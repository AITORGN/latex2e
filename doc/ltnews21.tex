%%%--- unfinished draft ---


% \iffalse meta-comment
%
% Copyright 2014
% The LaTeX3 Project and any individual authors listed elsewhere
% in this file.
%
% This file is part of the LaTeX base system.
% -------------------------------------------
%
% It may be distributed and/or modified under the
% conditions of the LaTeX Project Public License, either version 1.3c
% of this license or (at your option) any later version.
% The latest version of this license is in
%    http://www.latex-project.org/lppl.txt
% and version 1.3c or later is part of all distributions of LaTeX
% version 2005/12/01 or later.
%
% This file has the LPPL maintenance status "maintained".
%
% The list of all files belonging to the LaTeX base distribution is
% given in the file `manifest.txt'. See also `legal.txt' for additional
% information.
%
% The list of derived (unpacked) files belonging to the distribution
% and covered by LPPL is defined by the unpacking scripts (with
% extension .ins) which are part of the distribution.
%
% \fi
% Filename: ltnews21.tex
%
% This is issue 21 of LaTeX News.

\documentclass{ltnews}

\usepackage[T1]{fontenc}

\usepackage{lmodern,url}

\publicationmonth{???}
\publicationyear{2014}

\publicationissue{21}

\begin{document}

\maketitle

\section{Scheduled \LaTeX\ bug-fix release}

This issue of \LaTeX~News marks the second bug-fix release of
\LaTeXe\ since shifting to a new build system in 2009.
Provided sufficient changes are made, we expect to
repeat such releases yearly or bi-yearly to stay in sync with \TeX\ Live.

\section{Continued development}

\begin{itshape} % still from ltnews20

The \LaTeXe\ kernel is no longer being actively developed, as any non-negligible changes now could have dramatic backwards compatibility issues with old documents. Similarly, new features cannot be added to the kernel since any new documents written now would then be incompatible with legacy versions of \LaTeX{} of which there are many in the fields.

The situation on the package level is quite different though. While most of us have stopped developing packages for \LaTeXe{} there are many contributing developers that continue to enrich \LaTeXe{} by providing or extending add-on packages with new or better functionality.

However, the \LaTeX\ team certainly recognises that there are improvements to be made to the kernel code; over the last few years we have been working on building, expanding, and solidifying the \textsf{expl3} programming layer for future \LaTeX\ development. We are using \textsf{expl3} to build new interfaces for package development and tools for document design. Progress here is continuing.

\end{itshape}


\section{The tools directory}

\textit{document taking tools apart to support faster updates of individual packages in there}


\section{Release notes}

In addition to a few small documentation fixes, the following changes have been made to the \LaTeXe\ code; in accordance with the philosophy of minimising forwards and backwards compatibility problems, most of these will not be noticeable to the regular \LaTeX\ user.

\paragraph{???}

\paragraph{\texttt{fltrace} package}

explain the package usage


\paragraph{\texttt{multicol} updates}

1.8a support for balancing not overrunning a page
1.8b...

\paragraph{\texttt{tabularx} updates}
The restrictions on embedding \verb|\tabularx| \verb|\endtabularx| into the definition of a new
environment have been relaxed slightly. See the package documentation for details.

\paragraph{\texttt{color} updates}
The \verb|\nopagecolor| command suggested by Heiko Oberdiek that has been available 
for some years in the \textsf{pdftex} option has been added to the core package as suggested in 
graphics/3873. Currently this is supported in the driver files for \textsf{dvips} and \textsf{pdftex}.
Patches to support other drivers are welcome.

\end{document}

% \iffalse meta-comment
%
% Copyright 1993-2016
%
% The LaTeX3 Project and any individual authors listed elsewhere
% in this file.
%
% This file is part of the Standard LaTeX `Tools Bundle'.
% -------------------------------------------------------
%
% It may be distributed and/or modified under the
% conditions of the LaTeX Project Public License, either version 1.3c
% of this license or (at your option) any later version.
% The latest version of this license is in
%    http://www.latex-project.org/lppl.txt
% and version 1.3c or later is part of all distributions of LaTeX
% version 2005/12/01 or later.
%
% The list of all files belonging to the LaTeX `Tools Bundle' is
% given in the file `manifest.txt'.
%
% \fi
% \iffalse
%% File: tabularx.dtx Copyright (C) 1991-1999 2012 2014 2016 David Carlisle
%
%<*dtx>
          \ProvidesFile{tabularx.dtx}
%</dtx>
%<package>\NeedsTeXFormat{LaTeX2e}
%<package>\ProvidesPackage{tabularx}
%<driver> \ProvidesFile{tabularx.drv}
% \fi
%         \ProvidesFile{tabularx.dtx}
          [2016/02/03 v2.11 `tabularx' package (DPC)]
% \iffalse
%<*driver>
\documentclass{ltxdoc}
\usepackage[infoshow]{tabularx}
\begin{document}
 \DocInput{tabularx.dtx}
\end{document}
%</driver>
% \fi
%
% \changes{v1.00}{1992/01/30}{Initial version.}
% \changes{v1.01}{1992/07/07}{Re-issue for the new doc and docstrip.}
% \changes{v1.02}{1992/07/17}{Added some support for \cmd\verb}
% \changes{v1.03}{1992/08/17}
%    {Added \cs{ifnum0}!=`\{\cs{fi}\} brackets after report by
%     Andreas Maassen}
% \changes{v1.04}{1992/09/02}
%    {fixed \cmd\verb, and support footnotes.}
% \changes{v1.05}{1992/11/06}
%    {preserve all LaTeX counters}
% \changes{v1.06}{1993/08/02}
%    {(Martin Schroeder) Support the optional [t] or [b] argument.
%     Also now works with delarray.sty.}
% \changes{v1.07}{1993/08/27}
%    {Modifications to make this style compatible with calc.sty.}
% \changes{v2.00}{1994/03/14}
%    {Update to LaTeX2e}
% \changes{v2.01}{1994/05/22}
%    {New Tracing format.}
%
%
% %%%%%%%%%%%%%%%%%%%%%%%%%%%%%%%%%%%%%%%%%%%%%%%%%%%%%%%%%%%%%%%%%%%%%%
%
%
%
% \GetFileInfo{tabularx.dtx}
% \title{The \textsf{tabularx} package\thanks{This file
%        has version number \fileversion, last
%        revised \filedate.}}
% \author{David Carlisle}
% \date{\filedate}
% \MaintainedByLaTeXTeam{tools}
% \maketitle
% \DeleteShortVerb{\|}
% \MakeShortVerb{\"}
%
% \begin{abstract}
% A new environment, {\ttfamily tabularx}, is defined, which takes the
% same arguments as {\ttfamily tabular*}, but modifies the widths of
% certain columns, rather than the inter column space, to set a table
% with the requested total width. The columns that may stretch are
% marked with the new token {\ttfamily X} in the preamble argument.
%
% This package requires the {\ttfamily array} package.
% \end{abstract}
%
% \section{Introduction}
% This package implements a version of the {\ttfamily tabular}
% environment in which the widths of certain columns are calculated so
% that the table is a specified width. Requests for such an
% environment seem to occur quite regularly in {\ttfamily
% comp.text.tex}.
%
% \DescribeEnv{tabularx}
% "\begin{tabularx}{"\meta{width}"}["\meta{pos}"]{"\meta{preamble}"}"\\
% The arguments of "tabularx" are essentially the same as those of
% the standard "tabular*" environment. However rather than adding space
% between the columns to achieve the desired width, it adjusts the
% widths of some of the columns. The columns which are affected by the
% {\ttfamily tabularx} environment should be denoted with the letter
% {\ttfamily X} in the preamble argument. The {\ttfamily X} column
% specification will be converted to  "p{"\meta{some value}"}" once the
% correct column width has been calculated.
%
% \section{Examples}
%
% The following table is set with
% "\begin{tabularx}{250pt}{|c|X|c|X|} ...".
%
% \begin{center}
% \begin{tabularx}{250pt}{|c|X|c|X|}
% \hline
% \multicolumn{2}{|c|}{Multicolumn entry!}&
% THREE&
% FOUR\\
% \hline
% one&
% \raggedright\arraybackslash The width of this column depends on the
% width of the table.\footnote
% {You can now use {\ttfamily \bslash footnote} inside {\ttfamily
% tabularx}!}&
% three&
% \raggedright\arraybackslash Column four will act in the same way as
% column two, with the same width.\\
% \hline
% \end{tabularx}
% \end{center}
% If we change the first line to "\begin{tabularx}{300pt}{|c|X|c|X|}" we
% get:
% \begin{center}
% \begin{tabularx}{300pt}{|c|X|c|X|}
% \hline
% \multicolumn{2}{|c|}{Multicolumn entry!}&
% THREE&
% FOUR\\
% \hline
% one&
% \raggedright\arraybackslash The width of this column depends on the
% width of the table.&
% three&
% \raggedright\arraybackslash Column four will act in the same way as
% column two, with the same width.\\
% \hline
% \end{tabularx}
% \end{center}
%
% \edef\mytt{\expandafter\noexpand\csname
%             mdseries\endcsname\noexpand\ttfamily}
% \section{Differences between {\mytt tabularx} and {\mytt tabular*}}
% These two environments take the same arguments, to produce a table of
% a specified width. The main differences between them are:
% \begin{itemize}
% \item {\ttfamily tabularx} modifies the widths of the \emph{columns},
% whereas {\ttfamily tabular*} modifies the widths of the inter-column
% \emph{spaces}.
% \item {\ttfamily tabular} and {\ttfamily tabular*} environments may be
% nested with no restriction, however if one {\ttfamily tabularx}
% environment occurs inside another, then the inner one {\em must\/} be
% enclosed by "{ }".
% \item The body of the {\ttfamily tabularx} environment is in fact the
% argument to a command, and so certain constructions which are not
% allowed in command arguments (like "\verb") may not be used.\footnote
% {Since Version 1.02, {\ttfamily\bslash verb and \ttfamily\bslash
% verb*} may be used, but they may treat spaces incorrectly, and the
% argument can not contain an unmatched {\ttfamily\char`\{} or
% {\ttfamily\char`\}}, or a  {\ttfamily\char`\%} character.}
% \item {\ttfamily tabular*} uses a primitive capability of \TeX\ to
% modify the inter column space of an alignment. {\ttfamily tabularx}
% has to set the table several times as it searches for the best column
% widths, and is therefore much slower. Also the fact that the body is
% expanded several times may break certain \TeX\ constructs.
% \end{itemize}
%
% \section{Customising the behaviour of {\mytt tabularx}}
%
% \subsection{Terminal output}
% \DescribeMacro{\tracingtabularx}
% If this declaration is made, say in the document preamble, then all
% following {\ttfamily tabularx} environments will print information
% about column widths as they repeatedly re-set the tables to find the
% correct widths.
%
% As an alternative to using the "\tracingtabularx" declaration, either
% of the options "infoshow" or "debugshow" may be given, either in the
% "\usepackage" command that loads "tabularx", or as a global option
% in the "\documentclass" command.
%
% \subsection{The environment used to typeset the {\mytt X} columns}
% By default the {\ttfamily X} specification is turned into
% "p{"\meta{some value}"}". Such narrow columns often
% require a special format, this may be achieved using the ">" syntax
% of {\ttfamily array.sty}. So for example you may give a specification
% of ">{\small}X". Another format which is useful in narrow columns is
% ragged right, however \LaTeX's "\raggedright" macro redefines
% "\\" in a way which conflicts with its use in a tabular or array
% environments.
%\DescribeMacro{\arraybackslash}
% For this reason this package introduces the command "\arraybackslash",
% this may be used after a "\raggedright", "\raggedleft"  or
% "\centering" declaration. Thus a {\ttfamily tabularx} preamble may
% specify\\
% ">{\raggedright\arraybackslash}X".
%
% \DescribeMacro{\newcolumntype}
% These preamble specifications may of course be saved using the
% command, "\newcolumntype", defined in {\ttfamily array.sty}. Thus we
% may say\\
% "\newcolumntype{Y}{>{\small\raggedright\arraybackslash}X}"\\
% and then use {\ttfamily Y} in the {\ttfamily tabularx} preamble
% argument.
%
% \DescribeMacro{\tabularxcolumn}
% The {\ttfamily X} columns are set using the {\ttfamily p} column which
% corresponds  to "\parbox[t]". You may want them set using, say, the
% {\ttfamily m} column, which corresponds to "\parbox[c]". It is not
% possible to change the column type using the ">" syntax, so another
% system is provided.  "\tabularxcolumn" should be defined to be a macro
% with one argument, which expands to the {\ttfamily tabular} preamble
% specification that you want to correspond to {\ttfamily X}. The
% argument will be replaced by the calculated width of a column.
%
% The default is "\newcommand{\tabularxcolumn}[1]{p{#1}}". So we may
% change this with a command such as:\\
% "\renewcommand{\tabularxcolumn}[1]{>{\small}m{#1}}"
%
% \subsection{Column widths}
% Normally all {\ttfamily X} columns in a single table are set to the
% same width, however it is possible to make {\ttfamily tabularx} set
% them to different widths.
% A preamble argument of "{>{\hsize=.5\hsize}X>{\hsize=1.5\hsize}X}"
% specifies two columns, the second will be three times as wide as the
% first. However if you want to play games like this you should follow
% the following two rules.
% \begin{itemize}
% \item Make sure that the sum of the widths of all the {\ttfamily X}
% columns is unchanged. (In the above example, the new widths still add
% up to twice the default width, the same as two standard {\ttfamily X}
% columns.)
% \item Do not use "\multicolumn" entries which cross any {\ttfamily X}
% column.
% \end{itemize}
% As with most rules, these may be broken if you know what you are
% doing.
%
% \subsection{If the algorithm fails\ldots}
% It may be that the widths of the `normal' columns of the table
% already total more  than the requested total
% width. \textsf{tabularx} refuses to set the
% \texttt{X} columns to a negative width, so in this case you get a
% warning ``X Columns too narrow (table too wide)''.
%
% The \texttt{X} columns will in this case be set to a width of 1em
% and so the table itself will be wider than the requested total width
% given in the argument to the environment.
% This behaviour of the package can be customised slightly
% as noted in the documentation of the code section.
%
% \StopEventually{}
%
% \section{The Macros}
%
%    \begin{macrocode}
%<*package>
%    \end{macrocode}
%
% \changes{v2.00}{1994/02/07}
%    {New Option Handling}
%    \begin{macrocode}
\DeclareOption{infoshow}{\AtEndOfPackage\tracingtabularx}
\DeclareOption{debugshow}{\AtEndOfPackage\tracingtabularx}
\ProcessOptions
%    \end{macrocode}
%
% This requires {\ttfamily array.sty}.
% \changes{v2.00}{1994/02/07}
%    {Use LaTeX2e's \cmd{\RequirePackage} to load array}
%    \begin{macrocode}
\RequirePackage{array}[1994/02/03]
%    \end{macrocode}
%
% First some registers etc.\ that we need.
%    \begin{macrocode}
\newdimen\TX@col@width
\newdimen\TX@old@table
\newdimen\TX@old@col
\newdimen\TX@target
\newdimen\TX@delta
\newcount\TX@cols
\newif\ifTX@
%    \end{macrocode}
%
% Now a trick to get the body of an environment into a token register,
% without doing any expansion. This does not do any real checking of
% nested environments, so if you should need to nest one {\ttfamily
% tabularx} inside another, the inner one must be surrounded by "{ }".
%
% \begin{macro}{\tabularx}
% Prior to v1.06, this macro took two arguments, which were saved in
% separate registers before the table body was saved by "\TX@get@body".
% Unfortunately this disables the "[t]" optional argument. Now just save
% the width specification separately, then clear the token register
% "\toks@".
% Finally call "\TX@get@body" to begin saving the body of the table.
% The "{\ifnum0=`}\fi" was added at v1.03, to allow "tabularx" to appear
% inside a "\halign".^^A
% \setbox0=\hbox{\footnotesize"\iffalse{\fi\ifnum0=`}\fi"}^^A
% \setbox2=\hbox{\footnotesize"\ifnum0=`{}\fi"}^^A
% \footnote{This adds an extra level of grouping,
% which is not really needed. Instead, I could use \box0\ here, and
% \box2\ below, however the code here would then have to be moved after
% the first line, because of the footnote to page 386 of the \TeX{}Book,
% and I do not think I should be writing code that is so obscure as to
% be documented in a footnote in an appendix called ``Dirty Tricks''!}
%
% This mechanism of grabbing an environment body does have the
% disadvantage (shared with the AMS alignment environments) that you
% can not make extension environments by code such as
%\begin{verbatim}
%\newenvironment{foo}{\begin{tabularx}{XX}}{\end{tabularx}}
%\end{verbatim}
% as the code is looking for a literal string "\end{tabularx}" to stop
% scanning. Since version 2.02, one may avoid this problem by using
% "\tabularx" and "\endtabularx" directly in the definition:
%\begin{verbatim}
%\newenvironment{foo}{\tabularx{XX}}{\endtabularx}
%\end{verbatim}
% The scanner now looks for the end of the current environment ("foo" in
% this example.) There are some restrictions on this usage, the
% principal one being that "\endtabularx" must not be inside any "{ }" pairs
% ao that the code before "\endtabularx"  may be extracted and added to the table body
% (prior to version 2.09  "\endtabularx" had to be
% the \emph{first} token of the `end code' of the environment).
%    \begin{macrocode}
\def\tabularx#1{%
%    \end{macrocode}
% \changes{v2.02}{1995/03/20}
%    {New local setting of \cs{TX@}}
% \changes{v4.09}{1998/05/13}
%      {Use \cs{setlength}, so that calc extensions apply. tools/2793}
% Allow "\tabularx" "\endtabularx" (but not "\begin{tabularx}"
% "\end{tabularx}") to be used in "\newenvironment" definitions.
%    \begin{macrocode}
\edef\TX@{\@currenvir}%
  {\ifnum0=`}\fi
%    \end{macrocode}
% "\relax" added at v1.05 so that non-expandable length tokens, like
% "\textwidth" do not generate an extra space, and an overfull box.
% "\relax" removed again at v4.09 in favour of "\setlength" so if you
% use the calc package you can use a width of "(\textwidth-12pt)/2".
%    \begin{macrocode}
  \setlength\TX@target{#1}%
  \TX@typeout{Target width: #1 = \the\TX@target.}%
  \toks@{}\TX@get@body}
%    \end{macrocode}
% \end{macro}
%

%  \begin{macro}{\endtabularx}
% This does not do very much\ldots
% \changes{v2.02}{1995/03/20}
%    {Macro added}
%    \begin{macrocode}
\let\endtabularx\relax
%    \end{macrocode}
%  \end{macro}
%
% \begin{macro}{\TX@get@body}
% Place all tokens as far as the first "\end" into a token register.
% Then call "\TX@find@end" to see if we are at "\end{tabularx}".
%    \begin{macrocode}
\long\def\TX@get@body#1\end
  {\toks@\expandafter{\the\toks@#1}\TX@find@end}
%    \end{macrocode}
% \end{macro}
%
% \begin{macro}{\TX@find@end}
% If we are at "\end{tabularx}", call "\TX@endtabularx", otherwise
% add "\end{...}" to the register, and call "\TX@get@body" again.
%    \begin{macrocode}
\def\TX@find@end#1{%
  \def\@tempa{#1}%
  \ifx\@tempa\TX@\expandafter\TX@endtabularx
  \else\toks@\expandafter
    {\the\toks@\end{#1}}\expandafter\TX@get@body\fi}
%    \end{macrocode}
% \end{macro}
%
% \begin{macro}{\TX@find@endtabularxa}
% \changes{v2.09}{2014/04/22}{macro added}
% \changes{v2.10}{2014/05/13}{macro modified to test for missing \cs{endtabularx}}
% \changes{v2.11}{2016/01/03}{Fix to previous change to guard against empty arg 2. (Ulrike Fischer)}
% split up the end code, and extract the part that lives in the table body.
%    \begin{macrocode}
\long\def\TX@find@endtabularxa
       #1\endtabularx#2\endtabularx#3\TX@find@endtabularxa{%
  \ifx\TX@#2\relax\else
  \toks@\expandafter{\the\toks@#1}%
  \fi}
%    \end{macrocode}
% \end{macro}
% \begin{macro}{\TX@find@endtabularxb}
% \changes{v2.09}{2014/04/22}{macro added}
% \changes{v2.10}{2014/05/13}{macro modified to test for missing \cs{endtabularx}}
% split up the end code, and extract the part that lives outside the table body.
%    \begin{macrocode}
\long\def\TX@find@endtabularxb
       #1\endtabularx#2\endtabularx#3\TX@find@endtabularxb{%
  \ifx\TX@#2%
    \expandafter\@firstoftwo
  \else
     \expandafter\@secondoftwo
  \fi
  {#1}{#2}}
%    \end{macrocode}
% \end{macro}
%
% \begin{macro}{\TX@find@endtabularxbb}
% \changes{v2.10}{2014/05/13}{macro added}
% Helper to avoid needing 15 consecutive "expandafter"
%    \begin{macrocode}
\def\TX@find@endtabularxbb{%
  \expandafter\expandafter\expandafter
  \TX@find@endtabularxb
}
%    \end{macrocode}
% \end{macro}
%
% \begin{macro}{\TX@}
% The string {\ttfamily tabularx} as a macro for testing with "\ifx".
%    \begin{macrocode}
\def\TX@{tabularx}
%    \end{macrocode}
% \end{macro}
%
% Now that all the parts of the table specification are stored in
% registers, we can begin the work of setting the table.
%
% The algorithm for finding the correct column widths is as follows.
% Firstly set the table with each {\ttfamily X} column the width of the
% final table. Assuming that there is at least one {\ttfamily X} column,
% this will produce a table that is too wide. Divide the excess width by
% the number of {\ttfamily X} columns, and reduce the column width by
% this amount. Reset the table. If the table is not now the correct
% width, a "\multicolumn" entry must be `hiding' one of the {\ttfamily
% X} columns, and so there is one less {\ttfamily X} column affecting
% the width of the table. So we reduce by 1 the number of X columns and
% repeat the process.
%
% \begin{macro}{\TX@endtabularx}
% Although I have tried to make {\ttfamily tabularx} look like an
% environment, it is in fact a command, all the work is done by this
% macro.
%    \begin{macrocode}
\def\TX@endtabularx{%
%    \end{macrocode}
% \changes{v2.09}{2014/04/22}{call to \cs{TX@find@endtabularxa} added}
% \changes{v2.10}{2014/05/13}{macro modified to test for missing \cs{endtabularx}}
%    \begin{macrocode}
   \expandafter\expandafter\expandafter
     \TX@find@endtabularxa\csname end\TX@\endcsname
     \endtabularx\TX@\endtabularx\TX@find@endtabularxa
%    \end{macrocode}
% Define the {\ttfamily X} column, with an internal version of the
% "\newcolumntype" command. The "\expandafter" commands enable
% "\NC@newcol"  to get the {\em expansion} of\\
% "\tabularxcolumn{\TX@col@width}" as its
% argument. This will be the definition of an {\ttfamily X} column, as
% discussed in section 4.
%    \begin{macrocode}
  \expandafter\TX@newcol\expandafter{\tabularxcolumn{\TX@col@width}}%
%    \end{macrocode}
% Initialise the column width, and the number of {\ttfamily X} columns.
% The number of {\ttfamily X} columns is set to one, which means that
% the initial count will be one too high, but this value is decremented
% before it is used in the main loop.
%
% Since v1.02, switch the definition of "\verb".
%    \begin{macrocode}
  \let\verb\TX@verb
%    \end{macrocode}
% Since v1.05, save the values of all \LaTeX\ counters, the list
% "\cl@@ckpt" contains the names of all the \LaTeX\ counters that have
% been defined so far. We expand "\setcounter" at this point, as it
% results in fewer tokens being stored in "\TX@ckpt", but the actual
% resetting of the counters occurs when "\TX@ckpt" is expanded after
% each trial run.
% Actually since v1.07, use something equivalent to the expansion of the
% original definition of "\setcounter", so that "tabularx" works in
% conjunction with "calc.sty".
%    \begin{macrocode}
  \def\@elt##1{\global\value{##1}\the\value{##1}\relax}%
  \edef\TX@ckpt{\cl@@ckpt}%
  \let\@elt\relax
  \TX@old@table\maxdimen
  \TX@col@width\TX@target
  \global\TX@cols\@ne
%    \end{macrocode}
% Typeout some headings (unless this is disabled).
%    \begin{macrocode}
  \TX@typeout@
    {\@spaces Table Width\@spaces Column Width\@spaces X Columns}%
%    \end{macrocode}%
% First attempt. Modify the {\ttfamily X} definition to count {\ttfamily
% X} columns.
%    \begin{macrocode}
  \TX@trial{\def\NC@rewrite@X{%
          \global\advance\TX@cols\@ne\NC@find p{\TX@col@width}}}%
%    \end{macrocode}
% Repeatedly decrease column width until table is the correct width,
% or stops shrinking, or the columns become two narrow.
% If there are no multicolumn entries, this will only take one attempt.
%    \begin{macrocode}
  \loop
    \TX@arith
    \ifTX@
    \TX@trial{}%
  \repeat
%    \end{macrocode}
% One last time, with warnings back on (see appendix D)
% use {\ttfamily tabular*} to put it in a box of the right size, in case
% the algorithm failed to find the correct size.
%
% Since v1.04, locally make "\footnotetext" save its argument in a token
% register.
% Since v1.06, "\toks@" contains the preamble specification,
% and possible optional argument, as well as the table body.
%    \begin{macrocode}
  {\let\@footnotetext\TX@ftntext\let\@xfootnotenext\TX@xftntext
    \csname tabular*\expandafter\endcsname\expandafter\TX@target
      \the\toks@
    \csname endtabular*\endcsname}%
%    \end{macrocode}
% Now the alignment is finished, and the "}" has restored the original
% meaning of "\@footnotetext" expand the register "\TX@ftn" which will
% execute a series of\\
% "\footnotetext["\meta{num}"]{"\meta{note}"}"\\
% commands. We need to be careful about clearing the register as we may
% be inside a nested {\ttfamily tabularx}.
%    \begin{macrocode}
  \global\TX@ftn\expandafter{\expandafter}\the\TX@ftn
%    \end{macrocode}
% Now finish off the {\ttfamily tabularx} environment. Note that we need
% "\end{tabularx}" here as the "\end{tabularx}" in the user's
% file is never expanded. Now use "\TX@" rather than "tabularx".
% \changes{v2.02}{1995/03/20}
%    {Close the environment \cs{TX@} rather than `tabularx'}
%
% We also need to finish off the group started by "{\ifnum0=`}\fi" in
% the macro "\tabularx".
%    \begin{macrocode}
  \ifnum0=`{\fi}%
%    \end{macrocode}
% \changes{v2.09}{2014/04/22}{call to \cs{TX@find@endtabularxb} added}
% \changes{v2.10}{2014/05/13}{macro modified to test for missing \cs{endtabularx}}
%    \begin{macrocode}
   \expandafter\expandafter\expandafter
   \TX@find@endtabularxbb
    \expandafter\end\expandafter{\TX@}%
    \endtabularx\TX@\endtabularx\TX@find@endtabularxb
}
%    \end{macrocode}
% \end{macro}
%
%
% \begin{macro}{\TX@arith}
% Calculate the column width for the next try, setting the flag
% "\ifTX@" to false if the loop should be aborted.
%    \begin{macrocode}
\def\TX@arith{%
  \TX@false
%    \end{macrocode}
% \changes{v2.08}{2012/02/06}{Avoid exceeding maxdimen during trials}
%    \begin{macrocode}
  \@tempdimb\maxdimen
  \divide\@tempdimb\TX@cols
  \ifdim\TX@col@width>\@tempdimb
    \TX@typeout@{Don't exceed \maxdimen}%
    \wd\@tempboxa\maxdimen
  \fi
  \ifdim\TX@old@table=\wd\@tempboxa
%    \end{macrocode}
% If we have reduced the column width, but the table width has not
% changed, we stop the loop, and output the table (which will cause an
% over-full alignment) with the previous value of "\TX@col@width".
%    \begin{macrocode}
    \TX@col@width\TX@old@col
    \TX@typeout@{Reached minimum width, backing up.}%
  \else
%    \end{macrocode}
% Otherwise calculate the amount by which the current table is too wide.
%    \begin{macrocode}
    \dimen@\wd\@tempboxa
    \advance\dimen@ -\TX@target
    \ifdim\dimen@<\TX@delta
%    \end{macrocode}
% If this amount is less than "\TX@delta", stop. ("\TX@delta"
% should be non-zero otherwise we may miss the target due to rounding
% error.)
%    \begin{macrocode}
      \TX@typeout@{Reached target.}%
    \else
%    \end{macrocode}
% Reduce the number of effective {\ttfamily X} columns by one. (Checking
% that we do not get 0, as this would produce an error later.) Then
% divide excess width by the number of effective columns, and calculate
% the new column width. Temporarily store this value (times $-1$) in
% "\dimen@".
%    \begin{macrocode}
      \ifnum\TX@cols>\@ne
        \advance\TX@cols\m@ne
      \fi
      \divide\dimen@\TX@cols
      \advance\dimen@ -\TX@col@width
      \ifdim \dimen@ >\z@
%    \end{macrocode}
% If the new width would be too narrow, abort the loop. At the moment
% too narrow, means less than 0\,pt!
%
% Prior to v2.03, if the loop was aborted here, the X columns were left
% with the width of the previous run, but this may make the table far
% too wide as initial guesses are always too big. Now force to
% "\TX@error@width" which defaults to be 1em. If you want to
% get the old behaviour stick\\
%  "\renewcommand\TX@error@width{\TX@col@width}"\\
% in a package file loaded after \textsf{tabularx}.
% \changes{v2.03}{1997/02/20}{Improve warning message and force to 1em.}
%    \begin{macrocode}
        \PackageWarning{tabularx}%
           {X Columns too narrow (table too wide)\MessageBreak}%
        \TX@col@width\TX@error@width\relax
      \else
%    \end{macrocode}
% Otherwise save the old settings, and set the new column width. Set the
% flag to true so that the table will be set, and the loop will be
% executed again.
%    \begin{macrocode}
        \TX@old@col\TX@col@width
        \TX@old@table\wd\@tempboxa
        \TX@col@width-\dimen@
        \TX@true
      \fi
    \fi
  \fi}
%    \end{macrocode}
% \end{macro}
%
% \begin{macro}{\TX@error@width}
% \changes{v2.03}{1997/02/20}{macro added.}
% \changes{v2.04}{1997/02/26}{spurious brace removed.}
% If the calculated width is negative, use this instead.
%    \begin{macrocode}
\def\TX@error@width{1em}
%    \end{macrocode}
% \end{macro}
%
% \begin{macro}{\TX@delta}
% Accept a table that is within "\hfuzz" of the correct width.
%    \begin{macrocode}
\TX@delta\hfuzz
%    \end{macrocode}
% \end{macro}
%
% Initialise the {\ttfamily X} column. The definition can be empty here,
% as it is set for each {\ttfamily tabularx} environment.
%    \begin{macrocode}
\newcolumntype{X}{}
%    \end{macrocode}
%
% \begin{macro}{\tabularxcolumn}
% The default definition of {\ttfamily X} is "p{#1}".
%    \begin{macrocode}
\def\tabularxcolumn#1{p{#1}}
%    \end{macrocode}
% \end{macro}
%
% \begin{macro}{\TX@newcol}
% A little macro just used to cut down the number of "\expandafter"
% commands needed.
%    \begin{macrocode}
\def\TX@newcol{\newcol@{X}[0]}
%    \end{macrocode}
% \end{macro}
%
% \begin{macro}{\TX@trial}
% Make a test run.
%    \begin{macrocode}
\def\TX@trial#1{%
  \setbox\@tempboxa\hbox{%
%    \end{macrocode}
% Any extra commands. This is used on the first run to count the number
% of {\ttfamily X} columns.
%    \begin{macrocode}
    #1\relax
%    \end{macrocode}
% Since v1.04, make "\footnotetext" gobble its arguments. Also locally
% clear "\TX@vwarn" so that the warning is generated by the {\ttfamily
% final} run, and does not appear in the middle of the table if
% "\tracingtabularx".
%    \begin{macrocode}
  \let\@footnotetext\TX@trial@ftn
  \let\TX@vwarn\@empty
%    \end{macrocode}
% Do not nest {\ttfamily tabularx} environments during trial runs. This
% would waste time, and the global setting of "\TX@cols" would break the
% algorithm.
%    \begin{macrocode}
   \expandafter\let\expandafter\tabularx\csname tabular*\endcsname
   \expandafter\let\expandafter\endtabularx\csname endtabular*\endcsname
%    \end{macrocode}
% Added at v1.05: disable "\write"s during a trial run. This trick is
% from the \TeX{}Book.\footnote{Actually the \TeX{}Book trick does
% not work correctly, so changed for v2.05.}
% \changes{v2.05}{1997/09/18}
%    {New \cs{write} trick. tools/2607}
% \changes{v2.07}{1999/01/07}
%    {Newer \cs{write} trick. tools/2792}
%    \begin{macrocode}
   \def\write{\begingroup
     \def\let{\afterassignment\endgroup\toks@}%
        \afterassignment\let\count@}%
%    \end{macrocode}
% Turn off warnings (see appendix D). Also prevent them being turned
% back on by setting the parameter names to be registers.
%    \begin{macrocode}
    \hbadness\@M
    \hfuzz\maxdimen
    \let\hbadness\@tempcnta
    \let\hfuzz\@tempdima
%    \end{macrocode}
% Make the table, and finish the hbox.
% Since v1.06, "\toks@" contains the preamble specification,
% and possible optional argument, as well as the table body.
%    \begin{macrocode}
    \expandafter\tabular\the\toks@
    \endtabular}%
%    \end{macrocode}
% Since v1.05 reset all \LaTeX\ counters, by executing "\TX@ckpt".
%    \begin{macrocode}
  \TX@ckpt
%    \end{macrocode}
% Print some statistics.
% Added "\TX@align" in v1.05, to line up the columns.
%    \begin{macrocode}
  \TX@typeout@{\@spaces
     \expandafter\TX@align
        \the\wd\@tempboxa\space\space\space\space\space\@@
     \expandafter\TX@align
        \the\TX@col@width\space\space\space\space\space\@@
     \@spaces\the\TX@cols}}
%    \end{macrocode}
% \end{macro}
%
% \begin{macro}{\TX@align}
% Macro added at v1.05, to improve the printing of the tracing info.
%    \begin{macrocode}
\def\TX@align#1.#2#3#4#5#6#7#8#9\@@{%
  \ifnum#1<10 \space\fi
  \ifnum#1<100 \space\fi
  \ifnum#1<\@m\space\fi
  \ifnum#1<\@M\space\fi
  #1.#2#3#4#5#6#7#8\space\space}
%    \end{macrocode}
% \end{macro}
%
% \begin{macro}{\arraybackslash}
% "\\" hack.
% \changes{v2.08}{2012/02/06}{Don't define this if already defined by newer array package and use tabularnewline now this is in 2e.}
%    \begin{macrocode}
\ifx\arraybackslash\@undefined
\def\arraybackslash{\let\\\tabularnewline}
\fi
%    \end{macrocode}
% \end{macro}
%
% \begin{macro}{\tracingtabularx}
% Print statistics on column and table widths.
%    \begin{macrocode}
\def\tracingtabularx{%
  \def\TX@typeout{\PackageWarningNoLine{tabularx}}%
  \def\TX@typeout@##1{\typeout{(tabularx) ##1}}}
%    \end{macrocode}
% \end{macro}
%
% \begin{macro}{\TX@typeout}
% The default is to be to be quiet
%    \begin{macrocode}
\let\TX@typeout\@gobble
\let\TX@typeout@\@gobble
%    \end{macrocode}
% \end{macro}
%
% \begin{macro}{\TX@ftn}
% A token register for saving footnote texts.
%    \begin{macrocode}
\newtoks\TX@ftn
%    \end{macrocode}
% \end{macro}
%
% \begin{macro}{\TX@ftntext}
% \begin{macro}{\TX@xftntext}
% Inside the alignment just save up the footnote text in a token
% register.
%    \begin{macrocode}
\long\def\TX@ftntext#1{%
  \edef\@tempa{\the\TX@ftn\noexpand\footnotetext
                    [\the\csname c@\@mpfn\endcsname]}%
  \global\TX@ftn\expandafter{\@tempa{#1}}}%
\long\def\TX@xftntext[#1]#2{%
  \global\TX@ftn\expandafter{\the\TX@ftn\footnotetext[#1]{#2}}}
%    \end{macrocode}
% \end{macro}
% \end{macro}
%
% \begin{macro}{\TX@trial@ftn}
% On trial runs, gobble footnote texts.
%    \begin{macrocode}
\long\def\TX@trial@ftn#1{}
%    \end{macrocode}
% \end{macro}
%
% This last section was added at Version 1.02. Previous versions
% documented the fact that "\verb" did not work inside {\ttfamily
% tabularx}, but that did not stop people using it! This usually put
% \LaTeX\ into an irrecoverable error position, with error messages that
% did not mention the cause of the error. The `poor man's "\verb"' (and
% "\verb*") defined here is based on page 382 of the \TeX{}Book. As
% explained there, doing verbatim this way means that spaces are not
% treated correctly, and so "\verb*" may well be useless, however I
% consider this section of code to be error-recovery, rather than a real
% implementation of verbatim.
%
% The mechanism is quite general, and any macro which wants to allow a
% form of "\verb" to be used within its argument may
% "\let\verb=\TX@verb". (Making sure to restore the real definition
% later!)
%
% "\verb" and "\verb*" are subject to the following restrictions:
% \begin{enumerate}
% \item Spaces in the argument are not read verbatim, but may be skipped
%       according to \TeX's usual rules.
% \item Spaces will be added to the output after control words, even if
%       they were not present in the input.
% \item Unless the argument is a single space, any trailing space,
%       whether in the original argument, or added as in (2),
%       will be omitted.
% \item The argument must not end with "\", so "\verb|\|" is not
%      allowed, however, because of (3), "\verb|\ |" produces
%      "\".
% \item The argument must be balanced with respect to "{" and "}". So
%      "\verb|{|" is not allowed.
% \item A comment character like "%" will not appear verbatim. It will
%       act as usual, commenting out the rest of the input line!
% \item The combinations "?`" and "!`" will appear as
%       {\ttfamily?`} and {\ttfamily!`} if the {\ttfamily cmtt} font is
%       being used.
% \end{enumerate}
%
% \begin{macro}{\TX@verb}
% The internal definition of "\verb". Spaces will be replaced by "~", so
% for the star-form, "\let" "~" be \verb*| |, which we obtain as
% "\uppercase{*}". Use "{\ifnum0=`}\fi" rather than "\bgroup" to allow
% "&" to appear in the argument.
%    \begin{macrocode}
{\uccode`\*=`\ %
\uppercase{\gdef\TX@verb{%
  \leavevmode\null\TX@vwarn
  {\ifnum0=`}\fi\ttfamily\let\\\ignorespaces
  \@ifstar{\let~*\TX@vb}{\TX@vb}}}}
%    \end{macrocode}
% \end{macro}
%
% \begin{macro}{\TX@vb}
% Get the `almost verbatim' text using "\meaning". The `"!"' is added to
% the front of the user supplied text, to ensure that the whole argument
% does not consist of a single "{ }" group. \TeX\ would strip the outer
% braces from such a group. The `"!"' will be removed later.
%
% Originally I followed Knuth, and had "\def\@tempa{##1}", however this
% did not allow "#" to appear in the argument. So in v1.04, I changed
% this to to use a token register, and "\edef". This allows "#" appear,
% but makes each one appear twice!, so later we loop through, replacing
% "##" by "#".
%    \begin{macrocode}
\def\TX@vb#1{\def\@tempa##1#1{\toks@{##1}\edef\@tempa{\the\toks@}%
    \expandafter\TX@v\meaning\@tempa\\ \\\ifnum0=`{\fi}}\@tempa!}
%    \end{macrocode}
% \end{macro}
%
% \begin{macro}{\TX@v}
% Strip the initial segment of the "\meaning", including the `"!"'
% added earlier.
%    \begin{macrocode}
\def\TX@v#1!{\afterassignment\TX@vfirst\let\@tempa= }
%    \end{macrocode}
% \end{macro}
%
% As explained above we are going to replace "##" pairs by "#". To do
% this we need non-special "#" tokens.  Make "*" into a parameter
% token so that we can define macros with arguments. The normal meanings
% will be restored by the "\endgroup" later.
%    \begin{macrocode}
\begingroup
\catcode`\*=\catcode`\#
\catcode`\#=12
%    \end{macrocode}
%
% \begin{macro}{\TX@vfirst}
% As a special case, prevent the first character from being dropped.
% This makes "\verb*| |" produce \verb*| |. Then call "\TX@v@".
% This is slightly tricky since v1.04, as I have to ensure that an
% actual "#" rather than a command "\let" to "#" is passed on if the
% first character is "#".
%    \begin{macrocode}
\gdef\TX@vfirst{%
  \if\@tempa#%
    \def\@tempb{\TX@v@#}%
  \else
    \let\@tempb\TX@v@
    \if\@tempa\space~\else\@tempa\fi
  \fi
  \@tempb}
%    \end{macrocode}
% \end{macro}
%
% \begin{macro}{\TX@v@}
% Loop through the "\meaning", replacing all spaces by "~". If the last
% character is a space it is dropped, so that "\verb*|\LaTeX|" produces
% "\LaTeX" not \verb*|\LaTeX |. The rewritten tokens are then further
% processed to replace "##" pairs.
%    \begin{macrocode}
\gdef\TX@v@*1 *2{%
  \TX@v@hash*1##\relax\if*2\\\else~\expandafter\TX@v@\fi*2}
%    \end{macrocode}
% \end{macro}
%
% \begin{macro}{\TX@v@hash}
% The inner loop, replacing "##" by "#".
%    \begin{macrocode}
\gdef\TX@v@hash*1##*2{*1\ifx*2\relax\else#\expandafter\TX@v@hash\fi*2}
%    \end{macrocode}
% \end{macro}
%
% As promised, we now restore the normal meanings of "#" and "*".
%    \begin{macrocode}
\endgroup
%    \end{macrocode}
%
% \begin{macro}{\TX@vwarn}
% Warn the user the first time this "\verb" is used.
%    \begin{macrocode}
\def\TX@vwarn{%
  \@warning{\noexpand\verb may be unreliable inside tabularx}%
  \global\let\TX@vwarn\@empty}
%    \end{macrocode}
% \end{macro}
%
%    \begin{macrocode}
%</package>
%    \end{macrocode}
%
% \Finale
\endinput

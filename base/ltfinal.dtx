% \iffalse meta-comment
%
% Copyright 1993-2015
% The LaTeX3 Project and any individual authors listed elsewhere
% in this file.
%
% This file is part of the LaTeX base system.
% -------------------------------------------
%
% It may be distributed and/or modified under the
% conditions of the LaTeX Project Public License, either version 1.3c
% of this license or (at your option) any later version.
% The latest version of this license is in
%    http://www.latex-project.org/lppl.txt
% and version 1.3c or later is part of all distributions of LaTeX
% version 2005/12/01 or later.
%
% This file has the LPPL maintenance status "maintained".
%
% The list of all files belonging to the LaTeX base distribution is
% given in the file `manifest.txt'. See also `legal.txt' for additional
% information.
%
% The list of derived (unpacked) files belonging to the distribution
% and covered by LPPL is defined by the unpacking scripts (with
% extension .ins) which are part of the distribution.
%
% \fi
%
% \iffalse
%%% From File: ltfinal.dtx
%
%<*driver>
% \fi
\ProvidesFile{ltfinal.dtx}
             [2015/01/03 v2.0a LaTeX Kernel (Final Settings)]
% \iffalse
\documentclass{ltxdoc}
\GetFileInfo{ltfinal.dtx}
\title{\filename}
\date{\filedate}
\author{%
  Johannes Braams\and
  David Carlisle\and
  Alan Jeffrey\and
  Leslie Lamport\and
  Frank Mittelbach\and
  Chris Rowley\and
  Rainer Sch\"opf}
\begin{document}
\maketitle
 \DocInput{ltfinal.dtx}
\end{document}
%</driver>
% \fi
%
% \CheckSum{560}
%
% \section{Final settings}
% This section contains the final settings for \LaTeX.  It initialises
% some debugging and typesetting parameters, sets the default
% |\catcode|s and uc/lc codes, and inputs the hyphenation file.
%
% \StopEventually{}
%
% \changes{v0.1a}{1994/03/07}{Initial version, split from latex.dtx}
% \changes{v0.1a}{1994/03/07}{Remove oldcomments environment}
% \changes{v0.1c}{1994/04/21}{Added comments, set the catcodes of
%    128--255.}
% \changes{v0.1d}{1994/04/23}{Check that \cs{font@submax} is still zero}
% \changes{v0.1e}{1994/05/02}{Set all the catcodes}
% \changes{v0.1f}{1994/05/03}{Set the catcode of control-J to be
%    `other', for use in messages.}
% \changes{v0.1g}{1994/05/05}{Added empty errhelp.}
% \changes{v0.1h}{1994/05/13}{Added package ot1enc, and defined
%    \cs{@acci}, \cs{@accii} and \cs{@acciii}.}
% \changes{v0.1j}{1994/05/18}{Corrected the lccode for d-bar.}
% \changes{v0.1k}{1994/05/19}{Removed \cs{makeat...}}
% \changes{v1.0n}{1994/05/31}{Renamed lthyphen.* to lthyphen.*.}
% \changes{v1.0o}{1994/11/17}
%         {\cs{@tempa} to \cs{reserved@a}}
% \changes{v1.0p}{1994/12/01}
%         {Renamed lthyphen.* to hyphen.*.}
% \changes{v1.0r}{1995/06/05}
%         {Added \cs{MakeUppercase} and \cs{MakeLowercase}.}
% \changes{v1.0s}{1995/06/06}
%         {Made \cs{MakeUppercase} and \cs{MakeLowercase} brace their
%         argument.}
%
% \subsection{Debugging}
%
% By default, \LaTeX{} shows statistics:
%    \begin{macrocode}
%<*2ekernel>
\tracingstats1
%    \end{macrocode}
%
% \subsection{Typesetting parameters}
%
% \begin{macro}{\@lowpenalty}
% \begin{macro}{\@medpenalty}
% \begin{macro}{\@highpenalty}
%    These are penalties used internally.
%    \begin{macrocode}
\newcount\@lowpenalty
\newcount\@medpenalty
\newcount\@highpenalty
%    \end{macrocode}
% \end{macro}
% \end{macro}
% \end{macro}
%
%
%\begin{macro}{\newmarks}
% \changes{v2.0a}{2014/12/30}{macro added}
%\begin{macro}{\new@marks}
% \changes{v2.0a}{2014/12/30}{macro added}
% Allocate extended marks types if etex is active.
% Placed here at the end of the format
% to increase compatibility with count allocations
% in earlier releases.
%    \begin{macrocode}
%</2ekernel>
%<*2ekernel|latexrelease>
%<latexrelease>\IncludeInRelease{2015/01/01}%
%<latexrelease>                 {\newmarks}{Extended Allocation}{%
%    \end{macrocode}
%
%    \begin{macrocode}
\ifx\marks\@undefined\else
\newcount\marks@alloc
\def\newmarks{%
  \e@alloc\marks \e@alloc@chardef \mark@salloc\e@alloc@top\e@alloc@top}
\fi
%    \end{macrocode}
%
%    \begin{macrocode}
%</2ekernel|latexrelease>
%<latexrelease>}
%<latexrelease>\IncludeInRelease{0000/00/00}%
%<latexrelease>                 {\newmarks}{Extended Allocation}{%
%<latexrelease>\let\marks@alloc\@undefined
%<latexrelease>\let\newmarks\@undefined
%<latexrelease>}
%<*2ekernel>
%    \end{macrocode}
% \end{macro}
% \end{macro}
%
%\begin{macro}{\newXeTeXintercharclass}
% \changes{v2.0a}{2014/12/30}{macro added}
%\begin{macro}{\xe@alloc@intercharclass}
% \changes{v2.0a}{2014/12/30}{macro added}
% Allocate |\XeTeXintercharclass|  types if xetex is active.
% previously defined in |xetex.ini|.
%
% Placed here at the end of the format
% to increase compatibility with count allocations
% in earlier releases.
%    \begin{macrocode}
%</2ekernel>
%<*2ekernel|latexrelease>
%<latexrelease>\IncludeInRelease{2015/01/01}%
%<latexrelease>              {\newXeTeXintercharclass}{Extended Allocation}{%
%    \end{macrocode}
%
% Classes allocatedfrom 4 (1,2 and 3 are used by CJK), up to 254.
%    \begin{macrocode}
\ifx\XeTeXcharclass\@undefined
\else
\newcount\xe@alloc@intercharclass 
\xe@alloc@intercharclass=\thr@@ 
\def\newXeTeXintercharclass{%
 \e@alloc\XeTeXcharclass\chardef\xe@alloc@intercharclass\@cclv\@cclv}
\fi
%    \end{macrocode}
%
%    \begin{macrocode}
%</2ekernel|latexrelease>
%<latexrelease>}
%<latexrelease>\IncludeInRelease{0000/00/00}%
%<latexrelease>              {\newXeTeXintercharclass}{Extended Allocation}{%
%<latexrelease>\let\xe@alloc@intercharclass\@undefined
%<latexrelease>\let\newXeTeXintercharclass\@undefined
%<latexrelease>}
%<*2ekernel>
%    \end{macrocode}
% \end{macro}
% \end{macro}
%
%
% The default values of the picture and |\fbox| parameters:
%    \begin{macrocode}
\unitlength = 1pt
\fboxsep = 3pt
\fboxrule = .4pt
%    \end{macrocode}
% The saved value of \TeX's |\maxdepth|:
%    \begin{macrocode}
\@maxdepth       = \maxdepth
%    \end{macrocode}
% |\vsize| initialized because a |\clearpage| with |\vsize < \topskip|
%  causes trouble.
% |\@colroom| and |\@colht| also initialized because |\vsize| may be
%  set to them if a |\clearpage| is done before the |\begin{document}|
%
%    \begin{macrocode}
\vsize = 1000pt
\@colroom = \vsize
\@colht = \vsize
%    \end{macrocode}
% Initialise |\textheight| |\textwidth| and page style, to avoid
% internal errors if they are not set by the class.
% \changes{v0.1b}{1994/04/18}
%         {Initialise \cs{textheight}, \cs{textwidth} and page style}
%    \begin{macrocode}
\textheight=.5\maxdimen
\textwidth=\textheight
\ps@empty
%    \end{macrocode}
%
% \subsection{Lccodes for hyphenation}
%
% \changes{v1.1b}{1998/05/20}{Set up lccodes before loading
%    hyphenation files: pr/2639}
%    We set things up so that hyphenation files can assume that the
%    default (T1) lccodes are in use (at present this also sets up the
%    uccodes).
%    We temporarily define |\reserved@a| to apply |\reserved@c| to
%    all the numbers in the range of its arguments.
%    \begin{macrocode}
\def\reserved@a#1#2{%
   \@tempcnta#1\relax
   \@tempcntb#2\relax
   \reserved@b
}
\def\reserved@b{%
   \ifnum\@tempcnta>\@tempcntb\else
      \reserved@c\@tempcnta
      \advance\@tempcnta\@ne
      \expandafter\reserved@b
   \fi
}
%    \end{macrocode}
%    Depending on the \TeX{} version, we might not be allowed to do
%    this for non-ASCII characters.
% \changes{v1.0n}{1994/06/09}{For \TeX2, do not set codes for higher
%                   half of character table.}
%    \begin{macrocode}
\def\reserved@c#1{%
   \count@=#1\advance\count@ by -"20
   \uccode#1=\count@
   \lccode#1=#1
}
\reserved@a{`\a}{`\z}
\ifnum\inputlineno=\m@ne\else
  \reserved@a{"A0}{"BC}
  \reserved@a{"E0}{"FF}
\fi
%    \end{macrocode}
% The upper case characters need their |\uccode| and |\lccode| values
% set, and their |\sfcode| set to 999.
%    \begin{macrocode}
\def\reserved@c#1{%
   \count@=#1\advance\count@ by "20
   \uccode#1=#1
   \lccode#1=\count@
   \sfcode#1=999
}
\reserved@a{`\A}{`\Z}
\ifnum\inputlineno=\m@ne\else
  \reserved@a{"80}{"9C}
  \reserved@a{"C0}{"DF}
\fi
%    \end{macrocode}
% Well, it would be nice if that were correct, but unfortunately, the
% Cork encoding contains some odd slots whose uccode or lccode isn't
% quite what you'd expect.
%    \begin{macrocode}
\uccode`\^^Y=`\I     % dotless i
\lccode`\^^Y=`\^^Y   % dotless i
\uccode`\^^Z=`\J     % dotless j, ae in OT1
\lccode`\^^Z=`\^^Z   % dotless j, ae in OT1
\ifnum\inputlineno=\m@ne\else
  \lccode`\^^9d=`\i    % dotted I
  \uccode`\^^9d=`\^^9d % dotted I
  \lccode`\^^9e=`\^^9e % d-bar
  \uccode`\^^9e=`\^^d0 % d-bar
\fi
%    \end{macrocode}
% Finally here is one that helps hyphenation in the OT1 encoding.
% \changes{v1.0z}{1996/10/31}
%    {Added extra \cs{lcode}, hoping it does no harm in T1 (pr/1969)}
%    \begin{macrocode}
\lccode`\^^[=`\^^[   % oe in OT1
%    \end{macrocode}
%
% And we also set the |\lccode| of |\-| and |\textcompwordmark| so
% that they do not prevent hyphenation in the remainder of the word
% (as suggested by Lars Helstr\"om).
% \changes{v1.1e}{2003/10/13}
%    {Added extra \cs{lccode} for \cs{-} and \cs{textcompwordmark}}
%    \begin{macrocode}
\lccode`\- =`\-   % default hyphen char
\lccode 127=127   % alternate hyphen char
\lccode 23 =23    % textcompwordmark in T1
%    \end{macrocode}
%
% \changes{v2.0a}{2015/01/03}{Unicode data loading added}
%  The above is all well and good for $7$- and $8$-bit engines where the
%  assumption of T1 encodings is the basis for the hyphenation patterns.
%  That's not the case for the Unicode engines, where the assumption is
%  engine-native working. The file |ltunicode.ltx| contains data extracted
%  from the master Unicode Consortium information covering not only 
%  |\lccode| but also other related data. The |\lccode| part of that
%  at least needs to be loaded before hyphenation is tackled: Xe\TeX{}
%  follows the standard \TeX{} route of building patterns into the format.
%  Lua\TeX{} doesn't require this data be loaded \emph{here} but it does
%  need to be loaded somewhere. Rather than test for the Unicode engines
%  by name, the approach here is to look for the extended math mode handling
%  both provide: any other engine developed in this area will presumably also
%  provide |\Umathcode| (older Xe\TeX{} versions use |\XeTeXmathcode| so that
%  is covered too).
%    \begin{macrocode}
\ifnum 0%
  \ifx\Umathcode\@undefined\else 1\fi
  \ifx\XeTeXmathcode\@undefined\else 1\fi
  >\z@
  \message{ Unicode character data,}
  % \iffalse meta-comment
%
% Copyright 2014-2015
% The LaTeX3 Project and any individual authors listed elsewhere
% in this file.
%
% This file is part of the LaTeX base system.
% -------------------------------------------
%
% It may be distributed and/or modified under the
% conditions of the LaTeX Project Public License, either version 1.3c
% of this license or (at your option) any later version.
% The latest version of this license is in
%    http://www.latex-project.org/lppl.txt
% and version 1.3c or later is part of all distributions of LaTeX
% version 2005/12/01 or later.
%
% This file has the LPPL maintenance status "maintained".
%
% The list of all files belonging to the LaTeX base distribution is
% given in the file `manifest.txt'. See also `legal.txt' for additional
% information.
%
% The list of derived (unpacked) files belonging to the distribution
% and covered by LPPL is defined by the unpacking scripts (with
% extension .ins) which are part of the distribution.
%
% -----------------------------------------------------------------------------
%
% The same approach as used in \pkg{DocStrip}: if \cs{documentclass}
% is undefined then skip the driver, allowing the file to be used directly.
% This works as the \cs{fi} is only seen if \LaTeX{} is not in use. The odd
% \cs{jobname} business allows the extraction to work with \LaTeX{} provided
% an appropriate \texttt{.ins} file is set up.
%<*gobble>
\ifx\jobname\relax
  \let\documentclass\undefined
\fi
\begingroup\expandafter\expandafter\expandafter\endgroup
\expandafter\ifx\csname documentclass\endcsname\relax
\else
  \csname fi\endcsname
%</gobble>
%
%<*driver>
\ProvidesFile{ltunicode.dtx}
  [2015/01/01 v1.0 LaTeX Kernel (Unicode data)]
\documentclass{ltxdoc}
\begin{document}
\DocInput{\jobname.dtx}
\end{document}
%<*gobble>
\fi
%</gobble>
%</driver>
% \fi
%
% \GetFileInfo{ltunicode.dtx}
% \title{The \texttt{ltunicode.dtx} file\thanks
%     {This file has version number \fileversion, dated \filedate.}\\
%       for use with \LaTeXe}
% \author{The \LaTeX3 Project}
%
% \maketitle
%
% This script extracts data from the Unicode Consortium files
% \texttt{UnicodeData.txt}, \texttt{EastAsianWidth.txt} and
% \texttt{LineBreak.txt} to be used for setting up \LaTeXe{} with sane
% default settings when using the Xe\TeX{} and Lua\TeX{} engines. Details
% of the process are included in the code comments.
% 
% To create the extracted file, run this file in a location containing
% the three input data files using a plain \TeX{} system with the e-\TeX{}
% extensions enabled (\texttt{pdftex}, \texttt{xetex} or \texttt{luatex}
% in any modern \TeX{} distribution).
%
% \StopEventually{}
%
%    \begin{macrocode}
%<*script>
%    \end{macrocode}
%
% \section{General set up}
%
% The script is designed to work with plain \TeX{} and so |@| is made into
% a `letter' using the primitive approach.
%    \begin{macrocode}
\catcode`\@=11 %
%    \end{macrocode}
%
% \begin{macro}{\gobble}
% \begin{macro}{\firsttoken}
%   Standard utilities.
%    \begin{macrocode}
\long\def\gobble#1{}
\long\def\firsttoken#1#2\relax{#1}
%    \end{macrocode}
% \end{macro}
% \end{macro}
%
% \begin{macro}{\storedpar}
%   A simple piece of test set up: the final line of the read file will be
%   tokenized by \TeX{} as \cs{par} which can be tested by \cs{ifx} provided
%   we have an equivalent available.
%    \begin{macrocode}
\def\storedpar{\par}
%    \end{macrocode}
% \end{macro}
% 
% \begin{macro}{\return}
%   A stored |^^M| for string comparisons.
%    \begin{macrocode}
\begingroup
  \catcode`\^^M=12 %
  \gdef\return{^^M}%
\endgroup%
%    \end{macrocode}
% \end{macro}
%
% \begin{macro}{\sourceforhex}
% \begin{macro}{\sethex}
% \begin{macro}{\dohex}
% \begin{macro}{\hexdigit}
%   Some parts of the code here will need to be able to convert integers
%   to their hexadecimal equivalent. That is easiest to do for the requirements
%   here using a modified version of some code from Appendix~D of \emph{The
%   \TeX{}book}.
%    \begin{macrocode}
\newcount\sourceforhex
\def\sethex#1#2{%
  \def#1{}%
  \sourceforhex=#2\relax
  \ifnum\sourceforhex=0 %
    \def#1{0}%
  \else
    \dohex#1%
  \fi
}
\def\dohex#1{%
  \begingroup
    \count0=\sourceforhex
    \divide\sourceforhex by 16 %
    \ifnum\sourceforhex>0 %
      \dohex#1%
    \fi
    \count2=\sourceforhex
    \multiply\count2 by -16 %
    \advance\count0 by\count2
    \hexdigit#1%
  \expandafter\endgroup
  \expandafter\def\expandafter#1\expandafter{#1}%
}
\def\hexdigit#1{%
  \ifnum\count0<10 %
    \edef#1{#1\number\count0}%
  \else
    \advance\count0 by -10 %
    \edef#1{#1\ifcase\count0 A\or B\or C\or D\or E\or F\fi}%
  \fi
}
%    \end{macrocode}
% \end{macro}
% \end{macro}
% \end{macro}
% \end{macro}
%
% \begin{macro}{\unicoderead, \unicodewrite}
%   Set up the streams for data.
%    \begin{macrocode}
\newread\unicoderead
\newwrite\unicodewrite
%    \end{macrocode}
% \end{macro}
%
% \section{Verbatim copying}
%
% \begin{macro}{\verbatimcopy}
% \begin{macro}{\endverbatimcopy}
% \begin{macro}{\verbatimcopy@auxii}
% \begin{macro}{\verbatimcopy@auxii}
% \begin{macro}{\verbatim@endmarker}
%   Set up to read some material verbatim and write it to the output stream.
%   There needs to be a dedicated `clean up first line' macro, but other than
%   that life is simple enough.
%    \begin{macrocode}
\begingroup
  \catcode`\^^M=12 %
  \gdef\verbatimcopy{%
    \begingroup%
      \catcode`\^^M=12 %
      \catcode`\\=12 %
      \catcode`\{=12 %
      \catcode`\}=12 %
      \catcode`\#=12 %
      \catcode`\%=12 %
      \catcode`\ =12 %
      \endlinechar=`\^^M %
      \verbatimcopy@auxi
  }%
  \gdef\verbatimcopy@auxi#1^^M{%
    \expandafter\verbatimcopy@auxii\gobble#1^^M%
  }%
  \gdef\verbatimcopy@auxii#1^^M{%
    \def\temp{#1}%
    \ifx\temp\verbatim@endmarker%
      \expandafter\endgroup%
    \else%
      \ifx\temp\empty\else%
        \immediate\write\unicodewrite{#1}%
      \fi%
      \expandafter\verbatimcopy@auxii%
    \fi%
  }%
\endgroup%
\edef\verbatim@endmarker{\expandafter\gobble\string\\}
\edef\verbatim@endmarker{\verbatim@endmarker endverbatimcopy}
%    \end{macrocode}
% \end{macro}
% \end{macro}
% \end{macro}
% \end{macro}
% \end{macro}
%
% \section{File header section}
%
% With the mechanisms set up, open the data file for writing.
%    \begin{macrocode}
\immediate\openout\unicodewrite=ltunicode.ltx %
%    \end{macrocode}
% There are various lines that now need to go at the start of the file.
% First, there is some header information.
%    \begin{macrocode}
\verbatimcopy
%% This is the file `ltunicode.ltx',
%% generated using the script ltunicode.dtx.
%%
%% The data here are derived from the files
%%  - UnicodeData.txt
%%  - EastAsianWidth.txt
%%  - LineBreak.txt
%% which are maintained by the Unicode Consortium.
%%
%% Copyright 2014-2015
%% The LaTeX3 Project and any individual authors listed elsewhere
%% in this file.
%%
%% This file is part of the LaTeX base system.
%% -------------------------------------------
%%
%% It may be distributed and/or modified under the
%% conditions of the LaTeX Project Public License, either version 1.3c
%% of this license or (at your option) any later version.
%% The latest version of this license is in
%%    http://www.latex-project.org/lppl.txt
%% and version 1.3c or later is part of all distributions of LaTeX
%% version 2005/12/01 or later.
%%
%% This file has the LPPL maintenance status "maintained".
%%
%% The list of all files belonging to the LaTeX base distribution is
%% given in the file `manifest.txt'. See also `legal.txt' for additional
%% information.
\endverbatimcopy
%    \end{macrocode}
% Automatically include the current date.
%    \begin{macrocode}
\immediate\write\unicodewrite{%
  \expandafter\gobble\string\%\expandafter\gobble\string\%
  Generated on \the\year
    -\ifnum\month>9 \else 0\fi \the\month
    -\ifnum\day>9   \else 0\fi \the\day.
}
\immediate\write\unicodewrite{%
  \expandafter\gobble\string\%\expandafter\gobble\string\%
}
%    \end{macrocode}
%
% \section{Unicode character data}
%
% \begin{macro}{\parseunicodedata}
% \begin{macro}{\parseunicodedata@auxi}
% \begin{macro}{\parseunicodedata@auxii}
% \begin{macro}{\parseunicodedata@auxiii}
%   The first step of parsing a line of data is to check that it's not come
%   from a blank in the source, which will have been tokenized as \cs{par}.
%   Assuming that is not the case, there are lots of data items separated by
%   |;|. Of those, only a few are needed so they are picked out and everything
%   else is dropped.
%    \begin{macrocode}
\def\parseunicodedata#1{%
  \ifx#1\storedpar
  \else
    \expandafter\parseunicodedata@auxi#1\relax
  \fi
}
\def\parseunicodedata@auxi#1;#2;#3;#4;#5;#6;#7;#8;#9;{%
  \parseunicodedata@auxii#1;#3;
}
\def\parseunicodedata@auxii#1;#2;#3;#4;#5;#6;#7;#8\relax{%
  \parseunicodedata@auxiii{#1}{#2}{#6}{#7}%
}
%    \end{macrocode}
%   At this stage we have only four pieces of data
%   \begin{enumerate}
%     \item The code value
%     \item The general class
%     \item The uppercase mapping
%     \item The lowercase mapping
%   \end{enumerate}
%   where both one or both of the last two may be empty. Everything here could
%   be done in a single conditional within a \cs{write}, but that would be
%   tricky to follow. Instead, a series of defined auxiliaries are used to
%   show the flow. Notice that combining marks are treated as letters here
%   (the second `letter' test).
%    \begin{macrocode}
\def\parseunicodedata@auxiii#1#2#3#4{%
  \if L\firsttoken#2?\relax
    \expandafter\unicodeletter
  \else
    \if M\firsttoken#2?\relax
      \expandafter\expandafter\expandafter\unicodeletter
    \else
      \expandafter\expandafter\expandafter\unicodenonletter
    \fi
  \fi
    {#1}{#3}{#4}%
}
%    \end{macrocode}
% \end{macro}
% \end{macro}
% \end{macro}
% \end{macro}
%
% \begin{macro}{\unicodeletter, \unicodenonletter}
% \begin{macro}{\writeunicodedata}
%   For `letters', we always want to write the data to file, and the only
%   question here is if the character has case mappings or these point back
%   to the character itself.
%    \begin{macrocode}
\def\unicodeletter#1#2#3{%
  \writeunicodedata\L{#1}{#2}{#3}%
}
%    \end{macrocode}
%   Cased non-letters can also exist: they can be detected as they have at
%   least one case mapping. Write these in much the same way as letters.
%    \begin{macrocode}
\def\unicodenonletter#1#2#3{%
  \ifx\relax#2#3\relax
  \else
    \writeunicodedata\C{#1}{#2}{#3}%
  \fi
}
%    \end{macrocode}
%   Actually write the data. In all cases both upper- and lower-case mappings
%   are given, so there is a need to test that both were actually available and
%   if not set up to do nothing.
%    \begin{macrocode}
\def\writeunicodedata#1#2#3#4{%
  \immediate\write\unicodewrite{%
    \space\space
    \string#1\space
    #2 %
    \ifx\relax#3\relax
      #2 %
    \else
      #3 %
    \fi
    \ifx\relax#4\relax
      #2 %
    \else
      #4 %
    \fi
    \expandafter\gobble\string\%
  }%
}
%    \end{macrocode}
% \end{macro}
% \end{macro}
%
% There is now a lead-in section which creates the macros which take the
% processed data and do the code assignments. Everything is done within a
% group so that there is no need to worry about names.
%    \begin{macrocode}
\verbatimcopy
\begingroup
\endverbatimcopy
%    \end{macrocode}
% Cased non-letters simply need to have the case mappings set.
% For letters, there are a few things to sort out. First, the case mappings are
% defined as for non-letters. Category code is then set to $11$ before a check
% to see if this is an upper case letter. If it is then the \cs{sfcode} is set
% to $999$. Finally there is a need to deal with Unicode math codes, where base
% plane letters are class $7$ but supplementary plane letters are class~$1$.
% Older versions of Xe\TeX{} used a different name here: easy to pick up as
% we know that this primitive must be defined in some way. There is also an issue
% with the supplementary plane and older Xe\TeX{} versions, which is dealt with
% using a check at run time.
%    \begin{macrocode}
\verbatimcopy
  \def\C#1 #2 #3 {%
    \XeTeXcheck{#1}%
    \global\uccode"#1="#2 %
    \global\lccode"#1="#3 %
  }
  \def\L#1 #2 #3 {%
    \C #1 #2 #3 %
    \catcode"#1=11 %
    \ifnum"#1="#3 %
    \else
      \global\sfcode"#1=999 %
    \fi
    \ifnum"#1<"10000 %
      \global\Umathcode"#1="7"01"#1 %
    \else
      \global\Umathcode"#1="0"01"#1 %
    \fi    
  }
  \ifx\Umathcode\undefined
    \let\Umathcode\XeTeXmathcode
  \fi
  \def\XeTeXcheck#1{}
  \ifx\XeTeXversion\undefined
  \else
    \def\XeTeXcheck.#1.#2-#3\relax{#1}
     \ifnum\expandafter\XeTeXcheck\XeTeXrevision.-\relax>996 %
       \def\XeTeXcheck#1{}
     \else
       \def\XeTeXcheck#1{%
          \ifnum"#1>"FFFF %
            \long\def\XeTeXcheck##1\endgroup{\endgroup}
            \expandafter\XeTeXcheck
          \fi
       }
     \fi
  \fi
\endverbatimcopy
%    \end{macrocode}
% Read the data and write the resulting code assignments to the file.
%    \begin{macrocode}
\openin\unicoderead=UnicodeData.txt %
\ifeof\unicoderead
  \errmessage{Data file missing: UnicodeData.txt}%
\fi
\loop\unless\ifeof\unicoderead
  \read\unicoderead to \unicodedataline
  \parseunicodedata\unicodedataline
\repeat
%    \end{macrocode}
% End the group for setting character codes and assign a couple of special
% cases.
%    \begin{macrocode}
\verbatimcopy
\endgroup
\global\sfcode"2019=0 %
\global\sfcode"201D=0 %
\endverbatimcopy
%    \end{macrocode}
% Lua\TeX{} and older versions of Xe\TeX{} stop here: character classes are a
% Xe\TeX{}-only concept.
%    \begin{macrocode}
\verbatimcopy
\ifx\XeTeXchartoks\XeTeXcharclass
  \expandafter\endinput
\fi
\endverbatimcopy
%    \end{macrocode}
%
% \section{Xe\TeX{} Character classes}
% 
% The Xe\TeX{} engine includes the concept of character classes, which allow
% insertion of tokens into the input stream at defined boundaries. Setting
% up this data requires a two-part process as the information is split over
% two input files.
%
% \begin{macro}{\parseunicodedata}
% \begin{macro}{\parseunicodedata@auxi}
% \begin{macro}{\parseunicodedata@auxii}
%   The parsing system is redefined to parse a detokenized input line which
%   may be a comment starting with |#|. Assuming that is not the case, the
%   data line with start with a code point potentially forming part of a range.
%   The range is extracted and the width stored for each code point.
%    \begin{macrocode}
\def\parseunicodedata#1{%
  \ifx#1\return
  \else
    \if\expandafter\gobble\string\#\expandafter\firsttoken#1?\relax
    \else
      \expandafter\parseunicodedata@auxi#1\relax
    \fi
  \fi
}
\def\parseunicodedata@auxi#1;#2 #3\relax{%
  \parseunicodedata@auxii#1....\relax{#2}%
}
\def\parseunicodedata@auxii#1..#2..#3\relax#4{%
  \expandafter\gdef\csname EAW@#1\endcsname{#4}%
  \ifx\relax#2\relax
  \else
    \count@="#1 %
    \begingroup
      \loop
        \ifnum\count@<"#2 %
          \advance\count@\@ne
          \sethex\temp{\count@}%
          \expandafter\gdef\csname EAW@\temp\endcsname{#4}%
      \repeat
    \endgroup
  \fi
}
%    \end{macrocode}
% \end{macro}
% \end{macro}
% \end{macro}
%
% With the right parser in place, read the data file.
%    \begin{macrocode}
\openin\unicoderead=EastAsianWidth.txt %
\ifeof\unicoderead
  \errmessage{Data file missing: EastAsianWidth.txt}%
\fi
\loop\unless\ifeof\unicoderead
  \readline\unicoderead to \unicodedataline
  \parseunicodedata\unicodedataline
\repeat
%    \end{macrocode}
%
% \begin{macro}{\parseunicodedata@auxii}
% \begin{macro}{\parseunicodedata@auxiii}
% \begin{macro}{\parseunicodedata@auxiv}
% \begin{macro}{\ID}
% \begin{macro}{\OP}
% \begin{macro}{\CL}
% \begin{macro}{\EX}
% \begin{macro}{\IS}
% \begin{macro}{\NS}
% \begin{macro}{\CM}
%  The final file to read, |LineBreaking.txt|, uses the same format as
%  |EastAsianWidth.txt|. As such, only the final parts of the parser have to be
%  redefined.
%    \begin{macrocode}
\def\parseunicodedata@auxii#1..#2..#3\relax#4{%
  \parseunicodedata@auxiii{#1}{#4}%
  \ifx\relax#2\relax
  \else
    \count@="#1 %
    \begingroup
      \loop
        \ifnum\count@<"#2 %
          \advance\count@\@ne
          \sethex\temp{\count@}%
          \expandafter\parseunicodedata@auxiii\expandafter{\temp}{#4}%
      \repeat
    \endgroup
  \fi
}
%    \end{macrocode}
%   Adding data to the processed file depends on two factors: the
%   classification in the line-breaking file and (possibly) the width data
%   too. Any characters of class \texttt{ID} (ideograph) are stored: they
%   always need special treatment. For characters of classes \texttt{OP}
%   (opener), \texttt{CL} (closer), \texttt{EX} (exclamation), \texttt{IS}
%   (infix sep) and \texttt{NS} (non-starter) the data is stored if the
%   character is full, half or wide width. The same is true for 
%   \texttt{CM} (combining marks) characters, which need to be transparent
%   to the mechanism.
%    \begin{macrocode}
\def\parseunicodedata@auxiii#1#2{%
  \ifcsname #2\endcsname
    \ifnum\csname #2\endcsname=1 %
      \parseunicodedata@auxiv{#1}{#2}%
    \else
      \ifnum 0%
        \if F\csname EAW@#1\endcsname 1\fi
        \if H\csname EAW@#1\endcsname 1\fi
        \if W\csname EAW@#1\endcsname 1\fi
        >0 %
        \parseunicodedata@auxiv{#1}{#2}%
      \fi
    \fi
  \fi
}
\def\parseunicodedata@auxiv#1#2{%
  \immediate\write\unicodewrite{%
    \space\space
    \expandafter\string\csname #2\endcsname
    \space
    #1 %
    \expandafter\gobble\string\%
  }%
}
\def\ID{1}
\def\OP{2}
\def\CL{3}
\let\EX\CL
\let\IS\CL
\let\NS\CL
\def\CM{256}
%    \end{macrocode}
% \end{macro}
% \end{macro}
% \end{macro}
% \end{macro}
% \end{macro}
% \end{macro}
% \end{macro}
% \end{macro}
% \end{macro}
% \end{macro}
%
% Before actually reading the line breaking data file, the appropriate
% temporary code is added to the output. As described above, only a limited
% number of classes need to be covered: they are hard-coded as classes
% $1$, $2$ and $3$ following the convention adopted by plain Xe\TeX{}.
%    \begin{macrocode}
\verbatimcopy
\begingroup
  \def\ID#1 {\global\XeTeXcharclass"#1=1 \global\catcode"#1=11 }
  \def\OP#1 {\global\XeTeXcharclass"#1=2 }
  \def\CL#1 {\global\XeTeXcharclass"#1=3 }
  \def\EX#1 {\global\XeTeXcharclass"#1=3 }
  \def\IS#1 {\global\XeTeXcharclass"#1=3 }
  \def\NS#1 {\global\XeTeXcharclass"#1=3 }
  \def\CM#1 {\global\XeTeXcharclass"#1=256 }
\endverbatimcopy
%    \end{macrocode}
%
% Read the line breaking data and save to the output.
%    \begin{macrocode}
\openin\unicoderead=LineBreak.txt %
\ifeof\unicoderead
  \errmessage{Data file missing: LineBreak.txt}%
\fi
\loop\unless\ifeof\unicoderead
  \readline\unicoderead to \unicodedataline
  \parseunicodedata\unicodedataline
\repeat
%    \end{macrocode}
%
% Set up material to be inserted between character classes. Other than
% using \cs{hspace} here in place of \cs{hskip} this code is identical to
% that provided by plain Xe\TeX{}.
%    \begin{macrocode}
\verbatimcopy
\endgroup
\gdef\xtxHanGlue{\hspace{0pt plus 0.1em}}
\gdef\xtxHanSpace{\hspace{0.2em plus 0.2em minus 0.1em}}
\global\XeTeXinterchartoks 0 1 = {\xtxHanSpace}
\global\XeTeXinterchartoks 0 2 = {\xtxHanSpace}
\global\XeTeXinterchartoks 0 3 = {\nobreak\xtxHanSpace}
\global\XeTeXinterchartoks 1 0 = {\xtxHanSpace}
\global\XeTeXinterchartoks 2 0 = {\nobreak\xtxHanSpace}
\global\XeTeXinterchartoks 3 0 = {\xtxHanSpace}
\global\XeTeXinterchartoks 1 1 = {\xtxHanGlue}
\global\XeTeXinterchartoks 1 2 = {\xtxHanGlue}
\global\XeTeXinterchartoks 1 3 = {\nobreak\xtxHanGlue}
\global\XeTeXinterchartoks 2 1 = {\nobreak\xtxHanGlue}
\global\XeTeXinterchartoks 2 2 = {\nobreak\xtxHanGlue}
\global\XeTeXinterchartoks 2 3 = {\xtxHanGlue}
\global\XeTeXinterchartoks 3 1 = {\xtxHanGlue}
\global\XeTeXinterchartoks 3 2 = {\xtxHanGlue}
\global\XeTeXinterchartoks 3 3 = {\nobreak\xtxHanGlue}
\endverbatimcopy
%    \end{macrocode}
%
% Done: end the script.
%    \begin{macrocode}
\bye
%    \end{macrocode}
%    
%    \begin{macrocode}
%</script>
%    \end{macrocode}
\fi
%    \end{macrocode}
%
%  This is as good a place as any to active a few Xe\TeX{}-specific
%  settings
%    \begin{macrocode}
\ifx\XeTeXuseglyphmetrics\@undefined
\else
  \XeTeXuseglyphmetrics=1
  \XeTeXdashbreakstate=1
\fi
%    \end{macrocode}
%
% \subsection{Hyphenation}
%
% \changes{v0.1a}{1994/03/07}{move code here from lhyphen.dtx}
% \changes{v0.1a}{1994/03/07}
%         {use \cs{InputIfFileExists} not \cs{IfFileExists}}
% \changes{v1.0x}{1995/11/01}
%      {(DPC) Switch meaning of \cs{@addtofilelist} for cfg files}%
% The following code will be compiled into the format file. It checks
% for the existence of \texttt{hyphen.cfg} in inputs that file if
% found. Otherwise it inputs \texttt{hyphen.ltx}.  Note that these
% are loaded in \emph{before} the |\catcode|s are set, so local
% hyphenation files can use 8-bit input.
%
% We try to load the customized hyphenation description file.
%    \begin{macrocode}
\InputIfFileExists{hyphen.cfg}
           {\typeout{===========================================^^J%
                      Local configuration file hyphen.cfg used^^J%
                     ===========================================}%
             \def\@addtofilelist##1{\xdef\@filelist{\@filelist,##1}}%
           }
           {\InputIfFileExists{UShyphen.tex}%
   {\message{Loading hyphenation patterns for US english.}%
    \language=0
    \lefthyphenmin=2 \righthyphenmin=3 } % disallow x- or -xx breaks
   {\errhelp{The configuration for hyphenation is incorrectly
             installed.^^J%
             If you don't understand this error message you need
             to seek^^Jexpert advice.}%
    \errmessage{OOPS! I can't find any hyphenation patterns for
                US english.^^J \space Think of getting some or the
                latex2e setup will never succeed}\@@end}
\endinput
}
\let\@addtofilelist\@gobble
%    \end{macrocode}
%
%
%
% \subsection{Font loading}
%    Fonts loaded during the formatting process might already have
%    changed the |\font@submax| from |0pt| to something higher.
%    If so, we put out a bold warning.
% \changes{v0.1l}{1994/05/20}{Use new font warning commands}
%    \begin{macrocode}
% \changes{v1.1c}{2000/08/23}{Fix typo in warning}
\ifdim \font@submax >\z@
   \@font@warning{Size substitutions with differences\MessageBreak
                 up to \font@submax\space have occurred.\MessageBreak
                \MessageBreak
                Please check the transcript file
                carefully\MessageBreak
                and redo the format generation if necessary!
                \@gobbletwo}%
   \errhelp{Only stopped, to give you time to
            read the above message.}
   \errmessage{}
%    \end{macrocode}
%    We reset the macro. Otherwise every user will get a warning on
%    every job.
%    \begin{macrocode}
\def\font@submax{0pt}
\fi
%    \end{macrocode}
%
% \subsection{Input encoding}
%
% We temporarily define |\reserved@a| to apply |\reserved@c| to all the
% numbers in the range of its arguments.
%    \begin{macrocode}
\def\reserved@a#1#2{%
   \@tempcnta#1\relax
   \@tempcntb#2\relax
   \reserved@b
}
\def\reserved@b{%
   \ifnum\@tempcnta>\@tempcntb\else
      \reserved@c\@tempcnta
      \advance\@tempcnta\@ne
      \expandafter\reserved@b
   \fi
}
%    \end{macrocode}
% \changes{v0.1e}{1994/05/02}{Added setting the special catcodes.}
% \changes{v0.1f}{1994/05/02}{Set the catcode of control-J.}
% Set the special catcodes (although some of these are useless, since an
% error will have occurred if the catcodes have changed).  Note that
% |^^J| has catcode `other' for use in warning messages.
%    \begin{macrocode}
\catcode`\ =10
\catcode`\#=6
\catcode`\$=3
\catcode`\%=14
\catcode`\&=4
\catcode`\\=0
\catcode`\^=7
\catcode`\_=8
\catcode`\{=1
\catcode`\}=2
\catcode`\~=13
\catcode`\@=11
\catcode`\^^I=10
\catcode`\^^J=12
\catcode`\^^L=13
\catcode`\^^M=5
%    \end{macrocode}
% \changes{v0.1e}{1994/05/02}{Added setting the `other' catcodes.}
% Set the `other' catcodes.
%    \begin{macrocode}
\def\reserved@c#1{\catcode#1=12\relax}
\reserved@c{`\!}
\reserved@c{`\"}
\reserved@a{`\'}{`\?}
\reserved@c{`\[}
\reserved@c{`\]}
\reserved@c{`\`}
\reserved@c{`\|}
%    \end{macrocode}
% \changes{v0.1e}{1994/05/02}{Added setting the `letter' catcodes.}
% Set the `letter' catcodes.
%    \begin{macrocode}
\def\reserved@c#1{\catcode#1=11\relax}
\reserved@a{`\A}{`\Z}
\reserved@a{`\a}{`\z}
%    \end{macrocode}
% \changes{v0.1e}{1994/05/02}{Made slot 127 illegal}
% \changes{v1.0n}{1994/11/18}
%         {re-allow slots 127--255}
% All the characters in the range 0--31 and 127--255 are illegal,
% \emph{except} tab (|^^I|), nl (|^^J|), ff (|^^L|) and cr (|^^M|).
%
% Now allow 8-bit characters, although their use in this way is
% strongly discouraged. See |inputenc.dtx| for a supported mechanism
% for 8-bit input.
%    \begin{macrocode}
\def\reserved@c#1{\catcode#1=15\relax}
\reserved@a{0}{`\^^H}
\reserved@c{`\^^K}
\reserved@a{`\^^N}{31}
%\ifnum\inputlineno=\m@ne
  \catcode"7F=15
%\else
%  \reserved@a{"7F}{"FF}
%\fi
%    \end{macrocode}
%
% \subsection{Lccodes and uccodes}
%
% \changes{v1.1b}{1998/05/20}{Set up uc/lccodes after loading
%    hyphenation files: pr/2639}
%    We now again set up the default (T1) uc/lccodes.
%    The lower case characters need their |\uccode| and |\lccode| values
%    set. Some of this is a repeat of the set-up before loading
%    hyphenation files.
%    Depending on the \TeX{} version, we might not be allowed to do
%    this for non-ASCII characters.
% \changes{v1.0n}{1994/06/09}{For \TeX2, do not set codes for higher
%                   half of character table.}
% \changes{v2.0a}{2015/01/03}{Skip resetting codes with Unicode engines}
%   For the Unicode engines (Xe\TeX{} and Lua\TeX{}) there is no need to
%   do any of this: they use hyphenation data which does not alter any
%   of the set up and so this entire block is skipped.
%    \begin{macrocode}
\ifnum 0%
  \ifx\Umathcode\@undefined\else 1\fi
  \ifx\XeTeXmathcode\@undefined\else 1\fi
  >\z@
\else
\def\reserved@c#1{%
   \count@=#1\advance\count@ by -"20
   \uccode#1=\count@
   \lccode#1=#1
}
\reserved@a{`\a}{`\z}
\ifnum\inputlineno=\m@ne\else
  \reserved@a{"A0}{"BC}
  \reserved@a{"E0}{"FF}
\fi
%    \end{macrocode}
% The upper case characters need their |\uccode| and |\lccode| values
% set, and their |\sfcode| set to 999.
%    \begin{macrocode}
\def\reserved@c#1{%
   \count@=#1\advance\count@ by "20
   \uccode#1=#1
   \lccode#1=\count@
   \sfcode#1=999
}
\reserved@a{`\A}{`\Z}
\ifnum\inputlineno=\m@ne\else
  \reserved@a{"80}{"9C}
  \reserved@a{"C0}{"DF}
\fi
%    \end{macrocode}
% Well, it would be nice if that were correct, but unfortunately, the
% Cork encoding contains some odd slots whose uccode or lccode isn't
% quite what you'd expect.
%    \begin{macrocode}
\uccode`\^^Y=`\I     % dotless i
\lccode`\^^Y=`\^^Y   % dotless i
\uccode`\^^Z=`\J     % dotless j, ae in OT1
\lccode`\^^Z=`\^^Z   % dotless j, ae in OT1
\ifnum\inputlineno=\m@ne\else
  \lccode`\^^9d=`\i    % dotted I
  \uccode`\^^9d=`\^^9d % dotted I
  \lccode`\^^9e=`\^^9e % d-bar
  \uccode`\^^9e=`\^^d0 % d-bar
\fi
%    \end{macrocode}
% Finally here is one that helps hyphenation in the OT1 encoding.
% \changes{v1.0z}{1996/10/31}
%    {Added extra \cs{lcode}, hoping it does no harm in T1 (pr/1969)}
%    \begin{macrocode}
\lccode`\^^[=`\^^[   % oe in OT1
\fi % End of reset block for 8-bit engines
%    \end{macrocode}
%
% \begin{macro}{\MakeUppercase}
% \begin{macro}{\MakeUppercase}
% \begin{macro}{\@uclclist}
%
% \changes{v1.1a}{1997/10/20}{Removed \cs{aa} and \cs{AA} from
%    \cs{@uclclist} as these are macros.}
%
%    And whilst we're doing things with uc/lc tables, here are two
%    commands to upper- and lower-case a string.
%
%    \emph{Note} that this implementation is subject to change!  At
%    the moment we're not providing any way to extend the list of
%    uc/lc commands, since finding a good interface is difficult.
%    These commands have some nasty features, such as uppercasing
%    mathematics, environment names, labels, etc.  A much better
%    long-term solution is to use all-caps fonts, but these aren't
%    generally available.
%    \begin{macrocode}
\DeclareRobustCommand{\MakeUppercase}[1]{{%
      \def\i{I}\def\j{J}%
      \def\reserved@a##1##2{\let##1##2\reserved@a}%
      \expandafter\reserved@a\@uclclist\reserved@b{\reserved@b\@gobble}%
      \protected@edef\reserved@a{\uppercase{#1}}%
      \reserved@a
   }}
\DeclareRobustCommand{\MakeLowercase}[1]{{%
      \def\reserved@a##1##2{\let##2##1\reserved@a}%
      \expandafter\reserved@a\@uclclist\reserved@b{\reserved@b\@gobble}%
      \protected@edef\reserved@a{\lowercase{#1}}%
      \reserved@a
   }}
\def\@uclclist{\oe\OE\o\O\ae\AE
      \dh\DH\dj\DJ\l\L\ng\NG\ss\SS\th\TH}
%    \end{macrocode}
%    The above code works, but has the nasty side-effect that if you
%    say something like:
%\begin{verbatim}
%    \markboth{\MakeUppercase\contentsname}
%             {\MakeUppercase\contentsname}
%\end{verbatim}
%    then the uppercasing is only done to the first letter of the
%    contents name, since the mark expands out to:
%\begin{verbatim}
%    \mark{\protect\MakeUppercase Table of Contents}
%         {\protect\MakeUppercase Table of Contents}
%\end{verbatim}
%    In order to get round this, we redefine |\MakeUppercase| and
%    |\MakeLowercase| to grab their argument and brace it.  This is a
%    very low-level hack, and is \emph{not} recommended practice!
%    This is an instance of a general problem that makes it unsafe to
%    grab arguments unbraced, and probably needs a more general
%    solution.  For the moment though, this hack will do:
%    \begin{macrocode}
\protected@edef\MakeUppercase#1{\MakeUppercase{#1}}
\protected@edef\MakeLowercase#1{\MakeLowercase{#1}}
%    \end{macrocode}
% \end{macro}
% \end{macro}
% \end{macro}
%
% \changes{v1.0h}{1994/05/13}{Added output enc stuff}
% \changes{v1.0i}{1994/05/16}{moved output enc stuff to lfonts}
%
% \changes{v0.1a}{1994/03/07}{Add code from the old dump.dtx}
%
% \subsection{Applying Patch files}
% Between major releases, small patches will be distributed in
% files |ltpatch.ltx| which must be added at this point.
% \changes{v1.0m}{1994/06/08}{Add patch file system}
%    \begin{macrocode}
\IfFileExists{ltpatch.ltx}
  {\typeout{=================================^^J%
             Applying patch file ltpatch.ltx^^J%
            =================================}
   \def\fmtversion@topatch{unknown}
   \input{ltpatch.ltx}
   \ifx\fmtversion\fmtversion@topatch
      \ifx\patch@level\@undefined
        \typeout{^^J^^J^^J%
         !!!!!!!!!!!!!!!!!!!!!!!!!!!!!!!!!!!!!!!!!!!!!!!!!!!!!!^^J%
         !! Patch file `ltpatch.ltx' not suitable for this^^J%
         !! version of LaTeX.^^J^^J%
         !! Please check if initex found an old patch file:^^J%
         !! --- if so, rename it or delete it, and redo the^^J%
         !! initex run.^^J%
         !!!!!!!!!!!!!!!!!!!!!!!!!!!!!!!!!!!!!!!!!!!!!!!!!!!!!!^^J}%
        \batchmode \@@end
      \else
%    \end{macrocode}
% \changes{v1.0q}{1995/04/21}
%         {Allow initial patch level 0}
% \changes{v1.0t}{1995/06/13}
%         {Add patch level string more carefully}
% The code below adds the `patch level' string to the first |\typeout|
% in the startup banner.
%    \begin{macrocode}
        \def\fmtversion@topatch{0}%
        \ifx\fmtversion@topatch\patch@level\else
          \def\reserved@a\typeout##1##2\reserved@a{%
                 \typeout{##1 patch level \patch@level}##2}
          \everyjob\expandafter\expandafter\expandafter{%
             \expandafter\reserved@a\the\everyjob\reserved@a}
          \let\reserved@a\relax
          \the\everyjob
        \fi
      \fi
   \else
      \typeout{^^J^^J^^J%
     !!!!!!!!!!!!!!!!!!!!!!!!!!!!!!!!!!!!!!!!!!!!!!!!!!!!!!^^J%
     !! Patch file `ltpatch.ltx' (for version <\fmtversion@topatch>)^^J%
     !! is not suitable for version <\fmtversion> of LaTeX.^^J^^J%
     !! Please check if initex found an old patch file:^^J%
     !! --- if so, rename it or delete it, and redo the^^J%
     !!     initex run.^^J%
     !!!!!!!!!!!!!!!!!!!!!!!!!!!!!!!!!!!!!!!!!!!!!!!!!!!!!!^^J}%
       \batchmode \@@end
   \fi
   \let\fmtversion@topatch\relax
  }{}
%    \end{macrocode}
%
% \subsection{Freeing Memory}
%
% \begin{macro}{\reserved@a}
% \begin{macro}{\reserved@b}
% \changes{v1.0v}{1995/10/17}{reset here after the \cs{input} above}
% And just to make sure nobody relies on those definitions of
% |\reserved@b| and friends.
% These macros are reserved for use in the kernel. \emph{Do not use
% them as general scratch macros}.
%    \begin{macrocode}
\let\reserved@a\@filelist
\let\reserved@b=\@undefined
\let\reserved@c=\@undefined
\let\reserved@d=\@undefined
\let\reserved@e=\@undefined
\let\reserved@f=\@undefined
%    \end{macrocode}
% \end{macro}
% \end{macro}
%
% \begin{macro}{\toks}
% \changes{v1.0y}{1996/07/10}
%      {Free up memory from scratch registers /2213}
%    \begin{macrocode}
\toks0{}
\toks2{}
\toks4{}
\toks6{}
\toks8{}
%    \end{macrocode}
% \end{macro}
%
% \begin{macro}{\errhelp}
% \changes{v0.1g}{1994/05/05}{Set error help empty.}
% \changes{v1.1d}{2000/09/01}{Set error help empty at very end
%                             (pr/449 done correctly).}
% Empty the error help message, which may have some rubbish:
%    \begin{macrocode}
\errhelp{}
%    \end{macrocode}
% \end{macro}
%
% \subsection{Initialise file list}
%
% \begin{macro}{\@providesfile}
% \changes{v1.0v}{1995/10/17}{reset macro}
% Initialise for use in the document. During initex a modified version
% has been used which leaves debugging information for |latexbug.tex|.
%    \begin{macrocode}
\def\@providesfile#1[#2]{%
    \wlog{File: #1 #2}%
    \expandafter\xdef\csname ver@#1\endcsname{#2}%
  \endgroup}
%    \end{macrocode}
% \end{macro}
%
% \begin{macro}{\@filelist}
% \changes{v1.0w}{1995/10/19}{Move after \cs{reserved@a} setting:-)}
% \begin{macro}{\@addtofilelist}
% Reset |\@filelist| so files input while making the format are not
% listed. The list built up so far may take up a lot of memory and so
% it is moved to |\reserved@a| where it will be overwritten as soon
% as almost any \LaTeX\ command is issued in a class file.
% However the |latexbug.tex| program will be able to access this
% information and insert it into a bug report.
%    \begin{macrocode}
\let\@filelist\@gobble
\def\@addtofilelist#1{\xdef\@filelist{\@filelist,#1}}%
%    \end{macrocode}
% \end{macro}
% \end{macro}
%
% \subsection{Dumping the format}
%    Finally we make |@| into a letter, ensure the format will
% be in the `normal' error mode, and dump everything into the
%    format file.
% \changes{v1.0t}{1995/06/13}
%         {Call \cs{errorstopmode}}
%    \begin{macrocode}
\makeatother
\errorstopmode
\dump
%</2ekernel>
%    \end{macrocode}
%
% \Finale
%

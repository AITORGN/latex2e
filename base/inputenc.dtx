% \iffalse meta-comment
%
% Copyright 1993-2012
% LaTeX3 Project and any individual authors listed elsewhere
% in this file. 
% 
% This file is part of the LaTeX base system.
% -------------------------------------------
% 
% It may be distributed and/or modified under the
% conditions of the LaTeX Project Public License, either version 1.3c
% of this license or (at your option) any later version.
% The latest version of this license is in
%    http://www.latex-project.org/lppl.txt
% and version 1.3c or later is part of all distributions of LaTeX 
% version 2005/12/01 or later.
% 
% This file has the LPPL maintenance status "maintained".
% 
% The list of all files belonging to the LaTeX base distribution is
% given in the file `manifest.txt'. See also `legal.txt' for additional
% information.
% 
% The list of derived (unpacked) files belonging to the distribution 
% and covered by LPPL is defined by the unpacking scripts (with 
% extension .ins) which are part of the distribution.
% 
% \fi
%
% \iffalse
%<*driver>
\documentclass{ltxdoc}
\usepackage[ascii]{inputenc}
\GetFileInfo{inputenc.sty}
\title{\filename}
\date{\fileversion\space\filedate}
 \author{%
  Alan Jeffrey\and
  Frank Mittelbach}

\begin{document}
\maketitle
 \setlength\hfuzz{20pt}
 \DocInput{inputenc.dtx}
\end{document}
%</driver>
% \fi
%
% \CheckSum{3475}
%
%% \CharacterTable
%%  {Upper-case    \A\B\C\D\E\F\G\H\I\J\K\L\M\N\O\P\Q\R\S\T\U\V\W\X\Y\Z
%%   Lower-case    \a\b\c\d\e\f\g\h\i\j\k\l\m\n\o\p\q\r\s\t\u\v\w\x\y\z
%%   Digits        \0\1\2\3\4\5\6\7\8\9
%%   Exclamation   \!     Double quote  \"     Hash (number) \#
%%   Dollar        \$     Percent       \%     Ampersand     \&
%%   Acute accent  \'     Left paren    \(     Right paren   \)
%%   Asterisk      \*     Plus          \+     Comma         \,
%%   Minus         \-     Point         \.     Solidus       \/
%%   Colon         \:     Semicolon     \;     Less than     \<
%%   Equals        \=     Greater than  \>     Question mark \?
%%   Commercial at \@     Left bracket  \[     Backslash     \\
%%   Right bracket \]     Circumflex    \^     Underscore    \_
%%   Grave accent  \`     Left brace    \{     Vertical bar  \|
%%   Right brace   \}     Tilde         \~}
%
% \changes{v0.01}{1994/03/09}{Created file.}
% \changes{v0.02}{1994/07/14}{Replaced \cs{Dh} by \cs{DH} and \cs{Th}
%    by \cs{TH}.}
% \changes{v0.02}{1994/07/14}{Added \cs{ensuremath} to some math
%    commands.}
% \changes{v0.02}{1994/07/14}{Added \cs{inputencoding}.}
% \changes{v0.03}{1994/09/04}{Added \cs{DeclareInputComposite} and the
%    \cs{ProvidesCommand}s to the encoding files.}
% \changes{v0.03}{1994/09/04}{Removed the definition of the accent slots
%     in Latin-1.}
% \changes{v0.04}{1994/10/20}{Replaced \cs{DeclareInputCharacter} by
%     \cs{DeclareInputText} and \cs{DeclareInputMath}.}
% \changes{v0.04}{1994/10/20}{Removed \cs{DeclareInputComposite}.}
% \changes{v0.04}{1994/10/20}{Made many Latin-1 characters math-only.}
% \changes{v0.05}{1994/10/27}{Updated for the new version of ltoutenc.}
% \changes{v0.06}{1994/11/21}{Added \cs{textregistered}.}
% \changes{v0.06}{1994/11/21}{Added slot hex A0 to Latin-1.}
% \changes{v0.07}{1994/11/22}{Fixed typo A1 rather than A0.}
% \changes{v0.07}{1994/11/28}{Fixed where docstrip option had moved a
%    line.}
% \changes{v0.09}{1994/12/10}{Added `beta test' message.}
% \changes{v0.09}{1994/12/10}{Made letters active and undefined by
%    default, rather than illegal.}
%
% \changes{v0.9b}{1995/05/23}{Added Mac encoding, applemac.def}
% \changes{v0.9d}{1995/06/06}{Added cp437}
% \changes{v0.9f}{1995/09/21}{Added Next encoding, next.def}
% \changes{v0.9h}{1995/10/22}{Added Windows 3.1 ANSI encoding,
%                             ansinew.def}
% \changes{v0.9i}{1995/11/02}{Wrapped long lines}
% \changes{v0.9i}{1995/11/02}{Changed internal name \cs{a} to
% \cs{@tabacckludge} to protect against redefinition by malicious
% users.}
% \changes{v0.9j}{1995/11/14}{Remove \cs{endinput} so docstrip reaches
%    ansinew encoding}
% \changes{v0.9k}{1995/11/29}{Replaced uses of \cs{textsterling} with
%    \cs{pounds}.} 
% \changes{v0.9m}{1995/12/04}{Added German version}
% \changes{v0.9m}{1995/12/04}{Replaced \cs{textasciitilde} by
%    \cs{nobreakspace}.}
% \changes{v0.9m}{1995/12/04}{Made bullet and periodcentered text
%    glyphs rather than math glyphs.}
% \changes{v0.9m}{1995/12/04}{Added \cs{@tabacckludge} hacks.}
% \changes{v0.9o}{1996/02/14}
%      {Cedilla (\cs{c}\cs{ }) rather than (\cs{c}\{\}) latex/2077,
%      finished on 1996/10/28}
% \changes{v0.9o}{1996/02/14}
%      {0F0 corrected in cp850 latex/2080}
% \changes{v0.9o}{1996/02/14}
%      {0B2 corrected in latin2 latex/2079}
% \changes{v0.9r}{1996/05/10}{Added cp852.def}
% \changes{v0.9t}{1996/10/28}{Added cp865.def}
% \changes{v0.9t}{1996/10/28}
%      {Changed \cs{aa} and \cs{AA} to \cs{r} a and \cs{r} A}
% \changes{v0.9u}{1996/10/29}{Added more to cp865.def}
% \changes{v0.9y}{1997/04/30}{Added latin5.def (provided by
%       H. Turgut Uyar: uyar@cs.itu.edu.tr)}
% \changes{v0.9z}{1997/05/10}{Added latin3.def (provided by
%       J\"org Knappen and modified by Chris Rowley)}
% \changes{v0.91}{1997/08/19}{Put
%       \cs{makeatletter}\ldots\cs{makeatother} around all .def files.}
% \changes{v0.92}{1997/09/08}{Added decmulti.def
%       provided by M.Y. Chartoire. pr/2599}
% \changes{v0.93}{1997/11/23}{\cs{textperthousand} not
%       \cs{textpermill}. pr/2673}
% \changes{v0.94}{1997/12/17}{Made degree a text glyph}
% \changes{v0.94}{1997/12/17}{Added to and tidied documentation}
% \changes{v0.94}{1997/12/17}{Ogonek: changed
%       \cs{k}\{\} to \cs{k}\cs{ }}
% \changes{v0.94}{1997/12/17}{NOTE: for consistency, when available
%       the robust text-or-math internal form is always used --
%       THIS MAY CHANGE}
% \changes{v0.95}{1997/12/20}{Updated documentation}
% \changes{v0.97}{1998/03/05}{Spanish ords changed to text chars, pr/2579}
% \changes{v1.1b}{2006/03/04}{Number of normalisations in the LICR 
%                             representation (pr/3849)}
% \changes{v1.1c}{2006/11/18}{Added missing \cs{ProvidesFile} line for cp1257 (pr/3892)}
%
%
% \section{Introduction}
%
% This package allows the user to specify an input encoding (for
% example, ASCII, ISO Latin-1 or Macintosh) by saying:
% \begin{quote}
%    |\usepackage[|\emph{encoding name}|]{inputenc}|
% \end{quote}
% The encoding can also be selected in the document with:
% \begin{quote}
%    |\inputencoding{|\emph{encoding name}|}|
% \end{quote}
%    Originally this command was only to be used in vertical mode (with
%    the idea that it should be only  within a document when
%    using text from several documents to build up a composite work such
%    as a volume of journal articles.  However, usages in certain
%    languages suggested that it might be preferable to allow changing
%    the input encoding at any time, which is what is possible now
%    (though that is quite computing resource intensive).
% 
% The encodings provided by this package are:
% \begin{itemize}
% \item |ascii| ASCII encoding for the range 32--127 (all others are made
%        invalid, i.e., this really defines a 7-bit encoding).
% \item |latin1| ISO Latin-1 encoding.
% \item |latin2| ISO Latin-2 encoding.
% \item |latin3| ISO Latin-3 encoding.
% \item |latin4| ISO Latin-4 encoding.
% \item |latin5| ISO Latin-5 encoding.
% \item |latin9| ISO Latin-9 encoding.
% \item |latin10| ISO Latin-10 encoding.
% \item |decmulti| DEC Multinational Character Set encoding.
% \item |cp850| IBM 850 code page.
% \item |cp852| IBM 852 code page.
% \item |cp858| IBM 858 code page (this is 850 with Euro symbol).
% \item |cp437| IBM 437 code page.
% \item |cp437de| IBM 437 code page (German version).
% \item |cp865| IBM 865 code page.
% \item |applemac| Macintosh encoding.
% \item |macce| Macintosh Central European code page.
% \item |next| Next encoding.
% \item |cp1250| Windows 1250 (central and eastern Europe) code page.
% \item |cp1252| Windows 1252 (Western Europe) code page.
% \item |cp1257| Windows 1257 (Baltic) code page.
% \item |ansinew| Windows 3.1 ANSI encoding, extension of Latin-1
%       (synonym\footnote{It is now generated using the guards
%       \texttt{cp1252,ansinew} the latter only used for the provides
%       file line.} for |cp1252|).
% \item |utf8| Unicode UTF-8 encoding support.
% \end{itemize}
%
%
%
% \subsection{8-bit input encoding support}
%
% The \texttt{inputenc} package makes the upper 8-bit characters active and
%    assigns to all of them an error message. It then waits for the
%    input encoding definitions to change this set-up.  Similarly, whenever
%    |\inputencoding| is encountered in a document, first the upper
%    8-bit characters are set back to produce an error and then the
%    definitions for the new input encoding are loaded, changing some of the 
%    previous settings.
%
%
%
% Each encoding has an associated |.def| file, for example
% |latin1.def| which defines the behaviour of each input character,
% using the commands:
% \begin{quote}
%    |\DeclareInputText{|\emph{slot}|}{|\emph{text}|}| \\
%    |\DeclareInputMath{|\emph{slot}|}{|\emph{math}|}|
% \end{quote}
% This defines the input character \emph{slot} to be the
% \emph{text} material or \emph{math} material respectively.
% For example, |latin1.def| defines slots |"D6| (\AE)
% and |"B5| ($\mu$) by saying:
%\begin{verbatim}
%    \DeclareInputText{214}{\AE}
%    \DeclareInputMath{181}{\mu}
%\end{verbatim}
% Note that the \emph{commands} should be robust, and should not be
% dependent on the output encoding.  The same \emph{slot} should not
% have both a text and a math declaration for it. (This restriction
% may be removed in future releases of inputenc).
%
% The |.def| file may also define
% commands using the declarations:\\
% |\providecommand| or |\ProvideTextCommandDefault|.
% For example:
%\begin{verbatim}
%    \ProvideTextCommandDefault{\textonequarter}{\ensuremath{\frac14}}
%    \DeclareInputText{188}{\textonequarter}
%\end{verbatim}
%    The use of the `provide' forms here will ensure that a
%    better definition will not be over-written; their use is
%    recommended since, in general, the best definition depends on the
%    fonts available.
%    
%    See the documentation in |fntguide.tex| and |ltoutenc.dtx| for
%    details of how to declare text commands.
%
%
% \subsection{UTF-8 encoding support}
%
%
% The Unicode UTF-8 support works differently. It too uses a |.def| file
% (i.e., |utf8.def|) but this file does not contain code point declarations
% via |\DeclareInputText| or |\DeclareInputMath|. Instead it defines a number of
% parsing commands that parse UTF-8 characters and then provides the
% corresponding \LaTeX{} definitions (if possible).
%
% Unfortunately the number of Unicode characters that in theory could be
% contained in a document is enormous. Thus even with today's amount of
% computer memory it would be unrealistic to predefine all of them. Therefore the
% approach taken by \LaTeX{} is as follows:
% \begin{itemize}
% \item
%   At the start of the document (|\begin{document}|) it examines all font
%   encodings that are being used within the current document.
% \item
%   For each such font encoding it loads all known UTF-8 mappings that generate
%   characters from this font encoding.
% \item
%   All other UTF-8 characters remain undefined and will produce an error
%   message if they appear in the document.
% \end{itemize}
%
% The rationale behind this approach is that UTF-8 characters that do not
% correspond to any glyph within the used font encodings cannot be represented
% by \LaTeX{} anyway (without loading a font containing the glyph, which in
% turn should ``hopefully'' set up the corresponding UTF-8 mapping).
%
% This works well enough for the main Western languages for which \LaTeX{}
% has proper font encoding support, but currently already falls short on
% languages like Greek (which has some semi-official font support, but for
% which corresponding UTF-8 mappings still need to be defined).
%
% For some languages (such as Greek mentioned above) all that remains doing is
% to provide the necessary mappings and stick them into |utf8ienc.dtx|, so
% volunteers are welcome. For other languages that do not fit well into
% \LaTeX{} font selection scheme, e.g., Asian languages the outlined inputenc
% approach will not work. If that is the case one can try using Dominique
% Unruh's option |utf8x| for inputenc which has a somewhat different approach
% and encodes many more UTF-8 characters than the standard |utf8| option.
% However, we recommend to do so only if you really need such alphabets as
% there are problems with this extended approach which were precisely the
% reason that we decided to limit the support to what is properly supported
% within the boundaries of \LaTeX's font selection.
%
% If a UTF-8 mapping is missing and it is known to what \LaTeX{} definition it
% should map to, one can manually define it using a |\DeclareUnicodeCharacter|
% declaration. This declaration is available after inputenc has been loaded
% with the |utf8| option.
%
% The |\DeclareUnicodeCharacter| takes UTF-8 code point as its first argument
% (in form of a a hexadecimal number) and the definition that this maps to as
% its second argument. For example, the code point |00E4| which is
% ``LATIN SMALL LETTER A WITH DIAERESIS'' would be set up via:
%\begin{verbatim}
%   \DeclareUnicodeCharacter{00E4}{\"a}
%\end{verbatim}
% Conceptually the second argument should only contain ``encoding-specific
% commands'' as defined by \LaTeX{} font encoding concept, i.e., commands that
% automatically change behavior if the font encoding changes (see chapter~7 of
% the \LaTeX{} Companion for details).
%
% For details of the mappings per font encoding and some more technical
% information see the file \texttt{utf8ienc.dtx} that provides UTF-8 support
% using the \textsf{inputenc} package interface.
%
%
%
%
% \subsection{Error messages}
%
% In certain situations the inputenc package generates one of the following
% three error messages.
%
% \subsubsection{\normalfont\ttfamily Keyboard character used is undefined  in
% inputencoding `\meta{name}'}

% The document contains an 8-bit character that is not defined by the
% current input encoding in force. This means that either there is a
% mismatch between the document encoding that the document claims it
% is in (the option to inputenc) and the real encoding this document
% is encoded in. These days more often you find that UTF-8 is used as
% the encoding when saving a file in some text editor.
%
% Of course, it is also possible that the input encoding |.def| file is
% defective and the offending code point is simply missing from that file.
% Please check if the encoding file is one of the list above prior to
% reporting an error---on the net there are many additional encoding files
% supported by third parties.
%
%
%
% \subsubsection{\normalfont\ttfamily Cannot define Unicode char value < 00A0}
%
% This error message is shown if one tries to define a UTF-8 character
% with a code point lower than |00A0|. Those cannot be defined in
% \LaTeX{} through the |\DeclareUnicodeCharacter|.
%
%
% \subsubsection{\normalfont\ttfamily Unicode char \meta{charcode} not set up
% for use with LaTeX}
%
% This is the dreaded error message that one will receive if the
% document contains an UTF-8 character that isn't known to \LaTeX{}.
% It is quite possible that the character looks very unsuspicious and
% is rendered perfectly in the editor.
%
% For example, when entering a Euro symbol from the keyboard one may receive
% this error rather than a typeset symbol. But if this happens the reason is
% simply that the document doesn't load a font containing the Euro symbol,
% e.g., via the |textcomp| package. Thus \LaTeX{} does not know how to typeset
% one and therefore responds with this error message.
%
% However, even if \LaTeX{} can type that character in question it may not
% have been set up in which case you would need to do that yourself via
% |\DeclareUnicodeCharacter|. If you provide these declarations for a full
% font encoding then please contribute that work to this package so that
% others can benefit too.
%
% 
%
%

% \subsection{Programmers interface}
%
% To better support packages that manage their own character mappings and
% therefore have to react to input encoding changes, the following three
% commands have been added in version 1.1a:
%
% \DescribeMacro\inputencodingname This command stores the name of the current
% input encoding.
%
% \DescribeMacro\inpenc@prehook
% \DescribeMacro\inpenc@posthook These two are token registers that are
% executed whenever an |\inputencoding| change happens. The first is executed at
% the very beginning, i.e., with |\inputencodingname| still pointing to the
% encoding name currently in place while the second one is executed at the very
% end, i.e., when |\inputencoding| has build a new mapping.
%
% Packages making use of this new features should consider including the
% following line
%\begin{verbatim}
%   \NeedsTeXFormat{LaTeX2e}[2005/12/01]
%\end{verbatim}
% as these commands haven't been available in \textsf{inputenc} distributed
% with older releases of \LaTeX{}.
%

% \StopEventually{}
%
% \section{Announcing the files}
%
% We announce the files:
%    \begin{macrocode}
%<package>\NeedsTeXFormat{LaTeX2e}[1995/12/01]
%<package>\ProvidesPackage{inputenc}
%<ascii> \ProvidesFile{ascii.def}
%<latin1> \ProvidesFile{latin1.def}
%<latin2> \ProvidesFile{latin2.def}
%<latin3> \ProvidesFile{latin3.def}
%<latin4> \ProvidesFile{latin4.def}
%<latin5> \ProvidesFile{latin5.def}
%<latin9> \ProvidesFile{latin9.def}
%<latin10> \ProvidesFile{latin10.def}
%<decmulti> \ProvidesFile{decmulti.def}
%<cp850>  \ProvidesFile{cp850.def}
%<cp852>  \ProvidesFile{cp852.def}
%<cp858>  \ProvidesFile{cp858.def}
%<cp437>  \ProvidesFile{cp437.def}
%<cp437de>  \ProvidesFile{cp437de.def}
%<cp865>  \ProvidesFile{cp865.def} 
%<applemac>  \ProvidesFile{applemac.def}
%<applemacce>  \ProvidesFile{macce.def}
%<next>  \ProvidesFile{next.def}
%<ansinew>  \ProvidesFile{ansinew.def}
%<cp1252&!ansinew>  \ProvidesFile{cp1252.def}
%<cp1250>  \ProvidesFile{cp1250.def}
%<cp1257>  \ProvidesFile{cp1257.def}
   [2012/06/06 v1.1e Input encoding file]
%<cp850>%%
%<cp850>%% If you need a Euro symbol, try cp858 instead.
%<cp850>%%
%    \end{macrocode}
%
%
% \section{The package}
%
% \changes{v0.99c}{2002/11/11}{Added cp858 (pr/3464)}
%
% \changes{v0.04}{1994/10/20}{Improved coding of \cs{DeclareInputText}
%    and changed name from \cs{DeclareInputCharacter}.}
%
% \changes{v0.9g}{1995/10/19}{Replaced \cs{'} \cs{`} \cs{!=} by
%    \cs{a'} \cs{a`} \cs{a!=} in order to get correct accents in
%    a tabbing environment.}
%
% \changes{v0.9w}{1996/11/23}{Correct documentation
%    of \cs{@tabacckludge}}
%
% \changes{v0.9x}{1997/03/21}{Use decimal rather than hex
%  to avoid active character problems. latex/2451.}
%
% Before we start with the code, an important comment is in order:
% as you may or may not know, the |tabbing| environment changes the
% definition of the commands |\'|, |\`|, and |\=|. Outside such an
% environment these commands produce the corresponding accents, inside
% they are used for special text positioning, and the accents can be
% accessed using |\a'|, |\a`|, and |\a=|. Therefore we \emph{must} use
% the latter instead of the former in the second argument to
% |\DeclareInputText|, e.g. (from |latin1.def|):
% \begin{verbatim}
%     \DeclareInputText{224}{\@tabacckludge`a}
%\end{verbatim}
% The command |\@tabacckludge| is defined (in |ltoutenc.dtx|) in such
% a way that |\@tabacckludge'| will expand to the internal form of |\'|.
% Thus it is |\'| that is carried around \emph{internally} (the
% same applies to the other two accent commands).
%
% \begin{macro}{\DeclareInputText}
% \begin{macro}{\DeclareInputMath}
% \begin{macro}{\IeC}
%    These commands declare the expansion of an active character.  The
%    math declaration is the usual trick with |\uppercase|.
%    The text declaration is sneakier, since in text space matters.
%    We look to see if the definition ends in a macro, by checking
%    whether it's |\meaning| ends in a space.  If it does, then we
%    add an irrelevant |\IeC| and braces around the definition, in
%    order to avoid any space after the active char being gobbled up
%    once the text is written out to an auxiliary file.
%
%    The definition should contain only robust commands (and, for
%    correct ligatures and kerning, they must be defined via the
%    interfaces in the fontenc package).
%    
% \changes{v0.9b}{1995/05/23}{Added hackery with \cs{IeC} in order to
%       avoid space being gobbled.}
%
%    \begin{macrocode}
%<*package>
\def\DeclareInputMath#1{%
   \@inpenc@test
   \bgroup
      \uccode`\~#1%
      \uppercase{%
   \egroup
      \def~%
   }%
}
%    \end{macrocode}
%    \changes{v1.0a}{2003/01/17}{Now coding according to suggestion by David (pr/2004)}
%    \changes{v1.0b}{2003/12/29}{but better do it properly}
%    \changes{v1.0?}{2004/01/19}{or even correctly}
%    \begin{macrocode}
\def\DeclareInputText#1#2{%
   \def\reserved@a##1 ${}%
   \def\reserved@b{#2}%
   \ifcat_\expandafter\reserved@a\meaning\reserved@b$ $_%
      \DeclareInputMath{#1}{#2}%
   \else
      \DeclareInputMath{#1}{\IeC{#2}}%
   \fi
}
%    \end{macrocode}
%    The definition of |\IeC| was modified not to insert a |\protect|
%    unless it is needed, this means it works in |\hyphenation|
%    commands, and other such delicate places.  It was then further
%    changed to never insert a |\protect| as one is never needed; this 
%    makes it work in even more places.
% 
%    This still needs some attention.
%    
% \changes{v0.9m}{1995/12/12}{Modified \cs{IeC} in order to
%       work in \cs{hyphenation} latex/2004.}
%    
% \changes{v0.94}{1997/12/17}{Changed non-typeset case from
%       \cs{protect} to \cs{noexpand}: temporary fix.}
%       
%    \begin{macrocode}
\def\IeC{%
  \ifx\protect\@typeset@protect
    \expandafter\@firstofone
  \else
    \noexpand\IeC
  \fi
}
%    \end{macrocode}
% \end{macro}
% \end{macro}
% \end{macro}
%
% \begin{macro}{\inputencoding}
% \changes{v0.9q}{1996/05/09}
%      {Allow characters below 32 for latex/2071}
% \changes{v0.9q}{1996/05/09}
%      {Check the def file was reasonable for latex/2136}
% \changes{v0.94}{1997/12/17}{Changed to work only in outer vmode, see
%       latex/2608}
% \changes{v0.94}{1997/12/17}{Warning message reworded and line
%       number added, also for latex/2608}
% \changes{v0.95}{1997/12/20}{Changed to work in any vmode, for David}
%
%    This sets the encoding to be |#1|.  It first sets all the
%    characters 128--255 to be active (and sets their initial
%    definition to be |\@inpenc@undefined|).
%    It now also does this for some `low' codes below 32, but
%    misses out Null, control-I, control-J, control-L and control-M.
%    
%    It then inputs |#1.def|.  But it first sets up a test that
%    produces a warning message if no suitable definitions get read.
% 
%    \begin{macrocode}
\def\inputencoding#1{%
%    \end{macrocode}
%    We start with a hook to be executed before the encoding change
%    happens.
% \changes{v1.1a}{2006/02/22}{Added \cs{inpenc@prehook}}
%    \begin{macrocode}
  \the\inpenc@prehook
  \gdef\@inpenc@test{\global\let\@inpenc@test\relax}%
%    \end{macrocode}
%    Keyboard characters which don't get a definition will be mapped to
%    |\@inpenc@undefined| which gets a definition producing an error 
%    message indicating in which input encoding the current keyboard
%    character is undefined:
% \changes{v0.98}{1998/07/04}{Give better error message if key used
%    is undefined (pr/2845)}
% \changes{v0.993}{2000/01/24}{Fix error message for undefined chars
%    (pr/3158)}
%    \begin{macrocode}
  \edef\@inpenc@undefined{\noexpand\@inpenc@undefined@{#1}}%
%    \end{macrocode}
%    The |\edef| in the above definition is essential as |#1| may be
%    |\CurrentOption| in which case a later use would return incorrect
%    information (at best nothing).
%
%    For external lookup by other packages we also store the new encoding name
%    in a user accessible macro.
% \changes{v1.1a}{2006/02/22}{Added \cs{inputencodingname}}
%    \begin{macrocode}
  \edef\inputencodingname{#1}% 
%    \end{macrocode}
%
%    Now we make all potential input characters active.
% \changes{v0.994}{2000/01/27}{Allow change also in horizontal mode
%      (pr/2888)}
%    \begin{macrocode}
  \@inpenc@loop\^^A\^^H%
  \@inpenc@loop\^^K\^^K%
  \@inpenc@loop\^^N\^^_%
  \@inpenc@loop\^^?\^^ff%
%    \end{macrocode}
%
%    To be able to process the input encoding file in horizontal mode
%    we need to ensure that we don't get any stray spaces into the
%    horizontal mode or else we end up with extra space in the
%    paragraph.
% \changes{v0.998}{2001/05/25}{Suppress all spaces for horizontal mode 
%                              (pr/3273)}
% \changes{v1.0f}{2004/05/06}{Really do (pr/3273)}
% \changes{v1.1d}{2007/08/06}{Set \cs{endlinechar} properly (pr/3926)}
% \changes{ v1.1e}{2012/06/06}{Save and restore \cs{catcode} of @ (pr/...)}
%    \begin{macrocode}
  \xdef\saved@endlinechar@code{\the\endlinechar}%
  \endlinechar\m@ne
  \xdef\saved@space@catcode{\the\catcode`\ }%
  \catcode`\ 9\relax                      
  \xdef\saved@at@catcode{\the\catcode`\@}%
  \makeatletter
  \input{#1.def}%
  \endlinechar\saved@endlinechar@code\relax
  \catcode`\ \saved@space@catcode\relax
  \catcode`\@\saved@at@catcode\relax
%    \end{macrocode}
%
% If there have been no |\DeclareInputText| or |\DeclareInputMath|
% commands read then something is amiss.
%    \begin{macrocode}
  \ifx\@inpenc@test\relax\else
    \PackageWarning{inputenc}%
             {No characters defined\MessageBreak
              by input encoding change to `#1'\MessageBreak}%
  \fi
%    \end{macrocode}
%    We finish with a hook to be executed after the encoding change
%    happens.
% \changes{v1.1a}{2006/02/22}{Added \cs{inpenc@posthook}}
%    \begin{macrocode}
  \the\inpenc@posthook
}  
%    \end{macrocode}
% \end{macro}
%
% 
% \begin{macro}{\inpenc@prehook}
% \changes{v1.1a}{2006/02/22}{Hook added}
% \begin{macro}{\inpenc@posthook}
% \changes{v1.1a}{2006/02/22}{Hook added}
%    Two hooks to be executed before and after an encoding changes happened.
%    \begin{macrocode}
\newtoks\inpenc@prehook
\newtoks\inpenc@posthook
%    \end{macrocode}
% \end{macro}
% \end{macro}
%
% 
% \begin{macro}{\@inpenc@undefined@}
%    This command will assigned to any active character unless it
%    get a proper definition by the encoding. The argument is the
%    current encoding name.
% \changes{v0.993}{2000/01/24}{Fix error message for undefined chars
%    (pr/3158)}
%    \begin{macrocode}
\def\@inpenc@undefined@#1{\PackageError{inputenc}%
        {Keyboard character used is undefined\MessageBreak
         in inputencoding `#1'}%
       {You need to provide a  definition with
        \noexpand\DeclareInputText\MessageBreak or
        \noexpand\DeclareInputMath before using this key.}}%
%    \end{macrocode}
% \end{macro}
%
% \begin{macro}{\@inpenc@loop}
% \changes{v0.9q}{1996/05/09}
%      {Macro added}
% \changes{v0.9v}{1996/11/07}
%      {Exit from the top of loop fixed for latex/2257}
% Make characters |#1| to |#2| inclusive active and undefined.
%    \begin{macrocode}
\def\@inpenc@loop#1#2{%
  \@tempcnta`#1\relax
  \loop
    \catcode\@tempcnta\active
    \bgroup
      \uccode`\~\@tempcnta
      \uppercase{%
    \egroup
         \let~\@inpenc@undefined
      }%
  \ifnum\@tempcnta<`#2\relax
     \advance\@tempcnta\@ne
  \repeat}
%    \end{macrocode}
% \end{macro}
%    
%
%    Then for each option, we input that encoding file.
%    \begin{macrocode}
\DeclareOption*{\inputencoding{\CurrentOption}}
\ProcessOptions
%</package>
%    \end{macrocode}
%
%
% \section{Default definitions for characters}
%
%    Some input characters map to internal functions which are not in
%    either the |T1| or |OT1| font encoding. For this reason default
%    definitions are provided in the encoding file: these will be
%    used unless some other output encoding is used which supports
%    those glyphs.  In some cases this default definition has to be
%    simply an error message.
%
%    Note that this works reasonably well only because the encoding
%    files for both |OT1| and |T1| are loaded in the standard LaTeX
%    format. 
%
% \changes{v0.9a}{1995/04/23}{Default settings moved to own section}
%
% \changes{v0.9b}{1995/05/23}{Corrected ordmasc and ordfem which had
%    been switched by mistake.}
%
% \changes{v0.9k}{1995/11/29}{Moved \cs{textregistered} and
%    \cs{texttrademark} to the kernel.}
% \changes{v0.9k}{1995/11/29}{Added default commands for Next input
%    encoding.}
%
% \changes{v0.9m}{1995/12/04}{Added \cs{ensuremath} to definitions of
%    the fraction glyphs.}
%
% \changes{v0.9t}{1996/10/28}
%      {Added \cs{textblacksquare}}
% \changes{v0.9u}{1996/10/29}
%      {Corrected code for \cs{textblacksquare}}
%      
% \changes{v0.9u}{1996/10/29}
%      {Added cp865 and corrected cp850, cp852 and cp437 guards}
% \changes{v0.94}{1997/12/17}{Removed entries that were solely in
%       next.def}
% \changes{v0.995}{2000/05/22}{Added latin2 option for textdegree
%    (pr/3207) CAR}
%       
% The name |\textblacksquare| is derived from the AMS symbol name since
% Adobe seem not to want this symbol.  The default definition, as a
% rule, makes no claim to being a good design.
% 
% Some entries are repeated in case guards must all be on one line. 
%    \begin{macrocode}
%<*latin1|decmulti|latin2|latin3|latin4|latin5|latin9|applemacce|latin10>
\ProvideTextCommandDefault{\textdegree}{\ensuremath{{^\circ}}}
%</latin1|decmulti|latin2|latin3|latin4|latin5|latin9|applemacce|latin10>
%<*cp850|cp858|cp852|cp865|cp437|cp437de|applemac|cp1252|cp1250|cp1257|next>
\ProvideTextCommandDefault{\textdegree}{\ensuremath{{^\circ}}}
%</cp850|cp858|cp852|cp865|cp437|cp437de|applemac|cp1252|cp1250|cp1257|next>
%<*latin1|decmulti|latin3|latin5|cp850|cp858|cp852|cp1252|cp1257|next>
\ProvideTextCommandDefault{\textonehalf}{\ensuremath{\frac12}}
%</latin1|decmulti|latin3|latin5|cp850|cp858|cp852|cp1252|cp1257|next>
%<*latin1|decmulti|latin5|cp850|cp858|cp852|cp1252|cp1257|next>
\ProvideTextCommandDefault{\textonequarter}{\ensuremath{\frac14}}
%</latin1|decmulti|latin5|cp850|cp858|cp852|cp1252|cp1257|next>
%<*latin1|latin5|cp850|cp858|cp852|cp1252|cp1257|next>
\ProvideTextCommandDefault{\textthreequarters}{\ensuremath{\frac34}}
%</latin1|latin5|cp850|cp858|cp852|cp1252|cp1257|next>
%<*applemac|cp850|cp858|cp865|cp437|cp437de|cp1252|next>
\ProvideTextCommandDefault{\textflorin}{\textit{f}}
%</applemac|cp850|cp858|cp865|cp437|cp437de|cp1252|next>
%<*cp865|cp437|cp437de>
\ProvideTextCommandDefault{\textpeseta}{Pt}
%</cp865|cp437|cp437de>
%<*cp850|cp858|cp852|cp865|cp437|cp437de>
\ProvideTextCommandDefault{\textblacksquare}
{\vrule \@width .3em \@height .4em \@depth -.1em\relax}
%</cp850|cp858|cp852|cp865|cp437|cp437de>
%    \end{macrocode}
%
% \changes{v0.9k}{1995/11/29}{Added error messages for unavailable
%    characters.}
%
% Some commands can't be faked, so we have them generate an error
% message. 
%     \begin{macrocode}
%<*latin1|decmulti|latin5|latin9|cp850|cp858|cp865|cp437|cp437de>
\ProvideTextCommandDefault{\textcent}
   {\TextSymbolUnavailable\textcent}
\ProvideTextCommandDefault{\textyen}
   {\TextSymbolUnavailable\textyen}
%</latin1|decmulti|latin5|latin9|cp850|cp858|cp865|cp437|cp437de>
%<*applemac|cp1252|next>
\ProvideTextCommandDefault{\textcent}
   {\TextSymbolUnavailable\textcent}
\ProvideTextCommandDefault{\textyen}
   {\TextSymbolUnavailable\textyen}
%</applemac|cp1252|next>
%<*cp1257>
\ProvideTextCommandDefault{\textcent}
   {\TextSymbolUnavailable\textcent}
%</cp1257>
%<*latin9|cp1252|cp1257|latin10>
\ProvideTextCommandDefault{\texteuro}
   {\TextSymbolUnavailable\texteuro}
%</latin9|cp1252|cp1257|latin10>
%<*latin1|decmulti|latin2|latin3|latin4|latin5|cp850|cp858|cp852|cp865>
\ProvideTextCommandDefault{\textcurrency}
   {\TextSymbolUnavailable\textcurrency}
%</latin1|decmulti|latin2|latin3|latin4|latin5|cp850|cp858|cp852|cp865>
%<*applemac|cp1252|cp1250|cp1257|next>
\ProvideTextCommandDefault{\textcurrency}
   {\TextSymbolUnavailable\textcurrency}
%</applemac|cp1252|cp1250|cp1257|next>
%<*latin1|latin5|cp850|cp858|cp852|cp1252|cp1250|cp1257>
\ProvideTextCommandDefault{\textbrokenbar}
   {\TextSymbolUnavailable\textbrokenbar}
%</latin1|latin5|cp850|cp858|cp852|cp1252|cp1250|cp1257>
%<*latin3>
\ProvideTextCommandDefault{\textmalteseH}
   {\TextSymbolUnavailable\textmalteseH}
\ProvideTextCommandDefault{\textmalteseh}
   {\TextSymbolUnavailable\textmalteseh}
%</latin3>
%<*latin4>
\ProvideTextCommandDefault{\textkra}
  {\TextSymbolUnavailable\textkra}
\ProvideTextCommandDefault{\textTstroke}
  {\TextSymbolUnavailable\textTstroke}
\ProvideTextCommandDefault{\texttstroke}
  {\TextSymbolUnavailable\texttstroke}
%</latin4>
%<*cp1250|cp1252|cp1257|applemac|next>
\ProvideTextCommandDefault{\textperthousand}
   {\TextSymbolUnavailable\textperthousand}
%</cp1250|cp1252|cp1257|applemac|next>
%<*applemacce>
\ProvideTextCommandDefault{\textdiv}
   {\TextSymbolUnavailable\textdiv}
%</applemacce>
%    \end{macrocode}
%
% \changes{v0.9l}{1995/12/01}{Removed extraneous braces from the
%    `superior' glyphs.}
% \changes{v0.97}{1998/03/05}{Removed ords:
%    changed to text chars, pr/2579}
%    
% Characters that are supposed to be used only in math will be defined
% by |\providecommand| because \LaTeXe{} assumes that the font
% encoding for math fonts is static.
%
%    \begin{macrocode}
%<*latin1|decmulti|latin5|latin9|cp850|cp858|cp1252|cp1257|next>
\providecommand{\mathonesuperior}{{^1}}
%</latin1|decmulti|latin5|latin9|cp850|cp858|cp1252|cp1257|next>
%<*latin1|decmulti|latin3|latin5|latin9|cp850|cp858|cp1252|cp1257|next>
\providecommand{\maththreesuperior}{{^3}}
%</latin1|decmulti|latin3|latin5|latin9|cp850|cp858|cp1252|cp1257|next>
%<*latin1|decmulti|latin3|latin5|latin9|cp850|cp858|cp865|cp437|cp437de>
\providecommand{\mathtwosuperior}{{^2}}
%</latin1|decmulti|latin3|latin5|latin9|cp850|cp858|cp865|cp437|cp437de>
%<*cp1252|cp1257|next>
\providecommand{\mathtwosuperior}{{^2}}
%</cp1252|cp1257|next>
%<*cp865|cp437|cp437de>
\providecommand{\mathnsuperior}{{^n}}
%</cp865|cp437|cp437de>
%    \end{macrocode}
%
% \section{The ASCII encoding}
%
% The ASCII encoding only allows characters in the range 32--127, so
% we only need to provide a more or less empty |.def| file.
% But we suppress the warning that would normally appear if there are no
% encoding definitions.
%
% \changes{v1.1b}{2006/03/03}{Suppress unnecessary warning (pr/3849)}
%    \begin{macrocode}
%<ascii>\@inpenc@test
%    \end{macrocode}
%
% \changes{v0.94}{1997/12/17}{Merged latin1 with ansinew/cp1252}
%    
% \section{The ISO Latin-2 encoding}
%
% The ISO Latin-2 encoding file defines the characters
% in the ISO 8859-2 encoding. It was contributed by
% Petr Sojka (\texttt{sojka@muni.cz}) with small technical
% updates by Frank Mittelbach.
%
% \changes{v0.9e}{1995/08/31}{Redeclared "AD to be soft hyphen.}
% \changes{v0.91}{1997/08/19}{Replaced \cs{dh}/\cs{DH} by
%    \cs{dj}/\cs{DJ}.}
% \changes{v0.94}{1997/12/17}{Changed 176 to \cs{textdegree}}
%
%    \begin{macrocode}
%<*latin2>
\DeclareInputText{160}{\nobreakspace}
\DeclareInputText{176}{\textdegree}
\DeclareInputText{161}{\k A}
\DeclareInputText{177}{\k a}
\DeclareInputText{162}{\u{}}
\DeclareInputText{178}{\k\ }
\DeclareInputText{163}{\L}
\DeclareInputText{179}{\l}
\DeclareInputText{164}{\textcurrency}
\DeclareInputText{180}{\@tabacckludge'{}}
\DeclareInputText{165}{\v L}
\DeclareInputText{181}{\v l}
\DeclareInputText{166}{\@tabacckludge'S}
\DeclareInputText{182}{\@tabacckludge's}
\DeclareInputText{167}{\S}
\DeclareInputText{183}{\v{}}
\DeclareInputText{168}{\"{}}
\DeclareInputText{184}{\c\ }
\DeclareInputText{169}{\v S}
\DeclareInputText{185}{\v s}
\DeclareInputText{170}{\c S}
\DeclareInputText{186}{\c s}
\DeclareInputText{171}{\v T}
\DeclareInputText{187}{\v t}
\DeclareInputText{172}{\@tabacckludge'Z}
\DeclareInputText{188}{\@tabacckludge'z}
\DeclareInputText{173}{\-}
\DeclareInputText{189}{\H{}}
\DeclareInputText{174}{\v Z}
\DeclareInputText{190}{\v z}
\DeclareInputText{175}{\.Z}
\DeclareInputText{191}{\.z}
%    \end{macrocode}
%
%    \begin{macrocode}
\DeclareInputText{192}{\@tabacckludge'R}
\DeclareInputText{208}{\DJ}
\DeclareInputText{193}{\@tabacckludge'A}
\DeclareInputText{209}{\@tabacckludge'N}
\DeclareInputText{194}{\^A}
\DeclareInputText{210}{\v N}
\DeclareInputText{195}{\u A}
\DeclareInputText{211}{\@tabacckludge'O}
\DeclareInputText{196}{\"A}
\DeclareInputText{212}{\^O}
\DeclareInputText{197}{\@tabacckludge'L}
\DeclareInputText{213}{\H O}
\DeclareInputText{198}{\@tabacckludge'C}
\DeclareInputText{214}{\"O}
\DeclareInputText{199}{\c C}
\DeclareInputMath{215}{\times}
\DeclareInputText{200}{\v C}
\DeclareInputText{216}{\v R}
\DeclareInputText{201}{\@tabacckludge'E}
\DeclareInputText{217}{\r U}
\DeclareInputText{202}{\k E}
\DeclareInputText{218}{\@tabacckludge'U}
\DeclareInputText{203}{\"E}
\DeclareInputText{219}{\H U}
\DeclareInputText{204}{\v E}
\DeclareInputText{220}{\"U}
\DeclareInputText{205}{\@tabacckludge'I}
\DeclareInputText{221}{\@tabacckludge'Y}
\DeclareInputText{206}{\^I}
\DeclareInputText{222}{\c T}
\DeclareInputText{207}{\v D}
\DeclareInputText{223}{\ss}
%    \end{macrocode}
%
%    \begin{macrocode}
\DeclareInputText{224}{\@tabacckludge'r}
\DeclareInputText{240}{\dj}
\DeclareInputText{225}{\@tabacckludge'a}
\DeclareInputText{241}{\@tabacckludge'n}
\DeclareInputText{226}{\^a}
\DeclareInputText{242}{\v n}
\DeclareInputText{227}{\u a}
\DeclareInputText{243}{\@tabacckludge'o}
\DeclareInputText{228}{\"a}
\DeclareInputText{244}{\^o}
\DeclareInputText{229}{\@tabacckludge'l}
\DeclareInputText{245}{\H o}
\DeclareInputText{230}{\@tabacckludge'c}
\DeclareInputText{246}{\"o}
\DeclareInputText{231}{\c c}
\DeclareInputMath{247}{\div}
\DeclareInputText{232}{\v c}
\DeclareInputText{248}{\v r}
\DeclareInputText{233}{\@tabacckludge'e}
\DeclareInputText{249}{\r u}
\DeclareInputText{234}{\k e}
\DeclareInputText{250}{\@tabacckludge'u}
\DeclareInputText{235}{\"e}
\DeclareInputText{251}{\H u}
\DeclareInputText{236}{\v e}
\DeclareInputText{252}{\"u}
\DeclareInputText{237}{\@tabacckludge'\i}
\DeclareInputText{253}{\@tabacckludge'y}
\DeclareInputText{238}{\^\i}
\DeclareInputText{254}{\c t}
\DeclareInputText{239}{\v d}
\DeclareInputText{255}{\.{}}
%</latin2>
%    \end{macrocode}
%
% \section{The ISO Latin-3 encoding}
%
% The ISO Latin-3 encoding file defines the characters
% in the ISO 8859-3 encoding.  It was contributed by
% by J\"org Knappen (\texttt{joerg.knappen@uni-mainz.de}) and
% adapted by Chris Rowley.
% 
% It can be used for general purpose applications in
% typical office environments in the following languages:
% Afrikaans, Catalan, English, Esperanto, French, Galician, German,
% Italian, Maltese, and Turkish.
%
%    \begin{macrocode}
%<*latin3>
\DeclareInputText{160}{\nobreakspace}
\DeclareInputText{176}{\textdegree}
\DeclareInputText{161}{\textmalteseH}
\DeclareInputText{177}{\textmalteseh}
\DeclareInputText{162}{\u{}}
\DeclareInputMath{178}{\mathtwosuperior}
\DeclareInputText{163}{\pounds}
\DeclareInputMath{179}{\maththreesuperior}
\DeclareInputText{164}{\textcurrency}
\DeclareInputText{180}{\@tabacckludge'{}}
% \DeclareInputText{165}{\notdef}
\DeclareInputMath{181}{\mu}
\DeclareInputText{166}{\^H}
% NOT: \DeclareInputText{182}{h\llap{\^{}}} % \^h would be too tall
\DeclareInputText{182}{\^h}
\DeclareInputText{167}{\S}
\DeclareInputText{183}{\textperiodcentered}
\DeclareInputText{168}{\"{}}
\DeclareInputText{184}{\c\ }
\DeclareInputText{169}{\.I}
\DeclareInputText{185}{\i}
\DeclareInputText{170}{\c S}
\DeclareInputText{186}{\c s}
\DeclareInputText{171}{\u G}
\DeclareInputText{187}{\u g}
\DeclareInputText{172}{\^J}
\DeclareInputText{188}{\^\j}
\DeclareInputText{173}{\-}
\DeclareInputText{189}{\textonehalf}
% \DeclareInputText{174}{\notdef}
% \DeclareInputText{190}{\notdef}
\DeclareInputText{175}{\.Z}
\DeclareInputText{191}{\.z}
%    \end{macrocode}
%
%    \begin{macrocode}
\DeclareInputText{192}{\@tabacckludge`A}
% \DeclareInputText{208}{\notdef}
\DeclareInputText{193}{\@tabacckludge'A}
\DeclareInputText{209}{\~N}
\DeclareInputText{194}{\^A}
\DeclareInputText{210}{\@tabacckludge`O}
% \DeclareInputText{195}{\notdef}
\DeclareInputText{211}{\@tabacckludge'O}
\DeclareInputText{196}{\"A}
\DeclareInputText{212}{\^O}
\DeclareInputText{197}{\.C}
\DeclareInputText{213}{\.G}
\DeclareInputText{198}{\^C}
\DeclareInputText{214}{\"O}
\DeclareInputText{199}{\c C}
\DeclareInputMath{215}{\times}
\DeclareInputText{200}{\@tabacckludge`E}
\DeclareInputText{216}{\^G}
\DeclareInputText{201}{\@tabacckludge'E}
\DeclareInputText{217}{\@tabacckludge`U}
\DeclareInputText{202}{\^E}
\DeclareInputText{218}{\@tabacckludge'U}
\DeclareInputText{203}{\"E}
\DeclareInputText{219}{\^U}
\DeclareInputText{204}{\@tabacckludge`I}
\DeclareInputText{220}{\"U}
\DeclareInputText{205}{\@tabacckludge'I}
\DeclareInputText{221}{\u U}
\DeclareInputText{206}{\^I}
\DeclareInputText{222}{\^S}
\DeclareInputText{207}{\"I}
\DeclareInputText{223}{\ss}
%    \end{macrocode}
%
%    \begin{macrocode}
\DeclareInputText{224}{\@tabacckludge`a}
% \DeclareInputText{240}{\notdef}
\DeclareInputText{225}{\@tabacckludge'a}
\DeclareInputText{241}{\~n}
\DeclareInputText{226}{\^a}
\DeclareInputText{242}{\@tabacckludge`o}
% \DeclareInputText{227}{\notdef}
\DeclareInputText{243}{\@tabacckludge'o}
\DeclareInputText{228}{\"a}
\DeclareInputText{244}{\^o}
\DeclareInputText{229}{\.c}
\DeclareInputText{245}{\.g}
\DeclareInputText{230}{\^c}
\DeclareInputText{246}{\"o}
\DeclareInputText{231}{\c c}
\DeclareInputMath{247}{\div}
\DeclareInputText{232}{\@tabacckludge`e}
\DeclareInputText{248}{\^g}
\DeclareInputText{233}{\@tabacckludge'e}
\DeclareInputText{249}{\@tabacckludge`u}
\DeclareInputText{234}{\^e}
\DeclareInputText{250}{\@tabacckludge'u}
\DeclareInputText{235}{\"e}
\DeclareInputText{251}{\^u}
\DeclareInputText{236}{\@tabacckludge`\i}
\DeclareInputText{252}{\"u}
\DeclareInputText{237}{\@tabacckludge'\i}
\DeclareInputText{253}{\u u}
\DeclareInputText{238}{\^\i}
\DeclareInputText{254}{\^s}
\DeclareInputText{239}{\"\i}
\DeclareInputText{255}{\.{}}
%</latin3>
%    \end{macrocode}
%
% \section{The ISO Latin-4 encoding}
%
% The ISO Latin-4 encoding file defines the characters in the ISO
% 8859-4 encoding. It was contributed by Hana Skoumalov\'a 
% (\texttt{hana.skoumalova@ff.cuni.cz}).
%
% It was created for Estonian, Latvian, Lithuanian, Finnish, Lappish,
% Swedish, Norwegian, Danish, Icelandic and Greenlandic Inuit. Some
% alphabets, however, are incomplete (Greenlandic, Icelandic and Lappish).
% Some glyphs are not available in the fonts. For example, the
% Greenlandic character `kra' is not available at all;
% the Latvian and Lithuanian characters not
% contained in other encodings are constructed from components and
% they do not look good.
%
% \changes{v0.999}{2001/06/04}{Added \cs{textkra}, \cs{texttstroke} and
%                              \cs{textTstroke} (pr/3336)}
%    \begin{macrocode}
%<*latin4>
\DeclareInputText{160}{\nobreakspace}
\DeclareInputText{161}{\k A}
\DeclareInputText{162}{\textkra}%% Greenlandic Inuit
\DeclareInputText{163}{\c R}
\DeclareInputText{164}{\textcurrency}
\DeclareInputText{165}{\~I}
\DeclareInputText{166}{\c L}
\DeclareInputText{167}{\S}
\DeclareInputText{168}{\"{}}
\DeclareInputText{169}{\v S}
\DeclareInputText{170}{\@tabacckludge=E}
\DeclareInputText{171}{\c G}
\DeclareInputText{172}{\textTstroke}%% Northern Sami
\DeclareInputText{173}{\-}
\DeclareInputText{174}{\v Z}
\DeclareInputText{175}{\@tabacckludge={}}
\DeclareInputText{176}{\textdegree}
\DeclareInputText{177}{\k a}
\DeclareInputText{178}{\k\ }
\DeclareInputText{179}{\c r}
\DeclareInputText{180}{\@tabacckludge'{}}
\DeclareInputText{181}{\~\i}
\DeclareInputText{182}{\c l}
\DeclareInputText{183}{\v{}}
\DeclareInputText{184}{\c\ }
\DeclareInputText{185}{\v s}
\DeclareInputText{186}{\@tabacckludge=e}
\DeclareInputText{187}{\c g}
\DeclareInputText{188}{\texttstroke}%% Northern Sami
\DeclareInputText{189}{\NG}
\DeclareInputText{190}{\v z}
\DeclareInputText{191}{\ng}
%    \end{macrocode}
%
%    \begin{macrocode}
\DeclareInputText{192}{\@tabacckludge=A}
\DeclareInputText{193}{\@tabacckludge'A}
\DeclareInputText{194}{\^A}
\DeclareInputText{195}{\~A}
\DeclareInputText{196}{\"A}
\DeclareInputText{197}{\r A}
\DeclareInputText{198}{\AE}
\DeclareInputText{199}{\k I}
\DeclareInputText{200}{\v C}
\DeclareInputText{201}{\@tabacckludge'E}
\DeclareInputText{202}{\k E}
\DeclareInputText{203}{\"E}
\DeclareInputText{204}{\.{E}}
\DeclareInputText{205}{\@tabacckludge'I}
\DeclareInputText{206}{\^I}
\DeclareInputText{207}{\@tabacckludge=I}
\DeclareInputText{208}{\DJ}
\DeclareInputText{209}{\c N}
\DeclareInputText{210}{\@tabacckludge=O}
\DeclareInputText{211}{\c K}
\DeclareInputText{212}{\^O}
\DeclareInputText{213}{\~O}
\DeclareInputText{214}{\"O}
\DeclareInputMath{215}{\times}
\DeclareInputText{216}{\O}
\DeclareInputText{217}{\k U}
\DeclareInputText{218}{\@tabacckludge'U}
\DeclareInputText{219}{\^U}
\DeclareInputText{220}{\"U}
\DeclareInputText{221}{\~U}
\DeclareInputText{222}{\@tabacckludge=U}
\DeclareInputText{223}{\ss}
%    \end{macrocode}
%
%    \begin{macrocode}
\DeclareInputText{224}{\@tabacckludge=a}
\DeclareInputText{225}{\@tabacckludge'a}
\DeclareInputText{226}{\^a}
\DeclareInputText{227}{\~a}
\DeclareInputText{228}{\"a}
\DeclareInputText{229}{\r a}
\DeclareInputText{230}{\ae}
\DeclareInputText{231}{\k i}
\DeclareInputText{232}{\v c}
\DeclareInputText{233}{\@tabacckludge'e}
\DeclareInputText{234}{\k e}
\DeclareInputText{235}{\"e}
\DeclareInputText{236}{\.{e}}
\DeclareInputText{237}{\@tabacckludge'\i}
\DeclareInputText{238}{\^\i}
\DeclareInputText{239}{\@tabacckludge=\i}
\DeclareInputText{240}{\dj}
\DeclareInputText{241}{\c n}
\DeclareInputText{242}{\@tabacckludge=o}
\DeclareInputText{243}{\c k}
\DeclareInputText{244}{\^o}
\DeclareInputText{245}{\~o}
\DeclareInputText{246}{\"o}
\DeclareInputMath{247}{\div}
\DeclareInputText{248}{\o}
\DeclareInputText{249}{\k u}
\DeclareInputText{250}{\@tabacckludge'u}
\DeclareInputText{251}{\^u}
\DeclareInputText{252}{\"u}
\DeclareInputText{253}{\~u}
\DeclareInputText{254}{\@tabacckludge=u}
\DeclareInputText{255}{\.{}}
%</latin4>
%    \end{macrocode}
%
% \section{The ISO Latin-5 encoding}
%
% \changes{v0.991}{1999/08/23}{Corrected description for Latin-5.}
% The ISO Latin-5 encoding file defines the characters
% in the ISO 8859-9 encoding, which describes Latin Alphabet No. 5.
% It was contributed by H.~Turgut Uyar (\texttt{uyar@cs.itu.edu.tr});
% it is used for Turkish.
%
%    \begin{macrocode}
%<*latin5>
\DeclareInputText{160}{\nobreakspace}
\DeclareInputText{176}{\textdegree}
\DeclareInputText{161}{\textexclamdown}
\DeclareInputMath{177}{\pm}
\DeclareInputText{162}{\textcent}
\DeclareInputMath{178}{\mathtwosuperior}
\DeclareInputText{163}{\pounds}
\DeclareInputMath{179}{\maththreesuperior}
\DeclareInputText{164}{\textcurrency}
\DeclareInputText{180}{\@tabacckludge'{}}
\DeclareInputText{165}{\textyen}
\DeclareInputMath{181}{\mu}
\DeclareInputText{166}{\textbrokenbar}
\DeclareInputText{182}{\P}
\DeclareInputText{167}{\S}
\DeclareInputText{183}{\textperiodcentered}
\DeclareInputText{168}{\"{}}
\DeclareInputText{184}{\c\ }
\DeclareInputText{169}{\copyright}
\DeclareInputMath{185}{\mathonesuperior}
\DeclareInputText{170}{\textordfeminine}
\DeclareInputText{186}{\textordmasculine}
\DeclareInputText{171}{\guillemotleft}
\DeclareInputText{187}{\guillemotright}
\DeclareInputMath{172}{\lnot}
\DeclareInputText{188}{\textonequarter}
\DeclareInputText{173}{\-}
\DeclareInputText{189}{\textonehalf}
\DeclareInputText{174}{\textregistered}
\DeclareInputText{190}{\textthreequarters}
\DeclareInputText{175}{\@tabacckludge={}}
\DeclareInputText{191}{\textquestiondown}
%    \end{macrocode}
%
%    \begin{macrocode}
\DeclareInputText{192}{\@tabacckludge`A}
\DeclareInputText{208}{\u G}
\DeclareInputText{193}{\@tabacckludge'A}
\DeclareInputText{209}{\~N}
\DeclareInputText{194}{\^A}
\DeclareInputText{210}{\@tabacckludge`O}
\DeclareInputText{195}{\~A}
\DeclareInputText{211}{\@tabacckludge'O}
\DeclareInputText{196}{\"A}
\DeclareInputText{212}{\^O}
\DeclareInputText{197}{\r A}
\DeclareInputText{213}{\~O}
\DeclareInputText{198}{\AE}
\DeclareInputText{214}{\"O}
\DeclareInputText{199}{\c C}
\DeclareInputMath{215}{\times}
\DeclareInputText{200}{\@tabacckludge`E}
\DeclareInputText{216}{\O}
\DeclareInputText{201}{\@tabacckludge'E}
\DeclareInputText{217}{\@tabacckludge`U}
\DeclareInputText{202}{\^E}
\DeclareInputText{218}{\@tabacckludge'U}
\DeclareInputText{203}{\"E}
\DeclareInputText{219}{\^U}
\DeclareInputText{204}{\@tabacckludge`I}
\DeclareInputText{220}{\"U}
\DeclareInputText{205}{\@tabacckludge'I}
\DeclareInputText{221}{\.I}
\DeclareInputText{206}{\^I}
\DeclareInputText{222}{\c S}
\DeclareInputText{207}{\"I}
\DeclareInputText{223}{\ss}
%    \end{macrocode}
%
%    \begin{macrocode}
\DeclareInputText{224}{\@tabacckludge`a}
\DeclareInputText{240}{\u g}
\DeclareInputText{225}{\@tabacckludge'a}
\DeclareInputText{241}{\~n}
\DeclareInputText{226}{\^a}
\DeclareInputText{242}{\@tabacckludge`o}
\DeclareInputText{227}{\~a}
\DeclareInputText{243}{\@tabacckludge'o}
\DeclareInputText{228}{\"a}
\DeclareInputText{244}{\^o}
\DeclareInputText{229}{\r a}
\DeclareInputText{245}{\~o}
\DeclareInputText{230}{\ae}
\DeclareInputText{246}{\"o}
\DeclareInputText{231}{\c c}
\DeclareInputMath{247}{\div}
\DeclareInputText{232}{\@tabacckludge`e}
\DeclareInputText{248}{\o}
\DeclareInputText{233}{\@tabacckludge'e}
\DeclareInputText{249}{\@tabacckludge`u}
\DeclareInputText{234}{\^e}
\DeclareInputText{250}{\@tabacckludge'u}
\DeclareInputText{235}{\"e}
\DeclareInputText{251}{\^u}
\DeclareInputText{236}{\@tabacckludge`\i}
\DeclareInputText{252}{\"u}
\DeclareInputText{237}{\@tabacckludge'\i}
\DeclareInputText{253}{\i}
\DeclareInputText{238}{\^\i}
\DeclareInputText{254}{\c s}
\DeclareInputText{239}{\"\i}
\DeclareInputText{255}{\"y}
%</latin5>
%    \end{macrocode}
%
% \section{DEC Multinational Character Set}
%
% The DECMultinational character set, used by the OpenVMS operating
% system, is slightly different from the ISO Latin 1 character set.
%
% Reference: Digital Equipment Corporation VT330/VT340 Programmer
%    Reference Manual, Volume 1: Text Programming, page 22.
%
% This encoding was provided by  M.Y. Chartoire IPNL-IN2P3 \\
% \texttt{m.chartoire@ipnl.in2p3.fr}
%
%    \begin{macrocode}
%<*decmulti>
\DeclareInputText{176}{\textdegree}
\DeclareInputText{161}{\textexclamdown}
\DeclareInputMath{177}{\pm}
\DeclareInputText{162}{\textcent}
\DeclareInputMath{178}{\mathtwosuperior}
\DeclareInputText{163}{\pounds}
\DeclareInputMath{179}{\maththreesuperior}
\DeclareInputText{165}{\textyen}
\DeclareInputMath{181}{\mu}
\DeclareInputText{182}{\P}
\DeclareInputText{167}{\S}
\DeclareInputText{183}{\textperiodcentered}
\DeclareInputText{168}{\textcurrency}
\DeclareInputText{169}{\copyright}
\DeclareInputMath{185}{\mathonesuperior}
\DeclareInputText{170}{\textordfeminine}
\DeclareInputText{186}{\textordmasculine}
\DeclareInputText{171}{\guillemotleft}
\DeclareInputText{187}{\guillemotright}
\DeclareInputText{188}{\textonequarter}
\DeclareInputText{189}{\textonehalf}
\DeclareInputText{191}{\textquestiondown}
%    \end{macrocode}
%
%    \begin{macrocode}
\DeclareInputText{192}{\@tabacckludge`A}
\DeclareInputText{193}{\@tabacckludge'A}
\DeclareInputText{209}{\~N}
\DeclareInputText{194}{\^A}
\DeclareInputText{210}{\@tabacckludge`O}
\DeclareInputText{195}{\~A}
\DeclareInputText{211}{\@tabacckludge'O}
\DeclareInputText{196}{\"A}
\DeclareInputText{212}{\^O}
\DeclareInputText{197}{\r A}
\DeclareInputText{213}{\~O}
\DeclareInputText{198}{\AE}
\DeclareInputText{214}{\"O}
\DeclareInputText{199}{\c C}
\DeclareInputText{215}{\OE}
\DeclareInputText{200}{\@tabacckludge`E}
\DeclareInputText{216}{\O}
\DeclareInputText{201}{\@tabacckludge'E}
\DeclareInputText{217}{\@tabacckludge`U}
\DeclareInputText{202}{\^E}
\DeclareInputText{218}{\@tabacckludge'U}
\DeclareInputText{203}{\"E}
\DeclareInputText{219}{\^U}
\DeclareInputText{204}{\@tabacckludge`I}
\DeclareInputText{220}{\"U}
\DeclareInputText{205}{\@tabacckludge'I}
\DeclareInputText{221}{\"Y}
\DeclareInputText{206}{\^I}
\DeclareInputText{207}{\"I}
\DeclareInputText{223}{\ss}
%    \end{macrocode}
%
%    \begin{macrocode}
\DeclareInputText{224}{\@tabacckludge`a}
\DeclareInputText{225}{\@tabacckludge'a}
\DeclareInputText{241}{\~n}
\DeclareInputText{226}{\^a}
\DeclareInputText{242}{\@tabacckludge`o}
\DeclareInputText{227}{\~a}
\DeclareInputText{243}{\@tabacckludge'o}
\DeclareInputText{228}{\"a}
\DeclareInputText{244}{\^o}
\DeclareInputText{229}{\r a}
\DeclareInputText{245}{\~o}
\DeclareInputText{230}{\ae}
\DeclareInputText{246}{\"o}
\DeclareInputText{231}{\c c}
\DeclareInputText{247}{\oe}
\DeclareInputText{232}{\@tabacckludge`e}
\DeclareInputText{248}{\o}
\DeclareInputText{233}{\@tabacckludge'e}
\DeclareInputText{249}{\@tabacckludge`u}
\DeclareInputText{234}{\^e}
\DeclareInputText{250}{\@tabacckludge'u}
\DeclareInputText{235}{\"e}
\DeclareInputText{251}{\^u}
\DeclareInputText{236}{\@tabacckludge`\i}
\DeclareInputText{252}{\"u}
\DeclareInputText{237}{\@tabacckludge'\i}
\DeclareInputText{253}{\"y}
\DeclareInputText{238}{\^\i}
\DeclareInputText{239}{\"\i}
%</decmulti>
%    \end{macrocode}
%
%
% \section{The IBM code pages 850 and 858}
%
% This input encoding was contributed by
% Timo Knuutila (\texttt{knuutila@\linebreak[0]cs.utu.fi}),
% and edited by Christian Bartels
% (\texttt{ii140ba@\linebreak[0]vm1.rz.rwth-aachen.de}).
%
% The DOS graphics `letters' and a few
% other positions are ignored (left undefined).
%
% The 858 code page is identical to the 850 except that
%
% \changes{v0.9d}{1995/06/06}{Made changes to cp850 suggested by
%    Christian Bartels}
% \changes{v0.9s}{1995/08/31}
%      {Swapped ordfeminine and masculine. /2203}
%
%    \begin{macrocode}
%<*cp850|cp858>
\DeclareInputText{128}{\c C}
\DeclareInputText{144}{\@tabacckludge'E}
\DeclareInputText{129}{\"u}
\DeclareInputText{145}{\ae}
\DeclareInputText{130}{\@tabacckludge'e}
\DeclareInputText{146}{\AE}
\DeclareInputText{131}{\^a}
\DeclareInputText{147}{\^o}
\DeclareInputText{132}{\"a}
\DeclareInputText{148}{\"o}
\DeclareInputText{133}{\@tabacckludge`a}
\DeclareInputText{149}{\@tabacckludge`o}
\DeclareInputText{134}{\r a}
\DeclareInputText{150}{\^u}
\DeclareInputText{135}{\c c}
\DeclareInputText{151}{\@tabacckludge`u}
\DeclareInputText{136}{\^e}
\DeclareInputText{152}{\"y}
\DeclareInputText{137}{\"e}
\DeclareInputText{153}{\"O}
\DeclareInputText{138}{\@tabacckludge`e}
\DeclareInputText{154}{\"U}
\DeclareInputText{139}{\"\i}
\DeclareInputText{155}{\o}
\DeclareInputText{140}{\^\i}
\DeclareInputText{156}{\pounds}
\DeclareInputText{141}{\@tabacckludge`\i}
\DeclareInputText{157}{\O}
\DeclareInputText{142}{\"A}
\DeclareInputMath{158}{\times}
\DeclareInputText{143}{\r A}
\DeclareInputText{159}{\textflorin}
%    \end{macrocode}
%
%    \begin{macrocode}
\DeclareInputText{160}{\@tabacckludge'a}
\DeclareInputText{161}{\@tabacckludge'\i}
\DeclareInputText{162}{\@tabacckludge'o}
\DeclareInputText{163}{\@tabacckludge'u}
\DeclareInputText{164}{\~n}               %% "B0-"B4: DG
\DeclareInputText{165}{\~N}
\DeclareInputText{181}{\@tabacckludge'A}
\DeclareInputText{166}{\textordfeminine}
\DeclareInputText{182}{\^A}
\DeclareInputText{167}{\textordmasculine}
\DeclareInputText{183}{\@tabacckludge`A}
\DeclareInputText{168}{\textquestiondown}
\DeclareInputText{184}{\copyright}
\DeclareInputText{169}{\textregistered}
\DeclareInputMath{170}{\lnot}
\DeclareInputText{171}{\textonehalf}
\DeclareInputText{172}{\textonequarter}   %% "B9-"BC: DG
\DeclareInputText{173}{\textexclamdown}
\DeclareInputText{189}{\textcent}
\DeclareInputText{174}{\guillemotleft}
\DeclareInputText{190}{\textyen}
\DeclareInputText{175}{\guillemotright}
%    \end{macrocode}
%
%    \begin{macrocode}
% "BF-"C5: DG
\DeclareInputText{208}{\dh}
\DeclareInputText{209}{\DH}
\DeclareInputText{210}{\^E}
\DeclareInputText{211}{\"E}
\DeclareInputText{212}{\@tabacckludge`E}
%    \end{macrocode}
%    Here is the only point in which the two code pages differ!
%    \begin{macrocode}
%<-cp858>\DeclareInputText{213}{\i}
%<-cp850>\DeclareInputText{213}{\texteuro}
%    \end{macrocode}
%
%    \begin{macrocode}
\DeclareInputText{198}{\~a}
\DeclareInputText{214}{\@tabacckludge'I}
\DeclareInputText{199}{\~A}
\DeclareInputText{215}{\^I}
\DeclareInputText{216}{\"I}
% "C8-"CE: DG
% "D9-"DC: DG
\DeclareInputText{221}{\textbrokenbar}
\DeclareInputText{222}{\@tabacckludge`I}
\DeclareInputText{207}{\textcurrency}  %% "DF: DG
%    \end{macrocode}
%
%    \begin{macrocode}
\DeclareInputText{224}{\@tabacckludge'O}
\DeclareInputText{240}{\-}
\DeclareInputText{225}{\ss}
\DeclareInputMath{241}{\pm}
\DeclareInputText{226}{\^O}
% "F2: DG (not double underline, or equals?)
\DeclareInputText{227}{\@tabacckludge`O}
\DeclareInputText{243}{\textthreequarters}
\DeclareInputText{228}{\~o}
\DeclareInputText{244}{\P}
\DeclareInputText{229}{\~O}
\DeclareInputText{245}{\S}
\DeclareInputMath{230}{\mu}
\DeclareInputMath{246}{\div}
\DeclareInputText{231}{\th}
\DeclareInputText{247}{\c\ }
\DeclareInputText{232}{\TH}
\DeclareInputText{248}{\textdegree}
\DeclareInputText{233}{\@tabacckludge'U}
\DeclareInputText{249}{\"{}}
\DeclareInputText{234}{\^U}
\DeclareInputText{250}{\textperiodcentered}
\DeclareInputText{235}{\@tabacckludge`U}
\DeclareInputMath{251}{\mathonesuperior}
\DeclareInputText{236}{\@tabacckludge'y}
\DeclareInputMath{252}{\maththreesuperior}
\DeclareInputText{237}{\@tabacckludge'Y}
\DeclareInputMath{253}{\mathtwosuperior}
\DeclareInputText{238}{\@tabacckludge={}}
\DeclareInputText{254}{\textblacksquare} % right name?
\DeclareInputText{239}{\@tabacckludge'{}}
\DeclareInputText{255}{\nobreakspace}
%</cp850|cp858>
%    \end{macrocode}
%
% \section{The IBM code page 852}
%
% This input encoding was contributed by
% Petr Sojka (\texttt{sojka@\linebreak[0]Muni.cz}).
%
% \changes{v0.9t}{1996/10/28}{Added extra \cs{nobreakspace}: OK?}
% \changes{v0.9z}{1997/05/10}{Corrected typo in slot 213}
% \changes{v0.91}{1997/08/19}{Replaced \cs{dh}/\cs{DH} by
%    \cs{dj}/\cs{DJ}.}
% \changes{v0.99}{1999/04/14}{Changed 212, see pr/2992}
% 
%    \begin{macrocode}
%<*cp852>
\DeclareInputText{128}{\c C}
\DeclareInputText{144}{\@tabacckludge'E}
\DeclareInputText{129}{\"u}
\DeclareInputText{145}{\@tabacckludge'L}
\DeclareInputText{130}{\@tabacckludge'e}
\DeclareInputText{146}{\@tabacckludge'l}
\DeclareInputText{131}{\^a}
\DeclareInputText{147}{\^o}
\DeclareInputText{132}{\"a}
\DeclareInputText{148}{\"o}
\DeclareInputText{133}{\r u}
\DeclareInputText{149}{\v L}
\DeclareInputText{134}{\@tabacckludge'c}
\DeclareInputText{150}{\v l}
\DeclareInputText{135}{\c c}
\DeclareInputText{151}{\@tabacckludge'S}
\DeclareInputText{136}{\l}
\DeclareInputText{152}{\@tabacckludge's}
\DeclareInputText{137}{\"e}
\DeclareInputText{153}{\"O}
\DeclareInputText{138}{\H O}
\DeclareInputText{154}{\"U}
\DeclareInputText{139}{\H o}
\DeclareInputText{155}{\v T}
\DeclareInputText{140}{\^\i}
\DeclareInputText{156}{\v t}
\DeclareInputText{141}{\@tabacckludge'Z}
\DeclareInputText{157}{\L}
\DeclareInputText{142}{\"A}
\DeclareInputMath{158}{\times}
\DeclareInputText{143}{\@tabacckludge'C}
\DeclareInputText{159}{\v c}
%    \end{macrocode}
%
%    \begin{macrocode}
\DeclareInputText{160}{\@tabacckludge'a}
\DeclareInputText{161}{\@tabacckludge'\i}
\DeclareInputText{162}{\@tabacckludge'o}
\DeclareInputText{163}{\@tabacckludge'u}
\DeclareInputText{164}{\k A}
\DeclareInputText{165}{\k a}
\DeclareInputText{166}{\v Z}
\DeclareInputText{167}{\v z}
\DeclareInputText{168}{\k E}
\DeclareInputText{169}{\k e}
\DeclareInputMath{170}{\lnot}
\DeclareInputText{171}{\@tabacckludge'z}
\DeclareInputText{172}{\v C}
\DeclareInputText{173}{\c s}
\DeclareInputText{174}{\guillemotleft}
\DeclareInputText{175}{\guillemotright}
\DeclareInputText{181}{\@tabacckludge'A}
\DeclareInputText{182}{\^A}
\DeclareInputText{183}{\v E}
\DeclareInputText{184}{\c S}
\DeclareInputText{189}{\.Z}
\DeclareInputText{190}{\.z}
%    \end{macrocode}
%
%    \begin{macrocode}
\DeclareInputText{198}{\u A}
\DeclareInputText{199}{\u a}
\DeclareInputText{207}{\textcurrency}
\DeclareInputText{208}{\dj}
\DeclareInputText{209}{\DJ}
\DeclareInputText{210}{\v D}
\DeclareInputText{211}{\"E}
\DeclareInputText{212}{\v d} % d caron
\DeclareInputText{213}{\v N}
\DeclareInputText{214}{\@tabacckludge'I}
\DeclareInputText{215}{\^I}
\DeclareInputText{216}{\v e}
\DeclareInputText{221}{\c T}
\DeclareInputText{222}{\r U}
%    \end{macrocode}
%
%    \begin{macrocode}
\DeclareInputText{224}{\@tabacckludge'O}
\DeclareInputText{240}{\-}
\DeclareInputText{225}{\ss}
\DeclareInputText{241}{\H{}}
\DeclareInputText{226}{\^O}
\DeclareInputText{242}{\k\ }
\DeclareInputText{227}{\@tabacckludge'N}
\DeclareInputText{243}{\v{}}
\DeclareInputText{228}{\@tabacckludge'n}
\DeclareInputText{244}{\u{}}
\DeclareInputText{229}{\v n}
\DeclareInputText{245}{\S}
\DeclareInputText{230}{\v S}
\DeclareInputMath{246}{\div}
\DeclareInputText{231}{\v s}
\DeclareInputText{247}{\c\ }
\DeclareInputText{232}{\@tabacckludge'R}
\DeclareInputText{248}{\textdegree}
\DeclareInputText{233}{\@tabacckludge'U}
\DeclareInputText{249}{\"{}}
\DeclareInputText{234}{\@tabacckludge'r}
\DeclareInputText{250}{\.{}}
\DeclareInputText{235}{\H U}
\DeclareInputText{251}{\H u}
\DeclareInputText{236}{\@tabacckludge'y}
\DeclareInputText{252}{\v R}
\DeclareInputText{237}{\@tabacckludge'Y}
\DeclareInputText{253}{\v r}
\DeclareInputText{238}{\c t}
\DeclareInputText{254}{\textblacksquare} % right name?
\DeclareInputText{239}{\@tabacckludge'{}}
\DeclareInputText{255}{\nobreakspace}
%</cp852>
%    \end{macrocode}
%
% \section{The IBM code pages 437 and 865}
%
% This input encoding is based on work by\\
% Volker Kunert
% (\texttt{volker@\linebreak[0]numsun1.mathematik.uni-halle.de})\\
% and \texttt{bontus@\linebreak[0]al6000.physik.uni-siegen.de}.\\
% The changes for cp865 are based on work by S\o ren Sandmann
% (\texttt{sandmann@\linebreak[0]daimi.aau.dk}), with thanks to
% them all.
% 
% The DOS graphics `letters' and a few
% other positions are ignored (left undefined).
%
% Unfortunately, in cp437 there is no agreement as to whether slot E1
% should be `$\beta$' or `\ss', so we provide two variants,
% one (cp437) with `$\beta$' and one (cp437de) with `\ss'.
%
% \changes{v0.9e}{1995/08/31}{Added a 0 to all character codes}
%
% \changes{v0.9m}{1995/12/04}{Made uumlaut and pounds text characters}
% \changes{v0.9t}{1996/10/28}{Added \cs{textflorin} and \cs{textpeseta}}
% \changes{v0.9t}{1996/10/28}{Removed \cs{textbrokenbar}}
% \changes{v0.9t}{1996/10/28}{Removed \cs{textendash}}
% \changes{v0.9t}{1996/10/28}{Changed \cs{Theta} to \cs{Phi}}
% \changes{v0.9t}{1996/10/28}{Changed \cs{Pi} to \cs{pi}, perhaps}
% \changes{v0.9t}{1996/10/28}{Changed \cs{emptyset} to \cs{phi}}
% \changes{v0.9t}{1996/10/28}{Changed \cs{maththreesuperior} to
%                             \cs{mathnsuperior}}
% \changes{v0.94}{1997/12/17}{Changed 158 to \cs{DeclareInputText}}
% 
%    \begin{macrocode}
%<*cp437|cp437de|cp865>
\DeclareInputText{128}{\c C}
\DeclareInputText{129}{\"u}
\DeclareInputText{130}{\@tabacckludge'e}
\DeclareInputText{131}{\^a}
\DeclareInputText{132}{\"a}
\DeclareInputText{133}{\@tabacckludge`a}
\DeclareInputText{134}{\r a}
\DeclareInputText{135}{\c c}
\DeclareInputText{136}{\^e}
\DeclareInputText{137}{\"e}
\DeclareInputText{138}{\@tabacckludge`e}
\DeclareInputText{139}{\"\i}
\DeclareInputText{140}{\^\i}
\DeclareInputText{141}{\@tabacckludge`\i}
\DeclareInputText{142}{\"A}
\DeclareInputText{143}{\r A}
\DeclareInputText{144}{\@tabacckludge'E}
\DeclareInputText{145}{\ae}
\DeclareInputText{146}{\AE}
\DeclareInputText{147}{\^o}
\DeclareInputText{148}{\"o}
\DeclareInputText{149}{\@tabacckludge`o}
\DeclareInputText{150}{\^u}
\DeclareInputText{151}{\@tabacckludge`u}
\DeclareInputText{152}{\"y}
\DeclareInputText{153}{\"O}
\DeclareInputText{154}{\"U}
%</cp437|cp437de|cp865>
%<*cp437|cp437de>
\DeclareInputText{155}{\textcent}
\DeclareInputText{156}{\pounds}
\DeclareInputText{157}{\textyen}
%</cp437|cp437de>
%<*cp865>
\DeclareInputText{155}{\o} 
\DeclareInputText{156}{\pounds}
\DeclareInputText{157}{\O}
%</cp865>
%<*cp437|cp437de|cp865>
\DeclareInputText{158}{\textpeseta} % Pt
\DeclareInputText{159}{\textflorin}
%    \end{macrocode}
%
%    \begin{macrocode}
\DeclareInputText{160}{\@tabacckludge'a} % 160
\DeclareInputText{161}{\@tabacckludge'\i}
\DeclareInputText{162}{\@tabacckludge'o}
\DeclareInputText{163}{\@tabacckludge'u}
\DeclareInputText{164}{\~n}
\DeclareInputText{165}{\~N}
\DeclareInputText{166}{\textordfeminine}
\DeclareInputText{167}{\textordmasculine}
\DeclareInputText{168}{\textquestiondown}
%\DeclareInputText{169}{} % left upper corner
\DeclareInputMath{170}{\lnot}
\DeclareInputText{171}{\textonehalf}
\DeclareInputText{172}{\textonequarter}
\DeclareInputText{173}{\textexclamdown}
\DeclareInputText{174}{\guillemotleft}
%<cp437|cp437de>\DeclareInputText{175}{\guillemotright}
%<cp865>\DeclareInputText{175}{\textcurrency} 
%\DeclareInputText{176}{\textlightgraybox}
%\DeclareInputText{177}{\textgraybox}
%\DeclareInputText{178}{\textdarkgraybox}
%\DeclareInputMath{179}{} % vertical bar
%\DeclareInputText{180}{} % vertical bar with branch to left
%\DeclareInputText{181}{} % vertical bar with double branch to left
%\DeclareInputText{182}{} % double bar with single branch to left
%\DeclareInputText{183}{} % graphic
%\DeclareInputText{184}{} % graphic
%\DeclareInputMath{185}{} % vertical double bar with branch to left
%\DeclareInputMath{186}{} % vertical double bar
%\DeclareInputText{187}{} % double upper right corner
%\DeclareInputMath{188}{} % double lower right corner
%\DeclareInputText{189}{} % graphic
%\DeclareInputText{190}{} % graphic
%\DeclareInputMath{191}{\ensuremath{\rceil}}
%    \end{macrocode}
%
%    \begin{macrocode}
%\DeclareInputMath{192}{\ensuremath{\lfloor}}
%\DeclareInputText{193}{} % dash with branch up
%\DeclareInputText{194}{} % dash with branch down
%\DeclareInputText{195}{} % vertical bar with branch to right
%\DeclareInputText{196}{} % horizontal bar, not endash
%\DeclareInputText{197}{} % vertical bar crossed with dash
%\DeclareInputText{198}{} % graphic
%\DeclareInputText{199}{} % graphic
%\DeclareInputText{200}{} % double lower left corner
%\DeclareInputText{201}{} % double upper left corner
%\DeclareInputText{202}{} % double dash with branch up
%\DeclareInputText{203}{} % double dash with branch down
%\DeclareInputText{204}{} % double bar with branch right
%\DeclareInputText{205}{=} % double dash
%\DeclareInputText{206}{} % double bar crossing double dash
%\DeclareInputText{207}{} % graphic
%\DeclareInputMath{208}{}
%\DeclareInputText{209}{}
%\DeclareInputText{210}{}
%\DeclareInputText{211}{}
%\DeclareInputText{212}{}
%\DeclareInputText{213}{}
%\DeclareInputText{214}{}
%\DeclareInputText{215}{}
%\DeclareInputText{216}{}
%\DeclareInputMath{217}{\ensuremath{\rfloor}} % lower right corner
%\DeclareInputMath{218}{\ensuremath{\lceil}}  % upper left corner
%\DeclareInputText{219}{} % black box
%\DeclareInputText{220}{} % lower half of black box
%\DeclareInputText{221}{} % left bar 
%\DeclareInputText{222}{} % right bar
%\DeclareInputText{223}{} % upper half of black box
%    \end{macrocode}
%
%    \begin{macrocode}
\DeclareInputMath{224}{\alpha}
%</cp437|cp437de|cp865>
%<cp437|cp865>\DeclareInputMath{225}{\beta}
%<cp437de>\DeclareInputText{225}{\ss}
%<*cp437|cp437de|cp865>
\DeclareInputMath{226}{\Gamma}
\DeclareInputMath{227}{\pi}
\DeclareInputMath{228}{\Sigma}
\DeclareInputMath{229}{\sigma}
\DeclareInputMath{230}{\mu}
\DeclareInputMath{231}{\gamma}
\DeclareInputMath{232}{\Phi}
\DeclareInputMath{233}{\theta}
\DeclareInputMath{234}{\Omega}
\DeclareInputMath{235}{\delta}
\DeclareInputMath{236}{\infty}
\DeclareInputMath{237}{\phi}
\DeclareInputMath{238}{\varepsilon}
\DeclareInputMath{239}{\cap}
\DeclareInputMath{240}{\equiv}
\DeclareInputMath{241}{\pm}
\DeclareInputMath{242}{\geq}
\DeclareInputMath{243}{\leq}
%\DeclareInputMath{244}{}   % upper part of integral sign
%\DeclareInputMath{245}{}   % lower part of integral sign
\DeclareInputMath{246}{\div}
\DeclareInputMath{247}{\approx}  
\DeclareInputText{248}{\textdegree}
\DeclareInputText{249}{\textperiodcentered}
\DeclareInputText{250}{\textbullet}
\DeclareInputMath{251}{\surd}
\DeclareInputMath{252}{\mathnsuperior}
\DeclareInputMath{253}{\mathtwosuperior}
\DeclareInputText{254}{\textblacksquare} % right name?
\DeclareInputText{255}{\nobreakspace}
%</cp437|cp437de|cp865>
%    \end{macrocode}
%
% \section{The Macintosh encodings}
%
% This input encoding was contributed by
% Constantin Kahn (\texttt{kahn@\linebreak[0]math.toronto.edu}),
% with minor modifications by Alan Jeffrey.
%
% \changes{v0.09c}{1995/05/30}{Made mac encoding `active German quote
%    safe', and added the correct docstrip magic.}
%
%    \begin{macrocode}
%<*applemac>
\DeclareInputText{128}{\"A}
\DeclareInputText{129}{\r A}
\DeclareInputText{130}{\c C}
\DeclareInputText{131}{\@tabacckludge'E}
\DeclareInputText{132}{\~N}
\DeclareInputText{133}{\"O}
\DeclareInputText{134}{\"U}
\DeclareInputText{135}{\@tabacckludge'a}
\DeclareInputText{136}{\@tabacckludge`a}
\DeclareInputText{137}{\^a}
\DeclareInputText{138}{\"a}
\DeclareInputText{139}{\~a}
\DeclareInputText{140}{\r a}
\DeclareInputText{141}{\c c}
\DeclareInputText{142}{\@tabacckludge'e}
\DeclareInputText{143}{\@tabacckludge`e}
\DeclareInputText{144}{\^e}
\DeclareInputText{145}{\"e}
\DeclareInputText{146}{\@tabacckludge'\i}
\DeclareInputText{147}{\@tabacckludge`\i}
\DeclareInputText{148}{\^\i}
\DeclareInputText{149}{\"\i}
\DeclareInputText{150}{\~n}
\DeclareInputText{151}{\@tabacckludge'o}
\DeclareInputText{152}{\@tabacckludge`o}
\DeclareInputText{153}{\^o}
\DeclareInputText{154}{\"o}
\DeclareInputText{155}{\~o}
\DeclareInputText{156}{\@tabacckludge'u}
\DeclareInputText{157}{\@tabacckludge`u}
\DeclareInputText{158}{\^u}
\DeclareInputText{159}{\"u}
%    \end{macrocode}
%
%    \begin{macrocode}
\DeclareInputText{160}{\dag}
\DeclareInputText{161}{\textdegree}
\DeclareInputText{162}{\textcent}
\DeclareInputText{163}{\pounds}
\DeclareInputText{164}{\S}
\DeclareInputText{165}{\textbullet}
\DeclareInputText{166}{\P}
\DeclareInputText{167}{\ss}
\DeclareInputText{168}{\textregistered}
\DeclareInputText{169}{\copyright}
\DeclareInputText{170}{\texttrademark}
\DeclareInputText{171}{\@tabacckludge'{}}
\DeclareInputText{172}{\"{}}
\DeclareInputMath{173}{\neq}
\DeclareInputText{174}{\AE}
\DeclareInputText{175}{\O}
\DeclareInputMath{176}{\infty}
\DeclareInputMath{177}{\pm}
\DeclareInputMath{178}{\leq}
\DeclareInputMath{179}{\geq}
\DeclareInputText{180}{\textyen}
\DeclareInputMath{181}{\mu}
\DeclareInputMath{182}{\partial}
\DeclareInputMath{183}{\Sigma}
\DeclareInputMath{184}{\Pi}
\DeclareInputMath{185}{\pi}
\DeclareInputMath{186}{\int}
\DeclareInputText{187}{\textordfeminine}
\DeclareInputText{188}{\textordmasculine}
\DeclareInputMath{189}{\Omega}
\DeclareInputText{190}{\ae}
\DeclareInputText{191}{\o}
%    \end{macrocode}
%
%    \begin{macrocode}
\DeclareInputText{192}{\textquestiondown}
\DeclareInputText{193}{\textexclamdown}
\DeclareInputMath{194}{\lnot}
\DeclareInputMath{195}{\surd}
\DeclareInputText{196}{\textflorin}
\DeclareInputMath{197}{\approx}
\DeclareInputMath{198}{\Delta}
\DeclareInputText{199}{\guillemotleft}
\DeclareInputText{200}{\guillemotright}
\DeclareInputText{201}{\dots}
\DeclareInputText{202}{\nobreakspace}
\DeclareInputText{203}{\@tabacckludge`A}
\DeclareInputText{204}{\~A}
\DeclareInputText{205}{\~O}
\DeclareInputText{206}{\OE}
\DeclareInputText{207}{\oe}
\DeclareInputText{208}{\textendash}
\DeclareInputText{209}{\textemdash}
\DeclareInputText{210}{\textquotedblleft}
\DeclareInputText{211}{\textquotedblright}
\DeclareInputText{212}{\textquoteleft}
\DeclareInputText{213}{\textquoteright}
\DeclareInputMath{214}{\div}
\DeclareInputMath{215}{\diamond}
\DeclareInputText{216}{\"y}
\DeclareInputText{217}{\"Y}
\DeclareInputMath{218}{/}
\DeclareInputText{219}{\textcurrency}
\DeclareInputText{220}{\guilsinglleft}
\DeclareInputText{221}{\guilsinglright}
\DeclareInputText{222}{fi}
\DeclareInputText{223}{fl}
%    \end{macrocode}
%
%    \begin{macrocode}
\DeclareInputText{224}{\ddag}
\DeclareInputText{225}{\textperiodcentered}
\DeclareInputText{226}{\quotesinglbase}
\DeclareInputText{227}{\quotedblbase}
\DeclareInputText{228}{\textperthousand}
\DeclareInputText{229}{\^A}
\DeclareInputText{230}{\^E}
\DeclareInputText{231}{\@tabacckludge'A}
\DeclareInputText{232}{\"E}
\DeclareInputText{233}{\@tabacckludge`E}
\DeclareInputText{234}{\@tabacckludge'I}
\DeclareInputText{235}{\^I}
\DeclareInputText{236}{\"I}
\DeclareInputText{237}{\@tabacckludge`I}
\DeclareInputText{238}{\@tabacckludge'O}
\DeclareInputText{239}{\^O}
\DeclareInputText{240}{\textapplelogo}
\DeclareInputText{241}{\@tabacckludge`O}
\DeclareInputText{242}{\@tabacckludge'U}
\DeclareInputText{243}{\^U}
\DeclareInputText{244}{\@tabacckludge`U}
\DeclareInputText{245}{\i}
\DeclareInputText{246}{\^{}}
\DeclareInputText{247}{\~{}}
\DeclareInputText{248}{\@tabacckludge={}}
\DeclareInputText{249}{\u{}}
\DeclareInputText{250}{\.{}}
\DeclareInputText{251}{\r{}}
\DeclareInputText{252}{\c\ }
\DeclareInputText{253}{\H{}}
\DeclareInputText{254}{\k\ }
\DeclareInputText{255}{\v{}}
%</applemac>
%    \end{macrocode}
%
% This input encoding for the Apple Central European code page was
% contributed by Radek Tryc and Marcin Woli\'nski
%    \verb=<wolinski@mimuw.edu.pl>=.
%
% \changes{v0.99b}{2002/06/16}{Added macce encoding (pr/3433)}
% \changes{v1.0d}{2004/02/05}{Reordered code}
% \changes{v1.0g}{2004/05/22}{Changed \cs{textellipsis} to \cs{dots}
%    for consistency.}
% \changes{v1.0g}{2004/05/22}{Changed \cs{textdagger} to \cs{dag}
%    for consistency.}
% \changes{v1.0g}{2004/05/22}{Changed \cs{textparagraph} to \cs{P}
%    for consistency.}
% \changes{v1.0g}{2004/05/22}{Changed \cs{textsection} to \cs{S}
%    for consistency.}
% \changes{v1.0g}{2004/05/22}{Changed \cs{textcopyright} to \cs{copyright}
%    for consistency.}
%    \begin{macrocode}
%<*applemacce>
\DeclareInputText{128}{\"A}
\DeclareInputText{131}{\@tabacckludge'E}
\DeclareInputText{132}{\k A}
\DeclareInputText{133}{\"O}
\DeclareInputText{134}{\"U}
\DeclareInputText{136}{\k a}
\DeclareInputText{137}{\v C}
\DeclareInputText{138}{\"a}
\DeclareInputText{139}{\v c}
\DeclareInputText{140}{\@tabacckludge'C}
\DeclareInputText{141}{\@tabacckludge'c}
\DeclareInputText{143}{\@tabacckludge'Z}
\DeclareInputText{144}{\@tabacckludge'z}
\DeclareInputText{151}{\@tabacckludge'o}
\DeclareInputText{159}{\"u}
\DeclareInputText{154}{\"o}
\DeclareInputText{133}{\"O}
\DeclareInputText{134}{\"U}
\DeclareInputText{153}{\^o}
%    \end{macrocode}
%
%    \begin{macrocode}
\DeclareInputText{160}{\dag}
\DeclareInputText{161}{\textdegree}
\DeclareInputText{162}{\k E}
\DeclareInputText{163}{\pounds}
\DeclareInputText{164}{\S}
\DeclareInputText{165}{\textbullet}
\DeclareInputText{166}{\P}
\DeclareInputText{167}{\ss}
\DeclareInputText{168}{\textregistered}
\DeclareInputText{171}{\k e}
\DeclareInputText{193}{\@tabacckludge'N}
\DeclareInputText{169}{\copyright}
\DeclareInputText{184}{\l}
\DeclareInputText{196}{\@tabacckludge'n}
\DeclareInputText{199}{\guillemotleft}
\DeclareInputText{200}{\guillemotright}
\DeclareInputText{201}{\dots}
\DeclareInputText{202}{\nobreakspace}
\DeclareInputText{208}{\textendash}
\DeclareInputText{209}{\textemdash}
%    \end{macrocode}
%
% \changes{v1.0e}{2004/05/03}{Typo in \cs{textquotedblleft} (pr/3673)}
%    \begin{macrocode}
\DeclareInputText{210}{\textquotedblleft}
\DeclareInputText{211}{\textquotedblright}
\DeclareInputText{212}{\textquoteleft}
\DeclareInputText{213}{\textquoteright}
\DeclareInputText{214}{\textdiv}
\DeclareInputText{220}{\guilsinglleft}
\DeclareInputText{221}{\guilsinglright}
\DeclareInputText{222}{\v r}
%    \end{macrocode}
%
%    \begin{macrocode}
\DeclareInputText{226}{\quotesinglbase}
\DeclareInputText{227}{\quotedblbase}
\DeclareInputText{229}{\@tabacckludge'S}
\DeclareInputText{230}{\@tabacckludge's}
\DeclareInputText{238}{\@tabacckludge'O}
\DeclareInputText{239}{\^O}
\DeclareInputText{251}{\.Z}
\DeclareInputText{252}{\L}
\DeclareInputText{253}{\.z}
%</applemacce>
%    \end{macrocode}
%
% \section{The Next encoding}
%
% This input encoding is based on work by Stefan Ried
% (\texttt{stef@\linebreak[0]theo-phys.uni-essen.de} and Holger Uhr
% (\texttt{huhr@\linebreak[0]uni-paderborn.de}).
%
% Further extended by
% Jens Heise (\texttt{heisbeee@calvados.zrz.TU-Berlin.DE}).
%
% \changes{v0.9k}{1995/11/29}{Tidied up this encoding.}
%
% \changes{v0.9m}{1995/12/04}{Made fraction a math character.}
% \changes{v0.9n}{1995/12/10}{Made fraction a text character.}
%
% \changes{v0.94}{1997/12/17}{Changed \cs{textellipsis} to \cs{dots}
%    for consistency.}
% \changes{v0.94}{1997/12/17}{Changed \cs{textquotesinglbase} to
%    \cs{quotesinglbase}: this may be temporary.}
% \changes{v0.94}{1997/12/17}{Changed \cs{textquotedblbase} to
%    \cs{quotedblbase}: this may be temporary.}
% \changes{v0.96}{1998/03/02}{Fixed typo in slot 159.}
% \changes{v0.997}{2000/12/02}{Added all missing chars. (pr/3281)}
%    
%    \begin{macrocode}
%<*next>
\DeclareInputText{128}{\nobreakspace}
\DeclareInputText{129}{\@tabacckludge`A}
\DeclareInputText{130}{\@tabacckludge'A}
\DeclareInputText{131}{\^A}
\DeclareInputText{132}{\~A}
\DeclareInputText{133}{\"A}
\DeclareInputText{134}{\r A}
\DeclareInputText{135}{\c C}
\DeclareInputText{136}{\@tabacckludge`E}
\DeclareInputText{137}{\@tabacckludge'E}
\DeclareInputText{138}{\^E}
\DeclareInputText{139}{\"E}
\DeclareInputText{140}{\@tabacckludge`I}
\DeclareInputText{141}{\@tabacckludge'I}
\DeclareInputText{142}{\^I}
\DeclareInputText{143}{\"I}
\DeclareInputText{144}{\DH}
\DeclareInputText{145}{\~N}
\DeclareInputText{146}{\@tabacckludge`O}
\DeclareInputText{147}{\@tabacckludge'O}
\DeclareInputText{148}{\^O}
\DeclareInputText{149}{\~O}
\DeclareInputText{150}{\"O}
\DeclareInputText{151}{\@tabacckludge`U}
\DeclareInputText{152}{\@tabacckludge'U}
\DeclareInputText{153}{\^U}
\DeclareInputText{154}{\"U}
\DeclareInputText{155}{\@tabacckludge'Y}
\DeclareInputText{156}{\TH}
\DeclareInputMath{157}{\mu}
\DeclareInputMath{158}{\times}
\DeclareInputMath{159}{\div}
%    \end{macrocode}
%
%    \begin{macrocode}
\DeclareInputText{160}{\copyright}
\DeclareInputText{161}{\textexclamdown}
\DeclareInputText{162}{\textcent}
\DeclareInputText{163}{\pounds}
\DeclareInputMath{164}{/}
\DeclareInputText{165}{\textyen}
\DeclareInputText{166}{\textflorin}
\DeclareInputText{167}{\S}
\DeclareInputText{168}{\textcurrency}
\DeclareInputText{169}{\textquoteright}
\DeclareInputText{170}{\textquotedblleft}
\DeclareInputText{171}{\guillemotleft}
\DeclareInputText{172}{\guilsinglleft}
\DeclareInputText{173}{\guilsinglright}
\DeclareInputText{174}{fi}
\DeclareInputText{175}{fl}
\DeclareInputText{176}{\textregistered}
\DeclareInputText{177}{\textendash}
\DeclareInputText{178}{\dag}
\DeclareInputText{179}{\ddag}
\DeclareInputText{180}{\textperiodcentered}
\DeclareInputText{181}{\textbrokenbar}
\DeclareInputText{182}{\P}
\DeclareInputText{183}{\textbullet}
\DeclareInputText{184}{\quotesinglbase}
\DeclareInputText{185}{\quotedblbase}
\DeclareInputText{186}{\textquotedblright}
\DeclareInputText{187}{\guillemotright}
\DeclareInputText{188}{\dots}
\DeclareInputText{189}{\textperthousand}
\DeclareInputMath{190}{\lnot}
\DeclareInputText{191}{\textquestiondown}
%    \end{macrocode}
%
%    \begin{macrocode}
\DeclareInputMath{192}{\mathonesuperior}
\DeclareInputText{193}{\@tabacckludge`{}}
\DeclareInputText{194}{\@tabacckludge'{}}
\DeclareInputText{195}{\^{}}
\DeclareInputText{196}{\~{}}
\DeclareInputText{197}{\@tabacckludge={}}
\DeclareInputText{198}{\u{}}
\DeclareInputText{199}{\.{}}
\DeclareInputText{200}{\"{}}
\DeclareInputMath{201}{\mathtwosuperior}
\DeclareInputText{202}{\r{}}
\DeclareInputText{203}{\c\ }
\DeclareInputMath{204}{\maththreesuperior}
\DeclareInputText{205}{\H{}}
\DeclareInputText{206}{\k\ }
\DeclareInputText{207}{\v{}}
\DeclareInputText{208}{\textemdash}
\DeclareInputMath{209}{\pm}
\DeclareInputText{210}{\textonequarter}
\DeclareInputText{211}{\textonehalf}
\DeclareInputText{212}{\textthreequarters}
\DeclareInputText{213}{\@tabacckludge`a}
\DeclareInputText{214}{\@tabacckludge'a}
\DeclareInputText{215}{\^a}
\DeclareInputText{216}{\~a}
\DeclareInputText{217}{\"a}
\DeclareInputText{218}{\r a}
\DeclareInputText{219}{\c c}
\DeclareInputText{220}{\@tabacckludge`e}
\DeclareInputText{221}{\@tabacckludge'e}
\DeclareInputText{222}{\^e}
\DeclareInputText{223}{\"e}
%    \end{macrocode}
%
%    \begin{macrocode}
\DeclareInputText{224}{\@tabacckludge`\i}
\DeclareInputText{225}{\AE}
\DeclareInputText{226}{\@tabacckludge'\i}
\DeclareInputText{227}{\textordfeminine}
\DeclareInputText{228}{\^\i}
\DeclareInputText{229}{\"\i}
\DeclareInputText{230}{\dh}
\DeclareInputText{231}{\~n}
\DeclareInputText{232}{\L}
\DeclareInputText{233}{\O}
\DeclareInputText{234}{\OE}
\DeclareInputText{235}{\textordmasculine}
\DeclareInputText{236}{\@tabacckludge`o}
\DeclareInputText{237}{\@tabacckludge'o}
\DeclareInputText{238}{\^o}
\DeclareInputText{239}{\~o}
\DeclareInputText{240}{\"o}
\DeclareInputText{241}{\ae}
\DeclareInputText{242}{\@tabacckludge`u}
\DeclareInputText{243}{\@tabacckludge'u}
\DeclareInputText{244}{\^u}
\DeclareInputText{245}{\i}
\DeclareInputText{246}{\"u}
\DeclareInputText{247}{\@tabacckludge'y}
\DeclareInputText{248}{\l}
\DeclareInputText{249}{\o}
\DeclareInputText{250}{\oe}
\DeclareInputText{251}{\ss}
\DeclareInputText{252}{\th}
\DeclareInputText{253}{\"y}
%</next>
%    \end{macrocode}
%
%
%
% \changes{v0.9a}{1995/04/23}{\cs{textonequarter} and friends should
%                     be declared with \cs{DeclareInputText}}
% \section{The MS Windows ANSI encoding cp1252,\\
%          and the ISO Latin-1 and Latin-9 encodings}
%
% The MS Windows ANSI, cp 1252, input encoding was contributed by
% Berthold K.P. Horn (\texttt{bkph@\linebreak[0]ai.mit.edu}).
% 
% It has two very different names so the same code produces two files.
% These encoding files both define the characters in the
% MS Windows 3.1 ANSI encoding (Western Europe), also known as code
% page 1252, which is based on ISO Latin-1 but has important additions
% in the 128--159 range.
% 
% Designed for:
%       Danish, Dutch, English, Finnish, French, German, Icelandic,
%       Italian, Norwegian, Portuguese, Spanish, and Swedish.
%
% Note: Windows ANSI --- like Macintosh standard Roman encoding ---
%       has quotesingle at 39, and grave at 96 --- which is here
%       ignored.
%
% The ISO Latin-1 encoding file defines only the restricted range of
% characters available in the ISO~8859-1 encoding.
%
% The ISO~Latin-9 encoding file defines the characters in the
%    ISO~8859-15 encoding. It was contributed by Karsten Tinnefeld
%    (\texttt{karsten@tinnefeld.com}).
%    It differs only a small amount from ISO~Latin-1 and is a
%    replacement for it that contains a few characters that are needed for
%    French and Finnish. Further, a slot for the Euro currency sign has
%    been added and this could be the killer argument for many 8-bit
%    texts to be written in Latin-9 in the future.
%    
% According to a Linux man page, ISO~Latin-9 supports Albanian, Basque,
%    Breton, Catalan, Danish, Dutch,
% English, Estonian, Faroese, Finnish, French, Frisian, Galician, German, 
% Greenlandic, Icelandic, Irish Gaelic, Italian, Latin, Luxemburgish, 
% Norwegian, Portuguese, Rhaeto-Romanic, Scottish Gaelic, Spanish and 
% Swedish.
%
%    The characters added in |latin9.def| are (in \LaTeX{} notation):
% \begin{verbatim}
%     \texteuro, \v S   \v s   \v Z   \v z   \OE   \oe   \" Y
% \end{verbatim}
%    They displace the following characters from |latin1.def|:
% \begin{verbatim}
%       \textcurrency   \textbrokenbar   \"{}   \'{}   \c{}
%       \textonequarter   \textonehalf   \textthreequarters
% \end{verbatim}
% 
% \changes{v0.9e}{1995/08/31}{Redeclared "AD to be soft hyphen.}
% \changes{v0.9e}{1995/08/31}{Swapped ordfeminine and ordmasculine.}
%
% \changes{v0.9p}{1996/04/11}{ansinew 09f is \"Y not \"y, latex/2119}
%
% \changes{v0.91}{1997/08/19}{Exchanged codes for
%          \cs{textendash} and \cs{textemdash}.}
%                  
% \changes{v0.94}{1997/12/17}{Added cp1252 and merged latin1}
% \changes{v0.94}{1997/12/17}{Changed \cs{ldots} to \cs{dots}, this
%          should be undetectable since \cs{ldots} is not robust.}
%
% \changes{v0.99a}{2001/07/10}{Added latin9 (from Karsten Tinnefeld)}
% \changes{v0.99b}{2002/06/16}{Added code points 142,158 for cp1252 (pr/3441)}
% \changes{v0.99b}{2002/07/08}{Added code 128 (texteuro) for cp1252 (pr/3423)}
%                  
%    \begin{macrocode}
%<*cp1252>
\DeclareInputText{128}{\texteuro}
\DeclareInputText{130}{\quotesinglbase}
\DeclareInputText{131}{\textflorin}
\DeclareInputText{132}{\quotedblbase}
\DeclareInputText{133}{\dots}
\DeclareInputText{134}{\dag}
\DeclareInputText{135}{\ddag}
\DeclareInputText{136}{\^{}}
\DeclareInputText{137}{\textperthousand}
\DeclareInputText{138}{\v S}
\DeclareInputText{139}{\guilsinglleft}
\DeclareInputText{140}{\OE}
\DeclareInputText{142}{\v Z}
\DeclareInputText{145}{\textquoteleft}
\DeclareInputText{146}{\textquoteright}
\DeclareInputText{147}{\textquotedblleft}
\DeclareInputText{148}{\textquotedblright}
\DeclareInputText{149}{\textbullet}
\DeclareInputText{150}{\textendash}
\DeclareInputText{151}{\textemdash}
\DeclareInputText{152}{\~{}}
\DeclareInputText{153}{\texttrademark}
\DeclareInputText{154}{\v s}
\DeclareInputText{155}{\guilsinglright}
\DeclareInputText{156}{\oe}
\DeclareInputText{158}{\v z}
\DeclareInputText{159}{\"Y}
%</cp1252>
%    \end{macrocode}
%
%  This (somewhat confused) table is now even more disordered: first
%  we deal with those few characters that are different in latin9,
%  then with the rest.
%
%    \begin{macrocode}
%<*cp1252|latin1>
\DeclareInputText{164}{\textcurrency}
\DeclareInputText{166}{\textbrokenbar}
\DeclareInputText{168}{\"{}}
\DeclareInputText{180}{\@tabacckludge'{}}
\DeclareInputText{184}{\c\ }
\DeclareInputText{188}{\textonequarter}
\DeclareInputText{189}{\textonehalf}
\DeclareInputText{190}{\textthreequarters}
%</cp1252|latin1>
%    \end{macrocode}
%
%    \begin{macrocode}
%<*latin9>
\DeclareInputText{164}{\texteuro}
\DeclareInputText{166}{\v S}
\DeclareInputText{168}{\v s}
\DeclareInputText{180}{\v Z}
\DeclareInputText{184}{\v z}
\DeclareInputText{188}{\OE}
\DeclareInputText{189}{\oe}
\DeclareInputText{190}{\"Y}
%</latin9>
%    \end{macrocode}
%
%    \begin{macrocode}
%<*cp1252|latin1|latin9>
\DeclareInputText{160}{\nobreakspace}
\DeclareInputText{176}{\textdegree}
\DeclareInputText{161}{\textexclamdown}
\DeclareInputMath{177}{\pm}
\DeclareInputText{162}{\textcent}
\DeclareInputMath{178}{\mathtwosuperior}
\DeclareInputText{163}{\pounds}
\DeclareInputMath{179}{\maththreesuperior}
\DeclareInputText{165}{\textyen}
\DeclareInputMath{181}{\mu}
\DeclareInputText{182}{\P}
\DeclareInputText{167}{\S}
\DeclareInputText{183}{\textperiodcentered}
\DeclareInputText{169}{\copyright}
\DeclareInputMath{185}{\mathonesuperior}
\DeclareInputText{170}{\textordfeminine}
\DeclareInputText{186}{\textordmasculine}
\DeclareInputText{171}{\guillemotleft}
\DeclareInputText{187}{\guillemotright}
\DeclareInputMath{172}{\lnot}
\DeclareInputText{173}{\-}
\DeclareInputText{174}{\textregistered}
\DeclareInputText{175}{\@tabacckludge={}}
\DeclareInputText{191}{\textquestiondown}
%    \end{macrocode}
%
%    \begin{macrocode}
\DeclareInputText{192}{\@tabacckludge`A}
\DeclareInputText{208}{\DH}
\DeclareInputText{193}{\@tabacckludge'A}
\DeclareInputText{209}{\~N}
\DeclareInputText{194}{\^A}
\DeclareInputText{210}{\@tabacckludge`O}
\DeclareInputText{195}{\~A}
\DeclareInputText{211}{\@tabacckludge'O}
\DeclareInputText{196}{\"A}
\DeclareInputText{212}{\^O}
\DeclareInputText{197}{\r A}
\DeclareInputText{213}{\~O}
\DeclareInputText{198}{\AE}
\DeclareInputText{214}{\"O}
\DeclareInputText{199}{\c C}
\DeclareInputMath{215}{\times}
\DeclareInputText{200}{\@tabacckludge`E}
\DeclareInputText{216}{\O}
\DeclareInputText{201}{\@tabacckludge'E}
\DeclareInputText{217}{\@tabacckludge`U}
\DeclareInputText{202}{\^E}
\DeclareInputText{218}{\@tabacckludge'U}
\DeclareInputText{203}{\"E}
\DeclareInputText{219}{\^U}
\DeclareInputText{204}{\@tabacckludge`I}
\DeclareInputText{220}{\"U}
\DeclareInputText{205}{\@tabacckludge'I}
\DeclareInputText{221}{\@tabacckludge'Y}
\DeclareInputText{206}{\^I}
\DeclareInputText{222}{\TH}
\DeclareInputText{207}{\"I}
\DeclareInputText{223}{\ss}
%    \end{macrocode}
%
%    \begin{macrocode}
\DeclareInputText{224}{\@tabacckludge`a}
\DeclareInputText{240}{\dh}
\DeclareInputText{225}{\@tabacckludge'a}
\DeclareInputText{241}{\~n}
\DeclareInputText{226}{\^a}
\DeclareInputText{242}{\@tabacckludge`o}
\DeclareInputText{227}{\~a}
\DeclareInputText{243}{\@tabacckludge'o}
\DeclareInputText{228}{\"a}
\DeclareInputText{244}{\^o}
\DeclareInputText{229}{\r a}
\DeclareInputText{245}{\~o}
\DeclareInputText{230}{\ae}
\DeclareInputText{246}{\"o}
\DeclareInputText{231}{\c c}
\DeclareInputMath{247}{\div}
\DeclareInputText{232}{\@tabacckludge`e}
\DeclareInputText{248}{\o}
\DeclareInputText{233}{\@tabacckludge'e}
\DeclareInputText{249}{\@tabacckludge`u}
\DeclareInputText{234}{\^e}
\DeclareInputText{250}{\@tabacckludge'u}
\DeclareInputText{235}{\"e}
\DeclareInputText{251}{\^u}
\DeclareInputText{236}{\@tabacckludge`\i}
\DeclareInputText{252}{\"u}
\DeclareInputText{237}{\@tabacckludge'\i}
\DeclareInputText{253}{\@tabacckludge'y}
\DeclareInputText{238}{\^\i}
\DeclareInputText{254}{\th}
\DeclareInputText{239}{\"\i}
\DeclareInputText{255}{\"y}
%</cp1252|latin1|latin9>
%    \end{macrocode}
%
%    
%\section{The ISO 8859-16 (Latin10) encoding}
%
% \changes{v1.0c}{2004/02/04}{Added ISO 8859-16 Latin10 (pr/3568)}
%
%  This set of coded graphic characters is intended for use in data and
%  text processing applications and also for information interchange. The
%  set contains graphic characters used for general purpose applications in
%  typical office environments in at least the following languages:
%  Albanian, Croatian, English, Finnish, French, German, Hungarian, Irish
%  Gaelic (new orthography), Italian, Latin, Polish, Romanian, and
%  Slovenian. This set of coded graphic characters may be regarded as a
%  version of an 8-bit code according to ISO/IEC 2022 or ISO/IEC 4873 at
%  level 1. [ISO 8859-16:2001(E), p. 1]
%
%  ISO 8859-16 was primarily designed for single-byte encoding the Romanian
%  language. The UTF-8 charset is the preferred and in today's MIME software
%  more widely implemented encoding suitable for Romanian.
%
% Data for the \LaTeX{} support was mainly provided by  Ionel Ciob\^{i}c\u{a}
% with additions and corrections taken from\\ 
% \texttt{http://www.unicode.org/Public/MAPPINGS/ISO8859/8859-16.TXT}.
%
% \changes{v1.1b}{2006/03/03}{Corrections (pr/3849)}
%
%    \begin{macrocode}
%<*latin10>
%    \end{macrocode}
%    The ``comma below'' accent is provided here in a crude (better
%    than nothing) version.
%    \begin{macrocode}
\ProvideTextCommandDefault\textcommabelow[1]
  {\hmode@bgroup\ooalign{\null#1\crcr\hidewidth
     \raise-.31ex\hbox{\check@mathfonts
%    \end{macrocode}
%     Use |\sf@size| instead of |\ssf@size| if the comma looks too small:
%    \begin{macrocode}
                       \fontsize\ssf@size\z@
                       \math@fontsfalse\selectfont,}\hidewidth}\egroup}
%    \end{macrocode}
%    
%    \begin{macrocode}
\ProvideTextCommandDefault\textpm{\ensuremath\pm}
%    \end{macrocode}
%    But why only for this one encoding? (Answer: because it is a new encoding:
%    it contains only LICR objects --- Frank)
% \changes{v1.0g}{2004/05/22}{Changed \cs{textpm} to \cs{pm}}
% \changes{v1.1b}{2006/03/03}{Reverted back to \cs{textpm}
%    for consistency.}
%    \begin{macrocode}
\DeclareInputText{160}{\nobreakspace}
\DeclareInputText{161}{\k A}
\DeclareInputText{162}{\k a}
\DeclareInputText{163}{\L}
\DeclareInputText{164}{\texteuro}
\DeclareInputText{165}{\quotedblbase}
\DeclareInputText{166}{\v S}
\DeclareInputText{167}{\S}
\DeclareInputText{168}{\v s}
\DeclareInputText{169}{\textcopyright}
\DeclareInputText{170}{\textcommabelow S}
\DeclareInputText{171}{\guillemotleft}
\DeclareInputText{172}{\@tabacckludge'Z}
\DeclareInputText{173}{\-}
\DeclareInputText{174}{\@tabacckludge'z}
\DeclareInputText{175}{\.Z}
\DeclareInputText{176}{\textdegree}
\DeclareInputText{177}{\textpm}
\DeclareInputText{178}{\v C}
\DeclareInputText{179}{\l}
\DeclareInputText{180}{\v Z}
\DeclareInputText{181}{\textquotedblright}
\DeclareInputText{182}{\P}
\DeclareInputText{183}{\textperiodcentered}
\DeclareInputText{184}{\v z}
\DeclareInputText{185}{\v c}
\DeclareInputText{186}{\textcommabelow s}
\DeclareInputText{187}{\guillemotright}
\DeclareInputText{188}{\OE}
\DeclareInputText{189}{\oe}
\DeclareInputText{190}{\"Y}
\DeclareInputText{191}{\.z}
%    \end{macrocode}
%
%    \begin{macrocode}
\DeclareInputText{192}{\@tabacckludge`A}
\DeclareInputText{193}{\@tabacckludge'A}
\DeclareInputText{194}{\^A}
\DeclareInputText{195}{\u A}
\DeclareInputText{196}{\"A}
\DeclareInputText{197}{\@tabacckludge'C}
\DeclareInputText{198}{\AE}
\DeclareInputText{199}{\c C}
\DeclareInputText{200}{\@tabacckludge`E}
\DeclareInputText{201}{\@tabacckludge'E}
\DeclareInputText{202}{\^E}
\DeclareInputText{203}{\"E}
\DeclareInputText{204}{\@tabacckludge`I}
\DeclareInputText{205}{\@tabacckludge'I}
\DeclareInputText{206}{\^I}
\DeclareInputText{207}{\"I}
\DeclareInputText{208}{\DJ}
\DeclareInputText{209}{\@tabacckludge'N}
\DeclareInputText{210}{\@tabacckludge`O}
\DeclareInputText{211}{\@tabacckludge'O}
\DeclareInputText{212}{\^O}
\DeclareInputText{213}{\H O}
\DeclareInputText{214}{\"O}
\DeclareInputText{215}{\@tabacckludge'S}
\DeclareInputText{216}{\H U}
\DeclareInputText{217}{\@tabacckludge`U}
\DeclareInputText{218}{\@tabacckludge'U}
\DeclareInputText{219}{\^U}
\DeclareInputText{220}{\"U}
\DeclareInputText{221}{\k E}
\DeclareInputText{222}{\textcommabelow T}
%    \end{macrocode}
%
%    \begin{macrocode}
\DeclareInputText{223}{\ss}
\DeclareInputText{224}{\@tabacckludge`a}
\DeclareInputText{225}{\@tabacckludge'a}
\DeclareInputText{226}{\^a}
\DeclareInputText{227}{\u a}
\DeclareInputText{228}{\"a}
\DeclareInputText{229}{\@tabacckludge'c}
\DeclareInputText{230}{\ae}
\DeclareInputText{231}{\c c}
\DeclareInputText{232}{\@tabacckludge`e}
\DeclareInputText{233}{\@tabacckludge'e}
\DeclareInputText{234}{\^e}
\DeclareInputText{235}{\"e}
\DeclareInputText{236}{\@tabacckludge`\i}
\DeclareInputText{237}{\@tabacckludge'\i}
\DeclareInputText{238}{\^\i}
\DeclareInputText{239}{\"\i}
\DeclareInputText{240}{\dj}
\DeclareInputText{241}{\@tabacckludge'n}
\DeclareInputText{242}{\@tabacckludge`o}
\DeclareInputText{243}{\@tabacckludge'o}
\DeclareInputText{244}{\^o}
\DeclareInputText{245}{\H o}
\DeclareInputText{246}{\"o}
\DeclareInputText{247}{\@tabacckludge's}
\DeclareInputText{248}{\H u}
\DeclareInputText{249}{\@tabacckludge`u}
\DeclareInputText{250}{\@tabacckludge'u}
\DeclareInputText{251}{\^u}
\DeclareInputText{252}{\"u}
\DeclareInputText{253}{\k e}
\DeclareInputText{254}{\textcommabelow t}
\DeclareInputText{255}{\"y}
%</latin10>
%    \end{macrocode}
%
%\section{The MS Windows encoding cp1250}
%
%    This is an MS Windows encoding for Central and Eastern Europe,
%    known as Code Page 1250; it was provided by Marcin Woli\'nski 
%    \texttt{wolinski@melkor.mimuw.edu.pl} and edited by Chris Rowley
%    (see v0.94 below) to make it consistent with other files.
%
% \changes{v0.94}{1997/12/17}{Changed \cs{textellipsis} to \cs{dots}
%    for consistency.}
% \changes{v0.94}{1997/12/17}{Changed \cs{textdagger} to \cs{dag}
%    for consistency.}
% \changes{v0.94}{1997/12/17}{Changed \cs{textparagraph} to \cs{P}
%    for consistency.}
% \changes{v0.94}{1997/12/17}{Changed \cs{textsection} to \cs{S}
%    for consistency.}
% \changes{v0.99b}{2002/07/28}{Added code 128 (texteuro) for cp1250}
%    \begin{macrocode}
%<*cp1250>
\DeclareInputText{128}{\texteuro}
%\DeclareInputText{129}{} % n/u
\DeclareInputText{130}{\quotesinglbase}
%\DeclareInputText{131}{} % n/u
\DeclareInputText{132}{\quotedblbase} 
\DeclareInputText{133}{\dots}
\DeclareInputText{134}{\dag}
\DeclareInputText{135}{\ddag}
%\DeclareInputText{136}{} % n/u
\DeclareInputText{137}{\textperthousand}
\DeclareInputText{138}{\v S}
\DeclareInputText{139}{\guilsinglleft}
\DeclareInputText{140}{\@tabacckludge'S}
\DeclareInputText{141}{\v T}
\DeclareInputText{142}{\v Z}
\DeclareInputText{143}{\@tabacckludge'Z}
%\DeclareInputText{144}{} % n/u
\DeclareInputText{145}{\textquoteleft}
\DeclareInputText{146}{\textquoteright}
\DeclareInputText{147}{\textquotedblleft}
\DeclareInputText{148}{\textquotedblright}
\DeclareInputText{149}{\textbullet}
\DeclareInputText{150}{\textendash}
\DeclareInputText{151}{\textemdash}
%\DeclareInputText{152}{} % n/u
\DeclareInputText{153}{\texttrademark}
\DeclareInputText{154}{\v s}
\DeclareInputText{155}{\guilsinglright}
\DeclareInputText{156}{\@tabacckludge's}
\DeclareInputText{157}{\v t} % t caron (t') ?
\DeclareInputText{158}{\v z}
\DeclareInputText{159}{\@tabacckludge'z}
%    \end{macrocode}
%
%    \begin{macrocode}
\DeclareInputText{160}{\nobreakspace}
\DeclareInputText{161}{\v{}}
\DeclareInputText{162}{\u{}}
\DeclareInputText{163}{\L}
\DeclareInputText{164}{\textcurrency}
\DeclareInputText{165}{\k A}
\DeclareInputText{166}{\textbrokenbar}
\DeclareInputText{167}{\S}
\DeclareInputText{168}{\"{}}
\DeclareInputText{169}{\copyright}
\DeclareInputText{170}{\c S}
\DeclareInputText{171}{\guillemotleft}
\DeclareInputMath{172}{\lnot}
\DeclareInputText{173}{\-}% soft hyphen
\DeclareInputText{174}{\textregistered}
\DeclareInputText{175}{\.Z}
\DeclareInputText{176}{\textdegree}
\DeclareInputMath{177}{\pm}% plus-minus
\DeclareInputText{178}{\k\ }
\DeclareInputText{179}{\l}
\DeclareInputText{180}{\@tabacckludge'{}}
\DeclareInputMath{181}{\mu}% micro sign
\DeclareInputText{182}{\P}
\DeclareInputText{183}{\textperiodcentered}
\DeclareInputText{184}{\c\ }
\DeclareInputText{185}{\k a}
\DeclareInputText{186}{\c s}
\DeclareInputText{187}{\guillemotright}
\DeclareInputText{188}{\v L}% L caron (L') ?
\DeclareInputText{189}{\H{}}
\DeclareInputText{190}{\v l}% l caron (l') ? 
\DeclareInputText{191}{\.z}
%    \end{macrocode}
%
%    \begin{macrocode}
\DeclareInputText{192}{\@tabacckludge'R}
\DeclareInputText{193}{\@tabacckludge'A}
\DeclareInputText{194}{\^A}
\DeclareInputText{195}{\u A}
\DeclareInputText{196}{\"A}
\DeclareInputText{197}{\@tabacckludge'L}
\DeclareInputText{198}{\@tabacckludge'C}
\DeclareInputText{199}{\c C}
\DeclareInputText{200}{\v C}
\DeclareInputText{201}{\@tabacckludge'E}
\DeclareInputText{202}{\k E}
\DeclareInputText{203}{\"E}
\DeclareInputText{204}{\v E}
\DeclareInputText{205}{\@tabacckludge'I}
\DeclareInputText{206}{\^I}
\DeclareInputText{207}{\v D}
\DeclareInputText{208}{\DJ} % D stroke
\DeclareInputText{209}{\@tabacckludge'N}
\DeclareInputText{210}{\v N}
\DeclareInputText{211}{\@tabacckludge'O}
\DeclareInputText{212}{\^O}
\DeclareInputText{213}{\H O}
\DeclareInputText{214}{\"O}
\DeclareInputMath{215}{\times}
\DeclareInputText{216}{\v R}
\DeclareInputText{217}{\r U}
\DeclareInputText{218}{\@tabacckludge'U}
\DeclareInputText{219}{\H U}
\DeclareInputText{220}{\"U}
\DeclareInputText{221}{\@tabacckludge'Y}
\DeclareInputText{222}{\c T}
\DeclareInputText{223}{\ss}
%    \end{macrocode}
%
%    \begin{macrocode}
\DeclareInputText{224}{\@tabacckludge'r}
\DeclareInputText{225}{\@tabacckludge'a}
\DeclareInputText{226}{\^a}
\DeclareInputText{227}{\u a}
\DeclareInputText{228}{\"a}
\DeclareInputText{229}{\@tabacckludge'l}
\DeclareInputText{230}{\@tabacckludge'c}
\DeclareInputText{231}{\c c}
\DeclareInputText{232}{\v c}
\DeclareInputText{233}{\@tabacckludge'e}
\DeclareInputText{234}{\k e}
\DeclareInputText{235}{\"e}
\DeclareInputText{236}{\v e}
\DeclareInputText{237}{\@tabacckludge'\i}
\DeclareInputText{238}{\^\i}
\DeclareInputText{239}{\v d} % d caron (d') ?
\DeclareInputText{240}{\dj} % d stroke
\DeclareInputText{241}{\@tabacckludge'n}
\DeclareInputText{242}{\v n}
\DeclareInputText{243}{\@tabacckludge'o}
\DeclareInputText{244}{\^o}
\DeclareInputText{245}{\H o}
\DeclareInputText{246}{\"o}
\DeclareInputMath{247}{\div}
\DeclareInputText{248}{\v r}
\DeclareInputText{249}{\r u}
\DeclareInputText{250}{\@tabacckludge'u}
\DeclareInputText{251}{\H u}
\DeclareInputText{252}{\"u}
\DeclareInputText{253}{\@tabacckludge'y}
\DeclareInputText{254}{\c t}
\DeclareInputText{255}{\.{}}
%</cp1250>
%    \end{macrocode}
%
%
%\section{The MS Windows encoding cp1257}
%
%    This is an MS Windows encoding for Baltic languages,
%    known as Code Page 1257; provided by Heiko Oberdiek
%    using the mappings to Unicode in
%    \texttt{http://www.unicode.org/Public/MAPPINGS/VENDORS/MICSFT/WINDOWS/CP1257.TXT}
%    and
%    \texttt{http://www.microsoft.com/globaldev/reference/sbcs/1257.mspx}.
%
% \changes{v1.1b}{2006/03/03}{Large number of corrections (pr/3849)}
%    \begin{macrocode}
%<*cp1257>
\DeclareInputText{128}{\texteuro}
\DeclareInputText{130}{\quotesinglbase}
\DeclareInputText{132}{\quotedblbase}
\DeclareInputText{133}{\dots}
\DeclareInputText{134}{\dag}
\DeclareInputText{135}{\ddag}
\DeclareInputText{137}{\textperthousand}
\DeclareInputText{139}{\guilsinglleft}
\DeclareInputText{141}{\"{}}
\DeclareInputText{142}{\v{}}
\DeclareInputText{143}{\c\ }
\DeclareInputText{145}{\textquoteleft}
\DeclareInputText{146}{\textquoteright}
\DeclareInputText{147}{\textquotedblleft}
\DeclareInputText{148}{\textquotedblright}
\DeclareInputText{149}{\textbullet}
\DeclareInputText{150}{\textendash}
\DeclareInputText{151}{\textemdash}
\DeclareInputText{153}{\texttrademark}
\DeclareInputText{155}{\guilsinglright}
\DeclareInputText{157}{\@tabacckludge={}}
\DeclareInputText{158}{\k\ }
%    \end{macrocode}
%
%    \begin{macrocode}
\DeclareInputText{160}{\nobreakspace}
\DeclareInputText{162}{\textcent}
\DeclareInputText{163}{\pounds}
\DeclareInputText{164}{\textcurrency}
\DeclareInputText{166}{\textbrokenbar}
\DeclareInputText{167}{\S}
\DeclareInputText{168}{\O}
\DeclareInputText{169}{\copyright}
\DeclareInputText{170}{\c R}
\DeclareInputText{171}{\guillemotleft}
\DeclareInputMath{172}{\lnot}
\DeclareInputText{173}{\-}
\DeclareInputText{174}{\textregistered}
\DeclareInputText{175}{\AE}
\DeclareInputText{176}{\textdegree}
\DeclareInputMath{177}{\pm}
\DeclareInputMath{178}{\mathtwosuperior}
\DeclareInputMath{179}{\maththreesuperior}
\DeclareInputText{180}{\@tabacckludge'{}}
\DeclareInputMath{181}{\mu}
\DeclareInputText{182}{\P}
\DeclareInputText{183}{\textperiodcentered}
\DeclareInputText{184}{\o}
\DeclareInputMath{185}{\mathonesuperior}
\DeclareInputText{186}{\c r}
\DeclareInputText{187}{\guillemotright}
\DeclareInputText{188}{\textonequarter}
\DeclareInputText{189}{\textonehalf}
\DeclareInputText{190}{\textthreequarters}
\DeclareInputText{191}{\ae}
%    \end{macrocode}
%
%    \begin{macrocode}
\DeclareInputText{192}{\k A}
\DeclareInputText{193}{\k I}
\DeclareInputText{194}{\@tabacckludge=A}
\DeclareInputText{195}{\@tabacckludge'C}
\DeclareInputText{196}{\"A}
\DeclareInputText{197}{\r A}
\DeclareInputText{198}{\k E}
\DeclareInputText{199}{\@tabacckludge=E}
\DeclareInputText{200}{\v C}
\DeclareInputText{201}{\@tabacckludge'E}
\DeclareInputText{202}{\@tabacckludge'Z}
\DeclareInputText{203}{\.{E}}
\DeclareInputText{204}{\c G}
\DeclareInputText{205}{\c K}
\DeclareInputText{206}{\@tabacckludge=I}
\DeclareInputText{207}{\c L}
\DeclareInputText{208}{\v S}
\DeclareInputText{209}{\@tabacckludge'N}
\DeclareInputText{210}{\c N}
\DeclareInputText{211}{\@tabacckludge'O}
\DeclareInputText{212}{\@tabacckludge=O}
\DeclareInputText{213}{\~O}
\DeclareInputText{214}{\"O}
\DeclareInputMath{215}{\times}
\DeclareInputText{216}{\k U}
\DeclareInputText{217}{\L}
\DeclareInputText{218}{\@tabacckludge'S}
\DeclareInputText{219}{\@tabacckludge=U}
\DeclareInputText{220}{\"U}
\DeclareInputText{221}{\.Z}
\DeclareInputText{222}{\v Z}
\DeclareInputText{223}{\ss}
%    \end{macrocode}
%
%    \begin{macrocode}
\DeclareInputText{224}{\k a}
\DeclareInputText{225}{\k i}
\DeclareInputText{226}{\@tabacckludge=a}
\DeclareInputText{227}{\@tabacckludge'c}
\DeclareInputText{228}{\"a}
\DeclareInputText{229}{\r a}
\DeclareInputText{230}{\k e}
\DeclareInputText{231}{\@tabacckludge=e}
\DeclareInputText{232}{\v c}
\DeclareInputText{233}{\@tabacckludge'e}
\DeclareInputText{234}{\@tabacckludge'z}
\DeclareInputText{235}{\.{e}}
\DeclareInputText{236}{\c g}
\DeclareInputText{237}{\c k}
\DeclareInputText{238}{\@tabacckludge=\i}
\DeclareInputText{239}{\c l}
\DeclareInputText{240}{\v s}
\DeclareInputText{241}{\@tabacckludge'n}
\DeclareInputText{242}{\c n}
\DeclareInputText{243}{\@tabacckludge'o}
\DeclareInputText{244}{\@tabacckludge=o}
\DeclareInputText{245}{\~o}
\DeclareInputText{246}{\"o}
\DeclareInputMath{247}{\div}
\DeclareInputText{248}{\k u}
\DeclareInputText{249}{\l}
\DeclareInputText{250}{\@tabacckludge's}
\DeclareInputText{251}{\@tabacckludge=u}
\DeclareInputText{252}{\"u}
\DeclareInputText{253}{\.z}
\DeclareInputText{254}{\v z}
\DeclareInputText{255}{\.{}}
%</cp1257>
%    \end{macrocode}
%
% \DeleteShortVerb{\|}
% \Finale
\endinput

% \iffalse meta-comment
%
% Copyright 1993-2017
% The LaTeX3 Project and any individual authors listed elsewhere
% in this file.
%
% This file is part of the LaTeX base system.
% -------------------------------------------
%
% It may be distributed and/or modified under the
% conditions of the LaTeX Project Public License, either version 1.3c
% of this license or (at your option) any later version.
% The latest version of this license is in
%    https://www.latex-project.org/lppl.txt
% and version 1.3c or later is part of all distributions of LaTeX
% version 2005/12/01 or later.
%
% This file has the LPPL maintenance status "maintained".
%
% The list of all files belonging to the LaTeX base distribution is
% given in the file `manifest.txt'. See also `legal.txt' for additional
% information.
%
% The list of derived (unpacked) files belonging to the distribution
% and covered by LPPL is defined by the unpacking scripts (with
% extension .ins) which are part of the distribution.
%
% \fi
% \iffalse
%
%<class>\NeedsTeXFormat{LaTeX2e}
%<class>\ProvidesClass{ltxdoc}
%<class>         [2017/11/16 v2.0w Standard LaTeX documentation class]
%
%<*driver>
\documentclass{ltxdoc}
\GetFileInfo{ltxdoc.cls}
\providecommand\dst{\expandafter{\normalfont\scshape docstrip}}
\title{The file \texttt{ltxdoc.dtx} for use with
      \LaTeXe.\thanks{This file has version
           number \fileversion, dated \filedate.}\\[2pt]
      It contains the code for \texttt{ltxdoc.cls}}
\date{\filedate}
\author{David Carlisle}
\begin{document}
 \MaintainedByLaTeXTeam{latex}
 \maketitle
 \DocInput{ltxdoc.dtx}
\end{document}
%</driver>
%
% \fi
%
%
% \changes{v2.0i}{1994/04/29}{Update the documentation.}
% \changes{v2.0s}{1998/08/17}{(RmS) Documentation fixes.}
%
% \section{Documentation of the \LaTeX\ sources}
%
% This class file is designed for documenting the \LaTeX\ source files.
% You may however find it generally useful as a class for typesetting
% the documentation of files  produced in `doc' format.
%
% Each documented file in the standard distribution comes with extension
% |dtx|. The appropriate class package or initex file will be extracted
% from the source by the docstrip system. Each |dtx| file may be
% directly processed with \LaTeXe, for example
%\begin{verbatim}
% latex2e docclass.dtx
%\end{verbatim}
% would produce the documentation of the Class and package interface.
%
% Each file that is used in producing the \LaTeXe\ format (ie not
% including the standard class and packages) will be printed together in
% one document if you \LaTeX\ the file |sources2e.tex|. This has the
% advantage that one can produce a full index of macro usage across all
% the source files.
%
% If you need to customise the typesetting of any of these files, there
% are two options:
% \begin{itemize}
% \item You can use \dst\ with the module `driver' to extract a small
% \LaTeX\ file that you may edit to use whatever class or package
% options you require, before inputting the source file.
% \item You can create a file |ltxdoc.cfg|. This configuration file will
% be read whenever the |ltxdoc| class is used, and so can be used to
% customise the typesetting of all the source files, without having to
% edit lots of small driver files.
%\end{itemize}
%
% The second option is usually more convenient. Various possibilities
% are discussed in the next section.
%
% \section{Customisation}
%
% The simplest form of customisation is to pass more options to the
% |article| class which is loaded by |ltxdoc|. For instance if you wish
% all the documentation to be formatted for A4 paper, add the following
% line to |ltxdoc.cfg|:
%\begin{verbatim}
% \PassOptionsToClass{a4paper}{article}
%\end{verbatim}
%
% All the source files are in two parts, separated by |\StopEventually|.
% The first part (should) contain `user' documentation. The second part
% is a full documented listing of the source code. The |doc| package
% provides the command |\OnlyDescription| which suppresses the code
% listings. This may also be used in the configuration file, but as the
% |doc| package is read later, you must delay the execution of
% |\OnlyDescription| until after the |doc| package has been read. The
% simplest way is to use |\AtBeginDocument|. Thus you could put the
% following in your |ltxdoc.cfg|.
%\begin{verbatim}
% \AtBeginDocument{\OnlyDescription}
%\end{verbatim}
%
%
% If the full source listing |sources2e.tex| is processed, then an index
% and change history are produced by default, however indices are not
% normally produced for individual files.
%
% As an example, consider |ltclass.dtx|, which contains the sources for
% the new class and package interface commands.  With no |cfg|
% file, a 19~page document is produced. With the above configuration
% a slightly more readable document (4~pages) is produced.
%
% Conversely, if you really want to read the source listings in detail,
% you will want to have an index. Again the index commands provided by
% the doc package may be used, but their execution must be delayed.
%\begin{verbatim}
% \AtBeginDocument{\CodelineIndex\EnableCrossrefs}
% \AtEndDocument{\PrintIndex}
%\end{verbatim}
%
% The |doc| package writes index files to be sorted using MakeIndex with
% the |gind| style, so one would then use a command such as
%\begin{verbatim}
% makeindex -s gind.ist ltclass.idx
%\end{verbatim}
% and re-run \LaTeX.
%
% Similarly to print a Change history, you would add
%\begin{verbatim}
% \AtBeginDocument{\RecordChanges}
% \AtEndDocument{\PrintChanges}
%\end{verbatim}
% to |ltxdoc.cfg|, and use  MakeIndex with a command such as
%\begin{verbatim}
% makeindex -s gglo.ist -o ltclass.gls ltclass.glo
%\end{verbatim}
%
% Finally if you do not want to list all the sections of |source2e.tex|,
% you can use  |\includeonly| in the |cfg| file:
%\begin{verbatim}
% \includeonly{ltvers,ltboxes}
%\end{verbatim}
%
% \StopEventually{}
%
%
% \section{Options}
%
%    \begin{macrocode}
%<*class>
\DeclareOption{a5paper}{\@latexerr{Option not supported}%
   {}}
%    \end{macrocode}
%
%    \begin{macrocode}
\DeclareOption*{%
    \PassOptionsToClass  {\CurrentOption}{article}}
%    \end{macrocode}
%
% \section{Configuration}
% Input a local configuration file, if it exists.
%    \begin{macrocode}
\InputIfFileExists{ltxdoc.cfg}
           {\typeout{*************************************^^J%
                     * Local config file ltxdoc.cfg used^^J%
                     *************************************}}
           {}
%    \end{macrocode}
%
%
% \section{Option Processing}
%
%    \begin{macrocode}
\ProcessOptions
%    \end{macrocode}
%
% \section{Loading article and doc}
%
%    \begin{macrocode}
\LoadClass{article}
%    \end{macrocode}
%
%    \begin{macrocode}
\RequirePackage{doc}
%    \end{macrocode}
%
% Make \verb+|+ be a `short verb' character, but not in the document
% preamble, where an active character may interfere with packages that
% are loaded.
%    \begin{macrocode}
\AtBeginDocument{\MakeShortVerb{\|}}
%    \end{macrocode}
%
% As `doc' documents tend to have a lot of monospaced material,
% Set up some |tt| substitutions to occur silently.
% \changes{v2.0p}{1995/11/02}{Add font substitutions}
% \changes{v2.0t}{1999/04/17}{Replaced octal number, CAR}
%    \begin{macrocode}
\DeclareFontShape{OT1}{cmtt}{bx}{n}{<-> ssub * cmtt/m/n}{}
\DeclareFontFamily{OMS}{cmtt}{\skewchar\font 48}  % '60
\DeclareFontShape{OMS}{cmtt}{m}{n}{<-> ssub * cmsy/m/n}{}
\DeclareFontShape{OMS}{cmtt}{bx}{n}{<-> ssub * cmsy/b/n}{}
%    \end{macrocode}
% This substitution is in the standard fd file, but not silent.
%    \begin{macrocode}
\DeclareFontShape{OT1}{cmss}{m}{it}{<->ssub*cmss/m/sl}{}
%    \end{macrocode}
%
%    \begin{macrocode}
\CodelineNumbered
\DisableCrossrefs
%    \end{macrocode}
%
% Increase the text width slightly so that with the standard fonts
% 72 columns of code may appear in a |macrocode| environment.
% \changes{v2.0c}{1994/03/15}{Set \cs{textwidth}.}
%    \begin{macrocode}
\setlength{\textwidth}{355pt}
%    \end{macrocode}
%
% Increase the marginpar width slightly, for long command names.
% And increase the left margin by a similar amount
% \changes{v2.0l}
%      {1994/05/25}{Increase \cs{marginparwidth}}
% \changes{v2.0q}{1995/11/28}
%      {Increase \cs{marginparwidth} and page margin.}
%    \begin{macrocode}
\addtolength\marginparwidth{30pt}
\addtolength\oddsidemargin{20pt}
\addtolength\evensidemargin{20pt}
%    \end{macrocode}
%
%
%    \begin{macrocode}
\setcounter{StandardModuleDepth}{1}
%    \end{macrocode}
%
% \section{Useful abbreviations}
%
% |\cmd{\foo}| Prints |\foo| verbatim. It may be used inside moving
% arguments. It can \emph{not} be use to record commands that are defined as
%    ``|\outer|'' nor is it possible to use it on conditionals such as
%    |\iftrue| or  defined by |\newif|.
% |\cs{foo}| also prints |\foo|, for those who prefer that
% syntax. (This second form can be used to record all types of command so the
%    above restrictions do not apply.
% \begin{macro}{\cmd}
% \changes{v2.0k}{1994/05/21}{New definition, so \cmd\{ works.}
% \begin{macro}{\cs}
% \changes{v2.0d}{1994/03/17}{Add \cs{cs}}
%    \begin{macrocode}
\def\cmd#1{\cs{\expandafter\cmd@to@cs\string#1}}
\def\cmd@to@cs#1#2{\char\number`#2\relax}
\DeclareRobustCommand\cs[1]{\texttt{\char`\\#1}}
%    \end{macrocode}
% \end{macro}
% \end{macro}
%
% \changes{v2.0r}{1996/01/11}{Removed \cs{star} since useless pr/2039}
%
% \begin{macro}{\marg}
% \changes{v2.0d}{1994/03/17}{Add \cs{marg}}
%    |\marg{text}| prints \marg{text}, `mandatory argument'.
%    \begin{macrocode}
\providecommand\marg[1]{%
  {\ttfamily\char`\{}\meta{#1}{\ttfamily\char`\}}}
%    \end{macrocode}
% \end{macro}
%
% \begin{macro}{\oarg}
%    |\oarg{text}| prints \oarg{text}, `optional argument'.
%    \begin{macrocode}
\providecommand\oarg[1]{%
  {\ttfamily[}\meta{#1}{\ttfamily]}}
%    \end{macrocode}
% \end{macro}
%
% \begin{macro}{\parg}
%    |\parg{te,xt}| prints \parg{te,xt}, `picture mode argument'.
% \changes{v2.0h}{1994/04/28}{Add \cs{parg}}
% \changes{v2.0o}{1995/08/09}{Use \cs{meta} when showing arguments}
%    \begin{macrocode}
\providecommand\parg[1]{%
  {\ttfamily(}\meta{#1}{\ttfamily)}}
%    \end{macrocode}
% \end{macro}
%
%
% \section{Old Comments}
%
% The \LaTeXe\ sources contain a lot of code inherited from
% \LaTeX2.09. The comments in this code were not designed to be
% typeset, and do not contain the necessary \LaTeX\ markup. The
% \texttt{oldcomments} environment typesets these comments,
% automatically sensing when any control sequence appears, and
% implicitly adding the |\verb|. This procedure does not produce
% particularly beautiful pages, but it allows us to fully document new
% sections, and have some form of typeset comments on all the old
% code.
% \changes{v2.0e}{1994/03/18}{Use a fixed font.}
%
% Scan control names and put them in tt.
% Will actually (incorrectly) scan past |\\| but this does not matter as
% this is almost never followed by a letter in practice.
% (ie |\\foo|) would put |foo| in |\ttfamily|.
%    \begin{macrocode}
\def\oc@scan#1{%
  \ifx\oc@bslash#1%
                      \egroup\let\next\oc@bslash\else
  \ifcat a\noexpand#1%
                      #1\let\next\oc@scan\else
  \ifx\oc@percent#1%
                      \def\next{\char`\%\egroup}%
  \else
                      #1\let\next\egroup
  \fi\fi\fi\next}
%    \end{macrocode}
%
%    \begin{macrocode}
\def\oc@bslash{\bgroup\oc@ttf\char`\\\oc@scan}%
%    \end{macrocode}
%
%    \begin{macrocode}
\def\oc@verb#1{%
  \catcode`#1\active
  \uccode`\~`#1%
  \uppercase{\def~{{\oc@ttf\char`#1}}}}
%    \end{macrocode}
%
%    \begin{macrocode}
\begingroup
  \obeyspaces%
  \catcode`\/=\catcode`\\
  /catcode`/\/active
  /catcode`<=/catcode`{%
  /catcode`>=/catcode`}%
  /catcode`/{/active%
  /catcode`/}/active%
  /gdef/oldc< \end{oldcomments}>%
  /gdef/begmac<    \begin{macrocode}>%
  /gdef/obs</def <</oc@ttf/ >>>%
/endgroup%
%    \end{macrocode}
%
%    \begin{macrocode}
\begingroup
  \catcode`\/=\catcode`\\
  \catcode`\\=13
  /catcode`/|=/catcode`/%
  /catcode`/%=13
  /gdef/oldcomments{|
    /makeatletter
    /let/do/oc@verb/dospecials
    /frenchspacing/@vobeyspaces/obs
    /raggedright
    /oc@verb/>|
    /oc@verb/<|
    /let\/oc@bslash
    /let%/oc@percent
    /obeylines
    /parindent/z@
    /ttfamily/expandafter/let/expandafter/oc@ttf/the/font
    /rmfamily
    /hfuzz/maxdimen
    }
/endgroup
%    \end{macrocode}
%
%    \begin{macrocode}
\begingroup
  \sloppy%
  \obeylines%
  \gdef\oc@percent#1^^M{%
    \ifvmode%
    \def\commentline{#1}%
    \ifx\commentline\oldc%
    \end{oldcomments}%
    \else%
    \ifx\commentline\begmac%
    \begin{macrocode}%
    \else%
    \leavevmode%
    #1^^M%
    \fi\fi%
    \else%
    {\oc@ttf\char`\%}#1^^M%
    \fi}%
\endgroup%
%    \end{macrocode}
%
%
% \section{DocInclude}
%
%    \begin{macrocode}
\@addtoreset{CodelineNo}{part}
%    \end{macrocode}
%
% \begin{macro}{\DocInclude}
% More or less exactly the same as |\include|, but uses |\DocInput|
% on a |dtx| file, not |\input| on a |tex| file.
% \changes{v2.0b}{1994/03/14}{Rename from \cs{docinclude}}
% \changes{v2.0f}{1994/03/25}{Use \cs{part}}
% \changes{v2.0u}{1999/08/08}{Also works for .fdd (M. Schroeder)}
%    \begin{macrocode}
\def\partname{File}
%    \end{macrocode}
%
%    \begin{macrocode}
\newcommand*{\DocInclude}[1]{%
  \relax
  \clearpage
  \docincludeaux
  \IfFileExists{#1.fdd}%
    {\def\currentfile{#1.fdd}}%
    {\def\currentfile{#1.dtx}}%
  \ifnum\@auxout=\@partaux
    \@latexerr{\string\include\space cannot be nested}\@eha
  \else \@docinclude#1 \fi}
\def\@docinclude#1 {\clearpage
\if@filesw \immediate\write\@mainaux{\string\@input{#1.aux}}\fi
\@tempswatrue\if@partsw \@tempswafalse\edef\@tempb{#1}\@for
\@tempa:=\@partlist\do{\ifx\@tempa\@tempb\@tempswatrue\fi}\fi
\if@tempswa \let\@auxout\@partaux \if@filesw
\immediate\openout\@partaux #1.aux
\immediate\write\@partaux{\relax}\fi
%    \end{macrocode}
% We need to save (and later restore) various index-related
% commands which might be changed by the included file.
%    \begin{macrocode}
\let\@ltxdoc@PrintIndex\PrintIndex
\let\PrintIndex\relax
\let\@ltxdoc@PrintChanges\PrintChanges
\let\PrintChanges\relax
\let\@ltxdoc@theglossary\theglossary
\let\@ltxdoc@endtheglossary\endtheglossary
\part{\currentfile}%
  {\let\ttfamily\relax
  \xdef\filekey{\filekey, \thepart={\ttfamily\currentfile}}}%
\DocInput{\currentfile}%
\let\PrintIndex\@ltxdoc@PrintIndex
\let\PrintChanges\@ltxdoc@PrintChanges
\let\theglossary\@ltxdoc@theglossary
\let\endtheglossary\@ltxdoc@endtheglossary
\clearpage
\@writeckpt{#1}\if@filesw \immediate\closeout\@partaux \fi
\else\@nameuse{cp@#1}\fi\let\@auxout\@mainaux}
%    \end{macrocode}
%
%    \begin{macrocode}
\gdef\codeline@wrindex#1{\if@filesw
        \immediate\write\@indexfile
            {\string\indexentry{#1}%
            {\filesep\number\c@CodelineNo}}\fi}%
%    \end{macrocode}
% \end{macro}
%
%    \begin{macrocode}
\let\filesep\@empty
%    \end{macrocode}
%
%
%  \begin{macro}{\aalph}
% Special  form of |\alph| as currently |source2e.tex|
% includes more than 26 files
% \changes{v2.0n}{1994/05/27}{Use uppercase letters, for makeindex}.
%    \begin{macrocode}
\def\aalph#1{\@aalph{\csname c@#1\endcsname}}
\def\@aalph#1{%
  \ifcase#1\or a\or b\or c\or d\or e\or f\or g\or h\or i\or
         j\or k\or l\or m\or n\or o\or p\or q\or r\or s\or
         t\or u\or v\or w\or x\or y\or z\or A\or B\or C\or
         D\or E\or F\or G\or H\or I\or J\or K\or L\or M\or
         N\or O\or P\or Q\or R\or S\or T\or U\or V\or W\or
         X\or Y\or Z\else\@ctrerr\fi}
%    \end{macrocode}
%  \end{macro}
%
% \begin{macro}{\docincludeaux}
% \changes{v2.06}{1994/03/31}{Use \cs{footnotesize} in file key.}
% \changes{v2.0k}{1994/05/21}{Use \cs{aalph}}
% \changes{v2.0v}{2017/03/13}{Use \cs{parbox}[t] in file
%         key to maintain space between the text block and the page foot.}
%    \begin{macrocode}
\def\docincludeaux{%
  \def\thepart{\aalph{part}}\def\filesep{\thepart-}%
  \let\filekey\@gobble
  \g@addto@macro\index@prologue{%
    \gdef\@oddfoot{\parbox[t]{\textwidth}{\strut\footnotesize
       \raggedright{\bfseries File Key:} \filekey}}%
    \let\@evenfoot\@oddfoot}%
  \global\let\docincludeaux\relax
 \gdef\@oddfoot{%
   \expandafter\ifx\csname ver@\currentfile\endcsname\relax
    File \thepart: {\ttfamily\currentfile} %
   \else
    \GetFileInfo{\currentfile}%
    File \thepart: {\ttfamily\filename} %
    Date: \filedate\ %
    Version \fileversion
    \fi
    \hfill\thepage}%
 \let\@evenfoot\@oddfoot}%
%    \end{macrocode}
% \end{macro}
%
% \begin{macro}{\MaintainedByLaTeXTeam}
% \changes{v2.0v}{2015/03/25}{macro added}
% \changes{v2.0w}{2015/03/25}{use display block not footnote text}
% Generate boilerplate reference to bug database.
%    \begin{macrocode}
\def\MaintainedBy#1{\gdef\@maintainedby{#1}}
%    \end{macrocode}
%
%    \begin{macrocode}
\let\@maintainedby\@empty
%    \end{macrocode}
%
%    \begin{macrocode}
\def\MaintainedByLaTeXTeam#1{%
{\gdef\@maintainedby{%
This file is maintained by the \LaTeX{} Project team.\\%
Bug reports can be opened (category \texttt{#1}) at\\%
\url{https://latex-project.org/bugs.html}.}}}
%    \end{macrocode}
%
%    \begin{macrocode}
\def\@maketitle{%
  \newpage
  \null
  \vskip 2em%
  \begin{center}%
  \let \footnote \thanks
    {\LARGE \@title \par}%
    \vskip 1.5em%
    {\large
      \lineskip .5em%
      \begin{tabular}[t]{c}%
        \@author
      \end{tabular}\par}%
    \vskip 1em%
    {\large \@date}%
    \ifx\@maintainedby\@empty
    \else
    \vskip 1em%
    \fbox{\fbox{\begin{tabular}{@{}l@{}}\@maintainedby\end{tabular}}}%
    \fi
  \end{center}%
  \par
  \vskip 1.5em}
%    \end{macrocode}
% \end{macro}
%
% \begin{macro}{\url}
%    \begin{macrocode}
\providecommand\url{\texttt}
%    \end{macrocode}
% \end{macro}
%
%    \begin{macrocode}
\def\task#1#2{}
%</class>
%    \end{macrocode}
% \Finale
%

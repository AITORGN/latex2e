% \iffalse meta-comment
%
% Copyright 2014-2015
% The LaTeX3 Project and any individual authors listed elsewhere
% in this file.
%
% This file is part of the LaTeX base system.
% -------------------------------------------
%
% It may be distributed and/or modified under the
% conditions of the LaTeX Project Public License, either version 1.3c
% of this license or (at your option) any later version.
% The latest version of this license is in
%    http://www.latex-project.org/lppl.txt
% and version 1.3c or later is part of all distributions of LaTeX
% version 2005/12/01 or later.
%
% This file has the LPPL maintenance status "maintained".
%
% The list of all files belonging to the LaTeX base distribution is
% given in the file `manifest.txt'. See also `legal.txt' for additional
% information.
%
% The list of derived (unpacked) files belonging to the distribution
% and covered by LPPL is defined by the unpacking scripts (with
% extension .ins) which are part of the distribution.
%
% -----------------------------------------------------------------------------
%
% The same approach as used in \pkg{DocStrip}: if \cs{documentclass}
% is undefined then skip the driver, allowing the file to be used directly.
% This works as the \cs{fi} is only seen if \LaTeX{} is not in use. The odd
% \cs{jobname} business allows the extraction to work with \LaTeX{} provided
% an appropriate \texttt{.ins} file is set up.
%<*gobble>
\ifx\jobname\relax
  \let\documentclass\undefined
\fi
\begingroup\expandafter\expandafter\expandafter\endgroup
\expandafter\ifx\csname documentclass\endcsname\relax
\else
  \csname fi\endcsname
%</gobble>
%
%<*driver>
\ProvidesFile{ltunicode.dtx}
  [2015/03/25 v1.0a LaTeX Kernel (Unicode data)]
\documentclass{ltxdoc}
\begin{document}
\DocInput{\jobname.dtx}
\end{document}
%<*gobble>
\fi
%</gobble>
%</driver>
% \fi
%
% \GetFileInfo{ltunicode.dtx}
% \title{The \texttt{ltunicode.dtx} file\thanks
%     {This file has version number \fileversion, dated \filedate.}\\
%       for use with \LaTeXe}
% \author{The \LaTeX3 Project}
%
% \maketitle
%
% This script extracts data from the Unicode Consortium files
% \texttt{UnicodeData.txt}, \texttt{EastAsianWidth.txt} and
% \texttt{LineBreak.txt} to be used for setting up \LaTeXe{} with sane
% default settings when using the Xe\TeX{} and Lua\TeX{} engines. Details
% of the process are included in the code comments.
% 
% To create the extracted file, run this file in a location containing
% the three input data files using a plain \TeX{} system with the e-\TeX{}
% extensions enabled (\texttt{pdftex}, \texttt{xetex} or \texttt{luatex}
% in any modern \TeX{} distribution).
%
% \StopEventually{}
%
%    \begin{macrocode}
%<*script>
%    \end{macrocode}
%
% \section{General set up}
%
% The script is designed to work with plain \TeX{} and so |@| is made into
% a `letter' using the primitive approach.
%    \begin{macrocode}
\catcode`\@=11 %
%    \end{macrocode}
%
% \begin{macro}{\gobble}
% \begin{macro}{\firsttoken}
%   Standard utilities.
%    \begin{macrocode}
\long\def\gobble#1{}
\long\def\firsttoken#1#2\relax{#1}
%    \end{macrocode}
% \end{macro}
% \end{macro}
%
% \begin{macro}{\storedpar}
%   A simple piece of test set up: the final line of the read file will be
%   tokenized by \TeX{} as \cs{par} which can be tested by \cs{ifx} provided
%   we have an equivalent available.
%    \begin{macrocode}
\def\storedpar{\par}
%    \end{macrocode}
% \end{macro}
% 
% \begin{macro}{\return}
%   A stored |^^M| for string comparisons.
%    \begin{macrocode}
\begingroup
  \catcode`\^^M=12 %
  \gdef\return{^^M}%
\endgroup%
%    \end{macrocode}
% \end{macro}
%
% \begin{macro}{\sourceforhex}
% \begin{macro}{\sethex}
% \begin{macro}{\dohex}
% \begin{macro}{\hexdigit}
%   Some parts of the code here will need to be able to convert integers
%   to their hexadecimal equivalent. That is easiest to do for the requirements
%   here using a modified version of some code from Appendix~D of \emph{The
%   \TeX{}book}.
%    \begin{macrocode}
\newcount\sourceforhex
\def\sethex#1#2{%
  \def#1{}%
  \sourceforhex=#2\relax
  \ifnum\sourceforhex=0 %
    \def#1{0}%
  \else
    \dohex#1%
  \fi
}
\def\dohex#1{%
  \begingroup
    \count0=\sourceforhex
    \divide\sourceforhex by 16 %
    \ifnum\sourceforhex>0 %
      \dohex#1%
    \fi
    \count2=\sourceforhex
    \multiply\count2 by -16 %
    \advance\count0 by\count2
    \hexdigit#1%
  \expandafter\endgroup
  \expandafter\def\expandafter#1\expandafter{#1}%
}
\def\hexdigit#1{%
  \ifnum\count0<10 %
    \edef#1{#1\number\count0}%
  \else
    \advance\count0 by -10 %
    \edef#1{#1\ifcase\count0 A\or B\or C\or D\or E\or F\fi}%
  \fi
}
%    \end{macrocode}
% \end{macro}
% \end{macro}
% \end{macro}
% \end{macro}
%
% \begin{macro}{\unicoderead, \unicodewrite}
%   Set up the streams for data.
%    \begin{macrocode}
\newread\unicoderead
\newwrite\unicodewrite
%    \end{macrocode}
% \end{macro}
%
% \section{Verbatim copying}
%
% \begin{macro}{\verbatimcopy}
% \begin{macro}{\endverbatimcopy}
% \begin{macro}{\verbatimcopy@auxii}
% \begin{macro}{\verbatimcopy@auxii}
% \begin{macro}{\verbatim@endmarker}
%   Set up to read some material verbatim and write it to the output stream.
%   There needs to be a dedicated `clean up first line' macro, but other than
%   that life is simple enough.
%    \begin{macrocode}
\begingroup
  \catcode`\^^M=12 %
  \gdef\verbatimcopy{%
    \begingroup%
      \catcode`\^^M=12 %
      \catcode`\\=12 %
      \catcode`\{=12 %
      \catcode`\}=12 %
      \catcode`\#=12 %
      \catcode`\%=12 %
      \catcode`\ =12 %
      \endlinechar=`\^^M %
      \verbatimcopy@auxi
  }%
  \gdef\verbatimcopy@auxi#1^^M{%
    \expandafter\verbatimcopy@auxii\gobble#1^^M%
  }%
  \gdef\verbatimcopy@auxii#1^^M{%
    \def\temp{#1}%
    \ifx\temp\verbatim@endmarker%
      \expandafter\endgroup%
    \else%
      \ifx\temp\empty\else%
        \immediate\write\unicodewrite{#1}%
      \fi%
      \expandafter\verbatimcopy@auxii%
    \fi%
  }%
\endgroup%
\edef\verbatim@endmarker{\expandafter\gobble\string\\}
\edef\verbatim@endmarker{\verbatim@endmarker endverbatimcopy}
%    \end{macrocode}
% \end{macro}
% \end{macro}
% \end{macro}
% \end{macro}
% \end{macro}
%
% \section{File header section}
%
% With the mechanisms set up, open the data file for writing.
%    \begin{macrocode}
\immediate\openout\unicodewrite=ltunicode.ltx %
%    \end{macrocode}
% There are various lines that now need to go at the start of the file.
% First, there is some header information.
%    \begin{macrocode}
\verbatimcopy
%% This is the file `ltunicode.ltx',
%% generated using the script ltunicode.dtx.
%%
%% The data here are derived from the files
%%  - UnicodeData.txt
%%  - EastAsianWidth.txt
%%  - LineBreak.txt
%% which are maintained by the Unicode Consortium.
%%
%% Copyright 2014-2015
%% The LaTeX3 Project and any individual authors listed elsewhere
%% in this file.
%%
%% This file is part of the LaTeX base system.
%% -------------------------------------------
%%
%% It may be distributed and/or modified under the
%% conditions of the LaTeX Project Public License, either version 1.3c
%% of this license or (at your option) any later version.
%% The latest version of this license is in
%%    http://www.latex-project.org/lppl.txt
%% and version 1.3c or later is part of all distributions of LaTeX
%% version 2005/12/01 or later.
%%
%% This file has the LPPL maintenance status "maintained".
%%
%% The list of all files belonging to the LaTeX base distribution is
%% given in the file `manifest.txt'. See also `legal.txt' for additional
%% information.
\endverbatimcopy
%    \end{macrocode}
% Automatically include the current date.
%    \begin{macrocode}
\immediate\write\unicodewrite{%
  \expandafter\gobble\string\%\expandafter\gobble\string\%
  Generated on \the\year
    -\ifnum\month>9 \else 0\fi \the\month
    -\ifnum\day>9   \else 0\fi \the\day.
}
\immediate\write\unicodewrite{%
  \expandafter\gobble\string\%\expandafter\gobble\string\%
}
%    \end{macrocode}
%
% \section{Unicode character data}
%
% \begin{macro}{\parseunicodedata}
% \begin{macro}{\parseunicodedata@auxi}
% \begin{macro}{\parseunicodedata@auxii}
% \begin{macro}{\parseunicodedata@auxiii}
%   The first step of parsing a line of data is to check that it's not come
%   from a blank in the source, which will have been tokenized as \cs{par}.
%   Assuming that is not the case, there are lots of data items separated by
%   |;|. Of those, only a few are needed so they are picked out and everything
%   else is dropped.
%    \begin{macrocode}
\def\parseunicodedata#1{%
  \ifx#1\storedpar
  \else
    \expandafter\parseunicodedata@auxi#1\relax
  \fi
}
\def\parseunicodedata@auxi#1;#2;#3;#4;#5;#6;#7;#8;#9;{%
  \parseunicodedata@auxii#1;#3;
}
\def\parseunicodedata@auxii#1;#2;#3;#4;#5;#6;#7;#8\relax{%
  \parseunicodedata@auxiii{#1}{#2}{#6}{#7}%
}
%    \end{macrocode}
%   At this stage we have only four pieces of data
%   \begin{enumerate}
%     \item The code value
%     \item The general class
%     \item The uppercase mapping
%     \item The lowercase mapping
%   \end{enumerate}
%   where both one or both of the last two may be empty. Everything here could
%   be done in a single conditional within a \cs{write}, but that would be
%   tricky to follow. Instead, a series of defined auxiliaries are used to
%   show the flow. Notice that combining marks are treated as letters here
%   (the second `letter' test).
%    \begin{macrocode}
\def\parseunicodedata@auxiii#1#2#3#4{%
  \if L\firsttoken#2?\relax
    \expandafter\unicodeletter
  \else
    \if M\firsttoken#2?\relax
      \expandafter\expandafter\expandafter\unicodeletter
    \else
      \expandafter\expandafter\expandafter\unicodenonletter
    \fi
  \fi
    {#1}{#3}{#4}%
}
%    \end{macrocode}
% \end{macro}
% \end{macro}
% \end{macro}
% \end{macro}
%
% \begin{macro}{\unicodeletter, \unicodenonletter}
% \begin{macro}{\writeunicodedata}
%   For `letters', we always want to write the data to file, and the only
%   question here is if the character has case mappings or these point back
%   to the character itself.
%    \begin{macrocode}
\def\unicodeletter#1#2#3{%
  \writeunicodedata\L{#1}{#2}{#3}%
}
%    \end{macrocode}
%   Cased non-letters can also exist: they can be detected as they have at
%   least one case mapping. Write these in much the same way as letters.
%    \begin{macrocode}
\def\unicodenonletter#1#2#3{%
  \ifx\relax#2#3\relax
  \else
    \writeunicodedata\C{#1}{#2}{#3}%
  \fi
}
%    \end{macrocode}
%   Actually write the data. In all cases both upper- and lower-case mappings
%   are given, so there is a need to test that both were actually available and
%   if not set up to do nothing.
%    \begin{macrocode}
\def\writeunicodedata#1#2#3#4{%
  \immediate\write\unicodewrite{%
    \space\space
    \string#1\space
    #2 %
    \ifx\relax#3\relax
      #2 %
    \else
      #3 %
    \fi
    \ifx\relax#4\relax
      #2 %
    \else
      #4 %
    \fi
    \expandafter\gobble\string\%
  }%
}
%    \end{macrocode}
% \end{macro}
% \end{macro}
%
% There is now a lead-in section which creates the macros which take the
% processed data and do the code assignments. Everything is done within a
% group so that there is no need to worry about names.
%    \begin{macrocode}
\verbatimcopy
\begingroup
\endverbatimcopy
%    \end{macrocode}
% Cased non-letters simply need to have the case mappings set.
% For letters, there are a few things to sort out. First, the case mappings are
% defined as for non-letters. Category code is then set to $11$ before a check
% to see if this is an upper case letter. If it is then the \cs{sfcode} is set
% to $999$. Finally there is a need to deal with Unicode math codes, where base
% plane letters are class $7$ but supplementary plane letters are class~$1$.
% Older versions of Xe\TeX{} used a different name here: easy to pick up as
% we know that this primitive must be defined in some way. There is also an issue
% with the supplementary plane and older Xe\TeX{} versions, which is dealt with
% using a check at run time.
%    \begin{macrocode}
\verbatimcopy
  \def\C#1 #2 #3 {%
    \XeTeXcheck{#1}%
    \global\uccode"#1="#2 %
    \global\lccode"#1="#3 %
  }
  \def\L#1 #2 #3 {%
    \C #1 #2 #3 %
    \catcode"#1=11 %
    \ifnum"#1="#3 %
    \else
      \global\sfcode"#1=999 %
    \fi
    \ifnum"#1<"10000 %
      \global\Umathcode"#1="7"01"#1 %
    \else
      \global\Umathcode"#1="0"01"#1 %
    \fi    
  }
  \ifx\Umathcode\undefined
    \let\Umathcode\XeTeXmathcode
  \fi
  \def\XeTeXcheck#1{}
  \ifx\XeTeXversion\undefined
  \else
    \def\XeTeXcheck.#1.#2-#3\relax{#1}
     \ifnum\expandafter\XeTeXcheck\XeTeXrevision.-\relax>996 %
       \def\XeTeXcheck#1{}
     \else
       \def\XeTeXcheck#1{%
          \ifnum"#1>"FFFF %
            \long\def\XeTeXcheck##1\endgroup{\endgroup}
            \expandafter\XeTeXcheck
          \fi
       }
     \fi
  \fi
\endverbatimcopy
%    \end{macrocode}
% Read the data and write the resulting code assignments to the file.
%    \begin{macrocode}
\openin\unicoderead=UnicodeData.txt %
\ifeof\unicoderead
  \errmessage{Data file missing: UnicodeData.txt}%
\fi
\loop\unless\ifeof\unicoderead
  \read\unicoderead to \unicodedataline
  \parseunicodedata\unicodedataline
\repeat
%    \end{macrocode}
% End the group for setting character codes and assign a couple of special
% cases.
%    \begin{macrocode}
\verbatimcopy
\endgroup
\global\sfcode"2019=0 %
\global\sfcode"201D=0 %
\endverbatimcopy
%    \end{macrocode}
% Lua\TeX{} and older versions of Xe\TeX{} stop here: character classes are a
% Xe\TeX{}-only concept.
%    \begin{macrocode}
\verbatimcopy
\ifx\XeTeXchartoks\XeTeXcharclass
  \expandafter\endinput
\fi
\endverbatimcopy
%    \end{macrocode}
%
% \section{Xe\TeX{} Character classes}
% 
% The Xe\TeX{} engine includes the concept of character classes, which allow
% insertion of tokens into the input stream at defined boundaries. Setting
% up this data requires a two-part process as the information is split over
% two input files.
%
% \begin{macro}{\parseunicodedata}
% \begin{macro}{\parseunicodedata@auxi}
% \begin{macro}{\parseunicodedata@auxii}
%   The parsing system is redefined to parse a detokenized input line which
%   may be a comment starting with |#|. Assuming that is not the case, the
%   data line with start with a code point potentially forming part of a range.
%   The range is extracted and the width stored for each code point.
%    \begin{macrocode}
\def\parseunicodedata#1{%
  \ifx#1\return
  \else
    \if\expandafter\gobble\string\#\expandafter\firsttoken#1?\relax
    \else
      \expandafter\parseunicodedata@auxi#1\relax
    \fi
  \fi
}
\def\parseunicodedata@auxi#1;#2 #3\relax{%
  \parseunicodedata@auxii#1....\relax{#2}%
}
\def\parseunicodedata@auxii#1..#2..#3\relax#4{%
  \expandafter\gdef\csname EAW@#1\endcsname{#4}%
  \ifx\relax#2\relax
  \else
    \count@="#1 %
    \begingroup
      \loop
        \ifnum\count@<"#2 %
          \advance\count@\@ne
          \sethex\temp{\count@}%
          \expandafter\gdef\csname EAW@\temp\endcsname{#4}%
      \repeat
    \endgroup
  \fi
}
%    \end{macrocode}
% \end{macro}
% \end{macro}
% \end{macro}
%
% With the right parser in place, read the data file.
%    \begin{macrocode}
\openin\unicoderead=EastAsianWidth.txt %
\ifeof\unicoderead
  \errmessage{Data file missing: EastAsianWidth.txt}%
\fi
\loop\unless\ifeof\unicoderead
  \readline\unicoderead to \unicodedataline
  \parseunicodedata\unicodedataline
\repeat
%    \end{macrocode}
%
% \begin{macro}{\parseunicodedata@auxii}
% \begin{macro}{\parseunicodedata@auxiii}
% \begin{macro}{\parseunicodedata@auxiv}
% \begin{macro}{\ID}
% \begin{macro}{\OP}
% \begin{macro}{\CL}
% \begin{macro}{\EX}
% \begin{macro}{\IS}
% \begin{macro}{\NS}
% \begin{macro}{\CM}
%  The final file to read, |LineBreaking.txt|, uses the same format as\\
%  |EastAsianWidth.txt|. As such, only the final parts of the parser have to be
%  redefined.
%    \begin{macrocode}
\def\parseunicodedata@auxii#1..#2..#3\relax#4{%
  \parseunicodedata@auxiii{#1}{#4}%
  \ifx\relax#2\relax
  \else
    \count@="#1 %
    \begingroup
      \loop
        \ifnum\count@<"#2 %
          \advance\count@\@ne
          \sethex\temp{\count@}%
          \expandafter\parseunicodedata@auxiii\expandafter{\temp}{#4}%
      \repeat
    \endgroup
  \fi
}
%    \end{macrocode}
%   Adding data to the processed file depends on two factors: the
%   classification in the line-breaking file and (possibly) the width data
%   too. Any characters of class \texttt{ID} (ideograph) are stored: they
%   always need special treatment. For characters of classes \texttt{OP}
%   (opener), \texttt{CL} (closer), \texttt{EX} (exclamation), \texttt{IS}
%   (infix sep) and \texttt{NS} (non-starter) the data is stored if the
%   character is full, half or wide width. The same is true for 
%   \texttt{CM} (combining marks) characters, which need to be transparent
%   to the mechanism.
%    \begin{macrocode}
\def\parseunicodedata@auxiii#1#2{%
  \ifcsname #2\endcsname
    \ifnum\csname #2\endcsname=1 %
      \parseunicodedata@auxiv{#1}{#2}%
    \else
      \ifnum 0%
        \if F\csname EAW@#1\endcsname 1\fi
        \if H\csname EAW@#1\endcsname 1\fi
        \if W\csname EAW@#1\endcsname 1\fi
        >0 %
        \parseunicodedata@auxiv{#1}{#2}%
      \fi
    \fi
  \fi
}
\def\parseunicodedata@auxiv#1#2{%
  \immediate\write\unicodewrite{%
    \space\space
    \expandafter\string\csname #2\endcsname
    \space
    #1 %
    \expandafter\gobble\string\%
  }%
}
\def\ID{1}
\def\OP{2}
\def\CL{3}
\let\EX\CL
\let\IS\CL
\let\NS\CL
\def\CM{256}
%    \end{macrocode}
% \end{macro}
% \end{macro}
% \end{macro}
% \end{macro}
% \end{macro}
% \end{macro}
% \end{macro}
% \end{macro}
% \end{macro}
% \end{macro}
%
% Before actually reading the line breaking data file, the appropriate
% temporary code is added to the output. As described above, only a limited
% number of classes need to be covered: they are hard-coded as classes
% $1$, $2$ and $3$ following the convention adopted by plain Xe\TeX{}.
%    \begin{macrocode}
\verbatimcopy
\begingroup
  \def\ID#1 {\global\XeTeXcharclass"#1=1 \global\catcode"#1=11 }
  \def\OP#1 {\global\XeTeXcharclass"#1=2 }
  \def\CL#1 {\global\XeTeXcharclass"#1=3 }
  \def\EX#1 {\global\XeTeXcharclass"#1=3 }
  \def\IS#1 {\global\XeTeXcharclass"#1=3 }
  \def\NS#1 {\global\XeTeXcharclass"#1=3 }
  \def\CM#1 {\global\XeTeXcharclass"#1=256 }
\endverbatimcopy
%    \end{macrocode}
%
% Read the line breaking data and save to the output.
%    \begin{macrocode}
\openin\unicoderead=LineBreak.txt %
\ifeof\unicoderead
  \errmessage{Data file missing: LineBreak.txt}%
\fi
\loop\unless\ifeof\unicoderead
  \readline\unicoderead to \unicodedataline
  \parseunicodedata\unicodedataline
\repeat
%    \end{macrocode}
%
% \changes{v1.0a}{2015/03/25}{Use \cs{hskip} rather than \cs{hspace}
%   in glue settings}
% Set up material to be inserted between character classes.
% that provided by plain Xe\TeX{}. Using \cs{hskip} here means the code will
% work with plain as well as \LaTeXe{}.
%    \begin{macrocode}
\verbatimcopy
\endgroup
\gdef\xtxHanGlue{\hskip0pt plus 0.1em\relax}
\gdef\xtxHanSpace{\hskip0.2em plus 0.2em minus 0.1em\relax}
\global\XeTeXinterchartoks 0 1 = {\xtxHanSpace}
\global\XeTeXinterchartoks 0 2 = {\xtxHanSpace}
\global\XeTeXinterchartoks 0 3 = {\nobreak\xtxHanSpace}
\global\XeTeXinterchartoks 1 0 = {\xtxHanSpace}
\global\XeTeXinterchartoks 2 0 = {\nobreak\xtxHanSpace}
\global\XeTeXinterchartoks 3 0 = {\xtxHanSpace}
\global\XeTeXinterchartoks 1 1 = {\xtxHanGlue}
\global\XeTeXinterchartoks 1 2 = {\xtxHanGlue}
\global\XeTeXinterchartoks 1 3 = {\nobreak\xtxHanGlue}
\global\XeTeXinterchartoks 2 1 = {\nobreak\xtxHanGlue}
\global\XeTeXinterchartoks 2 2 = {\nobreak\xtxHanGlue}
\global\XeTeXinterchartoks 2 3 = {\xtxHanGlue}
\global\XeTeXinterchartoks 3 1 = {\xtxHanGlue}
\global\XeTeXinterchartoks 3 2 = {\xtxHanGlue}
\global\XeTeXinterchartoks 3 3 = {\nobreak\xtxHanGlue}
\endverbatimcopy
%    \end{macrocode}
%
% Done: end the script.
%    \begin{macrocode}
\bye
%    \end{macrocode}
%    
%    \begin{macrocode}
%</script>
%    \end{macrocode}
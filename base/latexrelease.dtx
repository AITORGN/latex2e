% \iffalse meta-comment
%
% Copyright 1993-2014
% The LaTeX3 Project and any individual authors listed elsewhere
% in this file.
%
% This file is part of the LaTeX base system.
% -------------------------------------------
%
% It may be distributed and/or modified under the
% conditions of the LaTeX Project Public License, either version 1.3c
% of this license or (at your option) any later version.
% The latest version of this license is in
%    http://www.latex-project.org/lppl.txt
% and version 1.3c or later is part of all distributions of LaTeX
% version 2005/12/01 or later.
%
% This file has the LPPL maintenance status "maintained".
%
% The list of all files belonging to the LaTeX base distribution is
% given in the file `manifest.txt'. See also `legal.txt' for additional
% information.
%
% The list of derived (unpacked) files belonging to the distribution
% and covered by LPPL is defined by the unpacking scripts (with
% extension .ins) which are part of the distribution.
%
% \fi
%
% \iffalse
%
%<*dtx>
          \ProvidesFile{latexrelease.dtx}
%</dtx>
%<driver>\ProvidesFile{latexrelease.drv}
%<latexrelease>\ProvidesPackage{latexrelease}
% \fi
%         \ProvidesFile{latexrelease.dtx}
          [2014/09/29 v1.1s fixes to LaTeX]
%
% \iffalse
%<*driver>
 \documentclass{ltxdoc}
 \newcommand\Lopt[1]{\textsf{#1}}
 \let\Lpack\Lopt
 \providecommand{\file}[1]{\texttt{#1}}
 \providecommand{\MF}{\textsf{Metafont}}
 \providecommand{\danger}{\marginpar[\hfill\protect\Huge!!]{\protect\Huge!!\hfill}}
 \begin{document}
 \DocInput{latexrelease.dtx}
 \end{document}
%</driver>
% \fi
%
% \CheckSum{88}
%
%
% \let\package\textsf
%
%
% \GetFileInfo{latexrelease.dtx}
%
% \title{The \Lpack{latexrelease} packages\thanks{This file
%         has version number \fileversion, last
%         revised \filedate.}}
% \author{...}
% \date{\filedate}
%  \maketitle
%
% \section{Introduction}
% somthing about how this is supposed to work\ldots
%
%
%
% \section{Release specific code}
% somethiung about |\\IncludeInRelease|
%
% \section{Currently supported options}
% |2014/05/01| undoes all patches.
%
% |2015/01/01| adds all currently inplemented patches.
% \section{Fixes added for 2005/12/01}
%
% \subsection{New Allocation Code}
% Previously |\newcount| and related commands were based on classic
% TeX and only allocated in the range 0--255. This was extended (in
% different ways) for e-\TeX\ in the |etex| package and in the
% |xelatex.ini| and |lualatex.ini| files used in those
% formats. Related to this the number of boxes allocated to store
% floats was limited. This was extended to a certain extent in the
% |morefloats| package (by xx and xx) but the new allocation
% incorporates float allocation directly and supports much larger
% float lists using the extended registers.
%
% The new code allocates registers in the full extended range (
% $2^{15}-1$ for etex and xelatex, $2^{16}-1$ for lualatex.
% In addition a new command |\extrafloats| is provided, to be documented\ldots
%
% |\newmarks| and |\newXeTeXintercharclass| \ldots
%
%
% \subsection{\texttt{\textbackslash textsubscript} not defined in
%    latex.ltx (pr/3492)}
%
%\begin{verbatim}
% >Number:         3492
% >Category:       latex
% >Synopsis:       \textsubscript not defined in latex.ltx
% >Arrival-Date:   Tue Jan 14 23:01:00 CET 2003
% >Originator:     tgakic@chem.tue.nl  (Ionel Mugurel Ciobica)
%
% I use \textsubscript much more often than \textsuperscript, and
% \textsubscript it is not defined in latex.ltx. Could you please
% consider including the definition of \textsubscript in the latex.ltx
% for the next versions of LaTeX.    Thank you.
%\end{verbatim}
%
% \subsection{\texttt{\textbackslash @} discards spaces when moving
%             (pr/3039)}
%
%\begin{verbatim}
% >Number:         3039
% >Category:       latex
% >Synopsis:       \@ discards spaces when moving
% >Arrival-Date:   Sat May 22 09:01:06 1999
% >Originator:     asnd@triumf.ca (Donald Arseneau)
% >Description:
% The \@ command expands to \spacefactor\@m in auxiliary files,
% which then ignores following spaces when it is reprocessed.
%\end{verbatim}
%
% \subsection{Wrong header for twocolumn (pr/2613)}
%
%\begin{verbatim}
% >Number:         2613
% >Category:       latex
% >Synopsis:       wrong headline for twocolumn
% >Arrival-Date:   Mon Sep 22 16:41:09 1997
% >Originator:     daniel@cs.uni-bonn.de (Daniel Reischert)
% >Description:
% When setting the document in two columns
% the headline shows the top mark of the second column,
% but it should show the top mark of the first column.
% \end{verbatim}
%
% Originally fixed in package \Lpack{fix2col} which was merged into
% this package. Documentation and code from this package have been
% merged into this file.
%
% \subsubsection{Notes on the Implementation Strategy}
%
% The standard \LaTeX\ twocolumn system works internally by making
% each column a separate `page' that is passed independently to \TeX's
% pagebreaker. (Unlike say the \package{multicol} package, where all
% columns are gathered together and then split into columns later,
% using |\vsplit|.) This means that the primitive \TeX\ marks that are
% normally used for header information, are globally reset after the
% first column. By default \LaTeX\ does nothing about this.
% A good solution is provided by Piet van Oostrum (building on earlier
% work of Joe Pallas) in his \package{fixmarks} package.
%
% After the first column box has been collected the mark information
% for that box is saved, so that any |\firstmark| can be
% `artificially' used to set the page-level marks after the second
% column has been collected. (The second column |\firstmark| is not
% normally required.) Unfortunately \TeX\ does not provide a direct
% way of knowing if any marks are in the page, |\firstmark| always has a
% value from previous pages, even if there is no mark in this page.
% The solution is to make a copy of the box and then |\vsplit| it
% so that any marks show up as |\splitfirstmark|.
%
% The use of |\vsplit| does mean that the output routine will globally
% change the value of |\splitfirstmark| and
% |\splitbotmark|. The \package{fixmarks} package goes to some trouble
% to save and restore these values so that the output routine does
% \emph{not} change the values. This part of \package{fixmarks} is not
% copied here as it is quite costly (having to be run on every page) and
% there is no reason why anyone writing code using |\vsplit| should
% allow the output routine to be triggered before the split marks have
% been accessed.
%
%
%
% \subsection{\texttt{\textbackslash setlength} produces error if
%   used with registers like \texttt{\textbackslash dimen0} (pr/3066)}
%
%\begin{verbatim}
% >Number:         3066
% >Category:       latex
% >Synopsis:       \setlength{\dimen0}{10pt}
% >Arrival-Date:   Tue Jul  6 15:01:06 1999
% >Originator:     oberdiek@ruf.uni-freiburg.de (Heiko Oberdiek)
% >Description:
% The current implementation of \setlength causes an error,
% because the length specification isn't terminated properly.
% More safe:
% \def\setlength#1#2{#1=#2\relax}
%\end{verbatim}

% \subsection{Fewer fragile commands}
%
%\begin{verbatim}
% >Number:         3816
% >Category:       latex
% >Synopsis:       Argument of \@sect has an extra }.
% >Arrival-Date:   Sat Oct 22 23:11:01 +0200 2005
% >Originator:     susi@uriah.heep.sax.de (Susanne Wunsch)
%
% Use of a \raisebox in \section{} produces the error message
% mentioned in the subject.
%
% PR latex/1738 described a similar problem, which has been solved
% 10 years ago. Protecting the \raisebox with \protect solved my
% problem as well, but wouldn't it make sense to have a similar fix
% as in the PR?
%
% It is particularly confusing, that an unprotected \raisebox in a
% \section*-environment works fine, while in a \section-environment
% produces error.
%\end{verbatim}
%
% While not technically a bug, in this day and age there are few
% reasons why commands taking optional arguments should not be robust.
%
% \subsubsection{Notes on the implementation strategy}
%
% Rather than changing the kernel macros to be robust, we have decided
% to add the macro \DescribeMacro{\MakeRobust}|\MakeRobust| in
% \Lpack{fixltx2e} so that users can easily turn fragile macros into
% robust ones. A macro |\foo| is made robust by doing the simple
% |\MakeRobust{\foo}|. \Lpack{fixltx2e} makes the following kernel
% macros robust: |\(|, |\)|, |\[|, |\]|,
%  |\makebox|, |\savebox|,
% |\framebox|, |\parbox|, |\rule| and |\raisebox|.
% 
% \ldots TODO \ldots  fleqn vesion of |\[\]| 
%
% \StopEventually{}
%
% \section{Implementation}
%
% We require at least a somewhat sane version of \LaTeXe{}. Earlier
% ones where really quite different from one another.
%    \begin{macrocode}
%<*latexrelease>
\NeedsTeXFormat{LaTeX2e}[1996/06/01]
%    \end{macrocode}
%
% \section{Setup}
%
% \begin{macro}{\IncludeInRelease}
%    \begin{macrocode}
%    \end{macrocode}
% \end{macro}
%
%    \begin{macrocode}
\DeclareOption*{\let\requestedpatchdate\CurrentOption}
\DeclareOption{latest}{\let\requestedpatchdate\latexreleaseversion}
%    \end{macrocode}
%
%    \begin{macrocode}
\ExecuteOptions{latest}
\ProcessOptions\relax
%    \end{macrocode}
%
% Sanity check options, it allows some non-legal dates but always
% ensures |requestedLaTeXdate| gets set to a number.
% Generate an error if there are any non digit tokens remaining after removing the |//|.
%    \begin{macrocode}
\def\reserved@a{%
\edef\requestedLaTeXdate{\the\count@}%
\reserved@b}
\def\reserved@b#1\\{%
\def\reserved@b{#1}%
\ifx\reserved@b\@empty\else
\PackageError{latexrelease}%
             {Unexpected option \requestedpatchdate}%
             {The option must be of the form yyyy/mm/dd}%
\fi}
\afterassignment\reserved@a
\count@\expandafter\@parse@version\expandafter0\requestedpatchdate//00\@nil\\
%    \end{macrocode}
%
% less precautions needed for |\fmtversion|
%    \begin{macrocode}
\edef\currentLaTeXdate{%
   \expandafter\@parse@version\fmtversion//00\@nil}
%    \end{macrocode}
%
%    \begin{macrocode}
\ifnum\requestedLaTeXdate=\currentLaTeXdate
\PackageWarningNoLine{latexrelease}{%
  Current format date selected, no patches applied.}
\expandafter\endinput
\fi
%    \end{macrocode}
%
% A newer version of latexrelease should have been distributed with the later format.
%    \begin{macrocode}
\ifnum\currentLaTeXdate>\expandafter\@parse@version\latexreleaseversion//00\@nil
\PackageWarningNoLine{latexrelease}{%
The current package is for an older LaTeX format:\MessageBreak
LaTeX \fmtversion\space\MessageBreak
Obtain a newer version of this package!}
\expandafter\endinput
\fi
%    \end{macrocode}
% can't patch into the future, could make this an error
% but it has some uses to control package updates
% so allow for now.
%    \begin{macrocode}
\ifnum\requestedLaTeXdate>\expandafter\@parse@version\latexreleaseversion//00\@nil
\PackageWarningNoLine{latexrelease}{%
The current package is for LaTeX \latexreleaseversion:\MessageBreak
It has no patches beyond that date\MessageBreak
There may be an updated version of this package available from CTAN.}
\expandafter\endinput
\fi
%    \end{macrocode}
%
% Update the format version to the requested date.
%    \begin{macrocode}
\let\fmtversion\requestedpatchdate
\let\currentLaTeXdate\requestedLaTeXdate
%    \end{macrocode}
%
%
% \section{Individual  Changes}
%
% The code for each change will be inserted at this point, extracted from the kernel source files.
%    
%    \begin{macrocode}
%</latexrelease>
%    \end{macrocode}
% \Finale
%
\endinput

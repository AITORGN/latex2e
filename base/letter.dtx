% \iffalse meta-comment
%
% Copyright 1993-2015
% The LaTeX3 Project and any individual authors listed elsewhere
% in this file.
%
% This file is part of the LaTeX base system.
% -------------------------------------------
%
% It may be distributed and/or modified under the
% conditions of the LaTeX Project Public License, either version 1.3c
% of this license or (at your option) any later version.
% The latest version of this license is in
%    http://www.latex-project.org/lppl.txt
% and version 1.3c or later is part of all distributions of LaTeX
% version 2005/12/01 or later.
%
% This file has the LPPL maintenance status "maintained".
%
% The list of all files belonging to the LaTeX base distribution is
% given in the file `manifest.txt'. See also `legal.txt' for additional
% information.
%
% The list of derived (unpacked) files belonging to the distribution
% and covered by LPPL is defined by the unpacking scripts (with
% extension .ins) which are part of the distribution.
%
% \fi
%
% \CheckSum{980}
%% \CharacterTable
%%  {Upper-case    \A\B\C\D\E\F\G\H\I\J\K\L\M\N\O\P\Q\R\S\T\U\V\W\X\Y\Z
%%   Lower-case    \a\b\c\d\e\f\g\h\i\j\k\l\m\n\o\p\q\r\s\t\u\v\w\x\y\z
%%   Digits        \0\1\2\3\4\5\6\7\8\9
%%   Exclamation   \!     Double quote  \"     Hash (number) \#
%%   Dollar        \$     Percent       \%     Ampersand     \&
%%   Acute accent  \'     Left paren    \(     Right paren   \)
%%   Asterisk      \*     Plus          \+     Comma         \,
%%   Minus         \-     Point         \.     Solidus       \/
%%   Colon         \:     Semicolon     \;     Less than     \<
%%   Equals        \=     Greater than  \>     Question mark \?
%%   Commercial at \@     Left bracket  \[     Backslash     \\
%%   Right bracket \]     Circumflex    \^     Underscore    \_
%%   Grave accent  \`     Left brace    \{     Vertical bar  \|
%%   Right brace   \}     Tilde         \~}
%
% \iffalse
% \section{Identification}
%
%
%    This document class can only be used with \LaTeXe, so we make
%    sure that an appropriate message is displayed when another \TeX{}
%    format is used.
%    \begin{macrocode}
%<+letter>\NeedsTeXFormat{LaTeX2e}[1996/06/01]
%    \end{macrocode}
%
%    Announce the Class name and its version.
%    \begin{macrocode}
%<+letter>\ProvidesClass{letter}
%<*driver>
       \ProvidesFile{letter.drv}
%</driver>
              [2014/09/29 v1.2z
%<+letter>               Standard LaTeX document class]
%    \end{macrocode}
%
% \section{The documentation driver file}
%
%    We have our own document class to format the \LaTeXe{}
%    documentation.
% \fi
% \changes{v1.0}{1993/12/07}{Use class ltxdoc document class}
% \iffalse
%    \begin{macrocode}
%<*driver>
]
\documentclass{ltxdoc}
%    \end{macrocode}
%
%    We don't want everything to appear in the index.
%    \begin{macrocode}
\DoNotIndex{\@Alph,\@alph,\@arabic,\@badmath}
\DoNotIndex{\@centercr}
\DoNotIndex{\@empty,\@ignoretrue}
\DoNotIndex{\@ixpt}
\DoNotIndex{\@M,\@minus,\@ne,\@plus}
\DoNotIndex{\\,\addtolength}
\DoNotIndex{\advance}
\DoNotIndex{\ast,\begin,\begingroup,\bfseries,\bgroup,\box}
\DoNotIndex{\bullet}
\DoNotIndex{\cdot,\cr,\day,\DeclareOption}
\DoNotIndex{\def,\DocInput,\documentclass}
\DoNotIndex{\DoNotIndex,\egroup,\ifx,\else,\fi,\endtrivlist}
\DoNotIndex{\EnableCrossrefs,\end,\end@dblfloat,\end@float,\endgroup}
\DoNotIndex{\endlist,\everycr,\ExecuteOptions}
\DoNotIndex{\filedate,\filename,\fileversion}
\DoNotIndex{\global,\halign,\hangindent,\hbox,\hfil,\hfill,\hrule}
\DoNotIndex{\hsize,\hskip,\hspace,\hss,\ifcase,\or,\fi}
\DoNotIndex{\ifvmode,\fi,\ifnum,\fi,\input}
\DoNotIndex{\kern,\leavevmode,\let,\leftmark}
\DoNotIndex{\list,\llap,\long,\m@ne,\m@th,\mark}
\DoNotIndex{\month,\newcommand,\newcounter,\newenvironment}
\DoNotIndex{\NeedsTeXFormat,\newdimen}
\DoNotIndex{\newpage,\nobreak,\noindent,\number}
\DoNotIndex{\p@}
\DoNotIndex{\pagestyle,\par}
\DoNotIndex{\penalty,\PrintChanges,\PrintIndex,\ProcessOptions}
\DoNotIndex{\protect,\ProvidesClass,\raggedbottom,\raggedright}
\DoNotIndex{\refstepcounter,\relax,\renewcommand,\reset@font}
\DoNotIndex{\rightmargin,\rlap,\rmfamily}
\DoNotIndex{\setbox,\setcounter,\setlength}
\DoNotIndex{\skip,\slshape,\space}
\DoNotIndex{\trivlist,\typeout,\tw@}
\DoNotIndex{\vskip,\vspace,\year,\z@}
%
\CodelineNumbered
\CodelineIndex
\EnableCrossrefs
\RecordChanges
%    \end{macrocode}
%    We use so many \file{docstrip} modules that we set the
%    \texttt{StandardModuleDepth} counter to 1.
%    \begin{macrocode}
\setcounter{StandardModuleDepth}{1}
%    \end{macrocode}
%    The following command retrieves the date and version information
%    from the file.
%    \begin{macrocode}
\GetFileInfo{letter.drv}
%    \end{macrocode}
%    Some commonly used abbreviations
%    \begin{macrocode}
\newcommand*{\Lopt}[1]{\textsf {#1}}
\newcommand*{\file}[1]{\texttt {#1}}
\newcommand*{\Lcount}[1]{\textsl {\small#1}}
\newcommand*{\pstyle}[1]{\textsl {#1}}
%    \end{macrocode}
%    We also want the full details.
%    \begin{macrocode}
\begin{document}
\DocInput{letter.dtx}
\PrintIndex
% ^^A\PrintChanges
\end{document}
%</driver>
%    \end{macrocode}
% \fi
%
% \changes{v1.1}{1993/12/18}{Changes to make it work with
%    compatibility mode.}
% \changes{v1.1a}{1993/12/19}{Removed float parms}
%
% \changes{v1.1c}{1994/01/17}{Moved definition of \cs{@textop} to
%    initialization section, to counteract setting of \cs{@texttop} in
%    \cs{raggedbottom}.}
%
% \changes{v1.1d}{1994/01/20}{Corrected margins: Now computed from
%    \cs{paperwidth} and \cs{textwidth}, so that the text is
%    horizontally centered.}
% \changes{v1.1e}{1994/02/03}{Corrected checksum}
%
% \changes{v1.2a}{1994/02/12}{Removed onecolumn and twocolumn options.}
% \changes{v1.2a}{1994/02/12}{Changed label producing commands in .aux
%    file.}
%
% \changes{v1.2b}{1994/02/28}{Moved driver further up.}
%
% \changes{v1.2c}{1994/03/08}{Always write .aux file.}
% \changes{v1.2f}{1994/04/29}{Use LaTeX instead of LaTeX2e in messages}
% \changes{v1.2f}{1994/04/29}{Removed \cs{typeout} message}
% \changes{v1.2g}{1994/05/01}{Removed the use of \cs{fileversion}
%    c.s.}
% \changes{v1.2j}{1994/06/01}{Corrected driver code.}
% \changes{v1.2k}{1994/06/17}{Various documentation enhancements}
% \changes{v1.2p}{1994/12/11}{Use \cs{newcommand*} to define
%    user-level commands that take an argument}
% \changes{v1.2q}{1995/05/17}{cleaned up \cs{changes entries}}
% \changes{v1.2q}{1995/05/18}{replaced \cs{hbox to} by \cs{hb@xt@}}
% \changes{v1.2r}{1995/05/23}{removed a superfluous brace}
%
% \iffalse
% Copyright (C) 1994 LaTeX3 project, Frank Mittelbach
% and Rainer Sch\"opf, all rights reserved.
% \fi
%
% \author{Leslie Lamport and Frank Mittelbach and Rainer Sch\"opf}
%
% \title{Standard Letter Document Class for \LaTeX{} version 2e}
%
%
% \MaintainedByLaTeXTeam{LaTeX}
% \maketitle
% \tableofcontents
%
% \StopEventually
%
%
% \section{Initial Code}
%
%    In this part we define a few commands that are used later on.
%
% \begin{macro}{\@ptsize}
%    This control sequence is used to store the second digit of the
%    pointsize we are typesetting in. So, normally, it's value is one
%    of 0, 1 or 2.
%    \begin{macrocode}
%<*letter>
\newcommand\@ptsize{}
%    \end{macrocode}
% \end{macro}
%
%
%
% \subsection{Setting Paper Sizes}
%
%    The variables |\paperwidth| and |\paperheight| should reflect the
%    physical paper size after trimming. For desk printer output this
%    is usually the real paper size since there is no post-processing.
% \changes{v1.0a}{1993/12/14}{Corrected typo, A4 is not 279 mm high}
%    \begin{macrocode}
\DeclareOption{a4paper}
   {\setlength\paperheight {297mm}%
    \setlength\paperwidth  {210mm}}
\DeclareOption{a5paper}
   {\setlength\paperheight {210mm}%
    \setlength\paperwidth  {148mm}}
\DeclareOption{b5paper}
   {\setlength\paperheight {250mm}%
    \setlength\paperwidth  {176mm}}
\DeclareOption{letterpaper}
   {\setlength\paperheight {11in}%
    \setlength\paperwidth  {8.5in}}
\DeclareOption{legalpaper}
   {\setlength\paperheight {14in}%
    \setlength\paperwidth  {8.5in}}
\DeclareOption{executivepaper}
   {\setlength\paperheight {10.5in}%
    \setlength\paperwidth  {7.25in}}
%    \end{macrocode}
%
% \changes{v1.2o}{1994/08/26}{Added the option \textsf{landscape}}
%    The option \Lopt{landscape} switches the values of |\paperheight|
%    and |\paperwidth|, assuming the dimensions were given for portrait
%    paper.
%    \begin{macrocode}
\DeclareOption{landscape}
   {\setlength\@tempdima   {\paperheight}%
    \setlength\paperheight {\paperwidth}%
    \setlength\paperwidth  {\@tempdima}}
%    \end{macrocode}
%
% \subsection{Choosing the type size}
%
%    The type size options are handled by defining |\@ptsize| to
%    contain the last digit of the size in question and branching on
%    |\ifcase| statements. This is done for historical reasons to stay
%    compatible with other packages that use the |\@ptsize| variable
%    to select special actions. It makes the declarations of size
%    options less than 10pt difficult, although one can probably use
%    \texttt{9} and \texttt{8} assuming that a class wont define both
%    \Lopt{8pt} and \Lopt{18pt} options.
%
%    \begin{macrocode}
\DeclareOption{10pt}{\renewcommand\@ptsize{0}}
\DeclareOption{11pt}{\renewcommand\@ptsize{1}}
\DeclareOption{12pt}{\renewcommand\@ptsize{2}}
%    \end{macrocode}
%
%
%  \subsection{Two-side or one-side printing}
%
%    Two-sided printing was not supported in the \LaTeX~2.09 version
%    of this document class.
% \changes{v1.2e}{1994/04/28}{Added oneside option and twoside option
%    in native mode}
%    \begin{macrocode}
\if@compatibility
  \DeclareOption{twoside}{\@latexerr{No `twoside' layout for letters}%
                                   \@eha}
\else
  \DeclareOption{twoside}{\@twosidetrue  \@mparswitchtrue}
\fi
%    \end{macrocode}
%
%
%    \begin{macrocode}
\DeclareOption{oneside}{\@twosidefalse \@mparswitchfalse}
%    \end{macrocode}
%
%  \subsection{Draft option}
%
%    If the user requests \Lopt{draft} we show any overfull boxes.
%    We could probably add some more interesting stuff to this option.
%    \begin{macrocode}
\DeclareOption{draft}{\setlength\overfullrule{5pt}}
\DeclareOption{final}{\setlength\overfullrule{0pt}}
%    \end{macrocode}
%
%
%  \subsection{Equation numbering on the left}
%
%    The option \Lopt{leqno} can be used to get the equation numbers
%    on the left side of the equation.
% \changes{v1.1c}{1994/01/17}{Option leqno now read from
%    \file{leqno.clo}.}
%    \begin{macrocode}
\DeclareOption{leqno}{\input{leqno.clo}}
%    \end{macrocode}
%
%  \subsection{Flush left displays}
%
%    The option \Lopt{fleqn} redefines the displayed math environments
%    in such a way that they come out flush left, with an indentation
%    of |\mathindent| from the prevailing left margin.
% \changes{v1.1}{1993/12/18}{Corrected typo in \cs{renewcommand},
%    doubled hash marks.}
% \changes{v1.1c}{1994/01/17}{Option fleqn now read from
%    \file{fleqn.clo}.}
%    \begin{macrocode}
\DeclareOption{fleqn}{\input{fleqn.clo}}
%    \end{macrocode}
%
%
% \section{Executing Options}
%
%    Here we execute the default options to initialize certain
%    variables.
% \changes{v1.2e}{1994/04/28}{Added execution of `oneside' option}
%    \begin{macrocode}
\ExecuteOptions{letterpaper,10pt,oneside,onecolumn,final}
%    \end{macrocode}
%
%    The |\ProcessOptions| command causes the execution of the code
%    for every option \Lopt{foo} which is declared and for which the
%    user typed the \Lopt{foo} option in his |\documentclass| command.
%    For every option \Lopt{bar} he typed, which is not declared, the
%    option is assumed to be a global option.  All options will be
%    passed as document options to any |\usepackage| command in the
%    document preamble.
%    \begin{macrocode}
\ProcessOptions
%    \end{macrocode}
%    Now that all the options have been executed we can load the
%    chosen class option file that contains all size dependent code.
%    \begin{macrocode}
\input{size1\@ptsize.clo}
%    \end{macrocode}
%
%
%  \section{Loading Packages}
%
%  The standard class files do not load additional packages.
%
%
% \section{Document Layout}
% \label{sec:maincode}
%
%  In this section we are finally dealing with the nasty typographical
%  details.
%
% \subsection{Paragraphing}
%
% \begin{macro}{\lineskip}
% \begin{macro}{\normallineskip}
%    These parameters control \TeX's behavior when two lines tend to
%    come too close together.
%    \begin{macrocode}
\setlength\lineskip{1\p@}
\setlength\normallineskip{1\p@}
%    \end{macrocode}
% \end{macro}
% \end{macro}
%
% \begin{macro}{\baselinestretch}
%    This is used as a multiplier for |\baselineskip|. The default is
%    to {\em not\/} stretch the baselines.
%    \begin{macrocode}
\renewcommand\baselinestretch{}
%    \end{macrocode}
% \end{macro}
%
% \begin{macro}{\parskip}
% \begin{macro}{\parindent}
%    |\parskip| gives extra vertical space between paragraphs and
%    |\parindent| is the width of the paragraph indentation.
%    Letters are typeset without paragraph indentation.
%    \begin{macrocode}
\setlength\parskip{0.7em}
\setlength\parindent{0\p@}
%    \end{macrocode}
% \end{macro}
% \end{macro}
%
% \begin{macro}{\@lowpenalty}
% \begin{macro}{\@medpenalty}
% \begin{macro}{\@highpenalty}
%    The commands |\nopagebreak| and |\nolinebreak| put in penalties
%    to discourage these breaks at the point they are put in.
%    They use |\@lowpenalty|, |\@medpenalty| or |\@highpenalty|,
%    dependent on their argument.
%    \begin{macrocode}
\@lowpenalty   51
\@medpenalty  151
\@highpenalty 301
%    \end{macrocode}
% \end{macro}
% \end{macro}
% \end{macro}
%
% \begin{macro}{\clubpenalty}
% \begin{macro}{\widowpenalty}
%    These penalties are used to discourage club and widow lines.
%    Because we use their default values we only show them here,
%    commented out.
%    \begin{macrocode}
% \clubpenalty  150
% \widowpenalty 150
%    \end{macrocode}
% \end{macro}
% \end{macro}
%
% \begin{macro}{\displaywidowpenalty}
% \begin{macro}{\predisplaypenalty}
% \begin{macro}{\postdisplaypenalty}
%    Discourage (but not so much) widows in front of a math display
%    and forbid breaking directly in front of a display. Allow break
%    after a display without a penalty. Again the default values are
%    used, therefore we only show them here.
%    \begin{macrocode}
% \displaywidowpenalty 50
% \predisplaypenalty   10000
% \postdisplaypenalty  0
%    \end{macrocode}
% \end{macro}
% \end{macro}
% \end{macro}
%
% \begin{macro}{\interlinepenalty}
%    Allow the breaking of a page in the middle of a paragraph.
%    \begin{macrocode}
% \interlinepenalty 0
%    \end{macrocode}
% \end{macro}
%
%
% \begin{macro}{\brokenpenalty}
%    We allow the breaking of a page after a hyphenated line.
%    \begin{macrocode}
% \brokenpenalty 0
%    \end{macrocode}
% \end{macro}
%
%
% \subsection{Page Layout}
%
%    All margin dimensions are measured from a point one inch from the
%    top and lefthand side of the page.
%
% \subsubsection{Vertical spacing}
%
% \begin{macro}{\headheight}
% \begin{macro}{\headsep}
%    The |\headheight| is the height of the box that will contain the
%    running head. The |\headsep| is the distance between the bottom
%    of the running head and the top of the text. |\topskip| is the
%    |\baselineskip| for the first line on a page, its value depends
%    on the size option that was specified. Therefore it is
%    initialized in one of the \file{size1x.clo} files.
%    \begin{macrocode}
\setlength\headheight{12\p@}
\setlength\headsep   {45\p@}
%    \end{macrocode}
% \end{macro}
% \end{macro}
%
% \begin{macro}{\footskip}
%    The distance from the baseline of the box which contains the
%    running footer to the baseline of last line of text is controlled
%    by the |\footskip|.
%    Bottom of page:
%    \begin{macrocode}
\setlength\footskip{25\p@}
%    \end{macrocode}
% \end{macro}
%
% \changes{v1.2n}{1994/07/13}{Setting of \cs{maxdepth} and
%    \cs{@maxdepth} already happens in the size.. files, removed
%    here}
%
% \subsubsection{The dimension of text}
%
% \begin{macro}{\textwidth}
%    When we are in compatibility mode we have to make sure that the
%    dimensions of the printed area are not different from what the
%    user was used to see.
%
% \changes{v1.2n}{1994/07/13}{Setting \cs{textwidth} to a fixed value
%    should only happen in compatibility mode}
%    \begin{macrocode}
\if@compatibility
  \setlength\textwidth{365\p@}
%    \end{macrocode}
% \end{macro}
%
% \begin{macro}{\textheight}
%    Now that we have computed the width of the text, we have to take
%    care of the height. The |\textheight| is the height of text
%    (including footnotes and figures, excluding running head and
%    foot).
% \changes{v1.2n}{1994/07/13}{Setting \cs{textheight} to a fixed value
%    should only happen in compatibility mode}
%    \begin{macrocode}
  \setlength\textheight{505\p@}
\fi
%    \end{macrocode}
%    In native mode we use the dimensions as they were computed by one
%    of the \Lopt{xxpt} options, together with one of the
%    \Lopt{..paper} options.
% \end{macro}
%
% \subsubsection{Margins}
%
% \begin{macro}{\oddsidemargin}
% \begin{macro}{\evensidemargin}
% \begin{macro}{\marginparwidth}
% \changes{v1.1d}{1994/01/20}{Corrected margins: Now computed from
%    \cs{paperwidth} and \cs{textwidth}.}  The values of
%    |\oddsidemargin| and |\evensidemargin| are computed from those of
%    |\paperwidth| and |\textwidth|.
% \changes{v1.2n}{1994/07/13}{Added fixed values for these parameters
%    in compatibility mode}
%    \begin{macrocode}
\if@compatibility
  \setlength\oddsidemargin{53pt}
  \setlength\evensidemargin{53pt}
  \setlength\marginparwidth{90pt}
\else
  \setlength\@tempdima{\paperwidth}
  \addtolength\@tempdima{-2in}
  \addtolength\@tempdima{-\textwidth}
  \setlength\oddsidemargin   {.5\@tempdima}
  \setlength\evensidemargin  {\oddsidemargin}
  \setlength\marginparwidth  {90\p@}
\fi
%    \end{macrocode}
% \end{macro}
% \end{macro}
% \end{macro}
%
% \begin{macro}{\marginparsep}
% \begin{macro}{\marginparpush}
%    The horizontal space between the main text and marginal notes is
%    determined by |\marginparsep|, the minimum vertical separation
%    between two marginal notes is controlled by |\marginparpush|.
%    \begin{macrocode}
\setlength\marginparsep {11\p@}
\setlength\marginparpush{5\p@}
%    \end{macrocode}
% \end{macro}
% \end{macro}
%
% \begin{macro}{\topmargin}
%    The |\topmargin| is the distance between the top of `the
%    printable area' --which is 1 inch below the top of the paper--
%    and the top of the box which contains the running head.
%
%    It can now be computed from the values set above.
%    \begin{macrocode}
\setlength\topmargin{27pt}
%    \end{macrocode}
% \end{macro}
%
%
% \subsubsection{Footnotes}
%
% \begin{macro}{\footnotesep}
%    |\footnotesep| is the height of the strut placed at the beginning
%    of every footnote. It equals the  height of a normal
%    |\footnotesize| strut in this
%    class, thus no extra space occurs between footnotes.
%    \begin{macrocode}
\setlength\footnotesep{12\p@}
%    \end{macrocode}
% \end{macro}
%
% \begin{macro}{\footins}
%    |\skip\footins| is the space between the last line of the main
%    text and the top of the first footnote.
%    \begin{macrocode}
\setlength{\skip\footins}{10\p@ \@plus 2\p@ \@minus 4\p@}
%    \end{macrocode}
% \end{macro}
%
%
% \subsection{Page Styles}
%
%    The page style \pstyle{foo} is defined by defining the command
%    |\ps@foo|.   This command should make only local definitions.
%    There should be no stray spaces in the definition, since they
%    could lead to mysterious extra spaces in the output (well, that's
%    something that should be always avoided).
%
% \begin{macro}{\@evenhead}
% \begin{macro}{\@oddhead}
% \begin{macro}{\@evenfoot}
% \begin{macro}{\@oddfoot}
%    The |\ps@...| command defines the macros |\@oddhead|,
%    |\@oddfoot|, |\@evenhead|, and |\@evenfoot| to define the running
%    heads and feet---e.g., |\@oddhead| is the macro to produce the
%    contents of the heading box for odd-numbered pages.  It is called
%    inside an |\hbox| of width |\textwidth|.
% \end{macro}
% \end{macro}
% \end{macro}
% \end{macro}
%
% \subsubsection{Marking conventions}
%
%    To make headings determined by the sectioning commands, the page
%    style defines the commands |\chaptermark|, |\sectionmark|,
%    \ldots, where |\chaptermark{|\meta{text}|}| is called by
%    |\chapter| to set a mark, and so on.
%
%    The |\...mark| commands and the |\...head| macros are defined
%    with the help of the following macros.  (All the |\...mark|
%    commands should be initialized to no-ops.)
%
%    \LaTeX{} extends \TeX's |\mark| facility by producing two kinds
%    of marks, a `left' and a `right' mark, using the following
%    commands:
%    \begin{flushleft}
%     |\markboth{|\meta{left}|}{|\meta{right}|}|: Adds both marks.
%
%     |\markright{|\meta{right}|}|: Adds a `right' mark.
%
%     |\leftmark|: Used in the |\@oddhead|, |\@oddfoot|, |\@evenhead|
%                  or |\@evenfoot| macros, it gets the current `left'
%                  mark.  |\leftmark| works like \TeX's |\botmark|
%                  command.
%
%     |\rightmark|: Used in the |\@oddhead|, |\@oddfoot|, |\@evenhead|
%                   or  |\@evenfoot| macros, it gets the current
%                   `right' mark. |\rightmark| works like \TeX's
%                   |\firstmark| command.
%    \end{flushleft}
%
%    The marking commands work reasonably well for right marks
%    `numbered within' left marks--e.g., the left mark is changed by a
%    |\chapter| command and the right mark is changed by a |\section|
%    command.  However, it does produce somewhat anomalous results if
%    two |\markboth|'s occur on the same page.
%
%
%    Commands like |\tableofcontents| that should set the marks in some
%    page styles use a |\@mkboth| command, which is |\let| by the
%    pagestyle command (|\ps@...|)  to |\markboth| for setting the
%    heading or to |\@gobbletwo| to do nothing.
%
%
% \subsubsection{Defining the page styles}
% \label{sec:pagestyle}
%
%    The pagestyles \pstyle{empty} and \pstyle{plain} are defined in
%    the \LaTeX{} kernel (\file{ltpage.dtx}), but these definitions
%    are changed to a simpler version for this document class.
%
% \begin{macro}{\ps@headings}
%    The definition of the page style \pstyle{headings} has to be
%    different for two sided printing than it is for one sided
%    printing.
%
% \changes{v1.2e}{1994/04/28}{Added twoside support}
%    \begin{macrocode}
\if@twoside
  \def\ps@headings{%
%    \end{macrocode}
%    The running feet are empty in this page style.
%    \begin{macrocode}
      \let\@oddfoot\@empty\let\@evenfoot\@empty
%    \end{macrocode}
%    The running head contains some information about this letter. The
%    head is the same for even and odd pages.
% \changes{v1.2i}{1994/05/12}{Made running heads the same on odd and
%    even pages}
% \changes{v1.2k}{1994/06/17}{Added a missing brace}
%    \begin{macrocode}
      \def\@oddhead{\slshape\headtoname{} \ignorespaces\toname
                    \hfil \@date
                    \hfil \pagename{} \thepage}%
      \let\@evenhead\@oddhead}
%    \end{macrocode}
%    For one sided printing we don't need to define |\@evenhead| so
%    the definition is somewhat simpler.
%    \begin{macrocode}
\else
  \def\ps@headings{%
      \let\@oddfoot\@empty
      \def\@oddhead{\slshape\headtoname{} \ignorespaces\toname
                    \hfil \@date
                    \hfil \pagename{} \thepage}}
\fi
%    \end{macrocode}
% \end{macro}
%
%
% \begin{macro}{\ps@empty}
%    The definition of the page style \pstyle{empty} is simple:
%    No running head or foot at all.
% \changes{v1.2e}{1994/04/28}{Also make even feet and head empty}
%    \begin{macrocode}
\def\ps@empty{%
      \let\@oddfoot\@empty\let\@oddhead\@empty
      \let\@evenfoot\@empty\let\@evenhead\@empty}
%    \end{macrocode}
% \end{macro}
%
% \begin{macro}{\ps@firstpage}
%    The page style \pstyle{firstpage} puts the telephone number
%    in the proper place for the letterhead. It should be adapted
%    to site conventions. The size of the number is determined
%    depending on the main size.
%    \begin{macrocode}
\def\ps@firstpage{%
     \let\@oddhead\@empty
     \def\@oddfoot{\raisebox{-45\p@}[\z@]{%
        \hb@xt@\textwidth{\hspace*{100\p@}%
          \ifcase \@ptsize\relax
             \normalsize
          \or
             \small
          \or
             \footnotesize
          \fi
        \fromlocation \hfill \telephonenum}}\hss}}
%    \end{macrocode}
% \end{macro}
%
% \begin{macro}{\ps@plain}
%    The definition of the page style \pstyle{plain} is again simple.
% \changes{v1.2e}{1994/04/28}{Added definition of \cs{@evenfoot}
%    because of twoside option}
% \changes{v1.2s}{1995/05/25}{replace \cs{rmfamily} with
%    \cs{normalfont} (PR 1578)}
%    \begin{macrocode}
\def\ps@plain{%
      \let\@oddhead\@empty
      \def\@oddfoot{\normalfont\hfil\thepage\hfil}%
      \def\@evenfoot{\normalfont\hfil\thepage\hfil}}
%    \end{macrocode}
% \end{macro}
%
%
%
% \section{Document Markup}
%
% \subsection{Global Declarations}
%
% \begin{macro}{\name}
% \begin{macro}{\signature}
% \begin{macro}{\address}
% \begin{macro}{\location}
% \begin{macro}{\telephone}
% The following declarations, shown with examples, give information
% about the sender:
% \begin{itemize}
%   \item |\name{Dr. L. User}| : to be used for the return address on
%          the envelope.
%    \begin{macrocode}
\newcommand*{\name}[1]{\def\fromname{#1}}
%    \end{macrocode}
%   \item |\signature{Larry User}| : goes after the closing.
%    \begin{macrocode}
\newcommand*{\signature}[1]{\def\fromsig{#1}}
%    \end{macrocode}
%   \item |\address{3245 Foo St.\\Gnu York}| : used as the return
%          address in the
%          letter and on the envelope.  If not declared, then an
%          institutional standard address is used.
%    \begin{macrocode}
\newcommand*{\address}[1]{\def\fromaddress{#1}}
%    \end{macrocode}
%   \item |\location{Room 374}| : Acts as modifier to the standard
%          institutional address.
%    \begin{macrocode}
\newcommand*{\location}[1]{\def\fromlocation{#1}}
%    \end{macrocode}
%   \item |\telephone{(415)123-4567}| : Just in case some style puts it
%          on the letter.
% \changes{v1.2m}{1994/06/23}{Removed typo (\#[] instead of [1])}
%    \begin{macrocode}
\newcommand*{\telephone}[1]{\def\telephonenum{#1}}
%    \end{macrocode}
% \end{itemize}
% \end{macro}
% \end{macro}
% \end{macro}
% \end{macro}
% \end{macro}
%
% \begin{macro}{\fromname}
% \begin{macro}{\fromsig}
% \begin{macro}{\fromaddress}
% \begin{macro}{\fromlocation}
% \begin{macro}{\telephonenum}
%    We make sure that the internal control sequences that are used to
%    store the information exist and are empty.
%    \begin{macrocode}
\name{}
\signature{}
\address{}
\location{}
\telephone{}
%    \end{macrocode}
% \end{macro}
% \end{macro}
% \end{macro}
% \end{macro}
% \end{macro}
%
% \begin{macro}{\makelabels}
%     The |\makelabels| declaration causes mailing labels to be made.
%    \begin{macrocode}
\newcommand*{\makelabels}{%
%    \end{macrocode}
%    At the beginning of the document, we need to activate the
%    |\@mlabel| and |\@startlabels| commands, as well as  write
%    |\@startlabels| to the |.aux| file.
%    \begin{macrocode}
  \AtBeginDocument{%
     \let\@startlabels\startlabels
     \let\@mlabel\mlabel
     \if@filesw
       \immediate\write\@mainaux{\string\@startlabels}\fi}%
%    \end{macrocode}
%    At the end of the document we need to write |\clearpage| to the
%    |.aux| file.
%    \begin{macrocode}
  \AtEndDocument{%
     \if@filesw\immediate\write\@mainaux{\string\clearpage}\fi}}
%    \end{macrocode}
%    |\makelabels| is allowed only before the |\begin{document}|
%    command.
%    \begin{macrocode}
\@onlypreamble\makelabels
%    \end{macrocode}
% \end{macro}
%
% \subsection{The generic letter commands}
%
% \begin{environment}{letter}
%     The letter environment creates a new letter, starting from page
%     $1$, with footnotes starting from $1$ as well.
%     (The first page is unnumbered.) It has a single argument, which
%     is the addressee and his address, as in
%\begin{verbatim}
%  \begin{letter}{Sam Jones \\
%                 Institute for Retarded Study\\
%                 Princeton, N.J.}
%\end{verbatim}
%     Local declarations, such as |\address|, can follow the
%     |\begin{letter}|.
% \changes{v1.2e}{1994/04/28}{When printing twosided a new letter
%    shouldn't start on the back of the previous one. Added insertion
%    of a blank page when necessary.}
% \changes{v1.2z}{1999/04/29}{Reset footnote counter to 0, (PR 2957).}
%    \begin{macrocode}
\newenvironment{letter}[1]
  {\newpage
    \if@twoside \ifodd\c@page
                \else\thispagestyle{empty}\null\newpage\fi
    \fi
    \c@page \@ne
    \c@footnote \z@
    \interlinepenalty=200 % smaller than the TeXbook value
%    \end{macrocode}
%    The |\leavevmode| and |\ignorespaces| commands are there for
%    protecting against an empty argument.
%    \begin{macrocode}
    \@processto{\leavevmode\ignorespaces #1}}
%    \end{macrocode}
%    The end of the environment possibly writes the address
%    information on the \file{.aux} file.
% \changes{v1.2p}{1994/12/11}{No longer redefine \cs{protect} but use
%    one of the available settings}
%    \begin{macrocode}
  {\stopletter\@@par\pagebreak\@@par
    \if@filesw
      \begingroup
        \let\\=\relax
        \let\protect\@unexpandable@protect
        \immediate\write\@auxout{%
          \string\@mlabel{\returnaddress}{\toname\\\toaddress}}%
      \endgroup
    \fi}
%    \end{macrocode}
% \end{environment}
%
%
% \begin{macro}{\@processto}
% \begin{macro}{\@xproc}
% \begin{macro}{\@yproc}
%    |\@processto| gets the |\toname| and |\toaddress| from the letter
%    environment's macro argument. |\@xproc| and |\@yproc| are auxiliary
%    macros.
%    \begin{macrocode}
\long\def\@processto#1{%
  \@xproc #1\\@@@%
  \ifx\toaddress\@empty
  \else
    \@yproc #1@@@%
  \fi}
%    \end{macrocode}
%
%    \begin{macrocode}
\long\def\@xproc #1\\#2@@@{\def\toname{#1}\def\toaddress{#2}}
%    \end{macrocode}
%
%    \begin{macrocode}
\long\def\@yproc #1\\#2@@@{\def\toaddress{#2}}
%    \end{macrocode}
% \end{macro}
% \end{macro}
% \end{macro}
%
%
% \subsubsection{Page breaking control}
%
% \begin{macro}{\stopbreaks}
%    When the command |\stopbreaks| is issued no page breaks should
%    occur until |\startbreaks| is called.
%    \begin{macrocode}
\newcommand*{\stopbreaks}{%
  \interlinepenalty\@M
  \def\par{\@@par\nobreak}%
  \let\\\@nobreakcr
  \let\vspace\@nobreakvspace}
%    \end{macrocode}
% \end{macro}
%
% \begin{macro}{\nobreakvspace}
% \changes{v1.2t}{1996/01/20}{Made robust pr/2049}
% \begin{macro}{\@nobreakvspacex}
% \begin{macro}{\@nobreakcr}
%    These are needed by |\stopbreaks|
%    \begin{macrocode}
\DeclareRobustCommand\@nobreakvspace
   {\@ifstar\@nobreakvspacex\@nobreakvspacex}
%    \end{macrocode}
%
%    \begin{macrocode}
\def\@nobreakvspacex#1{%
  \ifvmode
    \nobreak\vskip #1\relax
  \else
    \@bsphack\vadjust{\nobreak\vskip #1}\@esphack
  \fi}
%    \end{macrocode}
%
% \changes{v1.2w}{1997/01/07}
%     {Simplify definition, for ltspace 1.2u. /2341}
%    \begin{macrocode}
\def\@nobreakcr{\@ifstar{\@normalcr*}{\@normalcr*}}
%    \end{macrocode}
% \end{macro}
% \end{macro}
% \end{macro}
%
% \begin{macro}{\startbreaks}
%    This cancels the effect of |\stopbreaks|.
%    \begin{macrocode}
\newcommand*{\startbreaks}{%
  \let\\\@normalcr
  \interlinepenalty 200%
  \def\par{\@@par\penalty 200\relax}}
%    \end{macrocode}
% \end{macro}
%
%
% \begin{macro}{\longindentation}
%    The size of the indent to use before the closing of the letter.
%    \begin{macrocode}
\newdimen\longindentation
\longindentation=.5\textwidth
%    \end{macrocode}
% \end{macro}
%
% \begin{macro}{\indentedwidth}
%    The width of the closing of the letter.
%    \begin{macrocode}
\newdimen\indentedwidth
\indentedwidth=\textwidth
\advance\indentedwidth -\longindentation
%    \end{macrocode}
% \end{macro}
%
% \begin{macro}{\opening}
%     Text is begun with the |\opening| command, whose argument
%     generates the salutation, as in
%\begin{verbatim}
%      \opening{Dear Henry,}
%\end{verbatim}
%    This should produce everything up to and including the
%    `Dear Henry,' and a |\par| command that follows.
%    Since there's a |\vfil| at the bottom of every page,
%    it can add vertical fill to position a short letter.
%    It should use the following commands:
%   \begin{itemize}
%   \item |\toname| : name part of `to' address.
%                     Will be one line long.
%   \item |\toaddress| : address part of `to' address.
%                        The lines separated by |\\|.
%   \item |\fromname| : name of sender.
%   \item |\fromaddress| : argument of current |\address|
%     declaration-- null if none.  Should use standard institutional
%     address if null.
%   \item |\fromlocation| : argument of current |\location|
%     declaration--null if none.
%   \item |\telephonenum| : argument of current |\telephone|
%     declaration--null if none.
%    \end{itemize}
%    \begin{macrocode}
\newcommand*{\opening}[1]{\ifx\@empty\fromaddress
  \thispagestyle{firstpage}%
    {\raggedleft\@date\par}%
  \else  % home address
    \thispagestyle{empty}%
%    \end{macrocode}
% \changes{v1.2y}{1998/05/10}{Avoid space at right of tabular
%    (pr2794)}
%    \begin{macrocode}
    {\raggedleft\begin{tabular}{l@{}}\ignorespaces
      \fromaddress \\*[2\parskip]%
      \@date \end{tabular}\par}%
  \fi
  \vspace{2\parskip}%
  {\raggedright \toname \\ \toaddress \par}%
  \vspace{2\parskip}%
  #1\par\nobreak}
%    \end{macrocode}
% \end{macro}
%
% \begin{macro}{\closing}
%     The body of the letter follows, ended by a |\closing| command,
%     as in
%\begin{verbatim}
%            \closing{Yours truly,}
%\end{verbatim}
%     This command generates the closing matter, and the signature.
%     An obvious thing to do is to use a |\parbox| for the closing
%     and the signature.  Should use the following:
%    \begin{itemize}
%      \item |\fromsig| : argument of current |\signature| declaration
%                   or, if null, the |\fromname|.
%      \item |\stopbreaks| : a macro that inhibits page breaking.
%    \end{itemize}
%    \begin{macrocode}
\newcommand{\closing}[1]{\par\nobreak\vspace{\parskip}%
  \stopbreaks
  \noindent
  \ifx\@empty\fromaddress\else
  \hspace*{\longindentation}\fi
  \parbox{\indentedwidth}{\raggedright
       \ignorespaces #1\\[6\medskipamount]%
       \ifx\@empty\fromsig
           \fromname
       \else \fromsig \fi\strut}%
   \par}
%    \end{macrocode}
% \end{macro}
%
% \begin{macro}{\smallskipamount}
% \begin{macro}{\medskipamount}
% \begin{macro}{\bigskipamount}
%   Of these three, only |\medskipmount| is actually used above.
%    \begin{macrocode}
%\smallskipamount=.5\parskip
\medskipamount=\parskip
%\bigskipamount=2\parskip
%    \end{macrocode}
% \end{macro}
% \end{macro}
% \end{macro}
%
% \begin{macro}{\cc}
% \begin{macro}{\encl}
% \begin{macro}{\ps}
%    After the |\closing| you can put arbitrary stuff, which is typeset
%    with zero |\parindent| and no page breaking.  Commands designed
%    for use after the closing are:
%    \begin{verbatim}
%  \cc{Tinker\\Evers\\Chance}
%\end{verbatim}
%    which produces:\\
%    \begin{tabular}{ll}
%        cc: & Tinker\\
%            & Evers\\
%            & Chance
%    \end{tabular}\\
%    Note the obvious use of |\parbox|.
% \changes{v1.2s}{1995/05/25}{replace \cs{reset@font} with
%    \cs{normalfont}; remove \cs{rm}}
%    \begin{macrocode}
\newcommand*{\cc}[1]{%
  \par\noindent
  \parbox[t]{\textwidth}{%
    \@hangfrom{\normalfont\ccname: }%
    \ignorespaces #1\strut}\par}
%    \end{macrocode}
%\begin{verbatim}
%  \encl{Foo(2)\\Bar}
%\end{verbatim}
%    which produces:\\
%    \begin{tabular}{ll}
%        encl: & Foo(2)\\
%              & Bar
%    \end{tabular}
%    \begin{macrocode}
\newcommand*{\encl}[1]{%
  \par\noindent
  \parbox[t]{\textwidth}{%
    \@hangfrom{\normalfont\enclname: }%
    \ignorespaces #1\strut}\par}
%    \end{macrocode}
%
%    The only thing |\ps| needs to do is call |\startbreaks|,
%        which allows page breaking again.
%    \begin{macrocode}
\newcommand*{\ps}{\par\startbreaks}
%    \end{macrocode}
% \end{macro}
% \end{macro}
% \end{macro}
%
%
% \begin{macro}{\stopletter}
%     The |\stopletter| command is called by |\endletter| to do the
%     following:
%    \begin{itemize}
%     \item Add any desired  fill  or other material at the end of the
%       letter.
%     \item Define |\returnaddress| to be the return address for the
%       mailing label.
%       More precisely, it is the first argument of the |\mlabel|
%       command described below. It should be defined to null if the
%       return address doesn't appear on the labels.  Any command,
%       other than |\\|, that should not be expanded until the |\mlabel|
%       command is actually executed must be preceded by |\protect|.
%       Whenever possible, |\protect| commands in the definition of
%       |\returnaddress|---it's much more efficient that way.  In
%       particular, when the standard return address is used, you
%       should define |\returnaddress| to something like
%       |\protect\standardreturnaddress|.
%    \end{itemize}
%    \begin{macrocode}
\newcommand*{\stopletter}{}
%    \end{macrocode}
% \end{macro}
%
% \subsection{Customizing the labels}
%
%    Commands for generating the labels are put on the \file{.aux} file,
%    which is read in and processed by the |\end{document}| command.
%    You have to define the following two commands:
%    \begin{itemize}
%    \item |\startlabels| : Should reset the page layout parameters if
%      necessary.
%    \item |\mlabel|\marg{return address}\marg{to address} : Command to
%      generate a single label.
%    \end{itemize}
%
%
%
% \begin{macro}{\returnaddress}
%    The return address for the mailing labels can be stored in this
%    macro.
%    \begin{macrocode}
\newcommand*{\returnaddress}{}
%    \end{macrocode}
% \end{macro}
%
% \begin{macro}{\labelcount}
%    A register to count the labels
%    \begin{macrocode}
\newcount\labelcount
%    \end{macrocode}
% \end{macro}
%
% \begin{macro}{\startlabels}
%     The following |\startlabels| command sets things up for
%     producing labels in two columns of five 2" $\times$ 4-1/4"
%     labels each, suitable for reproducing onto Avery brand number
%     5352 address labels.
% \changes{v1.2q}{1995/03/14}{changed value of \cs{columnsep} by 1pt}
%    \begin{macrocode}
\newcommand*{\startlabels}{\labelcount\z@
  \pagestyle{empty}%
  \let\@texttop\relax
  \topmargin -50\p@
  \headsep \z@
  \oddsidemargin -35\p@
  \evensidemargin -35\p@
  \textheight 10in
  \@colht\textheight  \@colroom\textheight \vsize\textheight
  \textwidth 550\p@
  \columnsep 26\p@
  \ifcase \@ptsize\relax
    \normalsize
  \or
    \small
  \or
    \footnotesize
  \fi
  \baselineskip \z@
  \lineskip \z@
  \boxmaxdepth \z@
  \parindent \z@
  \twocolumn\relax}
%    \end{macrocode}
% \end{macro}
%
% \begin{macro}{\@startlabels}
%    |\@startlabels| is the command name that is written to the |.aux|
%    file. It is a no-op at first, and defined to be the same as
%    |\startlabels| in the |\begin{document}| hook.
%    \begin{macrocode}
\let\@startlabels=\relax
%    \end{macrocode}
% \end{macro}
%
% \begin{macro}{\mlabel}
%    This command prints an address label; it is used when the user
%    specified |\makelabels| in the preamble of his document. The
%    command |\mlabel| takes two arguments; the second argument is
%    supposed to be the address; the first argument can be used to
%    print a return address. In this document class we ignore the
%    first argument. Also the labels are supposed to be 2~inch high
%    and 3.6~inch wide. When your address labels have a different you
%    will have to define your own |\mlabel| command.
% \changes{v1.2p}{1994/12/11}{Redefined \cs{mlabel} to not use the
%    \cs{setbox} primitive}
% \changes{v1.2q}{1995/03/14}{changed width of the labels slightly to
%    prevent \LaTeX\ stuffing two on each line}
%    \begin{macrocode}
\newcommand*{\mlabel}[2]{%
  \parbox[b][2in][c]{262\p@}{\strut\ignorespaces #2}%
  }
%    \end{macrocode}
% \end{macro}
%
% \begin{macro}{\@mlabel}
%    The macro |\@mlabel| is written to the \file{.aux} file instead
%    of |\mlabel|.  This allows us to make it a no-op by default,
%    and then activate it in the |\begin{document}| hook.
%    \begin{macrocode}
\let\@mlabel=\@gobbletwo
%    \end{macrocode}
% \end{macro}
%
%
% \subsection{Lists}
%
% \subsubsection{General List Parameters}
%
%    The following commands are used to set the default values for the
%    list environment's parameters. See the \LaTeX{} manual for an
%    explanation of the meanings of the parameters.  Defaults for the
%    list environment are set as follows.  First, |\rightmargin|,
%    |\listparindent| and |\itemindent| are set to 0pt.  Then, for a
%    \texttt{K}th level list, the command |\@listK| is called, where
%    `\texttt{K}' denotes `\texttt{i}', `\texttt{ii}', ... ,
%    `\texttt{vi}'.  (I.e., |\@listiii| is called for a third-level
%    list.)  By convention, |\@listK| should set |\leftmargin| to
%    |\leftmarginK|.
%
% \begin{macro}{\leftmargin}
% \begin{macro}{\leftmargini}
% \begin{macro}{\leftmarginii}
% \begin{macro}{\leftmarginiii}
% \begin{macro}{\leftmarginiv}
% \begin{macro}{\leftmarginv}
% \begin{macro}{\leftmarginvi}
% For efficiency, level-one list's values are defined at top level, and
% |\@listi| is defined to set only |\leftmargin|.
%    \begin{macrocode}
\setlength\leftmargini  {2.5em}
%    \end{macrocode}
%    The following three are calculated so  that they are larger than
%    the sum of |\labelsep| and the width of the default labels (which
%    are `(m)', `vii.' and `M.').
%    \begin{macrocode}
\setlength\leftmarginii  {2.2em}
\setlength\leftmarginiii {1.87em}
\setlength\leftmarginiv  {1.7em}
\setlength\leftmarginv  {1em}
\setlength\leftmarginvi {1em}
%    \end{macrocode}
%    Here we set the top level leftmargin.
%    \begin{macrocode}
\setlength\leftmargin    {\leftmargini}
%    \end{macrocode}
% \end{macro}
% \end{macro}
% \end{macro}
% \end{macro}
% \end{macro}
% \end{macro}
% \end{macro}
%
% \begin{macro}{\labelsep}
% \begin{macro}{\labelwidth}
%    |\labelsep| is the distance between the label and the text of an
%    item; |\labelwidth| is the width of the label.
%    \begin{macrocode}
\setlength  \labelsep  {5\p@}
\setlength  \labelwidth{\leftmargini}
\addtolength\labelwidth{-\labelsep}
%    \end{macrocode}
% \end{macro}
% \end{macro}
%
% \begin{macro}{\partopsep}
%    When the user leaves a blank line before the environment an extra
%    vertical space of |\partopsep| is inserted, in addition to
%    |\parskip| and |\topsep|.
%    \begin{macrocode}
\setlength\partopsep{0\p@}
%    \end{macrocode}
% \end{macro}
%
% \begin{macro}{\@beginparpenalty}
% \begin{macro}{\@endparpenalty}
%    These penalties are inserted before and after a list or paragraph
%    environment. They are set to a bonus value to encourage page
%    breaking at these points.
% \begin{macro}{\@itempenalty}
%    This penalty is inserted between list items.
%    \begin{macrocode}
\@beginparpenalty -\@lowpenalty
\@endparpenalty   -\@lowpenalty
\@itempenalty     -\@lowpenalty
%    \end{macrocode}
% \end{macro}
% \end{macro}
% \end{macro}
%
% \begin{macro}{\@listI}
% \begin{macro}{\@listi}
% |\@listI| defines top level and |\@listi| values of
% |\leftmargin|, |\parsep|, |\topsep|, and |\itemsep|
%
%    \begin{macrocode}
\def\@listI{\setlength\leftmargin{\leftmargini}%
            \setlength\parsep {0\p@}%
            \setlength\topsep {.4em}%
            \setlength\itemsep{.4em}}
\let\@listi\@listI
%    \end{macrocode}
%    We have to initialize these parameters.
%    \begin{macrocode}
\@listi
%    \end{macrocode}
% \end{macro}
% \end{macro}
%
% \begin{macro}{\@listii}
% \begin{macro}{\@listiii}
% \begin{macro}{\@listiv}
% \begin{macro}{\@listv}
% \begin{macro}{\@listvi}
%    Here are the same macros for the higher level lists.
%    \begin{macrocode}
\def\@listii {\setlength  \leftmargin{\leftmarginii}%
              \setlength  \labelwidth{\leftmarginii}%
              \addtolength\labelwidth{-\labelsep}}
\def\@listiii{\setlength  \leftmargin{\leftmarginiii}%
              \setlength  \labelwidth{\leftmarginiii}%
              \addtolength\labelwidth{-\labelsep}%
              \setlength  \topsep    {.2em}%
              \setlength  \itemsep   {\topsep}}
\def\@listiv {\setlength  \leftmargin{\leftmarginiv}%
              \setlength  \labelwidth{\leftmarginiv}%
              \addtolength\labelwidth{-\labelsep}}
\def\@listv  {\setlength  \leftmargin{\leftmarginv}%
              \setlength  \labelwidth{\leftmarginv}%
              \addtolength\labelwidth{-\labelsep}}
\def\@listvi {\setlength  \leftmargin{\leftmarginvi}%
              \setlength  \labelwidth{\leftmarginvi}%
              \addtolength\labelwidth{-\labelsep}}
%    \end{macrocode}
% \end{macro}
% \end{macro}
% \end{macro}
% \end{macro}
% \end{macro}
%
% \subsubsection{Enumerate}
%
%    The enumerate environment uses  four counters: \Lcount{enumi},
%    \Lcount{enumii}, \Lcount{enumiii} and \Lcount{enumiv}, where
%    \Lcount{enumN} controls the numbering of the Nth level
%    enumeration.
%
% \begin{macro}{\theenumi}
% \begin{macro}{\theenumii}
% \begin{macro}{\theenumiii}
% \begin{macro}{\theenumiv}
%    The counters are already defined in the \LaTeX{} kernel
%    (\file{ltlists.dtx}), but their representation is changed here.
%
%    \begin{macrocode}
\renewcommand\theenumi{\@arabic\c@enumi}
\renewcommand\theenumii{\@alph\c@enumii}
\renewcommand\theenumiii{\@roman\c@enumiii}
\renewcommand\theenumiv{\@Alph\c@enumiv}
%    \end{macrocode}
% \end{macro}
% \end{macro}
% \end{macro}
% \end{macro}
%
% \begin{macro}{\labelenumi}
% \begin{macro}{\labelenumii}
% \begin{macro}{\labelenumiii}
% \begin{macro}{\labelenumiv}
%    The commands |\labelenumi| ... |\labelenumiv| generate the label
%    for each item.
%    \begin{macrocode}
\newcommand\labelenumi{\theenumi.}
\newcommand\labelenumii{(\theenumii)}
\newcommand\labelenumiii{\theenumiii.}
\newcommand\labelenumiv{\theenumiv.}
%    \end{macrocode}
% \end{macro}
% \end{macro}
% \end{macro}
% \end{macro}
%
% \begin{macro}{\p@enumii}
% \begin{macro}{\p@enumiii}
% \begin{macro}{\p@enumiv}
%    The expansion of |\p@enumN||\theenumN| defines the output of a
%    |\ref| command when referencing an item of the Nth level of an
%    enumerated list.
%    \begin{macrocode}
\renewcommand\p@enumii{\theenumi}
\renewcommand\p@enumiii{\theenumi(\theenumii)}
\renewcommand\p@enumiv{\p@enumiii\theenumiii}
%    \end{macrocode}
% \end{macro}
% \end{macro}
% \end{macro}
%
% \subsubsection{Itemize}
%
% \begin{macro}{\labelitemi}
% \begin{macro}{\labelitemii}
% \changes{v1.2h}{1994/05/06}{Inserted \cs{normalfont}}
% \changes{v1.2u}{1996/08/25}{replaced -{}- with \cs{textendash}}
% \begin{macro}{\labelitemiii}
% \begin{macro}{\labelitemiv}
% Itemization is controlled by  |\labelitemi|,
% |\labelitemii|, |\labelitemiii|, and |\labelitemiv|, which define
% the labels of the various itemization levels: the symbols used are
% bullet, bold en-dash, asterisk and centered dot.
%
% \changes{v1.2n}{1994/07/11}{renamed \cs{labelitemiiii} to
%    \cs{labelitemiii}}
% \changes{v1.2x}{1997/04/16}{Changed to \cs{textbullet},
%                 \cs{textasteriskcentered} and \cs{textperiodcentered}}
%    \begin{macrocode}
\newcommand\labelitemi{\textbullet}
\newcommand\labelitemii{\normalfont\bfseries \textendash}
\newcommand\labelitemiii{\textasteriskcentered}
\newcommand\labelitemiv{\textperiodcentered}
%    \end{macrocode}
% \end{macro}
% \end{macro}
% \end{macro}
% \end{macro}
%
% \subsubsection{Description}
%
% \begin{environment}{description}
%    The description environment is defined here -- while the itemize
%    and enumerate environments are defined in the \LaTeX{} kernel
%    (\file{ltlists.dtx}).
%
%    \begin{macrocode}
\newenvironment{description}
               {\list{}{\labelwidth\z@ \itemindent-\leftmargin
                        \let\makelabel\descriptionlabel}}
               {\endlist}
%    \end{macrocode}
% \end{environment}
%
% \begin{macro}{\descriptionlabel}
%    To change the formatting of the label, you must redefine
%    |\descriptionlabel|.
%
% \changes{v1.2q}{1995/03/14}{made command short}
%    \begin{macrocode}
\newcommand*{\descriptionlabel}[1]{\hspace\labelsep
                                \normalfont\bfseries #1}
%    \end{macrocode}
% \end{macro}
%
% \subsection{Defining new environments}
%
% \subsubsection{Verse}
%
% \begin{environment}{verse}
%   The verse environment is defined by making clever use of the
%   list environment's parameters.  The user types |\\| to end a line.
%   This is implemented by |\let|'ing |\\| equal |\@centercr|.
%
%    \begin{macrocode}
\newenvironment{verse}
               {\let\\=\@centercr
                \list{}{\setlength\itemsep{\z@}%
                        \setlength\itemindent{-15\p@}%
                        \setlength\listparindent{\itemindent}%
                        \setlength\rightmargin{\leftmargin}%
                        \addtolength\leftmargin{15\p@}}%
                \item[]}
               {\endlist}
%    \end{macrocode}
% \end{environment}
%
% \subsubsection{Quotation}
%
% \begin{environment}{quotation}
%   The quotation environment is also defined by making clever use of
%   the list environment's parameters. The lines in the environment
%   are set smaller than |\textwidth|. The first line of a paragraph
%   inside this environment is indented.
%
%    \begin{macrocode}
\newenvironment{quotation}
               {\list{}{\setlength\listparindent{1.5em}%
                        \setlength\itemindent{\listparindent}%
                        \setlength\rightmargin{\leftmargin}}%
                \item[]}
               {\endlist}
%    \end{macrocode}
% \end{environment}
%
% \subsubsection{Quote}
%
% \begin{environment}{quote}
%   The quote environment is like the quotation environment except
%   that paragraphs are not indented.
%
%    \begin{macrocode}
\newenvironment{quote}
               {\list{}{\setlength\rightmargin{\leftmargin}}%
                \item[]}
               {\endlist}
%    \end{macrocode}
% \end{environment}
%
% \subsubsection{Theorem}
%
%    This document class does not define it's own theorem environments,
%    the defaults, supplied by the \LaTeX{} kernel (\file{ltthm.dtx})
%    are available.
%
%
% \subsection{Setting parameters for existing environments}
%
% \subsubsection{Array and tabular}
%
% \begin{macro}{\arraycolsep}
%    The columns in an array environment are separated by
%    2|\arraycolsep|.
%    \begin{macrocode}
\setlength\arraycolsep{5\p@}
%    \end{macrocode}
% \end{macro}
%
% \begin{macro}{\tabcolsep}
%    The columns in an tabular environment are separated by
%    2|\tabcolsep|.
%    \begin{macrocode}
\setlength\tabcolsep{6\p@}
%    \end{macrocode}
% \end{macro}
%
% \begin{macro}{\arrayrulewidth}
%    The width of vertical rules in the array and tabular environments
%    is given by |\arrayrulewidth|.
%    \begin{macrocode}
\setlength\arrayrulewidth{.4\p@}
%    \end{macrocode}
% \end{macro}
%
% \begin{macro}{\doublerulesep}
%    The space between adjacent rules in the array and tabular
%    environments is given by |\doublerulesep|.
%    \begin{macrocode}
\setlength\doublerulesep{2\p@}
%    \end{macrocode}
% \end{macro}
%
% \subsubsection{Tabbing}
%
% \begin{macro}{\tabbingsep}
%    This controls the space that the |\'| command puts in. (See
%    \LaTeX{} manual for an explanation.)
%    \begin{macrocode}
\setlength\tabbingsep{\labelsep}
%    \end{macrocode}
% \end{macro}
%
% \subsubsection{Minipage}
%
% \begin{macro}{\@minipagerestore}
%    The macro |\@minipagerestore| is called upon entry to a minipage
%    environment to set up things that are to be handled differently
%    inside a minipage environment. In the current styles, it does
%    nothing.
% \end{macro}
%
% \begin{macro}{\@mpfootins}
%    Minipages have their own footnotes; |\skip||\@mpfootins| plays
%    same r\^ole for footnotes in a minipage as |\skip||\footins| does
%    for ordinary footnotes.
%
%    \begin{macrocode}
\skip\@mpfootins = \skip\footins
%    \end{macrocode}
% \end{macro}
%
% \subsubsection{Framed boxes}
%
% \begin{macro}{\fboxsep}
%    The space left by |\fbox| and |\framebox| between the box and the
%    text in it.
% \begin{macro}{\fboxrule}
%    The width of the rules in the box made by |\fbox| and |\framebox|.
%    \begin{macrocode}
\setlength\fboxsep{3\p@}
\setlength\fboxrule{.4\p@}
%    \end{macrocode}
% \end{macro}
% \end{macro}
%
% \subsubsection{Equation and eqnarray}
%
% \begin{macro}{\theequation}
%    The equation counter will be typeset using arabic numbers.
%    \begin{macrocode}
\renewcommand\theequation{\@arabic\c@equation}
%    \end{macrocode}
% \end{macro}
%
% \begin{macro}{\jot}
%    |\jot| is the extra space added between lines of an eqnarray
%    environment. The default value is used.
%    \begin{macrocode}
% \setlength\jot{3pt}
%    \end{macrocode}
% \end{macro}
%
% \begin{macro}{\@eqnnum}
%    The macro |\@eqnnum| defines how equation numbers are to appear in
%    equations. Again the default is used.
%
%    \begin{macrocode}
% \def\@eqnnum{(\theequation)}
%    \end{macrocode}
% \end{macro}
%
% \subsection{Font changing}
%
%    Here we supply the declarative font changing commands that were
%    common in \LaTeX\ version 2.09 and earlier. These commands work
%    in text mode \emph{and} in math mode. They are provided for
%    compatibility, but one should start using the |\text...| and
%    |\math...| commands instead. These commands are redefined using
%    |\@renewfontswitch|, a command with three arguments: the user
%    command to be defined; \LaTeX\ commands to execute in text mode
%    and \LaTeX\ commands to execute in math mode.
%
%  \begin{macro}{\rm}
%  \begin{macro}{\tt}
%  \begin{macro}{\sf}
%    The commands to change the family.
% \changes{v1.0a}{1993/12/14}{Added compatibility mode support.}
% \changes{v1.1}{1993/12/18}{Changed \cs{@newfontswitch} to
%    \cs{@renewfontswitch}, removed switch.}
% \changes{v1.2d}{1994/04/14}{\cs{@renewfontswitch} has become
%    \cs{DeclareOldFontCommand}}
%    \begin{macrocode}
\DeclareOldFontCommand{\rm}{\normalfont\rmfamily}{\mathrm}
\DeclareOldFontCommand{\sf}{\normalfont\sffamily}{\mathsf}
\DeclareOldFontCommand{\tt}{\normalfont\ttfamily}{\mathtt}
%    \end{macrocode}
%  \end{macro}
%  \end{macro}
%  \end{macro}
%
%  \begin{macro}{\bf}
%    The command to change to the bold series. One should use
%    |\mdseries| to explicitly switch back to medium series.
% \changes{v1.0a}{1993/12/14}{Added compatibility mode support.}
% \changes{v1.1}{1993/12/18}{Changed \cs{@newfontswitch} to
%    \cs{@renewfontswitch}, removed switch.}
%    \begin{macrocode}
\DeclareOldFontCommand{\bf}{\normalfont\bfseries}{\mathbf}
%    \end{macrocode}
%  \end{macro}
%
%  \begin{macro}{\sl}
%  \begin{macro}{\it}
%  \begin{macro}{\sc}
%
%    And the commands to change the shape of the font. The slanted and
%    small caps shapes are not available by default as math alphabets,
%    so those changes do nothing in math mode. One should use
%    |\upshape| to explicitly change back to the upright shape.
% \changes{v1.0a}{1993/12/14}{Added compatibility mode support.}
% \changes{v1.1}{1993/12/18}{Changed \cs{@newfontswitch} to
%    \cs{@renewfontswitch}, removed switch.}
% \changes{v1.1b}{1993/12/20}{Added forgotten re.}
%    \begin{macrocode}
\DeclareOldFontCommand{\it}{\normalfont\itshape}{\mathit}
\DeclareOldFontCommand{\sl}{\normalfont\slshape}{\relax}
\DeclareOldFontCommand{\sc}{\normalfont\scshape}{\relax}
%    \end{macrocode}
%  \end{macro}
%  \end{macro}
%  \end{macro}
%
% \begin{macro}{\cal}
% \changes{v1.0a}{1993/12/14}{Macro added}
% \begin{macro}{\mit}
% \changes{v1.0a}{1993/12/14}{Macro added}
%
%    The commands |\cal| and |\mit| should only be used in math mode,
%    outside math mode they have no effect. Currently the New Font
%    Selection Scheme defines these commands to generate warning
%    messages. Therefore we have to define them `by hand'.
% \changes{v1.2p}{1994/12/11}{Now define \cs{cal} and \cs{mit} using
%    \cs{DeclareRobustCommand*}}
%    \begin{macrocode}
\DeclareRobustCommand*{\cal}{\@fontswitch{\relax}{\mathcal}}
\DeclareRobustCommand*{\mit}{\@fontswitch{\relax}{\mathnormal}}
%    \end{macrocode}
%  \end{macro}
%  \end{macro}
%
% \subsection{Footnotes}
%
% \begin{macro}{\footnoterule}
%    Usually, footnotes are separated from the main body of the text
%    by a small rule. This rule is drawn by the macro |\footnoterule|.
%    We have to make sure that the rule takes no vertical space (see
%    \file{plain.tex}) so we compensate for the natural height of the
%    rule of 0.4pt by adding the right amount of vertical skip.
%
%    To prevent the rule from colliding with the footnote we first add
%    a little negative vertical skip, then we put the rule and make
%    sure we end up at the same point where we begun this operation.
%    \begin{macrocode}
\renewcommand\footnoterule{%
  \kern-\p@
  \hrule \@width .4\columnwidth
  \kern .6\p@}
%    \end{macrocode}
% \end{macro}
%
% \begin{macro}{\c@footnote}
%    A counter for footnotes.
%    \begin{macrocode}
% \newcounter{footnote}
%    \end{macrocode}
% \end{macro}
%
% \begin{macro}{\@makefntext}
% \changes{v1.2c}{1994/03/08}{Always call \cs{@makefnmark}.}
%    The footnote mechanism of \LaTeX{} calls the macro |\@makefntext|
%    to produce the actual footnote. The macro gets the text of the
%    footnote as its argument and should use |\@makefnmark| to produce
%    the mark of the footnote. The macro |\@makefntext| is called when
%    effectively inside a |\parbox| of width |\columnwidth| (i.e.,
%    with |\hsize| = |\columnwidth|).
%
%   An example of what can be achieved is given by the following piece
%   of \TeX\ code.
% \begin{verbatim}
%          \long\def\@makefntext#1{%
%             \@setpar{\@@par
%                      \@tempdima = \hsize
%                      \advance\@tempdima-10pt
%                      \parshape \@ne 10pt \@tempdima}%
%             \par
%             \parindent 1em\noindent
%             \hb@xt@\z@{\hss\@makefnmark}#1}
% \end{verbatim}
%    The effect of this definition is that all lines of the footnote
%    are indented by 10pt, while the first line of a new paragraph is
%    indented by 1em. To change these dimensions, just substitute the
%    desired value for `10pt' (in both places) or `1em'.  The mark is
%    flush right against the footnote.
%
%    In this document class we use a simpler macro, in which the
%    footnote text is set like an ordinary text paragraph, with no
%    indentation except on the first line of a paragraph, and the
%    first line of the footnote. Thus, all the macro must do is set
%    |\parindent| to the appropriate value for succeeding paragraphs
%    and put the proper indentation before the mark.
%
%    \begin{macrocode}
\long\def\@makefntext#1{%
    \noindent
    \hangindent 5\p@
    \hb@xt@5\p@{\hss\@makefnmark}#1}
%    \end{macrocode}
% \end{macro}
%
% \begin{macro}{\@makefnmark}
%    The footnote markers that are printed in the text to point to the
%    footnotes should be produced by the macro |\@makefnmark|. We use
%    the default definition for it.
%    \begin{macrocode}
%\def\@makefnmark{\hbox{$^{\@thefnmark}\m@th$}}
%    \end{macrocode}
% \end{macro}
%
% \subsection{Words}
%
% \begin{macro}{\ccname}
% \begin{macro}{\enclname}
% \begin{macro}{\pagename}
% \begin{macro}{\headtoname}
%     This document class is for documents prepared in the English
%     language. To prepare a version for another language, various
%     English words must be replaced.  All the English words that
%     require replacement are  defined below in command names.
%    \begin{macrocode}
\newcommand*{\ccname}{cc}
\newcommand*{\enclname}{encl}
\newcommand*{\pagename}{Page}
\newcommand*{\headtoname}{To}
%    \end{macrocode}
% \end{macro}
% \end{macro}
% \end{macro}
% \end{macro}
%
% \subsection{Date}
%
% \begin{macro}{\today}
%    This macro uses the \TeX\ primitives |\month|, |\day| and |\year|
%    to provide the date of the \LaTeX-run.
%    \begin{macrocode}
\newcommand*{\today}{\ifcase\month\or
  January\or February\or March\or April\or May\or June\or
  July\or August\or September\or October\or November\or December\fi
  \space\number\day, \number\year}
%    \end{macrocode}
% \end{macro}
%
% \subsection{Two column mode}
%
% \begin{macro}{\columnsep}
%    This gives the distance between two columns in two column mode.
%    \begin{macrocode}
\setlength\columnsep{10\p@}
%    \end{macrocode}
% \end{macro}
%
% \begin{macro}{\columnseprule}
%    This gives the width of the rule between two columns in two
%    column mode. We have no visible rule.
%    \begin{macrocode}
\setlength\columnseprule{0\p@}
%    \end{macrocode}
% \end{macro}
%
% \subsection{The page style}
%    We have \pstyle{plain} pages in this document class by
%    default. We use arabic page numbers.
%    \begin{macrocode}
\pagestyle{plain}
\pagenumbering{arabic}
%    \end{macrocode}
%
% \subsection{Single or double sided printing}
%
%    We don't try to make each page as long as all the others.
%    \begin{macrocode}
\raggedbottom
%    \end{macrocode}
%
% \begin{macro}{\@texttop}
%    The document class letter sets |\@texttop| to |\vskip| 0pt
%    \texttt{plus} .00006fil on the first page of a letter, which
%    centers a short letter on the page. This fil value may have to be
%    changed for other letterheads.  This setting has to be done after
%    |\raggedbottom| is executed, since the latter sets |\@texttop| to
%    |\relax|.
%    \begin{macrocode}
\def\@texttop{\ifnum\c@page=1\vskip \z@ plus.00006fil\relax\fi}
%    \end{macrocode}
% \end{macro}
%
%    We always start in one column mode.
%    \begin{macrocode}
\onecolumn
%</letter>
%    \end{macrocode}
%
%
%
% \Finale
%
\endinput

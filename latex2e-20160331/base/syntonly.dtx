% \iffalse meta-comment
%
% Copyright 1993-2016
% The LaTeX3 Project and any individual authors listed elsewhere
% in this file.
%
% This file is part of the LaTeX base system.
% -------------------------------------------
%
% It may be distributed and/or modified under the
% conditions of the LaTeX Project Public License, either version 1.3c
% of this license or (at your option) any later version.
% The latest version of this license is in
%    http://www.latex-project.org/lppl.txt
% and version 1.3c or later is part of all distributions of LaTeX
% version 2005/12/01 or later.
%
% This file has the LPPL maintenance status "maintained".
%
% The list of all files belonging to the LaTeX base distribution is
% given in the file `manifest.txt'. See also `legal.txt' for additional
% information.
%
% The list of derived (unpacked) files belonging to the distribution
% and covered by LPPL is defined by the unpacking scripts (with
% extension .ins) which are part of the distribution.
%
% \fi
%%
%
%
% \changes{v2.1a}{1994/03/10}{Update for \LaTeXe}
%
% \DoNotIndex{\@cclvi,\@cclv,\@gobble,\@gobbletwo,\@ifnextchar}
% \DoNotIndex{\@M,\@tempboxa,\advance,\alloc,\begingroup,\bgroup}
% \DoNotIndex{\box,\count@,\def,\do,\docdate,\egroup,\endgroup}
% \DoNotIndex{\fi,\filedate,\fileversion,\font,\fontdimen,\global}
% \DoNotIndex{\ifinner,\ifnum,\immediate,\let,\long,\loop,\m@ne}
% \DoNotIndex{\noexpand,\relax,\repeat,\scriptfont}
% \DoNotIndex{\scriptscriptfont,\setbox,\sixt@@n,\space,\textfont}
% \DoNotIndex{\toks,\toksdef,\write,\xdef,\z@}%
%
%
%\title{The file \texttt{syntonly.dtx} for use with
%      \LaTeXe.\thanks{This file has version
%           number \fileversion, dated \filedate.}\\[2pt]
%      It contains the code for \texttt{syntonly.sty}}
% \author{Frank Mittelbach \and Rainer Sch\"opf}
%
%
% \MaintainedByLaTeXTeam{latex}
% \maketitle
%
% This package implements the |\syntaxonly| declaration for \LaTeXe.
% This command can be used in the preamble for running a document
% through \LaTeX{} without actually getting any output.
%
% \StopEventually{}
%
%
% \section{Identification}
%
%  We identify the package and its current version.
% \changes{v2.1b}{1995/04/22}{Removed surplus \cs{typeout} lines}
% \changes{v2.1c}{1996/06/14}{Rearange \cs{GetFileInfo} useage.}
% \changes{v2.1d}{1998/08/17}{(RmS) Minor documentation fixes.}
%
%    \begin{macrocode}
%<package>\ProvidesPackage{syntonly}
%<*dtx>
          \ProvidesFile{syntonly.dtx}
%</dtx>
%<*package|dtx>
              [1999/09/17 v2.1e Standard LaTeX2e package]
%</package|dtx>
%    \end{macrocode}
%
%\iffalse
%
% \section{The documentation driver file}
%
% The next bit of code contains the documentation driver file for
% \TeX{}, i.e., the file that will produce the documentation you are
% currently reading. It will be extracted from this file by the \dst{}
% program.
%    \begin{macrocode}
%<*driver>
\documentclass{ltxdoc}
\GetFileInfo{syntonly.dtx}
\providecommand\dst{\expandafter{\normalfont\scshape docstrip}}
\begin{document}
   \DocInput{syntonly.dtx}
\end{document}
%</driver>
%    \end{macrocode}
%\fi
%
%
% \section{Implementation}
%
%    \begin{macrocode}
%<*package>
%    \end{macrocode}
%
% \begin{macro}{\dummyft@}
%    First of all we need to define the `dummy' font.
%    \begin{macrocode}
\font\dummyft@=dummy \relax
%    \end{macrocode}
% \end{macro}
% \changes{v2.1b}{1995/05/25}{Removed unneeded fontdimen settings for
%   dummyft@}
%
% \begin{macro}{\ifsyntax@}
% Now we can define the `syntax only' feature.  We define a switch
% |\if@syntax| so that any macro can always find out if it
% is really supposed to typeset text.  Its default is to run in
% normal mode.
%    \begin{macrocode}
\newif\ifsyntax@
\syntax@false
%    \end{macrocode}
% \end{macro}
%
% \begin{macro}{\syntaxonly}
%    The |\syntaxonly| macro sets up everything for syntax
%    checking.
%    \begin{macrocode}
\def\syntaxonly{%
%    \end{macrocode}
%    First of all it sets the |syntax@| switch to \texttt{true}.
%    \begin{macrocode}
   \syntax@true
%    \end{macrocode}
%    Then it globally sets all fonts to the dummy font.  These are:
%    the current font outside math mode,
%    \begin{macrocode}
   \global\dummyft@
%    \end{macrocode}
%    and the $3\times16$ math fonts for the $16$ math \emph{groups}.
%    We use a loop to set these.
%    \begin{macrocode}
   \count@\sixt@@n
   \loop
    \ifnum\count@ >\z@
     \advance\count@\m@ne
     \global\textfont\count@\dummyft@
     \global\scriptfont\count@\dummyft@
     \global\scriptscriptfont\count@\dummyft@
    \repeat
%    \end{macrocode}
%    Since all font changes occur either via |\selectfont| (in
%    text or |\mathversion| (for math mode) it is sufficient
%    to change these to no-ops.  In addition we must prevent
%    the loading of math fonts, this is done by making
%    |\getanddefine@fonts| a no-op.
%    \begin{macrocode}
   \global\let\selectfont\relax
   \global\let\mathversion\@gobble
   \global\let\getanddefine@fonts\@gobbletwo
%    \end{macrocode}
%    We prevent \TeX{} from complaining about the dummy font
%    having no characters.
%    \begin{macrocode}
   \tracinglostchars\z@
%    \end{macrocode}
%    Then we disable the output routine, and set |\frenchspacing|
%    (which is slightly faster than |\nonfrenchspacing|).
%    Finally we set |\hbadness| to $10000$ to avoid overfull box
%    messages.
%    \begin{macrocode}
   \nopages@
   \frenchspacing
   \hbadness\@M}
%    \end{macrocode}
% \end{macro}
%
% \changes{v2.1b}{1995/05/25}{Removed non-outer newtoks@: newtoks is no
%   longer outer}
%
% \begin{macro}{\nopages@}
% \changes{v1.0b}{1990/01/14}{Rewrite of the \cs{nopages@} macro to
%       correctly handle footnote insertions, floats and marginpars.}
%    The |\nopages@| macro disables the \LaTeX{} output routine.
%    To this end we define a very simple output routine that empties
%    the output \emph{and} footnote boxes (remember that the latter
%    are insertions.
%    \begin{macrocode}
\def\nopages@{%
  \output {\setbox\z@\box\@cclv
           \setbox\z@\box\footins
           \deadcycles\z@}%
%    \end{macrocode}
%    Then we protect it against definition by a style file.
% \changes{v2.1b}{1995/05/25}{newtoks is no longer outer}
%    \begin{macrocode}
  \newtoks\output
%    \end{macrocode}
%    But this is not enough: normally the \LaTeX{} output routine
%    is responsible for dealing with floating objects.
%    We therefore also redefine the internal macros for handling
%    floats and marginpars.
%    \begin{macrocode}
  \def\@xfloat##1[##2]{%
%    \end{macrocode}
%    There are a few things that have to be retained:
%    the definition of |\@captype| since it is used by the
%    |\caption| command,
% \changes{v1.0c}{1990/01/16}{Definition of \cs{@captype} added.}
%    \begin{macrocode}
    \def\@captype{##1}%
%    \end{macrocode}
%    the error message issued when not in outer paragraph mode,
%    \begin{macrocode}
    \ifinner\@parmoderr\fi
%    \end{macrocode}
%    and the |\@parboxrestore| command for the body of the
%    float.  This is necessary since it restores the original
%    definitions of important commands like |\par| or |\\|.
%    \begin{macrocode}
    \setbox\@tempboxa\vbox\bgroup\@parboxrestore}%
%    \end{macrocode}
%    |\end@float| must now only close the brace:
%    \begin{macrocode}
  \let\end@float\egroup
%    \end{macrocode}
%    The above would be enough also for two-column floats with the
%   kernel algorithm. However with the refined algorithm inside
%   \texttt{fixlxt2e} this doesn't any longer work, so there we also
%   need to explicitly overwrite the end macro for two-column floats
%   (the begin is still okay as it resolves to |\@xfloat| eventually).
%    \begin{macrocode}
  \let\end@dblfloat\egroup
%    \end{macrocode}
%    The redefinition of the |\marginpar| command is a bit more
%    complicated since we have to check for the optional argument.
%    First we redefine the command itself:
%    \begin{macrocode}
  \def\marginpar{\ifinner\@parmoderr\fi
%    \end{macrocode}
%    We open a group so that everything gathered in a temporary box
%    can easily be thrown away by closing it again (see below).
%    \begin{macrocode}
    \begingroup \@ifnextchar [\@xmpar\@ympar}
%    \end{macrocode}
%    |\@xmpar| and |\@ympar| are now defined similar to
%    |\@xfloat| above.
%    If an optional argument is present |\@xmpar| typesets
%    it in a temporary box that is thrown away later.
%    Then it calls up |\@ympar| to process |\marginpar|'s
%    argument.
% \changes{v1.0d}{1990/01/21}{Added forgotten \cs{vbox}.}
%    \begin{macrocode}
  \long\def\@xmpar[##1]{%
    \setbox\@tempboxa\vbox{\@parboxrestore ##1}\@ympar}%
%    \end{macrocode}
%    |\@ympar| gathers its argument in the same temporary box
%    and throws away its contents by closing the group opened up in
%    |\marginpar| above.
% \changes{v1.0d}{1990/01/21}{Added forgotten \cs{vbox}.}
%    \begin{macrocode}
  \long\def\@ympar##1{%
    \setbox\@tempboxa\vbox{\@parboxrestore ##1}\endgroup}%
%    \end{macrocode}
%    And that's all we had to do.
%    \begin{macrocode}
}
%    \end{macrocode}
% \end{macro}
%
% \begin{macro}{\@preamblecmds}
%    We disable the use of the |\syntaxonly| command after
%    |\begin{document}|
%    \begin{macrocode}
\@onlypreamble\syntaxonly
%</package>
%    \end{macrocode}
% \end{macro}
%
% \Finale

\endinput

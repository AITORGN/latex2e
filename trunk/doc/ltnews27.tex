% \iffalse meta-comment
%
% Copyright 2017
% The LaTeX3 Project and any individual authors listed elsewhere
% in this file.
%
% This file is part of the LaTeX base system.
% -------------------------------------------
%
% It may be distributed and/or modified under the
% conditions of the LaTeX Project Public License, either version 1.3c
% of this license or (at your option) any later version.
% The latest version of this license is in
%    http://www.latex-project.org/lppl.txt
% and version 1.3c or later is part of all distributions of LaTeX
% version 2005/12/01 or later.
%
% This file has the LPPL maintenance status "maintained".
%
% The list of all files belonging to the LaTeX base distribution is
% given in the file `manifest.txt'. See also `legal.txt' for additional
% information.
%
% The list of derived (unpacked) files belonging to the distribution
% and covered by LPPL is defined by the unpacking scripts (with
% extension .ins) which are part of the distribution.
%
% \fi
% Filename: ltnews27.tex
%
% This is issue 27 of LaTeX News.

\documentclass{ltnews}
\usepackage[T1]{fontenc}

\usepackage{lmodern,url,hologo}

\publicationmonth{May}
\publicationyear{2017}

\publicationissue{27}

\begin{document}

\maketitle
\tableofcontents

\setlength\rightskip{0pt plus 3em}

\section{ISO 8601 Date Format}
Since before the first releases of \LaTeX2e, \LaTeX\ has used a date
format in the form \textsc{yyyy/mm/dd}. This has many advantages over more
conventional formats, as it is easy to sort and avoids the unfortunate
ambiguity between different comunities as to whether 2017/01/02 is the
1st of February or 2nd of January.

However there is another date format, formalised by the
International standard, ISO~8601. The basic format defined by this
standard is functionally equivalent to the \LaTeX\ format, but using
\texttt{-} rather than \texttt{-}. This date format is now supported
in many Operating Systems and applications
(for example the \verb|date --iso-8601| command in linux and similar systems).

From this release, \LaTeX\ will accept ISO format date strings in the
date argument of \verb|\ProvidesPackage|, \verb|\usepackage| etc.
Currently we recommend that you do not use this format in any packages
that need to work with older \LaTeX\ releases. 
The \textsf{latexrelease} package may be used with older releases to
 add this functionality. This change is handled in a special way  by
 \textsf{latexrelease}: The package always adds support for ISO dates
whatever format date is requested, this is required so that the
necessary date comparisons may be made.

The new functionality can be seen in the startup banner which
advertsises \texttt{LaTeX2e 2017-05-01}.

\section{Further TU encoding improvements}
The 2017/01/01 release saw the introduction of the new TU encoding for
specifying Unicode fonts with \hologo{LuaTeX} and
\hologo{XeTeX}. There were a number of small corrections and additions
in the patch releases updating 2017/01/01, and a further addition in
this release, notably extended support for the dot-under accent,
\verb|\d|.


\section{Disabling Hyphenation}
The existing \LaTeX\ code for \verb|\verb| and \verb|verbatim| had some
issues when used with fonts that were not loaded with hyphenation
disabled via setting \verb|\hyphenchar| to $-1$. In this release these
verbatim environments use a \verb|\language| setting,
\verb|\l@nohyphenation|, that has no hyphenation patterns associated. 

The format ensures that  a language has been allocated with this name.
For most users this will in fact be no change as the standard
\textsf{babel} language has for a long time allocated a language with
this name.

In order that page breaks in \texttt{verbatim} do not influence the
language used in the page head and foot, the format now normalises the
language used in the output routine to a default language as described
below.
 

\section{Default Document Language}

A new integer parameter \verb|\document@default@language| is
introduced, this is initialised to $-1$ but is set at
\verb|\begin{document}| to the language in force at that time if it
  has not been set by preamble code. This is versy similar to the
  handling of the default color.  and is used in a similar way to
  normalise the settings for page head and foot as described above.
users should not normally need to set this explicitly but it expected that
language packages such as \textsf{babel} may set this if the default
behaviour is not suitable.
\end{document}

% \iffalse meta-comment
%
% Copyright 2015
% The LaTeX3 Project and any individual authors listed elsewhere
% in this file.
%
% This file is part of the LaTeX base system.
% -------------------------------------------
%
% It may be distributed and/or modified under the
% conditions of the LaTeX Project Public License, either version 1.3c
% of this license or (at your option) any later version.
% The latest version of this license is in
%    http://www.latex-project.org/lppl.txt
% and version 1.3c or later is part of all distributions of LaTeX
% version 2005/12/01 or later.
%
% This file has the LPPL maintenance status "maintained".
%
% The list of all files belonging to the LaTeX base distribution is
% given in the file `manifest.txt'. See also `legal.txt' for additional
% information.
%
% The list of derived (unpacked) files belonging to the distribution
% and covered by LPPL is defined by the unpacking scripts (with
% extension .ins) which are part of the distribution.
%
% \fi
% Filename: ltnews24.tex
%
% This is issue 24 of LaTeX News.

\documentclass{ltnews}
\usepackage[T1]{fontenc}

\usepackage{lmodern,url,hologo}

\makeatletter % -- provide command introduced in new release
              %    so this typesets with an old format

\publicationmonth{January}
\publicationyear{2016}

\publicationissue{24}

\begin{document}

\maketitle

\section{Unicode data}

As noted in \LaTeX{} News~22, the 2015/01/01 release of \LaTeX{} introduced
built-in support for extended \TeX{} systems. In particular, the kernel now
loads appropriate data from the Unicode Consortium to set \verb|\lccode|,
\verb|\uccode|, \verb|\catcode| and \verb|\sfcode| values in an automated
fashion for the entire Unicode range.

The initial approach taken by the team was to incorporate the existing model
used by (plain) \hologo{XeTeX} and to pre-process the ``raw'' Unicode data into
a ready-to-use form as \verb|unicode-letters.def|. However, the relationship
between Unicode Consortium and \TeX{} data structures is non-trivial and still
being explored. As such, it is preferable to directly parse the original
(\verb|.txt|) files at point of use. The team have therefore ``spun-out'' both
the data and the loading to a new generic package, \package{unicode-data}. This
package makes the original Unicode Consortium data files available in the
\verb|texmf| tree (in \verb|tex/generic/unicode-data|) and provides generic
loaders suitable for reading this data into the plain, \LaTeXe{} and other
formats.

At present, the following data files are included in this new package:
\begin{itemize}
  \item \verb|CaseFolding.txt|
  \item \verb|EastAsianWidth.txt|
  \item \verb|LineBreak.txt|
  \item \verb|MathClass.txt|
  \item \verb|SpecialCasing.txt|
  \item \verb|UnicodeData.txt|
\end{itemize}
These files are used either by \LaTeXe{} or by \package{expl3}
(\emph{i.e.}~they represent the set currently required by the team). The
Unicode Consortium provide various other data files and we are happy to add
these to the generic package, as this is intended to provide a single place
to collect this material in the \verb|texmf| tree. Such requests can be
mailed to the team as usual or logged at the package home page:
\url{https://github.com/latex3/unicode-data}.

\end{document}

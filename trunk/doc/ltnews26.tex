% \iffalse meta-comment
%
% Copyright 2016
% The LaTeX3 Project and any individual authors listed elsewhere
% in this file.
%
% This file is part of the LaTeX base system.
% -------------------------------------------
%
% It may be distributed and/or modified under the
% conditions of the LaTeX Project Public License, either version 1.3c
% of this license or (at your option) any later version.
% The latest version of this license is in
%    http://www.latex-project.org/lppl.txt
% and version 1.3c or later is part of all distributions of LaTeX
% version 2005/12/01 or later.
%
% This file has the LPPL maintenance status "maintained".
%
% The list of all files belonging to the LaTeX base distribution is
% given in the file `manifest.txt'. See also `legal.txt' for additional
% information.
%
% The list of derived (unpacked) files belonging to the distribution
% and covered by LPPL is defined by the unpacking scripts (with
% extension .ins) which are part of the distribution.
%
% \fi
% Filename: ltnews26.tex
%
% This is issue 26 of LaTeX News.

\documentclass{ltnews}
\usepackage[T1]{fontenc}

\usepackage{lmodern,url,hologo}


\publicationmonth{December}
\publicationyear{2016}

\publicationissue{26}

\begin{document}

\maketitle

\section{\eTeX{}}

In \LaTeX{} News~16 (December 2003) the team announced
\begin{quotation}
We expect that within the next two years, releases of \LaTeX{} will
change modestly in order to run best under an extended \TeX{} engine
that contains the \eTeX{} primitives, e.g., \eTeX{} or pdf\TeX{}.
\end{quotation}
and also said
\begin{quotation}
Although the current release does not \emph{require} \eTeX{} features, we
certainly recommend using an extended \TeX{}, especially if you need to debug
macros.
\end{quotation}

For many years the team have worked on the basis that users will have \eTeX{}
available but had not revisited the above statements formally. As of the
January 2017 release of \LaTeXe{}, \eTeX{} is \emph{required} to build the
format, and attempting to build a format without the extensions will fail.

Practically, modern \TeX{} distributions provide the extensions in all engines
other than the ``pure'' Knuth \texttt{tex}, and indeed parts of the
format-building process already require \eTeX{}, most notably some of the UTF-8
hyphenation patterns. As such, there should be no noticeable affect on users of
this change.

The team expect to make wider use of \eTeX{} within the kernel in future;
details will be announced where they impact on end users in a visible way.

\end{document}
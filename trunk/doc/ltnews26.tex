% \iffalse meta-comment
%
% Copyright 2016
% The LaTeX3 Project and any individual authors listed elsewhere
% in this file.
%
% This file is part of the LaTeX base system.
% -------------------------------------------
%
% It may be distributed and/or modified under the
% conditions of the LaTeX Project Public License, either version 1.3c
% of this license or (at your option) any later version.
% The latest version of this license is in
%    http://www.latex-project.org/lppl.txt
% and version 1.3c or later is part of all distributions of LaTeX
% version 2005/12/01 or later.
%
% This file has the LPPL maintenance status "maintained".
%
% The list of all files belonging to the LaTeX base distribution is
% given in the file `manifest.txt'. See also `legal.txt' for additional
% information.
%
% The list of derived (unpacked) files belonging to the distribution
% and covered by LPPL is defined by the unpacking scripts (with
% extension .ins) which are part of the distribution.
%
% \fi
% Filename: ltnews26.tex
%
% This is issue 26 of LaTeX News.

\documentclass{ltnews}
\usepackage[T1]{fontenc}

\usepackage{lmodern,url,hologo}


\publicationmonth{December}
\publicationyear{2016}

\publicationissue{26}

\begin{document}

\maketitle

\section{\eTeX{}}

In \LaTeX{} News~16 (December 2003) the team announced
\begin{quotation}
We expect that within the next two years, releases of \LaTeX{} will
change modestly in order to run best under an extended \TeX{} engine
that contains the \eTeX{} primitives, e.g., \eTeX{} or pdf\TeX{}.
\end{quotation}
and also said
\begin{quotation}
Although the current release does not \emph{require} \eTeX{} features, we
certainly recommend using an extended \TeX{}, especially if you need to debug
macros.
\end{quotation}

For many years the team have worked on the basis that users will have \eTeX{}
available but had not revisited the above statements formally. As of the
January 2017 release of \LaTeXe{}, \eTeX{} is \emph{required} to build the
format, and attempting to build a format without the extensions will fail.

Practically, modern \TeX{} distributions provide the extensions in all engines
other than the ``pure'' Knuth \texttt{tex}, and indeed parts of the
format-building process already require \eTeX{}, most notably some of the UTF-8
hyphenation patterns. As such, there should be no noticeable affect on users of
this change.

The team expect to make wider use of \eTeX{} within the kernel in future;
details will be announced where they impact on end users in a visible way.

\section{Default Encodings in \hologo{XeLaTeX} and \hologo{LuaLaTeX}}
The default encoding in \LaTeX\ has always been the original
127-character encoding ``OT1''.  For Unicode based \TeX\ engines, this
is not really suitable, and is especially problematic with
\hologo{XeLaTeX} as in the major distributions this is built with
Unicode base hyphenation patterns in the format.  In practice this has
not been a major problem as documents use the contributed
\textsf{fontspec} package in order to switch to a
Unicode encoded font..

In this release we are adding ``TU'' as a new supported
encoding in addition to the previously supported encodings such as OT1
and T1. This denotes a Unicode based font encoding. It is essentially
the same as the TU encoding that has been on trial with the
experimental \texttt{tuenc} option to \textsf{fontspec} for the past
year.

The \hologo{XeLaTeX} and \hologo{LuaLaTeX} formats will now default
to \texttt{TU} encoding and \texttt{lmr} (Latin Modern) family. In the
case of \hologo{LuaLaTeX} the contributed \textsf{luaotfload} Lua
module will be loaded at the start of each run to enable the loading
of OpenType fonts.

The \textsf{fontspec} package is being adjusted in a companion release
to recognise the new encoding defaullt arrangements.

Note that in practice no font supports the full Unicode range, and so
TU encoded fonts, unlike fonts specified for T1, may be expected to
be incomplete in various ways. In the current release the format has 
made some basic assumptions for (for example) default handling of
accent commands. The \textsf{fontspec} package has more control over
finer details of the font loading and control over OpenType
features. More facilities may be added as needed. Initially they will
be added in \textsf{fontspec} with some aspects possibly being moved
to the base format at a later date.

If for any reason you need to process a document with the previous
default OT1 encoding, you may switch encoding in the usual ways, for
example
\begin{verbatim}
\usepackage[T1]{fontenc}
\end{verbatim}
or you may roll back all the changes for this release by starting the
document with
\begin{verbatim}
\RequirePackage[2016/03/31]{latexrelease}
\end{verbatim}

\section{The \textsf{latexbug} package}

As expliained in more detail
\href{https://www.latex-project.org/bugs/}{at the \LaTeX\ Project
  website},
A new package, \textsf{latexbug}, has been produced to help produce
test files to accompany bug reports on the core \LaTeX\ distribution.
This is being published separately to CTAN at the same time as this
release. By using the \textsf{latexbug} package you can easily check
that the packages involved in the test are all part of the core
release. The \LaTeX\ project can not handle bug reports on contributed
packages, which should be directed to the package maintainer as given
in the package documentation.
\end{document}

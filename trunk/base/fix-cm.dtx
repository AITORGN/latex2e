% \iffalse meta-comment
%
% Copyright 1993-2016
% The LaTeX3 Project and any individual authors listed elsewhere
% in this file.
%
% This file is part of the LaTeX base system.
% -------------------------------------------
%
% It may be distributed and/or modified under the
% conditions of the LaTeX Project Public License, either version 1.3c
% of this license or (at your option) any later version.
% The latest version of this license is in
%    http://www.latex-project.org/lppl.txt
% and version 1.3c or later is part of all distributions of LaTeX
% version 2005/12/01 or later.
%
% This file has the LPPL maintenance status "maintained".
%
% The list of all files belonging to the LaTeX base distribution is
% given in the file `manifest.txt'. See also `legal.txt' for additional
% information.
%
% The list of derived (unpacked) files belonging to the distribution
% and covered by LPPL is defined by the unpacking scripts (with
% extension .ins) which are part of the distribution.
%
% \fi
%
% \iffalse
%
%<*dtx>
          \ProvidesFile{fix-cm.dtx}
%</dtx>
%<fix-cm>\NeedsTeXFormat{LaTeX2e}
%<fix-cm>\ProvidesPackage{fix-cm}
%<driver>\ProvidesFile{fix-cm.drv}
% \fi
%         \ProvidesFile{fix-cm.dtx}
          [2015/01/14 v1.1t fixes to LaTeX]
%
% \iffalse
%<*driver>
 \documentclass{ltxdoc}
 \newcommand\Lopt[1]{\textsf{#1}}
 \let\Lpack\Lopt
 \providecommand{\file}[1]{\texttt{#1}}
 \providecommand{\MF}{\textsf{Metafont}}
 \providecommand{\danger}{\marginpar[\hfill\protect\Huge!!]{\protect\Huge!!\hfill}}
 \begin{document}
 \DocInput{fix-cm.dtx}
 \end{document}
%</driver>
% \fi
%
% \CheckSum{161}
%
%
%
%
% \let\package\textsf
%
%
% \GetFileInfo{fix-cm.dtx}
%
% \title{The \Lpack{fix-cm} package\thanks{This file
%         has version number \fileversion, last
%         revised \filedate.}}
% \author{Frank Mittelbach, David Carlisle, Chris Rowley, Walter
%         Schmidt\thanks{Walter wrote \Lpack{fix-cm}}}
% \date{\filedate}
% \MaintainedByLaTeXTeam{latex}
%  \maketitle
%
% \begin{abstract}
%\Lpack{fix-cm} improves the definitions of the Computer Modern
% font families.
% \end{abstract}
%
% \tableofcontents
%
% \newpage
%
% \section{Introduction}
%
%
% \begin{sloppypar}
% To use the \Lpack{fix-cm} package,
% load \danger it \emph{before} \cmd{\documentclass},
% and use the command \cmd{\RequirePackage} to do so, rather than the
% normal \cmd{\usepackage}:
% \end{sloppypar}
% \begin{verse}
%   |\RequirePackage{fix-cm}|\\
%   |\documentclass| \dots
% \end{verse}
%
%
% \section{Using EC fonts (T1 encoding) makes my documents look
%             different}
%
% No I'm not trying to collect any cites from the news group
% discussion on this topic. In a nutshell, if one adds
%\begin{verbatim}
%\usepackage[T1]{fontenc}
%\end{verbatim}
% to a document that uses the Computer Modern typefaces,
% then not only the T1 encoding is used but the fonts
% used in the document look noticeably different. This is due to the fact
% that the EC fonts have more font series designs, e.g.\ a 14.4\,pt bold
% etc and those get used in the standard \texttt{.fd} files, while
% with Computer Modern (in OT1 encoding) such sizes were scaled
% versions of smaller sizes---with a noticeable different look and
% feel.
%
% So we provide a package \Lpack{fix-cm} to ensure that comparable
% definitions are used. In addition to that, the package
% \Lpack{fix-cm} also enables continuous scaling of the CM fonts.
% This package was written by Walter Schmidt.
%
%%
%^^A The documentation in this section was prepared by Walter Schmidt.
%
%
%
% \subsection{What \Lpack{fix-cm} does}
%

% Loading the package \Lpack{fix-cm} changes the font definitions of the
% Computer Modern fonts, in order to achieve the following effects:
% \begin{itemize}
%   \item
%   The appearance of the T1 and TS1 encoded CM fonts (aka `EC') is
%   made as similar as possible to the traditional (OT1 encoded) ones.
%   Particularly, a number of broken or ugly design sizes are no
%   longer used, the look of the bold sans serif typeface at large
%   sizes is considerably improved, and mismatches between the text
%   fonts and the corresponding math fonts are avoided.  As a side
%   effect, PostScript and PDF documents may become smaller, because
%   fewer fonts need to be embedded.
%   \item
%   The Computer Modern fonts are made available with arbitrary sizes.
%   \item
%   Only those design sizes of the fonts will be used, that are
%   normally available in Type1 format, too.  You need not load the
%   extra package \Lpack{cmmib57} for this purpose.
% \end{itemize}
% The package acts on the following font families:
% \begin{itemize}
%   \item
%   The text font families \file{cmr}, \file{cmss}, \file{cmtt} and
%   \file{cmvtt} with OT1, T1 and TS1 encoding.
%   \item
%   The default math fonts used by \LaTeX, i.e., the font families
%   \file{cmm} with encoding OML and \file{cms} with encoding OMS.
%   \item
%   The symbols used by the package \Lpack{latexsym}, i.e., the font
%   family \file{lasy}.
% \end{itemize}
% Note that the package does \emph{not} act on:
% \begin{itemize}
% \item Font families such as CM~Fibonacci, CM~Dunhill etc.,
%   which are provided for experimental purposes or for fun only.
% \item
%   CM text fonts with character sets other than Latin, e.g.,
%   Cyrillic.  Loading of the required font and encoding definitions
%   while the fonts are not actually used, would not be a good idea.
%   This should be addressed by particular packages or by changing the
%   standard FDs of these fonts.
% \item
%   Extra math fonts such as the AMS symbol fonts.  While
%   they match the style of Computer Modern, they are frequently used
%   in conjunction with other font families, too.  Thus,
%   \Lpack{fix-cm} is obviously not the right place to make sure that
%   they can be scaled continuously.  Ask the maintainers of these
%   fonts to provide this feature, which is badly needed!
% \item
%   The math extension font \file{cmex}.  Whether or not this font
%   should be scaled is a question of its own, and there are other
%   packages (\Lpack{exscale}, \Lpack{amsmath}, \Lpack{amsfonts}) to
%   take care of it.
% \end{itemize}
%
% \subsection{How to load the package}
% \begin{sloppypar}
% The package should be loaded \danger \emph{before} \cmd{\documentclass},
% using the command |\RequirePackage{fix-cm}|, rather than the
% normal \cmd{\usepackage}.
% Rationale:
% If the package is loaded in the preamble, a preceding package or
% even the code of the document class may have used any of the CM
% fonts already.  However, the definitions of those fonts, that are
% already in use, cannot be changed any more.
% \end{sloppypar}
%
% \subsection{Usage notes}
% In contrast to what you may expect, \Lpack{fix-cm} does \emph{not}
% ensure that line and page breaks stay the same, when you switch an
% existing document from OT1 to T1 encoding.  The package does not
% turn off all of the additional design sizes in the EC fonts
% collection: Those, that contribute considerably to the typographical
% quality and do not conflict with the math fonts,
% are---indeed---used.
%
% Be careful when using arbitrary, non-standard font sizes with
% applications that need bitmap fonts: You may end up \danger with
% lots of possibly huge \file{.pk} files.  Also, \MF{} chokes
% sometimes on extremely small or large sizes, because of arithmetic
% problems.
%
% \Lpack{fix-cm} supersedes the experimental packages \Lpack{cmsd} and
% \Lpack{fix-ec}, which are no longer distributed.
%
% The packages \Lpack{type1cm} and \Lpack{type1ec} must not be loaded
% additionally; they enable only continuous scaling.
%
%
%
%
% \StopEventually{}
%
% \section{Implementation}
%
% \subsection{Preliminaries}
% The \LaTeX{} kernel does not declare the font encoding TS1.
% However, we are going to set up font definitions for this encoding,
% so we have to declare it now.
%    \begin{macrocode}
%<*fix-cm>
\input{ts1enc.def}
%    \end{macrocode}
%
% In case the package is loaded in the preamble, any of the CM fonts may
% have been used already and cannot be redefined.  Yet we try to
% intercept at least the problem that is most likely to occur, i.e.,
% a hidden \cmd{\normalfont}.  Most of the standard definitions
% are ok, but those for T1 encoding and 10.95\,pt need to be removed:
%    \begin{macrocode}
\expandafter \let \csname T1/cmr/m/n/10.95\endcsname \relax
\expandafter \let \csname T1/cmss/m/n/10.95\endcsname \relax
\expandafter \let \csname T1/cmtt/m/n/10.95\endcsname \relax
\expandafter \let \csname T1/cmvtt/m/n/10.95\endcsname \relax
%    \end{macrocode}
%
% \Lpack{fix-cm} may still fail, if the EC fonts are preloaded in the
% \LaTeX{} format file. This situation is, however, very unlikely and could occur
% only with a customized format.
%
% The remainder of the package is enclosed in a group, where the catcodes
% are guaranteed to be appropriate for the processing of font definitions.
%    \begin{macrocode}
\begingroup
\nfss@catcodes
%    \end{macrocode}
%
% \subsection{T1 encoding}
%
% \paragraph{CM Roman}
%    \begin{macrocode}
\DeclareFontFamily{T1}{cmr}{}
\DeclareFontShape{T1}{cmr}{m}{n}{
        <-6>    ecrm0500
        <6-7>   ecrm0600
        <7-8>   ecrm0700
        <8-9>   ecrm0800
        <9-10>  ecrm0900
        <10-12> ecrm1000
        <12-17> ecrm1200
        <17->   ecrm1728
      }{}
\DeclareFontShape{T1}{cmr}{m}{sl}{
        <-6>    ecsl0500
        <6-7>   ecsl0600
        <7-8>   ecsl0700
        <8-9>   ecsl0800
        <9-10>  ecsl0900
        <10-12> ecsl1000
        <12-17> ecsl1200
        <17->   ecsl1728
      }{}
\DeclareFontShape{T1}{cmr}{m}{it}{
        <-8>    ecti0700
        <8-9>   ecti0800
        <9-10>  ecti0900
        <10-12> ecti1000
        <12-17> ecti1200
        <17->   ecti1728
      }{}
\DeclareFontShape{T1}{cmr}{m}{sc}{
        <-6>    eccc0500
        <6-7>   eccc0600
        <7-8>   eccc0700
        <8-9>   eccc0800
        <9-10>  eccc0900
        <10-12> eccc1000
        <12-17> eccc1200
        <17->   eccc1728
               }{}
\DeclareFontShape{T1}{cmr}{m}{ui}{
        <-8>    ecui0700
        <8-9>   ecui0800
        <9-10>  ecui0900
        <10-12> ecui1000
        <12-17> ecui1200
        <17->   ecui1728
      }{}
\DeclareFontShape{T1}{cmr}{b}{n}{
        <-6>    ecrb0500
        <6-7>   ecrb0600
        <7-8>   ecrb0700
        <8-9>   ecrb0800
        <9-10>  ecrb0900
        <10-12> ecrb1000
        <12-17> ecrb1200
        <17->   ecrb1728
      }{}
\DeclareFontShape{T1}{cmr}{bx}{n}{
        <-6>    ecbx0500
        <6-7>   ecbx0600
        <7-8>   ecbx0700
        <8-9>   ecbx0800
        <9-10>  ecbx0900
        <10-12> ecbx1000
        <12->   ecbx1200
      }{}
\DeclareFontShape{T1}{cmr}{bx}{sl}{
        <-6>    ecbl0500
        <6-7>   ecbl0600
        <7-8>   ecbl0700
        <8-9>   ecbl0800
        <9-10>  ecbl0900
        <10-12> ecbl1000
        <12->   ecbl1200
      }{}
\DeclareFontShape{T1}{cmr}{bx}{it}{
        <-8>    ecbi0700
        <8-9>   ecbi0800
        <9-10>  ecbi0900
        <10-12> ecbi1000
        <12->   ecbi1200
      }{}
\DeclareFontShape{T1}{cmr}{bx}{sc}{
        <-6>    ecxc0500
        <6-7>   ecxc0600
        <7-8>   ecxc0700
        <8-9>   ecxc0800
        <9-10>  ecxc0900
        <10-12> ecxc1000
        <12->   ecxc1200
      }{}
%
%    \end{macrocode}
%
% \paragraph{CM Sans}
%    \begin{macrocode}
\DeclareFontFamily{T1}{cmss}{}
\DeclareFontShape{T1}{cmss}{m}{n}{
        <-9>    ecss0800
        <9-10>  ecss0900
        <10-12> ecss1000
        <12-17> ecss1200
        <17->   ecss1728
      }{}
\DeclareFontShape{T1}{cmss}{m}{sl}{
        <-9>    ecsi0800
        <9-10>  ecsi0900
        <10-12> ecsi1000
        <12-17> ecsi1200
        <17->   ecsi1728
      }{}
\DeclareFontShape{T1}{cmss}{m}{it}
       {<->ssub*cmss/m/sl}{}
\DeclareFontShape{T1}{cmss}{m}{sc}
       {<->sub*cmr/m/sc}{}
\DeclareFontShape{T1}{cmss}{sbc}{n}{
        <->     ecssdc10
       }{}
\DeclareFontShape{T1}{cmss}{bx}{n}{
        <-10>   ecsx0900
        <10->   ecsx1000
      }{}
\DeclareFontShape{T1}{cmss}{bx}{sl}{
        <-10>   ecso0900
        <10->   ecso1000
      }{}
\DeclareFontShape{T1}{cmss}{bx}{it}
       {<->ssub*cmss/bx/sl}{}
%    \end{macrocode}
% The following substitutions are not provided in the default
% \file{.fd} files.  I have included them, so that you can
% easily use the EC fonts with the default bold series being
% \file{b} rather than \file{bx}.
%    \begin{macrocode}
\DeclareFontShape{T1}{cmss}{b}{n}
       {<->ssub*cmss/bx/n}{}
\DeclareFontShape{T1}{cmss}{b}{sl}
       {<->ssub*cmss/bx/sl}{}
\DeclareFontShape{T1}{cmss}{b}{it}
       {<->ssub*cmss/bx/sl}{}
%    \end{macrocode}
%
% \paragraph{CM Typewriter}
%    \begin{macrocode}
\DeclareFontFamily{T1}{cmtt}{\hyphenchar \font\m@ne}
\DeclareFontShape{T1}{cmtt}{m}{n}{
        <-9>    ectt0800
        <9-10>  ectt0900
        <10-12> ectt1000
        <12-17> ectt1200
        <17->   ectt1728
      }{}
\DeclareFontShape{T1}{cmtt}{m}{it}{
        <-9>    ecit0800
        <9-10>  ecit0900
        <10-12> ecit1000
        <12-17> ecit1200
        <17->   ecit1728
      }{}
\DeclareFontShape{T1}{cmtt}{m}{sl}{
        <-9>    ecst0800
        <9-10>  ecst0900
        <10-12> ecst1000
        <12-17> ecst1200
        <17->   ecst1728
      }{}
\DeclareFontShape{T1}{cmtt}{m}{sc}{
        <-9>    ectc0800
        <9-10>  ectc0900
        <10-12> ectc1000
        <12-17> ectc1200
        <17->   ectc1728
      }{}
\DeclareFontShape{T1}{cmtt}{bx}{n}
       {<->sub * cmtt/m/n}{}
\DeclareFontShape{T1}{cmtt}{bx}{it}
       {<->sub * cmtt/m/it}{}
\DeclareFontShape{T1}{cmtt}{bx}{sl}
       {<->sub * cmtt/m/sl}{}
%    \end{macrocode}
% Substitutions not provided in the default \file{.fd} files:
%    \begin{macrocode}
\DeclareFontShape{T1}{cmtt}{b}{n}
       {<->sub * cmtt/m/n}{}
\DeclareFontShape{T1}{cmtt}{b}{it}
       {<->sub * cmtt/m/it}{}
\DeclareFontShape{T1}{cmtt}{b}{sl}
       {<->sub * cmtt/m/sl}{}
%    \end{macrocode}
%
% \paragraph{CM Typewiter (var.)}
%    \begin{macrocode}
\DeclareFontFamily{T1}{cmvtt}{}
\DeclareFontShape{T1}{cmvtt}{m}{n}{
        <-9>    ecvt0800
        <9-10>  ecvt0900
        <10-12> ecvt1000
        <12-17> ecvt1200
        <17->   ecvt1728
      }{}
\DeclareFontShape{T1}{cmvtt}{m}{it}{
        <-9>    ecvi0800
        <9-10>  ecvi0900
        <10-12> ecvi1000
        <12-17> ecvi1200
        <17->   ecvi1728
      }{}
%    \end{macrocode}
%
% \subsection{TS1 encoding}
%
% \paragraph{CM Roman}
%    \begin{macrocode}
\DeclareFontFamily{TS1}{cmr}{\hyphenchar\font\m@ne}
\DeclareFontShape{TS1}{cmr}{m}{n}{
        <-6>    tcrm0500
        <6-7>   tcrm0600
        <7-8>   tcrm0700
        <8-9>   tcrm0800
        <9-10>  tcrm0900
        <10-12> tcrm1000
        <12-17> tcrm1200
        <17->   tcrm1728
      }{}
\DeclareFontShape{TS1}{cmr}{m}{sl}{
        <-6>    tcsl0500
        <6-7>   tcsl0600
        <7-8>   tcsl0700
        <8-9>   tcsl0800
        <9-10>  tcsl0900
        <10-12> tcsl1000
        <12-17> tcsl1200
        <17->   tcsl1728
      }{}
\DeclareFontShape{TS1}{cmr}{m}{it}{
        <-8>    tcti0700
        <8-9>   tcti0800
        <9-10>  tcti0900
        <10-12> tcti1000
        <12-17> tcti1200
        <17->   tcti1728
      }{}
\DeclareFontShape{TS1}{cmr}{m}{ui}{
        <-8>    tcui0700
        <8-9>   tcui0800
        <9-10>  tcui0900
        <10-12> tcui1000
        <12-17> tcui1200
        <17->   tcui1728
      }{}
\DeclareFontShape{TS1}{cmr}{b}{n}{
        <-6>    tcrb0500
        <6-7>   tcrb0600
        <7-8>   tcrb0700
        <8-9>   tcrb0800
        <9-10>  tcrb0900
        <10-12> tcrb1000
        <12-17> tcrb1200
        <17->   tcrb1728
      }{}
\DeclareFontShape{TS1}{cmr}{bx}{n}{
        <-6>    tcbx0500
        <6-7>   tcbx0600
        <7-8>   tcbx0700
        <8-9>   tcbx0800
        <9-10>  tcbx0900
        <10-12> tcbx1000
        <12->   tcbx1200
      }{}
\DeclareFontShape{TS1}{cmr}{bx}{sl}{
        <-6>    tcbl0500
        <6-7>   tcbl0600
        <7-8>   tcbl0700
        <8-9>   tcbl0800
        <9-10>  tcbl0900
        <10-12> tcbl1000
        <12->   tcbl1200
      }{}
\DeclareFontShape{TS1}{cmr}{bx}{it}{
        <-8>    tcbi0700
        <8-9>   tcbi0800
        <9-10>  tcbi0900
        <10-12> tcbi1000
        <12->   tcbi1200
      }{}
%    \end{macrocode}
%
% \paragraph{CM Sans}
%    \begin{macrocode}
\DeclareFontFamily{TS1}{cmss}{\hyphenchar\font\m@ne}
\DeclareFontShape{TS1}{cmss}{m}{n}{
        <-9>    tcss0800
        <9-10>  tcss0900
        <10-12> tcss1000
        <12-17> tcss1200
        <17->   tcss1728
      }{}
\DeclareFontShape{TS1}{cmss}{m}{it}
       {<->ssub*cmss/m/sl}{}
\DeclareFontShape{TS1}{cmss}{m}{sl}{
        <-9>    tcsi0800
        <9-10>  tcsi0900
        <10-12> tcsi1000
        <12-17> tcsi1200
        <17->   tcsi1728
      }{}
\DeclareFontShape{TS1}{cmss}{sbc}{n}{
        <->     tcssdc10
       }{}
\DeclareFontShape{TS1}{cmss}{bx}{n}{
        <-10>   tcsx0900
        <10->   tcsx1000
      }{}
\DeclareFontShape{TS1}{cmss}{bx}{sl}{
        <-10>   tcso0900
        <10->   tcso1000
      }{}
\DeclareFontShape{TS1}{cmss}{bx}{it}
       {<->ssub*cmss/bx/sl}{}
%    \end{macrocode}
% Substitutions not provided in the default \file{.fd} files:
%    \begin{macrocode}
\DeclareFontShape{TS1}{cmss}{b}{n}
       {<->ssub*cmss/bx/n}{}
\DeclareFontShape{TS1}{cmss}{b}{sl}
       {<->ssub*cmss/bx/sl}{}
\DeclareFontShape{TS1}{cmss}{b}{it}
       {<->ssub*cmss/bx/sl}{}
%    \end{macrocode}
%
% \paragraph{CM Typewriter}
%    \begin{macrocode}
\DeclareFontFamily{TS1}{cmtt}{\hyphenchar \font\m@ne}
\DeclareFontShape{TS1}{cmtt}{m}{n}{
        <-9>    tctt0800
        <9-10>  tctt0900
        <10-12> tctt1000
        <12-17> tctt1200
        <17->   tctt1728
      }{}
\DeclareFontShape{TS1}{cmtt}{m}{it}{
        <-9>    tcit0800
        <9-10>  tcit0900
        <10-12> tcit1000
        <12-17> tcit1200
        <17->   tcit1728
      }{}
\DeclareFontShape{TS1}{cmtt}{m}{sl}{
        <-9>    tcst0800
        <9-10>  tcst0900
        <10-12> tcst1000
        <12-17> tcst1200
        <17->   tcst1728
      }{}
\DeclareFontShape{TS1}{cmtt}{bx}{n}
       {<->sub * cmtt/m/n}{}
\DeclareFontShape{TS1}{cmtt}{bx}{it}
       {<->sub * cmtt/m/it}{}
\DeclareFontShape{TS1}{cmtt}{bx}{sl}
       {<->sub * cmtt/m/sl}{}
%    \end{macrocode}
% Substitutions not provided in the default \file{.fd} files:
%    \begin{macrocode}
\DeclareFontShape{TS1}{cmtt}{b}{n}
       {<->sub * cmtt/m/n}{}
\DeclareFontShape{TS1}{cmtt}{b}{it}
       {<->sub * cmtt/m/it}{}
\DeclareFontShape{TS1}{cmtt}{b}{sl}
       {<->sub * cmtt/m/sl}{}
%    \end{macrocode}
%
% \paragraph{CM Typewriter (var.)}
%    \begin{macrocode}
\DeclareFontFamily{TS1}{cmvtt}{}
\DeclareFontShape{TS1}{cmvtt}{m}{n}{
        <-9>    tcvt0800
        <9-10>  tcvt0900
        <10-12> tcvt1000
        <12-17> tcvt1200
        <17->   tcvi1728
      }{}
\DeclareFontShape{TS1}{cmvtt}{m}{it}{
        <-9>    tcvi0800
        <9-10>  tcvi0900
        <10-12> tcvi1000
        <12-17> tcvi1200
        <17->   tcvi1728
      }{}
%    \end{macrocode}
%
% \subsection{OT1 encoding}
%
% \paragraph{CM Roman}
%    \begin{macrocode}
\DeclareFontFamily{OT1}{cmr}{\hyphenchar\font45 }
\DeclareFontShape{OT1}{cmr}{m}{n}{
        <-6>    cmr5
        <6-7>   cmr6
        <7-8>   cmr7
        <8-9>   cmr8
        <9-10>  cmr9
        <10-12> cmr10
        <12-17> cmr12
        <17->   cmr17
      }{}
\DeclareFontShape{OT1}{cmr}{m}{sl}{
        <-9>    cmsl8
        <9-10>  cmsl9
        <10-12> cmsl10
        <12->   cmsl12
      }{}
\DeclareFontShape{OT1}{cmr}{m}{it}{
        <-8>    cmti7
        <8-9>   cmti8
        <9-10>  cmti9
        <10-12> cmti10
        <12->   cmti12
      }{}
\DeclareFontShape{OT1}{cmr}{m}{sc}{
        <->     cmcsc10
      }{}
\DeclareFontShape{OT1}{cmr}{m}{ui}{
        <->     cmu10
      }{}
\DeclareFontShape{OT1}{cmr}{b}{n}{
        <->     cmb10
      }{}
\DeclareFontShape{OT1}{cmr}{bx}{n}{
        <-6>    cmbx5
        <6-7>   cmbx6
        <7-8>   cmbx7
        <8-9>   cmbx8
        <9-10>  cmbx9
        <10-12> cmbx10
        <12->   cmbx12
      }{}
\DeclareFontShape{OT1}{cmr}{bx}{sl}{
        <->     cmbxsl10
      }{}
\DeclareFontShape{OT1}{cmr}{bx}{it}{
        <->     cmbxti10
      }{}
\DeclareFontShape{OT1}{cmr}{bx}{ui}
      {<->sub*cmr/m/ui}{}
%    \end{macrocode}
%
% \paragraph{CM Sans}
%    \begin{macrocode}
\DeclareFontFamily{OT1}{cmss}{\hyphenchar\font45 }
\DeclareFontShape{OT1}{cmss}{m}{n}{
        <-9>    cmss8
        <9-10>  cmss9
        <10-12> cmss10
        <12-17> cmss12
        <17->   cmss17
      }{}
\DeclareFontShape{OT1}{cmss}{m}{it}
      {<->sub*cmss/m/sl}{}
\DeclareFontShape{OT1}{cmss}{m}{sl}{
        <-9>    cmssi8
        <9-10>  cmssi9
        <10-12> cmssi10
        <12-17> cmssi12
        <17->   cmssi17
      }{}
\DeclareFontShape{OT1}{cmss}{m}{sc}
       {<->sub*cmr/m/sc}{}
\DeclareFontShape{OT1}{cmss}{m}{ui}
       {<->sub*cmr/m/ui}{}
\DeclareFontShape{OT1}{cmss}{sbc}{n}{
        <->     cmssdc10
      }{}
\DeclareFontShape{OT1}{cmss}{bx}{n}{
        <->     cmssbx10
      }{}
\DeclareFontShape{OT1}{cmss}{bx}{ui}
       {<->sub*cmr/bx/ui}{}
%    \end{macrocode}
%
% \paragraph{CM Typewriter}
%    \begin{macrocode}
\DeclareFontFamily{OT1}{cmtt}{\hyphenchar \font\m@ne}
\DeclareFontShape{OT1}{cmtt}{m}{n}{
        <-9>    cmtt8
        <9-10>  cmtt9
        <10-12> cmtt10
        <12->   cmtt12
      }{}
\DeclareFontShape{OT1}{cmtt}{m}{it}{
        <->     cmitt10
      }{}
\DeclareFontShape{OT1}{cmtt}{m}{sl}{
        <->     cmsltt10
      }{}
\DeclareFontShape{OT1}{cmtt}{m}{sc}{
        <->     cmtcsc10
      }{}
\DeclareFontShape{OT1}{cmtt}{m}{ui}
       {<->ssub*cmtt/m/it}{}
\DeclareFontShape{OT1}{cmtt}{bx}{n}
       {<->ssub*cmtt/m/n}{}
\DeclareFontShape{OT1}{cmtt}{bx}{it}
       {<->ssub*cmtt/m/it}{}
\DeclareFontShape{OT1}{cmtt}{bx}{ui}
       {<->ssub*cmtt/m/it}{}
%    \end{macrocode}
%
% \paragraph{CM Typewriter (var.)}
%    \begin{macrocode}
\DeclareFontFamily{OT1}{cmvtt}{\hyphenchar\font45 }
\DeclareFontShape{OT1}{cmvtt}{m}{n}{
        <->     cmvtt10
      }{}
\DeclareFontShape{OT1}{cmvtt}{m}{it}{
        <->     cmvtti10
      }{}
%    \end{macrocode}
%
% \subsection{OML and OMS encoded math fonts}
%    \begin{macrocode}
\DeclareFontFamily{OML}{cmm}{\skewchar\font127 }
\DeclareFontShape{OML}{cmm}{m}{it}{
        <-6>    cmmi5
        <6-7>   cmmi6
        <7-8>   cmmi7
        <8-9>   cmmi8
        <9-10>  cmmi9
        <10-12> cmmi10
        <12->   cmmi12
      }{}
\DeclareFontShape{OML}{cmm}{b}{it}{<-6>cmmib5<6-8>cmmib7<8->cmmib10}{}
\DeclareFontShape{OML}{cmm}{bx}{it}
       {<->ssub*cmm/b/it}{}
%    \end{macrocode}
%    \begin{macrocode}
\DeclareFontFamily{OMS}{cmsy}{\skewchar\font48 }
\DeclareFontShape{OMS}{cmsy}{m}{n}{
        <-6>    cmsy5
        <6-7>   cmsy6
        <7-8>   cmsy7
        <8-9>   cmsy8
        <9-10>  cmsy9
        <10->   cmsy10
      }{}
\DeclareFontShape{OMS}{cmsy}{b}{n}{<-6>cmbsy5<6-8>cmbsy7<8->cmbsy10}{}
%    \end{macrocode}
%
% \subsection{\LaTeX{} symbols}
%    \begin{macrocode}
\DeclareFontFamily{U}{lasy}{}
\DeclareFontShape{U}{lasy}{m}{n}{
        <-6>    lasy5
        <6-7>   lasy6
        <7-8>   lasy7
        <8-9>   lasy8
        <9-10>  lasy9
        <10->   lasy10
      }{}
\DeclareFontShape{U}{lasy}{b}{n}{
        <-10>   ssub * lasy/m/n
        <10->   lasyb10
      }{}
%    \end{macrocode}
%    \begin{macrocode}
\endgroup
%</fix-cm>
%    \end{macrocode}
%
% \Finale
%
\endinput

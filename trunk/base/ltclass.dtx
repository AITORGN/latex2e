% \iffalse meta-comment
%
% Copyright 1993-2016
% The LaTeX3 Project and any individual authors listed elsewhere
% in this file.
%
% This file is part of the LaTeX base system.
% -------------------------------------------
%
% It may be distributed and/or modified under the
% conditions of the LaTeX Project Public License, either version 1.3c
% of this license or (at your option) any later version.
% The latest version of this license is in
%    http://www.latex-project.org/lppl.txt
% and version 1.3c or later is part of all distributions of LaTeX
% version 2005/12/01 or later.
%
% This file has the LPPL maintenance status "maintained".
%
% The list of all files belonging to the LaTeX base distribution is
% given in the file `manifest.txt'. See also `legal.txt' for additional
% information.
%
% The list of derived (unpacked) files belonging to the distribution
% and covered by LPPL is defined by the unpacking scripts (with
% extension .ins) which are part of the distribution.
%
% \fi
%
% \iffalse
%%% From File: ltclass.dtx
%
%<*driver>
% \fi
\ProvidesFile{ltclass.dtx}
             [2016/10/02 v1.2a LaTeX Kernel (Class & Package Interface)]
% \iffalse
\documentclass{ltxdoc}
\GetFileInfo{ltclass.dtx}
\begin{document}
\title{The main structure of documents}
\author{Frank Mittelbach\and Chris Rowley\and Alan Jeffrey\and
        David Carlisle}
\date{\filedate}
 \MaintainedByLaTeXTeam{latex}
 \maketitle
 \DocInput{\filename}
\end{document}
%</driver>
% \fi
%
% \iffalse
% (C) Copyright Frank Mittelbach, Chris Rowley,
%               Alan Jeffrey and David Carlisle 1993-1998.
% All rights reserved.
% \fi
%
%
% \changes{v1.0f}{1994/05/22}{Use new warning and error commands}
% \changes{v1.0l}{1994/11/17}{\cs{@tempa} to \cs{reserved@a}}
% \changes{v1.0z}{1998/03/21}{Added to documentation of filecontents}
% \changes{v1.1c}{1998/08/17}{(RmS) Minor documentation fixes.}
%
% \section{Introduction}
%
% This file implements the following declarations, which replace
% |\documentstyle| in  \LaTeXe\ documents.
%
% Note that old documents containing |\documentstyle| will be run using
% a compatibility option---thus keeping everyone happy, we hope!
%
% The overall idea is that there are two types of `style files':
% `class files' which define elements and provide a default formatting
% for them;  and `packages' which provide extra functionality.  One
% difference between \LaTeXe\ and \LaTeX2.09 is that \LaTeXe\ packages
% may have options. Note that options to classes packages may be
% implemented such that they input files, but these file names are not
% necessarily directly related to the option name.
%
% \section{User interface}
%
% |\documentclass[|\meta{main-option-list}|]{|^^A
%   \meta{class}|}[|\meta{version}|]|
%
% There must be exactly one such declaration, and it must come first.
% The \meta{main-option-list} is a list of options which can modify the
% formatting of elements which are defined in the \meta{class} file
% as well as in all following |\usepackage| declarations (see below).
% The \meta{version} is a version number, beginning with a date in the
% format |YYYY/MM/DD|.  If an older version of the class is found, a
% warning is issued.
%
% \bigskip
%
% |\documentstyle[|\meta{main-option-list}|]{|^^A
%   \meta{class}|}[|\meta{version}|]|
%
% The |\documentstyle| declaration is kept in order to maintain upward
% compatibility with \LaTeX2.09 documents.  It is similar to
% |\documentclass|, but it causes all options in
% \meta{main-option-list} that the \meta{class} does not use to be
% passed to |\RequirePackage| after the options have been processed.
% This maintains compatibility with the 2.09 behaviour. Also a flag is
% set to indicate that the document is to be processed in \LaTeX2.09
% compatibility mode.  As far as most packages are concerned, this
% only affects the warnings and errors \LaTeX\ generates. This flag
% does affect the definition of font commands, and |\sloppy|.
%
% \bigskip
%
% |\usepackage[|\meta{package-option-list}|]{|^^A
%    \meta{package-list}|}[|\meta{version}|]|
%
% There can be any number of these declarations. All packages in
% \meta{package-list} are called with the same options.
%
% Each \meta{package} file defines new elements (or modifies those
% defined in the \meta{class}), and thus extends the range of documents
% which can be processed.
% The \meta{package-option-list} is a list of options which can modify
% the formatting of elements defined in the \meta{package} file.
% The \meta{version} is a version number, beginning with a date in the
% format |YYYY/MM/DD|.  If an older version of the package is found, a
% warning is issued.
%
% Each package is loaded only once.  If the same package is requested
% more than once, nothing happens, unless the package has been requested
% with options that were not given the first time it was loaded, in
% which case an error is produced.
%
% As well as processing the options given in the
% \meta{package-option-list}, each package processes the
% \meta{main-option-list}.  This means that options that affect all
% of the packages can be given globally, rather than repeated for every
% package.
%
% Note that class files have the extension |.cls|, packages have the
% extension |.sty|.
%
% \DescribeEnv{filecontents}
% The environment |filecontents| is intended for passing the contents
% of packages, options, or other files along with a document in a
% single file.
% It has one argument, which is the name of the file to create. If that
% file already exists (maybe only in the current directory if the OS
% supports a notion of a `current directory' or `default directory')
% then nothing happens
% (except for an information message) and the body of the environment
% is bypassed. Otherwise, the body of the environment is written
% verbatim to the file name given as the first argument, together with
% some comments about how it was produced.
%
% The environment is allowed only before |\documentclass| to ensure
% that all packages or options necessary for this particular run are
% present when needed.  The begin and end tags should each be on a
% line by itself.  There is also a star-form; this does not write
% extra comments into the file.
%
% \subsection{Option processing}
%
% When the options are processed, they are divided into two types: {\em
% local\/} and {\em global}:
% \begin{itemize}
%
% \item For a class, the options in the |\documentclass| command are
%    local.
%
% \item For a package, the options in the |\usepackage| command are
%    local, and the options in the |\documentclass| command are global.
%
% \end{itemize}
% The options for |\documentclass| and |\usepackage|
% are processed in the following way:
% \begin{enumerate}
%
% \item The local and global options that have been declared
%   (using |\DeclareOption| as  described below) are processed
%   first.
%
%  In the case of |\ProcessOptions|, they are processed in the order
%  that they were declared in the class or package.
%
%  In the case of |\ProcessOptions*|, they are processed in the order
%  that they appear in the option-lists. First the global options, and
%  then the local ones.
%
% \item Any remaining local options are dealt with using the default
%   option (declared using the |\DeclareOption*| declaration described
%   below).  For document classes, this usually does nothing, but
%   records the option on a list of unused options.
%   For packages, this usually produces an error.
%
% \end{enumerate}
% Finally, when |\begin{document}| is reached, if there are any global
% options which have not been used by either the class or any package,
% the system will produce a warning.
%
%
% \section{Class and Package interface}
%
% \subsection{Class name and version}
%
% \DescribeMacro\ProvidesClass
% A class can identify itself with the
% |\ProvidesClass{|\meta{name}|}[|\meta{version}|]| command.  The
% \meta{version} should begin with a date in the format |YYYY/MM/DD|.
%
% \subsection{Package name and version}
%
% \DescribeMacro\ProvidesPackage
% A package can identify itself with the
% |\ProvidesPackage|\marg{name}\oarg{version} command.  The
% \meta{version} should begin with a date in the format |YYYY/MM/DD|.
%
% \subsection{Requiring other packages}
%
% \DescribeMacro\RequirePackage
% Packages or classes can load other packages using\\
% |\RequirePackage|\oarg{options}\marg{name}\oarg{version}.\\
% If the package has already been loaded, then nothing happens unless
% the requested options are not a subset of the options with which it
% was loaded, in which case an error is called.
%
% \DescribeMacro\LoadClass
%  Similar to |\RequirePackage|, but for classes, may not be used in
%  package files.
%
% \DescribeMacro\PassOptionsToPackage
% Packages can pass options to other packages using:\\
% |\PassOptionsToPackage{|\meta{options}|}{|\meta{package}|}|.\\
% \DescribeMacro\PassOptionsToClass
% This adds the \meta{options} to the options list of any future
% |\RequirePackage| or |\usepackage| command.  For example:
% \begin{verbatim}
%    \PassOptionsToPackage{foo,bar}{fred}
%    \RequirePackage[baz]{fred}\end{verbatim}
% is the same as:
% \begin{verbatim}
%    \RequirePackage[foo,bar,baz]{fred}\end{verbatim}
%
% \DescribeMacro\LoadClassWithOptions
% |\LoadClassWithOptions|\marg{name}\oarg{version}:\\
% This is similar to
% |\LoadClass|, but it always calls class \meta{name} with
% exactly the same option list that is being used by the current class,
% rather than an option explicitly  supplied or passed on by
% |\PassOptionsToClass|.
% \DescribeMacro\RequirePackageWithOptions
% |\RequirePackageWithOptions| is the analogous command for packages.
%
% This is mainly intended to allow one class to simply build on another,
% for example:
%\begin{verbatim}
%   \LoadClassWithOptions{article}
%\end{verbatim}
%
% This should be contrasted with the slightly different construction
%\begin{verbatim}
%   \DeclareOption*{\PassOptionsToClass{\CurrentOption}{article}}
%   \ProcessOptions
%   \LoadClass{article}
%\end{verbatim}
%
% As used here, the effects are more or less the same, but the
% version using |\LoadClassWithOptions| is slightly quicker
% (and less to type).
% If, however, the class declares options of its own then
% the two constructions are different; compare, for example:
%\begin{verbatim}
%   \DeclareOption{landscape}{...}
%   \ProcessOptions
%   \LoadClassWithOptions{article}
%\end{verbatim}
% with:
%\begin{verbatim}
%   \DeclareOption{landscape}{...}
%   \DeclareOption*{\PassOptionsToClass{\CurrentOption}{article}}
%   \ProcessOptions
%   \LoadClass{article}
%\end{verbatim}
% In the first case, the \textsf{article} class will be called with
% option |landscape| precisely when the current class is called with
% this option; but in the second example it will
% not as in that case \textsf{article} is only passed options by the
% default option handler, which is not used for |landscape| as that
% option is explicitly declared.
%
% \DescribeMacro\@ifpackageloaded
% To find out if a package has already been loaded, use\\
% \DescribeMacro\@ifclassloaded
% |\@ifpackageloaded{|\meta{package}|}{|\meta{true}|}{|\meta{false}|}|.
%
% \DescribeMacro\@ifpackagelater
% \changes{v1.1i}{2013/07/07}{Correctly describe how the date in
%       \cs{@ifpackagelater} is used}
% To find out if a package has already been loaded with a version
% equal to or more
% recent than \meta{version}, use\\
% \DescribeMacro\@ifclasslater
% |\@ifpackagelater{|\meta{package}|}{|\meta{version}|}{|^^A
% \meta{true}|}{|\meta{false}|}|.
%
% \DescribeMacro\@ifpackagewith
% To find out if a package has already been loaded with at least the
% options \meta{options}, use
% \DescribeMacro\@ifclasswith
% |\@ifpackagewith{|\meta{package}|}{|\meta{options}|}{|^^A
% \meta{true}|}{|\meta{false}|}|.
%
% There exists one package that can't be tested with the above
% commands: the \texttt{fontenc} package pretends that it was never
% loaded to allow for repeated reloading with different options (see
% \texttt{ltoutenc.dtx} for details).
%
%
% \subsection{Declaring new options}
%
% Options for classes and packages are built using the same macros.
%
% \DescribeMacro\DeclareOption To define a builtin option, use
% |\DeclareOption{|\meta{name}|}{|\meta{code}|}|.
%
% \DescribeMacro{\DeclareOption*} To define the default action to
% perform for local options which have not been declared, use
% |\DeclareOption*{|\meta{code}|}|.
%
% {\em Note\/}: there should be no use of\\
%  |\RequirePackage|, |\DeclareOption|, |\DeclareOption*| or
%   |\ProcessOptions|\\
% inside |\DeclareOption| or |\DeclareOption*|.
%
% Possible uses for |\DeclareOption*| include:
%
% |\DeclareOption*{}|\\
%    Do nothing. Silently accept unknown options. (This suppresses the
%    usual warnings.)
%
% |\DeclareOption*{\@unkownoptionerror}|\\
%     Complain about unknown local options. (The initial setting for
%       package files.)
%
% |\DeclareOption*{\PassOptionsToPackage{\CurrentOption}|^^A
%                                     |{|\meta{pkg-name}|}|\\
% Handle the the current option by passing it on to the package
% \meta{pkg-name}, which will presumably be loaded via
% |\RequirePackage| later in the file. This is useful for building
% `extension' packages, that perhaps handle a couple of new options,
% but then pass everything else on to an existing package.
%
% |\DeclareOption*{\InputIfFileExists{xx-\CurrentOption.yyy}%|\\
% |               {}%|\\
% |               {\OptionNotUsed}}|\\
%  Handle the option foo by loading the file |xx-foo.yyy| if it
%  exists, otherwise do nothing, but declare that the option was not
%  used.
%  Actually the |\OptionNotUsed| declaration is only needed if this is
%  being used in class files, but does no harm in package files.
%
%
% \subsection{Safe Input Macros}
% \DescribeMacro{\InputIfFileExists}
%  |\InputIfFileExists{|\meta{file}|}{|\meta{then}|}{|\meta{else}|}|\\
% Inputs \meta{file} if it exists. Immediately before the input,
% \meta{then} is executed. Otherwise \meta{else} is executed.
%
% \DescribeMacro{\IfFileExists}
% As above, but does not input the file.
%
% One thing you might like to put in the \meta{else} clause is
%
% \DescribeMacro{\@missingfileerror}
% This starts an interactive request for a filename, supplying default
% extensions. Just hitting return causes the whole input to be skipped
% and entering |x| quits the current run,
%
% \DescribeMacro{\input}
% This has been redefined from the \LaTeX2.09 definition, in terms of
% the new commands |\InputIfFileExists| and |\@missingfileerror|.
%
%
% \DescribeMacro{\listfiles} Giving this declaration in the preamble
% causes a list of all files input via the `safe input' commands to be
% listed at the end. Any strings specified in the optional argument to
% |\ProvidesPackage| are listed alongside the file name. So files in
% standard (and other non-standard) distributions can put informative
% strings in this argument.
%
% \StopEventually{}
%
% \section{Implementation}
%
%    \begin{macrocode}
%<*2ekernel>
%    \end{macrocode}
%
%
% \changes{v0.2g}{1993/11/23}
%         {Various macros now moved to latex.tex.}
% \changes{v0.2g}{1993/11/23}
%         {Warnings and errors now directly coded.}
% \changes{v0.2h}{1993/11/28}
%         {Primitive filenames now terminated by space not \cs{relax}.}
% \changes{v0.2h}{1993/11/28}
%         {Directory syntax checing moved to dircheck.dtx}
% \changes{v0.2h}{1993/11/28}
%         {Assorted commands now in the kernel removed.}
% \changes{v0.2i}{1993/12/03}
%         {\cs{@onlypreamble}: Many commands declared.}
% \changes{v0.2i}{1993/12/03}
%         {Removed obsolete \cs{@documentclass}}
% \changes{v0.2o}{1993/12/13}
%         {Removed setting \cs{errorcontextlines}\ (now in latex.tex)}
% \changes{v0.2p}{1993/12/15}
%         {Removed extra `.'s from \cs{@@warning}s}
% \changes{v0.2s}{1994/01/17}
%         {Added many more \cs{@onlypreamble} commands}
% \changes{v0.2s}{1994/01/17}
%         {Wrapped long lines to column 72}
% \changes{v0.3a}{1994/03/02}
%         {Remove need for driver file}
% \changes{v0.3b}{1994/03/08}
%         {Modify driver code into `new style'}
% \changes{v0.3c}{1994/03/12}
%         {Change name from docclass to ltclass}
% \changes{v0.3h}{1994/04/25}
%         {Removed spurious extra `.'s at the end of error messages}
% \changes{v1.0a}{1994/04/29}
%         {Change version number to 1 (no other change)}
% \changes{v1.0k}{1994/11/03}
%         {Move \cs{@missingfileerror} to ltfiles}
%
% \begin{macro}{\if@compatibility}
%    The flag for compatibility mode.
%    \begin{macrocode}
\newif\if@compatibility
%    \end{macrocode}
% \end{macro}
%
% \begin{macro}{\@documentclasshook}
%    The hook called after the first |\documentclass| command.  By
%    default this checks to see if |\@normalsize| is undefined, and if
%    so, sets it to |\normalsize|.
% \changes{v0.2q}{1993/12/17}
%         {Macro added}
% \changes{v0.2z}{1994/02/10}
%         {Changed the name from \cs{@compatibility} to
%          \cs{@documentclasshook}, and added the check for whether
%          \cs{@normalsize} has been defined.  ASAJ.}
%    \begin{macrocode}
\def\@documentclasshook{%
   \ifx\@normalsize\@undefined
      \let\@normalsize\normalsize
   \fi
}
%    \end{macrocode}
% \end{macro}
%
%  \begin{macro}{\@declaredoptions}
%    This list is automatically built by |\DeclareOption|.
%    It is the list of options (separated by commas) declared in
%    the class or package file and it defines the order in which the
%    the corresponding |\ds@|\meta{option} commands are executed.
%    All local \meta{option}s which are not declared will be processed
%    in the order defined by the optional argument of |\documentclass|
%    or |\usepackage|.
%    \begin{macrocode}
\let\@declaredoptions\@empty
%    \end{macrocode}
%  \end{macro}
%
%  \begin{macro}{\@classoptionslist}
%    List of options of the main class.
% \changes{v1.0u}{1996/07/26}{made only preamble}
%    \begin{macrocode}
\let\@classoptionslist\relax
\@onlypreamble\@classoptionslist
%    \end{macrocode}
%  \end{macro}
%
%  \begin{macro}{\@unusedoptionlist}
% \changes{v1.0u}{1996/07/26}{made only preamble}
%    List of options of the main class that haven't been declared or
%    loaded as class option files.
%    \begin{macrocode}
\let\@unusedoptionlist\@empty
\@onlypreamble\@unusedoptionlist
%    \end{macrocode}
%  \end{macro}
%
%  \begin{macro}{\CurrentOption}
%    Name of current package or option.
% \changes{v0.2c}{1993/11/17}
%         {Name changed from \cs{@curroption}}
%    \begin{macrocode}
\let\CurrentOption\@empty
%    \end{macrocode}
%  \end{macro}
%
%  \begin{macro}{\@currname}
%    Name of current package or option.
%    \begin{macrocode}
\let\@currname\@empty
%    \end{macrocode}
%  \end{macro}
%
% \begin{macro}{\@currext}
%    The current file extension.
% \changes{v0.2a}{1993/11/14}{Name changed from \cs{@currextension}}
%    \begin{macrocode}
\global\let\@currext=\@empty
%    \end{macrocode}
% \end{macro}
%
% \begin{macro}{\@clsextension}
% \begin{macro}{\@pkgextension}
%    The two possible values of |\@currext|.
%    \begin{macrocode}
\def\@clsextension{cls}
\def\@pkgextension{sty}
\@onlypreamble\@clsextension
\@onlypreamble\@pkgextension
%    \end{macrocode}
% \end{macro}
% \end{macro}
%
% \begin{macro}{\@pushfilename}
% \begin{macro}{\@popfilename}
% \begin{macro}{\@currnamestack}
% Commands to push and pop the file name and extension. \\
% |#1| current name.      \\
% |#2| current extension. \\
% |#3| current catcode of |@|. \\
% |#4| Rest of the stack.
%    \begin{macrocode}
\def\@pushfilename{%
  \xdef\@currnamestack{%
    {\@currname}%
    {\@currext}%
    {\the\catcode`\@}%
    \@currnamestack}}
\@onlypreamble\@pushfilename
%    \end{macrocode}
%
%    \begin{macrocode}
\def\@popfilename{\expandafter\@p@pfilename\@currnamestack\@nil}
\@onlypreamble\@popfilename
%    \end{macrocode}
%
%    \begin{macrocode}
\def\@p@pfilename#1#2#3#4\@nil{%
  \gdef\@currname{#1}%
  \gdef\@currext{#2}%
  \catcode`\@#3\relax
  \gdef\@currnamestack{#4}}
\@onlypreamble\@p@pfilename
%    \end{macrocode}
%
%    \begin{macrocode}
\gdef\@currnamestack{}
\@onlypreamble\@currnamestack
%    \end{macrocode}
% \end{macro}
% \end{macro}
% \end{macro}
%
% \begin{macro}{\@ptionlist}
%    Returns the option list of the file.
%    \begin{macrocode}
\def\@ptionlist#1{%
  \@ifundefined{opt@#1}\@empty{\csname opt@#1\endcsname}}
\@onlypreamble\@ptionlist
%    \end{macrocode}
% \end{macro}
%
% \begin{macro}{\@ifpackageloaded}
% \begin{macro}{\@ifclassloaded}
%   |\@ifpackageloaded{|\meta{name}|}|
%  Checks to see whether a file has been loaded.
% \changes{v0.2t}{1994/01/18}
%         {Fix typo \cs{@pkgetension}}
%    \begin{macrocode}
\def\@ifpackageloaded{\@ifl@aded\@pkgextension}
\def\@ifclassloaded{\@ifl@aded\@clsextension}
\@onlypreamble\@ifpackageloaded
\@onlypreamble\@ifclassloaded
%    \end{macrocode}
%
%    \begin{macrocode}
\def\@ifl@aded#1#2{%
  \expandafter\ifx\csname ver@#2.#1\endcsname\relax
    \expandafter\@secondoftwo
  \else
    \expandafter\@firstoftwo
  \fi}
\@onlypreamble\@ifl@aded
%    \end{macrocode}
% \end{macro}
% \end{macro}
%
% \begin{macro}{\@ifpackagelater}
% \begin{macro}{\@ifclasslater}
% |\@ifpackagelater{|\meta{name}|}{YYYY/MM/DD}|
% Checks that the package loaded is more recent than the given date.
%    \begin{macrocode}
\def\@ifpackagelater{\@ifl@ter\@pkgextension}
\def\@ifclasslater{\@ifl@ter\@clsextension}
\@onlypreamble\@ifpackagelater
\@onlypreamble\@ifclasslater
%    \end{macrocode}
%
%    \begin{macrocode}
\def\@ifl@ter#1#2{%
  \expandafter\@ifl@t@r
    \csname ver@#2.#1\endcsname}
\@onlypreamble\@ifl@ter
%    \end{macrocode}
%
% This internal macro is also used in |\NeedsTeXFormat|.
% \changes{v0.2f}{1993/11/22}
%         {Added //00 so parsing never produces a runaway argument.}
%    \begin{macrocode}
\def\@ifl@t@r#1#2{%
  \ifnum\expandafter\@parse@version#1//00\@nil<%
        \expandafter\@parse@version#2//00\@nil
    \expandafter\@secondoftwo
  \else
    \expandafter\@firstoftwo
  \fi}
\@onlypreamble\@ifl@t@r
%    \end{macrocode}
%
% \changes{v1.1j}{2016/06/20}
%         {don't declare as \cs{@onlypreamble}}
%    \begin{macrocode}
\def\@parse@version#1/#2/#3#4#5\@nil{#1#2#3#4 }
%    \end{macrocode}
% \end{macro}
% \end{macro}
%
% \begin{macro}{\@ifpackagewith}
% \begin{macro}{\@ifclasswith}
% |\@ifpackagewith{|\meta{name}|}{|\meta{option-list}|}|
% Checks that \meta{option-list} is a subset of the options
% \textbf{with} which \meta{name} was loaded.
%    \begin{macrocode}
\def\@ifpackagewith{\@if@ptions\@pkgextension}
\def\@ifclasswith{\@if@ptions\@clsextension}
\@onlypreamble\@ifpackagewith
\@onlypreamble\@ifclasswith
%    \end{macrocode}
%
%    \begin{macrocode}
\def\@if@ptions#1#2{%
  \@expandtwoargs\@if@pti@ns{\@ptionlist{#2.#1}}}
\@onlypreamble\@if@ptions
%    \end{macrocode}
%
% Probably shouldn't use |\CurrentOption| here\ldots (changed to
% |\reserved@b|.)
% \changes{v0.2y}{1994/02/07}
%         {Add extra ,s so `two' is not matched with `twocolumn'}
% \changes{v1.1i}{2011/08/19}
%         {Re-jig definition after more stringent \cs{in@} test.}
%    \begin{macrocode}
%</2ekernel>
%<latexrelease>\IncludeInRelease{2016/12/01}%
%<latexrelease>                 {\@if@pti@ns}{Spaces in option clash check}%
%<*2ekernel|latexrelease>
\def\@if@pti@ns#1#2{%
 \let\reserved@a\@firstoftwo
%    \end{macrocode}
% \changes{v1.2a}{2016/10/02}
%         {Ignore spaces while checking for option clash}
%    \begin{macrocode}
 \edef\reserved@b{\zap@space#2 \@empty}%
 \@for\reserved@b:=\reserved@b\do{%
   \ifx\reserved@b\@empty
   \else
     \expandafter\in@\expandafter{\expandafter,\reserved@b,}{,#1,}%
     \ifin@
     \else
       \let\reserved@a\@secondoftwo
     \fi
   \fi
 }%
 \reserved@a}
%</2ekernel|latexrelease>
%<latexrelease>\EndIncludeInRelease
%<latexrelease>\IncludeInRelease{0000/00/00}%
%<latexrelease>                 {\@if@pti@ns}{Spaces in option clash check}%
%<latexrelease>\def\@if@pti@ns#1#2{%
%<latexrelease> \let\reserved@a\@firstoftwo
%<latexrelease> \@for\reserved@b:=#2\do{%
%<latexrelease>  \ifx\reserved@b\@empty
%<latexrelease>   \else
%<latexrelease>   \expandafter\in@\expandafter{\expandafter,\reserved@b,}{,#1,}%
%<latexrelease>    \ifin@
%<latexrelease>    \else
%<latexrelease>     \let\reserved@a\@secondoftwo
%<latexrelease>    \fi
%<latexrelease>  \fi
%<latexrelease> }%
%<latexrelease> \reserved@a}
%<*2ekernel>
%    \end{macrocode}
%
%    \begin{macrocode}
\@onlypreamble\@if@pti@ns
%    \end{macrocode}
% \end{macro}
% \end{macro}
%
% \begin{macro}{\ProvidesPackage}
%    Checks that the current filename is correct, and defines
%    |\ver@filename|.
% \changes{v0.3c}{1994/03/12}
%         {Add \cs{wlog}}
% \changes{v0.3c}{1994/03/12}
%         {use \cs{@gtempa}}
%    \begin{macrocode}
\def\ProvidesPackage#1{%
  \xdef\@gtempa{#1}%
  \ifx\@gtempa\@currname\else
    \@latex@warning@no@line{You have requested
      \@cls@pkg\space`\@currname',\MessageBreak
       but the \@cls@pkg\space provides `#1'}%
  \fi
  \@ifnextchar[\@pr@videpackage{\@pr@videpackage[]}}%]
\@onlypreamble\ProvidesPackage
%    \end{macrocode}
%
%    \begin{macrocode}
\def\@pr@videpackage[#1]{%
  \expandafter\xdef\csname ver@\@currname.\@currext\endcsname{#1}%
  \ifx\@currext\@clsextension
    \typeout{Document Class: \@gtempa\space#1}%
  \else
    \wlog{Package: \@gtempa\space#1}%
  \fi}
\@onlypreamble\@pr@videpackage
%    \end{macrocode}
% \end{macro}
%
% \begin{macro}{\ProvidesClass}
%    Like |\ProvidesPackage|, but for classes.
%    \begin{macrocode}
\let\ProvidesClass\ProvidesPackage
\@onlypreamble\ProvidesClass
%    \end{macrocode}
% \end{macro}
%
% \begin{macro}{\ProvidesFile}
%    Like |\ProvidesPackage|, but for arbitrary files. Do not apply
%    |\@onlypreamble| to these, as we may want to label files input
%    during the document.
% \changes{v0.2l}{1993/12/07}
%         {Macro added}
% \changes{v0.3c}{1994/03/12}
%         {Add \cs{wlog}}
% \changes{v0.3g}{1994/04/11}
%         {Protect against weird catcodes.}
% \begin{macro}{\@providesfile}
% \changes{v1.0r}{1995/10/17}
%         {Delay definition of \cs{ProvidesFile} till ltfinal}
% \changes{v1.1a}{1998/03/21}
%         {Allow \&. Internal/2702}
% \changes{v1.1d}{2001/05/25}{Explicitly set catcode of
%                              \cs{endlinechar} to 10 (pr/3334)}
% \changes{v1.1e}{2001/06/04}{But only if it is a char (pr/3334)}
% \changes{v1.1f}{2001/08/26}{Readded setting of space char (pr/3353)}
%    \begin{macrocode}
\def\ProvidesFile#1{%
  \begingroup
    \catcode`\ 10 %
    \ifnum \endlinechar<256 %
      \ifnum \endlinechar>\m@ne
        \catcode\endlinechar 10 %
      \fi
    \fi
    \@makeother\/%
    \@makeother\&%
%    \end{macrocode}
% \changes{v1.1g}{2004/01/28}{Use kernel version of
%                             \cs{@ifnextchar} (pr/3501)}
%    \begin{macrocode}
    \kernel@ifnextchar[{\@providesfile{#1}}{\@providesfile{#1}[]}}
%    \end{macrocode}
%
% During initex a special version of |\@providesfile| is used.
% The real definition is installed right at the end, in |ltfinal.dtx|.
%\begin{verbatim}
%\def\@providesfile#1[#2]{%
%    \wlog{File: #1 #2}%
%    \expandafter\xdef\csname ver@#1\endcsname{#2}%
%  \endgroup}
%    \end{macrocode}
%\end{verbatim}
% \end{macro}
% \end{macro}
%
% \begin{macro}{\PassOptionsToPackage}
% \begin{macro}{\PassOptionsToClass}
% If the package has been loaded, we check that it was first loaded with
% the options.  Otherwise we add the option list to that of the package.
%    \begin{macrocode}
\def\@pass@ptions#1#2#3{%
  \expandafter\xdef\csname opt@#3.#1\endcsname{%
    \@ifundefined{opt@#3.#1}\@empty
      {\csname opt@#3.#1\endcsname,}%
    \zap@space#2 \@empty}}
\@onlypreamble\@pass@ptions
%    \end{macrocode}
%
%    \begin{macrocode}
\def\PassOptionsToPackage{\@pass@ptions\@pkgextension}
\def\PassOptionsToClass{\@pass@ptions\@clsextension}
\@onlypreamble\PassOptionsToPackage
\@onlypreamble\PassOptionsToClass
%    \end{macrocode}
% \end{macro}
% \end{macro}
%
% \begin{macro}{\DeclareOption}
% \begin{macro}{\DeclareOption*}
%    Adds an option as a |\ds@| command, or the default |\default@ds|
%    command.
% \changes{v0.2c}{1993/11/17}
%         {Error checking added}
% \changes{v1.0m}{1995/04/21}
%         {Made long /1498}
% \changes{v1.0n}{1995/05/12}
%         {Use \cs{toks@} to remove need to double hash /1557}
%    \begin{macrocode}
\def\DeclareOption{%
  \let\@fileswith@pti@ns\@badrequireerror
  \@ifstar\@defdefault@ds\@declareoption}
\long\def\@declareoption#1#2{%
   \xdef\@declaredoptions{\@declaredoptions,#1}%
   \toks@{#2}%
   \expandafter\edef\csname ds@#1\endcsname{\the\toks@}}
\long\def\@defdefault@ds#1{%
  \toks@{#1}%
  \edef\default@ds{\the\toks@}}
\@onlypreamble\DeclareOption
\@onlypreamble\@declareoption
\@onlypreamble\@defdefault@ds
%    \end{macrocode}
% \end{macro}
% \end{macro}
%
% \begin{macro}{\OptionNotUsed}
% If we are in a class file, add |\CurrentOption| to the list of
% unused options. Otherwise, in a package file do nothing.
%    \begin{macrocode}
\def\OptionNotUsed{%
  \ifx\@currext\@clsextension
    \xdef\@unusedoptionlist{%
      \ifx\@unusedoptionlist\@empty\else\@unusedoptionlist,\fi
      \CurrentOption}%
  \fi}
\@onlypreamble\OptionNotUsed
%    \end{macrocode}
% \end{macro}
%
% \begin{macro}{\default@ds}
% The default default option code.
% Set by |\@onefilewithoptions| to either |\OptionNotUsed| for
% classes, or |\@unknownoptionerror| for packages. This may be reset
% in either case with |\DeclareOption*|.
%    \begin{macrocode}
% \let\default@ds\OptionNotUsed
%    \end{macrocode}
% \end{macro}
%
% \begin{macro}{\ProcessOptions}
% \begin{macro}{\ProcessOptions*}
% |\ProcessOptions| calls |\ds@option| for each known package option,
% then calls |\default@ds| for each option on the local options list.
% Finally resets all the declared options to |\relax|. The empty option
% does nothing, this has to be reset on the off chance it's set to
% |\relax| if an empty element gets into the |\@declaredoptions| list.
%
% The star form is similar but executes options given in the order
% specified in the document, not the order they are declared in the
% file. In the case of packages, global options are executed before
% local ones.
% \changes{v0.2a}{1993/11/14}
%         {Stop adding the global option list inside class files.}
% \changes{v0.2a}{1993/11/14}
%         {Optimise `empty option' code.}
% \changes{v0.2b}{1993/11/15}
%         {Star form added.}
% \changes{v0.2c}{1993/11/17}
%         {restoring \cs{@fileswith@pti@ns} added.}
%    \begin{macrocode}
\def\ProcessOptions{%
  \let\ds@\@empty
  \edef\@curroptions{\@ptionlist{\@currname.\@currext}}%
  \@ifstar\@xprocess@ptions\@process@ptions}
\@onlypreamble\ProcessOptions
%    \end{macrocode}
%
% \changes{v0.2y}{1994/02/07}
%         {Add extra ,s so `two' is not matched with `twocolumn'}
%    \begin{macrocode}
\def\@process@ptions{%
  \@for\CurrentOption:=\@declaredoptions\do{%
    \ifx\CurrentOption\@empty\else
      \@expandtwoargs\in@{,\CurrentOption,}{%
         ,\ifx\@currext\@clsextension\else\@classoptionslist,\fi
         \@curroptions,}%
      \ifin@
        \@use@ption
        \expandafter\let\csname ds@\CurrentOption\endcsname\@empty
      \fi
    \fi}%
  \@process@pti@ns}
\@onlypreamble\@process@ptions
%    \end{macrocode}
%
% \changes{v0.2y}{1994/02/07}
%         {Add extra ,s so `two' is not matched with `twocolumn'}
%    \begin{macrocode}
\def\@xprocess@ptions{%
  \ifx\@currext\@clsextension\else
    \@for\CurrentOption:=\@classoptionslist\do{%
      \ifx\CurrentOption\@empty\else
        \@expandtwoargs\in@{,\CurrentOption,}{,\@declaredoptions,}%
        \ifin@
          \@use@ption
          \expandafter\let\csname ds@\CurrentOption\endcsname\@empty
        \fi
      \fi}%
  \fi
  \@process@pti@ns}
\@onlypreamble\@xprocess@ptions
%    \end{macrocode}
%
% The common part of |\ProcessOptions| and |\ProcessOptions*|.
%    \begin{macrocode}
\def\@process@pti@ns{%
  \@for\CurrentOption:=\@curroptions\do{%
    \@ifundefined{ds@\CurrentOption}%
      {\@use@ption
       \default@ds}%
%    \end{macrocode}
% There should not be any non-empty definition of |\CurrentOption| at
% this point, as all the declared options were executed earlier. This is
% for compatibility with 2.09 styles which use |\def\ds@|\ldots\
% directly, and so have options which do not appear in
% |\@declaredoptions|.
%    \begin{macrocode}
      \@use@ption}%
%    \end{macrocode}
% Clear all the definitions for option code. First set all the declared
% options to |\relax|, then reset the `default' and `empty' options. and
% the lst of declared options.
%    \begin{macrocode}
  \@for\CurrentOption:=\@declaredoptions\do{%
    \expandafter\let\csname ds@\CurrentOption\endcsname\relax}%
%    \end{macrocode}
% \changes{v1.0r}{1995/10/17}
%         {Reset \cs{CurrentOption} for graphics/1873}
%    \begin{macrocode}
  \let\CurrentOption\@empty
  \let\@fileswith@pti@ns\@@fileswith@pti@ns
  \AtEndOfPackage{\let\@unprocessedoptions\relax}}
\@onlypreamble\@process@pti@ns
%    \end{macrocode}
% \end{macro}
% \end{macro}
%
% \begin{macro}{\@options}
% |\@options| is a synonym for |\ProcessOptions*| for upward
% compatibility with \LaTeX2.09 style files.
%    \begin{macrocode}
\def\@options{\ProcessOptions*}
\@onlypreamble\@options
%    \end{macrocode}
% \end{macro}
%
% \begin{macro}{\@use@ption}
% Execute the code for the current option.
% \changes{v0.2g}{1993/11/23}
%         {Name changed from \cs{@executeoption}}
% \changes{v1.0e}{1994/05/17}
%         {Execute option after removing from list, not before}
%    \begin{macrocode}
\def\@use@ption{%
  \@expandtwoargs\@removeelement\CurrentOption
  \@unusedoptionlist\@unusedoptionlist
  \csname ds@\CurrentOption\endcsname}
\@onlypreamble\@use@ption
%    \end{macrocode}
% \end{macro}
%
% \begin{macro}{\ExecuteOptions}
% |\ExecuteOptions{|\meta{option-list}|}| executes the code declared
% for each option.
% \changes{v0.2d}{1993/11/18}
%         {Use \cs{CurrentOption} not \cs{reserved@a}}
% \changes{v0.2k}{1993/12/06}
%         {Preserve \cs{CurrentOption}.}
%    \begin{macrocode}
%</2ekernel>
%<latexrelease>\IncludeInRelease{2016/12/01}%
%<latexrelease>                 {\@if@pti@ns}{Spaces in \ExecuteOptions}%
%<*2ekernel|latexrelease>
\def\ExecuteOptions#1{%
%    \end{macrocode}
% \changes{v1.2a}{2016/10/02}
%         {Ignore spaces in argument}
% Use |\@fortmp| here as it is anyway cleared during |\@for| loop
% so does not change any existing names.
%    \begin{macrocode}
  \edef\@fortmp{\zap@space#1 \@empty}%
  \def\reserved@a##1\@nil{%
    \@for\CurrentOption:=\@fortmp\do{\csname ds@\CurrentOption\endcsname}%
    \edef\CurrentOption{##1}}%
  \expandafter\reserved@a\CurrentOption\@nil}
%</2ekernel|latexrelease>
%<latexrelease>\EndIncludeInRelease
%<latexrelease>\IncludeInRelease{0000/00/00}%
%<latexrelease>                 {\@if@pti@ns}Spaces in \ExecuteOptions}%
%<latexrelease>\def\ExecuteOptions#1{%
%<latexrelease>  \def\reserved@a##1\@nil{%
%<latexrelease>    \@for\CurrentOption:=#1\do{\csname ds@\CurrentOption\endcsname}%
%<latexrelease>    \edef\CurrentOption{##1}}%
%<latexrelease>  \expandafter\reserved@a\CurrentOption\@nil}
%<*2ekernel>
%    \end{macrocode}
%
%    \begin{macrocode}
\@onlypreamble\ExecuteOptions
%    \end{macrocode}
% \end{macro}
%
% The top-level commands, which just set some parameters then call
% the internal command, |\@fileswithoptions|.
% \begin{macro}{\documentclass}
% \changes{v1.0q}{1995/06/19}
%         {Dont redefine \cs{usepackage} in compat mode for /1634}
% The main new-style class declaration.
%    \begin{macrocode}
\def\documentclass{%
  \let\documentclass\@twoclasseserror
  \if@compatibility\else\let\usepackage\RequirePackage\fi
  \@fileswithoptions\@clsextension}
\@onlypreamble\documentclass
%    \end{macrocode}
% \end{macro}
%
% \begin{macro}{\documentstyle}
% 2.09 style class `style' declaration.
% \changes{v0.2a}{1993/11/14}
%         {Added \cs{RequirePackage} \cs{@unusedoptionlist} stuff.}
% \changes{v0.2b}{1993/11/15}
%         {Modified to match \cs{ProcessOption*}}
% \changes{v0.2d}{1993/11/18}
%         {Modified \cs{RequirePackage} stuff.}
% \changes{v0.2n}{1993/12/09}
%         {input 209 compatibility file.}
% \changes{v0.2o}{1993/12/13}
%         {compatibility file now latex209.sty.}
% \changes{v0.2q}{1993/12/17}
%         {Match Alan's new code.}
% \changes{v0.2u}{1994/01/21}
%         {compatibility file now latex209.def.}
%    \begin{macrocode}
\def\documentstyle{%
  \makeatletter% \iffalse meta-comment
%
% Copyright 1993-2016
% The LaTeX3 Project and any individual authors listed elsewhere
% in this file.
%
% This file is part of the LaTeX base system.
% -------------------------------------------
%
% It may be distributed and/or modified under the
% conditions of the LaTeX Project Public License, either version 1.3c
% of this license or (at your option) any later version.
% The latest version of this license is in
%    http://www.latex-project.org/lppl.txt
% and version 1.3c or later is part of all distributions of LaTeX
% version 2005/12/01 or later.
%
% This file has the LPPL maintenance status "maintained".
%
% The list of all files belonging to the LaTeX base distribution is
% given in the file `manifest.txt'. See also `legal.txt' for additional
% information.
%
% The list of derived (unpacked) files belonging to the distribution
% and covered by LPPL is defined by the unpacking scripts (with
% extension .ins) which are part of the distribution.
%
% \fi
%
% \title{Compatibility mode for \LaTeXe{} emulating \LaTeX~2.09}
% \author{Alan Jeffrey and Frank Mittelbach}
% \date{1995/12/27}
%
% \CheckSum{717}
%
%
% \changes{v0.01}{1993/12/11}{Created the file, including:
%   setting the compatibility flag,
%   inputting oldlfont.sty,
%   setting the default encoding to be OT1, and
%   inputting the latex209.rc file}
% \changes{v0.02}{1993/12/12}{Changed the package filename to
%   latex209.sty, and added the provides-package command.}
% \changes{v0.03}{1993/12/16}{Added an empty mark, replaced
%   provides-package with provides-file, added the compatibility hook.}
% \changes{v0.04}{1993/12/16}{Moved oldlfont.sty out of the
%    compatibility hook and back into latex209.cmp.  Redefined
%    newfontswitch to ignore redefinitions.  Set the LaTeX 2e commands
%    to be errors.}
% \changes{v0.05}{1993/12/17}
%    {Removed the \cs{mark}, since it is now in the kernel.}
% \changes{v0.06}{1993/12/18}{Replaced the redefinition of
%    \cs{@newfontswitch} to a redefinition of \cs{@renewfontswitch}.
%    Added \cs{sloppy.}}
% \changes{v0.07}{1993/12/18}{Fixed a bug with \cs{@missingfileerror}.}
% \changes{v0.08}{1993/12/18}{Added the obsolete .sty files.}
% \changes{v0.09}{1993/12/20}{Removed art10.sty and friends.}
% \changes{v0.10}{1994/01/14}{Replaced latex209.rc by latex209.cfg.}
% \changes{v0.11}{1994/01/21}{Replaced latex209.cmp by latex209.def.
%    Moved half of oldlfont.dtx to here.  Split the package dst option
%    into head and tail.}
% \changes{v0.12}{1994/01/24}
%   {Added \cs{normalshape} and \cs{mediumseries}, and
%    declared the `oldlfont' option to stop oldlfont.sty from being
%    loaded.}
% \changes{v0.13}{1994/01/31}{removed setting of \cs{normalsize}. FMi}
% \changes{v0.14}{1994/02/07}{Added it back again.}
% \changes{v0.15}{1994/02/10}{Renamed \cs{@compatibility} to
%    \cs{@documentclasshook}.  Added the check for whether
%     \cs{normalsize} or \cs{@normalsize} needs defined.}
% \changes{v0.16}{1994/02/11}{Replaced the allocation of temporary
%    dimens for \cs{footheight}, \cs{@maxsep} and \cs{@dblmaxsep}
%     by real dimen variables.}
% \changes{v0.17}{1994/03/02}{Moved the documentation to the front, so
%     this file can be processed directly without a driver file.
%     Added \cs{@ptscale}, \cs{@halfmag}, \cs{@magscale}, and set the
%     default font to be CMR at 10pt.}
% \changes{v0.18}{1994/03/11}{Restored the old definition of \cs{verb}.
%     Set the catcodes of the non-alphanumerics.}
% \changes{v0.19}{1994/04/05}{Switched off more 2e features: \cs{lrbox},
%     \cs{width}, \cs{height}, \cs{depth} in box dimensions, and the new
%     optional arguments to \cs{parbox}, \cs{minipage} and
%     \cs{newcommand}.  The code was provided by DPC.  Fixed a misplaced
%     </head!>.  Made the \cs{ProvidesPackage} and \cs{ProvidesClass}
%     warnings log messages. Removed \cs{filedate}.}
% \changes{v0.20}{1994/04/20}
%      {Restored the 2.09 definition of \cs{@noligs}.}
% \changes{v0.21}{1994/04/24}
%      {Restored the 2.09 definition of \cs{@lquote}.}
% \changes{v0.22}{1994/05/02}{Added \cs{@latex@e@command}.}
% \changes{v0.23}{1994/05/11}{Added bezier.sty.}
% \changes{v0.24}{1994/05/14}{Removed \cs{@@@} and switched the box
%    commands back on, for use in packages.}
% \changes{v0.24}{1994/05/14}{Changed the 2e command error help.}
% \changes{v0.24}{1994/05/14}{Removed date from announcement of
%    2.09 mode.}
% \changes{v0.25}{1994/05/14}{Added the newlfont option, and rewrote the
%    oldlfont option.}
% \changes{v0.26}{1994/05/15}{Added the margid and nomargid options.}
% \changes{v0.27}{1994/05/16}{Fixed a bug with the margid option.}
% \changes{v0.28}{1994/05/16}{Fixed a bug with \cs{mediumseries}.}
% \changes{v0.29}{1994/05/17}{Fixed a bug with \cs{newlfont}.}
% \changes{v0.29}{1994/05/17}{Removed extra spaces from the missing
%    file error.}
% \changes{v0.29}{1994/05/17}{Made the bezier package use \cs{iffalse}
%    to comment itself out, rather than \%\%, which caused it to
%    appear in every 2.09 file.}
% \changes{v0.30}{1994/05/18}{Added \cs{@finalstrut}.}
% \changes{v0.31}{1994/05/20}{New definition of \cs{@finalstrut}.}
% \changes{v0.31}{1994/05/20}{Added the t1enc package.}
% \changes{v0.32}{1994/05/20}{Added SLiTeX.}
% \changes{v0.33}{1994/06/01}{Fixed bug with SLiTeX.}
% \changes{v0.34}{1994/08/22}{Replaced l2euser by usrguide.}
% \changes{v0.34}{1994/08/22}{Added a default definition for \cs{+}.}
% \changes{v0.35}{1994/09/23}
%                {Added spaces to the old font scale commands.}
% \changes{v0.36}{1994/10/17}{Added an empty \cs{mark} back again.}
% \changes{v0.37}{1994/10/20}{Corrected a typo.}
% \changes{v0.38}{1994/11/16}{Removed \cs{LaTeXe} from this list}
% \changes{v0.39}{1994/11/28}{Added old behaviour of floats and space.}
% \changes{v0.40}{1995/05/05}{Make \cs{verb} use \cs{tt} font in
%                             math mode.}
% \changes{v0.49}{1995/10/26}{Added code for fleqn.sty, leqno.sty,
%                             openbib.sty.}
% \changes{v0.50}{1995/12/08}{Switched of \cs{@inmathwarn}.}
% \changes{v0.53}{2015/02/22}{Dropped \cs{@no@font@optfalse} in various places
%                              - no longer provided by ltfsscmp.dtx.}
%
% \MaintainedByLaTeXTeam{latex}
% \maketitle
%
% \section{Introduction}
%
% The file |latex209.def| is read in by \LaTeXe{} whenever it finds a
% |\documentstyle| rather than |\documentclass| command at the
% beginning of the file.  This indicates a \LaTeX~2.09 document, which
% should be processed in {\em compatibility mode}.
%
% Any document which compiled under \LaTeX~2.09 should compile under
% compatibility mode, unless it uses low-level commands such as
% |\tenrm|.
%
% \section{The docstrip modules}
%
% The following modules are used in the implementation to direct
% docstrip in generating the external files:
% \begin{center}
% \begin{tabular}{ll}
%   driver & produce a documentation driver file \\
%   head    & produce the beginning of |latex209.def| \\
%   tail    & produce the end of |latex209.def| \\
%   article  & produce |article.sty| \\
%   book     & produce |book.sty| \\
%   report   & produce |report.sty| \\
%   slides   & produce |slides.sty| \\
%   letter   & produce |letter.sty| \\
%   bezier   & produce |bezier.sty| \\
%   fleqn    & produce |fleqn.sty| \\
%   leqno    & produce |leqno.sty| \\
%   openbib  & produce |openbib.sty|
% \end{tabular}
% \end{center}
% Between the |head| and |tail| of |latex209.def|, the code for
% |oldlfont.sty| is included, so \LaTeX~2.09 documents will
% automatically be run simulating the OFSS.
% \changes{v0.09}{1993/12/20}{Removed artN.sty, bkN.sty and repN.sty.}
% \changes{v0.11}{1994/01/21}{Split package into head and tail.}
% \changes{v0.23}{1994/05/11}{Added bezier option.}
%
% \StopEventually{}
%
% \section{Driver}
%
% This section contains the driver for this documentation.
%    \begin{macrocode}
%<*driver>
\documentclass{ltxdoc}
\DisableCrossrefs
% \OnlyDescription
\begin{document}
   \DocInput{latex209.dtx}
\end{document}
%</driver>
%    \end{macrocode}
%
% \section{Beginning of latex209.def}
%
% \changes{v0.11}{1994/01/21}{oldlfont.dtx is now also used to generate
%    latex209.dtx.}
%
% This section describes the beginning of the file |latex209.def|.
%    \begin{macrocode}
%<*head>
%    \end{macrocode}
%
% \subsection{Identification}
%
% This file needs to be run with \LaTeXe.
%    \begin{macrocode}
\NeedsTeXFormat{LaTeX2e}
%    \end{macrocode}
% Describe the file.
%    \begin{macrocode}
\ProvidesFile{latex209.def}[2015/02/22 v0.53 Standard LaTeX file]
%    \end{macrocode}
% \changes{v0.24}{1994/05/14}{Removed date.}
% \changes{v0.40}{1995/03/21}
%         {(DPC) Do not execute this file twice latex/1460}
% Announce compatibility mode to the user.
% \changes{v0.52}{1998/05/13}{Added experimental
%    long typeout to possibly avoid prs like 2807}
%    \begin{macrocode}
\if@compatibility
  \expandafter\endinput
\else
  \typeout{^^J\space
\@spaces\@spaces\space  Entering LaTeX 2.09 COMPATIBILITY MODE^^J\space
 *************************************************************^^J\space
 \space\space\space!!WARNING!!\space
 \space\space\space!!WARNING!!\space
 \space\space\space!!WARNING!!\space
 \space\space\space!!WARNING!!\space\space\space   ^^J\space
 ^^J\space
 This mode attempts to provide an emulation of the LaTeX 2.09^^J\space
 author environment so that OLD documents can be successfully^^J\space
 processed. It should NOT be used for NEW documents!^^J\space
 ^^J\space
 New documents should use Standard LaTeX conventions and start^^J\space
 with the \string\documentclass\space command.^^J\space
 ^^J\space
 Compatibility mode is UNLIKELY TO WORK with LaTeX 2.09 style^^J\space
 files that change any internal macros, especially not with^^J\space
 those that change the FONT SELECTION or OUTPUT ROUTINES.^^J\space
^^J\space
 Therefore such style files   MUST BE UPDATED to use^^J\space
\@spaces\@spaces\space        Current Standard LaTeX: LaTeX2e.^^J\space
 If you suspect that you may be using such a style file, which^^J\space
 is probably very, very old by now, then you should attempt to^^J\space
 get it updated by sending a copy of this error message to the^^J\space
 author of that file.^^J\space
 *************************************************************^^J}
 \fi
%    \end{macrocode}
%
% \subsection{Compatibility flag}
%
% \begin{macro}{\@compatibilitytrue}
%    \LaTeXe{} has a flag |\if@compatibility| which can be used by
%    document classes or packages to determine whether they are running
%    in compatibility mode or not.  This flag is set true by this file.
%    \begin{macrocode}
\@compatibilitytrue
%    \end{macrocode}
% \end{macro}
%
% \subsection{Removing features}
%
% \changes{v0.22}{1994/05/02}{Added \cs{latex@e@command}.}
% \changes{v0.36}{1994/10/17}{Redid switching off commands.}
% \changes{v0.38}{1994/11/16}{Removed \cs{LaTeXe} from this list}
%
% \begin{macro}{\usepackage}
% \begin{macro}{\listfiles}
% \begin{macro}{\ensuremath}
% \begin{macro}{\lrbox}
% \begin{macro}{\newcommand}
%    These \LaTeXe{} commands are switched off in compatibility mode.
%    This is done by saving the old definition, and redefining the
%    command to call |\@latex@e@error| before executing the old version.
%    \begin{macrocode}
\def\@tempa#1#2{%
   \expandafter\let\csname @@\string#1\endcsname#1%
   \edef#1{%
      \noexpand\@latex@e@error{\noexpand#2}%
      \expandafter\noexpand\csname @@\string#1\endcsname
   }%
}
\@tempa\usepackage\usepackage
\@tempa\listfiles\listfiles
\@tempa\ensuremath\ensuremath
\@tempa\lrbox{\begin{lrbox}}%
\@tempa\@xargdef{\newcommand{cmd}[args][def]}%
%    \end{macrocode}
% \end{macro}
% \end{macro}
% \end{macro}
% \end{macro}
% \end{macro}
%
% \changes{v0.22}{1994/05/02}
%         {Added \cs{if@latex@e@errors} and \cs{@@@}.}
% \changes{v0.24}
%         {1994/05/14}{Removed \cs{if@latex@e@errors} and \cs{@@@}.}
% \changes{v0.24}{1994/05/14}{Changed the error help.}
% \changes{v0.36}{1994/10/17}{Initialized \cs{@latex@e@error} to do
%    nothing, and switched it on at the begin document.}
%
% \begin{macro}{\@latex@e@error}
% \begin{macro}{\@latex@e@error@}
%   This error is produced if a user uses a \LaTeXe{} command in
%   compatibility mode.  This is to encourage users to move over to
%   using |\documentclass| as quickly as possible.  During the preamble
%   the error does nothing (so that packages can use \LaTeXe{} commands)
%   but it is redefined to be an error message at |\begin{document}|.
%    \begin{macrocode}
\let\@latex@e@error\@gobble
\def\@latex@e@error@#1{%
      \@latexerr{%
         LaTeX2e command \string#1\space in LaTeX 2.09 document%
      }{%
         This is a LaTeX 2.09 document, but it contains
         \string#1.^^J%
         If you want to use the new features of LaTeX2e,
         your document^^J%
         should begin with \string\documentclass\space
         rather than \string\documentstyle
      }%
}
%    \end{macrocode}
% \end{macro}
% \end{macro}
%
% \changes{v0.36}{1994/10/17}
%         {Allow 2e commands to be redefined once with \cs{newcommand}.}
% \changes{v0.40}{1995/03/21}{
%         (DPC) Add \cs{r} to the list of 2e commands latex/1424}
%
% \begin{macro}{\@ifdefinable}
% \begin{macro}{\@old@ifdefinable}
% \begin{macro}{\@@ifdefinable}
% \begin{macro}{\@latex@e@commands}
%    We trap the |\@notdefinable| error message to check to see if the
%    command is a \LaTeXe{} command, in which case we allow the
%    definition to happen.   We keep a list of commands which are
%    allowed to be redefined this way in |\@latex@e@commands|, and
%    remove an entry each time it is defined.
%    \begin{macrocode}
\let\@old@ifdefinable\@ifdefinable
\long\def\@ifdefinable#1{%
   \def\@tempa##1#1##2#1##3#1##4\@tempa{%
      \def\@latex@e@commands{##1##2}%
      ##3% ##3 will either be \iftrue or \iffalse
         \expandafter\@firstofone
      \else
         \expandafter\@old@ifdefinable\expandafter#1%
      \fi
   }%
   \expandafter\@tempa\@latex@e@commands#1\iftrue#1\iffalse#1\@tempa%
}
\let\@@ifdefinable\@ifdefinable
\def\@latex@e@commands{%
   \usepackage\listfiles\ensuremath\LaTeXe\lrbox
   \th\dh\ng\dj\TH\DH\NG\DJ\k\r\SS
   \guillemotleft\guillemotright\guilsinglleft
   \guilsinglright\quotedblbase\quotesinglbase
}
%    \end{macrocode}
% \end{macro}
% \end{macro}
% \end{macro}
% \end{macro}
%
% \begin{macro}{\@begin@tempboxa}
%    \changes{v0.22}{1994/05/02}{Commented out the redefinition of
%       \cs{@begin@tempboxa}.}
%    If we were to switch off the new |\width|, |\height| and |\depth|
%    commands, this is how to do it.  This isn't done, since these
%    commands may be used in packages.
%    \begin{verbatim}
%\long\def\@begin@tempboxa#1#2{%
%  \begingroup
%    \setbox\@tempboxa#1{{#2}}}
%    \end{verbatim}
% \end{macro}
%
% \subsection{Document class hook}
%
% \begin{macro}{\@documentclasshook}
% \changes{v0.15}{1994/02/10}{Renamed from \cs{@compatibility} to
%    \cs{@documentclasshook}.  Added the check for \cs{@normalsize} and
%    \cs{normalsize} being defined.}
% \changes{v0.22}{1994/05/02}{Moved switching off commands into the
%    document class hook.}
% \changes{v0.24}{1994/05/14}{Switched the box commands back on, for use
%    in packages.}
% \changes{v0.36}{1994/10/17}{Changed the way the 2e command error is
%    activated.}
% \changes{v0.51}{1996/05/24}{(DPC) Reimplemented for /2153.}
%    This macro is called by each use of |\documentclass|.  We define
%    it to define |\@normalsize| and |\normalsize| if necessary,
%    to input each unused option as a package, and to switch off the new
%    \LaTeXe{} commands.  However, we leave on the commands
%    |\settoheight|, |\settowidth| and the new options to |\parbox| and
%    |\minipage|, since these are likely to be used in packages.
%
% The intention of the strange |\normalsize| tests below are that after
% the |\documentstyle| command has completed, then
% if neither of the commands |\normalsize|
% nor |\@normalsize|  were defined by the main style or one of its
% `substyles' or `options', then |\@normalsize| will be undefined and
% |\normalsize| will generate an error saying it hasn't been defined.
%
% If the style defined either |\normalsize| or \@|normalsize| then
% these two commands will be |\let| equal to each other, with the
% definition given by the style file.
%
% If the style defines both |\normalsize| and |\@normalsize| then
% those two definitions are kept.
%    \begin{macrocode}
\def\@documentclasshook{%
  \RequirePackage\@unusedoptionlist
  \let\@unusedoptionlist\@empty
  \def\@tempa{\@normalsize}%
  \ifx\normalsize\@tempa
    \let\normalsize\@normalsize
  \fi
  \ifx\@normalsize\@undefined
    \let\@normalsize\normalsize
  \fi
  \ifx\normalsize\@undefined
    \let\normalsize\original@normalsize
  \fi
  \let\@latex@e@error\@latex@e@error@}
%    \end{macrocode}
% \end{macro}
%
% \begin{macro}{\original@normalsize}
% \changes{v0.51}{1996/05/24}{(DPC) Macro added /2153.}
% Save the original definition of |\normalsize| (which generates an
% error)
%    \begin{macrocode}
\let\original@normalsize\normalsize
%    \end{macrocode}
% \end{macro}
%
% \begin{macro}{\normalsize}
% \changes{v0.14}{1994/02/07}{Added \cs{normalsize}.}
%    Some styles don't define |\normalsize|, just |\@normalsize|.
%    \begin{macrocode}
\def\normalsize{\@normalsize}
%    \end{macrocode}
% \end{macro}
%
% \subsection{Compatibility with \LaTeX~2.09 document styles}
%
% \begin{macro}{\@missingfileerror}
%    If a |.cls| file is missing, we look to see if there is
%    a file of the same name with a |.sty| extension.
% \changes{v0.06}{1993/12/18}{Corrected a typo
%  \cs{@saved@missingfileerror}
%    should have been \cs{saved@missingfileerror}.}
% \changes{v0.07}{1993/12/18}{Corrected a typo, I'd forgotten to pass
%    the arguments of \cs{@missingfileerror} on to
%    \cs{saved@missingfileerror}.}
% \changes{v0.29}{1994/05/17}{Removed extraneous spaces.}
%    \begin{macrocode}
\@ifundefined{saved@missingfileerror}{
   \let\saved@missingfileerror=\@missingfileerror
}{}
\def\@missingfileerror#1#2{%
   \ifx#2\@clsextension
      \InputIfFileExists{#1.\@pkgextension}{%
         \wlog{Compatibility mode: loading #1.\@pkgextension
            \space rather than #1.#2.}%
      }{%
         \saved@missingfileerror{#1}{#2}%
      }%
   \else
      \saved@missingfileerror{#1}{#2}%
   \fi
}
%    \end{macrocode}
% \end{macro}
%
% \begin{macro}{\@obsoletefile}
%    For compatibility with the document styles which |\input| the
%    standard \LaTeX~2.09 document styles, we distribute
%    files called |article.sty|, |book.sty|, |report.sty|,
%    |slides.sty| and |letter.sty|.  These use the command
%    |\@obsoletefile|, which the \LaTeXe{} kernel defines to produce a
%    warning message.  We redefine it to just produce a message in the
%    log file, and to pass any options from the old filename to the
%    new filename.
%    \changes{v0.08}{1993/12/19}{Added this command.}
%    \changes{v0.10}{1994/01/14}{Added the option-passing.}
%    \begin{macrocode}
\def\@obsoletefile#1#2{%
   \expandafter\let\csname opt@#1\expandafter\endcsname
      \csname opt@\@currname.\@currext\endcsname
   \wlog{Compatibility mode: inputting `#1'
      instead of obsolete `#2'.}}
%    \end{macrocode}
% \end{macro}
%
% \begin{macro}{\footheight}
% \begin{macro}{\@maxsep}
% \begin{macro}{\@dblmaxsep}
%    \LaTeX~2.09 supported these parameters, so for compatibility with
%    old document styles we allocate them.
% \changes{v0.16}{1994/02/11}{Replaced the allocation of temporary
%    dimens for \cs{footheight}, \cs{@maxsep} and \cs{@dblmaxsep} by
%     real dimen variables.}
%    \begin{macrocode}
\newdimen\footheight
\newdimen\@maxsep
\newdimen\@dblmaxsep
%    \end{macrocode}
% \end{macro}
% \end{macro}
% \end{macro}
%
% \changes{v0.36}{1994/10/17}{Added an empty \cs{mark}.}
% \changes{v0.37}
%         {1994/10/20}{Corrected a type with the empty \cs{mark}.}
%
% \begin{macro}{\mark}
%    \LaTeX~2.09 initialized an empty mark.  Who knows, someone may have
%    relied on it:
%    \begin{macrocode}
\mark{{}{}}
%    \end{macrocode}
% \end{macro}
%
% \subsection{Layout}
%
% \begin{macro}{\sloppy}
% \changes{v0.06}{1993/12/18}{Added \cs{sloppy}}
%    There is a new version of |\sloppy| in \LaTeXe, so we restore the
%    old one.
%    \begin{macrocode}
\def\sloppy{\tolerance \@M \hfuzz .5\p@ \vfuzz .5\p@}
%    \end{macrocode}
% \end{macro}
%
% \begin{macro}{\@finalstrut}
% \changes{v0.30}{1994/05/18}{Added \cs{@finalstrut}.}
% \changes{v0.31}{1994/05/20}{New definition of \cs{@finalstrut}.}
%    The strut which is used in a footnote has changed.  This restores
%    the old definition.
%    \begin{macrocode}
\def\@finalstrut#1{\unskip\strut}
%    \end{macrocode}
% \end{macro}
%
% \begin{macro}{\@marginparreset}
% \begin{macro}{\@floatboxreset}
% \changes{v0.39}{1994/11/28}{Added old behaviour of floats and space.}
% \changes{v0.45}{1995/05/25}{(CAR) Changed method of restoring
%    old behaviour of floats and space.}
%    Restore the old spacing around floats.
%    \begin{macrocode}
\let \@marginparreset \@empty
\let \@floatboxreset \@empty
%    \end{macrocode}
% \end{macro}
% \end{macro}
%
% \begin{macro}{\proclaim}
% \changes{v0.41}{1995/04/24}
%         {Move here from ltplain.dtx}
% From plain \TeX.
%    \begin{macrocode}
\outer\def\proclaim #1. #2\par{%
  \medbreak
  \noindent{\bfseries#1.\enspace}{\slshape#2\par}%
  \ifdim\lastskip<\medskipamount
    \removelastskip\penalty55\medskip
  \fi}
%    \end{macrocode}
% \end{macro}
%
% \begin{macro}{\hang}
% \begin{macro}{\textindent}
% \changes{v0.42}{1995/04/27}
%         {Move here from ltplain.dtx}
% From plain \TeX.
%    \begin{macrocode}
\def\hang{\hangindent\parindent}
\def\textindent#1{\indent\llap{#1\enspace}\ignorespaces}
%    \end{macrocode}
% \end{macro}
% \end{macro}
% \begin{macro}{\ttraggedright}
% \changes{v0.41}{1995/04/24}
%         {Move here from ltplain.dtx}
%    \begin{macrocode}
\def\ttraggedright{\reset@font\ttfamily\rightskip\z@ plus2em\relax}
%    \end{macrocode}
% \end{macro}
%
%
% \begin{macro}{\@footnotemark}
% \changes{v0.43}{1995/05/12}
%         {macro added}
% \LaTeXe\ version has |\nobreak| to allow hyphenation.
%    \begin{macrocode}
\def\@footnotemark{%
  \leavevmode
  \ifhmode\edef\@x@sf{\the\spacefactor}\fi
  \@makefnmark
  \ifhmode\spacefactor\@x@sf\fi
  \relax}
%    \end{macrocode}
% \end{macro}
%
% \begin{macro}{\@textsuperscript}
% \changes{v0.48}{1995/07/07}
%         {macro added for latex/1722}
% Fudge this command to remove the text font command which
% is always the first thing in the argument. This is needed
% as in compatibility mode footnotes are processed in math mode,
% but the standard classes call |\@textsuperscript| in the definition
% of |\thanks|.
%    \begin{macrocode}
\def\@textsuperscript#1{$\m@th^{\@gobble#1}$}
%    \end{macrocode}
% \end{macro}
%
%
%
%  \begin{macro}{\@makefnmark}
% \changes{v0.44}{1995/05/20}{macro added}
%    \LaTeXe\ version uses |\textsuperscript| rather than
%    math mode.
%    \begin{macrocode}
\def\@makefnmark{\hbox{$^{\@thefnmark}\m@th$}}
%    \end{macrocode}
%  \end{macro}
%
%  \begin{macro}{\thempfootnote}
% \changes{v0.44}{1995/05/20}{macro added}
%    \LaTeXe\ version has an additional |\itshape| which would not
%    work (and would not make sense) in math mode.
%    \begin{macrocode}
\def\thempfootnote{\@alph\c@mpfootnote}
%    \end{macrocode}
%  \end{macro}
%
%  \begin{macro}{\@fnsymbol}
% \changes{v0.46}{1995/06/30}{macro added}
% \LaTeX\ version uses |\ensuremath| which does not work in
% compatibility mode.
%    \begin{macrocode}
\def\@fnsymbol#1{\ifcase#1\or *\or \dagger\or \ddagger\or
    \mathchar "278\or \mathchar "27B\or \|\or **\or \dagger\dagger
    \or \ddagger\ddagger \else\@ctrerr\fi}
%    \end{macrocode}
%  \end{macro}
%
% \begin{macro}{\@inmathwarn}
% \changes{v0.50}{1995/12/08}{Switched off \cs{@inmathwarn}}
% \LaTeX\ (1995/12/01) checks for text commands being used in math
% mode.  We switch this off in compatibility mode.
%    \begin{macrocode}
\let\@inmathwarn\@gobble
%    \end{macrocode}
% \end{macro}
%
% \subsection{Verbatim}
%
% \changes{v0.18}{1994/03/11}{Added the changes to \cs{verb}}
% \changes{v0.40}{1995/05/05}{Make \cs{verb} use \cs{tt} font in
%                             math mode.}
% \begin{macro}{\verb}
% \begin{macro}{\@sverb}
%    We restore the old definition of |\verb|, but using
%    |\verbatim@font| rather than |\tt|. The use of |\bgroup| and
%    |\egroup| allows us to prefix it with |\hbox| in math mode.
%    \begin{macrocode}
\def\verb{%
   \relax\ifmmode\hbox\fi\bgroup
      \@noligs
      \verbatim@font
      \let\do\@makeother \dospecials
      \@ifstar{\@sverb}{\@verb}%
}
\def\@sverb#1{%
   \def\@tempa ##1#1{\leavevmode\null##1\egroup}%
   \@tempa
}
%    \end{macrocode}
% \end{macro}
% \end{macro}
%
% \begin{macro}{\verbatim@nolig@list}
% \changes{v0.20}{1994/04/20}{Added the redefinition of
%    \cs{verbatim@nolig@list}.}
%    The only ligatures which should be switched off in 2.09 mode are
%    the Spanish punctuation.
%    \begin{macrocode}
\def\verbatim@nolig@list{\do\`}
%    \end{macrocode}
% \end{macro}
%
% \begin{macro}{\@lquote}
% \changes{v0.21}{1994/04/24}{Added the definition of \cs{@lquote}.}
%    We restore the old definition of |\@lquote| in case any packages
%    use it.
%    \begin{macrocode}
\def\@lquote{\leavevmode{\kern\z@}`}
%    \end{macrocode}
% \end{macro}
%
%
% \subsection{Character codes}
%
% \changes{v0.18}{1994/03/11}{Added the catcode changes}
%
% By default, \LaTeXe{} makes the input charactes 0--8, 11, 14--31 and
% 128--255 illegal.  In compatibility mode, we restore their old
% meanings.
%    \begin{macrocode}
\catcode0=9
\@tempcnta=1
\loop\ifnum\@tempcnta<32
   \catcode\@tempcnta=12
   \advance\@tempcnta by 1
\repeat%
\catcode`\^^I=10\relax%
\catcode`\^^L=13\relax%
\catcode`\^^M=5\relax%
\catcode127=15
\@tempcnta=128
\loop\ifnum\@tempcnta<256
   \catcode\@tempcnta=12
   \advance\@tempcnta by 1
\repeat
%    \end{macrocode}
%
% \subsection{Miscellaneous commands}
%
% \begin{macro}{\SLiTeX}
%    The \textsc{Sli\TeX} logo.
%    \begin{macrocode}
\DeclareRobustCommand{\SLiTeX}{{%
   \normalfont S\kern -.06em
   {\scshape l\kern -.035emi}\kern -.06em
   \TeX}}
%    \end{macrocode}
% \end{macro}
%
% \begin{macro}{\+}
%    The |\+| command should be defined, so that it can be used in
%    |\renewcommand|.
%    \begin{macrocode}
\let\+\@empty
%    \end{macrocode}
% \end{macro}
%
% \begin{macro}{\@cla}
% \begin{macro}{\@clb}
% \changes{v0.47}{1995/09/25}
%         {(DPC) Declare old \cs{cline} registers}
% \begin{macro}{\mscount}
% \LaTeX2.09 (and early versions of \LaTeXe) used these count registers
% in the definition of |\cline| and |\multispan|.
% Declare them here in case they were used for any other purposes.
%    \begin{macrocode}
\newcount\@cla
\newcount\@clb
\newcount\mscount
%    \end{macrocode}
% \end{macro}
% \end{macro}
% \end{macro}
%
% \begin{macro}{\@imakepicbox}
% \changes{v0.48}{1995/10/16}
%         {(DPC) emulate old behaviour of picture mode makebox}
% picture mode version
%    \begin{macrocode}
\long\def\@imakepicbox(#1,#2)[#3]#4{%
  \vbox to#2\unitlength
   {\let\mb@b\vss \let\mb@l\hss\let\mb@r\hss
    \let\mb@t\vss
    \@tfor\reserved@a :=#3\do{%
      \if s\reserved@a
        \let\mb@l\relax\let\mb@r\relax
      \else
        \expandafter\let\csname mb@\reserved@a\endcsname\relax
      \fi}%
    \mb@t
    \hb@xt@ #1\unitlength{\mb@l #4\mb@r}%
    \mb@b
%    \end{macrocode}
% This kern ensures that a |b| option aligns on the bottom of the
% text rather than the baseline. this is the documented behaviour in
% the \LaTeX Book. The kern is removed in compatibility mode.
%
% Remove kern for bug compatibility with 2.09.
%    \begin{macrocode}
%    \kern\z@
     }}
%    \end{macrocode}
% \end{macro}
%
% \begin{macro}{\supereject}
%    \begin{macrocode}
\def\supereject{\par\penalty-\@MM}
%    \end{macrocode}
%  \end{macro}
%
% \begin{macro}{\nofiles}
% \changes{v0.51}{1996/05/24}{(DPC) Old definition, without \cs{write}
% added to \cs{protected@write}, for latex/2146}
% This old version might change the vertical spacing when it is used.
% Some old document might depend on that changed spacing so\ldots
%    \begin{macrocode}
\def\nofiles{%
  \@fileswfalse
  \typeout{No auxiliary output files.^^J}%
  \long\def\protected@write##1##2##3{}%
  \let\makeindex\relax
  \let\makeglossary\relax}
%    \end{macrocode}
% \end{macro}
%
% \subsection{Packages and classes}
%
% \begin{macro}{\ProvidesPackage}
% \begin{macro}{\ProvidesClass}
%    We redefine |\ProvidesPackage| and |\ProvidesClass| to produce a
%    log message rather than a warning if they find an unexpected
%    file.
%    \begin{macrocode}
\def\ProvidesPackage#1{%
  \xdef\@gtempa{#1}%
  \ifx\@gtempa\@currname\else
    \wlog{Compatibility mode: \@cls@pkg\space`\@currname' requested,
       but `#1' provided.}%
  \fi
  \@ifnextchar[\@pr@videpackage{\@pr@videpackage[]}}%]
\let\ProvidesClass=\ProvidesPackage
%    \end{macrocode}
% \end{macro}
% \end{macro}
% That ends the head of |latex209.def|.
%    \begin{macrocode}
%</head>
%    \end{macrocode}
%
% \section{Middle of latex209.def}
%
% At this point, the code for |oldlfont.sty| is read in by the
% installation script.
%
% \section{End of latex209.def}
%
% This section describes the end of |latex209.def|.
%    \begin{macrocode}
%<*tail>
%    \end{macrocode}
%
% \subsection{Font commands}
%
% \changes{v0.12}{1994/01/24}{Added the oldlfont option.}
% \changes{v0.25}{1994/05/14}{Added the newlfont option, rewrote the
%    oldlfont option to change math grouping.}
% \changes{v0.26}{1994/05/15}{Added the margid and nomargid options.}
% \changes{v0.27}{1994/05/16}{Replaced \# by \#\# in margid.}
% \changes{v0.28}{1994/05/17}{Replaced \cs{input} newlfont.sty by
%    \cs{OptionNotUsed} in \cs{ds@newlfont}.}
%
% \begin{macro}{\ds@oldlfont}
% \begin{macro}{\ds@newlfont}
% \begin{macro}{\ds@margid}
% \begin{macro}{\ds@nomargid}
%    We declare |oldlfont|, |newlfont|, |margid| and |nomargid|
%    options to mimic the \LaTeX~2.09 NFSS1 options.
%    \begin{macrocode}
\def\ds@oldlfont{%
%FM   \@no@font@optfalse
   \let\math@bgroup\@empty
   \let\math@egroup\@empty
   \let\@@math@bgroup\math@bgroup
   \let\@@math@egroup\math@egroup
}
\def\ds@newlfont{%
%FM    \@no@font@optfalse
   \OptionNotUsed
}
\def\ds@margid{%
%FM   \@no@font@optfalse
  \let\math@bgroup\bgroup
  \def\math@egroup##1{##1\egroup}%
  \let \@@math@bgroup \math@bgroup
  \let \@@math@egroup \math@egroup
}
\let\ds@nomargid\ds@oldlfont
\@onlypreamble\ds@oldfont
\@onlypreamble\ds@newfont
\@onlypreamble\ds@margid
\@onlypreamble\ds@nomargid
%    \end{macrocode}
% \end{macro}
% \end{macro}
% \end{macro}
% \end{macro}
%
% \begin{macro}{\encodingdefault}
%    The default encoding for old documents is OT1 rather than T1.
%    \begin{macrocode}
\renewcommand{\encodingdefault}{OT1}
%    \end{macrocode}
% \end{macro}
%
% \begin{macro}{\cmex/m/n/10}
%    Just in case a document style relies on |\cmex/m/n/10| to exist
%    (which may have been hard-wired to |\fam3|) we load the font.
%    \begin{macrocode}
\expandafter\font\csname cmex/m/n/10\endcsname=cmex10
%    \end{macrocode}
% \end{macro}
%
%
% \changes{v0.12}{1994/01/24}
%    {Added \cs{normalshape} and \cs{mediumseries}.}
% \changes{v0.28}{1994/05/16}{\cs{mediumseries} was using
%    \cs{fontshape} rather than \cs{fontseries}.}
%
% \begin{macro}{\normalshape}
% \begin{macro}{\mediumseries}
%    These commands were used in older versions of NFSS.
%    \begin{macrocode}
\def\normalshape{\fontshape\shapedefault\selectfont}
\def\mediumseries{\fontseries\seriesdefault\selectfont}
%    \end{macrocode}
% \end{macro}
% \end{macro}
% \begin{macro}{\DeclareOldFontCommand}
%    We redefine |\DeclareOldFontCommand| to do nothing.  This means
%    that any new document classes will have their redefinitions of
%    |\rm|, |\bf| etc.~ignored.
% \changes{v0.06}{1993/12/18}{Replaced \cs{@newfontswitch} by
%    \cs{@renewfontswitch}.}
% \changes{v0.11}{1994/01/21}{Removed \cs{RequirePackage}{oldlfont}.}
% \changes{v0.19}{1994/04/05}{Replaced \cs{@renewfontswitch} by
%    \cs{DeclareOldFontCommand}.}
%    \begin{macrocode}
\def \DeclareOldFontCommand #1#2#3{%
  \wlog{Compatibility mode: definition
        of \string#1\space ignored.}%
}
%    \end{macrocode}
% \end{macro}
%
% \changes{v0.17}
%    {1994/03/02}{Added \cs{@halfmag}, \cs{@magscale} and \cs{@ptscale}}
% \changes{v0.35}
%         {1994/09/23}{Added spaces to the old font scale commands.}
%
% \begin{macro}{\@halfmag}
% \begin{macro}{\@magscale}
% \begin{macro}{\@ptscale}
%    Some font-specifying commands from \LaTeX~2.09.
%    \begin{macrocode}
\def\@halfmag{ scaled \magstephalf}
\def\@magscale#1{ scaled \magstep#1 }
\def\@ptscale#1{ scaled #100 }
%    \end{macrocode}
% \end{macro}
% \end{macro}
% \end{macro}
%
% \begin{macro}{\font}
%    The current font is set to be CMR 10pt, to match \LaTeX~2.09.
%    \begin{macrocode}
\fontencoding{OT1} \fontfamily{cmr}
\fontsize{10}{12} \fontseries{m} \fontshape{n}
\selectfont
%    \end{macrocode}
% \end{macro}
%
% \changes{v0.11}{1994/01/21}{Added the rest of this subsection, which
%    used to be in oldlfont.dtx.}
%
% \begin{macro}{\load}
%    The |\load| command is no longer needed, it is therefore
%    defined to do nothing.
%    \begin{macrocode}
\let\load\@gobbletwo
%    \end{macrocode}
% \end{macro}
%
%    Here are three delimiters which have be partly disabled by
%    NFSS2 (the small variants) since the corresponding fonts are
%    normally not preloaded as math symbol fonts.
%    \begin{macrocode}
\DeclareMathDelimiter{\lgroup} % extensible ( with sharper tips
     {\mathopen}{bold}{"28}{largesymbols}{"3A}
\DeclareMathDelimiter{\rgroup} % extensible ) with sharper tips
     {\mathclose}{bold}{"29}{largesymbols}{"3B}
\DeclareMathDelimiter{\bracevert} % the vertical bar that extends braces
     {\mathord}{typewriter}{"7C}{largesymbols}{"3E}
%    \end{macrocode}
%
%    In old documents we might find some usages of |\bffam| etc. Thus
%    we add the following code:
%    \begin{macrocode}
\let\bffam\symbold
\let\sffam\symsans
\let\itfam\symitalic
\let\ttfam\symtypewriter
\let\scfam\symsmallcaps
\let\slfam\symslanted
\let\rmfam\symoperators
%    \end{macrocode}
%
%    Below are the |\..pt| commands with hopefully the same
%    functionality as in the old \texttt{lfonts.tex}. Notice that the
%    |\baselineskip| parameter wasn't set by these commands so that
%    using them now shouldn't set this either. Thus we go low-level.
%    This means that the commands are now fragile but I think they
%    have been fragile before.
%    \begin{macrocode}
\newcommand\vpt   {\edef\f@size{\@vpt}\rm}
\newcommand\vipt  {\edef\f@size{\@vipt}\rm}
\newcommand\viipt {\edef\f@size{\@viipt}\rm}
\newcommand\viiipt{\edef\f@size{\@viiipt}\rm}
\newcommand\ixpt  {\edef\f@size{\@ixpt}\rm}
\newcommand\xpt   {\edef\f@size{\@xpt}\rm}
\newcommand\xipt  {\edef\f@size{\@xipt}\rm}
\newcommand\xiipt {\edef\f@size{\@xiipt}\rm}
\newcommand\xivpt {\edef\f@size{\@xivpt}\rm}
\newcommand\xviipt{\edef\f@size{\@xviipt}\rm}
\newcommand\xxpt  {\edef\f@size{\@xxpt}\rm}
\newcommand\xxvpt {\edef\f@size{\@xxvpt}\rm}
%    \end{macrocode}
%
% \subsection{User customization}
%
% For sites which customized their version of \LaTeX~2.09, we provide
% a file |latex209.cfg|, which is loaded every time we enter
% compatibility mode.  If the file doesn't exist, we don't do
% anything.
%    \begin{macrocode}
\InputIfFileExists{latex209.cfg}{}{}
%    \end{macrocode}
% That ends the file |latex209.def|.
%    \begin{macrocode}
%</tail>
%    \end{macrocode}
%
% \section{Obsolete style files}
%
% \changes{v0.08}{1993/12/19}{Added this section.}
% \changes{v0.09}{1993/12/20}{Removed artN.sty, bkN.sty and repN.sty.}
% \changes{v0.23}{1994/05/11}{Added bezier.sty.}
% \changes{v0.31}{1994/05/20}{Added t1enc.sty.}
%
% For each of the standard \LaTeX~2.09 document styles, we produce a
% file which points to the appropriate \LaTeXe{} document class file.
% This means that any styles which say |\input article.sty| should
% still work.
%
%    \begin{macrocode}
%<*article|book|report|letter|slides>
\NeedsTeXFormat{LaTeX2e}
%</article|book|report|letter|slides>
%<*article>
\@obsoletefile{article.cls}{article.sty}
\LoadClass{article}
%</article>
%<*book>
\@obsoletefile{book.cls}{book.sty}
\LoadClass{book}
%</book>
%<*report>
\@obsoletefile{report.cls}{report.sty}
\LoadClass{report}
%</report>
%<*letter>
\@obsoletefile{letter.cls}{letter.sty}
\LoadClass{letter}
%</letter>
%<*slides>
\@obsoletefile{slides.cls}{slides.sty}
\LoadClass{slides}
%</slides>
%    \end{macrocode}
% We also produce empty |fleqn.sty| and |leqno.sty| files in case
% anyone has |\input| one of them.
%    \begin{macrocode}
%<*fleqn>
\@obsoletefile{fleqn.clo}{fleqn.sty}
\input{fleqn.clo}
%</fleqn>
%<*leqno>
\@obsoletefile{leqno.clo}{leqno.sty}
\input{leqno.clo}
%</leqno>
%    \end{macrocode}
% We also produce an empty |openbib.sty| in case anyone has |\input|
% |openbib.sty|.  The |openbib| class option is now part of the kernel.
%    \begin{macrocode}
%<*openbib>
\iffalse

The openbib option is now part of LaTeX thus this package is no
longer necessary.  It is only retained for upward compatibility.
See the 2nd edition of the LaTeX book, or the file usrguide.tex
which comes with the LaTeX distribution, for more details.

\fi
%</openbib>
%    \end{macrocode}
% We also produce an empty |bezier.sty| in case anyone has |\input|
% |bezier.sty|.  The |\bezier| command is now part of the kernel.
%    \begin{macrocode}
%<*bezier>
\iffalse

The \bezier command is now part of LaTeX thus this package is no
longer necessary.  It is only retained for upward compatibility.
Also, please note that LaTeX now offers an extended bezier command
which automatically calculates the number of points needed for the
plot.  See the 2nd edition of the LaTeX book, or the file
usrguide.tex which comes with the LaTeX distribution, for more
details.

\fi
%</bezier>
%    \end{macrocode}
% We also produce a |t1enc| package, for compatibility with the
% Companion.  This has been replaced by the |fontenc| package.
%    \begin{macrocode}
%<*t1enc>
\NeedsTeXFormat{LaTeX2e}
\ProvidesPackage{t1enc}[1994/06/01 Standard LaTeX package]
\renewcommand{\encodingdefault}{T1}
\fontencoding{T1}\selectfont
%</t1enc>
%    \end{macrocode}
% \DeleteShortVerb{\|}
% \Finale
\endinput
\makeatother
  \documentclass}
\@onlypreamble\documentstyle
%    \end{macrocode}
% \end{macro}
%
% \begin{macro}{\RequirePackage}
% Load package if not already loaded.
%    \begin{macrocode}
\def\RequirePackage{%
  \@fileswithoptions\@pkgextension}
\@onlypreamble\RequirePackage
%    \end{macrocode}
% \end{macro}
%
% \begin{macro}{\LoadClass}
% Load class.
%    \begin{macrocode}
\def\LoadClass{%
  \ifx\@currext\@pkgextension
     \@latex@error
      {\noexpand\LoadClass in package file}%
      {You may only use \noexpand\LoadClass in a class file.}%
  \fi
  \@fileswithoptions\@clsextension}
\@onlypreamble\LoadClass
%    \end{macrocode}
% \end{macro}
%
% \begin{macro}{\@loadwithoptions}
% \changes{v1.0t}{1995/11/14}{macro added}
% Pass the current option list on to a class or package.
% |#1| is |\@|\emph{cls-or-pkg}|extension|,
% |#2| is |\RequirePackage| or |\LoadClass|,
% |#3| is the class or package to be loaded.
%    \begin{macrocode}
\def\@loadwithoptions#1#2#3{%
  \expandafter\let\csname opt@#3.#1\expandafter\endcsname
       \csname opt@\@currname.\@currext\endcsname
   #2{#3}}
\@onlypreamble\@loadwithoptions
%    \end{macrocode}
% \end{macro}
%
%
% \begin{macro}{\LoadClassWithOptions}
% \changes{v1.0t}{1995/11/14}{macro added}
% Load class `|#1|' with the current option list.
%    \begin{macrocode}
\def\LoadClassWithOptions{%
  \@loadwithoptions\@clsextension\LoadClass}
\@onlypreamble\LoadClassWithOptions
%    \end{macrocode}
% \end{macro}
%
% \begin{macro}{\RequirePackageWithOptions}
% \changes{v1.0t}{1995/11/14}{macro added}
% \changes{v1.0v}{1996/10/04}{Reset \cs{@unprocessedoptions} for /2269}
% Load package `|#1|' with the current option list.
%    \begin{macrocode}
\def\RequirePackageWithOptions{%
  \AtEndOfPackage{\let\@unprocessedoptions\relax}%
  \@loadwithoptions\@pkgextension\RequirePackage}
\@onlypreamble\RequirePackageWithOptions
%    \end{macrocode}
% \end{macro}
%
% \begin{macro}{\usepackage}
%    To begin with, |\usepackage| produces an error.  This is reset by
%    |\documentclass|.
% \changes{v0.2o}{1993/12/13}
%         {Fixed error handling}
% \changes{v1.0h}{1994/05/23}{Remove argument if possible}
%    \begin{macrocode}
\def\usepackage#1#{%
  \@latex@error
    {\noexpand \usepackage before \string\documentclass}%
    {\noexpand \usepackage may only appear in the document
      preamble, i.e.,\MessageBreak
      between \noexpand\documentclass and
      \string\begin{document}.}%
  \@gobble}
\@onlypreamble\usepackage
%    \end{macrocode}
% \end{macro}
%
% \begin{macro}{\NeedsTeXFormat}
% Check that the document is running on the correct system.
% \changes{v0.2a}{1993/11/14}
%         {made more robust for alternative syntax for other formats.}
% \changes{v0.2c}{1993/11/17}
%         {Name changed from \cs{NeedsFormat}}
% \changes{v0.2d}{1993/11/18}
%         {\cs{fmtname} \cs{fmtversion} not \cs{@}\ldots}
%    \begin{macrocode}
\def\NeedsTeXFormat#1{%
  \def\reserved@a{#1}%
  \ifx\reserved@a\fmtname
    \expandafter\@needsformat
  \else
     \@latex@error{This file needs format `\reserved@a'%
       \MessageBreak but this is `\fmtname'}{%
       The current input file will not be processed
       further,\MessageBreak
       because it was written for some other flavor of
       TeX.\MessageBreak\@ehd}%
%    \end{macrocode}
%    If the file is not meant to be processed by \LaTeXe{} we stop
%    inputting it, but we do not end the run. We just end inputting
%    the current file.
% \changes{v1.0h}{1994/05/23}
%     {Don't stop completely when format is wrong}
%    \begin{macrocode}
     \endinput \fi}
\@onlypreamble\NeedsTeXFormat
%    \end{macrocode}
%
%    \begin{macrocode}
\def\@needsformat{%
  \@ifnextchar[%]
    \@needsf@rmat
    {}}
\@onlypreamble\@needsformat
%    \end{macrocode}
%
% \changes{v1.0b}{1994/05/04}
%         {Changed wording of the warning}
%    \begin{macrocode}
\def\@needsf@rmat[#1]{%
    \@ifl@t@r\fmtversion{#1}{}%
    {\@latex@warning@no@line
        {You have requested release `#1' of LaTeX,\MessageBreak
         but only release `\fmtversion' is available}}}
\@onlypreamble\@needsf@rmat
%    \end{macrocode}
% \end{macro}
%
% \begin{macro}{\zap@space}
% |\zap@space foo|\meta{space}|\@empty| removes all spaces from |foo|
% that are not protected by |{ }| groups.
%    \begin{macrocode}
\def\zap@space#1 #2{%
  #1%
  \ifx#2\@empty\else\expandafter\zap@space\fi
  #2}
%    \end{macrocode}
% \end{macro}
%
% \begin{macro}{\@fileswithoptions}
% The common part of |\documentclass| and |\usepackage|.
%    \begin{macrocode}
\def\@fileswithoptions#1{%
  \@ifnextchar[%]
    {\@fileswith@ptions#1}%
    {\@fileswith@ptions#1[]}}
\@onlypreamble\@fileswithoptions
%    \end{macrocode}
%
% \changes{v0.2f}{1993/11/22}
%         {Made the default [] not [\cs{@unknownversion}]}
% \changes{v1.1h}{2007/08/05}
%         {Prevent loss of brackets PR/3965}
%    \begin{macrocode}
\def\@fileswith@ptions#1[#2]#3{%
  \@ifnextchar[%]
  {\@fileswith@pti@ns#1[{#2}]#3}%
  {\@fileswith@pti@ns#1[{#2}]#3[]}}
\@onlypreamble\@fileswith@ptions
%    \end{macrocode}
% Then we do some work.
%
% First of all, we define the global variables.
% Then we look to see if the file has already been loaded.
% If it has, we check that it was first loaded with at least the current
% options.
% If it has not, we add the current options to the package options,
% set the default version to be |0000/00/00|, and load the file if we
% can find it.
% Then we check the version number.
%
% Finally, we restore the old file name, reset the default option,
% and we set the catcode of |@|.
%
% For classes, we can immediately process the file. For other types,
% |#2| could be a comma separated list, so loop through, processing
% each one separately.
% \changes{v0.2q}{1993/12/17}
%         {Add \cs{@compatibility} hook}
% \changes{v0.2s}{1994/01/17}
%         {Modify to reduce parameter stack usage}
% \changes{v0.2y}{1994/02/07}
%         {Run \cs{@compatibility} on the first class to start
%          (not the first to finish) }
% \changes{v0.2z}{1994/02/10}
%         {Renamed \cs{@compatibility} to \cs{@documentclasshook}.
%          ASAJ.}
% \changes{v1.1h}{2007/08/05}
%         {Prevent loss of brackets PR/3965}
%    \begin{macrocode}
\def\@fileswith@pti@ns#1[#2]#3[#4]{%
  \ifx#1\@clsextension
    \ifx\@classoptionslist\relax
      \xdef\@classoptionslist{\zap@space#2 \@empty}%
      \def\reserved@a{%
        \@onefilewithoptions#3[{#2}][{#4}]#1%
        \@documentclasshook}%
    \else
      \def\reserved@a{%
        \@onefilewithoptions#3[{#2}][{#4}]#1}%
    \fi
  \else
%    \end{macrocode}
% build up a list of calls to |\@onefilewithoptions|
% (one for each package) without thrashing the parameter stack.
%    \begin{macrocode}
    \def\reserved@b##1,{%
      \ifx\@nil##1\relax\else
        \ifx\relax##1\relax\else
         \noexpand\@onefilewithoptions##1[{#2}][{#4}]%
         \noexpand\@pkgextension
        \fi
        \expandafter\reserved@b
      \fi}%
      \edef\reserved@a{\zap@space#3 \@empty}%
      \edef\reserved@a{\expandafter\reserved@b\reserved@a,\@nil,}%
  \fi
  \reserved@a}
\@onlypreamble\@fileswith@pti@ns
%    \end{macrocode}
%
% Have the main argument as |#1|, so we only need one |\expandafter|
% above.
% \changes{v0.2a}{1993/11/14}
%         {Moved resetting of \cs{default@ds}, \cs{ds@} and
%         \cs{@declaredoptions} here, from the end of
%         \cs{ProcessOptions}.}
% \changes{v0.2f}{1993/11/22}
%         {Made the initial version [] not [\cs{@unknownversion}]}
% \changes{v0.2m}{1993/12/07}
%         {Reset \cs{CurrentOption}}
%    \begin{macrocode}
\def\@onefilewithoptions#1[#2][#3]#4{%
  \@pushfilename
  \xdef\@currname{#1}%
  \global\let\@currext#4%
  \expandafter\let\csname\@currname.\@currext-h@@k\endcsname\@empty
  \let\CurrentOption\@empty
  \@reset@ptions
  \makeatletter
%    \end{macrocode}
% Grab everything in a macro, so the parameter stack is popped before
% any processing begins.
% \changes{v0.2s}{1994/01/17}
%         {Modify to reduce parameter stack usage}
% \changes{v1.1b}{1998/05/07}
%         {Modify help message for latex/2805}
%    \begin{macrocode}
  \def\reserved@a{%
    \@ifl@aded\@currext{#1}%
      {\@if@ptions\@currext{#1}{#2}{}%
        {\@latex@error
            {Option clash for \@cls@pkg\space #1}%
            {The package #1 has already been loaded
             with options:\MessageBreak
             \space\space[\@ptionlist{#1.\@currext}]\MessageBreak
             There has now been an attempt to load it
              with options\MessageBreak
             \space\space[#2]\MessageBreak
             Adding the global options:\MessageBreak
             \space\space
                  \@ptionlist{#1.\@currext},#2\MessageBreak
             to your \noexpand\documentclass declaration may fix this.%
             \MessageBreak
             Try typing \space <return> \space to proceed.}}}%
      {\@pass@ptions\@currext{#2}{#1}%
%    \end{macrocode}
% \changes{v0.3c}{1994/03/12}
%         {Do not use \cs{@pr@videpackage} to avoid typeout}
%    \begin{macrocode}
       \global\expandafter
       \let\csname ver@\@currname.\@currext\endcsname\@empty
       \InputIfFileExists
         {\@currname.\@currext}%
         {}%
         {\@missingfileerror\@currname\@currext}%
%    \end{macrocode}
% |\@unprocessedoptions| will generate an error for each specified
% option in a package unless a |\ProcessOptions| has appeared in the
% package file.
% \changes{v0.2v}{1994/01/29}
%         {All options raise error if no \cs{ProcessOptions} appears}
% \changes{v0.2x}{1994/02/02}
%         {Only run the hook and options check if the file was loaded.}
%    \begin{macrocode}
    \let\@unprocessedoptions\@@unprocessedoptions
    \csname\@currname.\@currext-h@@k\endcsname
    \expandafter\let\csname\@currname.\@currext-h@@k\endcsname
              \@undefined
    \@unprocessedoptions}
%    \end{macrocode}
%
%    \begin{macrocode}
    \@ifl@ter\@currext{#1}{#3}{}%
      {\@latex@warning@no@line
         {You have requested,\on@line,
          version\MessageBreak
            `#3' of \@cls@pkg\space #1,\MessageBreak
          but only version\MessageBreak
           `\csname ver@#1.\@currext\endcsname'\MessageBreak
          is available}}%
%    \end{macrocode}
% \changes{v0.2c}{1993/11/17}
%         {Added trap for two \cs{LoadClass} commands.}
%    \begin{macrocode}
    \ifx\@currext\@clsextension\let\LoadClass\@twoloadclasserror\fi
    \@popfilename
    \@reset@ptions}%
  \reserved@a}
\@onlypreamble\@onefilewithoptions
%    \end{macrocode}
% \end{macro}
%
% \begin{macro}{\@@fileswith@pti@ns}
% Save the definition (for error checking).
% \changes{v0.2c}{1993/11/17}
%         {Macro added}
%    \begin{macrocode}
\let\@@fileswith@pti@ns\@fileswith@pti@ns
\@onlypreamble\@@fileswith@pti@ns
%    \end{macrocode}
% \end{macro}
%
% \begin{macro}{\@reset@ptions}
% Reset the default option, and clear lists of declared options.
% \changes{v0.2a}{1993/11/14}{macro added}
%    \begin{macrocode}
\def\@reset@ptions{%
  \global\ifx\@currext\@clsextension
    \let\default@ds\OptionNotUsed
   \else
    \let\default@ds\@unknownoptionerror
  \fi
  \global\let\ds@\@empty
  \global\let\@declaredoptions\@empty}
\@onlypreamble\@reset@ptions
%    \end{macrocode}
% \end{macro}
%
% \subsection{Hooks}
%
% Allow code do be saved to be executed at specific later times.
%
% Save things in macros, I considered using toks registers, (and
% |\addto@hook| from the NFSS code, that would require stacking the
% contents in the case of required packages, so just generate a new
% macro for each package.
% \begin{macro}{\@begindocumenthook}
% \changes{v1.0s}{1995/10/20}
%         {Make setting conditional, for autoload version}
% \begin{macro}{\@enddocumenthook}
% Stuff to appear at the beginning or end of the document.
%    \begin{macrocode}
\ifx\@begindocumenthook\@undefined
  \let\@begindocumenthook\@empty
\fi
\let\@enddocumenthook\@empty
%    \end{macrocode}
% \end{macro}
% \end{macro}
%
% \begin{macro}{\g@addto@macro}
% Globally add to the end of a macro.
% \changes{v0.2a}{1993/11/14}{Made global}
% \changes{v0.2w}{1994/01/31}
%     {Use toks register to avoid `hash' problems}
% \changes{v1.0o}{1995/05/17}
%     {Make long for latex/1522}
% \changes{v1.0w}{1996/12/17}
%     {Use \cs{begingroup} to save making a mathord}
% \changes{v1.0x}{1997/02/05}
%     {missing percent /2402}
%    \begin{macrocode}
\long\def\g@addto@macro#1#2{%
  \begingroup
    \toks@\expandafter{#1#2}%
    \xdef#1{\the\toks@}%
  \endgroup}
%    \end{macrocode}
% \end{macro}
%
% \begin{macro}{\AtEndOfPackage}
% \begin{macro}{\AtEndOfClass}
% \begin{macro}{\AtBeginDocument}
% \begin{macro}{\AtEndDocument}
% The access functions.
% \changes{v0.2a}{1993/11/14}
%         {Included extension in the generated macro name for package
%         and class hooks.}
%    \begin{macrocode}
\def\AtEndOfPackage{%
  \expandafter\g@addto@macro\csname\@currname.\@currext-h@@k\endcsname}
\let\AtEndOfClass\AtEndOfPackage
\@onlypreamble\AtEndOfPackage
\@onlypreamble\AtEndOfClass
%    \end{macrocode}
%
%    \begin{macrocode}
\def\AtBeginDocument{\g@addto@macro\@begindocumenthook}
\def\AtEndDocument{\g@addto@macro\@enddocumenthook}
\@onlypreamble\AtBeginDocument
%    \end{macrocode}
% \end{macro}
% \end{macro}
% \end{macro}
% \end{macro}
%
%
% \begin{macro}{\@cls@pkg}
%    The current file type.
% \changes{v0.2i}{1993/12/03}
%         {Name changed to avoid clash with output routine.}
%    \begin{macrocode}
\def\@cls@pkg{%
  \ifx\@currext\@clsextension
    document class%
  \else
    package%
  \fi}
\@onlypreamble\@cls@pkg
%    \end{macrocode}
% \end{macro}
%
% \begin{macro}{\@unknownoptionerror}
% Bad option.
%    \begin{macrocode}
\def\@unknownoptionerror{%
  \@latex@error
    {Unknown option `\CurrentOption' for \@cls@pkg\space`\@currname'}%
    {The option `\CurrentOption' was not declared in
     \@cls@pkg\space`\@currname', perhaps you\MessageBreak
      misspelled its name.
     Try typing \space <return>
     \space to proceed.}}
\@onlypreamble\@unknownoptionerror
%    \end{macrocode}
% \end{macro}
%
% \begin{macro}{\@@unprocessedoptions}
% Declare an error for each option, unless a |\ProcessOptions| occurred.
% \changes{v0.2v}{1994/01/29}
%         {Macro added.}
% \changes{v1.0t}{1995/11/14}{Allow empty option}
%    \begin{macrocode}
\def\@@unprocessedoptions{%
  \ifx\@currext\@pkgextension
    \edef\@curroptions{\@ptionlist{\@currname.\@currext}}%
    \@for\CurrentOption:=\@curroptions\do{%
        \ifx\CurrentOption\@empty\else\@unknownoptionerror\fi}%
  \fi}
\@onlypreamble\@unprocessedoptions
\@onlypreamble\@@unprocessedoptions
%    \end{macrocode}
% \end{macro}
%
% \begin{macro}{\@badrequireerror}
% |\RequirePackage| or |\LoadClass| occurs in the options section.
% \changes{v0.2c}{1993/11/17}
%         {Macro added}
%    \begin{macrocode}
\def\@badrequireerror#1[#2]#3[#4]{%
  \@latex@error
    {\noexpand\RequirePackage or \noexpand\LoadClass
         in Options Section}%
    {The \@cls@pkg\space `\@currname' is defective.\MessageBreak
     It attempts to load `#3' in the options section, i.e.,\MessageBreak
     between \noexpand\DeclareOption and \string\ProcessOptions.}}
\@onlypreamble\@badrequireerror
%    \end{macrocode}
% \end{macro}
%
% \begin{macro}{\@twoloadclasserror}
% Two |\LoadClass| in a class.
% \changes{v0.2c}{1993/11/17}
%         {Macro added}
%    \begin{macrocode}
\def\@twoloadclasserror{%
  \@latex@error
    {Two \noexpand\LoadClass commands}%
    {You may only use one \noexpand\LoadClass in a class file}}
\@onlypreamble\@twoloadclasserror
%    \end{macrocode}
% \end{macro}
%
% \begin{macro}{\@twoclasseserror}
% Two |\documentclass| or |\documentstyle|.
% \changes{v0.2h}{1993/11/28}
%         {Macro added}
%    \begin{macrocode}
\def\@twoclasseserror#1#{%
  \@latex@error
    {Two \noexpand\documentclass or \noexpand\documentstyle commands}%
    {The document may only declare one class.}\@gobble}
\@onlypreamble\@twoclasseserror
%    \end{macrocode}
% \end{macro}
%
% \subsection{Providing shipment}
%
% \begin{macro}{\two@digits}
% Prefix a number less than 10 with `0'.
%    \begin{macrocode}
\def\two@digits#1{\ifnum#1<10 0\fi\number#1}
%    \end{macrocode}
% \end{macro}
%
%  \begin{macro}{\filecontents}
%  \begin{macro}{\endfilecontents}
%    This environment implements inline files.
%    The star-form does not write extra comments into the file.
%
% \changes{v0.2h}{1993/11/28}
%         {Don't globally allocate a write stream (always use 15)}
% \changes{v0.2r}{1993/12/19}{Different message when ignoring a file}
% \changes{v0.3g}{1994/04/11}
%         {Add star form,
%          dont write \cs{endinput} at the end of the file.}
% \changes{v1.0c}{1994/05/11}
%         {Add checks for form feed and tab}
% \changes{v1.0m}{1995/04/21}
%         {Close input check stream: latex/1487}
% \changes{v1.0p}{1995/05/25}{Delete \cs{filec@ntents} after preamble}
%    \begin{macrocode}
\begingroup%
\catcode`\*=11 %
\catcode`\^^M\active%
\catcode`\^^L\active\let^^L\relax%
\catcode`\^^I\active%
%    \end{macrocode}
%
%    \begin{macrocode}
\gdef\filecontents{\@tempswatrue\filec@ntents}%
\gdef\filecontents*{\@tempswafalse\filec@ntents}%
%    \end{macrocode}
%
%    \begin{macrocode}
\gdef\filec@ntents#1{%
  \openin\@inputcheck#1 %
  \ifeof\@inputcheck%
    \@latex@warning@no@line%
        {Writing file `\@currdir#1'}%
%    \end{macrocode}
%
% \changes{v1.0y}{1997/10/10}
%         {\cs{reserved@c} not \cs{verbatim@out} to save a csname}
%    \begin{macrocode}
    \chardef\reserved@c15 %
    \ch@ck7\reserved@c\write%
    \immediate\openout\reserved@c#1\relax%
  \else%
%    \end{macrocode}
%
% \changes{v1.0y}{1997/10/10}
%         {Use \cs{@gobbletwo}}
%    \begin{macrocode}
    \closein\@inputcheck%
    \@latex@warning@no@line%
            {File `#1' already exists on the system.\MessageBreak%
             Not generating it from this source}%
    \let\write\@gobbletwo%
    \let\closeout\@gobble%
  \fi%
  \if@tempswa%
%    \end{macrocode}
%
% \changes{v1.0y}{1997/10/10}
%         {\cs{@currenvir} in banner}
%    \begin{macrocode}
    \immediate\write\reserved@c{%
      \@percentchar\@percentchar\space%
          \expandafter\@gobble\string\LaTeX2e file `#1'^^J%
      \@percentchar\@percentchar\space  generated by the %
        `\@currenvir' \expandafter\@gobblefour\string\newenvironment^^J%
      \@percentchar\@percentchar\space from source `\jobname' on %
         \number\year/\two@digits\month/\two@digits\day.^^J%
      \@percentchar\@percentchar}%
  \fi%
  \let\do\@makeother\dospecials%
%    \end{macrocode}
%
% \changes{v1.0y}{1997/10/10}
%     {Check for text before or after \cs{end} environment. latex/2636}
%    \begin{macrocode}
  \edef\E{\@backslashchar end\string{\@currenvir\string}}%
  \edef\reserved@b{%
    \def\noexpand\reserved@b%
         ####1\E####2\E####3\relax}%
  \reserved@b{%
    \ifx\relax##3\relax%
%    \end{macrocode}
% There was no |\end{filecontents}|
%    \begin{macrocode}
      \immediate\write\reserved@c{##1}%
    \else%
%    \end{macrocode}
% There was a |\end{filecontents}|, so stop this time.
%    \begin{macrocode}
      \edef^^M{\noexpand\end{\@currenvir}}%
      \ifx\relax##1\relax%
      \else%
%    \end{macrocode}
% Text before the |\end|, write it with a warning.
%    \begin{macrocode}
          \@latex@warning{Writing text `##1' before %
             \string\end{\@currenvir}\MessageBreak as last line of #1}%
        \immediate\write\reserved@c{##1}%
      \fi%
      \ifx\relax##2\relax%
      \else%
%    \end{macrocode}
% Text after the |\end|, ignore it with a warning.
%    \begin{macrocode}
         \@latex@warning{%
           Ignoring text `##2' after \string\end{\@currenvir}}%
      \fi%
    \fi%
    ^^M}%
%    \end{macrocode}
%
%    \begin{macrocode}
  \catcode`\^^L\active%
  \let\L\@undefined%
  \def^^L{\@ifundefined L^^J^^J^^J}%
  \catcode`\^^I\active%
  \let\I\@undefined%
  \def^^I{\@ifundefined I\space\space}%
  \catcode`\^^M\active%
  \edef^^M##1^^M{%
    \noexpand\reserved@b##1\E\E\relax}}%
\endgroup%
%    \end{macrocode}
%
%    \begin{macrocode}
\begingroup
\catcode`|=\catcode`\%
\catcode`\%=12
\catcode`\*=11
\gdef\@percentchar{%}
\gdef\endfilecontents{|
  \immediate\closeout\reserved@c
  \def\T##1##2##3{|
  \ifx##1\@undefined\else
    \@latex@warning@no@line{##2 has been converted to Blank ##3e}|
  \fi}|
  \T\L{Form Feed}{Lin}|
  \T\I{Tab}{Spac}|
  \immediate\write\@unused{}}
\global\let\endfilecontents*\endfilecontents
\@onlypreamble\filecontents
\@onlypreamble\endfilecontents
\@onlypreamble\filecontents*
\@onlypreamble\endfilecontents*
\endgroup
\@onlypreamble\filec@ntents
%    \end{macrocode}
% \end{macro}
% \end{macro}
%
%
% \changes{v0.2f}{1993/11/22}
%         {\cs{@unknownversion} removed}
% \changes{v1.0j}{1994/10/18}
%         {Move \cs{listfiles} to ltfiles.dtx}
%
%    \begin{macrocode}
%</2ekernel>
%    \end{macrocode}
%
% \section{After Preamble}
% Finally we declare a package that allows all the commands declared
% above to be |\@onlypreamble| to be used after |\begin{document}|.
% \changes{v0.3f}{1994/03/16}
%         {Add pkgindoc package}
% \changes{v1.1a}{1998/03/21}
%         {Correct to new onlypreamble command list}
%    \begin{macrocode}
%<*afterpreamble>
\NeedsTeXFormat{LaTeX2e}
\ProvidesPackage{pkgindoc}
         [1994/10/20 v1.1 Package Interface in Document (DPC)]
\def\reserved@a#1\do\@classoptionslist#2\do\filec@ntents#3\relax{%
  \gdef\@preamblecmds{#1#3}}
\expandafter\reserved@a\@preamblecmds\relax
%</afterpreamble>
%    \end{macrocode}
%
% \Finale

% \iffalse meta-comment
%
% Copyright 1989-2008 Johannes L. Braams and any individual authors
% listed elsewhere in this file.  All rights reserved.
% 
% This file is part of the Babel system.
% --------------------------------------
% 
% It may be distributed and/or modified under the
% conditions of the LaTeX Project Public License, either version 1.3
% of this license or (at your option) any later version.
% The latest version of this license is in
%   http://www.latex-project.org/lppl.txt
% and version 1.3 or later is part of all distributions of LaTeX
% version 2003/12/01 or later.
% 
% This work has the LPPL maintenance status "maintained".
% 
% The Current Maintainer of this work is Johannes Braams.
% 
% The list of all files belonging to the Babel system is
% given in the file `manifest.bbl. See also `legal.bbl' for additional
% information.
% 
% The list of derived (unpacked) files belonging to the distribution
% and covered by LPPL is defined by the unpacking scripts (with
% extension .ins) which are part of the distribution.
% \fi
\documentclass[11pt]{article}
\usepackage[american,greek]{babel}
\languageattribute{greek}{polutoniko}
\usepackage{athnum,grmath}
\newcommand{\sg}{\selectlanguage{greek}}
\newcommand{\sa}{\selectlanguage{american}}
\begin{document}
\show\extrasgreek
%\show\extraspolutonikogreek
\selectlanguage{american}

\title{Writing Greek with the \ttfamily greek\rmfamily\  option of the
\ttfamily babel\rmfamily\ package}
\author{Apostolos Syropoulos\\
        366, 28th October Str.\\
        GR-671 00 Xanthi, GREECE\\
        e-mail: \texttt{apostolo@platon.ee.duth.gr}}
\date{October 15, 1997}
\maketitle
%%%%%%%%%%%%%%%%%%%%%%%%%%%%%%%%%%%%%%%%%%%%%%%%%%%%%%%%%%%%%%%%%%%%%%%%
\section{Overview}
The \texttt{greek} option of the \texttt{babel} package is an attempt to
make it possible for someone to write Greek text with \LaTeX. The current
version of the \texttt{greek} option supports the \sg polutonik'o \sa\ 
accentual system of the Greek language.
Moreover, there is now support for Greek numerals. One can produce easily 
valid Greek numerals both in uppercase and lowercase forms, e.g,
\textgreek{\greeknumeral{1997}}\ and \textgreek{\Greeknumeral{1997}}. The
labels in second and fourth level enumerations are lowercase
and uppercase Greek numerals correspondingly.
%%%%%%%%%%%%%%%%%%%%%%%%%%%%%%%%%%%%%%%%%%%%%%%%%%%%%%%%%%%%%%%%%%%%%%%%%
\section{Typing Greek Text}
\TeX\ understands only the basic ASCII characters, so it is not possible
to enter directly Greek letters. Instead, someone enters Latin letters
which are mapped to their Greek ``counterparts'' by \TeX. The following
table shows the transliteration employed:
\begin{center}
\begin{tabular}{|lllllllllllll|}\hline
\textgreek{a}&  
\textgreek{b}&   
\textgreek{g}&  
\textgreek{d}&  
\textgreek{e}&  
\textgreek{z}&  
\textgreek{h}&  
\textgreek{j}&   
\textgreek{i}&   
\textgreek{k}&   
\textgreek{l}&   
\textgreek{m}&
\textgreek{n}\\
a& b& g& d&  e&  z&  h&  j&  i&  k&  l&  m&  n\\
\hline    
\textgreek{x}&  
\textgreek{o}&  
\textgreek{p}&  
\textgreek{r}&  
\textgreek{sv}&  
\textgreek{t}&  
\textgreek{u}&  
\textgreek{f}&  
\textgreek{q}&  
\textgreek{y}&  
\textgreek{w}& 
\textgreek{c}& \hbox{ } \\
x&  o&  p&  r&  s&   
t&  u&  f&  q&  y&  w& c& \hbox{ }\\ \hline
\end{tabular}
\end{center}
Please, note that in order to produce the letter \textgreek{sv} in isolation
on has to type \texttt{sv}. This feature is due to the strong ligature
that \TeX\ employs. 
In the ``modern'' \textgreek{monotonik'o} accentual system only one accent is 
used---\textgreek{oxe'ia} (acute). In the traditional \textgreek{polutonik'o} 
accentual system we 
need more accents and breathing signs. We can produce an accented letter by
prefixing the letter with he symbol that denotes the accent, e.g.,
\texttt{>a'erac} produces the word \sg >a'erac.\sa\footnote{For the 
technically inclined reader, we must say that \TeX\ uses the ligature table of
the font in order to determine the character that corresponds to the
input character sequence.} Here are the symbols that are recognized: 

\begin{center}
\begin{tabular}{cccc}\hline
Accent & Symbol & Example & Output\\ \hline
acute  & \texttt{'} & \texttt{g'ata} & \textgreek{g'ata}\\
grave  & \texttt{`} & \texttt{dad`i} & \textgreek{dad`i}\\
circumflex & \verb+~+ & \verb+ful~hc+ & \sg\textgreek{ful~hc}\sa\\
rough breathing & \verb+<+ & \verb+<'otan+ & \sg\textgreek{<'otan}\sa\\
smooth breathing & \verb+>+ & \verb+>'aneu+ & \sg\textgreek{>'aneu}\sa\\
subscript & \texttt{|} & \verb+>anate'ilh|+ & \sg\textgreek{>anate'ilh|}\\
dieresis & \texttt{"}& \texttt{qa"ide'uh|c} & \sg\textgreek{qa"ide'uh|c}\\ 
\hline
\end{tabular}
\end{center}
Note that the subscript symbol is placed \textbf{after} the letter.
The last thing someone must know in order to be able to write normal Greek
text is the punctuation marks used in the language:
\begin{center}
\begin{tabular}{ccc}\hline
Punctuation Sign & Symbol & Output\\ \hline
period   & \texttt{.} & \sg\textgreek{.}\sa\\
semicolon & \texttt{;} & \sg\textgreek{;}\sa\\
exclamation mark & \texttt{!} & \sg\textgreek{!}\sa\\
comma & \texttt{,} & \sg\textgreek{,}\sa\\
colon & \texttt{:} & \sg\textgreek{:}\sa\\
question mark & \texttt{?} & \sg\textgreek{?}\sa\\
left apostrophe & \texttt{``} & \sg\textgreek{``}\sa\\
right apostrophe & \texttt{''} & \sg\textgreek{''}\sa\\
left quotation mark & \texttt{((} & \sg\textgreek{))}\sa\\
right quotation mark & \texttt{))} & \sg\textgreek{))}\sa\\ \hline
\end{tabular}
\end{center}
Using these conventions it is a straightforward exercise to write Greek
\textgreek{polutoniko} text. For example the following excerpt from 
\textgreek{D'uskoloc} of \textgreek{M'enandroc}
\sg
\begin{quote}
T'i f'hic? <Id`wn >enj'ede pa~id'' >eleuj'eran\\
t`ac plhs'ion N'umfac stefano~usan, S'wstrate,\\
>er~wn 'ap~hljec e>uj'uc?
\end{quote}
\sa can be produced by the following \LaTeX\ code:
\begin{center}
\begin{tabular}{l}
\verb+T'i f'hic? <Id`wn >enj'ede pa~id'' >eleuj'eran+\\
\verb+t`ac plhs'ion N'umfac stefano~usan, S'wstrate,+\\
\verb+>er~wn 'ap~hljec e>uj'uc?+
\end{tabular}
\end{center}
%%%%%%%%%%%%%%%%%%%%%%%%%%%%%%%%%%%%%%%%%%%%%%%%%%%%%%%%%%%%%%%%%%%%%
\section{Producing Greek Text}
Once the Greek language is selected with the command
\begin{center}
\verb+\selectlanguage{greek}+
\end{center}
whatever we type will be typeset with the Greek fonts. The command
\verb+\textlatin+ can be used for short passages in some language that
uses the Latin alphabet, while the the command \verb+\latintext+ changes
the base fonts to the ones used by languages that use the Latin alphabet.
However, all words will be hyphenated by following the Greek hyphenation
rules! Similar commands are available once someone has selected some
other language. The commands \verb+\textgreek+ and \verb+\greektext+
behave exactly like their ``latin'' counterparts. For example, the
word \textgreek{M'imhc} has been produced with the command 
\verb+\textgreek{M'imhc}+. Please note that certain symbols cannot have
their expected result for Greek text, unless someone has selected the Greek 
language, e.g., \verb+~+ is such a symbol.

As we have mentioned above this version of the \texttt{greek} option of the
\texttt{babel} package supports the use of Greek numerals. The commands
\verb+\greeknumeral+ and \verb+\Greeknumeral+ produce the lowercase and 
the uppercase Greek numeral, e.g., 
\begin{center}
\begin{tabular}{cc}\hline
Command & Output\\ \hline
\verb+\Greeknumeral{9999}+ & \sg\textgreek{\Greeknumeral{9999}}\\
\verb+\greeknumeral{9999}+ & \sg\textgreek{\greeknumeral{9999}}\\
\hline
\end{tabular}
\end{center}
In order to correctly typeset the greek numerals the greek option file
provides the following commands:
\begin{center}
\begin{tabular}{cc}\hline
Command & Output\\ \hline
\verb+\qoppa+ & \textgreek{\qoppa}\\
\verb+\sampi+ & \textgreek{\sampi}\\
\verb+\stigma+ & \textgreek{\stigma}\\
\hline
\end{tabular}
\end{center}

In traditional Greek typography the first paragraph after a header is
always indented, contrary to the habit of, say, American typography. This
effect can be achieved by using the package \verb+indentfirst+.

Additional greek symbols are available:
\begin{center}
\begin{tabular}{cc}\hline
Command & Output\\ \hline
\verb+\Digamma+ & \sg\textgreek{\Digamma}\\
\verb+\ddigamma+ & \sg\textgreek{\ddigamma}\\
%\verb+\tao+ & \sg\textgreek{\tao}\\
%\verb+\Qoppa+ & \sg\textgreek{\Qoppa}\\
%\verb|\VarQoppa| & \sg\textgreek{\VarQoppa}\\
%\verb+\varqoppa+ & \sg\textgreek{\varqoppa}\\
%\verb+\Sampi+ & \sg\textgreek{\Sampi}\\
%\verb|\Stigma| & \sg\textgreek{\Stigma}\\
\verb|\euro| & \sg\textgreek{\euro}\\
\verb|\permill| & \sg\textgreek{\permill}\\
\hline
\end{tabular}
\end{center}

The package \verb|athnum| provides the command \verb|\athnum|, with which
one can produce the so called \textit{Athenian numerals}:
\begin{center}
\begin{tabular}{cc}\hline
Command & Output\\ \hline
\verb|\athnum{1997}| & \sg\textgreek{\athnum{1997}}\\
\hline
\end{tabular}
\end{center}

The package \verb|grmath| renames the basic log-like functions with their
greek counterparts:
\begin{center}
\begin{tabular}{cc}\hline
Command & Output\\ \hline
\verb|$\sin^{2}x+\cos^{2}x=1$| & $\sin^{2}x+\cos^{2}y=1$\\
\hline
\end{tabular}
\end{center}
\end{document}

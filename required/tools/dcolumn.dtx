% \iffalse meta-comment
%
% Copyright 1993 1994 1995 1996 1997 1998 1999 2000 2001 2002 2003 2004 2005 2006
% The LaTeX3 Project and any individual authors listed elsewhere
% in this file. 
% 
% This file is part of the Standard LaTeX `Tools Bundle'.
% -------------------------------------------------------
% 
% It may be distributed and/or modified under the
% conditions of the LaTeX Project Public License, either version 1.3c
% of this license or (at your option) any later version.
% The latest version of this license is in
%    http://www.latex-project.org/lppl.txt
% and version 1.3c or later is part of all distributions of LaTeX 
% version 2005/12/01 or later.
% 
% The list of all files belonging to the LaTeX `Tools Bundle' is
% given in the file `manifest.txt'.
% 
% \fi
% \iffalse
%% File: dcolumn.dtx Copyright (C) 1992-1996 1999-2001 David Carlisle
%
%<*dtx>
          \ProvidesFile{dcolumn.dtx}
%</dtx>
%<package>\NeedsTeXFormat{LaTeX2e}
%<package>\ProvidesPackage{dcolumn}
%<driver>\ProvidesFile{dcolumn.drv}
% \fi
%         \ProvidesFile{dcolumn.dtx}
          [2001/05/28 v1.06 decimal alignment package (DPC)]
%
% \iffalse
%<*driver>
\documentclass{ltxdoc}
 \usepackage{dcolumn}
 \DeleteShortVerb{\|}
\begin{document}
 \DocInput{dcolumn.dtx}
\end{document}
%</driver>
% \fi
%
% \GetFileInfo{dcolumn.dtx}
%
% \title{The \textsf{dcolumn} package\thanks{This file
%         has version number \fileversion, last
%         revised \filedate.}}
% \author{David Carlisle}
% \date{\filedate}
% \author{David Carlisle}
% \maketitle
%
%
% \changes{v1.00}{1992/02/17}{Initial version}
% \changes{v1.01}{1992/06/12}{Re-issue for the new doc and docstrip.}
% \changes{v1.02}{1994/03/14}{Re-issue for LaTeX2e}
% \changes{v1.03}{1996/02/28}{New feature, after tools/2093}
% \changes{v1.05}{1999/07/05}
%      {Minor doc changes latex/3058}
% \changes{v1.06}{2001/05/28}
%      {More doc changes (suggested by BNB, done by CAR) latex/3315}
%
% \CheckSum{143}
%
%
% \begin{abstract}
% This package defines a system for defining columns of entries in an
% \texttt{array} or \texttt{tabular} which are to be aligned on a
% `decimal point'.
% \end{abstract}
%
% \MakeShortVerb{\"}
%
% This package defines {\tt D} to be a column specifier with three
% arguments.\\
% "D{"\meta{sep.tex}"}{"\meta{sep.dvi}"}{"\meta{decimal
% places}"}"
%
% \meta{sep.tex} should be a single character, this is used as the
% separator in the {\tt .tex} file. Thus it will usually be `{\tt.}' or
% `{\tt,}'.
%
% \mbox{\meta{sep.dvi}} is used as the separator in the output, this may
% be the same as the first argument, but may be any math-mode
% expression, such as "\cdot". It should be noted that \texttt{dcolumn}
% always uses math mode for the digits as well as the separator.
%
% \meta{decimal places} should be the maximum number of decimal places
% in the column. If this is negative, any number of decimal places can
% be used in the column, and all entries will be centred on 
% (the leading edge of) the 
% separator. Note that this can cause a column to be too wide, compare
% the first two columns in the example below. If this argument is
% positive, the column uses macros equivalent to "\rightdots"
% "\endrightdots" of {\tt array.sty}, otherwise the macros are
% essentially equivalent to "\centerdots" "\endcenterdots".
%
% You may not want to use all three entries in the {\tt array} or {\tt
% tabular} preamble, so you may define your own preamble specifiers
% using "\newcolumntype".
%
% For example we may say:
%
% \noindent"\newcolumntype{d}[1]{D{.}{\cdot}{#1}}"
%
% {\tt d} takes a single argument specifying the number of decimal
% places, and the {\tt .tex} file should use {\tt.}, with $\cdot$ being
% used in the output.
%
% \noindent"\newcolumntype{.}{D{.}{.}{-1}}"
%
% {\tt .} specifies a column of entries to be centred on the~$.$.
%
% \noindent"\newcolumntype{,}{D{,}{,}{2}}"
%
% {\tt ,} specifies takes a column of entries with at most two decimal
% places after a~$,$.
%
% \newcolumntype{d}[1]{D{.}{\cdot}{#1}}
% \newcolumntype{.}{D{.}{.}{-1}}
% \newcolumntype{,}{D{,}{,}{2}}
%
% The following table begins "\begin{tabular}{|d{-1}|d{2}|.|,|}"
%
% \begin{center}
% \begin{tabular}{|d{-1}|d{2}|.|,|}
% 1.2   & 1.2   &1.2    &1,2    \\
% 1.23  & 1.23  &12.5   &300,2  \\
% 1121.2& 1121.2&861.20 &674,29 \\
% 184   & 184   &10     &69     \\
% .4    & .4    &       &,4     \\
%       &       &.4     &
% \end{tabular}
% \end{center}
%
% Note that the first column, which had a negative \meta{decimal places}
% argument is wider than the second column, so that the decimal point
% appears in the middle of the column.
% Also note that this package deals correctly with entries with no
% decimal part, no integer part, and blank entries.
%
% If you have table headings (inserted with "\multicolumn{1}{c}{..}"
% to over-ride the "D" column type) then it may be that neither of the
% above `centred' or `right aligned' forms is quite what you want.
% \begin{center}\small
% \begin{tabular}[t]{|D..{-1}|D..{1}|D..{5.1}|}
%\multicolumn{1}{|c|}{head}&
%\multicolumn{1}{c|}{head}&
%\multicolumn{1}{c|}{head}\\[3pt]
% 1.2  & 1.2  &1.2 \\
% 11212.2& 11212.2&11212.2  \\
% .4    & .4    &.4         
% \end{tabular}
% \hfill
% \begin{tabular}[t]{|D..{-1}|D..{1}|D..{1.1}|}
%\multicolumn{1}{|c|}{wide heading}&
%\multicolumn{1}{c|}{wide heading}&
%\multicolumn{1}{c|}{wide heading}\\[3pt]
% 1.2  & 1.2  &1.2 \\
% .4    & .4    &.4  
% \end{tabular}
% \end{center}
%
% In both of these tables the first column is set with "D{.}{.}{-1}"
% to produce a column centered on the ".", and the second column is
% set with "D{.}{.}{1}" to produce a right aligned column.
%
% The centered column produces columns that are wider than necessary
% to fit in the numbers under a heading as it has to ensure that the
% decimal point is centred. The right aligned column two does not have
% this drawback, but under a wide heading a column of small right
% aligned figures looks a bit odd.
%
% In version v1.03 a third possibility is introduced. The third
% \meta{decimal places} argument may specify \emph{both} the number of
% digits to the left and to the right of the decimal place. The third
% column in the first table above is set with "D{.}{.}{5.1}" and in the
% second  table,  "D{.}{.}{1.1}", to specify
% `five places to the left and one to the right' and `one place to the
% left and% one to the right' respectively.  (You may use `,' or other
% tokens, not necessarily `.' in this argument.) The column of figures
% is then positioned such that a number with the specified numbers of
% digits is centred in the column.
% 
% This notation also enables columns that are centred on the mid-point
% of the separator, rather than its leading edge; for example 
% "D{+}{\,\pm\,}{3,3}" will give nice, symmetric layout of up to three
% digits on either side of a $\pm$ sign.
%
% \StopEventually{}
%
%
% \section{The Macros}
%
%    \begin{macrocode}
%<*package>
%    \end{macrocode}
%
% First we load {\tt array.sty} if it not already loaded.
%    \begin{macrocode}
\RequirePackage{array}
%    \end{macrocode}
%
% The basic ideas behind these macros are explained in the documentation
% for {\tt array.sty}. However they use three
% tricks which may be useful in other contexts.
% \begin{itemize}
% \item The separator is surrounded in extra "{ }", so that it is set
% with "\mathord" spacing, otherwise, for instance a `,' would have
% extra space after it.
% \item The separator is not given its special definition by making it
% active, as this would not work for an entry such as "& .5 &", as the
% first token of an alignment entry is read {\em before\/} the preamble
% part, incase it is an "\omit", in which case the preamble is to be
% omitted. Instead we switch the mathcode to (hex) 8000, which makes the
% token act as if it were active.
% \item Although \verb|\mathcode`.="8000|  makes {\tt.} act as if it
% were active, it is still not allowed in constructions such as
% "\def.{}", even in math-mode, so we have to construct an active
% version of the separator, this is done by making it the uppercase of
% "~", and then using the construct\\
% "\uppercase{\def~}{"\meta{definition}"}".\\
% Note that the \meta{definition} is not uppercased, so the definition
% can refer to the standard, non-active use of the separator.
% \end{itemize}
%
% \begin{macro}{\DC@}
% \changes{v1.03}{1996/02/28}{New feature, after tools/2093}
% Set up uppercase tables as required, and then grab the first part of
% the numerical argument into "\count@".
%    \begin{macrocode}
\def\DC@#1#2#3{%
  \uccode`\~=`#1\relax
  \m@th
  \afterassignment\DC@x\count@#3\relax{#1}{#2}}
%    \end{macrocode}
% \end{macro}
%
% \begin{macro}{\DC@x}
% \changes{v1.03}{1996/02/28}{Macro added}
% If "\count@" is negative, centre on the decimal point. If it is
% positive either "#1" will be empty in which case bad out decimal
% part to the number of digits specified by "\count@" or (new feature
% in v1.03) it is none empty in which case "\count@" contains the
% number of digits to the left of the point, and "#1" contains a junk
% token (probably ".") followed by the number of digits to the right
% of the point. In either of these latter cases, "\DC@right" is used.
%    \begin{macrocode}
\def\DC@x#1\relax#2#3{%
  \ifnum\z@>\count@
    \expandafter\DC@centre
  \else
    \expandafter\DC@right
  \fi
  {#2}{#3}{#1}}
%    \end{macrocode}
% \end{macro}
%
% \begin{macro}{\DC@centre}
% If centering on the decimal point, just need to box up the two halves.
%    \begin{macrocode}
\def\DC@centre#1#2#3{%
  \let\DC@end\DC@endcentre
  \uppercase{\def~}{$\egroup\setbox\tw@=\hbox\bgroup${#2}}%
  \setbox\tw@=\hbox{${\phantom{{#2}}}$}%
  \setbox\z@=\hbox\bgroup$\mathcode`#1="8000 }
%    \end{macrocode}
% \end{macro}
%
% \begin{macro}{\DC@endcentre}
% and then pad out the smaller of the two boxes so there is the same
% amount of stuff either side of the point.
%    \begin{macrocode}
\def\DC@endcentre{$\egroup
    \ifdim \wd\z@>\wd\tw@
      \setbox\tw@=\hbox to\wd\z@{\unhbox\tw@\hfill}%
    \else
      \setbox\z@=\hbox to\wd\tw@{\hfill\unhbox\z@}\fi
    \box\z@\box\tw@}
%    \end{macrocode}
% \end{macro}
%
% \begin{macro}{\DC@right}
% \changes{v1.03}{1996/02/28}{Re-implemented, after tools/2093}
% This deals with both the cases where a specified number of decimal
% places is given.
%    \begin{macrocode}
\def\DC@right#1#2#3{%
  \ifx\relax#3\relax
%    \end{macrocode}
% If "#3" is empty, add "\hfill" to right align the column, and 
% Just set "\DC@rl" to begin a group, so nothing fancy is done with
% the whole number part.
%    \begin{macrocode}
    \hfill
    \let\DC@rl\bgroup
  \else
%    \end{macrocode}
% Otherwise  set "\DC@rl" so that the whole number part is put in a
% box "\count@" times as wide as a digit.
% In order to share code with the other branch, then move "#3" (the
% number of decimal places) into "\count@" throwing away the `.' from
% the user syntax.
% \changes{v1.04}{1996/09/23}{Add \cs{hfill} so integer part
%               is still flush right if no decimal point used.}
%    \begin{macrocode}
    \edef\DC@rl{to\the\count@\dimen@ii\bgroup\hss\hfill}%
    \count@\@gobble#3\relax
  \fi
%    \end{macrocode}
%
%    \begin{macrocode}
  \let\DC@end\DC@endright
%    \end{macrocode}
% Box 2 contains the decimal part, set to "\dimen@" which is
% calculated below to be "\count@" times the width of a digit, plus
% the with of the `decimal point'.
%    \begin{macrocode}
  \uppercase{\def~}{$\egroup\setbox\tw@\hbox to\dimen@\bgroup${#2}}%
   \setbox\z@\hbox{$1$}\dimen@ii\wd\z@
   \dimen@\count@\dimen@ii
   \setbox\z@\hbox{${#2}$}\advance\dimen@\wd\z@
   \setbox\tw@\hbox to\dimen@{}%
%    \end{macrocode}
% Box 0 contains the whole number part, either just at its natural
% size for right aligned columns, or set to (the old value of)
% "\count@" times the width of a digit. "\DC@rl" defined above
% determines the two cases.
%    \begin{macrocode}
   \setbox\z@\hbox\DC@rl$\mathcode`#1="8000 }
%    \end{macrocode}
% \end{macro}
%
% \begin{macro}{\DC@endright}
% \changes{v1.03}{1996/02/28}{Re-implemented, after tools/2093}
% Just finish off the second box, and then put out both boxes.
%    \begin{macrocode}
\def\DC@endright{$\hfil\egroup\box\z@\box\tw@}
%    \end{macrocode}
% \end{macro}
%
% \begin{macro}{D}
% The user interface, define the {\tt D} column to take three arguments.
% For special purposes, you may need to directly access "\DC@" rather
% than the "D" column, eg to get a bold version you could use
%\begin{verbatim}
% \newcolumntype{E}[3]{>{\boldmath\DC@{#1}{#2}{#3}}c<{\DC@end}}
%\end{verbatim}
%    \begin{macrocode}
\newcolumntype{D}[3]{>{\DC@{#1}{#2}{#3}}c<{\DC@end}}
%</package>
%    \end{macrocode}
% \end{macro}
%
%
% \Finale
\endinput

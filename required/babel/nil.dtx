% \iffalse meta-comment
%
% Copyright 2012 Javier Bezos and any individual authors
% listed elsewhere in this file.  All rights reserved.
% 
% This file is part of the Babel system.
% --------------------------------------
% 
% It may be distributed and/or modified under the
% conditions of the LaTeX Project Public License, either version 1.3
% of this license or (at your option) any later version.
% The latest version of this license is in
%   http://www.latex-project.org/lppl.txt
% and version 1.3 or later is part of all distributions of LaTeX
% version 2003/12/01 or later.
% 
% This work has the LPPL maintenance status "maintained".
% 
% The Current Maintainer of this work is Javier Bezos.
% 
% The list of all files belonging to the Babel system is
% given in the file `manifest.bbl. See also `legal.bbl' for additional
% information.
% 
% The list of derived (unpacked) files belonging to the distribution
% and covered by LPPL is defined by the unpacking scripts (with
% extension .ins) which are part of the distribution.
% \fi
% \CheckSum{360}
% \iffalse
%    Tell the \LaTeX\ system who we are and write an entry on the
%    transcript.
%<*dtx>
\ProvidesFile{nil.dtx}
%</dtx>
%<code>\ProvidesLanguage{nil}
%\fi
%\ProvidesFile{nil.dtx}
        [2012/07/06 v0.1 Nil support from the babel system]
%\iffalse
%% File 'nil.dtx'
%% Babel package for LaTeX version 2e
%% Copyright (C) 1989 - 2005
%%           by Johannes Braams, TeXniek
%
%% Please report errors to: J.L. Braams
%%                          babel at braams.cistron.nl
%
%    This file is part of the babel system, it provides the source
%    code for the Nil language definition file.
%<*filedriver>
\documentclass{ltxdoc}
\newcommand*\TeXhax{\TeX hax}
\newcommand*\babel{\textsf{babel}}
\newcommand*\langvar{$\langle \mathit lang \rangle$}
\newcommand*\note[1]{}
\newcommand*\Lopt[1]{\textsf{#1}}
\newcommand*\file[1]{\texttt{#1}}
\begin{document}
 \DocInput{nil.dtx}
\end{document}
%</filedriver>
%\fi
% \GetFileInfo{nil.dtx}
%
%    \section{The `nil' language}
%
%    This `language' does, except setting the hyphenation patterns to
%    nohyphenation.
%
%    For this language currently no special definitions are needed or
%    available.
%
% \StopEventually{}
%
%    The macro |\LdfInit| takes care of preventing that this file is
%    loaded more than once, checking the category code of the
%    \texttt{@} sign, etc.
%    \begin{macrocode}
%<*code>
\LdfInit{nil}{datenil}
%    \end{macrocode}
%
%    When this file is read as an option, i.e. by the |\usepackage|
%    command, \texttt{nil} could be an `unknown' language in which
%    case we have to make it known. 
%    \begin{macrocode}
\ifx\l@nohyphenation\@undefined
   \@nopatterns{nil}
   \adddialect\l@nil0
\else
   \let\l@nil\l@nohyphenation
\fi
%    \end{macrocode}
%    This macro is used to store the values of the hyphenation
%    parameters |\lefthyphenmin| and |\righthyphenmin|.
%    \begin{macrocode}
\providehyphenmins{\CurrentOption}{\z@\z@}
%    \end{macrocode}
%  \end{macro}
%
%    The next step consists of defining commands to switch to (and
%    from) the `nil' language.
% \begin{macro}{\captionnil}
% \begin{macro}{\datenil}
%    \begin{macrocode}
\let\captionsnil\@empty
\let\datenil\@empty
%    \end{macrocode}
% \end{macro}
% \end{macro}
%  
%    The macro |\ldf@finish| takes care of looking for a
%    configuration file, setting the main language to be switched on
%    at |\begin{document}| and resetting the category code of
%    \texttt{@} to its original value.
%    \begin{macrocode}
\ldf@finish{nil}
%</code>
%    \end{macrocode}
%
% \Finale
%%
%% \CharacterTable
%%  {Upper-case    \A\B\C\D\E\F\G\H\I\J\K\L\M\N\O\P\Q\R\S\T\U\V\W\X\Y\Z
%%   Lower-case    \a\b\c\d\e\f\g\h\i\j\k\l\m\n\o\p\q\r\s\t\u\v\w\x\y\z
%%   Digits        \0\1\2\3\4\5\6\7\8\9
%%   Exclamation   \!     Double quote  \"     Hash (number) \#
%%   Dollar        \$     Percent       \%     Ampersand     \&
%%   Acute accent  \'     Left paren    \(     Right paren   \)
%%   Asterisk      \*     Plus          \+     Comma         \,
%%   Minus         \-     Point         \.     Solidus       \/
%%   Colon         \:     Semicolon     \;     Less than     \<
%%   Equals        \=     Greater than  \>     Question mark \?
%%   Commercial at \@     Left bracket  \[     Backslash     \\
%%   Right bracket \]     Circumflex    \^     Underscore    \_
%%   Grave accent  \`     Left brace    \{     Vertical bar  \|
%%   Right brace   \}     Tilde         \~}
%%
\endinput

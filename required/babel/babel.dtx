% \iffalse meta-comment
%
% Copyright 1989-2008 Johannes L. Braams and any individual authors
% listed elsewhere in this file.  All rights reserved.
% 
% This file is part of the Babel system.
% --------------------------------------
% 
% It may be distributed and/or modified under the
% conditions of the LaTeX Project Public License, either version 1.3
% of this license or (at your option) any later version.
% The latest version of this license is in
%   http://www.latex-project.org/lppl.txt
% and version 1.3 or later is part of all distributions of LaTeX
% version 2003/12/01 or later.
% 
% This work has the LPPL maintenance status "maintained".
% 
% The Current Maintainer of this work is Johannes Braams.
% 
% The list of all files belonging to the Babel system is
% given in the file `manifest.bbl. See also `legal.bbl' for additional
% information.
% 
% The list of derived (unpacked) files belonging to the distribution
% and covered by LPPL is defined by the unpacking scripts (with
% extension .ins) which are part of the distribution.
% \fi
% \CheckSum{3843}
%%
% \def\filename{babel.dtx}
% \let\thisfilename\filename
%
%\iffalse
% \changes{babel~3.5g}{1996/10/10}{We need at least \LaTeX\ from
%    December 1994}
% \changes{babel~3.6k}{1999/03/18}{We need at least \LaTeX\ from
%    June 1998}
%    \begin{macrocode}
%<+package>\NeedsTeXFormat{LaTeX2e}[2005/12/01]
%    \end{macrocode}
%
%% File 'babel.dtx'
%\fi
%%\ProvidesFile{babel.dtx}[2008/03/16 v3.8j The Babel package]
%\iffalse
%
% Babel DOCUMENT-STYLE option for LaTeX version 2.09 or plain TeX;
%% Babel package for LaTeX2e.
%
%% Copyright (C) 1989 -- 2008 by Johannes Braams,
%%                            TeXniek
%%                            all rights reserved.
%
%% Please report errors to: J.L. Braams
%%                          babel at braams.xs4all.nl
%<*filedriver>
\documentclass{ltxdoc}
\usepackage{supertabular}
\font\manual=logo10 % font used for the METAFONT logo, etc.
\newcommand*\MF{{\manual META}\-{\manual FONT}}
\newcommand*\TeXhax{\TeX hax}
\newcommand*\babel{\textsf{babel}}
\newcommand*\Babel{\textsf{Babel}}
\newcommand*\m[1]{\mbox{$\langle$\it#1\/$\rangle$}}
\newcommand*\langvar{\m{lang}}
\newcommand*\note[1]{}
\newcommand*\bsl{\protect\bslash}
\newcommand*\Lopt[1]{\textsf{#1}}
\newcommand*\Lenv[1]{\textsf{#1}}
\newcommand*\file[1]{\texttt{#1}}
\newcommand*\cls[1]{\texttt{#1}}
\newcommand*\pkg[1]{\texttt{#1}}
\begin{document}
 \DocInput{babel.dtx}
\end{document}
%</filedriver>
% \changes{babel~3.7a}{1997/04/16}{Make multiple loading of
%    \file{babel.def} impossible} 
%    \begin{macrocode}
%<*core>
\ifx\bbl@afterfi\@undefined
\else
  \bbl@afterfi\endinput
\fi
%</core>
%    \end{macrocode}
%<*dtx>
\ProvidesFile{babel.dtx}
%</dtx>
%\fi
%
% \GetFileInfo{babel.dtx}
%
% \changes{babel~2.0a}{1990/04/02}{Added text about \file{german.sty}}
% \changes{babel~2.0b}{1990/04/18}{Changed order of code to prevent
%    plain \TeX from seeing all of it}
% \changes{babel~2.1}{1990/04/24}{Modified user interface,
%    \cs{langTeX} no longer necessary}
% \changes{babel~2.1a}{1990/05/01}{Incorporated Nico's comments}
% \changes{babel~2.1b}{1990/05/01}{rename \cs{language} to
%    \cs{current@language}}
% \changes{babel~2.1c}{1990/05/22}{abstract for report fixed, missing
%    \texttt{\}}, found by Nicolas Brouard}
% \changes{babel~2.1d}{1990/07/04}{Missing right brace in definition of
%    abstract environment, found by Werenfried Spit}
% \changes{babel~2.1e}{1990/07/16}{Incorporated more comments from
%    Nico}
% \changes{babel~2.2}{1990/07/17}{Renamed \cs{newlanguage} to
%    \cs{addlanguage}}
% \changes{babel~2.2a}{1990/08/27}{Modified the documentation
%    somewhat}
% \changes{babel~3.0}{1991/04/23}{Moved part of the code to hyphen.doc
%    in preparation for \TeX~3.0}
% \changes{babel~3.0a}{1991/05/21}{Updated comments in various places}
% \changes{babel~3.0b}{1991/05/25}{Removed some problems in change log}
% \changes{babel~3.0c}{1991/07/15}{Renamed \file{babel.sty} and
%    \file{latexhax.sty} to \file{.com}}
% \changes{babel~3.1}{1991/10/31}{Added the support for active
%    characters and for extending a macro}
% \changes{babel~3.1}{1991/11/05}{Removed the need for
%    \file{latexhax}}
% \changes{babel~3.2}{1991/11/10}{Some Changes by br}
% \changes{babel~3.2a}{1992/02/15}{Fixups of the code and
%    documentation}
% \changes{babel~3.3}{1993/07/06}{Included driver file, and prepared
%    for distribution}
% \changes{babel~3.4}{1994/01/30}{Updated for \LaTeXe}
% \changes{babel~3.4}{1994/02/28}{Added language definition file for
%    bahasa}
% \changes{babel~3.4b}{1994/05/18}{Added a small driver to be able to
%    process just this file}
% \changes{babel~3.5a}{1995/02/03}{Provided common code to handle the
%    active double quote}
% \changes{babel~3.5c}{1995/06/14}{corrected a few typos (PR1652)}
% \changes{babel~3.5d}{1995/07/02}{Merged glyphs.dtx into this file}
% \changes{babel~3.5f}{1995/07/13}{repaired a typo}
% \changes{babel~3.5f}{1996/01/09}{replaced \cs{tmp}, \cs{bbl@tmp} and
%    \cs{bbl@temp} with \cs{bbl@tempa}}
% \changes{babel~3.5g}{1996/07/09}{replaced \cs{undefined} with
%    \cs{@undefined} to be consistent with \LaTeX}
% \changes{babel~3.7d}{1999/05/05}{Fixed a few typos in \cs{changes}
%    entries which made typesetting the code impossible}
% \changes{babel~3.7h}{2001/03/01}{Added a number of missing comment
%    characters which caused spurious white space}
% \changes{babel~3.8e}{2005/03/24}{Many enhancements to the text by
%    Andrew Young} 
%
% \title {Babel, a multilingual package for use with \LaTeX's standard
%    document classes\thanks{During the development ideas from Nico
%    Poppelier, Piet van Oostrum and many others have been used.
%    Bernd Raichle has provided many helpful suggestions.}}
%
% \author{Johannes Braams\\
%         Kersengaarde 33\\
%         2723 BP Zoetermeer\\
%         The Netherlands\\
%         \texttt{babel\char64 braams.xs4all.nl}}
%
% \date{Printed \today}
%
% \maketitle
%
%  \begin{abstract}
%    The standard distribution of \LaTeX\ contains a number of
%    document classes that are meant to be used, but also serve as
%    examples for other users to create their own document classes.
%    These document classes have become very popular among \LaTeX\
%    users. But it should be kept in mind that they were designed for
%    American tastes and typography. At one time they contained a
%    number of hard-wired texts. This report describes \babel{}, a
%    package that makes use of the new capabilities of \TeX\ version 3
%    to provide an environment in which documents can be typeset in
%    a non-American language, or in more than one language.
%  \end{abstract}
%
%  \begin{multicols}{2}
%    \tableofcontents
%  \end{multicols}
%
% \section{The user interface}\label{U-I}
%
%    The user interface of this package is quite simple. It consists
%    of a set of commands that switch from one language to another, and
%    a set of commands that deal with shorthands. It is also possible
%    to find out what the current language is.
%
%  \DescribeMacro{\selectlanguage}
%    When a user wants to switch from one language to another he can
%    do so using the macro |\selectlanguage|. This macro takes the
%    language, defined previously by a language definition file, as
%    its argument. It calls several macros that should be defined in
%    the language definition files to activate the special definitions
%    for the language chosen.
%
%  \DescribeEnv{otherlanguage}
%    The environment \Lenv{otherlanguage} does basically the same as
%    |\selectlanguage|, except the language change is local to the
%    environment. This environment is required for intermixing
%    left-to-right typesetting with right-to-left typesetting.
%    The language to switch to is specified as an
%    argument to |\begin{otherlanguage}|.
%
%  \DescribeMacro{\foreignlanguage}
%    The command |\foreignlanguage| takes two arguments; the second
%    argument is a phrase to be typeset according to the rules of the
%    language named in its first argument. This command only switches
%    the extra definitions and the hyphenation rules for the language,
%    \emph{not} the names and dates.
%
%  \DescribeEnv{otherlanguage*}
%    In the environment \Lenv{otherlanguage*} only the typesetting
%    is done according to the rules of the other language, but the
%    text-strings such as `figure', `table', etc. are left as they
%    were set outside this environment.
%
%  \DescribeEnv{hyphenrules}
%    The environment \Lenv{hyphenrules} can be used to select
%    \emph{only} the hyphenation rules to be used. This can for
%    instance be used to select `nohyphenation', provided that in
%    \file{language.dat} the `language' nohyphenation is defined by
%    loading \file{serohyph.tex}.
%
%  \DescribeMacro{\languagename}
%    The control sequence |\languagename| contains the name of the
%    current language.
%
%  \DescribeMacro{\iflanguage}
%    If more than one language is used, it might be necessary to know
%    which language is active at a specific time. This can be checked
%    by a call to |\iflanguage|. This macro takes three arguments.
%    The first argument is the name of a language; the second and
%    third arguments are the actions to take if the result of the test
%    is \texttt{true} or \texttt{false} respectively.
%
%  \DescribeMacro{\useshorthands}
%    The command |\useshorthands| initiates the definition of
%    user-defined shorthand sequences. It has one argument, the
%    character that starts these personal shorthands.
%
%  \DescribeMacro{\defineshorthand}
%     The command |\defineshorthand| takes two arguments: the first
%     is a one- or two-character shorthand sequence, and the second is
%     the code the shorthand should expand to.
%
%  \DescribeMacro{\aliasshorthand}
%    The command |\aliasshorthand| can be used to let another
%    character perform the same functions as the default shorthand
%    character. If one prefers for example to use the character |/|
%    over |"| in typing polish texts, this can be achieved by entering
%    |\aliasshorthand{"}{/}|. \emph{Please note} that the substitute
%    shorthand character must have been declared in the preamble of
%    your document, using a command such as |\useshorthands{/}| in this
%    example.
%
%  \DescribeMacro{\languageshorthands}
%     The command |\languageshorthands| can be used to switch the
%     shorthands on the language level. It takes one argument, the
%     name of a language. Note that for this to work the language
%     should have been specified as an option when loading the \babel\
%     package.
%
%  \DescribeMacro{\shorthandon}
%  \DescribeMacro{\shorthandoff}
%    It is sometimes necessary to switch a shorthand
%    character off temporarily, because it must be used in an
%    entirely different way. For this purpose, the user commands
%    |\shorthandoff| and |\shorthandon| are provided. They each take a
%    list of characters as their arguments. The command |\shorthandoff|
%    sets the |\catcode| for each of the characters in its argument to
%    other (12); the command |\shorthandon| sets the |\catcode| to
%    active (13). Both commands only work on `known'
%    shorthand characters. If a character is not known to be a
%    shorthand character its category code will be left unchanged.
%
%  \DescribeMacro{\languageattribute}
%    This is a user-level command, to be used in the preamble of a
%    document (after |\usepackage[...]{babel}|), that declares which
%    attributes are to be used for a given language. It takes two
%    arguments: the first is the name of the language; the second,
%    a (list of) attribute(s) to used.
%    The command checks whether the language is known in this document
%    and whether the attribute(s) are known for this language.
%
% \subsection{Languages supported by \Babel}
%
%    In the following table all the languages supported by \Babel\ are
%    listed, together with the names of the options with which you can
%    load \babel\ for each language.
%
%    \begin{center}
%      \tablehead{Language & Option(s)\\\hline}
%      \tabletail{\hline}
%      \begin{supertabular}{l p{8cm}}
%        Afrikaans  & afrikaans\\
%        Bahasa     & bahasa, indonesian, indon, bahasai,
%                     bahasam, malay, meyalu\
%        Basque     & basque\\
%        Breton     & breton\\
%        Bulgarian  & bulgarian\\
%        Catalan    & catalan\\
%        Croatian   & croatian\\
%        Czech      & czech\\
%        Danish     & danish\\
%        Dutch      & dutch\\
%        English    & english, USenglish, american, UKenglish,
%                     british, canadian, australian, newzealand\\
%        Esperanto  & esperanto\\
%        Estonian   & estonian\\
%        Finnish    & finnish\\
%        French     & french, francais, canadien, acadian\\
%        Galician   & galician\\
%        German     & austrian, german, germanb, ngerman, naustrian\\
%        Greek      & greek, polutonikogreek \\
%        Hebrew     & hebrew \\
%        Hungarian  & magyar, hungarian\\
%        Icelandic  & icelandic \\
%        Interlingua & interlingua \\
%        Irish Gaelic & irish\\
%        Italian    & italian\\
%^^A        Kannada    & kannada \\
%        Latin      & latin \\
%        Lower Sorbian & lowersorbian\\
%^^A        Devnagari  & nagari \\
%        North Sami & samin \\
%        Norwegian  & norsk, nynorsk\\
%        Polish     & polish\\
%        Portuguese & portuges, portuguese, brazilian, brazil\\
%        Romanian   & romanian\\
%        Russian    & russian\\
%^^A        Sanskrit   & sanskrit\\
%        Scottish Gaelic & scottish\\
%        Spanish    & spanish\\
%        Slovakian  & slovak\\
%        Slovenian  & slovene\\
%        Swedish    & swedish\\
%        Serbian    & serbian\\
%^^A        Tamil      & tamil \\
%        Turkish    & turkish\\
%        Ukrainian  & ukrainian\\
%        Upper Sorbian & uppersorbian\\
%        Welsh      & welsh\\
%      \end{supertabular}
%    \end{center}
%
%    For some languages \babel\ supports the options
%    \Lopt{activeacute} and \Lopt{activegrave}; for typestting Russian
%    texts, \babel\ knows about the options \Lopt{LWN} and \Lopt{LCY}
%    to specify the fontencoding of the cyrillic font used. Currently
%    only \Lopt{LWN} is supported.
%
% \subsection{Workarounds}
%
%    If you use the document class \cls{book} \emph{and} you use
%    |\ref| inside the argument of |\chapter|,
%    \LaTeX\ will keep complaining about an undefined
%    label. The reason is that the argument of |\ref| is passed through
%    |\uppercase| at some time during processing. To prevent such
%    problems, you could revert to using uppercase labels, or you can
%    use |\lowercase{\ref{foo}}| inside the argument of |\chapter|.
%
% \section{Changes for \LaTeXe}
%
%    With the advent of \LaTeXe\ the interface to \babel\ in the
%    preamble of the document has changed. With \LaTeX2.09 one used to
%    call up the \babel\ system with a line such as:
%
%\begin{verbatim}
%\documentstyle[dutch,english]{article}
%\end{verbatim}
%
%    which would tell \LaTeX\ that the document would be written in
%    two languages, Dutch and English, and that English would be the
%    first language in use.
%
%    The \LaTeXe\ way of providing the same information is:
%
%\begin{verbatim}
%\documentclass{article}
%\usepackage[dutch,english]{babel}
%\end{verbatim}
%
%    or, making \Lopt{dutch} and \Lopt{english} global options in
%    order to let other packages detect and use them:
%
%\begin{verbatim}
%\documentclass[dutch,english]{article}
%\usepackage{babel}
%\usepackage{varioref}
%\end{verbatim}
%
%    In this last example, the package \texttt{varioref} will also see
%    the options and will be able to use them.
%
% \section{Changes in \Babel\ version 3.7}
%
%    In \Babel\ version 3.7 a number of bugs that were found in
%    version~3.6 are fixed. Also a number of changes and additions
%    have occurred:
%    \begin{itemize}
%    \item Shorthands are expandable again. The disadvantage is that
%      one has to type |'{}a| when the acute accent is used as a
%      shorthand character. The advantage is that a number of other
%      problems (such as the breaking of ligatures, etc.) have
%      vanished.
%    \item Two new commands, |\shorthandon| and |\shorthandoff| have
%      been introduced to enable to temporarily switch off one or more
%      shorthands.
%^^A    \item Support for typesetting Sanskrit in transliteration is now
%^^A      available, thanks to Jun Takashima.
%^^A    \item Support for typesetting Kannada, Devnagari and Tamil is now
%^^A      available thanks to Jun Takashima.
%    \item Support for typesetting Greek has been enhanced. Code from
%      the \pkg{kdgreek} package (suggested by the author) was added
%      and |\greeknumeral| has been added.
%    \item Support for typesetting Basque is now available thanks to
%      Juan Aguirregabiria.
%    \item Support for typesetting Serbian with Latin script is now
%      available thanks to Dejan Muhamedagi\'{c} and Jankovic
%      Slobodan.
%    \item Support for typesetting Hebrew (and potential support for
%      typesetting other right-to-left written languages) is now
%      available thanks to Rama Porrat and Boris Lavva.
%    \item Support for typesetting Bulgarian is now available thanks to
%      Georgi Boshnakov.
%    \item Support for typesetting Latin is now available, thanks to
%      Claudio Beccari and Krzysztof Konrad \.Zelechowski.
%    \item Support for typesetting North Sami is now available, thanks
%      to Regnor Jernsletten.
%    \item The options \Lopt{canadian}, \Lopt{canadien} and
%      \Lopt{acadien} have been added for Canadian English and French
%      use.
%    \item A language attribute has been added to the |\mark...|
%      commands in order to make sure that a Greek header line comes
%      out right on the last page before a language switch.
%    \item Hyphenation pattern files are now read \emph{inside a
%      group}; therefore any changes a pattern file needs to make to
%      lowercase codes, uppercase codes, and category codes are kept
%      local to that group. If they are needed for the language, these
%      changes will need to be repeated and stored in |\extras...|
%    \item The concept of language attributes is introduced. It is
%      intended to give the user some control over the
%      features a language-definition file provides. Its
%      first use is for the Greek language, where the user can choose
%      the  $\pi o\lambda\upsilon\tau o\nu\kappa\acute{o}$
%      (``Polutoniko'' or multi-accented) Greek way of typesetting
%      texts. These attributes will possibly find wider use in future
%      releases.
%    \item The environment \Lenv{hyphenrules} is introduced.
%    \item The syntax of the file \file{language.dat} has been
%      extended to allow (optionally) specifying the font
%      encoding to be used while processing the patterns file.
%    \item The command |\providehyphenmins| should now be used in
%      language definition files in order to be able to keep any
%      settings provided by the pattern file.
%    \end{itemize}
%
% \section{Changes in \Babel\ version 3.6}
%
%    In \Babel\ version 3.6 a number of bugs that were found in
%    version~3.5 are fixed. Also a number of changes and additions
%    have occurred:
%    \begin{itemize}
%    \item A new environment \Lenv{otherlanguage*} is introduced. it
%      only switches the `specials', but leaves the `captions'
%      untouched.
%    \item The shorthands are no longer fully expandable. Some
%      problems could only be solved by peeking at the token following
%      an active character. The advantage is that |'{}a| works as
%      expected for languages that have the |'| active.
%    \item Support for typesetting french texts is much enhanced; the
%      file \file{francais.ldf} is now replaced by \file{frenchb.ldf}
%      which is maintained by Daniel Flipo.
%    \item Support for typesetting the russian language is again
%      available. The language definition file was originally
%      developed by Olga Lapko from CyrTUG. The fonts needed to
%      typeset the russian language are now part of the \babel\
%      distribution. The support is not yet up to the level which is
%      needed according to Olga, but this is a start.
%    \item Support for typesetting greek texts is now also
%      available. What is offered in this release is a first attempt;
%      it will be enhanced later on by Yannis Haralambous.
%    \item in \babel\ 3.6j some hooks have been added for the
%      development of support for Hebrew typesetting.
%    \item Support for typesetting texts in Afrikaans (a variant of
%      Dutch, spoken in South Africa) has been added to
%      \file{dutch.ldf}.
%    \item Support for typesetting Welsh texts is now available.
%    \item A new command |\aliasshorthand| is introduced. It seems
%      that in Poland various conventions are used to type the
%      necessary Polish letters. It is now possible to use the
%      character~|/| as a shorthand character instead of the
%      character~|"|, by issuing the command |\aliasshorthand{"}{/}|.
%    \item The shorthand mechanism now deals correctly with characters
%      that are already active.
%    \item Shorthand characters are made active at |\begin{document}|,
%      not earlier. This is to prevent problems with other packages.
%    \item A \emph{preambleonly} command |\substitutefontfamily| has
%      been added to create \file{.fd} files on the fly when the font
%      families of the Latin text differ from the families used for
%      the Cyrillic or Greek parts of the text.
%    \item Three new commands |\LdfInit|, |\ldf@quit| and
%      |\ldf@finish| are introduced that perform a number of standard
%      tasks.
%    \item In babel 3.6k the language Ukrainian has been added and the
%      support for Russian typesetting has been adapted to the package
%      'cyrillic' to be released with the December 1998 release of
%      \LaTeXe.
%    \end{itemize}
%
% \section{Changes in \Babel\ version 3.5}
%
%    In \Babel\ version 3.5 a lot of changes have been made when
%    compared with the previous release. Here is a list of the most
%    important ones:
%    \begin{itemize}
%    \item the selection of the language is delayed until
%      |\begin{document}|, which means you must
%      add appropriate |\selectlanguage| commands if you include
%      |\hyphenation| lists in the preamble of your document.
%    \item \babel\ now has a \Lenv{language} environment and a new
%      command |\foreignlanguage|;
%    \item the way active characters are dealt with is completely
%      changed. They are called `shorthands'; one can have three
%      levels of shorthands: on the user level, the language level, and
%      on `system level'. A consequence of the new way of handling
%      active characters is that they are now written to auxiliary
%      files `verbatim';
%    \item A language change now also writes information in the
%      \file{.aux} file, as the change might also affect typesetting
%      the table of contents. The consequence is that an .aux file
%      generated by a LaTeX format with babel preloaded gives errors
%      when read with a LaTeX format without babel; but I think this
%      probably doesn't occur;
%    \item \babel\ is now compatible with the \pkg{inputenc} and
%      \pkg{fontenc} packages;
%    \item the language definition files now have a new extension,
%      \file{ldf};
%    \item the syntax of the file \file{language.dat} is extended to
%      be compatible with the \pkg{french} package by Bernard Gaulle;
%    \item each language definition file looks for a configuration
%      file which has the same name, but the extension \file{.cfg}. It
%    can contain any valid \LaTeX\ code.
%    \end{itemize}
%
% \section{The interface between the core of \babel{} and the language
%    definition files}
%
%    In the core of the \babel{} system, several macros are defined
%    for use in language definition files. Their purpose
%    is to make a new language known.
%
%  \DescribeMacro{\addlanguage}
%    The macro |\addlanguage| is a non-outer version of the macro
%    |\newlanguage|, defined in \file{plain.tex} version~3.x. For
%    older versions of \file{plain.tex} and \file{lplain.tex} a
%    substitute definition is used.
%
%  \DescribeMacro{\adddialect}
%    The macro |\adddialect| can be used when two
%    languages can (or must) use the same hyphenation
%    patterns. This can also be useful for
%    languages for which no patterns are preloaded in the format. In
%    such cases the default behaviour of the \babel{} system is to
%    define this language as a `dialect' of the language for which the
%    patterns were loaded as |\language0|.
%
%    The language definition files must conform to a number of
%    conventions, because these files have to fill
%    in the gaps left by the common code in \file{babel.def}, i.\,e.,
%    the definitions of the macros that produce texts.  Also the
%    language-switching possibility which has been built into the
%    \babel{} system has its implications.
%
%    The following assumptions are made:
%   \begin{itemize}
%    \item Some of the language-specific definitions might be used by
%    plain \TeX\ users, so the files have to be coded so that they
%    can be read by both \LaTeX\ and plain \TeX. The current
%    format can be checked by looking at the value of the macro
%    |\fmtname|.
%
%    \item The common part of the \babel{} system redefines a number
%    of macros and environments (defined previously in the document
%    style) to put in the names of macros that replace the previously
%    hard-wired texts.  These macros have to be defined in the
%    language definition files.
%
%    \item The language definition files define five macros, used to
%    activate and deactivate the language-specific definitions.  These
%    macros are |\|\langvar|hyphenmins|, |\captions|\langvar,
%    |\date|\langvar, |\extras|\langvar\ and |\noextras|\langvar; where
%    \langvar\ is either the name of the language definition file or
%    the name of the \LaTeX\ option that is to be used. These
%    macros and their functions are discussed below.
%
%    \item When a language definition file is loaded, it can define
%    |\l@|\langvar\ to be a dialect of |\language0| when
%    |\l@|\langvar\ is undefined.
%
%    \item The language definition files can be read in the preamble of
%    the document, but also in the middle of document processing. This
%    means that they have to function independently of the current
%    |\catcode| of the \texttt{@}~sign.
%   \end{itemize}
%
%  \DescribeMacro{\providehyphenmins}
%    The macro |\providehyphenmins| should be used in the language
%    definition files to set the |\lefthyphenmin| and
%    |\righthyphenmin|. This macro will check whether these parameters
%    were provided by the hyphenation file before it takes any action.
%
%  \DescribeMacro{\langhyphenmins}
%    The macro |\|\langvar|hyphenmins| is used to store the values of
%    the |\lefthyphenmin| and |\righthyphenmin|.
%
%  \DescribeMacro{\captionslang}
%    The macro |\captions|\langvar\ defines the macros that
%    hold the texts to replace the original hard-wired texts.
%
%  \DescribeMacro{\datelang}
%    The macro |\date|\langvar\ defines |\today| and
%
%  \DescribeMacro{\extraslang}
%    The macro |\extras|\langvar\ contains all the extra definitions
%    needed for a specific language.
%
%  \DescribeMacro{\noextraslang}
%    Because we want to let the user switch
%    between languages, but we do not know what state \TeX\ might be in
%    after the execution of |\extras|\langvar, a macro that brings
%    \TeX\ into a predefined state is needed. It will be no surprise
%    that the name of this macro is |\noextras|\langvar.
%
%  \DescribeMacro{\bbl@declare@ttribute}
%    This is a command to be used in the language definition files for
%    declaring a language attribute. It takes three arguments: the
%    name of the language, the attribute to be defined, and the code
%    to be executed when the attribute is to be used.
%
%  \DescribeMacro{\main@language}
%    To postpone the activation of the definitions needed for a
%    language until the beginning of a document, all language
%    definition files should use |\main@language| instead of
%    |\selectlanguage|. This will just store the name of the language,
%    and the proper language will be activated at the start of the
%    document.
%
%  \DescribeMacro{\ProvidesLanguage}
%    The macro |\ProvidesLanguage| should be used to identify the
%    language definition files. Its syntax is similar to the syntax
%    of the \LaTeX\ command |\ProvidesPackage|.
%
%  \DescribeMacro{\LdfInit}
%    The macro |\LdfInit| performs a couple of standard checks that
%    must be made at the beginning of a language definition file,
%    such as checking the category code of the @-sign, preventing
%    the \file{.ldf} file from being processed twice, etc.
%
%  \DescribeMacro{\ldf@quit}
%    The macro |\ldf@quit| does work needed
%    if a \file{.ldf} file was processed
%    earlier. This includes resetting the category code
%    of the @-sign, preparing the language to be activated at
%    |\begin{document}| time, and ending the input stream.
%
%  \DescribeMacro{\ldf@finish}
%    The macro |\ldf@finish| does work needed
%    at the end of each \file{.ldf} file. This
%    includes resetting the category code of the @-sign,
%    loading a local configuration file, and preparing the language
%    to be activated at |\begin{document}| time.
%
%  \DescribeMacro{\loadlocalcfg}
%    After processing a language definition file,
%    \LaTeX\ can be instructed to load a local configuration
%    file. This file can, for instance, be used to add strings to
%    |\captions|\langvar\ to support local document
%    classes. The user will be informed that this
%    configuration file has been loaded. This macro is called by
%    |\ldf@finish|.
%
%  \DescribeMacro{\substitutefontfamily}
%    This command takes three arguments, a font encoding and two font
%    family names. It creates a font description file for the first
%    font in the given encoding. This \file{.fd} file will instruct
%    \LaTeX\ to use a font from the second family when a font from the
%    first family in the given encoding seems to be needed.
%
% \subsection{Support for active characters}
%
%    In quite a number of language definition files, active characters
%    are introduced. To facilitate this, some support macros are
%    provided.
%
% \DescribeMacro{\initiate@active@char}
%    The internal macro |\initiate@active@char| is used in language
%    definition files to instruct \LaTeX\ to give a character the
%    category code `active'. When a character has been made active it
%    will remain that way until the end of the document. Its
%    definition may vary.
%
% \DescribeMacro{\bbl@activate}
% \DescribeMacro{\bbl@deactivate}
%    The command |\bbl@activate| is used to change the way an active
%    character expands. |\bbl@activate| `switches on' the active
%    behaviour of the character. |\bbl@deactivate| lets the active
%    character expand to its former (mostly) non-active self.
%
% \DescribeMacro{\declare@shorthand}
%    The macro |\declare@shorthand| is used to define the various
%    shorthands. It takes three arguments: the name for the collection
%    of shorthands this definition belongs to; the character
%    (sequence) that makes up the shorthand, i.e.\ |~| or |"a|; and the
%    code to be executed when the shorthand is encountered.
%
% \DescribeMacro{\bbl@add@special}
% \DescribeMacro{\bbl@remove@special}
%    The \TeX book states: ``Plain \TeX\ includes a macro called
%    |\dospecials| that is
%    essentially a set macro, representing the set of all characters
%    that have a special category code.'' \cite[p.~380]{DEK} It is
%    used to set text `verbatim'.  To make this work if more
%    characters get a special category code, you have to add this
%    character to the macro |\dospecial|.  \LaTeX\ adds another macro
%    called |\@sanitize| representing the same character set, but
%    without the curly braces.  The macros
%    |\bbl@add@special|\meta{char} and
%    |\bbl@remove@special|\meta{char} add and remove the character
%    \meta{char} to these two sets.
%
% \subsection{Support for saving macro definitions}
%
%    Language definition files may want to \emph{re}define macros that
%    already exist. Therefor a mechanism for saving (and restoring)
%    the original definition of those macros is provided. We provide
%    two macros for this\footnote{This mechanism was introduced by
%    Bernd Raichle.}.
%
% \DescribeMacro{\babel@save} To save the current meaning of any
%    control sequence, the macro |\babel@save| is provided. It takes
%    one argument, \meta{csname}, the control sequence for which the
%    meaning has to be saved.
%
% \DescribeMacro{\babel@savevariable} A second macro is provided to
%    save the current value of a variable.  In this context, anything
%    that is allowed after the |\the| primitive is considered to be a
%    variable. The macro takes one argument, the \meta{variable}.
%
%    The effect of the preceding macros is to append a piece of code
%    to the current definition of |\originalTeX|. When
%    |\originalTeX| is expanded, this code restores the previous
%    definition of the control sequence or the previous value of the
%    variable.
%
% \subsection{Support for extending macros}
%
% \DescribeMacro{\addto}
%    The macro |\addto{|\meta{control sequence}|}{|\meta{\TeX\
%    code}|}| can be used to extend the definition of a macro. The
%    macro need not be defined. This macro can, for instance, be used
%    in adding instructions to a macro like |\extrasenglish|.
%
% \subsection{Macros common to a number of languages}
%
% \DescribeMacro{\allowhyphens}
%    In a couple of European languages compound words are used. This
%    means that when \TeX\ has to hyphenate such a compound word, it
%    only does so at the `\texttt{-}' that is used in such words. To
%    allow hyphenation in the rest of such a compound word, the macro
%    |\allowhyphens| can be used.
%
% \DescribeMacro{\set@low@box}
%    For some languages, quotes need to be lowered to the baseline. For
%    this purpose the macro |\set@low@box| is available. It takes one
%    argument and puts that argument in an |\hbox|, at the
%    baseline. The result is available in |\box0| for further
%    processing.
%
% \DescribeMacro{\save@sf@q}
%    Sometimes it is necessary to preserve the |\spacefactor|.  For
%    this purpose the macro |\save@sf@q| is available. It takes one
%    argument, saves the current spacefactor, executes the argument,
%    and restores the spacefactor.
%
% \DescribeMacro{\bbl@frenchspacing}
% \DescribeMacro{\bbl@nonfrenchspacing}
%    The commands |\bbl@frenchspacing| and |\bbl@nonfrenchspacing| can
%    be used to properly switch French spacing on and off.
%
% \section{Compatibility with \file{german.sty}}\label{l-h}
%
%    The file \file{german.sty} has been
%    one of the sources of inspiration for the \babel{}
%    system. Because of this I wanted to include \file{german.sty} in
%    the \babel{} system.  To be able to do that I had to allow for
%    one incompatibility: in the definition of the macro
%    |\selectlanguage| in \file{german.sty} the argument is used as the
%    {$\langle \it number \rangle$} for an |\ifcase|. So in this case
%    a call to |\selectlanguage| might look like
%    |\selectlanguage{\german}|.
%
%    In the definition of the macro |\selectlanguage| in
%    \file{babel.def} the argument is used as a part of other
%    macronames, so a call to |\selectlanguage| now looks like
%    |\selectlanguage{german}|.  Notice the absence of the escape
%    character.  As of version~3.1a of \babel{} both syntaxes are
%    allowed.
%
%    All other features of the original \file{german.sty} have been
%    copied into a new file, called \file{germanb.sty}\footnote{The
%    `b' is added to the name to distinguish the file from Partls'
%    file.}.
%
%    Although the \babel{} system was developed to be used with
%    \LaTeX, some of the features implemented in the language
%    definition files might be needed by plain \TeX\ users. Care has
%    been taken that all files in the system can be processed by plain
%    \TeX.
%
% \section{Compatibility with \file{ngerman.sty}}
%
%    When used with the options \Lopt{ngerman} or \Lopt{naustrian},
%    \babel{} will provide all features of the package \pkg{ngerman}.
%    There is however one exception:  The commands for special
%    hyphenation of double consonants (|"ff| etc.) and ck (|"ck|),
%    which are no longer required with the new German orthography, are
%    undefined. With the \pkg{ngerman} package, however, these
%    commands will generate appropriate warning messages only.
%
% \section{Compatibility with the \pkg{french} package}
%
%    It has been reported to me that the package \pkg{french} by
%    Bernard Gaulle (\texttt{gaulle@idris.fr}) works
%    together with \babel. On the other hand, it seems \emph{not} to
%    work well together with a lot of other packages. Therefore I have
%    decided to no longer load \file{french.ldf} by default. Instead,
%    when you want to use the package by Bernard Gaulle, you will have
%    to request it specifically, by passing either \Lopt{frenchle} or
%    \Lopt{frenchpro} as an option to \babel.
%
%\StopEventually{%
% \clearpage
% \let\filename\thisfilename
% \section{Conclusion}
%
%    A system of document options has been presented that enable the
%    user of \LaTeX\ to adapt the standard document classes of \LaTeX\
%    to the language he or she prefers to use. These options offer the
%    possibility of switching between languages in one document. The
%    basic interface consists of using one option, which is the same
%    for \emph{all} standard document classes.
%
%    In some cases the language definition files provide macros that
%    can be useful to plain \TeX\ users as well as to \LaTeX\ users.
%    The \babel{} system has been implemented so that it
%    can be used by both groups of users.
%
% \section{Acknowledgements}
%
%    I would like to thank all who volunteered as $\beta$-testers for
%    their time. I would like to mention Julio Sanchez who supplied
%    the option file for the Spanish language and Maurizio Codogno who
%    supplied the option file for the Italian language. Michel Goossens
%    supplied contributions for most of the other languages.  Nico
%    Poppelier helped polish the text of the documentation and
%    supplied parts of the macros for the Dutch language.  Paul
%    Wackers and Werenfried Spit helped find and repair bugs.
%
%    During the further development of the babel system I received
%    much help from Bernd Raichle, for which I am grateful.
%
%  \begin{thebibliography}{9}
%    \bibitem{DEK} Donald E. Knuth,
%      \emph{The \TeX book}, Addison-Wesley, 1986.
%    \bibitem{LLbook} Leslie Lamport,
%       \emph{\LaTeX, A document preparation System}, Addison-Wesley,
%       1986.
%    \bibitem{treebus} K.F. Treebus.
%       \emph{Tekstwijzer, een gids voor het grafisch verwerken van
%       tekst.}
%       SDU Uitgeverij ('s-Gravenhage, 1988). A Dutch book on layout
%       design and typography.
%    \bibitem{HP} Hubert Partl,
%      \emph{German \TeX}, \emph{TUGboat} 9 (1988) \#1, p.~70--72.
%     \bibitem{LLth} Leslie Lamport,
%       in: \TeXhax\ Digest, Volume 89, \#13, 17 February 1989.
%    \bibitem{BEP} Johannes Braams, Victor Eijkhout and Nico Poppelier,
%      \emph{The development of national \LaTeX\ styles},
%      \emph{TUGboat} 10 (1989) \#3, p.~401--406.
%    \bibitem{ilatex} Joachim Schrod,
%      \emph{International \LaTeX\ is ready to use},
%      \emph{TUGboat} 11 (1990) \#1, p.~87--90.
%  \end{thebibliography}
% }
%
% \section{Identification}
%
%    The file \file{babel.sty}\footnote{The file described in this
%    section is called \texttt{\filename}, has version
%    number~\fileversion\ and was last revised on~\filedate.} is meant
%    for \LaTeXe, therefor we make sure that the format file used is
%    the right one.
%
%  \begin{macro}{\ProvidesLanguage}
% \changes{babel~3.7a}{1997/03/18}{Added macro to prevent problems
%    with unexpected \cs{ProvidesFile} in plain formats because of
%    \babel.}
%    The identification code for each file is something that was
%    introduced in \LaTeXe. When the command |\ProvidesFile| does not
%    exist, a dummy definition is provided temporarily. For use in the
%    language definition file the command |\ProvidesLanguage| is
%    defined by \babel.
% \changes{babel~3.4e}{1994/06/24}{Redid the identification code,
%    provided dummy definition of \cs{ProvidesFile} for plain \TeX}
% \changes{babel~3.5f}{1995/07/26}{Store version in \cs{fileversion}}
% \changes{babel~3.5f}{1995/12/18}{Need to temporarily change the
%    definition of \cs{ProvidesFile} for December 1995 release}
% \changes{babel~3.5g}{1996/07/09}{Save a few csnames; use
%    \cs{bbl@tempa} instead of \cs{\@ProvidesFile} and store message
%    in \cs{toks8}}
%    \begin{macrocode}
%<*!package>
\ifx\ProvidesFile\@undefined
  \def\ProvidesFile#1[#2 #3 #4]{%
    \wlog{File: #1 #4 #3 <#2>}%
%<*kernel&patterns>
    \toks8{Babel <#3> and hyphenation patterns for }%
%</kernel&patterns>
    \let\ProvidesFile\@undefined
    }
%    \end{macrocode}
%    As an alternative for |\ProvidesFile| we define
%    |\ProvidesLanguage| here to be used in the language definition
%    files.
%    \begin{macrocode}
%<*kernel>
  \def\ProvidesLanguage#1[#2 #3 #4]{%
    \wlog{Language: #1 #4 #3 <#2>}%
    }
\else
%    \end{macrocode}
%    In this case we save the original definition of |\ProvidesFile| in
%    |\bbl@tempa| and restore it after we have stored the version of
%    the file in |\toks8|.
% \changes{babel~3.7a}{1997/11/04}{Removed superfluous braces}
%    \begin{macrocode}
%<*kernel&patterns>
  \let\bbl@tempa\ProvidesFile
  \def\ProvidesFile#1[#2 #3 #4]{%
    \toks8{Babel <#3> and hyphenation patterns for }%
    \bbl@tempa#1[#2 #3 #4]%
    \let\ProvidesFile\bbl@tempa}
%</kernel&patterns>
%    \end{macrocode}
%    When |\ProvidesFile| is defined we give |\ProvidesLanguage| a
%    similar definition.
%    \begin{macrocode}
  \def\ProvidesLanguage#1{%
    \begingroup
      \catcode`\ 10 %
      \@makeother\/%
      \@ifnextchar[%]
        {\@provideslanguage{#1}}{\@provideslanguage{#1}[]}}
  \def\@provideslanguage#1[#2]{%
    \wlog{Language: #1 #2}%
    \expandafter\xdef\csname ver@#1.ldf\endcsname{#2}%
    \endgroup}
%</kernel>
\fi
%</!package>
%    \end{macrocode}
%  \end{macro}
%
%    Identify each file that is produced from this source file.
% \changes{babel~3.4c}{1995/04/28}{lhyphen.cfg has become
%    lthyphen.cfg}
% \changes{babel~3.5b}{1995/01/25}{lthyphen.cfg has become hyphen.cfg}
%    \begin{macrocode}
%<+package>\ProvidesPackage{babel}
%<+core>\ProvidesFile{babel.def}
%<+kernel&patterns>\ProvidesFile{hyphen.cfg}
%<+kernel&!patterns>\ProvidesFile{switch.def}
%<+driver&!user>\ProvidesFile{babel.drv}
%<+driver&user>\ProvidesFile{user.drv}
                [2008/03/16 v3.8j %
%<+package>     The Babel package]
%<+core>         Babel common definitions]
%<+kernel>      Babel language switching mechanism]
%<+driver>]
%    \end{macrocode}
%
% \section{The Package File}
%
%    In order to make use of the features of \LaTeXe, the \babel\
%    system contains a package file, \file{babel.sty}. This file
%    is loaded by the |\usepackage| command and defines all the
%    language options known in the \babel\ system. It also takes care
%    of a number of compatibility issues with other packages.
%
%  \subsection{Language options}
%
% \changes{babel~3.6c}{1997/01/05}{When \cs{LdfInit} is undefined we
%    need to load \file{babel.def} from \file{babel.sty}}
% \changes{babel~3.6l}{1999/04/03}{Don't load \file{babel.def} now,
%    but rather define \cs{LdfInit} temporarily in order to load
%    \file{babel.def} at the right time, preventing problems with the
%    temporary definition of \cs{bbl@redefine}}
% \changes{babel~3.6r}{1999/04/12}{We \textbf{do} need to load
%    \file{babel.def} right now as \cs{ProvidesLanguage} needs to be
%    defined before the \file{.ldf} files are read and the reason for
%    for 3.6l has been removed}
%    \begin{macrocode}
%<*package>
\ifx\LdfInit\@undefined\input babel.def\relax\fi
%    \end{macrocode}
%
%    For all the languages supported we need to declare an option.
% \changes{babel~3.5a}{1995/03/14}{Changed extension of language
%    definition files to \texttt{ldf}}
% \changes{babel~3.5d}{1995/07/02}{Load language definition files
%    \emph{after} the check for the hyphenation patterns}
% \changes{babel~3.5g}{1996/10/04}{Added option \Lopt{afrikaans}}
% \changes{babel~3.7g}{2001/02/09}{Added option \Lopt{acadian}}
% \changes{babel~3.8c}{2004/06/12}{Added option \Lopt{australian}}
% \changes{babel~3.8h}{2005/11/23}{Added option \Lopt{albanian}}
%    \begin{macrocode}
\DeclareOption{acadian}{% \iffalse meta-comment
%
% Copyright 1989-2009 Johannes L. Braams and any individual authors
% listed elsewhere in this file.  All rights reserved.
% 
% This file is part of the Babel system.
% --------------------------------------
% 
% It may be distributed and/or modified under the
% conditions of the LaTeX Project Public License, either version 1.3
% of this license or (at your option) any later version.
% The latest version of this license is in
%   http://www.latex-project.org/lppl.txt
% and version 1.3 or later is part of all distributions of LaTeX
% version 2003/12/01 or later.
% 
% This work has the LPPL maintenance status "maintained".
% 
% The Current Maintainer of this work is Johannes Braams.
% 
% The list of all files belonging to the Babel system is
% given in the file `manifest.bbl. See also `legal.bbl' for additional
% information.
% 
% The list of derived (unpacked) files belonging to the distribution
% and covered by LPPL is defined by the unpacking scripts (with
% extension .ins) which are part of the distribution.
% \fi
% \CheckSum{2135}
%
% \iffalse
%    Tell the \LaTeX\ system who we are and write an entry on the
%    transcript. Nothing to write to the .cfg file, if generated.
%<*dtx>
\ProvidesFile{frenchb.dtx}
%</dtx>
% \changes{v2.1d}{2008/05/04}{Argument of \cs{ProvidesLanguage} changed
%     from `french' to `frenchb', otherwise \cs{listfiles} prints
%     no date/version information.  The bug with \cs{listfiles}
%     (introduced in v.1.5!), was pointed out by Ulrike Fischer.}
%<code>\ProvidesLanguage{frenchb}
%\ProvidesFile{frenchb.dtx}
%<*!cfg>
        [2009/03/16 v2.3d French support from the babel system]
%</!cfg>
%<*cfg>
%% frenchb.cfg: configuration file for frenchb.ldf
%% Daniel Flipo Daniel.Flipo at univ-lille1.fr
%</cfg>
%%    File `frenchb.dtx'
%%    Babel package for LaTeX version 2e
%%    Copyright (C) 1989 - 2009
%%              by Johannes Braams, TeXniek
%
%<*!cfg>
%%    Frenchb language Definition File
%%    Copyright (C) 1989 - 2009
%%              by Johannes Braams, TeXniek
%%                 Daniel Flipo, GUTenberg
%
%%    Please report errors to: Daniel Flipo, GUTenberg
%%                             Daniel.Flipo at univ-lille1.fr
%</!cfg>
%
%    This file is part of the babel system, it provides the source
%    code for the French language definition file.
%
%<*filedriver>
\documentclass[a4paper]{ltxdoc}
\DeclareFontEncoding{T1}{}{}
\DeclareFontSubstitution{T1}{lmr}{m}{n}
\DeclareTextCommand{\guillemotleft}{OT1}{%
  {\fontencoding{T1}\fontfamily{lmr}\selectfont\char19}}
\DeclareTextCommand{\guillemotright}{OT1}{%
  {\fontencoding{T1}\fontfamily{lmr}\selectfont\char20}}
\newcommand*\TeXhax{\TeX hax}
\newcommand*\babel{\textsf{babel}}
\newcommand*\langvar{$\langle \mathit lang \rangle$}
\newcommand*\note[1]{}
\newcommand*\Lopt[1]{\textsf{#1}}
\newcommand*\file[1]{\texttt{#1}}
\begin{document}
\setlength{\parindent}{0pt}
\begin{center}
  \textbf{\Large A Babel language definition file for French}\\[3mm]^^A\]
  Daniel \textsc{Flipo}\\
  \texttt{Daniel.Flipo@univ-lille1.fr}
\end{center}
 \RecordChanges
 \DocInput{frenchb.dtx}
\end{document}
%</filedriver>
% \fi
% \GetFileInfo{frenchb.dtx}
%
%  \section{The French language}
%
%    The file \file{\filename}\footnote{The file described in this
%    section has version number \fileversion\ and was last revised on
%    \filedate.}, defines all the language definition macros for the
%    French language.
%
%    Customisation for the French language is achieved following the
%    book ``Lexique des r\`egles typographiques en usage \`a
%    l'Imprimerie nationale'' troisi\`eme \'edition (1994),
%    ISBN-2-11-081075-0.
%
%    First version released: 1.1 (1996/05/31) as part of
%    \babel-3.6beta.
%
%    |frenchb| has been improved using helpful suggestions from many
%    people, mainly from Jacques Andr\'e, Michel Bovani, Thierry Bouche,
%    and Vincent Jalby.  Thanks to all of them!
%
%    This new version (2.x) has been designed to be used with \LaTeXe{}
%    and Plain\TeX{} formats only. \LaTeX-2.09 is no longer supported.
%    Changes between version 1.6 and \fileversion{} are listed in
%    subsection~\ref{ssec-changes} p.~\pageref{ssec-changes}.
%
%    An extensive documentation is available in French here:\\
%    |http://daniel.flipo.free.fr/frenchb|
%
%  \subsection{Basic interface}
%
%    In a multilingual document, some typographic rules are language
%    dependent, i.e. spaces before `double punctuation' (|:| |;| |!|
%    |?|) in French, others concern the general layout (i.e. layout of
%    lists, footnotes, indentation of first paragraphs of sections) and
%    should apply to the whole document.
%
%    Starting with version~2.2, |frenchb| behaves differently according
%    to \babel's \emph{main language} defined as the \emph{last}
%    option\footnote{Its name is kept in \texttt{\textbackslash
%           bbl@main@language}.} at \babel's loading.  When French is
%    not \babel's main language, |frenchb| no longer alters the global
%    layout of the document (even in parts where French is the current
%    language): the layout of lists, footnotes, indentation of first
%    paragraphs of sections are not customised by |frenchb|.
%
%    When French is loaded as the last option of \babel, |frenchb|
%    makes the following changes to the global layout, \emph{both in
%    French and in all other languages}\footnote{%
%       For each item, hooks are provided to reset standard
%       \LaTeX{} settings or to emulate the behavior of former versions
%       of \texttt{frenchb} (see command
%       \texttt{\textbackslash frenchbsetup\{\}},
%       section~\ref{ssec-custom}).}:
%    \begin{enumerate}
%    \item the first paragraph of each section is indented
%          (\LaTeX{} only);
%    \item the default items in itemize environment are set to `--'
%          instead of `\textbullet', and all vertical spacing and glue
%          is deleted; it is possible to change `--' to something else
%          (`---' for instance) using |\frenchbsetup{}|;
%    \item vertical spacing in general \LaTeX{} lists is
%          shortened;
%    \item footnotes are displayed ``\`a la fran\c{c}aise''.
%    \end{enumerate}
%
%    Regarding local typography, the command |\selectlanguage{french}|
%    switches to the French language\footnote{%
%      \texttt{\textbackslash selectlanguage\{francais\}}
%      and \texttt{\textbackslash selectlanguage\{frenchb\}} are kept
%      for backward compatibility but should no longer be used.},
%    with the following effects:
%    \begin{enumerate}
%    \item French hyphenation patterns are made active;
%    \item `double punctuation' (|:| |;| |!| |?|) is made
%           active%\footnote{Actually, they are active in the whole
%           document, only their expansions differ in French and
%           outside French} for correct spacing in French;
%    \item |\today| prints the date in French;
%    \item the caption names are translated into French
%          (\LaTeX{} only);
%    \item the space after |\dots| is removed in French.
%    \end{enumerate}
%
%    Some commands are provided in |frenchb| to make typesetting
%    easier:
%    \begin{enumerate}
%    \item French quotation marks can be entered using the commands
%          |\og| and |\fg| which work in \LaTeXe and Plain\TeX,
%          their appearance depending on what is available to draw
%          them; even if you use \LaTeXe{} \emph{and} |T1|-encoding,
%          you should refrain from entering them as
%          |<<~French quotation marks~>>|: |\og| and |\fg| provide
%          better horizontal spacing.
%          |\og| and |\fg| can be used outside French, they typeset
%          then English quotes `` and ''.
%    \item A command |\up| is provided to typeset superscripts like
%          |M\up{me}| (abbreviation for ``Madame''), |1\up{er}| (for
%          ``premier'').  Other commands are also provided for
%          ordinals: |\ier|, |\iere|, |\iers|, |\ieres|, |\ieme|,
%          |\iemes| (|3\iemes| prints 3\textsuperscript{es}).
%    \item Family names should be typeset in small capitals and never
%          be hyphenated, the macro |\bsc| (boxed small caps) does
%          this, e.g., |Leslie~\bsc{Lamport}| will produce
%          Leslie~\mbox{\textsc{Lamport}}. Note that composed names
%          (such as Dupont-Durant) may now be hyphenated on explicit
%          hyphens, this differs from |frenchb|~v.1.x.
%    \item Commands |\primo|, |\secundo|, |\tertio| and |\quarto|
%          print 1\textsuperscript{o}, 2\textsuperscript{o},
%          3\textsuperscript{o}, 4\textsuperscript{o}.
%          |\FrenchEnumerate{6}| prints 6\textsuperscript{o}.
%    \item Abbreviations for ``Num\'ero(s)'' and ``num\'ero(s)''
%          (N\textsuperscript{o} N\textsuperscript{os}
%          n\textsuperscript{o} and n\textsuperscript{os}~)
%          are obtained via the commands |\No|, |\Nos|, |\no|, |\nos|.
%    \item Two commands are provided to typeset the symbol for
%          ``degr\'e'': |\degre| prints the raw character and
%          |\degres| should be used to typeset temperatures (e.g.,
%          ``|20~\degres C|'' with an unbreakable space), or for
%          alcohols' strengths (e.g., ``|45\degres|'' with \emph{no}
%          space in French).
%    \item In math mode the comma has to be surrounded with
%          braces to avoid a spurious space being inserted after it,
%          in decimal numbers for instance (see the \TeX{}book p.~134).
%          The command |\DecimalMathComma| makes the comma be an
%          ordinary character \emph{in French only} (no space added);
%          as a counterpart, if |\DecimalMathComma| is active, an
%          explicit space has to be added in lists and intervals:
%          |$[0,\ 1]$|, |$(x,\ y)$|. |\StandardMathComma| switches back
%          to the standard behaviour of the comma.
%    \item A command |\nombre| was provided in 1.x versions to easily
%          format numbers in slices of three digits separated either
%          by a comma in English or with a space in French; |\nombre|
%          is now mapped to |\numprint| from \file{numprint.sty}, see
%          \file{numprint.pdf} for more information.
%    \item |frenchb| has been designed to take advantage of the |xspace|
%          package if present: adding |\usepackage{xspace}| in the
%          preamble will force macros like |\fg|, |\ier|, |\ieme|,
%          |\dots|, \dots, to respect the spaces you type after them,
%          for instance typing `|1\ier juin|' will print
%          `1\textsuperscript{er} juin' (no need for a forced space
%          after |1\ier|).
%    \end{enumerate}
%
%  \subsection{Customisation}
%  \label{ssec-custom}
%
%     Up to version 1.6, customisation of |frenchb| was achieved
%     by entering commands in \file{frenchb.cfg}.  This possibility
%     remains for compatibility, but \emph{should not longer be used}.
%     Version 2.0 introduces a new command |\frenchbsetup{}| using
%     the \file{keyval} syntax which should make it easier to choose
%     among the many options available. The command |\frenchbsetup{}|
%     is to appear in the preamble only (after loading \babel).
%
%     \vspace{.5\baselineskip}
%     |\frenchbsetup{ShowOptions}| prints all available options to
%     the \file{.log} file, it is just meant as a remainder of the
%     list of offered options. As usual with \file{keyval} syntax,
%     boolean options (as |ShowOptions|) can be entered as
%     |ShowOptions=true| or just |ShowOptions|, the `|=true|' part
%     can be omitted.
%
%     \vspace{.5\baselineskip}
%     The other options are listed below. Their default value is shown
%     between brackets, sometimes followed be a `\texttt{*}'.
%     The `\texttt{*}' means that the default shown applies when
%     |frenchb| is loaded as the \emph{last} option of \babel{}
%     ---\babel's \emph{main language}---, and is toggled otherwise:
%     \begin{itemize}
%     \item |StandardLayout=true [false*]| forces |frenchb| not to
%       interfere with the layout: no action on any kind of lists,
%       first paragraphs of sections are not indented (as in English),
%       no action on footnotes. This option replaces the former
%       command |\StandardLayout|.  It can be used to avoid conflicts
%       with classes or packages which customise lists or footnotes.
%     \item |GlobalLayoutFrench=false [true*]| can be used, when French
%       is the main language, to emulate what prior versions of
%       |frenchb| (pre-2.2) did: lists, and first paragraphs
%       of sections will be displayed the standard way in other
%       languages than French, and ``\`a la fran\c{c}aise'' in French.
%       Note that the layout of footnotes is language independent
%       anyway (see below |FrenchFootnotes| and |AutoSpaceFootnotes|).
%       This option replaces the former command |\FrenchLayout|.
%     \item |ReduceListSpacing=false [true*]|; |frenchb| normally
%       reduces the values of the vertical spaces used in the
%       environment |list| in French; setting this option to |false|
%       reverts to the standard settings of |list|.  This option
%       replaces the former command |\FrenchListSpacingfalse|.
%     \item |CompactItemize=false [true*]|; |frenchb| normally
%       suppresses any vertical space between items of |itemize| lists
%       in French; setting this option to |false| reverts to the
%       standard settings of |itemize| lists.  This option replaces
%       the former command |\FrenchItemizeSpacingfalse|.
%     \item |StandardItemLabels=true [false*]| when set to |true| this
%       option stops |frenchb| from changing the labels in |itemize|
%       lists in French.
%     \item |ItemLabels=\textemdash|, |\textbullet|, |\ding{43}|,
%       \dots, |[\textendash*]|; when |StandardItemLabels=false| (the
%       default), this option enables to choose the label used in
%       |itemize| lists for all levels.  The next three options do
%       the same but each one for one level only. Note that the
%       example |\ding{43}| requires |\usepackage{pifont}|.
%     \item |ItemLabeli=\textemdash|, |\textbullet|, |\ding{43}|,
%       \dots,|[\textendash*]|
%     \item |ItemLabelii=\textemdash|, |\textbullet|, |\ding{43}|,
%       \dots, |[\textendash*]|
%     \item |ItemLabeliii=\textemdash|, |\textbullet|, |\ding{43}|,
%       \dots, |[\textendash*]|
%     \item |ItemLabeliv=\textemdash|, |\textbullet|, |\ding{43}|,
%       \dots, |[\textendash*]|
%     \item |StandardLists=true [false*]| forbids |frenchb| to
%       customise any kind of list. Do activate the option
%       |StandardLists| when using classes or packages that customise
%       lists too (|enumitem|, |paralist|, \dots{}) to avoid conflicts.
%       This option is just a shorthand for |ReduceListSpacing=false|
%       and |CompactItemize=false| and |StandardItemLabels=true|.
%     \item |IndentFirst=false [true*]|; |frenchb| normally forces
%       indentation of the first paragraph of sections.
%       When this option is set to |false|, the first paragraph of
%       will look the same in French and in English (not indented).
%     \item |FrenchFootnotes=false [true*]| reverts to the standard
%       layout of footnotes. By default |frenchb| typesets leading
%       numbers as `1.\hspace{.5em}' instead of `$\hbox{}^1$', but
%       has no effect on footnotes numbered with symbols (as in the
%       |\thanks| command).  The former commands |\StandardFootnotes|
%       and |\FrenchFootnotes| are still there, |\StandardFootnotes|
%       can be useful when some footnotes are numbered with letters
%       (inside minipages for instance).
%     \item |AutoSpaceFootnotes=false [true*]| ; by default |frenchb|
%       adds a thin space in the running text before the number or
%       symbol calling the footnote.  Making this option |false|
%       reverts to the standard setting (no space added).
%     \item |FrenchSuperscripts=false [true]| ; then
%       |\up=\textsuperscript| (option added in version 2.1).
%       Should only be made |false| to recompile older documents.
%       By default |\up| now relies on |\fup| designed to produce
%       better looking superscripts.
%     \item |AutoSpacePunctuation=false [true]|; in French, the user
%       \emph{should} input a space before the four characters `|:;!?|'
%       but as many people forget about it (even among native French
%       writers!), the default behaviour of |frenchb| is to
%       automatically add a |\thinspace| before `|;|' `|!|' `|?|' and a
%       normal (unbreakable) space before~`|:|' (this is recommended by
%       the French Imprimerie nationale).  This is convenient in most
%       cases but can lead to addition of spurious spaces in URLs or in
%       MS-DOS paths but only if they are no typed using |\texttt| or
%       verbatim mode. When the current font is a monospaced
%       (typewriter) font, |AutoSpacePunctuation| is locally switched
%       to |false|, no spurious space is added in that case, so the
%       default behaviour of of |frenchb| in that area should be fine
%       in most circumstances.
%
%       Choosing |AutoSpacePunctuation=false| will ensure that
%       a proper space will be added before `|:;!?|' \emph{if and only
%       if} a (normal) space has been typed in. Those who are unsure
%       about their typing in this area should stick to the default
%       option and type |\string;| |\string:| |\string!| |\string?|
%       instead of |;| |:| |!| |?| in case an unwanted space is
%       added by |frenchb|.
%     \item |ThinColonSpace=true [false]| changes the normal
%       (unbreakable) space added before the colon `:' to a thin space,
%       so that the same amount of space is added before any of the
%       four double punctuation characters. The default setting is
%       supported by the French Imprimerie nationale.
%     \item |LowercaseSuperscripts=false [true]| ; by default |frenchb|
%       inhibits the uppercasing of superscripts (for instance when they
%       are moved to page headers). Making this option |false|
%       will disable this behaviour (not recommended).
%     \item |PartNameFull=false [true]|; when true, |frenchb| numbers
%       the title of |\part{}| commands as ``Premi\`ere partie'',
%       ``Deuxi\`eme partie'' and so on. With some classes which change
%       the|\part{}| command (AMS and SMF classes do so), you will get
%       ``Premi\`ere partie~I'', ``Deuxi\`eme partie~II'' instead;
%       when this occurs, this option should be set to |false|,
%       part titles will then be printed as ``Partie I'', ``Partie II''.
%     \item |og=|\texttt{\guillemotleft}, |fg=|\texttt{\guillemotright};
%       when guillemets characters are available on the keyboard
%       (through a compose key for instance), it is nice to use them
%       instead of typing |\og| and |\fg|. This option tells |frenchb|
%       which characters are opening and closing French guillemets
%       (they depend on the input encoding), then you can type either
%       \texttt{\guillemotleft{} guillemets \guillemotright}, or
%       \texttt{\guillemotleft{}guillemets\guillemotright} (with or
%       without spaces), to get properly typeset French quotes.
%       This option requires \file{inputenc} to be loaded with the
%       proper encoding, it works with 8-bits encodings (latin1,
%       latin9, ansinew,  applemac,\dots) and multi-byte encodings
%       (utf8 and utf8x).
%     \end{itemize}
%
%  \subsection{Hyphenation checks}
%  \label{ssec-hyphen}
%
%    Once you have built your format, a good precaution would be to
%    perform some basic tests about hyphenation in French. For
%    \LaTeXe{} I suggest this:
%    \begin{itemize}
%    \item run the following file, with the encoding suitable for
%      your machine (\textit{my-encoding} will be |latin1| for
%      \textsc{unix} machines, |ansinew| for PCs running~Windows,
%      |applemac| or |latin1| for Macintoshs, or |utf8|\dots\\[3mm]^^A\]
%      |%%% Test file for French hyphenation.|\\
%      |\documentclass{article}|\\
%      |\usepackage[|\textit{my-encoding}|]{inputenc}|\\
%      |\usepackage[T1]{fontenc} % Use LM fonts|\\
%      |\usepackage{lmodern}     % for French|\\
%      |\usepackage[frenchb]{babel}|\\
%      |\begin{document}|\\
%      |\showhyphens{signal container \'ev\'enement alg\`ebre}|\\
%      |\showhyphens{|\texttt{signal container \'ev\'enement
%                     alg\`ebre}|}|\\
%      |\end{document}|
%    \item check the hyphenations proposed by \TeX{} in your log-file;
%      in French you should get with both 7-bit and 8-bit encodings\\
%      \texttt{si-gnal contai-ner \'ev\'e-ne-ment al-g\`ebre}.\\
%      Do not care about how accented characters are displayed in the
%      log-file, what matters is the position of the `|-|' hyphen
%      signs \emph{only}.
%    \end{itemize}
%    If they are all correct, your installation (probably) works fine,
%    if one (or more) is (are) wrong, ask a local wizard to see what's
%    going wrong and perform the test again (or e-mail me about what
%    happens).\\
%    Frequent mismatches:
%    \begin{itemize}
%    \item you get |sig-nal con-tainer|, this probably means that the
%    hyphenation patterns you are using are for US-English, not for
%    French;
%    \item you get no hyphen at all in \texttt{\'ev\'e-ne-ment}, this
%    probably means that you are using CM fonts and the macro
%    |\accent| to produce accented characters.
%    Using 8-bits fonts with built-in accented characters avoids
%    this kind of mismatch.
%    \end{itemize}
%
%    \textbf{Options' order} -- Please remember that options are read
%    in the order they appear inside the |\frenchbsetup| command.
%    Someone wishing that |frenchb| leaves the layout of lists
%    and footnotes untouched but caring for indentation of first
%    paragraph of sections could choose
%    |\frenchbsetup{StandardLayout,IndentFirst}| and get the expected
%    layout. Choosing |\frenchbsetup{IndentFirst,StandardLayout}|
%    would not lead to the expected result: option |IndentFirst| would
%    be overwritten by |StandardLayout|.
%
%  \subsection{Changes}
%  \label{ssec-changes}
%
%  \subsubsection*{What's new in version 2.0?}
%
%    Here is the list of all changes:
%    \begin{itemize}
%    \item Support for \LaTeX-2.09 and for \LaTeXe{} in compatibility
%      mode has been dropped. This version is meant for \LaTeXe{} and
%      Plain based formats (like \file{bplain}). \LaTeXe{} formats
%      based on ml\TeX{} are no longer supported either (plenty of
%      good 8-bits fonts are available now, so T1 encoding should
%      be preferred for typesetting in French). A warning is issued
%      when OT1 encoding is in use at the |\begin{document}|.
%    \item Customisation should now be handled by command
%      |\frenchbsetup{}|, \file{frenchb.cfg} (kept for compatibility)
%      should no longer be used. See section~\ref{ssec-custom} for
%      the list of available options.
%    \item Captions in figures and table have changed in French: former
%      abbreviations ``Fig.'' and ``Tab.'' have been replaced by full
%      names ``Figure'' and ``Table''.  If this leads to formatting
%      problems in captions, you can add the following two commands to
%      your preamble (after loading \babel) to get the former captions\\
%      |\addto\captionsfrench{\def\figurename{{\scshape Fig.}}}|\\
%      |\addto\captionsfrench{\def\tablename{{\scshape Tab.}}}|.
%    \item The |\nombre| command is now provided by the \file{numprint}
%      package which has to be loaded \emph{after} \babel{} with the
%      option |autolanguage| if number formatting should depend on the
%      current language.
%    \item The |\bsc| command no longer uses an |\hbox| to stop
%      hyphenation of names but a |\kern0pt| instead. This change
%      enables \file{microtype} to fine tune the length of the
%      argument of |\bsc|; as a side-effect, compound names like
%      Dupont-Durand can now be hyphenated on  explicit hyphens.
%      You can get back to the former behaviour of |\bsc| by adding\\
%      |\renewcommand*{\bsc}[1]{\leavevmode\hbox{\scshape #1}}|\\
%      to the preamble of your document.
%    \item Footnotes are now displayed ``\`a la fran\c caise'' for the
%      whole document, except with an explicit\\
%      |\frenchbsetup{AutoSpaceFootnotes=false,FrenchFootnotes=false}|.\\
%      Add this command if you want standard footnotes. It is still
%      possible to revert locally to the standard layout of footnotes
%      by adding |\StandardFootnotes| (inside a |minipage| environment
%      for instance).
%    \end{itemize}
%
%  \subsubsection*{What's new in version 2.1?}
%
%      New command |\fup| to typeset better looking superscripts.
%      Former command |\up| is now defined as |\fup|, but an option
%      |\frenchbsetup{FrenchSuperscripts=false}| is provided for
%      backward compatibility.  |\fup| was designed using ideas from
%      Jacques Andr\'e, Thierry Bouche and Ren\'e Fritz, thanks to them!
%
%  \subsubsection*{What's new in version 2.2?}
%
%      Starting with version~2.2a, |frenchb| alters the layout of
%      lists, footnotes, and the indentation of first paragraphs of
%      sections) \emph{only if} French is the ``main language''
%      (i.e. babel's last language option). The layout is global for
%      the whole document: lists, etc. look the same in French and in
%      other languages, everything is typeset ``\`a la fran\c caise''
%      if French is the ``main language'', otherwise |frenchb| doesn't
%      change anything regarding lists, footnotes, and indentation of
%      paragraphs.
%
%  \subsubsection*{What's new in version 2.3?}
%
%      Starting with version~2.3a, |frenchb| no longer inserts spaces
%      automatically before `|:;!?|' when a typewriter font is in use;
%      this was suggested by Yannis Haralambous to prevent
%      spurious spaces in computer source code or expressions like
%      \texttt{C\string:/foo}, \texttt{http\string://foo.bar},
%      etc.  An option (|OriginalTypewriter|) is provided to get back
%      to the former behaviour of |frenchb|.
%
%      Another probably invisible change: lowercase conversion in
%      |\up{}| is now achieved by the \LaTeX{} command |\MakeLowercase|
%      instead of \TeX's |\lowercase| command.  This prevents error
%      messages when diacritics are used inside |\up{}| (diacritics
%      should \emph{never} be used in superscripts though!).
%
% \StopEventually{}
%
%  \subsection{File frenchb.cfg}
%  \label{sec-cfg}
%
%    \file{frenchb.cfg} is now a dummy file just kept for compatibility
%    with previous versions.
%
% \iffalse
%<*cfg>
% \fi
%    \begin{macrocode}
%%%%%%%%%%%%%%%%%%%%%%%%%%%%%%%%%%%%%%%%%%%%%%%%%%%%%%%%%%%%%%%%%%%%%%
%%%%%%%%%  WARNING: THIS  FILE SHOULD  NO  LONGER  BE  USED  %%%%%%%%%
%% If you want to customise frenchb, please DO NOT hack into the code!
%% Do no put any code in this file either, please use the new command
%% \frenchbsetup{} with the proper options to customise frenchb.
%% 
%% Add \frenchbsetup{ShowOptions} to your preamble to see the list of
%% available options and/or read the documentation.
%%%%%%%%%%%%%%%%%%%%%%%%%%%%%%%%%%%%%%%%%%%%%%%%%%%%%%%%%%%%%%%%%%%%%%
%    \end{macrocode}
% \iffalse
%</cfg>
% \fi
%
%  \section{\TeX{}nical details}
%
%  \subsection{Initial setup}
%
% \changes{v2.1d}{2008/05/02}{Argument of \cs{ProvidesLanguage} changed
%     above from `french' to `frenchb' (otherwise \cs{listfiles} prints
%     no date/version information).  The real name of current language
%     (french) as to be corrected before calling \cs{LdfInit}.}
%
% \iffalse
%<*code>
% \fi
%
%    While this file was read through the option \Lopt{frenchb} we make
%    it behave as if \Lopt{french} was specified.
%    \begin{macrocode}
\def\CurrentOption{french}
%    \end{macrocode}
%
%    The macro |\LdfInit| takes care of preventing that this file is
%    loaded more than once, checking the category code of the
%    \texttt{@} sign, etc.
%
%    \begin{macrocode}
\LdfInit\CurrentOption\datefrench
%    \end{macrocode}
%
% \changes{v2.1d}{2008/05/04}{Avoid warning ``\cs{end} occurred
%   when \cs{ifx} ... incomplete'' with LaTeX-2.09.}
%
%  \begin{macro}{\ifLaTeXe}
%    No support is provided for late \LaTeX-2.09: issue a warning
%    and exit if \LaTeX-2.09 is in use. Plain is still supported.
%    \begin{macrocode}
\newif\ifLaTeXe
\let\bbl@tempa\relax
\ifx\magnification\@undefined
   \ifx\@compatibilitytrue\@undefined
     \PackageError{frenchb.ldf}
        {LaTeX-2.09 format is no longer supported.\MessageBreak
         Aborting here}
        {Please upgrade to LaTeX2e!}
     \let\bbl@tempa\endinput
   \else
     \LaTeXetrue
   \fi
\fi
\bbl@tempa
%    \end{macrocode}
%  \end{macro}
%
%    Check if hyphenation patterns for the French language have been
%    loaded in language.dat; we allow for the names `french',
%    `francais', `canadien' or `acadian'. The latter two are both
%    names used in Canada for variants of French that are in use in
%    that country.
%
%    \begin{macrocode}
\ifx\l@french\@undefined
  \ifx\l@francais\@undefined
    \ifx\l@canadien\@undefined
      \ifx\l@acadian\@undefined
        \@nopatterns{French}
        \adddialect\l@french0
      \else
        \let\l@french\l@acadian
      \fi
    \else
      \let\l@french\l@canadien
    \fi
  \else
    \let\l@french\l@francais
  \fi
\fi
%    \end{macrocode}
%    Now |\l@french| is always defined.
%
%    The internal name for the French language is |french|;
%    |francais| and |frenchb| are synonymous for |french|:
%    first let both names use the same hyphenation patterns.
%    Later we will have to set aliases for |\captionsfrench|,
%    |\datefrench|, |\extrasfrench| and |\noextrasfrench|.
%    As French uses the standard values of |\lefthyphenmin| (2)
%    and |\righthyphenmin| (3), no special setting is required here.
%
%    \begin{macrocode}
\ifx\l@francais\@undefined
  \let\l@francais\l@french
\fi
\ifx\l@frenchb\@undefined
  \let\l@frenchb\l@french
\fi
%    \end{macrocode}
%    When this language definition file was loaded for one of the
%    Canadian versions of French we need to make sure that a suitable
%    hyphenation pattern register will be found by \TeX.
%    \begin{macrocode}
\ifx\l@canadien\@undefined
  \let\l@canadien\l@french
\fi
\ifx\l@acadian\@undefined
  \let\l@acadian\l@french
\fi
%    \end{macrocode}
%
%    This language definition can be loaded for different variants of
%    the French language. The `key' \babel\ macros are only defined
%    once, using `french' as the language name, but |frenchb| and
%    |francais| are synonymous.
%    \begin{macrocode}
\def\datefrancais{\datefrench}
\def\datefrenchb{\datefrench}
\def\extrasfrancais{\extrasfrench}
\def\extrasfrenchb{\extrasfrench}
\def\noextrasfrancais{\noextrasfrench}
\def\noextrasfrenchb{\noextrasfrench}
%    \end{macrocode}
%
% \begin{macro}{\extrasfrench}
% \begin{macro}{\noextrasfrench}
%    The macro |\extrasfrench| will perform all the extra
%    definitions needed for the French language.
%    The macro |\noextrasfrench| is used to cancel the actions of
%    |\extrasfrench|.\\
%    In French, character ``apostrophe'' is a letter in expressions
%    like |l'ambulance| (French  hyphenation patterns provide entries
%    for this kind of words).  This means that the |\lccode| of
%    ``apostrophe'' has to be non null in French for proper hyphenation
%    of those expressions, and has to be reset to null when exiting
%    French.
%
%    \begin{macrocode}
\@namedef{extras\CurrentOption}{\lccode`\'=`\'}
\@namedef{noextras\CurrentOption}{\lccode`\'=0}
%    \end{macrocode}
% \end{macro}
% \end{macro}
%
%    One more thing |\extrasfrench| needs to do is to make sure that
%    |\frenchspacing| is in effect.  |\noextrasfrench| will switch
%    |\frenchspacing| off again.
%    \begin{macrocode}
  \expandafter\addto\csname extras\CurrentOption\endcsname{%
    \bbl@frenchspacing}
  \expandafter\addto\csname noextras\CurrentOption\endcsname{%
    \bbl@nonfrenchspacing}
%    \end{macrocode}
%
%  \subsection{Punctuation}
%  \label{sec-punct}
%
%    As long as no better solution is available%
%    \footnote{Lua\TeX, or pdf\TeX{} might provide alternatives in
%       the future\dots},
%    the `double punctuation' characters (|;| |!| |?| and |:|) have to
%    be made |\active| for an automatic control of the amount of space
%    to insert before them. Before doing so, we have to save the
%    standard definition of |\@makecaption| (which includes two ':')
%    to compare it later to its definition at the |\begin{document}|.
%    \begin{macrocode}
\long\def\STD@makecaption#1#2{%
  \vskip\abovecaptionskip
  \sbox\@tempboxa{#1: #2}%
  \ifdim \wd\@tempboxa >\hsize
    #1: #2\par
  \else
    \global \@minipagefalse
    \hb@xt@\hsize{\hfil\box\@tempboxa\hfil}%
  \fi
  \vskip\belowcaptionskip}%
%    \end{macrocode}
%
%    We define a new `if' |\FBpunct@active| which will be made false
%    whenever a better alternative will be available. Another `if'
%    |\FBAutoSpacePunctuation| needs to be defined now.
%    \begin{macrocode}
\newif\ifFBpunct@active          \FBpunct@activetrue
\newif\ifFBAutoSpacePunctuation  \FBAutoSpacePunctuationtrue
%    \end{macrocode}
%    The following code makes the four characters |;| |!| |?| and |:|
%    `active' and provides their definitions.
%    \begin{macrocode}
\ifFBpunct@active
  \initiate@active@char{:}
  \initiate@active@char{;}
  \initiate@active@char{!}
  \initiate@active@char{?}
%    \end{macrocode}
%    We first tune the amount of space before \texttt{;}
%    \texttt{!}  \texttt{?} and \texttt{:}.  This should only happen
%    in horizontal mode, hence the test |\ifhmode|.
%
%    In horizontal mode, if a space has been typed before `;' we
%    remove it and put an unbreakable |\thinspace| instead. If no
%    space has been typed, we add |\FDP@thinspace| which will be
%    defined, up to the user's wishes, as an automatic added
%    thin space, or as |\@empty|.
%    \begin{macrocode}
  \declare@shorthand{french}{;}{%
      \ifhmode
      \ifdim\lastskip>\z@
          \unskip\penalty\@M\thinspace
          \else
            \FDP@thinspace
        \fi
      \fi
%    \end{macrocode}
%    Now we can insert a |;| character.
%    \begin{macrocode}
      \string;}
%    \end{macrocode}
%    The next three definitions are very similar.
%    \begin{macrocode}
  \declare@shorthand{french}{!}{%
      \ifhmode
        \ifdim\lastskip>\z@
          \unskip\penalty\@M\thinspace
        \else
          \FDP@thinspace
        \fi
      \fi
      \string!}
  \declare@shorthand{french}{?}{%
      \ifhmode
        \ifdim\lastskip>\z@
          \unskip\penalty\@M\thinspace
        \else
          \FDP@thinspace
        \fi
      \fi
      \string?}
%    \end{macrocode}
%    According to the I.N. specifications, the `:' requires a normal
%    space before it, but some people prefer a |\thinspace| (just
%    like the other three). We define |\Fcolonspace| to hold the
%    required amount of space (user customisable).
%    \begin{macrocode}
  \newcommand*{\Fcolonspace}{\space}
  \declare@shorthand{french}{:}{%
      \ifhmode
        \ifdim\lastskip>\z@
          \unskip\penalty\@M\Fcolonspace
        \else
          \FDP@colonspace
        \fi
      \fi
      \string:}
%    \end{macrocode}
%
% \changes{v2.3a}{2008/10/10}{\cs{NoAutoSpaceBeforeFDP} and
%    \cs{AutoSpaceBeforeFDP} now set the flag
%    \cs{ifFBAutoSpacePunctuation} accordingly (LaTeX only).}
%
%  \begin{macro}{\AutoSpaceBeforeFDP}
%  \begin{macro}{\NoAutoSpaceBeforeFDP}
%    |\FDP@thinspace| and |\FDP@space| are defined as unbreakable
%    spaces by |\autospace@beforeFDP| or as |\@empty| by
%    |\noautospace@beforeFDP| (internal commands), user commands
%    |\AutoSpaceBeforeFDP| and |\NoAutoSpaceBeforeFDP| do the same and
%    take care of the flag |\ifFBAutoSpacePunctuation| in \LaTeX{}.
%    Set the default now for Plain (done later for \LaTeX).
%    \begin{macrocode}
  \def\autospace@beforeFDP{%
          \def\FDP@thinspace{\penalty\@M\thinspace}%
          \def\FDP@colonspace{\penalty\@M\Fcolonspace}}
  \def\noautospace@beforeFDP{\let\FDP@thinspace\@empty
                            \let\FDP@colonspace\@empty}
  \ifLaTeXe
    \def\AutoSpaceBeforeFDP{\autospace@beforeFDP
                            \FBAutoSpacePunctuationtrue}
    \def\NoAutoSpaceBeforeFDP{\noautospace@beforeFDP
                              \FBAutoSpacePunctuationfalse}
  \else
    \let\AutoSpaceBeforeFDP\autospace@beforeFDP
    \let\NoAutoSpaceBeforeFDP\noautospace@beforeFDP
    \AutoSpaceBeforeFDP
  \fi
%    \end{macrocode}
% \end{macro}
% \end{macro}
%
% \changes{v2.3a}{2008/10/10}{In LaTeX, frenchb no longer adds spaces
%     before `double punctuation' characters in computer code.
%     Suggested by Yannis Haralambous.}
%
% \changes{v2.3c}{2009/02/07}{Commands \cs{ttfamily}, \cs{rmfamily}
%    and \cs{sffamily} have to be robust.  Bug introduced in 2.3a,
%    pointed out by Manuel P\'egouri\'e-Gonnard.}
%
%    In \LaTeXe{} |\ttfamily| (and hence |\texttt|) will be redefined
%    `AtBeginDocument' as |\ttfamilyFB| so that no space
%    is added before the four |; : ! ?| characters, even if
%    |AutoSpacePunctuation| is true.  |\rmfamily| and |\sffamily| need
%    to be redefined also (|\ttfamily| is not always used inside a
%    group, its effect can be cancelled by |\rmfamily| or |\sffamily|).
%
%    These redefinitions can be canceled if necessary, for instance to
%    recompile older documents, see option |OriginalTypewriter| below.
%    \begin{macrocode}
  \ifLaTeXe
    \let\ttfamilyORI\ttfamily
    \let\rmfamilyORI\rmfamily
    \let\sffamilyORI\sffamily
    \DeclareRobustCommand\ttfamilyFB{%
         \noautospace@beforeFDP\ttfamilyORI}%
    \DeclareRobustCommand\rmfamilyFB{%
         \ifFBAutoSpacePunctuation
            \autospace@beforeFDP\rmfamilyORI
         \else
            \noautospace@beforeFDP\rmfamilyORI
         \fi}%
    \DeclareRobustCommand\sffamilyFB{%
         \ifFBAutoSpacePunctuation
            \autospace@beforeFDP\sffamilyORI
         \else
            \noautospace@beforeFDP\sffamilyORI
         \fi}%
  \fi
%    \end{macrocode}
%
%    When the active characters appear in an environment where their
%    French behaviour is not wanted they should give an `expected'
%    result. Therefore we define shorthands at system level as well.
%    \begin{macrocode}
  \declare@shorthand{system}{:}{\string:}
  \declare@shorthand{system}{!}{\string!}
  \declare@shorthand{system}{?}{\string?}
  \declare@shorthand{system}{;}{\string;}
%    \end{macrocode}
%    We specify that the French group of shorthands should be used.
%    \begin{macrocode}
  \addto\extrasfrench{%
    \languageshorthands{french}%
%    \end{macrocode}
%    These characters are `turned on' once, later their definition may
%    vary. Don't misunderstand the following code: they keep being
%    active all along the document, even when leaving French.
%    \begin{macrocode}
    \bbl@activate{:}\bbl@activate{;}%
    \bbl@activate{!}\bbl@activate{?}%
  }
  \addto\noextrasfrench{%
  \bbl@deactivate{:}\bbl@deactivate{;}%
  \bbl@deactivate{!}\bbl@deactivate{?}}
\fi
%    \end{macrocode}
%
%  \subsection{French quotation marks}
%
%  \begin{macro}{\og}
%  \begin{macro}{\fg}
%    The top macros for quotation marks will be called |\og|
%    (``\underline{o}uvrez \underline{g}uillemets'') and |\fg|
%    (``\underline{f}ermez \underline{g}uillemets'').
%    Another option for typesetting quotes in multilingual texts
%    is to use the package |csquotes.sty| and its command |\enquote|.
%
%    \begin{macrocode}
\newcommand*{\og}{\@empty}
\newcommand*{\fg}{\@empty}
%    \end{macrocode}
%  \end{macro}
%  \end{macro}
%
%  \begin{macro}{\guillemotleft}
%  \begin{macro}{\guillemotright}
%    \LaTeX{} users are supposed to use 8-bit output encodings (T1,
%    LY1,\dots) to typeset French, those who still stick to OT1 should
%    call |aeguill.sty| or a similar package. In both cases the
%    commands |\guillemotleft| and |\guillemotright| will print the
%    French opening and closing quote characters from the output font.
%    For XeLaTeX, |\guillemotleft| and |\guillemotright| are defined
%    by package \file{xunicode.sty}.
%    We will check `AtBeginDocument' that the proper output encodings
%    are in use (see end of section~\ref{sec-keyval}).
%
%    We give the following definitions for Plain users only as a (poor)
%    fall-back, they are welcome to change them for anything better.
%    \begin{macrocode}
\ifLaTeXe
\else
  \ifx\guillemotleft\@undefined
    \def\guillemotleft{\leavevmode\raise0.25ex
                       \hbox{$\scriptscriptstyle\ll$}}
  \fi
  \ifx\guillemotright\@undefined
    \def\guillemotright{\raise0.25ex
                        \hbox{$\scriptscriptstyle\gg$}}
  \fi
  \let\xspace\relax
\fi
%    \end{macrocode}
%  \end{macro}
%  \end{macro}
%
%    The next step is to provide correct spacing after |\guillemotleft|
%    and before |\guillemotright|: a space precedes and follows
%    quotation marks but no line break is allowed neither \emph{after}
%    the opening one, nor \emph{before} the closing one.
%    |\FBguill@spacing| which does the spacing, has been fine tuned by
%    Thierry Bouche.  French quotes (including spacing) are printed by
%    |\FB@og| and |\FB@fg|, the expansion of the top level commands
%    |\og| and |\og| is different in and outside French.
%    We'll try to be smart to users of David~Carlisle's |xspace|
%    package: if this package is loaded there will be no need for |{}|
%    or |\ | to get a space after |\fg|, otherwise |\xspace| will be
%    defined as |\relax| (done at the end of this file).
%
%    \begin{macrocode}
\newcommand*{\FBguill@spacing}{\penalty\@M\hskip.8\fontdimen2\font
                                            plus.3\fontdimen3\font
                                           minus.8\fontdimen4\font}
\DeclareRobustCommand*{\FB@og}{\leavevmode
                               \guillemotleft\FBguill@spacing}
\DeclareRobustCommand*{\FB@fg}{\ifdim\lastskip>\z@\unskip\fi
                               \FBguill@spacing\guillemotright\xspace}
%    \end{macrocode}
%
%    The top level definitions for French quotation marks are switched
%    on and off through the |\extrasfrench| |\noextrasfrench|
%    mechanism. Outside French, |\og| and |\fg| will typeset standard
%    English opening and closing double quotes.
%
%    \begin{macrocode}
\ifLaTeXe
  \def\bbl@frenchguillemets{\renewcommand*{\og}{\FB@og}%
                            \renewcommand*{\fg}{\FB@fg}}
  \def\bbl@nonfrenchguillemets{\renewcommand*{\og}{\textquotedblleft}%
            \renewcommand*{\fg}{\ifdim\lastskip>\z@\unskip\fi
                                   \textquotedblright}}
\else
   \def\bbl@frenchguillemets{\let\og\FB@og
                             \let\fg\FB@fg}
   \def\bbl@nonfrenchguillemets{\def\og{``}%
                     \def\fg{\ifdim\lastskip>\z@\unskip\fi ''}}
\fi
\expandafter\addto\csname extras\CurrentOption\endcsname{%
  \bbl@frenchguillemets}
\expandafter\addto\csname noextras\CurrentOption\endcsname{%
  \bbl@nonfrenchguillemets}
%    \end{macrocode}
%
%  \subsection{Date in French}
%
% \begin{macro}{\datefrench}
%    The macro |\datefrench| redefines the command |\today| to
%    produce French dates.
%
% \changes{v2.0}{2006/11/06}{2 '\cs{relax}' added in
%    \cs{today}'s definition.}
%
% \changes{v2.1a}{2008/03/25}{\cs{today} changed (correction in 2.0
%    was wrong: \cs{today} was printed without spaces in toc).}
%
%    \begin{macrocode}
\@namedef{date\CurrentOption}{%
  \def\today{{\number\day}\ifnum1=\day {\ier}\fi \space
    \ifcase\month
      \or janvier\or f\'evrier\or mars\or avril\or mai\or juin\or
      juillet\or ao\^ut\or septembre\or octobre\or novembre\or
      d\'ecembre\fi
    \space \number\year}}
%    \end{macrocode}
% \end{macro}
%
%  \subsection{Extra utilities}
%
%    Let's provide the French user with some extra utilities.
%
% \changes{v2.1a}{2008/03/24}{Command \cs{fup} added to produce
%    better superscripts than \cs{textsuperscript}.}
%
%  \begin{macro}{\up}
%
% \changes{v2.1c}{2008/04/29}{Provide a temporary definition
%    (hyperref safe) of \cs{up} in case it has to be expanded in
%    the preamble (by beamer's \cs{title} command for instance).}
%
%  \begin{macro}{\fup}
%
% \changes{v2.1b}{2008/04/02}{Command \cs{fup} changed to use
%    real superscripts from fourier v. 1.6.}
%
% \changes{v2.2a}{2008/05/08}{\cs{newif} and \cs{newdimen} moved
%    before \cs{ifLaTeXe} to avoid an error with plainTeX.}
%
% \changes{v2.3a}{2008/09/30}{\cs{lowercase} changed to
%    \cs{MakeLowercase} as the former doesn't work for non ASCII
%    characters in encodings like applemac, utf-8,\dots}
%
%    |\up| eases the typesetting of superscripts like
%    `1\textsuperscript{er}'.  Up to version 2.0 of |frenchb| |\up| was
%    just a shortcut for |\textsuperscript| in \LaTeXe, but several
%    users complained that |\textsuperscript| typesets superscripts
%    too high and too big, so we now define |\fup| as an attempt to
%    produce better looking superscripts.  |\up| is defined as |\fup|
%    but can be redefined by |\frenchbsetup{FrenchSuperscripts=false}|
%    as |\textsuperscript| for compatibility with previous versions.
%
%    When a font has built-in superscripts, the best thing to do is
%    to just use them, otherwise |\fup| has to simulate superscripts
%    by scaling and raising ordinary letters.  Scaling is done using
%    package \file{scalefnt} which will be loaded at the end of
%    \babel's loading (|frenchb| being an option of babel, it cannot
%    load a package while being read).
%
%    \begin{macrocode}
\newif\ifFB@poorman
\newdimen\FB@Mht
\ifLaTeXe
  \AtEndOfPackage{\RequirePackage{scalefnt}}
%    \end{macrocode}
%    |\FB@up@fake| holds the definition of fake superscripts.
%    The scaling ratio is 0.65, raising is computed to put the top of
%    lower case letters (like `m') just under the top  of upper case
%    letters (like `M'), precisely 12\% down.  The chosen settings
%    look correct for most fonts, but can be tuned by the end-user
%    if necessary by changing |\FBsupR| and |\FBsupS| commands.
%
%    |\FB@lc| is defined as |\MakeLowercase| to inhibit the uppercasing
%    of superscripts (this may happen in page headers with the standard
%    classes but is wrong); |\FB@lc| can be redefined to do nothing
%    by option |LowercaseSuperscripts=false| of |\frenchbsetup{}|.
%    \begin{macrocode}
  \newcommand*{\FBsupR}{-0.12}
  \newcommand*{\FBsupS}{0.65}
  \newcommand*{\FB@lc}[1]{\MakeLowercase{#1}}
  \DeclareRobustCommand*{\FB@up@fake}[1]{%
    \settoheight{\FB@Mht}{M}%
    \addtolength{\FB@Mht}{\FBsupR \FB@Mht}%
    \addtolength{\FB@Mht}{-\FBsupS ex}%
    \raisebox{\FB@Mht}{\scalefont{\FBsupS}{\FB@lc{#1}}}%
    }
%    \end{macrocode}
%    The only packages I currently know to take advantage of real
%    superscripts are a) \file{xltxtra} used in conjunction with
%    XeLaTeX and OpenType fonts having the font feature
%    'VerticalPosition=Superior' (\file{xltxtra} defines
%    |\realsuperscript| and |\fakesuperscript|) and b) \file{fourier}
%    (from version 1.6) when Expert Utopia fonts are available.
%
%    |\FB@up| checks whether the current font is a Type1 `Expert'
%    (or `Pro') font with real superscripts or not (the code works
%    currently only with \file{fourier-1.6} but could work with any
%    Expert Type1 font with built-in superscripts, see below), and
%    decides to use real or fake superscripts.
%    It works as follows: the content of |\f@family| (family name of
%    the current font) is split by |\FB@split| into two pieces, the
%    first three characters (`|fut|' for Fourier, `|ppl|' for Adobe's
%    Palatino, \dots) stored in |\FB@firstthree| and the rest stored
%    in |\FB@suffix| which is expected to be `|x|' or `|j|' for expert
%    fonts.
%    \begin{macrocode}
  \def\FB@split#1#2#3#4\@nil{\def\FB@firstthree{#1#2#3}%
                             \def\FB@suffix{#4}}
  \def\FB@x{x}
  \def\FB@j{j}
  \DeclareRobustCommand*{\FB@up}[1]{%
    \bgroup \FB@poormantrue
      \expandafter\FB@split\f@family\@nil
%    \end{macrocode}
%    Then |\FB@up| looks for a \file{.fd} file named \file{t1fut-sup.fd}
%    (Fourier) or \file{t1ppl-sup.fd} (Palatino), etc. supposed to
%    define the subfamily (|fut-sup| or |ppl-sup|, etc.) giving access
%    to the built-in superscripts.  If the \file{.fd} file is not found
%    by |\IfFileExists|, |\FB@up| falls back on fake superscripts,
%    otherwise |\FB@suffix| is checked to decide whether to use fake or
%    real superscripts.
%    \begin{macrocode}
      \edef\reserved@a{\lowercase{%
         \noexpand\IfFileExists{\f@encoding\FB@firstthree -sup.fd}}}%
      \reserved@a
        {\ifx\FB@suffix\FB@x \FB@poormanfalse\fi
         \ifx\FB@suffix\FB@j \FB@poormanfalse\fi
         \ifFB@poorman \FB@up@fake{#1}%
         \else         \FB@up@real{#1}%
         \fi}%
        {\FB@up@fake{#1}}%
    \egroup}
%    \end{macrocode}
%    |\FB@up@real| just picks up the superscripts from the subfamily
%    (and forces lowercase).
%    \begin{macrocode}
  \newcommand*{\FB@up@real}[1]{\bgroup
       \fontfamily{\FB@firstthree -sup}\selectfont \FB@lc{#1}\egroup}
%    \end{macrocode}
%    |\fup| is now defined as |\FB@up| unless |\realsuperscript| is
%    defined (occurs with XeLaTeX calling \file{xltxtra.sty}).
%    \begin{macrocode}
  \DeclareRobustCommand*{\fup}[1]{%
    \@ifundefined{realsuperscript}%
      {\FB@up{#1}}%
      {\bgroup\let\fakesuperscript\FB@up@fake
            \realsuperscript{\FB@lc{#1}}\egroup}}
%    \end{macrocode}
%    Temporary definition of |up| (redefined `AtBeginDocument').
%    \begin{macrocode}
  \newcommand*{\up}{\relax}
%    \end{macrocode}
%    Poor man's definition of |\up| for Plain. In \LaTeXe,
%    |\up| will be defined as |\fup| or |\textsuperscript| later on
%    while processing the options of |\frenchbsetup{}|.
%    \begin{macrocode}
\else
  \newcommand*{\up}[1]{\leavevmode\raise1ex\hbox{\sevenrm #1}}
\fi
%    \end{macrocode}
%  \end{macro}
%  \end{macro}
%
%  \begin{macro}{\ieme}
%  \begin{macro}{\ier}
%  \begin{macro}{\iere}
%  \begin{macro}{\iemes}
%  \begin{macro}{\iers}
%  \begin{macro}{\ieres}
%  Some handy macros for those who don't know how to abbreviate ordinals:
%    \begin{macrocode}
\def\ieme{\up{\lowercase{e}}\xspace}
\def\iemes{\up{\lowercase{es}}\xspace}
\def\ier{\up{\lowercase{er}}\xspace}
\def\iers{\up{\lowercase{ers}}\xspace}
\def\iere{\up{\lowercase{re}}\xspace}
\def\ieres{\up{\lowercase{res}}\xspace}
%    \end{macrocode}
%  \end{macro}
%  \end{macro}
%  \end{macro}
%  \end{macro}
%  \end{macro}
%  \end{macro}
%
% \changes{v2.1c}{2008/04/29}{Added commands \cs{Nos} and \cs{nos}.}
%
%  \begin{macro}{\No}
%  \begin{macro}{\no}
%  \begin{macro}{\Nos}
%  \begin{macro}{\nos}
%  \begin{macro}{\primo}
%  \begin{macro}{\fprimo)}
%    And some more macros relying on |\up| for numbering,
%    first two support macros.
%    \begin{macrocode}
\newcommand*{\FrenchEnumerate}[1]{%
                       #1\up{\lowercase{o}}\kern+.3em}
\newcommand*{\FrenchPopularEnumerate}[1]{%
                       #1\up{\lowercase{o}})\kern+.3em}
%    \end{macrocode}
%
%    Typing |\primo| should result in `$1^{\rm o}$\kern+.3em',
%    \begin{macrocode}
\def\primo{\FrenchEnumerate1}
\def\secundo{\FrenchEnumerate2}
\def\tertio{\FrenchEnumerate3}
\def\quarto{\FrenchEnumerate4}
%    \end{macrocode}
%    while typing |\fprimo)| gives `1$^{\rm o}$)\kern+.3em.
%    \begin{macrocode}
\def\fprimo){\FrenchPopularEnumerate1}
\def\fsecundo){\FrenchPopularEnumerate2}
\def\ftertio){\FrenchPopularEnumerate3}
\def\fquarto){\FrenchPopularEnumerate4}
%    \end{macrocode}
%
%    Let's provide four macros for the common abbreviations
%    of ``Num\'ero''.
%    \begin{macrocode}
\DeclareRobustCommand*{\No}{N\up{\lowercase{o}}\kern+.2em}
\DeclareRobustCommand*{\no}{n\up{\lowercase{o}}\kern+.2em}
\DeclareRobustCommand*{\Nos}{N\up{\lowercase{os}}\kern+.2em}
\DeclareRobustCommand*{\nos}{n\up{\lowercase{os}}\kern+.2em}
%    \end{macrocode}
%  \end{macro}
%  \end{macro}
%  \end{macro}
%  \end{macro}
%  \end{macro}
%  \end{macro}
%
%  \begin{macro}{\bsc}
%    As family names should be written in small capitals and never be
%    hyphenated, we provide a command (its name comes from Boxed Small
%    Caps) to input them easily.  Note that this command has changed
%    with version~2 of |frenchb|: a |\kern0pt| is used instead of |\hbox|
%    because |\hbox| would break microtype's font expansion; as a
%    (positive?) side effect, composed names (such as Dupont-Durand)
%    can now be hyphenated on explicit hyphens.
%    Usage: |Jean~\bsc{Duchemin}|.
%
% \changes{v2.0}{2006/11/06}{\cs{hbox} dropped, replaced by
%    \cs{kern0pt}.}
%
%    \begin{macrocode}
\DeclareRobustCommand*{\bsc}[1]{\leavevmode\begingroup\kern0pt
                                           \scshape #1\endgroup}
\ifLaTeXe\else\let\scshape\relax\fi
%    \end{macrocode}
%  \end{macro}
%
%    Some definitions for special characters.  We won't define |\tilde|
%    as a Text Symbol not to conflict with the macro |\tilde| for math
%    mode and use the name |\tild| instead. Note that |\boi| may
%    \emph{not} be used in math mode, its name in math mode is
%    |\backslash|.  |\degre|  can be accessed by the command |\r{}|
%    for ring accent.
%
%    \begin{macrocode}
\ifLaTeXe
  \DeclareTextSymbol{\at}{T1}{64}
  \DeclareTextSymbol{\circonflexe}{T1}{94}
  \DeclareTextSymbol{\tild}{T1}{126}
  \DeclareTextSymbolDefault{\at}{T1}
  \DeclareTextSymbolDefault{\circonflexe}{T1}
  \DeclareTextSymbolDefault{\tild}{T1}
  \DeclareRobustCommand*{\boi}{\textbackslash}
  \DeclareRobustCommand*{\degre}{\r{}}
\else
  \def\T@one{T1}
  \ifx\f@encoding\T@one
    \newcommand*{\degre}{\char6}
  \else
    \newcommand*{\degre}{\char23}
  \fi
  \newcommand*{\at}{\char64}
  \newcommand*{\circonflexe}{\char94}
  \newcommand*{\tild}{\char126}
  \newcommand*{\boi}{$\backslash$}
\fi
%    \end{macrocode}
%
%  \begin{macro}{\degres}
%    We now define a macro |\degres| for typesetting the abbreviation
%    for `degrees' (as in `degrees Celsius'). As the bounding box of
%    the character `degree' has \emph{very} different widths in CM/EC
%    and PostScript fonts, we fix the width of the bounding box of
%    |\degres| to 0.3\,em, this lets the symbol `degree' stick to the
%    preceding (e.g., |45\degres|) or following character
%    (e.g., |20~\degres C|).
%
%    If the \TeX{} Companion fonts are available (\file{textcomp.sty}),
%    we pick up |\textdegree| from them instead of using emulating
%    `degrees' from the |\r{}| accent. Otherwise we overwrite the
%    (poor) definition of |\textdegree| given in \file{latin1.def},
%    \file{applemac.def} etc. (called by  \file{inputenc.sty}) by
%    our definition of |\degres|. We also advice the user (once only)
%    to use TS1-encoding.
%
% \changes{v2.1c}{2008/04/29}{Provide a temporary definition (hyperref
%    safe) of \cs{degres} in case it has to be expanded in the preamble
%    (by beamer's \cs{title} command for instance).}
%
%    \begin{macrocode}
\ifLaTeXe
  \newcommand*{\degres}{\degre}
  \def\Warning@degree@TSone{%
        \PackageWarning{frenchb.ldf}{%
           Degrees would look better in TS1-encoding:
           \MessageBreak add \protect
           \usepackage{textcomp} to the preamble.
           \MessageBreak Degrees used}}
  \AtBeginDocument{\expandafter\ifx\csname M@TS1\endcsname\relax
                     \DeclareRobustCommand*{\degres}{%
                       \leavevmode\hbox to 0.3em{\hss\degre\hss}%
                       \Warning@degree@TSone
                       \global\let\Warning@degree@TSone\relax}%
                      \let\textdegree\degres
                   \else
                     \DeclareRobustCommand*{\degres}{%
                         \hbox{\UseTextSymbol{TS1}{\textdegree}}}%
                   \fi}
\else
  \newcommand*{\degres}{%
    \leavevmode\hbox to 0.3em{\hss\degre\hss}}
\fi
%    \end{macrocode}
%  \end{macro}
%
%  \subsection{Formatting numbers}
%  \label{sec-numbers}
%
%  \begin{macro}{\DecimalMathComma}
%  \begin{macro}{\StandardMathComma}
%    As mentioned in the \TeX{}book p.~134, the comma is of type
%    |\mathpunct| in math mode: it is automatically followed by a
%    space. This is convenient in lists and intervals but
%    unpleasant when the comma is used as a decimal separator
%    in French: it has to be entered as |{,}|.
%    |\DecimalMathComma| makes the comma be an ordinary character
%    (of type |\mathord|) in French \emph{only} (no space added);
%    |\StandardMathComma| switches back to the standard behaviour
%    of the comma.
%    \begin{macrocode}
\newcount\std@mcc
\newcount\dec@mcc
\std@mcc=\mathcode`\,
\dec@mcc=\std@mcc
\@tempcnta=\std@mcc
\divide\@tempcnta by "1000
\multiply\@tempcnta by "1000
\advance\dec@mcc by -\@tempcnta
\newcommand*{\DecimalMathComma}{\iflanguage{french}%
                                 {\mathcode`\,=\dec@mcc}{}%
              \addto\extrasfrench{\mathcode`\,=\dec@mcc}}
\newcommand*{\StandardMathComma}{\mathcode`\,=\std@mcc
             \addto\extrasfrench{\mathcode`\,=\std@mcc}}
\expandafter\addto\csname noextras\CurrentOption\endcsname{%
   \mathcode`\,=\std@mcc}
%    \end{macrocode}
%  \end{macro}
%  \end{macro}
%
%  \begin{macro}{\nombre}
%
% \changes{v2.0}{2006/11/06}{\cs{nombre} requires now numprint.sty.}
%
%    The command |\nombre| is now borrowed from |numprint.sty| for
%    \LaTeXe.  There is no point to maintain the former tricky code
%    when a package is dedicated to do the same job and more.
%    For Plain based formats, |\nombre| no longer formats numbers,
%    it prints them as is and issues a warning about the change.
%
%    Fake command |\nombre| for Plain based formats, warning users of
%    |frenchb| v.1.x. of the change.
%    \begin{macrocode}
\newcommand*{\nombre}[1]{{#1}\message{%
     *** \noexpand\nombre no longer formats numbers\string! ***}}%
%    \end{macrocode}
%  \end{macro}
%
%    The next definitions only make sense for \LaTeXe. Let's cleanup
%    and exit if the format in Plain based.
%
%    \begin{macrocode}
\let\FBstop@here\relax
\def\FBclean@on@exit{\let\ifLaTeXe\@undefined
                     \let\LaTeXetrue\@undefined
                     \let\LaTeXefalse\@undefined}
\ifx\magnification\@undefined
\else
   \def\FBstop@here{\let\STD@makecaption\relax
                    \FBclean@on@exit
                    \ldf@quit\CurrentOption\endinput}
\fi
\FBstop@here
%    \end{macrocode}
%
%    What follows now is for \LaTeXe{} \emph{only}.
%    We redefine |\nombre| for \LaTeXe. A warning is issued
%    at the first call of |\nombre| if |\numprint| is not
%    defined, suggesting what to do.  The package |numprint|
%    is \emph{not} loaded automatically by |frenchb| because of
%    possible options conflict.
%
%    \begin{macrocode}
\renewcommand*{\nombre}[1]{\Warning@nombre\numprint{#1}}
\newcommand*{\Warning@nombre}{%
   \@ifundefined{numprint}%
      {\PackageWarning{frenchb.ldf}{%
         \protect\nombre\space now relies on package numprint.sty,
         \MessageBreak add \protect
         \usepackage[autolanguage]{numprint}\MessageBreak
         to your preamble *after* loading babel, \MessageBreak
         see file numprint.pdf for other options.\MessageBreak
         \protect\nombre\space called}%
       \global\let\Warning@nombre\relax
       \global\let\numprint\relax
      }{}%
}
%    \end{macrocode}
%
% \changes{v2.0c}{2007/06/25}{There is no need to define here
%    numprint's command \cs{npstylefrench}, it will be redefined
%    `AtBeginDocument' by \cs{FBprocess@options}.}
%
% \changes{v2.0c}{2007/06/25}{\cs{ThinSpaceInFrenchNumbers} added
%     for compatibility with frenchb-1.x.}
%
%    \begin{macrocode}
\newcommand*{\ThinSpaceInFrenchNumbers}{%
   \PackageWarning{frenchb.ldf}{%
         Type \protect\frenchbsetup{ThinSpaceInFrenchNumbers}
         \MessageBreak Command \protect\ThinSpaceInFrenchNumbers\space
         is no longer\MessageBreak  defined in frenchb v.2,}}
%    \end{macrocode}
%
%  \subsection{Caption names}
%
%    The next step consists of defining the French equivalents for
%    the \LaTeX{} caption names.
%
% \begin{macro}{\captionsfrench}
%    Let's first define  |\captionsfrench| which sets all strings used
%    in the four standard document classes provided with \LaTeX.
%
% \changes{v2.0}{2006/11/06}{`Fig.' changed to `Figure' and
%     `Tab.' to `Table'.}
%
% \changes{v2.0}{2006/12/15}{Set \cs{CaptionSeparator} in
%     \cs{extrasfrench} now instead of \cs{captionsfrench}
%     because it has to be reset when leaving French.}
%
%    \begin{macrocode}
\@namedef{captions\CurrentOption}{%
   \def\refname{R\'ef\'erences}%
   \def\abstractname{R\'esum\'e}%
   \def\bibname{Bibliographie}%
   \def\prefacename{Pr\'eface}%
   \def\chaptername{Chapitre}%
   \def\appendixname{Annexe}%
   \def\contentsname{Table des mati\`eres}%
   \def\listfigurename{Table des figures}%
   \def\listtablename{Liste des tableaux}%
   \def\indexname{Index}%
   \def\figurename{{\scshape Figure}}%
   \def\tablename{{\scshape Table}}%
%    \end{macrocode}
%   ``Premi\`ere partie'' instead of ``Part I''.
%    \begin{macrocode}
   \def\partname{\protect\@Fpt partie}%
   \def\@Fpt{{\ifcase\value{part}\or Premi\`ere\or Deuxi\`eme\or
   Troisi\`eme\or Quatri\`eme\or Cinqui\`eme\or Sixi\`eme\or
   Septi\`eme\or Huiti\`eme\or Neuvi\`eme\or Dixi\`eme\or Onzi\`eme\or
   Douzi\`eme\or Treizi\`eme\or Quatorzi\`eme\or Quinzi\`eme\or
   Seizi\`eme\or Dix-septi\`eme\or Dix-huiti\`eme\or Dix-neuvi\`eme\or
   Vingti\`eme\fi}\space\def\thepart{}}%
   \def\pagename{page}%
   \def\seename{{\emph{voir}}}%
   \def\alsoname{{\emph{voir aussi}}}%
   \def\enclname{P.~J. }%
   \def\ccname{Copie \`a }%
   \def\headtoname{}%
   \def\proofname{D\'emonstration}%
   \def\glossaryname{Glossaire}%
   }
%    \end{macrocode}
% \end{macro}
%
%    As some users who choose |frenchb| or |francais| as option of
%    \babel, might customise |\captionsfrenchb| or |\captionsfrancais|
%    in the preamble, we merge their changes at the |\begin{document}|
%    when they do so.
%    The other variants of French (canadien, acadian) are defined by
%    checking if the relevant option was used and then adding one extra
%    level of expansion.
%
%    \begin{macrocode}
\AtBeginDocument{\let\captions@French\captionsfrench
                 \@ifundefined{captionsfrenchb}%
                    {\let\captions@Frenchb\relax}%
                    {\let\captions@Frenchb\captionsfrenchb}%
                 \@ifundefined{captionsfrancais}%
                    {\let\captions@Francais\relax}%
                    {\let\captions@Francais\captionsfrancais}%
                 \def\captionsfrench{\captions@French
                        \captions@Francais\captions@Frenchb}%
                 \def\captionsfrancais{\captionsfrench}%
                 \def\captionsfrenchb{\captionsfrench}%
                 \iflanguage{french}{\captionsfrench}{}%
                }
\@ifpackagewith{babel}{canadien}{%
  \def\captionscanadien{\captionsfrench}%
  \def\datecanadien{\datefrench}%
  \def\extrascanadien{\extrasfrench}%
  \def\noextrascanadien{\noextrasfrench}%
  }{}
\@ifpackagewith{babel}{acadian}{%
  \def\captionsacadian{\captionsfrench}%
  \def\dateacadian{\datefrench}%
  \def\extrasacadian{\extrasfrench}%
  \def\noextrasacadian{\noextrasfrench}%
  }{}
%    \end{macrocode}
%
% \begin{macro}{\CaptionSeparator}
%    Let's consider now captions in figures and tables.
%    In French, captions in figures and tables should be printed with
%    endash (`--') instead of the standard `:'.
%
%    The standard definition of |\@makecaption| (e.g., the one provided
%    in article.cls, report.cls, book.cls which is frozen for \LaTeXe{}
%    according to Frank Mittelbach), has been saved in
%    |\STD@makecaption| before making `:' active
%    (see section~\ref{sec-punct}). `AtBeginDocument' we compare it to
%    its current definition (some classes like koma-script classes,
%    AMS classes, ua-thesis.cls\dots change it).
%    If they are identical, |frenchb| just adds a hook called
%    |\CaptionSeparator| to |\@makecaption|, |\CaptionSeparator|
%    defaults to `: ' as in the standard |\@makecaption|, and will be
%    changed to ` -- ' in French.
%    If the definitions differ, |frenchb| doesn't overwrite the changes,
%    but prints a message in the .log file.
%
%    \begin{macrocode}
\def\CaptionSeparator{\string:\space}
\long\def\FB@makecaption#1#2{%
  \vskip\abovecaptionskip
  \sbox\@tempboxa{#1\CaptionSeparator #2}%
  \ifdim \wd\@tempboxa >\hsize
    #1\CaptionSeparator #2\par
  \else
    \global \@minipagefalse
    \hb@xt@\hsize{\hfil\box\@tempboxa\hfil}%
  \fi
  \vskip\belowcaptionskip}
\AtBeginDocument{%
  \ifx\@makecaption\STD@makecaption
      \global\let\@makecaption\FB@makecaption
  \else
    \@ifundefined{@makecaption}{}%
       {\PackageWarning{frenchb.ldf}%
        {The definition of \protect\@makecaption\space
         has been changed,\MessageBreak
         frenchb will NOT customise it;\MessageBreak reported}%
       }%
  \fi
  \let\FB@makecaption\relax
  \let\STD@makecaption\relax
}
\expandafter\addto\csname extras\CurrentOption\endcsname{%
   \def\CaptionSeparator{\space\textendash\space}}
\expandafter\addto\csname noextras\CurrentOption\endcsname{%
    \def\CaptionSeparator{\string:\space}}
%    \end{macrocode}
% \end{macro}
%
%  \subsection{French lists}
%  \label{sec-lists}
%
%  \begin{macro}{\listFB}
%  \begin{macro}{\listORI}
%    Vertical spacing in general lists should be shorter in French
%    texts than the defaults provided by \LaTeX.
%    Note that the easy way, just changing values of vertical spacing
%    parameters when entering French and restoring them to their
%    defaults on exit would not work; as most lists are based on
%    |\list| we will define a variant of |\list| (|\listFB|) to
%    be used in French.
%
%    The amount of vertical space before and after a list is given by
%    |\topsep| + |\parskip| (+ |\partopsep| if the list starts a new
%    paragraph). IMHO, |\parskip| should be added \emph{only} when
%    the list starts a new paragraph, so I subtract |\parskip| from
%    |\topsep| and add it back to |\partopsep|; this will normally
%    make no difference because |\parskip|'s default value is 0pt, but
%    will be noticeable when |\parskip| is \emph{not} null.
%
%    |\endlist| is not redefined, but |\endlistORI| is provided for
%    the users who prefer to define their own lists from the original
%    command, they can code: |\begin{listORI}{}{} \end{listORI}|.
%    \begin{macrocode}
\let\listORI\list
\let\endlistORI\endlist
\def\FB@listsettings{%
      \setlength{\itemsep}{0.4ex plus 0.2ex minus 0.2ex}%
      \setlength{\parsep}{0.4ex plus 0.2ex minus 0.2ex}%
      \setlength{\topsep}{0.8ex plus 0.4ex minus 0.4ex}%
      \setlength{\partopsep}{0.4ex plus 0.2ex minus 0.2ex}%
%    \end{macrocode}
%    |\parskip| is of type `skip', its mean value only (\emph{not
%    the glue}) should be subtracted from |\topsep| and added to
%    |\partopsep|, so convert |\parskip| to a `dimen' using
%    |\@tempdima|.
%    \begin{macrocode}
      \@tempdima=\parskip
      \addtolength{\topsep}{-\@tempdima}%
      \addtolength{\partopsep}{\@tempdima}}%
\def\listFB#1#2{\listORI{#1}{\FB@listsettings #2}}%
\let\endlistFB\endlist
%    \end{macrocode}
%  \end{macro}
%  \end{macro}
%
%  \begin{macro}{\itemizeFB}
%  \begin{macro}{\itemizeORI}
%  \begin{macro}{\bbl@frenchlabelitems}
%  \begin{macro}{\bbl@nonfrenchlabelitems}
%    Let's now consider French itemize lists.  They differ from those
%    provided by the standard \LaTeXe{} classes:
%    \begin{itemize}
%      \item vertical spacing between items, before and after
%         the list, should be \emph{null} with \emph{no glue} added;
%      \item the item labels of a first level list should be vertically
%          aligned on the paragraph's first character (i.e. at
%          |\parindent| from the left margin);
%      \item the `\textbullet' is never used in French itemize-lists,
%          a long dash `--' is preferred for all levels. The item label
%          used in French is stored in |\FrenchLabelItem}|, it defaults
%          to `--' and can be changed using |\frenchbsetup{}| (see
%          section~\ref{sec-keyval}).
%    \end{itemize}
%
%    \begin{macrocode}
\newcommand*{\FrenchLabelItem}{\textendash}
\newcommand*{\Frlabelitemi}{\FrenchLabelItem}
\newcommand*{\Frlabelitemii}{\FrenchLabelItem}
\newcommand*{\Frlabelitemiii}{\FrenchLabelItem}
\newcommand*{\Frlabelitemiv}{\FrenchLabelItem}
%    \end{macrocode}
%    |\bbl@frenchlabelitems| saves current itemize labels and changes
%    them to their value in French. This code should never be executed
%    twice in a row, so we need a new flag that will be set and reset
%    by |\bbl@nonfrenchlabelitems| and |\bbl@frenchlabelitems|.
%    \begin{macrocode}
\newif\ifFB@enterFrench  \FB@enterFrenchtrue
\def\bbl@frenchlabelitems{%
  \ifFB@enterFrench
    \let\@ltiORI\labelitemi
    \let\@ltiiORI\labelitemii
    \let\@ltiiiORI\labelitemiii
    \let\@ltivORI\labelitemiv
    \let\labelitemi\Frlabelitemi
    \let\labelitemii\Frlabelitemii
    \let\labelitemiii\Frlabelitemiii
    \let\labelitemiv\Frlabelitemiv
    \FB@enterFrenchfalse
  \fi
}
\let\itemizeORI\itemize
\let\enditemizeORI\enditemize
\let\enditemizeFB\enditemize
\def\itemizeFB{%
    \ifnum \@itemdepth >\thr@@\@toodeep\else
      \advance\@itemdepth\@ne
      \edef\@itemitem{labelitem\romannumeral\the\@itemdepth}%
      \expandafter
      \listORI
      \csname\@itemitem\endcsname
      {\settowidth{\labelwidth}{\csname\@itemitem\endcsname}%
       \setlength{\leftmargin}{\labelwidth}%
       \addtolength{\leftmargin}{\labelsep}%
       \ifnum\@listdepth=0
         \setlength{\itemindent}{\parindent}%
       \else
         \addtolength{\leftmargin}{\parindent}%
       \fi
       \setlength{\itemsep}{\z@}%
       \setlength{\parsep}{\z@}%
       \setlength{\topsep}{\z@}%
       \setlength{\partopsep}{\z@}%
%    \end{macrocode}
%    |\parskip| is of type `skip', its mean value only (\emph{not
%    the glue}) should be subtracted from |\topsep| and added to
%    |\partopsep|, so convert |\parskip| to a `dimen' using
%    |\@tempdima|.
%    \begin{macrocode}
       \@tempdima=\parskip
       \addtolength{\topsep}{-\@tempdima}%
       \addtolength{\partopsep}{\@tempdima}}%
    \fi}
%    \end{macrocode}
%    The user's changes in labelitems are saved when leaving French for
%    further use when switching back to French.  This code should never
%    be executed twice in a row (toggle with |\bbl@frenchlabelitems|).
%    \begin{macrocode}
\def\bbl@nonfrenchlabelitems{%
  \ifFB@enterFrench
  \else
      \let\Frlabelitemi\labelitemi
      \let\Frlabelitemii\labelitemii
      \let\Frlabelitemiii\labelitemiii
      \let\Frlabelitemiv\labelitemiv
      \let\labelitemi\@ltiORI
      \let\labelitemii\@ltiiORI
      \let\labelitemiii\@ltiiiORI
      \let\labelitemiv\@ltivORI
      \FB@enterFrenchtrue
  \fi
}
%    \end{macrocode}
%  \end{macro}
%  \end{macro}
%  \end{macro}
%  \end{macro}
%
%  \subsection{French indentation of sections}
%  \label{sec-indent}
%
%  \begin{macro}{\bbl@frenchindent}
%  \begin{macro}{\bbl@nonfrenchindent}
%    In French the first paragraph of each section should be indented,
%    this is another difference with US-English. This is controlled by
%    the flag |\if@afterindent|.
%
% \changes{v2.3d}{2009/03/16}{Bug correction: previous versions of
%    frenchb set the flag \cs{if@afterindent} to false outside
%    French which is correct for English but wrong for some languages
%    like Spanish.  Pointed out by Juan Jos\'e Torrens.}
%
%    We need to save the value of the flag |\if@afterindent|
%    `AtBeginDocument' before eventually changing its value.
%
%    \begin{macrocode}
\AtBeginDocument{\ifx\@afterindentfalse\@afterindenttrue
                       \let\@aifORI\@afterindenttrue
                 \else \let\@aifORI\@afterindentfalse
                 \fi
}
\def\bbl@frenchindent{\let\@afterindentfalse\@afterindenttrue
                      \@afterindenttrue}
\def\bbl@nonfrenchindent{\let\@afterindentfalse\@aifORI
                         \@afterindentfalse}
%    \end{macrocode}
%  \end{macro}
%  \end{macro}
%
%  \subsection{Formatting footnotes}
%  \label{sec-footnotes}
%
% \changes{v2.0}{2006/11/06}{Footnotes are now printed
%     by default `\`a la fran\c caise' for the whole document.}
%
% \changes{v2.0b}{2007/04/18}{Footnotes: Just do nothing
%    (except warning) when the bigfoot package is loaded.}
%
%    The |bigfoot| package deeply changes the way footnotes are
%    handled. When |bigfoot| is loaded, we just warn the user that
%    |frenchb| will drop the customisation of footnotes.
%
%    The layout of footnotes is controlled by two flags
%    |\ifFBAutoSpaceFootnotes| and |\ifFBFrenchFootnotes| which are
%    set by options of |\frenchbsetup{}| (see section~\ref{sec-keyval}).
%    Notice that the layout of footnotes \emph{does not depend} on the
%    current language (just think of two footnotes on the same page
%    looking different because one was called in a French part, the
%    other one in English!).
%
%    When |\ifFBAutoSpaceFootnotes| is true, |\@footnotemark| (whose
%    definition is saved at the |\begin{document}| in order to include
%    any customisation that packages might have done) is redefined to
%    add a thin space before the number or symbol calling a footnote
%    (any space typed in is removed first).  This has no effect on
%    the layout of the footnote itself.
%
%    \begin{macrocode}
\AtBeginDocument{\@ifpackageloaded{bigfoot}%
                   {\PackageWarning{frenchb.ldf}%
                     {bigfoot package in use.\MessageBreak
                      frenchb will NOT customise footnotes;\MessageBreak
                      reported}}%
                   {\let\@footnotemarkORI\@footnotemark
                    \def\@footnotemarkFB{\leavevmode\unskip\unkern
                                         \,\@footnotemarkORI}%
                    \ifFBAutoSpaceFootnotes
                      \let\@footnotemark\@footnotemarkFB
                    \fi}%
                }
%    \end{macrocode}
%
%    We then define |\@makefntextFB|, a variant of |\@makefntext|
%    which is responsible for the layout of footnotes, to match the
%    specifications of the French `Imprimerie Nationale':  footnotes
%    will be indented by |\parindentFFN|, numbers (if any) typeset on
%    the baseline (instead of superscripts) and followed by a dot
%    and an half quad space. Whenever symbols are used to number
%    footnotes (as in |\thanks| for instance), we switch back to the
%    standard layout (the French layout of footnotes is meant for
%    footnotes numbered by Arabic or Roman digits).
%
% \changes{v2.0}{2006/11/06}{\cs{parindentFFN} not changed if
%    already defined (required by JA for cah-gut.cls).}
%
% \changes{v2.3b}{2008/12/06}{New commands \cs{dotFFN} and
%    \cs{kernFFN} for more flexibility (suggested by JA).}
%
%    The value of |\parindentFFN| will be redefined at the
%    |\begin{document}|, as the maximum of |\parindent| and 1.5em
%    \emph{unless} it has been set in the preamble (the weird value
%    10in is just for testing whether |\parindentFFN| has been set
%    or not).
%
%    \begin{macrocode}
\newcommand*{\dotFFN}{.}
\newcommand*{\kernFFN}{\kern .5em}
\newdimen\parindentFFN
\parindentFFN=10in
\def\ftnISsymbol{\@fnsymbol\c@footnote}
\long\def\@makefntextFB#1{\ifx\thefootnote\ftnISsymbol
                            \@makefntextORI{#1}%
                          \else
                            \parindent=\parindentFFN
                            \rule\z@\footnotesep
                            \setbox\@tempboxa\hbox{\@thefnmark}%
                            \ifdim\wd\@tempboxa>\z@
                              \llap{\@thefnmark}\dotFFN\kernFFN
                            \fi #1
                          \fi}%
%    \end{macrocode}
%
%    We save the standard definition of |\@makefntext| at the
%    |\begin{document}|, and then redefine |\@makefntext| according to
%    the value of flag |\ifFBFrenchFootnotes| (true or false).
%
%    \begin{macrocode}
\AtBeginDocument{\@ifpackageloaded{bigfoot}{}%
                  {\ifdim\parindentFFN<10in
                   \else
                      \parindentFFN=\parindent
                      \ifdim\parindentFFN<1.5em\parindentFFN=1.5em\fi
                   \fi
                   \let\@makefntextORI\@makefntext
                   \long\def\@makefntext#1{%
                      \ifFBFrenchFootnotes
                         \@makefntextFB{#1}%
                      \else
                         \@makefntextORI{#1}%
                      \fi}%
                  }%
                }
%    \end{macrocode}
%
%    For compatibility reasons, we provide definitions for the commands
%    dealing with the layout of footnotes in |frenchb| version~1.6.
%    |\frenchbsetup{}| (see in section \ref{sec-keyval}) should be
%    preferred for setting these options.  |\StandardFootnotes| may
%    still be used locally (in minipages for instance), that's why the
%    test |\ifFBFrenchFootnotes| is done inside |\@makefntext|.
%    \begin{macrocode}
\newcommand*{\AddThinSpaceBeforeFootnotes}{\FBAutoSpaceFootnotestrue}
\newcommand*{\FrenchFootnotes}{\FBFrenchFootnotestrue}
\newcommand*{\StandardFootnotes}{\FBFrenchFootnotesfalse}
%    \end{macrocode}
%
%  \subsection{Global layout}
%  \label{sec-global}
%
%    In multilingual documents, some typographic rules must depend
%    on the current language (e.g., hyphenation, typesetting of
%    numbers, spacing before double punctuation\dots), others should,
%    IMHO, be kept global to the document: especially the layout of
%    lists (see~\ref{sec-lists}) and footnotes
%    (see~\ref{sec-footnotes}), and the indentation of the first
%    paragraph of sections (see~\ref{sec-indent}).
%
%    From version 2.2 on, if |frenchb| is \babel's ``main language''
%    (i.e. last language option at \babel's loading), |frenchb|
%    customises the layout (i.e. lists, indentation of the first
%    paragraphs of sections and footnotes) in the whole document
%    regardless the current language.   On the other hand, if |frenchb|
%    is \emph{not} \babel's ``main language'', it leaves the layout
%    unchanged both in French and in other languages.
%
%  \begin{macro}{\FrenchLayout}
%  \begin{macro}{\StandardLayout}
%    The former commands |\FrenchLayout| and |\StandardLayout| are kept
%    for compatibility reasons but should no longer be used.
%
% \changes{v2.0g}{2008/03/23}{Flag \cs{ifFBStandardLayout} not checked
%     by \cs{FBprocess@options}, low-level flags have to be set
%     one by one.}
%
%    \begin{macrocode}
\newcommand*{\FrenchLayout}{%
    \FBGlobalLayoutFrenchtrue
    \PackageWarning{frenchb.ldf}%
    {\protect\FrenchLayout\space is obsolete.  Please use\MessageBreak
     \protect\frenchbsetup{GlobalLayoutFrench} instead.}%
}
\newcommand*{\StandardLayout}{%
  \FBReduceListSpacingfalse
  \FBCompactItemizefalse
  \FBStandardItemLabelstrue
  \FBIndentFirstfalse
  \FBFrenchFootnotesfalse
  \FBAutoSpaceFootnotesfalse
  \PackageWarning{frenchb.ldf}%
    {\protect\StandardLayout\space is obsolete.  Please use\MessageBreak
    \protect\frenchbsetup{StandardLayout} instead.}%
}
\@onlypreamble\FrenchLayout
\@onlypreamble\StandardLayout
%    \end{macrocode}
%  \end{macro}
%  \end{macro}
%
%  \subsection{Dots\dots}
%  \label{sec-dots}
%
%  \begin{macro}{\FBtextellipsis}
%    \LaTeXe's standard definition of |\dots| in text-mode is
%    |\textellipsis| which includes a |\kern| at the end;
%    this space is not wanted in some cases (before a closing brace
%    for instance) and |\kern| breaks hyphenation of the next word.
%    We define |\FBtextellipsis| for French (in \LaTeXe{} only).
%
%    The |\if| construction in the \LaTeXe{} definition of |\dots|
%    doesn't allow the use of |xspace| (|xspace| is always followed
%    by a |\fi|), so we use the AMS-\LaTeX{} construction of |\dots|;
%    this has to be done `AtBeginDocument' not to be overwritten
%    when \file{amsmath.sty} is loaded after \babel.
%
% \changes{v2.0}{2006/11/06}{Added special case for LY1 encoding,
%    see  bug report from Bruno Voisin (2004/05/18).}
%
%    LY1 has a ready made character for |\textellipsis|, it should be
%    used in French too (pointed out by Bruno Voisin).
%
%    \begin{macrocode}
\DeclareTextSymbol{\FBtextellipsis}{LY1}{133}
\DeclareTextCommandDefault{\FBtextellipsis}{%
    .\kern\fontdimen3\font.\kern\fontdimen3\font.\xspace}
%    \end{macrocode}
%    |\Mdots@| and |\Tdots@ORI| hold the definitions of |\dots| in
%    Math and Text mode. They default to those of amsmath-2.0, and
%    will revert to standard \LaTeX{} definitions `AtBeginDocument',
%    if amsmath has not been loaded. |\Mdots@| doesn't change when
%    switching from/to French, while |\Tdots@| is |\FBtextellipsis|
%    in French and |\Tdots@ORI| otherwise.
%    \begin{macrocode}
\newcommand*{\Tdots@ORI}{\@xp\textellipsis}
\newcommand*{\Tdots@}{\Tdots@ORI}
\newcommand*{\Mdots@}{\@xp\mdots@}
\AtBeginDocument{\DeclareRobustCommand*{\dots}{\relax
                 \csname\ifmmode M\else T\fi dots@\endcsname}%
                 \@ifundefined{@xp}{\let\@xp\relax}{}%
                 \@ifundefined{mdots@}{\let\Tdots@ORI\textellipsis
                                       \let\Mdots@\mathellipsis}{}}
\def\bbl@frenchdots{\let\Tdots@\FBtextellipsis}
\def\bbl@nonfrenchdots{\let\Tdots@\Tdots@ORI}
\expandafter\addto\csname extras\CurrentOption\endcsname{%
    \bbl@frenchdots}
\expandafter\addto\csname noextras\CurrentOption\endcsname{%
    \bbl@nonfrenchdots}
%    \end{macrocode}
%  \end{macro}
%
%  \subsection{Setup options: keyval stuff}
%  \label{sec-keyval}
%
% \changes{v2.0}{2006/11/06}{New command \cs{frenchbsetup} added
%     for global customisation.}
%
% \changes{v2.0c}{2007/06/25}{Option ThinSpaceInFrenchNumbers added.}
%
% \changes{v2.0d}{2007/07/15}{Options og and fg changed: limit
%     the definition to French so that quote characters can be used
%     in German.}
%
% \changes{v2.0e}{2007/10/05}{New option: StandardLists.}
%
% \changes{v2.0f}{2008/03/23}{Two typos corrected in
%    option StandardLists: [false] $\to$ [true] and
%    StandardLayout $\to$ StandardLists.}
%
% \changes{v2.0f}{2008/03/23}{StandardLayout option had no
%     effect on lists.  Test moved to \cs{FBprocess@options}.}
%
% \changes{v2.0g}{2008/03/23}{Revert previous change to
%     StandardLayout. This option must set the three flags
%     \cs{FBReduceListSpacingfalse}, \cs{FBCompactItemizefalse},
%     and \cs{FBStandardItemLabeltrue} instead of
%     \cs{FBStandardListstrue}, so that later options can still
%     change their value before executing \cs{FBprocess@options}.
%     Same thing for option StandardLists.}
%
% \changes{v2.1a}{2008/03/24}{New option: FrenchSuperscripts
%     to define \cs{up} as \cs{fup} or as \cs{textsuperscript}.}
%
% \changes{v2.1a}{2008/03/30}{New option: LowercaseSuperscripts.}
%
% \changes{v2.2a}{2008/05/08}{The global layout of the document is
%     no longer changed when frenchb is not the last option of babel
%     (\cs{bbl@main@language}). Suggested by Ulrike Fischer.}
%
% \changes{v2.2a}{2008/05/08}{Values of flags
%     \cs{ifFBReduceListSpacing}, \cs{ifFBCompactItemize},
%     \cs{ifFBStandardItemLabels}, \cs{ifFBIndentFirst},
%     \cs{ifFBFrenchFootnotes}, \cs{ifFBAutoSpaceFootnotes} changed:
%     default now means `StandardLayout', it will be changed to
%     `FrenchLayout' AtEndOfPackage only if french is
%     \cs{bbl@main@language}.}
%
% \changes{v2.2a}{2008/05/08}{When frenchb is babel's last option,
%     French becomes the document's main language, so
%     GlobalLayoutFrench applies.}
%
% \changes{v2.3a}{2008/10/10}{New option: OriginalTypewriter. Now
%    frenchb switches to \cs{noautospace@beforeFDP} when a tt-font is
%    in use.  When OriginalTypewriter is set to true, frenchb behaves
%    as in pre-2.3 versions.}
%
%    We first define a collection of conditionals with their defaults
%    (true or false).
%
%    \begin{macrocode}
\newif\ifFBStandardLayout           \FBStandardLayouttrue
\newif\ifFBGlobalLayoutFrench       \FBGlobalLayoutFrenchfalse
\newif\ifFBReduceListSpacing        \FBReduceListSpacingfalse
\newif\ifFBCompactItemize           \FBCompactItemizefalse
\newif\ifFBStandardItemLabels       \FBStandardItemLabelstrue
\newif\ifFBStandardLists            \FBStandardListstrue
\newif\ifFBIndentFirst              \FBIndentFirstfalse
\newif\ifFBFrenchFootnotes          \FBFrenchFootnotesfalse
\newif\ifFBAutoSpaceFootnotes       \FBAutoSpaceFootnotesfalse
\newif\ifFBOriginalTypewriter       \FBOriginalTypewriterfalse
\newif\ifFBThinColonSpace           \FBThinColonSpacefalse
\newif\ifFBThinSpaceInFrenchNumbers \FBThinSpaceInFrenchNumbersfalse
\newif\ifFBFrenchSuperscripts       \FBFrenchSuperscriptstrue
\newif\ifFBLowercaseSuperscripts    \FBLowercaseSuperscriptstrue
\newif\ifFBPartNameFull             \FBPartNameFulltrue
\newif\ifFBShowOptions              \FBShowOptionsfalse
%    \end{macrocode}
%
%    The defaults values of these flags have been set so that |frenchb|
%    does not change anything regarding the global layout.
%    |\bbl@main@language| (set by the last option of babel) controls
%    the global layout of the document.  We check the current language
%    `AtEndOfPackage' (it is |\bbl@main@language|); if it is French,
%    the values of some flags have to be changed to ensure a French
%    looking layout for the whole document (even in parts written in
%    languages other than French); the end-user will then be able to
%    customise the values of all these flags with |\frenchbsetup{}|.
%    \begin{macrocode}
\AtEndOfPackage{%
   \iflanguage{french}{\FBReduceListSpacingtrue
                       \FBCompactItemizetrue
                       \FBStandardItemLabelsfalse
                       \FBIndentFirsttrue
                       \FBFrenchFootnotestrue
                       \FBAutoSpaceFootnotestrue
                       \FBGlobalLayoutFrenchtrue}%
                      {}%
}
%    \end{macrocode}
%
%  \begin{macro}{\frenchbsetup}
%    From version 2.0 on, all setup options are handled by \emph{one}
%    command |\frenchbsetup| using the keyval syntax.
%    Let's now define this command which reads and sets the options
%    to be processed later (at |\begin{document}|) by
%    |\FBprocess@options|. It  can only be called in the preamble.
%    \begin{macrocode}
\newcommand*{\frenchbsetup}[1]{%
  \setkeys{FB}{#1}%
}%
\@onlypreamble\frenchbsetup
%    \end{macrocode}
%    |frenchb| being an option of babel, it cannot load a package
%    (keyval) while |frenchb.ldf| is read, so we defer the loading of
%    \file{keyval} and the options setup at the end of \babel's loading.
%
%    |StandardLayout| resets the layout in French to the standard layout
%    defined par the \LaTeX{} class and packages loaded. It deals with
%    lists, indentation of first paragraphs of sections and footnotes.
%    Other keys, entered \emph{after} |StandardLayout| in
%    |\frenchbsetup|, can overrule some of the |StandardLayout|
%     settings.
%
%    |GlobalLayoutFrench| forces the layout in French and (as far as
%    possible) outside French to meet the French typographic standards.
%
% \changes{v2.3d}{2009/03/16}{Warning added to \cs{GlobalLayoutFrench}
%    when French is not the main language.}
%
%    \begin{macrocode}
\AtEndOfPackage{%
    \RequirePackage{keyval}%
    \define@key{FB}{StandardLayout}[true]%
                      {\csname FBStandardLayout#1\endcsname
                       \ifFBStandardLayout
                         \FBReduceListSpacingfalse
                         \FBCompactItemizefalse
                         \FBStandardItemLabelstrue
                         \FBIndentFirstfalse
                         \FBFrenchFootnotesfalse
                         \FBAutoSpaceFootnotesfalse
                         \FBGlobalLayoutFrenchfalse
                       \else
                         \FBReduceListSpacingtrue
                         \FBCompactItemizetrue
                         \FBStandardItemLabelsfalse
                         \FBIndentFirsttrue
                         \FBFrenchFootnotestrue
                         \FBAutoSpaceFootnotestrue
                       \fi}%
    \define@key{FB}{GlobalLayoutFrench}[true]%
                      {\csname FBGlobalLayoutFrench#1\endcsname
                       \ifFBGlobalLayoutFrench
                          \iflanguage{french}%
                            {\FBReduceListSpacingtrue
                             \FBCompactItemizetrue
                             \FBStandardItemLabelsfalse
                             \FBIndentFirsttrue
                             \FBFrenchFootnotestrue
                             \FBAutoSpaceFootnotestrue}%
                            {\PackageWarning{frenchb.ldf}%
                              {Option `GlobalLayoutFrench' skipped:
                               \MessageBreak French is *not*
                               babel's last option.\MessageBreak}}%
                       \fi}%
    \define@key{FB}{ReduceListSpacing}[true]%
                      {\csname FBReduceListSpacing#1\endcsname}%
    \define@key{FB}{CompactItemize}[true]%
                      {\csname FBCompactItemize#1\endcsname}%
    \define@key{FB}{StandardItemLabels}[true]%
                      {\csname FBStandardItemLabels#1\endcsname}%
    \define@key{FB}{ItemLabels}{%
        \renewcommand*{\FrenchLabelItem}{#1}}%
    \define@key{FB}{ItemLabeli}{%
        \renewcommand*{\Frlabelitemi}{#1}}%
    \define@key{FB}{ItemLabelii}{%
        \renewcommand*{\Frlabelitemii}{#1}}%
    \define@key{FB}{ItemLabeliii}{%
        \renewcommand*{\Frlabelitemiii}{#1}}%
    \define@key{FB}{ItemLabeliv}{%
        \renewcommand*{\Frlabelitemiv}{#1}}%
    \define@key{FB}{StandardLists}[true]%
                      {\csname FBStandardLists#1\endcsname
                       \ifFBStandardLists
                         \FBReduceListSpacingfalse
                         \FBCompactItemizefalse
                         \FBStandardItemLabelstrue
                       \else
                         \FBReduceListSpacingtrue
                         \FBCompactItemizetrue
                         \FBStandardItemLabelsfalse
                       \fi}%
    \define@key{FB}{IndentFirst}[true]%
                      {\csname FBIndentFirst#1\endcsname}%
    \define@key{FB}{FrenchFootnotes}[true]%
                      {\csname FBFrenchFootnotes#1\endcsname}%
    \define@key{FB}{AutoSpaceFootnotes}[true]%
                      {\csname FBAutoSpaceFootnotes#1\endcsname}%
    \define@key{FB}{AutoSpacePunctuation}[true]%
                      {\csname FBAutoSpacePunctuation#1\endcsname}%
    \define@key{FB}{OriginalTypewriter}[true]%
                      {\csname FBOriginalTypewriter#1\endcsname}%
    \define@key{FB}{ThinColonSpace}[true]%
                      {\csname FBThinColonSpace#1\endcsname}%
    \define@key{FB}{ThinSpaceInFrenchNumbers}[true]%
                      {\csname FBThinSpaceInFrenchNumbers#1\endcsname}%
    \define@key{FB}{FrenchSuperscripts}[true]%
                      {\csname FBFrenchSuperscripts#1\endcsname}
    \define@key{FB}{LowercaseSuperscripts}[true]%
                      {\csname FBLowercaseSuperscripts#1\endcsname}
    \define@key{FB}{PartNameFull}[true]%
                      {\csname FBPartNameFull#1\endcsname}%
    \define@key{FB}{ShowOptions}[true]%
                      {\csname FBShowOptions#1\endcsname}%
%    \end{macrocode}
%    Inputing French quotes as \emph{single characters} when they are
%    available on the keyboard (through a compose key for instance)
%    is more comfortable than typing |\og| and |\fg|.
%    The purpose of the following code is to map the French quote
%    characters to |\og\ignorespaces| and |{\fg}| respectively when
%    the current language is French, and to |\guillemotleft| and
%    |\guillemotright| otherwise (think of German quotes); thus correct
%    unbreakable spaces will be added automatically to French quotes.
%    The quote characters typed in depend on the input encoding,
%    it can be single-byte (latin1, latin9, applemac,\dots) or
%    multi-bytes (utf-8, utf8x).  We first check whether XeTeX is used
%    or not, if not the package |inputenc| has to be loaded before the
%    |\begin{document}| with the proper coding option, so we check if
%    |\DeclareInputText| is defined.
%    \begin{macrocode}
    \define@key{FB}{og}{%
       \newcommand*{\FB@@og}{\iflanguage{french}%
                               {\FB@og\ignorespaces}{\guillemotleft}}%
       \expandafter\ifx\csname XeTeXrevision\endcsname\relax
         \AtBeginDocument{%
           \@ifundefined{DeclareInputText}%
             {\PackageWarning{frenchb.ldf}%
               {Option `og' requires package inputenc.\MessageBreak}%
             }%
             {\@ifundefined{uc@dclc}%
%    \end{macrocode}
%    if |\uc@dclc| is undefined, utf8x is not loaded\dots
%    \begin{macrocode}
               {\@ifundefined{DeclareUnicodeCharacter}%
%    \end{macrocode}
%    if |\DeclareUnicodeCharacter| is undefined, utf8 is not loaded
%    either, we assume 8-bit character input encoding.
%    Package MULEenc.sty (from CJK) defines |\mule@def| to map
%    characters to control sequences.
%    \begin{macrocode}
                  {\@tempcnta`#1\relax
                     \@ifundefined{mule@def}%
                       {\DeclareInputText{\the\@tempcnta}{\FB@@og}}%
                       {\mule@def{11}{{\FB@@og}}}%
                  }%
%    \end{macrocode}
%    utf8 loaded, use |\DeclareUnicodeCharacter|,
%    \begin{macrocode}
                  {\DeclareUnicodeCharacter{00AB}{\FB@@og}}%
               }%
%    \end{macrocode}
%    utf8x loaded, use |\uc@dclc|,
%    \begin{macrocode}
               {\uc@dclc{171}{default}{\FB@@og}}%
             }%
         }%
%    \end{macrocode}
%    XeTeX in use, the following trick for defining the active quote
%    character is borrowed from \file{inputenc.dtx}.
%    \begin{macrocode}
       \else
         \catcode`#1=\active
         \bgroup
           \uccode`\~`#1%
           \uppercase{%
         \egroup
         \def~%
         }{\FB@@og}%
       \fi
    }%
%    \end{macrocode}
%    Same code for the closing quote.
%    \begin{macrocode}
    \define@key{FB}{fg}{%
       \newcommand*{\FB@@fg}{\iflanguage{french}%
                               {\FB@fg}{\guillemotright}}%
       \expandafter\ifx\csname XeTeXrevision\endcsname\relax
         \AtBeginDocument{%
           \@ifundefined{DeclareInputText}%
             {\PackageWarning{frenchb.ldf}%
               {Option `fg' requires package inputenc.\MessageBreak}%
             }%
             {\@ifundefined{uc@dclc}%
               {\@ifundefined{DeclareUnicodeCharacter}%
                  {\@tempcnta`#1\relax
                     \@ifundefined{mule@def}%
                       {\DeclareInputText{\the\@tempcnta}{{\FB@@fg}}}%
                       {\mule@def{27}{{\FB@@fg}}}%
                  }%
                  {\DeclareUnicodeCharacter{00BB}{{\FB@@fg}}}%
               }%
               {\uc@dclc{187}{default}{{\FB@@fg}}}%
             }%
         }%
       \else
         \catcode`#1=\active
         \bgroup
           \uccode`\~`#1%
           \uppercase{%
         \egroup
         \def~%
         }{{\FB@@fg}}%
       \fi
    }%
}
%    \end{macrocode}
%  \end{macro}
%
% \begin{macro}{\FBprocess@options}
%    |\FBprocess@options| processes the options, it is called \emph{once}
%    at |\begin{document}|.
%    \begin{macrocode}
\newcommand*{\FBprocess@options}{%
%    \end{macrocode}
%    Nothing has to be done here for |StandardLayout| and
%    |StandardLists| (the involved flags have already been set in
%    |\frenchbsetup{}| or before (at babel's EndOfPackage).
%
%    The next three options deal with the layout of lists in French.
%
%    |ReduceListSpacing| reduces the vertical spaces between list
%    items in French (done by changing |\list| to |\listFB|).
%    When |GlobalLayoutFrench| is true (the default), the same is
%    done outside French except for languages that force a different
%    setting.
%    \begin{macrocode}
  \ifFBReduceListSpacing
    \addto\extrasfrench{\let\list\listFB
                        \let\endlist\endlistFB}%
    \addto\noextrasfrench{\ifFBGlobalLayoutFrench
                            \let\list\listFB
                            \let\endlist\endlistFB
                          \else
                            \let\list\listORI
                            \let\endlist\endlistORI
                          \fi}%
  \else
    \addto\extrasfrench{\let\list\listORI
                        \let\endlist\endlistORI}%
    \addto\noextrasfrench{\let\list\listORI
                          \let\endlist\endlistORI}%
  \fi
%    \end{macrocode}
%
%    |CompactItemize| suppresses the vertical spacing between list
%    items in French (done by changing |\itemize| to |\itemizeFB|).
%    When |GlobalLayoutFrench| is true the same is done outside French.
%    \begin{macrocode}
  \ifFBCompactItemize
    \addto\extrasfrench{\let\itemize\itemizeFB
                        \let\enditemize\enditemizeFB}%
    \addto\noextrasfrench{\ifFBGlobalLayoutFrench
                             \let\itemize\itemizeFB
                             \let\enditemize\enditemizeFB
                          \else
                             \let\itemize\itemizeORI
                             \let\enditemize\enditemizeORI
                          \fi}%
  \else
    \addto\extrasfrench{\let\itemize\itemizeORI
                        \let\enditemize\enditemizeORI}%
    \addto\noextrasfrench{\let\itemize\itemizeORI
                          \let\enditemize\enditemizeORI}%
  \fi
%    \end{macrocode}
%
%    |StandardItemLabels| resets labelitems in French to their
%    standard values set by the \LaTeX{} class and packages loaded.
%    When |GlobalLayoutFrench| is true labelitems are identical inside
%    and outside French.
%    \begin{macrocode}
  \ifFBStandardItemLabels
    \addto\extrasfrench{\bbl@nonfrenchlabelitems}%
    \addto\noextrasfrench{\bbl@nonfrenchlabelitems}%
  \else
    \addto\extrasfrench{\bbl@frenchlabelitems}%
    \addto\noextrasfrench{\ifFBGlobalLayoutFrench
                            \bbl@frenchlabelitems
                          \else
                            \bbl@nonfrenchlabelitems
                          \fi}%
  \fi
%    \end{macrocode}
%
%    |IndentFirst| forces the first paragraphs of sections to be
%    indented just like the other ones in French.
%    When |GlobalLayoutFrench| is true (the default), the same is
%    done outside French except for languages that force a different
%    setting.
%    \begin{macrocode}
  \ifFBIndentFirst
    \addto\extrasfrench{\bbl@frenchindent}%
    \addto\noextrasfrench{\ifFBGlobalLayoutFrench
                             \bbl@frenchindent
                          \else
                             \bbl@nonfrenchindent
                          \fi}%
  \else
    \addto\extrasfrench{\bbl@nonfrenchindent}%
    \addto\noextrasfrench{\bbl@nonfrenchindent}%
  \fi
%    \end{macrocode}
%
%    The layout of footnotes is handled at the |\begin{document}|
%    depending on the values of flags |FrenchFootnotes|
%    and |AutoSpaceFootnotes| (see section~\ref{sec-footnotes}),
%    nothing has to be done here for footnotes.
%
%    |AutoSpacePunctuation| adds an unbreakable space (in French only)
%    before the four active characters (:;!?) even if none has been
%    typed before them.
%    \begin{macrocode}
  \ifFBAutoSpacePunctuation
     \autospace@beforeFDP
  \else
     \noautospace@beforeFDP
  \fi
%    \end{macrocode}
%
%    When |OriginalTypewriter| is set to |false| (the default),
%    |\ttfamily|, |\rmfamily| and |\sffamily| are redefined as
%    |\ttfamilyFB|, |\rmfamilyFB| and |\sffamilyFB| respectively
%    to prevent addition of automatic spaces before the four active
%    characters in computer code.
%    \begin{macrocode}
  \ifFBOriginalTypewriter
  \else
     \let\ttfamily\ttfamilyFB
     \let\rmfamily\rmfamilyFB
     \let\sffamily\sffamilyFB
  \fi
%    \end{macrocode}
%
%    |ThinColonSpace| changes the normal unbreakable space typeset in
%     French before `:' to a thin space.
%    \begin{macrocode}
  \ifFBThinColonSpace\renewcommand*{\Fcolonspace}{\thinspace}\fi
%    \end{macrocode}
%
%    When |true|, |ThinSpaceInFrenchNumbers| redefines |numprint.sty|'s
%    command |\npstylefrench| to set |\npthousandsep| to |\,|
%    (thinspace) instead of |~| (default) . This option has no effect
%    if package |numprint.sty| is not loaded with `|autolanguage|'.
%    As old versions of |numprint.sty| did not define |\npstylefrench|,
%    we have to provide this command.
%    \begin{macrocode}
  \@ifpackageloaded{numprint}%
  {\ifnprt@autolanguage
     \providecommand*{\npstylefrench}{}%
     \ifFBThinSpaceInFrenchNumbers
       \renewcommand*\npstylefrench{%
          \npthousandsep{\,}%
          \npdecimalsign{,}%
          \npproductsign{\cdot}%
          \npunitseparator{\,}%
          \npdegreeseparator{}%
          \nppercentseparator{\nprt@unitsep}%
          }%
     \else
       \renewcommand*\npstylefrench{%
          \npthousandsep{~}%
          \npdecimalsign{,}%
          \npproductsign{\cdot}%
          \npunitseparator{\,}%
          \npdegreeseparator{}%
          \nppercentseparator{\nprt@unitsep}%
          }%
     \fi
     \npaddtolanguage{french}{french}%
   \fi}{}%
%    \end{macrocode}
%
%    |FrenchSuperscripts|: if |true| |\up=\fup|, else
%    |\up=\textsuperscript|. Anyway |\up*=\FB@up@fake|. The star-form
%    |\up*{}| is provided for fonts that lack some superior letters:
%    Adobe Jenson Pro and Utopia Expert have no ``g superior'' for
%    instance.
%    \begin{macrocode}
  \ifFBFrenchSuperscripts
    \DeclareRobustCommand*{\up}{\@ifstar{\FB@up@fake}{\fup}}%
  \else
    \DeclareRobustCommand*{\up}{\@ifstar{\FB@up@fake}%
                                        {\textsuperscript}}%
  \fi
%    \end{macrocode}
%
%    |LowercaseSuperscripts|: if |true| let |\FB@lc| be |\lowercase|,
%     else |\FB@lc| is redefined to do nothing.
%    \begin{macrocode}
  \ifFBLowercaseSuperscripts
  \else
    \renewcommand*{\FB@lc}[1]{##1}%
  \fi
%    \end{macrocode}
%
%    |PartNameFull|: if |false|, redefine |\partname|.
%    \begin{macrocode}
  \ifFBPartNameFull
  \else\addto\captionsfrench{\def\partname{Partie}}\fi
%    \end{macrocode}
%
%    |ShowOptions|: if |true|, print the list of all options to the
%    \file{.log} file.
%    \begin{macrocode}
  \ifFBShowOptions
    \GenericWarning{* }{%
     * **** List of possible options for frenchb ****\MessageBreak
     [Default values between brackets when frenchb is loaded *LAST*]%
     \MessageBreak
     ShowOptions=true [false]\MessageBreak
     StandardLayout=true [false]\MessageBreak
     GlobalLayoutFrench=false [true]\MessageBreak
     StandardLists=true [false]\MessageBreak
     ReduceListSpacing=false [true]\MessageBreak
     CompactItemize=false [true]\MessageBreak
     StandardItemLabels=true [false]\MessageBreak
     ItemLabels=\textemdash, \textbullet,
        \protect\ding{43},... [\textendash]\MessageBreak
     ItemLabeli=\textemdash, \textbullet,
        \protect\ding{43},... [\textendash]\MessageBreak
     ItemLabelii=\textemdash, \textbullet,
        \protect\ding{43},... [\textendash]\MessageBreak
     ItemLabeliii=\textemdash, \textbullet,
        \protect\ding{43},... [\textendash]\MessageBreak
     ItemLabeliv=\textemdash, \textbullet,
        \protect\ding{43},... [\textendash]\MessageBreak
     IndentFirst=false [true]\MessageBreak
     FrenchFootnotes=false [true]\MessageBreak
     AutoSpaceFootnotes=false [true]\MessageBreak
     AutoSpacePunctuation=false [true]\MessageBreak
     OriginalTypewriter=true [false]\MessageBreak
     ThinColonSpace=true [false]\MessageBreak
     ThinSpaceInFrenchNumbers=true [false]\MessageBreak
     FrenchSuperscripts=false [true]\MessageBreak
     LowercaseSuperscripts=false [true]\MessageBreak
     PartNameFull=false [true]\MessageBreak
     og= <left quote character>, fg= <right quote character>
     \MessageBreak
     *********************************************
     \MessageBreak\protect\frenchbsetup{ShowOptions}}
  \fi
}
%    \end{macrocode}
%  \end{macro}
%
% \changes{v2.0}{2006/12/15}{AtBeginDocument, save again the
%    definitions of the `list' and `itemize' environments and the
%    values of labelitems.  As of frenchb v.1.6, `ORI' values were
%    set when reading frenchb.ldf, later changes were ignored.}
%
% \changes{v2.0}{2006/12/06}{Added warning for OT1 encoding.}
%
% \changes{v2.1b}{2008/04/07}{Disable some commands in bookmarks.}
%
%    At |\begin{document}| we save again the definitions of the `list'
%    and `itemize' environments and the values of labelitems so that
%    all changes made in the preamble are taken into account in
%    languages other than French and in French with the StandardLayout
%    option.  We also have to provide an |\xspace| command in case the
%    |xspace.sty| package is not loaded.
%
%    \begin{macrocode}
\AtBeginDocument{%
   \let\listORI\list
   \let\endlistORI\endlist
   \let\itemizeORI\itemize
   \let\enditemizeORI\enditemize
   \let\@ltiORI\labelitemi
   \let\@ltiiORI\labelitemii
   \let\@ltiiiORI\labelitemiii
   \let\@ltivORI\labelitemiv
   \providecommand*{\xspace}{\relax}%
%    \end{macrocode}
%    Let's redefine some commands in \file{hyperref}'s bookmarks.
%    \begin{macrocode}
   \@ifundefined{pdfstringdefDisableCommands}{}%
     {\pdfstringdefDisableCommands{%
        \let\up\relax
        \def\ieme{e\xspace}%
        \def\iemes{es\xspace}%
        \def\ier{er\xspace}%
        \def\iers{ers\xspace}%
        \def\iere{re\xspace}%
        \def\ieres{res\xspace}%
        \def\FrenchEnumerate#1{#1\degre\space}%
        \def\FrenchPopularEnumerate#1{#1\degre)\space}%
        \def\No{N\degre\space}%
        \def\no{n\degre\space}%
        \def\Nos{N\degre\space}%
        \def\nos{n\degre\space}%
        \def\og{\guillemotleft\space}%
        \def\fg{\space\guillemotright}%
        \let\bsc\textsc
        \let\degres\degre
     }}%
%    \end{macrocode}
%    It is time to process the options set with |\frenchboptions{}|.
%    Then execute either |\extrasfrench| and |\captionsfrench| or
%    |\noextrasfrench| according to the current language at the
%    |\begin{document}| (these three commands are updated by
%    |\FBprocess@options|).
%    \begin{macrocode}
   \FBprocess@options
   \iflanguage{french}{\extrasfrench\captionsfrench}{\noextrasfrench}%
%    \end{macrocode}
%    Some warnings are issued when output font encodings are not
%    properly set. With XeLaTeX, \file{fontspec.sty} and
%    \file{xunicode.sty} should be loaded; with (pdf)\LaTeX, a warning
%    is issued when OT1 encoding is in use at the |\begin{document}|.
%    Mind that |\encodingdefault| is defined as `long', defining
%    |\FBOTone| with |\newcommand*| would fail!
%    \begin{macrocode}
   \expandafter\ifx\csname XeTeXrevision\endcsname\relax
      \begingroup \newcommand{\FBOTone}{OT1}%
      \ifx\encodingdefault\FBOTone
        \PackageWarning{frenchb.ldf}%
           {OT1 encoding should not be used for French.
            \MessageBreak
            Add \protect\usepackage[T1]{fontenc} to the
            preamble\MessageBreak of your document,}
      \fi
     \endgroup
   \else
     \@ifundefined{DeclareUTFcharacter}%
       {\PackageWarning{frenchb.ldf}%
         {Add \protect\usepackage{fontspec} *and*\MessageBreak
          \protect\usepackage{xunicode} to the preamble\MessageBreak
          of your document,}}%
       {}%
    \fi
}
%    \end{macrocode}
%
%  \subsection{Clean up and exit}
%
%    Load |frenchb.cfg| (should do nothing, just for compatibility).
%    \begin{macrocode}
\loadlocalcfg{frenchb}
%    \end{macrocode}
%    Final cleaning.
%    The macro |\ldf@quit| takes care for setting the main language
%    to be switched on at |\begin{document}| and resetting the
%    category code of \texttt{@} to its original value.
%    The config file searched for has to be |frenchb.cfg|, and
%    |\CurrentOption| has been set to `french', so
%    |\ldf@finish\CurrentOption| cannot be used: we first load
%    |frenchb.cfg|, then call |\ldf@quit\CurrentOption|.
%    \begin{macrocode}
\FBclean@on@exit
\ldf@quit\CurrentOption
%    \end{macrocode}
% \iffalse
%</code>
%<*dtx>
% \fi
%%
%% \CharacterTable
%%  {Upper-case    \A\B\C\D\E\F\G\H\I\J\K\L\M\N\O\P\Q\R\S\T\U\V\W\X\Y\Z
%%   Lower-case    \a\b\c\d\e\f\g\h\i\j\k\l\m\n\o\p\q\r\s\t\u\v\w\x\y\z
%%   Digits        \0\1\2\3\4\5\6\7\8\9
%%   Exclamation   \!     Double quote  \"     Hash (number) \#
%%   Dollar        \$     Percent       \%     Ampersand     \&
%%   Acute accent  \'     Left paren    \(     Right paren   \)
%%   Asterisk      \*     Plus          \+     Comma         \,
%%   Minus         \-     Point         \.     Solidus       \/
%%   Colon         \:     Semicolon     \;     Less than     \<
%%   Equals        \=     Greater than  \>     Question mark \?
%%   Commercial at \@     Left bracket  \[     Backslash     \\
%%   Right bracket \]     Circumflex    \^     Underscore    \_
%%   Grave accent  \`     Left brace    \{     Vertical bar  \|
%%   Right brace   \}     Tilde         \~}
%%
% \iffalse
%</dtx>
% \fi
%
% \Finale
\endinput
}
\DeclareOption{albanian}{% \iffalse meta-comment
%
% Copyright 1989-2007 Johannes L. Braams and any individual authors
% listed elsewhere in this file.  All rights reserved.
% 
% This file is part of the Babel system.
% --------------------------------------
% 
% It may be distributed and/or modified under the
% conditions of the LaTeX Project Public License, either version 1.3
% of this license or (at your option) any later version.
% The latest version of this license is in
%   http://www.latex-project.org/lppl.txt
% and version 1.3 or later is part of all distributions of LaTeX
% version 2003/12/01 or later.
% 
% This work has the LPPL maintenance status "maintained".
% 
% The Current Maintainer of this work is Johannes Braams.
% 
% The list of all files belonging to the Babel system is
% given in the file `manifest.bbl. See also `legal.bbl' for additional
% information.
% 
% The list of derived (unpacked) files belonging to the distribution
% and covered by LPPL is defined by the unpacking scripts (with
% extension .ins) which are part of the distribution.
% \fi
% \CheckSum{251}
% \iffalse
%    Tell the \LaTeX\ system who we are and write an entry on the
%    transcript.
%<*dtx>
\ProvidesFile{albanian.dtx}
%</dtx>
%<code>\ProvidesLanguage{albanian}
%\fi
%\ProvidesFile{albanian.dtx}
       [2007/10/20 v1.0c albanian support from the babel system]
%\iffalse
% Babel package for LaTeX version 2e
% Copyright (C) 1989 - 2007
%           by Johannes Braams, TeXniek
%
% Please report errors to: J.L. Braams
%                          babel at braamsdot xs4all dot nl
%
%    This file is part of the babel system, it provides the source
%    code for the albanian language definition file.  A contribution
%    was made by Adi Zaimi (adizaimi at yahoo.com).
%
%<*filedriver>
\documentclass{ltxdoc}
\newcommand*\TeXhax{\TeX hax}
\newcommand*\babel{\textsf{babel}}
\newcommand*\langvar{$\langle \it lang \rangle$}
\newcommand*\note[1]{}
\newcommand*\Lopt[1]{\textsf{#1}}
\newcommand*\file[1]{\texttt{#1}}
\begin{document}
 \DocInput{albanian.dtx}
\end{document}
%</filedriver>
%\fi
% \GetFileInfo{albanian.dtx}
% \changes{albanian-1.0a}{2005/11/21}{Started first version of the file}
% \changes{albanian-1.0b}{2005/12/10}{A number of corrections in the
%    translations from Adi Zaimi}
% \changes{albanian-1.0c}{2006/06/05}{Small documentation fix}
%
%  \section{The Albanian language}
%
%    The file \file{\filename}\footnote{The file described in this
%    section has version number \fileversion\ and was last revised on
%    \filedate} defines all the language definition macros for the
%    Albanian language.  
%
%    Albanian is written in a latin script, but it has 36 letters, 
%    9 which are diletters (dh, gj, ll, nj, rr, sh, th, xh, zh), 
%    and two extra special characters.
%
%    For this language the character |"| is made active. In
%    table~\ref{tab:albanian-quote} an overview is given of its
%    purpose. 
%
%    \begin{table}[htb]
%     \begin{center}
%     \begin{tabular}{lp{8cm}}
%      |"c| & |\"c|, also implemented for the uppercase           \\
%      |"-| & an explicit hyphen sign, allowing hyphenation
%                  in the rest of the word.                        \\
%      \verb="|= & disable ligature at this position               \\
%      |""| & like |"-|, but producing no hyphen sign
%                  (for compund words with hyphen, e.g.\ |x-""y|). \\
%      |"`| & for Albanian left double quotes (looks like ,,).      \\
%      |"'| & for Albanian right double quotes.                     \\
%      |"<| & for French left double quotes (similar to $<<$).     \\
%      |">| & for French right double quotes (similar to $>>$).    \\
%     \end{tabular}
%     \caption{The extra definitions made
%              by \file{albanian.ldf}}\label{tab:albanian-quote}
%     \end{center}
%    \end{table}
%
%    Apart from defining shorthands we need to make sure that the
%    first paragraph of each section is intended. Furthermore the
%    following new math operators are defined (|\tg|, |\ctg|,
%    |\arctg|, |\arcctg|, |\sh|, |\ch|, |\th|, |\cth|, |\arsh|,
%    |\arch|, |\arth|, |\arcth|, |\Prob|, |\Expect|, |\Variance|).
%
% \StopEventually{}
%
%    The macro |\LdfInit| takes care of preventing that this file is
%    loaded more than once, checking the category code of the
%    \texttt{@} sign, etc.
%    \begin{macrocode}
%<*code>
\LdfInit{albanian}\captionsalbanian
%    \end{macrocode}
%
%    When this file is read as an option, i.e. by the |\usepackage|
%    command, \texttt{albanian} will be an `unknown' language in which
%    case we have to make it known. So we check for the existence of
%    |\l@albanian| to see whether we have to do something here.
%
%    \begin{macrocode}
\ifx\l@albanian\@undefined
    \@nopatterns{Albanian}
    \adddialect\l@albanian0\fi
%    \end{macrocode}
%
%    The next step consists of defining commands to switch to (and
%    from) the Albanian language.
%
%  \begin{macro}{\captionsalbanian}
%    The macro |\captionsalbanian| defines all strings used
%    in the four standard documentclasses provided with \LaTeX.
%    \begin{macrocode}
\addto\captionsalbanian{%
  \def\prefacename{Parathenia}%
  \def\refname{Referencat}%
  \def\abstractname{P\"ermbledhja}%
  \def\bibname{Bibliografia}%
  \def\chaptername{Kapitulli}%
  \def\appendixname{Shtesa}%
  \def\contentsname{P\"ermbajta}%
  \def\listfigurename{Figurat}%
  \def\listtablename{Tabelat}%
  \def\indexname{Indeksi}%
  \def\figurename{Figura}%
  \def\tablename{Tabela}%
  \def\partname{Pjesa}%
  \def\enclname{Lidhja}%
  \def\ccname{Kopja}%
  \def\headtoname{P\"er}%
  \def\pagename{Faqe}%
  \def\seename{shiko}%
  \def\alsoname{shiko dhe}%
  \def\proofname{V\"ertetim}%
  \def\glossaryname{P\"erhasja e Fjal\"eve}% 
  }%
%    \end{macrocode}
%  \end{macro}
%
%  \begin{macro}{\datealbanian}
%    The macro |\datealbanian| redefines the command |\today| to
%    produce Albanian dates.
%    \begin{macrocode}
\def\datealbanian{%
  \def\today{\number\day~\ifcase\month\or
    Janar\or Shkurt\or Mars\or Prill\or Maj\or
    Qershor\or Korrik\or Gusht\or Shtator\or Tetor\or N\"entor\or
    Dhjetor\fi \space \number\year}}
%    \end{macrocode}
%  \end{macro}
%
%  \begin{macro}{\extrasalbanian}
%  \begin{macro}{\noextrasalbanian}
%    The macro |\extrasalbanian| will perform all the extra
%    definitions needed for the Albanian language. The macro
%    |\noextrasalbanian| is used to cancel the actions of
%    |\extrasalbanian|.  
%
%    For Albanian the \texttt{"} character is made active. This is done
%    once, later on its definition may vary. Other languages in the
%    same document may also use the \texttt{"} character for
%    shorthands; we specify that the albanian group of shorthands
%    should be used.
%
%    \begin{macrocode}
\initiate@active@char{"}
\addto\extrasalbanian{\languageshorthands{albanian}}
\addto\extrasalbanian{\bbl@activate{"}}
%    \end{macrocode}
%    Don't forget to turn the shorthands off again.
%    \begin{macrocode}
\addto\noextrasalbanian{\bbl@deactivate{"}}
%    \end{macrocode}
%    First we define shorthands to facilitate the occurence of letters
%    such as \v{c}.
%    \begin{macrocode}
\declare@shorthand{albanian}{"c}{\textormath{\v c}{\check c}}
\declare@shorthand{albanian}{"e}{\textormath{\v e}{\check e}}
\declare@shorthand{albanian}{"C}{\textormath{\v C}{\check C}}
\declare@shorthand{albanian}{"E}{\textormath{\v E}{\check E}}
%    \end{macrocode}
%
%    Then we define access to two forms of quotation marks, similar
%    to the german and french quotation marks.
%    \begin{macrocode}
\declare@shorthand{albanian}{"`}{%
  \textormath{\quotedblbase{}}{\mbox{\quotedblbase}}}
\declare@shorthand{albanian}{"'}{%
  \textormath{\textquotedblleft{}}{\mbox{\textquotedblleft}}}
\declare@shorthand{albanian}{"<}{%
  \textormath{\guillemotleft{}}{\mbox{\guillemotleft}}}
\declare@shorthand{albanian}{">}{%
  \textormath{\guillemotright{}}{\mbox{\guillemotright}}}
%    \end{macrocode}
%    then we define two shorthands to be able to specify hyphenation
%    breakpoints that behave a little different from |\-|.
%    \begin{macrocode}
\declare@shorthand{albanian}{"-}{\nobreak-\bbl@allowhyphens}
\declare@shorthand{albanian}{""}{\hskip\z@skip}
%    \end{macrocode}
%    And we want to have a shorthand for disabling a ligature.
%    \begin{macrocode}
\declare@shorthand{albanian}{"|}{%
  \textormath{\discretionary{-}{}{\kern.03em}}{}}
%    \end{macrocode}
%  \end{macro}
%  \end{macro}
%
%  \begin{macro}{\bbl@frenchindent}
%  \begin{macro}{\bbl@nonfrenchindent}
%    In albanian the first paragraph of each section should be indented.
%    Add this code only in \LaTeX.
%    \begin{macrocode}
\ifx\fmtname plain \else
  \let\@aifORI\@afterindentfalse
  \def\bbl@frenchindent{\let\@afterindentfalse\@afterindenttrue
                        \@afterindenttrue}
  \def\bbl@nonfrenchindent{\let\@afterindentfalse\@aifORI
                          \@afterindentfalse}
  \addto\extrasalbanian{\bbl@frenchindent}
  \addto\noextrasalbanian{\bbl@nonfrenchindent}
\fi
%    \end{macrocode}
%  \end{macro}
%  \end{macro}
%
%  \begin{macro}{\mathalbanian}
%    Some math functions in Albanian math books have other names:
%    e.g. |sinh| in Albanian is written as |sh| etc. So we define a
%    number of new math operators.
%    \begin{macrocode}
\def\sh{\mathop{\operator@font sh}\nolimits} % same as \sinh
\def\ch{\mathop{\operator@font ch}\nolimits} % same as \cosh
\def\th{\mathop{\operator@font th}\nolimits} % same as \tanh
\def\cth{\mathop{\operator@font cth}\nolimits} % same as \coth
\def\arsh{\mathop{\operator@font arsh}\nolimits}
\def\arch{\mathop{\operator@font arch}\nolimits}
\def\arth{\mathop{\operator@font arth}\nolimits}
\def\arcth{\mathop{\operator@font arcth}\nolimits}
\def\tg{\mathop{\operator@font tg}\nolimits} % same as \tan
\def\ctg{\mathop{\operator@font ctg}\nolimits} % same as \cot
\def\arctg{\mathop{\operator@font arctg}\nolimits} % same as \arctan
\def\arcctg{\mathop{\operator@font arcctg}\nolimits}
\def\Prob{\mathop{\mathsf P\hskip0pt}\nolimits}
\def\Expect{\mathop{\mathsf E\hskip0pt}\nolimits}
\def\Variance{\mathop{\mathsf D\hskip0pt}\nolimits}
%    \end{macrocode}
%  \end{macro}
%
%    The macro |\ldf@finish| takes care of looking for a
%    configuration file, setting the main language to be switched on
%    at |\begin{document}| and resetting the category code of
%    \texttt{@} to its original value.
%    \begin{macrocode}
\ldf@finish{albanian}
%</code>
%    \end{macrocode}
%
% \Finale
%% \CharacterTable
%%  {Upper-case    \A\B\C\D\E\F\G\H\I\J\K\L\M\N\O\P\Q\R\S\T\U\V\W\X\Y\Z
%%   Lower-case    \a\b\c\d\e\f\g\h\i\j\k\l\m\n\o\p\q\r\s\t\u\v\w\x\y\z
%%   Digits        \0\1\2\3\4\5\6\7\8\9
%%   Exclamation   \!     Double quote  \"     Hash (number) \#
%%   Dollar        \$     Percent       \%     Ampersand     \&
%%   Acute accent  \'     Left paren    \(     Right paren   \)
%%   Asterisk      \*     Plus          \+     Comma         \,
%%   Minus         \-     Point         \.     Solidus       \/
%%   Colon         \:     Semicolon     \;     Less than     \<
%%   Equals        \=     Greater than  \>     Question mark \?
%%   Commercial at \@     Left bracket  \[     Backslash     \\
%%   Right bracket \]     Circumflex    \^     Underscore    \_
%%   Grave accent  \`     Left brace    \{     Vertical bar  \|
%%   Right brace   \}     Tilde         \~}
%%
\endinput

}
\DeclareOption{afrikaans}{%%
%% This file will generate fast loadable files and documentation
%% driver files from the doc files in this package when run through
%% LaTeX or TeX.
%%
%% Copyright 1989-2005 Johannes L. Braams and any individual authors
%% listed elsewhere in this file.  All rights reserved.
%% 
%% This file is part of the Babel system.
%% --------------------------------------
%% 
%% It may be distributed and/or modified under the
%% conditions of the LaTeX Project Public License, either version 1.3
%% of this license or (at your option) any later version.
%% The latest version of this license is in
%%   http://www.latex-project.org/lppl.txt
%% and version 1.3 or later is part of all distributions of LaTeX
%% version 2003/12/01 or later.
%% 
%% This work has the LPPL maintenance status "maintained".
%% 
%% The Current Maintainer of this work is Johannes Braams.
%% 
%% The list of all files belonging to the LaTeX base distribution is
%% given in the file `manifest.bbl. See also `legal.bbl' for additional
%% information.
%% 
%% The list of derived (unpacked) files belonging to the distribution
%% and covered by LPPL is defined by the unpacking scripts (with
%% extension .ins) which are part of the distribution.
%%
%% --------------- start of docstrip commands ------------------
%%
\def\filedate{1999/04/11}
\def\batchfile{dutch.ins}
\input docstrip.tex

{\ifx\generate\undefined
\Msg{**********************************************}
\Msg{*}
\Msg{* This installation requires docstrip}
\Msg{* version 2.3c or later.}
\Msg{*}
\Msg{* An older version of docstrip has been input}
\Msg{*}
\Msg{**********************************************}
\errhelp{Move or rename old docstrip.tex.}
\errmessage{Old docstrip in input path}
\batchmode
\csname @@end\endcsname
\fi}

\declarepreamble\mainpreamble
This is a generated file.

Copyright 1989-2005 Johannes L. Braams and any individual authors
listed elsewhere in this file.  All rights reserved.

This file was generated from file(s) of the Babel system.
---------------------------------------------------------

It may be distributed and/or modified under the
conditions of the LaTeX Project Public License, either version 1.3
of this license or (at your option) any later version.
The latest version of this license is in
  http://www.latex-project.org/lppl.txt
and version 1.3 or later is part of all distributions of LaTeX
version 2003/12/01 or later.

This work has the LPPL maintenance status "maintained".

The Current Maintainer of this work is Johannes Braams.

This file may only be distributed together with a copy of the Babel
system. You may however distribute the Babel system without
such generated files.

The list of all files belonging to the Babel distribution is
given in the file `manifest.bbl'. See also `legal.bbl for additional
information.

The list of derived (unpacked) files belonging to the distribution
and covered by LPPL is defined by the unpacking scripts (with
extension .ins) which are part of the distribution.
\endpreamble

\declarepreamble\fdpreamble
This is a generated file.

Copyright 1989-2005 Johannes L. Braams and any individual authors
listed elsewhere in this file.  All rights reserved.

This file was generated from file(s) of the Babel system.
---------------------------------------------------------

It may be distributed and/or modified under the
conditions of the LaTeX Project Public License, either version 1.3
of this license or (at your option) any later version.
The latest version of this license is in
  http://www.latex-project.org/lppl.txt
and version 1.3 or later is part of all distributions of LaTeX
version 2003/12/01 or later.

This work has the LPPL maintenance status "maintained".

The Current Maintainer of this work is Johannes Braams.

This file may only be distributed together with a copy of the Babel
system. You may however distribute the Babel system without
such generated files.

The list of all files belonging to the Babel distribution is
given in the file `manifest.bbl'. See also `legal.bbl for additional
information.

In particular, permission is granted to customize the declarations in
this file to serve the needs of your installation.

However, NO PERMISSION is granted to distribute a modified version
of this file under its original name.

\endpreamble

\keepsilent

\usedir{tex/generic/babel} 

\usepreamble\mainpreamble
\generate{\file{dutch.ldf}{\from{dutch.dtx}{code}}
          }
\usepreamble\fdpreamble

\ifToplevel{
\Msg{***********************************************************}
\Msg{*}
\Msg{* To finish the installation you have to move the following}
\Msg{* files into a directory searched by TeX:}
\Msg{*}
\Msg{* \space\space All *.def, *.fd, *.ldf, *.sty}
\Msg{*}
\Msg{* To produce the documentation run the files ending with}
\Msg{* '.dtx' and `.fdd' through LaTeX.}
\Msg{*}
\Msg{* Happy TeXing}
\Msg{***********************************************************}
}
 
\endinput
}
\DeclareOption{american}{%%
%% This file will generate fast loadable files and documentation
%% driver files from the doc files in this package when run through
%% LaTeX or TeX.
%%
%% Copyright 1989-2005 Johannes L. Braams and any individual authors
%% listed elsewhere in this file.  All rights reserved.
%% 
%% This file is part of the Babel system.
%% --------------------------------------
%% 
%% It may be distributed and/or modified under the
%% conditions of the LaTeX Project Public License, either version 1.3
%% of this license or (at your option) any later version.
%% The latest version of this license is in
%%   http://www.latex-project.org/lppl.txt
%% and version 1.3 or later is part of all distributions of LaTeX
%% version 2003/12/01 or later.
%% 
%% This work has the LPPL maintenance status "maintained".
%% 
%% The Current Maintainer of this work is Johannes Braams.
%% 
%% The list of all files belonging to the LaTeX base distribution is
%% given in the file `manifest.bbl. See also `legal.bbl' for additional
%% information.
%% 
%% The list of derived (unpacked) files belonging to the distribution
%% and covered by LPPL is defined by the unpacking scripts (with
%% extension .ins) which are part of the distribution.
%%
%% --------------- start of docstrip commands ------------------
%%
\def\filedate{1999/04/11}
\def\batchfile{english.ins}
\input docstrip.tex

{\ifx\generate\undefined
\Msg{**********************************************}
\Msg{*}
\Msg{* This installation requires docstrip}
\Msg{* version 2.3c or later.}
\Msg{*}
\Msg{* An older version of docstrip has been input}
\Msg{*}
\Msg{**********************************************}
\errhelp{Move or rename old docstrip.tex.}
\errmessage{Old docstrip in input path}
\batchmode
\csname @@end\endcsname
\fi}

\declarepreamble\mainpreamble
This is a generated file.

Copyright 1989-2005 Johannes L. Braams and any individual authors
listed elsewhere in this file.  All rights reserved.

This file was generated from file(s) of the Babel system.
---------------------------------------------------------

It may be distributed and/or modified under the
conditions of the LaTeX Project Public License, either version 1.3
of this license or (at your option) any later version.
The latest version of this license is in
  http://www.latex-project.org/lppl.txt
and version 1.3 or later is part of all distributions of LaTeX
version 2003/12/01 or later.

This work has the LPPL maintenance status "maintained".

The Current Maintainer of this work is Johannes Braams.

This file may only be distributed together with a copy of the Babel
system. You may however distribute the Babel system without
such generated files.

The list of all files belonging to the Babel distribution is
given in the file `manifest.bbl'. See also `legal.bbl for additional
information.

The list of derived (unpacked) files belonging to the distribution
and covered by LPPL is defined by the unpacking scripts (with
extension .ins) which are part of the distribution.
\endpreamble

\declarepreamble\fdpreamble
This is a generated file.

Copyright 1989-2005 Johannes L. Braams and any individual authors
listed elsewhere in this file.  All rights reserved.

This file was generated from file(s) of the Babel system.
---------------------------------------------------------

It may be distributed and/or modified under the
conditions of the LaTeX Project Public License, either version 1.3
of this license or (at your option) any later version.
The latest version of this license is in
  http://www.latex-project.org/lppl.txt
and version 1.3 or later is part of all distributions of LaTeX
version 2003/12/01 or later.

This work has the LPPL maintenance status "maintained".

The Current Maintainer of this work is Johannes Braams.

This file may only be distributed together with a copy of the Babel
system. You may however distribute the Babel system without
such generated files.

The list of all files belonging to the Babel distribution is
given in the file `manifest.bbl'. See also `legal.bbl for additional
information.

In particular, permission is granted to customize the declarations in
this file to serve the needs of your installation.

However, NO PERMISSION is granted to distribute a modified version
of this file under its original name.

\endpreamble

\keepsilent

\usedir{tex/generic/babel} 

\usepreamble\mainpreamble
\generate{\file{english.ldf}{\from{english.dtx}{code}}
          }
\usepreamble\fdpreamble

\ifToplevel{
\Msg{***********************************************************}
\Msg{*}
\Msg{* To finish the installation you have to move the following}
\Msg{* files into a directory searched by TeX:}
\Msg{*}
\Msg{* \space\space All *.def, *.fd, *.ldf, *.sty}
\Msg{*}
\Msg{* To produce the documentation run the files ending with}
\Msg{* '.dtx' and `.fdd' through LaTeX.}
\Msg{*}
\Msg{* Happy TeXing}
\Msg{***********************************************************}
}
 
\endinput
}
\DeclareOption{australian}{%%
%% This file will generate fast loadable files and documentation
%% driver files from the doc files in this package when run through
%% LaTeX or TeX.
%%
%% Copyright 1989-2005 Johannes L. Braams and any individual authors
%% listed elsewhere in this file.  All rights reserved.
%% 
%% This file is part of the Babel system.
%% --------------------------------------
%% 
%% It may be distributed and/or modified under the
%% conditions of the LaTeX Project Public License, either version 1.3
%% of this license or (at your option) any later version.
%% The latest version of this license is in
%%   http://www.latex-project.org/lppl.txt
%% and version 1.3 or later is part of all distributions of LaTeX
%% version 2003/12/01 or later.
%% 
%% This work has the LPPL maintenance status "maintained".
%% 
%% The Current Maintainer of this work is Johannes Braams.
%% 
%% The list of all files belonging to the LaTeX base distribution is
%% given in the file `manifest.bbl. See also `legal.bbl' for additional
%% information.
%% 
%% The list of derived (unpacked) files belonging to the distribution
%% and covered by LPPL is defined by the unpacking scripts (with
%% extension .ins) which are part of the distribution.
%%
%% --------------- start of docstrip commands ------------------
%%
\def\filedate{1999/04/11}
\def\batchfile{english.ins}
\input docstrip.tex

{\ifx\generate\undefined
\Msg{**********************************************}
\Msg{*}
\Msg{* This installation requires docstrip}
\Msg{* version 2.3c or later.}
\Msg{*}
\Msg{* An older version of docstrip has been input}
\Msg{*}
\Msg{**********************************************}
\errhelp{Move or rename old docstrip.tex.}
\errmessage{Old docstrip in input path}
\batchmode
\csname @@end\endcsname
\fi}

\declarepreamble\mainpreamble
This is a generated file.

Copyright 1989-2005 Johannes L. Braams and any individual authors
listed elsewhere in this file.  All rights reserved.

This file was generated from file(s) of the Babel system.
---------------------------------------------------------

It may be distributed and/or modified under the
conditions of the LaTeX Project Public License, either version 1.3
of this license or (at your option) any later version.
The latest version of this license is in
  http://www.latex-project.org/lppl.txt
and version 1.3 or later is part of all distributions of LaTeX
version 2003/12/01 or later.

This work has the LPPL maintenance status "maintained".

The Current Maintainer of this work is Johannes Braams.

This file may only be distributed together with a copy of the Babel
system. You may however distribute the Babel system without
such generated files.

The list of all files belonging to the Babel distribution is
given in the file `manifest.bbl'. See also `legal.bbl for additional
information.

The list of derived (unpacked) files belonging to the distribution
and covered by LPPL is defined by the unpacking scripts (with
extension .ins) which are part of the distribution.
\endpreamble

\declarepreamble\fdpreamble
This is a generated file.

Copyright 1989-2005 Johannes L. Braams and any individual authors
listed elsewhere in this file.  All rights reserved.

This file was generated from file(s) of the Babel system.
---------------------------------------------------------

It may be distributed and/or modified under the
conditions of the LaTeX Project Public License, either version 1.3
of this license or (at your option) any later version.
The latest version of this license is in
  http://www.latex-project.org/lppl.txt
and version 1.3 or later is part of all distributions of LaTeX
version 2003/12/01 or later.

This work has the LPPL maintenance status "maintained".

The Current Maintainer of this work is Johannes Braams.

This file may only be distributed together with a copy of the Babel
system. You may however distribute the Babel system without
such generated files.

The list of all files belonging to the Babel distribution is
given in the file `manifest.bbl'. See also `legal.bbl for additional
information.

In particular, permission is granted to customize the declarations in
this file to serve the needs of your installation.

However, NO PERMISSION is granted to distribute a modified version
of this file under its original name.

\endpreamble

\keepsilent

\usedir{tex/generic/babel} 

\usepreamble\mainpreamble
\generate{\file{english.ldf}{\from{english.dtx}{code}}
          }
\usepreamble\fdpreamble

\ifToplevel{
\Msg{***********************************************************}
\Msg{*}
\Msg{* To finish the installation you have to move the following}
\Msg{* files into a directory searched by TeX:}
\Msg{*}
\Msg{* \space\space All *.def, *.fd, *.ldf, *.sty}
\Msg{*}
\Msg{* To produce the documentation run the files ending with}
\Msg{* '.dtx' and `.fdd' through LaTeX.}
\Msg{*}
\Msg{* Happy TeXing}
\Msg{***********************************************************}
}
 
\endinput
}
%    \end{macrocode}
%    Austrian is really a dialect of German.
% \changes{babel~3.6i}{1997/02/20}{Added the \Lopt{Basque} option}
%    \begin{macrocode}
\DeclareOption{austrian}{% \iffalse meta-comm

% Copyright 1989-2008 Johannes L. Braams and any individual auth
% listed elsewhere in this file.  All rights reserv

% This file is part of the Babel syst
% -----------------------------------

% It may be distributed and/or modified under
% conditions of the LaTeX Project Public License, either version
% of this license or (at your option) any later versi
% The latest version of this license is
%   http://www.latex-project.org/lppl.
% and version 1.3 or later is part of all distributions of La
% version 2003/12/01 or lat

% This work has the LPPL maintenance status "maintaine

% The Current Maintainer of this work is Johannes Braa

% The list of all files belonging to the Babel system
% given in the file `manifest.bbl. See also `legal.bbl' for additio
% informati

% The list of derived (unpacked) files belonging to the distribut
% and covered by LPPL is defined by the unpacking scripts (w
% extension .ins) which are part of the distributi
%
% \CheckSum{3

% \iffa
%    Tell the \LaTeX\ system who we are and write an entry on
%    transcri
%<*d
\ProvidesFile{germanb.d
%</d
%<code>\ProvidesLanguage{germa
%
%\ProvidesFile{germanb.d
        [2008/06/01 v2.6m German support from the babel syst
%\iffa
%% File `germanb.d
%% Babel package for LaTeX version
%% Copyright (C) 1989 - 2
%%           by Johannes Braams, TeXn

%% Germanb Language Definition F
%% Copyright (C) 1989 - 2
%%           by Bernd Raichle raichle at azu.Informatik.Uni-Stuttgart
%%              Johannes Braams, TeXn
% This file is based on german.tex version 2.
%                       by Bernd Raichle, Hubert Partl et.

%% Please report errors to: J.L. Bra
%%                          babel at braams.xs4all

%<*filedriv
\documentclass{ltxd
\font\manual=logo10 % font used for the METAFONT logo, e
\newcommand*\MF{{\manual META}\-{\manual FON
\newcommand*\TeXhax{\TeX h
\newcommand*\babel{\textsf{babe
\newcommand*\langvar{$\langle \it lang \rangl
\newcommand*\note[1
\newcommand*\Lopt[1]{\textsf{#
\newcommand*\file[1]{\texttt{#
\begin{docume
 \DocInput{germanb.d
\end{docume
%</filedriv
%
% \GetFileInfo{germanb.d

% \changes{germanb-1.0a}{1990/05/14}{Incorporated Nico's commen
% \changes{germanb-1.0b}{1990/05/22}{fixed typo in definition
%    austrian language found by Werenfried S
%    \texttt{nspit@fys.ruu.n
% \changes{germanb-1.0c}{1990/07/16}{Fixed some typ
% \changes{germanb-1.1}{1990/07/30}{When using PostScript fonts w
%    the Adobe fontencoding, the dieresis-accent is located elsewhe
%    modified co
% \changes{germanb-1.1a}{1990/08/27}{Modified the documentat
%    somewh
% \changes{germanb-2.0}{1991/04/23}{Modified for babel 3
% \changes{germanb-2.0a}{1991/05/25}{Removed some problems in cha
%    l
% \changes{germanb-2.1}{1991/05/29}{Removed bug found by van der Me
% \changes{germanb-2.2}{1991/06/11}{Removed global assignmen
%    brought uptodate with \file{german.tex} v2.
% \changes{germanb-2.2a}{1991/07/15}{Renamed \file{babel.sty}
%    \file{babel.co
% \changes{germanb-2.3}{1991/11/05}{Rewritten parts of the code to
%    the new features of babel version 3
% \changes{germanb-2.3e}{1991/11/10}{Brought up-to-date w
%    \file{german.tex} v2.3e (plus some bug fixes) [b
% \changes{germanb-2.5}{1994/02/08}{Update or \LaTe
% \changes{germanb-2.5c}{1994/06/26}{Removed the use of \cs{fileda
%    and moved the identification after the loading
%    \file{babel.de
% \changes{germanb-2.6a}{1995/02/15}{Moved the identification to
%    top of the fi
% \changes{germanb-2.6a}{1995/02/15}{Rewrote the code that handles
%    active double quote charact
% \changes{germanb-2.6d}{1996/07/10}{Replaced \cs{undefined} w
%    \cs{@undefined} and \cs{empty} with \cs{@empty} for consiste
%    with \LaTe
% \changes{germanb-2.6d}{1996/10/10}{Moved the definition
%    \cs{atcatcode} right to the beginning

%  \section{The German langua

%    The file \file{\filename}\footnote{The file described in t
%    section has version number \fileversion\ and was last revised
%    \filedate.}  defines all the language definition macros for
%    German language as well as for the Austrian dialect of t
%    language\footnote{This file is a re-implementation of Hub
%    Partl's \file{german.sty} version 2.5b, see~\cite{HP}

%    For this language the character |"| is made active.
%    table~\ref{tab:german-quote} an overview is given of
%    purpose. One of the reasons for this is that in the Ger
%    language some character combinations change when a word is bro
%    between the combination. Also the vertical placement of
%    umlaut can be controlled this w
%    \begin{table}[h
%     \begin{cent
%     \begin{tabular}{lp{8c
%      |"a| & |\"a|, also implemented for the ot
%                  lowercase and uppercase vowels.
%      |"s| & to produce the German \ss{} (like |\ss{}|).
%      |"z| & to produce the German \ss{} (like |\ss{}|).
%      |"ck|& for |ck| to be hyphenated as |k-k|.
%      |"ff|& for |ff| to be hyphenated as |ff-
%                  this is also implemented for l, m, n, p, r and
%      |"S| & for |SS| to be |\uppercase{"s}|.
%      |"Z| & for |SZ| to be |\uppercase{"z}|.
%      \verb="|= & disable ligature at this position.
%      |"-| & an explicit hyphen sign, allowing hyphenat
%             in the rest of the word.
%      |""| & like |"-|, but producing no hyphen s
%             (for compund words with hyphen, e.g.\ |x-""y|).
%      |"~| & for a compound word mark without a breakpoint.
%      |"=| & for a compound word mark with a breakpoint, allow
%             hyphenation in the composing words.
%      |"`| & for German left double quotes (looks like ,,).
%      |"'| & for German right double quotes.
%      |"<| & for French left double quotes (similar to $<<$).
%      |">| & for French right double quotes (similar to $>>$).
%     \end{tabul
%     \caption{The extra definitions m
%              by \file{german.ldf}}\label{tab:german-quo
%     \end{cent
%    \end{tab
%    The quotes in table~\ref{tab:german-quote} can also be typeset
%    using the commands in table~\ref{tab:more-quot
%    \begin{table}[h
%     \begin{cent
%     \begin{tabular}{lp{8c
%      |\glqq| & for German left double quotes (looks like ,,).
%      |\grqq| & for German right double quotes (looks like ``).
%      |\glq|  & for German left single quotes (looks like ,).
%      |\grq|  & for German right single quotes (looks like `).
%      |\flqq| & for French left double quotes (similar to $<<$).
%      |\frqq| & for French right double quotes (similar to $>>$)
%      |\flq|  & for (French) left single quotes (similar to $<$).
%      |\frq|  & for (French) right single quotes (similar to $>$).
%      |\dq|   & the original quotes character (|"|).
%     \end{tabul
%     \caption{More commands which produce quotes, defi
%              by \file{german.ldf}}\label{tab:more-quo
%     \end{cent
%    \end{tab

% \StopEventuall

%    When this file was read through the option \Lopt{germanb} we m
%    it behave as if \Lopt{german} was specifi
% \changes{german-2.6l}{2008/03/17}{Making germanb behave like ger
%    needs some more work besides defining \cs{CurrentOptio
% \changes{germanb-2.6m}{2008/06/01}{Correted a ty
%    \begin{macroco
\def\bbl@tempa{germa
\ifx\CurrentOption\bbl@te
  \def\CurrentOption{germ
  \ifx\l@german\@undefi
    \@nopatterns{Germ
    \adddialect\l@germ

  \let\l@germanb\l@ger
  \AtBeginDocumen
    \let\captionsgermanb\captionsger
    \let\dategermanb\dateger
    \let\extrasgermanb\extrasger
    \let\noextrasgermanb\noextrasger


%    \end{macroco

%    The macro |\LdfInit| takes care of preventing that this file
%    loaded more than once, checking the category code of
%    \texttt{@} sign, e
% \changes{germanb-2.6d}{1996/11/02}{Now use \cs{LdfInit} to perf
%    initial check
%    \begin{macroco
%<*co
\LdfInit\CurrentOption{captions\CurrentOpti
%    \end{macroco

%    When this file is read as an option, i.e., by the |\usepacka
%    command, \texttt{german} will be an `unknown' language, so
%    have to make it known.  So we check for the existence
%    |\l@german| to see whether we have to do something he

% \changes{germanb-2.0}{1991/04/23}{Now use \cs{adddialect}
%    language undefin
% \changes{germanb-2.2d}{1991/10/27}{Removed use of \cs{@ifundefine
% \changes{germanb-2.3e}{1991/11/10}{Added warning, if no ger
%    patterns load
% \changes{germanb-2.5c}{1994/06/26}{Now use \cs{@nopatterns}
%    produce the warni
%    \begin{macroco
\ifx\l@german\@undefi
  \@nopatterns{Germ
  \adddialect\l@germ

%    \end{macroco

%    For the Austrian version of these definitions we just add anot
%    languag
% \changes{germanb-2.0}{1991/04/23}{Now use \cs{adddialect}
%    austri
%    \begin{macroco
\adddialect\l@austrian\l@ger
%    \end{macroco

%    The next step consists of defining commands to switch to (
%    from) the German langua

%  \begin{macro}{\captionsgerm
%  \begin{macro}{\captionsaustri
%    Either the macro |\captionsgerman| or the ma
%    |\captionsaustrian| will define all strings used in the f
%    standard document classes provided with \LaT

% \changes{germanb-2.2}{1991/06/06}{Removed \cs{global} definitio
% \changes{germanb-2.2}{1991/06/06}{\cs{pagename} should
%    \cs{headpagenam
% \changes{germanb-2.3e}{1991/11/10}{Added \cs{prefacenam
%    \cs{seename} and \cs{alsonam
% \changes{germanb-2.4}{1993/07/15}{\cs{headpagename} should
%    \cs{pagenam
% \changes{germanb-2.6b}{1995/07/04}{Added \cs{proofname}
%    AMS-\LaT
% \changes{germanb-2.6d}{1996/07/10}{Construct control sequence on
%    f
% \changes{germanb-2.6j}{2000/09/20}{Added \cs{glossarynam
%    \begin{macroco
\@namedef{captions\CurrentOption
  \def\prefacename{Vorwor
  \def\refname{Literatu
  \def\abstractname{Zusammenfassun
  \def\bibname{Literaturverzeichni
  \def\chaptername{Kapite
  \def\appendixname{Anhan
  \def\contentsname{Inhaltsverzeichnis}%    % oder nur: Inh
  \def\listfigurename{Abbildungsverzeichni
  \def\listtablename{Tabellenverzeichni
  \def\indexname{Inde
  \def\figurename{Abbildun
  \def\tablename{Tabelle}%                  % oder: Ta
  \def\partname{Tei
  \def\enclname{Anlage(n)}%                 % oder: Beilage
  \def\ccname{Verteiler}%                   % oder: Kopien
  \def\headtoname{A
  \def\pagename{Seit
  \def\seename{sieh
  \def\alsoname{siehe auc
  \def\proofname{Bewei
  \def\glossaryname{Glossa

%    \end{macroco
%  \end{mac
%  \end{mac

%  \begin{macro}{\dategerm
%    The macro |\dategerman| redefines the comm
%    |\today| to produce German dat
% \changes{germanb-2.3e}{1991/11/10}{Added \cs{month@germa
% \changes{germanb-2.6f}{1997/10/01}{Use \cs{edef} to def
%    \cs{today} to save memo
% \changes{germanb-2.6f}{1998/03/28}{use \cs{def} instead
%    \cs{ede
%    \begin{macroco
\def\month@german{\ifcase\month
  Januar\or Februar\or M\"arz\or April\or Mai\or Juni
  Juli\or August\or September\or Oktober\or November\or Dezember\
\def\dategerman{\def\today{\number\day.~\month@ger
    \space\number\yea
%    \end{macroco
%  \end{mac

%  \begin{macro}{\dateaustri
%    The macro |\dateaustrian| redefines the comm
%    |\today| to produce Austrian version of the German dat
% \changes{germanb-2.6f}{1997/10/01}{Use \cs{edef} to def
%    \cs{today} to save memo
% \changes{germanb-2.6f}{1998/03/28}{use \cs{def} instead
%    \cs{ede
%    \begin{macroco
\def\dateaustrian{\def\today{\number\day.~\ifnum1=\mo
  J\"anner\else \month@german\fi \space\number\yea
%    \end{macroco
%  \end{mac

%  \begin{macro}{\extrasgerm
%  \begin{macro}{\extrasaustri
% \changes{germanb-2.0b}{1991/05/29}{added some comment chars
%    prevent white spa
% \changes{germanb-2.2}{1991/06/11}{Save all redefined macr
%  \begin{macro}{\noextrasgerm
%  \begin{macro}{\noextrasaustri
% \changes{germanb-1.1}{1990/07/30}{Added \cs{dieresi
% \changes{germanb-2.0b}{1991/05/29}{added some comment chars
%    prevent white spa
% \changes{germanb-2.2}{1991/06/11}{Try to restore everything to
%    former sta
% \changes{germanb-2.6d}{1996/07/10}{Construct control seque
%    \cs{extrasgerman} or \cs{extrasaustrian} on the f

%    Either the macro |\extrasgerman| or the macros |\extrasaustri
%    will perform all the extra definitions needed for the Ger
%    language. The macro |\noextrasgerman| is used to cancel
%    actions of |\extrasgerman

%    For German (as well as for Dutch) the \texttt{"} character
%    made active. This is done once, later on its definition may va
%    \begin{macroco
\initiate@active@char
\@namedef{extras\CurrentOption
  \languageshorthands{germa
\expandafter\addto\csname extras\CurrentOption\endcsnam
  \bbl@activate{
%    \end{macroco
%    Don't forget to turn the shorthands off aga
% \changes{germanb-2.6i}{1999/12/16}{Deactivate shorthands ouside
%    Germ
%    \begin{macroco
\addto\noextrasgerman{\bbl@deactivate{
%    \end{macroco

% \changes{germanb-2.6a}{1995/02/15}{All the code to handle the act
%    double quote has been moved to \file{babel.de

%    In order for \TeX\ to be able to hyphenate German words wh
%    contain `\ss' (in the \texttt{OT1} position |^^Y|) we have
%    give the character a nonzero |\lccode| (see Appendix H, the \
%    boo
% \changes{germanb-2.6c}{1996/04/08}{Use decimal number instead
%    hat-notation as the hat may be activat
%    \begin{macroco
\expandafter\addto\csname extras\CurrentOption\endcsnam
  \babel@savevariable{\lccode2
  \lccode25=
%    \end{macroco
% \changes{germanb-2.6a}{1995/02/15}{Removeed \cs{3} as it is
%    longer in \file{german.ld

%    The umlaut accent macro |\"| is changed to lower the umlaut do
%    The redefinition is done with the help of |\umlautlo
%    \begin{macroco
\expandafter\addto\csname extras\CurrentOption\endcsnam
  \babel@save\"\umlautl
\@namedef{noextras\CurrentOption}{\umlauthi
%    \end{macroco
%    The german hyphenation patterns can be used with |\lefthyphenm
%    and |\righthyphenmin| set to
% \changes{germanb-2.6a}{1995/05/13}{use \cs{germanhyphenmins} to st
%    the correct valu
% \changes{germanb-2.6j}{2000/09/22}{Now use \cs{providehyphenmins}
%    provide a default val
%    \begin{macroco
\providehyphenmins{\CurrentOption}{\tw@\t
%    \end{macroco
%    For German texts we need to make sure that |\frenchspacing|
%    turned
% \changes{germanb-2.6k}{2001/01/26}{Turn frenchspacing on, as
%    \texttt{german.st
%    \begin{macroco
\expandafter\addto\csname extras\CurrentOption\endcsnam
  \bbl@frenchspaci
\expandafter\addto\csname noextras\CurrentOption\endcsnam
  \bbl@nonfrenchspaci
%    \end{macroco
%  \end{mac
%  \end{mac
%  \end{mac
%  \end{mac

% \changes{germanb-2.6a}{1995/02/15}{\cs{umlautlow}
%    \cs{umlauthigh} moved to \file{glyphs.dtx}, as well
%    \cs{newumlaut} (now \cs{lower@umlau

%    The code above is necessary because we need an extra act
%    character. This character is then used as indicated
%    table~\ref{tab:german-quot

%    To be able to define the function of |"|, we first defin
%    couple of `support' macr

% \changes{germanb-2.3e}{1991/11/10}{Added \cs{save@sf@q} macro
%    rewrote all quote macros to use
% \changes{germanb-2.3h}{1991/02/16}{moved definition
%    \cs{allowhyphens}, \cs{set@low@box} and \cs{save@sf@q}
%    \file{babel.co
% \changes{germanb-2.6a}{1995/02/15}{Moved all quotation characters
%    \file{glyphs.dt

%  \begin{macro}{\
%    We save the original double quote character in |\dq| to k
%    it available, the math accent |\"| can now be typed as |
%    \begin{macroco
\begingroup \catcode`\
\def\x{\endgr
  \def\@SS{\mathchar"701
  \def\dq{

%    \end{macroco
%  \end{mac
% \changes{germanb-2.6c}{1996/01/24}{Moved \cs{german@dq@disc}
%    babel.def, calling it \cs{bbl@dis

% \changes{germanb-2.6a}{1995/02/15}{Use \cs{ddot} instead
%    \cs{@MATHUMLAU

%    Now we can define the doublequote macros: the umlau
% \changes{germanb-2.6c}{1996/05/30}{added the \cs{allowhyphen
%    \begin{macroco
\declare@shorthand{german}{"a}{\textormath{\"{a}\allowhyphens}{\ddot
\declare@shorthand{german}{"o}{\textormath{\"{o}\allowhyphens}{\ddot
\declare@shorthand{german}{"u}{\textormath{\"{u}\allowhyphens}{\ddot
\declare@shorthand{german}{"A}{\textormath{\"{A}\allowhyphens}{\ddot
\declare@shorthand{german}{"O}{\textormath{\"{O}\allowhyphens}{\ddot
\declare@shorthand{german}{"U}{\textormath{\"{U}\allowhyphens}{\ddot
%    \end{macroco
%    trem
%    \begin{macroco
\declare@shorthand{german}{"e}{\textormath{\"{e}}{\ddot
\declare@shorthand{german}{"E}{\textormath{\"{E}}{\ddot
\declare@shorthand{german}{"i}{\textormath{\"{\i
                              {\ddot\imat
\declare@shorthand{german}{"I}{\textormath{\"{I}}{\ddot
%    \end{macroco
%    german es-zet (sharp
% \changes{germanb-2.6f}{1997/05/08}{use \cs{SS} instead
%    \texttt{SS}, removed braces after \cs{ss
%    \begin{macroco
\declare@shorthand{german}{"s}{\textormath{\ss}{\@SS{
\declare@shorthand{german}{"S}{\
\declare@shorthand{german}{"z}{\textormath{\ss}{\@SS{
\declare@shorthand{german}{"Z}{
%    \end{macroco
%    german and french quot
%    \begin{macroco
\declare@shorthand{german}{"`}{\gl
\declare@shorthand{german}{"'}{\gr
\declare@shorthand{german}{"<}{\fl
\declare@shorthand{german}{">}{\fr
%    \end{macroco
%    discretionary comma
%    \begin{macroco
\declare@shorthand{german}{"c}{\textormath{\bbl@disc ck}{
\declare@shorthand{german}{"C}{\textormath{\bbl@disc CK}{
\declare@shorthand{german}{"F}{\textormath{\bbl@disc F{FF}}{
\declare@shorthand{german}{"l}{\textormath{\bbl@disc l{ll}}{
\declare@shorthand{german}{"L}{\textormath{\bbl@disc L{LL}}{
\declare@shorthand{german}{"m}{\textormath{\bbl@disc m{mm}}{
\declare@shorthand{german}{"M}{\textormath{\bbl@disc M{MM}}{
\declare@shorthand{german}{"n}{\textormath{\bbl@disc n{nn}}{
\declare@shorthand{german}{"N}{\textormath{\bbl@disc N{NN}}{
\declare@shorthand{german}{"p}{\textormath{\bbl@disc p{pp}}{
\declare@shorthand{german}{"P}{\textormath{\bbl@disc P{PP}}{
\declare@shorthand{german}{"r}{\textormath{\bbl@disc r{rr}}{
\declare@shorthand{german}{"R}{\textormath{\bbl@disc R{RR}}{
\declare@shorthand{german}{"t}{\textormath{\bbl@disc t{tt}}{
\declare@shorthand{german}{"T}{\textormath{\bbl@disc T{TT}}{
%    \end{macroco
%    We need to treat |"f| a bit differently in order to preserve
%    ff-ligatur
% \changes{germanb-2.6f}{1998/06/15}{Copied the coding for \texttt{
%    from german.dtx version 2.5
%    \begin{macroco
\declare@shorthand{german}{"f}{\textormath{\bbl@discff}{
\def\bbl@discff{\penalty
  \afterassignment\bbl@insertff \let\bbl@nextff
\def\bbl@insertf
  \if f\bbl@nex
    \expandafter\@firstoftwo\else\expandafter\@secondoftwo
  {\relax\discretionary{ff-}{f}{ff}\allowhyphens}{f\bbl@nextf
\let\bbl@nextf
%    \end{macroco
%    and some additional comman
%    \begin{macroco
\declare@shorthand{german}{"-}{\nobreak\-\bbl@allowhyphe
\declare@shorthand{german}{"|
  \textormath{\penalty\@M\discretionary{-}{}{\kern.03e
              \allowhyphens}
\declare@shorthand{german}{""}{\hskip\z@sk
\declare@shorthand{german}{"~}{\textormath{\leavevmode\hbox{-}}{
\declare@shorthand{german}{"=}{\penalty\@M-\hskip\z@sk
%    \end{macroco

%  \begin{macro}{\mdq
%  \begin{macro}{\mdqo
%  \begin{macro}{\
%    All that's left to do now is to  define a couple of comma
%    for reasons of compatibility with \file{german.st
% \changes{germanb-2.6f}{1998/06/07}{Now use \cs{shorthandon}
%    \cs{shorthandoff
%    \begin{macroco
\def\mdqon{\shorthandon{
\def\mdqoff{\shorthandoff{
\def\ck{\allowhyphens\discretionary{k-}{k}{ck}\allowhyphe
%    \end{macroco
%  \end{mac
%  \end{mac
%  \end{mac

%    The macro |\ldf@finish| takes care of looking fo
%    configuration file, setting the main language to be switched
%    at |\begin{document}| and resetting the category code
%    \texttt{@} to its original val
% \changes{germanb-2.6d}{1996/11/02}{Now use \cs{ldf@finish} to w
%    u
%    \begin{macroco
\ldf@finish\CurrentOpt
%</co
%    \end{macroco

% \Fin

%% \CharacterTa
%%  {Upper-case    \A\B\C\D\E\F\G\H\I\J\K\L\M\N\O\P\Q\R\S\T\U\V\W\X\
%%   Lower-case    \a\b\c\d\e\f\g\h\i\j\k\l\m\n\o\p\q\r\s\t\u\v\w\x\
%%   Digits        \0\1\2\3\4\5\6\7\
%%   Exclamation   \!     Double quote  \"     Hash (number)
%%   Dollar        \$     Percent       \%     Ampersand
%%   Acute accent  \'     Left paren    \(     Right paren
%%   Asterisk      \*     Plus          \+     Comma
%%   Minus         \-     Point         \.     Solidus
%%   Colon         \:     Semicolon     \;     Less than
%%   Equals        \=     Greater than  \>     Question mark
%%   Commercial at \@     Left bracket  \[     Backslash
%%   Right bracket \]     Circumflex    \^     Underscore
%%   Grave accent  \`     Left brace    \{     Vertical bar
%%   Right brace   \}     Tilde

\endin
}
%    \end{macrocode}
% \changes{babel~3.8h}{2005/11/23}{added synonyms \Lopt{indonesian},
%    \Lopt{indon} and \Lopt{bahasai} for the original bahasa
%    (indonesia) support}
% \changes{babel~3.8h}{2005/11/23}{added \Lopt{malay}, \Lopt{meyaluy}
%    and \Lopt{bahasam} for the Bahasa Malaysia support}
%    \begin{macrocode}
\DeclareOption{bahasa}{\input{bahasai.ldf}}
\DeclareOption{indonesian}{\input{bahasai.ldf}}
\DeclareOption{indon}{\input{bahasai.ldf}}
\DeclareOption{bahasai}{\input{bahasai.ldf}}
\DeclareOption{malay}{% \iffalse meta-com

% Copyright 1989-2008 Johannes L. Braams and any individual aut
% listed elsewhere in this file.  All rights reser

% This file is part of the Babel sys
% ----------------------------------

% It may be distributed and/or modified under
% conditions of the LaTeX Project Public License, either version
% of this license or (at your option) any later vers
% The latest version of this license i
%   http://www.latex-project.org/lppl
% and version 1.3 or later is part of all distributions of L
% version 2003/12/01 or la

% This work has the LPPL maintenance status "maintain

% The Current Maintainer of this work is Johannes Bra

% The list of all files belonging to the Babel syste
% given in the file `manifest.bbl. See also `legal.bbl' for additi
% informat

% The list of derived (unpacked) files belonging to the distribu
% and covered by LPPL is defined by the unpacking scripts (
% extension .ins) which are part of the distribut
%
% \CheckSum{
%\iff
%    Tell the \LaTeX\ system who we are and write an entry on
%    transcr
%<*
\ProvidesFile{bahasam.
%</
%<code>\ProvidesLanguage{baha

%\ProvidesFile{bahasam.
       [2008/01/27 v1.0k Bahasa Malaysia support from the babel sys
%\iff
%% File `bahasam.
%% Babel package for LaTeX versio
%% Copyright (C) 1989 -
%%           by Johannes Braams, TeX

%% Bahasa Malaysia Language Definition
%% Copyright (C) 1994 -
%%           by J"org Knappen, (joerg.knappen at alpha.ntp.springer
%              Terry Mart (mart at vkpmzd.kph.uni-mainz
%              Institut f\"ur Kernph
%              Johannes Gutenberg-Universit\"at M
%              D-55099 M
%              Ger

%% Copyright (C) 2005,
%%           by Bob Margolis, (bob.margolis at ntlworld.
%              derived from J"ork Knappen's work - see ab
%%           [With help from Awangku Merali Pengiran Mohamed (Saraw
%               gratefully acknowled
%               Yate
%

%% Please report errors to: Bob Marg
%%                          bob.margolis at ntlworld
%%                          J.L. Br
%%                          babel at braams.xs4al

%    This file is part of the babel system, it provides the so
%    code for the  Bahasa Malaysia language defini
%    file.  The original version of this file was written by T
%    Mart (mart@vkpmzd.kph.uni-mainz.de) and J"org Kna
%    (knappen@vkpmzd.kph.uni-mainz.
%<*filedri
\documentclass{ltx
\newcommand*\TeXhax{\TeX
\newcommand*\babel{\textsf{bab
\newcommand*\langvar{$\langle \it lang \rang
\newcommand*\note[
\newcommand*\Lopt[1]{\textsf{
\newcommand*\file[1]{\texttt{
\begin{docum
 \DocInput{bahasam.
\end{docum
%</filedri

% \GetFileInfo{bahasam.

% \changes{bahasa-0.9c}{1994/06/26}{Removed the use of \cs{filed
%    and moved identification after the loading of \file{babel.d
% \changes{bahasa-1.0d}{1996/07/10}{Replaced \cs{undefined}
%    \cs{@undefined} and \cs{empty} with \cs{@empty} for consist
%    with \LaT
% \changes{bahasa-1.0e}{1996/10/10}{Moved the definitio
%    \cs{atcatcode} right to the beginni
% \changes{bahasam-0.9f}{2005/11/22}{A number of changes to make
%    specific to Bahasa Maya

%  \section{The Bahasa Malaysia langu

%    The file \file{\filename}\footnote{The file described in
%    section has version number \fileversion\ and was last revise
%    \filedate.}  defines all the language definition macros for
%    Bahasa Malaysia language. Bahasa just m
%    `language' in Bahasa Malaysia. A number of terms differ from those
%    in bahasa indone

%    For this language currently no special definitions are neede
%    availa

% \StopEventual

%    The macro |\LdfInit| takes care of preventing that this fil
%    loaded more than once, checking the category code of
%    \texttt{@} sign,
% \changes{bahasa-1.0e}{1996/11/02}{Now use \cs{LdfInit} to per
%    initial che
% \changes{bahasam-v1.0j}{2005/11/23}{Make it possible that this
%    is loaded by variuos opti
%    \begin{macroc
%<*c
\LdfInit\CurrentOption{date\CurrentOpt
%    \end{macroc

%    When this file is read as an option, i.e. by the |\usepack
%    command, \texttt{bahasa} could be an `unknown' language in w
%    case we have to make it known. So we check for the existenc
%    |\l@bahasa| to see whether we have to do something h

%    For both Bahasa Malaysia and Bahasa Indonesia the same se
%    hyphenation patterns can be used which are available in the
%    \file{inhyph.tex}. However it could be loaded using any of
%    possible Babel options fot the Malaysian and Indone
%    languase. So first we try to find out whether this is the c

% \changes{bahasa-0.9c}{1994/06/26}{Now use \cs{@patterns} to pro
%    the warn
%    \begin{macroc
\ifx\l@malay\@undef
  \ifx\l@meyalu\@undef
    \ifx\l@bahasam\@undef
      \ifx\l@bahasa\@undef
        \ifx\l@bahasai\@undef
          \ifx\l@indon\@undef
            \ifx\l@indonesian\@undef
              \@nopatterns{Bahasa Malay
              \adddialect\l@malay0\r
            \
              \let\l@malay\l@indone

          \
            \let\l@malay\l@i

        \
          \let\l@malay\l@bah

      \
        \let\l@malay\l@ba

    \
      \let\l@malay\l@bah

  \
    \let\l@malay\l@me


%    \end{macroc

%    Now that we are sure the |\l@malay| has some valid definitio
%    need to make sure that a name to access the hyphenation patte
%    corresponding to the option used, is availa
%    \begin{macroc
\expandafter\expandafter\expandafter
  \expandafter\cs
  \expandafter l\expandafter @\CurrentOption\endcs
  \l@m
%    \end{macroc

%    The next step consists of defining commands to switch to
%    from) the Bahasa langu

% \begin{macro}{\captionsbaha
%    The macro |\captionsbahasam| defines all strings used in the
%    standard documentclasses provided with \La
% \changes{bahasa-1.0b}{1995/07/04}{Added \cs{proofname}
%    AMS-\La
% \changes{bahasa-1.0d}{1996/07/09}{Replaced `Proof' by `Bu
%    (PR2214
% \changes{bahasa-1.0h}{2000/09/19}{Added \cs{glossaryna
% \changes{bahasa-1.0i}{2003/11/17}{Inserted translation for Gloss
% \changes{bahasam-1.0k}{2008/01/27}{Inserted changes from Awangku Mera
%    \begin{macroc
\@namedef{captions\CurrentOptio
  \def\prefacename{Praka
  \def\refname{Rujuk
  \def\abstractname{Abstrak}% (sometime it's called 'intis
                              %  or 'ikhtis
  \def\bibname{Bibliogra
  \def\chaptername{B
  \def\appendixname{Lampir
  \def\contentsname{Kandung
  \def\listfigurename{Senarai Gamb
  \def\listtablename{Senarai Jadu
  \def\indexname{Inde
  \def\figurename{Gamb
  \def\tablename{Jadu
  \def\partname{Bahagi
%  Subject:  Per
%  From:
  \def\enclname{Lampir
  \def\ccname{sk}% (short form for 'Salinan Kepa
  \def\headtoname{Kepa
  \def\pagename{Halam
%  Notes (Endnotes): Cat
  \def\seename{sila  ruj
  \def\alsoname{rujuk ju
  \def\proofname{Buk
  \def\glossaryname{Istil

%    \end{macroc
% \end{ma

% \begin{macro}{\datebaha
%    The macro |\datebahasam| redefines the command |\today| to pro
%    Bahasa Malaysian da
% \changes{bahasa-1.0f}{1997/10/01}{Use \cs{edef} to define \cs{tod
% \changes{bahasa~1.0f}{1998/03/28}{use \cs{def} instead of \cs{e
%    to save mem
% \changes{bahasa-1.0g}{1999/03/12}{Februari should be spelle
%    Pebru
% \changes{bahasam-1.0k}{2008/01/27}{Februari restored to BM spelli
%    see Collins Kamus Dwibahasa 2
%    \begin{macroc
\@namedef{date\CurrentOptio
  \def\today{\number\day~\ifcase\mont
    Januari\or Februari\or Mac\or April\or Mei\or Ju
    Julai\or Ogos\or September\or Oktober\or November\or Disembe
    \space \number\ye
%    \end{macroc
% \end{ma


% \begin{macro}{\extrasbaha
% \begin{macro}{\noextrasbaha
%    The macro |\extrasbahasa| will perform all the extra definit
%    needed for the Bahasa language. The macro |\extrasbahasa| is
%    to cancel the actions of |\extrasbahasa|.  For the moment t
%    macros are empty but they are defined for compatibility with
%    other language definition fi

%    \begin{macroc
\@namedef{extras\CurrentOptio
\@namedef{noextras\CurrentOptio
%    \end{macroc
% \end{ma
% \end{ma

%  \begin{macro}{\bahasamhyphenm
%    The bahasam hyphenation patterns should be used
%    |\lefthyphenmin| set to~2 and |\righthyphenmin| set t
% \changes{bahasa-1.0e}{1996/08/07}{use \cs{bahasamhyphenmins} to s
%    the correct val
% \changes{bahasa-1.0h}{2000/09/22}{Now use \cs{providehyphenmins
%    provide a default va
%    \begin{macroc
\providehyphenmins{\CurrentOption}{\tw@\
%    \end{macroc
%  \end{ma

%    The macro |\ldf@finish| takes care of looking f
%    configuration file, setting the main language to be switche
%    at |\begin{document}| and resetting the category cod
%    \texttt{@} to its original va
% \changes{bahasa-1.0e}{1996/11/02}{Now use \cs{ldf@finish} to wrap
%    \begin{macroc
\ldf@finish{\CurrentOpt
%</c
%    \end{macroc

% \Fi

%% \CharacterT
%%  {Upper-case    \A\B\C\D\E\F\G\H\I\J\K\L\M\N\O\P\Q\R\S\T\U\V\W\X
%%   Lower-case    \a\b\c\d\e\f\g\h\i\j\k\l\m\n\o\p\q\r\s\t\u\v\w\x
%%   Digits        \0\1\2\3\4\5\6\7
%%   Exclamation   \!     Double quote  \"     Hash (number
%%   Dollar        \$     Percent       \%     Ampersand
%%   Acute accent  \'     Left paren    \(     Right paren
%%   Asterisk      \*     Plus          \+     Comma
%%   Minus         \-     Point         \.     Solidus
%%   Colon         \:     Semicolon     \;     Less than
%%   Equals        \=     Greater than  \>     Question mar
%%   Commercial at \@     Left bracket  \[     Backslash
%%   Right bracket \]     Circumflex    \^     Underscore
%%   Grave accent  \`     Left brace    \{     Vertical bar
%%   Right brace   \}     Tilde

\endi
}
\DeclareOption{meyalu}{% \iffalse meta-com

% Copyright 1989-2008 Johannes L. Braams and any individual aut
% listed elsewhere in this file.  All rights reser

% This file is part of the Babel sys
% ----------------------------------

% It may be distributed and/or modified under
% conditions of the LaTeX Project Public License, either version
% of this license or (at your option) any later vers
% The latest version of this license i
%   http://www.latex-project.org/lppl
% and version 1.3 or later is part of all distributions of L
% version 2003/12/01 or la

% This work has the LPPL maintenance status "maintain

% The Current Maintainer of this work is Johannes Bra

% The list of all files belonging to the Babel syste
% given in the file `manifest.bbl. See also `legal.bbl' for additi
% informat

% The list of derived (unpacked) files belonging to the distribu
% and covered by LPPL is defined by the unpacking scripts (
% extension .ins) which are part of the distribut
%
% \CheckSum{
%\iff
%    Tell the \LaTeX\ system who we are and write an entry on
%    transcr
%<*
\ProvidesFile{bahasam.
%</
%<code>\ProvidesLanguage{baha

%\ProvidesFile{bahasam.
       [2008/01/27 v1.0k Bahasa Malaysia support from the babel sys
%\iff
%% File `bahasam.
%% Babel package for LaTeX versio
%% Copyright (C) 1989 -
%%           by Johannes Braams, TeX

%% Bahasa Malaysia Language Definition
%% Copyright (C) 1994 -
%%           by J"org Knappen, (joerg.knappen at alpha.ntp.springer
%              Terry Mart (mart at vkpmzd.kph.uni-mainz
%              Institut f\"ur Kernph
%              Johannes Gutenberg-Universit\"at M
%              D-55099 M
%              Ger

%% Copyright (C) 2005,
%%           by Bob Margolis, (bob.margolis at ntlworld.
%              derived from J"ork Knappen's work - see ab
%%           [With help from Awangku Merali Pengiran Mohamed (Saraw
%               gratefully acknowled
%               Yate
%

%% Please report errors to: Bob Marg
%%                          bob.margolis at ntlworld
%%                          J.L. Br
%%                          babel at braams.xs4al

%    This file is part of the babel system, it provides the so
%    code for the  Bahasa Malaysia language defini
%    file.  The original version of this file was written by T
%    Mart (mart@vkpmzd.kph.uni-mainz.de) and J"org Kna
%    (knappen@vkpmzd.kph.uni-mainz.
%<*filedri
\documentclass{ltx
\newcommand*\TeXhax{\TeX
\newcommand*\babel{\textsf{bab
\newcommand*\langvar{$\langle \it lang \rang
\newcommand*\note[
\newcommand*\Lopt[1]{\textsf{
\newcommand*\file[1]{\texttt{
\begin{docum
 \DocInput{bahasam.
\end{docum
%</filedri

% \GetFileInfo{bahasam.

% \changes{bahasa-0.9c}{1994/06/26}{Removed the use of \cs{filed
%    and moved identification after the loading of \file{babel.d
% \changes{bahasa-1.0d}{1996/07/10}{Replaced \cs{undefined}
%    \cs{@undefined} and \cs{empty} with \cs{@empty} for consist
%    with \LaT
% \changes{bahasa-1.0e}{1996/10/10}{Moved the definitio
%    \cs{atcatcode} right to the beginni
% \changes{bahasam-0.9f}{2005/11/22}{A number of changes to make
%    specific to Bahasa Maya

%  \section{The Bahasa Malaysia langu

%    The file \file{\filename}\footnote{The file described in
%    section has version number \fileversion\ and was last revise
%    \filedate.}  defines all the language definition macros for
%    Bahasa Malaysia language. Bahasa just m
%    `language' in Bahasa Malaysia. A number of terms differ from those
%    in bahasa indone

%    For this language currently no special definitions are neede
%    availa

% \StopEventual

%    The macro |\LdfInit| takes care of preventing that this fil
%    loaded more than once, checking the category code of
%    \texttt{@} sign,
% \changes{bahasa-1.0e}{1996/11/02}{Now use \cs{LdfInit} to per
%    initial che
% \changes{bahasam-v1.0j}{2005/11/23}{Make it possible that this
%    is loaded by variuos opti
%    \begin{macroc
%<*c
\LdfInit\CurrentOption{date\CurrentOpt
%    \end{macroc

%    When this file is read as an option, i.e. by the |\usepack
%    command, \texttt{bahasa} could be an `unknown' language in w
%    case we have to make it known. So we check for the existenc
%    |\l@bahasa| to see whether we have to do something h

%    For both Bahasa Malaysia and Bahasa Indonesia the same se
%    hyphenation patterns can be used which are available in the
%    \file{inhyph.tex}. However it could be loaded using any of
%    possible Babel options fot the Malaysian and Indone
%    languase. So first we try to find out whether this is the c

% \changes{bahasa-0.9c}{1994/06/26}{Now use \cs{@patterns} to pro
%    the warn
%    \begin{macroc
\ifx\l@malay\@undef
  \ifx\l@meyalu\@undef
    \ifx\l@bahasam\@undef
      \ifx\l@bahasa\@undef
        \ifx\l@bahasai\@undef
          \ifx\l@indon\@undef
            \ifx\l@indonesian\@undef
              \@nopatterns{Bahasa Malay
              \adddialect\l@malay0\r
            \
              \let\l@malay\l@indone

          \
            \let\l@malay\l@i

        \
          \let\l@malay\l@bah

      \
        \let\l@malay\l@ba

    \
      \let\l@malay\l@bah

  \
    \let\l@malay\l@me


%    \end{macroc

%    Now that we are sure the |\l@malay| has some valid definitio
%    need to make sure that a name to access the hyphenation patte
%    corresponding to the option used, is availa
%    \begin{macroc
\expandafter\expandafter\expandafter
  \expandafter\cs
  \expandafter l\expandafter @\CurrentOption\endcs
  \l@m
%    \end{macroc

%    The next step consists of defining commands to switch to
%    from) the Bahasa langu

% \begin{macro}{\captionsbaha
%    The macro |\captionsbahasam| defines all strings used in the
%    standard documentclasses provided with \La
% \changes{bahasa-1.0b}{1995/07/04}{Added \cs{proofname}
%    AMS-\La
% \changes{bahasa-1.0d}{1996/07/09}{Replaced `Proof' by `Bu
%    (PR2214
% \changes{bahasa-1.0h}{2000/09/19}{Added \cs{glossaryna
% \changes{bahasa-1.0i}{2003/11/17}{Inserted translation for Gloss
% \changes{bahasam-1.0k}{2008/01/27}{Inserted changes from Awangku Mera
%    \begin{macroc
\@namedef{captions\CurrentOptio
  \def\prefacename{Praka
  \def\refname{Rujuk
  \def\abstractname{Abstrak}% (sometime it's called 'intis
                              %  or 'ikhtis
  \def\bibname{Bibliogra
  \def\chaptername{B
  \def\appendixname{Lampir
  \def\contentsname{Kandung
  \def\listfigurename{Senarai Gamb
  \def\listtablename{Senarai Jadu
  \def\indexname{Inde
  \def\figurename{Gamb
  \def\tablename{Jadu
  \def\partname{Bahagi
%  Subject:  Per
%  From:
  \def\enclname{Lampir
  \def\ccname{sk}% (short form for 'Salinan Kepa
  \def\headtoname{Kepa
  \def\pagename{Halam
%  Notes (Endnotes): Cat
  \def\seename{sila  ruj
  \def\alsoname{rujuk ju
  \def\proofname{Buk
  \def\glossaryname{Istil

%    \end{macroc
% \end{ma

% \begin{macro}{\datebaha
%    The macro |\datebahasam| redefines the command |\today| to pro
%    Bahasa Malaysian da
% \changes{bahasa-1.0f}{1997/10/01}{Use \cs{edef} to define \cs{tod
% \changes{bahasa~1.0f}{1998/03/28}{use \cs{def} instead of \cs{e
%    to save mem
% \changes{bahasa-1.0g}{1999/03/12}{Februari should be spelle
%    Pebru
% \changes{bahasam-1.0k}{2008/01/27}{Februari restored to BM spelli
%    see Collins Kamus Dwibahasa 2
%    \begin{macroc
\@namedef{date\CurrentOptio
  \def\today{\number\day~\ifcase\mont
    Januari\or Februari\or Mac\or April\or Mei\or Ju
    Julai\or Ogos\or September\or Oktober\or November\or Disembe
    \space \number\ye
%    \end{macroc
% \end{ma


% \begin{macro}{\extrasbaha
% \begin{macro}{\noextrasbaha
%    The macro |\extrasbahasa| will perform all the extra definit
%    needed for the Bahasa language. The macro |\extrasbahasa| is
%    to cancel the actions of |\extrasbahasa|.  For the moment t
%    macros are empty but they are defined for compatibility with
%    other language definition fi

%    \begin{macroc
\@namedef{extras\CurrentOptio
\@namedef{noextras\CurrentOptio
%    \end{macroc
% \end{ma
% \end{ma

%  \begin{macro}{\bahasamhyphenm
%    The bahasam hyphenation patterns should be used
%    |\lefthyphenmin| set to~2 and |\righthyphenmin| set t
% \changes{bahasa-1.0e}{1996/08/07}{use \cs{bahasamhyphenmins} to s
%    the correct val
% \changes{bahasa-1.0h}{2000/09/22}{Now use \cs{providehyphenmins
%    provide a default va
%    \begin{macroc
\providehyphenmins{\CurrentOption}{\tw@\
%    \end{macroc
%  \end{ma

%    The macro |\ldf@finish| takes care of looking f
%    configuration file, setting the main language to be switche
%    at |\begin{document}| and resetting the category cod
%    \texttt{@} to its original va
% \changes{bahasa-1.0e}{1996/11/02}{Now use \cs{ldf@finish} to wrap
%    \begin{macroc
\ldf@finish{\CurrentOpt
%</c
%    \end{macroc

% \Fi

%% \CharacterT
%%  {Upper-case    \A\B\C\D\E\F\G\H\I\J\K\L\M\N\O\P\Q\R\S\T\U\V\W\X
%%   Lower-case    \a\b\c\d\e\f\g\h\i\j\k\l\m\n\o\p\q\r\s\t\u\v\w\x
%%   Digits        \0\1\2\3\4\5\6\7
%%   Exclamation   \!     Double quote  \"     Hash (number
%%   Dollar        \$     Percent       \%     Ampersand
%%   Acute accent  \'     Left paren    \(     Right paren
%%   Asterisk      \*     Plus          \+     Comma
%%   Minus         \-     Point         \.     Solidus
%%   Colon         \:     Semicolon     \;     Less than
%%   Equals        \=     Greater than  \>     Question mar
%%   Commercial at \@     Left bracket  \[     Backslash
%%   Right bracket \]     Circumflex    \^     Underscore
%%   Grave accent  \`     Left brace    \{     Vertical bar
%%   Right brace   \}     Tilde

\endi
}
\DeclareOption{bahasam}{% \iffalse meta-com

% Copyright 1989-2008 Johannes L. Braams and any individual aut
% listed elsewhere in this file.  All rights reser

% This file is part of the Babel sys
% ----------------------------------

% It may be distributed and/or modified under
% conditions of the LaTeX Project Public License, either version
% of this license or (at your option) any later vers
% The latest version of this license i
%   http://www.latex-project.org/lppl
% and version 1.3 or later is part of all distributions of L
% version 2003/12/01 or la

% This work has the LPPL maintenance status "maintain

% The Current Maintainer of this work is Johannes Bra

% The list of all files belonging to the Babel syste
% given in the file `manifest.bbl. See also `legal.bbl' for additi
% informat

% The list of derived (unpacked) files belonging to the distribu
% and covered by LPPL is defined by the unpacking scripts (
% extension .ins) which are part of the distribut
%
% \CheckSum{
%\iff
%    Tell the \LaTeX\ system who we are and write an entry on
%    transcr
%<*
\ProvidesFile{bahasam.
%</
%<code>\ProvidesLanguage{baha

%\ProvidesFile{bahasam.
       [2008/01/27 v1.0k Bahasa Malaysia support from the babel sys
%\iff
%% File `bahasam.
%% Babel package for LaTeX versio
%% Copyright (C) 1989 -
%%           by Johannes Braams, TeX

%% Bahasa Malaysia Language Definition
%% Copyright (C) 1994 -
%%           by J"org Knappen, (joerg.knappen at alpha.ntp.springer
%              Terry Mart (mart at vkpmzd.kph.uni-mainz
%              Institut f\"ur Kernph
%              Johannes Gutenberg-Universit\"at M
%              D-55099 M
%              Ger

%% Copyright (C) 2005,
%%           by Bob Margolis, (bob.margolis at ntlworld.
%              derived from J"ork Knappen's work - see ab
%%           [With help from Awangku Merali Pengiran Mohamed (Saraw
%               gratefully acknowled
%               Yate
%

%% Please report errors to: Bob Marg
%%                          bob.margolis at ntlworld
%%                          J.L. Br
%%                          babel at braams.xs4al

%    This file is part of the babel system, it provides the so
%    code for the  Bahasa Malaysia language defini
%    file.  The original version of this file was written by T
%    Mart (mart@vkpmzd.kph.uni-mainz.de) and J"org Kna
%    (knappen@vkpmzd.kph.uni-mainz.
%<*filedri
\documentclass{ltx
\newcommand*\TeXhax{\TeX
\newcommand*\babel{\textsf{bab
\newcommand*\langvar{$\langle \it lang \rang
\newcommand*\note[
\newcommand*\Lopt[1]{\textsf{
\newcommand*\file[1]{\texttt{
\begin{docum
 \DocInput{bahasam.
\end{docum
%</filedri

% \GetFileInfo{bahasam.

% \changes{bahasa-0.9c}{1994/06/26}{Removed the use of \cs{filed
%    and moved identification after the loading of \file{babel.d
% \changes{bahasa-1.0d}{1996/07/10}{Replaced \cs{undefined}
%    \cs{@undefined} and \cs{empty} with \cs{@empty} for consist
%    with \LaT
% \changes{bahasa-1.0e}{1996/10/10}{Moved the definitio
%    \cs{atcatcode} right to the beginni
% \changes{bahasam-0.9f}{2005/11/22}{A number of changes to make
%    specific to Bahasa Maya

%  \section{The Bahasa Malaysia langu

%    The file \file{\filename}\footnote{The file described in
%    section has version number \fileversion\ and was last revise
%    \filedate.}  defines all the language definition macros for
%    Bahasa Malaysia language. Bahasa just m
%    `language' in Bahasa Malaysia. A number of terms differ from those
%    in bahasa indone

%    For this language currently no special definitions are neede
%    availa

% \StopEventual

%    The macro |\LdfInit| takes care of preventing that this fil
%    loaded more than once, checking the category code of
%    \texttt{@} sign,
% \changes{bahasa-1.0e}{1996/11/02}{Now use \cs{LdfInit} to per
%    initial che
% \changes{bahasam-v1.0j}{2005/11/23}{Make it possible that this
%    is loaded by variuos opti
%    \begin{macroc
%<*c
\LdfInit\CurrentOption{date\CurrentOpt
%    \end{macroc

%    When this file is read as an option, i.e. by the |\usepack
%    command, \texttt{bahasa} could be an `unknown' language in w
%    case we have to make it known. So we check for the existenc
%    |\l@bahasa| to see whether we have to do something h

%    For both Bahasa Malaysia and Bahasa Indonesia the same se
%    hyphenation patterns can be used which are available in the
%    \file{inhyph.tex}. However it could be loaded using any of
%    possible Babel options fot the Malaysian and Indone
%    languase. So first we try to find out whether this is the c

% \changes{bahasa-0.9c}{1994/06/26}{Now use \cs{@patterns} to pro
%    the warn
%    \begin{macroc
\ifx\l@malay\@undef
  \ifx\l@meyalu\@undef
    \ifx\l@bahasam\@undef
      \ifx\l@bahasa\@undef
        \ifx\l@bahasai\@undef
          \ifx\l@indon\@undef
            \ifx\l@indonesian\@undef
              \@nopatterns{Bahasa Malay
              \adddialect\l@malay0\r
            \
              \let\l@malay\l@indone

          \
            \let\l@malay\l@i

        \
          \let\l@malay\l@bah

      \
        \let\l@malay\l@ba

    \
      \let\l@malay\l@bah

  \
    \let\l@malay\l@me


%    \end{macroc

%    Now that we are sure the |\l@malay| has some valid definitio
%    need to make sure that a name to access the hyphenation patte
%    corresponding to the option used, is availa
%    \begin{macroc
\expandafter\expandafter\expandafter
  \expandafter\cs
  \expandafter l\expandafter @\CurrentOption\endcs
  \l@m
%    \end{macroc

%    The next step consists of defining commands to switch to
%    from) the Bahasa langu

% \begin{macro}{\captionsbaha
%    The macro |\captionsbahasam| defines all strings used in the
%    standard documentclasses provided with \La
% \changes{bahasa-1.0b}{1995/07/04}{Added \cs{proofname}
%    AMS-\La
% \changes{bahasa-1.0d}{1996/07/09}{Replaced `Proof' by `Bu
%    (PR2214
% \changes{bahasa-1.0h}{2000/09/19}{Added \cs{glossaryna
% \changes{bahasa-1.0i}{2003/11/17}{Inserted translation for Gloss
% \changes{bahasam-1.0k}{2008/01/27}{Inserted changes from Awangku Mera
%    \begin{macroc
\@namedef{captions\CurrentOptio
  \def\prefacename{Praka
  \def\refname{Rujuk
  \def\abstractname{Abstrak}% (sometime it's called 'intis
                              %  or 'ikhtis
  \def\bibname{Bibliogra
  \def\chaptername{B
  \def\appendixname{Lampir
  \def\contentsname{Kandung
  \def\listfigurename{Senarai Gamb
  \def\listtablename{Senarai Jadu
  \def\indexname{Inde
  \def\figurename{Gamb
  \def\tablename{Jadu
  \def\partname{Bahagi
%  Subject:  Per
%  From:
  \def\enclname{Lampir
  \def\ccname{sk}% (short form for 'Salinan Kepa
  \def\headtoname{Kepa
  \def\pagename{Halam
%  Notes (Endnotes): Cat
  \def\seename{sila  ruj
  \def\alsoname{rujuk ju
  \def\proofname{Buk
  \def\glossaryname{Istil

%    \end{macroc
% \end{ma

% \begin{macro}{\datebaha
%    The macro |\datebahasam| redefines the command |\today| to pro
%    Bahasa Malaysian da
% \changes{bahasa-1.0f}{1997/10/01}{Use \cs{edef} to define \cs{tod
% \changes{bahasa~1.0f}{1998/03/28}{use \cs{def} instead of \cs{e
%    to save mem
% \changes{bahasa-1.0g}{1999/03/12}{Februari should be spelle
%    Pebru
% \changes{bahasam-1.0k}{2008/01/27}{Februari restored to BM spelli
%    see Collins Kamus Dwibahasa 2
%    \begin{macroc
\@namedef{date\CurrentOptio
  \def\today{\number\day~\ifcase\mont
    Januari\or Februari\or Mac\or April\or Mei\or Ju
    Julai\or Ogos\or September\or Oktober\or November\or Disembe
    \space \number\ye
%    \end{macroc
% \end{ma


% \begin{macro}{\extrasbaha
% \begin{macro}{\noextrasbaha
%    The macro |\extrasbahasa| will perform all the extra definit
%    needed for the Bahasa language. The macro |\extrasbahasa| is
%    to cancel the actions of |\extrasbahasa|.  For the moment t
%    macros are empty but they are defined for compatibility with
%    other language definition fi

%    \begin{macroc
\@namedef{extras\CurrentOptio
\@namedef{noextras\CurrentOptio
%    \end{macroc
% \end{ma
% \end{ma

%  \begin{macro}{\bahasamhyphenm
%    The bahasam hyphenation patterns should be used
%    |\lefthyphenmin| set to~2 and |\righthyphenmin| set t
% \changes{bahasa-1.0e}{1996/08/07}{use \cs{bahasamhyphenmins} to s
%    the correct val
% \changes{bahasa-1.0h}{2000/09/22}{Now use \cs{providehyphenmins
%    provide a default va
%    \begin{macroc
\providehyphenmins{\CurrentOption}{\tw@\
%    \end{macroc
%  \end{ma

%    The macro |\ldf@finish| takes care of looking f
%    configuration file, setting the main language to be switche
%    at |\begin{document}| and resetting the category cod
%    \texttt{@} to its original va
% \changes{bahasa-1.0e}{1996/11/02}{Now use \cs{ldf@finish} to wrap
%    \begin{macroc
\ldf@finish{\CurrentOpt
%</c
%    \end{macroc

% \Fi

%% \CharacterT
%%  {Upper-case    \A\B\C\D\E\F\G\H\I\J\K\L\M\N\O\P\Q\R\S\T\U\V\W\X
%%   Lower-case    \a\b\c\d\e\f\g\h\i\j\k\l\m\n\o\p\q\r\s\t\u\v\w\x
%%   Digits        \0\1\2\3\4\5\6\7
%%   Exclamation   \!     Double quote  \"     Hash (number
%%   Dollar        \$     Percent       \%     Ampersand
%%   Acute accent  \'     Left paren    \(     Right paren
%%   Asterisk      \*     Plus          \+     Comma
%%   Minus         \-     Point         \.     Solidus
%%   Colon         \:     Semicolon     \;     Less than
%%   Equals        \=     Greater than  \>     Question mar
%%   Commercial at \@     Left bracket  \[     Backslash
%%   Right bracket \]     Circumflex    \^     Underscore
%%   Grave accent  \`     Left brace    \{     Vertical bar
%%   Right brace   \}     Tilde

\endi
}
\DeclareOption{basque}{% \iffalse meta-com

% Copyright 1989-2005 Johannes L. Braams and any individual aut
% listed elsewhere in this file.  All rights reser

% This file is part of the Babel sys
% ----------------------------------

% It may be distributed and/or modified under
% conditions of the LaTeX Project Public License, either version
% of this license or (at your option) any later vers
% The latest version of this license i
%   http://www.latex-project.org/lppl
% and version 1.3 or later is part of all distributions of L
% version 2003/12/01 or la

% This work has the LPPL maintenance status "maintain

% The Current Maintainer of this work is Johannes Bra

% The list of all files belonging to the Babel syste
% given in the file `manifest.bbl. See also `legal.bbl' for additi
% informat

% The list of derived (unpacked) files belonging to the distribu
% and covered by LPPL is defined by the unpacking scripts (
% extension .ins) which are part of the distribut
%
% \CheckSum{
% \iff
%    Tell the \LaTeX\ system who we are and write an entry on
%    transcr
%<*
\ProvidesFile{basque.
%</
%<code>\ProvidesLanguage{bas

%\ProvidesFile{basque.
        [2005/03/29 v1.0f Basque support from the babel sys
%\iff
%% File `Basque.
%% Babel package for LaTeX versio
%% Copyright (C) 1989 -
%%           by Johannes Braams, TeX

%% Basque Language Definition
%% Copyright (C) 1997 -
%%           by Juan M. Aguirregab
%              The University of the Basque Cou
%              Dept. of Theoretical Phy
%              Apdo.
%              E-48080 Bi
%              S
%              tel: +34 946 012
%              fax: +34 944 648
%              e-mail: wtpagagj at lg.eh
%              WWW:    http://tp.lc.ehu.es/jma.
% based on
% Spanish Language Definition
% Copyright (C) 1991 -
%           by Julio San
%              GMV
%              c/ Isaac Newto
%              PTM - Tres Ca
%              E-28760 Ma
%              S
%              tel: +34 1 807 2
%              fax +34 1 807 2
%              jsanchez at gm

% Acknowledgements: I am indebted to Zunbeltz Iza
%                   who suggested the use of \discretionary i

%% Please report errors to: Juan M. Aguirregabiria <wtpagagj at lg.ehu
%%                          (or J.L. Braams <babel at braams.cistron.

%    This file is part of the babel system, it provides the so
%    code for the basque language definition f
%<*filedri
\documentclass{ltx
\newcommand*\TeXhax{\TeX
\newcommand*\babel{\textsf{bab
\newcommand*\langvar{$\langle \it lang \rang
\newcommand*\note[
\newcommand*\Lopt[1]{\textsf{
\newcommand*\file[1]{\texttt{
\begin{docum
 \DocInput{basque.
\end{docum
%</filedri

% \GetFileInfo{basque.

%  \section{The Basque langu

%    The file \file{\filename}\footnote{The file described in
%    section has version number \fileversion\ and was last revise
%    \filedate. The original author is Juan M. Aguirregabi
%    (\texttt{wtpagagj@lg.ehu.es}) and is based on the Spanish
%    by Julio S\'anc
%    (\texttt{jsanchez@gmv.es}).} defines all the language defini
%    macro's for the Basque langu

%    For this language the characters |~| and |"| are
%    active. In table~\ref{tab:basque-quote} an overview is give
%    their purp
%    \begin{table}[
%     \cente
%     \begin{tabular}{lp{8
%      \verb="|= & disable ligature at this positio
%      |"-| & an explicit hyphen sign, allowing hyphena
%             in the rest of the wor
%      |\-| & like the old |\-|, but allowing hyphena
%             in the rest of the word
%      |"<| & for French left double quotes (similar to $<<$
%      |">| & for French right double quotes (similar to $>>$
%      |~n| & a n with tilde. Works for uppercase
%     \end{tabu
%     \caption{The extra definitions made by \file{basque.l
%     \label{tab:basque-qu
%    \end{ta
%    These active accent characters behave according to their orig
%    definitions if not followed by one of the characters indicate
%    that ta

% \changes{basque-1.0f}{2002/01/07}{Changed url's for the patt
%    f
%    This option includes support for working with extended, 8
%    fonts, if available. Support is base
%    providing an appropriate definition for the accent macro
%    entry to the Basque language. This is automatically don
%    \LaTeXe\ or NFSS2. If T1 encoding is chosen, and provided
%    adequate hyphenation patterns\footnote{One source for
%    patterns is the archive at \texttt{tp.lc.ehu.es} that ca
%    accessed by anonymous FTP o
%    \texttt{http://tp.lc.ehu.es/jma/basque.html}} are availab
%    The easiest way to use the new encoding with \LaTeXe{
%    to load the package \texttt{t1enc} with |\usepackage|. This
%    be done before loading \ba

% \StopEventual

%    The macro |\LdfInit| takes care of preventing that this fil
%    loaded more than once, checking the category code of
%    \texttt{@} sign,
%    \begin{macroc
%<*c
\LdfInit{basque}\captionsba
%    \end{macroc

%    When this file is read as an option, i.e. by the |\usepack
%    command, \texttt{basque} could be an `unknown' language in w
%    case we have to make it known.  So we check for the existenc
%    |\l@basque| to see whether we have to do something h

%    \begin{macroc
\ifx\l@basque\@undef
  \@nopatterns{Bas
  \adddialect\l@bas

%    \end{macroc

%    The next step consists of defining commands to switch to
%    from) the Basque langu

% \begin{macro}{\captionsbas
%    The macro |\captionsbasque| defines all strings used
%    the four standard documentclasses provided with \La
% \changes{basque-1.0e}{2000/09/19}{Added \cs{glossaryna
% \changes{basque-1.0f}{2002/01/07}{Added translation for Gloss
%    \begin{macroc
\addto\captionsbasq
  \def\prefacename{Hitzaurr
  \def\refname{Erreferentzi
  \def\abstractname{Laburpe
  \def\bibname{Bibliograf
  \def\chaptername{Kapitul
  \def\appendixname{Eranski
  \def\contentsname{Gaien Aurkibid
  \def\listfigurename{Irudien Zerren
  \def\listtablename{Taulen Zerren
  \def\indexname{Kontzeptuen Aurkibid
  \def\figurename{Irud
  \def\tablename{Tau
  \def\partname{Ata
  \def\enclname{Erants
  \def\ccname{Kop
  \def\headtoname{No
  \def\pagename{Orr
  \def\seename{Iku
  \def\alsoname{Ikusi, halab
  \def\proofname{Frogape
  \def\glossaryname{Glosari

%    \end{macroc
% \end{ma

% \begin{macro}{\datebas
%    The macro |\datebasque| redefines the command |\today
%    produce Ba
% \changes{basque-1.0b}{1997/10/01}{Use \cs{edef} to define \cs{to
%    to save mem
% \changes{basque-1.0b}{1998/03/28}{use \cs{def} instead of \cs{ed
% \changes{basque-1.0c}{1999/11/22}{fixed typo in April's n
%    \begin{macroc
\def\datebasq
  \def\today{\number\year.eko\space\ifcase\mont
    urtarrilaren\or otsailaren\or martxoaren\or apirilare
    maiatzaren\or ekainaren\or uztailaren\or abuztuare
    irailaren\or urriaren\or azaroare
    abenduaren\fi~\number\d
%    \end{macroc
% \end{ma

% \begin{macro}{\extrasbas
% \begin{macro}{\noextrasbas
%    The macro |\extrasbasque| will perform all the extra definit
%    needed for the Basque language. The macro |\noextrasbasque
%    used to cancel the actions of |\extrasbasque|. For Basque,
%    characters are made active or are redefined. In particular,
%    \texttt{"} character and the |~| character receive
%    meanings. Therefore these characters have to be treate
%    `special' characte

%    \begin{macroc
\addto\extrasbasque{\languageshorthands{basq
\initiate@active@cha
\initiate@active@cha
\addto\extrasbasq
  \bbl@activate
  \bbl@activate
%    \end{macroc
%    Don't forget to turn the shorthands off ag
% \changes{basque-1.0d}{1999/12/16}{Deactivate shorthands ousid
%    Bas
%    \begin{macroc
\addto\noextrasbas
  \bbl@deactivate{"}\bbl@deactivate
%    \end{macroc

%    Apart from the active characters some other macros get a
%    definition. Therefore we store the current one to be abl
%    restore them la
%    \begin{macroc
\addto\extrasbasq
  \babel@sav
  \babel@sav
  \def\"{\protect\@umla
  \def\~{\protect\@til
%    \end{macroc
% \end{ma
% \end{ma

%  \begin{macro}{\basquehyphenm
%    Basque hyphenation uses |\lefthyphenmin| and |\righthyphen
%    both set t
% \changes{basque-1.0e}{2000/09/22}{Now use \cs{providehyphenmins
%    provide a default va
%    \begin{macroc
\providehyphenmins{\CurrentOption}{\tw@\
%    \end{macroc
% \end{ma

%  \begin{macro}{\diere
%  \begin{macro}{\textti
%    The original definition of |\"| is stored as |\dieresia|, bec
%    the we do not know what is its definition, since it depend
%    the encoding we are using or on special macros that the
%    might have loaded. The expansion of the macro might use the \
%    |\accent| primitive using some particular accent that the
%    provides or might check if a combined accent exists in the f
%    These two cases happen with respectively OT1 and T1 encodi
%    For this reason we save the definition of |\"| and use tha
%    the definition of other macros. We do likewise for |\'|
%    |\~|. The present coding of this option file is incorrect in
%    it can break when the encoding changes. We do not
%    |\tilde| as the macro name because it is already define
%    |\mathacce
%    \begin{macroc
\let\dieres
\let\texttil
%    \end{macroc
%  \end{ma
%  \end{ma

%  \begin{macro}{\@uml
%  \begin{macro}{\@ti
%    We check the encoding and if not using T1, we make the acc
%    expand but enabling hyphenation beyond the accent. If this is
%    case, not all break positions will be found in words that con
%    accents, but this is a limitation in \TeX. An unsolved pro
%    here is that the encoding can change at any time. The definit
%    below are made in such a way that a change between two 256-
%    encodings are supported, but changes between a 128-char a
%    256-char encoding are not properly supported. We check if T
%    in use. If not, we will give a warning and proceed redefining
%    accent macros so that \TeX{} at least finds the breaks that
%    not too close to the accent. The warning will only be printe
%    the log f
%    \begin{macroc
\ifx\DeclareFontShape\@undef
  \wlog{Warning: You are using an old La
  \wlog{Some word breaks will not be fou
  \def\@umlaut#1{\allowhyphens\dieresia{#1}\allowhyph
  \def\@tilde#1{\allowhyphens\texttilde{#1}\allowhyph
\
  \edef\bbl@next
  \ifx\f@encoding\bbl@
    \let\@umlaut\dier
    \let\@tilde\textt
  \
    \wlog{Warning: You are using encoding \f@encoding\s
      instead of
    \wlog{Some word breaks will not be fou
    \def\@umlaut#1{\allowhyphens\dieresia{#1}\allowhyph
    \def\@tilde#1{\allowhyphens\texttilde{#1}\allowhyph


%    \end{macroc
%  \end{ma
%  \end{ma

%     Now we can define our shorthands: the french quo
% \changes{basque-1.0b}{1997/04/03}{Removed empty groups a
%    guillemot character
%    \begin{macroc
\declare@shorthand{basque}{"
  \textormath{\guillemotleft}{\mbox{\guillemotlef
\declare@shorthand{basque}{"
  \textormath{\guillemotright}{\mbox{\guillemotrigh
%    \end{macroc
%    ordinals\footnote{The code for the ordinals was taken from
%    answer provided by Raymond
%    (\texttt{raymond@math.berkeley.edu}) to a question by Joseph
%    (\texttt{yogi@cs.ubc.ca}) in \texttt{comp.text.tex
%    \begin{macroc
  \declare@shorthand{basque}{'
    \textormath{\textquotedblright}{\sp\bgroup\prim@
%    \end{macroc
%    til
%    \begin{macroc
\declare@shorthand{basque}{~n}{\textormath{\~n}{\@tilde
\declare@shorthand{basque}{~N}{\textormath{\~N}{\@tilde
%    \end{macroc
%    and some additional comma

%    The shorthand |"-| should be used in places where a word cont
%    an explictit hyphenation character. According to the Academ
%    the Basque language, when a word break occurs at an expl
%    hyphen it must appear \emph{both} at the end of the first
%    \emph{and} at the beginning of the second l
% \changes{changes-1.0f}{2002/01/07}{The hyphen char needs to ap
%    at the beginning of the line as we
%    \begin{macroc
\declare@shorthand{basque}{"
  \nobreak\discretionary{-}{-}{-}\bbl@allowhyph
\declare@shorthand{basque}{"
  \textormath{\nobreak\discretionary{-}{}{\kern.03
              \allowhyphens
%    \end{macroc

%    The macro |\ldf@finish| takes care of looking f
%    configuration file, setting the main language to be switche
%    at |\begin{document}| and resetting the category cod
%    \texttt{@} to its original va
%    \begin{macroc
\ldf@finish{bas
%</c
%    \end{macroc

% \Fi


%% \CharacterT
%%  {Upper-case    \A\B\C\D\E\F\G\H\I\J\K\L\M\N\O\P\Q\R\S\T\U\V\W\X
%%   Lower-case    \a\b\c\d\e\f\g\h\i\j\k\l\m\n\o\p\q\r\s\t\u\v\w\x
%%   Digits        \0\1\2\3\4\5\6\7
%%   Exclamation   \!     Double quote  \"     Hash (number
%%   Dollar        \$     Percent       \%     Ampersand
%%   Acute accent  \'     Left paren    \(     Right paren
%%   Asterisk      \*     Plus          \+     Comma
%%   Minus         \-     Point         \.     Solidus
%%   Colon         \:     Semicolon     \;     Less than
%%   Equals        \=     Greater than  \>     Question mar
%%   Commercial at \@     Left bracket  \[     Backslash
%%   Right bracket \]     Circumflex    \^     Underscore
%%   Grave accent  \`     Left brace    \{     Vertical bar
%%   Right brace   \}     Tilde

\endi
}
\DeclareOption{brazil}{% \iffalse meta-comment
%
% Copyright 1989-2008 Johannes L. Braams and any individual authors
% listed elsewhere in this file.  All rights reserved.
% 
% This file is part of the Babel system.
% --------------------------------------
% 
% It may be distributed and/or modified under the
% conditions of the LaTeX Project Public License, either version 1.3
% of this license or (at your option) any later version.
% The latest version of this license is in
%   http://www.latex-project.org/lppl.txt
% and version 1.3 or later is part of all distributions of LaTeX
% version 2003/12/01 or later.
% 
% This work has the LPPL maintenance status "maintained".
% 
% The Current Maintainer of this work is Johannes Braams.
% 
% The list of all files belonging to the Babel system is
% given in the file `manifest.bbl. See also `legal.bbl' for additional
% information.
% 
% The list of derived (unpacked) files belonging to the distribution
% and covered by LPPL is defined by the unpacking scripts (with
% extension .ins) which are part of the distribution.
% \fi
% \CheckSum{320}
% \iffalse
%    Tell the \LaTeX\ system who we are and write an entry on the
%    transcript.
%<*dtx>
\ProvidesFile{portuges.dtx}
%</dtx>
%<code>\ProvidesLanguage{portuges}
%\fi
%\ProvidesFile{portuges.dtx}
        [2008/03/18 v1.2q Portuguese support from the babel system]
%\iffalse
%% File `portuges.dtx'
%% Babel package for LaTeX version 2e
%% Copyright (C) 1989 - 2008
%%           by Johannes Braams, TeXniek
%
%% Portuguese Language Definition File
%% Copyright (C) 1989 - 2008
%%           by Johannes Braams, TeXniek
%
%% Please report errors to: J.L. Braams
%%                          babel at braams.cistron.nl
%
%    This file is part of the babel system, it provides the source
%    code for the Portuguese language definition file.  The Portuguese
%    words were contributed by Jose Pedro Ramalhete, (JRAMALHE@CERNVM
%    or Jose-Pedro_Ramalhete@MACMAIL).
%
%    Arnaldo Viegas de Lima <arnaldo@VNET.IBM.COM> contributed
%    brasilian translations and suggestions for enhancements.
%<*filedriver>
\documentclass{ltxdoc}
\newcommand*\TeXhax{\TeX hax}
\newcommand*\babel{\textsf{babel}}
\newcommand*\langvar{$\langle \it lang \rangle$}
\newcommand*\note[1]{}
\newcommand*\Lopt[1]{\textsf{#1}}
\newcommand*\file[1]{\texttt{#1}}
\begin{document}
 \DocInput{portuges.dtx}
\end{document}
%</filedriver>
%\fi
%
% \GetFileInfo{portuges.dtx}
%
% \changes{portuges-1.0a}{1991/07/15}{Renamed \file{babel.sty} in
%    \file{babel.com}}
% \changes{portuges-1.1}{1992/02/16}{Brought up-to-date with babel 3.2a}
% \changes{portuges-1.2}{1994/02/26}{Update for \LaTeXe}
% \changes{portuges-1.2d}{1994/06/26}{Removed the use of \cs{filedate}
%    and moved identification after the loading of \file{babel.def}}
% \changes{portuges-1.2g}{1995/06/04}{Enhanced support for brasilian}
% \changes{portuges-1.2j}{1996/07/11}{Replaced \cs{undefined} with
%    \cs{@undefined} and \cs{empty} with \cs{@empty} for consistency
%    with \LaTeX} 
% \changes{portuges-1.2j}{1996/10/10}{Moved the definition of
%    \cs{atcatcode} right to the beginning.}
%
%  \section{The Portuguese language}
%
%    The file \file{\filename}\footnote{The file described in this
%    section has version number \fileversion\ and was last revised on
%    \filedate.  Contributions were made by Jose Pedro Ramalhete
%    (\texttt{JRAMALHE@CERNVM} or
%    \texttt{Jose-Pedro\_Ramalhete@MACMAIL}) and Arnaldo Viegas de
%    Lima \texttt{arnaldo@VNET.IBM.COM}.}  defines all the
%    language-specific macros for the Portuguese language as well as
%    for the Brasilian version of this language.
%
%    For this language the character |"| is made active. In
%    table~\ref{tab:port-quote} an overview is given of its purpose.
%
%    \begin{table}[htb]
%     \centering
%     \begin{tabular}{lp{8cm}}
%       \verb="|= & disable ligature at this position.\\
%        |"-| & an explicit hyphen sign, allowing hyphenation
%               in the rest of the word.\\
%        |""| & like \verb="-=, but producing no hyphen sign (for
%              words that should break at some sign such as
%              ``entrada/salida.''\\
%        |"<| & for French left double quotes (similar to $<<$).\\
%        |">| & for French right double quotes (similar to $>>$).\\
%        |\-| & like the old |\-|, but allowing hyphenation
%               in the rest of the word. \\
%     \end{tabular}
%     \caption{The extra definitions made by \file{portuges.ldf}}
%     \label{tab:port-quote}
%    \end{table}
%
% \StopEventually{}
%
%    The macro |\LdfInit| takes care of preventing that this file is
%    loaded more than once, checking the category code of the
%    \texttt{@} sign, etc.
% \changes{portuges-1.2j}{1996/11/03}{Now use \cs{LdfInit} to perform
%    initial checks} 
%    \begin{macrocode}
%<*code>
\LdfInit\CurrentOption{captions\CurrentOption}
%    \end{macrocode}
%
%    When this file is read as an option, i.e. by the |\usepackage|
%    command, \texttt{portuges} will be an `unknown' language in which
%    case we have to make it known. So we check for the existence of
%    |\l@portuges| to see whether we have to do something here. Since
%    it is possible to load this file with any of the following four
%    options to babel: \Lopt{portuges}, \Lopt{portuguese},
%    \Lopt{brazil} and \Lopt{brazilian} we also allow that the
%    hyphenation patterns are loaded under any of these four names. We
%    just have to find out which one was used.
%
% \changes{portuges-1.0b}{1991/10/29}{Removed use of cs{@ifundefined}}
% \changes{portuges-1.1}{1992/02/16}{Added a warning when no
%    hyphenation patterns were loaded.}
% \changes{portuges-1.2d}{1994/06/26}{Now use \cs{@nopatterns} to
%    produce the warning}
%    \begin{macrocode}
\ifx\l@portuges\@undefined
  \ifx\l@portuguese\@undefined
    \ifx\l@brazil\@undefined
      \ifx\l@brazilian\@undefined
        \@nopatterns{Portuguese}
        \adddialect\l@portuges0
      \else
        \let\l@portuges\l@brazilian
      \fi
    \else
      \let\l@portuges\l@brazil
    \fi
  \else
    \let\l@portuges\l@portuguese
  \fi
\fi
%    \end{macrocode}
%    By now |\l@portuges| is defined. When the language definition
%    file was loaded under a different name we make sure that the
%    hyphenation patterns can be found.
%    \begin{macrocode}
\expandafter\ifx\csname l@\CurrentOption\endcsname\relax
  \expandafter\let\csname l@\CurrentOption\endcsname\l@portuges
\fi
%    \end{macrocode}
%
%    Now we have to decide whether this language definition file was
%    loaded for Portuguese or Brasilian use. This can be done by
%    checking the contents of |\CurrentOption|. When it doesn't
%    contain either `portuges' or `portuguese' we make |\bbl@tempb|
%    empty. 
%    \begin{macrocode}
\def\bbl@tempa{portuguese}
\ifx\CurrentOption\bbl@tempa
  \let\bbl@tempb\@empty
\else
  \def\bbl@tempa{portuges}
  \ifx\CurrentOption\bbl@tempa
    \let\bbl@tempb\@empty
  \else
    \def\bbl@tempb{brazil}
  \fi
\fi
\ifx\bbl@tempb\@empty
%    \end{macrocode}
%
%    The next step consists of defining commands to switch to (and from)
%    the Portuguese language.
%
% \begin{macro}{\captionsportuges}
%    The macro |\captionsportuges| defines all strings used
%    in the four standard documentclasses provided with \LaTeX.
% \changes{portuges-1.1}{1992/02/16}{Added \cs{seename}, \cs{alsoname}
%    and \cs{prefacename}}
% \changes{portuges-1.1}{1993/07/15}{\cs{headpagename} should be
%    \cs{pagename}}
% \changes{portuges-1.2e}{1994/11/09}{Added a few missing
%    translations}
% \changes{portuges-1.2h}{1995/07/04}{Added \cs{proofname} for
%    AMS-\LaTeX}
% \changes{portuges-1.2i}{1995/11/25}{Substituted `Prova' for `Proof'}
%    \begin{macrocode}
  \@namedef{captions\CurrentOption}{%
    \def\prefacename{Pref\'acio}%
    \def\refname{Refer\^encias}%
    \def\abstractname{Resumo}%
    \def\bibname{Bibliografia}%
    \def\chaptername{Cap\'{\i}tulo}%
    \def\appendixname{Ap\^endice}%
%    \end{macrocode}
%    Some discussion took place around the correct translations for
%    `Table of Contents' and `Index'. the translations differ for
%    Portuguese and Brasilian based the following history:
%    \begin{quote}
%      The whole issue is that some books without a real index at the
%      end misused the term `\'Indice' as table of contents. Then,
%      what happens is that some books apeared with `\'Indice' at the
%      begining and a `\'Indice Remissivo' at the end. Remissivo is a
%      redundant word in this case, but was introduced to make up the
%      difference. So in Brasil people started using `Sum\'ario' and
%      `\'Indice Remissivo'. In Portugal this seems not to be very
%      common, therefore we chose `\'Indice' instead of `\'Indice
%      Remissivo'.
%    \end{quote}
%    \begin{macrocode}
    \def\contentsname{Conte\'udo}%
    \def\listfigurename{Lista de Figuras}%
    \def\listtablename{Lista de Tabelas}%
    \def\indexname{\'Indice}%
    \def\figurename{Figura}%
    \def\tablename{Tabela}%
    \def\partname{Parte}%
    \def\enclname{Anexo}%
    \def\ccname{Com c\'opia a}%
    \def\headtoname{Para}%
    \def\pagename{P\'agina}%
    \def\seename{ver}%
    \def\alsoname{ver tamb\'em}%
%    \end{macrocode}
%    An alternate term for `Proof' could be `Prova'.
% \changes{portuges-1.2m}{2000/09/20}{Added \cs{glossaryname}}
% \changes{portuges-1.2p}{2003/05/23}{Substituted `Gloss\'ario' for
%    `Glossary'}
%    \begin{macrocode}
    \def\proofname{Demonstra\c{c}\~ao}%
    \def\glossaryname{Gloss\'ario}%
    }
%    \end{macrocode}
% \end{macro}
%
% \begin{macro}{\dateportuges}
%    The macro |\dateportuges| redefines the command |\today| to
%    produce Portuguese dates.
% \changes{portuges-1.2k}{1997/10/01}{Use \cs{edef} to define
%    \cs{today} to save memory}
% \changes{portuges-1.2k}{1998/03/28}{use \cs{def} instead of
%    \cs{edef}} 
% \changes{portuges-1.2n}{2001/01/27}{Removed spurious space after
%    Dezembro}
%    \begin{macrocode}
  \@namedef{date\CurrentOption}{%
    \def\today{\number\day\space de\space\ifcase\month\or
      Janeiro\or Fevereiro\or Mar\c{c}o\or Abril\or Maio\or Junho\or
      Julho\or Agosto\or Setembro\or Outubro\or Novembro\or Dezembro%
      \fi
      \space de\space\number\year}}
\else
%    \end{macrocode}
% \end{macro}
%
%    For the Brasilian version of these definitions we just add a
%    ``dialect''. 
%    \begin{macrocode}
  \expandafter
    \adddialect\csname l@\CurrentOption\endcsname\l@portuges
%    \end{macrocode}
%
% \begin{macro}{\captionsbrazil}
% \changes{portuges-1.2g}{1995/06/04}{The captions for brasilian and
%    portuguese are different now}
%
%    The ``captions'' are different for both versions of the language,
%    so we define the macro |\captionsbrazil| here.
% \changes{portuges-1.2i}{1995/11/25}{Added \cs{proofname} for
%    AMS-\LaTeX}
% \changes{portuges-1.2m}{2000/09/20}{Added \cs{glossaryname}}
% \changes{portuges-1.2q}{2008/03/18}{Substituted `Gloss\'ario' for
%    `Glossary'}
%    \begin{macrocode}
  \@namedef{captions\CurrentOption}{%
    \def\prefacename{Pref\'acio}%
    \def\refname{Refer\^encias}%
    \def\abstractname{Resumo}%
    \def\bibname{Refer\^encias Bibliogr\'aficas}%
    \def\chaptername{Cap\'{\i}tulo}%
    \def\appendixname{Ap\^endice}%
    \def\contentsname{Sum\'ario}%
    \def\listfigurename{Lista de Figuras}%
    \def\listtablename{Lista de Tabelas}%
    \def\indexname{\'Indice Remissivo}%
    \def\figurename{Figura}%
    \def\tablename{Tabela}%
    \def\partname{Parte}%
    \def\enclname{Anexo}%
    \def\ccname{C\'opia para}%
    \def\headtoname{Para}%
    \def\pagename{P\'agina}%
    \def\seename{veja}%
    \def\alsoname{veja tamb\'em}%
    \def\proofname{Demonstra\c{c}\~ao}%
    \def\glossaryname{Gloss\'ario}%
    }
%    \end{macrocode}
% \end{macro}
%
% \begin{macro}{\datebrazil}
%    The macro |\datebrazil| redefines the command
%    |\today| to produce Brasilian dates, for which the names
%    of the months are not capitalized.
% \changes{portuges-1.2k}{1997/10/01}{Use \cs{edef} to define
%    \cs{today} to save memory}
% \changes{portuges-1.2k}{1998/03/28}{use \cs{def} instead of
%    \cs{edef}} 
% \changes{portuges-1.2n}{2001/01/27}{Removed spurious space after
%    dezembro}
%    \begin{macrocode}
  \@namedef{date\CurrentOption}{%
    \def\today{\number\day\space de\space\ifcase\month\or
      janeiro\or fevereiro\or mar\c{c}o\or abril\or maio\or junho\or
      julho\or agosto\or setembro\or outubro\or novembro\or dezembro%
      \fi
      \space de\space\number\year}}
\fi
%    \end{macrocode}
% \end{macro}
%
%  \begin{macro}{\portugeshyphenmins}
% \changes{portuges-1.2g}{1995/06/04}{Added setting of hyphenmin
%    values}
%    Set correct values for |\lefthyphenmin| and |\righthyphenmin|.
% \changes{portuges-1.2m}{2000/09/22}{Now use \cs{providehyphenmins} to
%    provide a default value}
% \changes{portuges-1.2o}{2001/02/16}{Set \cs{righthyphenmin} to 3 if
%    not provided by the pattern file.}
%    \begin{macrocode}
\providehyphenmins{\CurrentOption}{\tw@\thr@@}
%    \end{macrocode}
%  \end{macro}
%
% \begin{macro}{\extrasportuges}
% \changes{portuges-1.2g}{1995/06/04}{Added the definition of some
%    \texttt{"} shorthands}
% \begin{macro}{\noextrasportuges}
%    The macro |\extrasportuges| will perform all the extra
%    definitions needed for the Portuguese language. The macro
%    |\noextrasportuges| is used to cancel the actions of
%    |\extrasportuges|.
%
%    For Portuguese the \texttt{"} character is made active. This is
%    done once, later on its definition may vary. Other languages in
%    the same document may also use the \texttt{"} character for
%    shorthands; we specify that the portuguese group of shorthands
%    should be used.
%
%    \begin{macrocode}
\initiate@active@char{"}
\@namedef{extras\CurrentOption}{\languageshorthands{portuges}}
\expandafter\addto\csname extras\CurrentOption\endcsname{%
  \bbl@activate{"}}
%    \end{macrocode}
%    Don't forget to turn the shorthands off again.
% \changes{portuges-1.2m}{1999/12/17}{Deactivate shorthands ouside of
%    Basque}
%    \begin{macrocode}
\addto\noextrasportuges{\bbl@deactivate{"}}
%    \end{macrocode}
%    First we define access to the guillemets for quotations,
% \changes{portuges-1.2k}{1997/04/03}{Removed empty groups after
%    guillemot characters}
%    \begin{macrocode}
\declare@shorthand{portuges}{"<}{%
  \textormath{\guillemotleft}{\mbox{\guillemotleft}}}
\declare@shorthand{portuges}{">}{%
  \textormath{\guillemotright}{\mbox{\guillemotright}}}
%    \end{macrocode}
%    then we define two shorthands to be able to specify hyphenation
%    breakpoints that behave a little different from |\-|.
%    \begin{macrocode}
\declare@shorthand{portuges}{"-}{\nobreak-\bbl@allowhyphens}
\declare@shorthand{portuges}{""}{\hskip\z@skip}
%    \end{macrocode}
%    And we want to have a shorthand for disabling a ligature.
%    \begin{macrocode}
\declare@shorthand{portuges}{"|}{%
  \textormath{\discretionary{-}{}{\kern.03em}}{}}
%    \end{macrocode}
% \end{macro}
% \end{macro}
%
%  \begin{macro}{\-}
%
%    All that is left now is the redefinition of |\-|. The new version
%    of |\-| should indicate an extra hyphenation position, while
%    allowing other hyphenation positions to be generated
%    automatically. The standard behaviour of \TeX\ in this respect is
%    very unfortunate for languages such as Dutch and German, where
%    long compound words are quite normal and all one needs is a means
%    to indicate an extra hyphenation position on top of the ones that
%    \TeX\ can generate from the hyphenation patterns.
%    \begin{macrocode}
\expandafter\addto\csname extras\CurrentOption\endcsname{%
  \babel@save\-}
\expandafter\addto\csname extras\CurrentOption\endcsname{%
  \def\-{\allowhyphens\discretionary{-}{}{}\allowhyphens}}
%    \end{macrocode}
%  \end{macro}
%
%  \begin{macro}{\ord}
% \changes{portuges-1.2g}{1995/06/04}{Added macro}
%  \begin{macro}{\ro}
% \changes{portuges-1.2g}{1995/06/04}{Added macro}
%  \begin{macro}{\orda}
% \changes{portuges-1.2g}{1995/06/04}{Added macro}
%  \begin{macro}{\ra}
% \changes{portuges-1.2g}{1995/06/04}{Added macro}
%    We also provide an easy way to typeset ordinals, both in the male
%    (|\ord| or |\ro|) and the female (|orda| or |\ra|) form.
%    \begin{macrocode}
\def\ord{$^{\rm o}$}
\def\orda{$^{\rm a}$}
\let\ro\ord\let\ra\orda
%    \end{macrocode}
%  \end{macro}
%  \end{macro}
%  \end{macro}
%  \end{macro}
%
%    The macro |\ldf@finish| takes care of looking for a
%    configuration file, setting the main language to be switched on
%    at |\begin{document}| and resetting the category code of
%    \texttt{@} to its original value.
% \changes{portuges-1.2j}{1996/11/03}{ow use \cs{ldf@finish} to wrap
%    up} 
%    \begin{macrocode}
\ldf@finish\CurrentOption
%</code>
%    \end{macrocode}
%
% \Finale
%%
%% \CharacterTable
%%  {Upper-case    \A\B\C\D\E\F\G\H\I\J\K\L\M\N\O\P\Q\R\S\T\U\V\W\X\Y\Z
%%   Lower-case    \a\b\c\d\e\f\g\h\i\j\k\l\m\n\o\p\q\r\s\t\u\v\w\x\y\z
%%   Digits        \0\1\2\3\4\5\6\7\8\9
%%   Exclamation   \!     Double quote  \"     Hash (number) \#
%%   Dollar        \$     Percent       \%     Ampersand     \&
%%   Acute accent  \'     Left paren    \(     Right paren   \)
%%   Asterisk      \*     Plus          \+     Comma         \,
%%   Minus         \-     Point         \.     Solidus       \/
%%   Colon         \:     Semicolon     \;     Less than     \<
%%   Equals        \=     Greater than  \>     Question mark \?
%%   Commercial at \@     Left bracket  \[     Backslash     \\
%%   Right bracket \]     Circumflex    \^     Underscore    \_
%%   Grave accent  \`     Left brace    \{     Vertical bar  \|
%%   Right brace   \}     Tilde         \~}
%%
\endinput
}
%    \end{macrocode}
% \changes{babel~3.5b}{1995/05/25}{Added \Lopt{brazilian} as
%    alternative for \Lopt{brazil}}
%    \begin{macrocode}
\DeclareOption{brazilian}{% \iffalse meta-comment
%
% Copyright 1989-2008 Johannes L. Braams and any individual authors
% listed elsewhere in this file.  All rights reserved.
% 
% This file is part of the Babel system.
% --------------------------------------
% 
% It may be distributed and/or modified under the
% conditions of the LaTeX Project Public License, either version 1.3
% of this license or (at your option) any later version.
% The latest version of this license is in
%   http://www.latex-project.org/lppl.txt
% and version 1.3 or later is part of all distributions of LaTeX
% version 2003/12/01 or later.
% 
% This work has the LPPL maintenance status "maintained".
% 
% The Current Maintainer of this work is Johannes Braams.
% 
% The list of all files belonging to the Babel system is
% given in the file `manifest.bbl. See also `legal.bbl' for additional
% information.
% 
% The list of derived (unpacked) files belonging to the distribution
% and covered by LPPL is defined by the unpacking scripts (with
% extension .ins) which are part of the distribution.
% \fi
% \CheckSum{320}
% \iffalse
%    Tell the \LaTeX\ system who we are and write an entry on the
%    transcript.
%<*dtx>
\ProvidesFile{portuges.dtx}
%</dtx>
%<code>\ProvidesLanguage{portuges}
%\fi
%\ProvidesFile{portuges.dtx}
        [2008/03/18 v1.2q Portuguese support from the babel system]
%\iffalse
%% File `portuges.dtx'
%% Babel package for LaTeX version 2e
%% Copyright (C) 1989 - 2008
%%           by Johannes Braams, TeXniek
%
%% Portuguese Language Definition File
%% Copyright (C) 1989 - 2008
%%           by Johannes Braams, TeXniek
%
%% Please report errors to: J.L. Braams
%%                          babel at braams.cistron.nl
%
%    This file is part of the babel system, it provides the source
%    code for the Portuguese language definition file.  The Portuguese
%    words were contributed by Jose Pedro Ramalhete, (JRAMALHE@CERNVM
%    or Jose-Pedro_Ramalhete@MACMAIL).
%
%    Arnaldo Viegas de Lima <arnaldo@VNET.IBM.COM> contributed
%    brasilian translations and suggestions for enhancements.
%<*filedriver>
\documentclass{ltxdoc}
\newcommand*\TeXhax{\TeX hax}
\newcommand*\babel{\textsf{babel}}
\newcommand*\langvar{$\langle \it lang \rangle$}
\newcommand*\note[1]{}
\newcommand*\Lopt[1]{\textsf{#1}}
\newcommand*\file[1]{\texttt{#1}}
\begin{document}
 \DocInput{portuges.dtx}
\end{document}
%</filedriver>
%\fi
%
% \GetFileInfo{portuges.dtx}
%
% \changes{portuges-1.0a}{1991/07/15}{Renamed \file{babel.sty} in
%    \file{babel.com}}
% \changes{portuges-1.1}{1992/02/16}{Brought up-to-date with babel 3.2a}
% \changes{portuges-1.2}{1994/02/26}{Update for \LaTeXe}
% \changes{portuges-1.2d}{1994/06/26}{Removed the use of \cs{filedate}
%    and moved identification after the loading of \file{babel.def}}
% \changes{portuges-1.2g}{1995/06/04}{Enhanced support for brasilian}
% \changes{portuges-1.2j}{1996/07/11}{Replaced \cs{undefined} with
%    \cs{@undefined} and \cs{empty} with \cs{@empty} for consistency
%    with \LaTeX} 
% \changes{portuges-1.2j}{1996/10/10}{Moved the definition of
%    \cs{atcatcode} right to the beginning.}
%
%  \section{The Portuguese language}
%
%    The file \file{\filename}\footnote{The file described in this
%    section has version number \fileversion\ and was last revised on
%    \filedate.  Contributions were made by Jose Pedro Ramalhete
%    (\texttt{JRAMALHE@CERNVM} or
%    \texttt{Jose-Pedro\_Ramalhete@MACMAIL}) and Arnaldo Viegas de
%    Lima \texttt{arnaldo@VNET.IBM.COM}.}  defines all the
%    language-specific macros for the Portuguese language as well as
%    for the Brasilian version of this language.
%
%    For this language the character |"| is made active. In
%    table~\ref{tab:port-quote} an overview is given of its purpose.
%
%    \begin{table}[htb]
%     \centering
%     \begin{tabular}{lp{8cm}}
%       \verb="|= & disable ligature at this position.\\
%        |"-| & an explicit hyphen sign, allowing hyphenation
%               in the rest of the word.\\
%        |""| & like \verb="-=, but producing no hyphen sign (for
%              words that should break at some sign such as
%              ``entrada/salida.''\\
%        |"<| & for French left double quotes (similar to $<<$).\\
%        |">| & for French right double quotes (similar to $>>$).\\
%        |\-| & like the old |\-|, but allowing hyphenation
%               in the rest of the word. \\
%     \end{tabular}
%     \caption{The extra definitions made by \file{portuges.ldf}}
%     \label{tab:port-quote}
%    \end{table}
%
% \StopEventually{}
%
%    The macro |\LdfInit| takes care of preventing that this file is
%    loaded more than once, checking the category code of the
%    \texttt{@} sign, etc.
% \changes{portuges-1.2j}{1996/11/03}{Now use \cs{LdfInit} to perform
%    initial checks} 
%    \begin{macrocode}
%<*code>
\LdfInit\CurrentOption{captions\CurrentOption}
%    \end{macrocode}
%
%    When this file is read as an option, i.e. by the |\usepackage|
%    command, \texttt{portuges} will be an `unknown' language in which
%    case we have to make it known. So we check for the existence of
%    |\l@portuges| to see whether we have to do something here. Since
%    it is possible to load this file with any of the following four
%    options to babel: \Lopt{portuges}, \Lopt{portuguese},
%    \Lopt{brazil} and \Lopt{brazilian} we also allow that the
%    hyphenation patterns are loaded under any of these four names. We
%    just have to find out which one was used.
%
% \changes{portuges-1.0b}{1991/10/29}{Removed use of cs{@ifundefined}}
% \changes{portuges-1.1}{1992/02/16}{Added a warning when no
%    hyphenation patterns were loaded.}
% \changes{portuges-1.2d}{1994/06/26}{Now use \cs{@nopatterns} to
%    produce the warning}
%    \begin{macrocode}
\ifx\l@portuges\@undefined
  \ifx\l@portuguese\@undefined
    \ifx\l@brazil\@undefined
      \ifx\l@brazilian\@undefined
        \@nopatterns{Portuguese}
        \adddialect\l@portuges0
      \else
        \let\l@portuges\l@brazilian
      \fi
    \else
      \let\l@portuges\l@brazil
    \fi
  \else
    \let\l@portuges\l@portuguese
  \fi
\fi
%    \end{macrocode}
%    By now |\l@portuges| is defined. When the language definition
%    file was loaded under a different name we make sure that the
%    hyphenation patterns can be found.
%    \begin{macrocode}
\expandafter\ifx\csname l@\CurrentOption\endcsname\relax
  \expandafter\let\csname l@\CurrentOption\endcsname\l@portuges
\fi
%    \end{macrocode}
%
%    Now we have to decide whether this language definition file was
%    loaded for Portuguese or Brasilian use. This can be done by
%    checking the contents of |\CurrentOption|. When it doesn't
%    contain either `portuges' or `portuguese' we make |\bbl@tempb|
%    empty. 
%    \begin{macrocode}
\def\bbl@tempa{portuguese}
\ifx\CurrentOption\bbl@tempa
  \let\bbl@tempb\@empty
\else
  \def\bbl@tempa{portuges}
  \ifx\CurrentOption\bbl@tempa
    \let\bbl@tempb\@empty
  \else
    \def\bbl@tempb{brazil}
  \fi
\fi
\ifx\bbl@tempb\@empty
%    \end{macrocode}
%
%    The next step consists of defining commands to switch to (and from)
%    the Portuguese language.
%
% \begin{macro}{\captionsportuges}
%    The macro |\captionsportuges| defines all strings used
%    in the four standard documentclasses provided with \LaTeX.
% \changes{portuges-1.1}{1992/02/16}{Added \cs{seename}, \cs{alsoname}
%    and \cs{prefacename}}
% \changes{portuges-1.1}{1993/07/15}{\cs{headpagename} should be
%    \cs{pagename}}
% \changes{portuges-1.2e}{1994/11/09}{Added a few missing
%    translations}
% \changes{portuges-1.2h}{1995/07/04}{Added \cs{proofname} for
%    AMS-\LaTeX}
% \changes{portuges-1.2i}{1995/11/25}{Substituted `Prova' for `Proof'}
%    \begin{macrocode}
  \@namedef{captions\CurrentOption}{%
    \def\prefacename{Pref\'acio}%
    \def\refname{Refer\^encias}%
    \def\abstractname{Resumo}%
    \def\bibname{Bibliografia}%
    \def\chaptername{Cap\'{\i}tulo}%
    \def\appendixname{Ap\^endice}%
%    \end{macrocode}
%    Some discussion took place around the correct translations for
%    `Table of Contents' and `Index'. the translations differ for
%    Portuguese and Brasilian based the following history:
%    \begin{quote}
%      The whole issue is that some books without a real index at the
%      end misused the term `\'Indice' as table of contents. Then,
%      what happens is that some books apeared with `\'Indice' at the
%      begining and a `\'Indice Remissivo' at the end. Remissivo is a
%      redundant word in this case, but was introduced to make up the
%      difference. So in Brasil people started using `Sum\'ario' and
%      `\'Indice Remissivo'. In Portugal this seems not to be very
%      common, therefore we chose `\'Indice' instead of `\'Indice
%      Remissivo'.
%    \end{quote}
%    \begin{macrocode}
    \def\contentsname{Conte\'udo}%
    \def\listfigurename{Lista de Figuras}%
    \def\listtablename{Lista de Tabelas}%
    \def\indexname{\'Indice}%
    \def\figurename{Figura}%
    \def\tablename{Tabela}%
    \def\partname{Parte}%
    \def\enclname{Anexo}%
    \def\ccname{Com c\'opia a}%
    \def\headtoname{Para}%
    \def\pagename{P\'agina}%
    \def\seename{ver}%
    \def\alsoname{ver tamb\'em}%
%    \end{macrocode}
%    An alternate term for `Proof' could be `Prova'.
% \changes{portuges-1.2m}{2000/09/20}{Added \cs{glossaryname}}
% \changes{portuges-1.2p}{2003/05/23}{Substituted `Gloss\'ario' for
%    `Glossary'}
%    \begin{macrocode}
    \def\proofname{Demonstra\c{c}\~ao}%
    \def\glossaryname{Gloss\'ario}%
    }
%    \end{macrocode}
% \end{macro}
%
% \begin{macro}{\dateportuges}
%    The macro |\dateportuges| redefines the command |\today| to
%    produce Portuguese dates.
% \changes{portuges-1.2k}{1997/10/01}{Use \cs{edef} to define
%    \cs{today} to save memory}
% \changes{portuges-1.2k}{1998/03/28}{use \cs{def} instead of
%    \cs{edef}} 
% \changes{portuges-1.2n}{2001/01/27}{Removed spurious space after
%    Dezembro}
%    \begin{macrocode}
  \@namedef{date\CurrentOption}{%
    \def\today{\number\day\space de\space\ifcase\month\or
      Janeiro\or Fevereiro\or Mar\c{c}o\or Abril\or Maio\or Junho\or
      Julho\or Agosto\or Setembro\or Outubro\or Novembro\or Dezembro%
      \fi
      \space de\space\number\year}}
\else
%    \end{macrocode}
% \end{macro}
%
%    For the Brasilian version of these definitions we just add a
%    ``dialect''. 
%    \begin{macrocode}
  \expandafter
    \adddialect\csname l@\CurrentOption\endcsname\l@portuges
%    \end{macrocode}
%
% \begin{macro}{\captionsbrazil}
% \changes{portuges-1.2g}{1995/06/04}{The captions for brasilian and
%    portuguese are different now}
%
%    The ``captions'' are different for both versions of the language,
%    so we define the macro |\captionsbrazil| here.
% \changes{portuges-1.2i}{1995/11/25}{Added \cs{proofname} for
%    AMS-\LaTeX}
% \changes{portuges-1.2m}{2000/09/20}{Added \cs{glossaryname}}
% \changes{portuges-1.2q}{2008/03/18}{Substituted `Gloss\'ario' for
%    `Glossary'}
%    \begin{macrocode}
  \@namedef{captions\CurrentOption}{%
    \def\prefacename{Pref\'acio}%
    \def\refname{Refer\^encias}%
    \def\abstractname{Resumo}%
    \def\bibname{Refer\^encias Bibliogr\'aficas}%
    \def\chaptername{Cap\'{\i}tulo}%
    \def\appendixname{Ap\^endice}%
    \def\contentsname{Sum\'ario}%
    \def\listfigurename{Lista de Figuras}%
    \def\listtablename{Lista de Tabelas}%
    \def\indexname{\'Indice Remissivo}%
    \def\figurename{Figura}%
    \def\tablename{Tabela}%
    \def\partname{Parte}%
    \def\enclname{Anexo}%
    \def\ccname{C\'opia para}%
    \def\headtoname{Para}%
    \def\pagename{P\'agina}%
    \def\seename{veja}%
    \def\alsoname{veja tamb\'em}%
    \def\proofname{Demonstra\c{c}\~ao}%
    \def\glossaryname{Gloss\'ario}%
    }
%    \end{macrocode}
% \end{macro}
%
% \begin{macro}{\datebrazil}
%    The macro |\datebrazil| redefines the command
%    |\today| to produce Brasilian dates, for which the names
%    of the months are not capitalized.
% \changes{portuges-1.2k}{1997/10/01}{Use \cs{edef} to define
%    \cs{today} to save memory}
% \changes{portuges-1.2k}{1998/03/28}{use \cs{def} instead of
%    \cs{edef}} 
% \changes{portuges-1.2n}{2001/01/27}{Removed spurious space after
%    dezembro}
%    \begin{macrocode}
  \@namedef{date\CurrentOption}{%
    \def\today{\number\day\space de\space\ifcase\month\or
      janeiro\or fevereiro\or mar\c{c}o\or abril\or maio\or junho\or
      julho\or agosto\or setembro\or outubro\or novembro\or dezembro%
      \fi
      \space de\space\number\year}}
\fi
%    \end{macrocode}
% \end{macro}
%
%  \begin{macro}{\portugeshyphenmins}
% \changes{portuges-1.2g}{1995/06/04}{Added setting of hyphenmin
%    values}
%    Set correct values for |\lefthyphenmin| and |\righthyphenmin|.
% \changes{portuges-1.2m}{2000/09/22}{Now use \cs{providehyphenmins} to
%    provide a default value}
% \changes{portuges-1.2o}{2001/02/16}{Set \cs{righthyphenmin} to 3 if
%    not provided by the pattern file.}
%    \begin{macrocode}
\providehyphenmins{\CurrentOption}{\tw@\thr@@}
%    \end{macrocode}
%  \end{macro}
%
% \begin{macro}{\extrasportuges}
% \changes{portuges-1.2g}{1995/06/04}{Added the definition of some
%    \texttt{"} shorthands}
% \begin{macro}{\noextrasportuges}
%    The macro |\extrasportuges| will perform all the extra
%    definitions needed for the Portuguese language. The macro
%    |\noextrasportuges| is used to cancel the actions of
%    |\extrasportuges|.
%
%    For Portuguese the \texttt{"} character is made active. This is
%    done once, later on its definition may vary. Other languages in
%    the same document may also use the \texttt{"} character for
%    shorthands; we specify that the portuguese group of shorthands
%    should be used.
%
%    \begin{macrocode}
\initiate@active@char{"}
\@namedef{extras\CurrentOption}{\languageshorthands{portuges}}
\expandafter\addto\csname extras\CurrentOption\endcsname{%
  \bbl@activate{"}}
%    \end{macrocode}
%    Don't forget to turn the shorthands off again.
% \changes{portuges-1.2m}{1999/12/17}{Deactivate shorthands ouside of
%    Basque}
%    \begin{macrocode}
\addto\noextrasportuges{\bbl@deactivate{"}}
%    \end{macrocode}
%    First we define access to the guillemets for quotations,
% \changes{portuges-1.2k}{1997/04/03}{Removed empty groups after
%    guillemot characters}
%    \begin{macrocode}
\declare@shorthand{portuges}{"<}{%
  \textormath{\guillemotleft}{\mbox{\guillemotleft}}}
\declare@shorthand{portuges}{">}{%
  \textormath{\guillemotright}{\mbox{\guillemotright}}}
%    \end{macrocode}
%    then we define two shorthands to be able to specify hyphenation
%    breakpoints that behave a little different from |\-|.
%    \begin{macrocode}
\declare@shorthand{portuges}{"-}{\nobreak-\bbl@allowhyphens}
\declare@shorthand{portuges}{""}{\hskip\z@skip}
%    \end{macrocode}
%    And we want to have a shorthand for disabling a ligature.
%    \begin{macrocode}
\declare@shorthand{portuges}{"|}{%
  \textormath{\discretionary{-}{}{\kern.03em}}{}}
%    \end{macrocode}
% \end{macro}
% \end{macro}
%
%  \begin{macro}{\-}
%
%    All that is left now is the redefinition of |\-|. The new version
%    of |\-| should indicate an extra hyphenation position, while
%    allowing other hyphenation positions to be generated
%    automatically. The standard behaviour of \TeX\ in this respect is
%    very unfortunate for languages such as Dutch and German, where
%    long compound words are quite normal and all one needs is a means
%    to indicate an extra hyphenation position on top of the ones that
%    \TeX\ can generate from the hyphenation patterns.
%    \begin{macrocode}
\expandafter\addto\csname extras\CurrentOption\endcsname{%
  \babel@save\-}
\expandafter\addto\csname extras\CurrentOption\endcsname{%
  \def\-{\allowhyphens\discretionary{-}{}{}\allowhyphens}}
%    \end{macrocode}
%  \end{macro}
%
%  \begin{macro}{\ord}
% \changes{portuges-1.2g}{1995/06/04}{Added macro}
%  \begin{macro}{\ro}
% \changes{portuges-1.2g}{1995/06/04}{Added macro}
%  \begin{macro}{\orda}
% \changes{portuges-1.2g}{1995/06/04}{Added macro}
%  \begin{macro}{\ra}
% \changes{portuges-1.2g}{1995/06/04}{Added macro}
%    We also provide an easy way to typeset ordinals, both in the male
%    (|\ord| or |\ro|) and the female (|orda| or |\ra|) form.
%    \begin{macrocode}
\def\ord{$^{\rm o}$}
\def\orda{$^{\rm a}$}
\let\ro\ord\let\ra\orda
%    \end{macrocode}
%  \end{macro}
%  \end{macro}
%  \end{macro}
%  \end{macro}
%
%    The macro |\ldf@finish| takes care of looking for a
%    configuration file, setting the main language to be switched on
%    at |\begin{document}| and resetting the category code of
%    \texttt{@} to its original value.
% \changes{portuges-1.2j}{1996/11/03}{ow use \cs{ldf@finish} to wrap
%    up} 
%    \begin{macrocode}
\ldf@finish\CurrentOption
%</code>
%    \end{macrocode}
%
% \Finale
%%
%% \CharacterTable
%%  {Upper-case    \A\B\C\D\E\F\G\H\I\J\K\L\M\N\O\P\Q\R\S\T\U\V\W\X\Y\Z
%%   Lower-case    \a\b\c\d\e\f\g\h\i\j\k\l\m\n\o\p\q\r\s\t\u\v\w\x\y\z
%%   Digits        \0\1\2\3\4\5\6\7\8\9
%%   Exclamation   \!     Double quote  \"     Hash (number) \#
%%   Dollar        \$     Percent       \%     Ampersand     \&
%%   Acute accent  \'     Left paren    \(     Right paren   \)
%%   Asterisk      \*     Plus          \+     Comma         \,
%%   Minus         \-     Point         \.     Solidus       \/
%%   Colon         \:     Semicolon     \;     Less than     \<
%%   Equals        \=     Greater than  \>     Question mark \?
%%   Commercial at \@     Left bracket  \[     Backslash     \\
%%   Right bracket \]     Circumflex    \^     Underscore    \_
%%   Grave accent  \`     Left brace    \{     Vertical bar  \|
%%   Right brace   \}     Tilde         \~}
%%
\endinput
}
\DeclareOption{breton}{% \iffalse meta-comment
%
% Copyright 1989-2005 Johannes L. Braams and any individual authors
% listed elsewhere in this file.  All rights reserved.
% 
% This file is part of the Babel system.
% --------------------------------------
% 
% It may be distributed and/or modified under the
% conditions of the LaTeX Project Public License, either version 1.3
% of this license or (at your option) any later version.
% The latest version of this license is in
%   http://www.latex-project.org/lppl.txt
% and version 1.3 or later is part of all distributions of LaTeX
% version 2003/12/01 or later.
% 
% This work has the LPPL maintenance status "maintained".
% 
% The Current Maintainer of this work is Johannes Braams.
% 
% The list of all files belonging to the Babel system is
% given in the file `manifest.bbl. See also `legal.bbl' for additional
% information.
% 
% The list of derived (unpacked) files belonging to the distribution
% and covered by LPPL is defined by the unpacking scripts (with
% extension .ins) which are part of the distribution.
% \fi
% \CheckSum{263}
%
% \iffalse
%    Tell the \LaTeX\ system who we are and write an entry on the
%    transcript.
%<*dtx>
\ProvidesFile{breton.dtx}
%</dtx>
%<code>\ProvidesLanguage{breton}
%\fi
%\ProvidesFile{breton.dtx}
        [2005/03/29 v1.0h Breton support from the babel system]
%\iffalse
%% File `breton.dtx'
%% Babel package for LaTeX version 2e
%% Copyright (C) 1989 - 2005
%%           by Johannes Braams, TeXniek
%
%% Breton Language Definition File
%% Copyright (C) 1994 - 2005
%%           by Christian Rolland
%              Universite de Bretagne occidentale
%              Departement d'informatique
%              6, avenue Le Gorgeu
%              BP 452
%              29275 Brest Cedex -- FRANCE
%              Christian.Rolland at univ-brest.fr (Internet)
%
%% Please report errors to: J.L. Braams
%%                          babel at braams.cistron.nl
%
%    This file is part of the babel system, it provides the source
%    code for the Breton language definition file. It is based on the
%    language definition file for french, version 4.5c
%<*filedriver>
\documentclass{ltxdoc}
\newcommand*\TeXhax{\TeX hax}
\newcommand*\babel{\textsf{babel}}
\newcommand*\langvar{$\langle \it lang \rangle$}
\newcommand*\note[1]{}
\newcommand*\Lopt[1]{\textsf{#1}}
\newcommand*\file[1]{\texttt{#1}}
\begin{document}
 \DocInput{breton.dtx}
\end{document}
%</filedriver>
%\fi
% \GetFileInfo{breton.dtx}
% \changes{breton-1.0e}{1996/10/10}{Replaced \cs{undefined} with
%    \cs{@undefined} and \cs{empty} with \cs{@empty} for consistency
%    with \LaTeX, moved the definition of \cs{atcatcode} right to the
%    beginning.} 
%
% \changes{breton-1.0}{1994/09/21}{First release}
%
%  \section{The Breton language}
%
%    The file \file{\filename}\footnote{The file described in this
%    section has version number \fileversion\ and was last revised on
%    \filedate.} defines all the language-specific macros for the Breton
%    language. 
%
%    There are not really typographic rules for the Breton
%    language. It is a local language (it's one of the celtic
%    languages) which is spoken in Brittany (West of France). So we
%    have a synthesis between french typographic rules and english
%    typographic rules. The characters \texttt{:}, \texttt{;},
%    \texttt{!} and \texttt{?} are made active in order to get a
%    whitespace automatically before these characters.
%
% \StopEventually{}
%
%    The macro |\LdfInit| takes care of preventing that this file is
%    loaded more than once, checking the category code of the
%    \texttt{@} sign, etc.
% \changes{breton-1.0e}{1996/11/02}{Now use \cs{LdfInit} to perform
%    initial checks} 
%    \begin{macrocode}
%<*code>
\LdfInit{breton}\captionsbreton
%    \end{macrocode}
%
%    When this file is read as an option, i.e. by the |\usepackage|
%    command, \texttt{breton} will be an `unknown' language in which
%    case we have to make it known.  So we check for the existence of
%    |\l@breton| to see whether we have to do something here.
%
%    \begin{macrocode}
\ifx\l@breton\@undefined
    \@nopatterns{Breton}
    \adddialect\l@breton0\fi
%    \end{macrocode}
%    The next step consists of defining commands to switch to the
%    English language. The reason for this is that a user might want
%    to switch back and forth between languages.
%
% \begin{macro}{\captionsbreton}
%    The macro |\captionsbreton| defines all strings used in the
%    four standard document classes provided with \LaTeX.
% \changes{breton-1.0b}{1995/07/04}{Added \cs{proofname} for
%    AMS-\LaTeX}
% \changes{breton-1.0h}{2000/09/19}{Added \cs{glossaryname}}
%    \begin{macrocode}
\addto\captionsbreton{%
  \def\prefacename{Rakskrid}%
  \def\refname{Daveenno\`u}%
  \def\abstractname{Dvierra\~n}%
  \def\bibname{Lennadurezh}%
  \def\chaptername{Pennad}%
  \def\appendixname{Stagadenn}%
  \def\contentsname{Taolenn}%
  \def\listfigurename{Listenn ar Figurenno\`u}%
  \def\listtablename{Listenn an taolenno\`u}%
  \def\indexname{Meneger}%
  \def\figurename{Figurenn}%
  \def\tablename{Taolenn}%
  \def\partname{Lodenn}%
  \def\enclname{Diello\`u kevret}%
  \def\ccname{Eilskrid da}%
  \def\headtoname{evit}
  \def\pagename{Pajenn}%
  \def\seename{Gwelout}%
  \def\alsoname{Gwelout ivez}%
  \def\proofname{Proof}%  <-- needs translation
  \def\glossaryname{Glossary}% <-- Needs translation
}
%    \end{macrocode}
% \end{macro}
%
% \begin{macro}{\datebreton}
%    The macro |\datebreton| redefines the command
%    |\today| to produce Breton dates.
% \changes{breton-1.0f}{1997/10/01}{Use \cs{edef} to define \cs{today}
%    to save memory}
% \changes{breton-1.0f}{1998/03/28}{use \cs{def} instead of \cs{edef}}
%    \begin{macrocode}
\def\datebreton{%
  \def\today{\ifnum\day=1\relax 1\/$^{\rm a\tilde{n}}$\else
    \number\day\fi \space a\space viz\space\ifcase\month\or
    Genver\or C'hwevrer\or Meurzh\or Ebrel\or Mae\or Mezheven\or
    Gouere\or Eost\or Gwengolo\or Here\or Du\or Kerzu\fi
    \space\number\year}}
%    \end{macrocode}
% \end{macro}
%
% \begin{macro}{\extrasbreton}
% \begin{macro}{\noextrasbreton}
%    The macro |\extrasbreton| will perform all the extra
%    definitions needed for the Breton language. The macro
%    |\noextrasbreton| is used to cancel the actions of
%    |\extrasbreton|.
%
%    The category code of the characters \texttt{:}, \texttt{;},
%    \texttt{!} and \texttt{?} is made |\active| to insert a little
%    white space.
% \changes{breton-1.0b}{1995/03/07}{Use the new mechanism for dealing
%    with active chars}
%    \begin{macrocode}
\initiate@active@char{:}
\initiate@active@char{;}
\initiate@active@char{!}
\initiate@active@char{?}
%    \end{macrocode}
%    We specify that the breton group of shorthands should be used.
%    \begin{macrocode}
\addto\extrasbreton{\languageshorthands{breton}}
%    \end{macrocode}
%    These characters are `turned on' once, later their definition may
%    vary. 
%    \begin{macrocode}
\addto\extrasbreton{%
  \bbl@activate{:}\bbl@activate{;}%
  \bbl@activate{!}\bbl@activate{?}}
%    \end{macrocode}
%    Don't forget to turn the shorthands off again.
% \changes{breton-1.0g}{1999/12/16}{Deactivate shorthands ouside of Breton}
%    \begin{macrocode}
\addto\noextrasbreton{%
 \bbl@deactivate{:}\bbl@deactivate{;}%
 \bbl@deactivate{!}\bbl@deactivate{?}}
%    \end{macrocode}
%
%    The last thing |\extrasbreton| needs to do is to make sure that
%    |\frenchspacing| is in effect.  If this is not the case the
%    execution of |\noextrasbreton| will switch it of again.
%    \begin{macrocode}
\addto\extrasbreton{\bbl@frenchspacing}
\addto\noextrasbreton{\bbl@nonfrenchspacing}
%    \end{macrocode}
% \end{macro}
% \end{macro}
%
% \begin{macro}{\breton@sh@;@}
%    We have to reduce the amount of white space before \texttt{;},
%    \texttt{:} and \texttt{!} when the user types a space in front of
%    these characters. This should only happen outside mathmode, hence
%    the test with |\ifmmode|.
%
%    \begin{macrocode}
\declare@shorthand{breton}{;}{%
    \ifmmode
      \string;\space
    \else\relax
%    \end{macrocode}
%    In horizontal mode we check for the presence of a `space' and
%    replace it by a |\thinspace|.
%    \begin{macrocode}
      \ifhmode
        \ifdim\lastskip>\z@
          \unskip\penalty\@M\thinspace
        \fi
      \fi
      \string;\space
    \fi}%
%    \end{macrocode}
% \end{macro}
%
% \begin{macro}{\breton@sh@:@}
% \begin{macro}{\breton@sh@!@}
%    Because these definitions are very similar only one is displayed
%    in a way that the definition can be easily checked.
%    \begin{macrocode}
\declare@shorthand{breton}{:}{%
  \ifmmode\string:\space
  \else\relax
    \ifhmode
      \ifdim\lastskip>\z@\unskip\penalty\@M\thinspace\fi
    \fi
    \string:\space
  \fi}
\declare@shorthand{breton}{!}{%
  \ifmmode\string!\space
  \else\relax
    \ifhmode
      \ifdim\lastskip>\z@\unskip\penalty\@M\thinspace\fi
    \fi
    \string!\space
  \fi}
%    \end{macrocode}
% \end{macro}
% \end{macro}
%
% \begin{macro}{\breton@sh@?@}
%    For the question mark something different has to be done. In this
%    case the amount of white space that replaces the space character
%    depends on the dimensions of the font.
%    \begin{macrocode}
\declare@shorthand{breton}{?}{%
  \ifmmode
    \string?\space
  \else\relax
    \ifhmode
      \ifdim\lastskip>\z@
        \unskip
        \kern\fontdimen2\font
        \kern-1.4\fontdimen3\font
      \fi
    \fi
    \string?\space
  \fi}
%    \end{macrocode}
% \end{macro}
%
%    All that is left to do now is provide the breton user with some
%    extra utilities.
%
%    Some definitions for special characters.
%    \begin{macrocode}
\DeclareTextSymbol{\at}{OT1}{64}
\DeclareTextSymbol{\at}{T1}{64}
\DeclareTextSymbolDefault{\at}{OT1}
\DeclareTextSymbol{\boi}{OT1}{92}
\DeclareTextSymbol{\boi}{T1}{16}
\DeclareTextSymbolDefault{\boi}{OT1}
\DeclareTextSymbol{\circonflexe}{OT1}{94}
\DeclareTextSymbol{\circonflexe}{T1}{2}
\DeclareTextSymbolDefault{\circonflexe}{OT1}
\DeclareTextSymbol{\tild}{OT1}{126}
\DeclareTextSymbol{\tild}{T1}{3}
\DeclareTextSymbolDefault{\tild}{OT1}
\DeclareTextSymbol{\degre}{OT1}{23}
\DeclareTextSymbol{\degre}{T1}{6}
\DeclareTextSymbolDefault{\degre}{OT1}
%    \end{macrocode}
%
%    The following macros are used in the redefinition of |\^| and
%    |\"| to handle the letter i.
% \changes{breton-1.0c}{1995/07/07}{Postpone the 
%    \cs{DeclareTextCompositeCommand}s untill \cs{AtBeginDocument}}
%
%    \begin{macrocode}
\AtBeginDocument{%
  \DeclareTextCompositeCommand{\^}{OT1}{i}{\^\i}
  \DeclareTextCompositeCommand{\"}{OT1}{i}{\"\i}} 
%    \end{macrocode}
%
%    And some more macros for numbering.
%    \begin{macrocode}
\def\kentan{1\/${}^{\rm a\tilde{n}}$}
\def\eil{2\/${}^{\rm l}$}
\def\re{\/${}^{\rm re}$}
\def\trede{3\re}
\def\pevare{4\re}
\def\vet{\/${}^{\rm vet}$}
\def\pempvet{5\vet}
%    \end{macrocode}
%
%    The macro |\ldf@finish| takes care of looking for a
%    configuration file, setting the main language to be switched on
%    at |\begin{document}| and resetting the category code of
%    \texttt{@} to its original value.
% \changes{breton-1.0e}{1996/11/02}{Now use \cs{ldf@finish} to wrap
%    up} 
%    \begin{macrocode}
\ldf@finish{breton}
%</code>
%    \end{macrocode}
%
% \Finale
%%
%% \CharacterTable
%%  {Upper-case    \A\B\C\D\E\F\G\H\I\J\K\L\M\N\O\P\Q\R\S\T\U\V\W\X\Y\Z
%%   Lower-case    \a\b\c\d\e\f\g\h\i\j\k\l\m\n\o\p\q\r\s\t\u\v\w\x\y\z
%%   Digits        \0\1\2\3\4\5\6\7\8\9
%%   Exclamation   \!     Double quote  \"     Hash (number) \#
%%   Dollar        \$     Percent       \%     Ampersand     \&
%%   Acute accent  \'     Left paren    \(     Right paren   \)
%%   Asterisk      \*     Plus          \+     Comma         \,
%%   Minus         \-     Point         \.     Solidus       \/
%%   Colon         \:     Semicolon     \;     Less than     \<
%%   Equals        \=     Greater than  \>     Question mark \?
%%   Commercial at \@     Left bracket  \[     Backslash     \\
%%   Right bracket \]     Circumflex    \^     Underscore    \_
%%   Grave accent  \`     Left brace    \{     Vertical bar  \|
%%   Right brace   \}     Tilde         \~}
%%
\endinput
}
%    \end{macrocode}
% \changes{babel~3.5d}{1995/07/02}{Added \Lopt{british} as an
%    alternative for \Lopt{english} with a preference for british
%    hyphenation}
% \changes{babel~3.7f}{2000/09/21}{Added the \Lopt{bulgarian} option}
% \changes{babel~3.7g}{2001/02/07}{Added option \Lopt{canadian}}
% \changes{babel~3.7g}{2001/02/09}{Added option \Lopt{canadien}}
%    \begin{macrocode}
\DeclareOption{british}{%%
%% This file will generate fast loadable files and documentation
%% driver files from the doc files in this package when run through
%% LaTeX or TeX.
%%
%% Copyright 1989-2005 Johannes L. Braams and any individual authors
%% listed elsewhere in this file.  All rights reserved.
%% 
%% This file is part of the Babel system.
%% --------------------------------------
%% 
%% It may be distributed and/or modified under the
%% conditions of the LaTeX Project Public License, either version 1.3
%% of this license or (at your option) any later version.
%% The latest version of this license is in
%%   http://www.latex-project.org/lppl.txt
%% and version 1.3 or later is part of all distributions of LaTeX
%% version 2003/12/01 or later.
%% 
%% This work has the LPPL maintenance status "maintained".
%% 
%% The Current Maintainer of this work is Johannes Braams.
%% 
%% The list of all files belonging to the LaTeX base distribution is
%% given in the file `manifest.bbl. See also `legal.bbl' for additional
%% information.
%% 
%% The list of derived (unpacked) files belonging to the distribution
%% and covered by LPPL is defined by the unpacking scripts (with
%% extension .ins) which are part of the distribution.
%%
%% --------------- start of docstrip commands ------------------
%%
\def\filedate{1999/04/11}
\def\batchfile{english.ins}
\input docstrip.tex

{\ifx\generate\undefined
\Msg{**********************************************}
\Msg{*}
\Msg{* This installation requires docstrip}
\Msg{* version 2.3c or later.}
\Msg{*}
\Msg{* An older version of docstrip has been input}
\Msg{*}
\Msg{**********************************************}
\errhelp{Move or rename old docstrip.tex.}
\errmessage{Old docstrip in input path}
\batchmode
\csname @@end\endcsname
\fi}

\declarepreamble\mainpreamble
This is a generated file.

Copyright 1989-2005 Johannes L. Braams and any individual authors
listed elsewhere in this file.  All rights reserved.

This file was generated from file(s) of the Babel system.
---------------------------------------------------------

It may be distributed and/or modified under the
conditions of the LaTeX Project Public License, either version 1.3
of this license or (at your option) any later version.
The latest version of this license is in
  http://www.latex-project.org/lppl.txt
and version 1.3 or later is part of all distributions of LaTeX
version 2003/12/01 or later.

This work has the LPPL maintenance status "maintained".

The Current Maintainer of this work is Johannes Braams.

This file may only be distributed together with a copy of the Babel
system. You may however distribute the Babel system without
such generated files.

The list of all files belonging to the Babel distribution is
given in the file `manifest.bbl'. See also `legal.bbl for additional
information.

The list of derived (unpacked) files belonging to the distribution
and covered by LPPL is defined by the unpacking scripts (with
extension .ins) which are part of the distribution.
\endpreamble

\declarepreamble\fdpreamble
This is a generated file.

Copyright 1989-2005 Johannes L. Braams and any individual authors
listed elsewhere in this file.  All rights reserved.

This file was generated from file(s) of the Babel system.
---------------------------------------------------------

It may be distributed and/or modified under the
conditions of the LaTeX Project Public License, either version 1.3
of this license or (at your option) any later version.
The latest version of this license is in
  http://www.latex-project.org/lppl.txt
and version 1.3 or later is part of all distributions of LaTeX
version 2003/12/01 or later.

This work has the LPPL maintenance status "maintained".

The Current Maintainer of this work is Johannes Braams.

This file may only be distributed together with a copy of the Babel
system. You may however distribute the Babel system without
such generated files.

The list of all files belonging to the Babel distribution is
given in the file `manifest.bbl'. See also `legal.bbl for additional
information.

In particular, permission is granted to customize the declarations in
this file to serve the needs of your installation.

However, NO PERMISSION is granted to distribute a modified version
of this file under its original name.

\endpreamble

\keepsilent

\usedir{tex/generic/babel} 

\usepreamble\mainpreamble
\generate{\file{english.ldf}{\from{english.dtx}{code}}
          }
\usepreamble\fdpreamble

\ifToplevel{
\Msg{***********************************************************}
\Msg{*}
\Msg{* To finish the installation you have to move the following}
\Msg{* files into a directory searched by TeX:}
\Msg{*}
\Msg{* \space\space All *.def, *.fd, *.ldf, *.sty}
\Msg{*}
\Msg{* To produce the documentation run the files ending with}
\Msg{* '.dtx' and `.fdd' through LaTeX.}
\Msg{*}
\Msg{* Happy TeXing}
\Msg{***********************************************************}
}
 
\endinput
}
\DeclareOption{bulgarian}{% \iffalse meta-com

% Copyright 1989-2008 Johannes L. Braams and any individual aut
% listed elsewhere in this file.  All rights reser

% This file is part of the Babel sys
% ----------------------------------

% It may be distributed and/or modified under
% conditions of the LaTeX Project Public License, either version
% of this license or (at your option) any later vers
% The latest version of this license i
%   http://www.latex-project.org/lppl
% and version 1.3 or later is part of all distributions of L
% version 2003/12/01 or la

% This work has the LPPL maintenance status "maintain

% The Current Maintainer of this work is Johannes Bra

% The list of all files belonging to the Babel syste
% given in the file `manifest.bbl. See also `legal.bbl' for additi
% informat

% The list of derived (unpacked) files belonging to the distribu
% and covered by LPPL is defined by the unpacking scripts (
% extension .ins) which are part of the distribut
%
%    \CheckSum{1
%
%    \iff
%    Tell the \LaTeX\ system who we are and write an entry on
%    transcr
%<*
\ProvidesFile{bulgarian.
%</
%<code>\ProvidesLanguage{bulgar
          [2008/03/21 v1.0g Bulgarian support from the babel sys

%% File `bulgarian.
%% Babel package for LaTeX versio
%% Copyright (C) 1989-
%%               by Johannes Braams,TeX

%% Bulgarian Language Definition
%%               Copyright (C) 1995-
%%               by Georgi.Boshnakov <georgi.boshnakov at umist.ac
%%                  Johannes Braams,TeX

%% Adapted from russianb
%%               by Georgi.Boshnakov <georgi.boshnakov at umist.ac

%% Please report errors to:J.L.Br
%%               babel at braams.xs4al

%<*filedri
\documentclass{ltx
\newcommand\TeXhax{\TeX
\newcommand\babel{\textsf{bab
\newcommand\langvar{$\langle\it lang \rang
\newcommand\note[
\newcommand\Lopt[1]{\textsf{
\newcommand\file[1]{\texttt{
\newcommand\pkg[1]{\texttt{
\begin{docum
 \DocInput{bulgarian.
\end{docum
%</filedri

% \GetFileInfo{bulgarian.

%    \changes{bulgarian-0.99}{2000/06/1
%          This is a prerelease version of this f
%          Features needing further testing are remov
%
%    \section{The Bulgarian langu
%
%    The file \file{\filename}\footnote{The file described in
%    section has version number \fileversion\ and was last revise
%    \filedate. This file was initially derived from the August-
%    version of \file{russianb.dt
%
%    It is (reasonably) backward compatible with the 1994/
%    (non-babel) bulgarian style (bulgaria.sty) by Ge
%    Boshnakov---files prepared for that style should com
%    successfully (with vastly improved appearance due to usag
%    standard fonts).} provides the language-specific macros for
%    Bulgarian langua
%
%    Users should take note of the vaious ``cyrillic'' da
%    available now (see below). These should remove many cause
%    headache. Also, although by default the Bulgarian quotation m
%    will appear automatically when typesetting in Bulgarian, i
%    better to use the new commands \cs{"'} and \cs{"'} w
%    explicitly typeset t
%    Note: automatic switch to Bulgarian quotation is withd
%    for the moment and may not be reintroduced at
%
%    For this language the character |"| is made active
%    table~\ref{tab:bulgarian-quote} an overview is given of
%    purp
%
%    \begin{table}[
%      \begin{cen
%      \begin{tabular}{lp{8
%       \verb="|= & disable ligature at this position.
%       |"-| & an explicit hyphen sign, allowing hyphena
%                       in the rest of the word.
%       |"---| & Cyrillic emdash in plain text.
%       |"--~| & Cyrillic emdash in compound names (surnames).
%       |"--*| & Cyrillic emdash for denoting direct speech.
%       |""|  & like |"-|, but producing no hyphen
%                       (for compound words with hyphen, e.g.\ |x-
%                       or some other signs  as ``disable/enable'').
%       |"~|   & for a compound word mark without a breakpoint.
%       |"=|   & for a compound word mark with a breakpoint, allo
%                       hyphenation in the composing words.
%       |",|   & thinspace for initials with a breakp
%                       in following surname.
%       |"`|   & for German left double qu
%                       (looks like ,\kern-0.08em,).
%       |"'|   & for German right double quotes (looks like ``).      \\%^
%       |"<|   & for French left double quotes  (looks like $<\!\!<$
%       |">|   & for French right double quotes (looks like $>\!\!>$
%      \end{tabu
%      \caption{The extra definitions
%                by\file{bulgarian}}\label{tab:bulgarian-qu
%      \end{cen
%    \end{ta
%
%    The quotes in table~\ref{tab:bulgarian-quote} can also be typ
%    by using the commands in table~\ref{tab:bmore-quo
%
%    \begin{table}[
%      \begin{cen
%      \begin{tabular}{lp{8
%       |\cdash---| & Cyrillic emdash in plain text.
%       |\cdash--~| & Cyrillic emdash in compound names (surnames).
%       |\cdash--*| & Cyrillic emdash for denoting direct speech.
%       |\glqq|     & for German left double qu
%                         (looks like,\kern-0.08em,).
%       |\grqq|     & for German right double quotes (looks like ``).\\^^
%       |\flqq|     & for French left double quotes (looks like $<\!\!<$
%       |\frqq|     & for French right double quotes (looks like $>\!\!>$
%       |\dq|       & the original quotes character (|
%      \end{tabu
%      \caption{More commands which produce quotes, def
%                by \babel}\label{tab:bmore-qu
%      \end{cen
%    \end{ta
%
%    The French quotes are also available as ligatures `|<<|'
%    `|>>|' in 8-bit Cyrillic font encodings (\texttt{L
%    \texttt{X2}, \texttt{T2*}) and as `|<|' and `|>|' character
%    7-bit Cyrillic font encodings (\texttt{OT2} and \texttt{LW
%
%    The quotation marks traditionally used in Bulgarian were borr
%    from German o they keep their original names. French quota
%    marks may be seen as well in older boo

% \StopEventual

%    The macro |\LdfInit| takes care of preventing that this fil
%    loaded more than once, checking the category code of
%    \texttt{@} sign, e
%
%    \begin{macroc
%<*c
\LdfInit{bulgarian}{captionsbulgar
%    \end{macroc
%
%    When this file is read as an option, i.e., by the |\usepack
%    command, \texttt{bulgarian} will be an `unknown' language
%    which case we have to make it known. So we check for
%    existence of |\l@bulgarian| to see whether we have t
%    something he
%
%    \begin{macroc
\ifx\l@bulgarian\@undef
  \@nopatterns{Bulgar
  \adddialect\l@bulgar

%    \end{macroc
%
%
%
% \begin{macro}{\latinencod

%    We need to know the encoding for text that is supposed t
%    which is active at the end of the \babel\ package. If
%    \pkg{fontenc} package is loaded later, then \ldots too
%
%    \begin{macroc
\let\latinencoding\cf@enco
%    \end{macroc
%
% \end{ma

%    The user may choose between different available Cyri
%    encodings---e.g., \texttt{X2}, \texttt{LCY}, or \texttt{L
%    If the user wants to use a font encoding other than the def
%    (\texttt{T2A}), he has to load the corresponding
%    \emph{before} \file{bulgarian.st
%    This may be done in the following
%
%    \begin{verba
%      \usepackage[LCY,OT1]{fontenc}     %overwrite the default encod
%      \usepackage[english,bulgarian]{ba
%    \end{verba
%    \un
%
%    Note: most people would prefer the \texttt{T2A} to \texttt{
%    because \texttt{X2} does not contain Latin letters, and  u
%    should be very careful to switch the language every time t
%    want to typeset a Latin word inside a Bulgarian phrase or
%    versa. On the other hand, switching the language is a
%    practice anyway. With a decent text processing program it
%    not involve more work than switching between the Bulgarian
%    English keyboard. Moreover that the far most common disrup
%    occurs  as a result of forgetting to switch back to cyri
%    keyboa
%
%    We parse the |\cdp@list| containing the encodings know
%    \LaTeX\ in the order they were loaded. We set
%    |\cyrillicencoding| to the \emph{last} loaded encoding in
%    list of supported Cyrillic encodings: \texttt{OT2}, \texttt{L
%    \texttt{LCY}, \texttt{X2}, \texttt{T2C}, \texttt{T
%    \texttt{T2A}, if a
%
%
%    \begin{macroc
\def\reserved@a#1
   \edef\reserved@b{
   \edef\reserved@c{
   \ifx\reserved@b\reserv
     \let\cyrillicencoding\reserv

\def\cdp@elt#1#2#3
   \reserved@a{#1}{O
   \reserved@a{#1}{L
   \reserved@a{#1}{L
   \reserved@a{#1}{
   \reserved@a{#1}{T
   \reserved@a{#1}{T
   \reserved@a{#1}{T
\cdp@
%    \end{macroc
%
%    Now, if |\cyrillicencoding| is undefined, then the user did
%    load any of supported encodings. So, we have to
%    |\cyrillicencoding| to some default value. We test the pres
%    of the encoding definition files in the order from
%    preferable to more preferable encodings. We use the lower
%    names (i.e., \file{lcyenc.def} instead of \file{LCYenc.de
%
%    \begin{macroc
\ifx\cyrillicencoding\undef
  \IfFileExists{ot2enc.def}{\def\cyrillicencoding{OT2}}\r
  \IfFileExists{lwnenc.def}{\def\cyrillicencoding{LWN}}\r
  \IfFileExists{lcyenc.def}{\def\cyrillicencoding{LCY}}\r
  \IfFileExists{x2enc.def}{\def\cyrillicencoding{X2}}\r
  \IfFileExists{t2cenc.def}{\def\cyrillicencoding{T2C}}\r
  \IfFileExists{t2benc.def}{\def\cyrillicencoding{T2B}}\r
  \IfFileExists{t2aenc.def}{\def\cyrillicencoding{T2A}}\r
%    \end{macroc
%
%    If |\cyrillicencoding| is still undefined, then the user s
%    not to  have a properly installed distribution. A fatal er
%
%    \begin{macroc
\ifx\cyrillicencoding\undef
    \PackageError{bab
    {No Cyrillic encoding definition files were fou
    {Your installation is incomplete. \MessageB
    You need at least one of the following files: \MessageB
    \space\s
    x2enc.def, t2aenc.def, t2benc.def, t2cenc.def, \MessageB
    \space\s
    lcyenc.def, lwnenc.def, ot2enc.de
  \
%    \end{macroc
%
%    We avoid |\usepackage[\cyrillicencoding]{fontenc}| becaus
%    don't want to force the switch of |\encodingdefau
%
%    \begin{macroc
    \lower
      \expandafter{\expandafter\input\cyrillicencoding enc.def\rel


%    \end{macroc
%
%    \begin{verba
%      \PackageInfo{ba
%        {Using `\cyrillicencoding' as a default Cyrillic encodi
%    \end{verba
%    \un
%
%
%
%    \begin{macroc
\DeclareRobustCommand{\Bulgaria
  \fontencoding\cyrillicencoding\select
  \let\encodingdefault\cyrillicenco
  \expandafter\set@hyphenmins\bulgarianhyphen
  \language\l@bulgar
\DeclareRobustCommand{\Englis
  \fontencoding\latinencoding\select
  \let\encodingdefault\latinenco
  \expandafter\set@hyphenmins\englishhyphen
  \language\l@engl
\let\Bul\Bulga
\let\Bg\Bulga
\let\cyrillictext\Bulga
\let\cyr\Bulga
\let\Eng\Eng
\def\selectenglanguage{\selectlanguage{engli
\def\selectbglanguage{\selectlanguage{bulgari
%    \end{macroc
%
%    Since the \texttt{X2} encoding does not contain Latin letters
%    should make some redefinitions of \LaTeX\ macros which implici
%    produce Latin lett
%
%
%    \begin{macroc
\expandafter\ifx\csname T@X2\endcsname\relax\
%    \end{macroc
%
%    We put |\latinencoding| in braces to avoid problems with |\@a
%    inside minipages (e.g., footnotes inside minipages) w
%    |\@alph| is expanded and we get for example `|\fontencoding O
%    (|\fontencoding| is robu
%
% \changes{bulgarian-1.0c}{2003/06/14}{Added missing closing br
%    \begin{macroc
  \def\@Alph@eng#1{{\fontencoding{\latinencoding}\selectf
      \ifcase#1\or A\or B\or C\or D\or E\or F\or G\or H\or I\or
      K\or L\or M\or N\or O\or P\or Q\or R\or S\or T\or U\or V\or
      X\or Y\or Z\else \@ctrerr\f
  \def\@alph@eng#1{{\fontencoding{\latinencoding}\select
      \ifcase#1\or a\or b\or c\or d\or e\or f\or g\or h\or i\or
      k\or l\or m\or n\or o\or p\or q\or r\or s\or t\or u\or v\or
      x\or y\or z\else \@ctrerr\f
  \let\@Alph\@Alph
  \let\@alph\@alph
%    \end{macroc
%
%    Unfortunately, the commands |\AA| and |\aa| are not enco
%    dependent in \LaTeX\ (unlike e.g., |\oe| or |\DH|). They
%    defined as |\r{A}| and |\r{a}|. This leads to unpredict
%    results when the font encoding does not contain the Latin let
%    `A' and `a' (like \texttt{X2
%
%    \begin{macroc
  \DeclareTextSymbolDefault{\AA}{
  \DeclareTextSymbolDefault{\aa}{
  \DeclareTextCommand{\AA}{OT1}{\
  \DeclareTextCommand{\aa}{OT1}{\

%    \end{macroc
%
%    The following block redefines the character class of upper
%    Greek letters and some accents, if it is equal to 7 (vari
%    family), to avoid incorrect results if the font encoding in
%    math family does not contain these characters in places of
%    encoding. The code was taken from |amsmath.dtx|. See comments
%    further explanation the
%
%    \begin{macroc
\begingroup\catcode`\
% uppercase greek lett
\def\@tempa#1{\expandafter\@tempb\meaning#1\relax\relax\relax\r
 "0000\@ni
\def\@tempb#1"#2#3#4#5#6\@nil
\ifnum"#2=7 \count@"1#3#4#5\r
\ifnum\count@<"1000 \else \global\mathchardef#7="0#3#4#5\relax

\@tempa\Gamma\@tempa\Delta\@tempa\Theta\@tempa\Lambda\@temp
\@tempa\Pi\@tempa\Sigma\@tempa\Upsilon\@tempa\Phi\@tempa
\@tempa\O
% some acce
\def\@tempa#1#2\@nil{\def\@tempc{#1}}\def\@tempb{\mathacc
\expandafter\@tempa\hat\relax\relax\
\ifx\@tempb\@t
\def\@tempa#1\@nil{
\def\@tempb#1{\afterassignment\@tempa\mathchardef\@temp
\def\do#1"
\def\@tempd#1{\expandafter\@tempb#1\
 \ifnum\@tempc>
 \xdef#1{\mathaccent"\expandafter\do\meaning\@tempc\spa

\@tempd\hat\@tempd\check\@tempd\tilde\@tempd\acute\@tempd\g
\@tempd\dot\@tempd\ddot\@tempd\breve\@tempd

\endg
%    \end{macroc
%
%    The user should use the \pkg{inputenc} package when any 8
%    Cyrillic font encoding is used, selecting one of the Cyri
%    input encodings. We do not assume any default input encoding
%    the user should explicitly call the \pkg{inputenc} packag
%    |\usepackage{inputenc}|. We also removed |\AtBeginDocument|
%    \pkg{inputenc} should be used before \bab
%
%    \begin{macroc
\@ifpackageloaded{inputenc}
\def\reserved@a{L
\ifx\reserved@a\cyrillicencoding\
\def\reserved@a{O
\ifx\reserved@a\cyrillicencoding\
\PackageWarning{bab
{No input encoding specified for Bulgarian language}\fi
%    \end{macroc
%
%    Now we define two commands that offer the possibility to sw
%    between Cyrillic and Roman encodi
%
% \begin{macro}{\cyrillict
% \begin{macro}{\latint

%    The command |\cyrillictext| will switch from Latin font enco
%    to the Cyrillic font encoding, the command |\latintext| swit
%    back. This assumes that the `normal' font encoding is a L
%    one. These commands are \emph{declarations}, for shorter pe
%    of text the commands |\textlatin| and |\textcyrillic| ca
%    u
%
%    We comment out |\latintext| since it is defined in the cor
%    babel (babel.def). We add the shorthand |\lat| for |\latinte
%    Note that |\cyrillictext| has been defined ab
%
%    \begin{macroc
% \DeclareRobustCommand{\latintex
% \fontencoding{\latinencoding}\select
%   \def\encodingdefault{\latinencodi
\let\lat\latin
%    \end{macroc
%
% \end{ma
% \end{ma

% \begin{macro}{\textcyril
% \begin{macro}{\textla

%    These commands take an argument which is then typeset using
%    requested font encod
%    |\textlatin| is commented out since it is defined in the cor
%    babel. (It is defined there with |\DeclareRobustCommand| inste
%
%    \begin{macroc
\DeclareTextFontCommand{\textcyrillic}{\cyrillict
% \DeclareTextFontCommand{\textlatin}{\latint
%    \end{macroc
%
% \end{ma
% \end{ma

%    The next step consists of defining commands to switch to (
%    from) the Bulgarian langu
%
% \begin{macro}{\captionsbulgar

%    The macro |\captionsbulgarian| defines all strings used in
%    four standard document classes provided with \LaTeX. The
%    commands |\cyr| and |\lat| activate Cyrillic resp. Latin encod
%
%    \begin{macroc
\addto\captionsbulgari
  \def\prefacena
    {\cyr\CYRP\cyrr\cyre\cyrd\cyrg\cyro\cyrv\cyro\cyr
  \def\refna
    {\cyr\CYRL\cyri\cyrt\cyre\cyrr\cyra\cyrt\cyru\cyrr\cyr
  \def\abstractna
    {\cyr\CYRA\cyrb\cyrs\cyrt\cyrr\cyra\cyrk\cyr
  \def\bibna
    {\cyr\CYRB\cyri\cyrb\cyrl\cyri\cyro\cyrg\cyrr\cyra\cyrf\cyri\cyry
  \def\chapterna
    {\cyr\CYRG\cyrl\cyra\cyrv\cyr
  \def\appendixna
    {\cyr\CYRP\cyrr\cyri\cyrl\cyro\cyrzh\cyre\cyrn\cyri\cyr
  \def\contentsna
    {\cyr\CYRS\cyrhrdsn\cyrd\cyrhrdsn\cyrr\cyrzh\cyra\cyrn\cyri\cyr
  \def\listfigurena
    {\cyr\CYRS\cyrp\cyri\cyrs\cyrhrdsn\cyrk\ \cyrn\cyra\ \cyrf\cyri\cyrg\cyru\cyrr\cyri\cyrt\cyr
  \def\listtablena
    {\cyr\CYRS\cyrp\cyri\cyrs\cyrhrdsn\cyrk\ \cyrn\cyra\ \cyrt\cyra\cyrb\cyrl\cyri\cyrc\cyri\cyrt\cyr
  \def\indexna
    {\cyr\CYRA\cyrz\cyrb\cyru\cyrch\cyre\cyrn\ \cyru\cyrk\cyra\cyrz\cyra\cyrt\cyre\cyr
  \def\authorna
    {\cyr\CYRI\cyrm\cyre\cyrn\cyre\cyrn\ \cyru\cyrk\cyra\cyrz\cyra\cyrt\cyre\cyr
  \def\figurena
    {\cyr\CYRF\cyri\cyrg\cyru\cyrr\cyr
  \def\tablena
    {\cyr\CYRT\cyra\cyrb\cyrl\cyri\cyrc\cyr
  \def\partna
    {\cyr\CYRCH\cyra\cyrs\cyr
  \def\enclna
    {\cyr\CYRP\cyrr\cyri\cyrl\cyro\cyrzh\cyre\cyrn\cyri\cyry
  \def\ccna
    {\cyr\cyrk\cyro\cyrp\cyri\cyry
  \def\headtona
    {\cyr\CYRZ\cyr
  \def\pagena
    {\cyr\CYRS\cyrt\cyrr
  \def\seena
    {\cyr\cyrv\cyrzh
  \def\alsona
    {\cyr\cyrv\cyrzh.\ \cyrs\cyrhrdsn\cyrshch\cyro\ \cyr
  \def\proofname{Proof}% <-- Needs transla
  \def\glossaryname{Glossary}% <-- Needs transla

%    \end{macrocode}
% \end{ma
%%
% \begin{macro}{\datebulgar

%    The macro |\datebulgarian| redefines the command |\today
%    produce Bulgarian da
%    It also provides the command |\todayRoman| which produces
%    date with the month in capital roman numerals, a popular fo
%    for dates in Bulgari
%
%    \begin{macroc
\def\datebulgari
  \def\month@bulgarian{\ifcase\month
    \cyrya\cyrn\cyru\cyra\cyrr\cyri
    \cyrf\cyre\cyrv\cyrr\cyru\cyra\cyrr\cyri
    \cyrm\cyra\cyrr\cyrt
    \cyra\cyrp\cyrr\cyri\cyrl
    \cyrm\cyra\cyrishrt
    \cyryu\cyrn\cyri
    \cyryu\cyrl\cyri
    \cyra\cyrv\cyrg\cyru\cyrs\cyrt
    \cyrs\cyre\cyrp\cyrt\cyre\cyrm\cyrv\cyrr\cyri
    \cyro\cyrk\cyrt\cyro\cyrm\cyrv\cyrr\cyri
    \cyrn\cyro\cyre\cyrm\cyrv\cyrr\cyri
    \cyrd\cyre\cyrk\cyre\cyrm\cyrv\cyrr\cyri\
  \def\month@Roman{\expandafter\@Roman\mon
  \def\today{\number\day~\month@bulgarian\ \number\year~\cyr
  \def\todayRoman{\number\day.\,\month@Roman.\,\number\year~\cyr

%    \end{macroc
%
% \end{ma

%
%
% \begin{macro}{\todayRo

%    The month is often written with roman numbers in Bulgarian da
%    Here we define date in this for
%
%    \begin{macroc
\def\Romannumeral#1{\uppercase\expandafter{\romannumeral
\def\todayRoman{\number\day.\Romannumeral{\month}.\number\year~\cy
%    \end{macroc
%
% \end{ma

% \begin{macro}{\extrasbulgar

%    The macro |\extrasbulgarian| will perform all the e
%    definitions needed for the Bulgarian langu
%    The macro |\noextrasbulgarian| is used to cancel the action
%    |\extrasbulgaria
%
%    The first action we define is to switch on the selected Cyri
%    encoding whenever we enter `bulgari
%
%    \begin{macroc
\addto\extrasbulgarian{\cyrillict
%    \end{macroc
%
%    When the encoding definition file was processed by \LaTeX\
%    current font encoding is stored in |\latinencoding|, assu
%    that \LaTeX\ uses \texttt{T1} or \texttt{OT1
%    default. Therefore we switch back to |\latinencoding| when
%    the Bulgarian language is no longer `activ
%
%    \begin{macroc
\addto\noextrasbulgarian{\latint
%    \end{macroc
%
%    For Bulgarian the \texttt{"} character also is made act
%
%    \begin{macroc
\initiate@active@cha
%    \end{macroc
%
%    The code above is necessary because we need extra ac
%    characters. The character |"| is used as indicate
%    table~\ref{tab:bulgarian-quote}. We specify that the Bulga
%    group of shorthands should be us
%
%    \begin{macroc
\addto\extrasbulgarian{\languageshorthands{bulgari
%    \end{macroc
%
%    These characters are `turned on' once, later their definition
%    v
%
%    \begin{macroc
\addto\extrasbulgari
  \bbl@activate
\addto\noextrasbulgari
  \bbl@deactivate
%    \end{macroc
%
%    The \texttt{X2} and \texttt{T2*} encodings do not con
%    |spanish_shriek| and |spanish_query| symbols; as a conseque
%    the ligatures `|?`|' and `|!`|' do not work with them (t
%    characters are useless for Cyrillic texts anyway). But we de
%    the shorthands to emulate these ligatures (optional
%
%    We do not use |\latinencoding| here (but instead explicitly
%    \texttt{OT1}) because the user may choose \texttt{T2A} to be
%    primary encoding, but it does not contain these charact
%
%    \begin{macroc
%<*spanishl
\declare@shorthand{bulgarian}{?`}{\UseTextSymbol{OT1}\textquestiond
\declare@shorthand{bulgarian}{!`}{\UseTextSymbol{OT1}\textexclamd
%</spanishl
%    \end{macroc
%
%    To be able to define the function of `|"|', we first defi
%    couple of `support' mac
%
% \begin{macro}{

%    We save the original double quote character in |\dq| to keep
%    available, the math accent |\"|can now be typed as `|
% \changes{bulgarian-1.0c}{2003/04/10}{repaired t
%    \begin{macroc
\begingroup \catcode`
\def\reserved@a{\endg
  \def\@SS{\mathchar"7
  \def\dq
\reserv
%    \end{macroc
%
% \end{ma

%    Now we can define the doublequote macros: german and fr
%    quotes. We use definitions of these quotes made in babel.
%    The french quotes are contained in the \texttt{T2*} encodi
%
%    \begin{macroc
\declare@shorthand{bulgarian}{"`}{\g
\declare@shorthand{bulgarian}{"'}{\g
\declare@shorthand{bulgarian}{"<}{\f
\declare@shorthand{bulgarian}{">}{\f
%    \end{macroc
%
%    Some additional comma
%
%    \begin{macroc
\declare@shorthand{bulgarian}{""}{\hskip\z@s
\declare@shorthand{bulgarian}{"~}{\textormath{\leavevmode\hbox{-}}
\declare@shorthand{bulgarian}{"=}{\nobreak-\hskip\z@s
\declare@shorthand{bulgarian}{"
\textormath{\nobreak\discretionary{-}{}{\kern.03
\allowhyphens
%    \end{macroc
%
%    The next two macros for |"-| and |"---| are somewhat differ
%    We must check whether the second token is a hyphen charac
%
%    \begin{macroc
\declare@shorthand{bulgarian}{"
%    \end{macroc
%
%    If the next token is `|-|', we typeset an emdash, otherwi
%    hyphen s
%
%    \begin{macroc
  \def\bulgarian@sh@t
    \if\bulgarian@sh@next-\expandafter\bulgarian@sh@em
    \else\expandafter\bulgarian@sh@hyphe

%    \end{macroc
%
%    \TeX\ looks for the next token after the first `|-|': the mea
%    of this token is written to |\bulgarian@sh@next|
%    |\bulgarian@sh@tmp| is cal
%
%    \begin{macroc
  \futurelet\bulgarian@sh@next\bulgarian@sh@
%    \end{macroc
%
%    Here are the definitions of hyphen and emdash. First the hyp
%
%    \begin{macroc
\def\bulgarian@sh@hyphen{\nobreak\-\bbl@allowhyph
%    \end{macroc
%
%    For the emdash definition, there are the two parameters: we
%    `eat' two last hyphen signs of our emdash \do
%
%    \begin{macroc
\def\bulgarian@sh@emdash#1#2{\cdash-#
%    \end{macroc
%
% \begin{macro}{\cd

%    \dots\ these two parameters are useful for another ma
%    |\cda
%
% \changes{bulgarian-1.0e}{2006/03/31}{Two occurences of \cmd{t
%    were changed into tab followed by e
%    \begin{macroc
\ifx\cdash\undefined % should be defined ear
\def\cdash#1#2#3{\def\tempx@{
\def\tempa@{-}\def\tempb@{~}\def\tempc@
 \ifx\tempx@\tempa@\@Acdash\
  \ifx\tempx@\tempb@\@Bcdash\
   \ifx\tempx@\tempc@\@Ccdash\
    \errmessage{Wrong usage of cdash}\fi\fi
%    \end{macroc
%
%  second parameter (or third for |\cdash|) shows what kind of em
%  to create in next
%  \begin{cen
%  \begin{tabular}{@{}p{.1\hsize}@{}p{.9\hsize}
%   |"---| & ordinary (plain) Cyrillic emdash inside t
%   an unbreakable thinspace will be inserted before only in cas
%   a \textit{space} before the dash (it is necessary for dashes a
%   display maths formulae: there could be lists, enumerations et
%   started with ``---where $a$ is ...'' i.e., the dash starts a li
%   (Firstly there were planned rather soft rules for user:he may
%   a space before the dash or not. But it is difficult to place
%   thinspace automatically, i.e., by checking modes because a
%   display formulae \TeX{} uses horizontal mode. Maybe there i
%   misunderstanding? Maybe there is another way?) After a
%   a breakable thinspace is always placed
%  \end{tabu
%  \end{cen
%
%
%    \begin{macroc
% What is more grammatically: .2em or .2\fontdimen6\f
\def\@Acdash{\ifdim\lastskip>\z@\unskip\nobreak\hskip.2e
\cyrdash\hskip.2em\ignorespac
%    \end{macroc
%
%      \begin{cen
%      \begin{tabular}{@{}p{.1\hsize}@{}p{.9\hsize}
%       |"--~| & emdash in compound names or surn
%       (like Mendeleev--Klapeiron); this dash has no space charac
%       around; after the dash some space is a
%       |\exhyphenalty
%      \end{tabu
%      \end{cen
%
%    \begin{macroc
\def\@Bcdash{\leavevmode\ifdim\lastskip>\z@\unski
 \nobreak\cyrdash\penalty\exhyphenpenalty\hskip\z@skip\ignorespac
%    \end{macroc
%
%    \begin{cen
%    \begin{tabular}{@{}p{.1\hsize}@{}p{.9\hsize}
%     |"--*| & for denoting direct speech (a space like |\ens
%     must follow the emdash)
%    \end{tabu
%    \end{cen
%
%    \begin{macroc
\def\@Ccdash{\leavev
 \nobreak\cyrdash\nobreak\hskip.35em\ignorespac

%    \end{macroc
%
% \end{ma

%
%
% \begin{macro}{\cyrd

%    Finally the macro for ``body'' of the Cyrillic emd
%    The |\cyrdash| macro will be defined in case this macro ha
%    been defined in a fontenc file. For T2*fonts, cyrdash wil
%    placed in the code of the English emdash thus it uses liga
%    |--
%
%    \begin{macroc
% Is there an IF necess
\ifx\cyrdash\undef
\def\cyrdash{\hbox to.8em{--\hss

%    \end{macroc
%
% \end{ma

%    Here a really new macro---to place thinspace between initi
%    This macro used instead of |\,| allows hyphenation in
%    following surn
%
%    \begin{macroc
\declare@shorthand{bulgarian}{",}{\nobreak\hskip.2em\ignorespa
%    \end{macroc
%
%    The Bulgarian hyphenation patterns can be used
%    |\lefthyphenmin| and |\righthyphenmin| set t
% \changes{bulgarian-1.0b}{2000/09/22}{Now use \cs{providehyphenmins
%    provide a default va
%    \begin{macroc
\providehyphenmins{\CurrentOption}{\tw@\

%    \end{macroc
%
%    Now the action |\extrasbulgarian| has to execute is to make
%    that the command |\frenchspacing| is in effect. If this is
%    the case the execution of |\noextrasbulgarian| will switch it
%    ag
%
%    \begin{macroc
\addto\extrasbulgarian{\bbl@frenchspac
\addto\noextrasbulgarian{\bbl@nonfrenchspac
%    \end{macroc
%
%    Make the double quotes produce the traditional qu
%    used in Bulgarian texts (these are the German quote
%
%    \begin{macroc
% \initiate@active@cha
%  \initiate@active@cha
% \addto\extrasbulgari
%   \bbl@activate
% \addto\extrasbulgari
%   \bbl@activate
% \addto\noextrasbulgari
%   \bbl@deactivate
% \addto\noextrasbulgari
%   \bbl@deactivate
% \def\mlron{\bbl@activate{`}\bbl@activate
% \def\mlroff{\bbl@deactivate{`}\bbl@deactivate
% \declare@shorthand{bulgarian}{``}{\g
% \declare@shorthand{bulgarian}{''}{\g
%    \end{macroc
%
% \end{ma

%    Next we add a new enumeration style for Bulgarian manuscripts
%    Cyrillic letters,and later on we define some math operator name
%    accordance with Bulgarian typesetting traditi
%
% \begin{macro}{\@Alph@

%    We begin by defining |\@Alph@bul| which works like |\@Alph|,
%    produces (uppercase) Cyrillic letters intead of Latin ones.
%    letters ISHRT, HRDSN and SFTSN are skipped, as usual for
%    enumerat
%
%    \begin{macroc
\def\enumBul{\let\@Alph\@Alph@bul \let\@alph\@alph@
\def\enumEng{\let\@Alph\@Alph@eng \let\@alph\@alph@
\def\enumLat{\let\@Alph\@Alph@eng \let\@alph\@alph@
\addto\extrasbulgarian{\enum
\addto\noextrasbulgarian{\enum
\def\@Alph@bul
  \ifcase#
  \CYRA\or \CYRB\or \CYRV\or \CYRG\or \CYRD\or \CYRE\or \CYRZ
  \CYRZ\or \CYRI\or \CYRK\or \CYRL\or \CYRM\or \CYRN\or \CYR
  \CYRP\or \CYRR\or \CYRS\or \CYRT\or \CYRU\or \CYRF\or \CYR
  \CYRC\or \CYRCH\or \CYRSH\or \CYRSHCH\or \CYRYU\or \CYRYA\
  \@ctrer

\def\@Alph@eng
  \ifcase#1
  A\or B\or C\or D\or E\or F\or G\or H\or I\or J\or K\or L\or M
  N\or O\or P\or Q\or R\or S\or T\or U\or V\or W\or X\or Y\or Z\
  \@ctrer

%    \end{macroc
%
% \end{ma

%
%
% \begin{macro}{\@alph@

%    The macro |\@alph@bul| is similar to |\@Alph@b
%    it produces lowercase Bulgarian lett
%
%    \begin{macroc
\def\@alph@bul
  \ifcase#1
  \cyra\or \cyrb\or \cyrv\or \cyrg\or \cyrd\or \cyre\or \cyrzh
  \cyrz\or \cyri\or \cyrk\or \cyrl\or \cyrm\or \cyrn\or \cyro
  \cyrp\or \cyrr\or \cyrs\or \cyrt\or \cyru\or \cyrf\or \cyrh
  \cyrc\or \cyrch\or \cyrsh\or \cyrshch\or \cyryu\or \cyrya\
  \@ctrer

\def\@alph@eng
  \ifcase#1
  a\or b\or c\or d\or e\or f\or g\or h\or i\or j\or k\or l\or m
  n\or o\or p\or q\or r\or s\or t\or u\or v\or w\or x\or y\or z\
  \@ctrer

%    \end{macroc
%
% \end{ma

%    Set up default Cyrillic math alphabets. To use Cyrillic let
%    in math mode user should load the |textmath| pac
%    \emph{before} loading fontenc package (or \babel). Note,tha
%    default Cyrillic letters are taken from upright font in math
%    (unlike Latin letter
%
%    \begin{macroc
%\RequirePackage{textm
\@ifundefined{sym\cyrillicencoding letters}
\SetSymbolFont{\cyrillicencoding letters}{bold}\cyrillicenco
  \rmdefault\bfdefault\updef
\DeclareSymbolFontAlphabet\cyrmathrm{\cyrillicencoding lett
%    \end{macroc
%
%    And we need a few commands to be able to switch to diffe
%    varia
%
%    \begin{macroc
\DeclareMathAlphabet\cyrmathbf\cyrillicenco
  \rmdefault\bfdefault\updef
\DeclareMathAlphabet\cyrmathsf\cyrillicenco
  \sfdefault\mddefault\updef
\DeclareMathAlphabet\cyrmathit\cyrillicenco
  \rmdefault\mddefault\itdef
\DeclareMathAlphabet\cyrmathtt\cyrillicenco
  \ttdefault\mddefault\updef
\SetMathAlphabet\cyrmathsf{bold}\cyrillicenco
  \sfdefault\bfdefault\updef
\SetMathAlphabet\cyrmathit{bold}\cyrillicenco
  \rmdefault\bfdefault\itdef

%    \end{macroc
%
%    Some math functions in Bulgarian math books have other na
%    e.g., \texttt{sinh} in Bulgarian is written as \texttt
%    etc. So we define a number of new math operat
%
%    |\si
%
%    \begin{macroc
\def\sh{\mathop{\operator@font sh}\nolim
%    \end{macroc
%
%    |\co
%
%    \begin{macroc
\def\ch{\mathop{\operator@font ch}\nolim
%    \end{macroc
%
%    |\t
%
%    \begin{macroc
\def\tg{\mathop{\operator@font tg}\nolim
%    \end{macroc
%
%    |\arct
%
%    \begin{macroc
\def\arctg{\mathop{\operator@font arctg}\nolim
%    \end{macroc
%
%    |\arcc
%
%    \begin{macroc
\def\arcctg{\mathop{\operator@font arcctg}\nolim
%    \end{macroc
%
%    The following macro conflicts with |\th| defined in Lat
%    encod
%    |\ta
% \changes{bulgarian-1.0d}{2004/05/21}{Change definition of \cs
%    only for this langu
%    \begin{macroc
\addto\extrasrussi
  \babel@save{\
  \let\ltx@t
  \def\th{\textormath{\ltx@
                     {\mathop{\operator@font th}\nolimit

%    \end{macroc
%
%    |\c
%
%    \begin{macroc
\def\ctg{\mathop{\operator@font ctg}\nolim
%    \end{macroc
%
%    |\co
%
%    \begin{macroc
\def\cth{\mathop{\operator@font cth}\nolim
%    \end{macroc
%
%    |\c
%
%    \begin{macroc
\def\cosec{\mathop{\operator@font cosec}\nolim
%    \end{macroc
%
%    This is for compatibility with older Bulgarian packa
%
%    \begin{macroc
\DeclareRobustCommand{\N
    \ifmmode{\nfss@text{\textnumero}}\else\textnumero
%    \end{macroc
%
%    The macro |\ldf@finish| takes care of looking for a configura
%    file, setting the main language to be switched o
%    |\begin{document}| and resetting the category code of \textt
%    to its original val
%
%    \begin{macroc
\ldf@finish{bulgar
%</c
%    \end{macroc
%
% \Fi

%% \CharacterT
%%  {Upper-case    \A\B\C\D\E\F\G\H\I\J\K\L\M\N\O\P\Q\R\S\T\U\V\W\X
%%   Lower-case    \a\b\c\d\e\f\g\h\i\j\k\l\m\n\o\p\q\r\s\t\u\v\w\x
%%   Digits        \0\1\2\3\4\5\6\7
%%   Exclamation   \!     Double quote \"    Hash (number
%%   Dollar        \$     Percent      \%    Ampersand
%%   Acute accent  \'     Left paren   \(    Right paren
%%   Asterisk      \*     Plus         \+    Comma
%%   Minus         \-     Point        \.    Solidus
%%   Colon         \:     Semicolon    \;    Less than
%%   Equals        \=     Greater than \>    Question mar
%%   Commercial at \@     Left bracket \[    Backslash
%%   Right bracket \]     Circumflex   \^    Underscore
%%   Grave accent  \`     Left brace   \{    Vertical bar
%%   Right brace   \}     Tilde

\endi
}
\DeclareOption{canadian}{%%
%% This file will generate fast loadable files and documentation
%% driver files from the doc files in this package when run through
%% LaTeX or TeX.
%%
%% Copyright 1989-2005 Johannes L. Braams and any individual authors
%% listed elsewhere in this file.  All rights reserved.
%% 
%% This file is part of the Babel system.
%% --------------------------------------
%% 
%% It may be distributed and/or modified under the
%% conditions of the LaTeX Project Public License, either version 1.3
%% of this license or (at your option) any later version.
%% The latest version of this license is in
%%   http://www.latex-project.org/lppl.txt
%% and version 1.3 or later is part of all distributions of LaTeX
%% version 2003/12/01 or later.
%% 
%% This work has the LPPL maintenance status "maintained".
%% 
%% The Current Maintainer of this work is Johannes Braams.
%% 
%% The list of all files belonging to the LaTeX base distribution is
%% given in the file `manifest.bbl. See also `legal.bbl' for additional
%% information.
%% 
%% The list of derived (unpacked) files belonging to the distribution
%% and covered by LPPL is defined by the unpacking scripts (with
%% extension .ins) which are part of the distribution.
%%
%% --------------- start of docstrip commands ------------------
%%
\def\filedate{1999/04/11}
\def\batchfile{english.ins}
\input docstrip.tex

{\ifx\generate\undefined
\Msg{**********************************************}
\Msg{*}
\Msg{* This installation requires docstrip}
\Msg{* version 2.3c or later.}
\Msg{*}
\Msg{* An older version of docstrip has been input}
\Msg{*}
\Msg{**********************************************}
\errhelp{Move or rename old docstrip.tex.}
\errmessage{Old docstrip in input path}
\batchmode
\csname @@end\endcsname
\fi}

\declarepreamble\mainpreamble
This is a generated file.

Copyright 1989-2005 Johannes L. Braams and any individual authors
listed elsewhere in this file.  All rights reserved.

This file was generated from file(s) of the Babel system.
---------------------------------------------------------

It may be distributed and/or modified under the
conditions of the LaTeX Project Public License, either version 1.3
of this license or (at your option) any later version.
The latest version of this license is in
  http://www.latex-project.org/lppl.txt
and version 1.3 or later is part of all distributions of LaTeX
version 2003/12/01 or later.

This work has the LPPL maintenance status "maintained".

The Current Maintainer of this work is Johannes Braams.

This file may only be distributed together with a copy of the Babel
system. You may however distribute the Babel system without
such generated files.

The list of all files belonging to the Babel distribution is
given in the file `manifest.bbl'. See also `legal.bbl for additional
information.

The list of derived (unpacked) files belonging to the distribution
and covered by LPPL is defined by the unpacking scripts (with
extension .ins) which are part of the distribution.
\endpreamble

\declarepreamble\fdpreamble
This is a generated file.

Copyright 1989-2005 Johannes L. Braams and any individual authors
listed elsewhere in this file.  All rights reserved.

This file was generated from file(s) of the Babel system.
---------------------------------------------------------

It may be distributed and/or modified under the
conditions of the LaTeX Project Public License, either version 1.3
of this license or (at your option) any later version.
The latest version of this license is in
  http://www.latex-project.org/lppl.txt
and version 1.3 or later is part of all distributions of LaTeX
version 2003/12/01 or later.

This work has the LPPL maintenance status "maintained".

The Current Maintainer of this work is Johannes Braams.

This file may only be distributed together with a copy of the Babel
system. You may however distribute the Babel system without
such generated files.

The list of all files belonging to the Babel distribution is
given in the file `manifest.bbl'. See also `legal.bbl for additional
information.

In particular, permission is granted to customize the declarations in
this file to serve the needs of your installation.

However, NO PERMISSION is granted to distribute a modified version
of this file under its original name.

\endpreamble

\keepsilent

\usedir{tex/generic/babel} 

\usepreamble\mainpreamble
\generate{\file{english.ldf}{\from{english.dtx}{code}}
          }
\usepreamble\fdpreamble

\ifToplevel{
\Msg{***********************************************************}
\Msg{*}
\Msg{* To finish the installation you have to move the following}
\Msg{* files into a directory searched by TeX:}
\Msg{*}
\Msg{* \space\space All *.def, *.fd, *.ldf, *.sty}
\Msg{*}
\Msg{* To produce the documentation run the files ending with}
\Msg{* '.dtx' and `.fdd' through LaTeX.}
\Msg{*}
\Msg{* Happy TeXing}
\Msg{***********************************************************}
}
 
\endinput
}
\DeclareOption{canadien}{% \iffalse meta-comment
%
% Copyright 1989-2009 Johannes L. Braams and any individual authors
% listed elsewhere in this file.  All rights reserved.
% 
% This file is part of the Babel system.
% --------------------------------------
% 
% It may be distributed and/or modified under the
% conditions of the LaTeX Project Public License, either version 1.3
% of this license or (at your option) any later version.
% The latest version of this license is in
%   http://www.latex-project.org/lppl.txt
% and version 1.3 or later is part of all distributions of LaTeX
% version 2003/12/01 or later.
% 
% This work has the LPPL maintenance status "maintained".
% 
% The Current Maintainer of this work is Johannes Braams.
% 
% The list of all files belonging to the Babel system is
% given in the file `manifest.bbl. See also `legal.bbl' for additional
% information.
% 
% The list of derived (unpacked) files belonging to the distribution
% and covered by LPPL is defined by the unpacking scripts (with
% extension .ins) which are part of the distribution.
% \fi
% \CheckSum{2135}
%
% \iffalse
%    Tell the \LaTeX\ system who we are and write an entry on the
%    transcript. Nothing to write to the .cfg file, if generated.
%<*dtx>
\ProvidesFile{frenchb.dtx}
%</dtx>
% \changes{v2.1d}{2008/05/04}{Argument of \cs{ProvidesLanguage} changed
%     from `french' to `frenchb', otherwise \cs{listfiles} prints
%     no date/version information.  The bug with \cs{listfiles}
%     (introduced in v.1.5!), was pointed out by Ulrike Fischer.}
%<code>\ProvidesLanguage{frenchb}
%\ProvidesFile{frenchb.dtx}
%<*!cfg>
        [2009/03/16 v2.3d French support from the babel system]
%</!cfg>
%<*cfg>
%% frenchb.cfg: configuration file for frenchb.ldf
%% Daniel Flipo Daniel.Flipo at univ-lille1.fr
%</cfg>
%%    File `frenchb.dtx'
%%    Babel package for LaTeX version 2e
%%    Copyright (C) 1989 - 2009
%%              by Johannes Braams, TeXniek
%
%<*!cfg>
%%    Frenchb language Definition File
%%    Copyright (C) 1989 - 2009
%%              by Johannes Braams, TeXniek
%%                 Daniel Flipo, GUTenberg
%
%%    Please report errors to: Daniel Flipo, GUTenberg
%%                             Daniel.Flipo at univ-lille1.fr
%</!cfg>
%
%    This file is part of the babel system, it provides the source
%    code for the French language definition file.
%
%<*filedriver>
\documentclass[a4paper]{ltxdoc}
\DeclareFontEncoding{T1}{}{}
\DeclareFontSubstitution{T1}{lmr}{m}{n}
\DeclareTextCommand{\guillemotleft}{OT1}{%
  {\fontencoding{T1}\fontfamily{lmr}\selectfont\char19}}
\DeclareTextCommand{\guillemotright}{OT1}{%
  {\fontencoding{T1}\fontfamily{lmr}\selectfont\char20}}
\newcommand*\TeXhax{\TeX hax}
\newcommand*\babel{\textsf{babel}}
\newcommand*\langvar{$\langle \mathit lang \rangle$}
\newcommand*\note[1]{}
\newcommand*\Lopt[1]{\textsf{#1}}
\newcommand*\file[1]{\texttt{#1}}
\begin{document}
\setlength{\parindent}{0pt}
\begin{center}
  \textbf{\Large A Babel language definition file for French}\\[3mm]^^A\]
  Daniel \textsc{Flipo}\\
  \texttt{Daniel.Flipo@univ-lille1.fr}
\end{center}
 \RecordChanges
 \DocInput{frenchb.dtx}
\end{document}
%</filedriver>
% \fi
% \GetFileInfo{frenchb.dtx}
%
%  \section{The French language}
%
%    The file \file{\filename}\footnote{The file described in this
%    section has version number \fileversion\ and was last revised on
%    \filedate.}, defines all the language definition macros for the
%    French language.
%
%    Customisation for the French language is achieved following the
%    book ``Lexique des r\`egles typographiques en usage \`a
%    l'Imprimerie nationale'' troisi\`eme \'edition (1994),
%    ISBN-2-11-081075-0.
%
%    First version released: 1.1 (1996/05/31) as part of
%    \babel-3.6beta.
%
%    |frenchb| has been improved using helpful suggestions from many
%    people, mainly from Jacques Andr\'e, Michel Bovani, Thierry Bouche,
%    and Vincent Jalby.  Thanks to all of them!
%
%    This new version (2.x) has been designed to be used with \LaTeXe{}
%    and Plain\TeX{} formats only. \LaTeX-2.09 is no longer supported.
%    Changes between version 1.6 and \fileversion{} are listed in
%    subsection~\ref{ssec-changes} p.~\pageref{ssec-changes}.
%
%    An extensive documentation is available in French here:\\
%    |http://daniel.flipo.free.fr/frenchb|
%
%  \subsection{Basic interface}
%
%    In a multilingual document, some typographic rules are language
%    dependent, i.e. spaces before `double punctuation' (|:| |;| |!|
%    |?|) in French, others concern the general layout (i.e. layout of
%    lists, footnotes, indentation of first paragraphs of sections) and
%    should apply to the whole document.
%
%    Starting with version~2.2, |frenchb| behaves differently according
%    to \babel's \emph{main language} defined as the \emph{last}
%    option\footnote{Its name is kept in \texttt{\textbackslash
%           bbl@main@language}.} at \babel's loading.  When French is
%    not \babel's main language, |frenchb| no longer alters the global
%    layout of the document (even in parts where French is the current
%    language): the layout of lists, footnotes, indentation of first
%    paragraphs of sections are not customised by |frenchb|.
%
%    When French is loaded as the last option of \babel, |frenchb|
%    makes the following changes to the global layout, \emph{both in
%    French and in all other languages}\footnote{%
%       For each item, hooks are provided to reset standard
%       \LaTeX{} settings or to emulate the behavior of former versions
%       of \texttt{frenchb} (see command
%       \texttt{\textbackslash frenchbsetup\{\}},
%       section~\ref{ssec-custom}).}:
%    \begin{enumerate}
%    \item the first paragraph of each section is indented
%          (\LaTeX{} only);
%    \item the default items in itemize environment are set to `--'
%          instead of `\textbullet', and all vertical spacing and glue
%          is deleted; it is possible to change `--' to something else
%          (`---' for instance) using |\frenchbsetup{}|;
%    \item vertical spacing in general \LaTeX{} lists is
%          shortened;
%    \item footnotes are displayed ``\`a la fran\c{c}aise''.
%    \end{enumerate}
%
%    Regarding local typography, the command |\selectlanguage{french}|
%    switches to the French language\footnote{%
%      \texttt{\textbackslash selectlanguage\{francais\}}
%      and \texttt{\textbackslash selectlanguage\{frenchb\}} are kept
%      for backward compatibility but should no longer be used.},
%    with the following effects:
%    \begin{enumerate}
%    \item French hyphenation patterns are made active;
%    \item `double punctuation' (|:| |;| |!| |?|) is made
%           active%\footnote{Actually, they are active in the whole
%           document, only their expansions differ in French and
%           outside French} for correct spacing in French;
%    \item |\today| prints the date in French;
%    \item the caption names are translated into French
%          (\LaTeX{} only);
%    \item the space after |\dots| is removed in French.
%    \end{enumerate}
%
%    Some commands are provided in |frenchb| to make typesetting
%    easier:
%    \begin{enumerate}
%    \item French quotation marks can be entered using the commands
%          |\og| and |\fg| which work in \LaTeXe and Plain\TeX,
%          their appearance depending on what is available to draw
%          them; even if you use \LaTeXe{} \emph{and} |T1|-encoding,
%          you should refrain from entering them as
%          |<<~French quotation marks~>>|: |\og| and |\fg| provide
%          better horizontal spacing.
%          |\og| and |\fg| can be used outside French, they typeset
%          then English quotes `` and ''.
%    \item A command |\up| is provided to typeset superscripts like
%          |M\up{me}| (abbreviation for ``Madame''), |1\up{er}| (for
%          ``premier'').  Other commands are also provided for
%          ordinals: |\ier|, |\iere|, |\iers|, |\ieres|, |\ieme|,
%          |\iemes| (|3\iemes| prints 3\textsuperscript{es}).
%    \item Family names should be typeset in small capitals and never
%          be hyphenated, the macro |\bsc| (boxed small caps) does
%          this, e.g., |Leslie~\bsc{Lamport}| will produce
%          Leslie~\mbox{\textsc{Lamport}}. Note that composed names
%          (such as Dupont-Durant) may now be hyphenated on explicit
%          hyphens, this differs from |frenchb|~v.1.x.
%    \item Commands |\primo|, |\secundo|, |\tertio| and |\quarto|
%          print 1\textsuperscript{o}, 2\textsuperscript{o},
%          3\textsuperscript{o}, 4\textsuperscript{o}.
%          |\FrenchEnumerate{6}| prints 6\textsuperscript{o}.
%    \item Abbreviations for ``Num\'ero(s)'' and ``num\'ero(s)''
%          (N\textsuperscript{o} N\textsuperscript{os}
%          n\textsuperscript{o} and n\textsuperscript{os}~)
%          are obtained via the commands |\No|, |\Nos|, |\no|, |\nos|.
%    \item Two commands are provided to typeset the symbol for
%          ``degr\'e'': |\degre| prints the raw character and
%          |\degres| should be used to typeset temperatures (e.g.,
%          ``|20~\degres C|'' with an unbreakable space), or for
%          alcohols' strengths (e.g., ``|45\degres|'' with \emph{no}
%          space in French).
%    \item In math mode the comma has to be surrounded with
%          braces to avoid a spurious space being inserted after it,
%          in decimal numbers for instance (see the \TeX{}book p.~134).
%          The command |\DecimalMathComma| makes the comma be an
%          ordinary character \emph{in French only} (no space added);
%          as a counterpart, if |\DecimalMathComma| is active, an
%          explicit space has to be added in lists and intervals:
%          |$[0,\ 1]$|, |$(x,\ y)$|. |\StandardMathComma| switches back
%          to the standard behaviour of the comma.
%    \item A command |\nombre| was provided in 1.x versions to easily
%          format numbers in slices of three digits separated either
%          by a comma in English or with a space in French; |\nombre|
%          is now mapped to |\numprint| from \file{numprint.sty}, see
%          \file{numprint.pdf} for more information.
%    \item |frenchb| has been designed to take advantage of the |xspace|
%          package if present: adding |\usepackage{xspace}| in the
%          preamble will force macros like |\fg|, |\ier|, |\ieme|,
%          |\dots|, \dots, to respect the spaces you type after them,
%          for instance typing `|1\ier juin|' will print
%          `1\textsuperscript{er} juin' (no need for a forced space
%          after |1\ier|).
%    \end{enumerate}
%
%  \subsection{Customisation}
%  \label{ssec-custom}
%
%     Up to version 1.6, customisation of |frenchb| was achieved
%     by entering commands in \file{frenchb.cfg}.  This possibility
%     remains for compatibility, but \emph{should not longer be used}.
%     Version 2.0 introduces a new command |\frenchbsetup{}| using
%     the \file{keyval} syntax which should make it easier to choose
%     among the many options available. The command |\frenchbsetup{}|
%     is to appear in the preamble only (after loading \babel).
%
%     \vspace{.5\baselineskip}
%     |\frenchbsetup{ShowOptions}| prints all available options to
%     the \file{.log} file, it is just meant as a remainder of the
%     list of offered options. As usual with \file{keyval} syntax,
%     boolean options (as |ShowOptions|) can be entered as
%     |ShowOptions=true| or just |ShowOptions|, the `|=true|' part
%     can be omitted.
%
%     \vspace{.5\baselineskip}
%     The other options are listed below. Their default value is shown
%     between brackets, sometimes followed be a `\texttt{*}'.
%     The `\texttt{*}' means that the default shown applies when
%     |frenchb| is loaded as the \emph{last} option of \babel{}
%     ---\babel's \emph{main language}---, and is toggled otherwise:
%     \begin{itemize}
%     \item |StandardLayout=true [false*]| forces |frenchb| not to
%       interfere with the layout: no action on any kind of lists,
%       first paragraphs of sections are not indented (as in English),
%       no action on footnotes. This option replaces the former
%       command |\StandardLayout|.  It can be used to avoid conflicts
%       with classes or packages which customise lists or footnotes.
%     \item |GlobalLayoutFrench=false [true*]| can be used, when French
%       is the main language, to emulate what prior versions of
%       |frenchb| (pre-2.2) did: lists, and first paragraphs
%       of sections will be displayed the standard way in other
%       languages than French, and ``\`a la fran\c{c}aise'' in French.
%       Note that the layout of footnotes is language independent
%       anyway (see below |FrenchFootnotes| and |AutoSpaceFootnotes|).
%       This option replaces the former command |\FrenchLayout|.
%     \item |ReduceListSpacing=false [true*]|; |frenchb| normally
%       reduces the values of the vertical spaces used in the
%       environment |list| in French; setting this option to |false|
%       reverts to the standard settings of |list|.  This option
%       replaces the former command |\FrenchListSpacingfalse|.
%     \item |CompactItemize=false [true*]|; |frenchb| normally
%       suppresses any vertical space between items of |itemize| lists
%       in French; setting this option to |false| reverts to the
%       standard settings of |itemize| lists.  This option replaces
%       the former command |\FrenchItemizeSpacingfalse|.
%     \item |StandardItemLabels=true [false*]| when set to |true| this
%       option stops |frenchb| from changing the labels in |itemize|
%       lists in French.
%     \item |ItemLabels=\textemdash|, |\textbullet|, |\ding{43}|,
%       \dots, |[\textendash*]|; when |StandardItemLabels=false| (the
%       default), this option enables to choose the label used in
%       |itemize| lists for all levels.  The next three options do
%       the same but each one for one level only. Note that the
%       example |\ding{43}| requires |\usepackage{pifont}|.
%     \item |ItemLabeli=\textemdash|, |\textbullet|, |\ding{43}|,
%       \dots,|[\textendash*]|
%     \item |ItemLabelii=\textemdash|, |\textbullet|, |\ding{43}|,
%       \dots, |[\textendash*]|
%     \item |ItemLabeliii=\textemdash|, |\textbullet|, |\ding{43}|,
%       \dots, |[\textendash*]|
%     \item |ItemLabeliv=\textemdash|, |\textbullet|, |\ding{43}|,
%       \dots, |[\textendash*]|
%     \item |StandardLists=true [false*]| forbids |frenchb| to
%       customise any kind of list. Do activate the option
%       |StandardLists| when using classes or packages that customise
%       lists too (|enumitem|, |paralist|, \dots{}) to avoid conflicts.
%       This option is just a shorthand for |ReduceListSpacing=false|
%       and |CompactItemize=false| and |StandardItemLabels=true|.
%     \item |IndentFirst=false [true*]|; |frenchb| normally forces
%       indentation of the first paragraph of sections.
%       When this option is set to |false|, the first paragraph of
%       will look the same in French and in English (not indented).
%     \item |FrenchFootnotes=false [true*]| reverts to the standard
%       layout of footnotes. By default |frenchb| typesets leading
%       numbers as `1.\hspace{.5em}' instead of `$\hbox{}^1$', but
%       has no effect on footnotes numbered with symbols (as in the
%       |\thanks| command).  The former commands |\StandardFootnotes|
%       and |\FrenchFootnotes| are still there, |\StandardFootnotes|
%       can be useful when some footnotes are numbered with letters
%       (inside minipages for instance).
%     \item |AutoSpaceFootnotes=false [true*]| ; by default |frenchb|
%       adds a thin space in the running text before the number or
%       symbol calling the footnote.  Making this option |false|
%       reverts to the standard setting (no space added).
%     \item |FrenchSuperscripts=false [true]| ; then
%       |\up=\textsuperscript| (option added in version 2.1).
%       Should only be made |false| to recompile older documents.
%       By default |\up| now relies on |\fup| designed to produce
%       better looking superscripts.
%     \item |AutoSpacePunctuation=false [true]|; in French, the user
%       \emph{should} input a space before the four characters `|:;!?|'
%       but as many people forget about it (even among native French
%       writers!), the default behaviour of |frenchb| is to
%       automatically add a |\thinspace| before `|;|' `|!|' `|?|' and a
%       normal (unbreakable) space before~`|:|' (this is recommended by
%       the French Imprimerie nationale).  This is convenient in most
%       cases but can lead to addition of spurious spaces in URLs or in
%       MS-DOS paths but only if they are no typed using |\texttt| or
%       verbatim mode. When the current font is a monospaced
%       (typewriter) font, |AutoSpacePunctuation| is locally switched
%       to |false|, no spurious space is added in that case, so the
%       default behaviour of of |frenchb| in that area should be fine
%       in most circumstances.
%
%       Choosing |AutoSpacePunctuation=false| will ensure that
%       a proper space will be added before `|:;!?|' \emph{if and only
%       if} a (normal) space has been typed in. Those who are unsure
%       about their typing in this area should stick to the default
%       option and type |\string;| |\string:| |\string!| |\string?|
%       instead of |;| |:| |!| |?| in case an unwanted space is
%       added by |frenchb|.
%     \item |ThinColonSpace=true [false]| changes the normal
%       (unbreakable) space added before the colon `:' to a thin space,
%       so that the same amount of space is added before any of the
%       four double punctuation characters. The default setting is
%       supported by the French Imprimerie nationale.
%     \item |LowercaseSuperscripts=false [true]| ; by default |frenchb|
%       inhibits the uppercasing of superscripts (for instance when they
%       are moved to page headers). Making this option |false|
%       will disable this behaviour (not recommended).
%     \item |PartNameFull=false [true]|; when true, |frenchb| numbers
%       the title of |\part{}| commands as ``Premi\`ere partie'',
%       ``Deuxi\`eme partie'' and so on. With some classes which change
%       the|\part{}| command (AMS and SMF classes do so), you will get
%       ``Premi\`ere partie~I'', ``Deuxi\`eme partie~II'' instead;
%       when this occurs, this option should be set to |false|,
%       part titles will then be printed as ``Partie I'', ``Partie II''.
%     \item |og=|\texttt{\guillemotleft}, |fg=|\texttt{\guillemotright};
%       when guillemets characters are available on the keyboard
%       (through a compose key for instance), it is nice to use them
%       instead of typing |\og| and |\fg|. This option tells |frenchb|
%       which characters are opening and closing French guillemets
%       (they depend on the input encoding), then you can type either
%       \texttt{\guillemotleft{} guillemets \guillemotright}, or
%       \texttt{\guillemotleft{}guillemets\guillemotright} (with or
%       without spaces), to get properly typeset French quotes.
%       This option requires \file{inputenc} to be loaded with the
%       proper encoding, it works with 8-bits encodings (latin1,
%       latin9, ansinew,  applemac,\dots) and multi-byte encodings
%       (utf8 and utf8x).
%     \end{itemize}
%
%  \subsection{Hyphenation checks}
%  \label{ssec-hyphen}
%
%    Once you have built your format, a good precaution would be to
%    perform some basic tests about hyphenation in French. For
%    \LaTeXe{} I suggest this:
%    \begin{itemize}
%    \item run the following file, with the encoding suitable for
%      your machine (\textit{my-encoding} will be |latin1| for
%      \textsc{unix} machines, |ansinew| for PCs running~Windows,
%      |applemac| or |latin1| for Macintoshs, or |utf8|\dots\\[3mm]^^A\]
%      |%%% Test file for French hyphenation.|\\
%      |\documentclass{article}|\\
%      |\usepackage[|\textit{my-encoding}|]{inputenc}|\\
%      |\usepackage[T1]{fontenc} % Use LM fonts|\\
%      |\usepackage{lmodern}     % for French|\\
%      |\usepackage[frenchb]{babel}|\\
%      |\begin{document}|\\
%      |\showhyphens{signal container \'ev\'enement alg\`ebre}|\\
%      |\showhyphens{|\texttt{signal container \'ev\'enement
%                     alg\`ebre}|}|\\
%      |\end{document}|
%    \item check the hyphenations proposed by \TeX{} in your log-file;
%      in French you should get with both 7-bit and 8-bit encodings\\
%      \texttt{si-gnal contai-ner \'ev\'e-ne-ment al-g\`ebre}.\\
%      Do not care about how accented characters are displayed in the
%      log-file, what matters is the position of the `|-|' hyphen
%      signs \emph{only}.
%    \end{itemize}
%    If they are all correct, your installation (probably) works fine,
%    if one (or more) is (are) wrong, ask a local wizard to see what's
%    going wrong and perform the test again (or e-mail me about what
%    happens).\\
%    Frequent mismatches:
%    \begin{itemize}
%    \item you get |sig-nal con-tainer|, this probably means that the
%    hyphenation patterns you are using are for US-English, not for
%    French;
%    \item you get no hyphen at all in \texttt{\'ev\'e-ne-ment}, this
%    probably means that you are using CM fonts and the macro
%    |\accent| to produce accented characters.
%    Using 8-bits fonts with built-in accented characters avoids
%    this kind of mismatch.
%    \end{itemize}
%
%    \textbf{Options' order} -- Please remember that options are read
%    in the order they appear inside the |\frenchbsetup| command.
%    Someone wishing that |frenchb| leaves the layout of lists
%    and footnotes untouched but caring for indentation of first
%    paragraph of sections could choose
%    |\frenchbsetup{StandardLayout,IndentFirst}| and get the expected
%    layout. Choosing |\frenchbsetup{IndentFirst,StandardLayout}|
%    would not lead to the expected result: option |IndentFirst| would
%    be overwritten by |StandardLayout|.
%
%  \subsection{Changes}
%  \label{ssec-changes}
%
%  \subsubsection*{What's new in version 2.0?}
%
%    Here is the list of all changes:
%    \begin{itemize}
%    \item Support for \LaTeX-2.09 and for \LaTeXe{} in compatibility
%      mode has been dropped. This version is meant for \LaTeXe{} and
%      Plain based formats (like \file{bplain}). \LaTeXe{} formats
%      based on ml\TeX{} are no longer supported either (plenty of
%      good 8-bits fonts are available now, so T1 encoding should
%      be preferred for typesetting in French). A warning is issued
%      when OT1 encoding is in use at the |\begin{document}|.
%    \item Customisation should now be handled by command
%      |\frenchbsetup{}|, \file{frenchb.cfg} (kept for compatibility)
%      should no longer be used. See section~\ref{ssec-custom} for
%      the list of available options.
%    \item Captions in figures and table have changed in French: former
%      abbreviations ``Fig.'' and ``Tab.'' have been replaced by full
%      names ``Figure'' and ``Table''.  If this leads to formatting
%      problems in captions, you can add the following two commands to
%      your preamble (after loading \babel) to get the former captions\\
%      |\addto\captionsfrench{\def\figurename{{\scshape Fig.}}}|\\
%      |\addto\captionsfrench{\def\tablename{{\scshape Tab.}}}|.
%    \item The |\nombre| command is now provided by the \file{numprint}
%      package which has to be loaded \emph{after} \babel{} with the
%      option |autolanguage| if number formatting should depend on the
%      current language.
%    \item The |\bsc| command no longer uses an |\hbox| to stop
%      hyphenation of names but a |\kern0pt| instead. This change
%      enables \file{microtype} to fine tune the length of the
%      argument of |\bsc|; as a side-effect, compound names like
%      Dupont-Durand can now be hyphenated on  explicit hyphens.
%      You can get back to the former behaviour of |\bsc| by adding\\
%      |\renewcommand*{\bsc}[1]{\leavevmode\hbox{\scshape #1}}|\\
%      to the preamble of your document.
%    \item Footnotes are now displayed ``\`a la fran\c caise'' for the
%      whole document, except with an explicit\\
%      |\frenchbsetup{AutoSpaceFootnotes=false,FrenchFootnotes=false}|.\\
%      Add this command if you want standard footnotes. It is still
%      possible to revert locally to the standard layout of footnotes
%      by adding |\StandardFootnotes| (inside a |minipage| environment
%      for instance).
%    \end{itemize}
%
%  \subsubsection*{What's new in version 2.1?}
%
%      New command |\fup| to typeset better looking superscripts.
%      Former command |\up| is now defined as |\fup|, but an option
%      |\frenchbsetup{FrenchSuperscripts=false}| is provided for
%      backward compatibility.  |\fup| was designed using ideas from
%      Jacques Andr\'e, Thierry Bouche and Ren\'e Fritz, thanks to them!
%
%  \subsubsection*{What's new in version 2.2?}
%
%      Starting with version~2.2a, |frenchb| alters the layout of
%      lists, footnotes, and the indentation of first paragraphs of
%      sections) \emph{only if} French is the ``main language''
%      (i.e. babel's last language option). The layout is global for
%      the whole document: lists, etc. look the same in French and in
%      other languages, everything is typeset ``\`a la fran\c caise''
%      if French is the ``main language'', otherwise |frenchb| doesn't
%      change anything regarding lists, footnotes, and indentation of
%      paragraphs.
%
%  \subsubsection*{What's new in version 2.3?}
%
%      Starting with version~2.3a, |frenchb| no longer inserts spaces
%      automatically before `|:;!?|' when a typewriter font is in use;
%      this was suggested by Yannis Haralambous to prevent
%      spurious spaces in computer source code or expressions like
%      \texttt{C\string:/foo}, \texttt{http\string://foo.bar},
%      etc.  An option (|OriginalTypewriter|) is provided to get back
%      to the former behaviour of |frenchb|.
%
%      Another probably invisible change: lowercase conversion in
%      |\up{}| is now achieved by the \LaTeX{} command |\MakeLowercase|
%      instead of \TeX's |\lowercase| command.  This prevents error
%      messages when diacritics are used inside |\up{}| (diacritics
%      should \emph{never} be used in superscripts though!).
%
% \StopEventually{}
%
%  \subsection{File frenchb.cfg}
%  \label{sec-cfg}
%
%    \file{frenchb.cfg} is now a dummy file just kept for compatibility
%    with previous versions.
%
% \iffalse
%<*cfg>
% \fi
%    \begin{macrocode}
%%%%%%%%%%%%%%%%%%%%%%%%%%%%%%%%%%%%%%%%%%%%%%%%%%%%%%%%%%%%%%%%%%%%%%
%%%%%%%%%  WARNING: THIS  FILE SHOULD  NO  LONGER  BE  USED  %%%%%%%%%
%% If you want to customise frenchb, please DO NOT hack into the code!
%% Do no put any code in this file either, please use the new command
%% \frenchbsetup{} with the proper options to customise frenchb.
%% 
%% Add \frenchbsetup{ShowOptions} to your preamble to see the list of
%% available options and/or read the documentation.
%%%%%%%%%%%%%%%%%%%%%%%%%%%%%%%%%%%%%%%%%%%%%%%%%%%%%%%%%%%%%%%%%%%%%%
%    \end{macrocode}
% \iffalse
%</cfg>
% \fi
%
%  \section{\TeX{}nical details}
%
%  \subsection{Initial setup}
%
% \changes{v2.1d}{2008/05/02}{Argument of \cs{ProvidesLanguage} changed
%     above from `french' to `frenchb' (otherwise \cs{listfiles} prints
%     no date/version information).  The real name of current language
%     (french) as to be corrected before calling \cs{LdfInit}.}
%
% \iffalse
%<*code>
% \fi
%
%    While this file was read through the option \Lopt{frenchb} we make
%    it behave as if \Lopt{french} was specified.
%    \begin{macrocode}
\def\CurrentOption{french}
%    \end{macrocode}
%
%    The macro |\LdfInit| takes care of preventing that this file is
%    loaded more than once, checking the category code of the
%    \texttt{@} sign, etc.
%
%    \begin{macrocode}
\LdfInit\CurrentOption\datefrench
%    \end{macrocode}
%
% \changes{v2.1d}{2008/05/04}{Avoid warning ``\cs{end} occurred
%   when \cs{ifx} ... incomplete'' with LaTeX-2.09.}
%
%  \begin{macro}{\ifLaTeXe}
%    No support is provided for late \LaTeX-2.09: issue a warning
%    and exit if \LaTeX-2.09 is in use. Plain is still supported.
%    \begin{macrocode}
\newif\ifLaTeXe
\let\bbl@tempa\relax
\ifx\magnification\@undefined
   \ifx\@compatibilitytrue\@undefined
     \PackageError{frenchb.ldf}
        {LaTeX-2.09 format is no longer supported.\MessageBreak
         Aborting here}
        {Please upgrade to LaTeX2e!}
     \let\bbl@tempa\endinput
   \else
     \LaTeXetrue
   \fi
\fi
\bbl@tempa
%    \end{macrocode}
%  \end{macro}
%
%    Check if hyphenation patterns for the French language have been
%    loaded in language.dat; we allow for the names `french',
%    `francais', `canadien' or `acadian'. The latter two are both
%    names used in Canada for variants of French that are in use in
%    that country.
%
%    \begin{macrocode}
\ifx\l@french\@undefined
  \ifx\l@francais\@undefined
    \ifx\l@canadien\@undefined
      \ifx\l@acadian\@undefined
        \@nopatterns{French}
        \adddialect\l@french0
      \else
        \let\l@french\l@acadian
      \fi
    \else
      \let\l@french\l@canadien
    \fi
  \else
    \let\l@french\l@francais
  \fi
\fi
%    \end{macrocode}
%    Now |\l@french| is always defined.
%
%    The internal name for the French language is |french|;
%    |francais| and |frenchb| are synonymous for |french|:
%    first let both names use the same hyphenation patterns.
%    Later we will have to set aliases for |\captionsfrench|,
%    |\datefrench|, |\extrasfrench| and |\noextrasfrench|.
%    As French uses the standard values of |\lefthyphenmin| (2)
%    and |\righthyphenmin| (3), no special setting is required here.
%
%    \begin{macrocode}
\ifx\l@francais\@undefined
  \let\l@francais\l@french
\fi
\ifx\l@frenchb\@undefined
  \let\l@frenchb\l@french
\fi
%    \end{macrocode}
%    When this language definition file was loaded for one of the
%    Canadian versions of French we need to make sure that a suitable
%    hyphenation pattern register will be found by \TeX.
%    \begin{macrocode}
\ifx\l@canadien\@undefined
  \let\l@canadien\l@french
\fi
\ifx\l@acadian\@undefined
  \let\l@acadian\l@french
\fi
%    \end{macrocode}
%
%    This language definition can be loaded for different variants of
%    the French language. The `key' \babel\ macros are only defined
%    once, using `french' as the language name, but |frenchb| and
%    |francais| are synonymous.
%    \begin{macrocode}
\def\datefrancais{\datefrench}
\def\datefrenchb{\datefrench}
\def\extrasfrancais{\extrasfrench}
\def\extrasfrenchb{\extrasfrench}
\def\noextrasfrancais{\noextrasfrench}
\def\noextrasfrenchb{\noextrasfrench}
%    \end{macrocode}
%
% \begin{macro}{\extrasfrench}
% \begin{macro}{\noextrasfrench}
%    The macro |\extrasfrench| will perform all the extra
%    definitions needed for the French language.
%    The macro |\noextrasfrench| is used to cancel the actions of
%    |\extrasfrench|.\\
%    In French, character ``apostrophe'' is a letter in expressions
%    like |l'ambulance| (French  hyphenation patterns provide entries
%    for this kind of words).  This means that the |\lccode| of
%    ``apostrophe'' has to be non null in French for proper hyphenation
%    of those expressions, and has to be reset to null when exiting
%    French.
%
%    \begin{macrocode}
\@namedef{extras\CurrentOption}{\lccode`\'=`\'}
\@namedef{noextras\CurrentOption}{\lccode`\'=0}
%    \end{macrocode}
% \end{macro}
% \end{macro}
%
%    One more thing |\extrasfrench| needs to do is to make sure that
%    |\frenchspacing| is in effect.  |\noextrasfrench| will switch
%    |\frenchspacing| off again.
%    \begin{macrocode}
  \expandafter\addto\csname extras\CurrentOption\endcsname{%
    \bbl@frenchspacing}
  \expandafter\addto\csname noextras\CurrentOption\endcsname{%
    \bbl@nonfrenchspacing}
%    \end{macrocode}
%
%  \subsection{Punctuation}
%  \label{sec-punct}
%
%    As long as no better solution is available%
%    \footnote{Lua\TeX, or pdf\TeX{} might provide alternatives in
%       the future\dots},
%    the `double punctuation' characters (|;| |!| |?| and |:|) have to
%    be made |\active| for an automatic control of the amount of space
%    to insert before them. Before doing so, we have to save the
%    standard definition of |\@makecaption| (which includes two ':')
%    to compare it later to its definition at the |\begin{document}|.
%    \begin{macrocode}
\long\def\STD@makecaption#1#2{%
  \vskip\abovecaptionskip
  \sbox\@tempboxa{#1: #2}%
  \ifdim \wd\@tempboxa >\hsize
    #1: #2\par
  \else
    \global \@minipagefalse
    \hb@xt@\hsize{\hfil\box\@tempboxa\hfil}%
  \fi
  \vskip\belowcaptionskip}%
%    \end{macrocode}
%
%    We define a new `if' |\FBpunct@active| which will be made false
%    whenever a better alternative will be available. Another `if'
%    |\FBAutoSpacePunctuation| needs to be defined now.
%    \begin{macrocode}
\newif\ifFBpunct@active          \FBpunct@activetrue
\newif\ifFBAutoSpacePunctuation  \FBAutoSpacePunctuationtrue
%    \end{macrocode}
%    The following code makes the four characters |;| |!| |?| and |:|
%    `active' and provides their definitions.
%    \begin{macrocode}
\ifFBpunct@active
  \initiate@active@char{:}
  \initiate@active@char{;}
  \initiate@active@char{!}
  \initiate@active@char{?}
%    \end{macrocode}
%    We first tune the amount of space before \texttt{;}
%    \texttt{!}  \texttt{?} and \texttt{:}.  This should only happen
%    in horizontal mode, hence the test |\ifhmode|.
%
%    In horizontal mode, if a space has been typed before `;' we
%    remove it and put an unbreakable |\thinspace| instead. If no
%    space has been typed, we add |\FDP@thinspace| which will be
%    defined, up to the user's wishes, as an automatic added
%    thin space, or as |\@empty|.
%    \begin{macrocode}
  \declare@shorthand{french}{;}{%
      \ifhmode
      \ifdim\lastskip>\z@
          \unskip\penalty\@M\thinspace
          \else
            \FDP@thinspace
        \fi
      \fi
%    \end{macrocode}
%    Now we can insert a |;| character.
%    \begin{macrocode}
      \string;}
%    \end{macrocode}
%    The next three definitions are very similar.
%    \begin{macrocode}
  \declare@shorthand{french}{!}{%
      \ifhmode
        \ifdim\lastskip>\z@
          \unskip\penalty\@M\thinspace
        \else
          \FDP@thinspace
        \fi
      \fi
      \string!}
  \declare@shorthand{french}{?}{%
      \ifhmode
        \ifdim\lastskip>\z@
          \unskip\penalty\@M\thinspace
        \else
          \FDP@thinspace
        \fi
      \fi
      \string?}
%    \end{macrocode}
%    According to the I.N. specifications, the `:' requires a normal
%    space before it, but some people prefer a |\thinspace| (just
%    like the other three). We define |\Fcolonspace| to hold the
%    required amount of space (user customisable).
%    \begin{macrocode}
  \newcommand*{\Fcolonspace}{\space}
  \declare@shorthand{french}{:}{%
      \ifhmode
        \ifdim\lastskip>\z@
          \unskip\penalty\@M\Fcolonspace
        \else
          \FDP@colonspace
        \fi
      \fi
      \string:}
%    \end{macrocode}
%
% \changes{v2.3a}{2008/10/10}{\cs{NoAutoSpaceBeforeFDP} and
%    \cs{AutoSpaceBeforeFDP} now set the flag
%    \cs{ifFBAutoSpacePunctuation} accordingly (LaTeX only).}
%
%  \begin{macro}{\AutoSpaceBeforeFDP}
%  \begin{macro}{\NoAutoSpaceBeforeFDP}
%    |\FDP@thinspace| and |\FDP@space| are defined as unbreakable
%    spaces by |\autospace@beforeFDP| or as |\@empty| by
%    |\noautospace@beforeFDP| (internal commands), user commands
%    |\AutoSpaceBeforeFDP| and |\NoAutoSpaceBeforeFDP| do the same and
%    take care of the flag |\ifFBAutoSpacePunctuation| in \LaTeX{}.
%    Set the default now for Plain (done later for \LaTeX).
%    \begin{macrocode}
  \def\autospace@beforeFDP{%
          \def\FDP@thinspace{\penalty\@M\thinspace}%
          \def\FDP@colonspace{\penalty\@M\Fcolonspace}}
  \def\noautospace@beforeFDP{\let\FDP@thinspace\@empty
                            \let\FDP@colonspace\@empty}
  \ifLaTeXe
    \def\AutoSpaceBeforeFDP{\autospace@beforeFDP
                            \FBAutoSpacePunctuationtrue}
    \def\NoAutoSpaceBeforeFDP{\noautospace@beforeFDP
                              \FBAutoSpacePunctuationfalse}
  \else
    \let\AutoSpaceBeforeFDP\autospace@beforeFDP
    \let\NoAutoSpaceBeforeFDP\noautospace@beforeFDP
    \AutoSpaceBeforeFDP
  \fi
%    \end{macrocode}
% \end{macro}
% \end{macro}
%
% \changes{v2.3a}{2008/10/10}{In LaTeX, frenchb no longer adds spaces
%     before `double punctuation' characters in computer code.
%     Suggested by Yannis Haralambous.}
%
% \changes{v2.3c}{2009/02/07}{Commands \cs{ttfamily}, \cs{rmfamily}
%    and \cs{sffamily} have to be robust.  Bug introduced in 2.3a,
%    pointed out by Manuel P\'egouri\'e-Gonnard.}
%
%    In \LaTeXe{} |\ttfamily| (and hence |\texttt|) will be redefined
%    `AtBeginDocument' as |\ttfamilyFB| so that no space
%    is added before the four |; : ! ?| characters, even if
%    |AutoSpacePunctuation| is true.  |\rmfamily| and |\sffamily| need
%    to be redefined also (|\ttfamily| is not always used inside a
%    group, its effect can be cancelled by |\rmfamily| or |\sffamily|).
%
%    These redefinitions can be canceled if necessary, for instance to
%    recompile older documents, see option |OriginalTypewriter| below.
%    \begin{macrocode}
  \ifLaTeXe
    \let\ttfamilyORI\ttfamily
    \let\rmfamilyORI\rmfamily
    \let\sffamilyORI\sffamily
    \DeclareRobustCommand\ttfamilyFB{%
         \noautospace@beforeFDP\ttfamilyORI}%
    \DeclareRobustCommand\rmfamilyFB{%
         \ifFBAutoSpacePunctuation
            \autospace@beforeFDP\rmfamilyORI
         \else
            \noautospace@beforeFDP\rmfamilyORI
         \fi}%
    \DeclareRobustCommand\sffamilyFB{%
         \ifFBAutoSpacePunctuation
            \autospace@beforeFDP\sffamilyORI
         \else
            \noautospace@beforeFDP\sffamilyORI
         \fi}%
  \fi
%    \end{macrocode}
%
%    When the active characters appear in an environment where their
%    French behaviour is not wanted they should give an `expected'
%    result. Therefore we define shorthands at system level as well.
%    \begin{macrocode}
  \declare@shorthand{system}{:}{\string:}
  \declare@shorthand{system}{!}{\string!}
  \declare@shorthand{system}{?}{\string?}
  \declare@shorthand{system}{;}{\string;}
%    \end{macrocode}
%    We specify that the French group of shorthands should be used.
%    \begin{macrocode}
  \addto\extrasfrench{%
    \languageshorthands{french}%
%    \end{macrocode}
%    These characters are `turned on' once, later their definition may
%    vary. Don't misunderstand the following code: they keep being
%    active all along the document, even when leaving French.
%    \begin{macrocode}
    \bbl@activate{:}\bbl@activate{;}%
    \bbl@activate{!}\bbl@activate{?}%
  }
  \addto\noextrasfrench{%
  \bbl@deactivate{:}\bbl@deactivate{;}%
  \bbl@deactivate{!}\bbl@deactivate{?}}
\fi
%    \end{macrocode}
%
%  \subsection{French quotation marks}
%
%  \begin{macro}{\og}
%  \begin{macro}{\fg}
%    The top macros for quotation marks will be called |\og|
%    (``\underline{o}uvrez \underline{g}uillemets'') and |\fg|
%    (``\underline{f}ermez \underline{g}uillemets'').
%    Another option for typesetting quotes in multilingual texts
%    is to use the package |csquotes.sty| and its command |\enquote|.
%
%    \begin{macrocode}
\newcommand*{\og}{\@empty}
\newcommand*{\fg}{\@empty}
%    \end{macrocode}
%  \end{macro}
%  \end{macro}
%
%  \begin{macro}{\guillemotleft}
%  \begin{macro}{\guillemotright}
%    \LaTeX{} users are supposed to use 8-bit output encodings (T1,
%    LY1,\dots) to typeset French, those who still stick to OT1 should
%    call |aeguill.sty| or a similar package. In both cases the
%    commands |\guillemotleft| and |\guillemotright| will print the
%    French opening and closing quote characters from the output font.
%    For XeLaTeX, |\guillemotleft| and |\guillemotright| are defined
%    by package \file{xunicode.sty}.
%    We will check `AtBeginDocument' that the proper output encodings
%    are in use (see end of section~\ref{sec-keyval}).
%
%    We give the following definitions for Plain users only as a (poor)
%    fall-back, they are welcome to change them for anything better.
%    \begin{macrocode}
\ifLaTeXe
\else
  \ifx\guillemotleft\@undefined
    \def\guillemotleft{\leavevmode\raise0.25ex
                       \hbox{$\scriptscriptstyle\ll$}}
  \fi
  \ifx\guillemotright\@undefined
    \def\guillemotright{\raise0.25ex
                        \hbox{$\scriptscriptstyle\gg$}}
  \fi
  \let\xspace\relax
\fi
%    \end{macrocode}
%  \end{macro}
%  \end{macro}
%
%    The next step is to provide correct spacing after |\guillemotleft|
%    and before |\guillemotright|: a space precedes and follows
%    quotation marks but no line break is allowed neither \emph{after}
%    the opening one, nor \emph{before} the closing one.
%    |\FBguill@spacing| which does the spacing, has been fine tuned by
%    Thierry Bouche.  French quotes (including spacing) are printed by
%    |\FB@og| and |\FB@fg|, the expansion of the top level commands
%    |\og| and |\og| is different in and outside French.
%    We'll try to be smart to users of David~Carlisle's |xspace|
%    package: if this package is loaded there will be no need for |{}|
%    or |\ | to get a space after |\fg|, otherwise |\xspace| will be
%    defined as |\relax| (done at the end of this file).
%
%    \begin{macrocode}
\newcommand*{\FBguill@spacing}{\penalty\@M\hskip.8\fontdimen2\font
                                            plus.3\fontdimen3\font
                                           minus.8\fontdimen4\font}
\DeclareRobustCommand*{\FB@og}{\leavevmode
                               \guillemotleft\FBguill@spacing}
\DeclareRobustCommand*{\FB@fg}{\ifdim\lastskip>\z@\unskip\fi
                               \FBguill@spacing\guillemotright\xspace}
%    \end{macrocode}
%
%    The top level definitions for French quotation marks are switched
%    on and off through the |\extrasfrench| |\noextrasfrench|
%    mechanism. Outside French, |\og| and |\fg| will typeset standard
%    English opening and closing double quotes.
%
%    \begin{macrocode}
\ifLaTeXe
  \def\bbl@frenchguillemets{\renewcommand*{\og}{\FB@og}%
                            \renewcommand*{\fg}{\FB@fg}}
  \def\bbl@nonfrenchguillemets{\renewcommand*{\og}{\textquotedblleft}%
            \renewcommand*{\fg}{\ifdim\lastskip>\z@\unskip\fi
                                   \textquotedblright}}
\else
   \def\bbl@frenchguillemets{\let\og\FB@og
                             \let\fg\FB@fg}
   \def\bbl@nonfrenchguillemets{\def\og{``}%
                     \def\fg{\ifdim\lastskip>\z@\unskip\fi ''}}
\fi
\expandafter\addto\csname extras\CurrentOption\endcsname{%
  \bbl@frenchguillemets}
\expandafter\addto\csname noextras\CurrentOption\endcsname{%
  \bbl@nonfrenchguillemets}
%    \end{macrocode}
%
%  \subsection{Date in French}
%
% \begin{macro}{\datefrench}
%    The macro |\datefrench| redefines the command |\today| to
%    produce French dates.
%
% \changes{v2.0}{2006/11/06}{2 '\cs{relax}' added in
%    \cs{today}'s definition.}
%
% \changes{v2.1a}{2008/03/25}{\cs{today} changed (correction in 2.0
%    was wrong: \cs{today} was printed without spaces in toc).}
%
%    \begin{macrocode}
\@namedef{date\CurrentOption}{%
  \def\today{{\number\day}\ifnum1=\day {\ier}\fi \space
    \ifcase\month
      \or janvier\or f\'evrier\or mars\or avril\or mai\or juin\or
      juillet\or ao\^ut\or septembre\or octobre\or novembre\or
      d\'ecembre\fi
    \space \number\year}}
%    \end{macrocode}
% \end{macro}
%
%  \subsection{Extra utilities}
%
%    Let's provide the French user with some extra utilities.
%
% \changes{v2.1a}{2008/03/24}{Command \cs{fup} added to produce
%    better superscripts than \cs{textsuperscript}.}
%
%  \begin{macro}{\up}
%
% \changes{v2.1c}{2008/04/29}{Provide a temporary definition
%    (hyperref safe) of \cs{up} in case it has to be expanded in
%    the preamble (by beamer's \cs{title} command for instance).}
%
%  \begin{macro}{\fup}
%
% \changes{v2.1b}{2008/04/02}{Command \cs{fup} changed to use
%    real superscripts from fourier v. 1.6.}
%
% \changes{v2.2a}{2008/05/08}{\cs{newif} and \cs{newdimen} moved
%    before \cs{ifLaTeXe} to avoid an error with plainTeX.}
%
% \changes{v2.3a}{2008/09/30}{\cs{lowercase} changed to
%    \cs{MakeLowercase} as the former doesn't work for non ASCII
%    characters in encodings like applemac, utf-8,\dots}
%
%    |\up| eases the typesetting of superscripts like
%    `1\textsuperscript{er}'.  Up to version 2.0 of |frenchb| |\up| was
%    just a shortcut for |\textsuperscript| in \LaTeXe, but several
%    users complained that |\textsuperscript| typesets superscripts
%    too high and too big, so we now define |\fup| as an attempt to
%    produce better looking superscripts.  |\up| is defined as |\fup|
%    but can be redefined by |\frenchbsetup{FrenchSuperscripts=false}|
%    as |\textsuperscript| for compatibility with previous versions.
%
%    When a font has built-in superscripts, the best thing to do is
%    to just use them, otherwise |\fup| has to simulate superscripts
%    by scaling and raising ordinary letters.  Scaling is done using
%    package \file{scalefnt} which will be loaded at the end of
%    \babel's loading (|frenchb| being an option of babel, it cannot
%    load a package while being read).
%
%    \begin{macrocode}
\newif\ifFB@poorman
\newdimen\FB@Mht
\ifLaTeXe
  \AtEndOfPackage{\RequirePackage{scalefnt}}
%    \end{macrocode}
%    |\FB@up@fake| holds the definition of fake superscripts.
%    The scaling ratio is 0.65, raising is computed to put the top of
%    lower case letters (like `m') just under the top  of upper case
%    letters (like `M'), precisely 12\% down.  The chosen settings
%    look correct for most fonts, but can be tuned by the end-user
%    if necessary by changing |\FBsupR| and |\FBsupS| commands.
%
%    |\FB@lc| is defined as |\MakeLowercase| to inhibit the uppercasing
%    of superscripts (this may happen in page headers with the standard
%    classes but is wrong); |\FB@lc| can be redefined to do nothing
%    by option |LowercaseSuperscripts=false| of |\frenchbsetup{}|.
%    \begin{macrocode}
  \newcommand*{\FBsupR}{-0.12}
  \newcommand*{\FBsupS}{0.65}
  \newcommand*{\FB@lc}[1]{\MakeLowercase{#1}}
  \DeclareRobustCommand*{\FB@up@fake}[1]{%
    \settoheight{\FB@Mht}{M}%
    \addtolength{\FB@Mht}{\FBsupR \FB@Mht}%
    \addtolength{\FB@Mht}{-\FBsupS ex}%
    \raisebox{\FB@Mht}{\scalefont{\FBsupS}{\FB@lc{#1}}}%
    }
%    \end{macrocode}
%    The only packages I currently know to take advantage of real
%    superscripts are a) \file{xltxtra} used in conjunction with
%    XeLaTeX and OpenType fonts having the font feature
%    'VerticalPosition=Superior' (\file{xltxtra} defines
%    |\realsuperscript| and |\fakesuperscript|) and b) \file{fourier}
%    (from version 1.6) when Expert Utopia fonts are available.
%
%    |\FB@up| checks whether the current font is a Type1 `Expert'
%    (or `Pro') font with real superscripts or not (the code works
%    currently only with \file{fourier-1.6} but could work with any
%    Expert Type1 font with built-in superscripts, see below), and
%    decides to use real or fake superscripts.
%    It works as follows: the content of |\f@family| (family name of
%    the current font) is split by |\FB@split| into two pieces, the
%    first three characters (`|fut|' for Fourier, `|ppl|' for Adobe's
%    Palatino, \dots) stored in |\FB@firstthree| and the rest stored
%    in |\FB@suffix| which is expected to be `|x|' or `|j|' for expert
%    fonts.
%    \begin{macrocode}
  \def\FB@split#1#2#3#4\@nil{\def\FB@firstthree{#1#2#3}%
                             \def\FB@suffix{#4}}
  \def\FB@x{x}
  \def\FB@j{j}
  \DeclareRobustCommand*{\FB@up}[1]{%
    \bgroup \FB@poormantrue
      \expandafter\FB@split\f@family\@nil
%    \end{macrocode}
%    Then |\FB@up| looks for a \file{.fd} file named \file{t1fut-sup.fd}
%    (Fourier) or \file{t1ppl-sup.fd} (Palatino), etc. supposed to
%    define the subfamily (|fut-sup| or |ppl-sup|, etc.) giving access
%    to the built-in superscripts.  If the \file{.fd} file is not found
%    by |\IfFileExists|, |\FB@up| falls back on fake superscripts,
%    otherwise |\FB@suffix| is checked to decide whether to use fake or
%    real superscripts.
%    \begin{macrocode}
      \edef\reserved@a{\lowercase{%
         \noexpand\IfFileExists{\f@encoding\FB@firstthree -sup.fd}}}%
      \reserved@a
        {\ifx\FB@suffix\FB@x \FB@poormanfalse\fi
         \ifx\FB@suffix\FB@j \FB@poormanfalse\fi
         \ifFB@poorman \FB@up@fake{#1}%
         \else         \FB@up@real{#1}%
         \fi}%
        {\FB@up@fake{#1}}%
    \egroup}
%    \end{macrocode}
%    |\FB@up@real| just picks up the superscripts from the subfamily
%    (and forces lowercase).
%    \begin{macrocode}
  \newcommand*{\FB@up@real}[1]{\bgroup
       \fontfamily{\FB@firstthree -sup}\selectfont \FB@lc{#1}\egroup}
%    \end{macrocode}
%    |\fup| is now defined as |\FB@up| unless |\realsuperscript| is
%    defined (occurs with XeLaTeX calling \file{xltxtra.sty}).
%    \begin{macrocode}
  \DeclareRobustCommand*{\fup}[1]{%
    \@ifundefined{realsuperscript}%
      {\FB@up{#1}}%
      {\bgroup\let\fakesuperscript\FB@up@fake
            \realsuperscript{\FB@lc{#1}}\egroup}}
%    \end{macrocode}
%    Temporary definition of |up| (redefined `AtBeginDocument').
%    \begin{macrocode}
  \newcommand*{\up}{\relax}
%    \end{macrocode}
%    Poor man's definition of |\up| for Plain. In \LaTeXe,
%    |\up| will be defined as |\fup| or |\textsuperscript| later on
%    while processing the options of |\frenchbsetup{}|.
%    \begin{macrocode}
\else
  \newcommand*{\up}[1]{\leavevmode\raise1ex\hbox{\sevenrm #1}}
\fi
%    \end{macrocode}
%  \end{macro}
%  \end{macro}
%
%  \begin{macro}{\ieme}
%  \begin{macro}{\ier}
%  \begin{macro}{\iere}
%  \begin{macro}{\iemes}
%  \begin{macro}{\iers}
%  \begin{macro}{\ieres}
%  Some handy macros for those who don't know how to abbreviate ordinals:
%    \begin{macrocode}
\def\ieme{\up{\lowercase{e}}\xspace}
\def\iemes{\up{\lowercase{es}}\xspace}
\def\ier{\up{\lowercase{er}}\xspace}
\def\iers{\up{\lowercase{ers}}\xspace}
\def\iere{\up{\lowercase{re}}\xspace}
\def\ieres{\up{\lowercase{res}}\xspace}
%    \end{macrocode}
%  \end{macro}
%  \end{macro}
%  \end{macro}
%  \end{macro}
%  \end{macro}
%  \end{macro}
%
% \changes{v2.1c}{2008/04/29}{Added commands \cs{Nos} and \cs{nos}.}
%
%  \begin{macro}{\No}
%  \begin{macro}{\no}
%  \begin{macro}{\Nos}
%  \begin{macro}{\nos}
%  \begin{macro}{\primo}
%  \begin{macro}{\fprimo)}
%    And some more macros relying on |\up| for numbering,
%    first two support macros.
%    \begin{macrocode}
\newcommand*{\FrenchEnumerate}[1]{%
                       #1\up{\lowercase{o}}\kern+.3em}
\newcommand*{\FrenchPopularEnumerate}[1]{%
                       #1\up{\lowercase{o}})\kern+.3em}
%    \end{macrocode}
%
%    Typing |\primo| should result in `$1^{\rm o}$\kern+.3em',
%    \begin{macrocode}
\def\primo{\FrenchEnumerate1}
\def\secundo{\FrenchEnumerate2}
\def\tertio{\FrenchEnumerate3}
\def\quarto{\FrenchEnumerate4}
%    \end{macrocode}
%    while typing |\fprimo)| gives `1$^{\rm o}$)\kern+.3em.
%    \begin{macrocode}
\def\fprimo){\FrenchPopularEnumerate1}
\def\fsecundo){\FrenchPopularEnumerate2}
\def\ftertio){\FrenchPopularEnumerate3}
\def\fquarto){\FrenchPopularEnumerate4}
%    \end{macrocode}
%
%    Let's provide four macros for the common abbreviations
%    of ``Num\'ero''.
%    \begin{macrocode}
\DeclareRobustCommand*{\No}{N\up{\lowercase{o}}\kern+.2em}
\DeclareRobustCommand*{\no}{n\up{\lowercase{o}}\kern+.2em}
\DeclareRobustCommand*{\Nos}{N\up{\lowercase{os}}\kern+.2em}
\DeclareRobustCommand*{\nos}{n\up{\lowercase{os}}\kern+.2em}
%    \end{macrocode}
%  \end{macro}
%  \end{macro}
%  \end{macro}
%  \end{macro}
%  \end{macro}
%  \end{macro}
%
%  \begin{macro}{\bsc}
%    As family names should be written in small capitals and never be
%    hyphenated, we provide a command (its name comes from Boxed Small
%    Caps) to input them easily.  Note that this command has changed
%    with version~2 of |frenchb|: a |\kern0pt| is used instead of |\hbox|
%    because |\hbox| would break microtype's font expansion; as a
%    (positive?) side effect, composed names (such as Dupont-Durand)
%    can now be hyphenated on explicit hyphens.
%    Usage: |Jean~\bsc{Duchemin}|.
%
% \changes{v2.0}{2006/11/06}{\cs{hbox} dropped, replaced by
%    \cs{kern0pt}.}
%
%    \begin{macrocode}
\DeclareRobustCommand*{\bsc}[1]{\leavevmode\begingroup\kern0pt
                                           \scshape #1\endgroup}
\ifLaTeXe\else\let\scshape\relax\fi
%    \end{macrocode}
%  \end{macro}
%
%    Some definitions for special characters.  We won't define |\tilde|
%    as a Text Symbol not to conflict with the macro |\tilde| for math
%    mode and use the name |\tild| instead. Note that |\boi| may
%    \emph{not} be used in math mode, its name in math mode is
%    |\backslash|.  |\degre|  can be accessed by the command |\r{}|
%    for ring accent.
%
%    \begin{macrocode}
\ifLaTeXe
  \DeclareTextSymbol{\at}{T1}{64}
  \DeclareTextSymbol{\circonflexe}{T1}{94}
  \DeclareTextSymbol{\tild}{T1}{126}
  \DeclareTextSymbolDefault{\at}{T1}
  \DeclareTextSymbolDefault{\circonflexe}{T1}
  \DeclareTextSymbolDefault{\tild}{T1}
  \DeclareRobustCommand*{\boi}{\textbackslash}
  \DeclareRobustCommand*{\degre}{\r{}}
\else
  \def\T@one{T1}
  \ifx\f@encoding\T@one
    \newcommand*{\degre}{\char6}
  \else
    \newcommand*{\degre}{\char23}
  \fi
  \newcommand*{\at}{\char64}
  \newcommand*{\circonflexe}{\char94}
  \newcommand*{\tild}{\char126}
  \newcommand*{\boi}{$\backslash$}
\fi
%    \end{macrocode}
%
%  \begin{macro}{\degres}
%    We now define a macro |\degres| for typesetting the abbreviation
%    for `degrees' (as in `degrees Celsius'). As the bounding box of
%    the character `degree' has \emph{very} different widths in CM/EC
%    and PostScript fonts, we fix the width of the bounding box of
%    |\degres| to 0.3\,em, this lets the symbol `degree' stick to the
%    preceding (e.g., |45\degres|) or following character
%    (e.g., |20~\degres C|).
%
%    If the \TeX{} Companion fonts are available (\file{textcomp.sty}),
%    we pick up |\textdegree| from them instead of using emulating
%    `degrees' from the |\r{}| accent. Otherwise we overwrite the
%    (poor) definition of |\textdegree| given in \file{latin1.def},
%    \file{applemac.def} etc. (called by  \file{inputenc.sty}) by
%    our definition of |\degres|. We also advice the user (once only)
%    to use TS1-encoding.
%
% \changes{v2.1c}{2008/04/29}{Provide a temporary definition (hyperref
%    safe) of \cs{degres} in case it has to be expanded in the preamble
%    (by beamer's \cs{title} command for instance).}
%
%    \begin{macrocode}
\ifLaTeXe
  \newcommand*{\degres}{\degre}
  \def\Warning@degree@TSone{%
        \PackageWarning{frenchb.ldf}{%
           Degrees would look better in TS1-encoding:
           \MessageBreak add \protect
           \usepackage{textcomp} to the preamble.
           \MessageBreak Degrees used}}
  \AtBeginDocument{\expandafter\ifx\csname M@TS1\endcsname\relax
                     \DeclareRobustCommand*{\degres}{%
                       \leavevmode\hbox to 0.3em{\hss\degre\hss}%
                       \Warning@degree@TSone
                       \global\let\Warning@degree@TSone\relax}%
                      \let\textdegree\degres
                   \else
                     \DeclareRobustCommand*{\degres}{%
                         \hbox{\UseTextSymbol{TS1}{\textdegree}}}%
                   \fi}
\else
  \newcommand*{\degres}{%
    \leavevmode\hbox to 0.3em{\hss\degre\hss}}
\fi
%    \end{macrocode}
%  \end{macro}
%
%  \subsection{Formatting numbers}
%  \label{sec-numbers}
%
%  \begin{macro}{\DecimalMathComma}
%  \begin{macro}{\StandardMathComma}
%    As mentioned in the \TeX{}book p.~134, the comma is of type
%    |\mathpunct| in math mode: it is automatically followed by a
%    space. This is convenient in lists and intervals but
%    unpleasant when the comma is used as a decimal separator
%    in French: it has to be entered as |{,}|.
%    |\DecimalMathComma| makes the comma be an ordinary character
%    (of type |\mathord|) in French \emph{only} (no space added);
%    |\StandardMathComma| switches back to the standard behaviour
%    of the comma.
%    \begin{macrocode}
\newcount\std@mcc
\newcount\dec@mcc
\std@mcc=\mathcode`\,
\dec@mcc=\std@mcc
\@tempcnta=\std@mcc
\divide\@tempcnta by "1000
\multiply\@tempcnta by "1000
\advance\dec@mcc by -\@tempcnta
\newcommand*{\DecimalMathComma}{\iflanguage{french}%
                                 {\mathcode`\,=\dec@mcc}{}%
              \addto\extrasfrench{\mathcode`\,=\dec@mcc}}
\newcommand*{\StandardMathComma}{\mathcode`\,=\std@mcc
             \addto\extrasfrench{\mathcode`\,=\std@mcc}}
\expandafter\addto\csname noextras\CurrentOption\endcsname{%
   \mathcode`\,=\std@mcc}
%    \end{macrocode}
%  \end{macro}
%  \end{macro}
%
%  \begin{macro}{\nombre}
%
% \changes{v2.0}{2006/11/06}{\cs{nombre} requires now numprint.sty.}
%
%    The command |\nombre| is now borrowed from |numprint.sty| for
%    \LaTeXe.  There is no point to maintain the former tricky code
%    when a package is dedicated to do the same job and more.
%    For Plain based formats, |\nombre| no longer formats numbers,
%    it prints them as is and issues a warning about the change.
%
%    Fake command |\nombre| for Plain based formats, warning users of
%    |frenchb| v.1.x. of the change.
%    \begin{macrocode}
\newcommand*{\nombre}[1]{{#1}\message{%
     *** \noexpand\nombre no longer formats numbers\string! ***}}%
%    \end{macrocode}
%  \end{macro}
%
%    The next definitions only make sense for \LaTeXe. Let's cleanup
%    and exit if the format in Plain based.
%
%    \begin{macrocode}
\let\FBstop@here\relax
\def\FBclean@on@exit{\let\ifLaTeXe\@undefined
                     \let\LaTeXetrue\@undefined
                     \let\LaTeXefalse\@undefined}
\ifx\magnification\@undefined
\else
   \def\FBstop@here{\let\STD@makecaption\relax
                    \FBclean@on@exit
                    \ldf@quit\CurrentOption\endinput}
\fi
\FBstop@here
%    \end{macrocode}
%
%    What follows now is for \LaTeXe{} \emph{only}.
%    We redefine |\nombre| for \LaTeXe. A warning is issued
%    at the first call of |\nombre| if |\numprint| is not
%    defined, suggesting what to do.  The package |numprint|
%    is \emph{not} loaded automatically by |frenchb| because of
%    possible options conflict.
%
%    \begin{macrocode}
\renewcommand*{\nombre}[1]{\Warning@nombre\numprint{#1}}
\newcommand*{\Warning@nombre}{%
   \@ifundefined{numprint}%
      {\PackageWarning{frenchb.ldf}{%
         \protect\nombre\space now relies on package numprint.sty,
         \MessageBreak add \protect
         \usepackage[autolanguage]{numprint}\MessageBreak
         to your preamble *after* loading babel, \MessageBreak
         see file numprint.pdf for other options.\MessageBreak
         \protect\nombre\space called}%
       \global\let\Warning@nombre\relax
       \global\let\numprint\relax
      }{}%
}
%    \end{macrocode}
%
% \changes{v2.0c}{2007/06/25}{There is no need to define here
%    numprint's command \cs{npstylefrench}, it will be redefined
%    `AtBeginDocument' by \cs{FBprocess@options}.}
%
% \changes{v2.0c}{2007/06/25}{\cs{ThinSpaceInFrenchNumbers} added
%     for compatibility with frenchb-1.x.}
%
%    \begin{macrocode}
\newcommand*{\ThinSpaceInFrenchNumbers}{%
   \PackageWarning{frenchb.ldf}{%
         Type \protect\frenchbsetup{ThinSpaceInFrenchNumbers}
         \MessageBreak Command \protect\ThinSpaceInFrenchNumbers\space
         is no longer\MessageBreak  defined in frenchb v.2,}}
%    \end{macrocode}
%
%  \subsection{Caption names}
%
%    The next step consists of defining the French equivalents for
%    the \LaTeX{} caption names.
%
% \begin{macro}{\captionsfrench}
%    Let's first define  |\captionsfrench| which sets all strings used
%    in the four standard document classes provided with \LaTeX.
%
% \changes{v2.0}{2006/11/06}{`Fig.' changed to `Figure' and
%     `Tab.' to `Table'.}
%
% \changes{v2.0}{2006/12/15}{Set \cs{CaptionSeparator} in
%     \cs{extrasfrench} now instead of \cs{captionsfrench}
%     because it has to be reset when leaving French.}
%
%    \begin{macrocode}
\@namedef{captions\CurrentOption}{%
   \def\refname{R\'ef\'erences}%
   \def\abstractname{R\'esum\'e}%
   \def\bibname{Bibliographie}%
   \def\prefacename{Pr\'eface}%
   \def\chaptername{Chapitre}%
   \def\appendixname{Annexe}%
   \def\contentsname{Table des mati\`eres}%
   \def\listfigurename{Table des figures}%
   \def\listtablename{Liste des tableaux}%
   \def\indexname{Index}%
   \def\figurename{{\scshape Figure}}%
   \def\tablename{{\scshape Table}}%
%    \end{macrocode}
%   ``Premi\`ere partie'' instead of ``Part I''.
%    \begin{macrocode}
   \def\partname{\protect\@Fpt partie}%
   \def\@Fpt{{\ifcase\value{part}\or Premi\`ere\or Deuxi\`eme\or
   Troisi\`eme\or Quatri\`eme\or Cinqui\`eme\or Sixi\`eme\or
   Septi\`eme\or Huiti\`eme\or Neuvi\`eme\or Dixi\`eme\or Onzi\`eme\or
   Douzi\`eme\or Treizi\`eme\or Quatorzi\`eme\or Quinzi\`eme\or
   Seizi\`eme\or Dix-septi\`eme\or Dix-huiti\`eme\or Dix-neuvi\`eme\or
   Vingti\`eme\fi}\space\def\thepart{}}%
   \def\pagename{page}%
   \def\seename{{\emph{voir}}}%
   \def\alsoname{{\emph{voir aussi}}}%
   \def\enclname{P.~J. }%
   \def\ccname{Copie \`a }%
   \def\headtoname{}%
   \def\proofname{D\'emonstration}%
   \def\glossaryname{Glossaire}%
   }
%    \end{macrocode}
% \end{macro}
%
%    As some users who choose |frenchb| or |francais| as option of
%    \babel, might customise |\captionsfrenchb| or |\captionsfrancais|
%    in the preamble, we merge their changes at the |\begin{document}|
%    when they do so.
%    The other variants of French (canadien, acadian) are defined by
%    checking if the relevant option was used and then adding one extra
%    level of expansion.
%
%    \begin{macrocode}
\AtBeginDocument{\let\captions@French\captionsfrench
                 \@ifundefined{captionsfrenchb}%
                    {\let\captions@Frenchb\relax}%
                    {\let\captions@Frenchb\captionsfrenchb}%
                 \@ifundefined{captionsfrancais}%
                    {\let\captions@Francais\relax}%
                    {\let\captions@Francais\captionsfrancais}%
                 \def\captionsfrench{\captions@French
                        \captions@Francais\captions@Frenchb}%
                 \def\captionsfrancais{\captionsfrench}%
                 \def\captionsfrenchb{\captionsfrench}%
                 \iflanguage{french}{\captionsfrench}{}%
                }
\@ifpackagewith{babel}{canadien}{%
  \def\captionscanadien{\captionsfrench}%
  \def\datecanadien{\datefrench}%
  \def\extrascanadien{\extrasfrench}%
  \def\noextrascanadien{\noextrasfrench}%
  }{}
\@ifpackagewith{babel}{acadian}{%
  \def\captionsacadian{\captionsfrench}%
  \def\dateacadian{\datefrench}%
  \def\extrasacadian{\extrasfrench}%
  \def\noextrasacadian{\noextrasfrench}%
  }{}
%    \end{macrocode}
%
% \begin{macro}{\CaptionSeparator}
%    Let's consider now captions in figures and tables.
%    In French, captions in figures and tables should be printed with
%    endash (`--') instead of the standard `:'.
%
%    The standard definition of |\@makecaption| (e.g., the one provided
%    in article.cls, report.cls, book.cls which is frozen for \LaTeXe{}
%    according to Frank Mittelbach), has been saved in
%    |\STD@makecaption| before making `:' active
%    (see section~\ref{sec-punct}). `AtBeginDocument' we compare it to
%    its current definition (some classes like koma-script classes,
%    AMS classes, ua-thesis.cls\dots change it).
%    If they are identical, |frenchb| just adds a hook called
%    |\CaptionSeparator| to |\@makecaption|, |\CaptionSeparator|
%    defaults to `: ' as in the standard |\@makecaption|, and will be
%    changed to ` -- ' in French.
%    If the definitions differ, |frenchb| doesn't overwrite the changes,
%    but prints a message in the .log file.
%
%    \begin{macrocode}
\def\CaptionSeparator{\string:\space}
\long\def\FB@makecaption#1#2{%
  \vskip\abovecaptionskip
  \sbox\@tempboxa{#1\CaptionSeparator #2}%
  \ifdim \wd\@tempboxa >\hsize
    #1\CaptionSeparator #2\par
  \else
    \global \@minipagefalse
    \hb@xt@\hsize{\hfil\box\@tempboxa\hfil}%
  \fi
  \vskip\belowcaptionskip}
\AtBeginDocument{%
  \ifx\@makecaption\STD@makecaption
      \global\let\@makecaption\FB@makecaption
  \else
    \@ifundefined{@makecaption}{}%
       {\PackageWarning{frenchb.ldf}%
        {The definition of \protect\@makecaption\space
         has been changed,\MessageBreak
         frenchb will NOT customise it;\MessageBreak reported}%
       }%
  \fi
  \let\FB@makecaption\relax
  \let\STD@makecaption\relax
}
\expandafter\addto\csname extras\CurrentOption\endcsname{%
   \def\CaptionSeparator{\space\textendash\space}}
\expandafter\addto\csname noextras\CurrentOption\endcsname{%
    \def\CaptionSeparator{\string:\space}}
%    \end{macrocode}
% \end{macro}
%
%  \subsection{French lists}
%  \label{sec-lists}
%
%  \begin{macro}{\listFB}
%  \begin{macro}{\listORI}
%    Vertical spacing in general lists should be shorter in French
%    texts than the defaults provided by \LaTeX.
%    Note that the easy way, just changing values of vertical spacing
%    parameters when entering French and restoring them to their
%    defaults on exit would not work; as most lists are based on
%    |\list| we will define a variant of |\list| (|\listFB|) to
%    be used in French.
%
%    The amount of vertical space before and after a list is given by
%    |\topsep| + |\parskip| (+ |\partopsep| if the list starts a new
%    paragraph). IMHO, |\parskip| should be added \emph{only} when
%    the list starts a new paragraph, so I subtract |\parskip| from
%    |\topsep| and add it back to |\partopsep|; this will normally
%    make no difference because |\parskip|'s default value is 0pt, but
%    will be noticeable when |\parskip| is \emph{not} null.
%
%    |\endlist| is not redefined, but |\endlistORI| is provided for
%    the users who prefer to define their own lists from the original
%    command, they can code: |\begin{listORI}{}{} \end{listORI}|.
%    \begin{macrocode}
\let\listORI\list
\let\endlistORI\endlist
\def\FB@listsettings{%
      \setlength{\itemsep}{0.4ex plus 0.2ex minus 0.2ex}%
      \setlength{\parsep}{0.4ex plus 0.2ex minus 0.2ex}%
      \setlength{\topsep}{0.8ex plus 0.4ex minus 0.4ex}%
      \setlength{\partopsep}{0.4ex plus 0.2ex minus 0.2ex}%
%    \end{macrocode}
%    |\parskip| is of type `skip', its mean value only (\emph{not
%    the glue}) should be subtracted from |\topsep| and added to
%    |\partopsep|, so convert |\parskip| to a `dimen' using
%    |\@tempdima|.
%    \begin{macrocode}
      \@tempdima=\parskip
      \addtolength{\topsep}{-\@tempdima}%
      \addtolength{\partopsep}{\@tempdima}}%
\def\listFB#1#2{\listORI{#1}{\FB@listsettings #2}}%
\let\endlistFB\endlist
%    \end{macrocode}
%  \end{macro}
%  \end{macro}
%
%  \begin{macro}{\itemizeFB}
%  \begin{macro}{\itemizeORI}
%  \begin{macro}{\bbl@frenchlabelitems}
%  \begin{macro}{\bbl@nonfrenchlabelitems}
%    Let's now consider French itemize lists.  They differ from those
%    provided by the standard \LaTeXe{} classes:
%    \begin{itemize}
%      \item vertical spacing between items, before and after
%         the list, should be \emph{null} with \emph{no glue} added;
%      \item the item labels of a first level list should be vertically
%          aligned on the paragraph's first character (i.e. at
%          |\parindent| from the left margin);
%      \item the `\textbullet' is never used in French itemize-lists,
%          a long dash `--' is preferred for all levels. The item label
%          used in French is stored in |\FrenchLabelItem}|, it defaults
%          to `--' and can be changed using |\frenchbsetup{}| (see
%          section~\ref{sec-keyval}).
%    \end{itemize}
%
%    \begin{macrocode}
\newcommand*{\FrenchLabelItem}{\textendash}
\newcommand*{\Frlabelitemi}{\FrenchLabelItem}
\newcommand*{\Frlabelitemii}{\FrenchLabelItem}
\newcommand*{\Frlabelitemiii}{\FrenchLabelItem}
\newcommand*{\Frlabelitemiv}{\FrenchLabelItem}
%    \end{macrocode}
%    |\bbl@frenchlabelitems| saves current itemize labels and changes
%    them to their value in French. This code should never be executed
%    twice in a row, so we need a new flag that will be set and reset
%    by |\bbl@nonfrenchlabelitems| and |\bbl@frenchlabelitems|.
%    \begin{macrocode}
\newif\ifFB@enterFrench  \FB@enterFrenchtrue
\def\bbl@frenchlabelitems{%
  \ifFB@enterFrench
    \let\@ltiORI\labelitemi
    \let\@ltiiORI\labelitemii
    \let\@ltiiiORI\labelitemiii
    \let\@ltivORI\labelitemiv
    \let\labelitemi\Frlabelitemi
    \let\labelitemii\Frlabelitemii
    \let\labelitemiii\Frlabelitemiii
    \let\labelitemiv\Frlabelitemiv
    \FB@enterFrenchfalse
  \fi
}
\let\itemizeORI\itemize
\let\enditemizeORI\enditemize
\let\enditemizeFB\enditemize
\def\itemizeFB{%
    \ifnum \@itemdepth >\thr@@\@toodeep\else
      \advance\@itemdepth\@ne
      \edef\@itemitem{labelitem\romannumeral\the\@itemdepth}%
      \expandafter
      \listORI
      \csname\@itemitem\endcsname
      {\settowidth{\labelwidth}{\csname\@itemitem\endcsname}%
       \setlength{\leftmargin}{\labelwidth}%
       \addtolength{\leftmargin}{\labelsep}%
       \ifnum\@listdepth=0
         \setlength{\itemindent}{\parindent}%
       \else
         \addtolength{\leftmargin}{\parindent}%
       \fi
       \setlength{\itemsep}{\z@}%
       \setlength{\parsep}{\z@}%
       \setlength{\topsep}{\z@}%
       \setlength{\partopsep}{\z@}%
%    \end{macrocode}
%    |\parskip| is of type `skip', its mean value only (\emph{not
%    the glue}) should be subtracted from |\topsep| and added to
%    |\partopsep|, so convert |\parskip| to a `dimen' using
%    |\@tempdima|.
%    \begin{macrocode}
       \@tempdima=\parskip
       \addtolength{\topsep}{-\@tempdima}%
       \addtolength{\partopsep}{\@tempdima}}%
    \fi}
%    \end{macrocode}
%    The user's changes in labelitems are saved when leaving French for
%    further use when switching back to French.  This code should never
%    be executed twice in a row (toggle with |\bbl@frenchlabelitems|).
%    \begin{macrocode}
\def\bbl@nonfrenchlabelitems{%
  \ifFB@enterFrench
  \else
      \let\Frlabelitemi\labelitemi
      \let\Frlabelitemii\labelitemii
      \let\Frlabelitemiii\labelitemiii
      \let\Frlabelitemiv\labelitemiv
      \let\labelitemi\@ltiORI
      \let\labelitemii\@ltiiORI
      \let\labelitemiii\@ltiiiORI
      \let\labelitemiv\@ltivORI
      \FB@enterFrenchtrue
  \fi
}
%    \end{macrocode}
%  \end{macro}
%  \end{macro}
%  \end{macro}
%  \end{macro}
%
%  \subsection{French indentation of sections}
%  \label{sec-indent}
%
%  \begin{macro}{\bbl@frenchindent}
%  \begin{macro}{\bbl@nonfrenchindent}
%    In French the first paragraph of each section should be indented,
%    this is another difference with US-English. This is controlled by
%    the flag |\if@afterindent|.
%
% \changes{v2.3d}{2009/03/16}{Bug correction: previous versions of
%    frenchb set the flag \cs{if@afterindent} to false outside
%    French which is correct for English but wrong for some languages
%    like Spanish.  Pointed out by Juan Jos\'e Torrens.}
%
%    We need to save the value of the flag |\if@afterindent|
%    `AtBeginDocument' before eventually changing its value.
%
%    \begin{macrocode}
\AtBeginDocument{\ifx\@afterindentfalse\@afterindenttrue
                       \let\@aifORI\@afterindenttrue
                 \else \let\@aifORI\@afterindentfalse
                 \fi
}
\def\bbl@frenchindent{\let\@afterindentfalse\@afterindenttrue
                      \@afterindenttrue}
\def\bbl@nonfrenchindent{\let\@afterindentfalse\@aifORI
                         \@afterindentfalse}
%    \end{macrocode}
%  \end{macro}
%  \end{macro}
%
%  \subsection{Formatting footnotes}
%  \label{sec-footnotes}
%
% \changes{v2.0}{2006/11/06}{Footnotes are now printed
%     by default `\`a la fran\c caise' for the whole document.}
%
% \changes{v2.0b}{2007/04/18}{Footnotes: Just do nothing
%    (except warning) when the bigfoot package is loaded.}
%
%    The |bigfoot| package deeply changes the way footnotes are
%    handled. When |bigfoot| is loaded, we just warn the user that
%    |frenchb| will drop the customisation of footnotes.
%
%    The layout of footnotes is controlled by two flags
%    |\ifFBAutoSpaceFootnotes| and |\ifFBFrenchFootnotes| which are
%    set by options of |\frenchbsetup{}| (see section~\ref{sec-keyval}).
%    Notice that the layout of footnotes \emph{does not depend} on the
%    current language (just think of two footnotes on the same page
%    looking different because one was called in a French part, the
%    other one in English!).
%
%    When |\ifFBAutoSpaceFootnotes| is true, |\@footnotemark| (whose
%    definition is saved at the |\begin{document}| in order to include
%    any customisation that packages might have done) is redefined to
%    add a thin space before the number or symbol calling a footnote
%    (any space typed in is removed first).  This has no effect on
%    the layout of the footnote itself.
%
%    \begin{macrocode}
\AtBeginDocument{\@ifpackageloaded{bigfoot}%
                   {\PackageWarning{frenchb.ldf}%
                     {bigfoot package in use.\MessageBreak
                      frenchb will NOT customise footnotes;\MessageBreak
                      reported}}%
                   {\let\@footnotemarkORI\@footnotemark
                    \def\@footnotemarkFB{\leavevmode\unskip\unkern
                                         \,\@footnotemarkORI}%
                    \ifFBAutoSpaceFootnotes
                      \let\@footnotemark\@footnotemarkFB
                    \fi}%
                }
%    \end{macrocode}
%
%    We then define |\@makefntextFB|, a variant of |\@makefntext|
%    which is responsible for the layout of footnotes, to match the
%    specifications of the French `Imprimerie Nationale':  footnotes
%    will be indented by |\parindentFFN|, numbers (if any) typeset on
%    the baseline (instead of superscripts) and followed by a dot
%    and an half quad space. Whenever symbols are used to number
%    footnotes (as in |\thanks| for instance), we switch back to the
%    standard layout (the French layout of footnotes is meant for
%    footnotes numbered by Arabic or Roman digits).
%
% \changes{v2.0}{2006/11/06}{\cs{parindentFFN} not changed if
%    already defined (required by JA for cah-gut.cls).}
%
% \changes{v2.3b}{2008/12/06}{New commands \cs{dotFFN} and
%    \cs{kernFFN} for more flexibility (suggested by JA).}
%
%    The value of |\parindentFFN| will be redefined at the
%    |\begin{document}|, as the maximum of |\parindent| and 1.5em
%    \emph{unless} it has been set in the preamble (the weird value
%    10in is just for testing whether |\parindentFFN| has been set
%    or not).
%
%    \begin{macrocode}
\newcommand*{\dotFFN}{.}
\newcommand*{\kernFFN}{\kern .5em}
\newdimen\parindentFFN
\parindentFFN=10in
\def\ftnISsymbol{\@fnsymbol\c@footnote}
\long\def\@makefntextFB#1{\ifx\thefootnote\ftnISsymbol
                            \@makefntextORI{#1}%
                          \else
                            \parindent=\parindentFFN
                            \rule\z@\footnotesep
                            \setbox\@tempboxa\hbox{\@thefnmark}%
                            \ifdim\wd\@tempboxa>\z@
                              \llap{\@thefnmark}\dotFFN\kernFFN
                            \fi #1
                          \fi}%
%    \end{macrocode}
%
%    We save the standard definition of |\@makefntext| at the
%    |\begin{document}|, and then redefine |\@makefntext| according to
%    the value of flag |\ifFBFrenchFootnotes| (true or false).
%
%    \begin{macrocode}
\AtBeginDocument{\@ifpackageloaded{bigfoot}{}%
                  {\ifdim\parindentFFN<10in
                   \else
                      \parindentFFN=\parindent
                      \ifdim\parindentFFN<1.5em\parindentFFN=1.5em\fi
                   \fi
                   \let\@makefntextORI\@makefntext
                   \long\def\@makefntext#1{%
                      \ifFBFrenchFootnotes
                         \@makefntextFB{#1}%
                      \else
                         \@makefntextORI{#1}%
                      \fi}%
                  }%
                }
%    \end{macrocode}
%
%    For compatibility reasons, we provide definitions for the commands
%    dealing with the layout of footnotes in |frenchb| version~1.6.
%    |\frenchbsetup{}| (see in section \ref{sec-keyval}) should be
%    preferred for setting these options.  |\StandardFootnotes| may
%    still be used locally (in minipages for instance), that's why the
%    test |\ifFBFrenchFootnotes| is done inside |\@makefntext|.
%    \begin{macrocode}
\newcommand*{\AddThinSpaceBeforeFootnotes}{\FBAutoSpaceFootnotestrue}
\newcommand*{\FrenchFootnotes}{\FBFrenchFootnotestrue}
\newcommand*{\StandardFootnotes}{\FBFrenchFootnotesfalse}
%    \end{macrocode}
%
%  \subsection{Global layout}
%  \label{sec-global}
%
%    In multilingual documents, some typographic rules must depend
%    on the current language (e.g., hyphenation, typesetting of
%    numbers, spacing before double punctuation\dots), others should,
%    IMHO, be kept global to the document: especially the layout of
%    lists (see~\ref{sec-lists}) and footnotes
%    (see~\ref{sec-footnotes}), and the indentation of the first
%    paragraph of sections (see~\ref{sec-indent}).
%
%    From version 2.2 on, if |frenchb| is \babel's ``main language''
%    (i.e. last language option at \babel's loading), |frenchb|
%    customises the layout (i.e. lists, indentation of the first
%    paragraphs of sections and footnotes) in the whole document
%    regardless the current language.   On the other hand, if |frenchb|
%    is \emph{not} \babel's ``main language'', it leaves the layout
%    unchanged both in French and in other languages.
%
%  \begin{macro}{\FrenchLayout}
%  \begin{macro}{\StandardLayout}
%    The former commands |\FrenchLayout| and |\StandardLayout| are kept
%    for compatibility reasons but should no longer be used.
%
% \changes{v2.0g}{2008/03/23}{Flag \cs{ifFBStandardLayout} not checked
%     by \cs{FBprocess@options}, low-level flags have to be set
%     one by one.}
%
%    \begin{macrocode}
\newcommand*{\FrenchLayout}{%
    \FBGlobalLayoutFrenchtrue
    \PackageWarning{frenchb.ldf}%
    {\protect\FrenchLayout\space is obsolete.  Please use\MessageBreak
     \protect\frenchbsetup{GlobalLayoutFrench} instead.}%
}
\newcommand*{\StandardLayout}{%
  \FBReduceListSpacingfalse
  \FBCompactItemizefalse
  \FBStandardItemLabelstrue
  \FBIndentFirstfalse
  \FBFrenchFootnotesfalse
  \FBAutoSpaceFootnotesfalse
  \PackageWarning{frenchb.ldf}%
    {\protect\StandardLayout\space is obsolete.  Please use\MessageBreak
    \protect\frenchbsetup{StandardLayout} instead.}%
}
\@onlypreamble\FrenchLayout
\@onlypreamble\StandardLayout
%    \end{macrocode}
%  \end{macro}
%  \end{macro}
%
%  \subsection{Dots\dots}
%  \label{sec-dots}
%
%  \begin{macro}{\FBtextellipsis}
%    \LaTeXe's standard definition of |\dots| in text-mode is
%    |\textellipsis| which includes a |\kern| at the end;
%    this space is not wanted in some cases (before a closing brace
%    for instance) and |\kern| breaks hyphenation of the next word.
%    We define |\FBtextellipsis| for French (in \LaTeXe{} only).
%
%    The |\if| construction in the \LaTeXe{} definition of |\dots|
%    doesn't allow the use of |xspace| (|xspace| is always followed
%    by a |\fi|), so we use the AMS-\LaTeX{} construction of |\dots|;
%    this has to be done `AtBeginDocument' not to be overwritten
%    when \file{amsmath.sty} is loaded after \babel.
%
% \changes{v2.0}{2006/11/06}{Added special case for LY1 encoding,
%    see  bug report from Bruno Voisin (2004/05/18).}
%
%    LY1 has a ready made character for |\textellipsis|, it should be
%    used in French too (pointed out by Bruno Voisin).
%
%    \begin{macrocode}
\DeclareTextSymbol{\FBtextellipsis}{LY1}{133}
\DeclareTextCommandDefault{\FBtextellipsis}{%
    .\kern\fontdimen3\font.\kern\fontdimen3\font.\xspace}
%    \end{macrocode}
%    |\Mdots@| and |\Tdots@ORI| hold the definitions of |\dots| in
%    Math and Text mode. They default to those of amsmath-2.0, and
%    will revert to standard \LaTeX{} definitions `AtBeginDocument',
%    if amsmath has not been loaded. |\Mdots@| doesn't change when
%    switching from/to French, while |\Tdots@| is |\FBtextellipsis|
%    in French and |\Tdots@ORI| otherwise.
%    \begin{macrocode}
\newcommand*{\Tdots@ORI}{\@xp\textellipsis}
\newcommand*{\Tdots@}{\Tdots@ORI}
\newcommand*{\Mdots@}{\@xp\mdots@}
\AtBeginDocument{\DeclareRobustCommand*{\dots}{\relax
                 \csname\ifmmode M\else T\fi dots@\endcsname}%
                 \@ifundefined{@xp}{\let\@xp\relax}{}%
                 \@ifundefined{mdots@}{\let\Tdots@ORI\textellipsis
                                       \let\Mdots@\mathellipsis}{}}
\def\bbl@frenchdots{\let\Tdots@\FBtextellipsis}
\def\bbl@nonfrenchdots{\let\Tdots@\Tdots@ORI}
\expandafter\addto\csname extras\CurrentOption\endcsname{%
    \bbl@frenchdots}
\expandafter\addto\csname noextras\CurrentOption\endcsname{%
    \bbl@nonfrenchdots}
%    \end{macrocode}
%  \end{macro}
%
%  \subsection{Setup options: keyval stuff}
%  \label{sec-keyval}
%
% \changes{v2.0}{2006/11/06}{New command \cs{frenchbsetup} added
%     for global customisation.}
%
% \changes{v2.0c}{2007/06/25}{Option ThinSpaceInFrenchNumbers added.}
%
% \changes{v2.0d}{2007/07/15}{Options og and fg changed: limit
%     the definition to French so that quote characters can be used
%     in German.}
%
% \changes{v2.0e}{2007/10/05}{New option: StandardLists.}
%
% \changes{v2.0f}{2008/03/23}{Two typos corrected in
%    option StandardLists: [false] $\to$ [true] and
%    StandardLayout $\to$ StandardLists.}
%
% \changes{v2.0f}{2008/03/23}{StandardLayout option had no
%     effect on lists.  Test moved to \cs{FBprocess@options}.}
%
% \changes{v2.0g}{2008/03/23}{Revert previous change to
%     StandardLayout. This option must set the three flags
%     \cs{FBReduceListSpacingfalse}, \cs{FBCompactItemizefalse},
%     and \cs{FBStandardItemLabeltrue} instead of
%     \cs{FBStandardListstrue}, so that later options can still
%     change their value before executing \cs{FBprocess@options}.
%     Same thing for option StandardLists.}
%
% \changes{v2.1a}{2008/03/24}{New option: FrenchSuperscripts
%     to define \cs{up} as \cs{fup} or as \cs{textsuperscript}.}
%
% \changes{v2.1a}{2008/03/30}{New option: LowercaseSuperscripts.}
%
% \changes{v2.2a}{2008/05/08}{The global layout of the document is
%     no longer changed when frenchb is not the last option of babel
%     (\cs{bbl@main@language}). Suggested by Ulrike Fischer.}
%
% \changes{v2.2a}{2008/05/08}{Values of flags
%     \cs{ifFBReduceListSpacing}, \cs{ifFBCompactItemize},
%     \cs{ifFBStandardItemLabels}, \cs{ifFBIndentFirst},
%     \cs{ifFBFrenchFootnotes}, \cs{ifFBAutoSpaceFootnotes} changed:
%     default now means `StandardLayout', it will be changed to
%     `FrenchLayout' AtEndOfPackage only if french is
%     \cs{bbl@main@language}.}
%
% \changes{v2.2a}{2008/05/08}{When frenchb is babel's last option,
%     French becomes the document's main language, so
%     GlobalLayoutFrench applies.}
%
% \changes{v2.3a}{2008/10/10}{New option: OriginalTypewriter. Now
%    frenchb switches to \cs{noautospace@beforeFDP} when a tt-font is
%    in use.  When OriginalTypewriter is set to true, frenchb behaves
%    as in pre-2.3 versions.}
%
%    We first define a collection of conditionals with their defaults
%    (true or false).
%
%    \begin{macrocode}
\newif\ifFBStandardLayout           \FBStandardLayouttrue
\newif\ifFBGlobalLayoutFrench       \FBGlobalLayoutFrenchfalse
\newif\ifFBReduceListSpacing        \FBReduceListSpacingfalse
\newif\ifFBCompactItemize           \FBCompactItemizefalse
\newif\ifFBStandardItemLabels       \FBStandardItemLabelstrue
\newif\ifFBStandardLists            \FBStandardListstrue
\newif\ifFBIndentFirst              \FBIndentFirstfalse
\newif\ifFBFrenchFootnotes          \FBFrenchFootnotesfalse
\newif\ifFBAutoSpaceFootnotes       \FBAutoSpaceFootnotesfalse
\newif\ifFBOriginalTypewriter       \FBOriginalTypewriterfalse
\newif\ifFBThinColonSpace           \FBThinColonSpacefalse
\newif\ifFBThinSpaceInFrenchNumbers \FBThinSpaceInFrenchNumbersfalse
\newif\ifFBFrenchSuperscripts       \FBFrenchSuperscriptstrue
\newif\ifFBLowercaseSuperscripts    \FBLowercaseSuperscriptstrue
\newif\ifFBPartNameFull             \FBPartNameFulltrue
\newif\ifFBShowOptions              \FBShowOptionsfalse
%    \end{macrocode}
%
%    The defaults values of these flags have been set so that |frenchb|
%    does not change anything regarding the global layout.
%    |\bbl@main@language| (set by the last option of babel) controls
%    the global layout of the document.  We check the current language
%    `AtEndOfPackage' (it is |\bbl@main@language|); if it is French,
%    the values of some flags have to be changed to ensure a French
%    looking layout for the whole document (even in parts written in
%    languages other than French); the end-user will then be able to
%    customise the values of all these flags with |\frenchbsetup{}|.
%    \begin{macrocode}
\AtEndOfPackage{%
   \iflanguage{french}{\FBReduceListSpacingtrue
                       \FBCompactItemizetrue
                       \FBStandardItemLabelsfalse
                       \FBIndentFirsttrue
                       \FBFrenchFootnotestrue
                       \FBAutoSpaceFootnotestrue
                       \FBGlobalLayoutFrenchtrue}%
                      {}%
}
%    \end{macrocode}
%
%  \begin{macro}{\frenchbsetup}
%    From version 2.0 on, all setup options are handled by \emph{one}
%    command |\frenchbsetup| using the keyval syntax.
%    Let's now define this command which reads and sets the options
%    to be processed later (at |\begin{document}|) by
%    |\FBprocess@options|. It  can only be called in the preamble.
%    \begin{macrocode}
\newcommand*{\frenchbsetup}[1]{%
  \setkeys{FB}{#1}%
}%
\@onlypreamble\frenchbsetup
%    \end{macrocode}
%    |frenchb| being an option of babel, it cannot load a package
%    (keyval) while |frenchb.ldf| is read, so we defer the loading of
%    \file{keyval} and the options setup at the end of \babel's loading.
%
%    |StandardLayout| resets the layout in French to the standard layout
%    defined par the \LaTeX{} class and packages loaded. It deals with
%    lists, indentation of first paragraphs of sections and footnotes.
%    Other keys, entered \emph{after} |StandardLayout| in
%    |\frenchbsetup|, can overrule some of the |StandardLayout|
%     settings.
%
%    |GlobalLayoutFrench| forces the layout in French and (as far as
%    possible) outside French to meet the French typographic standards.
%
% \changes{v2.3d}{2009/03/16}{Warning added to \cs{GlobalLayoutFrench}
%    when French is not the main language.}
%
%    \begin{macrocode}
\AtEndOfPackage{%
    \RequirePackage{keyval}%
    \define@key{FB}{StandardLayout}[true]%
                      {\csname FBStandardLayout#1\endcsname
                       \ifFBStandardLayout
                         \FBReduceListSpacingfalse
                         \FBCompactItemizefalse
                         \FBStandardItemLabelstrue
                         \FBIndentFirstfalse
                         \FBFrenchFootnotesfalse
                         \FBAutoSpaceFootnotesfalse
                         \FBGlobalLayoutFrenchfalse
                       \else
                         \FBReduceListSpacingtrue
                         \FBCompactItemizetrue
                         \FBStandardItemLabelsfalse
                         \FBIndentFirsttrue
                         \FBFrenchFootnotestrue
                         \FBAutoSpaceFootnotestrue
                       \fi}%
    \define@key{FB}{GlobalLayoutFrench}[true]%
                      {\csname FBGlobalLayoutFrench#1\endcsname
                       \ifFBGlobalLayoutFrench
                          \iflanguage{french}%
                            {\FBReduceListSpacingtrue
                             \FBCompactItemizetrue
                             \FBStandardItemLabelsfalse
                             \FBIndentFirsttrue
                             \FBFrenchFootnotestrue
                             \FBAutoSpaceFootnotestrue}%
                            {\PackageWarning{frenchb.ldf}%
                              {Option `GlobalLayoutFrench' skipped:
                               \MessageBreak French is *not*
                               babel's last option.\MessageBreak}}%
                       \fi}%
    \define@key{FB}{ReduceListSpacing}[true]%
                      {\csname FBReduceListSpacing#1\endcsname}%
    \define@key{FB}{CompactItemize}[true]%
                      {\csname FBCompactItemize#1\endcsname}%
    \define@key{FB}{StandardItemLabels}[true]%
                      {\csname FBStandardItemLabels#1\endcsname}%
    \define@key{FB}{ItemLabels}{%
        \renewcommand*{\FrenchLabelItem}{#1}}%
    \define@key{FB}{ItemLabeli}{%
        \renewcommand*{\Frlabelitemi}{#1}}%
    \define@key{FB}{ItemLabelii}{%
        \renewcommand*{\Frlabelitemii}{#1}}%
    \define@key{FB}{ItemLabeliii}{%
        \renewcommand*{\Frlabelitemiii}{#1}}%
    \define@key{FB}{ItemLabeliv}{%
        \renewcommand*{\Frlabelitemiv}{#1}}%
    \define@key{FB}{StandardLists}[true]%
                      {\csname FBStandardLists#1\endcsname
                       \ifFBStandardLists
                         \FBReduceListSpacingfalse
                         \FBCompactItemizefalse
                         \FBStandardItemLabelstrue
                       \else
                         \FBReduceListSpacingtrue
                         \FBCompactItemizetrue
                         \FBStandardItemLabelsfalse
                       \fi}%
    \define@key{FB}{IndentFirst}[true]%
                      {\csname FBIndentFirst#1\endcsname}%
    \define@key{FB}{FrenchFootnotes}[true]%
                      {\csname FBFrenchFootnotes#1\endcsname}%
    \define@key{FB}{AutoSpaceFootnotes}[true]%
                      {\csname FBAutoSpaceFootnotes#1\endcsname}%
    \define@key{FB}{AutoSpacePunctuation}[true]%
                      {\csname FBAutoSpacePunctuation#1\endcsname}%
    \define@key{FB}{OriginalTypewriter}[true]%
                      {\csname FBOriginalTypewriter#1\endcsname}%
    \define@key{FB}{ThinColonSpace}[true]%
                      {\csname FBThinColonSpace#1\endcsname}%
    \define@key{FB}{ThinSpaceInFrenchNumbers}[true]%
                      {\csname FBThinSpaceInFrenchNumbers#1\endcsname}%
    \define@key{FB}{FrenchSuperscripts}[true]%
                      {\csname FBFrenchSuperscripts#1\endcsname}
    \define@key{FB}{LowercaseSuperscripts}[true]%
                      {\csname FBLowercaseSuperscripts#1\endcsname}
    \define@key{FB}{PartNameFull}[true]%
                      {\csname FBPartNameFull#1\endcsname}%
    \define@key{FB}{ShowOptions}[true]%
                      {\csname FBShowOptions#1\endcsname}%
%    \end{macrocode}
%    Inputing French quotes as \emph{single characters} when they are
%    available on the keyboard (through a compose key for instance)
%    is more comfortable than typing |\og| and |\fg|.
%    The purpose of the following code is to map the French quote
%    characters to |\og\ignorespaces| and |{\fg}| respectively when
%    the current language is French, and to |\guillemotleft| and
%    |\guillemotright| otherwise (think of German quotes); thus correct
%    unbreakable spaces will be added automatically to French quotes.
%    The quote characters typed in depend on the input encoding,
%    it can be single-byte (latin1, latin9, applemac,\dots) or
%    multi-bytes (utf-8, utf8x).  We first check whether XeTeX is used
%    or not, if not the package |inputenc| has to be loaded before the
%    |\begin{document}| with the proper coding option, so we check if
%    |\DeclareInputText| is defined.
%    \begin{macrocode}
    \define@key{FB}{og}{%
       \newcommand*{\FB@@og}{\iflanguage{french}%
                               {\FB@og\ignorespaces}{\guillemotleft}}%
       \expandafter\ifx\csname XeTeXrevision\endcsname\relax
         \AtBeginDocument{%
           \@ifundefined{DeclareInputText}%
             {\PackageWarning{frenchb.ldf}%
               {Option `og' requires package inputenc.\MessageBreak}%
             }%
             {\@ifundefined{uc@dclc}%
%    \end{macrocode}
%    if |\uc@dclc| is undefined, utf8x is not loaded\dots
%    \begin{macrocode}
               {\@ifundefined{DeclareUnicodeCharacter}%
%    \end{macrocode}
%    if |\DeclareUnicodeCharacter| is undefined, utf8 is not loaded
%    either, we assume 8-bit character input encoding.
%    Package MULEenc.sty (from CJK) defines |\mule@def| to map
%    characters to control sequences.
%    \begin{macrocode}
                  {\@tempcnta`#1\relax
                     \@ifundefined{mule@def}%
                       {\DeclareInputText{\the\@tempcnta}{\FB@@og}}%
                       {\mule@def{11}{{\FB@@og}}}%
                  }%
%    \end{macrocode}
%    utf8 loaded, use |\DeclareUnicodeCharacter|,
%    \begin{macrocode}
                  {\DeclareUnicodeCharacter{00AB}{\FB@@og}}%
               }%
%    \end{macrocode}
%    utf8x loaded, use |\uc@dclc|,
%    \begin{macrocode}
               {\uc@dclc{171}{default}{\FB@@og}}%
             }%
         }%
%    \end{macrocode}
%    XeTeX in use, the following trick for defining the active quote
%    character is borrowed from \file{inputenc.dtx}.
%    \begin{macrocode}
       \else
         \catcode`#1=\active
         \bgroup
           \uccode`\~`#1%
           \uppercase{%
         \egroup
         \def~%
         }{\FB@@og}%
       \fi
    }%
%    \end{macrocode}
%    Same code for the closing quote.
%    \begin{macrocode}
    \define@key{FB}{fg}{%
       \newcommand*{\FB@@fg}{\iflanguage{french}%
                               {\FB@fg}{\guillemotright}}%
       \expandafter\ifx\csname XeTeXrevision\endcsname\relax
         \AtBeginDocument{%
           \@ifundefined{DeclareInputText}%
             {\PackageWarning{frenchb.ldf}%
               {Option `fg' requires package inputenc.\MessageBreak}%
             }%
             {\@ifundefined{uc@dclc}%
               {\@ifundefined{DeclareUnicodeCharacter}%
                  {\@tempcnta`#1\relax
                     \@ifundefined{mule@def}%
                       {\DeclareInputText{\the\@tempcnta}{{\FB@@fg}}}%
                       {\mule@def{27}{{\FB@@fg}}}%
                  }%
                  {\DeclareUnicodeCharacter{00BB}{{\FB@@fg}}}%
               }%
               {\uc@dclc{187}{default}{{\FB@@fg}}}%
             }%
         }%
       \else
         \catcode`#1=\active
         \bgroup
           \uccode`\~`#1%
           \uppercase{%
         \egroup
         \def~%
         }{{\FB@@fg}}%
       \fi
    }%
}
%    \end{macrocode}
%  \end{macro}
%
% \begin{macro}{\FBprocess@options}
%    |\FBprocess@options| processes the options, it is called \emph{once}
%    at |\begin{document}|.
%    \begin{macrocode}
\newcommand*{\FBprocess@options}{%
%    \end{macrocode}
%    Nothing has to be done here for |StandardLayout| and
%    |StandardLists| (the involved flags have already been set in
%    |\frenchbsetup{}| or before (at babel's EndOfPackage).
%
%    The next three options deal with the layout of lists in French.
%
%    |ReduceListSpacing| reduces the vertical spaces between list
%    items in French (done by changing |\list| to |\listFB|).
%    When |GlobalLayoutFrench| is true (the default), the same is
%    done outside French except for languages that force a different
%    setting.
%    \begin{macrocode}
  \ifFBReduceListSpacing
    \addto\extrasfrench{\let\list\listFB
                        \let\endlist\endlistFB}%
    \addto\noextrasfrench{\ifFBGlobalLayoutFrench
                            \let\list\listFB
                            \let\endlist\endlistFB
                          \else
                            \let\list\listORI
                            \let\endlist\endlistORI
                          \fi}%
  \else
    \addto\extrasfrench{\let\list\listORI
                        \let\endlist\endlistORI}%
    \addto\noextrasfrench{\let\list\listORI
                          \let\endlist\endlistORI}%
  \fi
%    \end{macrocode}
%
%    |CompactItemize| suppresses the vertical spacing between list
%    items in French (done by changing |\itemize| to |\itemizeFB|).
%    When |GlobalLayoutFrench| is true the same is done outside French.
%    \begin{macrocode}
  \ifFBCompactItemize
    \addto\extrasfrench{\let\itemize\itemizeFB
                        \let\enditemize\enditemizeFB}%
    \addto\noextrasfrench{\ifFBGlobalLayoutFrench
                             \let\itemize\itemizeFB
                             \let\enditemize\enditemizeFB
                          \else
                             \let\itemize\itemizeORI
                             \let\enditemize\enditemizeORI
                          \fi}%
  \else
    \addto\extrasfrench{\let\itemize\itemizeORI
                        \let\enditemize\enditemizeORI}%
    \addto\noextrasfrench{\let\itemize\itemizeORI
                          \let\enditemize\enditemizeORI}%
  \fi
%    \end{macrocode}
%
%    |StandardItemLabels| resets labelitems in French to their
%    standard values set by the \LaTeX{} class and packages loaded.
%    When |GlobalLayoutFrench| is true labelitems are identical inside
%    and outside French.
%    \begin{macrocode}
  \ifFBStandardItemLabels
    \addto\extrasfrench{\bbl@nonfrenchlabelitems}%
    \addto\noextrasfrench{\bbl@nonfrenchlabelitems}%
  \else
    \addto\extrasfrench{\bbl@frenchlabelitems}%
    \addto\noextrasfrench{\ifFBGlobalLayoutFrench
                            \bbl@frenchlabelitems
                          \else
                            \bbl@nonfrenchlabelitems
                          \fi}%
  \fi
%    \end{macrocode}
%
%    |IndentFirst| forces the first paragraphs of sections to be
%    indented just like the other ones in French.
%    When |GlobalLayoutFrench| is true (the default), the same is
%    done outside French except for languages that force a different
%    setting.
%    \begin{macrocode}
  \ifFBIndentFirst
    \addto\extrasfrench{\bbl@frenchindent}%
    \addto\noextrasfrench{\ifFBGlobalLayoutFrench
                             \bbl@frenchindent
                          \else
                             \bbl@nonfrenchindent
                          \fi}%
  \else
    \addto\extrasfrench{\bbl@nonfrenchindent}%
    \addto\noextrasfrench{\bbl@nonfrenchindent}%
  \fi
%    \end{macrocode}
%
%    The layout of footnotes is handled at the |\begin{document}|
%    depending on the values of flags |FrenchFootnotes|
%    and |AutoSpaceFootnotes| (see section~\ref{sec-footnotes}),
%    nothing has to be done here for footnotes.
%
%    |AutoSpacePunctuation| adds an unbreakable space (in French only)
%    before the four active characters (:;!?) even if none has been
%    typed before them.
%    \begin{macrocode}
  \ifFBAutoSpacePunctuation
     \autospace@beforeFDP
  \else
     \noautospace@beforeFDP
  \fi
%    \end{macrocode}
%
%    When |OriginalTypewriter| is set to |false| (the default),
%    |\ttfamily|, |\rmfamily| and |\sffamily| are redefined as
%    |\ttfamilyFB|, |\rmfamilyFB| and |\sffamilyFB| respectively
%    to prevent addition of automatic spaces before the four active
%    characters in computer code.
%    \begin{macrocode}
  \ifFBOriginalTypewriter
  \else
     \let\ttfamily\ttfamilyFB
     \let\rmfamily\rmfamilyFB
     \let\sffamily\sffamilyFB
  \fi
%    \end{macrocode}
%
%    |ThinColonSpace| changes the normal unbreakable space typeset in
%     French before `:' to a thin space.
%    \begin{macrocode}
  \ifFBThinColonSpace\renewcommand*{\Fcolonspace}{\thinspace}\fi
%    \end{macrocode}
%
%    When |true|, |ThinSpaceInFrenchNumbers| redefines |numprint.sty|'s
%    command |\npstylefrench| to set |\npthousandsep| to |\,|
%    (thinspace) instead of |~| (default) . This option has no effect
%    if package |numprint.sty| is not loaded with `|autolanguage|'.
%    As old versions of |numprint.sty| did not define |\npstylefrench|,
%    we have to provide this command.
%    \begin{macrocode}
  \@ifpackageloaded{numprint}%
  {\ifnprt@autolanguage
     \providecommand*{\npstylefrench}{}%
     \ifFBThinSpaceInFrenchNumbers
       \renewcommand*\npstylefrench{%
          \npthousandsep{\,}%
          \npdecimalsign{,}%
          \npproductsign{\cdot}%
          \npunitseparator{\,}%
          \npdegreeseparator{}%
          \nppercentseparator{\nprt@unitsep}%
          }%
     \else
       \renewcommand*\npstylefrench{%
          \npthousandsep{~}%
          \npdecimalsign{,}%
          \npproductsign{\cdot}%
          \npunitseparator{\,}%
          \npdegreeseparator{}%
          \nppercentseparator{\nprt@unitsep}%
          }%
     \fi
     \npaddtolanguage{french}{french}%
   \fi}{}%
%    \end{macrocode}
%
%    |FrenchSuperscripts|: if |true| |\up=\fup|, else
%    |\up=\textsuperscript|. Anyway |\up*=\FB@up@fake|. The star-form
%    |\up*{}| is provided for fonts that lack some superior letters:
%    Adobe Jenson Pro and Utopia Expert have no ``g superior'' for
%    instance.
%    \begin{macrocode}
  \ifFBFrenchSuperscripts
    \DeclareRobustCommand*{\up}{\@ifstar{\FB@up@fake}{\fup}}%
  \else
    \DeclareRobustCommand*{\up}{\@ifstar{\FB@up@fake}%
                                        {\textsuperscript}}%
  \fi
%    \end{macrocode}
%
%    |LowercaseSuperscripts|: if |true| let |\FB@lc| be |\lowercase|,
%     else |\FB@lc| is redefined to do nothing.
%    \begin{macrocode}
  \ifFBLowercaseSuperscripts
  \else
    \renewcommand*{\FB@lc}[1]{##1}%
  \fi
%    \end{macrocode}
%
%    |PartNameFull|: if |false|, redefine |\partname|.
%    \begin{macrocode}
  \ifFBPartNameFull
  \else\addto\captionsfrench{\def\partname{Partie}}\fi
%    \end{macrocode}
%
%    |ShowOptions|: if |true|, print the list of all options to the
%    \file{.log} file.
%    \begin{macrocode}
  \ifFBShowOptions
    \GenericWarning{* }{%
     * **** List of possible options for frenchb ****\MessageBreak
     [Default values between brackets when frenchb is loaded *LAST*]%
     \MessageBreak
     ShowOptions=true [false]\MessageBreak
     StandardLayout=true [false]\MessageBreak
     GlobalLayoutFrench=false [true]\MessageBreak
     StandardLists=true [false]\MessageBreak
     ReduceListSpacing=false [true]\MessageBreak
     CompactItemize=false [true]\MessageBreak
     StandardItemLabels=true [false]\MessageBreak
     ItemLabels=\textemdash, \textbullet,
        \protect\ding{43},... [\textendash]\MessageBreak
     ItemLabeli=\textemdash, \textbullet,
        \protect\ding{43},... [\textendash]\MessageBreak
     ItemLabelii=\textemdash, \textbullet,
        \protect\ding{43},... [\textendash]\MessageBreak
     ItemLabeliii=\textemdash, \textbullet,
        \protect\ding{43},... [\textendash]\MessageBreak
     ItemLabeliv=\textemdash, \textbullet,
        \protect\ding{43},... [\textendash]\MessageBreak
     IndentFirst=false [true]\MessageBreak
     FrenchFootnotes=false [true]\MessageBreak
     AutoSpaceFootnotes=false [true]\MessageBreak
     AutoSpacePunctuation=false [true]\MessageBreak
     OriginalTypewriter=true [false]\MessageBreak
     ThinColonSpace=true [false]\MessageBreak
     ThinSpaceInFrenchNumbers=true [false]\MessageBreak
     FrenchSuperscripts=false [true]\MessageBreak
     LowercaseSuperscripts=false [true]\MessageBreak
     PartNameFull=false [true]\MessageBreak
     og= <left quote character>, fg= <right quote character>
     \MessageBreak
     *********************************************
     \MessageBreak\protect\frenchbsetup{ShowOptions}}
  \fi
}
%    \end{macrocode}
%  \end{macro}
%
% \changes{v2.0}{2006/12/15}{AtBeginDocument, save again the
%    definitions of the `list' and `itemize' environments and the
%    values of labelitems.  As of frenchb v.1.6, `ORI' values were
%    set when reading frenchb.ldf, later changes were ignored.}
%
% \changes{v2.0}{2006/12/06}{Added warning for OT1 encoding.}
%
% \changes{v2.1b}{2008/04/07}{Disable some commands in bookmarks.}
%
%    At |\begin{document}| we save again the definitions of the `list'
%    and `itemize' environments and the values of labelitems so that
%    all changes made in the preamble are taken into account in
%    languages other than French and in French with the StandardLayout
%    option.  We also have to provide an |\xspace| command in case the
%    |xspace.sty| package is not loaded.
%
%    \begin{macrocode}
\AtBeginDocument{%
   \let\listORI\list
   \let\endlistORI\endlist
   \let\itemizeORI\itemize
   \let\enditemizeORI\enditemize
   \let\@ltiORI\labelitemi
   \let\@ltiiORI\labelitemii
   \let\@ltiiiORI\labelitemiii
   \let\@ltivORI\labelitemiv
   \providecommand*{\xspace}{\relax}%
%    \end{macrocode}
%    Let's redefine some commands in \file{hyperref}'s bookmarks.
%    \begin{macrocode}
   \@ifundefined{pdfstringdefDisableCommands}{}%
     {\pdfstringdefDisableCommands{%
        \let\up\relax
        \def\ieme{e\xspace}%
        \def\iemes{es\xspace}%
        \def\ier{er\xspace}%
        \def\iers{ers\xspace}%
        \def\iere{re\xspace}%
        \def\ieres{res\xspace}%
        \def\FrenchEnumerate#1{#1\degre\space}%
        \def\FrenchPopularEnumerate#1{#1\degre)\space}%
        \def\No{N\degre\space}%
        \def\no{n\degre\space}%
        \def\Nos{N\degre\space}%
        \def\nos{n\degre\space}%
        \def\og{\guillemotleft\space}%
        \def\fg{\space\guillemotright}%
        \let\bsc\textsc
        \let\degres\degre
     }}%
%    \end{macrocode}
%    It is time to process the options set with |\frenchboptions{}|.
%    Then execute either |\extrasfrench| and |\captionsfrench| or
%    |\noextrasfrench| according to the current language at the
%    |\begin{document}| (these three commands are updated by
%    |\FBprocess@options|).
%    \begin{macrocode}
   \FBprocess@options
   \iflanguage{french}{\extrasfrench\captionsfrench}{\noextrasfrench}%
%    \end{macrocode}
%    Some warnings are issued when output font encodings are not
%    properly set. With XeLaTeX, \file{fontspec.sty} and
%    \file{xunicode.sty} should be loaded; with (pdf)\LaTeX, a warning
%    is issued when OT1 encoding is in use at the |\begin{document}|.
%    Mind that |\encodingdefault| is defined as `long', defining
%    |\FBOTone| with |\newcommand*| would fail!
%    \begin{macrocode}
   \expandafter\ifx\csname XeTeXrevision\endcsname\relax
      \begingroup \newcommand{\FBOTone}{OT1}%
      \ifx\encodingdefault\FBOTone
        \PackageWarning{frenchb.ldf}%
           {OT1 encoding should not be used for French.
            \MessageBreak
            Add \protect\usepackage[T1]{fontenc} to the
            preamble\MessageBreak of your document,}
      \fi
     \endgroup
   \else
     \@ifundefined{DeclareUTFcharacter}%
       {\PackageWarning{frenchb.ldf}%
         {Add \protect\usepackage{fontspec} *and*\MessageBreak
          \protect\usepackage{xunicode} to the preamble\MessageBreak
          of your document,}}%
       {}%
    \fi
}
%    \end{macrocode}
%
%  \subsection{Clean up and exit}
%
%    Load |frenchb.cfg| (should do nothing, just for compatibility).
%    \begin{macrocode}
\loadlocalcfg{frenchb}
%    \end{macrocode}
%    Final cleaning.
%    The macro |\ldf@quit| takes care for setting the main language
%    to be switched on at |\begin{document}| and resetting the
%    category code of \texttt{@} to its original value.
%    The config file searched for has to be |frenchb.cfg|, and
%    |\CurrentOption| has been set to `french', so
%    |\ldf@finish\CurrentOption| cannot be used: we first load
%    |frenchb.cfg|, then call |\ldf@quit\CurrentOption|.
%    \begin{macrocode}
\FBclean@on@exit
\ldf@quit\CurrentOption
%    \end{macrocode}
% \iffalse
%</code>
%<*dtx>
% \fi
%%
%% \CharacterTable
%%  {Upper-case    \A\B\C\D\E\F\G\H\I\J\K\L\M\N\O\P\Q\R\S\T\U\V\W\X\Y\Z
%%   Lower-case    \a\b\c\d\e\f\g\h\i\j\k\l\m\n\o\p\q\r\s\t\u\v\w\x\y\z
%%   Digits        \0\1\2\3\4\5\6\7\8\9
%%   Exclamation   \!     Double quote  \"     Hash (number) \#
%%   Dollar        \$     Percent       \%     Ampersand     \&
%%   Acute accent  \'     Left paren    \(     Right paren   \)
%%   Asterisk      \*     Plus          \+     Comma         \,
%%   Minus         \-     Point         \.     Solidus       \/
%%   Colon         \:     Semicolon     \;     Less than     \<
%%   Equals        \=     Greater than  \>     Question mark \?
%%   Commercial at \@     Left bracket  \[     Backslash     \\
%%   Right bracket \]     Circumflex    \^     Underscore    \_
%%   Grave accent  \`     Left brace    \{     Vertical bar  \|
%%   Right brace   \}     Tilde         \~}
%%
% \iffalse
%</dtx>
% \fi
%
% \Finale
\endinput
}
\DeclareOption{catalan}{% \iffalse meta-com

% Copyright 1989-2005 Johannes L. Braams and any individual aut
% listed elsewhere in this file.  All rights reser

% This file is part of the Babel sys
% ----------------------------------

% It may be distributed and/or modified under
% conditions of the LaTeX Project Public License, either version
% of this license or (at your option) any later vers
% The latest version of this license i
%   http://www.latex-project.org/lppl
% and version 1.3 or later is part of all distributions of L
% version 2003/12/01 or la

% This work has the LPPL maintenance status "maintain

% The Current Maintainer of this work is Johannes Bra

% The list of all files belonging to the Babel syste
% given in the file `manifest.bbl. See also `legal.bbl' for additi
% informat

% The list of derived (unpacked) files belonging to the distribu
% and covered by LPPL is defined by the unpacking scripts (
% extension .ins) which are part of the distribut
%
% \CheckSum{

% \iff
%    Tell the \LaTeX\ system who we are and write an entry on
%    transcr
%<*
\ProvidesFile{catalan.
%</
%<code>\ProvidesLanguage{cata

%\ProvidesFile{catalan.
        [2005/03/29 v2.2p Catalan support from the babel sys
%\iff
%% File `catalan.
%% Babel package for LaTeX versio
%% Copyright (C) 1989 -
%%           by Johannes Braams, TeX

%% Catalan Language Definition
%% Copyright (C) 1991 -
%%           by Goncal Badenes <badenes at imec
%%              Johannes Braams, TeX

%% Please report errors to: J.L. Braams babel at braams.cistro

%    This file is part of the babel system, it provides the so
%    code for the Catalan language definition f
%    This file was developped out of spanish.sty and suggestion
%    Goncal Badenes <badenes at imec.be> and Joerg Kna
%    <knappen at vkpmzd.kph.uni-mainz.

%    The file spanish.sty was written by Julio Sanc
%    (jsanchez@gmv.es) The code for the catalan language has
%    removed and now is in this f
%<*filedri
\documentclass{ltx
\newcommand*\TeXhax{\TeX
\newcommand*\babel{\textsf{bab
\newcommand*\langvar{$\langle \it lang \rang
\newcommand*\note[
\newcommand*\Lopt[1]{\textsf{
\newcommand*\file[1]{\texttt{
\begin{docum
 \DocInput{catalan.
\end{docum
%</filedri


% \GetFileInfo{catalan.

% \changes{catalan-2.0b}{1993/09/23}{Incorporated the changes
%    \file{spanish.s
% \changes{catalan-2.1}{1994/02/27}{Update for \LaT
% \changes{catalan-2.1d}{1994/06/26}{Removed the use of \cs{filed
%    and moved identification after the loading of \file{babel.d
% \changes{catalan-2.2b}{1995/07/04}{Made the activation of the g
%    and acute accents optio
% \changes{catalan-2.2c}{1995/07/08}{Removed the use of the tilde
%    cata
% \changes{catalan-2.2f}{1996/07/13}{Replaced \cs{undefined}
%    \cs{@undefined} and \cs{empty} with \cs{@empty} for consist
%    with \La
% \changes{catalan-2.2g}{1996/10/10}{Moved the definitio
%    \cs{atcatcode} right to the beginni
% \changes{catalan-2.2k}{1999/05/05}{A wrong \cs{changes} entry
%    typesetting impossi

%  \section{The Catalan langu

%    The file \file{\filename}\footnote{The file described in
%    section has version number \fileversion\ and was last revise
%    \filedate.}  defines all the language-specific macro's for
%    Catalan langu

%    For this language only the double quote character (|"|) is
%    active by default. In table~\ref{tab:catalan-quote-def
%    overview is given of the new macros defined and the new mean
%    of |"|. Additionally to that, the user can explicitly acti
%    the acute accent or apostrophe (|'|) and/or the grave ac
%    (|`|) characters by using the \Lopt{activeacute}
%    \Lopt{activegrave} options. In that case, the definitions s
%    in table~\ref{tab:catalan-quote-opt} also be
%    available\footnote{Please note that if the acute accent chara
%    is active, it is necessary to take special care of co
%    apostrophes in a way which cannot be confounded
%    accents. Therefore, it is necessary to type \texttt{l'\{\}es
%    instead of \texttt{l'estri

%    \begin{table}[
%     \cente
%     \begin{tabular}{lp{8
%      |\l.l|   & geminated-l digraph (simila
%               l$\cdot$l). |\L.L| produces the uppercase versio
%      |\lgem|  & geminated-l digraph (simila
%               l$\cdot$l). |\Lgem| produces the uppercase versio
%      |\up| & Macro to help typing raised ordinals, like {1\r
%               1ex\hbox{\small er}}. Takes one argumen
%      |\-| & like the old |\-|, but allowing hyphena
%               in the rest of the word
%      |"i| & i with diaeresis, allowing hyphena
%             in the rest of the word. Valid for the following vow
%               i, u (both lowercase and uppercase
%      |"c| & c-cedilla (\c{c}). Valid for both uppercase
%               lowercase
%      |"l| & geminated-l digraph (simila
%               l$\cdot$l). Valid for both uppercase and lowercase
%      |"<| & French left double quotes (similar to $<<$
%      |">| & French right double quotes (similar to $>>$
%      |"-| & explicit hyphen sign, allowing hyphena
%               in the rest of the wor
%      \verb="|= & disable ligature at this posit
%     \end{tabu
%     \caption{Extra definitions made by file \file{catalan.
%       (activated by defau
%     \label{tab:catalan-quote-
%    \end{ta

%    \begin{table}[
%     \cente
%     \begin{tabular}{lp{8
%      |'e| & acute accented a, allowing hyphena
%             in the rest of the word. Valid for the follo
%             vowels: e, i, o, u (both lowercase and uppercase
%      |`a| & grave accented a, allowing hyphena
%             in the rest of the word. Valid for the follo
%             vowels: a, e, o (both lowercase and upperca
%     \end{tabu
%     \caption{Extra definitions made by file \file{catalan.
%       (activated only when using the options \Lopt{activeacute}
%       \Lopt{activegrav
%     \label{tab:catalan-quote-
%    \end{ta
%    These active accents characters behave according to their orig
%    definitions if not followed by one of the characters indicate
%    that ta

% \StopEventual

% \changes{catalan-2.0}{1993/07/11}{Removed code to
%    \file{latexhax.c

%    The macro |\LdfInit| takes care of preventing that this fil
%    loaded more than once, checking the category code of
%    \texttt{@} sign,
% \changes{catalan-2.2g}{1996/11/02}{Now use \cs{LdfInit} to per
%    initial chec
%    \begin{macroc
%<*c
\LdfInit{catalan}\captionscat
%    \end{macroc

%    When this file is read as an option, i.e. by the |\usepack
%    command, \texttt{catalan} could be an `unknown' language in w
%    case we have to make it known.  So we check for the existenc
%    |\l@catalan| to see whether we have to do something h

% \changes{catalan-2.1d}{1994/06/26}{Now use \cs{@nopatterns
%    produce the warn
%    \begin{macroc
\ifx\l@catalan\@undef
  \@nopatterns{Cata
  \adddialect\l@cata

%    \end{macroc

%    The next step consists of defining commands to switch to
%    from) the Catalan langu

%  \begin{macro}{\catalanhyphenm
%    This macro is used to store the correct values of the hyphena
%    parameters |\lefthyphenmin| and |\righthyphenm
% \changes{catalan-2.2n}{2001/02/19}{Set the hyphenation parame
%    both to two as required by \texttt{cahyph.t
%    \begin{macroc
\providehyphenmins{catalan}{\tw@\
%    \end{macroc
%  \end{ma

% \begin{macro}{\captionscata
%    The macro |\captionscatalan| defines all strings
%    in the four standard documentclasses provided with \La
% \changes{catalan-1.1}{1993/07/11}{\cs{headpagename} shoul
%    \cs{pagena
% \changes{catalan-2.0}{1993/07/11}{Added some na
% \changes{catalan-2.1d}{1994/11/09}{Added a few missing translati
% \changes{catalan-2.2b}{1995/07/03}{Added \cs{proofname}
%    AMS-\La
% \changes{catalan-2.2d}{1995/07/10}{added translation of Pr
% \changes{catalan-2.2d}{1995/11/15}{Translations revi
% \changes{catalan-2.2m}{2000/09/19}{Added \cs{glossaryna
% \changes{catalan-2.2p}{2003/11/17}{Inserted translation
%    Gloss
%    \begin{macroc
\addto\captionscatal
  \def\prefacename{Pr\`ol
  \def\refname{Refer\`enci
  \def\abstractname{Res
  \def\bibname{Bibliograf
  \def\chaptername{Cap\'{\i}t
  \def\appendixname{Ap\`end
  \def\contentsname{\'Ind
  \def\listfigurename{\'Index de figur
  \def\listtablename{\'Index de taul
  \def\indexname{\'Index alfab\`et
  \def\figurename{Figu
  \def\tablename{Tau
  \def\partname{Pa
  \def\enclname{Adju
  \def\ccname{C\`opies
  \def\headtoname
  \def\pagename{P\`agi
  \def\seename{Veg
  \def\alsoname{Vegeu tamb\
  \def\proofname{Demostraci\
  \def\glossaryname{Glossa

%    \end{macroc
% \end{ma

% \begin{macro}{\datecata
%    The macro |\datecatalan| redefines the command |\today
%    produce Catalan dates. Months are writte
%    lowercase\footnote{This seems to be the common practice. See
%    example: E.~Coromina, \emph{El 9 Nou: Manual de redacci\
%    estil}, Ed.~Eumo, Vic, 19
% \changes{catalan-2.2b}{1995/06/18}{Month names in lowerc
% \changes{catalan-2.2i}{1997/10/01}{Use \cs{edef} to define \cs{to
%    to save mem
% \changes{catalan-2.2i}{1998/03/28}{use \cs{def} instead of \cs{ed
%    \begin{macroc
\def\datecatal
  \def\today{\number\day~\ifcase\mont
    de gener\or de febrer\or de mar\c{c}\or d'abril\or de mai
    de juny\or de juliol\or d'agost\or de setembre\or d'octubr
    de novembre\or de desembr
    \space de~\number\ye
%    \end{macroc
% \end{ma

% \begin{macro}{\extrascata
% \changes{catalan-2.0}{1993/07/11}{Macro completely rewrit
% \changes{catalan-2.2a}{1995/03/11}{Handling of active charac
%    completely rewrit

% \begin{macro}{\noextrascata
% \changes{catalan-2.0}{1993/07/11}{Macro completely rewrit

%    The macro |\extrascatalan| will perform all the extra definit
%    needed for the Catalan language.  The macro |\noextrascatalan
%    used to cancel the actions of |\extrascatal

% \changes{catalan-2.2e}{1995/11/10}{Now give the apostrop
%    lowercase c
%    To improve hyphenation we give the grave character (\texttt{'
%    non-zero lower case code; when we do that \TeX\ will find
%    breakpoints in words that contain this character in its r\^ol
%    apostrop
%    \begin{macroc
\addto\extrascatal
  \lccode`'
\addto\noextrascatal
  \lccode`
%    \end{macroc

%    For Catalan, some characters are made active or are redefined
%    particular, the \texttt{"} character receives a new meaning;
%    can also happen for the \texttt{'} character and the \textt
%    character when the options \Lopt{activegrave} an
%    \Lopt{activeacute} are specif

% \changes{catalan-2.2b}{1995/07/07}{Make activating the ac
%    characters optio
% \changes{catalan-2.2e}{1995/08/17}{Need to split up
%    \cs{@ifpackagewith} stateme
%    \begin{macroc
\addto\extrascatalan{\languageshorthands{catal
\initiate@active@cha
\addto\extrascatalan{\bbl@activate
%    \end{macroc
%    Because the grave character is being used in constructs suc
%    |\catcode``=\active| it needs to have it's original category cod
%    when the auxiliary file is being read. Note that this fil
%    read twice, once at the beginning of the document; then ther
%    no problem; but the second time it is read at the end of
%    document to check whether any labels changes. It's this se
%    time round that the actived grave character leads to e
%    messa
% \changes{catalan-2.2l}{1999/11/29}{Make sure that the grave ac
%    has catcode 12 \emph{before} it is made \cs{acti
%    \begin{macroc
\@ifpackagewith{babel}{activegrav
  \AtBeginDocume
    \if@filesw\immediate\write\@auxout{\catcode096=12}
  \initiate@active@char

\@ifpackagewith{babel}{activegrav
  \addto\extrascatalan{\bbl@activate{

\@ifpackagewith{babel}{activeacut
  \initiate@active@char

\@ifpackagewith{babel}{activeacut
  \addto\extrascatalan{\bbl@activate{

%    \end{macroc
%    Now make sure that the characters that have been turned
%    shorthanfd characters expand to `normal' characters outside
%    catalan environm
% \changes{catalan-2.2l}{1999/12/16}{Don't forget do deactivate
%    shortha
%    \begin{macroc
\addto\noextrascatalan{\bbl@deactivate
\@ifpackagewith{babel}{activegrav
  \addto\noextrascatalan{\bbl@deactivate{`}
\@ifpackagewith{babel}{activeacut
  \addto\noextrascatalan{\bbl@deactivate{'}
%    \end{macroc

% \changes{catalan-2.2a}{1995/03/11}{All the code for handling ac
%    characters is now moved to \file{babel.d

%    Apart from the active characters some other macros get a
%    definition. Therefore we store the current ones to be
%    to restore them la
%    When their current meanings are saved, we can safely rede
%    t

%    We provide new definitions for the accent macros when on
%    both of the options \Lopt{activegrave} or \Lopt{activeac
%    were specif

% \changes{catalan-2.2h}{1997/01/08}{Added some comment sign
%    prevent unwanted spaces in the outp
%    \begin{macroc
\addto\extrascatal
  \babel@sav
  \def\"{\protect\@umlau
\@ifpackagewith{babel}{activegrav
  \babel@sav
  \addto\extrascatalan{\def\`{\protect\@gra

\@ifpackagewith{babel}{activeacut
  \babel@sav
  \addto\extrascatalan{\def\'{\protect\@acu

%    \end{macroc
% \end{ma
% \end{ma

%    All the code above is necessary because we need a few e
%    active characters. These characters are then used as indicate
%    tables~\ref{tab:catalan-quote-
%    and~\ref{tab:catalan-quote-o

%  \begin{macro}{\diere
%  \begin{macro}{\textac
% \changes{catalan-2.1d}{1994/06/26}{Renamed from \cs{acute} as
%    is a \cs{mathacce
%  \begin{macro}{\textgr

%    The original definition of |\"| is stored as |\dieresis|, bec
%    the definition of |\"| might not be the default plain \
%    one. If the user uses \textsc{PostScript} fonts with the A
%    font encoding the \texttt{"} character is not in the
%    position as in Knuth's font encoding. In this case |\"| will
%    be defined as |\accent"7F 1|, but as |\accent'310 #1|. Somet
%    similar happens when using fonts that follow the
%    encoding. For this reason we save the definition of |\"| and
%    that in the definition of other macros. We do likewise for |
%    and |
%    \begin{macroc
\let\dieres
\@ifpackagewith{babel}{activegrave}{\let\textgrave\
\@ifpackagewith{babel}{activeacute}{\let\textacute\
%    \end{macroc
%  \end{ma
%  \end{ma
%  \end{ma

%  \begin{macro}{\@uml
%  \begin{macro}{\@ac
%  \begin{macro}{\@gr
%    We check the encoding and if not using T1, we make the acc
%    expand but enabling hyphenation beyond the accent. If this is
%    case, not all break positions will be found in words that con
%    accents, but this is a limitation in \TeX. An unsolved pro
%    here is that the encoding can change at any time. The definit
%    below are made in such a way that a change between two 256-
%    encodings are supported, but changes between a 128-char a
%    256-char encoding are not properly supported. We check if T
%    in use. If not, we will give a warning and proceed redefining
%    accent macros so that \TeX{} at least finds the breaks that
%    not too close to the accent. The warning will only be printe
%    the log f

%    \begin{macroc
\ifx\DeclareFontShape\@undef
  \wlog{Warning: You are using an old La
  \wlog{Some word breaks will not be fou
  \def\@umlaut#1{\allowhyphens\dieresis{#1}\allowhyph
  \@ifpackagewith{babel}{activeacut
    \def\@acute#1{\allowhyphens\textacute{#1}\allowhyphens
  \@ifpackagewith{babel}{activegrav
    \def\@grave#1{\allowhyphens\textgrave{#1}\allowhyphens
\
  \ifx\f@encoding\bbl@t
    \let\@umlaut\dier
    \@ifpackagewith{babel}{activeacut
      \let\@acute\textacut
    \@ifpackagewith{babel}{activegrav
      \let\@grave\textgrav
  \
    \wlog{Warning: You are using encoding \f@encoding\s
      instead of
    \wlog{Some word breaks will not be fou
    \def\@umlaut#1{\allowhyphens\dieresis{#1}\allowhyph
    \@ifpackagewith{babel}{activeacut
      \def\@acute#1{\allowhyphens\textacute{#1}\allowhyphens
    \@ifpackagewith{babel}{activegrav
      \def\@grave#1{\allowhyphens\textgrave{#1}\allowhyphens


%    \end{macroc
%    If the user setup has extended fonts, the Ferguson macros
%    required to be defined. We check for their existance and
%    defined, expand to whatever they are defined to. For insta
%    |\'a| would check for the existance of a |\@ac@a| macro. I
%    assumed to expand to the code of the accented letter.  If i
%    not defined, we assume that no extended codes are available
%    expand to the original definition but enabling hyphenation be
%    the accent. This is as best as we can do. It is better if
%    have extended fonts or ML-\TeX{} because the hyphena
%    algorithm can work on the whole word. The following macros
%    directly derived from ML-\TeX{}.\footnote{A problem is perce
%    here with these macros when used in a multilingual environ
%    where extended hyphenation patterns are available for some
%    not all languages. Assume that no extended patterns exist at
%    site for French and that \file{french.sty} would adopt
%    scheme too. In that case, \mbox{\texttt{'e}} in French w
%    produce the combined accented letter, but hyphenation aroun
%    would be suppressed. Both language options would nee
%    independent method to know whether they have extended patt
%    available. The precise impact of this problem and the poss
%    solutions are under stu
%  \end{ma
%  \end{ma
%  \end{ma

% \changes{catalan-2.2a}{1995/03/14}{All the code to deal with ac
%    characters is now in \file{babel.de

%    Now we can define our shorthands: the diaeresis and `
%    geminada'' supp
%    \begin{macroc
\declare@shorthand{catalan}{"i}{\textormath{\@umlaut\i}{\ddot\ima
\declare@shorthand{catalan}{"l}{\lge
\declare@shorthand{catalan}{"u}{\textormath{\@umlaut u}{\ddot
\declare@shorthand{catalan}{"I}{\textormath{\@umlaut I}{\ddot
\declare@shorthand{catalan}{"L}{\Lge
\declare@shorthand{catalan}{"U}{\textormath{\@umlaut U}{\ddot
%    \end{macroc
%    cedi
% \changes{catalan-2.2c}{1995/07/08}{cedile now produced by do
%    quote shorth
%    \begin{macroc
\declare@shorthand{catalan}{"c}{\textormath{\c c}{^{\prime}
\declare@shorthand{catalan}{"C}{\textormath{\c C}{^{\prime}
%    \end{macroc
%    `french' quote charact
% \changes{catalan-2.2c}{1995/07/08}{Added shorthands for guillem
% \changes{catalan-2.2i}{1997/04/03}{Removed empty groups a
%    guillemot characte
%    \begin{macroc
\declare@shorthand{catalan}{"
  \textormath{\guillemotleft}{\mbox{\guillemotlef
\declare@shorthand{catalan}{"
  \textormath{\guillemotright}{\mbox{\guillemotrigh
%    \end{macroc
%     grave acce
% \changes{catalan-2.2e}{1996/03/05}{Added `{}` as an axtra shorth
%    \begin{macroc
\@ifpackagewith{babel}{activegrav
  \declare@shorthand{catalan}{`a}{\textormath{\@grave a}{\grave
  \declare@shorthand{catalan}{`e}{\textormath{\@grave e}{\grave
  \declare@shorthand{catalan}{`o}{\textormath{\@grave o}{\grave
  \declare@shorthand{catalan}{`A}{\textormath{\@grave A}{\grave
  \declare@shorthand{catalan}{`E}{\textormath{\@grave E}{\grave
  \declare@shorthand{catalan}{`O}{\textormath{\@grave O}{\grave
  \declare@shorthand{catalan}{``}{\textquotedblleft

%    \end{macroc
%     acute acce
% \changes{catalan-2.2b}{1995/07/03}{Changed mathmode definitio
%    acute shorthands to expand to a single prime followed by the
%    character in the in
% \changes{catalan-2.2e}{1995/09/05}{Added vertical bar as argumen
%    active ac
%    \begin{macroc
\@ifpackagewith{babel}{activeacut
  \declare@shorthand{catalan}{'a}{\textormath{\@acute a}{^{\prime}
  \declare@shorthand{catalan}{'e}{\textormath{\@acute e}{^{\prime}
  \declare@shorthand{catalan}{'i}{\textormath{\@acute\i{}}{^{\prime}
  \declare@shorthand{catalan}{'o}{\textormath{\@acute o}{^{\prime}
  \declare@shorthand{catalan}{'u}{\textormath{\@acute u}{^{\prime}
  \declare@shorthand{catalan}{'A}{\textormath{\@acute A}{^{\prime}
  \declare@shorthand{catalan}{'E}{\textormath{\@acute E}{^{\prime}
  \declare@shorthand{catalan}{'I}{\textormath{\@acute I}{^{\prime}
  \declare@shorthand{catalan}{'O}{\textormath{\@acute O}{^{\prime}
  \declare@shorthand{catalan}{'U}{\textormath{\@acute U}{^{\prime}
  \declare@shorthand{catalan}{'
    \textormath{\csname normal@char\string'\endcsname}{^{\prim
%    \end{macroc
%         the acute acc
% \changes{catalan-2.2c}{1995/07/08}{Added '{}' as an axtra shorth
%    removed 'n as a shorth
%    \begin{macroc
  \declare@shorthand{catalan}{'
    \textormath{\textquotedblright}{\sp\bgroup\prim@

%    \end{macroc
%    and finally, some support definit
%    \begin{macroc
\declare@shorthand{catalan}{"-}{\nobreak-\bbl@allowhyph
\declare@shorthand{catalan}{"
  \textormath{\nobreak\discretionary{-}{}{\kern.03
              \allowhyphens
%    \end{macroc

%  \begin{macro}

%    All that is left now is the redefinition of |\-|. The new ver
%    of |\-| should indicate an extra hyphenation position, w
%    allowing other hyphenation positions to be gener
%    automatically. The standard behaviour of \TeX\ in this respec
%    unfortunate for Catalan but not as much as for Dutch or Ger
%    where long compound words are quite normal and all one needs
%    means to indicate an extra hyphenation position on top of
%    ones that \TeX\ can generate from the hyphena
%    patterns. However, the average length of words in Catalan m
%    this desirable and so it is kept h

%    \begin{macroc
\addto\extrascatal
  \babel@save{
  \def\-{\bbl@allowhyphens\discretionary{-}{}{}\bbl@allowhyphe
%    \end{macroc
%  \end{ma

%  \begin{macro}{\l
%  \begin{macro}{\L
% \changes{catalan-2.2b}{1995/06/18}{Added support for typing
%    catalan ``ela geminada'' with the macros \cs{lgem} and \cs{Lg
% \changes{catalan-2.2f}{1996/09/20}{Added a check for math mod
%    the use of \cs{lgem} and \cs{Lgem} in math mode is not sensib

%    Here we define a macro for typing the catalan ``ela gemina
%    (geminated l). The macros |\lgem| and |\Lgem| have been ch
%    for its lowercase and uppercase representat
%    respectively\footnote{The macro names \cs{ll} and \cs{LL}
%    not taken because of the fact that \cs{ll} is already use
%    mathematical mod

%    The code used in the actual macro used is a combination of
%    one proposed by Feruglio and Fuster\footnote{G.~Valiente
%    R.~Fuster, Typesetting Catalan Texts with \TeX, \emph{TUGb
%    \textbf{14}(3), 1993.} and the proposal\footnote{G. Valie
%    Modern Catalan Typographical Conventions, \emph{TUGb
%    \textbf{16}(3), 1995.} from Valiente presented at the \TeX\ U
%    Group Annual Meeting in 1995. This last proposal has not
%    fully implemented due to its limitation to CM fo
%    \begin{macroc
\newdimen\leftllkern \newdimen\rightllkern \newdimen\raisel
\def\lg
  \ifm
    \csname normal@char\string"\endcsnam
  \
    \leftllkern=0pt\rightllkern=0pt\raiselldim=
    \setbox0\hbox{l}\setbox1\hbox{l\/}\setbox2\hbox
    \advance\raiselldim by \the\fontdimen5\the\
    \advance\raiselldim by -\
    \leftllkern=-.25\
    \advance\leftllkern by \
    \advance\leftllkern by -\
    \rightllkern=-.25\
    \advance\rightllkern by -\
    \advance\rightllkern by \
    \allowhyphens\discretionary{l-}
    {\hbox{l}\kern\leftllkern\raise\raiselldim\hbox
      \kern\rightllkern\hbox{l}}\allowhyp


\def\Lg
  \ifm
    \csname normal@char\string"\endcsnam
  \
    \leftllkern=0pt\rightllkern=0pt\raiselldim=
    \setbox0\hbox{L}\setbox1\hbox{L\/}\setbox2\hbox
    \advance\raiselldim by .5\
    \advance\raiselldim by -.5\
    \leftllkern=-.125\
    \advance\leftllkern by \
    \advance\leftllkern by -\
    \rightllkern=-\
    \divide\rightllkern b
    \advance\rightllkern by -\
    \advance\rightllkern by \
    \allowhyphens\discretionary{L-}
    {\hbox{L}\kern\leftllkern\raise\raiselldim\hbox
      \kern\rightllkern\hbox{L}}\allowhyp


%    \end{macroc
%  \end{ma
%  \end{ma

%  \begin{macro}{\
%  \begin{macro}{\
% \changes{catalan-2.2e}{1996/06/26}{Added redefinition of \cs{l}
%    \cs
%    It seems to be the most natural way of entering the `
%    geminda'' to use the sequences |\l.l| and |\L.L|. These are
%    really macro's by themselves but the macros |\l| and |\L|
%    delimited arguments. Therefor we define two macros that chec
%    the next character is a period. If not the ``polish l'' wil
%    typeset, otherwise a ``ela geminada'' will be typeset and
%    next two tokens will be `eat
% \changes{catalan-2.2o}{2003/09/19}{Postpone the redefinitio
%    \cs{l} and \cs{L} until begin document to prevent overwritin
%    font
%    \begin{macroc
\AtBeginDocume
  \let\lsla
  \let\Lsla
  \DeclareRobustCommand\l{\@ifnextchar.\bbl@l\lsl
  \DeclareRobustCommand\L{\@ifnextchar.\bbl@L\Lsla
\def\bbl@l#1#2{\l
\def\bbl@L#1#2{\L
%    \end{macroc
%  \end{ma
%  \end{ma

%  \begin{macro}{

%    A macro for typesetting things like 1\raise1ex\hbox{\small er
%    proposed by Raymon Seroul\footnote{This macro has been borr
%    from francais.d
% \changes{catalan-2.2b}{1995/06/18}{Added definition of m
%    \cs{up}, which can be used to type ordin
% \changes{catalan-2.2e}{1996/02/29}{Now use \cs{textsuperscript}
%    make \cs{up} rob
%    \begin{macroc
\DeclareRobustCommand*{\up}[1]{\textsuperscript{
%    \end{macroc
%  \end{ma

%    The macro |\ldf@finish| takes care of looking f
%    configuration file, setting the main language to be switche
%    at |\begin{document}| and resetting the category cod
%    \texttt{@} to its original va
% \changes{catalan-2.2g}{1996/11/02}{Now use \cs{ldf@finish} to
%
%    \begin{macroc
\ldf@finish{cata
%</c
%    \end{macroc

% \Fi
%% \CharacterT
%%  {Upper-case    \A\B\C\D\E\F\G\H\I\J\K\L\M\N\O\P\Q\R\S\T\U\V\W\X
%%   Lower-case    \a\b\c\d\e\f\g\h\i\j\k\l\m\n\o\p\q\r\s\t\u\v\w\x
%%   Digits        \0\1\2\3\4\5\6\7
%%   Exclamation   \!     Double quote  \"     Hash (number
%%   Dollar        \$     Percent       \%     Ampersand
%%   Acute accent  \'     Left paren    \(     Right paren
%%   Asterisk      \*     Plus          \+     Comma
%%   Minus         \-     Point         \.     Solidus
%%   Colon         \:     Semicolon     \;     Less than
%%   Equals        \=     Greater than  \>     Question mar
%%   Commercial at \@     Left bracket  \[     Backslash
%%   Right bracket \]     Circumflex    \^     Underscore
%%   Grave accent  \`     Left brace    \{     Vertical bar
%%   Right brace   \}     Tilde

\endi
}
\DeclareOption{croatian}{% \iffalse meta-comment
%
% Copyright 1989-2005 Johannes L. Braams and any individual authors
% listed elsewhere in this file.  All rights reserved.
% 
% This file is part of the Babel system.
% --------------------------------------
% 
% It may be distributed and/or modified under the
% conditions of the LaTeX Project Public License, either version 1.3
% of this license or (at your option) any later version.
% The latest version of this license is in
%   http://www.latex-project.org/lppl.txt
% and version 1.3 or later is part of all distributions of LaTeX
% version 2003/12/01 or later.
% 
% This work has the LPPL maintenance status "maintained".
% 
% The Current Maintainer of this work is Johannes Braams.
% 
% The list of all files belonging to the Babel system is
% given in the file `manifest.bbl. See also `legal.bbl' for additional
% information.
% 
% The list of derived (unpacked) files belonging to the distribution
% and covered by LPPL is defined by the unpacking scripts (with
% extension .ins) which are part of the distribution.
% \fi
% \CheckSum{87}
% \iffalse
%    Tell the \LaTeX\ system who we are and write an entry on the
%    transcript.
%<*dtx>
\ProvidesFile{croatian.dtx}
%</dtx>
%<code>\ProvidesLanguage{croatian}
%\fi
%\ProvidesFile{croatian.dtx}
       [2005/03/29 v1.3l Croatian support from the babel system]
%\iffalse
%% File `croatian.dtx'
%% Babel package for LaTeX version 2e
%% Copyright (C) 1989 - 2005
%%           by Johannes Braams, TeXniek
%
%% Please report errors to: J.L. Braams
%%                          babel at braams.cistron.nl
%
%    This file is part of the babel system, it provides the source
%    code for the Croatian language definition file.  A contribution
%    was made by Alan Pai\'{c} (paica@cernvm.cern.ch)
%<*filedriver>
\documentclass{ltxdoc}
\newcommand*\TeXhax{\TeX hax}
\newcommand*\babel{\textsf{babel}}
\newcommand*\langvar{$\langle \it lang \rangle$}
\newcommand*\note[1]{}
\newcommand*\Lopt[1]{\textsf{#1}}
\newcommand*\file[1]{\texttt{#1}}
\begin{document}
 \DocInput{croatian.dtx}
\end{document}
%</filedriver>
%\fi
% \GetFileInfo{croatian.dtx}
%
% \changes{croatian-1.0a}{1991/07/15}{Renamed \file{babel.sty} in
%    \file{babel.com}}
% \changes{croatian-1.0c}{1992/01/25}{Removed some typos}
% \changes{croatian-1.1}{1992/02/15}{Brought up-to-date with babel 3.2a}
% \changes{croatian-1.3}{1994/02/27}{Update for \LaTeXe}
% \changes{croatian-1.3g}{1996/10/10}{Replaced \cs{undefined} with
%    \cs{@undefined} and \cs{empty} with \cs{@empty} for consistency
%    with \LaTeX, moved the definition of \cs{atcatcode} right to the
%    beginning.}
%
%  \section{The Croatian language}
%
%    The file \file{\filename}\footnote{The file described in this
%    section has version number \fileversion\ and was last revised on
%    \filedate.  A contribution was made by Alan Pai\'{c}
%    (\texttt{paica@cernvm.cern.ch}).}  defines all the
%    language definition macros for the Croatian language.
%
%    For this language currently no special definitions are needed or
%    available.
%
% \StopEventually{}
%
%    The macro |\LdfInit| takes care of preventing that this file is
%    loaded more than once, checking the category code of the
%    \texttt{@} sign, etc.
% \changes{croatian-1.3g}{1996/11/02}{Now use \cs{LdfInit} to perform
%    initial checks} 
%    \begin{macrocode}
%<*code>
\LdfInit{croatian}\captionscroatian
%    \end{macrocode}
%
%    When this file is read as an option, i.e. by the |\usepackage|
%    command, \texttt{croatian} will be an `unknown' language in which
%    case we have to make it known. So we check for the existence of
%    |\l@croatian| to see whether we have to do something here.
%
% \changes{croatian-1.0b}{1991/10/07}{Removed use of
%    \cs{@ifundefined}}
% \changes{croatian-1.1}{1992/02/15}{Added a warning when no
%    hyphenation patterns were loaded.}
%    \begin{macrocode}
\ifx\l@croatian\@undefined
    \@nopatterns{Croatian}
    \adddialect\l@croatian0\fi
%    \end{macrocode}
%
%    The next step consists of defining commands to switch to (and
%    from) the Croatian language.
%
%  \begin{macro}{\captionscroatian}
%    The macro |\captionscroatian| defines all strings used
%    in the four standard documentclasses provided with \LaTeX.
% \changes{croatian-1.1}{1992/02/15}{Added \cs{seename}, 
%    \cs{alsoname} and \cs{prefacename}}
% \changes{croatian-1.2}{1993/07/11}{\cs{headpagename} should be
%    \cs{pagename}}
% \changes{croatian-1.3d}{1995/05/08}{Added a few translations}
% \changes{croatian-1.3e}{1995/07/04}{Added \cs{proofname} for
%    AMS-\LaTeX}
% \changes{croatian-1.3f}{1996/01/18}{Added translation of Proof}
% \changes{croatian-1.3i}{1997/09/11}{Replaced some of the
%    translations with `better' words} 
% \changes{croatian-1.3k}{2000/09/19}{Added \cs{glossaryname}}
% \changes{croatian-1.3l}{2003/11/17}{Inserted translation for Glossary}
%    \begin{macrocode}
\addto\captionscroatian{%
  \def\prefacename{Predgovor}%
  \def\refname{Literatura}%
  \def\abstractname{Sa\v{z}etak}%
  \def\bibname{Bibliografija}%
  \def\chaptername{Poglavlje}%
  \def\appendixname{Dodatak}%
  \def\contentsname{Sadr\v{z}aj}%
  \def\listfigurename{Popis slika}%
  \def\listtablename{Popis tablica}%
  \def\indexname{Indeks}%
  \def\figurename{Slika}%
  \def\tablename{Tablica}%
  \def\partname{Dio}%
  \def\enclname{Prilozi}%
  \def\ccname{Kopije}%
  \def\headtoname{Prima}%
  \def\pagename{Stranica}%
  \def\seename{Vidjeti}%
  \def\alsoname{Vidjeti i}%
  \def\proofname{Dokaz}%
  \def\glossaryname{Kazalo}%
  }%
%    \end{macrocode}
%  \end{macro}
%
%  \begin{macro}{\datecroatian}
%    The macro |\datecroatian| redefines the command |\today| to
%    produce Croatian dates.
% \changes{croatian-1.3f}{1996/01/18}{in croatian language, the
%    genitive for the name of the month has to be used}
% \changes{croatian-1.3h}{1997/02/06}{\texttt{sijev\{c\}nja} should be
%    \texttt{seij\cs{v}\{c\}nja} and there should be a period after
%    the year}
% \changes{croatian-1.3i}{1997/10/01}{Use \cs{edef} to define
%    \cs{today} to save memory}
% \changes{croatian-1.3i}{1998/03/28}{use \cs{def} instead of
%    \cs{edef}}
% \changes{croatian-1.3j}{1999/03/12}{changed \cs{od} into \cs{or}}
%    \begin{macrocode}
\def\datecroatian{%
  \def\today{\number\day.~\ifcase\month\or
    sije\v{c}nja\or velja\v{c}e\or o\v{z}ujka\or travnja\or svibnja\or
    lipnja\or srpnja\or kolovoza\or rujna\or listopada\or studenog\or
    prosinca\fi \space \number\year.}}
%    \end{macrocode}
%  \end{macro}
%
%  \begin{macro}{\extrascroatian}
%  \begin{macro}{\noextrascroatian}
%    The macro |\extrascroatian| will perform all the extra
%    definitions needed for the Croatian language. The macro
%    |\noextrascroatian| is used to cancel the actions of
%    |\extrascroatian|.  For the moment these macros are empty but
%    they are defined for compatibility with the other language
%    definition files.
%
%    \begin{macrocode}
\addto\extrascroatian{}
\addto\noextrascroatian{}
%    \end{macrocode}
%  \end{macro}
%  \end{macro}
%
%    The macro |\ldf@finish| takes care of looking for a
%    configuration file, setting the main language to be switched on
%    at |\begin{document}| and resetting the category code of
%    \texttt{@} to its original value.
% \changes{croatian-1.3g}{1996/11/02}{Now use \cs{ldf@finish} to wrap
%    up} 
%    \begin{macrocode}
\ldf@finish{croatian}
%</code>
%    \end{macrocode}
%
% \Finale
%% \CharacterTable
%%  {Upper-case    \A\B\C\D\E\F\G\H\I\J\K\L\M\N\O\P\Q\R\S\T\U\V\W\X\Y\Z
%%   Lower-case    \a\b\c\d\e\f\g\h\i\j\k\l\m\n\o\p\q\r\s\t\u\v\w\x\y\z
%%   Digits        \0\1\2\3\4\5\6\7\8\9
%%   Exclamation   \!     Double quote  \"     Hash (number) \#
%%   Dollar        \$     Percent       \%     Ampersand     \&
%%   Acute accent  \'     Left paren    \(     Right paren   \)
%%   Asterisk      \*     Plus          \+     Comma         \,
%%   Minus         \-     Point         \.     Solidus       \/
%%   Colon         \:     Semicolon     \;     Less than     \<
%%   Equals        \=     Greater than  \>     Question mark \?
%%   Commercial at \@     Left bracket  \[     Backslash     \\
%%   Right bracket \]     Circumflex    \^     Underscore    \_
%%   Grave accent  \`     Left brace    \{     Vertical bar  \|
%%   Right brace   \}     Tilde         \~}
%%
\endinput
}
\DeclareOption{czech}{% \iffalse meta-comment
%
% Copyright 1989-2008 Johannes L. Braams and any individual authors
% listed elsewhere in this file.  All rights reserved.
% 
% This file is part of the Babel system.
% --------------------------------------
% 
% It may be distributed and/or modified under the
% conditions of the LaTeX Project Public License, either version 1.3
% of this license or (at your option) any later version.
% The latest version of this license is in
%   http://www.latex-project.org/lppl.txt
% and version 1.3 or later is part of all distributions of LaTeX
% version 2003/12/01 or later.
% 
% This work has the LPPL maintenance status "maintained".
% 
% The Current Maintainer of this work is Johannes Braams.
% 
% The list of all files belonging to the Babel system is
% given in the file `manifest.bbl. See also `legal.bbl' for additional
% information.
% 
% The list of derived (unpacked) files belonging to the distribution
% and covered by LPPL is defined by the unpacking scripts (with
% extension .ins) which are part of the distribution.
% \fi
% \CheckSum{1142}
%
% \iffalse
%    Tell the \LaTeX\ system who we are and write an entry on the
%    transcript.
%<*dtx>
\ProvidesFile{czech.dtx}
%</dtx>
%<+code>\ProvidesLanguage{czech}
%\fi
%\ProvidesFile{czech.dtx}
        [2008/07/06 v3.1a Czech support from the babel system]
%\iffalse
%% File `czech.dtx'
%% Babel package for LaTeX version 2e
%% Copyright (C) 1989 - 2008
%%           by Johannes Braams, TeXniek
%
%% Copyright (C) 2005, 2008
%%           by Petr Tesa\v r\'ik (babel at tesarici.cz)
%%
%% Czech Language Definition File
% This file is also based on CSLaTeX
%                       by Ji\v r\'i Zlatu\v ska, Zden\v ek Wagner,
%                          Jaroslav \v Snajdr and Petr Ol\v s\'ak.
%
%% Please report errors to: Petr Tesa\v r\'ik
%%                          babel at tesarici.cz
%
%    This file is part of the babel system, it provides the source
%    code for the Czech language definition file.
%<*filedriver>
\documentclass{ltxdoc}
\newcommand*\TeXhax{\TeX hax}
\newcommand*\babel{\textsf{babel}}
\newcommand*\langvar{$\langle \it lang \rangle$}
\newcommand*\note[1]{}
\newcommand*\Lopt[1]{\textsf{#1}}
\newcommand*\file[1]{\texttt{#1}}
\begin{document}
 \DocInput{czech.dtx}
\end{document}
%</filedriver>
%\fi
%
% \def\CS{$\cal C\kern-.1667em
%   \lower.5ex\hbox{$\cal S$}\kern-.075em$}
%
% \GetFileInfo{czech.dtx}
%
% \changes{czech-1.0a}{1991/07/15}{Renamed babel.sty in babel.com}
% \changes{czech-1.1}{1992/02/15}{Brought up-to-date with babel 3.2a}
% \changes{czech-1.2}{1993/07/11}{Included some features from Kasal's
%    czech.sty}
% \changes{czech-1.3}{1994/02/27}{Update for \LaTeXe}
% \changes{czech-1.3d}{1994/06/26}{Removed the use of \cs{filedate}
%    and moved identification after the loading of \file{babel.def}}
% \changes{czech-1.3h}{1996/10/10}{Replaced \cs{undefined} with
%    \cs{@undefined} and \cs{empty} with \cs{@empty} for consistency
%    with \LaTeX, moved the definition of \cs{atcatcode} right to the
%    beginning.}
% \changes{czech-3.0}{2005/09/10}{Implemented the functionality of
%    \CS\LaTeX's czech.sty.  The version number was bumped to 3.0
%    to minimize confusion by being higher than the last version
%    of \CS\LaTeX.}
%
%  \section{The Czech Language}
%
%    The file \file{\filename}\footnote{The file described in this
%    section has version number \fileversion\ and was last revised on
%    \filedate.  It was rewritten by Petr Tesa\v r\'ik
%    (\texttt{babel@tesarici.cz}).} defines all the language definition
%    macros for the Czech language.  It is meant as a replacement of
%    \CS\LaTeX, the most-widely used standard for typesetting Czech
%    documents in \LaTeX.
%
%  \subsection{Usage}
%    For this language |\frenchspacing| is set.
%
%    Additionally, two macros are defined  |\q| and |\w| for easy
%    access to two accents are defined.
%
%    The command |\q| is used with the letters (\texttt{t},
%    \texttt{d}, \texttt{l}, and \texttt{L}) and adds a \texttt{'} to
%    them to simulate a `hook' that should be there.  The result looks
%    like t\kern-2pt\char'47. The command |\w| is used to put the
%    ring-accent which appears in \aa ngstr\o m over the letters
%    \texttt{u} and \texttt{U}.
%
%  \subsection{Compatibility}
%
%    Great care has been taken to ensure backward compatibility with
%    \CS\LaTeX.  In particular, documents which load this file with
%    |\usepackage{czech}| should produce identical output with no
%    modifications to the source.  Additionally, all the \CS\LaTeX{}
%    options are recognized:
%
%    \label{tab:cslatex-options}
%    \begin{list}{}
%     {\def\makelabel#1{\sbox0{\Lopt{#1}}%
%        \ifdim\wd0>\labelwidth
%          \setbox0\vbox{\box0\hbox{}} \wd0=0pt \fi
%        \box0\hfil}
%      \setlength{\labelwidth}{2\parindent}
%      \setlength{\leftmargin}{2\parindent}
%      \setlength{\rightmargin}{\parindent}}
%     \item[IL2, T1, OT1]
%       These options set the default font encoding.  Please note
%       that their use is deprecated. You should use the |fontenc|
%       package to select font encoding.
%
%     \item[split, nosplit]
%       These options control whether hyphenated words are
%       automatically split according to Czech typesetting rules.
%       With the \Lopt{split} option ``je-li'' is hyphenated as
%       ``je-/-li''. The \Lopt{nosplit} option disables this behavior.
%
%       The use of this option is strongly discouraged, as it breaks
%       too many common things---hyphens cannot be used in labels,
%       negative arguments to \TeX{} primitives will not work in
%       horizontal mode (use \cs{minus} as a workaround), and there are
%       a few other peculiarities with using this mode.
%
%     \item[nocaptions]
%
%       This option was used in \CS\LaTeX{} to set up Czech/Slovak
%       typesetting rules, but leave the original captions and dates.
%       The recommended way to achieve this is to use English as the main
%       language of the document and use the environment |otherlanguage*|
%       for Czech text.
%
%     \item[olduv]
%       There are two version of \cs{uv}.  The older one allows the use
%       of \cs{verb} inside the quotes but breaks any respective kerning
%       with the quotes (like that in \CS{} fonts).  The newer one honors
%       the kerning in the font but does not allow \cs{verb} inside the
%       quotes.
%
%       The new version is used by default in \LaTeXe{} and the old version
%       is used with plain \TeX.  You may use \Lopt{olduv} to override the
%       default in \LaTeXe.
%       
%     \item[cstex]
%       This option was used to include the commands \cs{csprimeson} and
%       \cs{csprimesoff}.  Since these commands are always included now,
%       it has been removed and the empty definition lasts for compatibility.
%    \end{list}
%
% \StopEventually{}
%
%  \subsection{Implementation}
%
%    The macro |\LdfInit| takes care of preventing that this file is
%    loaded more than once, checking the category code of the
%    \texttt{@} sign, etc.
% \changes{czech-1.3h}{1996/11/02}{Now use \cs{LdfInit} to perform
%    initial checks} 
%    \begin{macrocode}
%<*code>
\LdfInit\CurrentOption{date\CurrentOption}
%    \end{macrocode}
%
%    When this file is read as an option, i.e. by the |\usepackage|
%    command, \texttt{czech} might be an `unknown' language in which
%    case we have to make it known. So we check for the existence of
%    |\l@czech| to see whether we have to do something here.
%
% \changes{czech-1.0b}{1991/10/27}{Removed use of \cs{@ifundefined}}
% \changes{czech-1.1}{1992/02/15}{Added a warning when no hyphenation
%    patterns were loaded.}
% \changes{czech-1.3d}{1994/06/26}{Now use \cs{@nopatterns} to produce
%    the warning}
%    \begin{macrocode}
\ifx\l@czech\@undefined
    \@nopatterns{Czech}
    \adddialect\l@czech0\fi
%    \end{macrocode}
%
%    We need to define these macros early in the process.
%
%    \begin{macrocode}
\def\cs@iltw@{IL2}
\newif\ifcs@splithyphens
\cs@splithyphensfalse
%    \end{macrocode}
%
%    If Babel is not loaded, we provide compatibility with \CS\LaTeX.
%    However, if macro \cs{@ifpackageloaded} is not defined, we assume
%    to be loaded from plain and provide compatibility with csplain.
%    Of course, this does not work well with \LaTeX$\:$2.09, but I
%    doubt anyone will ever want to use this file with \LaTeX$\:$2.09.
%
%    \begin{macrocode}
\ifx\@ifpackageloaded\@undefined
  \let\cs@compat@plain\relax
  \message{csplain compatibility mode}
\else
  \@ifpackageloaded{babel}{}{%
    \let\cs@compat@latex\relax
    \message{cslatex compatibility mode}}
\fi
\ifx\cs@compat@latex\relax
  \ProvidesPackage{czech}[2008/07/06 v3.1a CSTeX Czech style]
%    \end{macrocode}
%
%    Declare \CS\LaTeX{} options (see also the descriptions on page
%    \pageref{tab:cslatex-options}).
%
%    \begin{macrocode}
  \DeclareOption{IL2}{\def\encodingdefault{IL2}}
  \DeclareOption {T1}{\def\encodingdefault {T1}}
  \DeclareOption{OT1}{\def\encodingdefault{OT1}}
  \DeclareOption{nosplit}{\cs@splithyphensfalse}
  \DeclareOption{split}{\cs@splithyphenstrue}
  \DeclareOption{nocaptions}{\let\cs@nocaptions=\relax}
  \DeclareOption{olduv}{\let\cs@olduv=\relax}
  \DeclareOption{cstex}{\relax}
%    \end{macrocode}
%
%    Make |IL2| encoding the default.  This can be overriden with
%    the other font encoding options.
%    \begin{macrocode}
  \ExecuteOptions{\cs@iltw@}
%    \end{macrocode}
%
%    Now, process the user-supplied options.
%    \begin{macrocode}
  \ProcessOptions
%    \end{macrocode}
%
%    Standard \LaTeXe{} does not include the IL2 encoding in the format.
%    The encoding can be loaded later using the fontenc package, but
%    \CS\LaTeX{} included IL2 by default.  This means existing documents
%    for \CS\LaTeX{} do not load that package, so load the encoding
%    ourselves in compatibility mode.
%
%    \begin{macrocode}
  \ifx\encodingdefault\cs@iltw@
    \input il2enc.def
  \fi
%    \end{macrocode}
%
%    Restore the definition of \cs{CurrentOption}, clobbered by processing
%    the options.
%
%    \begin{macrocode}
  \def\CurrentOption{czech}
\fi
%    \end{macrocode}
%
%    The next step consists of defining commands to switch to (and
%    from) the Czech language.
%
%  \begin{macro}{\captionsczech}
%    The macro \cs{captionsczech} defines all strings used in the four
%    standard documentclasses provided with \LaTeX.
%
% \changes{czech-1.1}{1992/02/15}{Added \cs{seename}, \cs{alsoname}
%    and \cs{prefacename}}
% \changes{czech-1.3f}{1995/07/04}{Added \cs{proofname} for AMS-\LaTeX}
% \changes{czech-1.3g}{1996/02/12}{Fixed two errors and provided
%    translation for `proof'} 
% \changes{czech-1.3j}{2000/09/19}{Added \cs{glossaryname}}
% \changes{czech-1.3k}{2004/02/18}{Added translation for Glossary}
% \changes{czech-3.0}{2005/11/25}{Updated some translations.  Former
%    translations were: `Dodatek' for \cs{appendixname} and `Index'
%    for \cs{indexname}. Also removed spurious colon at the end of
%    \cs{ccname}.}
%
%    \begin{macrocode}
\@namedef{captions\CurrentOption}{%
  \def\prefacename{P\v{r}edmluva}%
  \def\refname{Reference}%
  \def\abstractname{Abstrakt}%
  \def\bibname{Literatura}%
  \def\chaptername{Kapitola}%
  \def\appendixname{P\v{r}\'{\i}loha}%
  \def\contentsname{Obsah}%
  \def\listfigurename{Seznam obr\'azk\r{u}}%
  \def\listtablename{Seznam tabulek}%
  \def\indexname{Rejst\v{r}\'{\i}k}%
  \def\figurename{Obr\'azek}%
  \def\tablename{Tabulka}%
  \def\partname{\v{C}\'ast}%
  \def\enclname{P\v{r}\'{\i}loha}%
  \def\ccname{Na v\v{e}dom\'{\i}}%
  \def\headtoname{Komu}%
  \def\pagename{Strana}%
  \def\seename{viz}%
  \def\alsoname{viz tak\'e}%
  \def\proofname{D\r{u}kaz}%
  \def\glossaryname{Slovn\'{\i}k}%
  }%
%    \end{macrocode}
%  \end{macro}
%
%  \begin{macro}{\dateczech}
%    The macro \cs{dateczech} redefines the command \cs{today}
%    to produce Czech dates.
%
%    \CS\LaTeX{} allows line break between the day and the month.
%    However, this behavior has been agreed upon to be a bad thing by
%    the csTeX mailing list in December 2005 and has not been adopted.
%    
% \changes{czech-1.3i}{1997/10/01}{Use \cs{edef} to define \cs{today}
%    to save memory}
% \changes{czech-1.3i}{1998/03/28}{Use \cs{def} instead of \cs{edef}}
%    \begin{macrocode}
\@namedef{date\CurrentOption}{%
  \def\today{\number\day.~\ifcase\month\or ledna\or \'unora\or
    b\v{r}ezna\or dubna\or kv\v{e}tna\or \v{c}ervna\or \v{c}ervence\or
    srpna\or z\'a\v{r}\'\i\or \v{r}\'{\i}jna\or listopadu\or
    prosince\fi \space\number\year}}
%    \end{macrocode}
%  \end{macro}
%
%  \begin{macro}{\extrasczech}
%  \begin{macro}{\noextrasczech}
%    The macro |\extrasczech| will perform all the extra definitions
%    needed for the Czech language. The macro |\noextrasczech| is used
%    to cancel the actions of |\extrasczech|.  This means saving the
%    meaning of two one-letter control sequences before defining them.
%
%    For Czech texts \cs{frenchspacing} should be in effect.  Language
%    group for shorthands is also set here.
% \changes{czech-1.3e}{1995/03/14}{now use \cs{bbl@frenchspacing} and
%    \cs{bbl@nonfrenchspacing}}
% \changes{czech-3.1}{2006/10/07}{move \cs{languageshorthands} here,
%    so that the language group is always initialized correctly}
%    \begin{macrocode}
\expandafter\addto\csname extras\CurrentOption\endcsname{%
  \bbl@frenchspacing
  \languageshorthands{czech}}
\expandafter\addto\csname noextras\CurrentOption\endcsname{%
  \bbl@nonfrenchspacing}
%    \end{macrocode}
%
% \changes{czech-1.1a}{1992/07/07}{Removed typo, \cs{q} was restored
%    twice, once too many.}
% \changes{czech-1.3e}{1995/03/15}{Use \LaTeX's \cs{v} and \cs{r}
%    accent commands}
%    \begin{macrocode}
\expandafter\addto\csname extras\CurrentOption\endcsname{%
  \babel@save\q\let\q\v
  \babel@save\w\let\w\r}
%    \end{macrocode}
%  \end{macro}
%  \end{macro}
%
%  \begin{macro}{\sq}
%  \begin{macro}{\dq}
%    We save the original single and double quote characters in
%    \cs{sq} and \cs{dq} to make them available later.
%    \begin{macrocode}
\begingroup\catcode`\"=12\catcode`\'=12
\def\x{\endgroup
  \def\sq{'}
  \def\dq{"}}
\x
%    \end{macrocode}
%  \end{macro}
%  \end{macro}
%
% \changes{czech-3.0}{2005/12/10}{Added default for setting hyphenmin
%   parameters.  Values taken from \CS\LaTeX.}
%    This macro is used to store the correct values of the hyphenation
%    parameters |\lefthyphenmin| and |\righthyphenmin|.
%    \begin{macrocode}
\providehyphenmins{\CurrentOption}{\tw@\thr@@}
%    \end{macrocode}
%
%  \begin{macro}{\v}
%    \LaTeX's normal |\v| accent places a caron over the letter that
%    follows it (\v{o}). This is not what we want for the letters d,
%    t, l and L; for those the accent should change shape. This is
%    acheived by the following.
%    \begin{macrocode}
\AtBeginDocument{%
  \DeclareTextCompositeCommand{\v}{OT1}{t}{%
    t\kern-.23em\raise.24ex\hbox{'}}
  \DeclareTextCompositeCommand{\v}{OT1}{d}{%
    d\kern-.13em\raise.24ex\hbox{'}}
  \DeclareTextCompositeCommand{\v}{OT1}{l}{\lcaron{}}
  \DeclareTextCompositeCommand{\v}{OT1}{L}{\Lcaron{}}}
%    \end{macrocode}
%
%  \begin{macro}{\lcaron}
%  \begin{macro}{\Lcaron}
%    For the letters \texttt{l} and \texttt{L} we want to disinguish
%    between normal fonts and monospaced fonts.
%    \begin{macrocode}
\def\lcaron{%
  \setbox0\hbox{M}\setbox\tw@\hbox{i}%
  \ifdim\wd0>\wd\tw@\relax
    l\kern-.13em\raise.24ex\hbox{'}\kern-.11em%
  \else
    l\raise.45ex\hbox to\z@{\kern-.35em '\hss}%
  \fi}
\def\Lcaron{%
  \setbox0\hbox{M}\setbox\tw@\hbox{i}%
  \ifdim\wd0>\wd\tw@\relax
    L\raise.24ex\hbox to\z@{\kern-.28em'\hss}%
  \else
    L\raise.45ex\hbox to\z@{\kern-.40em '\hss}%
  \fi}
%    \end{macrocode}
%  \end{macro}
%  \end{macro}
%  \end{macro}
%
%    Initialize active quotes.  \CS\LaTeX{} provides a way of
%    converting English-style quotes into Czech-style ones.  Both
%    single and double quotes are affected, i.e. |``text''| is
%    converted to something like |,,text``| and |`text'| is converted
%    to |,text`|.  This conversion can be switched on and off with
%    \cs{csprimeson} and \cs{csprimesoff}.\footnote{By the way, the
%    names of these macros are misleading, because the handling of
%    primes in math mode is rather marginal, the most important thing
%    being the handling of quotes\ldots}
%
%    These quotes present various troubles, e.g. the kerning is broken,
%    apostrophes are converted to closing single quote, some primitives
%    are broken (most notably the |\catcode`\|\meta{char} syntax will
%    not work any more), and writing them to \file{.aux} files cannot
%    be handled correctly.  For these reasons, these commands are only
%    available in \CS\LaTeX{} compatibility mode.
%
%    \begin{macrocode}
\ifx\cs@compat@latex\relax
  \let\cs@ltxprim@s\prim@s
  \def\csprimeson{%
    \catcode`\`\active \catcode`\'\active \let\prim@s\bbl@prim@s}
  \def\csprimesoff{%
    \catcode`\`12 \catcode`\'12 \let\prim@s\cs@ltxprim@s}
  \begingroup\catcode`\`\active
  \def\x{\endgroup
    \def`{\futurelet\cs@next\cs@openquote}
    \def\cs@openquote{%
      \ifx`\cs@next \expandafter\cs@opendq
      \else \expandafter\clq
      \fi}%
  }\x
  \begingroup\catcode`\'\active
  \def\x{\endgroup
    \def'{\textormath{\futurelet\cs@next\cs@closequote}
                     {^\bgroup\prim@s}}
    \def\cs@closequote{%
      \ifx'\cs@next \expandafter\cs@closedq
      \else \expandafter\crq
      \fi}%
  }\x
  \def\cs@opendq{\clqq\let\cs@next= }
  \def\cs@closedq{\crqq\let\cs@next= }
%    \end{macrocode}
%
%    The way I recommend for typesetting quotes in Czech documents
%    is to use shorthands similar to those used in German.
%    
%    \begin{macrocode}
\else
  \initiate@active@char{"}
  \expandafter\addto\csname extras\CurrentOption\endcsname{%
    \bbl@activate{"}}
  \expandafter\addto\csname noextras\CurrentOption\endcsname{%
    \bbl@deactivate{"}}
  \declare@shorthand{czech}{"`}{\clqq}
  \declare@shorthand{czech}{"'}{\crqq}
  \declare@shorthand{czech}{"<}{\flqq}
  \declare@shorthand{czech}{">}{\frqq}
  \declare@shorthand{czech}{"=}{\cs@splithyphen}
\fi
%    \end{macrocode}
%
%  \begin{macro}{\clqq}
%    This is the CS opening quote, which is similar to the German
%    quote (\cs{glqq}) but the kerning is different.
%
%    For the OT1 encoding, the quote is constructed from the right
%    double quote (i.e. the ``Opening quotes'' character) by moving
%    it down to the baseline and shifting it to the right, or to the
%    left if italic correction is positive.
%
%    For T1, the ``German Opening quotes'' is used.  It is moved to
%    the right and the total width is enlarged.  This is done in an
%    attempt to minimize the difference between the OT1 and T1
%    versions.
%
% \changes{3.0}{2006/04/20}{Added \cs{leavevmode} to allow an opening
%   quote at the beginning of a paragraph}
%    \begin{macrocode}
\ProvideTextCommand{\clqq}{OT1}{%
  \set@low@box{\textquotedblright}%
  \setbox\@ne=\hbox{l\/}\dimen\@ne=\wd\@ne
  \setbox\@ne=\hbox{l}\advance\dimen\@ne-\wd\@ne
  \leavevmode
  \ifdim\dimen\@ne>\z@\kern-.1em\box\z@\kern.1em
    \else\kern.1em\box\z@\kern-.1em\fi\allowhyphens}
\ProvideTextCommand{\clqq}{T1}
  {\kern.1em\quotedblbase\kern-.0158em\relax}
\ProvideTextCommandDefault{\clqq}{\UseTextSymbol{OT1}\clqq}
%    \end{macrocode}
%  \end{macro}
%
%  \begin{macro}{\crqq}
%    For OT1, the CS closing quote is basically the same as
%    \cs{grqq}, only the \cs{textormath} macro is not used, because
%    as far as I know, \cs{grqq} does not work in math mode anyway.
%
%    For T1, the character is slightly wider and shifted to the
%    right to match its OT1 counterpart.
%
%    \begin{macrocode}
\ProvideTextCommand{\crqq}{OT1}
  {\save@sf@q{\nobreak\kern-.07em\textquotedblleft\kern.07em}}
\ProvideTextCommand{\crqq}{T1}
  {\save@sf@q{\nobreak\kern.06em\textquotedblleft\kern.024em}}
\ProvideTextCommandDefault{\crqq}{\UseTextSymbol{OT1}\crqq}
%    \end{macrocode}
%  \end{macro}
%
%  \begin{macro}{\clq}
%  \begin{macro}{\crq}
%
%    Single CS quotes are similar to double quotes (see the
%    discussion above).
%
%    \begin{macrocode}
\ProvideTextCommand{\clq}{OT1}
  {\set@low@box{\textquoteright}\box\z@\kern.04em\allowhyphens}
\ProvideTextCommand{\clq}{T1}
  {\quotesinglbase\kern-.0428em\relax}
\ProvideTextCommandDefault{\clq}{\UseTextSymbol{OT1}\clq}
\ProvideTextCommand{\crq}{OT1}
  {\save@sf@q{\nobreak\textquoteleft\kern.17em}}
\ProvideTextCommand{\crq}{T1}
  {\save@sf@q{\nobreak\textquoteleft\kern.17em}}
\ProvideTextCommandDefault{\crq}{\UseTextSymbol{OT1}\crq}
%    \end{macrocode}
%  \end{macro}
%  \end{macro}
%
%  \begin{macro}{\uv}
%    There are two versions of \cs{uv}.  The older one opens a group
%    and uses \cs{aftergroup} to typeset the closing quotes.  This
%    version allows using \cs{verb} inside the quotes, because the
%    enclosed text is not passed as an argument, but unfortunately
%    it breaks any kerning with the quotes.  Although the kerning
%    with the opening quote could be fixed, the kerning with the
%    closing quote cannot.
%
%    The newer version is defined as a command with one parameter.
%    It preserves kerning but since the quoted text is passed as an
%    argument, it cannot contain \cs{verb}.
%
%    Decide which version of \cs{uv} should be used.  For sake
%    of compatibility, we use the older version with plain \TeX{}
%    and the newer version with \LaTeXe.
%    \begin{macrocode}
\ifx\cs@compat@plain\@undefined\else\let\cs@olduv=\relax\fi
\ifx\cs@olduv\@undefined
  \DeclareRobustCommand\uv[1]{{\leavevmode\clqq#1\crqq}}
\else
  \DeclareRobustCommand\uv{\bgroup\aftergroup\closequotes
    \leavevmode\clqq\let\cs@next=}
  \def\closequotes{\unskip\crqq\relax}
\fi
%    \end{macrocode}
%  \end{macro}
%
%
%  \begin{macro}{\cs@wordlen}
%    Declare a counter to hold the length of the word after the
%    hyphen.
%
%    \begin{macrocode}
\newcount\cs@wordlen
%    \end{macrocode}
%  \end{macro}
%
%  \begin{macro}{\cs@hyphen}
%  \begin{macro}{\cs@endash}
%  \begin{macro}{\cs@emdash}
%    Store the original hyphen in a macro. Ditto for the ligatures.
%
% \changes{czech-3.1}{2006/10/07}{ensure correct catcode for the
%    saved hyphen}
%    \begin{macrocode}
\begingroup\catcode`\-12
\def\x{\endgroup
  \def\cs@hyphen{-}
  \def\cs@endash{--}
  \def\cs@emdash{---}
%    \end{macrocode}
%  \end{macro}
%  \end{macro}
%  \end{macro}
%
%  \begin{macro}{\cs@boxhyphen}
%    Provide a non-breakable hyphen to be used when a compound word
%    is too short to be split, i.e. the second part is shorter than
%    \cs{righthyphenmin}.
%
%    \begin{macrocode}
  \def\cs@boxhyphen{\hbox{-}}
%    \end{macrocode}
%  \end{macro}
%
%  \begin{macro}{\cs@splithyphen}
%    The macro \cs{cs@splithyphen} inserts a split hyphen, while
%    allowing both parts of the compound word to be hyphenated at
%    other places too.
%
%    \begin{macrocode}
  \def\cs@splithyphen{\kern\z@
    \discretionary{-}{\char\hyphenchar\the\font}{-}\nobreak\hskip\z@}
}\x
%    \end{macrocode}
%  \end{macro}
%
%  \begin{macro}{-}
%    To minimize the effects of activating the hyphen character,
%    the active definition expands to the non-active character
%    in all cases where hyphenation cannot occur, i.e. if not
%    typesetting (check \cs{protect}), not in horizontal mode,
%    or in inner horizontal mode.
%
%    \begin{macrocode}
\initiate@active@char{-}
\declare@shorthand{czech}{-}{%
  \ifx\protect\@typeset@protect
    \ifhmode
      \ifinner
        \bbl@afterelse\bbl@afterelse\bbl@afterelse\cs@hyphen
      \else
        \bbl@afterfi\bbl@afterelse\bbl@afterelse\cs@firsthyphen
      \fi
    \else
      \bbl@afterfi\bbl@afterelse\cs@hyphen
    \fi
  \else
    \bbl@afterfi\cs@hyphen
  \fi}
%    \end{macrocode}
%  \end{macro}
%
%  \begin{macro}{\cs@firsthyphen}
%  \begin{macro}{\cs@firsthyph@n}
%  \begin{macro}{\cs@secondhyphen}
%  \begin{macro}{\cs@secondhyph@n}
%    If we encounter a hyphen, check whether it is followed
%    by a second or a third hyphen and if so, insert the
%    corresponding ligature.
%
%    If we don't find a hyphen, the token found will be placed
%    in \cs{cs@token} for further analysis, and it will also stay
%    in the input.
%
%    \begin{macrocode}
\begingroup\catcode`\-\active
\def\x{\endgroup
  \def\cs@firsthyphen{\futurelet\cs@token\cs@firsthyph@n}
  \def\cs@firsthyph@n{%
    \ifx -\cs@token
      \bbl@afterelse\cs@secondhyphen
    \else
      \bbl@afterfi\cs@checkhyphen
    \fi}
  \def\cs@secondhyphen ##1{%
    \futurelet\cs@token\cs@secondhyph@n}
  \def\cs@secondhyph@n{%
    \ifx -\cs@token
      \bbl@afterelse\cs@emdash\@gobble
    \else
      \bbl@afterfi\cs@endash
    \fi}
}\x
%    \end{macrocode}
%  \end{macro}
%  \end{macro}
%  \end{macro}
%  \end{macro}
%
%  \begin{macro}{\cs@checkhyphen}
%    Check that hyphenation is enabled, and if so, start analyzing
%    the rest of the word, i.e. initialize \cs{cs@word} and \cs{cs@wordlen}
%    and start processing input with \cs{cs@scanword}.
%
%    \begin{macrocode}
\def\cs@checkhyphen{%
  \ifnum\expandafter\hyphenchar\the\font=`\-
    \def\cs@word{}\cs@wordlen\z@
    \bbl@afterelse\cs@scanword
  \else
    \cs@hyphen
  \fi}
%    \end{macrocode}
%  \end{macro}
%
%  \begin{macro}{\cs@scanword}
%  \begin{macro}{\cs@continuescan}
%  \begin{macro}{\cs@gettoken}
%  \begin{macro}{\cs@gett@ken}
%    Each token is first analyzed with \cs{cs@scanword}, which expands
%    the token and passes the first token of the result to
%    \cs{cs@gett@ken}. If the expanded token is not identical to the
%    unexpanded one, presume that it might be expanded further and
%    pass it back to \cs{cs@scanword} until you get an unexpandable
%    token. Then analyze it in \cs{cs@examinetoken}.
%
%    The \cs{cs@continuescan} macro does the same thing as
%    \cs{cs@scanword}, but it does not require the first token to be
%    in \cs{cs@token} already.
%
%    \begin{macrocode}
\def\cs@scanword{\let\cs@lasttoken= \cs@token\expandafter\cs@gettoken}
\def\cs@continuescan{\let\cs@lasttoken\@undefined\expandafter\cs@gettoken}
\def\cs@gettoken{\futurelet\cs@token\cs@gett@ken}
\def\cs@gett@ken{%
  \ifx\cs@token\cs@lasttoken \def\cs@next{\cs@examinetoken}%
  \else \def\cs@next{\cs@scanword}%
  \fi \cs@next}
%    \end{macrocode}
%  \end{macro}
%  \end{macro}
%  \end{macro}
%  \end{macro}
%
%  \begin{macro}{cs@examinetoken}
%    Examine the token in \cs{cs@token}:
%
%    \begin{itemize}
%    \item
%      If it is a letter (catcode 11) or other (catcode 12), add it
%      to \cs{cs@word} with \cs{cs@addparam}.
%
%    \item
%      If it is the \cs{char} primitive, add it with \cs{cs@expandchar}.
%
%    \item
%      If the token starts or ends a group, ignore it with
%      \cs{cs@ignoretoken}.
%
%    \item
%      Otherwise analyze the meaning of the token with
%      \cs{cs@checkchardef} to detect primitives defined with
%      \cs{chardef}.
%
%    \end{itemize}
%
%    \begin{macrocode}
\def\cs@examinetoken{%
  \ifcat A\cs@token
    \def\cs@next{\cs@addparam}%
  \else\ifcat 0\cs@token
    \def\cs@next{\cs@addparam}%
  \else\ifx\char\cs@token
    \def\cs@next{\afterassignment\cs@expandchar\let\cs@token= }%
  \else\ifx\bgroup\cs@token
    \def\cs@next{\cs@ignoretoken\bgroup}%
  \else\ifx\egroup\cs@token
    \def\cs@next{\cs@ignoretoken\egroup}%
  \else\ifx\begingroup\cs@token
    \def\cs@next{\cs@ignoretoken\begingroup}%
  \else\ifx\endgroup\cs@token
    \def\cs@next{\cs@ignoretoken\endgroup}%
  \else
    \def\cs@next{\expandafter\expandafter\expandafter\cs@checkchardef
      \expandafter\meaning\expandafter\cs@token\string\char\end}%
  \fi\fi\fi\fi\fi\fi\fi\cs@next}
%    \end{macrocode}
%  \end{macro}
%
%  \begin{macro}{\cs@checkchardef}
%    Check the meaning of a token and if it is a primitive defined
%    with \cs{chardef}, pass it to \cs{\cs@examinechar} as if it were
%    a \cs{char} sequence. Otherwise, there are no more word characters,
%    so do the final actions in \cs{cs@nosplit}.
%
%    \begin{macrocode}
\expandafter\def\expandafter\cs@checkchardef
  \expandafter#\expandafter1\string\char#2\end{%
    \def\cs@token{#1}%
    \ifx\cs@token\@empty
      \def\cs@next{\afterassignment\cs@examinechar\let\cs@token= }%
    \else
      \def\cs@next{\cs@nosplit}%
    \fi \cs@next}
%    \end{macrocode}
%  \end{macro}
%
%  \begin{macro}{\cs@ignoretoken}
%    Add a token to \cs{cs@word} but do not update the \cs{cs@wordlen}
%    counter. This is mainly useful for group starting and ending
%    primitives, which need to be preserved, but do not affect the word
%    boundary.
%
%    \begin{macrocode}
\def\cs@ignoretoken#1{%
  \edef\cs@word{\cs@word#1}%
  \afterassignment\cs@continuescan\let\cs@token= }
%    \end{macrocode}
%  \end{macro}
%
%  \begin{macro}{cs@addparam}
%    Add a token to \cs{cs@word} and check its lccode. Note that
%    this macro can only be used for tokens which can be passed as
%    a parameter.
%
%    \begin{macrocode}
\def\cs@addparam#1{%
  \edef\cs@word{\cs@word#1}%
  \cs@checkcode{\lccode`#1}}
%    \end{macrocode}
%  \end{macro}
%
%  \begin{macro}{\cs@expandchar}
%  \begin{macro}{\cs@examinechar}
%    Add a \cs{char} sequence to \cs{cs@word} and check its lccode.
%    The charcode is first parsed in \cs{cs@expandchar} and then the
%    resulting \cs{chardef}-defined sequence is analyzed in
%    \cs{cs@examinechar}.
%
%    \begin{macrocode}
\def\cs@expandchar{\afterassignment\cs@examinechar\chardef\cs@token=}
\def\cs@examinechar{%
  \edef\cs@word{\cs@word\char\the\cs@token\space}%
  \cs@checkcode{\lccode\cs@token}}
%    \end{macrocode}
%  \end{macro}
%  \end{macro}
%
%  \begin{macro}{\cs@checkcode}
%    Check the lccode of a character. If it is zero, it does not count
%    to the current word, so finish it with \cs{cs@nosplit}. Otherwise
%    update the \cs{cs@wordlen} counter and go on scanning the word
%    with \cs{cs@continuescan}. When enough characters are gathered in
%    \cs{cs@word} to allow word break, insert the split hyphen and
%    finish.
%
%    \begin{macrocode}
\def\cs@checkcode#1{%
  \ifnum0=#1
    \def\cs@next{\cs@nosplit}%
  \else
    \advance\cs@wordlen\@ne
    \ifnum\righthyphenmin>\the\cs@wordlen
      \def\cs@next{\cs@continuescan}%
    \else
      \cs@splithyphen
      \def\cs@next{\cs@word}%
    \fi
  \fi \cs@next}
%    \end{macrocode}
%  \end{macro}
%
%  \begin{macro}{\cs@nosplit}
%    Insert a non-breakable hyphen followed by the saved word.
%
%    \begin{macrocode}
\def\cs@nosplit{\cs@boxhyphen\cs@word}
%    \end{macrocode}
%  \end{macro}
%
%  \begin{macro}{\cs@hyphen}
%    The \cs{minus} sequence can be used where the active hyphen
%    does not work, e.g. in arguments to \TeX{} primitives in outer
%    horizontal mode.
%
%    \begin{macrocode}
\let\minus\cs@hyphen
%    \end{macrocode}
%  \end{macro}

%  \begin{macro}{\standardhyphens}
%  \begin{macro}{\splithyphens}
%    These macros control whether split hyphens are allowed in Czech
%    and/or Slovak texts. You may use them in any language, but the
%    split hyphen is only activated for Czech and Slovak.
%
% \changes{czech-3.1}{2006/10/07}{activate with split hyphens and
%    deactivate with standard hyphens, not vice versa}
%    \begin{macrocode}
\def\standardhyphens{\cs@splithyphensfalse\cs@deactivatehyphens}
\def\splithyphens{\cs@splithyphenstrue\cs@activatehyphens}
%    \end{macrocode}
%  \end{macro}
%  \end{macro}
%
%  \begin{macro}{\cs@splitattr}
%    Now we declare the |split| language attribute.  This is
%    similar to the |split| package option of cslatex, but it
%    only affects Czech, not Slovak.
%
% \changes{czech-3.1}{2006/10/07}{attribute added}
%    \begin{macrocode}
\def\cs@splitattr{\babel@save\ifcs@splithyphens\splithyphens}
\bbl@declare@ttribute{czech}{split}{%
  \addto\extrasczech{\cs@splitattr}}
%    \end{macrocode}
%  \end{macro}
%
%  \begin{macro}{\cs@activatehyphens}
%  \begin{macro}{\cs@deactivatehyphens}
%    These macros are defined as \cs{relax} by default to prevent
%    activating/deactivating the hyphen character. They are redefined
%    when the language is switched to Czech/Slovak. At that moment
%    the hyphen is also activated if split hyphens were requested with
%    \cs{splithyphens}.
%
%    When the language is de-activated, de-activate the hyphen and
%    restore the bogus definitions of these macros.
%
%    \begin{macrocode}
\let\cs@activatehyphens\relax
\let\cs@deactivatehyphens\relax
\expandafter\addto\csname extras\CurrentOption\endcsname{%
  \def\cs@activatehyphens{\bbl@activate{-}}%
  \def\cs@deactivatehyphens{\bbl@deactivate{-}}%
  \ifcs@splithyphens\cs@activatehyphens\fi}
\expandafter\addto\csname noextras\CurrentOption\endcsname{%
  \cs@deactivatehyphens
  \let\cs@activatehyphens\relax
  \let\cs@deactivatehyphens\relax}
%    \end{macrocode}
%  \end{macro}
%  \end{macro}
%
%  \begin{macro}{\cs@looseness}
%  \begin{macro}{\looseness}
%    One of the most common situations where an active hyphen will not
%    work properly is the \cs{looseness} primitive. Change its definition
%    so that it deactivates the hyphen if needed.
%
%    \begin{macrocode}
\let\cs@looseness\looseness
\def\looseness{%
  \ifcs@splithyphens
    \cs@deactivatehyphens\afterassignment\cs@activatehyphens \fi
  \cs@looseness}
%    \end{macrocode}
%  \end{macro}
%  \end{macro}
%
%  \begin{macro}{\cs@selectlanguage}
%  \begin{macro}{\cs@main@language}
%    Specifying the |nocaptions| option means that captions and dates
%    are not redefined by default, but they can be switched on later
%    with \cs{captionsczech} and/or \cs{dateczech}. 
%
%    We mimic this behavior by redefining \cs{selectlanguage}.  This
%    macro is called once at the beginning of the document to set the
%    main language of the document.  If this is \cs{cs@main@language},
%    it disables the macros for setting captions and date.  In any
%    case, it restores the original definition of \cs{selectlanguage}
%    and expands it.
%
%    The definition of \cs{selectlanguage} can be shared between Czech
%    and Slovak; the actual language is stored in \cs{cs@main@language}.
%
%    \begin{macrocode}
\ifx\cs@nocaptions\@undefined\else
  \edef\cs@main@language{\CurrentOption}
  \ifx\cs@origselect\@undefined
    \let\cs@origselect=\selectlanguage
    \def\selectlanguage{%
      \let\selectlanguage\cs@origselect
      \ifx\bbl@main@language\cs@main@language
        \expandafter\cs@selectlanguage
      \else
        \expandafter\selectlanguage
      \fi}
    \def\cs@selectlanguage{%
      \cs@tempdisable{captions}%
      \cs@tempdisable{date}%
      \selectlanguage}
%    \end{macrocode}
%
%  \begin{macro}{\cs@tempdisable}
%    \cs{cs@tempdisable} disables a language setup macro temporarily,
%    i.e. the macro with the name of \meta{\#1}|\bbl@main@language|
%    just restores the original definition and purges the saved macro
%    from memory.
%
%    \begin{macrocode}
    \def\cs@tempdisable#1{%
      \def\@tempa{cs@#1}%
      \def\@tempb{#1\bbl@main@language}%
      \expandafter\expandafter\expandafter\let
        \expandafter \csname\expandafter \@tempa \expandafter\endcsname
        \csname \@tempb \endcsname
      \expandafter\edef\csname \@tempb \endcsname{%
        \let \expandafter\noexpand \csname \@tempb \endcsname
          \expandafter\noexpand \csname \@tempa \endcsname
        \let \expandafter\noexpand\csname \@tempa \endcsname
          \noexpand\@undefined}}
%    \end{macrocode}
%  \end{macro}
%
%    These macros are not needed, once the initialization is over.
%
%    \begin{macrocode}
    \@onlypreamble\cs@main@language
    \@onlypreamble\cs@origselect
    \@onlypreamble\cs@selectlanguage
    \@onlypreamble\cs@tempdisable
  \fi
\fi
%    \end{macrocode}
%  \end{macro}
%  \end{macro}
%
%    The encoding of mathematical fonts should be changed to IL2.  This
%    allows to use accented letter in some font families.  Besides,
%    documents do not use CM fonts if there are equivalents in CS-fonts,
%    so there is no need to have both bitmaps of CM-font and CS-font.
%
%    \cs{@font@warning} and \cs{@font@info} are temporarily redefined
%    to avoid annoying font warnings.
%
%    \begin{macrocode}
\ifx\cs@compat@plain\@undefined
\ifx\cs@check@enc\@undefined\else
  \def\cs@check@enc{
    \ifx\encodingdefault\cs@iltw@
      \let\cs@warn\@font@warning \let\@font@warning\@gobble
      \let\cs@info\@font@info    \let\@font@info\@gobble
      \SetSymbolFont{operators}{normal}{\cs@iltw@}{cmr}{m}{n}
      \SetSymbolFont{operators}{bold}{\cs@iltw@}{cmr}{bx}{n}
      \SetMathAlphabet\mathbf{normal}{\cs@iltw@}{cmr}{bx}{n}
      \SetMathAlphabet\mathit{normal}{\cs@iltw@}{cmr}{m}{it}
      \SetMathAlphabet\mathrm{normal}{\cs@iltw@}{cmr}{m}{n}
      \SetMathAlphabet\mathsf{normal}{\cs@iltw@}{cmss}{m}{n}
      \SetMathAlphabet\mathtt{normal}{\cs@iltw@}{cmtt}{m}{n}
      \SetMathAlphabet\mathbf{bold}{\cs@iltw@}{cmr}{bx}{n}
      \SetMathAlphabet\mathit{bold}{\cs@iltw@}{cmr}{bx}{it}
      \SetMathAlphabet\mathrm{bold}{\cs@iltw@}{cmr}{bx}{n}
      \SetMathAlphabet\mathsf{bold}{\cs@iltw@}{cmss}{bx}{n}
      \SetMathAlphabet\mathtt{bold}{\cs@iltw@}{cmtt}{m}{n}
      \let\@font@warning\cs@warn \let\cs@warn\@undefined
      \let\@font@info\cs@info    \let\cs@info\@undefined
    \fi
    \let\cs@check@enc\@undefined}
  \AtBeginDocument{\cs@check@enc}
\fi
\fi
%    \end{macrocode}
%
%  \begin{macro}{cs@undoiltw@}
%
%    The thing is that \LaTeXe{} core only supports the T1 encoding
%    and does not bother changing the uc/lc/sfcodes when encoding
%    is switched. :( However, the IL2 encoding \emph{does} change
%    these codes, so if encoding is switched back from IL2, we must
%    also undo the effect of this change to be compatible with
%    \LaTeXe.  OK, this is not the right\textsuperscript{TM} solution
%    but it works.  Cheers to Petr Ol\v s\'ak.
%
%    \begin{macrocode}
\def\cs@undoiltw@{%
  \uccode158=208 \lccode158=158 \sfcode158=1000
  \sfcode159=1000
  \uccode165=133 \lccode165=165 \sfcode165=1000
  \uccode169=137 \lccode169=169 \sfcode169=1000
  \uccode171=139 \lccode171=171 \sfcode171=1000
  \uccode174=142 \lccode174=174 \sfcode174=1000
  \uccode181=149
  \uccode185=153
  \uccode187=155
  \uccode190=0   \lccode190=0
  \uccode254=222 \lccode254=254 \sfcode254=1000
  \uccode255=223 \lccode255=255 \sfcode255=1000}
%    \end{macrocode}
%  \end{macro}
%
%  \begin{macro}{@@enc@update}
%
%    Redefine the \LaTeXe{} internal function \cs{@@enc@update} to
%    change the encodings correctly.
%
%    \begin{macrocode}
\ifx\cs@enc@update\@undefined
\ifx\@@enc@update\@undefined\else
  \let\cs@enc@update\@@enc@update
  \def\@@enc@update{\ifx\cf@encoding\cs@iltw@\cs@undoiltw@\fi
    \cs@enc@update
    \expandafter\ifnum\csname l@\languagename\endcsname=\the\language
      \expandafter\ifx
      \csname l@\languagename:\f@encoding\endcsname\relax
      \else
        \expandafter\expandafter\expandafter\let
          \expandafter\csname
          \expandafter l\expandafter @\expandafter\languagename
          \expandafter\endcsname\csname l@\languagename:\f@encoding\endcsname
      \fi
      \language=\csname l@\languagename\endcsname\relax
    \fi}
\fi\fi
%    \end{macrocode}
%  \end{macro}
%
%    The macro |\ldf@finish| takes care of looking for a
%    configuration file, setting the main language to be switched on
%    at |\begin{document}| and resetting the category code of
%    \texttt{@} to its original value.
% \changes{czech-1.3h}{1996/11/02}{Now use \cs{ldf@finish} to wrap up}
%    \begin{macrocode}
\ldf@finish\CurrentOption
%</code>
%    \end{macrocode}
%
% \Finale
%%
%% \CharacterTable
%%  {Upper-case    \A\B\C\D\E\F\G\H\I\J\K\L\M\N\O\P\Q\R\S\T\U\V\W\X\Y\Z
%%   Lower-case    \a\b\c\d\e\f\g\h\i\j\k\l\m\n\o\p\q\r\s\t\u\v\w\x\y\z
%%   Digits        \0\1\2\3\4\5\6\7\8\9
%%   Exclamation   \!     Double quote  \"     Hash (number) \#
%%   Dollar        \$     Percent       \%     Ampersand     \&
%%   Acute accent  \'     Left paren    \(     Right paren   \)
%%   Asterisk      \*     Plus          \+     Comma         \,
%%   Minus         \-     Point         \.     Solidus       \/
%%   Colon         \:     Semicolon     \;     Less than     \<
%%   Equals        \=     Greater than  \>     Question mark \?
%%   Commercial at \@     Left bracket  \[     Backslash     \\
%%   Right bracket \]     Circumflex    \^     Underscore    \_
%%   Grave accent  \`     Left brace    \{     Vertical bar  \|
%%   Right brace   \}     Tilde         \~}
%%
\endinput
}
\DeclareOption{danish}{%%
%% This file will generate fast loadable files and documentation
%% driver files from the doc files in this package when run through
%% LaTeX or TeX.
%%
%% Copyright 1989-2009 Johannes L. Braams and any individual authors
%% listed elsewhere in this file.  All rights reserved.
%% 
%% This file is part of the Babel system.
%% --------------------------------------
%% 
%% It may be distributed and/or modified under the
%% conditions of the LaTeX Project Public License, either version 1.3
%% of this license or (at your option) any later version.
%% The latest version of this license is in
%%   http://www.latex-project.org/lppl.txt
%% and version 1.3 or later is part of all distributions of LaTeX
%% version 2003/12/01 or later.
%% 
%% This work has the LPPL maintenance status "maintained".
%% 
%% The Current Maintainer of this work is Johannes Braams.
%% 
%% The list of all files belonging to the LaTeX base distribution is
%% given in the file `manifest.bbl. See also `legal.bbl' for additional
%% information.
%% 
%% The list of derived (unpacked) files belonging to the distribution
%% and covered by LPPL is defined by the unpacking scripts (with
%% extension .ins) which are part of the distribution.
%%
%% --------------- start of docstrip commands ------------------
%%
\def\filedate{1999/04/11}
\def\batchfile{danish.ins}
\input docstrip.tex

{\ifx\generate\undefined
\Msg{**********************************************}
\Msg{*}
\Msg{* This installation requires docstrip}
\Msg{* version 2.3c or later.}
\Msg{*}
\Msg{* An older version of docstrip has been input}
\Msg{*}
\Msg{**********************************************}
\errhelp{Move or rename old docstrip.tex.}
\errmessage{Old docstrip in input path}
\batchmode
\csname @@end\endcsname
\fi}

\declarepreamble\mainpreamble
This is a generated file.

Copyright 1989-2009 Johannes L. Braams and any individual authors
listed elsewhere in this file.  All rights reserved.

This file was generated from file(s) of the Babel system.
---------------------------------------------------------

It may be distributed and/or modified under the
conditions of the LaTeX Project Public License, either version 1.3
of this license or (at your option) any later version.
The latest version of this license is in
  http://www.latex-project.org/lppl.txt
and version 1.3 or later is part of all distributions of LaTeX
version 2003/12/01 or later.

This work has the LPPL maintenance status "maintained".

The Current Maintainer of this work is Johannes Braams.

This file may only be distributed together with a copy of the Babel
system. You may however distribute the Babel system without
such generated files.

The list of all files belonging to the Babel distribution is
given in the file `manifest.bbl'. See also `legal.bbl for additional
information.

The list of derived (unpacked) files belonging to the distribution
and covered by LPPL is defined by the unpacking scripts (with
extension .ins) which are part of the distribution.
\endpreamble

\declarepreamble\fdpreamble
This is a generated file.

Copyright 1989-2009 Johannes L. Braams and any individual authors
listed elsewhere in this file.  All rights reserved.

This file was generated from file(s) of the Babel system.
---------------------------------------------------------

It may be distributed and/or modified under the
conditions of the LaTeX Project Public License, either version 1.3
of this license or (at your option) any later version.
The latest version of this license is in
  http://www.latex-project.org/lppl.txt
and version 1.3 or later is part of all distributions of LaTeX
version 2003/12/01 or later.

This work has the LPPL maintenance status "maintained".

The Current Maintainer of this work is Johannes Braams.

This file may only be distributed together with a copy of the Babel
system. You may however distribute the Babel system without
such generated files.

The list of all files belonging to the Babel distribution is
given in the file `manifest.bbl'. See also `legal.bbl for additional
information.

In particular, permission is granted to customize the declarations in
this file to serve the needs of your installation.

However, NO PERMISSION is granted to distribute a modified version
of this file under its original name.

\endpreamble

\keepsilent

\usedir{tex/generic/babel} 

\usepreamble\mainpreamble
\generate{\file{danish.ldf}{\from{danish.dtx}{code}}
          }
\usepreamble\fdpreamble

\ifToplevel{
\Msg{***********************************************************}
\Msg{*}
\Msg{* To finish the installation you have to move the following}
\Msg{* files into a directory searched by TeX:}
\Msg{*}
\Msg{* \space\space All *.def, *.fd, *.ldf, *.sty}
\Msg{*}
\Msg{* To produce the documentation run the files ending with}
\Msg{* '.dtx' and `.fdd' through LaTeX.}
\Msg{*}
\Msg{* Happy TeXing}
\Msg{***********************************************************}
}
 
\endinput
}
\DeclareOption{dutch}{%%
%% This file will generate fast loadable files and documentation
%% driver files from the doc files in this package when run through
%% LaTeX or TeX.
%%
%% Copyright 1989-2005 Johannes L. Braams and any individual authors
%% listed elsewhere in this file.  All rights reserved.
%% 
%% This file is part of the Babel system.
%% --------------------------------------
%% 
%% It may be distributed and/or modified under the
%% conditions of the LaTeX Project Public License, either version 1.3
%% of this license or (at your option) any later version.
%% The latest version of this license is in
%%   http://www.latex-project.org/lppl.txt
%% and version 1.3 or later is part of all distributions of LaTeX
%% version 2003/12/01 or later.
%% 
%% This work has the LPPL maintenance status "maintained".
%% 
%% The Current Maintainer of this work is Johannes Braams.
%% 
%% The list of all files belonging to the LaTeX base distribution is
%% given in the file `manifest.bbl. See also `legal.bbl' for additional
%% information.
%% 
%% The list of derived (unpacked) files belonging to the distribution
%% and covered by LPPL is defined by the unpacking scripts (with
%% extension .ins) which are part of the distribution.
%%
%% --------------- start of docstrip commands ------------------
%%
\def\filedate{1999/04/11}
\def\batchfile{dutch.ins}
\input docstrip.tex

{\ifx\generate\undefined
\Msg{**********************************************}
\Msg{*}
\Msg{* This installation requires docstrip}
\Msg{* version 2.3c or later.}
\Msg{*}
\Msg{* An older version of docstrip has been input}
\Msg{*}
\Msg{**********************************************}
\errhelp{Move or rename old docstrip.tex.}
\errmessage{Old docstrip in input path}
\batchmode
\csname @@end\endcsname
\fi}

\declarepreamble\mainpreamble
This is a generated file.

Copyright 1989-2005 Johannes L. Braams and any individual authors
listed elsewhere in this file.  All rights reserved.

This file was generated from file(s) of the Babel system.
---------------------------------------------------------

It may be distributed and/or modified under the
conditions of the LaTeX Project Public License, either version 1.3
of this license or (at your option) any later version.
The latest version of this license is in
  http://www.latex-project.org/lppl.txt
and version 1.3 or later is part of all distributions of LaTeX
version 2003/12/01 or later.

This work has the LPPL maintenance status "maintained".

The Current Maintainer of this work is Johannes Braams.

This file may only be distributed together with a copy of the Babel
system. You may however distribute the Babel system without
such generated files.

The list of all files belonging to the Babel distribution is
given in the file `manifest.bbl'. See also `legal.bbl for additional
information.

The list of derived (unpacked) files belonging to the distribution
and covered by LPPL is defined by the unpacking scripts (with
extension .ins) which are part of the distribution.
\endpreamble

\declarepreamble\fdpreamble
This is a generated file.

Copyright 1989-2005 Johannes L. Braams and any individual authors
listed elsewhere in this file.  All rights reserved.

This file was generated from file(s) of the Babel system.
---------------------------------------------------------

It may be distributed and/or modified under the
conditions of the LaTeX Project Public License, either version 1.3
of this license or (at your option) any later version.
The latest version of this license is in
  http://www.latex-project.org/lppl.txt
and version 1.3 or later is part of all distributions of LaTeX
version 2003/12/01 or later.

This work has the LPPL maintenance status "maintained".

The Current Maintainer of this work is Johannes Braams.

This file may only be distributed together with a copy of the Babel
system. You may however distribute the Babel system without
such generated files.

The list of all files belonging to the Babel distribution is
given in the file `manifest.bbl'. See also `legal.bbl for additional
information.

In particular, permission is granted to customize the declarations in
this file to serve the needs of your installation.

However, NO PERMISSION is granted to distribute a modified version
of this file under its original name.

\endpreamble

\keepsilent

\usedir{tex/generic/babel} 

\usepreamble\mainpreamble
\generate{\file{dutch.ldf}{\from{dutch.dtx}{code}}
          }
\usepreamble\fdpreamble

\ifToplevel{
\Msg{***********************************************************}
\Msg{*}
\Msg{* To finish the installation you have to move the following}
\Msg{* files into a directory searched by TeX:}
\Msg{*}
\Msg{* \space\space All *.def, *.fd, *.ldf, *.sty}
\Msg{*}
\Msg{* To produce the documentation run the files ending with}
\Msg{* '.dtx' and `.fdd' through LaTeX.}
\Msg{*}
\Msg{* Happy TeXing}
\Msg{***********************************************************}
}
 
\endinput
}
%    \end{macrocode}
% \changes{babel~3.5b}{1995/06/06}{Added the \Lopt{estonian} option}
%    \begin{macrocode}
\DeclareOption{english}{%%
%% This file will generate fast loadable files and documentation
%% driver files from the doc files in this package when run through
%% LaTeX or TeX.
%%
%% Copyright 1989-2005 Johannes L. Braams and any individual authors
%% listed elsewhere in this file.  All rights reserved.
%% 
%% This file is part of the Babel system.
%% --------------------------------------
%% 
%% It may be distributed and/or modified under the
%% conditions of the LaTeX Project Public License, either version 1.3
%% of this license or (at your option) any later version.
%% The latest version of this license is in
%%   http://www.latex-project.org/lppl.txt
%% and version 1.3 or later is part of all distributions of LaTeX
%% version 2003/12/01 or later.
%% 
%% This work has the LPPL maintenance status "maintained".
%% 
%% The Current Maintainer of this work is Johannes Braams.
%% 
%% The list of all files belonging to the LaTeX base distribution is
%% given in the file `manifest.bbl. See also `legal.bbl' for additional
%% information.
%% 
%% The list of derived (unpacked) files belonging to the distribution
%% and covered by LPPL is defined by the unpacking scripts (with
%% extension .ins) which are part of the distribution.
%%
%% --------------- start of docstrip commands ------------------
%%
\def\filedate{1999/04/11}
\def\batchfile{english.ins}
\input docstrip.tex

{\ifx\generate\undefined
\Msg{**********************************************}
\Msg{*}
\Msg{* This installation requires docstrip}
\Msg{* version 2.3c or later.}
\Msg{*}
\Msg{* An older version of docstrip has been input}
\Msg{*}
\Msg{**********************************************}
\errhelp{Move or rename old docstrip.tex.}
\errmessage{Old docstrip in input path}
\batchmode
\csname @@end\endcsname
\fi}

\declarepreamble\mainpreamble
This is a generated file.

Copyright 1989-2005 Johannes L. Braams and any individual authors
listed elsewhere in this file.  All rights reserved.

This file was generated from file(s) of the Babel system.
---------------------------------------------------------

It may be distributed and/or modified under the
conditions of the LaTeX Project Public License, either version 1.3
of this license or (at your option) any later version.
The latest version of this license is in
  http://www.latex-project.org/lppl.txt
and version 1.3 or later is part of all distributions of LaTeX
version 2003/12/01 or later.

This work has the LPPL maintenance status "maintained".

The Current Maintainer of this work is Johannes Braams.

This file may only be distributed together with a copy of the Babel
system. You may however distribute the Babel system without
such generated files.

The list of all files belonging to the Babel distribution is
given in the file `manifest.bbl'. See also `legal.bbl for additional
information.

The list of derived (unpacked) files belonging to the distribution
and covered by LPPL is defined by the unpacking scripts (with
extension .ins) which are part of the distribution.
\endpreamble

\declarepreamble\fdpreamble
This is a generated file.

Copyright 1989-2005 Johannes L. Braams and any individual authors
listed elsewhere in this file.  All rights reserved.

This file was generated from file(s) of the Babel system.
---------------------------------------------------------

It may be distributed and/or modified under the
conditions of the LaTeX Project Public License, either version 1.3
of this license or (at your option) any later version.
The latest version of this license is in
  http://www.latex-project.org/lppl.txt
and version 1.3 or later is part of all distributions of LaTeX
version 2003/12/01 or later.

This work has the LPPL maintenance status "maintained".

The Current Maintainer of this work is Johannes Braams.

This file may only be distributed together with a copy of the Babel
system. You may however distribute the Babel system without
such generated files.

The list of all files belonging to the Babel distribution is
given in the file `manifest.bbl'. See also `legal.bbl for additional
information.

In particular, permission is granted to customize the declarations in
this file to serve the needs of your installation.

However, NO PERMISSION is granted to distribute a modified version
of this file under its original name.

\endpreamble

\keepsilent

\usedir{tex/generic/babel} 

\usepreamble\mainpreamble
\generate{\file{english.ldf}{\from{english.dtx}{code}}
          }
\usepreamble\fdpreamble

\ifToplevel{
\Msg{***********************************************************}
\Msg{*}
\Msg{* To finish the installation you have to move the following}
\Msg{* files into a directory searched by TeX:}
\Msg{*}
\Msg{* \space\space All *.def, *.fd, *.ldf, *.sty}
\Msg{*}
\Msg{* To produce the documentation run the files ending with}
\Msg{* '.dtx' and `.fdd' through LaTeX.}
\Msg{*}
\Msg{* Happy TeXing}
\Msg{***********************************************************}
}
 
\endinput
}
\DeclareOption{esperanto}{% \iffalse meta-comment
%
% Copyright 1989-2007 Johannes L. Braams and any individual authors
% listed elsewhere in this file.  All rights reserved.
% 
% This file is part of the Babel system.
% --------------------------------------
% 
% It may be distributed and/or modified under the
% conditions of the LaTeX Project Public License, either version 1.3
% of this license or (at your option) any later version.
% The latest version of this license is in
%   http://www.latex-project.org/lppl.txt
% and version 1.3 or later is part of all distributions of LaTeX
% version 2003/12/01 or later.
% 
% This work has the LPPL maintenance status "maintained".
% 
% The Current Maintainer of this work is Johannes Braams.
% 
% The list of all files belonging to the Babel system is
% given in the file `manifest.bbl. See also `legal.bbl' for additional
% information.
% 
% The list of derived (unpacked) files belonging to the distribution
% and covered by LPPL is defined by the unpacking scripts (with
% extension .ins) which are part of the distribution.
% \fi
% \CheckSum{261}
%\iffalse
%    Tell the \LaTeX\ system who we are and write an entry on the
%    transcript.
%<*dtx>
\ProvidesFile{esperanto.dtx}
%</dtx>
%<code>\ProvidesLanguage{esperanto}
%\fi
%\ProvidesFile{esperant.dtx}
        [2007/10/20 v1.4t Esperanto support from the babel system]
%\iffalse
%% File 'esperanto.dtx'
%% Babel package for LaTeX version 2e
%% Copyright (C) 1989 - 2007
%%           by Johannes Braams, TeXniek
%
%% Please report errors to: J.L. Braams
%%                          babel at braams dot xs4all dot nl
%
%    This file is part of the babel system, it provides the source
%    code for the Esperanto language definition file.  A contribution
%    was made by Ruiz-Altaba Marti (ruizaltb@cernvm.cern.ch) Code from
%    esperant.sty version 1.1 by Joerg Knappen
%    (\texttt{knappen@vkpmzd.kph.uni-mainz.de}) was included in
%    version 1.2.
%<*filedriver>
\documentclass{ltxdoc}
\newcommand*\TeXhax{\TeX hax}
\newcommand*\babel{\textsf{babel}}
\newcommand*\langvar{$\langle \it lang \rangle$}
\newcommand*\note[1]{}
\newcommand*\Lopt[1]{\textsf{#1}}
\newcommand*\file[1]{\texttt{#1}}
\begin{document}
 \DocInput{esperanto.dtx}
\end{document}
%</filedriver>
%\fi
% \GetFileInfo{esperanto.dtx}
%
% \changes{esperanto-1.0a}{1991/07/15}{Renamed \file{babel.sty} in
%    \file{babel.com}}
% \changes{esperanto-1.1}{1992/02/15}{Brought up-to-date with
%    babel~3.2a}
% \changes{esperanto-1.2}{1992/02/18}{Included code from
%    \texttt{esperant.sty}}
% \changes{esperanto-1.4a}{1994/02/04}{Updated for \LaTeXe}
% \changes{esperanto-1.4d}{1994/06/25}{Removed the use of
%    \cs{filedate}, moved Identification after loading of
%    \file{babel.def}}
% \changes{esperanto-1.4e}{1995/02/09}{Moved identification code to
%    the top of the file}
% \changes{esperanto-1.4f}{1995/06/14}{Corrected typos (PR1652)}
% \changes{esperanto-1.4i}{1996/07/10}{Replaced \cs{undefined} with
%    \cs{@undefined} and \cs{empty} with \cs{@empty} for consistency
%    with \LaTeX} 
% \changes{esperanto-1.4i}{1996/10/10}{Moved the definition of
%    \cs{atcatcode} right to the beginning.}
%
%  \section{The Esperanto language}
%
%    The file \file{\filename}\footnote{The file described in this
%    section has version number \fileversion\ and was last revised on
%    \filedate. A contribution was made by Ruiz-Altaba Marti
%    (\texttt{ruizaltb@cernvm.cern.ch}). Code from the file
%    \texttt{esperant.sty} by J\"org Knappen
%    (\texttt{knappen@vkpmzd.kph.uni-mainz.de}) was included.} defines
%    all the language-specific macros for the Esperanto language.
%
%    For this language the character |^| is made active.
%    In table~\ref{tab:esp-act} an overview is given of its purpose.
% \changes{esperanto-1.4j}{1997/01/06}{fixed typo in table caption
%    (funtion instead of function)} 
% \begin{table}[htb]
%    \centering
%     \begin{tabular}{lp{8cm}}
%      |^c| & gives \^c with hyphenation in the rest of the word
%             allowed, this works for c, C, g, G, H, J, s, S, z, Z\\
%      |^h| & prevents h\llap{\^{}} from becoming too tall\\
%      |^j| & gives \^\j\\
%      |^u| & gives \u u, with hyphenation in the rest of the word
%                   allowed\\
%      |^U| & gives \u U, with hyphenation in the rest of the word
%                   allowed\\
%      \verb=^|= & inserts a |\discretionary{-}{}{}|\\
%      \end{tabular}
%      \caption{The functions of the active character for Esperanto.}
%    \label{tab:esp-act}
% \end{table}
%
%  \StopEventually{}
%
%    The macro |\LdfInit| takes care of preventing that this file is
%    loaded more than once, checking the category code of the
%    \texttt{@} sign, etc.
% \changes{esperanto-1.4i}{1996/11/02}{Now use \cs{LdfInit} to perform
%    initial checks}
%    \begin{macrocode}
%<*code>
\LdfInit{esperanto}\captionsesperanto
%    \end{macrocode}
%
%    When this file is read as an option, i.e. by the |\usepackage|
%    command, \texttt{esperanto} will be an `unknown' language in
%    which case we have to make it known. So we check for the
%    existence of |\l@esperanto| to see whether we have to do
%    something here.
%
% \changes{esperanto-1.0b}{1991/10/29}{Removed use of
%    \cs{makeatletter}}
% \changes{esperanto-1.1}{1992/02/15}{Added a warning when no
%    hyphenation patterns were loaded.}
% \changes{esperanto-1.4d}{1994/06/25}{Use \cs{@nopatterns} for the
%    warning}
%    \begin{macrocode}
\ifx\l@esperanto\@undefined
  \@nopatterns{Esperanto}
  \adddialect\l@esperanto0\fi
%    \end{macrocode}
%
%    The next step consists of defining commands to switch to the
%    Esperanto language. The reason for this is that a user might want
%    to switch back and forth between languages.
%
% \begin{macro}{\captionsesperanto}
%    The macro |\captionsesperanto| defines all strings used
%    in the four standard documentclasses provided with \LaTeX.
% \changes{esperanto-1.1}{1992/02/15}{Added \cs{seename},
%    \cs{alsoname} and \cs{prefacename}}
% \changes{esperanto-1.3}{1993/07/10}{Repaired a number of mistakes,
%    indicated by D. Ederveen}
% \changes{esperanto-1.3}{1993/07/15}{\cs{headpagename} should be
%    \cs{pagename}}
% \changes{esperanto-1.4a}{1994/02/04}{added missing closing brace}
% \changes{esperanto-1.4g}{1995/07/04}{Added \cs{proofname} for
%    AMS-\LaTeX}
% \changes{esperanto-1.4i}{1996/07/06}{Replaced `Proof' by `Pruvo' 
%    PR 2207} 
% \changes{esperanto-1.4p}{2000/09/19}{Added \cs{glossaryname}}
% \changes{esperanto-1.4q}{2002/01/07}{Added translation for Glossary}
%    \begin{macrocode}
\addto\captionsesperanto{%
  \def\prefacename{Anta\u{u}parolo}%
  \def\refname{Cita\^\j{}oj}%
  \def\abstractname{Resumo}%
  \def\bibname{Bibliografio}%
  \def\chaptername{{\^C}apitro}%
  \def\appendixname{Apendico}%
  \def\contentsname{Enhavo}%
  \def\listfigurename{Listo de figuroj}%
  \def\listtablename{Listo de tabeloj}%
  \def\indexname{Indekso}%
  \def\figurename{Figuro}%
  \def\tablename{Tabelo}%
  \def\partname{Parto}%
  \def\enclname{Aldono(j)}%
  \def\ccname{Kopie al}%
  \def\headtoname{Al}%
  \def\pagename{Pa\^go}%
  \def\subjectname{Temo}%
  \def\seename{vidu}%   a^u: vd.
  \def\alsoname{vidu anka\u{u}}% a^u vd. anka\u{u}
  \def\proofname{Pruvo}%
  \def\glossaryname{Glosaro}%
  }
%    \end{macrocode}
% \end{macro}
%
% \begin{macro}{\dateesperanto}
%    The macro |\dateesperanto| redefines the command |\today| to
%    produce Esperanto dates.
% \changes{esperanto-1.3}{1993/07/10}{Removed the capitals from
%    \cs{today}}
% \changes{esperanto-1.4k}{1997/10/01}{Use \cs{edef} to define
%    \cs{today} to save memory}
% \changes{esperanto-1.4k}{1998/03/27}{Removed Rthe use of \cs{edef}
%    again} 
%    \begin{macrocode}
\def\dateesperanto{%
  \def\today{\number\day{--a}~de~\ifcase\month\or
    januaro\or februaro\or marto\or aprilo\or majo\or junio\or
    julio\or a\u{u}gusto\or septembro\or oktobro\or novembro\or
    decembro\fi,\space \number\year}}
%    \end{macrocode}
% \end{macro}
%
% \begin{macro}{\extrasesperanto}
% \begin{macro}{\noextrasesperanto}
%    The macro |\extrasesperanto| performs all the extra definitions
%    needed for the Esperanto language. The macro |\noextrasesperanto|
%    is used to cancel the actions of |\extrasesperanto|.
%
%    For Esperanto the |^| character is made active. This is done
%    once, later on its definition may vary.
%
%    \begin{macrocode}
\initiate@active@char{^}
%    \end{macrocode}
%    Because the character |^| is used in math mode with quite a
%    different purpose we need to add an extra level of evaluation to
%    the definition of the active |^|. It checks whether math mode is
%    active; if so the shorthand mechanism is bypassed by a direct
%    call of |\normal@char^|.
% \changes{esperanto-1.4n}{1999/09/30}{Added a check for math mode to
%    the definition of the shorthand character}
% \changes{esperant0-1.4o}{1999/12/18}{Moved the check for math to
%    babel.def}
%    \begin{macrocode}
\addto\extrasesperanto{\languageshorthands{esperanto}}
\addto\extrasesperanto{\bbl@activate{^}}
\addto\noextrasesperanto{\bbl@deactivate{^}}
%    \end{macrocode}
% \end{macro}
% \end{macro}
%
%    In order to prevent problems with the active |^| we add a
%    shorthand on system level which expands to a `normal |^|.
% \changes{esperanto-1.4l}{1999/04/11}{Added a shorthand definition on
%    system level}
%    \begin{macrocode}
\declare@shorthand{system}{^}{\csname normal@char\string^\endcsname}
%    \end{macrocode}
%    And here are the uses of the active |^|:
% \changes{esperanto-1.4h}{1995/07/27}{Added a few shorthands}
%    \begin{macrocode}
\declare@shorthand{esperanto}{^c}{\^{c}\allowhyphens}
\declare@shorthand{esperanto}{^C}{\^{C}\allowhyphens}
\declare@shorthand{esperanto}{^g}{\^{g}\allowhyphens}
\declare@shorthand{esperanto}{^G}{\^{G}\allowhyphens}
\declare@shorthand{esperanto}{^h}{h\llap{\^{}}\allowhyphens}
\declare@shorthand{esperanto}{^H}{\^{H}\allowhyphens}
\declare@shorthand{esperanto}{^j}{\^{\j}\allowhyphens}
\declare@shorthand{esperanto}{^J}{\^{J}\allowhyphens}
\declare@shorthand{esperanto}{^s}{\^{s}\allowhyphens}
\declare@shorthand{esperanto}{^S}{\^{S}\allowhyphens}
\declare@shorthand{esperanto}{^u}{\u u\allowhyphens}
\declare@shorthand{esperanto}{^U}{\u U\allowhyphens}
\declare@shorthand{esperanto}{^|}{\discretionary{-}{}{}\allowhyphens}
%    \end{macrocode}
%
% \begin{macro}{\Esper}
% \begin{macro}{\esper}
%    In \file{esperant.sty} J\"org Knappen provides the macros
%    |\esper| and |\Esper| that can be used instead of |\alph| and
%    |\Alph|. These macros are available in this file as well.
%
%    Their definition takes place in two steps. First the toplevel.
%    \begin{macrocode}
\def\esper#1{\@esper{\@nameuse{c@#1}}}
\def\Esper#1{\@Esper{\@nameuse{c@#1}}}
%    \end{macrocode}
%    Then the second level.
% \changes{esperanto-1.4q}{2002/01/08}{Removed the extra level of
%    expansion for more than five items, as was done in \LaTeX}
% \changes{esperanto-1.4t}{2006/06/05}{Added the missing `r' in these
%    macros} 
%    \begin{macrocode}
\def\@esper#1{%
  \ifcase#1\or a\or b\or c\or \^c\or d\or e\or f\or g\or \^g\or
    h\or h\llap{\^{}}\or i\or j\or \^\j\or k\or l\or m\or n\or o\or
    p\or r\or s\or \^s\or t\or u\or \u{u}\or v\or z\else\@ctrerr\fi}
\def\@Esper#1{%
  \ifcase#1\or A\or B\or C\or \^C\or D\or E\or F\or G\or \^G\or
    H\or \^H\or I\or J\or \^J\or K\or L\or M\or N\or O\or
    P\or R\or S\or \^S\or T\or U\or \u{U}\or V\or Z\else\@ctrerr\fi}
%    \end{macrocode}
% \end{macro}
% \end{macro}
%
% \begin{macro}{\hodiau}
% \begin{macro}{\hodiaun}
%    In \file{esperant.sty} J\"org Knappen provides two alternative
%    macros for |\today|, |\hodiau| and |\hodiaun|. The second macro
%    produces an accusative version of the date in Esperanto.
%    \begin{macrocode}
\addto\dateesperanto{\def\hodiau{la \today}}
\def\hodiaun{la \number\day --an~de~\ifcase\month\or
  januaro\or februaro\or marto\or aprilo\or majo\or junio\or
  julio\or a\u{u}gusto\or septembro\or oktobro\or novembro\or
  decembro\fi, \space \number\year}
%    \end{macrocode}
% \end{macro}
% \end{macro}
%
%    The macro |\ldf@finish| takes care of looking for a
%    configuration file, setting the main language to be switched on
%    at |\begin{document}| and resetting the category code of
%    \texttt{@} to its original value.
% \changes{esperanto-1.4i}{1996/11/02}{Now use \cs{ldf@finish} to wrap
%    up} 
%    \begin{macrocode}
\ldf@finish{esperanto}
%</code>
%    \end{macrocode}
%
% \Finale
%%
%% \CharacterTable
%%  {Upper-case    \A\B\C\D\E\F\G\H\I\J\K\L\M\N\O\P\Q\R\S\T\U\V\W\X\Y\Z
%%   Lower-case    \a\b\c\d\e\f\g\h\i\j\k\l\m\n\o\p\q\r\s\t\u\v\w\x\y\z
%%   Digits        \0\1\2\3\4\5\6\7\8\9
%%   Exclamation   \!     Double quote  \"     Hash (number) \#
%%   Dollar        \$     Percent       \%     Ampersand     \&
%%   Acute accent  \'     Left paren    \(     Right paren   \)
%%   Asterisk      \*     Plus          \+     Comma         \,
%%   Minus         \-     Point         \.     Solidus       \/
%%   Colon         \:     Semicolon     \;     Less than     \<
%%   Equals        \=     Greater than  \>     Question mark \?
%%   Commercial at \@     Left bracket  \[     Backslash     \\
%%   Right bracket \]     Circumflex    \^     Underscore    \_
%%   Grave accent  \`     Left brace    \{     Vertical bar  \|
%%   Right brace   \}     Tilde         \~}
%%
\endinput
}
\DeclareOption{estonian}{% \iffalse meta-comment
%
% Copyright 1989-2005 Johannes L. Braams and any individual authors
% listed elsewhere in this file.  All rights reserved.
% 
% This file is part of the Babel system.
% --------------------------------------
% 
% It may be distributed and/or modified under the
% conditions of the LaTeX Project Public License, either version 1.3
% of this license or (at your option) any later version.
% The latest version of this license is in
%   http://www.latex-project.org/lppl.txt
% and version 1.3 or later is part of all distributions of LaTeX
% version 2003/12/01 or later.
% 
% This work has the LPPL maintenance status "maintained".
% 
% The Current Maintainer of this work is Johannes Braams.
% 
% The list of all files belonging to the Babel system is
% given in the file `manifest.bbl. See also `legal.bbl' for additional
% information.
% 
% The list of derived (unpacked) files belonging to the distribution
% and covered by LPPL is defined by the unpacking scripts (with
% extension .ins) which are part of the distribution.
% \fi
% \CheckSum{330}
% \iffalse
%    Tell the \LaTeX\ system who we are and write an entry on the
%    transcript.
%<*dtx>
\ProvidesFile{estonian.dtx}
%</dtx>
%<code>\ProvidesLanguage{estonian}
%\fi
%\ProvidesFile{estonian.dtx}
        [2005/03/30 v1.0h Estonian support from the babel system]
%\iffalse
%% File `estonian.dtx'
%% Babel package for LaTeX version 2e
%% Copyright (C) 1989 - 2005
%%           by Johannes Braams, TeXniek
%
%% Estonian language Definition File
%% Copyright (C) 1991 - 2005
%%           by Enn Saar, Tartu Astrophysical Observatory
%              Tartu Astrophysical Observatory
%              EE-2444 T\~oravere
%              Estonia
%              tel: +372 7 410 267
%              fax: +372 7 410 205
%              saar@aai.ee
%
%              Johannes Braams, TeXniek
%
%% Please report errors to: Enn Saar saar at aai.ee
%%                          (or J.L. Braams babel atbraams.cistron.nl
%
%    This file is part of the babel system, it provides the source
%    code for the Estonian language definition file.  The original
%    version of this file was written by Enn Saar,
%    (saar@aai.ee).
%<*filedriver>
\documentclass{ltxdoc}
\newcommand*\TeXhax{\TeX hax}
\newcommand*\babel{\textsf{babel}}
\newcommand*\langvar{$\langle \it lang \rangle$}
\newcommand*\note[1]{}
\newcommand*\Lopt[1]{\textsf{#1}}
\newcommand*\file[1]{\texttt{#1}}
\begin{document}
 \DocInput{estonian.dtx}
\end{document}
%</filedriver>
%\fi
%
% \changes{estonian-1.0b}{1995/06/16}{corrected typos}
% \changes{estonian-1.0e}{1996/10/10}{Replaced \cs{undefined} with
%    \cs{@undefined} and \cs{empty} with \cs{@empty} for consistency
%    with \LaTeX, moved the definition of \cs{atcatcode} right to the
%    beginning.}
%
% \GetFileInfo{estonian.dtx}
%
% \section{The Estonian language}
%
%    The file \file{\filename}\footnote{The file described in this
%    section has version number \fileversion\ and was last revised on
%    \filedate. The original author is Enn Saar,
%    (\texttt{saar@aai.ee}).}  defines the language definition macro's
%    for the Estonian language.
%
%    This file was written as part of the TWGML project, and borrows
%    heavily from the \babel\ German and Spanish language files
%    \file{germanb.ldf} and \file{spanish.ldf}.
%
%    Estonian has the same umlauts as German (\"a, \"o, \"u), but in
%    addition to this, we have also \~o, and two recent characters
%    \v s and \v z, so we need at least two active characters.
%    We shall use |"| and |~| to type Estonian accents on ASCII
%    keyboards (in the 7-bit character world). Their use is given in
%    table~\ref{tab:estonian-quote}.
%    \begin{table}[htb]
%     \begin{center}
%     \begin{tabular}{lp{8cm}}
%      |~o| & |\~o|, (and uppercase); \\
%      |"a| & |\"a|, (and uppercase); \\
%      |"o| & |\"o|, (and uppercase); \\
%      |"u| & |\"u|, (and uppercase); \\
%      |~s| & |\v s|, (and uppercase); \\
%      |~z| & |\v z|, (and uppercase); \\
%      \verb="|= & disable ligature at this position;\\
%      |"-| & an explicit hyphen sign, allowing hyphenation
%                  in the rest of the word;\\
%      |\-| & like the old |\-|, but allowing hyphenation
%             in the rest of the word; \\
%      |"`| & for Estonian low left double quotes (same as German);\\
%      |"'| & for Estonian right double quotes;\\
%      |"<| & for French left double quotes (also rather popular)\\
%      |">| & for French right double quotes.\\
%     \end{tabular}
%     \caption{The extra definitions made
%              by \file{estonian.ldf}}\label{tab:estonian-quote}
%     \end{center}
%    \end{table}
%    These active accent characters behave according to their original
%    definitions if not followed by one of the characters indicated in
%    that table; the original quote character can be typed using the
%    macro |\dq|.
%
%    We support also the T1 output encoding (and Cork-encoded text
%    input).  You can choose the T1 encoding by the command
%    |\usepackage[T1]{fontenc}|.  This package must be loaded before
%    \babel. As the standard Estonian hyphenation file
%    \file{eehyph.tex} is in the Cork encoding, choosing this encoding
%    will give you better hyphenation.
%
%    As mentioned in the Spanish style file, it may happen that some
%    packages fail (usually in a \cs{message}). In this case you
%    should change the order of the \cs{usepackage} declarations
%    or the order of the style options in \cs{documentclass}.
%
% \StopEventually{}
%
% \subsection{Implementation}
%
%    The macro |\LdfInit| takes care of preventing that this file is
%    loaded more than once, checking the category code of the
%    \texttt{@} sign, etc.
% \changes{estonian-1.0e}{1996/10/30}{Now use \cs{LdfInit} to perform
%    initial checks} 
%    \begin{macrocode}
%<*code>
\LdfInit{estonian}\captionsestonian
%    \end{macrocode}
%
%    If Estonian is not included in the format file (does not have
%    hyphenation patterns), we shall use English hyphenation.
%
%    \begin{macrocode}
\ifx\l@estonian\@undefined
  \@nopatterns{Estonian}
  \adddialect\l@estonian0
\fi
%    \end{macrocode}
%
%    Now come the commands to switch to (and from) Estonian.
%
%  \begin{macro}{\captionsestonian}
%    The macro |\captionsestonian| defines all strings used in the
%    four standard documentclasses provided with \LaTeX.
%
% \changes{estonian-1.0c}{1995/07/04}{Added \cs{proofname} for
%    AMS-\LaTeX}
% \changes{estonian-1.0d}{1995/07/27}{Added translation of `Proof'}
% \changes{estonian-1.0h}{2000/09/20}{Added \cs{glossaryname}}
%    \begin{macrocode}
\addto\captionsestonian{%
  \def\prefacename{Sissejuhatus}%
  \def\refname{Viited}%
  \def\bibname{Kirjandus}%
  \def\appendixname{Lisa}%
  \def\contentsname{Sisukord}%
  \def\listfigurename{Joonised}%
  \def\listtablename{Tabelid}%
  \def\indexname{Indeks}%
  \def\figurename{Joonis}%
  \def\tablename{Tabel}%
  \def\partname{Osa}%
  \def\enclname{Lisa(d)}%
  \def\ccname{Koopia(d)}%
  \def\headtoname{}%
  \def\pagename{Lk.}%
  \def\seename{vt.}%
  \def\alsoname{vt. ka}%
  \def\proofname{Korrektuur}%
  \def\glossaryname{Glossary}% <-- Needs translation
  }
%    \end{macrocode}
%
%    These captions contain accented characters.
%
%    \begin{macrocode}
\begingroup \catcode`\"\active
\def\x{\endgroup
\addto\captionsestonian{%
  \def\abstractname{Kokkuv~ote}%
  \def\chaptername{Peat"ukk}}}
\x
%    \end{macrocode}
%  \end{macro}
%
%  \begin{macro}{\dateestonian}
%    The macro |\dateestonian| redefines the command |\today| to
%    produce Estonian dates.
% \changes{estonian-1.0f}{1997/10/01}{Use \cs{edef} to define
%    \cs{today} to save memory}
% \changes{estonian-1.0f}{1998/03/28}{use \cs{def} instead of
%    \cs{edef}} 
%    \begin{macrocode}
\begingroup \catcode`\"\active
\def\x{\endgroup
  \def\month@estonian{\ifcase\month\or
    jaanuar\or veebruar\or m"arts\or aprill\or mai\or juuni\or
    juuli\or august\or september\or oktoober\or november\or
    detsember\fi}}
\x
\def\dateestonian{%
  \def\today{\number\day.\space\month@estonian
    \space\number\year.\space a.}}
%    \end{macrocode}
%  \end{macro}
%
%  \begin{macro}{\extrasestonian}
%  \begin{macro}{\noextrasestonian}
%    The macro |\extrasestonian| will perform all the extra
%    definitions needed for Estonian. The macro |\noextrasestonian| is
%    used to cancel the actions of |\extrasestonian|. For Estonian,
%    |"| is made active and has to be treated as `special' (|~| is
%    active already).
%
%    \begin{macrocode}
\initiate@active@char{"}
\initiate@active@char{~}
\addto\extrasestonian{\languageshorthands{estonian}}
\addto\extrasestonian{\bbl@activate{"}\bbl@activate{~}}
%    \end{macrocode}
%    Store the original macros, and redefine accents.
%
%    \begin{macrocode}
\addto\extrasestonian{\babel@save\"\umlautlow\babel@save\~\tildelow}
%    \end{macrocode}
%
% \changes{estonian-1.0d}{1995/07/27}{Removed the code that changes
%    category, lower case, uper case and space factor codes }
%
%    Estonian does not use extra spaces after sentences.
%
%    \begin{macrocode}
\addto\extrasestonian{\bbl@frenchspacing}
\addto\noextrasestonian{\bbl@nonfrenchspacing}
%    \end{macrocode}
%  \end{macro}
%  \end{macro}
%
%  \begin{macro}{\estonianhyphenmins}
%     For Estonian, |\lefthyphenmin| and |\righthyphenmin| are
%    both~2. 
% \changes{estonian-1.0h}{2000/09/22}{Now use \cs{providehyphenmins} to
%    provide a default value}
%    \begin{macrocode}
\providehyphenmins{\CurrentOption}{\tw@\tw@}
%    \end{macrocode}
%  \end{macro}
%
%  \begin{macro}{\tildelow}
%  \begin{macro}{\gentilde}
%  \begin{macro}{\newtilde}
%  \begin{macro}{\newcheck}
%    The standard \TeX\ accents are too high for Estonian typography,
%    we have to lower them (following the \babel\ German style).  For
%    a detailed explanation see the file \file{glyphs.dtx}.
%
%    \begin{macrocode}
\def\tildelow{\def\~{\protect\gentilde}}
\def\gentilde#1{\if#1o\newtilde{#1}\else\if#1O\newtilde{#1}%
    \else\newcheck{#1}%
    \fi\fi}
\def\newtilde#1{\leavevmode\allowhyphens
  {\U@D 1ex%
  {\setbox\z@\hbox{\char126}\dimen@ -.45ex\advance\dimen@\ht\z@
  \ifdim 1ex<\dimen@ \fontdimen5\font\dimen@ \fi}%
  \accent126\fontdimen5\font\U@D #1}\allowhyphens}
\def\newcheck#1{\leavevmode\allowhyphens
  {\U@D 1ex%
  {\setbox\z@\hbox{\char20}\dimen@ -.45ex\advance\dimen@\ht\z@
  \ifdim 1ex<\dimen@ \fontdimen5\font\dimen@ \fi}%
  \accent20\fontdimen5\font\U@D #1}\allowhyphens}
%    \end{macrocode}
%  \end{macro}
%  \end{macro}
%  \end{macro}
%  \end{macro}
%
%    We save the double quote character in |\dq|, and  tilde in |\til|,
%    and store the original definitions of |\"| and |\~| as |\dieresis|
%    and |\texttilde|.
%
%    \begin{macrocode}
\begingroup \catcode`\"12
\edef\x{\endgroup
  \def\noexpand\dq{"}
  \def\noexpand\til{~}}
\x
\let\dieresis\"
\let\texttilde\~
%    \end{macrocode}
%
%    This part follows closely \file{spanish.ldf}. We check the
%    encoding and if it is T1, we have to tell \TeX\ about our
%    redefined accents.
%
% \changes{estonian-1.0g}{1999/04/11}{use \cs{bbl@t@one} instead of
%    \cs{bbl@next}} 
%    \begin{macrocode}
\ifx\f@encoding\bbl@t@one
  \let\@umlaut\dieresis
  \let\@tilde\texttilde
  \DeclareTextComposite{\~}{T1}{s}{178}
  \DeclareTextComposite{\~}{T1}{S}{146}
  \DeclareTextComposite{\~}{T1}{z}{186}
  \DeclareTextComposite{\~}{T1}{Z}{154}
  \DeclareTextComposite{\"}{T1}{'}{17}
  \DeclareTextComposite{\"}{T1}{`}{18}
  \DeclareTextComposite{\"}{T1}{<}{19}
  \DeclareTextComposite{\"}{T1}{>}{20}
%    \end{macrocode}
%
%    If the encoding differs from T1, we expand the accents, enabling
%    hyphenation beyond the accent. In this case \TeX\ will not find
%    all possible breaks, and we have to warn people.
%
%    \begin{macrocode}
\else
  \wlog{Warning: Hyphenation would work better for the T1 encoding.}
  \let\@umlaut\newumlaut
  \let\@tilde\gentilde
\fi
%    \end{macrocode}
%
%     Now we define the shorthands.
%
% \changes{estonian-1.0d}{1995/07/27}{The second argument was missing
%    in the definition of some of the double-quote shorthands}
%    \begin{macrocode}
\declare@shorthand{estonian}{"a}{\textormath{\"{a}}{\ddot a}}
\declare@shorthand{estonian}{"A}{\textormath{\"{A}}{\ddot A}}
\declare@shorthand{estonian}{"o}{\textormath{\"{o}}{\ddot o}}
\declare@shorthand{estonian}{"O}{\textormath{\"{O}}{\ddot O}}
\declare@shorthand{estonian}{"u}{\textormath{\"{u}}{\ddot u}}
\declare@shorthand{estonian}{"U}{\textormath{\"{U}}{\ddot U}}
%    \end{macrocode}
%    german and french quotes,
% \changes{estonian-1.0f}{1997/04/03}{Removed empty groups after
%    double quote and guillemot characters}
%    \begin{macrocode}
\declare@shorthand{estonian}{"`}{%
  \textormath{\quotedblbase}{\mbox{\quotedblbase}}}
\declare@shorthand{estonian}{"'}{%
  \textormath{\textquotedblleft}{\mbox{\textquotedblleft}}}
\declare@shorthand{estonian}{"<}{%
  \textormath{\guillemotleft}{\mbox{\guillemotleft}}}
\declare@shorthand{estonian}{">}{%
  \textormath{\guillemotright}{\mbox{\guillemotright}}}
%    \end{macrocode}
%    
%    \begin{macrocode}
\declare@shorthand{estonian}{~o}{\textormath{\@tilde o}{\tilde o}}
\declare@shorthand{estonian}{~O}{\textormath{\@tilde O}{\tilde O}}
\declare@shorthand{estonian}{~s}{\textormath{\@tilde s}{\check s}}
\declare@shorthand{estonian}{~S}{\textormath{\@tilde S}{\check S}}
\declare@shorthand{estonian}{~z}{\textormath{\@tilde z}{\check z}}
\declare@shorthand{estonian}{~Z}{\textormath{\@tilde Z}{\check Z}}
%    \end{macrocode}
%    and some additional commands:
%    \begin{macrocode}
\declare@shorthand{estonian}{"-}{\nobreak\-\bbl@allowhyphens}
\declare@shorthand{estonian}{"|}{%
  \textormath{\nobreak\discretionary{-}{}{\kern.03em}%
              \allowhyphens}{}}
\declare@shorthand{estonian}{""}{\dq}
\declare@shorthand{estonian}{~~}{\til}
%    \end{macrocode}
%
%    The macro |\ldf@finish| takes care of looking for a
%    configuration file, setting the main language to be switched on
%    at |\begin{document}| and resetting the category code of
%    \texttt{@} to its original value.
% \changes{estonian-1.0e}{1996/11/02}{Now use \cs{ldf@finish} to wrap
%    up} 
%    \begin{macrocode}
\ldf@finish{estonian}
%</code>
%    \end{macrocode}
%
% \Finale
%%
%% \CharacterTable
%%  {Upper-case    \A\B\C\D\E\F\G\H\I\J\K\L\M\N\O\P\Q\R\S\T\U\V\W\X\Y\Z
%%   Lower-case    \a\b\c\d\e\f\g\h\i\j\k\l\m\n\o\p\q\r\s\t\u\v\w\x\y\z
%%   Digits        \0\1\2\3\4\5\6\7\8\9
%%   Exclamation   \!     Double quote  \"     Hash (number) \#
%%   Dollar        \$     Percent       \%     Ampersand     \&
%%   Acute accent  \'     Left paren    \(     Right paren   \)
%%   Asterisk      \*     Plus          \+     Comma         \,
%%   Minus         \-     Point         \.     Solidus       \/
%%   Colon         \:     Semicolon     \;     Less than     \<
%%   Equals        \=     Greater than  \>     Question mark \?
%%   Commercial at \@     Left bracket  \[     Backslash     \\
%%   Right bracket \]     Circumflex    \^     Underscore    \_
%%   Grave accent  \`     Left brace    \{     Vertical bar  \|
%%   Right brace   \}     Tilde         \~}
%%
\endinput
}
\DeclareOption{finnish}{%%
%% This file will generate fast loadable files and documentation
%% driver files from the doc files in this package when run through
%% LaTeX or TeX.
%%
%% Copyright 1989-2007 Johannes L. Braams and any individual authors
%% listed elsewhere in this file.  All rights reserved.
%% 
%% This file is part of the Babel system.
%% --------------------------------------
%% 
%% It may be distributed and/or modified under the
%% conditions of the LaTeX Project Public License, either version 1.3
%% of this license or (at your option) any later version.
%% The latest version of this license is in
%%   http://www.latex-project.org/lppl.txt
%% and version 1.3 or later is part of all distributions of LaTeX
%% version 2003/12/01 or later.
%% 
%% This work has the LPPL maintenance status "maintained".
%% 
%% The Current Maintainer of this work is Johannes Braams.
%% 
%% The list of all files belonging to the LaTeX base distribution is
%% given in the file `manifest.bbl. See also `legal.bbl' for additional
%% information.
%% 
%% The list of derived (unpacked) files belonging to the distribution
%% and covered by LPPL is defined by the unpacking scripts (with
%% extension .ins) which are part of the distribution.
%%
%% --------------- start of docstrip commands ------------------
%%
\def\filedate{1999/04/11}
\def\batchfile{finnish.ins}
\input docstrip.tex

{\ifx\generate\undefined
\Msg{**********************************************}
\Msg{*}
\Msg{* This installation requires docstrip}
\Msg{* version 2.3c or later.}
\Msg{*}
\Msg{* An older version of docstrip has been input}
\Msg{*}
\Msg{**********************************************}
\errhelp{Move or rename old docstrip.tex.}
\errmessage{Old docstrip in input path}
\batchmode
\csname @@end\endcsname
\fi}

\declarepreamble\mainpreamble
This is a generated file.

Copyright 1989-2007 Johannes L. Braams and any individual authors
listed elsewhere in this file.  All rights reserved.

This file was generated from file(s) of the Babel system.
---------------------------------------------------------

It may be distributed and/or modified under the
conditions of the LaTeX Project Public License, either version 1.3
of this license or (at your option) any later version.
The latest version of this license is in
  http://www.latex-project.org/lppl.txt
and version 1.3 or later is part of all distributions of LaTeX
version 2003/12/01 or later.

This work has the LPPL maintenance status "maintained".

The Current Maintainer of this work is Johannes Braams.

This file may only be distributed together with a copy of the Babel
system. You may however distribute the Babel system without
such generated files.

The list of all files belonging to the Babel distribution is
given in the file `manifest.bbl'. See also `legal.bbl for additional
information.

The list of derived (unpacked) files belonging to the distribution
and covered by LPPL is defined by the unpacking scripts (with
extension .ins) which are part of the distribution.
\endpreamble

\declarepreamble\fdpreamble
This is a generated file.

Copyright 1989-2005 Johannes L. Braams and any individual authors
listed elsewhere in this file.  All rights reserved.

This file was generated from file(s) of the Babel system.
---------------------------------------------------------

It may be distributed and/or modified under the
conditions of the LaTeX Project Public License, either version 1.3
of this license or (at your option) any later version.
The latest version of this license is in
  http://www.latex-project.org/lppl.txt
and version 1.3 or later is part of all distributions of LaTeX
version 2003/12/01 or later.

This work has the LPPL maintenance status "maintained".

The Current Maintainer of this work is Johannes Braams.

This file may only be distributed together with a copy of the Babel
system. You may however distribute the Babel system without
such generated files.

The list of all files belonging to the Babel distribution is
given in the file `manifest.bbl'. See also `legal.bbl for additional
information.

In particular, permission is granted to customize the declarations in
this file to serve the needs of your installation.

However, NO PERMISSION is granted to distribute a modified version
of this file under its original name.

\endpreamble

\keepsilent

\usedir{tex/generic/babel} 

\usepreamble\mainpreamble
\generate{\file{finnish.ldf}{\from{finnish.dtx}{code}}
          }
\usepreamble\fdpreamble

\ifToplevel{
\Msg{***********************************************************}
\Msg{*}
\Msg{* To finish the installation you have to move the following}
\Msg{* files into a directory searched by TeX:}
\Msg{*}
\Msg{* \space\space All *.def, *.fd, *.ldf, *.sty}
\Msg{*}
\Msg{* To produce the documentation run the files ending with}
\Msg{* '.dtx' and `.fdd' through LaTeX.}
\Msg{*}
\Msg{* Happy TeXing}
\Msg{***********************************************************}
}
 
\endinput
}
%    \end{macrocode}
%    The \babel\ support or French used to be stored in
%    \file{francais.ldf}; therefor the \LaTeX2.09 option used to be
%    \Lopt{francais}. The hyphenation patterns may be loaded as either
%    `french' or as `francais'.
% \changes{babel~3.5f}{1996/01/10}{Now use the file \file{frenchb.ldf}
%    from Daniel Flipo for french support}
% \changes{babel~3.6e}{1997/01/08}{Added option \Lopt{frenchb} an
%    alias for \Lopt{francais}}
%    \begin{macrocode}
\DeclareOption{francais}{% \iffalse meta-comment
%
% Copyright 1989-2009 Johannes L. Braams and any individual authors
% listed elsewhere in this file.  All rights reserved.
% 
% This file is part of the Babel system.
% --------------------------------------
% 
% It may be distributed and/or modified under the
% conditions of the LaTeX Project Public License, either version 1.3
% of this license or (at your option) any later version.
% The latest version of this license is in
%   http://www.latex-project.org/lppl.txt
% and version 1.3 or later is part of all distributions of LaTeX
% version 2003/12/01 or later.
% 
% This work has the LPPL maintenance status "maintained".
% 
% The Current Maintainer of this work is Johannes Braams.
% 
% The list of all files belonging to the Babel system is
% given in the file `manifest.bbl. See also `legal.bbl' for additional
% information.
% 
% The list of derived (unpacked) files belonging to the distribution
% and covered by LPPL is defined by the unpacking scripts (with
% extension .ins) which are part of the distribution.
% \fi
% \CheckSum{2135}
%
% \iffalse
%    Tell the \LaTeX\ system who we are and write an entry on the
%    transcript. Nothing to write to the .cfg file, if generated.
%<*dtx>
\ProvidesFile{frenchb.dtx}
%</dtx>
% \changes{v2.1d}{2008/05/04}{Argument of \cs{ProvidesLanguage} changed
%     from `french' to `frenchb', otherwise \cs{listfiles} prints
%     no date/version information.  The bug with \cs{listfiles}
%     (introduced in v.1.5!), was pointed out by Ulrike Fischer.}
%<code>\ProvidesLanguage{frenchb}
%\ProvidesFile{frenchb.dtx}
%<*!cfg>
        [2009/03/16 v2.3d French support from the babel system]
%</!cfg>
%<*cfg>
%% frenchb.cfg: configuration file for frenchb.ldf
%% Daniel Flipo Daniel.Flipo at univ-lille1.fr
%</cfg>
%%    File `frenchb.dtx'
%%    Babel package for LaTeX version 2e
%%    Copyright (C) 1989 - 2009
%%              by Johannes Braams, TeXniek
%
%<*!cfg>
%%    Frenchb language Definition File
%%    Copyright (C) 1989 - 2009
%%              by Johannes Braams, TeXniek
%%                 Daniel Flipo, GUTenberg
%
%%    Please report errors to: Daniel Flipo, GUTenberg
%%                             Daniel.Flipo at univ-lille1.fr
%</!cfg>
%
%    This file is part of the babel system, it provides the source
%    code for the French language definition file.
%
%<*filedriver>
\documentclass[a4paper]{ltxdoc}
\DeclareFontEncoding{T1}{}{}
\DeclareFontSubstitution{T1}{lmr}{m}{n}
\DeclareTextCommand{\guillemotleft}{OT1}{%
  {\fontencoding{T1}\fontfamily{lmr}\selectfont\char19}}
\DeclareTextCommand{\guillemotright}{OT1}{%
  {\fontencoding{T1}\fontfamily{lmr}\selectfont\char20}}
\newcommand*\TeXhax{\TeX hax}
\newcommand*\babel{\textsf{babel}}
\newcommand*\langvar{$\langle \mathit lang \rangle$}
\newcommand*\note[1]{}
\newcommand*\Lopt[1]{\textsf{#1}}
\newcommand*\file[1]{\texttt{#1}}
\begin{document}
\setlength{\parindent}{0pt}
\begin{center}
  \textbf{\Large A Babel language definition file for French}\\[3mm]^^A\]
  Daniel \textsc{Flipo}\\
  \texttt{Daniel.Flipo@univ-lille1.fr}
\end{center}
 \RecordChanges
 \DocInput{frenchb.dtx}
\end{document}
%</filedriver>
% \fi
% \GetFileInfo{frenchb.dtx}
%
%  \section{The French language}
%
%    The file \file{\filename}\footnote{The file described in this
%    section has version number \fileversion\ and was last revised on
%    \filedate.}, defines all the language definition macros for the
%    French language.
%
%    Customisation for the French language is achieved following the
%    book ``Lexique des r\`egles typographiques en usage \`a
%    l'Imprimerie nationale'' troisi\`eme \'edition (1994),
%    ISBN-2-11-081075-0.
%
%    First version released: 1.1 (1996/05/31) as part of
%    \babel-3.6beta.
%
%    |frenchb| has been improved using helpful suggestions from many
%    people, mainly from Jacques Andr\'e, Michel Bovani, Thierry Bouche,
%    and Vincent Jalby.  Thanks to all of them!
%
%    This new version (2.x) has been designed to be used with \LaTeXe{}
%    and Plain\TeX{} formats only. \LaTeX-2.09 is no longer supported.
%    Changes between version 1.6 and \fileversion{} are listed in
%    subsection~\ref{ssec-changes} p.~\pageref{ssec-changes}.
%
%    An extensive documentation is available in French here:\\
%    |http://daniel.flipo.free.fr/frenchb|
%
%  \subsection{Basic interface}
%
%    In a multilingual document, some typographic rules are language
%    dependent, i.e. spaces before `double punctuation' (|:| |;| |!|
%    |?|) in French, others concern the general layout (i.e. layout of
%    lists, footnotes, indentation of first paragraphs of sections) and
%    should apply to the whole document.
%
%    Starting with version~2.2, |frenchb| behaves differently according
%    to \babel's \emph{main language} defined as the \emph{last}
%    option\footnote{Its name is kept in \texttt{\textbackslash
%           bbl@main@language}.} at \babel's loading.  When French is
%    not \babel's main language, |frenchb| no longer alters the global
%    layout of the document (even in parts where French is the current
%    language): the layout of lists, footnotes, indentation of first
%    paragraphs of sections are not customised by |frenchb|.
%
%    When French is loaded as the last option of \babel, |frenchb|
%    makes the following changes to the global layout, \emph{both in
%    French and in all other languages}\footnote{%
%       For each item, hooks are provided to reset standard
%       \LaTeX{} settings or to emulate the behavior of former versions
%       of \texttt{frenchb} (see command
%       \texttt{\textbackslash frenchbsetup\{\}},
%       section~\ref{ssec-custom}).}:
%    \begin{enumerate}
%    \item the first paragraph of each section is indented
%          (\LaTeX{} only);
%    \item the default items in itemize environment are set to `--'
%          instead of `\textbullet', and all vertical spacing and glue
%          is deleted; it is possible to change `--' to something else
%          (`---' for instance) using |\frenchbsetup{}|;
%    \item vertical spacing in general \LaTeX{} lists is
%          shortened;
%    \item footnotes are displayed ``\`a la fran\c{c}aise''.
%    \end{enumerate}
%
%    Regarding local typography, the command |\selectlanguage{french}|
%    switches to the French language\footnote{%
%      \texttt{\textbackslash selectlanguage\{francais\}}
%      and \texttt{\textbackslash selectlanguage\{frenchb\}} are kept
%      for backward compatibility but should no longer be used.},
%    with the following effects:
%    \begin{enumerate}
%    \item French hyphenation patterns are made active;
%    \item `double punctuation' (|:| |;| |!| |?|) is made
%           active%\footnote{Actually, they are active in the whole
%           document, only their expansions differ in French and
%           outside French} for correct spacing in French;
%    \item |\today| prints the date in French;
%    \item the caption names are translated into French
%          (\LaTeX{} only);
%    \item the space after |\dots| is removed in French.
%    \end{enumerate}
%
%    Some commands are provided in |frenchb| to make typesetting
%    easier:
%    \begin{enumerate}
%    \item French quotation marks can be entered using the commands
%          |\og| and |\fg| which work in \LaTeXe and Plain\TeX,
%          their appearance depending on what is available to draw
%          them; even if you use \LaTeXe{} \emph{and} |T1|-encoding,
%          you should refrain from entering them as
%          |<<~French quotation marks~>>|: |\og| and |\fg| provide
%          better horizontal spacing.
%          |\og| and |\fg| can be used outside French, they typeset
%          then English quotes `` and ''.
%    \item A command |\up| is provided to typeset superscripts like
%          |M\up{me}| (abbreviation for ``Madame''), |1\up{er}| (for
%          ``premier'').  Other commands are also provided for
%          ordinals: |\ier|, |\iere|, |\iers|, |\ieres|, |\ieme|,
%          |\iemes| (|3\iemes| prints 3\textsuperscript{es}).
%    \item Family names should be typeset in small capitals and never
%          be hyphenated, the macro |\bsc| (boxed small caps) does
%          this, e.g., |Leslie~\bsc{Lamport}| will produce
%          Leslie~\mbox{\textsc{Lamport}}. Note that composed names
%          (such as Dupont-Durant) may now be hyphenated on explicit
%          hyphens, this differs from |frenchb|~v.1.x.
%    \item Commands |\primo|, |\secundo|, |\tertio| and |\quarto|
%          print 1\textsuperscript{o}, 2\textsuperscript{o},
%          3\textsuperscript{o}, 4\textsuperscript{o}.
%          |\FrenchEnumerate{6}| prints 6\textsuperscript{o}.
%    \item Abbreviations for ``Num\'ero(s)'' and ``num\'ero(s)''
%          (N\textsuperscript{o} N\textsuperscript{os}
%          n\textsuperscript{o} and n\textsuperscript{os}~)
%          are obtained via the commands |\No|, |\Nos|, |\no|, |\nos|.
%    \item Two commands are provided to typeset the symbol for
%          ``degr\'e'': |\degre| prints the raw character and
%          |\degres| should be used to typeset temperatures (e.g.,
%          ``|20~\degres C|'' with an unbreakable space), or for
%          alcohols' strengths (e.g., ``|45\degres|'' with \emph{no}
%          space in French).
%    \item In math mode the comma has to be surrounded with
%          braces to avoid a spurious space being inserted after it,
%          in decimal numbers for instance (see the \TeX{}book p.~134).
%          The command |\DecimalMathComma| makes the comma be an
%          ordinary character \emph{in French only} (no space added);
%          as a counterpart, if |\DecimalMathComma| is active, an
%          explicit space has to be added in lists and intervals:
%          |$[0,\ 1]$|, |$(x,\ y)$|. |\StandardMathComma| switches back
%          to the standard behaviour of the comma.
%    \item A command |\nombre| was provided in 1.x versions to easily
%          format numbers in slices of three digits separated either
%          by a comma in English or with a space in French; |\nombre|
%          is now mapped to |\numprint| from \file{numprint.sty}, see
%          \file{numprint.pdf} for more information.
%    \item |frenchb| has been designed to take advantage of the |xspace|
%          package if present: adding |\usepackage{xspace}| in the
%          preamble will force macros like |\fg|, |\ier|, |\ieme|,
%          |\dots|, \dots, to respect the spaces you type after them,
%          for instance typing `|1\ier juin|' will print
%          `1\textsuperscript{er} juin' (no need for a forced space
%          after |1\ier|).
%    \end{enumerate}
%
%  \subsection{Customisation}
%  \label{ssec-custom}
%
%     Up to version 1.6, customisation of |frenchb| was achieved
%     by entering commands in \file{frenchb.cfg}.  This possibility
%     remains for compatibility, but \emph{should not longer be used}.
%     Version 2.0 introduces a new command |\frenchbsetup{}| using
%     the \file{keyval} syntax which should make it easier to choose
%     among the many options available. The command |\frenchbsetup{}|
%     is to appear in the preamble only (after loading \babel).
%
%     \vspace{.5\baselineskip}
%     |\frenchbsetup{ShowOptions}| prints all available options to
%     the \file{.log} file, it is just meant as a remainder of the
%     list of offered options. As usual with \file{keyval} syntax,
%     boolean options (as |ShowOptions|) can be entered as
%     |ShowOptions=true| or just |ShowOptions|, the `|=true|' part
%     can be omitted.
%
%     \vspace{.5\baselineskip}
%     The other options are listed below. Their default value is shown
%     between brackets, sometimes followed be a `\texttt{*}'.
%     The `\texttt{*}' means that the default shown applies when
%     |frenchb| is loaded as the \emph{last} option of \babel{}
%     ---\babel's \emph{main language}---, and is toggled otherwise:
%     \begin{itemize}
%     \item |StandardLayout=true [false*]| forces |frenchb| not to
%       interfere with the layout: no action on any kind of lists,
%       first paragraphs of sections are not indented (as in English),
%       no action on footnotes. This option replaces the former
%       command |\StandardLayout|.  It can be used to avoid conflicts
%       with classes or packages which customise lists or footnotes.
%     \item |GlobalLayoutFrench=false [true*]| can be used, when French
%       is the main language, to emulate what prior versions of
%       |frenchb| (pre-2.2) did: lists, and first paragraphs
%       of sections will be displayed the standard way in other
%       languages than French, and ``\`a la fran\c{c}aise'' in French.
%       Note that the layout of footnotes is language independent
%       anyway (see below |FrenchFootnotes| and |AutoSpaceFootnotes|).
%       This option replaces the former command |\FrenchLayout|.
%     \item |ReduceListSpacing=false [true*]|; |frenchb| normally
%       reduces the values of the vertical spaces used in the
%       environment |list| in French; setting this option to |false|
%       reverts to the standard settings of |list|.  This option
%       replaces the former command |\FrenchListSpacingfalse|.
%     \item |CompactItemize=false [true*]|; |frenchb| normally
%       suppresses any vertical space between items of |itemize| lists
%       in French; setting this option to |false| reverts to the
%       standard settings of |itemize| lists.  This option replaces
%       the former command |\FrenchItemizeSpacingfalse|.
%     \item |StandardItemLabels=true [false*]| when set to |true| this
%       option stops |frenchb| from changing the labels in |itemize|
%       lists in French.
%     \item |ItemLabels=\textemdash|, |\textbullet|, |\ding{43}|,
%       \dots, |[\textendash*]|; when |StandardItemLabels=false| (the
%       default), this option enables to choose the label used in
%       |itemize| lists for all levels.  The next three options do
%       the same but each one for one level only. Note that the
%       example |\ding{43}| requires |\usepackage{pifont}|.
%     \item |ItemLabeli=\textemdash|, |\textbullet|, |\ding{43}|,
%       \dots,|[\textendash*]|
%     \item |ItemLabelii=\textemdash|, |\textbullet|, |\ding{43}|,
%       \dots, |[\textendash*]|
%     \item |ItemLabeliii=\textemdash|, |\textbullet|, |\ding{43}|,
%       \dots, |[\textendash*]|
%     \item |ItemLabeliv=\textemdash|, |\textbullet|, |\ding{43}|,
%       \dots, |[\textendash*]|
%     \item |StandardLists=true [false*]| forbids |frenchb| to
%       customise any kind of list. Do activate the option
%       |StandardLists| when using classes or packages that customise
%       lists too (|enumitem|, |paralist|, \dots{}) to avoid conflicts.
%       This option is just a shorthand for |ReduceListSpacing=false|
%       and |CompactItemize=false| and |StandardItemLabels=true|.
%     \item |IndentFirst=false [true*]|; |frenchb| normally forces
%       indentation of the first paragraph of sections.
%       When this option is set to |false|, the first paragraph of
%       will look the same in French and in English (not indented).
%     \item |FrenchFootnotes=false [true*]| reverts to the standard
%       layout of footnotes. By default |frenchb| typesets leading
%       numbers as `1.\hspace{.5em}' instead of `$\hbox{}^1$', but
%       has no effect on footnotes numbered with symbols (as in the
%       |\thanks| command).  The former commands |\StandardFootnotes|
%       and |\FrenchFootnotes| are still there, |\StandardFootnotes|
%       can be useful when some footnotes are numbered with letters
%       (inside minipages for instance).
%     \item |AutoSpaceFootnotes=false [true*]| ; by default |frenchb|
%       adds a thin space in the running text before the number or
%       symbol calling the footnote.  Making this option |false|
%       reverts to the standard setting (no space added).
%     \item |FrenchSuperscripts=false [true]| ; then
%       |\up=\textsuperscript| (option added in version 2.1).
%       Should only be made |false| to recompile older documents.
%       By default |\up| now relies on |\fup| designed to produce
%       better looking superscripts.
%     \item |AutoSpacePunctuation=false [true]|; in French, the user
%       \emph{should} input a space before the four characters `|:;!?|'
%       but as many people forget about it (even among native French
%       writers!), the default behaviour of |frenchb| is to
%       automatically add a |\thinspace| before `|;|' `|!|' `|?|' and a
%       normal (unbreakable) space before~`|:|' (this is recommended by
%       the French Imprimerie nationale).  This is convenient in most
%       cases but can lead to addition of spurious spaces in URLs or in
%       MS-DOS paths but only if they are no typed using |\texttt| or
%       verbatim mode. When the current font is a monospaced
%       (typewriter) font, |AutoSpacePunctuation| is locally switched
%       to |false|, no spurious space is added in that case, so the
%       default behaviour of of |frenchb| in that area should be fine
%       in most circumstances.
%
%       Choosing |AutoSpacePunctuation=false| will ensure that
%       a proper space will be added before `|:;!?|' \emph{if and only
%       if} a (normal) space has been typed in. Those who are unsure
%       about their typing in this area should stick to the default
%       option and type |\string;| |\string:| |\string!| |\string?|
%       instead of |;| |:| |!| |?| in case an unwanted space is
%       added by |frenchb|.
%     \item |ThinColonSpace=true [false]| changes the normal
%       (unbreakable) space added before the colon `:' to a thin space,
%       so that the same amount of space is added before any of the
%       four double punctuation characters. The default setting is
%       supported by the French Imprimerie nationale.
%     \item |LowercaseSuperscripts=false [true]| ; by default |frenchb|
%       inhibits the uppercasing of superscripts (for instance when they
%       are moved to page headers). Making this option |false|
%       will disable this behaviour (not recommended).
%     \item |PartNameFull=false [true]|; when true, |frenchb| numbers
%       the title of |\part{}| commands as ``Premi\`ere partie'',
%       ``Deuxi\`eme partie'' and so on. With some classes which change
%       the|\part{}| command (AMS and SMF classes do so), you will get
%       ``Premi\`ere partie~I'', ``Deuxi\`eme partie~II'' instead;
%       when this occurs, this option should be set to |false|,
%       part titles will then be printed as ``Partie I'', ``Partie II''.
%     \item |og=|\texttt{\guillemotleft}, |fg=|\texttt{\guillemotright};
%       when guillemets characters are available on the keyboard
%       (through a compose key for instance), it is nice to use them
%       instead of typing |\og| and |\fg|. This option tells |frenchb|
%       which characters are opening and closing French guillemets
%       (they depend on the input encoding), then you can type either
%       \texttt{\guillemotleft{} guillemets \guillemotright}, or
%       \texttt{\guillemotleft{}guillemets\guillemotright} (with or
%       without spaces), to get properly typeset French quotes.
%       This option requires \file{inputenc} to be loaded with the
%       proper encoding, it works with 8-bits encodings (latin1,
%       latin9, ansinew,  applemac,\dots) and multi-byte encodings
%       (utf8 and utf8x).
%     \end{itemize}
%
%  \subsection{Hyphenation checks}
%  \label{ssec-hyphen}
%
%    Once you have built your format, a good precaution would be to
%    perform some basic tests about hyphenation in French. For
%    \LaTeXe{} I suggest this:
%    \begin{itemize}
%    \item run the following file, with the encoding suitable for
%      your machine (\textit{my-encoding} will be |latin1| for
%      \textsc{unix} machines, |ansinew| for PCs running~Windows,
%      |applemac| or |latin1| for Macintoshs, or |utf8|\dots\\[3mm]^^A\]
%      |%%% Test file for French hyphenation.|\\
%      |\documentclass{article}|\\
%      |\usepackage[|\textit{my-encoding}|]{inputenc}|\\
%      |\usepackage[T1]{fontenc} % Use LM fonts|\\
%      |\usepackage{lmodern}     % for French|\\
%      |\usepackage[frenchb]{babel}|\\
%      |\begin{document}|\\
%      |\showhyphens{signal container \'ev\'enement alg\`ebre}|\\
%      |\showhyphens{|\texttt{signal container \'ev\'enement
%                     alg\`ebre}|}|\\
%      |\end{document}|
%    \item check the hyphenations proposed by \TeX{} in your log-file;
%      in French you should get with both 7-bit and 8-bit encodings\\
%      \texttt{si-gnal contai-ner \'ev\'e-ne-ment al-g\`ebre}.\\
%      Do not care about how accented characters are displayed in the
%      log-file, what matters is the position of the `|-|' hyphen
%      signs \emph{only}.
%    \end{itemize}
%    If they are all correct, your installation (probably) works fine,
%    if one (or more) is (are) wrong, ask a local wizard to see what's
%    going wrong and perform the test again (or e-mail me about what
%    happens).\\
%    Frequent mismatches:
%    \begin{itemize}
%    \item you get |sig-nal con-tainer|, this probably means that the
%    hyphenation patterns you are using are for US-English, not for
%    French;
%    \item you get no hyphen at all in \texttt{\'ev\'e-ne-ment}, this
%    probably means that you are using CM fonts and the macro
%    |\accent| to produce accented characters.
%    Using 8-bits fonts with built-in accented characters avoids
%    this kind of mismatch.
%    \end{itemize}
%
%    \textbf{Options' order} -- Please remember that options are read
%    in the order they appear inside the |\frenchbsetup| command.
%    Someone wishing that |frenchb| leaves the layout of lists
%    and footnotes untouched but caring for indentation of first
%    paragraph of sections could choose
%    |\frenchbsetup{StandardLayout,IndentFirst}| and get the expected
%    layout. Choosing |\frenchbsetup{IndentFirst,StandardLayout}|
%    would not lead to the expected result: option |IndentFirst| would
%    be overwritten by |StandardLayout|.
%
%  \subsection{Changes}
%  \label{ssec-changes}
%
%  \subsubsection*{What's new in version 2.0?}
%
%    Here is the list of all changes:
%    \begin{itemize}
%    \item Support for \LaTeX-2.09 and for \LaTeXe{} in compatibility
%      mode has been dropped. This version is meant for \LaTeXe{} and
%      Plain based formats (like \file{bplain}). \LaTeXe{} formats
%      based on ml\TeX{} are no longer supported either (plenty of
%      good 8-bits fonts are available now, so T1 encoding should
%      be preferred for typesetting in French). A warning is issued
%      when OT1 encoding is in use at the |\begin{document}|.
%    \item Customisation should now be handled by command
%      |\frenchbsetup{}|, \file{frenchb.cfg} (kept for compatibility)
%      should no longer be used. See section~\ref{ssec-custom} for
%      the list of available options.
%    \item Captions in figures and table have changed in French: former
%      abbreviations ``Fig.'' and ``Tab.'' have been replaced by full
%      names ``Figure'' and ``Table''.  If this leads to formatting
%      problems in captions, you can add the following two commands to
%      your preamble (after loading \babel) to get the former captions\\
%      |\addto\captionsfrench{\def\figurename{{\scshape Fig.}}}|\\
%      |\addto\captionsfrench{\def\tablename{{\scshape Tab.}}}|.
%    \item The |\nombre| command is now provided by the \file{numprint}
%      package which has to be loaded \emph{after} \babel{} with the
%      option |autolanguage| if number formatting should depend on the
%      current language.
%    \item The |\bsc| command no longer uses an |\hbox| to stop
%      hyphenation of names but a |\kern0pt| instead. This change
%      enables \file{microtype} to fine tune the length of the
%      argument of |\bsc|; as a side-effect, compound names like
%      Dupont-Durand can now be hyphenated on  explicit hyphens.
%      You can get back to the former behaviour of |\bsc| by adding\\
%      |\renewcommand*{\bsc}[1]{\leavevmode\hbox{\scshape #1}}|\\
%      to the preamble of your document.
%    \item Footnotes are now displayed ``\`a la fran\c caise'' for the
%      whole document, except with an explicit\\
%      |\frenchbsetup{AutoSpaceFootnotes=false,FrenchFootnotes=false}|.\\
%      Add this command if you want standard footnotes. It is still
%      possible to revert locally to the standard layout of footnotes
%      by adding |\StandardFootnotes| (inside a |minipage| environment
%      for instance).
%    \end{itemize}
%
%  \subsubsection*{What's new in version 2.1?}
%
%      New command |\fup| to typeset better looking superscripts.
%      Former command |\up| is now defined as |\fup|, but an option
%      |\frenchbsetup{FrenchSuperscripts=false}| is provided for
%      backward compatibility.  |\fup| was designed using ideas from
%      Jacques Andr\'e, Thierry Bouche and Ren\'e Fritz, thanks to them!
%
%  \subsubsection*{What's new in version 2.2?}
%
%      Starting with version~2.2a, |frenchb| alters the layout of
%      lists, footnotes, and the indentation of first paragraphs of
%      sections) \emph{only if} French is the ``main language''
%      (i.e. babel's last language option). The layout is global for
%      the whole document: lists, etc. look the same in French and in
%      other languages, everything is typeset ``\`a la fran\c caise''
%      if French is the ``main language'', otherwise |frenchb| doesn't
%      change anything regarding lists, footnotes, and indentation of
%      paragraphs.
%
%  \subsubsection*{What's new in version 2.3?}
%
%      Starting with version~2.3a, |frenchb| no longer inserts spaces
%      automatically before `|:;!?|' when a typewriter font is in use;
%      this was suggested by Yannis Haralambous to prevent
%      spurious spaces in computer source code or expressions like
%      \texttt{C\string:/foo}, \texttt{http\string://foo.bar},
%      etc.  An option (|OriginalTypewriter|) is provided to get back
%      to the former behaviour of |frenchb|.
%
%      Another probably invisible change: lowercase conversion in
%      |\up{}| is now achieved by the \LaTeX{} command |\MakeLowercase|
%      instead of \TeX's |\lowercase| command.  This prevents error
%      messages when diacritics are used inside |\up{}| (diacritics
%      should \emph{never} be used in superscripts though!).
%
% \StopEventually{}
%
%  \subsection{File frenchb.cfg}
%  \label{sec-cfg}
%
%    \file{frenchb.cfg} is now a dummy file just kept for compatibility
%    with previous versions.
%
% \iffalse
%<*cfg>
% \fi
%    \begin{macrocode}
%%%%%%%%%%%%%%%%%%%%%%%%%%%%%%%%%%%%%%%%%%%%%%%%%%%%%%%%%%%%%%%%%%%%%%
%%%%%%%%%  WARNING: THIS  FILE SHOULD  NO  LONGER  BE  USED  %%%%%%%%%
%% If you want to customise frenchb, please DO NOT hack into the code!
%% Do no put any code in this file either, please use the new command
%% \frenchbsetup{} with the proper options to customise frenchb.
%% 
%% Add \frenchbsetup{ShowOptions} to your preamble to see the list of
%% available options and/or read the documentation.
%%%%%%%%%%%%%%%%%%%%%%%%%%%%%%%%%%%%%%%%%%%%%%%%%%%%%%%%%%%%%%%%%%%%%%
%    \end{macrocode}
% \iffalse
%</cfg>
% \fi
%
%  \section{\TeX{}nical details}
%
%  \subsection{Initial setup}
%
% \changes{v2.1d}{2008/05/02}{Argument of \cs{ProvidesLanguage} changed
%     above from `french' to `frenchb' (otherwise \cs{listfiles} prints
%     no date/version information).  The real name of current language
%     (french) as to be corrected before calling \cs{LdfInit}.}
%
% \iffalse
%<*code>
% \fi
%
%    While this file was read through the option \Lopt{frenchb} we make
%    it behave as if \Lopt{french} was specified.
%    \begin{macrocode}
\def\CurrentOption{french}
%    \end{macrocode}
%
%    The macro |\LdfInit| takes care of preventing that this file is
%    loaded more than once, checking the category code of the
%    \texttt{@} sign, etc.
%
%    \begin{macrocode}
\LdfInit\CurrentOption\datefrench
%    \end{macrocode}
%
% \changes{v2.1d}{2008/05/04}{Avoid warning ``\cs{end} occurred
%   when \cs{ifx} ... incomplete'' with LaTeX-2.09.}
%
%  \begin{macro}{\ifLaTeXe}
%    No support is provided for late \LaTeX-2.09: issue a warning
%    and exit if \LaTeX-2.09 is in use. Plain is still supported.
%    \begin{macrocode}
\newif\ifLaTeXe
\let\bbl@tempa\relax
\ifx\magnification\@undefined
   \ifx\@compatibilitytrue\@undefined
     \PackageError{frenchb.ldf}
        {LaTeX-2.09 format is no longer supported.\MessageBreak
         Aborting here}
        {Please upgrade to LaTeX2e!}
     \let\bbl@tempa\endinput
   \else
     \LaTeXetrue
   \fi
\fi
\bbl@tempa
%    \end{macrocode}
%  \end{macro}
%
%    Check if hyphenation patterns for the French language have been
%    loaded in language.dat; we allow for the names `french',
%    `francais', `canadien' or `acadian'. The latter two are both
%    names used in Canada for variants of French that are in use in
%    that country.
%
%    \begin{macrocode}
\ifx\l@french\@undefined
  \ifx\l@francais\@undefined
    \ifx\l@canadien\@undefined
      \ifx\l@acadian\@undefined
        \@nopatterns{French}
        \adddialect\l@french0
      \else
        \let\l@french\l@acadian
      \fi
    \else
      \let\l@french\l@canadien
    \fi
  \else
    \let\l@french\l@francais
  \fi
\fi
%    \end{macrocode}
%    Now |\l@french| is always defined.
%
%    The internal name for the French language is |french|;
%    |francais| and |frenchb| are synonymous for |french|:
%    first let both names use the same hyphenation patterns.
%    Later we will have to set aliases for |\captionsfrench|,
%    |\datefrench|, |\extrasfrench| and |\noextrasfrench|.
%    As French uses the standard values of |\lefthyphenmin| (2)
%    and |\righthyphenmin| (3), no special setting is required here.
%
%    \begin{macrocode}
\ifx\l@francais\@undefined
  \let\l@francais\l@french
\fi
\ifx\l@frenchb\@undefined
  \let\l@frenchb\l@french
\fi
%    \end{macrocode}
%    When this language definition file was loaded for one of the
%    Canadian versions of French we need to make sure that a suitable
%    hyphenation pattern register will be found by \TeX.
%    \begin{macrocode}
\ifx\l@canadien\@undefined
  \let\l@canadien\l@french
\fi
\ifx\l@acadian\@undefined
  \let\l@acadian\l@french
\fi
%    \end{macrocode}
%
%    This language definition can be loaded for different variants of
%    the French language. The `key' \babel\ macros are only defined
%    once, using `french' as the language name, but |frenchb| and
%    |francais| are synonymous.
%    \begin{macrocode}
\def\datefrancais{\datefrench}
\def\datefrenchb{\datefrench}
\def\extrasfrancais{\extrasfrench}
\def\extrasfrenchb{\extrasfrench}
\def\noextrasfrancais{\noextrasfrench}
\def\noextrasfrenchb{\noextrasfrench}
%    \end{macrocode}
%
% \begin{macro}{\extrasfrench}
% \begin{macro}{\noextrasfrench}
%    The macro |\extrasfrench| will perform all the extra
%    definitions needed for the French language.
%    The macro |\noextrasfrench| is used to cancel the actions of
%    |\extrasfrench|.\\
%    In French, character ``apostrophe'' is a letter in expressions
%    like |l'ambulance| (French  hyphenation patterns provide entries
%    for this kind of words).  This means that the |\lccode| of
%    ``apostrophe'' has to be non null in French for proper hyphenation
%    of those expressions, and has to be reset to null when exiting
%    French.
%
%    \begin{macrocode}
\@namedef{extras\CurrentOption}{\lccode`\'=`\'}
\@namedef{noextras\CurrentOption}{\lccode`\'=0}
%    \end{macrocode}
% \end{macro}
% \end{macro}
%
%    One more thing |\extrasfrench| needs to do is to make sure that
%    |\frenchspacing| is in effect.  |\noextrasfrench| will switch
%    |\frenchspacing| off again.
%    \begin{macrocode}
  \expandafter\addto\csname extras\CurrentOption\endcsname{%
    \bbl@frenchspacing}
  \expandafter\addto\csname noextras\CurrentOption\endcsname{%
    \bbl@nonfrenchspacing}
%    \end{macrocode}
%
%  \subsection{Punctuation}
%  \label{sec-punct}
%
%    As long as no better solution is available%
%    \footnote{Lua\TeX, or pdf\TeX{} might provide alternatives in
%       the future\dots},
%    the `double punctuation' characters (|;| |!| |?| and |:|) have to
%    be made |\active| for an automatic control of the amount of space
%    to insert before them. Before doing so, we have to save the
%    standard definition of |\@makecaption| (which includes two ':')
%    to compare it later to its definition at the |\begin{document}|.
%    \begin{macrocode}
\long\def\STD@makecaption#1#2{%
  \vskip\abovecaptionskip
  \sbox\@tempboxa{#1: #2}%
  \ifdim \wd\@tempboxa >\hsize
    #1: #2\par
  \else
    \global \@minipagefalse
    \hb@xt@\hsize{\hfil\box\@tempboxa\hfil}%
  \fi
  \vskip\belowcaptionskip}%
%    \end{macrocode}
%
%    We define a new `if' |\FBpunct@active| which will be made false
%    whenever a better alternative will be available. Another `if'
%    |\FBAutoSpacePunctuation| needs to be defined now.
%    \begin{macrocode}
\newif\ifFBpunct@active          \FBpunct@activetrue
\newif\ifFBAutoSpacePunctuation  \FBAutoSpacePunctuationtrue
%    \end{macrocode}
%    The following code makes the four characters |;| |!| |?| and |:|
%    `active' and provides their definitions.
%    \begin{macrocode}
\ifFBpunct@active
  \initiate@active@char{:}
  \initiate@active@char{;}
  \initiate@active@char{!}
  \initiate@active@char{?}
%    \end{macrocode}
%    We first tune the amount of space before \texttt{;}
%    \texttt{!}  \texttt{?} and \texttt{:}.  This should only happen
%    in horizontal mode, hence the test |\ifhmode|.
%
%    In horizontal mode, if a space has been typed before `;' we
%    remove it and put an unbreakable |\thinspace| instead. If no
%    space has been typed, we add |\FDP@thinspace| which will be
%    defined, up to the user's wishes, as an automatic added
%    thin space, or as |\@empty|.
%    \begin{macrocode}
  \declare@shorthand{french}{;}{%
      \ifhmode
      \ifdim\lastskip>\z@
          \unskip\penalty\@M\thinspace
          \else
            \FDP@thinspace
        \fi
      \fi
%    \end{macrocode}
%    Now we can insert a |;| character.
%    \begin{macrocode}
      \string;}
%    \end{macrocode}
%    The next three definitions are very similar.
%    \begin{macrocode}
  \declare@shorthand{french}{!}{%
      \ifhmode
        \ifdim\lastskip>\z@
          \unskip\penalty\@M\thinspace
        \else
          \FDP@thinspace
        \fi
      \fi
      \string!}
  \declare@shorthand{french}{?}{%
      \ifhmode
        \ifdim\lastskip>\z@
          \unskip\penalty\@M\thinspace
        \else
          \FDP@thinspace
        \fi
      \fi
      \string?}
%    \end{macrocode}
%    According to the I.N. specifications, the `:' requires a normal
%    space before it, but some people prefer a |\thinspace| (just
%    like the other three). We define |\Fcolonspace| to hold the
%    required amount of space (user customisable).
%    \begin{macrocode}
  \newcommand*{\Fcolonspace}{\space}
  \declare@shorthand{french}{:}{%
      \ifhmode
        \ifdim\lastskip>\z@
          \unskip\penalty\@M\Fcolonspace
        \else
          \FDP@colonspace
        \fi
      \fi
      \string:}
%    \end{macrocode}
%
% \changes{v2.3a}{2008/10/10}{\cs{NoAutoSpaceBeforeFDP} and
%    \cs{AutoSpaceBeforeFDP} now set the flag
%    \cs{ifFBAutoSpacePunctuation} accordingly (LaTeX only).}
%
%  \begin{macro}{\AutoSpaceBeforeFDP}
%  \begin{macro}{\NoAutoSpaceBeforeFDP}
%    |\FDP@thinspace| and |\FDP@space| are defined as unbreakable
%    spaces by |\autospace@beforeFDP| or as |\@empty| by
%    |\noautospace@beforeFDP| (internal commands), user commands
%    |\AutoSpaceBeforeFDP| and |\NoAutoSpaceBeforeFDP| do the same and
%    take care of the flag |\ifFBAutoSpacePunctuation| in \LaTeX{}.
%    Set the default now for Plain (done later for \LaTeX).
%    \begin{macrocode}
  \def\autospace@beforeFDP{%
          \def\FDP@thinspace{\penalty\@M\thinspace}%
          \def\FDP@colonspace{\penalty\@M\Fcolonspace}}
  \def\noautospace@beforeFDP{\let\FDP@thinspace\@empty
                            \let\FDP@colonspace\@empty}
  \ifLaTeXe
    \def\AutoSpaceBeforeFDP{\autospace@beforeFDP
                            \FBAutoSpacePunctuationtrue}
    \def\NoAutoSpaceBeforeFDP{\noautospace@beforeFDP
                              \FBAutoSpacePunctuationfalse}
  \else
    \let\AutoSpaceBeforeFDP\autospace@beforeFDP
    \let\NoAutoSpaceBeforeFDP\noautospace@beforeFDP
    \AutoSpaceBeforeFDP
  \fi
%    \end{macrocode}
% \end{macro}
% \end{macro}
%
% \changes{v2.3a}{2008/10/10}{In LaTeX, frenchb no longer adds spaces
%     before `double punctuation' characters in computer code.
%     Suggested by Yannis Haralambous.}
%
% \changes{v2.3c}{2009/02/07}{Commands \cs{ttfamily}, \cs{rmfamily}
%    and \cs{sffamily} have to be robust.  Bug introduced in 2.3a,
%    pointed out by Manuel P\'egouri\'e-Gonnard.}
%
%    In \LaTeXe{} |\ttfamily| (and hence |\texttt|) will be redefined
%    `AtBeginDocument' as |\ttfamilyFB| so that no space
%    is added before the four |; : ! ?| characters, even if
%    |AutoSpacePunctuation| is true.  |\rmfamily| and |\sffamily| need
%    to be redefined also (|\ttfamily| is not always used inside a
%    group, its effect can be cancelled by |\rmfamily| or |\sffamily|).
%
%    These redefinitions can be canceled if necessary, for instance to
%    recompile older documents, see option |OriginalTypewriter| below.
%    \begin{macrocode}
  \ifLaTeXe
    \let\ttfamilyORI\ttfamily
    \let\rmfamilyORI\rmfamily
    \let\sffamilyORI\sffamily
    \DeclareRobustCommand\ttfamilyFB{%
         \noautospace@beforeFDP\ttfamilyORI}%
    \DeclareRobustCommand\rmfamilyFB{%
         \ifFBAutoSpacePunctuation
            \autospace@beforeFDP\rmfamilyORI
         \else
            \noautospace@beforeFDP\rmfamilyORI
         \fi}%
    \DeclareRobustCommand\sffamilyFB{%
         \ifFBAutoSpacePunctuation
            \autospace@beforeFDP\sffamilyORI
         \else
            \noautospace@beforeFDP\sffamilyORI
         \fi}%
  \fi
%    \end{macrocode}
%
%    When the active characters appear in an environment where their
%    French behaviour is not wanted they should give an `expected'
%    result. Therefore we define shorthands at system level as well.
%    \begin{macrocode}
  \declare@shorthand{system}{:}{\string:}
  \declare@shorthand{system}{!}{\string!}
  \declare@shorthand{system}{?}{\string?}
  \declare@shorthand{system}{;}{\string;}
%    \end{macrocode}
%    We specify that the French group of shorthands should be used.
%    \begin{macrocode}
  \addto\extrasfrench{%
    \languageshorthands{french}%
%    \end{macrocode}
%    These characters are `turned on' once, later their definition may
%    vary. Don't misunderstand the following code: they keep being
%    active all along the document, even when leaving French.
%    \begin{macrocode}
    \bbl@activate{:}\bbl@activate{;}%
    \bbl@activate{!}\bbl@activate{?}%
  }
  \addto\noextrasfrench{%
  \bbl@deactivate{:}\bbl@deactivate{;}%
  \bbl@deactivate{!}\bbl@deactivate{?}}
\fi
%    \end{macrocode}
%
%  \subsection{French quotation marks}
%
%  \begin{macro}{\og}
%  \begin{macro}{\fg}
%    The top macros for quotation marks will be called |\og|
%    (``\underline{o}uvrez \underline{g}uillemets'') and |\fg|
%    (``\underline{f}ermez \underline{g}uillemets'').
%    Another option for typesetting quotes in multilingual texts
%    is to use the package |csquotes.sty| and its command |\enquote|.
%
%    \begin{macrocode}
\newcommand*{\og}{\@empty}
\newcommand*{\fg}{\@empty}
%    \end{macrocode}
%  \end{macro}
%  \end{macro}
%
%  \begin{macro}{\guillemotleft}
%  \begin{macro}{\guillemotright}
%    \LaTeX{} users are supposed to use 8-bit output encodings (T1,
%    LY1,\dots) to typeset French, those who still stick to OT1 should
%    call |aeguill.sty| or a similar package. In both cases the
%    commands |\guillemotleft| and |\guillemotright| will print the
%    French opening and closing quote characters from the output font.
%    For XeLaTeX, |\guillemotleft| and |\guillemotright| are defined
%    by package \file{xunicode.sty}.
%    We will check `AtBeginDocument' that the proper output encodings
%    are in use (see end of section~\ref{sec-keyval}).
%
%    We give the following definitions for Plain users only as a (poor)
%    fall-back, they are welcome to change them for anything better.
%    \begin{macrocode}
\ifLaTeXe
\else
  \ifx\guillemotleft\@undefined
    \def\guillemotleft{\leavevmode\raise0.25ex
                       \hbox{$\scriptscriptstyle\ll$}}
  \fi
  \ifx\guillemotright\@undefined
    \def\guillemotright{\raise0.25ex
                        \hbox{$\scriptscriptstyle\gg$}}
  \fi
  \let\xspace\relax
\fi
%    \end{macrocode}
%  \end{macro}
%  \end{macro}
%
%    The next step is to provide correct spacing after |\guillemotleft|
%    and before |\guillemotright|: a space precedes and follows
%    quotation marks but no line break is allowed neither \emph{after}
%    the opening one, nor \emph{before} the closing one.
%    |\FBguill@spacing| which does the spacing, has been fine tuned by
%    Thierry Bouche.  French quotes (including spacing) are printed by
%    |\FB@og| and |\FB@fg|, the expansion of the top level commands
%    |\og| and |\og| is different in and outside French.
%    We'll try to be smart to users of David~Carlisle's |xspace|
%    package: if this package is loaded there will be no need for |{}|
%    or |\ | to get a space after |\fg|, otherwise |\xspace| will be
%    defined as |\relax| (done at the end of this file).
%
%    \begin{macrocode}
\newcommand*{\FBguill@spacing}{\penalty\@M\hskip.8\fontdimen2\font
                                            plus.3\fontdimen3\font
                                           minus.8\fontdimen4\font}
\DeclareRobustCommand*{\FB@og}{\leavevmode
                               \guillemotleft\FBguill@spacing}
\DeclareRobustCommand*{\FB@fg}{\ifdim\lastskip>\z@\unskip\fi
                               \FBguill@spacing\guillemotright\xspace}
%    \end{macrocode}
%
%    The top level definitions for French quotation marks are switched
%    on and off through the |\extrasfrench| |\noextrasfrench|
%    mechanism. Outside French, |\og| and |\fg| will typeset standard
%    English opening and closing double quotes.
%
%    \begin{macrocode}
\ifLaTeXe
  \def\bbl@frenchguillemets{\renewcommand*{\og}{\FB@og}%
                            \renewcommand*{\fg}{\FB@fg}}
  \def\bbl@nonfrenchguillemets{\renewcommand*{\og}{\textquotedblleft}%
            \renewcommand*{\fg}{\ifdim\lastskip>\z@\unskip\fi
                                   \textquotedblright}}
\else
   \def\bbl@frenchguillemets{\let\og\FB@og
                             \let\fg\FB@fg}
   \def\bbl@nonfrenchguillemets{\def\og{``}%
                     \def\fg{\ifdim\lastskip>\z@\unskip\fi ''}}
\fi
\expandafter\addto\csname extras\CurrentOption\endcsname{%
  \bbl@frenchguillemets}
\expandafter\addto\csname noextras\CurrentOption\endcsname{%
  \bbl@nonfrenchguillemets}
%    \end{macrocode}
%
%  \subsection{Date in French}
%
% \begin{macro}{\datefrench}
%    The macro |\datefrench| redefines the command |\today| to
%    produce French dates.
%
% \changes{v2.0}{2006/11/06}{2 '\cs{relax}' added in
%    \cs{today}'s definition.}
%
% \changes{v2.1a}{2008/03/25}{\cs{today} changed (correction in 2.0
%    was wrong: \cs{today} was printed without spaces in toc).}
%
%    \begin{macrocode}
\@namedef{date\CurrentOption}{%
  \def\today{{\number\day}\ifnum1=\day {\ier}\fi \space
    \ifcase\month
      \or janvier\or f\'evrier\or mars\or avril\or mai\or juin\or
      juillet\or ao\^ut\or septembre\or octobre\or novembre\or
      d\'ecembre\fi
    \space \number\year}}
%    \end{macrocode}
% \end{macro}
%
%  \subsection{Extra utilities}
%
%    Let's provide the French user with some extra utilities.
%
% \changes{v2.1a}{2008/03/24}{Command \cs{fup} added to produce
%    better superscripts than \cs{textsuperscript}.}
%
%  \begin{macro}{\up}
%
% \changes{v2.1c}{2008/04/29}{Provide a temporary definition
%    (hyperref safe) of \cs{up} in case it has to be expanded in
%    the preamble (by beamer's \cs{title} command for instance).}
%
%  \begin{macro}{\fup}
%
% \changes{v2.1b}{2008/04/02}{Command \cs{fup} changed to use
%    real superscripts from fourier v. 1.6.}
%
% \changes{v2.2a}{2008/05/08}{\cs{newif} and \cs{newdimen} moved
%    before \cs{ifLaTeXe} to avoid an error with plainTeX.}
%
% \changes{v2.3a}{2008/09/30}{\cs{lowercase} changed to
%    \cs{MakeLowercase} as the former doesn't work for non ASCII
%    characters in encodings like applemac, utf-8,\dots}
%
%    |\up| eases the typesetting of superscripts like
%    `1\textsuperscript{er}'.  Up to version 2.0 of |frenchb| |\up| was
%    just a shortcut for |\textsuperscript| in \LaTeXe, but several
%    users complained that |\textsuperscript| typesets superscripts
%    too high and too big, so we now define |\fup| as an attempt to
%    produce better looking superscripts.  |\up| is defined as |\fup|
%    but can be redefined by |\frenchbsetup{FrenchSuperscripts=false}|
%    as |\textsuperscript| for compatibility with previous versions.
%
%    When a font has built-in superscripts, the best thing to do is
%    to just use them, otherwise |\fup| has to simulate superscripts
%    by scaling and raising ordinary letters.  Scaling is done using
%    package \file{scalefnt} which will be loaded at the end of
%    \babel's loading (|frenchb| being an option of babel, it cannot
%    load a package while being read).
%
%    \begin{macrocode}
\newif\ifFB@poorman
\newdimen\FB@Mht
\ifLaTeXe
  \AtEndOfPackage{\RequirePackage{scalefnt}}
%    \end{macrocode}
%    |\FB@up@fake| holds the definition of fake superscripts.
%    The scaling ratio is 0.65, raising is computed to put the top of
%    lower case letters (like `m') just under the top  of upper case
%    letters (like `M'), precisely 12\% down.  The chosen settings
%    look correct for most fonts, but can be tuned by the end-user
%    if necessary by changing |\FBsupR| and |\FBsupS| commands.
%
%    |\FB@lc| is defined as |\MakeLowercase| to inhibit the uppercasing
%    of superscripts (this may happen in page headers with the standard
%    classes but is wrong); |\FB@lc| can be redefined to do nothing
%    by option |LowercaseSuperscripts=false| of |\frenchbsetup{}|.
%    \begin{macrocode}
  \newcommand*{\FBsupR}{-0.12}
  \newcommand*{\FBsupS}{0.65}
  \newcommand*{\FB@lc}[1]{\MakeLowercase{#1}}
  \DeclareRobustCommand*{\FB@up@fake}[1]{%
    \settoheight{\FB@Mht}{M}%
    \addtolength{\FB@Mht}{\FBsupR \FB@Mht}%
    \addtolength{\FB@Mht}{-\FBsupS ex}%
    \raisebox{\FB@Mht}{\scalefont{\FBsupS}{\FB@lc{#1}}}%
    }
%    \end{macrocode}
%    The only packages I currently know to take advantage of real
%    superscripts are a) \file{xltxtra} used in conjunction with
%    XeLaTeX and OpenType fonts having the font feature
%    'VerticalPosition=Superior' (\file{xltxtra} defines
%    |\realsuperscript| and |\fakesuperscript|) and b) \file{fourier}
%    (from version 1.6) when Expert Utopia fonts are available.
%
%    |\FB@up| checks whether the current font is a Type1 `Expert'
%    (or `Pro') font with real superscripts or not (the code works
%    currently only with \file{fourier-1.6} but could work with any
%    Expert Type1 font with built-in superscripts, see below), and
%    decides to use real or fake superscripts.
%    It works as follows: the content of |\f@family| (family name of
%    the current font) is split by |\FB@split| into two pieces, the
%    first three characters (`|fut|' for Fourier, `|ppl|' for Adobe's
%    Palatino, \dots) stored in |\FB@firstthree| and the rest stored
%    in |\FB@suffix| which is expected to be `|x|' or `|j|' for expert
%    fonts.
%    \begin{macrocode}
  \def\FB@split#1#2#3#4\@nil{\def\FB@firstthree{#1#2#3}%
                             \def\FB@suffix{#4}}
  \def\FB@x{x}
  \def\FB@j{j}
  \DeclareRobustCommand*{\FB@up}[1]{%
    \bgroup \FB@poormantrue
      \expandafter\FB@split\f@family\@nil
%    \end{macrocode}
%    Then |\FB@up| looks for a \file{.fd} file named \file{t1fut-sup.fd}
%    (Fourier) or \file{t1ppl-sup.fd} (Palatino), etc. supposed to
%    define the subfamily (|fut-sup| or |ppl-sup|, etc.) giving access
%    to the built-in superscripts.  If the \file{.fd} file is not found
%    by |\IfFileExists|, |\FB@up| falls back on fake superscripts,
%    otherwise |\FB@suffix| is checked to decide whether to use fake or
%    real superscripts.
%    \begin{macrocode}
      \edef\reserved@a{\lowercase{%
         \noexpand\IfFileExists{\f@encoding\FB@firstthree -sup.fd}}}%
      \reserved@a
        {\ifx\FB@suffix\FB@x \FB@poormanfalse\fi
         \ifx\FB@suffix\FB@j \FB@poormanfalse\fi
         \ifFB@poorman \FB@up@fake{#1}%
         \else         \FB@up@real{#1}%
         \fi}%
        {\FB@up@fake{#1}}%
    \egroup}
%    \end{macrocode}
%    |\FB@up@real| just picks up the superscripts from the subfamily
%    (and forces lowercase).
%    \begin{macrocode}
  \newcommand*{\FB@up@real}[1]{\bgroup
       \fontfamily{\FB@firstthree -sup}\selectfont \FB@lc{#1}\egroup}
%    \end{macrocode}
%    |\fup| is now defined as |\FB@up| unless |\realsuperscript| is
%    defined (occurs with XeLaTeX calling \file{xltxtra.sty}).
%    \begin{macrocode}
  \DeclareRobustCommand*{\fup}[1]{%
    \@ifundefined{realsuperscript}%
      {\FB@up{#1}}%
      {\bgroup\let\fakesuperscript\FB@up@fake
            \realsuperscript{\FB@lc{#1}}\egroup}}
%    \end{macrocode}
%    Temporary definition of |up| (redefined `AtBeginDocument').
%    \begin{macrocode}
  \newcommand*{\up}{\relax}
%    \end{macrocode}
%    Poor man's definition of |\up| for Plain. In \LaTeXe,
%    |\up| will be defined as |\fup| or |\textsuperscript| later on
%    while processing the options of |\frenchbsetup{}|.
%    \begin{macrocode}
\else
  \newcommand*{\up}[1]{\leavevmode\raise1ex\hbox{\sevenrm #1}}
\fi
%    \end{macrocode}
%  \end{macro}
%  \end{macro}
%
%  \begin{macro}{\ieme}
%  \begin{macro}{\ier}
%  \begin{macro}{\iere}
%  \begin{macro}{\iemes}
%  \begin{macro}{\iers}
%  \begin{macro}{\ieres}
%  Some handy macros for those who don't know how to abbreviate ordinals:
%    \begin{macrocode}
\def\ieme{\up{\lowercase{e}}\xspace}
\def\iemes{\up{\lowercase{es}}\xspace}
\def\ier{\up{\lowercase{er}}\xspace}
\def\iers{\up{\lowercase{ers}}\xspace}
\def\iere{\up{\lowercase{re}}\xspace}
\def\ieres{\up{\lowercase{res}}\xspace}
%    \end{macrocode}
%  \end{macro}
%  \end{macro}
%  \end{macro}
%  \end{macro}
%  \end{macro}
%  \end{macro}
%
% \changes{v2.1c}{2008/04/29}{Added commands \cs{Nos} and \cs{nos}.}
%
%  \begin{macro}{\No}
%  \begin{macro}{\no}
%  \begin{macro}{\Nos}
%  \begin{macro}{\nos}
%  \begin{macro}{\primo}
%  \begin{macro}{\fprimo)}
%    And some more macros relying on |\up| for numbering,
%    first two support macros.
%    \begin{macrocode}
\newcommand*{\FrenchEnumerate}[1]{%
                       #1\up{\lowercase{o}}\kern+.3em}
\newcommand*{\FrenchPopularEnumerate}[1]{%
                       #1\up{\lowercase{o}})\kern+.3em}
%    \end{macrocode}
%
%    Typing |\primo| should result in `$1^{\rm o}$\kern+.3em',
%    \begin{macrocode}
\def\primo{\FrenchEnumerate1}
\def\secundo{\FrenchEnumerate2}
\def\tertio{\FrenchEnumerate3}
\def\quarto{\FrenchEnumerate4}
%    \end{macrocode}
%    while typing |\fprimo)| gives `1$^{\rm o}$)\kern+.3em.
%    \begin{macrocode}
\def\fprimo){\FrenchPopularEnumerate1}
\def\fsecundo){\FrenchPopularEnumerate2}
\def\ftertio){\FrenchPopularEnumerate3}
\def\fquarto){\FrenchPopularEnumerate4}
%    \end{macrocode}
%
%    Let's provide four macros for the common abbreviations
%    of ``Num\'ero''.
%    \begin{macrocode}
\DeclareRobustCommand*{\No}{N\up{\lowercase{o}}\kern+.2em}
\DeclareRobustCommand*{\no}{n\up{\lowercase{o}}\kern+.2em}
\DeclareRobustCommand*{\Nos}{N\up{\lowercase{os}}\kern+.2em}
\DeclareRobustCommand*{\nos}{n\up{\lowercase{os}}\kern+.2em}
%    \end{macrocode}
%  \end{macro}
%  \end{macro}
%  \end{macro}
%  \end{macro}
%  \end{macro}
%  \end{macro}
%
%  \begin{macro}{\bsc}
%    As family names should be written in small capitals and never be
%    hyphenated, we provide a command (its name comes from Boxed Small
%    Caps) to input them easily.  Note that this command has changed
%    with version~2 of |frenchb|: a |\kern0pt| is used instead of |\hbox|
%    because |\hbox| would break microtype's font expansion; as a
%    (positive?) side effect, composed names (such as Dupont-Durand)
%    can now be hyphenated on explicit hyphens.
%    Usage: |Jean~\bsc{Duchemin}|.
%
% \changes{v2.0}{2006/11/06}{\cs{hbox} dropped, replaced by
%    \cs{kern0pt}.}
%
%    \begin{macrocode}
\DeclareRobustCommand*{\bsc}[1]{\leavevmode\begingroup\kern0pt
                                           \scshape #1\endgroup}
\ifLaTeXe\else\let\scshape\relax\fi
%    \end{macrocode}
%  \end{macro}
%
%    Some definitions for special characters.  We won't define |\tilde|
%    as a Text Symbol not to conflict with the macro |\tilde| for math
%    mode and use the name |\tild| instead. Note that |\boi| may
%    \emph{not} be used in math mode, its name in math mode is
%    |\backslash|.  |\degre|  can be accessed by the command |\r{}|
%    for ring accent.
%
%    \begin{macrocode}
\ifLaTeXe
  \DeclareTextSymbol{\at}{T1}{64}
  \DeclareTextSymbol{\circonflexe}{T1}{94}
  \DeclareTextSymbol{\tild}{T1}{126}
  \DeclareTextSymbolDefault{\at}{T1}
  \DeclareTextSymbolDefault{\circonflexe}{T1}
  \DeclareTextSymbolDefault{\tild}{T1}
  \DeclareRobustCommand*{\boi}{\textbackslash}
  \DeclareRobustCommand*{\degre}{\r{}}
\else
  \def\T@one{T1}
  \ifx\f@encoding\T@one
    \newcommand*{\degre}{\char6}
  \else
    \newcommand*{\degre}{\char23}
  \fi
  \newcommand*{\at}{\char64}
  \newcommand*{\circonflexe}{\char94}
  \newcommand*{\tild}{\char126}
  \newcommand*{\boi}{$\backslash$}
\fi
%    \end{macrocode}
%
%  \begin{macro}{\degres}
%    We now define a macro |\degres| for typesetting the abbreviation
%    for `degrees' (as in `degrees Celsius'). As the bounding box of
%    the character `degree' has \emph{very} different widths in CM/EC
%    and PostScript fonts, we fix the width of the bounding box of
%    |\degres| to 0.3\,em, this lets the symbol `degree' stick to the
%    preceding (e.g., |45\degres|) or following character
%    (e.g., |20~\degres C|).
%
%    If the \TeX{} Companion fonts are available (\file{textcomp.sty}),
%    we pick up |\textdegree| from them instead of using emulating
%    `degrees' from the |\r{}| accent. Otherwise we overwrite the
%    (poor) definition of |\textdegree| given in \file{latin1.def},
%    \file{applemac.def} etc. (called by  \file{inputenc.sty}) by
%    our definition of |\degres|. We also advice the user (once only)
%    to use TS1-encoding.
%
% \changes{v2.1c}{2008/04/29}{Provide a temporary definition (hyperref
%    safe) of \cs{degres} in case it has to be expanded in the preamble
%    (by beamer's \cs{title} command for instance).}
%
%    \begin{macrocode}
\ifLaTeXe
  \newcommand*{\degres}{\degre}
  \def\Warning@degree@TSone{%
        \PackageWarning{frenchb.ldf}{%
           Degrees would look better in TS1-encoding:
           \MessageBreak add \protect
           \usepackage{textcomp} to the preamble.
           \MessageBreak Degrees used}}
  \AtBeginDocument{\expandafter\ifx\csname M@TS1\endcsname\relax
                     \DeclareRobustCommand*{\degres}{%
                       \leavevmode\hbox to 0.3em{\hss\degre\hss}%
                       \Warning@degree@TSone
                       \global\let\Warning@degree@TSone\relax}%
                      \let\textdegree\degres
                   \else
                     \DeclareRobustCommand*{\degres}{%
                         \hbox{\UseTextSymbol{TS1}{\textdegree}}}%
                   \fi}
\else
  \newcommand*{\degres}{%
    \leavevmode\hbox to 0.3em{\hss\degre\hss}}
\fi
%    \end{macrocode}
%  \end{macro}
%
%  \subsection{Formatting numbers}
%  \label{sec-numbers}
%
%  \begin{macro}{\DecimalMathComma}
%  \begin{macro}{\StandardMathComma}
%    As mentioned in the \TeX{}book p.~134, the comma is of type
%    |\mathpunct| in math mode: it is automatically followed by a
%    space. This is convenient in lists and intervals but
%    unpleasant when the comma is used as a decimal separator
%    in French: it has to be entered as |{,}|.
%    |\DecimalMathComma| makes the comma be an ordinary character
%    (of type |\mathord|) in French \emph{only} (no space added);
%    |\StandardMathComma| switches back to the standard behaviour
%    of the comma.
%    \begin{macrocode}
\newcount\std@mcc
\newcount\dec@mcc
\std@mcc=\mathcode`\,
\dec@mcc=\std@mcc
\@tempcnta=\std@mcc
\divide\@tempcnta by "1000
\multiply\@tempcnta by "1000
\advance\dec@mcc by -\@tempcnta
\newcommand*{\DecimalMathComma}{\iflanguage{french}%
                                 {\mathcode`\,=\dec@mcc}{}%
              \addto\extrasfrench{\mathcode`\,=\dec@mcc}}
\newcommand*{\StandardMathComma}{\mathcode`\,=\std@mcc
             \addto\extrasfrench{\mathcode`\,=\std@mcc}}
\expandafter\addto\csname noextras\CurrentOption\endcsname{%
   \mathcode`\,=\std@mcc}
%    \end{macrocode}
%  \end{macro}
%  \end{macro}
%
%  \begin{macro}{\nombre}
%
% \changes{v2.0}{2006/11/06}{\cs{nombre} requires now numprint.sty.}
%
%    The command |\nombre| is now borrowed from |numprint.sty| for
%    \LaTeXe.  There is no point to maintain the former tricky code
%    when a package is dedicated to do the same job and more.
%    For Plain based formats, |\nombre| no longer formats numbers,
%    it prints them as is and issues a warning about the change.
%
%    Fake command |\nombre| for Plain based formats, warning users of
%    |frenchb| v.1.x. of the change.
%    \begin{macrocode}
\newcommand*{\nombre}[1]{{#1}\message{%
     *** \noexpand\nombre no longer formats numbers\string! ***}}%
%    \end{macrocode}
%  \end{macro}
%
%    The next definitions only make sense for \LaTeXe. Let's cleanup
%    and exit if the format in Plain based.
%
%    \begin{macrocode}
\let\FBstop@here\relax
\def\FBclean@on@exit{\let\ifLaTeXe\@undefined
                     \let\LaTeXetrue\@undefined
                     \let\LaTeXefalse\@undefined}
\ifx\magnification\@undefined
\else
   \def\FBstop@here{\let\STD@makecaption\relax
                    \FBclean@on@exit
                    \ldf@quit\CurrentOption\endinput}
\fi
\FBstop@here
%    \end{macrocode}
%
%    What follows now is for \LaTeXe{} \emph{only}.
%    We redefine |\nombre| for \LaTeXe. A warning is issued
%    at the first call of |\nombre| if |\numprint| is not
%    defined, suggesting what to do.  The package |numprint|
%    is \emph{not} loaded automatically by |frenchb| because of
%    possible options conflict.
%
%    \begin{macrocode}
\renewcommand*{\nombre}[1]{\Warning@nombre\numprint{#1}}
\newcommand*{\Warning@nombre}{%
   \@ifundefined{numprint}%
      {\PackageWarning{frenchb.ldf}{%
         \protect\nombre\space now relies on package numprint.sty,
         \MessageBreak add \protect
         \usepackage[autolanguage]{numprint}\MessageBreak
         to your preamble *after* loading babel, \MessageBreak
         see file numprint.pdf for other options.\MessageBreak
         \protect\nombre\space called}%
       \global\let\Warning@nombre\relax
       \global\let\numprint\relax
      }{}%
}
%    \end{macrocode}
%
% \changes{v2.0c}{2007/06/25}{There is no need to define here
%    numprint's command \cs{npstylefrench}, it will be redefined
%    `AtBeginDocument' by \cs{FBprocess@options}.}
%
% \changes{v2.0c}{2007/06/25}{\cs{ThinSpaceInFrenchNumbers} added
%     for compatibility with frenchb-1.x.}
%
%    \begin{macrocode}
\newcommand*{\ThinSpaceInFrenchNumbers}{%
   \PackageWarning{frenchb.ldf}{%
         Type \protect\frenchbsetup{ThinSpaceInFrenchNumbers}
         \MessageBreak Command \protect\ThinSpaceInFrenchNumbers\space
         is no longer\MessageBreak  defined in frenchb v.2,}}
%    \end{macrocode}
%
%  \subsection{Caption names}
%
%    The next step consists of defining the French equivalents for
%    the \LaTeX{} caption names.
%
% \begin{macro}{\captionsfrench}
%    Let's first define  |\captionsfrench| which sets all strings used
%    in the four standard document classes provided with \LaTeX.
%
% \changes{v2.0}{2006/11/06}{`Fig.' changed to `Figure' and
%     `Tab.' to `Table'.}
%
% \changes{v2.0}{2006/12/15}{Set \cs{CaptionSeparator} in
%     \cs{extrasfrench} now instead of \cs{captionsfrench}
%     because it has to be reset when leaving French.}
%
%    \begin{macrocode}
\@namedef{captions\CurrentOption}{%
   \def\refname{R\'ef\'erences}%
   \def\abstractname{R\'esum\'e}%
   \def\bibname{Bibliographie}%
   \def\prefacename{Pr\'eface}%
   \def\chaptername{Chapitre}%
   \def\appendixname{Annexe}%
   \def\contentsname{Table des mati\`eres}%
   \def\listfigurename{Table des figures}%
   \def\listtablename{Liste des tableaux}%
   \def\indexname{Index}%
   \def\figurename{{\scshape Figure}}%
   \def\tablename{{\scshape Table}}%
%    \end{macrocode}
%   ``Premi\`ere partie'' instead of ``Part I''.
%    \begin{macrocode}
   \def\partname{\protect\@Fpt partie}%
   \def\@Fpt{{\ifcase\value{part}\or Premi\`ere\or Deuxi\`eme\or
   Troisi\`eme\or Quatri\`eme\or Cinqui\`eme\or Sixi\`eme\or
   Septi\`eme\or Huiti\`eme\or Neuvi\`eme\or Dixi\`eme\or Onzi\`eme\or
   Douzi\`eme\or Treizi\`eme\or Quatorzi\`eme\or Quinzi\`eme\or
   Seizi\`eme\or Dix-septi\`eme\or Dix-huiti\`eme\or Dix-neuvi\`eme\or
   Vingti\`eme\fi}\space\def\thepart{}}%
   \def\pagename{page}%
   \def\seename{{\emph{voir}}}%
   \def\alsoname{{\emph{voir aussi}}}%
   \def\enclname{P.~J. }%
   \def\ccname{Copie \`a }%
   \def\headtoname{}%
   \def\proofname{D\'emonstration}%
   \def\glossaryname{Glossaire}%
   }
%    \end{macrocode}
% \end{macro}
%
%    As some users who choose |frenchb| or |francais| as option of
%    \babel, might customise |\captionsfrenchb| or |\captionsfrancais|
%    in the preamble, we merge their changes at the |\begin{document}|
%    when they do so.
%    The other variants of French (canadien, acadian) are defined by
%    checking if the relevant option was used and then adding one extra
%    level of expansion.
%
%    \begin{macrocode}
\AtBeginDocument{\let\captions@French\captionsfrench
                 \@ifundefined{captionsfrenchb}%
                    {\let\captions@Frenchb\relax}%
                    {\let\captions@Frenchb\captionsfrenchb}%
                 \@ifundefined{captionsfrancais}%
                    {\let\captions@Francais\relax}%
                    {\let\captions@Francais\captionsfrancais}%
                 \def\captionsfrench{\captions@French
                        \captions@Francais\captions@Frenchb}%
                 \def\captionsfrancais{\captionsfrench}%
                 \def\captionsfrenchb{\captionsfrench}%
                 \iflanguage{french}{\captionsfrench}{}%
                }
\@ifpackagewith{babel}{canadien}{%
  \def\captionscanadien{\captionsfrench}%
  \def\datecanadien{\datefrench}%
  \def\extrascanadien{\extrasfrench}%
  \def\noextrascanadien{\noextrasfrench}%
  }{}
\@ifpackagewith{babel}{acadian}{%
  \def\captionsacadian{\captionsfrench}%
  \def\dateacadian{\datefrench}%
  \def\extrasacadian{\extrasfrench}%
  \def\noextrasacadian{\noextrasfrench}%
  }{}
%    \end{macrocode}
%
% \begin{macro}{\CaptionSeparator}
%    Let's consider now captions in figures and tables.
%    In French, captions in figures and tables should be printed with
%    endash (`--') instead of the standard `:'.
%
%    The standard definition of |\@makecaption| (e.g., the one provided
%    in article.cls, report.cls, book.cls which is frozen for \LaTeXe{}
%    according to Frank Mittelbach), has been saved in
%    |\STD@makecaption| before making `:' active
%    (see section~\ref{sec-punct}). `AtBeginDocument' we compare it to
%    its current definition (some classes like koma-script classes,
%    AMS classes, ua-thesis.cls\dots change it).
%    If they are identical, |frenchb| just adds a hook called
%    |\CaptionSeparator| to |\@makecaption|, |\CaptionSeparator|
%    defaults to `: ' as in the standard |\@makecaption|, and will be
%    changed to ` -- ' in French.
%    If the definitions differ, |frenchb| doesn't overwrite the changes,
%    but prints a message in the .log file.
%
%    \begin{macrocode}
\def\CaptionSeparator{\string:\space}
\long\def\FB@makecaption#1#2{%
  \vskip\abovecaptionskip
  \sbox\@tempboxa{#1\CaptionSeparator #2}%
  \ifdim \wd\@tempboxa >\hsize
    #1\CaptionSeparator #2\par
  \else
    \global \@minipagefalse
    \hb@xt@\hsize{\hfil\box\@tempboxa\hfil}%
  \fi
  \vskip\belowcaptionskip}
\AtBeginDocument{%
  \ifx\@makecaption\STD@makecaption
      \global\let\@makecaption\FB@makecaption
  \else
    \@ifundefined{@makecaption}{}%
       {\PackageWarning{frenchb.ldf}%
        {The definition of \protect\@makecaption\space
         has been changed,\MessageBreak
         frenchb will NOT customise it;\MessageBreak reported}%
       }%
  \fi
  \let\FB@makecaption\relax
  \let\STD@makecaption\relax
}
\expandafter\addto\csname extras\CurrentOption\endcsname{%
   \def\CaptionSeparator{\space\textendash\space}}
\expandafter\addto\csname noextras\CurrentOption\endcsname{%
    \def\CaptionSeparator{\string:\space}}
%    \end{macrocode}
% \end{macro}
%
%  \subsection{French lists}
%  \label{sec-lists}
%
%  \begin{macro}{\listFB}
%  \begin{macro}{\listORI}
%    Vertical spacing in general lists should be shorter in French
%    texts than the defaults provided by \LaTeX.
%    Note that the easy way, just changing values of vertical spacing
%    parameters when entering French and restoring them to their
%    defaults on exit would not work; as most lists are based on
%    |\list| we will define a variant of |\list| (|\listFB|) to
%    be used in French.
%
%    The amount of vertical space before and after a list is given by
%    |\topsep| + |\parskip| (+ |\partopsep| if the list starts a new
%    paragraph). IMHO, |\parskip| should be added \emph{only} when
%    the list starts a new paragraph, so I subtract |\parskip| from
%    |\topsep| and add it back to |\partopsep|; this will normally
%    make no difference because |\parskip|'s default value is 0pt, but
%    will be noticeable when |\parskip| is \emph{not} null.
%
%    |\endlist| is not redefined, but |\endlistORI| is provided for
%    the users who prefer to define their own lists from the original
%    command, they can code: |\begin{listORI}{}{} \end{listORI}|.
%    \begin{macrocode}
\let\listORI\list
\let\endlistORI\endlist
\def\FB@listsettings{%
      \setlength{\itemsep}{0.4ex plus 0.2ex minus 0.2ex}%
      \setlength{\parsep}{0.4ex plus 0.2ex minus 0.2ex}%
      \setlength{\topsep}{0.8ex plus 0.4ex minus 0.4ex}%
      \setlength{\partopsep}{0.4ex plus 0.2ex minus 0.2ex}%
%    \end{macrocode}
%    |\parskip| is of type `skip', its mean value only (\emph{not
%    the glue}) should be subtracted from |\topsep| and added to
%    |\partopsep|, so convert |\parskip| to a `dimen' using
%    |\@tempdima|.
%    \begin{macrocode}
      \@tempdima=\parskip
      \addtolength{\topsep}{-\@tempdima}%
      \addtolength{\partopsep}{\@tempdima}}%
\def\listFB#1#2{\listORI{#1}{\FB@listsettings #2}}%
\let\endlistFB\endlist
%    \end{macrocode}
%  \end{macro}
%  \end{macro}
%
%  \begin{macro}{\itemizeFB}
%  \begin{macro}{\itemizeORI}
%  \begin{macro}{\bbl@frenchlabelitems}
%  \begin{macro}{\bbl@nonfrenchlabelitems}
%    Let's now consider French itemize lists.  They differ from those
%    provided by the standard \LaTeXe{} classes:
%    \begin{itemize}
%      \item vertical spacing between items, before and after
%         the list, should be \emph{null} with \emph{no glue} added;
%      \item the item labels of a first level list should be vertically
%          aligned on the paragraph's first character (i.e. at
%          |\parindent| from the left margin);
%      \item the `\textbullet' is never used in French itemize-lists,
%          a long dash `--' is preferred for all levels. The item label
%          used in French is stored in |\FrenchLabelItem}|, it defaults
%          to `--' and can be changed using |\frenchbsetup{}| (see
%          section~\ref{sec-keyval}).
%    \end{itemize}
%
%    \begin{macrocode}
\newcommand*{\FrenchLabelItem}{\textendash}
\newcommand*{\Frlabelitemi}{\FrenchLabelItem}
\newcommand*{\Frlabelitemii}{\FrenchLabelItem}
\newcommand*{\Frlabelitemiii}{\FrenchLabelItem}
\newcommand*{\Frlabelitemiv}{\FrenchLabelItem}
%    \end{macrocode}
%    |\bbl@frenchlabelitems| saves current itemize labels and changes
%    them to their value in French. This code should never be executed
%    twice in a row, so we need a new flag that will be set and reset
%    by |\bbl@nonfrenchlabelitems| and |\bbl@frenchlabelitems|.
%    \begin{macrocode}
\newif\ifFB@enterFrench  \FB@enterFrenchtrue
\def\bbl@frenchlabelitems{%
  \ifFB@enterFrench
    \let\@ltiORI\labelitemi
    \let\@ltiiORI\labelitemii
    \let\@ltiiiORI\labelitemiii
    \let\@ltivORI\labelitemiv
    \let\labelitemi\Frlabelitemi
    \let\labelitemii\Frlabelitemii
    \let\labelitemiii\Frlabelitemiii
    \let\labelitemiv\Frlabelitemiv
    \FB@enterFrenchfalse
  \fi
}
\let\itemizeORI\itemize
\let\enditemizeORI\enditemize
\let\enditemizeFB\enditemize
\def\itemizeFB{%
    \ifnum \@itemdepth >\thr@@\@toodeep\else
      \advance\@itemdepth\@ne
      \edef\@itemitem{labelitem\romannumeral\the\@itemdepth}%
      \expandafter
      \listORI
      \csname\@itemitem\endcsname
      {\settowidth{\labelwidth}{\csname\@itemitem\endcsname}%
       \setlength{\leftmargin}{\labelwidth}%
       \addtolength{\leftmargin}{\labelsep}%
       \ifnum\@listdepth=0
         \setlength{\itemindent}{\parindent}%
       \else
         \addtolength{\leftmargin}{\parindent}%
       \fi
       \setlength{\itemsep}{\z@}%
       \setlength{\parsep}{\z@}%
       \setlength{\topsep}{\z@}%
       \setlength{\partopsep}{\z@}%
%    \end{macrocode}
%    |\parskip| is of type `skip', its mean value only (\emph{not
%    the glue}) should be subtracted from |\topsep| and added to
%    |\partopsep|, so convert |\parskip| to a `dimen' using
%    |\@tempdima|.
%    \begin{macrocode}
       \@tempdima=\parskip
       \addtolength{\topsep}{-\@tempdima}%
       \addtolength{\partopsep}{\@tempdima}}%
    \fi}
%    \end{macrocode}
%    The user's changes in labelitems are saved when leaving French for
%    further use when switching back to French.  This code should never
%    be executed twice in a row (toggle with |\bbl@frenchlabelitems|).
%    \begin{macrocode}
\def\bbl@nonfrenchlabelitems{%
  \ifFB@enterFrench
  \else
      \let\Frlabelitemi\labelitemi
      \let\Frlabelitemii\labelitemii
      \let\Frlabelitemiii\labelitemiii
      \let\Frlabelitemiv\labelitemiv
      \let\labelitemi\@ltiORI
      \let\labelitemii\@ltiiORI
      \let\labelitemiii\@ltiiiORI
      \let\labelitemiv\@ltivORI
      \FB@enterFrenchtrue
  \fi
}
%    \end{macrocode}
%  \end{macro}
%  \end{macro}
%  \end{macro}
%  \end{macro}
%
%  \subsection{French indentation of sections}
%  \label{sec-indent}
%
%  \begin{macro}{\bbl@frenchindent}
%  \begin{macro}{\bbl@nonfrenchindent}
%    In French the first paragraph of each section should be indented,
%    this is another difference with US-English. This is controlled by
%    the flag |\if@afterindent|.
%
% \changes{v2.3d}{2009/03/16}{Bug correction: previous versions of
%    frenchb set the flag \cs{if@afterindent} to false outside
%    French which is correct for English but wrong for some languages
%    like Spanish.  Pointed out by Juan Jos\'e Torrens.}
%
%    We need to save the value of the flag |\if@afterindent|
%    `AtBeginDocument' before eventually changing its value.
%
%    \begin{macrocode}
\AtBeginDocument{\ifx\@afterindentfalse\@afterindenttrue
                       \let\@aifORI\@afterindenttrue
                 \else \let\@aifORI\@afterindentfalse
                 \fi
}
\def\bbl@frenchindent{\let\@afterindentfalse\@afterindenttrue
                      \@afterindenttrue}
\def\bbl@nonfrenchindent{\let\@afterindentfalse\@aifORI
                         \@afterindentfalse}
%    \end{macrocode}
%  \end{macro}
%  \end{macro}
%
%  \subsection{Formatting footnotes}
%  \label{sec-footnotes}
%
% \changes{v2.0}{2006/11/06}{Footnotes are now printed
%     by default `\`a la fran\c caise' for the whole document.}
%
% \changes{v2.0b}{2007/04/18}{Footnotes: Just do nothing
%    (except warning) when the bigfoot package is loaded.}
%
%    The |bigfoot| package deeply changes the way footnotes are
%    handled. When |bigfoot| is loaded, we just warn the user that
%    |frenchb| will drop the customisation of footnotes.
%
%    The layout of footnotes is controlled by two flags
%    |\ifFBAutoSpaceFootnotes| and |\ifFBFrenchFootnotes| which are
%    set by options of |\frenchbsetup{}| (see section~\ref{sec-keyval}).
%    Notice that the layout of footnotes \emph{does not depend} on the
%    current language (just think of two footnotes on the same page
%    looking different because one was called in a French part, the
%    other one in English!).
%
%    When |\ifFBAutoSpaceFootnotes| is true, |\@footnotemark| (whose
%    definition is saved at the |\begin{document}| in order to include
%    any customisation that packages might have done) is redefined to
%    add a thin space before the number or symbol calling a footnote
%    (any space typed in is removed first).  This has no effect on
%    the layout of the footnote itself.
%
%    \begin{macrocode}
\AtBeginDocument{\@ifpackageloaded{bigfoot}%
                   {\PackageWarning{frenchb.ldf}%
                     {bigfoot package in use.\MessageBreak
                      frenchb will NOT customise footnotes;\MessageBreak
                      reported}}%
                   {\let\@footnotemarkORI\@footnotemark
                    \def\@footnotemarkFB{\leavevmode\unskip\unkern
                                         \,\@footnotemarkORI}%
                    \ifFBAutoSpaceFootnotes
                      \let\@footnotemark\@footnotemarkFB
                    \fi}%
                }
%    \end{macrocode}
%
%    We then define |\@makefntextFB|, a variant of |\@makefntext|
%    which is responsible for the layout of footnotes, to match the
%    specifications of the French `Imprimerie Nationale':  footnotes
%    will be indented by |\parindentFFN|, numbers (if any) typeset on
%    the baseline (instead of superscripts) and followed by a dot
%    and an half quad space. Whenever symbols are used to number
%    footnotes (as in |\thanks| for instance), we switch back to the
%    standard layout (the French layout of footnotes is meant for
%    footnotes numbered by Arabic or Roman digits).
%
% \changes{v2.0}{2006/11/06}{\cs{parindentFFN} not changed if
%    already defined (required by JA for cah-gut.cls).}
%
% \changes{v2.3b}{2008/12/06}{New commands \cs{dotFFN} and
%    \cs{kernFFN} for more flexibility (suggested by JA).}
%
%    The value of |\parindentFFN| will be redefined at the
%    |\begin{document}|, as the maximum of |\parindent| and 1.5em
%    \emph{unless} it has been set in the preamble (the weird value
%    10in is just for testing whether |\parindentFFN| has been set
%    or not).
%
%    \begin{macrocode}
\newcommand*{\dotFFN}{.}
\newcommand*{\kernFFN}{\kern .5em}
\newdimen\parindentFFN
\parindentFFN=10in
\def\ftnISsymbol{\@fnsymbol\c@footnote}
\long\def\@makefntextFB#1{\ifx\thefootnote\ftnISsymbol
                            \@makefntextORI{#1}%
                          \else
                            \parindent=\parindentFFN
                            \rule\z@\footnotesep
                            \setbox\@tempboxa\hbox{\@thefnmark}%
                            \ifdim\wd\@tempboxa>\z@
                              \llap{\@thefnmark}\dotFFN\kernFFN
                            \fi #1
                          \fi}%
%    \end{macrocode}
%
%    We save the standard definition of |\@makefntext| at the
%    |\begin{document}|, and then redefine |\@makefntext| according to
%    the value of flag |\ifFBFrenchFootnotes| (true or false).
%
%    \begin{macrocode}
\AtBeginDocument{\@ifpackageloaded{bigfoot}{}%
                  {\ifdim\parindentFFN<10in
                   \else
                      \parindentFFN=\parindent
                      \ifdim\parindentFFN<1.5em\parindentFFN=1.5em\fi
                   \fi
                   \let\@makefntextORI\@makefntext
                   \long\def\@makefntext#1{%
                      \ifFBFrenchFootnotes
                         \@makefntextFB{#1}%
                      \else
                         \@makefntextORI{#1}%
                      \fi}%
                  }%
                }
%    \end{macrocode}
%
%    For compatibility reasons, we provide definitions for the commands
%    dealing with the layout of footnotes in |frenchb| version~1.6.
%    |\frenchbsetup{}| (see in section \ref{sec-keyval}) should be
%    preferred for setting these options.  |\StandardFootnotes| may
%    still be used locally (in minipages for instance), that's why the
%    test |\ifFBFrenchFootnotes| is done inside |\@makefntext|.
%    \begin{macrocode}
\newcommand*{\AddThinSpaceBeforeFootnotes}{\FBAutoSpaceFootnotestrue}
\newcommand*{\FrenchFootnotes}{\FBFrenchFootnotestrue}
\newcommand*{\StandardFootnotes}{\FBFrenchFootnotesfalse}
%    \end{macrocode}
%
%  \subsection{Global layout}
%  \label{sec-global}
%
%    In multilingual documents, some typographic rules must depend
%    on the current language (e.g., hyphenation, typesetting of
%    numbers, spacing before double punctuation\dots), others should,
%    IMHO, be kept global to the document: especially the layout of
%    lists (see~\ref{sec-lists}) and footnotes
%    (see~\ref{sec-footnotes}), and the indentation of the first
%    paragraph of sections (see~\ref{sec-indent}).
%
%    From version 2.2 on, if |frenchb| is \babel's ``main language''
%    (i.e. last language option at \babel's loading), |frenchb|
%    customises the layout (i.e. lists, indentation of the first
%    paragraphs of sections and footnotes) in the whole document
%    regardless the current language.   On the other hand, if |frenchb|
%    is \emph{not} \babel's ``main language'', it leaves the layout
%    unchanged both in French and in other languages.
%
%  \begin{macro}{\FrenchLayout}
%  \begin{macro}{\StandardLayout}
%    The former commands |\FrenchLayout| and |\StandardLayout| are kept
%    for compatibility reasons but should no longer be used.
%
% \changes{v2.0g}{2008/03/23}{Flag \cs{ifFBStandardLayout} not checked
%     by \cs{FBprocess@options}, low-level flags have to be set
%     one by one.}
%
%    \begin{macrocode}
\newcommand*{\FrenchLayout}{%
    \FBGlobalLayoutFrenchtrue
    \PackageWarning{frenchb.ldf}%
    {\protect\FrenchLayout\space is obsolete.  Please use\MessageBreak
     \protect\frenchbsetup{GlobalLayoutFrench} instead.}%
}
\newcommand*{\StandardLayout}{%
  \FBReduceListSpacingfalse
  \FBCompactItemizefalse
  \FBStandardItemLabelstrue
  \FBIndentFirstfalse
  \FBFrenchFootnotesfalse
  \FBAutoSpaceFootnotesfalse
  \PackageWarning{frenchb.ldf}%
    {\protect\StandardLayout\space is obsolete.  Please use\MessageBreak
    \protect\frenchbsetup{StandardLayout} instead.}%
}
\@onlypreamble\FrenchLayout
\@onlypreamble\StandardLayout
%    \end{macrocode}
%  \end{macro}
%  \end{macro}
%
%  \subsection{Dots\dots}
%  \label{sec-dots}
%
%  \begin{macro}{\FBtextellipsis}
%    \LaTeXe's standard definition of |\dots| in text-mode is
%    |\textellipsis| which includes a |\kern| at the end;
%    this space is not wanted in some cases (before a closing brace
%    for instance) and |\kern| breaks hyphenation of the next word.
%    We define |\FBtextellipsis| for French (in \LaTeXe{} only).
%
%    The |\if| construction in the \LaTeXe{} definition of |\dots|
%    doesn't allow the use of |xspace| (|xspace| is always followed
%    by a |\fi|), so we use the AMS-\LaTeX{} construction of |\dots|;
%    this has to be done `AtBeginDocument' not to be overwritten
%    when \file{amsmath.sty} is loaded after \babel.
%
% \changes{v2.0}{2006/11/06}{Added special case for LY1 encoding,
%    see  bug report from Bruno Voisin (2004/05/18).}
%
%    LY1 has a ready made character for |\textellipsis|, it should be
%    used in French too (pointed out by Bruno Voisin).
%
%    \begin{macrocode}
\DeclareTextSymbol{\FBtextellipsis}{LY1}{133}
\DeclareTextCommandDefault{\FBtextellipsis}{%
    .\kern\fontdimen3\font.\kern\fontdimen3\font.\xspace}
%    \end{macrocode}
%    |\Mdots@| and |\Tdots@ORI| hold the definitions of |\dots| in
%    Math and Text mode. They default to those of amsmath-2.0, and
%    will revert to standard \LaTeX{} definitions `AtBeginDocument',
%    if amsmath has not been loaded. |\Mdots@| doesn't change when
%    switching from/to French, while |\Tdots@| is |\FBtextellipsis|
%    in French and |\Tdots@ORI| otherwise.
%    \begin{macrocode}
\newcommand*{\Tdots@ORI}{\@xp\textellipsis}
\newcommand*{\Tdots@}{\Tdots@ORI}
\newcommand*{\Mdots@}{\@xp\mdots@}
\AtBeginDocument{\DeclareRobustCommand*{\dots}{\relax
                 \csname\ifmmode M\else T\fi dots@\endcsname}%
                 \@ifundefined{@xp}{\let\@xp\relax}{}%
                 \@ifundefined{mdots@}{\let\Tdots@ORI\textellipsis
                                       \let\Mdots@\mathellipsis}{}}
\def\bbl@frenchdots{\let\Tdots@\FBtextellipsis}
\def\bbl@nonfrenchdots{\let\Tdots@\Tdots@ORI}
\expandafter\addto\csname extras\CurrentOption\endcsname{%
    \bbl@frenchdots}
\expandafter\addto\csname noextras\CurrentOption\endcsname{%
    \bbl@nonfrenchdots}
%    \end{macrocode}
%  \end{macro}
%
%  \subsection{Setup options: keyval stuff}
%  \label{sec-keyval}
%
% \changes{v2.0}{2006/11/06}{New command \cs{frenchbsetup} added
%     for global customisation.}
%
% \changes{v2.0c}{2007/06/25}{Option ThinSpaceInFrenchNumbers added.}
%
% \changes{v2.0d}{2007/07/15}{Options og and fg changed: limit
%     the definition to French so that quote characters can be used
%     in German.}
%
% \changes{v2.0e}{2007/10/05}{New option: StandardLists.}
%
% \changes{v2.0f}{2008/03/23}{Two typos corrected in
%    option StandardLists: [false] $\to$ [true] and
%    StandardLayout $\to$ StandardLists.}
%
% \changes{v2.0f}{2008/03/23}{StandardLayout option had no
%     effect on lists.  Test moved to \cs{FBprocess@options}.}
%
% \changes{v2.0g}{2008/03/23}{Revert previous change to
%     StandardLayout. This option must set the three flags
%     \cs{FBReduceListSpacingfalse}, \cs{FBCompactItemizefalse},
%     and \cs{FBStandardItemLabeltrue} instead of
%     \cs{FBStandardListstrue}, so that later options can still
%     change their value before executing \cs{FBprocess@options}.
%     Same thing for option StandardLists.}
%
% \changes{v2.1a}{2008/03/24}{New option: FrenchSuperscripts
%     to define \cs{up} as \cs{fup} or as \cs{textsuperscript}.}
%
% \changes{v2.1a}{2008/03/30}{New option: LowercaseSuperscripts.}
%
% \changes{v2.2a}{2008/05/08}{The global layout of the document is
%     no longer changed when frenchb is not the last option of babel
%     (\cs{bbl@main@language}). Suggested by Ulrike Fischer.}
%
% \changes{v2.2a}{2008/05/08}{Values of flags
%     \cs{ifFBReduceListSpacing}, \cs{ifFBCompactItemize},
%     \cs{ifFBStandardItemLabels}, \cs{ifFBIndentFirst},
%     \cs{ifFBFrenchFootnotes}, \cs{ifFBAutoSpaceFootnotes} changed:
%     default now means `StandardLayout', it will be changed to
%     `FrenchLayout' AtEndOfPackage only if french is
%     \cs{bbl@main@language}.}
%
% \changes{v2.2a}{2008/05/08}{When frenchb is babel's last option,
%     French becomes the document's main language, so
%     GlobalLayoutFrench applies.}
%
% \changes{v2.3a}{2008/10/10}{New option: OriginalTypewriter. Now
%    frenchb switches to \cs{noautospace@beforeFDP} when a tt-font is
%    in use.  When OriginalTypewriter is set to true, frenchb behaves
%    as in pre-2.3 versions.}
%
%    We first define a collection of conditionals with their defaults
%    (true or false).
%
%    \begin{macrocode}
\newif\ifFBStandardLayout           \FBStandardLayouttrue
\newif\ifFBGlobalLayoutFrench       \FBGlobalLayoutFrenchfalse
\newif\ifFBReduceListSpacing        \FBReduceListSpacingfalse
\newif\ifFBCompactItemize           \FBCompactItemizefalse
\newif\ifFBStandardItemLabels       \FBStandardItemLabelstrue
\newif\ifFBStandardLists            \FBStandardListstrue
\newif\ifFBIndentFirst              \FBIndentFirstfalse
\newif\ifFBFrenchFootnotes          \FBFrenchFootnotesfalse
\newif\ifFBAutoSpaceFootnotes       \FBAutoSpaceFootnotesfalse
\newif\ifFBOriginalTypewriter       \FBOriginalTypewriterfalse
\newif\ifFBThinColonSpace           \FBThinColonSpacefalse
\newif\ifFBThinSpaceInFrenchNumbers \FBThinSpaceInFrenchNumbersfalse
\newif\ifFBFrenchSuperscripts       \FBFrenchSuperscriptstrue
\newif\ifFBLowercaseSuperscripts    \FBLowercaseSuperscriptstrue
\newif\ifFBPartNameFull             \FBPartNameFulltrue
\newif\ifFBShowOptions              \FBShowOptionsfalse
%    \end{macrocode}
%
%    The defaults values of these flags have been set so that |frenchb|
%    does not change anything regarding the global layout.
%    |\bbl@main@language| (set by the last option of babel) controls
%    the global layout of the document.  We check the current language
%    `AtEndOfPackage' (it is |\bbl@main@language|); if it is French,
%    the values of some flags have to be changed to ensure a French
%    looking layout for the whole document (even in parts written in
%    languages other than French); the end-user will then be able to
%    customise the values of all these flags with |\frenchbsetup{}|.
%    \begin{macrocode}
\AtEndOfPackage{%
   \iflanguage{french}{\FBReduceListSpacingtrue
                       \FBCompactItemizetrue
                       \FBStandardItemLabelsfalse
                       \FBIndentFirsttrue
                       \FBFrenchFootnotestrue
                       \FBAutoSpaceFootnotestrue
                       \FBGlobalLayoutFrenchtrue}%
                      {}%
}
%    \end{macrocode}
%
%  \begin{macro}{\frenchbsetup}
%    From version 2.0 on, all setup options are handled by \emph{one}
%    command |\frenchbsetup| using the keyval syntax.
%    Let's now define this command which reads and sets the options
%    to be processed later (at |\begin{document}|) by
%    |\FBprocess@options|. It  can only be called in the preamble.
%    \begin{macrocode}
\newcommand*{\frenchbsetup}[1]{%
  \setkeys{FB}{#1}%
}%
\@onlypreamble\frenchbsetup
%    \end{macrocode}
%    |frenchb| being an option of babel, it cannot load a package
%    (keyval) while |frenchb.ldf| is read, so we defer the loading of
%    \file{keyval} and the options setup at the end of \babel's loading.
%
%    |StandardLayout| resets the layout in French to the standard layout
%    defined par the \LaTeX{} class and packages loaded. It deals with
%    lists, indentation of first paragraphs of sections and footnotes.
%    Other keys, entered \emph{after} |StandardLayout| in
%    |\frenchbsetup|, can overrule some of the |StandardLayout|
%     settings.
%
%    |GlobalLayoutFrench| forces the layout in French and (as far as
%    possible) outside French to meet the French typographic standards.
%
% \changes{v2.3d}{2009/03/16}{Warning added to \cs{GlobalLayoutFrench}
%    when French is not the main language.}
%
%    \begin{macrocode}
\AtEndOfPackage{%
    \RequirePackage{keyval}%
    \define@key{FB}{StandardLayout}[true]%
                      {\csname FBStandardLayout#1\endcsname
                       \ifFBStandardLayout
                         \FBReduceListSpacingfalse
                         \FBCompactItemizefalse
                         \FBStandardItemLabelstrue
                         \FBIndentFirstfalse
                         \FBFrenchFootnotesfalse
                         \FBAutoSpaceFootnotesfalse
                         \FBGlobalLayoutFrenchfalse
                       \else
                         \FBReduceListSpacingtrue
                         \FBCompactItemizetrue
                         \FBStandardItemLabelsfalse
                         \FBIndentFirsttrue
                         \FBFrenchFootnotestrue
                         \FBAutoSpaceFootnotestrue
                       \fi}%
    \define@key{FB}{GlobalLayoutFrench}[true]%
                      {\csname FBGlobalLayoutFrench#1\endcsname
                       \ifFBGlobalLayoutFrench
                          \iflanguage{french}%
                            {\FBReduceListSpacingtrue
                             \FBCompactItemizetrue
                             \FBStandardItemLabelsfalse
                             \FBIndentFirsttrue
                             \FBFrenchFootnotestrue
                             \FBAutoSpaceFootnotestrue}%
                            {\PackageWarning{frenchb.ldf}%
                              {Option `GlobalLayoutFrench' skipped:
                               \MessageBreak French is *not*
                               babel's last option.\MessageBreak}}%
                       \fi}%
    \define@key{FB}{ReduceListSpacing}[true]%
                      {\csname FBReduceListSpacing#1\endcsname}%
    \define@key{FB}{CompactItemize}[true]%
                      {\csname FBCompactItemize#1\endcsname}%
    \define@key{FB}{StandardItemLabels}[true]%
                      {\csname FBStandardItemLabels#1\endcsname}%
    \define@key{FB}{ItemLabels}{%
        \renewcommand*{\FrenchLabelItem}{#1}}%
    \define@key{FB}{ItemLabeli}{%
        \renewcommand*{\Frlabelitemi}{#1}}%
    \define@key{FB}{ItemLabelii}{%
        \renewcommand*{\Frlabelitemii}{#1}}%
    \define@key{FB}{ItemLabeliii}{%
        \renewcommand*{\Frlabelitemiii}{#1}}%
    \define@key{FB}{ItemLabeliv}{%
        \renewcommand*{\Frlabelitemiv}{#1}}%
    \define@key{FB}{StandardLists}[true]%
                      {\csname FBStandardLists#1\endcsname
                       \ifFBStandardLists
                         \FBReduceListSpacingfalse
                         \FBCompactItemizefalse
                         \FBStandardItemLabelstrue
                       \else
                         \FBReduceListSpacingtrue
                         \FBCompactItemizetrue
                         \FBStandardItemLabelsfalse
                       \fi}%
    \define@key{FB}{IndentFirst}[true]%
                      {\csname FBIndentFirst#1\endcsname}%
    \define@key{FB}{FrenchFootnotes}[true]%
                      {\csname FBFrenchFootnotes#1\endcsname}%
    \define@key{FB}{AutoSpaceFootnotes}[true]%
                      {\csname FBAutoSpaceFootnotes#1\endcsname}%
    \define@key{FB}{AutoSpacePunctuation}[true]%
                      {\csname FBAutoSpacePunctuation#1\endcsname}%
    \define@key{FB}{OriginalTypewriter}[true]%
                      {\csname FBOriginalTypewriter#1\endcsname}%
    \define@key{FB}{ThinColonSpace}[true]%
                      {\csname FBThinColonSpace#1\endcsname}%
    \define@key{FB}{ThinSpaceInFrenchNumbers}[true]%
                      {\csname FBThinSpaceInFrenchNumbers#1\endcsname}%
    \define@key{FB}{FrenchSuperscripts}[true]%
                      {\csname FBFrenchSuperscripts#1\endcsname}
    \define@key{FB}{LowercaseSuperscripts}[true]%
                      {\csname FBLowercaseSuperscripts#1\endcsname}
    \define@key{FB}{PartNameFull}[true]%
                      {\csname FBPartNameFull#1\endcsname}%
    \define@key{FB}{ShowOptions}[true]%
                      {\csname FBShowOptions#1\endcsname}%
%    \end{macrocode}
%    Inputing French quotes as \emph{single characters} when they are
%    available on the keyboard (through a compose key for instance)
%    is more comfortable than typing |\og| and |\fg|.
%    The purpose of the following code is to map the French quote
%    characters to |\og\ignorespaces| and |{\fg}| respectively when
%    the current language is French, and to |\guillemotleft| and
%    |\guillemotright| otherwise (think of German quotes); thus correct
%    unbreakable spaces will be added automatically to French quotes.
%    The quote characters typed in depend on the input encoding,
%    it can be single-byte (latin1, latin9, applemac,\dots) or
%    multi-bytes (utf-8, utf8x).  We first check whether XeTeX is used
%    or not, if not the package |inputenc| has to be loaded before the
%    |\begin{document}| with the proper coding option, so we check if
%    |\DeclareInputText| is defined.
%    \begin{macrocode}
    \define@key{FB}{og}{%
       \newcommand*{\FB@@og}{\iflanguage{french}%
                               {\FB@og\ignorespaces}{\guillemotleft}}%
       \expandafter\ifx\csname XeTeXrevision\endcsname\relax
         \AtBeginDocument{%
           \@ifundefined{DeclareInputText}%
             {\PackageWarning{frenchb.ldf}%
               {Option `og' requires package inputenc.\MessageBreak}%
             }%
             {\@ifundefined{uc@dclc}%
%    \end{macrocode}
%    if |\uc@dclc| is undefined, utf8x is not loaded\dots
%    \begin{macrocode}
               {\@ifundefined{DeclareUnicodeCharacter}%
%    \end{macrocode}
%    if |\DeclareUnicodeCharacter| is undefined, utf8 is not loaded
%    either, we assume 8-bit character input encoding.
%    Package MULEenc.sty (from CJK) defines |\mule@def| to map
%    characters to control sequences.
%    \begin{macrocode}
                  {\@tempcnta`#1\relax
                     \@ifundefined{mule@def}%
                       {\DeclareInputText{\the\@tempcnta}{\FB@@og}}%
                       {\mule@def{11}{{\FB@@og}}}%
                  }%
%    \end{macrocode}
%    utf8 loaded, use |\DeclareUnicodeCharacter|,
%    \begin{macrocode}
                  {\DeclareUnicodeCharacter{00AB}{\FB@@og}}%
               }%
%    \end{macrocode}
%    utf8x loaded, use |\uc@dclc|,
%    \begin{macrocode}
               {\uc@dclc{171}{default}{\FB@@og}}%
             }%
         }%
%    \end{macrocode}
%    XeTeX in use, the following trick for defining the active quote
%    character is borrowed from \file{inputenc.dtx}.
%    \begin{macrocode}
       \else
         \catcode`#1=\active
         \bgroup
           \uccode`\~`#1%
           \uppercase{%
         \egroup
         \def~%
         }{\FB@@og}%
       \fi
    }%
%    \end{macrocode}
%    Same code for the closing quote.
%    \begin{macrocode}
    \define@key{FB}{fg}{%
       \newcommand*{\FB@@fg}{\iflanguage{french}%
                               {\FB@fg}{\guillemotright}}%
       \expandafter\ifx\csname XeTeXrevision\endcsname\relax
         \AtBeginDocument{%
           \@ifundefined{DeclareInputText}%
             {\PackageWarning{frenchb.ldf}%
               {Option `fg' requires package inputenc.\MessageBreak}%
             }%
             {\@ifundefined{uc@dclc}%
               {\@ifundefined{DeclareUnicodeCharacter}%
                  {\@tempcnta`#1\relax
                     \@ifundefined{mule@def}%
                       {\DeclareInputText{\the\@tempcnta}{{\FB@@fg}}}%
                       {\mule@def{27}{{\FB@@fg}}}%
                  }%
                  {\DeclareUnicodeCharacter{00BB}{{\FB@@fg}}}%
               }%
               {\uc@dclc{187}{default}{{\FB@@fg}}}%
             }%
         }%
       \else
         \catcode`#1=\active
         \bgroup
           \uccode`\~`#1%
           \uppercase{%
         \egroup
         \def~%
         }{{\FB@@fg}}%
       \fi
    }%
}
%    \end{macrocode}
%  \end{macro}
%
% \begin{macro}{\FBprocess@options}
%    |\FBprocess@options| processes the options, it is called \emph{once}
%    at |\begin{document}|.
%    \begin{macrocode}
\newcommand*{\FBprocess@options}{%
%    \end{macrocode}
%    Nothing has to be done here for |StandardLayout| and
%    |StandardLists| (the involved flags have already been set in
%    |\frenchbsetup{}| or before (at babel's EndOfPackage).
%
%    The next three options deal with the layout of lists in French.
%
%    |ReduceListSpacing| reduces the vertical spaces between list
%    items in French (done by changing |\list| to |\listFB|).
%    When |GlobalLayoutFrench| is true (the default), the same is
%    done outside French except for languages that force a different
%    setting.
%    \begin{macrocode}
  \ifFBReduceListSpacing
    \addto\extrasfrench{\let\list\listFB
                        \let\endlist\endlistFB}%
    \addto\noextrasfrench{\ifFBGlobalLayoutFrench
                            \let\list\listFB
                            \let\endlist\endlistFB
                          \else
                            \let\list\listORI
                            \let\endlist\endlistORI
                          \fi}%
  \else
    \addto\extrasfrench{\let\list\listORI
                        \let\endlist\endlistORI}%
    \addto\noextrasfrench{\let\list\listORI
                          \let\endlist\endlistORI}%
  \fi
%    \end{macrocode}
%
%    |CompactItemize| suppresses the vertical spacing between list
%    items in French (done by changing |\itemize| to |\itemizeFB|).
%    When |GlobalLayoutFrench| is true the same is done outside French.
%    \begin{macrocode}
  \ifFBCompactItemize
    \addto\extrasfrench{\let\itemize\itemizeFB
                        \let\enditemize\enditemizeFB}%
    \addto\noextrasfrench{\ifFBGlobalLayoutFrench
                             \let\itemize\itemizeFB
                             \let\enditemize\enditemizeFB
                          \else
                             \let\itemize\itemizeORI
                             \let\enditemize\enditemizeORI
                          \fi}%
  \else
    \addto\extrasfrench{\let\itemize\itemizeORI
                        \let\enditemize\enditemizeORI}%
    \addto\noextrasfrench{\let\itemize\itemizeORI
                          \let\enditemize\enditemizeORI}%
  \fi
%    \end{macrocode}
%
%    |StandardItemLabels| resets labelitems in French to their
%    standard values set by the \LaTeX{} class and packages loaded.
%    When |GlobalLayoutFrench| is true labelitems are identical inside
%    and outside French.
%    \begin{macrocode}
  \ifFBStandardItemLabels
    \addto\extrasfrench{\bbl@nonfrenchlabelitems}%
    \addto\noextrasfrench{\bbl@nonfrenchlabelitems}%
  \else
    \addto\extrasfrench{\bbl@frenchlabelitems}%
    \addto\noextrasfrench{\ifFBGlobalLayoutFrench
                            \bbl@frenchlabelitems
                          \else
                            \bbl@nonfrenchlabelitems
                          \fi}%
  \fi
%    \end{macrocode}
%
%    |IndentFirst| forces the first paragraphs of sections to be
%    indented just like the other ones in French.
%    When |GlobalLayoutFrench| is true (the default), the same is
%    done outside French except for languages that force a different
%    setting.
%    \begin{macrocode}
  \ifFBIndentFirst
    \addto\extrasfrench{\bbl@frenchindent}%
    \addto\noextrasfrench{\ifFBGlobalLayoutFrench
                             \bbl@frenchindent
                          \else
                             \bbl@nonfrenchindent
                          \fi}%
  \else
    \addto\extrasfrench{\bbl@nonfrenchindent}%
    \addto\noextrasfrench{\bbl@nonfrenchindent}%
  \fi
%    \end{macrocode}
%
%    The layout of footnotes is handled at the |\begin{document}|
%    depending on the values of flags |FrenchFootnotes|
%    and |AutoSpaceFootnotes| (see section~\ref{sec-footnotes}),
%    nothing has to be done here for footnotes.
%
%    |AutoSpacePunctuation| adds an unbreakable space (in French only)
%    before the four active characters (:;!?) even if none has been
%    typed before them.
%    \begin{macrocode}
  \ifFBAutoSpacePunctuation
     \autospace@beforeFDP
  \else
     \noautospace@beforeFDP
  \fi
%    \end{macrocode}
%
%    When |OriginalTypewriter| is set to |false| (the default),
%    |\ttfamily|, |\rmfamily| and |\sffamily| are redefined as
%    |\ttfamilyFB|, |\rmfamilyFB| and |\sffamilyFB| respectively
%    to prevent addition of automatic spaces before the four active
%    characters in computer code.
%    \begin{macrocode}
  \ifFBOriginalTypewriter
  \else
     \let\ttfamily\ttfamilyFB
     \let\rmfamily\rmfamilyFB
     \let\sffamily\sffamilyFB
  \fi
%    \end{macrocode}
%
%    |ThinColonSpace| changes the normal unbreakable space typeset in
%     French before `:' to a thin space.
%    \begin{macrocode}
  \ifFBThinColonSpace\renewcommand*{\Fcolonspace}{\thinspace}\fi
%    \end{macrocode}
%
%    When |true|, |ThinSpaceInFrenchNumbers| redefines |numprint.sty|'s
%    command |\npstylefrench| to set |\npthousandsep| to |\,|
%    (thinspace) instead of |~| (default) . This option has no effect
%    if package |numprint.sty| is not loaded with `|autolanguage|'.
%    As old versions of |numprint.sty| did not define |\npstylefrench|,
%    we have to provide this command.
%    \begin{macrocode}
  \@ifpackageloaded{numprint}%
  {\ifnprt@autolanguage
     \providecommand*{\npstylefrench}{}%
     \ifFBThinSpaceInFrenchNumbers
       \renewcommand*\npstylefrench{%
          \npthousandsep{\,}%
          \npdecimalsign{,}%
          \npproductsign{\cdot}%
          \npunitseparator{\,}%
          \npdegreeseparator{}%
          \nppercentseparator{\nprt@unitsep}%
          }%
     \else
       \renewcommand*\npstylefrench{%
          \npthousandsep{~}%
          \npdecimalsign{,}%
          \npproductsign{\cdot}%
          \npunitseparator{\,}%
          \npdegreeseparator{}%
          \nppercentseparator{\nprt@unitsep}%
          }%
     \fi
     \npaddtolanguage{french}{french}%
   \fi}{}%
%    \end{macrocode}
%
%    |FrenchSuperscripts|: if |true| |\up=\fup|, else
%    |\up=\textsuperscript|. Anyway |\up*=\FB@up@fake|. The star-form
%    |\up*{}| is provided for fonts that lack some superior letters:
%    Adobe Jenson Pro and Utopia Expert have no ``g superior'' for
%    instance.
%    \begin{macrocode}
  \ifFBFrenchSuperscripts
    \DeclareRobustCommand*{\up}{\@ifstar{\FB@up@fake}{\fup}}%
  \else
    \DeclareRobustCommand*{\up}{\@ifstar{\FB@up@fake}%
                                        {\textsuperscript}}%
  \fi
%    \end{macrocode}
%
%    |LowercaseSuperscripts|: if |true| let |\FB@lc| be |\lowercase|,
%     else |\FB@lc| is redefined to do nothing.
%    \begin{macrocode}
  \ifFBLowercaseSuperscripts
  \else
    \renewcommand*{\FB@lc}[1]{##1}%
  \fi
%    \end{macrocode}
%
%    |PartNameFull|: if |false|, redefine |\partname|.
%    \begin{macrocode}
  \ifFBPartNameFull
  \else\addto\captionsfrench{\def\partname{Partie}}\fi
%    \end{macrocode}
%
%    |ShowOptions|: if |true|, print the list of all options to the
%    \file{.log} file.
%    \begin{macrocode}
  \ifFBShowOptions
    \GenericWarning{* }{%
     * **** List of possible options for frenchb ****\MessageBreak
     [Default values between brackets when frenchb is loaded *LAST*]%
     \MessageBreak
     ShowOptions=true [false]\MessageBreak
     StandardLayout=true [false]\MessageBreak
     GlobalLayoutFrench=false [true]\MessageBreak
     StandardLists=true [false]\MessageBreak
     ReduceListSpacing=false [true]\MessageBreak
     CompactItemize=false [true]\MessageBreak
     StandardItemLabels=true [false]\MessageBreak
     ItemLabels=\textemdash, \textbullet,
        \protect\ding{43},... [\textendash]\MessageBreak
     ItemLabeli=\textemdash, \textbullet,
        \protect\ding{43},... [\textendash]\MessageBreak
     ItemLabelii=\textemdash, \textbullet,
        \protect\ding{43},... [\textendash]\MessageBreak
     ItemLabeliii=\textemdash, \textbullet,
        \protect\ding{43},... [\textendash]\MessageBreak
     ItemLabeliv=\textemdash, \textbullet,
        \protect\ding{43},... [\textendash]\MessageBreak
     IndentFirst=false [true]\MessageBreak
     FrenchFootnotes=false [true]\MessageBreak
     AutoSpaceFootnotes=false [true]\MessageBreak
     AutoSpacePunctuation=false [true]\MessageBreak
     OriginalTypewriter=true [false]\MessageBreak
     ThinColonSpace=true [false]\MessageBreak
     ThinSpaceInFrenchNumbers=true [false]\MessageBreak
     FrenchSuperscripts=false [true]\MessageBreak
     LowercaseSuperscripts=false [true]\MessageBreak
     PartNameFull=false [true]\MessageBreak
     og= <left quote character>, fg= <right quote character>
     \MessageBreak
     *********************************************
     \MessageBreak\protect\frenchbsetup{ShowOptions}}
  \fi
}
%    \end{macrocode}
%  \end{macro}
%
% \changes{v2.0}{2006/12/15}{AtBeginDocument, save again the
%    definitions of the `list' and `itemize' environments and the
%    values of labelitems.  As of frenchb v.1.6, `ORI' values were
%    set when reading frenchb.ldf, later changes were ignored.}
%
% \changes{v2.0}{2006/12/06}{Added warning for OT1 encoding.}
%
% \changes{v2.1b}{2008/04/07}{Disable some commands in bookmarks.}
%
%    At |\begin{document}| we save again the definitions of the `list'
%    and `itemize' environments and the values of labelitems so that
%    all changes made in the preamble are taken into account in
%    languages other than French and in French with the StandardLayout
%    option.  We also have to provide an |\xspace| command in case the
%    |xspace.sty| package is not loaded.
%
%    \begin{macrocode}
\AtBeginDocument{%
   \let\listORI\list
   \let\endlistORI\endlist
   \let\itemizeORI\itemize
   \let\enditemizeORI\enditemize
   \let\@ltiORI\labelitemi
   \let\@ltiiORI\labelitemii
   \let\@ltiiiORI\labelitemiii
   \let\@ltivORI\labelitemiv
   \providecommand*{\xspace}{\relax}%
%    \end{macrocode}
%    Let's redefine some commands in \file{hyperref}'s bookmarks.
%    \begin{macrocode}
   \@ifundefined{pdfstringdefDisableCommands}{}%
     {\pdfstringdefDisableCommands{%
        \let\up\relax
        \def\ieme{e\xspace}%
        \def\iemes{es\xspace}%
        \def\ier{er\xspace}%
        \def\iers{ers\xspace}%
        \def\iere{re\xspace}%
        \def\ieres{res\xspace}%
        \def\FrenchEnumerate#1{#1\degre\space}%
        \def\FrenchPopularEnumerate#1{#1\degre)\space}%
        \def\No{N\degre\space}%
        \def\no{n\degre\space}%
        \def\Nos{N\degre\space}%
        \def\nos{n\degre\space}%
        \def\og{\guillemotleft\space}%
        \def\fg{\space\guillemotright}%
        \let\bsc\textsc
        \let\degres\degre
     }}%
%    \end{macrocode}
%    It is time to process the options set with |\frenchboptions{}|.
%    Then execute either |\extrasfrench| and |\captionsfrench| or
%    |\noextrasfrench| according to the current language at the
%    |\begin{document}| (these three commands are updated by
%    |\FBprocess@options|).
%    \begin{macrocode}
   \FBprocess@options
   \iflanguage{french}{\extrasfrench\captionsfrench}{\noextrasfrench}%
%    \end{macrocode}
%    Some warnings are issued when output font encodings are not
%    properly set. With XeLaTeX, \file{fontspec.sty} and
%    \file{xunicode.sty} should be loaded; with (pdf)\LaTeX, a warning
%    is issued when OT1 encoding is in use at the |\begin{document}|.
%    Mind that |\encodingdefault| is defined as `long', defining
%    |\FBOTone| with |\newcommand*| would fail!
%    \begin{macrocode}
   \expandafter\ifx\csname XeTeXrevision\endcsname\relax
      \begingroup \newcommand{\FBOTone}{OT1}%
      \ifx\encodingdefault\FBOTone
        \PackageWarning{frenchb.ldf}%
           {OT1 encoding should not be used for French.
            \MessageBreak
            Add \protect\usepackage[T1]{fontenc} to the
            preamble\MessageBreak of your document,}
      \fi
     \endgroup
   \else
     \@ifundefined{DeclareUTFcharacter}%
       {\PackageWarning{frenchb.ldf}%
         {Add \protect\usepackage{fontspec} *and*\MessageBreak
          \protect\usepackage{xunicode} to the preamble\MessageBreak
          of your document,}}%
       {}%
    \fi
}
%    \end{macrocode}
%
%  \subsection{Clean up and exit}
%
%    Load |frenchb.cfg| (should do nothing, just for compatibility).
%    \begin{macrocode}
\loadlocalcfg{frenchb}
%    \end{macrocode}
%    Final cleaning.
%    The macro |\ldf@quit| takes care for setting the main language
%    to be switched on at |\begin{document}| and resetting the
%    category code of \texttt{@} to its original value.
%    The config file searched for has to be |frenchb.cfg|, and
%    |\CurrentOption| has been set to `french', so
%    |\ldf@finish\CurrentOption| cannot be used: we first load
%    |frenchb.cfg|, then call |\ldf@quit\CurrentOption|.
%    \begin{macrocode}
\FBclean@on@exit
\ldf@quit\CurrentOption
%    \end{macrocode}
% \iffalse
%</code>
%<*dtx>
% \fi
%%
%% \CharacterTable
%%  {Upper-case    \A\B\C\D\E\F\G\H\I\J\K\L\M\N\O\P\Q\R\S\T\U\V\W\X\Y\Z
%%   Lower-case    \a\b\c\d\e\f\g\h\i\j\k\l\m\n\o\p\q\r\s\t\u\v\w\x\y\z
%%   Digits        \0\1\2\3\4\5\6\7\8\9
%%   Exclamation   \!     Double quote  \"     Hash (number) \#
%%   Dollar        \$     Percent       \%     Ampersand     \&
%%   Acute accent  \'     Left paren    \(     Right paren   \)
%%   Asterisk      \*     Plus          \+     Comma         \,
%%   Minus         \-     Point         \.     Solidus       \/
%%   Colon         \:     Semicolon     \;     Less than     \<
%%   Equals        \=     Greater than  \>     Question mark \?
%%   Commercial at \@     Left bracket  \[     Backslash     \\
%%   Right bracket \]     Circumflex    \^     Underscore    \_
%%   Grave accent  \`     Left brace    \{     Vertical bar  \|
%%   Right brace   \}     Tilde         \~}
%%
% \iffalse
%</dtx>
% \fi
%
% \Finale
\endinput
}
\DeclareOption{frenchb}{% \iffalse meta-comment
%
% Copyright 1989-2009 Johannes L. Braams and any individual authors
% listed elsewhere in this file.  All rights reserved.
% 
% This file is part of the Babel system.
% --------------------------------------
% 
% It may be distributed and/or modified under the
% conditions of the LaTeX Project Public License, either version 1.3
% of this license or (at your option) any later version.
% The latest version of this license is in
%   http://www.latex-project.org/lppl.txt
% and version 1.3 or later is part of all distributions of LaTeX
% version 2003/12/01 or later.
% 
% This work has the LPPL maintenance status "maintained".
% 
% The Current Maintainer of this work is Johannes Braams.
% 
% The list of all files belonging to the Babel system is
% given in the file `manifest.bbl. See also `legal.bbl' for additional
% information.
% 
% The list of derived (unpacked) files belonging to the distribution
% and covered by LPPL is defined by the unpacking scripts (with
% extension .ins) which are part of the distribution.
% \fi
% \CheckSum{2135}
%
% \iffalse
%    Tell the \LaTeX\ system who we are and write an entry on the
%    transcript. Nothing to write to the .cfg file, if generated.
%<*dtx>
\ProvidesFile{frenchb.dtx}
%</dtx>
% \changes{v2.1d}{2008/05/04}{Argument of \cs{ProvidesLanguage} changed
%     from `french' to `frenchb', otherwise \cs{listfiles} prints
%     no date/version information.  The bug with \cs{listfiles}
%     (introduced in v.1.5!), was pointed out by Ulrike Fischer.}
%<code>\ProvidesLanguage{frenchb}
%\ProvidesFile{frenchb.dtx}
%<*!cfg>
        [2009/03/16 v2.3d French support from the babel system]
%</!cfg>
%<*cfg>
%% frenchb.cfg: configuration file for frenchb.ldf
%% Daniel Flipo Daniel.Flipo at univ-lille1.fr
%</cfg>
%%    File `frenchb.dtx'
%%    Babel package for LaTeX version 2e
%%    Copyright (C) 1989 - 2009
%%              by Johannes Braams, TeXniek
%
%<*!cfg>
%%    Frenchb language Definition File
%%    Copyright (C) 1989 - 2009
%%              by Johannes Braams, TeXniek
%%                 Daniel Flipo, GUTenberg
%
%%    Please report errors to: Daniel Flipo, GUTenberg
%%                             Daniel.Flipo at univ-lille1.fr
%</!cfg>
%
%    This file is part of the babel system, it provides the source
%    code for the French language definition file.
%
%<*filedriver>
\documentclass[a4paper]{ltxdoc}
\DeclareFontEncoding{T1}{}{}
\DeclareFontSubstitution{T1}{lmr}{m}{n}
\DeclareTextCommand{\guillemotleft}{OT1}{%
  {\fontencoding{T1}\fontfamily{lmr}\selectfont\char19}}
\DeclareTextCommand{\guillemotright}{OT1}{%
  {\fontencoding{T1}\fontfamily{lmr}\selectfont\char20}}
\newcommand*\TeXhax{\TeX hax}
\newcommand*\babel{\textsf{babel}}
\newcommand*\langvar{$\langle \mathit lang \rangle$}
\newcommand*\note[1]{}
\newcommand*\Lopt[1]{\textsf{#1}}
\newcommand*\file[1]{\texttt{#1}}
\begin{document}
\setlength{\parindent}{0pt}
\begin{center}
  \textbf{\Large A Babel language definition file for French}\\[3mm]^^A\]
  Daniel \textsc{Flipo}\\
  \texttt{Daniel.Flipo@univ-lille1.fr}
\end{center}
 \RecordChanges
 \DocInput{frenchb.dtx}
\end{document}
%</filedriver>
% \fi
% \GetFileInfo{frenchb.dtx}
%
%  \section{The French language}
%
%    The file \file{\filename}\footnote{The file described in this
%    section has version number \fileversion\ and was last revised on
%    \filedate.}, defines all the language definition macros for the
%    French language.
%
%    Customisation for the French language is achieved following the
%    book ``Lexique des r\`egles typographiques en usage \`a
%    l'Imprimerie nationale'' troisi\`eme \'edition (1994),
%    ISBN-2-11-081075-0.
%
%    First version released: 1.1 (1996/05/31) as part of
%    \babel-3.6beta.
%
%    |frenchb| has been improved using helpful suggestions from many
%    people, mainly from Jacques Andr\'e, Michel Bovani, Thierry Bouche,
%    and Vincent Jalby.  Thanks to all of them!
%
%    This new version (2.x) has been designed to be used with \LaTeXe{}
%    and Plain\TeX{} formats only. \LaTeX-2.09 is no longer supported.
%    Changes between version 1.6 and \fileversion{} are listed in
%    subsection~\ref{ssec-changes} p.~\pageref{ssec-changes}.
%
%    An extensive documentation is available in French here:\\
%    |http://daniel.flipo.free.fr/frenchb|
%
%  \subsection{Basic interface}
%
%    In a multilingual document, some typographic rules are language
%    dependent, i.e. spaces before `double punctuation' (|:| |;| |!|
%    |?|) in French, others concern the general layout (i.e. layout of
%    lists, footnotes, indentation of first paragraphs of sections) and
%    should apply to the whole document.
%
%    Starting with version~2.2, |frenchb| behaves differently according
%    to \babel's \emph{main language} defined as the \emph{last}
%    option\footnote{Its name is kept in \texttt{\textbackslash
%           bbl@main@language}.} at \babel's loading.  When French is
%    not \babel's main language, |frenchb| no longer alters the global
%    layout of the document (even in parts where French is the current
%    language): the layout of lists, footnotes, indentation of first
%    paragraphs of sections are not customised by |frenchb|.
%
%    When French is loaded as the last option of \babel, |frenchb|
%    makes the following changes to the global layout, \emph{both in
%    French and in all other languages}\footnote{%
%       For each item, hooks are provided to reset standard
%       \LaTeX{} settings or to emulate the behavior of former versions
%       of \texttt{frenchb} (see command
%       \texttt{\textbackslash frenchbsetup\{\}},
%       section~\ref{ssec-custom}).}:
%    \begin{enumerate}
%    \item the first paragraph of each section is indented
%          (\LaTeX{} only);
%    \item the default items in itemize environment are set to `--'
%          instead of `\textbullet', and all vertical spacing and glue
%          is deleted; it is possible to change `--' to something else
%          (`---' for instance) using |\frenchbsetup{}|;
%    \item vertical spacing in general \LaTeX{} lists is
%          shortened;
%    \item footnotes are displayed ``\`a la fran\c{c}aise''.
%    \end{enumerate}
%
%    Regarding local typography, the command |\selectlanguage{french}|
%    switches to the French language\footnote{%
%      \texttt{\textbackslash selectlanguage\{francais\}}
%      and \texttt{\textbackslash selectlanguage\{frenchb\}} are kept
%      for backward compatibility but should no longer be used.},
%    with the following effects:
%    \begin{enumerate}
%    \item French hyphenation patterns are made active;
%    \item `double punctuation' (|:| |;| |!| |?|) is made
%           active%\footnote{Actually, they are active in the whole
%           document, only their expansions differ in French and
%           outside French} for correct spacing in French;
%    \item |\today| prints the date in French;
%    \item the caption names are translated into French
%          (\LaTeX{} only);
%    \item the space after |\dots| is removed in French.
%    \end{enumerate}
%
%    Some commands are provided in |frenchb| to make typesetting
%    easier:
%    \begin{enumerate}
%    \item French quotation marks can be entered using the commands
%          |\og| and |\fg| which work in \LaTeXe and Plain\TeX,
%          their appearance depending on what is available to draw
%          them; even if you use \LaTeXe{} \emph{and} |T1|-encoding,
%          you should refrain from entering them as
%          |<<~French quotation marks~>>|: |\og| and |\fg| provide
%          better horizontal spacing.
%          |\og| and |\fg| can be used outside French, they typeset
%          then English quotes `` and ''.
%    \item A command |\up| is provided to typeset superscripts like
%          |M\up{me}| (abbreviation for ``Madame''), |1\up{er}| (for
%          ``premier'').  Other commands are also provided for
%          ordinals: |\ier|, |\iere|, |\iers|, |\ieres|, |\ieme|,
%          |\iemes| (|3\iemes| prints 3\textsuperscript{es}).
%    \item Family names should be typeset in small capitals and never
%          be hyphenated, the macro |\bsc| (boxed small caps) does
%          this, e.g., |Leslie~\bsc{Lamport}| will produce
%          Leslie~\mbox{\textsc{Lamport}}. Note that composed names
%          (such as Dupont-Durant) may now be hyphenated on explicit
%          hyphens, this differs from |frenchb|~v.1.x.
%    \item Commands |\primo|, |\secundo|, |\tertio| and |\quarto|
%          print 1\textsuperscript{o}, 2\textsuperscript{o},
%          3\textsuperscript{o}, 4\textsuperscript{o}.
%          |\FrenchEnumerate{6}| prints 6\textsuperscript{o}.
%    \item Abbreviations for ``Num\'ero(s)'' and ``num\'ero(s)''
%          (N\textsuperscript{o} N\textsuperscript{os}
%          n\textsuperscript{o} and n\textsuperscript{os}~)
%          are obtained via the commands |\No|, |\Nos|, |\no|, |\nos|.
%    \item Two commands are provided to typeset the symbol for
%          ``degr\'e'': |\degre| prints the raw character and
%          |\degres| should be used to typeset temperatures (e.g.,
%          ``|20~\degres C|'' with an unbreakable space), or for
%          alcohols' strengths (e.g., ``|45\degres|'' with \emph{no}
%          space in French).
%    \item In math mode the comma has to be surrounded with
%          braces to avoid a spurious space being inserted after it,
%          in decimal numbers for instance (see the \TeX{}book p.~134).
%          The command |\DecimalMathComma| makes the comma be an
%          ordinary character \emph{in French only} (no space added);
%          as a counterpart, if |\DecimalMathComma| is active, an
%          explicit space has to be added in lists and intervals:
%          |$[0,\ 1]$|, |$(x,\ y)$|. |\StandardMathComma| switches back
%          to the standard behaviour of the comma.
%    \item A command |\nombre| was provided in 1.x versions to easily
%          format numbers in slices of three digits separated either
%          by a comma in English or with a space in French; |\nombre|
%          is now mapped to |\numprint| from \file{numprint.sty}, see
%          \file{numprint.pdf} for more information.
%    \item |frenchb| has been designed to take advantage of the |xspace|
%          package if present: adding |\usepackage{xspace}| in the
%          preamble will force macros like |\fg|, |\ier|, |\ieme|,
%          |\dots|, \dots, to respect the spaces you type after them,
%          for instance typing `|1\ier juin|' will print
%          `1\textsuperscript{er} juin' (no need for a forced space
%          after |1\ier|).
%    \end{enumerate}
%
%  \subsection{Customisation}
%  \label{ssec-custom}
%
%     Up to version 1.6, customisation of |frenchb| was achieved
%     by entering commands in \file{frenchb.cfg}.  This possibility
%     remains for compatibility, but \emph{should not longer be used}.
%     Version 2.0 introduces a new command |\frenchbsetup{}| using
%     the \file{keyval} syntax which should make it easier to choose
%     among the many options available. The command |\frenchbsetup{}|
%     is to appear in the preamble only (after loading \babel).
%
%     \vspace{.5\baselineskip}
%     |\frenchbsetup{ShowOptions}| prints all available options to
%     the \file{.log} file, it is just meant as a remainder of the
%     list of offered options. As usual with \file{keyval} syntax,
%     boolean options (as |ShowOptions|) can be entered as
%     |ShowOptions=true| or just |ShowOptions|, the `|=true|' part
%     can be omitted.
%
%     \vspace{.5\baselineskip}
%     The other options are listed below. Their default value is shown
%     between brackets, sometimes followed be a `\texttt{*}'.
%     The `\texttt{*}' means that the default shown applies when
%     |frenchb| is loaded as the \emph{last} option of \babel{}
%     ---\babel's \emph{main language}---, and is toggled otherwise:
%     \begin{itemize}
%     \item |StandardLayout=true [false*]| forces |frenchb| not to
%       interfere with the layout: no action on any kind of lists,
%       first paragraphs of sections are not indented (as in English),
%       no action on footnotes. This option replaces the former
%       command |\StandardLayout|.  It can be used to avoid conflicts
%       with classes or packages which customise lists or footnotes.
%     \item |GlobalLayoutFrench=false [true*]| can be used, when French
%       is the main language, to emulate what prior versions of
%       |frenchb| (pre-2.2) did: lists, and first paragraphs
%       of sections will be displayed the standard way in other
%       languages than French, and ``\`a la fran\c{c}aise'' in French.
%       Note that the layout of footnotes is language independent
%       anyway (see below |FrenchFootnotes| and |AutoSpaceFootnotes|).
%       This option replaces the former command |\FrenchLayout|.
%     \item |ReduceListSpacing=false [true*]|; |frenchb| normally
%       reduces the values of the vertical spaces used in the
%       environment |list| in French; setting this option to |false|
%       reverts to the standard settings of |list|.  This option
%       replaces the former command |\FrenchListSpacingfalse|.
%     \item |CompactItemize=false [true*]|; |frenchb| normally
%       suppresses any vertical space between items of |itemize| lists
%       in French; setting this option to |false| reverts to the
%       standard settings of |itemize| lists.  This option replaces
%       the former command |\FrenchItemizeSpacingfalse|.
%     \item |StandardItemLabels=true [false*]| when set to |true| this
%       option stops |frenchb| from changing the labels in |itemize|
%       lists in French.
%     \item |ItemLabels=\textemdash|, |\textbullet|, |\ding{43}|,
%       \dots, |[\textendash*]|; when |StandardItemLabels=false| (the
%       default), this option enables to choose the label used in
%       |itemize| lists for all levels.  The next three options do
%       the same but each one for one level only. Note that the
%       example |\ding{43}| requires |\usepackage{pifont}|.
%     \item |ItemLabeli=\textemdash|, |\textbullet|, |\ding{43}|,
%       \dots,|[\textendash*]|
%     \item |ItemLabelii=\textemdash|, |\textbullet|, |\ding{43}|,
%       \dots, |[\textendash*]|
%     \item |ItemLabeliii=\textemdash|, |\textbullet|, |\ding{43}|,
%       \dots, |[\textendash*]|
%     \item |ItemLabeliv=\textemdash|, |\textbullet|, |\ding{43}|,
%       \dots, |[\textendash*]|
%     \item |StandardLists=true [false*]| forbids |frenchb| to
%       customise any kind of list. Do activate the option
%       |StandardLists| when using classes or packages that customise
%       lists too (|enumitem|, |paralist|, \dots{}) to avoid conflicts.
%       This option is just a shorthand for |ReduceListSpacing=false|
%       and |CompactItemize=false| and |StandardItemLabels=true|.
%     \item |IndentFirst=false [true*]|; |frenchb| normally forces
%       indentation of the first paragraph of sections.
%       When this option is set to |false|, the first paragraph of
%       will look the same in French and in English (not indented).
%     \item |FrenchFootnotes=false [true*]| reverts to the standard
%       layout of footnotes. By default |frenchb| typesets leading
%       numbers as `1.\hspace{.5em}' instead of `$\hbox{}^1$', but
%       has no effect on footnotes numbered with symbols (as in the
%       |\thanks| command).  The former commands |\StandardFootnotes|
%       and |\FrenchFootnotes| are still there, |\StandardFootnotes|
%       can be useful when some footnotes are numbered with letters
%       (inside minipages for instance).
%     \item |AutoSpaceFootnotes=false [true*]| ; by default |frenchb|
%       adds a thin space in the running text before the number or
%       symbol calling the footnote.  Making this option |false|
%       reverts to the standard setting (no space added).
%     \item |FrenchSuperscripts=false [true]| ; then
%       |\up=\textsuperscript| (option added in version 2.1).
%       Should only be made |false| to recompile older documents.
%       By default |\up| now relies on |\fup| designed to produce
%       better looking superscripts.
%     \item |AutoSpacePunctuation=false [true]|; in French, the user
%       \emph{should} input a space before the four characters `|:;!?|'
%       but as many people forget about it (even among native French
%       writers!), the default behaviour of |frenchb| is to
%       automatically add a |\thinspace| before `|;|' `|!|' `|?|' and a
%       normal (unbreakable) space before~`|:|' (this is recommended by
%       the French Imprimerie nationale).  This is convenient in most
%       cases but can lead to addition of spurious spaces in URLs or in
%       MS-DOS paths but only if they are no typed using |\texttt| or
%       verbatim mode. When the current font is a monospaced
%       (typewriter) font, |AutoSpacePunctuation| is locally switched
%       to |false|, no spurious space is added in that case, so the
%       default behaviour of of |frenchb| in that area should be fine
%       in most circumstances.
%
%       Choosing |AutoSpacePunctuation=false| will ensure that
%       a proper space will be added before `|:;!?|' \emph{if and only
%       if} a (normal) space has been typed in. Those who are unsure
%       about their typing in this area should stick to the default
%       option and type |\string;| |\string:| |\string!| |\string?|
%       instead of |;| |:| |!| |?| in case an unwanted space is
%       added by |frenchb|.
%     \item |ThinColonSpace=true [false]| changes the normal
%       (unbreakable) space added before the colon `:' to a thin space,
%       so that the same amount of space is added before any of the
%       four double punctuation characters. The default setting is
%       supported by the French Imprimerie nationale.
%     \item |LowercaseSuperscripts=false [true]| ; by default |frenchb|
%       inhibits the uppercasing of superscripts (for instance when they
%       are moved to page headers). Making this option |false|
%       will disable this behaviour (not recommended).
%     \item |PartNameFull=false [true]|; when true, |frenchb| numbers
%       the title of |\part{}| commands as ``Premi\`ere partie'',
%       ``Deuxi\`eme partie'' and so on. With some classes which change
%       the|\part{}| command (AMS and SMF classes do so), you will get
%       ``Premi\`ere partie~I'', ``Deuxi\`eme partie~II'' instead;
%       when this occurs, this option should be set to |false|,
%       part titles will then be printed as ``Partie I'', ``Partie II''.
%     \item |og=|\texttt{\guillemotleft}, |fg=|\texttt{\guillemotright};
%       when guillemets characters are available on the keyboard
%       (through a compose key for instance), it is nice to use them
%       instead of typing |\og| and |\fg|. This option tells |frenchb|
%       which characters are opening and closing French guillemets
%       (they depend on the input encoding), then you can type either
%       \texttt{\guillemotleft{} guillemets \guillemotright}, or
%       \texttt{\guillemotleft{}guillemets\guillemotright} (with or
%       without spaces), to get properly typeset French quotes.
%       This option requires \file{inputenc} to be loaded with the
%       proper encoding, it works with 8-bits encodings (latin1,
%       latin9, ansinew,  applemac,\dots) and multi-byte encodings
%       (utf8 and utf8x).
%     \end{itemize}
%
%  \subsection{Hyphenation checks}
%  \label{ssec-hyphen}
%
%    Once you have built your format, a good precaution would be to
%    perform some basic tests about hyphenation in French. For
%    \LaTeXe{} I suggest this:
%    \begin{itemize}
%    \item run the following file, with the encoding suitable for
%      your machine (\textit{my-encoding} will be |latin1| for
%      \textsc{unix} machines, |ansinew| for PCs running~Windows,
%      |applemac| or |latin1| for Macintoshs, or |utf8|\dots\\[3mm]^^A\]
%      |%%% Test file for French hyphenation.|\\
%      |\documentclass{article}|\\
%      |\usepackage[|\textit{my-encoding}|]{inputenc}|\\
%      |\usepackage[T1]{fontenc} % Use LM fonts|\\
%      |\usepackage{lmodern}     % for French|\\
%      |\usepackage[frenchb]{babel}|\\
%      |\begin{document}|\\
%      |\showhyphens{signal container \'ev\'enement alg\`ebre}|\\
%      |\showhyphens{|\texttt{signal container \'ev\'enement
%                     alg\`ebre}|}|\\
%      |\end{document}|
%    \item check the hyphenations proposed by \TeX{} in your log-file;
%      in French you should get with both 7-bit and 8-bit encodings\\
%      \texttt{si-gnal contai-ner \'ev\'e-ne-ment al-g\`ebre}.\\
%      Do not care about how accented characters are displayed in the
%      log-file, what matters is the position of the `|-|' hyphen
%      signs \emph{only}.
%    \end{itemize}
%    If they are all correct, your installation (probably) works fine,
%    if one (or more) is (are) wrong, ask a local wizard to see what's
%    going wrong and perform the test again (or e-mail me about what
%    happens).\\
%    Frequent mismatches:
%    \begin{itemize}
%    \item you get |sig-nal con-tainer|, this probably means that the
%    hyphenation patterns you are using are for US-English, not for
%    French;
%    \item you get no hyphen at all in \texttt{\'ev\'e-ne-ment}, this
%    probably means that you are using CM fonts and the macro
%    |\accent| to produce accented characters.
%    Using 8-bits fonts with built-in accented characters avoids
%    this kind of mismatch.
%    \end{itemize}
%
%    \textbf{Options' order} -- Please remember that options are read
%    in the order they appear inside the |\frenchbsetup| command.
%    Someone wishing that |frenchb| leaves the layout of lists
%    and footnotes untouched but caring for indentation of first
%    paragraph of sections could choose
%    |\frenchbsetup{StandardLayout,IndentFirst}| and get the expected
%    layout. Choosing |\frenchbsetup{IndentFirst,StandardLayout}|
%    would not lead to the expected result: option |IndentFirst| would
%    be overwritten by |StandardLayout|.
%
%  \subsection{Changes}
%  \label{ssec-changes}
%
%  \subsubsection*{What's new in version 2.0?}
%
%    Here is the list of all changes:
%    \begin{itemize}
%    \item Support for \LaTeX-2.09 and for \LaTeXe{} in compatibility
%      mode has been dropped. This version is meant for \LaTeXe{} and
%      Plain based formats (like \file{bplain}). \LaTeXe{} formats
%      based on ml\TeX{} are no longer supported either (plenty of
%      good 8-bits fonts are available now, so T1 encoding should
%      be preferred for typesetting in French). A warning is issued
%      when OT1 encoding is in use at the |\begin{document}|.
%    \item Customisation should now be handled by command
%      |\frenchbsetup{}|, \file{frenchb.cfg} (kept for compatibility)
%      should no longer be used. See section~\ref{ssec-custom} for
%      the list of available options.
%    \item Captions in figures and table have changed in French: former
%      abbreviations ``Fig.'' and ``Tab.'' have been replaced by full
%      names ``Figure'' and ``Table''.  If this leads to formatting
%      problems in captions, you can add the following two commands to
%      your preamble (after loading \babel) to get the former captions\\
%      |\addto\captionsfrench{\def\figurename{{\scshape Fig.}}}|\\
%      |\addto\captionsfrench{\def\tablename{{\scshape Tab.}}}|.
%    \item The |\nombre| command is now provided by the \file{numprint}
%      package which has to be loaded \emph{after} \babel{} with the
%      option |autolanguage| if number formatting should depend on the
%      current language.
%    \item The |\bsc| command no longer uses an |\hbox| to stop
%      hyphenation of names but a |\kern0pt| instead. This change
%      enables \file{microtype} to fine tune the length of the
%      argument of |\bsc|; as a side-effect, compound names like
%      Dupont-Durand can now be hyphenated on  explicit hyphens.
%      You can get back to the former behaviour of |\bsc| by adding\\
%      |\renewcommand*{\bsc}[1]{\leavevmode\hbox{\scshape #1}}|\\
%      to the preamble of your document.
%    \item Footnotes are now displayed ``\`a la fran\c caise'' for the
%      whole document, except with an explicit\\
%      |\frenchbsetup{AutoSpaceFootnotes=false,FrenchFootnotes=false}|.\\
%      Add this command if you want standard footnotes. It is still
%      possible to revert locally to the standard layout of footnotes
%      by adding |\StandardFootnotes| (inside a |minipage| environment
%      for instance).
%    \end{itemize}
%
%  \subsubsection*{What's new in version 2.1?}
%
%      New command |\fup| to typeset better looking superscripts.
%      Former command |\up| is now defined as |\fup|, but an option
%      |\frenchbsetup{FrenchSuperscripts=false}| is provided for
%      backward compatibility.  |\fup| was designed using ideas from
%      Jacques Andr\'e, Thierry Bouche and Ren\'e Fritz, thanks to them!
%
%  \subsubsection*{What's new in version 2.2?}
%
%      Starting with version~2.2a, |frenchb| alters the layout of
%      lists, footnotes, and the indentation of first paragraphs of
%      sections) \emph{only if} French is the ``main language''
%      (i.e. babel's last language option). The layout is global for
%      the whole document: lists, etc. look the same in French and in
%      other languages, everything is typeset ``\`a la fran\c caise''
%      if French is the ``main language'', otherwise |frenchb| doesn't
%      change anything regarding lists, footnotes, and indentation of
%      paragraphs.
%
%  \subsubsection*{What's new in version 2.3?}
%
%      Starting with version~2.3a, |frenchb| no longer inserts spaces
%      automatically before `|:;!?|' when a typewriter font is in use;
%      this was suggested by Yannis Haralambous to prevent
%      spurious spaces in computer source code or expressions like
%      \texttt{C\string:/foo}, \texttt{http\string://foo.bar},
%      etc.  An option (|OriginalTypewriter|) is provided to get back
%      to the former behaviour of |frenchb|.
%
%      Another probably invisible change: lowercase conversion in
%      |\up{}| is now achieved by the \LaTeX{} command |\MakeLowercase|
%      instead of \TeX's |\lowercase| command.  This prevents error
%      messages when diacritics are used inside |\up{}| (diacritics
%      should \emph{never} be used in superscripts though!).
%
% \StopEventually{}
%
%  \subsection{File frenchb.cfg}
%  \label{sec-cfg}
%
%    \file{frenchb.cfg} is now a dummy file just kept for compatibility
%    with previous versions.
%
% \iffalse
%<*cfg>
% \fi
%    \begin{macrocode}
%%%%%%%%%%%%%%%%%%%%%%%%%%%%%%%%%%%%%%%%%%%%%%%%%%%%%%%%%%%%%%%%%%%%%%
%%%%%%%%%  WARNING: THIS  FILE SHOULD  NO  LONGER  BE  USED  %%%%%%%%%
%% If you want to customise frenchb, please DO NOT hack into the code!
%% Do no put any code in this file either, please use the new command
%% \frenchbsetup{} with the proper options to customise frenchb.
%% 
%% Add \frenchbsetup{ShowOptions} to your preamble to see the list of
%% available options and/or read the documentation.
%%%%%%%%%%%%%%%%%%%%%%%%%%%%%%%%%%%%%%%%%%%%%%%%%%%%%%%%%%%%%%%%%%%%%%
%    \end{macrocode}
% \iffalse
%</cfg>
% \fi
%
%  \section{\TeX{}nical details}
%
%  \subsection{Initial setup}
%
% \changes{v2.1d}{2008/05/02}{Argument of \cs{ProvidesLanguage} changed
%     above from `french' to `frenchb' (otherwise \cs{listfiles} prints
%     no date/version information).  The real name of current language
%     (french) as to be corrected before calling \cs{LdfInit}.}
%
% \iffalse
%<*code>
% \fi
%
%    While this file was read through the option \Lopt{frenchb} we make
%    it behave as if \Lopt{french} was specified.
%    \begin{macrocode}
\def\CurrentOption{french}
%    \end{macrocode}
%
%    The macro |\LdfInit| takes care of preventing that this file is
%    loaded more than once, checking the category code of the
%    \texttt{@} sign, etc.
%
%    \begin{macrocode}
\LdfInit\CurrentOption\datefrench
%    \end{macrocode}
%
% \changes{v2.1d}{2008/05/04}{Avoid warning ``\cs{end} occurred
%   when \cs{ifx} ... incomplete'' with LaTeX-2.09.}
%
%  \begin{macro}{\ifLaTeXe}
%    No support is provided for late \LaTeX-2.09: issue a warning
%    and exit if \LaTeX-2.09 is in use. Plain is still supported.
%    \begin{macrocode}
\newif\ifLaTeXe
\let\bbl@tempa\relax
\ifx\magnification\@undefined
   \ifx\@compatibilitytrue\@undefined
     \PackageError{frenchb.ldf}
        {LaTeX-2.09 format is no longer supported.\MessageBreak
         Aborting here}
        {Please upgrade to LaTeX2e!}
     \let\bbl@tempa\endinput
   \else
     \LaTeXetrue
   \fi
\fi
\bbl@tempa
%    \end{macrocode}
%  \end{macro}
%
%    Check if hyphenation patterns for the French language have been
%    loaded in language.dat; we allow for the names `french',
%    `francais', `canadien' or `acadian'. The latter two are both
%    names used in Canada for variants of French that are in use in
%    that country.
%
%    \begin{macrocode}
\ifx\l@french\@undefined
  \ifx\l@francais\@undefined
    \ifx\l@canadien\@undefined
      \ifx\l@acadian\@undefined
        \@nopatterns{French}
        \adddialect\l@french0
      \else
        \let\l@french\l@acadian
      \fi
    \else
      \let\l@french\l@canadien
    \fi
  \else
    \let\l@french\l@francais
  \fi
\fi
%    \end{macrocode}
%    Now |\l@french| is always defined.
%
%    The internal name for the French language is |french|;
%    |francais| and |frenchb| are synonymous for |french|:
%    first let both names use the same hyphenation patterns.
%    Later we will have to set aliases for |\captionsfrench|,
%    |\datefrench|, |\extrasfrench| and |\noextrasfrench|.
%    As French uses the standard values of |\lefthyphenmin| (2)
%    and |\righthyphenmin| (3), no special setting is required here.
%
%    \begin{macrocode}
\ifx\l@francais\@undefined
  \let\l@francais\l@french
\fi
\ifx\l@frenchb\@undefined
  \let\l@frenchb\l@french
\fi
%    \end{macrocode}
%    When this language definition file was loaded for one of the
%    Canadian versions of French we need to make sure that a suitable
%    hyphenation pattern register will be found by \TeX.
%    \begin{macrocode}
\ifx\l@canadien\@undefined
  \let\l@canadien\l@french
\fi
\ifx\l@acadian\@undefined
  \let\l@acadian\l@french
\fi
%    \end{macrocode}
%
%    This language definition can be loaded for different variants of
%    the French language. The `key' \babel\ macros are only defined
%    once, using `french' as the language name, but |frenchb| and
%    |francais| are synonymous.
%    \begin{macrocode}
\def\datefrancais{\datefrench}
\def\datefrenchb{\datefrench}
\def\extrasfrancais{\extrasfrench}
\def\extrasfrenchb{\extrasfrench}
\def\noextrasfrancais{\noextrasfrench}
\def\noextrasfrenchb{\noextrasfrench}
%    \end{macrocode}
%
% \begin{macro}{\extrasfrench}
% \begin{macro}{\noextrasfrench}
%    The macro |\extrasfrench| will perform all the extra
%    definitions needed for the French language.
%    The macro |\noextrasfrench| is used to cancel the actions of
%    |\extrasfrench|.\\
%    In French, character ``apostrophe'' is a letter in expressions
%    like |l'ambulance| (French  hyphenation patterns provide entries
%    for this kind of words).  This means that the |\lccode| of
%    ``apostrophe'' has to be non null in French for proper hyphenation
%    of those expressions, and has to be reset to null when exiting
%    French.
%
%    \begin{macrocode}
\@namedef{extras\CurrentOption}{\lccode`\'=`\'}
\@namedef{noextras\CurrentOption}{\lccode`\'=0}
%    \end{macrocode}
% \end{macro}
% \end{macro}
%
%    One more thing |\extrasfrench| needs to do is to make sure that
%    |\frenchspacing| is in effect.  |\noextrasfrench| will switch
%    |\frenchspacing| off again.
%    \begin{macrocode}
  \expandafter\addto\csname extras\CurrentOption\endcsname{%
    \bbl@frenchspacing}
  \expandafter\addto\csname noextras\CurrentOption\endcsname{%
    \bbl@nonfrenchspacing}
%    \end{macrocode}
%
%  \subsection{Punctuation}
%  \label{sec-punct}
%
%    As long as no better solution is available%
%    \footnote{Lua\TeX, or pdf\TeX{} might provide alternatives in
%       the future\dots},
%    the `double punctuation' characters (|;| |!| |?| and |:|) have to
%    be made |\active| for an automatic control of the amount of space
%    to insert before them. Before doing so, we have to save the
%    standard definition of |\@makecaption| (which includes two ':')
%    to compare it later to its definition at the |\begin{document}|.
%    \begin{macrocode}
\long\def\STD@makecaption#1#2{%
  \vskip\abovecaptionskip
  \sbox\@tempboxa{#1: #2}%
  \ifdim \wd\@tempboxa >\hsize
    #1: #2\par
  \else
    \global \@minipagefalse
    \hb@xt@\hsize{\hfil\box\@tempboxa\hfil}%
  \fi
  \vskip\belowcaptionskip}%
%    \end{macrocode}
%
%    We define a new `if' |\FBpunct@active| which will be made false
%    whenever a better alternative will be available. Another `if'
%    |\FBAutoSpacePunctuation| needs to be defined now.
%    \begin{macrocode}
\newif\ifFBpunct@active          \FBpunct@activetrue
\newif\ifFBAutoSpacePunctuation  \FBAutoSpacePunctuationtrue
%    \end{macrocode}
%    The following code makes the four characters |;| |!| |?| and |:|
%    `active' and provides their definitions.
%    \begin{macrocode}
\ifFBpunct@active
  \initiate@active@char{:}
  \initiate@active@char{;}
  \initiate@active@char{!}
  \initiate@active@char{?}
%    \end{macrocode}
%    We first tune the amount of space before \texttt{;}
%    \texttt{!}  \texttt{?} and \texttt{:}.  This should only happen
%    in horizontal mode, hence the test |\ifhmode|.
%
%    In horizontal mode, if a space has been typed before `;' we
%    remove it and put an unbreakable |\thinspace| instead. If no
%    space has been typed, we add |\FDP@thinspace| which will be
%    defined, up to the user's wishes, as an automatic added
%    thin space, or as |\@empty|.
%    \begin{macrocode}
  \declare@shorthand{french}{;}{%
      \ifhmode
      \ifdim\lastskip>\z@
          \unskip\penalty\@M\thinspace
          \else
            \FDP@thinspace
        \fi
      \fi
%    \end{macrocode}
%    Now we can insert a |;| character.
%    \begin{macrocode}
      \string;}
%    \end{macrocode}
%    The next three definitions are very similar.
%    \begin{macrocode}
  \declare@shorthand{french}{!}{%
      \ifhmode
        \ifdim\lastskip>\z@
          \unskip\penalty\@M\thinspace
        \else
          \FDP@thinspace
        \fi
      \fi
      \string!}
  \declare@shorthand{french}{?}{%
      \ifhmode
        \ifdim\lastskip>\z@
          \unskip\penalty\@M\thinspace
        \else
          \FDP@thinspace
        \fi
      \fi
      \string?}
%    \end{macrocode}
%    According to the I.N. specifications, the `:' requires a normal
%    space before it, but some people prefer a |\thinspace| (just
%    like the other three). We define |\Fcolonspace| to hold the
%    required amount of space (user customisable).
%    \begin{macrocode}
  \newcommand*{\Fcolonspace}{\space}
  \declare@shorthand{french}{:}{%
      \ifhmode
        \ifdim\lastskip>\z@
          \unskip\penalty\@M\Fcolonspace
        \else
          \FDP@colonspace
        \fi
      \fi
      \string:}
%    \end{macrocode}
%
% \changes{v2.3a}{2008/10/10}{\cs{NoAutoSpaceBeforeFDP} and
%    \cs{AutoSpaceBeforeFDP} now set the flag
%    \cs{ifFBAutoSpacePunctuation} accordingly (LaTeX only).}
%
%  \begin{macro}{\AutoSpaceBeforeFDP}
%  \begin{macro}{\NoAutoSpaceBeforeFDP}
%    |\FDP@thinspace| and |\FDP@space| are defined as unbreakable
%    spaces by |\autospace@beforeFDP| or as |\@empty| by
%    |\noautospace@beforeFDP| (internal commands), user commands
%    |\AutoSpaceBeforeFDP| and |\NoAutoSpaceBeforeFDP| do the same and
%    take care of the flag |\ifFBAutoSpacePunctuation| in \LaTeX{}.
%    Set the default now for Plain (done later for \LaTeX).
%    \begin{macrocode}
  \def\autospace@beforeFDP{%
          \def\FDP@thinspace{\penalty\@M\thinspace}%
          \def\FDP@colonspace{\penalty\@M\Fcolonspace}}
  \def\noautospace@beforeFDP{\let\FDP@thinspace\@empty
                            \let\FDP@colonspace\@empty}
  \ifLaTeXe
    \def\AutoSpaceBeforeFDP{\autospace@beforeFDP
                            \FBAutoSpacePunctuationtrue}
    \def\NoAutoSpaceBeforeFDP{\noautospace@beforeFDP
                              \FBAutoSpacePunctuationfalse}
  \else
    \let\AutoSpaceBeforeFDP\autospace@beforeFDP
    \let\NoAutoSpaceBeforeFDP\noautospace@beforeFDP
    \AutoSpaceBeforeFDP
  \fi
%    \end{macrocode}
% \end{macro}
% \end{macro}
%
% \changes{v2.3a}{2008/10/10}{In LaTeX, frenchb no longer adds spaces
%     before `double punctuation' characters in computer code.
%     Suggested by Yannis Haralambous.}
%
% \changes{v2.3c}{2009/02/07}{Commands \cs{ttfamily}, \cs{rmfamily}
%    and \cs{sffamily} have to be robust.  Bug introduced in 2.3a,
%    pointed out by Manuel P\'egouri\'e-Gonnard.}
%
%    In \LaTeXe{} |\ttfamily| (and hence |\texttt|) will be redefined
%    `AtBeginDocument' as |\ttfamilyFB| so that no space
%    is added before the four |; : ! ?| characters, even if
%    |AutoSpacePunctuation| is true.  |\rmfamily| and |\sffamily| need
%    to be redefined also (|\ttfamily| is not always used inside a
%    group, its effect can be cancelled by |\rmfamily| or |\sffamily|).
%
%    These redefinitions can be canceled if necessary, for instance to
%    recompile older documents, see option |OriginalTypewriter| below.
%    \begin{macrocode}
  \ifLaTeXe
    \let\ttfamilyORI\ttfamily
    \let\rmfamilyORI\rmfamily
    \let\sffamilyORI\sffamily
    \DeclareRobustCommand\ttfamilyFB{%
         \noautospace@beforeFDP\ttfamilyORI}%
    \DeclareRobustCommand\rmfamilyFB{%
         \ifFBAutoSpacePunctuation
            \autospace@beforeFDP\rmfamilyORI
         \else
            \noautospace@beforeFDP\rmfamilyORI
         \fi}%
    \DeclareRobustCommand\sffamilyFB{%
         \ifFBAutoSpacePunctuation
            \autospace@beforeFDP\sffamilyORI
         \else
            \noautospace@beforeFDP\sffamilyORI
         \fi}%
  \fi
%    \end{macrocode}
%
%    When the active characters appear in an environment where their
%    French behaviour is not wanted they should give an `expected'
%    result. Therefore we define shorthands at system level as well.
%    \begin{macrocode}
  \declare@shorthand{system}{:}{\string:}
  \declare@shorthand{system}{!}{\string!}
  \declare@shorthand{system}{?}{\string?}
  \declare@shorthand{system}{;}{\string;}
%    \end{macrocode}
%    We specify that the French group of shorthands should be used.
%    \begin{macrocode}
  \addto\extrasfrench{%
    \languageshorthands{french}%
%    \end{macrocode}
%    These characters are `turned on' once, later their definition may
%    vary. Don't misunderstand the following code: they keep being
%    active all along the document, even when leaving French.
%    \begin{macrocode}
    \bbl@activate{:}\bbl@activate{;}%
    \bbl@activate{!}\bbl@activate{?}%
  }
  \addto\noextrasfrench{%
  \bbl@deactivate{:}\bbl@deactivate{;}%
  \bbl@deactivate{!}\bbl@deactivate{?}}
\fi
%    \end{macrocode}
%
%  \subsection{French quotation marks}
%
%  \begin{macro}{\og}
%  \begin{macro}{\fg}
%    The top macros for quotation marks will be called |\og|
%    (``\underline{o}uvrez \underline{g}uillemets'') and |\fg|
%    (``\underline{f}ermez \underline{g}uillemets'').
%    Another option for typesetting quotes in multilingual texts
%    is to use the package |csquotes.sty| and its command |\enquote|.
%
%    \begin{macrocode}
\newcommand*{\og}{\@empty}
\newcommand*{\fg}{\@empty}
%    \end{macrocode}
%  \end{macro}
%  \end{macro}
%
%  \begin{macro}{\guillemotleft}
%  \begin{macro}{\guillemotright}
%    \LaTeX{} users are supposed to use 8-bit output encodings (T1,
%    LY1,\dots) to typeset French, those who still stick to OT1 should
%    call |aeguill.sty| or a similar package. In both cases the
%    commands |\guillemotleft| and |\guillemotright| will print the
%    French opening and closing quote characters from the output font.
%    For XeLaTeX, |\guillemotleft| and |\guillemotright| are defined
%    by package \file{xunicode.sty}.
%    We will check `AtBeginDocument' that the proper output encodings
%    are in use (see end of section~\ref{sec-keyval}).
%
%    We give the following definitions for Plain users only as a (poor)
%    fall-back, they are welcome to change them for anything better.
%    \begin{macrocode}
\ifLaTeXe
\else
  \ifx\guillemotleft\@undefined
    \def\guillemotleft{\leavevmode\raise0.25ex
                       \hbox{$\scriptscriptstyle\ll$}}
  \fi
  \ifx\guillemotright\@undefined
    \def\guillemotright{\raise0.25ex
                        \hbox{$\scriptscriptstyle\gg$}}
  \fi
  \let\xspace\relax
\fi
%    \end{macrocode}
%  \end{macro}
%  \end{macro}
%
%    The next step is to provide correct spacing after |\guillemotleft|
%    and before |\guillemotright|: a space precedes and follows
%    quotation marks but no line break is allowed neither \emph{after}
%    the opening one, nor \emph{before} the closing one.
%    |\FBguill@spacing| which does the spacing, has been fine tuned by
%    Thierry Bouche.  French quotes (including spacing) are printed by
%    |\FB@og| and |\FB@fg|, the expansion of the top level commands
%    |\og| and |\og| is different in and outside French.
%    We'll try to be smart to users of David~Carlisle's |xspace|
%    package: if this package is loaded there will be no need for |{}|
%    or |\ | to get a space after |\fg|, otherwise |\xspace| will be
%    defined as |\relax| (done at the end of this file).
%
%    \begin{macrocode}
\newcommand*{\FBguill@spacing}{\penalty\@M\hskip.8\fontdimen2\font
                                            plus.3\fontdimen3\font
                                           minus.8\fontdimen4\font}
\DeclareRobustCommand*{\FB@og}{\leavevmode
                               \guillemotleft\FBguill@spacing}
\DeclareRobustCommand*{\FB@fg}{\ifdim\lastskip>\z@\unskip\fi
                               \FBguill@spacing\guillemotright\xspace}
%    \end{macrocode}
%
%    The top level definitions for French quotation marks are switched
%    on and off through the |\extrasfrench| |\noextrasfrench|
%    mechanism. Outside French, |\og| and |\fg| will typeset standard
%    English opening and closing double quotes.
%
%    \begin{macrocode}
\ifLaTeXe
  \def\bbl@frenchguillemets{\renewcommand*{\og}{\FB@og}%
                            \renewcommand*{\fg}{\FB@fg}}
  \def\bbl@nonfrenchguillemets{\renewcommand*{\og}{\textquotedblleft}%
            \renewcommand*{\fg}{\ifdim\lastskip>\z@\unskip\fi
                                   \textquotedblright}}
\else
   \def\bbl@frenchguillemets{\let\og\FB@og
                             \let\fg\FB@fg}
   \def\bbl@nonfrenchguillemets{\def\og{``}%
                     \def\fg{\ifdim\lastskip>\z@\unskip\fi ''}}
\fi
\expandafter\addto\csname extras\CurrentOption\endcsname{%
  \bbl@frenchguillemets}
\expandafter\addto\csname noextras\CurrentOption\endcsname{%
  \bbl@nonfrenchguillemets}
%    \end{macrocode}
%
%  \subsection{Date in French}
%
% \begin{macro}{\datefrench}
%    The macro |\datefrench| redefines the command |\today| to
%    produce French dates.
%
% \changes{v2.0}{2006/11/06}{2 '\cs{relax}' added in
%    \cs{today}'s definition.}
%
% \changes{v2.1a}{2008/03/25}{\cs{today} changed (correction in 2.0
%    was wrong: \cs{today} was printed without spaces in toc).}
%
%    \begin{macrocode}
\@namedef{date\CurrentOption}{%
  \def\today{{\number\day}\ifnum1=\day {\ier}\fi \space
    \ifcase\month
      \or janvier\or f\'evrier\or mars\or avril\or mai\or juin\or
      juillet\or ao\^ut\or septembre\or octobre\or novembre\or
      d\'ecembre\fi
    \space \number\year}}
%    \end{macrocode}
% \end{macro}
%
%  \subsection{Extra utilities}
%
%    Let's provide the French user with some extra utilities.
%
% \changes{v2.1a}{2008/03/24}{Command \cs{fup} added to produce
%    better superscripts than \cs{textsuperscript}.}
%
%  \begin{macro}{\up}
%
% \changes{v2.1c}{2008/04/29}{Provide a temporary definition
%    (hyperref safe) of \cs{up} in case it has to be expanded in
%    the preamble (by beamer's \cs{title} command for instance).}
%
%  \begin{macro}{\fup}
%
% \changes{v2.1b}{2008/04/02}{Command \cs{fup} changed to use
%    real superscripts from fourier v. 1.6.}
%
% \changes{v2.2a}{2008/05/08}{\cs{newif} and \cs{newdimen} moved
%    before \cs{ifLaTeXe} to avoid an error with plainTeX.}
%
% \changes{v2.3a}{2008/09/30}{\cs{lowercase} changed to
%    \cs{MakeLowercase} as the former doesn't work for non ASCII
%    characters in encodings like applemac, utf-8,\dots}
%
%    |\up| eases the typesetting of superscripts like
%    `1\textsuperscript{er}'.  Up to version 2.0 of |frenchb| |\up| was
%    just a shortcut for |\textsuperscript| in \LaTeXe, but several
%    users complained that |\textsuperscript| typesets superscripts
%    too high and too big, so we now define |\fup| as an attempt to
%    produce better looking superscripts.  |\up| is defined as |\fup|
%    but can be redefined by |\frenchbsetup{FrenchSuperscripts=false}|
%    as |\textsuperscript| for compatibility with previous versions.
%
%    When a font has built-in superscripts, the best thing to do is
%    to just use them, otherwise |\fup| has to simulate superscripts
%    by scaling and raising ordinary letters.  Scaling is done using
%    package \file{scalefnt} which will be loaded at the end of
%    \babel's loading (|frenchb| being an option of babel, it cannot
%    load a package while being read).
%
%    \begin{macrocode}
\newif\ifFB@poorman
\newdimen\FB@Mht
\ifLaTeXe
  \AtEndOfPackage{\RequirePackage{scalefnt}}
%    \end{macrocode}
%    |\FB@up@fake| holds the definition of fake superscripts.
%    The scaling ratio is 0.65, raising is computed to put the top of
%    lower case letters (like `m') just under the top  of upper case
%    letters (like `M'), precisely 12\% down.  The chosen settings
%    look correct for most fonts, but can be tuned by the end-user
%    if necessary by changing |\FBsupR| and |\FBsupS| commands.
%
%    |\FB@lc| is defined as |\MakeLowercase| to inhibit the uppercasing
%    of superscripts (this may happen in page headers with the standard
%    classes but is wrong); |\FB@lc| can be redefined to do nothing
%    by option |LowercaseSuperscripts=false| of |\frenchbsetup{}|.
%    \begin{macrocode}
  \newcommand*{\FBsupR}{-0.12}
  \newcommand*{\FBsupS}{0.65}
  \newcommand*{\FB@lc}[1]{\MakeLowercase{#1}}
  \DeclareRobustCommand*{\FB@up@fake}[1]{%
    \settoheight{\FB@Mht}{M}%
    \addtolength{\FB@Mht}{\FBsupR \FB@Mht}%
    \addtolength{\FB@Mht}{-\FBsupS ex}%
    \raisebox{\FB@Mht}{\scalefont{\FBsupS}{\FB@lc{#1}}}%
    }
%    \end{macrocode}
%    The only packages I currently know to take advantage of real
%    superscripts are a) \file{xltxtra} used in conjunction with
%    XeLaTeX and OpenType fonts having the font feature
%    'VerticalPosition=Superior' (\file{xltxtra} defines
%    |\realsuperscript| and |\fakesuperscript|) and b) \file{fourier}
%    (from version 1.6) when Expert Utopia fonts are available.
%
%    |\FB@up| checks whether the current font is a Type1 `Expert'
%    (or `Pro') font with real superscripts or not (the code works
%    currently only with \file{fourier-1.6} but could work with any
%    Expert Type1 font with built-in superscripts, see below), and
%    decides to use real or fake superscripts.
%    It works as follows: the content of |\f@family| (family name of
%    the current font) is split by |\FB@split| into two pieces, the
%    first three characters (`|fut|' for Fourier, `|ppl|' for Adobe's
%    Palatino, \dots) stored in |\FB@firstthree| and the rest stored
%    in |\FB@suffix| which is expected to be `|x|' or `|j|' for expert
%    fonts.
%    \begin{macrocode}
  \def\FB@split#1#2#3#4\@nil{\def\FB@firstthree{#1#2#3}%
                             \def\FB@suffix{#4}}
  \def\FB@x{x}
  \def\FB@j{j}
  \DeclareRobustCommand*{\FB@up}[1]{%
    \bgroup \FB@poormantrue
      \expandafter\FB@split\f@family\@nil
%    \end{macrocode}
%    Then |\FB@up| looks for a \file{.fd} file named \file{t1fut-sup.fd}
%    (Fourier) or \file{t1ppl-sup.fd} (Palatino), etc. supposed to
%    define the subfamily (|fut-sup| or |ppl-sup|, etc.) giving access
%    to the built-in superscripts.  If the \file{.fd} file is not found
%    by |\IfFileExists|, |\FB@up| falls back on fake superscripts,
%    otherwise |\FB@suffix| is checked to decide whether to use fake or
%    real superscripts.
%    \begin{macrocode}
      \edef\reserved@a{\lowercase{%
         \noexpand\IfFileExists{\f@encoding\FB@firstthree -sup.fd}}}%
      \reserved@a
        {\ifx\FB@suffix\FB@x \FB@poormanfalse\fi
         \ifx\FB@suffix\FB@j \FB@poormanfalse\fi
         \ifFB@poorman \FB@up@fake{#1}%
         \else         \FB@up@real{#1}%
         \fi}%
        {\FB@up@fake{#1}}%
    \egroup}
%    \end{macrocode}
%    |\FB@up@real| just picks up the superscripts from the subfamily
%    (and forces lowercase).
%    \begin{macrocode}
  \newcommand*{\FB@up@real}[1]{\bgroup
       \fontfamily{\FB@firstthree -sup}\selectfont \FB@lc{#1}\egroup}
%    \end{macrocode}
%    |\fup| is now defined as |\FB@up| unless |\realsuperscript| is
%    defined (occurs with XeLaTeX calling \file{xltxtra.sty}).
%    \begin{macrocode}
  \DeclareRobustCommand*{\fup}[1]{%
    \@ifundefined{realsuperscript}%
      {\FB@up{#1}}%
      {\bgroup\let\fakesuperscript\FB@up@fake
            \realsuperscript{\FB@lc{#1}}\egroup}}
%    \end{macrocode}
%    Temporary definition of |up| (redefined `AtBeginDocument').
%    \begin{macrocode}
  \newcommand*{\up}{\relax}
%    \end{macrocode}
%    Poor man's definition of |\up| for Plain. In \LaTeXe,
%    |\up| will be defined as |\fup| or |\textsuperscript| later on
%    while processing the options of |\frenchbsetup{}|.
%    \begin{macrocode}
\else
  \newcommand*{\up}[1]{\leavevmode\raise1ex\hbox{\sevenrm #1}}
\fi
%    \end{macrocode}
%  \end{macro}
%  \end{macro}
%
%  \begin{macro}{\ieme}
%  \begin{macro}{\ier}
%  \begin{macro}{\iere}
%  \begin{macro}{\iemes}
%  \begin{macro}{\iers}
%  \begin{macro}{\ieres}
%  Some handy macros for those who don't know how to abbreviate ordinals:
%    \begin{macrocode}
\def\ieme{\up{\lowercase{e}}\xspace}
\def\iemes{\up{\lowercase{es}}\xspace}
\def\ier{\up{\lowercase{er}}\xspace}
\def\iers{\up{\lowercase{ers}}\xspace}
\def\iere{\up{\lowercase{re}}\xspace}
\def\ieres{\up{\lowercase{res}}\xspace}
%    \end{macrocode}
%  \end{macro}
%  \end{macro}
%  \end{macro}
%  \end{macro}
%  \end{macro}
%  \end{macro}
%
% \changes{v2.1c}{2008/04/29}{Added commands \cs{Nos} and \cs{nos}.}
%
%  \begin{macro}{\No}
%  \begin{macro}{\no}
%  \begin{macro}{\Nos}
%  \begin{macro}{\nos}
%  \begin{macro}{\primo}
%  \begin{macro}{\fprimo)}
%    And some more macros relying on |\up| for numbering,
%    first two support macros.
%    \begin{macrocode}
\newcommand*{\FrenchEnumerate}[1]{%
                       #1\up{\lowercase{o}}\kern+.3em}
\newcommand*{\FrenchPopularEnumerate}[1]{%
                       #1\up{\lowercase{o}})\kern+.3em}
%    \end{macrocode}
%
%    Typing |\primo| should result in `$1^{\rm o}$\kern+.3em',
%    \begin{macrocode}
\def\primo{\FrenchEnumerate1}
\def\secundo{\FrenchEnumerate2}
\def\tertio{\FrenchEnumerate3}
\def\quarto{\FrenchEnumerate4}
%    \end{macrocode}
%    while typing |\fprimo)| gives `1$^{\rm o}$)\kern+.3em.
%    \begin{macrocode}
\def\fprimo){\FrenchPopularEnumerate1}
\def\fsecundo){\FrenchPopularEnumerate2}
\def\ftertio){\FrenchPopularEnumerate3}
\def\fquarto){\FrenchPopularEnumerate4}
%    \end{macrocode}
%
%    Let's provide four macros for the common abbreviations
%    of ``Num\'ero''.
%    \begin{macrocode}
\DeclareRobustCommand*{\No}{N\up{\lowercase{o}}\kern+.2em}
\DeclareRobustCommand*{\no}{n\up{\lowercase{o}}\kern+.2em}
\DeclareRobustCommand*{\Nos}{N\up{\lowercase{os}}\kern+.2em}
\DeclareRobustCommand*{\nos}{n\up{\lowercase{os}}\kern+.2em}
%    \end{macrocode}
%  \end{macro}
%  \end{macro}
%  \end{macro}
%  \end{macro}
%  \end{macro}
%  \end{macro}
%
%  \begin{macro}{\bsc}
%    As family names should be written in small capitals and never be
%    hyphenated, we provide a command (its name comes from Boxed Small
%    Caps) to input them easily.  Note that this command has changed
%    with version~2 of |frenchb|: a |\kern0pt| is used instead of |\hbox|
%    because |\hbox| would break microtype's font expansion; as a
%    (positive?) side effect, composed names (such as Dupont-Durand)
%    can now be hyphenated on explicit hyphens.
%    Usage: |Jean~\bsc{Duchemin}|.
%
% \changes{v2.0}{2006/11/06}{\cs{hbox} dropped, replaced by
%    \cs{kern0pt}.}
%
%    \begin{macrocode}
\DeclareRobustCommand*{\bsc}[1]{\leavevmode\begingroup\kern0pt
                                           \scshape #1\endgroup}
\ifLaTeXe\else\let\scshape\relax\fi
%    \end{macrocode}
%  \end{macro}
%
%    Some definitions for special characters.  We won't define |\tilde|
%    as a Text Symbol not to conflict with the macro |\tilde| for math
%    mode and use the name |\tild| instead. Note that |\boi| may
%    \emph{not} be used in math mode, its name in math mode is
%    |\backslash|.  |\degre|  can be accessed by the command |\r{}|
%    for ring accent.
%
%    \begin{macrocode}
\ifLaTeXe
  \DeclareTextSymbol{\at}{T1}{64}
  \DeclareTextSymbol{\circonflexe}{T1}{94}
  \DeclareTextSymbol{\tild}{T1}{126}
  \DeclareTextSymbolDefault{\at}{T1}
  \DeclareTextSymbolDefault{\circonflexe}{T1}
  \DeclareTextSymbolDefault{\tild}{T1}
  \DeclareRobustCommand*{\boi}{\textbackslash}
  \DeclareRobustCommand*{\degre}{\r{}}
\else
  \def\T@one{T1}
  \ifx\f@encoding\T@one
    \newcommand*{\degre}{\char6}
  \else
    \newcommand*{\degre}{\char23}
  \fi
  \newcommand*{\at}{\char64}
  \newcommand*{\circonflexe}{\char94}
  \newcommand*{\tild}{\char126}
  \newcommand*{\boi}{$\backslash$}
\fi
%    \end{macrocode}
%
%  \begin{macro}{\degres}
%    We now define a macro |\degres| for typesetting the abbreviation
%    for `degrees' (as in `degrees Celsius'). As the bounding box of
%    the character `degree' has \emph{very} different widths in CM/EC
%    and PostScript fonts, we fix the width of the bounding box of
%    |\degres| to 0.3\,em, this lets the symbol `degree' stick to the
%    preceding (e.g., |45\degres|) or following character
%    (e.g., |20~\degres C|).
%
%    If the \TeX{} Companion fonts are available (\file{textcomp.sty}),
%    we pick up |\textdegree| from them instead of using emulating
%    `degrees' from the |\r{}| accent. Otherwise we overwrite the
%    (poor) definition of |\textdegree| given in \file{latin1.def},
%    \file{applemac.def} etc. (called by  \file{inputenc.sty}) by
%    our definition of |\degres|. We also advice the user (once only)
%    to use TS1-encoding.
%
% \changes{v2.1c}{2008/04/29}{Provide a temporary definition (hyperref
%    safe) of \cs{degres} in case it has to be expanded in the preamble
%    (by beamer's \cs{title} command for instance).}
%
%    \begin{macrocode}
\ifLaTeXe
  \newcommand*{\degres}{\degre}
  \def\Warning@degree@TSone{%
        \PackageWarning{frenchb.ldf}{%
           Degrees would look better in TS1-encoding:
           \MessageBreak add \protect
           \usepackage{textcomp} to the preamble.
           \MessageBreak Degrees used}}
  \AtBeginDocument{\expandafter\ifx\csname M@TS1\endcsname\relax
                     \DeclareRobustCommand*{\degres}{%
                       \leavevmode\hbox to 0.3em{\hss\degre\hss}%
                       \Warning@degree@TSone
                       \global\let\Warning@degree@TSone\relax}%
                      \let\textdegree\degres
                   \else
                     \DeclareRobustCommand*{\degres}{%
                         \hbox{\UseTextSymbol{TS1}{\textdegree}}}%
                   \fi}
\else
  \newcommand*{\degres}{%
    \leavevmode\hbox to 0.3em{\hss\degre\hss}}
\fi
%    \end{macrocode}
%  \end{macro}
%
%  \subsection{Formatting numbers}
%  \label{sec-numbers}
%
%  \begin{macro}{\DecimalMathComma}
%  \begin{macro}{\StandardMathComma}
%    As mentioned in the \TeX{}book p.~134, the comma is of type
%    |\mathpunct| in math mode: it is automatically followed by a
%    space. This is convenient in lists and intervals but
%    unpleasant when the comma is used as a decimal separator
%    in French: it has to be entered as |{,}|.
%    |\DecimalMathComma| makes the comma be an ordinary character
%    (of type |\mathord|) in French \emph{only} (no space added);
%    |\StandardMathComma| switches back to the standard behaviour
%    of the comma.
%    \begin{macrocode}
\newcount\std@mcc
\newcount\dec@mcc
\std@mcc=\mathcode`\,
\dec@mcc=\std@mcc
\@tempcnta=\std@mcc
\divide\@tempcnta by "1000
\multiply\@tempcnta by "1000
\advance\dec@mcc by -\@tempcnta
\newcommand*{\DecimalMathComma}{\iflanguage{french}%
                                 {\mathcode`\,=\dec@mcc}{}%
              \addto\extrasfrench{\mathcode`\,=\dec@mcc}}
\newcommand*{\StandardMathComma}{\mathcode`\,=\std@mcc
             \addto\extrasfrench{\mathcode`\,=\std@mcc}}
\expandafter\addto\csname noextras\CurrentOption\endcsname{%
   \mathcode`\,=\std@mcc}
%    \end{macrocode}
%  \end{macro}
%  \end{macro}
%
%  \begin{macro}{\nombre}
%
% \changes{v2.0}{2006/11/06}{\cs{nombre} requires now numprint.sty.}
%
%    The command |\nombre| is now borrowed from |numprint.sty| for
%    \LaTeXe.  There is no point to maintain the former tricky code
%    when a package is dedicated to do the same job and more.
%    For Plain based formats, |\nombre| no longer formats numbers,
%    it prints them as is and issues a warning about the change.
%
%    Fake command |\nombre| for Plain based formats, warning users of
%    |frenchb| v.1.x. of the change.
%    \begin{macrocode}
\newcommand*{\nombre}[1]{{#1}\message{%
     *** \noexpand\nombre no longer formats numbers\string! ***}}%
%    \end{macrocode}
%  \end{macro}
%
%    The next definitions only make sense for \LaTeXe. Let's cleanup
%    and exit if the format in Plain based.
%
%    \begin{macrocode}
\let\FBstop@here\relax
\def\FBclean@on@exit{\let\ifLaTeXe\@undefined
                     \let\LaTeXetrue\@undefined
                     \let\LaTeXefalse\@undefined}
\ifx\magnification\@undefined
\else
   \def\FBstop@here{\let\STD@makecaption\relax
                    \FBclean@on@exit
                    \ldf@quit\CurrentOption\endinput}
\fi
\FBstop@here
%    \end{macrocode}
%
%    What follows now is for \LaTeXe{} \emph{only}.
%    We redefine |\nombre| for \LaTeXe. A warning is issued
%    at the first call of |\nombre| if |\numprint| is not
%    defined, suggesting what to do.  The package |numprint|
%    is \emph{not} loaded automatically by |frenchb| because of
%    possible options conflict.
%
%    \begin{macrocode}
\renewcommand*{\nombre}[1]{\Warning@nombre\numprint{#1}}
\newcommand*{\Warning@nombre}{%
   \@ifundefined{numprint}%
      {\PackageWarning{frenchb.ldf}{%
         \protect\nombre\space now relies on package numprint.sty,
         \MessageBreak add \protect
         \usepackage[autolanguage]{numprint}\MessageBreak
         to your preamble *after* loading babel, \MessageBreak
         see file numprint.pdf for other options.\MessageBreak
         \protect\nombre\space called}%
       \global\let\Warning@nombre\relax
       \global\let\numprint\relax
      }{}%
}
%    \end{macrocode}
%
% \changes{v2.0c}{2007/06/25}{There is no need to define here
%    numprint's command \cs{npstylefrench}, it will be redefined
%    `AtBeginDocument' by \cs{FBprocess@options}.}
%
% \changes{v2.0c}{2007/06/25}{\cs{ThinSpaceInFrenchNumbers} added
%     for compatibility with frenchb-1.x.}
%
%    \begin{macrocode}
\newcommand*{\ThinSpaceInFrenchNumbers}{%
   \PackageWarning{frenchb.ldf}{%
         Type \protect\frenchbsetup{ThinSpaceInFrenchNumbers}
         \MessageBreak Command \protect\ThinSpaceInFrenchNumbers\space
         is no longer\MessageBreak  defined in frenchb v.2,}}
%    \end{macrocode}
%
%  \subsection{Caption names}
%
%    The next step consists of defining the French equivalents for
%    the \LaTeX{} caption names.
%
% \begin{macro}{\captionsfrench}
%    Let's first define  |\captionsfrench| which sets all strings used
%    in the four standard document classes provided with \LaTeX.
%
% \changes{v2.0}{2006/11/06}{`Fig.' changed to `Figure' and
%     `Tab.' to `Table'.}
%
% \changes{v2.0}{2006/12/15}{Set \cs{CaptionSeparator} in
%     \cs{extrasfrench} now instead of \cs{captionsfrench}
%     because it has to be reset when leaving French.}
%
%    \begin{macrocode}
\@namedef{captions\CurrentOption}{%
   \def\refname{R\'ef\'erences}%
   \def\abstractname{R\'esum\'e}%
   \def\bibname{Bibliographie}%
   \def\prefacename{Pr\'eface}%
   \def\chaptername{Chapitre}%
   \def\appendixname{Annexe}%
   \def\contentsname{Table des mati\`eres}%
   \def\listfigurename{Table des figures}%
   \def\listtablename{Liste des tableaux}%
   \def\indexname{Index}%
   \def\figurename{{\scshape Figure}}%
   \def\tablename{{\scshape Table}}%
%    \end{macrocode}
%   ``Premi\`ere partie'' instead of ``Part I''.
%    \begin{macrocode}
   \def\partname{\protect\@Fpt partie}%
   \def\@Fpt{{\ifcase\value{part}\or Premi\`ere\or Deuxi\`eme\or
   Troisi\`eme\or Quatri\`eme\or Cinqui\`eme\or Sixi\`eme\or
   Septi\`eme\or Huiti\`eme\or Neuvi\`eme\or Dixi\`eme\or Onzi\`eme\or
   Douzi\`eme\or Treizi\`eme\or Quatorzi\`eme\or Quinzi\`eme\or
   Seizi\`eme\or Dix-septi\`eme\or Dix-huiti\`eme\or Dix-neuvi\`eme\or
   Vingti\`eme\fi}\space\def\thepart{}}%
   \def\pagename{page}%
   \def\seename{{\emph{voir}}}%
   \def\alsoname{{\emph{voir aussi}}}%
   \def\enclname{P.~J. }%
   \def\ccname{Copie \`a }%
   \def\headtoname{}%
   \def\proofname{D\'emonstration}%
   \def\glossaryname{Glossaire}%
   }
%    \end{macrocode}
% \end{macro}
%
%    As some users who choose |frenchb| or |francais| as option of
%    \babel, might customise |\captionsfrenchb| or |\captionsfrancais|
%    in the preamble, we merge their changes at the |\begin{document}|
%    when they do so.
%    The other variants of French (canadien, acadian) are defined by
%    checking if the relevant option was used and then adding one extra
%    level of expansion.
%
%    \begin{macrocode}
\AtBeginDocument{\let\captions@French\captionsfrench
                 \@ifundefined{captionsfrenchb}%
                    {\let\captions@Frenchb\relax}%
                    {\let\captions@Frenchb\captionsfrenchb}%
                 \@ifundefined{captionsfrancais}%
                    {\let\captions@Francais\relax}%
                    {\let\captions@Francais\captionsfrancais}%
                 \def\captionsfrench{\captions@French
                        \captions@Francais\captions@Frenchb}%
                 \def\captionsfrancais{\captionsfrench}%
                 \def\captionsfrenchb{\captionsfrench}%
                 \iflanguage{french}{\captionsfrench}{}%
                }
\@ifpackagewith{babel}{canadien}{%
  \def\captionscanadien{\captionsfrench}%
  \def\datecanadien{\datefrench}%
  \def\extrascanadien{\extrasfrench}%
  \def\noextrascanadien{\noextrasfrench}%
  }{}
\@ifpackagewith{babel}{acadian}{%
  \def\captionsacadian{\captionsfrench}%
  \def\dateacadian{\datefrench}%
  \def\extrasacadian{\extrasfrench}%
  \def\noextrasacadian{\noextrasfrench}%
  }{}
%    \end{macrocode}
%
% \begin{macro}{\CaptionSeparator}
%    Let's consider now captions in figures and tables.
%    In French, captions in figures and tables should be printed with
%    endash (`--') instead of the standard `:'.
%
%    The standard definition of |\@makecaption| (e.g., the one provided
%    in article.cls, report.cls, book.cls which is frozen for \LaTeXe{}
%    according to Frank Mittelbach), has been saved in
%    |\STD@makecaption| before making `:' active
%    (see section~\ref{sec-punct}). `AtBeginDocument' we compare it to
%    its current definition (some classes like koma-script classes,
%    AMS classes, ua-thesis.cls\dots change it).
%    If they are identical, |frenchb| just adds a hook called
%    |\CaptionSeparator| to |\@makecaption|, |\CaptionSeparator|
%    defaults to `: ' as in the standard |\@makecaption|, and will be
%    changed to ` -- ' in French.
%    If the definitions differ, |frenchb| doesn't overwrite the changes,
%    but prints a message in the .log file.
%
%    \begin{macrocode}
\def\CaptionSeparator{\string:\space}
\long\def\FB@makecaption#1#2{%
  \vskip\abovecaptionskip
  \sbox\@tempboxa{#1\CaptionSeparator #2}%
  \ifdim \wd\@tempboxa >\hsize
    #1\CaptionSeparator #2\par
  \else
    \global \@minipagefalse
    \hb@xt@\hsize{\hfil\box\@tempboxa\hfil}%
  \fi
  \vskip\belowcaptionskip}
\AtBeginDocument{%
  \ifx\@makecaption\STD@makecaption
      \global\let\@makecaption\FB@makecaption
  \else
    \@ifundefined{@makecaption}{}%
       {\PackageWarning{frenchb.ldf}%
        {The definition of \protect\@makecaption\space
         has been changed,\MessageBreak
         frenchb will NOT customise it;\MessageBreak reported}%
       }%
  \fi
  \let\FB@makecaption\relax
  \let\STD@makecaption\relax
}
\expandafter\addto\csname extras\CurrentOption\endcsname{%
   \def\CaptionSeparator{\space\textendash\space}}
\expandafter\addto\csname noextras\CurrentOption\endcsname{%
    \def\CaptionSeparator{\string:\space}}
%    \end{macrocode}
% \end{macro}
%
%  \subsection{French lists}
%  \label{sec-lists}
%
%  \begin{macro}{\listFB}
%  \begin{macro}{\listORI}
%    Vertical spacing in general lists should be shorter in French
%    texts than the defaults provided by \LaTeX.
%    Note that the easy way, just changing values of vertical spacing
%    parameters when entering French and restoring them to their
%    defaults on exit would not work; as most lists are based on
%    |\list| we will define a variant of |\list| (|\listFB|) to
%    be used in French.
%
%    The amount of vertical space before and after a list is given by
%    |\topsep| + |\parskip| (+ |\partopsep| if the list starts a new
%    paragraph). IMHO, |\parskip| should be added \emph{only} when
%    the list starts a new paragraph, so I subtract |\parskip| from
%    |\topsep| and add it back to |\partopsep|; this will normally
%    make no difference because |\parskip|'s default value is 0pt, but
%    will be noticeable when |\parskip| is \emph{not} null.
%
%    |\endlist| is not redefined, but |\endlistORI| is provided for
%    the users who prefer to define their own lists from the original
%    command, they can code: |\begin{listORI}{}{} \end{listORI}|.
%    \begin{macrocode}
\let\listORI\list
\let\endlistORI\endlist
\def\FB@listsettings{%
      \setlength{\itemsep}{0.4ex plus 0.2ex minus 0.2ex}%
      \setlength{\parsep}{0.4ex plus 0.2ex minus 0.2ex}%
      \setlength{\topsep}{0.8ex plus 0.4ex minus 0.4ex}%
      \setlength{\partopsep}{0.4ex plus 0.2ex minus 0.2ex}%
%    \end{macrocode}
%    |\parskip| is of type `skip', its mean value only (\emph{not
%    the glue}) should be subtracted from |\topsep| and added to
%    |\partopsep|, so convert |\parskip| to a `dimen' using
%    |\@tempdima|.
%    \begin{macrocode}
      \@tempdima=\parskip
      \addtolength{\topsep}{-\@tempdima}%
      \addtolength{\partopsep}{\@tempdima}}%
\def\listFB#1#2{\listORI{#1}{\FB@listsettings #2}}%
\let\endlistFB\endlist
%    \end{macrocode}
%  \end{macro}
%  \end{macro}
%
%  \begin{macro}{\itemizeFB}
%  \begin{macro}{\itemizeORI}
%  \begin{macro}{\bbl@frenchlabelitems}
%  \begin{macro}{\bbl@nonfrenchlabelitems}
%    Let's now consider French itemize lists.  They differ from those
%    provided by the standard \LaTeXe{} classes:
%    \begin{itemize}
%      \item vertical spacing between items, before and after
%         the list, should be \emph{null} with \emph{no glue} added;
%      \item the item labels of a first level list should be vertically
%          aligned on the paragraph's first character (i.e. at
%          |\parindent| from the left margin);
%      \item the `\textbullet' is never used in French itemize-lists,
%          a long dash `--' is preferred for all levels. The item label
%          used in French is stored in |\FrenchLabelItem}|, it defaults
%          to `--' and can be changed using |\frenchbsetup{}| (see
%          section~\ref{sec-keyval}).
%    \end{itemize}
%
%    \begin{macrocode}
\newcommand*{\FrenchLabelItem}{\textendash}
\newcommand*{\Frlabelitemi}{\FrenchLabelItem}
\newcommand*{\Frlabelitemii}{\FrenchLabelItem}
\newcommand*{\Frlabelitemiii}{\FrenchLabelItem}
\newcommand*{\Frlabelitemiv}{\FrenchLabelItem}
%    \end{macrocode}
%    |\bbl@frenchlabelitems| saves current itemize labels and changes
%    them to their value in French. This code should never be executed
%    twice in a row, so we need a new flag that will be set and reset
%    by |\bbl@nonfrenchlabelitems| and |\bbl@frenchlabelitems|.
%    \begin{macrocode}
\newif\ifFB@enterFrench  \FB@enterFrenchtrue
\def\bbl@frenchlabelitems{%
  \ifFB@enterFrench
    \let\@ltiORI\labelitemi
    \let\@ltiiORI\labelitemii
    \let\@ltiiiORI\labelitemiii
    \let\@ltivORI\labelitemiv
    \let\labelitemi\Frlabelitemi
    \let\labelitemii\Frlabelitemii
    \let\labelitemiii\Frlabelitemiii
    \let\labelitemiv\Frlabelitemiv
    \FB@enterFrenchfalse
  \fi
}
\let\itemizeORI\itemize
\let\enditemizeORI\enditemize
\let\enditemizeFB\enditemize
\def\itemizeFB{%
    \ifnum \@itemdepth >\thr@@\@toodeep\else
      \advance\@itemdepth\@ne
      \edef\@itemitem{labelitem\romannumeral\the\@itemdepth}%
      \expandafter
      \listORI
      \csname\@itemitem\endcsname
      {\settowidth{\labelwidth}{\csname\@itemitem\endcsname}%
       \setlength{\leftmargin}{\labelwidth}%
       \addtolength{\leftmargin}{\labelsep}%
       \ifnum\@listdepth=0
         \setlength{\itemindent}{\parindent}%
       \else
         \addtolength{\leftmargin}{\parindent}%
       \fi
       \setlength{\itemsep}{\z@}%
       \setlength{\parsep}{\z@}%
       \setlength{\topsep}{\z@}%
       \setlength{\partopsep}{\z@}%
%    \end{macrocode}
%    |\parskip| is of type `skip', its mean value only (\emph{not
%    the glue}) should be subtracted from |\topsep| and added to
%    |\partopsep|, so convert |\parskip| to a `dimen' using
%    |\@tempdima|.
%    \begin{macrocode}
       \@tempdima=\parskip
       \addtolength{\topsep}{-\@tempdima}%
       \addtolength{\partopsep}{\@tempdima}}%
    \fi}
%    \end{macrocode}
%    The user's changes in labelitems are saved when leaving French for
%    further use when switching back to French.  This code should never
%    be executed twice in a row (toggle with |\bbl@frenchlabelitems|).
%    \begin{macrocode}
\def\bbl@nonfrenchlabelitems{%
  \ifFB@enterFrench
  \else
      \let\Frlabelitemi\labelitemi
      \let\Frlabelitemii\labelitemii
      \let\Frlabelitemiii\labelitemiii
      \let\Frlabelitemiv\labelitemiv
      \let\labelitemi\@ltiORI
      \let\labelitemii\@ltiiORI
      \let\labelitemiii\@ltiiiORI
      \let\labelitemiv\@ltivORI
      \FB@enterFrenchtrue
  \fi
}
%    \end{macrocode}
%  \end{macro}
%  \end{macro}
%  \end{macro}
%  \end{macro}
%
%  \subsection{French indentation of sections}
%  \label{sec-indent}
%
%  \begin{macro}{\bbl@frenchindent}
%  \begin{macro}{\bbl@nonfrenchindent}
%    In French the first paragraph of each section should be indented,
%    this is another difference with US-English. This is controlled by
%    the flag |\if@afterindent|.
%
% \changes{v2.3d}{2009/03/16}{Bug correction: previous versions of
%    frenchb set the flag \cs{if@afterindent} to false outside
%    French which is correct for English but wrong for some languages
%    like Spanish.  Pointed out by Juan Jos\'e Torrens.}
%
%    We need to save the value of the flag |\if@afterindent|
%    `AtBeginDocument' before eventually changing its value.
%
%    \begin{macrocode}
\AtBeginDocument{\ifx\@afterindentfalse\@afterindenttrue
                       \let\@aifORI\@afterindenttrue
                 \else \let\@aifORI\@afterindentfalse
                 \fi
}
\def\bbl@frenchindent{\let\@afterindentfalse\@afterindenttrue
                      \@afterindenttrue}
\def\bbl@nonfrenchindent{\let\@afterindentfalse\@aifORI
                         \@afterindentfalse}
%    \end{macrocode}
%  \end{macro}
%  \end{macro}
%
%  \subsection{Formatting footnotes}
%  \label{sec-footnotes}
%
% \changes{v2.0}{2006/11/06}{Footnotes are now printed
%     by default `\`a la fran\c caise' for the whole document.}
%
% \changes{v2.0b}{2007/04/18}{Footnotes: Just do nothing
%    (except warning) when the bigfoot package is loaded.}
%
%    The |bigfoot| package deeply changes the way footnotes are
%    handled. When |bigfoot| is loaded, we just warn the user that
%    |frenchb| will drop the customisation of footnotes.
%
%    The layout of footnotes is controlled by two flags
%    |\ifFBAutoSpaceFootnotes| and |\ifFBFrenchFootnotes| which are
%    set by options of |\frenchbsetup{}| (see section~\ref{sec-keyval}).
%    Notice that the layout of footnotes \emph{does not depend} on the
%    current language (just think of two footnotes on the same page
%    looking different because one was called in a French part, the
%    other one in English!).
%
%    When |\ifFBAutoSpaceFootnotes| is true, |\@footnotemark| (whose
%    definition is saved at the |\begin{document}| in order to include
%    any customisation that packages might have done) is redefined to
%    add a thin space before the number or symbol calling a footnote
%    (any space typed in is removed first).  This has no effect on
%    the layout of the footnote itself.
%
%    \begin{macrocode}
\AtBeginDocument{\@ifpackageloaded{bigfoot}%
                   {\PackageWarning{frenchb.ldf}%
                     {bigfoot package in use.\MessageBreak
                      frenchb will NOT customise footnotes;\MessageBreak
                      reported}}%
                   {\let\@footnotemarkORI\@footnotemark
                    \def\@footnotemarkFB{\leavevmode\unskip\unkern
                                         \,\@footnotemarkORI}%
                    \ifFBAutoSpaceFootnotes
                      \let\@footnotemark\@footnotemarkFB
                    \fi}%
                }
%    \end{macrocode}
%
%    We then define |\@makefntextFB|, a variant of |\@makefntext|
%    which is responsible for the layout of footnotes, to match the
%    specifications of the French `Imprimerie Nationale':  footnotes
%    will be indented by |\parindentFFN|, numbers (if any) typeset on
%    the baseline (instead of superscripts) and followed by a dot
%    and an half quad space. Whenever symbols are used to number
%    footnotes (as in |\thanks| for instance), we switch back to the
%    standard layout (the French layout of footnotes is meant for
%    footnotes numbered by Arabic or Roman digits).
%
% \changes{v2.0}{2006/11/06}{\cs{parindentFFN} not changed if
%    already defined (required by JA for cah-gut.cls).}
%
% \changes{v2.3b}{2008/12/06}{New commands \cs{dotFFN} and
%    \cs{kernFFN} for more flexibility (suggested by JA).}
%
%    The value of |\parindentFFN| will be redefined at the
%    |\begin{document}|, as the maximum of |\parindent| and 1.5em
%    \emph{unless} it has been set in the preamble (the weird value
%    10in is just for testing whether |\parindentFFN| has been set
%    or not).
%
%    \begin{macrocode}
\newcommand*{\dotFFN}{.}
\newcommand*{\kernFFN}{\kern .5em}
\newdimen\parindentFFN
\parindentFFN=10in
\def\ftnISsymbol{\@fnsymbol\c@footnote}
\long\def\@makefntextFB#1{\ifx\thefootnote\ftnISsymbol
                            \@makefntextORI{#1}%
                          \else
                            \parindent=\parindentFFN
                            \rule\z@\footnotesep
                            \setbox\@tempboxa\hbox{\@thefnmark}%
                            \ifdim\wd\@tempboxa>\z@
                              \llap{\@thefnmark}\dotFFN\kernFFN
                            \fi #1
                          \fi}%
%    \end{macrocode}
%
%    We save the standard definition of |\@makefntext| at the
%    |\begin{document}|, and then redefine |\@makefntext| according to
%    the value of flag |\ifFBFrenchFootnotes| (true or false).
%
%    \begin{macrocode}
\AtBeginDocument{\@ifpackageloaded{bigfoot}{}%
                  {\ifdim\parindentFFN<10in
                   \else
                      \parindentFFN=\parindent
                      \ifdim\parindentFFN<1.5em\parindentFFN=1.5em\fi
                   \fi
                   \let\@makefntextORI\@makefntext
                   \long\def\@makefntext#1{%
                      \ifFBFrenchFootnotes
                         \@makefntextFB{#1}%
                      \else
                         \@makefntextORI{#1}%
                      \fi}%
                  }%
                }
%    \end{macrocode}
%
%    For compatibility reasons, we provide definitions for the commands
%    dealing with the layout of footnotes in |frenchb| version~1.6.
%    |\frenchbsetup{}| (see in section \ref{sec-keyval}) should be
%    preferred for setting these options.  |\StandardFootnotes| may
%    still be used locally (in minipages for instance), that's why the
%    test |\ifFBFrenchFootnotes| is done inside |\@makefntext|.
%    \begin{macrocode}
\newcommand*{\AddThinSpaceBeforeFootnotes}{\FBAutoSpaceFootnotestrue}
\newcommand*{\FrenchFootnotes}{\FBFrenchFootnotestrue}
\newcommand*{\StandardFootnotes}{\FBFrenchFootnotesfalse}
%    \end{macrocode}
%
%  \subsection{Global layout}
%  \label{sec-global}
%
%    In multilingual documents, some typographic rules must depend
%    on the current language (e.g., hyphenation, typesetting of
%    numbers, spacing before double punctuation\dots), others should,
%    IMHO, be kept global to the document: especially the layout of
%    lists (see~\ref{sec-lists}) and footnotes
%    (see~\ref{sec-footnotes}), and the indentation of the first
%    paragraph of sections (see~\ref{sec-indent}).
%
%    From version 2.2 on, if |frenchb| is \babel's ``main language''
%    (i.e. last language option at \babel's loading), |frenchb|
%    customises the layout (i.e. lists, indentation of the first
%    paragraphs of sections and footnotes) in the whole document
%    regardless the current language.   On the other hand, if |frenchb|
%    is \emph{not} \babel's ``main language'', it leaves the layout
%    unchanged both in French and in other languages.
%
%  \begin{macro}{\FrenchLayout}
%  \begin{macro}{\StandardLayout}
%    The former commands |\FrenchLayout| and |\StandardLayout| are kept
%    for compatibility reasons but should no longer be used.
%
% \changes{v2.0g}{2008/03/23}{Flag \cs{ifFBStandardLayout} not checked
%     by \cs{FBprocess@options}, low-level flags have to be set
%     one by one.}
%
%    \begin{macrocode}
\newcommand*{\FrenchLayout}{%
    \FBGlobalLayoutFrenchtrue
    \PackageWarning{frenchb.ldf}%
    {\protect\FrenchLayout\space is obsolete.  Please use\MessageBreak
     \protect\frenchbsetup{GlobalLayoutFrench} instead.}%
}
\newcommand*{\StandardLayout}{%
  \FBReduceListSpacingfalse
  \FBCompactItemizefalse
  \FBStandardItemLabelstrue
  \FBIndentFirstfalse
  \FBFrenchFootnotesfalse
  \FBAutoSpaceFootnotesfalse
  \PackageWarning{frenchb.ldf}%
    {\protect\StandardLayout\space is obsolete.  Please use\MessageBreak
    \protect\frenchbsetup{StandardLayout} instead.}%
}
\@onlypreamble\FrenchLayout
\@onlypreamble\StandardLayout
%    \end{macrocode}
%  \end{macro}
%  \end{macro}
%
%  \subsection{Dots\dots}
%  \label{sec-dots}
%
%  \begin{macro}{\FBtextellipsis}
%    \LaTeXe's standard definition of |\dots| in text-mode is
%    |\textellipsis| which includes a |\kern| at the end;
%    this space is not wanted in some cases (before a closing brace
%    for instance) and |\kern| breaks hyphenation of the next word.
%    We define |\FBtextellipsis| for French (in \LaTeXe{} only).
%
%    The |\if| construction in the \LaTeXe{} definition of |\dots|
%    doesn't allow the use of |xspace| (|xspace| is always followed
%    by a |\fi|), so we use the AMS-\LaTeX{} construction of |\dots|;
%    this has to be done `AtBeginDocument' not to be overwritten
%    when \file{amsmath.sty} is loaded after \babel.
%
% \changes{v2.0}{2006/11/06}{Added special case for LY1 encoding,
%    see  bug report from Bruno Voisin (2004/05/18).}
%
%    LY1 has a ready made character for |\textellipsis|, it should be
%    used in French too (pointed out by Bruno Voisin).
%
%    \begin{macrocode}
\DeclareTextSymbol{\FBtextellipsis}{LY1}{133}
\DeclareTextCommandDefault{\FBtextellipsis}{%
    .\kern\fontdimen3\font.\kern\fontdimen3\font.\xspace}
%    \end{macrocode}
%    |\Mdots@| and |\Tdots@ORI| hold the definitions of |\dots| in
%    Math and Text mode. They default to those of amsmath-2.0, and
%    will revert to standard \LaTeX{} definitions `AtBeginDocument',
%    if amsmath has not been loaded. |\Mdots@| doesn't change when
%    switching from/to French, while |\Tdots@| is |\FBtextellipsis|
%    in French and |\Tdots@ORI| otherwise.
%    \begin{macrocode}
\newcommand*{\Tdots@ORI}{\@xp\textellipsis}
\newcommand*{\Tdots@}{\Tdots@ORI}
\newcommand*{\Mdots@}{\@xp\mdots@}
\AtBeginDocument{\DeclareRobustCommand*{\dots}{\relax
                 \csname\ifmmode M\else T\fi dots@\endcsname}%
                 \@ifundefined{@xp}{\let\@xp\relax}{}%
                 \@ifundefined{mdots@}{\let\Tdots@ORI\textellipsis
                                       \let\Mdots@\mathellipsis}{}}
\def\bbl@frenchdots{\let\Tdots@\FBtextellipsis}
\def\bbl@nonfrenchdots{\let\Tdots@\Tdots@ORI}
\expandafter\addto\csname extras\CurrentOption\endcsname{%
    \bbl@frenchdots}
\expandafter\addto\csname noextras\CurrentOption\endcsname{%
    \bbl@nonfrenchdots}
%    \end{macrocode}
%  \end{macro}
%
%  \subsection{Setup options: keyval stuff}
%  \label{sec-keyval}
%
% \changes{v2.0}{2006/11/06}{New command \cs{frenchbsetup} added
%     for global customisation.}
%
% \changes{v2.0c}{2007/06/25}{Option ThinSpaceInFrenchNumbers added.}
%
% \changes{v2.0d}{2007/07/15}{Options og and fg changed: limit
%     the definition to French so that quote characters can be used
%     in German.}
%
% \changes{v2.0e}{2007/10/05}{New option: StandardLists.}
%
% \changes{v2.0f}{2008/03/23}{Two typos corrected in
%    option StandardLists: [false] $\to$ [true] and
%    StandardLayout $\to$ StandardLists.}
%
% \changes{v2.0f}{2008/03/23}{StandardLayout option had no
%     effect on lists.  Test moved to \cs{FBprocess@options}.}
%
% \changes{v2.0g}{2008/03/23}{Revert previous change to
%     StandardLayout. This option must set the three flags
%     \cs{FBReduceListSpacingfalse}, \cs{FBCompactItemizefalse},
%     and \cs{FBStandardItemLabeltrue} instead of
%     \cs{FBStandardListstrue}, so that later options can still
%     change their value before executing \cs{FBprocess@options}.
%     Same thing for option StandardLists.}
%
% \changes{v2.1a}{2008/03/24}{New option: FrenchSuperscripts
%     to define \cs{up} as \cs{fup} or as \cs{textsuperscript}.}
%
% \changes{v2.1a}{2008/03/30}{New option: LowercaseSuperscripts.}
%
% \changes{v2.2a}{2008/05/08}{The global layout of the document is
%     no longer changed when frenchb is not the last option of babel
%     (\cs{bbl@main@language}). Suggested by Ulrike Fischer.}
%
% \changes{v2.2a}{2008/05/08}{Values of flags
%     \cs{ifFBReduceListSpacing}, \cs{ifFBCompactItemize},
%     \cs{ifFBStandardItemLabels}, \cs{ifFBIndentFirst},
%     \cs{ifFBFrenchFootnotes}, \cs{ifFBAutoSpaceFootnotes} changed:
%     default now means `StandardLayout', it will be changed to
%     `FrenchLayout' AtEndOfPackage only if french is
%     \cs{bbl@main@language}.}
%
% \changes{v2.2a}{2008/05/08}{When frenchb is babel's last option,
%     French becomes the document's main language, so
%     GlobalLayoutFrench applies.}
%
% \changes{v2.3a}{2008/10/10}{New option: OriginalTypewriter. Now
%    frenchb switches to \cs{noautospace@beforeFDP} when a tt-font is
%    in use.  When OriginalTypewriter is set to true, frenchb behaves
%    as in pre-2.3 versions.}
%
%    We first define a collection of conditionals with their defaults
%    (true or false).
%
%    \begin{macrocode}
\newif\ifFBStandardLayout           \FBStandardLayouttrue
\newif\ifFBGlobalLayoutFrench       \FBGlobalLayoutFrenchfalse
\newif\ifFBReduceListSpacing        \FBReduceListSpacingfalse
\newif\ifFBCompactItemize           \FBCompactItemizefalse
\newif\ifFBStandardItemLabels       \FBStandardItemLabelstrue
\newif\ifFBStandardLists            \FBStandardListstrue
\newif\ifFBIndentFirst              \FBIndentFirstfalse
\newif\ifFBFrenchFootnotes          \FBFrenchFootnotesfalse
\newif\ifFBAutoSpaceFootnotes       \FBAutoSpaceFootnotesfalse
\newif\ifFBOriginalTypewriter       \FBOriginalTypewriterfalse
\newif\ifFBThinColonSpace           \FBThinColonSpacefalse
\newif\ifFBThinSpaceInFrenchNumbers \FBThinSpaceInFrenchNumbersfalse
\newif\ifFBFrenchSuperscripts       \FBFrenchSuperscriptstrue
\newif\ifFBLowercaseSuperscripts    \FBLowercaseSuperscriptstrue
\newif\ifFBPartNameFull             \FBPartNameFulltrue
\newif\ifFBShowOptions              \FBShowOptionsfalse
%    \end{macrocode}
%
%    The defaults values of these flags have been set so that |frenchb|
%    does not change anything regarding the global layout.
%    |\bbl@main@language| (set by the last option of babel) controls
%    the global layout of the document.  We check the current language
%    `AtEndOfPackage' (it is |\bbl@main@language|); if it is French,
%    the values of some flags have to be changed to ensure a French
%    looking layout for the whole document (even in parts written in
%    languages other than French); the end-user will then be able to
%    customise the values of all these flags with |\frenchbsetup{}|.
%    \begin{macrocode}
\AtEndOfPackage{%
   \iflanguage{french}{\FBReduceListSpacingtrue
                       \FBCompactItemizetrue
                       \FBStandardItemLabelsfalse
                       \FBIndentFirsttrue
                       \FBFrenchFootnotestrue
                       \FBAutoSpaceFootnotestrue
                       \FBGlobalLayoutFrenchtrue}%
                      {}%
}
%    \end{macrocode}
%
%  \begin{macro}{\frenchbsetup}
%    From version 2.0 on, all setup options are handled by \emph{one}
%    command |\frenchbsetup| using the keyval syntax.
%    Let's now define this command which reads and sets the options
%    to be processed later (at |\begin{document}|) by
%    |\FBprocess@options|. It  can only be called in the preamble.
%    \begin{macrocode}
\newcommand*{\frenchbsetup}[1]{%
  \setkeys{FB}{#1}%
}%
\@onlypreamble\frenchbsetup
%    \end{macrocode}
%    |frenchb| being an option of babel, it cannot load a package
%    (keyval) while |frenchb.ldf| is read, so we defer the loading of
%    \file{keyval} and the options setup at the end of \babel's loading.
%
%    |StandardLayout| resets the layout in French to the standard layout
%    defined par the \LaTeX{} class and packages loaded. It deals with
%    lists, indentation of first paragraphs of sections and footnotes.
%    Other keys, entered \emph{after} |StandardLayout| in
%    |\frenchbsetup|, can overrule some of the |StandardLayout|
%     settings.
%
%    |GlobalLayoutFrench| forces the layout in French and (as far as
%    possible) outside French to meet the French typographic standards.
%
% \changes{v2.3d}{2009/03/16}{Warning added to \cs{GlobalLayoutFrench}
%    when French is not the main language.}
%
%    \begin{macrocode}
\AtEndOfPackage{%
    \RequirePackage{keyval}%
    \define@key{FB}{StandardLayout}[true]%
                      {\csname FBStandardLayout#1\endcsname
                       \ifFBStandardLayout
                         \FBReduceListSpacingfalse
                         \FBCompactItemizefalse
                         \FBStandardItemLabelstrue
                         \FBIndentFirstfalse
                         \FBFrenchFootnotesfalse
                         \FBAutoSpaceFootnotesfalse
                         \FBGlobalLayoutFrenchfalse
                       \else
                         \FBReduceListSpacingtrue
                         \FBCompactItemizetrue
                         \FBStandardItemLabelsfalse
                         \FBIndentFirsttrue
                         \FBFrenchFootnotestrue
                         \FBAutoSpaceFootnotestrue
                       \fi}%
    \define@key{FB}{GlobalLayoutFrench}[true]%
                      {\csname FBGlobalLayoutFrench#1\endcsname
                       \ifFBGlobalLayoutFrench
                          \iflanguage{french}%
                            {\FBReduceListSpacingtrue
                             \FBCompactItemizetrue
                             \FBStandardItemLabelsfalse
                             \FBIndentFirsttrue
                             \FBFrenchFootnotestrue
                             \FBAutoSpaceFootnotestrue}%
                            {\PackageWarning{frenchb.ldf}%
                              {Option `GlobalLayoutFrench' skipped:
                               \MessageBreak French is *not*
                               babel's last option.\MessageBreak}}%
                       \fi}%
    \define@key{FB}{ReduceListSpacing}[true]%
                      {\csname FBReduceListSpacing#1\endcsname}%
    \define@key{FB}{CompactItemize}[true]%
                      {\csname FBCompactItemize#1\endcsname}%
    \define@key{FB}{StandardItemLabels}[true]%
                      {\csname FBStandardItemLabels#1\endcsname}%
    \define@key{FB}{ItemLabels}{%
        \renewcommand*{\FrenchLabelItem}{#1}}%
    \define@key{FB}{ItemLabeli}{%
        \renewcommand*{\Frlabelitemi}{#1}}%
    \define@key{FB}{ItemLabelii}{%
        \renewcommand*{\Frlabelitemii}{#1}}%
    \define@key{FB}{ItemLabeliii}{%
        \renewcommand*{\Frlabelitemiii}{#1}}%
    \define@key{FB}{ItemLabeliv}{%
        \renewcommand*{\Frlabelitemiv}{#1}}%
    \define@key{FB}{StandardLists}[true]%
                      {\csname FBStandardLists#1\endcsname
                       \ifFBStandardLists
                         \FBReduceListSpacingfalse
                         \FBCompactItemizefalse
                         \FBStandardItemLabelstrue
                       \else
                         \FBReduceListSpacingtrue
                         \FBCompactItemizetrue
                         \FBStandardItemLabelsfalse
                       \fi}%
    \define@key{FB}{IndentFirst}[true]%
                      {\csname FBIndentFirst#1\endcsname}%
    \define@key{FB}{FrenchFootnotes}[true]%
                      {\csname FBFrenchFootnotes#1\endcsname}%
    \define@key{FB}{AutoSpaceFootnotes}[true]%
                      {\csname FBAutoSpaceFootnotes#1\endcsname}%
    \define@key{FB}{AutoSpacePunctuation}[true]%
                      {\csname FBAutoSpacePunctuation#1\endcsname}%
    \define@key{FB}{OriginalTypewriter}[true]%
                      {\csname FBOriginalTypewriter#1\endcsname}%
    \define@key{FB}{ThinColonSpace}[true]%
                      {\csname FBThinColonSpace#1\endcsname}%
    \define@key{FB}{ThinSpaceInFrenchNumbers}[true]%
                      {\csname FBThinSpaceInFrenchNumbers#1\endcsname}%
    \define@key{FB}{FrenchSuperscripts}[true]%
                      {\csname FBFrenchSuperscripts#1\endcsname}
    \define@key{FB}{LowercaseSuperscripts}[true]%
                      {\csname FBLowercaseSuperscripts#1\endcsname}
    \define@key{FB}{PartNameFull}[true]%
                      {\csname FBPartNameFull#1\endcsname}%
    \define@key{FB}{ShowOptions}[true]%
                      {\csname FBShowOptions#1\endcsname}%
%    \end{macrocode}
%    Inputing French quotes as \emph{single characters} when they are
%    available on the keyboard (through a compose key for instance)
%    is more comfortable than typing |\og| and |\fg|.
%    The purpose of the following code is to map the French quote
%    characters to |\og\ignorespaces| and |{\fg}| respectively when
%    the current language is French, and to |\guillemotleft| and
%    |\guillemotright| otherwise (think of German quotes); thus correct
%    unbreakable spaces will be added automatically to French quotes.
%    The quote characters typed in depend on the input encoding,
%    it can be single-byte (latin1, latin9, applemac,\dots) or
%    multi-bytes (utf-8, utf8x).  We first check whether XeTeX is used
%    or not, if not the package |inputenc| has to be loaded before the
%    |\begin{document}| with the proper coding option, so we check if
%    |\DeclareInputText| is defined.
%    \begin{macrocode}
    \define@key{FB}{og}{%
       \newcommand*{\FB@@og}{\iflanguage{french}%
                               {\FB@og\ignorespaces}{\guillemotleft}}%
       \expandafter\ifx\csname XeTeXrevision\endcsname\relax
         \AtBeginDocument{%
           \@ifundefined{DeclareInputText}%
             {\PackageWarning{frenchb.ldf}%
               {Option `og' requires package inputenc.\MessageBreak}%
             }%
             {\@ifundefined{uc@dclc}%
%    \end{macrocode}
%    if |\uc@dclc| is undefined, utf8x is not loaded\dots
%    \begin{macrocode}
               {\@ifundefined{DeclareUnicodeCharacter}%
%    \end{macrocode}
%    if |\DeclareUnicodeCharacter| is undefined, utf8 is not loaded
%    either, we assume 8-bit character input encoding.
%    Package MULEenc.sty (from CJK) defines |\mule@def| to map
%    characters to control sequences.
%    \begin{macrocode}
                  {\@tempcnta`#1\relax
                     \@ifundefined{mule@def}%
                       {\DeclareInputText{\the\@tempcnta}{\FB@@og}}%
                       {\mule@def{11}{{\FB@@og}}}%
                  }%
%    \end{macrocode}
%    utf8 loaded, use |\DeclareUnicodeCharacter|,
%    \begin{macrocode}
                  {\DeclareUnicodeCharacter{00AB}{\FB@@og}}%
               }%
%    \end{macrocode}
%    utf8x loaded, use |\uc@dclc|,
%    \begin{macrocode}
               {\uc@dclc{171}{default}{\FB@@og}}%
             }%
         }%
%    \end{macrocode}
%    XeTeX in use, the following trick for defining the active quote
%    character is borrowed from \file{inputenc.dtx}.
%    \begin{macrocode}
       \else
         \catcode`#1=\active
         \bgroup
           \uccode`\~`#1%
           \uppercase{%
         \egroup
         \def~%
         }{\FB@@og}%
       \fi
    }%
%    \end{macrocode}
%    Same code for the closing quote.
%    \begin{macrocode}
    \define@key{FB}{fg}{%
       \newcommand*{\FB@@fg}{\iflanguage{french}%
                               {\FB@fg}{\guillemotright}}%
       \expandafter\ifx\csname XeTeXrevision\endcsname\relax
         \AtBeginDocument{%
           \@ifundefined{DeclareInputText}%
             {\PackageWarning{frenchb.ldf}%
               {Option `fg' requires package inputenc.\MessageBreak}%
             }%
             {\@ifundefined{uc@dclc}%
               {\@ifundefined{DeclareUnicodeCharacter}%
                  {\@tempcnta`#1\relax
                     \@ifundefined{mule@def}%
                       {\DeclareInputText{\the\@tempcnta}{{\FB@@fg}}}%
                       {\mule@def{27}{{\FB@@fg}}}%
                  }%
                  {\DeclareUnicodeCharacter{00BB}{{\FB@@fg}}}%
               }%
               {\uc@dclc{187}{default}{{\FB@@fg}}}%
             }%
         }%
       \else
         \catcode`#1=\active
         \bgroup
           \uccode`\~`#1%
           \uppercase{%
         \egroup
         \def~%
         }{{\FB@@fg}}%
       \fi
    }%
}
%    \end{macrocode}
%  \end{macro}
%
% \begin{macro}{\FBprocess@options}
%    |\FBprocess@options| processes the options, it is called \emph{once}
%    at |\begin{document}|.
%    \begin{macrocode}
\newcommand*{\FBprocess@options}{%
%    \end{macrocode}
%    Nothing has to be done here for |StandardLayout| and
%    |StandardLists| (the involved flags have already been set in
%    |\frenchbsetup{}| or before (at babel's EndOfPackage).
%
%    The next three options deal with the layout of lists in French.
%
%    |ReduceListSpacing| reduces the vertical spaces between list
%    items in French (done by changing |\list| to |\listFB|).
%    When |GlobalLayoutFrench| is true (the default), the same is
%    done outside French except for languages that force a different
%    setting.
%    \begin{macrocode}
  \ifFBReduceListSpacing
    \addto\extrasfrench{\let\list\listFB
                        \let\endlist\endlistFB}%
    \addto\noextrasfrench{\ifFBGlobalLayoutFrench
                            \let\list\listFB
                            \let\endlist\endlistFB
                          \else
                            \let\list\listORI
                            \let\endlist\endlistORI
                          \fi}%
  \else
    \addto\extrasfrench{\let\list\listORI
                        \let\endlist\endlistORI}%
    \addto\noextrasfrench{\let\list\listORI
                          \let\endlist\endlistORI}%
  \fi
%    \end{macrocode}
%
%    |CompactItemize| suppresses the vertical spacing between list
%    items in French (done by changing |\itemize| to |\itemizeFB|).
%    When |GlobalLayoutFrench| is true the same is done outside French.
%    \begin{macrocode}
  \ifFBCompactItemize
    \addto\extrasfrench{\let\itemize\itemizeFB
                        \let\enditemize\enditemizeFB}%
    \addto\noextrasfrench{\ifFBGlobalLayoutFrench
                             \let\itemize\itemizeFB
                             \let\enditemize\enditemizeFB
                          \else
                             \let\itemize\itemizeORI
                             \let\enditemize\enditemizeORI
                          \fi}%
  \else
    \addto\extrasfrench{\let\itemize\itemizeORI
                        \let\enditemize\enditemizeORI}%
    \addto\noextrasfrench{\let\itemize\itemizeORI
                          \let\enditemize\enditemizeORI}%
  \fi
%    \end{macrocode}
%
%    |StandardItemLabels| resets labelitems in French to their
%    standard values set by the \LaTeX{} class and packages loaded.
%    When |GlobalLayoutFrench| is true labelitems are identical inside
%    and outside French.
%    \begin{macrocode}
  \ifFBStandardItemLabels
    \addto\extrasfrench{\bbl@nonfrenchlabelitems}%
    \addto\noextrasfrench{\bbl@nonfrenchlabelitems}%
  \else
    \addto\extrasfrench{\bbl@frenchlabelitems}%
    \addto\noextrasfrench{\ifFBGlobalLayoutFrench
                            \bbl@frenchlabelitems
                          \else
                            \bbl@nonfrenchlabelitems
                          \fi}%
  \fi
%    \end{macrocode}
%
%    |IndentFirst| forces the first paragraphs of sections to be
%    indented just like the other ones in French.
%    When |GlobalLayoutFrench| is true (the default), the same is
%    done outside French except for languages that force a different
%    setting.
%    \begin{macrocode}
  \ifFBIndentFirst
    \addto\extrasfrench{\bbl@frenchindent}%
    \addto\noextrasfrench{\ifFBGlobalLayoutFrench
                             \bbl@frenchindent
                          \else
                             \bbl@nonfrenchindent
                          \fi}%
  \else
    \addto\extrasfrench{\bbl@nonfrenchindent}%
    \addto\noextrasfrench{\bbl@nonfrenchindent}%
  \fi
%    \end{macrocode}
%
%    The layout of footnotes is handled at the |\begin{document}|
%    depending on the values of flags |FrenchFootnotes|
%    and |AutoSpaceFootnotes| (see section~\ref{sec-footnotes}),
%    nothing has to be done here for footnotes.
%
%    |AutoSpacePunctuation| adds an unbreakable space (in French only)
%    before the four active characters (:;!?) even if none has been
%    typed before them.
%    \begin{macrocode}
  \ifFBAutoSpacePunctuation
     \autospace@beforeFDP
  \else
     \noautospace@beforeFDP
  \fi
%    \end{macrocode}
%
%    When |OriginalTypewriter| is set to |false| (the default),
%    |\ttfamily|, |\rmfamily| and |\sffamily| are redefined as
%    |\ttfamilyFB|, |\rmfamilyFB| and |\sffamilyFB| respectively
%    to prevent addition of automatic spaces before the four active
%    characters in computer code.
%    \begin{macrocode}
  \ifFBOriginalTypewriter
  \else
     \let\ttfamily\ttfamilyFB
     \let\rmfamily\rmfamilyFB
     \let\sffamily\sffamilyFB
  \fi
%    \end{macrocode}
%
%    |ThinColonSpace| changes the normal unbreakable space typeset in
%     French before `:' to a thin space.
%    \begin{macrocode}
  \ifFBThinColonSpace\renewcommand*{\Fcolonspace}{\thinspace}\fi
%    \end{macrocode}
%
%    When |true|, |ThinSpaceInFrenchNumbers| redefines |numprint.sty|'s
%    command |\npstylefrench| to set |\npthousandsep| to |\,|
%    (thinspace) instead of |~| (default) . This option has no effect
%    if package |numprint.sty| is not loaded with `|autolanguage|'.
%    As old versions of |numprint.sty| did not define |\npstylefrench|,
%    we have to provide this command.
%    \begin{macrocode}
  \@ifpackageloaded{numprint}%
  {\ifnprt@autolanguage
     \providecommand*{\npstylefrench}{}%
     \ifFBThinSpaceInFrenchNumbers
       \renewcommand*\npstylefrench{%
          \npthousandsep{\,}%
          \npdecimalsign{,}%
          \npproductsign{\cdot}%
          \npunitseparator{\,}%
          \npdegreeseparator{}%
          \nppercentseparator{\nprt@unitsep}%
          }%
     \else
       \renewcommand*\npstylefrench{%
          \npthousandsep{~}%
          \npdecimalsign{,}%
          \npproductsign{\cdot}%
          \npunitseparator{\,}%
          \npdegreeseparator{}%
          \nppercentseparator{\nprt@unitsep}%
          }%
     \fi
     \npaddtolanguage{french}{french}%
   \fi}{}%
%    \end{macrocode}
%
%    |FrenchSuperscripts|: if |true| |\up=\fup|, else
%    |\up=\textsuperscript|. Anyway |\up*=\FB@up@fake|. The star-form
%    |\up*{}| is provided for fonts that lack some superior letters:
%    Adobe Jenson Pro and Utopia Expert have no ``g superior'' for
%    instance.
%    \begin{macrocode}
  \ifFBFrenchSuperscripts
    \DeclareRobustCommand*{\up}{\@ifstar{\FB@up@fake}{\fup}}%
  \else
    \DeclareRobustCommand*{\up}{\@ifstar{\FB@up@fake}%
                                        {\textsuperscript}}%
  \fi
%    \end{macrocode}
%
%    |LowercaseSuperscripts|: if |true| let |\FB@lc| be |\lowercase|,
%     else |\FB@lc| is redefined to do nothing.
%    \begin{macrocode}
  \ifFBLowercaseSuperscripts
  \else
    \renewcommand*{\FB@lc}[1]{##1}%
  \fi
%    \end{macrocode}
%
%    |PartNameFull|: if |false|, redefine |\partname|.
%    \begin{macrocode}
  \ifFBPartNameFull
  \else\addto\captionsfrench{\def\partname{Partie}}\fi
%    \end{macrocode}
%
%    |ShowOptions|: if |true|, print the list of all options to the
%    \file{.log} file.
%    \begin{macrocode}
  \ifFBShowOptions
    \GenericWarning{* }{%
     * **** List of possible options for frenchb ****\MessageBreak
     [Default values between brackets when frenchb is loaded *LAST*]%
     \MessageBreak
     ShowOptions=true [false]\MessageBreak
     StandardLayout=true [false]\MessageBreak
     GlobalLayoutFrench=false [true]\MessageBreak
     StandardLists=true [false]\MessageBreak
     ReduceListSpacing=false [true]\MessageBreak
     CompactItemize=false [true]\MessageBreak
     StandardItemLabels=true [false]\MessageBreak
     ItemLabels=\textemdash, \textbullet,
        \protect\ding{43},... [\textendash]\MessageBreak
     ItemLabeli=\textemdash, \textbullet,
        \protect\ding{43},... [\textendash]\MessageBreak
     ItemLabelii=\textemdash, \textbullet,
        \protect\ding{43},... [\textendash]\MessageBreak
     ItemLabeliii=\textemdash, \textbullet,
        \protect\ding{43},... [\textendash]\MessageBreak
     ItemLabeliv=\textemdash, \textbullet,
        \protect\ding{43},... [\textendash]\MessageBreak
     IndentFirst=false [true]\MessageBreak
     FrenchFootnotes=false [true]\MessageBreak
     AutoSpaceFootnotes=false [true]\MessageBreak
     AutoSpacePunctuation=false [true]\MessageBreak
     OriginalTypewriter=true [false]\MessageBreak
     ThinColonSpace=true [false]\MessageBreak
     ThinSpaceInFrenchNumbers=true [false]\MessageBreak
     FrenchSuperscripts=false [true]\MessageBreak
     LowercaseSuperscripts=false [true]\MessageBreak
     PartNameFull=false [true]\MessageBreak
     og= <left quote character>, fg= <right quote character>
     \MessageBreak
     *********************************************
     \MessageBreak\protect\frenchbsetup{ShowOptions}}
  \fi
}
%    \end{macrocode}
%  \end{macro}
%
% \changes{v2.0}{2006/12/15}{AtBeginDocument, save again the
%    definitions of the `list' and `itemize' environments and the
%    values of labelitems.  As of frenchb v.1.6, `ORI' values were
%    set when reading frenchb.ldf, later changes were ignored.}
%
% \changes{v2.0}{2006/12/06}{Added warning for OT1 encoding.}
%
% \changes{v2.1b}{2008/04/07}{Disable some commands in bookmarks.}
%
%    At |\begin{document}| we save again the definitions of the `list'
%    and `itemize' environments and the values of labelitems so that
%    all changes made in the preamble are taken into account in
%    languages other than French and in French with the StandardLayout
%    option.  We also have to provide an |\xspace| command in case the
%    |xspace.sty| package is not loaded.
%
%    \begin{macrocode}
\AtBeginDocument{%
   \let\listORI\list
   \let\endlistORI\endlist
   \let\itemizeORI\itemize
   \let\enditemizeORI\enditemize
   \let\@ltiORI\labelitemi
   \let\@ltiiORI\labelitemii
   \let\@ltiiiORI\labelitemiii
   \let\@ltivORI\labelitemiv
   \providecommand*{\xspace}{\relax}%
%    \end{macrocode}
%    Let's redefine some commands in \file{hyperref}'s bookmarks.
%    \begin{macrocode}
   \@ifundefined{pdfstringdefDisableCommands}{}%
     {\pdfstringdefDisableCommands{%
        \let\up\relax
        \def\ieme{e\xspace}%
        \def\iemes{es\xspace}%
        \def\ier{er\xspace}%
        \def\iers{ers\xspace}%
        \def\iere{re\xspace}%
        \def\ieres{res\xspace}%
        \def\FrenchEnumerate#1{#1\degre\space}%
        \def\FrenchPopularEnumerate#1{#1\degre)\space}%
        \def\No{N\degre\space}%
        \def\no{n\degre\space}%
        \def\Nos{N\degre\space}%
        \def\nos{n\degre\space}%
        \def\og{\guillemotleft\space}%
        \def\fg{\space\guillemotright}%
        \let\bsc\textsc
        \let\degres\degre
     }}%
%    \end{macrocode}
%    It is time to process the options set with |\frenchboptions{}|.
%    Then execute either |\extrasfrench| and |\captionsfrench| or
%    |\noextrasfrench| according to the current language at the
%    |\begin{document}| (these three commands are updated by
%    |\FBprocess@options|).
%    \begin{macrocode}
   \FBprocess@options
   \iflanguage{french}{\extrasfrench\captionsfrench}{\noextrasfrench}%
%    \end{macrocode}
%    Some warnings are issued when output font encodings are not
%    properly set. With XeLaTeX, \file{fontspec.sty} and
%    \file{xunicode.sty} should be loaded; with (pdf)\LaTeX, a warning
%    is issued when OT1 encoding is in use at the |\begin{document}|.
%    Mind that |\encodingdefault| is defined as `long', defining
%    |\FBOTone| with |\newcommand*| would fail!
%    \begin{macrocode}
   \expandafter\ifx\csname XeTeXrevision\endcsname\relax
      \begingroup \newcommand{\FBOTone}{OT1}%
      \ifx\encodingdefault\FBOTone
        \PackageWarning{frenchb.ldf}%
           {OT1 encoding should not be used for French.
            \MessageBreak
            Add \protect\usepackage[T1]{fontenc} to the
            preamble\MessageBreak of your document,}
      \fi
     \endgroup
   \else
     \@ifundefined{DeclareUTFcharacter}%
       {\PackageWarning{frenchb.ldf}%
         {Add \protect\usepackage{fontspec} *and*\MessageBreak
          \protect\usepackage{xunicode} to the preamble\MessageBreak
          of your document,}}%
       {}%
    \fi
}
%    \end{macrocode}
%
%  \subsection{Clean up and exit}
%
%    Load |frenchb.cfg| (should do nothing, just for compatibility).
%    \begin{macrocode}
\loadlocalcfg{frenchb}
%    \end{macrocode}
%    Final cleaning.
%    The macro |\ldf@quit| takes care for setting the main language
%    to be switched on at |\begin{document}| and resetting the
%    category code of \texttt{@} to its original value.
%    The config file searched for has to be |frenchb.cfg|, and
%    |\CurrentOption| has been set to `french', so
%    |\ldf@finish\CurrentOption| cannot be used: we first load
%    |frenchb.cfg|, then call |\ldf@quit\CurrentOption|.
%    \begin{macrocode}
\FBclean@on@exit
\ldf@quit\CurrentOption
%    \end{macrocode}
% \iffalse
%</code>
%<*dtx>
% \fi
%%
%% \CharacterTable
%%  {Upper-case    \A\B\C\D\E\F\G\H\I\J\K\L\M\N\O\P\Q\R\S\T\U\V\W\X\Y\Z
%%   Lower-case    \a\b\c\d\e\f\g\h\i\j\k\l\m\n\o\p\q\r\s\t\u\v\w\x\y\z
%%   Digits        \0\1\2\3\4\5\6\7\8\9
%%   Exclamation   \!     Double quote  \"     Hash (number) \#
%%   Dollar        \$     Percent       \%     Ampersand     \&
%%   Acute accent  \'     Left paren    \(     Right paren   \)
%%   Asterisk      \*     Plus          \+     Comma         \,
%%   Minus         \-     Point         \.     Solidus       \/
%%   Colon         \:     Semicolon     \;     Less than     \<
%%   Equals        \=     Greater than  \>     Question mark \?
%%   Commercial at \@     Left bracket  \[     Backslash     \\
%%   Right bracket \]     Circumflex    \^     Underscore    \_
%%   Grave accent  \`     Left brace    \{     Vertical bar  \|
%%   Right brace   \}     Tilde         \~}
%%
% \iffalse
%</dtx>
% \fi
%
% \Finale
\endinput
}
%    \end{macrocode}
%    With \LaTeXe\ we can now also use the option \Lopt{french} and
%    still call the file \file{frenchb.ldf}.
% \changes{babel~3.5d}{1995/07/02}{Load \file{french.ldf} when it is
%    found instead of \file{frenchb.ldf}}
% \changes{babel~3.7j}{2003/06/07}{\emph{only} load
%    \file{frenchb.ldf}}
%    \begin{macrocode}
\DeclareOption{french}{% \iffalse meta-comment
%
% Copyright 1989-2009 Johannes L. Braams and any individual authors
% listed elsewhere in this file.  All rights reserved.
% 
% This file is part of the Babel system.
% --------------------------------------
% 
% It may be distributed and/or modified under the
% conditions of the LaTeX Project Public License, either version 1.3
% of this license or (at your option) any later version.
% The latest version of this license is in
%   http://www.latex-project.org/lppl.txt
% and version 1.3 or later is part of all distributions of LaTeX
% version 2003/12/01 or later.
% 
% This work has the LPPL maintenance status "maintained".
% 
% The Current Maintainer of this work is Johannes Braams.
% 
% The list of all files belonging to the Babel system is
% given in the file `manifest.bbl. See also `legal.bbl' for additional
% information.
% 
% The list of derived (unpacked) files belonging to the distribution
% and covered by LPPL is defined by the unpacking scripts (with
% extension .ins) which are part of the distribution.
% \fi
% \CheckSum{2135}
%
% \iffalse
%    Tell the \LaTeX\ system who we are and write an entry on the
%    transcript. Nothing to write to the .cfg file, if generated.
%<*dtx>
\ProvidesFile{frenchb.dtx}
%</dtx>
% \changes{v2.1d}{2008/05/04}{Argument of \cs{ProvidesLanguage} changed
%     from `french' to `frenchb', otherwise \cs{listfiles} prints
%     no date/version information.  The bug with \cs{listfiles}
%     (introduced in v.1.5!), was pointed out by Ulrike Fischer.}
%<code>\ProvidesLanguage{frenchb}
%\ProvidesFile{frenchb.dtx}
%<*!cfg>
        [2009/03/16 v2.3d French support from the babel system]
%</!cfg>
%<*cfg>
%% frenchb.cfg: configuration file for frenchb.ldf
%% Daniel Flipo Daniel.Flipo at univ-lille1.fr
%</cfg>
%%    File `frenchb.dtx'
%%    Babel package for LaTeX version 2e
%%    Copyright (C) 1989 - 2009
%%              by Johannes Braams, TeXniek
%
%<*!cfg>
%%    Frenchb language Definition File
%%    Copyright (C) 1989 - 2009
%%              by Johannes Braams, TeXniek
%%                 Daniel Flipo, GUTenberg
%
%%    Please report errors to: Daniel Flipo, GUTenberg
%%                             Daniel.Flipo at univ-lille1.fr
%</!cfg>
%
%    This file is part of the babel system, it provides the source
%    code for the French language definition file.
%
%<*filedriver>
\documentclass[a4paper]{ltxdoc}
\DeclareFontEncoding{T1}{}{}
\DeclareFontSubstitution{T1}{lmr}{m}{n}
\DeclareTextCommand{\guillemotleft}{OT1}{%
  {\fontencoding{T1}\fontfamily{lmr}\selectfont\char19}}
\DeclareTextCommand{\guillemotright}{OT1}{%
  {\fontencoding{T1}\fontfamily{lmr}\selectfont\char20}}
\newcommand*\TeXhax{\TeX hax}
\newcommand*\babel{\textsf{babel}}
\newcommand*\langvar{$\langle \mathit lang \rangle$}
\newcommand*\note[1]{}
\newcommand*\Lopt[1]{\textsf{#1}}
\newcommand*\file[1]{\texttt{#1}}
\begin{document}
\setlength{\parindent}{0pt}
\begin{center}
  \textbf{\Large A Babel language definition file for French}\\[3mm]^^A\]
  Daniel \textsc{Flipo}\\
  \texttt{Daniel.Flipo@univ-lille1.fr}
\end{center}
 \RecordChanges
 \DocInput{frenchb.dtx}
\end{document}
%</filedriver>
% \fi
% \GetFileInfo{frenchb.dtx}
%
%  \section{The French language}
%
%    The file \file{\filename}\footnote{The file described in this
%    section has version number \fileversion\ and was last revised on
%    \filedate.}, defines all the language definition macros for the
%    French language.
%
%    Customisation for the French language is achieved following the
%    book ``Lexique des r\`egles typographiques en usage \`a
%    l'Imprimerie nationale'' troisi\`eme \'edition (1994),
%    ISBN-2-11-081075-0.
%
%    First version released: 1.1 (1996/05/31) as part of
%    \babel-3.6beta.
%
%    |frenchb| has been improved using helpful suggestions from many
%    people, mainly from Jacques Andr\'e, Michel Bovani, Thierry Bouche,
%    and Vincent Jalby.  Thanks to all of them!
%
%    This new version (2.x) has been designed to be used with \LaTeXe{}
%    and Plain\TeX{} formats only. \LaTeX-2.09 is no longer supported.
%    Changes between version 1.6 and \fileversion{} are listed in
%    subsection~\ref{ssec-changes} p.~\pageref{ssec-changes}.
%
%    An extensive documentation is available in French here:\\
%    |http://daniel.flipo.free.fr/frenchb|
%
%  \subsection{Basic interface}
%
%    In a multilingual document, some typographic rules are language
%    dependent, i.e. spaces before `double punctuation' (|:| |;| |!|
%    |?|) in French, others concern the general layout (i.e. layout of
%    lists, footnotes, indentation of first paragraphs of sections) and
%    should apply to the whole document.
%
%    Starting with version~2.2, |frenchb| behaves differently according
%    to \babel's \emph{main language} defined as the \emph{last}
%    option\footnote{Its name is kept in \texttt{\textbackslash
%           bbl@main@language}.} at \babel's loading.  When French is
%    not \babel's main language, |frenchb| no longer alters the global
%    layout of the document (even in parts where French is the current
%    language): the layout of lists, footnotes, indentation of first
%    paragraphs of sections are not customised by |frenchb|.
%
%    When French is loaded as the last option of \babel, |frenchb|
%    makes the following changes to the global layout, \emph{both in
%    French and in all other languages}\footnote{%
%       For each item, hooks are provided to reset standard
%       \LaTeX{} settings or to emulate the behavior of former versions
%       of \texttt{frenchb} (see command
%       \texttt{\textbackslash frenchbsetup\{\}},
%       section~\ref{ssec-custom}).}:
%    \begin{enumerate}
%    \item the first paragraph of each section is indented
%          (\LaTeX{} only);
%    \item the default items in itemize environment are set to `--'
%          instead of `\textbullet', and all vertical spacing and glue
%          is deleted; it is possible to change `--' to something else
%          (`---' for instance) using |\frenchbsetup{}|;
%    \item vertical spacing in general \LaTeX{} lists is
%          shortened;
%    \item footnotes are displayed ``\`a la fran\c{c}aise''.
%    \end{enumerate}
%
%    Regarding local typography, the command |\selectlanguage{french}|
%    switches to the French language\footnote{%
%      \texttt{\textbackslash selectlanguage\{francais\}}
%      and \texttt{\textbackslash selectlanguage\{frenchb\}} are kept
%      for backward compatibility but should no longer be used.},
%    with the following effects:
%    \begin{enumerate}
%    \item French hyphenation patterns are made active;
%    \item `double punctuation' (|:| |;| |!| |?|) is made
%           active%\footnote{Actually, they are active in the whole
%           document, only their expansions differ in French and
%           outside French} for correct spacing in French;
%    \item |\today| prints the date in French;
%    \item the caption names are translated into French
%          (\LaTeX{} only);
%    \item the space after |\dots| is removed in French.
%    \end{enumerate}
%
%    Some commands are provided in |frenchb| to make typesetting
%    easier:
%    \begin{enumerate}
%    \item French quotation marks can be entered using the commands
%          |\og| and |\fg| which work in \LaTeXe and Plain\TeX,
%          their appearance depending on what is available to draw
%          them; even if you use \LaTeXe{} \emph{and} |T1|-encoding,
%          you should refrain from entering them as
%          |<<~French quotation marks~>>|: |\og| and |\fg| provide
%          better horizontal spacing.
%          |\og| and |\fg| can be used outside French, they typeset
%          then English quotes `` and ''.
%    \item A command |\up| is provided to typeset superscripts like
%          |M\up{me}| (abbreviation for ``Madame''), |1\up{er}| (for
%          ``premier'').  Other commands are also provided for
%          ordinals: |\ier|, |\iere|, |\iers|, |\ieres|, |\ieme|,
%          |\iemes| (|3\iemes| prints 3\textsuperscript{es}).
%    \item Family names should be typeset in small capitals and never
%          be hyphenated, the macro |\bsc| (boxed small caps) does
%          this, e.g., |Leslie~\bsc{Lamport}| will produce
%          Leslie~\mbox{\textsc{Lamport}}. Note that composed names
%          (such as Dupont-Durant) may now be hyphenated on explicit
%          hyphens, this differs from |frenchb|~v.1.x.
%    \item Commands |\primo|, |\secundo|, |\tertio| and |\quarto|
%          print 1\textsuperscript{o}, 2\textsuperscript{o},
%          3\textsuperscript{o}, 4\textsuperscript{o}.
%          |\FrenchEnumerate{6}| prints 6\textsuperscript{o}.
%    \item Abbreviations for ``Num\'ero(s)'' and ``num\'ero(s)''
%          (N\textsuperscript{o} N\textsuperscript{os}
%          n\textsuperscript{o} and n\textsuperscript{os}~)
%          are obtained via the commands |\No|, |\Nos|, |\no|, |\nos|.
%    \item Two commands are provided to typeset the symbol for
%          ``degr\'e'': |\degre| prints the raw character and
%          |\degres| should be used to typeset temperatures (e.g.,
%          ``|20~\degres C|'' with an unbreakable space), or for
%          alcohols' strengths (e.g., ``|45\degres|'' with \emph{no}
%          space in French).
%    \item In math mode the comma has to be surrounded with
%          braces to avoid a spurious space being inserted after it,
%          in decimal numbers for instance (see the \TeX{}book p.~134).
%          The command |\DecimalMathComma| makes the comma be an
%          ordinary character \emph{in French only} (no space added);
%          as a counterpart, if |\DecimalMathComma| is active, an
%          explicit space has to be added in lists and intervals:
%          |$[0,\ 1]$|, |$(x,\ y)$|. |\StandardMathComma| switches back
%          to the standard behaviour of the comma.
%    \item A command |\nombre| was provided in 1.x versions to easily
%          format numbers in slices of three digits separated either
%          by a comma in English or with a space in French; |\nombre|
%          is now mapped to |\numprint| from \file{numprint.sty}, see
%          \file{numprint.pdf} for more information.
%    \item |frenchb| has been designed to take advantage of the |xspace|
%          package if present: adding |\usepackage{xspace}| in the
%          preamble will force macros like |\fg|, |\ier|, |\ieme|,
%          |\dots|, \dots, to respect the spaces you type after them,
%          for instance typing `|1\ier juin|' will print
%          `1\textsuperscript{er} juin' (no need for a forced space
%          after |1\ier|).
%    \end{enumerate}
%
%  \subsection{Customisation}
%  \label{ssec-custom}
%
%     Up to version 1.6, customisation of |frenchb| was achieved
%     by entering commands in \file{frenchb.cfg}.  This possibility
%     remains for compatibility, but \emph{should not longer be used}.
%     Version 2.0 introduces a new command |\frenchbsetup{}| using
%     the \file{keyval} syntax which should make it easier to choose
%     among the many options available. The command |\frenchbsetup{}|
%     is to appear in the preamble only (after loading \babel).
%
%     \vspace{.5\baselineskip}
%     |\frenchbsetup{ShowOptions}| prints all available options to
%     the \file{.log} file, it is just meant as a remainder of the
%     list of offered options. As usual with \file{keyval} syntax,
%     boolean options (as |ShowOptions|) can be entered as
%     |ShowOptions=true| or just |ShowOptions|, the `|=true|' part
%     can be omitted.
%
%     \vspace{.5\baselineskip}
%     The other options are listed below. Their default value is shown
%     between brackets, sometimes followed be a `\texttt{*}'.
%     The `\texttt{*}' means that the default shown applies when
%     |frenchb| is loaded as the \emph{last} option of \babel{}
%     ---\babel's \emph{main language}---, and is toggled otherwise:
%     \begin{itemize}
%     \item |StandardLayout=true [false*]| forces |frenchb| not to
%       interfere with the layout: no action on any kind of lists,
%       first paragraphs of sections are not indented (as in English),
%       no action on footnotes. This option replaces the former
%       command |\StandardLayout|.  It can be used to avoid conflicts
%       with classes or packages which customise lists or footnotes.
%     \item |GlobalLayoutFrench=false [true*]| can be used, when French
%       is the main language, to emulate what prior versions of
%       |frenchb| (pre-2.2) did: lists, and first paragraphs
%       of sections will be displayed the standard way in other
%       languages than French, and ``\`a la fran\c{c}aise'' in French.
%       Note that the layout of footnotes is language independent
%       anyway (see below |FrenchFootnotes| and |AutoSpaceFootnotes|).
%       This option replaces the former command |\FrenchLayout|.
%     \item |ReduceListSpacing=false [true*]|; |frenchb| normally
%       reduces the values of the vertical spaces used in the
%       environment |list| in French; setting this option to |false|
%       reverts to the standard settings of |list|.  This option
%       replaces the former command |\FrenchListSpacingfalse|.
%     \item |CompactItemize=false [true*]|; |frenchb| normally
%       suppresses any vertical space between items of |itemize| lists
%       in French; setting this option to |false| reverts to the
%       standard settings of |itemize| lists.  This option replaces
%       the former command |\FrenchItemizeSpacingfalse|.
%     \item |StandardItemLabels=true [false*]| when set to |true| this
%       option stops |frenchb| from changing the labels in |itemize|
%       lists in French.
%     \item |ItemLabels=\textemdash|, |\textbullet|, |\ding{43}|,
%       \dots, |[\textendash*]|; when |StandardItemLabels=false| (the
%       default), this option enables to choose the label used in
%       |itemize| lists for all levels.  The next three options do
%       the same but each one for one level only. Note that the
%       example |\ding{43}| requires |\usepackage{pifont}|.
%     \item |ItemLabeli=\textemdash|, |\textbullet|, |\ding{43}|,
%       \dots,|[\textendash*]|
%     \item |ItemLabelii=\textemdash|, |\textbullet|, |\ding{43}|,
%       \dots, |[\textendash*]|
%     \item |ItemLabeliii=\textemdash|, |\textbullet|, |\ding{43}|,
%       \dots, |[\textendash*]|
%     \item |ItemLabeliv=\textemdash|, |\textbullet|, |\ding{43}|,
%       \dots, |[\textendash*]|
%     \item |StandardLists=true [false*]| forbids |frenchb| to
%       customise any kind of list. Do activate the option
%       |StandardLists| when using classes or packages that customise
%       lists too (|enumitem|, |paralist|, \dots{}) to avoid conflicts.
%       This option is just a shorthand for |ReduceListSpacing=false|
%       and |CompactItemize=false| and |StandardItemLabels=true|.
%     \item |IndentFirst=false [true*]|; |frenchb| normally forces
%       indentation of the first paragraph of sections.
%       When this option is set to |false|, the first paragraph of
%       will look the same in French and in English (not indented).
%     \item |FrenchFootnotes=false [true*]| reverts to the standard
%       layout of footnotes. By default |frenchb| typesets leading
%       numbers as `1.\hspace{.5em}' instead of `$\hbox{}^1$', but
%       has no effect on footnotes numbered with symbols (as in the
%       |\thanks| command).  The former commands |\StandardFootnotes|
%       and |\FrenchFootnotes| are still there, |\StandardFootnotes|
%       can be useful when some footnotes are numbered with letters
%       (inside minipages for instance).
%     \item |AutoSpaceFootnotes=false [true*]| ; by default |frenchb|
%       adds a thin space in the running text before the number or
%       symbol calling the footnote.  Making this option |false|
%       reverts to the standard setting (no space added).
%     \item |FrenchSuperscripts=false [true]| ; then
%       |\up=\textsuperscript| (option added in version 2.1).
%       Should only be made |false| to recompile older documents.
%       By default |\up| now relies on |\fup| designed to produce
%       better looking superscripts.
%     \item |AutoSpacePunctuation=false [true]|; in French, the user
%       \emph{should} input a space before the four characters `|:;!?|'
%       but as many people forget about it (even among native French
%       writers!), the default behaviour of |frenchb| is to
%       automatically add a |\thinspace| before `|;|' `|!|' `|?|' and a
%       normal (unbreakable) space before~`|:|' (this is recommended by
%       the French Imprimerie nationale).  This is convenient in most
%       cases but can lead to addition of spurious spaces in URLs or in
%       MS-DOS paths but only if they are no typed using |\texttt| or
%       verbatim mode. When the current font is a monospaced
%       (typewriter) font, |AutoSpacePunctuation| is locally switched
%       to |false|, no spurious space is added in that case, so the
%       default behaviour of of |frenchb| in that area should be fine
%       in most circumstances.
%
%       Choosing |AutoSpacePunctuation=false| will ensure that
%       a proper space will be added before `|:;!?|' \emph{if and only
%       if} a (normal) space has been typed in. Those who are unsure
%       about their typing in this area should stick to the default
%       option and type |\string;| |\string:| |\string!| |\string?|
%       instead of |;| |:| |!| |?| in case an unwanted space is
%       added by |frenchb|.
%     \item |ThinColonSpace=true [false]| changes the normal
%       (unbreakable) space added before the colon `:' to a thin space,
%       so that the same amount of space is added before any of the
%       four double punctuation characters. The default setting is
%       supported by the French Imprimerie nationale.
%     \item |LowercaseSuperscripts=false [true]| ; by default |frenchb|
%       inhibits the uppercasing of superscripts (for instance when they
%       are moved to page headers). Making this option |false|
%       will disable this behaviour (not recommended).
%     \item |PartNameFull=false [true]|; when true, |frenchb| numbers
%       the title of |\part{}| commands as ``Premi\`ere partie'',
%       ``Deuxi\`eme partie'' and so on. With some classes which change
%       the|\part{}| command (AMS and SMF classes do so), you will get
%       ``Premi\`ere partie~I'', ``Deuxi\`eme partie~II'' instead;
%       when this occurs, this option should be set to |false|,
%       part titles will then be printed as ``Partie I'', ``Partie II''.
%     \item |og=|\texttt{\guillemotleft}, |fg=|\texttt{\guillemotright};
%       when guillemets characters are available on the keyboard
%       (through a compose key for instance), it is nice to use them
%       instead of typing |\og| and |\fg|. This option tells |frenchb|
%       which characters are opening and closing French guillemets
%       (they depend on the input encoding), then you can type either
%       \texttt{\guillemotleft{} guillemets \guillemotright}, or
%       \texttt{\guillemotleft{}guillemets\guillemotright} (with or
%       without spaces), to get properly typeset French quotes.
%       This option requires \file{inputenc} to be loaded with the
%       proper encoding, it works with 8-bits encodings (latin1,
%       latin9, ansinew,  applemac,\dots) and multi-byte encodings
%       (utf8 and utf8x).
%     \end{itemize}
%
%  \subsection{Hyphenation checks}
%  \label{ssec-hyphen}
%
%    Once you have built your format, a good precaution would be to
%    perform some basic tests about hyphenation in French. For
%    \LaTeXe{} I suggest this:
%    \begin{itemize}
%    \item run the following file, with the encoding suitable for
%      your machine (\textit{my-encoding} will be |latin1| for
%      \textsc{unix} machines, |ansinew| for PCs running~Windows,
%      |applemac| or |latin1| for Macintoshs, or |utf8|\dots\\[3mm]^^A\]
%      |%%% Test file for French hyphenation.|\\
%      |\documentclass{article}|\\
%      |\usepackage[|\textit{my-encoding}|]{inputenc}|\\
%      |\usepackage[T1]{fontenc} % Use LM fonts|\\
%      |\usepackage{lmodern}     % for French|\\
%      |\usepackage[frenchb]{babel}|\\
%      |\begin{document}|\\
%      |\showhyphens{signal container \'ev\'enement alg\`ebre}|\\
%      |\showhyphens{|\texttt{signal container \'ev\'enement
%                     alg\`ebre}|}|\\
%      |\end{document}|
%    \item check the hyphenations proposed by \TeX{} in your log-file;
%      in French you should get with both 7-bit and 8-bit encodings\\
%      \texttt{si-gnal contai-ner \'ev\'e-ne-ment al-g\`ebre}.\\
%      Do not care about how accented characters are displayed in the
%      log-file, what matters is the position of the `|-|' hyphen
%      signs \emph{only}.
%    \end{itemize}
%    If they are all correct, your installation (probably) works fine,
%    if one (or more) is (are) wrong, ask a local wizard to see what's
%    going wrong and perform the test again (or e-mail me about what
%    happens).\\
%    Frequent mismatches:
%    \begin{itemize}
%    \item you get |sig-nal con-tainer|, this probably means that the
%    hyphenation patterns you are using are for US-English, not for
%    French;
%    \item you get no hyphen at all in \texttt{\'ev\'e-ne-ment}, this
%    probably means that you are using CM fonts and the macro
%    |\accent| to produce accented characters.
%    Using 8-bits fonts with built-in accented characters avoids
%    this kind of mismatch.
%    \end{itemize}
%
%    \textbf{Options' order} -- Please remember that options are read
%    in the order they appear inside the |\frenchbsetup| command.
%    Someone wishing that |frenchb| leaves the layout of lists
%    and footnotes untouched but caring for indentation of first
%    paragraph of sections could choose
%    |\frenchbsetup{StandardLayout,IndentFirst}| and get the expected
%    layout. Choosing |\frenchbsetup{IndentFirst,StandardLayout}|
%    would not lead to the expected result: option |IndentFirst| would
%    be overwritten by |StandardLayout|.
%
%  \subsection{Changes}
%  \label{ssec-changes}
%
%  \subsubsection*{What's new in version 2.0?}
%
%    Here is the list of all changes:
%    \begin{itemize}
%    \item Support for \LaTeX-2.09 and for \LaTeXe{} in compatibility
%      mode has been dropped. This version is meant for \LaTeXe{} and
%      Plain based formats (like \file{bplain}). \LaTeXe{} formats
%      based on ml\TeX{} are no longer supported either (plenty of
%      good 8-bits fonts are available now, so T1 encoding should
%      be preferred for typesetting in French). A warning is issued
%      when OT1 encoding is in use at the |\begin{document}|.
%    \item Customisation should now be handled by command
%      |\frenchbsetup{}|, \file{frenchb.cfg} (kept for compatibility)
%      should no longer be used. See section~\ref{ssec-custom} for
%      the list of available options.
%    \item Captions in figures and table have changed in French: former
%      abbreviations ``Fig.'' and ``Tab.'' have been replaced by full
%      names ``Figure'' and ``Table''.  If this leads to formatting
%      problems in captions, you can add the following two commands to
%      your preamble (after loading \babel) to get the former captions\\
%      |\addto\captionsfrench{\def\figurename{{\scshape Fig.}}}|\\
%      |\addto\captionsfrench{\def\tablename{{\scshape Tab.}}}|.
%    \item The |\nombre| command is now provided by the \file{numprint}
%      package which has to be loaded \emph{after} \babel{} with the
%      option |autolanguage| if number formatting should depend on the
%      current language.
%    \item The |\bsc| command no longer uses an |\hbox| to stop
%      hyphenation of names but a |\kern0pt| instead. This change
%      enables \file{microtype} to fine tune the length of the
%      argument of |\bsc|; as a side-effect, compound names like
%      Dupont-Durand can now be hyphenated on  explicit hyphens.
%      You can get back to the former behaviour of |\bsc| by adding\\
%      |\renewcommand*{\bsc}[1]{\leavevmode\hbox{\scshape #1}}|\\
%      to the preamble of your document.
%    \item Footnotes are now displayed ``\`a la fran\c caise'' for the
%      whole document, except with an explicit\\
%      |\frenchbsetup{AutoSpaceFootnotes=false,FrenchFootnotes=false}|.\\
%      Add this command if you want standard footnotes. It is still
%      possible to revert locally to the standard layout of footnotes
%      by adding |\StandardFootnotes| (inside a |minipage| environment
%      for instance).
%    \end{itemize}
%
%  \subsubsection*{What's new in version 2.1?}
%
%      New command |\fup| to typeset better looking superscripts.
%      Former command |\up| is now defined as |\fup|, but an option
%      |\frenchbsetup{FrenchSuperscripts=false}| is provided for
%      backward compatibility.  |\fup| was designed using ideas from
%      Jacques Andr\'e, Thierry Bouche and Ren\'e Fritz, thanks to them!
%
%  \subsubsection*{What's new in version 2.2?}
%
%      Starting with version~2.2a, |frenchb| alters the layout of
%      lists, footnotes, and the indentation of first paragraphs of
%      sections) \emph{only if} French is the ``main language''
%      (i.e. babel's last language option). The layout is global for
%      the whole document: lists, etc. look the same in French and in
%      other languages, everything is typeset ``\`a la fran\c caise''
%      if French is the ``main language'', otherwise |frenchb| doesn't
%      change anything regarding lists, footnotes, and indentation of
%      paragraphs.
%
%  \subsubsection*{What's new in version 2.3?}
%
%      Starting with version~2.3a, |frenchb| no longer inserts spaces
%      automatically before `|:;!?|' when a typewriter font is in use;
%      this was suggested by Yannis Haralambous to prevent
%      spurious spaces in computer source code or expressions like
%      \texttt{C\string:/foo}, \texttt{http\string://foo.bar},
%      etc.  An option (|OriginalTypewriter|) is provided to get back
%      to the former behaviour of |frenchb|.
%
%      Another probably invisible change: lowercase conversion in
%      |\up{}| is now achieved by the \LaTeX{} command |\MakeLowercase|
%      instead of \TeX's |\lowercase| command.  This prevents error
%      messages when diacritics are used inside |\up{}| (diacritics
%      should \emph{never} be used in superscripts though!).
%
% \StopEventually{}
%
%  \subsection{File frenchb.cfg}
%  \label{sec-cfg}
%
%    \file{frenchb.cfg} is now a dummy file just kept for compatibility
%    with previous versions.
%
% \iffalse
%<*cfg>
% \fi
%    \begin{macrocode}
%%%%%%%%%%%%%%%%%%%%%%%%%%%%%%%%%%%%%%%%%%%%%%%%%%%%%%%%%%%%%%%%%%%%%%
%%%%%%%%%  WARNING: THIS  FILE SHOULD  NO  LONGER  BE  USED  %%%%%%%%%
%% If you want to customise frenchb, please DO NOT hack into the code!
%% Do no put any code in this file either, please use the new command
%% \frenchbsetup{} with the proper options to customise frenchb.
%% 
%% Add \frenchbsetup{ShowOptions} to your preamble to see the list of
%% available options and/or read the documentation.
%%%%%%%%%%%%%%%%%%%%%%%%%%%%%%%%%%%%%%%%%%%%%%%%%%%%%%%%%%%%%%%%%%%%%%
%    \end{macrocode}
% \iffalse
%</cfg>
% \fi
%
%  \section{\TeX{}nical details}
%
%  \subsection{Initial setup}
%
% \changes{v2.1d}{2008/05/02}{Argument of \cs{ProvidesLanguage} changed
%     above from `french' to `frenchb' (otherwise \cs{listfiles} prints
%     no date/version information).  The real name of current language
%     (french) as to be corrected before calling \cs{LdfInit}.}
%
% \iffalse
%<*code>
% \fi
%
%    While this file was read through the option \Lopt{frenchb} we make
%    it behave as if \Lopt{french} was specified.
%    \begin{macrocode}
\def\CurrentOption{french}
%    \end{macrocode}
%
%    The macro |\LdfInit| takes care of preventing that this file is
%    loaded more than once, checking the category code of the
%    \texttt{@} sign, etc.
%
%    \begin{macrocode}
\LdfInit\CurrentOption\datefrench
%    \end{macrocode}
%
% \changes{v2.1d}{2008/05/04}{Avoid warning ``\cs{end} occurred
%   when \cs{ifx} ... incomplete'' with LaTeX-2.09.}
%
%  \begin{macro}{\ifLaTeXe}
%    No support is provided for late \LaTeX-2.09: issue a warning
%    and exit if \LaTeX-2.09 is in use. Plain is still supported.
%    \begin{macrocode}
\newif\ifLaTeXe
\let\bbl@tempa\relax
\ifx\magnification\@undefined
   \ifx\@compatibilitytrue\@undefined
     \PackageError{frenchb.ldf}
        {LaTeX-2.09 format is no longer supported.\MessageBreak
         Aborting here}
        {Please upgrade to LaTeX2e!}
     \let\bbl@tempa\endinput
   \else
     \LaTeXetrue
   \fi
\fi
\bbl@tempa
%    \end{macrocode}
%  \end{macro}
%
%    Check if hyphenation patterns for the French language have been
%    loaded in language.dat; we allow for the names `french',
%    `francais', `canadien' or `acadian'. The latter two are both
%    names used in Canada for variants of French that are in use in
%    that country.
%
%    \begin{macrocode}
\ifx\l@french\@undefined
  \ifx\l@francais\@undefined
    \ifx\l@canadien\@undefined
      \ifx\l@acadian\@undefined
        \@nopatterns{French}
        \adddialect\l@french0
      \else
        \let\l@french\l@acadian
      \fi
    \else
      \let\l@french\l@canadien
    \fi
  \else
    \let\l@french\l@francais
  \fi
\fi
%    \end{macrocode}
%    Now |\l@french| is always defined.
%
%    The internal name for the French language is |french|;
%    |francais| and |frenchb| are synonymous for |french|:
%    first let both names use the same hyphenation patterns.
%    Later we will have to set aliases for |\captionsfrench|,
%    |\datefrench|, |\extrasfrench| and |\noextrasfrench|.
%    As French uses the standard values of |\lefthyphenmin| (2)
%    and |\righthyphenmin| (3), no special setting is required here.
%
%    \begin{macrocode}
\ifx\l@francais\@undefined
  \let\l@francais\l@french
\fi
\ifx\l@frenchb\@undefined
  \let\l@frenchb\l@french
\fi
%    \end{macrocode}
%    When this language definition file was loaded for one of the
%    Canadian versions of French we need to make sure that a suitable
%    hyphenation pattern register will be found by \TeX.
%    \begin{macrocode}
\ifx\l@canadien\@undefined
  \let\l@canadien\l@french
\fi
\ifx\l@acadian\@undefined
  \let\l@acadian\l@french
\fi
%    \end{macrocode}
%
%    This language definition can be loaded for different variants of
%    the French language. The `key' \babel\ macros are only defined
%    once, using `french' as the language name, but |frenchb| and
%    |francais| are synonymous.
%    \begin{macrocode}
\def\datefrancais{\datefrench}
\def\datefrenchb{\datefrench}
\def\extrasfrancais{\extrasfrench}
\def\extrasfrenchb{\extrasfrench}
\def\noextrasfrancais{\noextrasfrench}
\def\noextrasfrenchb{\noextrasfrench}
%    \end{macrocode}
%
% \begin{macro}{\extrasfrench}
% \begin{macro}{\noextrasfrench}
%    The macro |\extrasfrench| will perform all the extra
%    definitions needed for the French language.
%    The macro |\noextrasfrench| is used to cancel the actions of
%    |\extrasfrench|.\\
%    In French, character ``apostrophe'' is a letter in expressions
%    like |l'ambulance| (French  hyphenation patterns provide entries
%    for this kind of words).  This means that the |\lccode| of
%    ``apostrophe'' has to be non null in French for proper hyphenation
%    of those expressions, and has to be reset to null when exiting
%    French.
%
%    \begin{macrocode}
\@namedef{extras\CurrentOption}{\lccode`\'=`\'}
\@namedef{noextras\CurrentOption}{\lccode`\'=0}
%    \end{macrocode}
% \end{macro}
% \end{macro}
%
%    One more thing |\extrasfrench| needs to do is to make sure that
%    |\frenchspacing| is in effect.  |\noextrasfrench| will switch
%    |\frenchspacing| off again.
%    \begin{macrocode}
  \expandafter\addto\csname extras\CurrentOption\endcsname{%
    \bbl@frenchspacing}
  \expandafter\addto\csname noextras\CurrentOption\endcsname{%
    \bbl@nonfrenchspacing}
%    \end{macrocode}
%
%  \subsection{Punctuation}
%  \label{sec-punct}
%
%    As long as no better solution is available%
%    \footnote{Lua\TeX, or pdf\TeX{} might provide alternatives in
%       the future\dots},
%    the `double punctuation' characters (|;| |!| |?| and |:|) have to
%    be made |\active| for an automatic control of the amount of space
%    to insert before them. Before doing so, we have to save the
%    standard definition of |\@makecaption| (which includes two ':')
%    to compare it later to its definition at the |\begin{document}|.
%    \begin{macrocode}
\long\def\STD@makecaption#1#2{%
  \vskip\abovecaptionskip
  \sbox\@tempboxa{#1: #2}%
  \ifdim \wd\@tempboxa >\hsize
    #1: #2\par
  \else
    \global \@minipagefalse
    \hb@xt@\hsize{\hfil\box\@tempboxa\hfil}%
  \fi
  \vskip\belowcaptionskip}%
%    \end{macrocode}
%
%    We define a new `if' |\FBpunct@active| which will be made false
%    whenever a better alternative will be available. Another `if'
%    |\FBAutoSpacePunctuation| needs to be defined now.
%    \begin{macrocode}
\newif\ifFBpunct@active          \FBpunct@activetrue
\newif\ifFBAutoSpacePunctuation  \FBAutoSpacePunctuationtrue
%    \end{macrocode}
%    The following code makes the four characters |;| |!| |?| and |:|
%    `active' and provides their definitions.
%    \begin{macrocode}
\ifFBpunct@active
  \initiate@active@char{:}
  \initiate@active@char{;}
  \initiate@active@char{!}
  \initiate@active@char{?}
%    \end{macrocode}
%    We first tune the amount of space before \texttt{;}
%    \texttt{!}  \texttt{?} and \texttt{:}.  This should only happen
%    in horizontal mode, hence the test |\ifhmode|.
%
%    In horizontal mode, if a space has been typed before `;' we
%    remove it and put an unbreakable |\thinspace| instead. If no
%    space has been typed, we add |\FDP@thinspace| which will be
%    defined, up to the user's wishes, as an automatic added
%    thin space, or as |\@empty|.
%    \begin{macrocode}
  \declare@shorthand{french}{;}{%
      \ifhmode
      \ifdim\lastskip>\z@
          \unskip\penalty\@M\thinspace
          \else
            \FDP@thinspace
        \fi
      \fi
%    \end{macrocode}
%    Now we can insert a |;| character.
%    \begin{macrocode}
      \string;}
%    \end{macrocode}
%    The next three definitions are very similar.
%    \begin{macrocode}
  \declare@shorthand{french}{!}{%
      \ifhmode
        \ifdim\lastskip>\z@
          \unskip\penalty\@M\thinspace
        \else
          \FDP@thinspace
        \fi
      \fi
      \string!}
  \declare@shorthand{french}{?}{%
      \ifhmode
        \ifdim\lastskip>\z@
          \unskip\penalty\@M\thinspace
        \else
          \FDP@thinspace
        \fi
      \fi
      \string?}
%    \end{macrocode}
%    According to the I.N. specifications, the `:' requires a normal
%    space before it, but some people prefer a |\thinspace| (just
%    like the other three). We define |\Fcolonspace| to hold the
%    required amount of space (user customisable).
%    \begin{macrocode}
  \newcommand*{\Fcolonspace}{\space}
  \declare@shorthand{french}{:}{%
      \ifhmode
        \ifdim\lastskip>\z@
          \unskip\penalty\@M\Fcolonspace
        \else
          \FDP@colonspace
        \fi
      \fi
      \string:}
%    \end{macrocode}
%
% \changes{v2.3a}{2008/10/10}{\cs{NoAutoSpaceBeforeFDP} and
%    \cs{AutoSpaceBeforeFDP} now set the flag
%    \cs{ifFBAutoSpacePunctuation} accordingly (LaTeX only).}
%
%  \begin{macro}{\AutoSpaceBeforeFDP}
%  \begin{macro}{\NoAutoSpaceBeforeFDP}
%    |\FDP@thinspace| and |\FDP@space| are defined as unbreakable
%    spaces by |\autospace@beforeFDP| or as |\@empty| by
%    |\noautospace@beforeFDP| (internal commands), user commands
%    |\AutoSpaceBeforeFDP| and |\NoAutoSpaceBeforeFDP| do the same and
%    take care of the flag |\ifFBAutoSpacePunctuation| in \LaTeX{}.
%    Set the default now for Plain (done later for \LaTeX).
%    \begin{macrocode}
  \def\autospace@beforeFDP{%
          \def\FDP@thinspace{\penalty\@M\thinspace}%
          \def\FDP@colonspace{\penalty\@M\Fcolonspace}}
  \def\noautospace@beforeFDP{\let\FDP@thinspace\@empty
                            \let\FDP@colonspace\@empty}
  \ifLaTeXe
    \def\AutoSpaceBeforeFDP{\autospace@beforeFDP
                            \FBAutoSpacePunctuationtrue}
    \def\NoAutoSpaceBeforeFDP{\noautospace@beforeFDP
                              \FBAutoSpacePunctuationfalse}
  \else
    \let\AutoSpaceBeforeFDP\autospace@beforeFDP
    \let\NoAutoSpaceBeforeFDP\noautospace@beforeFDP
    \AutoSpaceBeforeFDP
  \fi
%    \end{macrocode}
% \end{macro}
% \end{macro}
%
% \changes{v2.3a}{2008/10/10}{In LaTeX, frenchb no longer adds spaces
%     before `double punctuation' characters in computer code.
%     Suggested by Yannis Haralambous.}
%
% \changes{v2.3c}{2009/02/07}{Commands \cs{ttfamily}, \cs{rmfamily}
%    and \cs{sffamily} have to be robust.  Bug introduced in 2.3a,
%    pointed out by Manuel P\'egouri\'e-Gonnard.}
%
%    In \LaTeXe{} |\ttfamily| (and hence |\texttt|) will be redefined
%    `AtBeginDocument' as |\ttfamilyFB| so that no space
%    is added before the four |; : ! ?| characters, even if
%    |AutoSpacePunctuation| is true.  |\rmfamily| and |\sffamily| need
%    to be redefined also (|\ttfamily| is not always used inside a
%    group, its effect can be cancelled by |\rmfamily| or |\sffamily|).
%
%    These redefinitions can be canceled if necessary, for instance to
%    recompile older documents, see option |OriginalTypewriter| below.
%    \begin{macrocode}
  \ifLaTeXe
    \let\ttfamilyORI\ttfamily
    \let\rmfamilyORI\rmfamily
    \let\sffamilyORI\sffamily
    \DeclareRobustCommand\ttfamilyFB{%
         \noautospace@beforeFDP\ttfamilyORI}%
    \DeclareRobustCommand\rmfamilyFB{%
         \ifFBAutoSpacePunctuation
            \autospace@beforeFDP\rmfamilyORI
         \else
            \noautospace@beforeFDP\rmfamilyORI
         \fi}%
    \DeclareRobustCommand\sffamilyFB{%
         \ifFBAutoSpacePunctuation
            \autospace@beforeFDP\sffamilyORI
         \else
            \noautospace@beforeFDP\sffamilyORI
         \fi}%
  \fi
%    \end{macrocode}
%
%    When the active characters appear in an environment where their
%    French behaviour is not wanted they should give an `expected'
%    result. Therefore we define shorthands at system level as well.
%    \begin{macrocode}
  \declare@shorthand{system}{:}{\string:}
  \declare@shorthand{system}{!}{\string!}
  \declare@shorthand{system}{?}{\string?}
  \declare@shorthand{system}{;}{\string;}
%    \end{macrocode}
%    We specify that the French group of shorthands should be used.
%    \begin{macrocode}
  \addto\extrasfrench{%
    \languageshorthands{french}%
%    \end{macrocode}
%    These characters are `turned on' once, later their definition may
%    vary. Don't misunderstand the following code: they keep being
%    active all along the document, even when leaving French.
%    \begin{macrocode}
    \bbl@activate{:}\bbl@activate{;}%
    \bbl@activate{!}\bbl@activate{?}%
  }
  \addto\noextrasfrench{%
  \bbl@deactivate{:}\bbl@deactivate{;}%
  \bbl@deactivate{!}\bbl@deactivate{?}}
\fi
%    \end{macrocode}
%
%  \subsection{French quotation marks}
%
%  \begin{macro}{\og}
%  \begin{macro}{\fg}
%    The top macros for quotation marks will be called |\og|
%    (``\underline{o}uvrez \underline{g}uillemets'') and |\fg|
%    (``\underline{f}ermez \underline{g}uillemets'').
%    Another option for typesetting quotes in multilingual texts
%    is to use the package |csquotes.sty| and its command |\enquote|.
%
%    \begin{macrocode}
\newcommand*{\og}{\@empty}
\newcommand*{\fg}{\@empty}
%    \end{macrocode}
%  \end{macro}
%  \end{macro}
%
%  \begin{macro}{\guillemotleft}
%  \begin{macro}{\guillemotright}
%    \LaTeX{} users are supposed to use 8-bit output encodings (T1,
%    LY1,\dots) to typeset French, those who still stick to OT1 should
%    call |aeguill.sty| or a similar package. In both cases the
%    commands |\guillemotleft| and |\guillemotright| will print the
%    French opening and closing quote characters from the output font.
%    For XeLaTeX, |\guillemotleft| and |\guillemotright| are defined
%    by package \file{xunicode.sty}.
%    We will check `AtBeginDocument' that the proper output encodings
%    are in use (see end of section~\ref{sec-keyval}).
%
%    We give the following definitions for Plain users only as a (poor)
%    fall-back, they are welcome to change them for anything better.
%    \begin{macrocode}
\ifLaTeXe
\else
  \ifx\guillemotleft\@undefined
    \def\guillemotleft{\leavevmode\raise0.25ex
                       \hbox{$\scriptscriptstyle\ll$}}
  \fi
  \ifx\guillemotright\@undefined
    \def\guillemotright{\raise0.25ex
                        \hbox{$\scriptscriptstyle\gg$}}
  \fi
  \let\xspace\relax
\fi
%    \end{macrocode}
%  \end{macro}
%  \end{macro}
%
%    The next step is to provide correct spacing after |\guillemotleft|
%    and before |\guillemotright|: a space precedes and follows
%    quotation marks but no line break is allowed neither \emph{after}
%    the opening one, nor \emph{before} the closing one.
%    |\FBguill@spacing| which does the spacing, has been fine tuned by
%    Thierry Bouche.  French quotes (including spacing) are printed by
%    |\FB@og| and |\FB@fg|, the expansion of the top level commands
%    |\og| and |\og| is different in and outside French.
%    We'll try to be smart to users of David~Carlisle's |xspace|
%    package: if this package is loaded there will be no need for |{}|
%    or |\ | to get a space after |\fg|, otherwise |\xspace| will be
%    defined as |\relax| (done at the end of this file).
%
%    \begin{macrocode}
\newcommand*{\FBguill@spacing}{\penalty\@M\hskip.8\fontdimen2\font
                                            plus.3\fontdimen3\font
                                           minus.8\fontdimen4\font}
\DeclareRobustCommand*{\FB@og}{\leavevmode
                               \guillemotleft\FBguill@spacing}
\DeclareRobustCommand*{\FB@fg}{\ifdim\lastskip>\z@\unskip\fi
                               \FBguill@spacing\guillemotright\xspace}
%    \end{macrocode}
%
%    The top level definitions for French quotation marks are switched
%    on and off through the |\extrasfrench| |\noextrasfrench|
%    mechanism. Outside French, |\og| and |\fg| will typeset standard
%    English opening and closing double quotes.
%
%    \begin{macrocode}
\ifLaTeXe
  \def\bbl@frenchguillemets{\renewcommand*{\og}{\FB@og}%
                            \renewcommand*{\fg}{\FB@fg}}
  \def\bbl@nonfrenchguillemets{\renewcommand*{\og}{\textquotedblleft}%
            \renewcommand*{\fg}{\ifdim\lastskip>\z@\unskip\fi
                                   \textquotedblright}}
\else
   \def\bbl@frenchguillemets{\let\og\FB@og
                             \let\fg\FB@fg}
   \def\bbl@nonfrenchguillemets{\def\og{``}%
                     \def\fg{\ifdim\lastskip>\z@\unskip\fi ''}}
\fi
\expandafter\addto\csname extras\CurrentOption\endcsname{%
  \bbl@frenchguillemets}
\expandafter\addto\csname noextras\CurrentOption\endcsname{%
  \bbl@nonfrenchguillemets}
%    \end{macrocode}
%
%  \subsection{Date in French}
%
% \begin{macro}{\datefrench}
%    The macro |\datefrench| redefines the command |\today| to
%    produce French dates.
%
% \changes{v2.0}{2006/11/06}{2 '\cs{relax}' added in
%    \cs{today}'s definition.}
%
% \changes{v2.1a}{2008/03/25}{\cs{today} changed (correction in 2.0
%    was wrong: \cs{today} was printed without spaces in toc).}
%
%    \begin{macrocode}
\@namedef{date\CurrentOption}{%
  \def\today{{\number\day}\ifnum1=\day {\ier}\fi \space
    \ifcase\month
      \or janvier\or f\'evrier\or mars\or avril\or mai\or juin\or
      juillet\or ao\^ut\or septembre\or octobre\or novembre\or
      d\'ecembre\fi
    \space \number\year}}
%    \end{macrocode}
% \end{macro}
%
%  \subsection{Extra utilities}
%
%    Let's provide the French user with some extra utilities.
%
% \changes{v2.1a}{2008/03/24}{Command \cs{fup} added to produce
%    better superscripts than \cs{textsuperscript}.}
%
%  \begin{macro}{\up}
%
% \changes{v2.1c}{2008/04/29}{Provide a temporary definition
%    (hyperref safe) of \cs{up} in case it has to be expanded in
%    the preamble (by beamer's \cs{title} command for instance).}
%
%  \begin{macro}{\fup}
%
% \changes{v2.1b}{2008/04/02}{Command \cs{fup} changed to use
%    real superscripts from fourier v. 1.6.}
%
% \changes{v2.2a}{2008/05/08}{\cs{newif} and \cs{newdimen} moved
%    before \cs{ifLaTeXe} to avoid an error with plainTeX.}
%
% \changes{v2.3a}{2008/09/30}{\cs{lowercase} changed to
%    \cs{MakeLowercase} as the former doesn't work for non ASCII
%    characters in encodings like applemac, utf-8,\dots}
%
%    |\up| eases the typesetting of superscripts like
%    `1\textsuperscript{er}'.  Up to version 2.0 of |frenchb| |\up| was
%    just a shortcut for |\textsuperscript| in \LaTeXe, but several
%    users complained that |\textsuperscript| typesets superscripts
%    too high and too big, so we now define |\fup| as an attempt to
%    produce better looking superscripts.  |\up| is defined as |\fup|
%    but can be redefined by |\frenchbsetup{FrenchSuperscripts=false}|
%    as |\textsuperscript| for compatibility with previous versions.
%
%    When a font has built-in superscripts, the best thing to do is
%    to just use them, otherwise |\fup| has to simulate superscripts
%    by scaling and raising ordinary letters.  Scaling is done using
%    package \file{scalefnt} which will be loaded at the end of
%    \babel's loading (|frenchb| being an option of babel, it cannot
%    load a package while being read).
%
%    \begin{macrocode}
\newif\ifFB@poorman
\newdimen\FB@Mht
\ifLaTeXe
  \AtEndOfPackage{\RequirePackage{scalefnt}}
%    \end{macrocode}
%    |\FB@up@fake| holds the definition of fake superscripts.
%    The scaling ratio is 0.65, raising is computed to put the top of
%    lower case letters (like `m') just under the top  of upper case
%    letters (like `M'), precisely 12\% down.  The chosen settings
%    look correct for most fonts, but can be tuned by the end-user
%    if necessary by changing |\FBsupR| and |\FBsupS| commands.
%
%    |\FB@lc| is defined as |\MakeLowercase| to inhibit the uppercasing
%    of superscripts (this may happen in page headers with the standard
%    classes but is wrong); |\FB@lc| can be redefined to do nothing
%    by option |LowercaseSuperscripts=false| of |\frenchbsetup{}|.
%    \begin{macrocode}
  \newcommand*{\FBsupR}{-0.12}
  \newcommand*{\FBsupS}{0.65}
  \newcommand*{\FB@lc}[1]{\MakeLowercase{#1}}
  \DeclareRobustCommand*{\FB@up@fake}[1]{%
    \settoheight{\FB@Mht}{M}%
    \addtolength{\FB@Mht}{\FBsupR \FB@Mht}%
    \addtolength{\FB@Mht}{-\FBsupS ex}%
    \raisebox{\FB@Mht}{\scalefont{\FBsupS}{\FB@lc{#1}}}%
    }
%    \end{macrocode}
%    The only packages I currently know to take advantage of real
%    superscripts are a) \file{xltxtra} used in conjunction with
%    XeLaTeX and OpenType fonts having the font feature
%    'VerticalPosition=Superior' (\file{xltxtra} defines
%    |\realsuperscript| and |\fakesuperscript|) and b) \file{fourier}
%    (from version 1.6) when Expert Utopia fonts are available.
%
%    |\FB@up| checks whether the current font is a Type1 `Expert'
%    (or `Pro') font with real superscripts or not (the code works
%    currently only with \file{fourier-1.6} but could work with any
%    Expert Type1 font with built-in superscripts, see below), and
%    decides to use real or fake superscripts.
%    It works as follows: the content of |\f@family| (family name of
%    the current font) is split by |\FB@split| into two pieces, the
%    first three characters (`|fut|' for Fourier, `|ppl|' for Adobe's
%    Palatino, \dots) stored in |\FB@firstthree| and the rest stored
%    in |\FB@suffix| which is expected to be `|x|' or `|j|' for expert
%    fonts.
%    \begin{macrocode}
  \def\FB@split#1#2#3#4\@nil{\def\FB@firstthree{#1#2#3}%
                             \def\FB@suffix{#4}}
  \def\FB@x{x}
  \def\FB@j{j}
  \DeclareRobustCommand*{\FB@up}[1]{%
    \bgroup \FB@poormantrue
      \expandafter\FB@split\f@family\@nil
%    \end{macrocode}
%    Then |\FB@up| looks for a \file{.fd} file named \file{t1fut-sup.fd}
%    (Fourier) or \file{t1ppl-sup.fd} (Palatino), etc. supposed to
%    define the subfamily (|fut-sup| or |ppl-sup|, etc.) giving access
%    to the built-in superscripts.  If the \file{.fd} file is not found
%    by |\IfFileExists|, |\FB@up| falls back on fake superscripts,
%    otherwise |\FB@suffix| is checked to decide whether to use fake or
%    real superscripts.
%    \begin{macrocode}
      \edef\reserved@a{\lowercase{%
         \noexpand\IfFileExists{\f@encoding\FB@firstthree -sup.fd}}}%
      \reserved@a
        {\ifx\FB@suffix\FB@x \FB@poormanfalse\fi
         \ifx\FB@suffix\FB@j \FB@poormanfalse\fi
         \ifFB@poorman \FB@up@fake{#1}%
         \else         \FB@up@real{#1}%
         \fi}%
        {\FB@up@fake{#1}}%
    \egroup}
%    \end{macrocode}
%    |\FB@up@real| just picks up the superscripts from the subfamily
%    (and forces lowercase).
%    \begin{macrocode}
  \newcommand*{\FB@up@real}[1]{\bgroup
       \fontfamily{\FB@firstthree -sup}\selectfont \FB@lc{#1}\egroup}
%    \end{macrocode}
%    |\fup| is now defined as |\FB@up| unless |\realsuperscript| is
%    defined (occurs with XeLaTeX calling \file{xltxtra.sty}).
%    \begin{macrocode}
  \DeclareRobustCommand*{\fup}[1]{%
    \@ifundefined{realsuperscript}%
      {\FB@up{#1}}%
      {\bgroup\let\fakesuperscript\FB@up@fake
            \realsuperscript{\FB@lc{#1}}\egroup}}
%    \end{macrocode}
%    Temporary definition of |up| (redefined `AtBeginDocument').
%    \begin{macrocode}
  \newcommand*{\up}{\relax}
%    \end{macrocode}
%    Poor man's definition of |\up| for Plain. In \LaTeXe,
%    |\up| will be defined as |\fup| or |\textsuperscript| later on
%    while processing the options of |\frenchbsetup{}|.
%    \begin{macrocode}
\else
  \newcommand*{\up}[1]{\leavevmode\raise1ex\hbox{\sevenrm #1}}
\fi
%    \end{macrocode}
%  \end{macro}
%  \end{macro}
%
%  \begin{macro}{\ieme}
%  \begin{macro}{\ier}
%  \begin{macro}{\iere}
%  \begin{macro}{\iemes}
%  \begin{macro}{\iers}
%  \begin{macro}{\ieres}
%  Some handy macros for those who don't know how to abbreviate ordinals:
%    \begin{macrocode}
\def\ieme{\up{\lowercase{e}}\xspace}
\def\iemes{\up{\lowercase{es}}\xspace}
\def\ier{\up{\lowercase{er}}\xspace}
\def\iers{\up{\lowercase{ers}}\xspace}
\def\iere{\up{\lowercase{re}}\xspace}
\def\ieres{\up{\lowercase{res}}\xspace}
%    \end{macrocode}
%  \end{macro}
%  \end{macro}
%  \end{macro}
%  \end{macro}
%  \end{macro}
%  \end{macro}
%
% \changes{v2.1c}{2008/04/29}{Added commands \cs{Nos} and \cs{nos}.}
%
%  \begin{macro}{\No}
%  \begin{macro}{\no}
%  \begin{macro}{\Nos}
%  \begin{macro}{\nos}
%  \begin{macro}{\primo}
%  \begin{macro}{\fprimo)}
%    And some more macros relying on |\up| for numbering,
%    first two support macros.
%    \begin{macrocode}
\newcommand*{\FrenchEnumerate}[1]{%
                       #1\up{\lowercase{o}}\kern+.3em}
\newcommand*{\FrenchPopularEnumerate}[1]{%
                       #1\up{\lowercase{o}})\kern+.3em}
%    \end{macrocode}
%
%    Typing |\primo| should result in `$1^{\rm o}$\kern+.3em',
%    \begin{macrocode}
\def\primo{\FrenchEnumerate1}
\def\secundo{\FrenchEnumerate2}
\def\tertio{\FrenchEnumerate3}
\def\quarto{\FrenchEnumerate4}
%    \end{macrocode}
%    while typing |\fprimo)| gives `1$^{\rm o}$)\kern+.3em.
%    \begin{macrocode}
\def\fprimo){\FrenchPopularEnumerate1}
\def\fsecundo){\FrenchPopularEnumerate2}
\def\ftertio){\FrenchPopularEnumerate3}
\def\fquarto){\FrenchPopularEnumerate4}
%    \end{macrocode}
%
%    Let's provide four macros for the common abbreviations
%    of ``Num\'ero''.
%    \begin{macrocode}
\DeclareRobustCommand*{\No}{N\up{\lowercase{o}}\kern+.2em}
\DeclareRobustCommand*{\no}{n\up{\lowercase{o}}\kern+.2em}
\DeclareRobustCommand*{\Nos}{N\up{\lowercase{os}}\kern+.2em}
\DeclareRobustCommand*{\nos}{n\up{\lowercase{os}}\kern+.2em}
%    \end{macrocode}
%  \end{macro}
%  \end{macro}
%  \end{macro}
%  \end{macro}
%  \end{macro}
%  \end{macro}
%
%  \begin{macro}{\bsc}
%    As family names should be written in small capitals and never be
%    hyphenated, we provide a command (its name comes from Boxed Small
%    Caps) to input them easily.  Note that this command has changed
%    with version~2 of |frenchb|: a |\kern0pt| is used instead of |\hbox|
%    because |\hbox| would break microtype's font expansion; as a
%    (positive?) side effect, composed names (such as Dupont-Durand)
%    can now be hyphenated on explicit hyphens.
%    Usage: |Jean~\bsc{Duchemin}|.
%
% \changes{v2.0}{2006/11/06}{\cs{hbox} dropped, replaced by
%    \cs{kern0pt}.}
%
%    \begin{macrocode}
\DeclareRobustCommand*{\bsc}[1]{\leavevmode\begingroup\kern0pt
                                           \scshape #1\endgroup}
\ifLaTeXe\else\let\scshape\relax\fi
%    \end{macrocode}
%  \end{macro}
%
%    Some definitions for special characters.  We won't define |\tilde|
%    as a Text Symbol not to conflict with the macro |\tilde| for math
%    mode and use the name |\tild| instead. Note that |\boi| may
%    \emph{not} be used in math mode, its name in math mode is
%    |\backslash|.  |\degre|  can be accessed by the command |\r{}|
%    for ring accent.
%
%    \begin{macrocode}
\ifLaTeXe
  \DeclareTextSymbol{\at}{T1}{64}
  \DeclareTextSymbol{\circonflexe}{T1}{94}
  \DeclareTextSymbol{\tild}{T1}{126}
  \DeclareTextSymbolDefault{\at}{T1}
  \DeclareTextSymbolDefault{\circonflexe}{T1}
  \DeclareTextSymbolDefault{\tild}{T1}
  \DeclareRobustCommand*{\boi}{\textbackslash}
  \DeclareRobustCommand*{\degre}{\r{}}
\else
  \def\T@one{T1}
  \ifx\f@encoding\T@one
    \newcommand*{\degre}{\char6}
  \else
    \newcommand*{\degre}{\char23}
  \fi
  \newcommand*{\at}{\char64}
  \newcommand*{\circonflexe}{\char94}
  \newcommand*{\tild}{\char126}
  \newcommand*{\boi}{$\backslash$}
\fi
%    \end{macrocode}
%
%  \begin{macro}{\degres}
%    We now define a macro |\degres| for typesetting the abbreviation
%    for `degrees' (as in `degrees Celsius'). As the bounding box of
%    the character `degree' has \emph{very} different widths in CM/EC
%    and PostScript fonts, we fix the width of the bounding box of
%    |\degres| to 0.3\,em, this lets the symbol `degree' stick to the
%    preceding (e.g., |45\degres|) or following character
%    (e.g., |20~\degres C|).
%
%    If the \TeX{} Companion fonts are available (\file{textcomp.sty}),
%    we pick up |\textdegree| from them instead of using emulating
%    `degrees' from the |\r{}| accent. Otherwise we overwrite the
%    (poor) definition of |\textdegree| given in \file{latin1.def},
%    \file{applemac.def} etc. (called by  \file{inputenc.sty}) by
%    our definition of |\degres|. We also advice the user (once only)
%    to use TS1-encoding.
%
% \changes{v2.1c}{2008/04/29}{Provide a temporary definition (hyperref
%    safe) of \cs{degres} in case it has to be expanded in the preamble
%    (by beamer's \cs{title} command for instance).}
%
%    \begin{macrocode}
\ifLaTeXe
  \newcommand*{\degres}{\degre}
  \def\Warning@degree@TSone{%
        \PackageWarning{frenchb.ldf}{%
           Degrees would look better in TS1-encoding:
           \MessageBreak add \protect
           \usepackage{textcomp} to the preamble.
           \MessageBreak Degrees used}}
  \AtBeginDocument{\expandafter\ifx\csname M@TS1\endcsname\relax
                     \DeclareRobustCommand*{\degres}{%
                       \leavevmode\hbox to 0.3em{\hss\degre\hss}%
                       \Warning@degree@TSone
                       \global\let\Warning@degree@TSone\relax}%
                      \let\textdegree\degres
                   \else
                     \DeclareRobustCommand*{\degres}{%
                         \hbox{\UseTextSymbol{TS1}{\textdegree}}}%
                   \fi}
\else
  \newcommand*{\degres}{%
    \leavevmode\hbox to 0.3em{\hss\degre\hss}}
\fi
%    \end{macrocode}
%  \end{macro}
%
%  \subsection{Formatting numbers}
%  \label{sec-numbers}
%
%  \begin{macro}{\DecimalMathComma}
%  \begin{macro}{\StandardMathComma}
%    As mentioned in the \TeX{}book p.~134, the comma is of type
%    |\mathpunct| in math mode: it is automatically followed by a
%    space. This is convenient in lists and intervals but
%    unpleasant when the comma is used as a decimal separator
%    in French: it has to be entered as |{,}|.
%    |\DecimalMathComma| makes the comma be an ordinary character
%    (of type |\mathord|) in French \emph{only} (no space added);
%    |\StandardMathComma| switches back to the standard behaviour
%    of the comma.
%    \begin{macrocode}
\newcount\std@mcc
\newcount\dec@mcc
\std@mcc=\mathcode`\,
\dec@mcc=\std@mcc
\@tempcnta=\std@mcc
\divide\@tempcnta by "1000
\multiply\@tempcnta by "1000
\advance\dec@mcc by -\@tempcnta
\newcommand*{\DecimalMathComma}{\iflanguage{french}%
                                 {\mathcode`\,=\dec@mcc}{}%
              \addto\extrasfrench{\mathcode`\,=\dec@mcc}}
\newcommand*{\StandardMathComma}{\mathcode`\,=\std@mcc
             \addto\extrasfrench{\mathcode`\,=\std@mcc}}
\expandafter\addto\csname noextras\CurrentOption\endcsname{%
   \mathcode`\,=\std@mcc}
%    \end{macrocode}
%  \end{macro}
%  \end{macro}
%
%  \begin{macro}{\nombre}
%
% \changes{v2.0}{2006/11/06}{\cs{nombre} requires now numprint.sty.}
%
%    The command |\nombre| is now borrowed from |numprint.sty| for
%    \LaTeXe.  There is no point to maintain the former tricky code
%    when a package is dedicated to do the same job and more.
%    For Plain based formats, |\nombre| no longer formats numbers,
%    it prints them as is and issues a warning about the change.
%
%    Fake command |\nombre| for Plain based formats, warning users of
%    |frenchb| v.1.x. of the change.
%    \begin{macrocode}
\newcommand*{\nombre}[1]{{#1}\message{%
     *** \noexpand\nombre no longer formats numbers\string! ***}}%
%    \end{macrocode}
%  \end{macro}
%
%    The next definitions only make sense for \LaTeXe. Let's cleanup
%    and exit if the format in Plain based.
%
%    \begin{macrocode}
\let\FBstop@here\relax
\def\FBclean@on@exit{\let\ifLaTeXe\@undefined
                     \let\LaTeXetrue\@undefined
                     \let\LaTeXefalse\@undefined}
\ifx\magnification\@undefined
\else
   \def\FBstop@here{\let\STD@makecaption\relax
                    \FBclean@on@exit
                    \ldf@quit\CurrentOption\endinput}
\fi
\FBstop@here
%    \end{macrocode}
%
%    What follows now is for \LaTeXe{} \emph{only}.
%    We redefine |\nombre| for \LaTeXe. A warning is issued
%    at the first call of |\nombre| if |\numprint| is not
%    defined, suggesting what to do.  The package |numprint|
%    is \emph{not} loaded automatically by |frenchb| because of
%    possible options conflict.
%
%    \begin{macrocode}
\renewcommand*{\nombre}[1]{\Warning@nombre\numprint{#1}}
\newcommand*{\Warning@nombre}{%
   \@ifundefined{numprint}%
      {\PackageWarning{frenchb.ldf}{%
         \protect\nombre\space now relies on package numprint.sty,
         \MessageBreak add \protect
         \usepackage[autolanguage]{numprint}\MessageBreak
         to your preamble *after* loading babel, \MessageBreak
         see file numprint.pdf for other options.\MessageBreak
         \protect\nombre\space called}%
       \global\let\Warning@nombre\relax
       \global\let\numprint\relax
      }{}%
}
%    \end{macrocode}
%
% \changes{v2.0c}{2007/06/25}{There is no need to define here
%    numprint's command \cs{npstylefrench}, it will be redefined
%    `AtBeginDocument' by \cs{FBprocess@options}.}
%
% \changes{v2.0c}{2007/06/25}{\cs{ThinSpaceInFrenchNumbers} added
%     for compatibility with frenchb-1.x.}
%
%    \begin{macrocode}
\newcommand*{\ThinSpaceInFrenchNumbers}{%
   \PackageWarning{frenchb.ldf}{%
         Type \protect\frenchbsetup{ThinSpaceInFrenchNumbers}
         \MessageBreak Command \protect\ThinSpaceInFrenchNumbers\space
         is no longer\MessageBreak  defined in frenchb v.2,}}
%    \end{macrocode}
%
%  \subsection{Caption names}
%
%    The next step consists of defining the French equivalents for
%    the \LaTeX{} caption names.
%
% \begin{macro}{\captionsfrench}
%    Let's first define  |\captionsfrench| which sets all strings used
%    in the four standard document classes provided with \LaTeX.
%
% \changes{v2.0}{2006/11/06}{`Fig.' changed to `Figure' and
%     `Tab.' to `Table'.}
%
% \changes{v2.0}{2006/12/15}{Set \cs{CaptionSeparator} in
%     \cs{extrasfrench} now instead of \cs{captionsfrench}
%     because it has to be reset when leaving French.}
%
%    \begin{macrocode}
\@namedef{captions\CurrentOption}{%
   \def\refname{R\'ef\'erences}%
   \def\abstractname{R\'esum\'e}%
   \def\bibname{Bibliographie}%
   \def\prefacename{Pr\'eface}%
   \def\chaptername{Chapitre}%
   \def\appendixname{Annexe}%
   \def\contentsname{Table des mati\`eres}%
   \def\listfigurename{Table des figures}%
   \def\listtablename{Liste des tableaux}%
   \def\indexname{Index}%
   \def\figurename{{\scshape Figure}}%
   \def\tablename{{\scshape Table}}%
%    \end{macrocode}
%   ``Premi\`ere partie'' instead of ``Part I''.
%    \begin{macrocode}
   \def\partname{\protect\@Fpt partie}%
   \def\@Fpt{{\ifcase\value{part}\or Premi\`ere\or Deuxi\`eme\or
   Troisi\`eme\or Quatri\`eme\or Cinqui\`eme\or Sixi\`eme\or
   Septi\`eme\or Huiti\`eme\or Neuvi\`eme\or Dixi\`eme\or Onzi\`eme\or
   Douzi\`eme\or Treizi\`eme\or Quatorzi\`eme\or Quinzi\`eme\or
   Seizi\`eme\or Dix-septi\`eme\or Dix-huiti\`eme\or Dix-neuvi\`eme\or
   Vingti\`eme\fi}\space\def\thepart{}}%
   \def\pagename{page}%
   \def\seename{{\emph{voir}}}%
   \def\alsoname{{\emph{voir aussi}}}%
   \def\enclname{P.~J. }%
   \def\ccname{Copie \`a }%
   \def\headtoname{}%
   \def\proofname{D\'emonstration}%
   \def\glossaryname{Glossaire}%
   }
%    \end{macrocode}
% \end{macro}
%
%    As some users who choose |frenchb| or |francais| as option of
%    \babel, might customise |\captionsfrenchb| or |\captionsfrancais|
%    in the preamble, we merge their changes at the |\begin{document}|
%    when they do so.
%    The other variants of French (canadien, acadian) are defined by
%    checking if the relevant option was used and then adding one extra
%    level of expansion.
%
%    \begin{macrocode}
\AtBeginDocument{\let\captions@French\captionsfrench
                 \@ifundefined{captionsfrenchb}%
                    {\let\captions@Frenchb\relax}%
                    {\let\captions@Frenchb\captionsfrenchb}%
                 \@ifundefined{captionsfrancais}%
                    {\let\captions@Francais\relax}%
                    {\let\captions@Francais\captionsfrancais}%
                 \def\captionsfrench{\captions@French
                        \captions@Francais\captions@Frenchb}%
                 \def\captionsfrancais{\captionsfrench}%
                 \def\captionsfrenchb{\captionsfrench}%
                 \iflanguage{french}{\captionsfrench}{}%
                }
\@ifpackagewith{babel}{canadien}{%
  \def\captionscanadien{\captionsfrench}%
  \def\datecanadien{\datefrench}%
  \def\extrascanadien{\extrasfrench}%
  \def\noextrascanadien{\noextrasfrench}%
  }{}
\@ifpackagewith{babel}{acadian}{%
  \def\captionsacadian{\captionsfrench}%
  \def\dateacadian{\datefrench}%
  \def\extrasacadian{\extrasfrench}%
  \def\noextrasacadian{\noextrasfrench}%
  }{}
%    \end{macrocode}
%
% \begin{macro}{\CaptionSeparator}
%    Let's consider now captions in figures and tables.
%    In French, captions in figures and tables should be printed with
%    endash (`--') instead of the standard `:'.
%
%    The standard definition of |\@makecaption| (e.g., the one provided
%    in article.cls, report.cls, book.cls which is frozen for \LaTeXe{}
%    according to Frank Mittelbach), has been saved in
%    |\STD@makecaption| before making `:' active
%    (see section~\ref{sec-punct}). `AtBeginDocument' we compare it to
%    its current definition (some classes like koma-script classes,
%    AMS classes, ua-thesis.cls\dots change it).
%    If they are identical, |frenchb| just adds a hook called
%    |\CaptionSeparator| to |\@makecaption|, |\CaptionSeparator|
%    defaults to `: ' as in the standard |\@makecaption|, and will be
%    changed to ` -- ' in French.
%    If the definitions differ, |frenchb| doesn't overwrite the changes,
%    but prints a message in the .log file.
%
%    \begin{macrocode}
\def\CaptionSeparator{\string:\space}
\long\def\FB@makecaption#1#2{%
  \vskip\abovecaptionskip
  \sbox\@tempboxa{#1\CaptionSeparator #2}%
  \ifdim \wd\@tempboxa >\hsize
    #1\CaptionSeparator #2\par
  \else
    \global \@minipagefalse
    \hb@xt@\hsize{\hfil\box\@tempboxa\hfil}%
  \fi
  \vskip\belowcaptionskip}
\AtBeginDocument{%
  \ifx\@makecaption\STD@makecaption
      \global\let\@makecaption\FB@makecaption
  \else
    \@ifundefined{@makecaption}{}%
       {\PackageWarning{frenchb.ldf}%
        {The definition of \protect\@makecaption\space
         has been changed,\MessageBreak
         frenchb will NOT customise it;\MessageBreak reported}%
       }%
  \fi
  \let\FB@makecaption\relax
  \let\STD@makecaption\relax
}
\expandafter\addto\csname extras\CurrentOption\endcsname{%
   \def\CaptionSeparator{\space\textendash\space}}
\expandafter\addto\csname noextras\CurrentOption\endcsname{%
    \def\CaptionSeparator{\string:\space}}
%    \end{macrocode}
% \end{macro}
%
%  \subsection{French lists}
%  \label{sec-lists}
%
%  \begin{macro}{\listFB}
%  \begin{macro}{\listORI}
%    Vertical spacing in general lists should be shorter in French
%    texts than the defaults provided by \LaTeX.
%    Note that the easy way, just changing values of vertical spacing
%    parameters when entering French and restoring them to their
%    defaults on exit would not work; as most lists are based on
%    |\list| we will define a variant of |\list| (|\listFB|) to
%    be used in French.
%
%    The amount of vertical space before and after a list is given by
%    |\topsep| + |\parskip| (+ |\partopsep| if the list starts a new
%    paragraph). IMHO, |\parskip| should be added \emph{only} when
%    the list starts a new paragraph, so I subtract |\parskip| from
%    |\topsep| and add it back to |\partopsep|; this will normally
%    make no difference because |\parskip|'s default value is 0pt, but
%    will be noticeable when |\parskip| is \emph{not} null.
%
%    |\endlist| is not redefined, but |\endlistORI| is provided for
%    the users who prefer to define their own lists from the original
%    command, they can code: |\begin{listORI}{}{} \end{listORI}|.
%    \begin{macrocode}
\let\listORI\list
\let\endlistORI\endlist
\def\FB@listsettings{%
      \setlength{\itemsep}{0.4ex plus 0.2ex minus 0.2ex}%
      \setlength{\parsep}{0.4ex plus 0.2ex minus 0.2ex}%
      \setlength{\topsep}{0.8ex plus 0.4ex minus 0.4ex}%
      \setlength{\partopsep}{0.4ex plus 0.2ex minus 0.2ex}%
%    \end{macrocode}
%    |\parskip| is of type `skip', its mean value only (\emph{not
%    the glue}) should be subtracted from |\topsep| and added to
%    |\partopsep|, so convert |\parskip| to a `dimen' using
%    |\@tempdima|.
%    \begin{macrocode}
      \@tempdima=\parskip
      \addtolength{\topsep}{-\@tempdima}%
      \addtolength{\partopsep}{\@tempdima}}%
\def\listFB#1#2{\listORI{#1}{\FB@listsettings #2}}%
\let\endlistFB\endlist
%    \end{macrocode}
%  \end{macro}
%  \end{macro}
%
%  \begin{macro}{\itemizeFB}
%  \begin{macro}{\itemizeORI}
%  \begin{macro}{\bbl@frenchlabelitems}
%  \begin{macro}{\bbl@nonfrenchlabelitems}
%    Let's now consider French itemize lists.  They differ from those
%    provided by the standard \LaTeXe{} classes:
%    \begin{itemize}
%      \item vertical spacing between items, before and after
%         the list, should be \emph{null} with \emph{no glue} added;
%      \item the item labels of a first level list should be vertically
%          aligned on the paragraph's first character (i.e. at
%          |\parindent| from the left margin);
%      \item the `\textbullet' is never used in French itemize-lists,
%          a long dash `--' is preferred for all levels. The item label
%          used in French is stored in |\FrenchLabelItem}|, it defaults
%          to `--' and can be changed using |\frenchbsetup{}| (see
%          section~\ref{sec-keyval}).
%    \end{itemize}
%
%    \begin{macrocode}
\newcommand*{\FrenchLabelItem}{\textendash}
\newcommand*{\Frlabelitemi}{\FrenchLabelItem}
\newcommand*{\Frlabelitemii}{\FrenchLabelItem}
\newcommand*{\Frlabelitemiii}{\FrenchLabelItem}
\newcommand*{\Frlabelitemiv}{\FrenchLabelItem}
%    \end{macrocode}
%    |\bbl@frenchlabelitems| saves current itemize labels and changes
%    them to their value in French. This code should never be executed
%    twice in a row, so we need a new flag that will be set and reset
%    by |\bbl@nonfrenchlabelitems| and |\bbl@frenchlabelitems|.
%    \begin{macrocode}
\newif\ifFB@enterFrench  \FB@enterFrenchtrue
\def\bbl@frenchlabelitems{%
  \ifFB@enterFrench
    \let\@ltiORI\labelitemi
    \let\@ltiiORI\labelitemii
    \let\@ltiiiORI\labelitemiii
    \let\@ltivORI\labelitemiv
    \let\labelitemi\Frlabelitemi
    \let\labelitemii\Frlabelitemii
    \let\labelitemiii\Frlabelitemiii
    \let\labelitemiv\Frlabelitemiv
    \FB@enterFrenchfalse
  \fi
}
\let\itemizeORI\itemize
\let\enditemizeORI\enditemize
\let\enditemizeFB\enditemize
\def\itemizeFB{%
    \ifnum \@itemdepth >\thr@@\@toodeep\else
      \advance\@itemdepth\@ne
      \edef\@itemitem{labelitem\romannumeral\the\@itemdepth}%
      \expandafter
      \listORI
      \csname\@itemitem\endcsname
      {\settowidth{\labelwidth}{\csname\@itemitem\endcsname}%
       \setlength{\leftmargin}{\labelwidth}%
       \addtolength{\leftmargin}{\labelsep}%
       \ifnum\@listdepth=0
         \setlength{\itemindent}{\parindent}%
       \else
         \addtolength{\leftmargin}{\parindent}%
       \fi
       \setlength{\itemsep}{\z@}%
       \setlength{\parsep}{\z@}%
       \setlength{\topsep}{\z@}%
       \setlength{\partopsep}{\z@}%
%    \end{macrocode}
%    |\parskip| is of type `skip', its mean value only (\emph{not
%    the glue}) should be subtracted from |\topsep| and added to
%    |\partopsep|, so convert |\parskip| to a `dimen' using
%    |\@tempdima|.
%    \begin{macrocode}
       \@tempdima=\parskip
       \addtolength{\topsep}{-\@tempdima}%
       \addtolength{\partopsep}{\@tempdima}}%
    \fi}
%    \end{macrocode}
%    The user's changes in labelitems are saved when leaving French for
%    further use when switching back to French.  This code should never
%    be executed twice in a row (toggle with |\bbl@frenchlabelitems|).
%    \begin{macrocode}
\def\bbl@nonfrenchlabelitems{%
  \ifFB@enterFrench
  \else
      \let\Frlabelitemi\labelitemi
      \let\Frlabelitemii\labelitemii
      \let\Frlabelitemiii\labelitemiii
      \let\Frlabelitemiv\labelitemiv
      \let\labelitemi\@ltiORI
      \let\labelitemii\@ltiiORI
      \let\labelitemiii\@ltiiiORI
      \let\labelitemiv\@ltivORI
      \FB@enterFrenchtrue
  \fi
}
%    \end{macrocode}
%  \end{macro}
%  \end{macro}
%  \end{macro}
%  \end{macro}
%
%  \subsection{French indentation of sections}
%  \label{sec-indent}
%
%  \begin{macro}{\bbl@frenchindent}
%  \begin{macro}{\bbl@nonfrenchindent}
%    In French the first paragraph of each section should be indented,
%    this is another difference with US-English. This is controlled by
%    the flag |\if@afterindent|.
%
% \changes{v2.3d}{2009/03/16}{Bug correction: previous versions of
%    frenchb set the flag \cs{if@afterindent} to false outside
%    French which is correct for English but wrong for some languages
%    like Spanish.  Pointed out by Juan Jos\'e Torrens.}
%
%    We need to save the value of the flag |\if@afterindent|
%    `AtBeginDocument' before eventually changing its value.
%
%    \begin{macrocode}
\AtBeginDocument{\ifx\@afterindentfalse\@afterindenttrue
                       \let\@aifORI\@afterindenttrue
                 \else \let\@aifORI\@afterindentfalse
                 \fi
}
\def\bbl@frenchindent{\let\@afterindentfalse\@afterindenttrue
                      \@afterindenttrue}
\def\bbl@nonfrenchindent{\let\@afterindentfalse\@aifORI
                         \@afterindentfalse}
%    \end{macrocode}
%  \end{macro}
%  \end{macro}
%
%  \subsection{Formatting footnotes}
%  \label{sec-footnotes}
%
% \changes{v2.0}{2006/11/06}{Footnotes are now printed
%     by default `\`a la fran\c caise' for the whole document.}
%
% \changes{v2.0b}{2007/04/18}{Footnotes: Just do nothing
%    (except warning) when the bigfoot package is loaded.}
%
%    The |bigfoot| package deeply changes the way footnotes are
%    handled. When |bigfoot| is loaded, we just warn the user that
%    |frenchb| will drop the customisation of footnotes.
%
%    The layout of footnotes is controlled by two flags
%    |\ifFBAutoSpaceFootnotes| and |\ifFBFrenchFootnotes| which are
%    set by options of |\frenchbsetup{}| (see section~\ref{sec-keyval}).
%    Notice that the layout of footnotes \emph{does not depend} on the
%    current language (just think of two footnotes on the same page
%    looking different because one was called in a French part, the
%    other one in English!).
%
%    When |\ifFBAutoSpaceFootnotes| is true, |\@footnotemark| (whose
%    definition is saved at the |\begin{document}| in order to include
%    any customisation that packages might have done) is redefined to
%    add a thin space before the number or symbol calling a footnote
%    (any space typed in is removed first).  This has no effect on
%    the layout of the footnote itself.
%
%    \begin{macrocode}
\AtBeginDocument{\@ifpackageloaded{bigfoot}%
                   {\PackageWarning{frenchb.ldf}%
                     {bigfoot package in use.\MessageBreak
                      frenchb will NOT customise footnotes;\MessageBreak
                      reported}}%
                   {\let\@footnotemarkORI\@footnotemark
                    \def\@footnotemarkFB{\leavevmode\unskip\unkern
                                         \,\@footnotemarkORI}%
                    \ifFBAutoSpaceFootnotes
                      \let\@footnotemark\@footnotemarkFB
                    \fi}%
                }
%    \end{macrocode}
%
%    We then define |\@makefntextFB|, a variant of |\@makefntext|
%    which is responsible for the layout of footnotes, to match the
%    specifications of the French `Imprimerie Nationale':  footnotes
%    will be indented by |\parindentFFN|, numbers (if any) typeset on
%    the baseline (instead of superscripts) and followed by a dot
%    and an half quad space. Whenever symbols are used to number
%    footnotes (as in |\thanks| for instance), we switch back to the
%    standard layout (the French layout of footnotes is meant for
%    footnotes numbered by Arabic or Roman digits).
%
% \changes{v2.0}{2006/11/06}{\cs{parindentFFN} not changed if
%    already defined (required by JA for cah-gut.cls).}
%
% \changes{v2.3b}{2008/12/06}{New commands \cs{dotFFN} and
%    \cs{kernFFN} for more flexibility (suggested by JA).}
%
%    The value of |\parindentFFN| will be redefined at the
%    |\begin{document}|, as the maximum of |\parindent| and 1.5em
%    \emph{unless} it has been set in the preamble (the weird value
%    10in is just for testing whether |\parindentFFN| has been set
%    or not).
%
%    \begin{macrocode}
\newcommand*{\dotFFN}{.}
\newcommand*{\kernFFN}{\kern .5em}
\newdimen\parindentFFN
\parindentFFN=10in
\def\ftnISsymbol{\@fnsymbol\c@footnote}
\long\def\@makefntextFB#1{\ifx\thefootnote\ftnISsymbol
                            \@makefntextORI{#1}%
                          \else
                            \parindent=\parindentFFN
                            \rule\z@\footnotesep
                            \setbox\@tempboxa\hbox{\@thefnmark}%
                            \ifdim\wd\@tempboxa>\z@
                              \llap{\@thefnmark}\dotFFN\kernFFN
                            \fi #1
                          \fi}%
%    \end{macrocode}
%
%    We save the standard definition of |\@makefntext| at the
%    |\begin{document}|, and then redefine |\@makefntext| according to
%    the value of flag |\ifFBFrenchFootnotes| (true or false).
%
%    \begin{macrocode}
\AtBeginDocument{\@ifpackageloaded{bigfoot}{}%
                  {\ifdim\parindentFFN<10in
                   \else
                      \parindentFFN=\parindent
                      \ifdim\parindentFFN<1.5em\parindentFFN=1.5em\fi
                   \fi
                   \let\@makefntextORI\@makefntext
                   \long\def\@makefntext#1{%
                      \ifFBFrenchFootnotes
                         \@makefntextFB{#1}%
                      \else
                         \@makefntextORI{#1}%
                      \fi}%
                  }%
                }
%    \end{macrocode}
%
%    For compatibility reasons, we provide definitions for the commands
%    dealing with the layout of footnotes in |frenchb| version~1.6.
%    |\frenchbsetup{}| (see in section \ref{sec-keyval}) should be
%    preferred for setting these options.  |\StandardFootnotes| may
%    still be used locally (in minipages for instance), that's why the
%    test |\ifFBFrenchFootnotes| is done inside |\@makefntext|.
%    \begin{macrocode}
\newcommand*{\AddThinSpaceBeforeFootnotes}{\FBAutoSpaceFootnotestrue}
\newcommand*{\FrenchFootnotes}{\FBFrenchFootnotestrue}
\newcommand*{\StandardFootnotes}{\FBFrenchFootnotesfalse}
%    \end{macrocode}
%
%  \subsection{Global layout}
%  \label{sec-global}
%
%    In multilingual documents, some typographic rules must depend
%    on the current language (e.g., hyphenation, typesetting of
%    numbers, spacing before double punctuation\dots), others should,
%    IMHO, be kept global to the document: especially the layout of
%    lists (see~\ref{sec-lists}) and footnotes
%    (see~\ref{sec-footnotes}), and the indentation of the first
%    paragraph of sections (see~\ref{sec-indent}).
%
%    From version 2.2 on, if |frenchb| is \babel's ``main language''
%    (i.e. last language option at \babel's loading), |frenchb|
%    customises the layout (i.e. lists, indentation of the first
%    paragraphs of sections and footnotes) in the whole document
%    regardless the current language.   On the other hand, if |frenchb|
%    is \emph{not} \babel's ``main language'', it leaves the layout
%    unchanged both in French and in other languages.
%
%  \begin{macro}{\FrenchLayout}
%  \begin{macro}{\StandardLayout}
%    The former commands |\FrenchLayout| and |\StandardLayout| are kept
%    for compatibility reasons but should no longer be used.
%
% \changes{v2.0g}{2008/03/23}{Flag \cs{ifFBStandardLayout} not checked
%     by \cs{FBprocess@options}, low-level flags have to be set
%     one by one.}
%
%    \begin{macrocode}
\newcommand*{\FrenchLayout}{%
    \FBGlobalLayoutFrenchtrue
    \PackageWarning{frenchb.ldf}%
    {\protect\FrenchLayout\space is obsolete.  Please use\MessageBreak
     \protect\frenchbsetup{GlobalLayoutFrench} instead.}%
}
\newcommand*{\StandardLayout}{%
  \FBReduceListSpacingfalse
  \FBCompactItemizefalse
  \FBStandardItemLabelstrue
  \FBIndentFirstfalse
  \FBFrenchFootnotesfalse
  \FBAutoSpaceFootnotesfalse
  \PackageWarning{frenchb.ldf}%
    {\protect\StandardLayout\space is obsolete.  Please use\MessageBreak
    \protect\frenchbsetup{StandardLayout} instead.}%
}
\@onlypreamble\FrenchLayout
\@onlypreamble\StandardLayout
%    \end{macrocode}
%  \end{macro}
%  \end{macro}
%
%  \subsection{Dots\dots}
%  \label{sec-dots}
%
%  \begin{macro}{\FBtextellipsis}
%    \LaTeXe's standard definition of |\dots| in text-mode is
%    |\textellipsis| which includes a |\kern| at the end;
%    this space is not wanted in some cases (before a closing brace
%    for instance) and |\kern| breaks hyphenation of the next word.
%    We define |\FBtextellipsis| for French (in \LaTeXe{} only).
%
%    The |\if| construction in the \LaTeXe{} definition of |\dots|
%    doesn't allow the use of |xspace| (|xspace| is always followed
%    by a |\fi|), so we use the AMS-\LaTeX{} construction of |\dots|;
%    this has to be done `AtBeginDocument' not to be overwritten
%    when \file{amsmath.sty} is loaded after \babel.
%
% \changes{v2.0}{2006/11/06}{Added special case for LY1 encoding,
%    see  bug report from Bruno Voisin (2004/05/18).}
%
%    LY1 has a ready made character for |\textellipsis|, it should be
%    used in French too (pointed out by Bruno Voisin).
%
%    \begin{macrocode}
\DeclareTextSymbol{\FBtextellipsis}{LY1}{133}
\DeclareTextCommandDefault{\FBtextellipsis}{%
    .\kern\fontdimen3\font.\kern\fontdimen3\font.\xspace}
%    \end{macrocode}
%    |\Mdots@| and |\Tdots@ORI| hold the definitions of |\dots| in
%    Math and Text mode. They default to those of amsmath-2.0, and
%    will revert to standard \LaTeX{} definitions `AtBeginDocument',
%    if amsmath has not been loaded. |\Mdots@| doesn't change when
%    switching from/to French, while |\Tdots@| is |\FBtextellipsis|
%    in French and |\Tdots@ORI| otherwise.
%    \begin{macrocode}
\newcommand*{\Tdots@ORI}{\@xp\textellipsis}
\newcommand*{\Tdots@}{\Tdots@ORI}
\newcommand*{\Mdots@}{\@xp\mdots@}
\AtBeginDocument{\DeclareRobustCommand*{\dots}{\relax
                 \csname\ifmmode M\else T\fi dots@\endcsname}%
                 \@ifundefined{@xp}{\let\@xp\relax}{}%
                 \@ifundefined{mdots@}{\let\Tdots@ORI\textellipsis
                                       \let\Mdots@\mathellipsis}{}}
\def\bbl@frenchdots{\let\Tdots@\FBtextellipsis}
\def\bbl@nonfrenchdots{\let\Tdots@\Tdots@ORI}
\expandafter\addto\csname extras\CurrentOption\endcsname{%
    \bbl@frenchdots}
\expandafter\addto\csname noextras\CurrentOption\endcsname{%
    \bbl@nonfrenchdots}
%    \end{macrocode}
%  \end{macro}
%
%  \subsection{Setup options: keyval stuff}
%  \label{sec-keyval}
%
% \changes{v2.0}{2006/11/06}{New command \cs{frenchbsetup} added
%     for global customisation.}
%
% \changes{v2.0c}{2007/06/25}{Option ThinSpaceInFrenchNumbers added.}
%
% \changes{v2.0d}{2007/07/15}{Options og and fg changed: limit
%     the definition to French so that quote characters can be used
%     in German.}
%
% \changes{v2.0e}{2007/10/05}{New option: StandardLists.}
%
% \changes{v2.0f}{2008/03/23}{Two typos corrected in
%    option StandardLists: [false] $\to$ [true] and
%    StandardLayout $\to$ StandardLists.}
%
% \changes{v2.0f}{2008/03/23}{StandardLayout option had no
%     effect on lists.  Test moved to \cs{FBprocess@options}.}
%
% \changes{v2.0g}{2008/03/23}{Revert previous change to
%     StandardLayout. This option must set the three flags
%     \cs{FBReduceListSpacingfalse}, \cs{FBCompactItemizefalse},
%     and \cs{FBStandardItemLabeltrue} instead of
%     \cs{FBStandardListstrue}, so that later options can still
%     change their value before executing \cs{FBprocess@options}.
%     Same thing for option StandardLists.}
%
% \changes{v2.1a}{2008/03/24}{New option: FrenchSuperscripts
%     to define \cs{up} as \cs{fup} or as \cs{textsuperscript}.}
%
% \changes{v2.1a}{2008/03/30}{New option: LowercaseSuperscripts.}
%
% \changes{v2.2a}{2008/05/08}{The global layout of the document is
%     no longer changed when frenchb is not the last option of babel
%     (\cs{bbl@main@language}). Suggested by Ulrike Fischer.}
%
% \changes{v2.2a}{2008/05/08}{Values of flags
%     \cs{ifFBReduceListSpacing}, \cs{ifFBCompactItemize},
%     \cs{ifFBStandardItemLabels}, \cs{ifFBIndentFirst},
%     \cs{ifFBFrenchFootnotes}, \cs{ifFBAutoSpaceFootnotes} changed:
%     default now means `StandardLayout', it will be changed to
%     `FrenchLayout' AtEndOfPackage only if french is
%     \cs{bbl@main@language}.}
%
% \changes{v2.2a}{2008/05/08}{When frenchb is babel's last option,
%     French becomes the document's main language, so
%     GlobalLayoutFrench applies.}
%
% \changes{v2.3a}{2008/10/10}{New option: OriginalTypewriter. Now
%    frenchb switches to \cs{noautospace@beforeFDP} when a tt-font is
%    in use.  When OriginalTypewriter is set to true, frenchb behaves
%    as in pre-2.3 versions.}
%
%    We first define a collection of conditionals with their defaults
%    (true or false).
%
%    \begin{macrocode}
\newif\ifFBStandardLayout           \FBStandardLayouttrue
\newif\ifFBGlobalLayoutFrench       \FBGlobalLayoutFrenchfalse
\newif\ifFBReduceListSpacing        \FBReduceListSpacingfalse
\newif\ifFBCompactItemize           \FBCompactItemizefalse
\newif\ifFBStandardItemLabels       \FBStandardItemLabelstrue
\newif\ifFBStandardLists            \FBStandardListstrue
\newif\ifFBIndentFirst              \FBIndentFirstfalse
\newif\ifFBFrenchFootnotes          \FBFrenchFootnotesfalse
\newif\ifFBAutoSpaceFootnotes       \FBAutoSpaceFootnotesfalse
\newif\ifFBOriginalTypewriter       \FBOriginalTypewriterfalse
\newif\ifFBThinColonSpace           \FBThinColonSpacefalse
\newif\ifFBThinSpaceInFrenchNumbers \FBThinSpaceInFrenchNumbersfalse
\newif\ifFBFrenchSuperscripts       \FBFrenchSuperscriptstrue
\newif\ifFBLowercaseSuperscripts    \FBLowercaseSuperscriptstrue
\newif\ifFBPartNameFull             \FBPartNameFulltrue
\newif\ifFBShowOptions              \FBShowOptionsfalse
%    \end{macrocode}
%
%    The defaults values of these flags have been set so that |frenchb|
%    does not change anything regarding the global layout.
%    |\bbl@main@language| (set by the last option of babel) controls
%    the global layout of the document.  We check the current language
%    `AtEndOfPackage' (it is |\bbl@main@language|); if it is French,
%    the values of some flags have to be changed to ensure a French
%    looking layout for the whole document (even in parts written in
%    languages other than French); the end-user will then be able to
%    customise the values of all these flags with |\frenchbsetup{}|.
%    \begin{macrocode}
\AtEndOfPackage{%
   \iflanguage{french}{\FBReduceListSpacingtrue
                       \FBCompactItemizetrue
                       \FBStandardItemLabelsfalse
                       \FBIndentFirsttrue
                       \FBFrenchFootnotestrue
                       \FBAutoSpaceFootnotestrue
                       \FBGlobalLayoutFrenchtrue}%
                      {}%
}
%    \end{macrocode}
%
%  \begin{macro}{\frenchbsetup}
%    From version 2.0 on, all setup options are handled by \emph{one}
%    command |\frenchbsetup| using the keyval syntax.
%    Let's now define this command which reads and sets the options
%    to be processed later (at |\begin{document}|) by
%    |\FBprocess@options|. It  can only be called in the preamble.
%    \begin{macrocode}
\newcommand*{\frenchbsetup}[1]{%
  \setkeys{FB}{#1}%
}%
\@onlypreamble\frenchbsetup
%    \end{macrocode}
%    |frenchb| being an option of babel, it cannot load a package
%    (keyval) while |frenchb.ldf| is read, so we defer the loading of
%    \file{keyval} and the options setup at the end of \babel's loading.
%
%    |StandardLayout| resets the layout in French to the standard layout
%    defined par the \LaTeX{} class and packages loaded. It deals with
%    lists, indentation of first paragraphs of sections and footnotes.
%    Other keys, entered \emph{after} |StandardLayout| in
%    |\frenchbsetup|, can overrule some of the |StandardLayout|
%     settings.
%
%    |GlobalLayoutFrench| forces the layout in French and (as far as
%    possible) outside French to meet the French typographic standards.
%
% \changes{v2.3d}{2009/03/16}{Warning added to \cs{GlobalLayoutFrench}
%    when French is not the main language.}
%
%    \begin{macrocode}
\AtEndOfPackage{%
    \RequirePackage{keyval}%
    \define@key{FB}{StandardLayout}[true]%
                      {\csname FBStandardLayout#1\endcsname
                       \ifFBStandardLayout
                         \FBReduceListSpacingfalse
                         \FBCompactItemizefalse
                         \FBStandardItemLabelstrue
                         \FBIndentFirstfalse
                         \FBFrenchFootnotesfalse
                         \FBAutoSpaceFootnotesfalse
                         \FBGlobalLayoutFrenchfalse
                       \else
                         \FBReduceListSpacingtrue
                         \FBCompactItemizetrue
                         \FBStandardItemLabelsfalse
                         \FBIndentFirsttrue
                         \FBFrenchFootnotestrue
                         \FBAutoSpaceFootnotestrue
                       \fi}%
    \define@key{FB}{GlobalLayoutFrench}[true]%
                      {\csname FBGlobalLayoutFrench#1\endcsname
                       \ifFBGlobalLayoutFrench
                          \iflanguage{french}%
                            {\FBReduceListSpacingtrue
                             \FBCompactItemizetrue
                             \FBStandardItemLabelsfalse
                             \FBIndentFirsttrue
                             \FBFrenchFootnotestrue
                             \FBAutoSpaceFootnotestrue}%
                            {\PackageWarning{frenchb.ldf}%
                              {Option `GlobalLayoutFrench' skipped:
                               \MessageBreak French is *not*
                               babel's last option.\MessageBreak}}%
                       \fi}%
    \define@key{FB}{ReduceListSpacing}[true]%
                      {\csname FBReduceListSpacing#1\endcsname}%
    \define@key{FB}{CompactItemize}[true]%
                      {\csname FBCompactItemize#1\endcsname}%
    \define@key{FB}{StandardItemLabels}[true]%
                      {\csname FBStandardItemLabels#1\endcsname}%
    \define@key{FB}{ItemLabels}{%
        \renewcommand*{\FrenchLabelItem}{#1}}%
    \define@key{FB}{ItemLabeli}{%
        \renewcommand*{\Frlabelitemi}{#1}}%
    \define@key{FB}{ItemLabelii}{%
        \renewcommand*{\Frlabelitemii}{#1}}%
    \define@key{FB}{ItemLabeliii}{%
        \renewcommand*{\Frlabelitemiii}{#1}}%
    \define@key{FB}{ItemLabeliv}{%
        \renewcommand*{\Frlabelitemiv}{#1}}%
    \define@key{FB}{StandardLists}[true]%
                      {\csname FBStandardLists#1\endcsname
                       \ifFBStandardLists
                         \FBReduceListSpacingfalse
                         \FBCompactItemizefalse
                         \FBStandardItemLabelstrue
                       \else
                         \FBReduceListSpacingtrue
                         \FBCompactItemizetrue
                         \FBStandardItemLabelsfalse
                       \fi}%
    \define@key{FB}{IndentFirst}[true]%
                      {\csname FBIndentFirst#1\endcsname}%
    \define@key{FB}{FrenchFootnotes}[true]%
                      {\csname FBFrenchFootnotes#1\endcsname}%
    \define@key{FB}{AutoSpaceFootnotes}[true]%
                      {\csname FBAutoSpaceFootnotes#1\endcsname}%
    \define@key{FB}{AutoSpacePunctuation}[true]%
                      {\csname FBAutoSpacePunctuation#1\endcsname}%
    \define@key{FB}{OriginalTypewriter}[true]%
                      {\csname FBOriginalTypewriter#1\endcsname}%
    \define@key{FB}{ThinColonSpace}[true]%
                      {\csname FBThinColonSpace#1\endcsname}%
    \define@key{FB}{ThinSpaceInFrenchNumbers}[true]%
                      {\csname FBThinSpaceInFrenchNumbers#1\endcsname}%
    \define@key{FB}{FrenchSuperscripts}[true]%
                      {\csname FBFrenchSuperscripts#1\endcsname}
    \define@key{FB}{LowercaseSuperscripts}[true]%
                      {\csname FBLowercaseSuperscripts#1\endcsname}
    \define@key{FB}{PartNameFull}[true]%
                      {\csname FBPartNameFull#1\endcsname}%
    \define@key{FB}{ShowOptions}[true]%
                      {\csname FBShowOptions#1\endcsname}%
%    \end{macrocode}
%    Inputing French quotes as \emph{single characters} when they are
%    available on the keyboard (through a compose key for instance)
%    is more comfortable than typing |\og| and |\fg|.
%    The purpose of the following code is to map the French quote
%    characters to |\og\ignorespaces| and |{\fg}| respectively when
%    the current language is French, and to |\guillemotleft| and
%    |\guillemotright| otherwise (think of German quotes); thus correct
%    unbreakable spaces will be added automatically to French quotes.
%    The quote characters typed in depend on the input encoding,
%    it can be single-byte (latin1, latin9, applemac,\dots) or
%    multi-bytes (utf-8, utf8x).  We first check whether XeTeX is used
%    or not, if not the package |inputenc| has to be loaded before the
%    |\begin{document}| with the proper coding option, so we check if
%    |\DeclareInputText| is defined.
%    \begin{macrocode}
    \define@key{FB}{og}{%
       \newcommand*{\FB@@og}{\iflanguage{french}%
                               {\FB@og\ignorespaces}{\guillemotleft}}%
       \expandafter\ifx\csname XeTeXrevision\endcsname\relax
         \AtBeginDocument{%
           \@ifundefined{DeclareInputText}%
             {\PackageWarning{frenchb.ldf}%
               {Option `og' requires package inputenc.\MessageBreak}%
             }%
             {\@ifundefined{uc@dclc}%
%    \end{macrocode}
%    if |\uc@dclc| is undefined, utf8x is not loaded\dots
%    \begin{macrocode}
               {\@ifundefined{DeclareUnicodeCharacter}%
%    \end{macrocode}
%    if |\DeclareUnicodeCharacter| is undefined, utf8 is not loaded
%    either, we assume 8-bit character input encoding.
%    Package MULEenc.sty (from CJK) defines |\mule@def| to map
%    characters to control sequences.
%    \begin{macrocode}
                  {\@tempcnta`#1\relax
                     \@ifundefined{mule@def}%
                       {\DeclareInputText{\the\@tempcnta}{\FB@@og}}%
                       {\mule@def{11}{{\FB@@og}}}%
                  }%
%    \end{macrocode}
%    utf8 loaded, use |\DeclareUnicodeCharacter|,
%    \begin{macrocode}
                  {\DeclareUnicodeCharacter{00AB}{\FB@@og}}%
               }%
%    \end{macrocode}
%    utf8x loaded, use |\uc@dclc|,
%    \begin{macrocode}
               {\uc@dclc{171}{default}{\FB@@og}}%
             }%
         }%
%    \end{macrocode}
%    XeTeX in use, the following trick for defining the active quote
%    character is borrowed from \file{inputenc.dtx}.
%    \begin{macrocode}
       \else
         \catcode`#1=\active
         \bgroup
           \uccode`\~`#1%
           \uppercase{%
         \egroup
         \def~%
         }{\FB@@og}%
       \fi
    }%
%    \end{macrocode}
%    Same code for the closing quote.
%    \begin{macrocode}
    \define@key{FB}{fg}{%
       \newcommand*{\FB@@fg}{\iflanguage{french}%
                               {\FB@fg}{\guillemotright}}%
       \expandafter\ifx\csname XeTeXrevision\endcsname\relax
         \AtBeginDocument{%
           \@ifundefined{DeclareInputText}%
             {\PackageWarning{frenchb.ldf}%
               {Option `fg' requires package inputenc.\MessageBreak}%
             }%
             {\@ifundefined{uc@dclc}%
               {\@ifundefined{DeclareUnicodeCharacter}%
                  {\@tempcnta`#1\relax
                     \@ifundefined{mule@def}%
                       {\DeclareInputText{\the\@tempcnta}{{\FB@@fg}}}%
                       {\mule@def{27}{{\FB@@fg}}}%
                  }%
                  {\DeclareUnicodeCharacter{00BB}{{\FB@@fg}}}%
               }%
               {\uc@dclc{187}{default}{{\FB@@fg}}}%
             }%
         }%
       \else
         \catcode`#1=\active
         \bgroup
           \uccode`\~`#1%
           \uppercase{%
         \egroup
         \def~%
         }{{\FB@@fg}}%
       \fi
    }%
}
%    \end{macrocode}
%  \end{macro}
%
% \begin{macro}{\FBprocess@options}
%    |\FBprocess@options| processes the options, it is called \emph{once}
%    at |\begin{document}|.
%    \begin{macrocode}
\newcommand*{\FBprocess@options}{%
%    \end{macrocode}
%    Nothing has to be done here for |StandardLayout| and
%    |StandardLists| (the involved flags have already been set in
%    |\frenchbsetup{}| or before (at babel's EndOfPackage).
%
%    The next three options deal with the layout of lists in French.
%
%    |ReduceListSpacing| reduces the vertical spaces between list
%    items in French (done by changing |\list| to |\listFB|).
%    When |GlobalLayoutFrench| is true (the default), the same is
%    done outside French except for languages that force a different
%    setting.
%    \begin{macrocode}
  \ifFBReduceListSpacing
    \addto\extrasfrench{\let\list\listFB
                        \let\endlist\endlistFB}%
    \addto\noextrasfrench{\ifFBGlobalLayoutFrench
                            \let\list\listFB
                            \let\endlist\endlistFB
                          \else
                            \let\list\listORI
                            \let\endlist\endlistORI
                          \fi}%
  \else
    \addto\extrasfrench{\let\list\listORI
                        \let\endlist\endlistORI}%
    \addto\noextrasfrench{\let\list\listORI
                          \let\endlist\endlistORI}%
  \fi
%    \end{macrocode}
%
%    |CompactItemize| suppresses the vertical spacing between list
%    items in French (done by changing |\itemize| to |\itemizeFB|).
%    When |GlobalLayoutFrench| is true the same is done outside French.
%    \begin{macrocode}
  \ifFBCompactItemize
    \addto\extrasfrench{\let\itemize\itemizeFB
                        \let\enditemize\enditemizeFB}%
    \addto\noextrasfrench{\ifFBGlobalLayoutFrench
                             \let\itemize\itemizeFB
                             \let\enditemize\enditemizeFB
                          \else
                             \let\itemize\itemizeORI
                             \let\enditemize\enditemizeORI
                          \fi}%
  \else
    \addto\extrasfrench{\let\itemize\itemizeORI
                        \let\enditemize\enditemizeORI}%
    \addto\noextrasfrench{\let\itemize\itemizeORI
                          \let\enditemize\enditemizeORI}%
  \fi
%    \end{macrocode}
%
%    |StandardItemLabels| resets labelitems in French to their
%    standard values set by the \LaTeX{} class and packages loaded.
%    When |GlobalLayoutFrench| is true labelitems are identical inside
%    and outside French.
%    \begin{macrocode}
  \ifFBStandardItemLabels
    \addto\extrasfrench{\bbl@nonfrenchlabelitems}%
    \addto\noextrasfrench{\bbl@nonfrenchlabelitems}%
  \else
    \addto\extrasfrench{\bbl@frenchlabelitems}%
    \addto\noextrasfrench{\ifFBGlobalLayoutFrench
                            \bbl@frenchlabelitems
                          \else
                            \bbl@nonfrenchlabelitems
                          \fi}%
  \fi
%    \end{macrocode}
%
%    |IndentFirst| forces the first paragraphs of sections to be
%    indented just like the other ones in French.
%    When |GlobalLayoutFrench| is true (the default), the same is
%    done outside French except for languages that force a different
%    setting.
%    \begin{macrocode}
  \ifFBIndentFirst
    \addto\extrasfrench{\bbl@frenchindent}%
    \addto\noextrasfrench{\ifFBGlobalLayoutFrench
                             \bbl@frenchindent
                          \else
                             \bbl@nonfrenchindent
                          \fi}%
  \else
    \addto\extrasfrench{\bbl@nonfrenchindent}%
    \addto\noextrasfrench{\bbl@nonfrenchindent}%
  \fi
%    \end{macrocode}
%
%    The layout of footnotes is handled at the |\begin{document}|
%    depending on the values of flags |FrenchFootnotes|
%    and |AutoSpaceFootnotes| (see section~\ref{sec-footnotes}),
%    nothing has to be done here for footnotes.
%
%    |AutoSpacePunctuation| adds an unbreakable space (in French only)
%    before the four active characters (:;!?) even if none has been
%    typed before them.
%    \begin{macrocode}
  \ifFBAutoSpacePunctuation
     \autospace@beforeFDP
  \else
     \noautospace@beforeFDP
  \fi
%    \end{macrocode}
%
%    When |OriginalTypewriter| is set to |false| (the default),
%    |\ttfamily|, |\rmfamily| and |\sffamily| are redefined as
%    |\ttfamilyFB|, |\rmfamilyFB| and |\sffamilyFB| respectively
%    to prevent addition of automatic spaces before the four active
%    characters in computer code.
%    \begin{macrocode}
  \ifFBOriginalTypewriter
  \else
     \let\ttfamily\ttfamilyFB
     \let\rmfamily\rmfamilyFB
     \let\sffamily\sffamilyFB
  \fi
%    \end{macrocode}
%
%    |ThinColonSpace| changes the normal unbreakable space typeset in
%     French before `:' to a thin space.
%    \begin{macrocode}
  \ifFBThinColonSpace\renewcommand*{\Fcolonspace}{\thinspace}\fi
%    \end{macrocode}
%
%    When |true|, |ThinSpaceInFrenchNumbers| redefines |numprint.sty|'s
%    command |\npstylefrench| to set |\npthousandsep| to |\,|
%    (thinspace) instead of |~| (default) . This option has no effect
%    if package |numprint.sty| is not loaded with `|autolanguage|'.
%    As old versions of |numprint.sty| did not define |\npstylefrench|,
%    we have to provide this command.
%    \begin{macrocode}
  \@ifpackageloaded{numprint}%
  {\ifnprt@autolanguage
     \providecommand*{\npstylefrench}{}%
     \ifFBThinSpaceInFrenchNumbers
       \renewcommand*\npstylefrench{%
          \npthousandsep{\,}%
          \npdecimalsign{,}%
          \npproductsign{\cdot}%
          \npunitseparator{\,}%
          \npdegreeseparator{}%
          \nppercentseparator{\nprt@unitsep}%
          }%
     \else
       \renewcommand*\npstylefrench{%
          \npthousandsep{~}%
          \npdecimalsign{,}%
          \npproductsign{\cdot}%
          \npunitseparator{\,}%
          \npdegreeseparator{}%
          \nppercentseparator{\nprt@unitsep}%
          }%
     \fi
     \npaddtolanguage{french}{french}%
   \fi}{}%
%    \end{macrocode}
%
%    |FrenchSuperscripts|: if |true| |\up=\fup|, else
%    |\up=\textsuperscript|. Anyway |\up*=\FB@up@fake|. The star-form
%    |\up*{}| is provided for fonts that lack some superior letters:
%    Adobe Jenson Pro and Utopia Expert have no ``g superior'' for
%    instance.
%    \begin{macrocode}
  \ifFBFrenchSuperscripts
    \DeclareRobustCommand*{\up}{\@ifstar{\FB@up@fake}{\fup}}%
  \else
    \DeclareRobustCommand*{\up}{\@ifstar{\FB@up@fake}%
                                        {\textsuperscript}}%
  \fi
%    \end{macrocode}
%
%    |LowercaseSuperscripts|: if |true| let |\FB@lc| be |\lowercase|,
%     else |\FB@lc| is redefined to do nothing.
%    \begin{macrocode}
  \ifFBLowercaseSuperscripts
  \else
    \renewcommand*{\FB@lc}[1]{##1}%
  \fi
%    \end{macrocode}
%
%    |PartNameFull|: if |false|, redefine |\partname|.
%    \begin{macrocode}
  \ifFBPartNameFull
  \else\addto\captionsfrench{\def\partname{Partie}}\fi
%    \end{macrocode}
%
%    |ShowOptions|: if |true|, print the list of all options to the
%    \file{.log} file.
%    \begin{macrocode}
  \ifFBShowOptions
    \GenericWarning{* }{%
     * **** List of possible options for frenchb ****\MessageBreak
     [Default values between brackets when frenchb is loaded *LAST*]%
     \MessageBreak
     ShowOptions=true [false]\MessageBreak
     StandardLayout=true [false]\MessageBreak
     GlobalLayoutFrench=false [true]\MessageBreak
     StandardLists=true [false]\MessageBreak
     ReduceListSpacing=false [true]\MessageBreak
     CompactItemize=false [true]\MessageBreak
     StandardItemLabels=true [false]\MessageBreak
     ItemLabels=\textemdash, \textbullet,
        \protect\ding{43},... [\textendash]\MessageBreak
     ItemLabeli=\textemdash, \textbullet,
        \protect\ding{43},... [\textendash]\MessageBreak
     ItemLabelii=\textemdash, \textbullet,
        \protect\ding{43},... [\textendash]\MessageBreak
     ItemLabeliii=\textemdash, \textbullet,
        \protect\ding{43},... [\textendash]\MessageBreak
     ItemLabeliv=\textemdash, \textbullet,
        \protect\ding{43},... [\textendash]\MessageBreak
     IndentFirst=false [true]\MessageBreak
     FrenchFootnotes=false [true]\MessageBreak
     AutoSpaceFootnotes=false [true]\MessageBreak
     AutoSpacePunctuation=false [true]\MessageBreak
     OriginalTypewriter=true [false]\MessageBreak
     ThinColonSpace=true [false]\MessageBreak
     ThinSpaceInFrenchNumbers=true [false]\MessageBreak
     FrenchSuperscripts=false [true]\MessageBreak
     LowercaseSuperscripts=false [true]\MessageBreak
     PartNameFull=false [true]\MessageBreak
     og= <left quote character>, fg= <right quote character>
     \MessageBreak
     *********************************************
     \MessageBreak\protect\frenchbsetup{ShowOptions}}
  \fi
}
%    \end{macrocode}
%  \end{macro}
%
% \changes{v2.0}{2006/12/15}{AtBeginDocument, save again the
%    definitions of the `list' and `itemize' environments and the
%    values of labelitems.  As of frenchb v.1.6, `ORI' values were
%    set when reading frenchb.ldf, later changes were ignored.}
%
% \changes{v2.0}{2006/12/06}{Added warning for OT1 encoding.}
%
% \changes{v2.1b}{2008/04/07}{Disable some commands in bookmarks.}
%
%    At |\begin{document}| we save again the definitions of the `list'
%    and `itemize' environments and the values of labelitems so that
%    all changes made in the preamble are taken into account in
%    languages other than French and in French with the StandardLayout
%    option.  We also have to provide an |\xspace| command in case the
%    |xspace.sty| package is not loaded.
%
%    \begin{macrocode}
\AtBeginDocument{%
   \let\listORI\list
   \let\endlistORI\endlist
   \let\itemizeORI\itemize
   \let\enditemizeORI\enditemize
   \let\@ltiORI\labelitemi
   \let\@ltiiORI\labelitemii
   \let\@ltiiiORI\labelitemiii
   \let\@ltivORI\labelitemiv
   \providecommand*{\xspace}{\relax}%
%    \end{macrocode}
%    Let's redefine some commands in \file{hyperref}'s bookmarks.
%    \begin{macrocode}
   \@ifundefined{pdfstringdefDisableCommands}{}%
     {\pdfstringdefDisableCommands{%
        \let\up\relax
        \def\ieme{e\xspace}%
        \def\iemes{es\xspace}%
        \def\ier{er\xspace}%
        \def\iers{ers\xspace}%
        \def\iere{re\xspace}%
        \def\ieres{res\xspace}%
        \def\FrenchEnumerate#1{#1\degre\space}%
        \def\FrenchPopularEnumerate#1{#1\degre)\space}%
        \def\No{N\degre\space}%
        \def\no{n\degre\space}%
        \def\Nos{N\degre\space}%
        \def\nos{n\degre\space}%
        \def\og{\guillemotleft\space}%
        \def\fg{\space\guillemotright}%
        \let\bsc\textsc
        \let\degres\degre
     }}%
%    \end{macrocode}
%    It is time to process the options set with |\frenchboptions{}|.
%    Then execute either |\extrasfrench| and |\captionsfrench| or
%    |\noextrasfrench| according to the current language at the
%    |\begin{document}| (these three commands are updated by
%    |\FBprocess@options|).
%    \begin{macrocode}
   \FBprocess@options
   \iflanguage{french}{\extrasfrench\captionsfrench}{\noextrasfrench}%
%    \end{macrocode}
%    Some warnings are issued when output font encodings are not
%    properly set. With XeLaTeX, \file{fontspec.sty} and
%    \file{xunicode.sty} should be loaded; with (pdf)\LaTeX, a warning
%    is issued when OT1 encoding is in use at the |\begin{document}|.
%    Mind that |\encodingdefault| is defined as `long', defining
%    |\FBOTone| with |\newcommand*| would fail!
%    \begin{macrocode}
   \expandafter\ifx\csname XeTeXrevision\endcsname\relax
      \begingroup \newcommand{\FBOTone}{OT1}%
      \ifx\encodingdefault\FBOTone
        \PackageWarning{frenchb.ldf}%
           {OT1 encoding should not be used for French.
            \MessageBreak
            Add \protect\usepackage[T1]{fontenc} to the
            preamble\MessageBreak of your document,}
      \fi
     \endgroup
   \else
     \@ifundefined{DeclareUTFcharacter}%
       {\PackageWarning{frenchb.ldf}%
         {Add \protect\usepackage{fontspec} *and*\MessageBreak
          \protect\usepackage{xunicode} to the preamble\MessageBreak
          of your document,}}%
       {}%
    \fi
}
%    \end{macrocode}
%
%  \subsection{Clean up and exit}
%
%    Load |frenchb.cfg| (should do nothing, just for compatibility).
%    \begin{macrocode}
\loadlocalcfg{frenchb}
%    \end{macrocode}
%    Final cleaning.
%    The macro |\ldf@quit| takes care for setting the main language
%    to be switched on at |\begin{document}| and resetting the
%    category code of \texttt{@} to its original value.
%    The config file searched for has to be |frenchb.cfg|, and
%    |\CurrentOption| has been set to `french', so
%    |\ldf@finish\CurrentOption| cannot be used: we first load
%    |frenchb.cfg|, then call |\ldf@quit\CurrentOption|.
%    \begin{macrocode}
\FBclean@on@exit
\ldf@quit\CurrentOption
%    \end{macrocode}
% \iffalse
%</code>
%<*dtx>
% \fi
%%
%% \CharacterTable
%%  {Upper-case    \A\B\C\D\E\F\G\H\I\J\K\L\M\N\O\P\Q\R\S\T\U\V\W\X\Y\Z
%%   Lower-case    \a\b\c\d\e\f\g\h\i\j\k\l\m\n\o\p\q\r\s\t\u\v\w\x\y\z
%%   Digits        \0\1\2\3\4\5\6\7\8\9
%%   Exclamation   \!     Double quote  \"     Hash (number) \#
%%   Dollar        \$     Percent       \%     Ampersand     \&
%%   Acute accent  \'     Left paren    \(     Right paren   \)
%%   Asterisk      \*     Plus          \+     Comma         \,
%%   Minus         \-     Point         \.     Solidus       \/
%%   Colon         \:     Semicolon     \;     Less than     \<
%%   Equals        \=     Greater than  \>     Question mark \?
%%   Commercial at \@     Left bracket  \[     Backslash     \\
%%   Right bracket \]     Circumflex    \^     Underscore    \_
%%   Grave accent  \`     Left brace    \{     Vertical bar  \|
%%   Right brace   \}     Tilde         \~}
%%
% \iffalse
%</dtx>
% \fi
%
% \Finale
\endinput
}%
\DeclareOption{galician}{% \iffalse meta-comment
%
% Copyright 1989-2008 Johannes L. Braams and any individual authors
% listed elsewhere in this file.  All rights reserved.
% 
% This file is part of the Babel system.
% --------------------------------------
% 
% It may be distributed and/or modified under the
% conditions of the LaTeX Project Public License, either version 1.3
% of this license or (at your option) any later version.
% The latest version of this license is in
%   http://www.latex-project.org/lppl.txt
% and version 1.3 or later is part of all distributions of LaTeX
% version 2003/12/01 or later.
% 
% This work has the LPPL maintenance status "maintained".
% 
% The Current Maintainer of this work is Johannes Braams.
% 
% The list of all files belonging to the Babel system is
% given in the file `manifest.bbl. See also `legal.bbl' for additional
% information.
% 
% The list of derived (unpacked) files belonging to the distribution
% and covered by LPPL is defined by the unpacking scripts (with
% extension .ins) which are part of the distribution.
% \fi
% \CheckSum{2264}
% \ProvidesFile{galician.dtx}
%       [2007/10/11 v4.3b Galician support from the babel system] 
%\iffalse
%% File `galician.dtx'
%% Babel package for LaTeX version 2e
%% Copyright (C) 1989 - 2008
%%           by Johannes Braams, TeXniek
%
%% Galcian Language Definition File
%% Copyright (C) 1989 - 2006
%%           by Manuel Carriba mcarriba at eunetcom.net
%%           Johannes Braams, TeXniek
%% Copyright (C) 2007 - 2008
%%           by Javier A. M\'ugica
%%           Johannes Braams, TeXniek
%
%% Please report errors to: Javier A. Mugica (preferably)
%%                          jmugica at digi21.net
%%                          J.L. Braams
%%                          babel at braams.xs4all.nl
%
%    This file is part of the babel system, it provides the source
%    code for the Galician language definition file.
%    The original version of the file spanish.dtx was written by Javier Bezos.
%
%<*filedriver>
\documentclass[galician,a4paper]{ltxdoc}
\usepackage[activeacute]{babel}
%Small suport for the input encoding
\catcode`\�=\active  \def�{\'a}
\catcode`\�=\active  \def�{\'e}
\catcode`\�=\active  \def�{\'\i}
\catcode`\�=\active  \def�{\'o}
\catcode`\�=\active  \def�{\'u}
\catcode`\�=\active  \def�{\~n}
%end

%\usepackage{ps}

\newcommand*\TeXhax{\TeX hax}
\newcommand*\babel{\textsf{babel}}
\newcommand*\langvar{$\langle \it lang \rangle$}
\newcommand*\note[1]{}
\newcommand*\Lopt[1]{\textsf{#1}}
\newcommand*\file[1]{\texttt{#1}}

\setlength{\arrayrulewidth}{2\arrayrulewidth}
\newcommand\toprule{\cline{1-2}\\[-2ex]}
\newcommand\botrule{\\[.6ex]\cline{1-2}}
\newcommand\hmk{$\string|$}

\newcommand\New[1]{%
  \leavevmode\marginpar{\raggedleft\sffamily Novo en #1}}

\newcommand\nm[1]{\unskip\,$^{#1}$}
\newcommand\nt[1]{\quad$^{#1}$\,\ignorespaces}

\makeatletter
  \renewcommand\@biblabel{}
\makeatother

\newcommand\DOT[1]{\msc{DOT},~#1}
\newcommand\DTL[1]{\msc{DTL},~#1}
\newcommand\MEA[1]{\msc{MEA},~#1}

\raggedright
\addtolength{\oddsidemargin}{-2pc}
\addtolength{\textwidth}{2pc}
\setlength{\parindent}{1em}

\OnlyDescription
\begin{document}
   \DocInput{galician.dtx}
\end{document}
%</filedriver>
%\fi
%
% \begingroup
% 
% \catcode`\{=11
% \catcode`\[=1
% 
% \gdef\ignoringuser[^^A
%   \bgroup
%   \let\end{userdtx\fi
%   \let\end{userdrv\fi
%   \aftergroup\endgroup
%   \catcode`\{=11
%   \catcode`f=14 ^^A kills \if... and \fi
%   \catcode`l=14 ^^A kills \else
%   \catcode`o=14 ^^A kills \or
%   \catcode`p=14 ^^A kills \repeat
%   \iffalse}
%   
% \endgroup
% 
% ^^A\def\langdeffile{}
% \newenvironment{userdtx}
%   {\ifx\langdeffile\undefined
%    \else\expandafter\ignoringuser
%    \fi}{}
% 
% \newenvironment{userdrv}
%   {\ifx\langdeffile\undefined
%    \expandafter\ignoringuser
%    \else\fi}{}
%
% \begin{userdtx}
%^^A ======= Beginning of text as typeset by galician.dtx =========
%
% \title{Estilo \textsf{galician} para o sistema \babel\\[10pt]
%       \large Versi�n 4.3, de 29 de xaneiro do 2007}
%
% \author{Javier M�gica\footnote{O autor orixinal foi Javier Bezos.
%   Foi traduciro por min do castel�n en xaneiro do 2007, e desde 
%   ent�n e mantido e actualizado por min.}}
%
% \date{} ^^A@#
%
% \maketitle
% 
% {\small\tableofcontents}
%
%% \section{\textsf{galician} coma lingua principal}
% 
% En \babel{} consid�rase que a �ltima lingua citada en |\usepackage| e |\documentclass|,
% por esta orde, � a lingua principal. Se a lingua principal � |galician|, act�vase o grupo
% |\layoutgalician| que adapta varios elementos �s usos tipogr'aficos galegos do seguinte modo:
%
% \begin{itemize}
%
% \item[$\diamond$] |enumerate| e |itemize|\vadjust{\nobreak}\\[1ex]
%
% O primeiro usa a seguinte secuencia:\vadjust{\nobreak}\\
% \quad 1.\\
% \qquad \emph{a})\\
% \quad\qquad 1)\\
% \qquad\qquad \emph{a$'$})\\
% O segundo a seguinte:\\
% \quad\leavevmode\hbox to 1.2ex
%     {\hss\vrule height .95ex width .8ex depth -.15ex\hss}\\
% \qquad\textbullet\\
% \quad\qquad $\circ$\\
% \qquad\qquad $\diamond$
%
% D�as �rdenes permiten outros estilos en |itemize|: con |\galiciandashitems| c�mbiase a raias en t�do-los niveis
% e con |\galiciansignitems|, a \textbullet{} $\circ$
% $\diamond$ $\triangleright$.
%
% \item[$\diamond$] |\alph| e |\Alph|\\[1ex]
%
% Incl�en o e'ne, pero non k, j, w nin y.
%
% \item[$\diamond$] |\fnsymbol|\\[1ex]
%
% Empr�ganse un, dous, tres... asteriscos (*, **, ***, etc.),
% en lugar da sucesi'on angloamericana de cruces, barras,
% etc.\footnote{\DOT{162}.}
%
% \item[$\diamond$] |\guillemotleft| e |\guillemotright|\\[1ex]
%
% As comi�as latinas para |OT1| son menos angulosas e xen�ranse cunhas puntas de frecha |lasy|.
%
% \item[$\diamond$] |\roman|\\[1ex]
%
% Os n�meros romanos en min�scula son propios das obras inglesas e non se empregan no continente.
% Por iso se redefine |\roman| para que produza versalitas.
%
% \begin{quote}\small
% \textbf{Nota.} MakeIndex non pode entender a forma na que
% |\roman| escribe o n'umero de p'axina, polo que
% elimina as li�as afectadas. Por iso o aquivo |.idx|
% ten que ser convertido antes de ser procesado con MakeIndex.
% Con este paquete proporci�nase a utilidade |glromidx.tex| que se
% encarga diso. Simplemente componse ese arquivo con \LaTeX{}
% e a continuaci'on resp�ndese �s preguntas que se formulan.
% Este proceso non � necesario se non se introduciu ningunha
% entrada de 'indice en p'axinas numeradas con |\roman| (o que ser'a
% o m�is normal). Se un s'imbolo propio de \emph{MakeIndex} xenerara
% problemas, debe encerrarse entre chaves: \verb={"|}=.
% \end{quote}
%
% \item[$\diamond$] |\section|, |\subsection|, etc.\\[1ex]
%
% Os n'umeros nos t'itulos est'an seguidos dun punto tanto no texto coma
% no 'indice. Adem'ais, o primeiro par�grafo tra-lo t�tulo non elimina a sangr'ia
% (de novo, un costume angloamericano).
% 
% \end{itemize}
%
% Estes cambios funcionan coas clases est'andar ~---con otras
% tal vez alg�n deles non te�a efecto--- e persisten durante
% todo o documento (non se poden desactivar). Ning�n deles �
% necesario para compo�er o documento, a�nda que naturalmente
% o resultado ser'a distinto.
% 
% \begin{itemize}
%
% \item[$\diamond$] |\selectgalician*|\vadjust{\nobreak}\\[1ex]
% 
% Se non se desexan estes cambios, chega con usar no
% pre'ambulo |\selectgalician*| (con asterisco) ou <<borralos>> con:
%\begin{verbatim}
% \let\layoutgalician\relax
%\end{verbatim}
% 
% \end{itemize}
%
% \section{Descripci'on}
%
% \subsection{Traducci�ns}
% 
% Certas ordes def�nense para proporcionar traducci�ns
% � galego dalg�ns t�rminos, tal e como se describe no cadro 1.
% 
% \begin{table}
% \center\small
% \caption{Traducci�ns}
% \vspace{1.5ex}
% \begin{tabular}{l@{\hspace{3em}}l}
% \toprule
% |\refname|        & Referencias\\
% |\abstractname|   & Resumo\\
% |\bibname|        & Bibliograf'ia\\
% |\chaptername|    & Cap'itulo\\
% |\appendixname|   & Ap'endice\\
% |\contentsname|   & 'Indice xeral\nm{a}\\
% |\listfigurename| & 'Indice de figuras\\
% |\listtablename|  & 'Indice de cadros\\
% |\indexname|      & 'Indice alfab'etico\\
% |\figurename|     & Figura\\
% |\tablename|      & Cadro\\
% |\partname|       & Parte\\
% |\enclname|       & Adxunto\\
% |\ccname|         & Copia a\\
% |\headtoname|     & A\\
% |\pagename|       & P'axina\\
% |\seename|        & v'exase\\
% |\alsoname|       & v'exase tam'en\\
% |\proofname|      & Demostraci'on
% \botrule
% \end{tabular}
%
% \vspace{1.5ex}
%
% \begin{minipage}{6cm}\footnotesize
% \nt{a} Pero s'olo <<'Indice>> en \textsf{article}.
% \end{minipage}
% \end{table}
% 
%  Non existe para o �ndice unha terminolox'ia unificada. Tal vez
%  \emph{'Indice xeral} � o que m'ais se usa, as'i que a
%  iso me ate�o salvo en |article|, onde se comp�n coma secci'on
%  e polo tanto resulta algo ostentoso.\footnote{\'O contrario que en
%  ingl'es, en galego o 'indice por antonomasia � o xeral.}
% 
%  Para o 'indice alfab'etico tense proposto \emph{'Indice de materias}
%  ou \emph{'Indice anal'itico}, a�nda que estes 'indices non soamente
%  adoitan inclu�r materias, mais tam�n nomes; \emph{'Indice alfab'etico}
%  � m'ais preciso.\footnote{\'E a usada en \DOT{300} as'i coma na meirande
%  parte dos libros que consultei � azar nunha biblioteca. (J.B.)}
% 
%  En canto �s de cadros e figuras, tam'en � posible dicir
%  \emph{lista}, pero par�ceme preferible \emph{'indice}, que
%  implica a correspondencia coas p'axinas.
% 
%  \emph{Table} debe traducirse por \emph{cadro}, xa que \emph{t�boa}
%  � un \emph{falso amigo};\,\footnote{V'exanse as definici�ns
%  do \msc{DGII} e \DTL{67 ss.}} esa � a pr'actica tradicional.
%  (Por exemplo, <<cadro de estados medievais>> frente a <<t�boa de logaritmos>>.)
%
%  As traducci�n escr�bense con min'usculas, salvo a inicial. Ev�tase
%  o anglicismo de comenzar con mai'usculas os substantivos.\footnote{\DOT{197}.}
%
%  A orde |\today| d� a data actual. Con |\galiciandatedo| e |\galiciandatede|
%  �ptase por \textit{do} (predeterminado) ou \textit{de}.
%
% \subsection{Abreviaci�ns}
% 
%  (O que en \babel{} denom�nase `shorthands'.) A lista completa
%  p�dese atopar no cadro~2. Nos seguintes apartados daranse m�is
%  detalles sobre algunhas delas.
%
% \begin{table}[!t]
% \center\small
% \caption{Abreviaci�ns}
% \vspace{1.5ex}
% \begin{tabular}{l@{\hspace{3em}}l}
% \toprule
% |'a 'e 'i 'o 'u| & 'a 'e 'i 'o 'u\\
% |'A 'E 'I 'O 'U| & 'A 'E 'I 'O 'U\\
% |'n 'N|          & 'n 'N\nm{a}\\
% |"u "U|          & "u "U\\
% |"i "I|          & "i "I\\
% |"a "A "o "O|    & Ordinais: 1"a, 1"A, 1"o, 1"O\\
% |"rr "RR|        & rr, pero -r cando se divide\\
% |"-|             & Coma |\-|, pero permite m'ais divisi�ns\\
% |"=|             & Coma |-|, pero permite m�is divisi�ns\nm{b}\\
% |"~|             & Gui'on estil'istico\nm{c}\\
% |~- ~-- ~---|    & Coma |-|, |--| e |---|, pero sen divisi'on\\
% |""|             & Permite m�is divisi�ns antes e despois\nm{d}\\
% |"/|             & Unha barra algo m'ais baixa\\
% \verb+"|+        & Divide un logotipo\nm{e}\\
% |<< >>|          & << >>\\
% |"< ">|          & |\begin{quoting}| |\end{quoting}|\nm{f}\\
% |?` !`|          & ?` !`\nm{g}\\
% |"? "!|          & "? "! ali�ados coa li�a base\nm{h}
% \botrule
% \end{tabular}
%
% \vspace{1.5ex}
%
% \begin{minipage}{11cm}
% \footnotesize
% \nt{a} A forma |~n| debe considerarse en extinci'on.
% \nt{b} |"=| ven a ser o mesmo que |""-""|.
% \nt{c} Esta abreviaci'on ten un uso distinto noutras
% linguas. \nt{d} Coma en <<entrada/sa�da>>.
% \nt{e} Carece de uso en galego. \nt{f} V'exase sec.~2.7. 
% \nt{g} Non proporcionadas por este paquete, mais por cada tipo;
% figuran aqu'i coma simple recordatorio. \nt{h} 'Utiles en
% r'otulos en mai'usculas.
% \end{minipage}
% \end{table}
% 
% Para poder usar ap'ostrofos coma abreviaci�ns de acentos
% � necesaria a opci'on |activeacute| en |\usepackage|.
% Pode cambiarse este comportamento coa orde |\gl@acuteactive|
% no arquivo de configuraci'on |galician.cfg|; nese caso os 
% ap'ostrofos act�vanse sempre.
% 
% Os caracteres usados coma abreviaci�ns comp�rtanse coma
% outras ordes de \TeX{} e polo tanto faise caso omiso
% dos espacios que poidan seguir: \verb*|' a| � o mesmo que |'a|.
% Eso tam'en implica que tras eses caracteres non pode vir unha
% chave de peche e que deber'a escribirse |{... '{}}| en lugar 
% de |{... '}|; en modo matem'atico non hai problema e |$x^{a'}$|
% ($x^{a'}$) � v'alido.
% 
% \begin{itemize}
%
% \item[$\diamond$]
% |\deactivatetilden|\\[1ex]
% 
% Esta orde desactiva as abreviaci�ns |~n| e |~N| debido �s problemas
% que presentan. Pode usarse no arquivo de configuraci'on (v'exase
% m'ais abaixo).
% 
% % \item[$\diamond$]
% |\galiciandeactivate{<caracteres>}|\\[1ex]
% 
% Permite desactivar as abreviaci�ns correspondentes �s caracteres dados.
% Para evitar entrar en conflicto con outras linguas, � sair de |galician|
% react�vanse,\footnote{O punto para os decimais non � estrictamente
% unha abreviaci'on e non se reactiva.} polo que se se desexa que 
% persista c�mpre engadi-la orde a |\shorthandsgalician| con |\addto|.
% A orde |\renewcommand\shorthandsgalician{}| � unha variante optimizada de
%\begin{verbatim}
% \addto\shorthandsgalician{\galiciandeactivate{.'"~<>}}
%\end{verbatim}
% e � o recomendado se se desea prescindir do mecanismo de abreviaci�ns.
% \end{itemize}
%
% \subsection{Coma decimal}
%
% En \textsf{galician} t�mase partido, coma en castel�n, por separar
% a parte enteira e a parte decimal mediante unha coma. O punto �
% tradicional en ingl�s, non en galego, e adem�is chegouse a unha
% normalizaci�n internacional pola que os milleiros sep�ranse por
% un espacio fino e os decimais con coma.
%
% Dado que \TeX\ usa a coma coma separador en intervalos ou expresi�ns
% similares, o que engade un espacio fino, \textsf{galician} converte 
% todo punto en modo matem'atico nunha coma sempre que estea seguido
% dunha cifra, pero non notras circunstancias:
% \begin{quote}\small\begin{tabbing}
% |$1\,234.567\,890$|     \quad \=  $1\,234.567\,890$\\
% |$f(1,2)=12.34.$|        \> $f(1,2)=12.34.$\\
% |$1{.}000$|              \> $1{.}000$, mais\\
% |1.000|                  \> 1.000, pois non � modo matem'atico.
% \end{tabbing}\end{quote}
%
% As ordes |\decimalcomma| e |\decimalpoint| establecen se se usa
% unha coma, que � o valor predeterminado, ou un punto, mentras que
% |\galiciandecimal{<math>}| permite darlle unha definici'on
% arbitraria.\footnote{Internamente o mecanismo � o dunha abreviaci'on,
% e p�dese desactivar coma as outras.}
%
% \subsection{Divisi'on de palabras}
%
% \textsf{Galician} comproba a codificaci'on no momento no que se
% emprega un acento: se a codificaci'on � |OT1| t�manse medidas para 
% facilitar a divisi'on, que pese a todo nunca ser'a perfecta, mentres 
% que con |T1| acc�dese directamente � car'acter correspondente.
%
% Para matizar a divisi'on de palabras hai catro posibilidades, d�as
% delas co m'etodo de abreviaci�ns:
% \begin{itemize}
% \item a orde |\-| � un gui'on opcional que non permite m'ais 
% divisi�ns (coma en \TeX),
%
% \item |"-| � similar pero permite m'ais divisi�ns,
%
% \item un |-| � un gui'on que non permite m'ais divisi�ns nin
% antes nin despois (coma en \TeX), e
% 
% \item |"=| � o equivalente que s'i as permite,\footnote{Non
% � unha boa idea usar esta orde, pero en medidas moi curtas
% puede resultar necesario.}
%
% \end{itemize}
% Por exemplo (coas posibles divisi�ns marcadas con \hmk):
% \begin{quote}\small\begin{tabbing}
% |Zaragoza-Barcelona|\qquad \= Zaragoza-\hmk Barcelona\\
% |Zaragoza"=Barcelona| \>
%    Za\hmk ra\hmk go\hmk za-\hmk Bar\hmk ce\hmk lo\hmk na\\
% |semi\-aperto| \> semi\hmk aperto\\
% |semi"-aperto| \> se\hmk mi\hmk aper\hmk to.\footnotemark
% \end{tabbing}\footnotetext{Xusto antes e desp�is de
% {\ttfamily\string"\string-} e {\ttfamily\string"\string=}
% apl�canse os correspondentes valores de
% {\ttfamily\string\...hyphenmin}, o que implica que a
% divis'on semia\hmk perto non � posible.
% Este � un comportamento correcto.}
% \end{quote}
%
% Adem'ais hai abreviaci�ns que evitan divisi�ns: |~-|, que resulta
% 'util para expresar unha serie de n'umeros sen que o gui'on os
% divida (12~-14, |12~-14|), e |~---|, que � a forma que debe usarse
% para abrir incisos con raias, xa que do contrario pode haber unha
% divisi'on entre a raia de abrir e a palabra que lle sigue:
% \begin{quote}\small\begin{tabbing}
% |Os concertos ~---ou academias--- que organiz'ou...|
% \end{tabbing}\end{quote}
% Mentras que este gui'on evita toda posible divisi'on nos elementos
% que une, a raia (---) e a semirraia (--) perm�tenas nas palabras
% que a precedan ou sigan.
% 
% A abreviaci'on |"~| �sase cando se quere que o gui'on tam'en apareza
% � comienzo da seguinte li�a. Por exemplo:
% \begin{quote}\small\begin{tabbing}
% |infra"~vermello|  \quad \= in\hmk fra-ver\hmk mello, pero infra-\hmk-vermello.
% \end{tabbing}\end{quote}
%
% Outra abreviaci'on � |"rr| que sirve para o 'unico cambio de escritura do
% castel�n en caso de haber divisi'on.\footnote{En galego non atopei un 
% pronunciamento � respecto. A \msc{RAE} indica que � a'nadir un prefixo 
% que reamta en vocal a unha palabra que comenza con \emph{r}, esta 'ultima debe 
% dobrarse a menos que se unan por un gui'on. Por exemplo:
% |extra"{}rradio| div�dese en ex\hmk trarra\hmk dio, mais extra-\hmk radio.}
%
% \subsection{Ordinais}
% 
% As abreviaturas sempre levan punto, salvo algunhas nas que se
% substit�e por unha barra (e salvo as siglas e s'imbolos, naturalmente),
% que precede �s letras voladitas.\footnote{\DTL{196}. V'exase tam'en \DOT{222 e 227}.}
% Por ello, \textsf{galician} proporciona a orde |\sptext| que facilita a
% creaci'on destas abreviaturas. Por exemplo: |adm\sptext{'on}|, que d�
% adm\sptext{'on}. Hai catro abreviaci�ns asociadas a ordinais:
% |"a|, |"A|, |"o| e |"O| que equivalen a |\sptext{a}|, etc.\footnote{Moitos
% tipos engaden un pequeno subli�ado que debe evitarse, e polo tanto non se debe
% escribi-los ordinais con \textsf{inputenc}.}
%
% Para axustar o tamano o mellor posible, �sase o de 'indices en curso.
% Esto funciona ben salvo para tama'nos moi grandes ou moi pequenos, onde
% os resultados son meramente aceptables. 
%
% En Plain \TeX{} exec�tase a orde |\sptextfont| para a letra voladita,
% de xeito que |{\bf\let\sptextfont\bf 1"o}| d� o resultado correcto
% (|\mit| se � para cursiva). Para usar un tipo novo con |\sptext| hai
% que definir tam'en as variantes matem'aticas con |\newfam|.
% 
% \subsection{Funci�ns matem'aticas}
%
% En castel�n, tradicionalmente form�ronse as abreviaci�ns
% do que en \TeX\ co�ecense coma operadores a partir do nome castel�n,
% o que implica a presencia do acento en l'im, m'ax, m'in, 'inf e m'od.
% Parece que isto e tam�n o m�is adecuado para o galego.
%
% Con |galician| p�dense seguir varios criterios por medio das seguintes
% ordes:
% \begin{itemize}
% \item[$\diamond$] |\accentedoperators| |\unaccentedoperators|\\[1ex]
% Activa ou desactiva os acentos.
% Por omisi'on acent'uanse, coma por exemplo: $\lim_{x\to 0}(1/x)$
% (|$\lim_{x\to 0}(1/x)$|).
% 
% \item[$\diamond$] |\spacedoperators| |\unspacedoperators|\\[1ex]
% Activa ou desactiva o espacio entre "<arc"> e la funci'on. Por omisi'on
% non se espacia.
% \end{itemize}
% 
% O i sen punto tam'en � accesible directamente en modo matem'atico
% coa orde |\dotlessi|, de forma que se pode escribir |\acute{\dotlessi}|.
% Por exemplo, 
% |$V_{\mathbf{cr\acute{\dotlessi}t}}$| d� $V_{\mathbf{cr\acute{\dotlessi}t}}$.
%
% Tam'en eng�dense |\sen|, |\cosec|, |\arcsen|, |\tx|, |\arctx|, e |senh|,
% que dan as funci�ns respectivas. Adem�is |\sin|,\New{galician~4.3}
% |sinh| e |arcsin| producen os mesmos resultados que |\sen|, |\senh|
% e |\arcsen| respectivamente. Outras funci�ns trigonom'etricas
% at�panse almacenadas no par'ametro |\galicianoperators|,
% que inicialmente incl�e cotx e txh. Deste xeito p�dense cambiar por
% outras e engadir m�is, coma por exemplo:
%\begin{verbatim}
% \renewcommand{\galicianoperators}{cotan arc\,ctx tanh}
%\end{verbatim}
% (separadas con espacio). Cando se selecciona |galician| cr�anse
% ordes con eses nomes e que dan esas funci�ns (sempre con |\nolimits|).
% Adem'ais das letras sen acentuar ac�ptanse as ordes |\,| e |\acute|, que
% se pasan por alto para forma-lo nome. Por exemplo, |arc\,ctg| escribirase
% no documento coma |\arcctg|, |M\acute{a}x| coma |\Max| e |cr\acute{i}t|
% coma |\crit| (hai que usar |i| e non |\dotlessi|). A orde |\,| responde
% a |\|(|un|)|spacedoperators|, e |\acute| a |\|(|un|)|accentedoperators|.
% 
% |\renewcommand{\galicianoperators}| haber� de estar no pre'ambulo do documento,
% despois de |\sepackage[galician]{babel}| e antes de |\selectgalician| ou de 
% |\begin{document}|.
%
% \subsection{Entrecomi�ados}
% 
% O entorno |quoting| entrecomi�a un texto, engadindo comi�as de seguir �
% comienzo de cada par�grafo ou no seu interior.\footnote{P�dese atopar unha
% detallada exposici'on das comi�as en \DTL{44 ss.} De al� tomouse alg'un exemplo.}
% Tam'en p�dense empregrar as abreviaci�ns |"<| e |">| que se limitan a chamar
% a |quoting|, que por ser entorno considera os seus cambios internos coma locais.
% (� dicir, |"< ... ">| implica |{"< ... ">}|.) As abreviaci�ns |<<| e |>>|
% contin'uan dando sen m'ais os caracteres de comi�as de abrir e pechar.
% 
% Por exemplo:
%\begin{verbatim}
% "<Ch�manse "<comi�as de seguir"> as que son de peche, pero col�canse
% � comenzo de cada par�grafo cando se transcribe un texto
% entrecomi�ado con m'ais dun par'agrafo.
% 
% No seu interior, coma de costume, �sanse as inglesas.">
%\end{verbatim}
% que ten por resultad:
% \begin{quotation}\small
% "<Ch�manse "<comillas de seguir"> as que son de peche, pero col�canse
% � comenzo de cada par'agrafo cando se transcribe un texto
% entrecomi�ado con m'ais dun par'agrafo.
% 
% No seu interior, coma de costume, �sanse as inglesas.">
% \end{quotation}
% 
% Este entorno p�dese redefinir. Por exemplo:
%\begin{verbatim}
% \renewenvironment{quoting}{\itshape}{}
%\end{verbatim}
% pero non implica un novo par'agrafo, xa que est'a pensado
% para ser usado tam'en no texto.
%
% En caso de inclu�r uns entornos |quoting| dentro doutros, modif�canse
% as comi�as dos niveis interiores, que tam'en engadense �s de seguir:
%\begin{verbatim}
% "<O di'alogo desenvolveuse desta forma:
% 
% "<---Eu no fun ---grit'ou Antonio.
% 
% ---Mais colaboraches ---asegur'ou Rafael">.
% 
% Mais al'i ning�en se aclarou.">
%\end{verbatim}
% 
% \begin{quotation}\small
% "<O di'alogo desenvolveuse desta forma:
% 
% "<---Eu no fun ---grit'ou Antonio.
% 
% ---Mais colaboraches ---asegur'ou Rafael">.
% 
% Mais al'i ning�en se aclarou.">
% \end{quotation}
%
% \begin{itemize}
% \item[$\diamond$]
% |\lquoti| |\rquoti| |\lquotii| |\rquotii| |\lquotiii| 
% |\rquotiii|\\[1ex]
% 
% Controlan as comi�as en |quoting|, segundo o nivel no que nos
% atopemos. |\lquoti| son as comi�as de abrir m'ais exteriores,
% |\lquotii| as de segundo nivel, etc., e o mesmo para as de pechar
% con |\rquoti|... Para as de seguir sempre se usan as de pechar.
% Os valores predefinidos est'an no cadro 3.
% \begin{table}
% \center\small
% \caption{Entrecomi�ados}
% \vspace{1.5ex}
% \begin{tabular}{l@{\hspace{5em}}l}
% \toprule
% |\lquoti|   &|<<|\\
% |\rquoti|   &|>>|\\
% |\lquotii|  &|``|\\
% |\rquotii|  &|''|\\
% |\lquotiii| &|`|\\
% |\rquotiii| &|'|
% \botrule
% \end{tabular}
% \end{table}
% 
% \newenvironment{dialog}
%   {\def\lquoti{}\begin{quoting}---\ignorespaces}
%   {\def\rquoti{}\end{quoting}}
% 
% As comi�as de seguir tam'en empr�ganse en di'alogos, incluso
% se non as hai de abrir e pechar. Coa axuda do seguinte entorno,
%\begin{verbatim}
% \newenvironment{dialog}
%   {\def\lquoti{}\begin{quoting}---\ignorespaces}
%   {\def\rquoti{}\end{quoting}}
%\end{verbatim}
% podemos obter
% \begin{quotation}\small
% \begin{dialog}%
% O di'alogo desenvolveuse deste xeito:
% 
% "<---Eu non fun ---grit'ou Antonio.
% 
% ---Mais colaboraches ---asegur'ou Rafael">.
%
% Mais al'i ningu�n se aclarou.
% \end{dialog}
% \end{quotation}
%
% \item[$\diamond$]
% |\activatequoting \deactivatequoting|\\[1ex]
% 
% As incompatibilidades potenciais destas abreviaci�ns son enormes.
% Por exemplo, en \textsf{ifthen} canc�lanse as comparaci�ns entre
% n'umeros;\,\footnote{E en |\string\ifnum|, etc. usado polos
% desenroladores nos paquetes.} tam'en resultan inoperantes |@>>>|
% e |@<<<| de \textsf{amstex}.\footnote{A�nda que neste caso cabe
% empregar os sin'onimos |@)))| e |@(((|.} Por iso, dase a posibilidade
% de cancelalas e reactivalas con estas ordes, a�nda que se se
% est'a a usar con \textsf{xmltex} xa se desactivan por completo de
% modo autom'atico. O entorno |quoting| sempre permanece disponible.
% \footnote{Alg�ns tipos pos�en as ligaduras |<{}<| e |>{}>| de
% forma interna para xenerar os caracteres de comi�as, polo que
% neles tamb'en podemos usalos siempre.}
%
% \end{itemize}
% 
% \subsection{Espaciado}
% 
% O espaciado en galego af�stase relativamente pouco do ingl'es;
% Non obstante, o espacio tra-los signos de puntuaci'on debe 
% de ser o mesmo que o que hai entre palabras. Ou dito en
% t'erminos de \TeX, |\frenchspacing| est'a activo.
% 
% Tam'en noutros dous sitios hai diferencias. O primero son os
% puntos suspensivos, para os que se redefine |\dots| e |\cdots|,\New{galician 4.3}
% que os dan menos espaciados, e para |\dots| no texto eng�dese a
% continuaci�n un espacio igual que o que seguir�a a un punto.
% Por exemplo:
% \begin{quote}\small\begin{tabbing}
% |\dots non sei se canta\dots ou chora.|\quad \dots non sei se canta\dots ou chora.\\
% |$f(x_1,\dots,x_n,y_1,\ldots,y_m)$|\quad $f(x_1,\dots,x_n,y_1,\ldots,y_m)$\\
% |$1+2+\cdots+n$|\quad $1+2+\cdots+n$
% \end{tabbing}\end{quote}
% Tam'en poder'ianse escribir os tres puntos sen m'ais... , e na
% pr'actica non hai diferencia. |galician| realiza as modificaci�ns
% a moi baixo nivel, para que sexan compatibles cos cambios que tam�n
% introduce |amsmath| nestes comandos.
%
% O segundo sitio � un espacio fino antes do signo |\%| (que m'ais
% exactamente � |\,|, logo p�dese <<recuperar>> co seu oposto |\!|,
% se |\%| no segue a unha cifra).
%
% \subsection{Pequenas e medianas versais}
%
% |\lsc| p�dese empregar para letras en versalitas. Tam�n existe
% |\msc|,\New{galician 4.3} que selecciona un tipo lixeiramente maior.
% Por exemplo:
% \begin{quote}\small\begin{tabbing}
% |\lsc{DOG}, \lsc{RenFe}| \quad \= No \lsc{DOG} publicouse que \lsc{ReNFe} deber'a...\\
% |\msc{DOG}, \msc{ReNFe}| \quad \= No \msc{DOG} publicouse que \msc{ReNFe} deber'a...
% \end{tabbing}\end{quote}
% Tam'en pode ser 'util para alguns usos dos n'umeros romanos:
% \begin{quote}\small\begin{tabbing}
% |s'eculo \msc{XVII}| \quad \= s'eculo \msc{XVII}\\
% |cap'itulo \msc{II}| \quad \= cap'itulo \msc{II}.
% \end{tabbing}\end{quote}
% 
% Para evitar que con un tipo que carece de versalitas acabe
% aparecendo (por substituci'on) un texto de min'usculas int�ntase
% usar nestes casos as versais \emph{reais} dun tama'no menor.
% Queda simplemente aceptable, pero � mellor que nada. (\LaTeX{}
% tende a substitu�r versalitas por versalitas, pero hai
% excepci�ns, como coas negritas.)
%
% \subsection{Miscel'anea}
%
% \begin{itemize}
% \item[$\bullet$] P�dese escribir |\'i| para |\'{\i}|.
%
% \item[$\bullet$] Hai unha abreviaci'on adicional como utilidade
% tipogr'afica m'ais que especificamente galega. En certos
% tipos, coma Times, o extremo inferior da barra est'a na li�a
% de base e expresi�ns coma <<am/pm>> resultan pouco est'eticas.
% |"/| produce unha barra que, de ser necesario, b�ixase lexeiramente.
% Computer Modern ten unha barra ben dese'nada e non � posible
% ilustrar aqu'i este punto, pero en todo caso escribir'iase |am"/pm|.^^A
% \footnote{En \MEA{141} rec�rrese a unha soluci'on que � a 'unica
% sinxela en programas de maquetaci'on: usar un corpo menor. Pero
% con \TeX{} � moito m'ais doado automatiza-las tareas.}
%
% \end{itemize}
%
% \section{Selecci'on da lingua}
% 
% Por omisi'on, \babel{} deixa <<dormidas>> as linguas ata que se
% chega a |\begin{document}| co fin de evitar conflictos polas
% abreviaci�ns; a cambio, pr�vase da posibilidade de usar as linguas
% no pre'ambulo en ordes coma |\savebox|, |\title|, |\newtheorem|, etc.
% 
% A orde |\selectgalician| permite activar |galician| coas s�as
% extensi�ns e abreviaci�ns antes de |\begin{document}|.\footnote{^^A
% Alg�ns detalles, que apenas afectan a \texttt{galician}, seguen sen
% activarse ata o comenzo do documento.}
% Deste xeito, poder'iamos dicir
%\begin{verbatim}
% \documentclass{book}
% \usepackage[T1]{fontenc}
% \usepackage[galician]{babel}
% \usepackage[latin1]{inputenc}
% \usepackage[centerlast]{caption2}
% ... % M'ais paquetes
% 
% \selectgalician
% 
% \title{T'itulo}
% \author{Autor}
% \newcommand{\pste}{para"-psicoloxicamente}
% \newsavebox{\mybox}
% \savebox{\mybox}{m'ais cosas}
% ...   % M'ais definici�ns
%
% \begin{document}
%\end{verbatim}
%
% \section{Adaptaci'on}
% \subsection{Configuraci'on}
%
% Nas s�as 'ultimas versions, \babel{} proporciona a posibilidade
% de cargar autom'aticamente un arquivo co mesmo nome que o
% principal, pero con extensi'on |.cfg|. \textsf{Galician}
% proporciona unhas poucas ordes para seren usadas neste arquivo:
% \begin{itemize}
% \item[$\diamond$] |\gl@activeacute|\\[1ex]
% Activa as abreviaci�ns con ap'ostrofos, sen que sexa
% necesario inclu�r |activeacute| coma opci'on en |\usepackage|.
%
% \item[$\diamond$] |\gl@enumerate{<leveli>}|%
%      |{<levelii>}{<leveliii>}{<leveliv>}|\\[1ex] 
% Cambia os valores preestablecidos por |galician| para
% |enumerate|. \textit{leveln} consiste nunha letra, que
% indica qu'e formato ter'a o n'umero, seguida de cualquera
% texto. A letra ten que ser: |1| (ar'abigo), |a|
% (min'uscula \emph{cursiva}\,\footnote{A letra � cursiva
% mais non os signos que a poidan seguir. M'ais ben debera
% dicirse destacada, xa que se usa |\string\emph|. V'exase
% \DTL{11}.}), |A| (versal), |i| (romano \emph{versalita},
% con |\msc|), |I| (romano versal) ou finalmente |o| (ordinal\,
% \footnote{O normal � non engadir ning'un signo tras ordinal.}).
%
% Esta orde non est'a pensada para facer cambios elaborados, se
% non solamente para meros reaxustes. Os valores preestablecidos 
% equivalen a
%\begin{verbatim}
% \gl@enumerate{1.}{a)}{1)}{a$'$)}
%\end{verbatim}
%
% \item[$\diamond$] |\gl@itemize{<leveli>}|%
%      |{<levelii>}{<leveliii>}{<leveliv>}|\\[1ex] 
% O mesmo para |itemize|, s'olo que os argumentos �sanse
% de forma literal. Os valores orixinais de \LaTeX{} son
% similares a
%\begin{verbatim}
% \gl@itemize{\textbullet}{\normalfont\bfseries\textendash}
%    {\textasteriskcentered}{\textperiodcentered}
%\end{verbatim}
%
% \item[$\diamond$] |\gl@operators|\\[1ex]
% Todo o relativo a operadores canc�lase con
%\begin{verbatim}
% \let\gl@operators\relax
%\end{verbatim}
% � boa idea inclu�lo se no se van usar, xa que aforra memoria.
%
% \end{itemize}
%
% Outros axustes 'utiles neste contexto son |\galicianoperators|,
% |\selectgalician| e |\deactivatequoting|.
%
%
% Recordemos que t�do-los cambios operados dende este arquivo
% restan compatibilidade � documento, polo que se se distrib�e
% conven adxuntarlo co entorno |filecontents|.
%
% \subsection{Outros cambios}
% 
% \begin{itemize}
% 
% \item A orde |\addto| permite
% cambiar algunha das convenci�ns internas. Esto resulta
% interesante coas traducci�ns, xa que as formas proporcionadas
% poden non ser as desexadas. Para iso � necesario que
% |galician| non estea seleccionado. Por exemplo, para cambiar
% \emph{'Indice de figuras} por \emph{Lista de figuras}:
%\begin{verbatim}
% \addto\captionsgalician{%
%   \def\listfigurename{Lista de figuras}}
%\end{verbatim}
%
% \item Para voltar elimina-la sangr'ia tra-la secci'on:
%\begin{verbatim}
% \def\@afterindentfalse{\let\if@afterindent\iffalse}
% \@afterindentfalse
%\end{verbatim}
%
% \item Para que |\roman| proporcione n'umeros romanos en
% min'uscula, segundo a forma inglesa:
%\begin{verbatim}
% \def\@roman#1{\romannumeral #1};
%\end{verbatim}
% e para que emprege |\lsc| en troques de |\msc|: 
%\begin{verbatim}
% \def\gl@roman##1{\protect\gl@lsc{\romannumeral##1}}
%\end{verbatim}
%
% \item
% Os extras at�panse organizados en varios grupos: 
% |\textgalician|, |\mathgalician|, |\shorthandsgalician|
% e |\layoutgalician|. Poden cancelarse con:
%\begin{verbatim}
%  \renewcommand\textgalician{}
%\end{verbatim}
% 
% \end{itemize}
%
% \section{Formatos distintos a \LaTeXe}
% 
% O estilo |galician| funciona con outros formatos, a�nda que con un
% subconxunto das funci�ns disponibles en \LaTeXe{}. Con Plain hai 
% que facer
%\begin{verbatim}
% \input galician.sty
%\end{verbatim}
% e con \LaTeX2.09, incluir |galician| entre as opci�ns de estilo.
% 
% Incl�ense: traducci�ns, case todas as abreviaci�ns, coma decimal,
% utilidades para a divisi'on de palabras, ordinais nunha versi'on
% simplificada (e non moi elegante), funci�ns matem'aticas,
% entrecomi�ados en \LaTeX2.09, espaciado e |\'i|.
% A selecci'on da lingua � inmediata � cargar o arquivo.
% 
% En cambio non est'an disponibles: entrecomi�ados en Plain, 
% |\lsc|, |\msc| nin as adaptaci�ns proporcionadas por |\layoutgalician|.
% 
% A partir desta versi'on, o arquivo de configuraci'on lese sempre,
% polo que aqueles que xa estean escritos espec'ificamante para \LaTeXe{}
% poden presentar problemas se se usan con outros formatos. Se as
% versi�ns que se usan non son moi antigas, p�dese comprobar o formato
% coa variable |\fmtname|, que vale |LaTeX2e| ou |plain|. Por exemplo,
%\begin{verbatim}
% \def\temp{LaTeX2e}
% \ifx\temp\fmtname
%   ...
% \fi
%\end{verbatim}
% 
% \section{Bibliograf'ias}
% 
% O arquivo |glbst.tex| que se xenera con \textsf{galician}
% serve para que a utilidade \textsf{custom-bib} traballe en
% conxunci'on con \babel. Define unha serie de ordes, que poden
% consultarse no propio arquivo, que se utilizan para as
% traducci�ns se se selecciona |babel| coma lingua � xenerar un
% estilo bibliogr'afico. 
% 
% \section{Incompatibilidades con versi�ns anteriores}
% 
% ^^A\begin{itemize}
% 
% O t'ermino correspondente a |\tablename| estaba traducido
% incorrectamente coma <<T�boa>>. Coma queira que <<t�boa>> � a
% palabra con que pode aparecer no propio texto, ou ben pode
% haber un artigo feminino ante |\tablename|, pode reponerse o
% valor antigo con:
%\begin{verbatim}
% \addto\captionsgalician{%   
%   \def\tablename{T�boa}%
%   \def\listtablename{\'Indice de t�boas}}
%\end{verbatim}
%
% ^^A\end{itemize}
%
% \section*{Referencias}
% \addcontentsline{toc}{section}{Referencias}
%
% \begingroup
% \small
% \leftskip1.5cm \parindent-1.5cm
%
% \makebox[1.5cm][l]{\msc{DGII}}\textit{Diccionario de galego},
% Ir Indo. Vigo, 2004.
%
% \makebox[1.5cm][l]{\msc{DOT}}Jos'e Mart'inez de Sousa,
%   \textit{Diccionario de ortograf'ia t'ecnica}, 
%   Germ'an S'anchez Ruip'erez/Pir'amide. Madrid, 1987.
%   (Biblioteca del libro.)
%
% \makebox[1.5cm][l]{\msc{DTL}}Jos'e Mart'inez de Sousa,
%   \textit{Diccionario de tipograf'ia y del libro}, 
%   Paraninfo. Madrid, 3"a ed., 1992.
%
% \makebox[1.5cm][l]{\msc{MEA}}Jos'e Mart'inez de Sousa,
%   \textit{Manual de edici'on y autoedici'on}, 
%   Pir'amide. Madrid 1994.
%
% \makebox[1.5cm][l]{\msc{OELG}}Xos'e Feix'o Cid,
%   \textit{Ortograf�a e estilo da lingua galega}, 
%   Pir'amide. Vigo, 2002.
%
% \leftskip0pt \parindent0pt \vspace{6pt}
%
% {\itshape
% Como normalmente o primeiro contacto con \TeX{} � por unha tesis,
% cito libros que est'an relacionados co tema �s que tiven acceso.
% Est'an por orde de preferencia; en  particular, os dous 'ultimos
% par�cenme pouco recomendables. (J. B.)}
% 
% \parindent-1.5pc \leftskip1.5pc  \vspace{3pt}
%
%   Umberto Eco,
%   \textit{C'omo se hace una tesis}, Gedisa.
%   Barcelona 1982.
%
%   Antonia Rigo Arnavat e Gabriel Genesc\`a Due'nas,
%   \textit{C'omo presentar una tesis y trabajos de investigaci'on},
%   Eumo-Octaedro. Barcelona, 2002.
%
%   Prudenci Comes,
%   \textit{Gu'ia para la redacci'on y presentaci'on de trabajos
%   cient'ificos, informes t'ecnicos y tesinas}, 
%   Oikos-Tau. Barcelona, 1971.
%
%   Javier Lasso de la Vega,
%   \textit{C'omo se hace una tesis doctoral}, 
%   Fundacion Universitaria Espa~nola. Madrid, 1977.
%
%   Jos'e Romera Castillo e outros,
%   \textit{Manual de estilo}, 
%   Universidad Nacional de Educaci'on a Distancia. Madrid, 1996.
%
%   Restituto Sierra Bravo,
%   \textit{Tesis doctorales y trabajos de investigaci'on cient'ifica}, 
%   Paraninfo. Madrid, 1986.
%
% \parindent0pc \leftskip0pc \vspace{6pt}
%
% {\itshape
% Para outras cuesti�ns tipogr'aficas, as referencias
% usadas son, entre outras:}
%
% \parindent-1.5pc \leftskip1.5pc \vspace{3pt}
% 
%   Javier Bezos,
%   \textit{Tipograf'ia espa'nola con \TeX}, documento electr'onico
%   disponible en 
%   \textsf{http://perso.wanadoo.es/jbezos/tipografia.html}. 
%
%   Ra'ul Cabanes Mart'inez,
%   <<El sistema internacional de unidades: ese desconocido>>,
%   \textit{Mundo Electr'onico}, n"o 142, p'axs.~119~-125. 1984.
%
%   \textit{The Chicago Manual of Style}, University of Chicago
%   Press, 14"a~ed., esp.~p'axs.~333~-335. Chicago, 1993.
%
%   Jos'e Fern'andez Castillo,
%   \textit{Normas para correctores y compositores tip'ografos},
%   Espasa-Calpe. Madrid, 1959.
%
%   IRANOR [AENOR], Normas \msc{UNE} n'umeros 5010 (<<Signos 
%   matem'aticos>>), 5028 (<<S'imbolos 
%   geom'etricos>>) e 5029 (<<Impresi'on de los
%   s'imbolos de magnitudes y unidades y de los n'umeros>>). 
%   [Obsoletas.]
%
%   Real Academia Espa'nola,
%   \textit{Esbozo de una nueva gram'atica de la
%   lengua espa'nola}, Espasa-Calpe. Madrid, 1973.
%   
%   V.\ Mart'inez Sicluna,
%   \textit{Teor'ia y pr'actica de la tipograf'ia},
%   Gustavo Gili. Barcelona, 1945.
%
%   Jos'e Mart'inez de Sousa,
%   \textit{Diccionario de ortograf'ia de la lengua espa'nola},
%   Paraninfo. Madrid, 1996.
%
%   Juan Mart'inez Val, \textit{Tipograf'ia pr'actica},
%   Laberinto. Madrid, 2002.
%
%   Juan Jos'e Morato, \textit{Gu'ia pr'actica del compositor 
%   tipogr'afico}, Hernando, 2"a ed. Madrid, 1908 (1"a ed., 1900,
%   3"a ed., 1933).
%
%   Marion Neubauer,
%   <<Feinheiten bei wissenschaftlichen Publikationen>>,
%   \textit{Die \TeX nisches Kom\"odie},  parte I, vol. 8, n"o 4,
%   p'axs. 23-40. 1996; parte II, vol. 9, n"o 1, p'axs.~25~-44. 1997.
%
%   Jos'e Polo,
%   \textit{Ortograf'ia y ciencia del lenguaje},  Paraninfo. Madrid, 1974.
%
%   Pedro Valle,
%   \textit{C'omo corregir sin ofender}, Lumen. Buenos Aires, 1998.
%
%   Hugh C. Wolfe, <<S'imbolos, unidades y nomenclatura>>, 
%   \textit{Enciclopedia de F'isica}, t.~2, p'axs.~1423~-1451.
%    dir. Rita G. Lerner e George L. Trigg, Alianza. Madrid, 1987,
%
% \endgroup
%
% \end{userdtx}
% 
% \begin{userdrv}
%^^A  ======= Beginning of text as typeset by user.drv =========
%
% \GetFileInfo{galician.dtx}
%
% \section{This file}
%
% This file defines all the language-specific macros for the
% Galician language. The file galician.dtx was translated in
% January 2007 by Javier A. M\'ugica from spanish.dtx. It was 
% given the version number 4.3, based on the version for spanish.dtx
% at those times, that was 4.2b. The original author from
% v4.0 to 4.2b was Javier Bezos. Previous versions were written
% by Julio S\'anchez.

% I decided to make \emph{tabula rasa} of all |\changes|
% logs. Only changes from spanish 4.2b to galician 4.3 and thereafter
% are documented. The change history for the original spanish.dtx
% can be found in that file.
% 
% \section{The Galcian language}
%
% Custumization is made following mainly the books on the subject
% by Jos\'e Mart\'\i nez de Sousa and Xos\'e Feix\'o Cid.
% By typesetting |galician.dtx| directly you will get the full
% documentation (regrettably is in Galician only, but it is
% pretty long). References in this part refers to that document.
% There are several aditional features documented in the Galician
% version only.
% 
% This style provides:
% \begin{itemize}
% \item Translations following the International \LaTeX{} 
% conventions, as well as |\today|.
% 
% \item Shorthands listed in Table~\ref{tab:galician-quote-def}.
% Examples in subsection~3.4 are illustrative. Note that
% |"~| has a special meaning in \textsf{galician}
% different to other languages, and is used mainly in linguistic
% contexts.
% 
% \begin{table}[htb]
% \centering
% \begin{tabular}{lp{8cm}}
% |'a|      & acute accented a. Also for: e, i, o, u (both
%             lowercase and uppercase).\\
% |'n|      & \~n (also uppercase).\\
% |~n|      & \~n (also uppercase). Deprecated.\\
% |"u|      & \"u (also uppercase).\\
% |"i|      & \"i (also uppercase).\\
% |"a|      & Ordinal numbers (also  |"A|, |"o|, |"O|).\\
% |"rr|     & rr, but -r when hyphenated\\
% |"-|      & Like |\-|, but allowing hyphenation in the rest 
%             the word.\\
% |"=|      & Like |-|, but allowing hyphenation in the rest 
%             the word.\\
% |"~|      & The hyphen is repeated at the very beginning of 
%             the next line if the word is hyphenated at this
%             point.\\
% |""|      & Like |"-| but producing no hyphen sign.\\
% |~-|      & Like |-| but with no break after the hyphen. Also for:
%             en-dashes (|~--|) and em-dashes (|~---|). \\
% |"/|      & A slash slightly lowered, if necessary.\\
% \verb+"|+ & disable ligatures at this point.\\
% |<<|      & Left guillemets.\\
% |>>|      & Right guillemets.\\
% |"<|      & |\begin{quoting}|. (See text.)\\
% |">|      & |\end{quoting}|. (See text.)
% \end{tabular}
% \caption{Extra definitions made by file \file{galician.ldf}}
% \label{tab:galician-quote-def}
% \end{table}
% 
% \item |\deactivatetilden| deactivates the |~n| and |~N| shorthands.
% 
% \item \emph{In math mode} a dot followed by a digit is replaced
% by a decimal comma.
% 
% \item Galicians ordinals and abbeviations with |\sptext|
% as, for instance, |1\sptext{o}|. The
% preceptive dot is included.
% 
% \item Accented functions: l\'\i m, m\'ax, m\'\i n, m\'od. You may
% globally omit the accents with |\unaccentedoperators|. Spaced
% functions: arc\,cos, etc. You may globally kill that space with
% |\unspacedoperators|. |\dotlessi| is provided for use in math mode.
% 
% \item A |quoting| environment and a related pair of shorthands |<<|
% and |>>|. The command
% |\deactivatequoting| deactivates these shorthand in case 
% you want to use |<| and |>| in some AMS commands and numerical
% comparisons.
% 
% \item The command |\selectgalician| selects the |galician| language
% \emph{and} its shorthands. (Intended for the preamble.)
% 
% \item |\frenchspacing| is used.
% 
% \item |\dots| is redefined. It is now equal to typing tree points
% in a row (it preserves the space following).
% 
% \item There is a small space before |\%|.
% 
% \item |\msc| provides lowercase small caps. (See subsection~3.10.)
% \end{itemize}
% 
% Just in case |galician| is the main language, the group 
% |\layoutgalician| is activated, which modifies the standard
% classes through the whole document (it cannot be deactivated)
% in the following way:
% \begin{itemize}
% \item Both |enumerate| and |itemize| are adapted to Galician rules.
% 
% \item Both |\alph| and |\Alph| include \textit{\~n} after \textit{n}.
% 
% \item Symbol footmarks are one, two, three, etc., asteriscs.
% 
% \item |OT1| guillemets are generated with two |lasy| symbols instead
% of small |\ll| and |\gg|.
% 
% \item |\roman| is redefined to write small caps roman numerals, since
% lowercase roman numerals are not allowed. However, \textit{MakeIndex}
% rejects entries containing pages in that format. The |.idx| file must
% be preprocessed if the document has this kind of entries with
% the provided |romanidx.tex| tool---just \TeX{} it and follow the
% instructions.
% 
% \item There is a dot after section numbers in titles and toc.
% \end{itemize}
% This group is ignored if you write |\selectgalician*| in the
% preamble.
% 
% Some additional commands are provided to be used in the
% |galician.cfg| file:
% \begin{itemize}
% \item With |\gl@activeacute| acute accents are always active,
% overriding the default \textsf{babel} behaviour.
% 
% \item |\gl@enumerate| sets the labels to be used by |enumerate|. The
% same applies to |\gl@itemize| and |itemize|.
% 
% \item |\gl@operators| stores the operator commands.
% All of them are canceled with
%\begin{verbatim}
% \let\gl@operators\relax
%\end{verbatim}
% \end{itemize}
% The commands |\deactivatequoting|, |\deactivatetilden| and
% |\selectgalician| may be used in this file, too.
% 
% A subset of these commands is provided for
% use in Plain \TeX{} (with |\input galician.sty|).
%
% \end{userdrv}
% 
%\StopEventually{}
%
%^^A ========== End of manual ===============
%
% \begin{userdtx}
% \section{The Code}
% \end{userdtx}
% 
% \begin{userdrv}
% \subsection{The Code}
% \end{userdrv}
% 
% \changes{galician~4.3}{07/01/22}{Reversed "<, "> and <<, >>.}
% \changes{galician~4.3}{07/01/22}{Removed the Spanish et sign.}
% \changes{galician~4.3}{07/01/26}{Removed the shorthand for \c{c}. It
% existed in medieval galician-portuguese, but that does not seem a
% reason to be included here.}
%
%    This file provides definition for both \LaTeXe{} and non
%    \LaTeXe{} formats.
%
%    Identify the |ldf| file.
%    
%    \begin{macrocode}
%<*code>
\ProvidesLanguage{galician.ldf}
       [2007/01/29 v4.3 Galician support from the babel system]
%    \end{macrocode}
%
%    The macro |\LdfInit| takes care of preventing that this file is
%    loaded more than once, checking the category code of the
%    \texttt{@} sign, etc.
%    When this file is read as an option, i.e. by the |\usepackage|
%    command, \texttt{galician} will be an `unknown' language in which
%    case we have to make it known.  So we check for the existence of
%    |\l@galician| to see whether we have to do something here.
%
%    \begin{macrocode}
\LdfInit{galician}\captionsgalician
\ifx\undefined\l@galician
  \@nopatterns{Galician}
  \adddialect\l@galician0
\fi
%    \end{macrocode}
%
%    We define some tools which will be used in that style file:
%    (1) we make sure that |~| is active, (2) |\gl@delayed| delays
%    the expansion of the code in conditionals (in fact, quite similar
%    to |\bbl@afterfi|).
%
%    \begin{macrocode}
\edef\gl@savedcatcodes{%
  \catcode`\noexpand\~=\the\catcode`\~
  \catcode`\noexpand\"=\the\catcode`\"}
\catcode`\~=\active
\catcode`\"=12
\long\def\gl@delayed#1\then#2\else#3\fi{%
  #1%
    \expandafter\@firstoftwo
  \else
    \expandafter\@secondoftwo
  \fi
  {#2}{#3}}
%    \end{macrocode}
%
%    Two tests are introduced. The first one tells us if the format is
%    \LaTeXe{}, and the second one if the format is Plain or any other.
%    If both are false, the format is \LaTeX2.09{}.
%
%    \begin{macrocode}
\gl@delayed
\expandafter\ifx\csname documentclass\endcsname\relax\then
  \let\ifes@LaTeXe\iffalse
\else
  \let\ifes@LaTeXe\iftrue
\fi
\gl@delayed
\expandafter\ifx\csname newenvironment\endcsname\relax\then
  \let\ifes@plain\iftrue
\else
  \let\ifes@plain\iffalse
\fi
%    \end{macrocode}
%
% Translations for captions.
%
%    \begin{macrocode}
\addto\captionsgalician{%
  \def\prefacename{Prefacio}%
  \def\refname{Referencias}%
  \def\abstractname{Resumo}%
  \def\bibname{Bibliograf\'{\i}a}%
  \def\chaptername{Cap\'{\i}tulo}%
  \def\appendixname{Ap\'endice}%
  \def\listfigurename{\'Indice de figuras}%
  \def\listtablename{\'Indice de cadros}%
  \def\indexname{\'Indice alfab\'etico}%
  \def\figurename{Figura}%
  \def\tablename{Cadro}%
  \def\partname{Parte}%
  \def\enclname{Adxunto}%
  \def\ccname{Copia a}%
  \def\headtoname{A}%
  \def\pagename{P\'axina}%
  \def\seename{v\'exase}%
  \def\alsoname{v\'exase tam\'en}%
  \def\proofname{Demostraci\'on}%
  \def\glossaryname{Glosario}}
  
\expandafter\ifx\csname chapter\endcsname\relax
  \addto\captionsgalician{\def\contentsname{\'Indice}}
\else
  \addto\captionsgalician{\def\contentsname{\'Indice xeral}}
\fi
%    \end{macrocode}
%
%    And the date.
% \changes{galician~4.3}{07/01/21}{Set the default to \textit{do}}
%    \begin{macrocode}
\def\dategalician{%
 \def\today{\the\day~de \ifcase\month\or xaneiro\or febreiro\or
      marzo\or abril\or maio\or xu\~no\or xullo\or agosto\or
      setembro\or outubro\or novembro\or decembro\fi 
      \ \ifnum\year>1999\gl@yearl\else de\fi~\the\year}}
\def\galiciandatedo{\def\gl@yearl{do}}
\def\galiciandatede{\def\gl@yearl{de}}
\galiciandatedo
%    \end{macrocode}
%
%    The basic macros to select the language, in the preamble or the
%    config file. Use of |\selectlanguage| should be avoided at this
%    early stage because the active chars are not yet
%    active. |\selectgalician| makes them active.
%
%    \begin{macrocode}
\def\selectgalician{%
  \def\selectgalician{%
    \def\selectgalician{%
      \PackageWarning{galician}{Extra \string\selectgalician ignored}}%
    \gl@select}}

\@onlypreamble\selectgalician

\def\gl@select{%
  \let\gl@select\@undefined
  \selectlanguage{galician}%
  \catcode`\"\active\catcode`\~=\active}
%    \end{macrocode}
%
%    Instead of joining all the extras directly in |\extrasgalician|,
%    we subdivide them in three further groups.
%    \begin{macrocode}
\def\extrasgalician{%
  \textgalician
  \mathgalician
  \ifx\shorthandsgalician\@empty
    \galiciandeactivate{."'~<>}%
    \languageshorthands{none}%
  \else
    \shorthandsgalician
  \fi}
\def\noextrasgalician{%
  \ifx\textgalician\@empty\else
    \notextgalician
  \fi
  \ifx\mathgalician\@empty\else
    \nomathgalician
  \fi
  \ifx\shorthandsgalician\@empty\else
    \noshorthandsgalician
  \fi
  \gl@reviveshorthands}
%    \end{macrocode}
%
%    And the first of these sub-groups is defined.
%
%    \begin{macrocode}
\addto\textgalician{%
  \babel@save\sptext
  \def\sptext{\protect\gl@sptext}}    
%    \end{macrocode}
%
%    The definition of |\sptext| is more elaborated than that of
%    |\textsuperscript|. With uppercase superscript text
%    the scriptscriptsize is used. The mandatory dot is already
%    included. There are two versions, depending on the
%    format.
%
%    \begin{macrocode}
\ifes@LaTeXe   %<<<<<<
  \newcommand\gl@sptext[1]{%
    {.\setbox\z@\hbox{8}\dimen@\ht\z@
     \csname S@\f@size\endcsname
     \edef\@tempa{\def\noexpand\@tempc{#1}%
       \lowercase{\def\noexpand\@tempb{#1}}}\@tempa
     \ifx\@tempb\@tempc
       \fontsize\sf@size\z@
       \selectfont
       \advance\dimen@-1.15ex
     \else
       \fontsize\ssf@size\z@
       \selectfont
       \advance\dimen@-1.5ex
     \fi
     \math@fontsfalse\raise\dimen@\hbox{#1}}}
\else          %<<<<<<
  \let\sptextfont\rm
  \newcommand\gl@sptext[1]{%
    {.\setbox\z@\hbox{8}\dimen@\ht\z@
     \edef\@tempa{\def\noexpand\@tempc{#1}%
       \lowercase{\def\noexpand\@tempb{#1}}}\@tempa
     \ifx\@tempb\@tempc
       \advance\dimen@-0.75ex
       \raise\dimen@\hbox{$\scriptstyle\sptextfont#1$}%
     \else
       \advance\dimen@-0.8ex
       \raise\dimen@\hbox{$\scriptscriptstyle\sptextfont#1$}%
     \fi}}
\fi            %<<<<<<
%    \end{macrocode}
%
%    Now, lowercase small caps. First, we test if there are actual
%    small caps for the current font. If not, faked small caps are
%    used. \msc tries a slightly larger font.
%    Javier B. wrote: ``The \cs{selectfont} in \cs{gl@lsc} could
%    seem redundant, but it's not''. I cannot see how it can't be
%    redundant (it is the last thing executed by |\scshape|), but
%    I keep it.
%
%    \changes{galician~4.3}{07/00/26}{Added \cs{\msc}}
%    \begin{macrocode}
\ifes@LaTeXe   %<<<<<<
  \addto\textgalician{%
    \babel@save\lsc
    \def\lsc{\protect\gl@lsc}
    \babel@save\msc
    \def\msc{\protect\gl@msc}}
        
        \def\gl@@msc{\expandafter\@tempdima\f@size pt \divide\@tempdima by 200 \multiply\@tempdima by 219 
                \edef\f@size{\strip@pt\@tempdima}\selectfont}
        \def\gl@msc{\let\gl@do@msc\gl@@msc\lsc}
        \let\gl@do@msc\relax

  \def\gl@lsc#1{%
    \leavevmode
    \hbox{\gl@do@msc\scshape\selectfont
       \expandafter\ifx\csname\f@encoding/\f@family/\f@series
           /n/\f@size\expandafter\endcsname
         \csname\curr@fontshape/\f@size\endcsname
         \csname S@\f@size\endcsname
         \fontsize\sf@size\z@\selectfont
           \PackageInfo{galician}{Replacing undefined sc font\MessageBreak
                                 shape by faked small caps}%
         \MakeUppercase{#1}%
       \else
         \MakeLowercase{#1}%
       \fi}\let\gl@do@msc\relax}
\fi            %<<<<<<
%    \end{macrocode}
%
%    The |quoting| environment. This part is not available
%    in Plain, hence the test. Overriding the default |\everypar| is
%    a bit tricky.
%
%    \begin{macrocode}
\newif\ifgl@listquot

\ifes@plain\else %<<<<<<
  \csname newtoks\endcsname\gl@quottoks
  \csname newcount\endcsname\gl@quotdepth

        \ifx\quoting\c@undefined\def\next{\let\next\relax\newenvironment}
        \else\def\next{\PackageInfo{galician}{Redefining quoting}\let\next\relax\renewenvironment}
        \fi
  \next{quoting}
   {\leavevmode
    \advance\gl@quotdepth1
    \csname lquot\romannumeral\gl@quotdepth\endcsname%
    \ifnum\gl@quotdepth=\@ne
      \gl@listquotfalse
      \let\gl@quotpar\everypar
      \let\everypar\gl@quottoks
      \everypar\expandafter{\the\gl@quotpar}%
      \gl@quotpar{\the\everypar
        \ifgl@listquot\global\gl@listquotfalse\else\gl@quotcont\fi}%
    \fi
    \toks@\expandafter{\gl@quotcont}%
    \edef\gl@quotcont{\the\toks@
       \expandafter\noexpand
       \csname rquot\romannumeral\gl@quotdepth\endcsname}}
   {\csname rquot\romannumeral\gl@quotdepth\endcsname}

  \def\lquoti{\guillemotleft{}}
  \def\rquoti{\guillemotright{}}
  \def\lquotii{``}
  \def\rquotii{''}
  \def\lquotiii{`}
  \def\rquotiii{'}

  \let\gl@quotcont\@empty
%    \end{macrocode}
%
%    If there is a margin par inside quoting, we don't add the
%    quotes. |\gl@listqout| stores the quotes to be used before
%    item labels; otherwise they could appear after the labels.
%
%    \begin{macrocode}  
  \addto\@marginparreset{\let\gl@quotcont\@empty}

  \def\gl@listquot{%
    \csname rquot\romannumeral\gl@quotdepth\endcsname
    \global\gl@listquottrue}
\fi            %<<<<<<
%    \end{macrocode}
%
%    Now, the |\frenchspacing|, followed by |\...dots| and |\%|
%                Instead of redefining |\ldots| and |\cdots|, we redefine |\ldotp|
%    and |\cdotp|, so that this is compatible with amsmath.
%    In LaTeX we also redefine |\textellipsis|, and for plain or
%    other we redefine |\dots|.
%
% \changes{galician~4.3}{07/01/27}{\cs{...} is removed and instead
% \cs{dots} and \cs{dots} are changed, by redefining \cs{ldotc},
% \cs{dotc} and \cs{textellipsis} or \cs{dots}}
%    \begin{macrocode}
\addto\textgalician{\bbl@frenchspacing}
\addto\notextgalician{\bbl@nonfrenchspacing}

\mathchardef\gl@cdot="0201
\ifes@LaTeXe            %<<<<<<
\addto\textgalician{%
  \babel@save\textellipsis
  \babel@save\ldotp
  \babel@save\cdotp%
        \def\textellipsis{\hbox{...}\spacefactor\sfcode`.{} }%
        \mathchardef\ldotp="013A%
        \mathchardef\cdotp="0201%
}
\else            %<<<<<<
\addto\textgalician{%
  \babel@save\dots
  \babel@save\ldotp
  \babel@save\cdotp
  \mathchardef\ldotp="013A%
        \mathchardef\cdotp="0201%
        \def\dots{\ifmmode\ldots\else...\spacefactor\sfcode`.{} \fi}%
}
\fi            %<<<<<<

\ifes@LaTeXe   %<<<<<<  
  \addto\textgalician{%
    \let\percentsign\%%
    \babel@save\%%
    \def\%{\unskip\,\percentsign{}}}
\else
  \addto\textgalician{%
    \let\percentsign\%%
    \babel@save\%%
    \def\%{\unskip\ifmmode\,\else$\m@th\,$\fi\percentsign{}}}
\fi
%    \end{macrocode}
%    
%    We follow with the math group. It's not easy to add an accent
%    in an operator. The difficulty is that we must avoid using
%    text (that is, |\mbox|) because we have no control on font
%    and size, and at time we should access |\i|, which is a text
%    command forbidden in math mode. |\dotlessi| must be
%    converted to uppercase if necessary in \LaTeXe. There are
%    two versions, depending on the format.
%
%    \begin{macrocode}
\addto\mathgalician{%
  \babel@save\dotlessi
  \def\dotlessi{\protect\gl@dotlessi}}

\let\nomathgalician\relax %% Unused, but called

\ifes@LaTeXe   %<<<<<< 
  \def\gl@texti{\i}
  \addto\@uclclist{\dotlessi\gl@texti}
\fi            %<<<<<<

\ifes@LaTeXe   %<<<<<<
  \def\gl@dotlessi{%
    \ifmmode
      {\ifnum\mathgroup=\m@ne
         \imath
       \else
         \count@\escapechar \escapechar=\m@ne
         \expandafter\expandafter\expandafter
           \split@name\expandafter\string\the\textfont\mathgroup\@nil
         \escapechar=\count@
         \@ifundefined{\f@encoding\string\i}%
           {\edef\f@encoding{\string?}}{}%
         \expandafter\count@\the\csname\f@encoding\string\i\endcsname
         \advance\count@"7000
         \mathchar\count@
       \fi}%
    \else
      \i
    \fi}
\else          %<<<<<<
  \def\gl@dotlessi{%
    \ifmmode
      \mathchar"7010
    \else
      \i
    \fi}
\fi            %<<<<<<
%    \end{macrocode}
%
%   The switches for accents and spaces in math.
%
% \changes{galician~4.3}{07/01/21}{Set the default to \cs{unspacedoperators}}
%    \begin{macrocode}
\def\accentedoperators{%
  \def\gl@op@ac##1{\acute{##1}}%
  \def\gl@op@i{\acute{\dotlessi}}}
\def\unaccentedoperators{%
  \def\gl@op@ac##1{##1}%
  \def\gl@op@i{i}}
\accentedoperators

\def\spacedoperators{\let\gl@op@sp\,}
\def\unspacedoperators{\let\gl@op@sp\@empty}
\unspacedoperators
%    \end{macrocode}
%
%    The operators are stored in |\gl@operators|, which in turn is
%    included in the math group. Since |\operator@font| is
%    defined in \LaTeXe{} only, we need to define them in the plain variant.
%    
% \changes{galician~4.3}{07/01/21}{\cs{sin}, \cs{arcsin} and \cs{sinh}
% are set to produce the same as \cs{sen}, \cs{arcsen} and \cs{senh}}
% \changes{galician~4.3}{07/01/27}{cosec and senh moved from
% \cs{galicianoperators} to the main group}
%
%    \begin{macrocode}
\addto\mathgalician{%
  \gl@operators}

\ifes@LaTeXe\else %<<<<<<
  \let\operator@font\rm
  \def\@empty{}
\fi            %<<<<<<

\def\gl@operators{%
  \babel@save\lim        \def\lim{\mathop{\operator@font l\protect\gl@op@i m}}%
  \babel@save\limsup  \def\limsup{\mathop{\operator@font l\gl@op@i m\,sup}}%
  \babel@save\liminf  \def\liminf{\mathop{\operator@font l\gl@op@i m\,inf}}%
  \babel@save\max     \def\max{\mathop{\operator@font m\gl@op@ac ax}}%
  \babel@save\inf     \def\inf{\mathop{\operator@font \protect\gl@op@i nf}}%
  \babel@save\min     \def\min{\mathop{\operator@font m\protect\gl@op@i n}}%
  \babel@save\bmod
  \def\bmod{%
    \nonscript\mskip-\medmuskip\mkern5mu%
    \mathbin{\operator@font m\gl@op@ac od}\penalty900\mkern5mu%
    \nonscript\mskip-\medmuskip}%
  \babel@save\pmod
  \def\pmod##1{%
    \allowbreak\mkern18mu({\operator@font m\gl@op@ac od}\,\,##1)}%
  \def\gl@a##1 {%
    \gl@delayed
    \if^##1^\then  %  is it empty? do nothing and continue
      \gl@a
    \else
      \gl@delayed
      \if&##1\then % is it &? do nothing and finish
      \else
        \begingroup
          \let\,\@empty % \, is ignored when def'ing the macro name
          \let\acute\@firstofone % same
          \edef\gl@b{\expandafter\noexpand\csname##1\endcsname}%
          \def\,{\noexpand\gl@op@sp}%
          \def\acute####1{%
            \if i####1%
              \noexpand\gl@op@i
            \else
              \noexpand\gl@op@ac####1%
            \fi}%
          \edef\gl@a{\endgroup
            \noexpand\babel@save\expandafter\noexpand\gl@b
            \def\expandafter\noexpand\gl@b{%
                    \mathop{\noexpand\operator@font##1}\nolimits}}%
          \gl@a % It restores itself
        \gl@a
      \fi
    \fi}%
  \let\gl@b\galicianoperators
  \addto\gl@b{ }%
  \expandafter\gl@a\gl@b sen tx cosec arc\,sen arc\,cos arc\,tx senh & %\, will be set to \gl@op@sp
  %
  \babel@save\sin    \let\sin\sen
        \babel@save\arcsin \let\arcsin\arcsen
        \babel@save\sinh   \let\sinh\senh
}

\def\galicianoperators{cotx txh}
%    \end{macrocode}
%
%    Now comes the text shorthands. They are grouped in
%    |\shorthandsgalician| and this style performs some
%    operations before the babel shortands are called.
%    The goals are to allow espression like |$a^{x'}$|
%    and to deactivate the shorthands making them of
%    category `other'. After providing a |\'i| shorthand,
%    the new macros are defined.
%    
%    \begin{macrocode}
\DeclareTextCompositeCommand{\'}{OT1}{i}{\@tabacckludge'{\i}}

\def\gl@set@shorthand#1{%
  \expandafter\edef\csname gl@savecat\string#1\endcsname
     {\the\catcode`#1}%
  \initiate@active@char{#1}%
  \catcode`#1=\csname gl@savecat\string#1\endcsname\relax
  \expandafter\let\csname gl@math\string#1\expandafter\endcsname
    \csname normal@char\string#1\endcsname}

\def\gl@use@shorthand{%
  \gl@delayed
  \ifx\thepage\relax\then
    \string
  \else{%
    \gl@delayed
    \ifx\protect\@unexpandable@protect\then
      \noexpand
    \else
      \gl@use@sh
    \fi}%
  \fi}

\def\gl@text@sh#1{\csname active@char\string#1\endcsname}
\def\gl@math@sh#1{\csname gl@math\string#1\endcsname}

\def\gl@use@sh{%
  \gl@delayed
  \if@safe@actives\then
    \string
  \else{%
    \gl@delayed
    \ifmmode\then
      \gl@math@sh
    \else
      \gl@text@sh
    \fi}%
  \fi}

\gdef\gl@activate#1{%
  \begingroup
    \lccode`\~=`#1
    \lowercase{%
  \endgroup
  \def~{\gl@use@shorthand~}}}

\def\galiciandeactivate#1{%
  \@tfor\@tempa:=#1\do{\expandafter\gl@spdeactivate\@tempa}}

\def\gl@spdeactivate#1{%
  \if.#1%
    \mathcode`\.=\gl@period@code
  \else
    \begingroup
      \lccode`\~=`#1
      \lowercase{%
    \endgroup
    \expandafter\let\expandafter~%
      \csname normal@char\string#1\endcsname}%
    \catcode`#1\csname gl@savecat\string#1\endcsname\relax
  \fi}

\def\gl@reviveshorthands{%
  \gl@restore{"}\gl@restore{~}%
  \gl@restore{<}\gl@restore{>}%
  \gl@quoting}

\def\gl@restore#1{%
  \catcode`#1=\active
  \begingroup
    \lccode`\~=`#1
    \lowercase{%
  \endgroup
  \bbl@deactivate{~}}}
%    \end{macrocode}
%
%    But \textsf{galician} allows two category codes for |'|,
%    so both should be taken into account in \cs{bbl@pr@m@s}.
%    
%    \begin{macrocode}
\begingroup
\catcode`\'=12
\lccode`~=`' \lccode`'=`'
\lowercase{%
\gdef\bbl@pr@m@s{%
  \gl@delayed
  \ifx~\@let@token\then
    \pr@@@s
  \else
    {\gl@delayed
     \ifx'\@let@token\then
       \pr@@@s
     \else
       {\gl@delayed
        \ifx^\@let@token\then
          \pr@@@t
        \else
          \egroup
        \fi}%
     \fi}%
  \fi}}
\endgroup
%    \end{macrocode}
%
%    \begin{macrocode}
\expandafter\ifx\csname @tabacckludge\endcsname\relax
  \let\gl@tak\a
\else
  \let\gl@tak\@tabacckludge
\fi

\ifes@LaTeXe   %<<<<<<
  \def\@tabacckludge#1{\expandafter\gl@tak\string#1}
  \let\a\@tabacckludge
\else\ifes@plain %<<<<<<
  \def\@tabacckludge#1{\csname\string#1\endcsname}
\else          %<<<<<<
  \def\@tabacckludge#1{\csname a\string#1\endcsname}
\fi\fi         %<<<<<<    

\expandafter\ifx\csname add@accent\endcsname\relax
  \def\add@accent#1#2{\accent#1 #2}
\fi
%    \end{macrocode}
%
%    Instead of redefining |\'|, we redefine the internal macro for the OT1 encoding.
%    \begin{macrocode}
\ifes@LaTeXe   %<<<<<<
  \def\gl@accent#1#2#3{%
    \expandafter\@text@composite
    \csname OT1\string#1\endcsname#3\@empty\@text@composite
    {\bbl@allowhyphens\add@accent{#2}{#3}\bbl@allowhyphens
     \setbox\@tempboxa\hbox{#3%
       \global\mathchardef\accent@spacefactor\spacefactor}%
     \spacefactor\accent@spacefactor}}
\else          %<<<<<<
  \def\gl@accent#1#2#3{%
    \bbl@allowhyphens\add@accent{#2}{#3}\bbl@allowhyphens
    \spacefactor\sfcode`#3 }
\fi            %<<<<<<
%    \end{macrocode}
%
%    The shorthands are activated in the aux file. Now, we begin
%    the shorthands group.
%
%    \begin{macrocode}
\addto\shorthandsgalician{\languageshorthands{galician}}
\let\noshorthandsgalician\relax
%    \end{macrocode}
%
%    First, decimal comma.
%
%    \begin{macrocode}
\def\galiciandecimal#1{\def\gl@decimal{{#1}}}
\def\decimalcomma{\galiciandecimal{,}}
\def\decimalpoint{\galiciandecimal{.}}
\decimalcomma

\gl@set@shorthand{.}

\@namedef{gl@math\string.}{%
  \@ifnextchar\egroup
    {\mathchar\gl@period@code\relax}%
    {\gl@text@sh.}}

\declare@shorthand{system}{.}{\mathchar\gl@period@code\relax}
%    \end{macrocode}
%
%    \begin{macrocode}
\addto\shorthandsgalician{%
  \mathchardef\gl@period@code\the\mathcode`\.%
  \babel@savevariable{\mathcode`\.}%
  \mathcode`\.="8000 %
  \gl@activate{.}}

\AtBeginDocument{%
  \catcode`\.=12
  \if@filesw
    \immediate\write\@mainaux{%
    \string\catcode`\string\.=12}%
  \fi}

\declare@shorthand{galician}{.1}{\gl@decimal1}
\declare@shorthand{galician}{.2}{\gl@decimal2}
\declare@shorthand{galician}{.3}{\gl@decimal3}
\declare@shorthand{galician}{.4}{\gl@decimal4}
\declare@shorthand{galician}{.5}{\gl@decimal5}
\declare@shorthand{galician}{.6}{\gl@decimal6}
\declare@shorthand{galician}{.7}{\gl@decimal7}
\declare@shorthand{galician}{.8}{\gl@decimal8}
\declare@shorthand{galician}{.9}{\gl@decimal9}
\declare@shorthand{galician}{.0}{\gl@decimal0}
%    \end{macrocode}
%
%     Now accents and tools
%
%    \begin{macrocode}
\gl@set@shorthand{"}
\def\gl@umlaut#1{%
  \bbl@allowhyphens\add@accent{127}#1\bbl@allowhyphens
  \spacefactor\sfcode`#1 }
%    \end{macrocode}
%
%    We override the default |"| of babel, intended for german.
%
%    \begin{macrocode}
\ifes@LaTeXe   %<<<<<<
  \addto\shorthandsgalician{%
    \gl@activate{"}%
    \gl@activate{~}%
    \babel@save\bbl@umlauta
    \let\bbl@umlauta\gl@umlaut
    \expandafter\babel@save\csname OT1\string\~\endcsname
    \expandafter\def\csname OT1\string\~\endcsname{\gl@accent\~{126}}%
    \expandafter\babel@save\csname OT1\string\'\endcsname
    \expandafter\def\csname OT1\string\'\endcsname{\gl@accent\'{19}}}
\else          %<<<<<<
  \addto\shorthandsgalician{%
    \gl@activate{"}%
    \gl@activate{~}%
    \babel@save\bbl@umlauta
    \let\bbl@umlauta\gl@umlaut
    \babel@save\~%
    \def\~{\gl@accent\~{126}}%
    \babel@save\'%
    \def\'#1{\if#1i\gl@accent\'{19}\i\else\gl@accent\'{19}{#1}\fi}}
\fi            %<<<<<<
%    \end{macrocode}
% \changes{galician~4.3}{07/01/22}{Removed the shorthands "er and "ER,
% they don't exist in galician.}
%    \begin{macrocode}
\declare@shorthand{galician}{"a}{\protect\gl@sptext{a}}
\declare@shorthand{galician}{"A}{\protect\gl@sptext{A}}
\declare@shorthand{galician}{"o}{\protect\gl@sptext{o}}
\declare@shorthand{galician}{"O}{\protect\gl@sptext{O}}

\declare@shorthand{galician}{"u}{\"u}
\declare@shorthand{galician}{"U}{\"U}
\declare@shorthand{galician}{"i}{\"i}
\declare@shorthand{galician}{"I}{\"I}

\declare@shorthand{galician}{"<}{\begin{quoting}}
\declare@shorthand{galician}{">}{\end{quoting}}
\declare@shorthand{galician}{"-}{\bbl@allowhyphens\-\bbl@allowhyphens}
\declare@shorthand{galician}{"=}%
  {\bbl@allowhyphens\char\hyphenchar\font\hskip\z@skip}
\declare@shorthand{galician}{"~}
  {\bbl@allowhyphens\discretionary{\char\hyphenchar\font}%
       {\char\hyphenchar\font}{\char\hyphenchar\font}\bbl@allowhyphens}
\declare@shorthand{galician}{"r}
  {\bbl@allowhyphens\discretionary{\char\hyphenchar\font}%
       {}{r}\bbl@allowhyphens}
\declare@shorthand{galician}{"R}
  {\bbl@allowhyphens\discretionary{\char\hyphenchar\font}%
       {}{R}\bbl@allowhyphens}
\declare@shorthand{galician}{""}{\hskip\z@skip}
\declare@shorthand{galician}{"/}
  {\setbox\z@\hbox{/}%
   \dimen@\ht\z@
   \advance\dimen@-1ex
   \advance\dimen@\dp\z@
   \dimen@.31\dimen@
   \advance\dimen@-\dp\z@
   \ifdim\dimen@>0pt
     \kern.01em\lower\dimen@\box\z@\kern.03em
   \else
     \box\z@
   \fi}
\declare@shorthand{galician}{"?}
  {\setbox\z@\hbox{?`}%
   \leavevmode\raise\dp\z@\box\z@}
\declare@shorthand{galician}{"!}
  {\setbox\z@\hbox{!`}%
   \leavevmode\raise\dp\z@\box\z@}

\gl@set@shorthand{~}
\declare@shorthand{galician}{~n}{\~n}
\declare@shorthand{galician}{~N}{\~N}
\declare@shorthand{galician}{~-}{%
  \leavevmode
  \bgroup
  \let\@sptoken\gl@dashes  % This assignation changes the
  \@ifnextchar-%             \@ifnextchar behaviour
    {\gl@dashes}%
    {\hbox{\char\hyphenchar\font}\egroup}}
\def\gl@dashes-{%
  \@ifnextchar-%
    {\bbl@allowhyphens\hbox{---}\bbl@allowhyphens\egroup\@gobble}%
    {\bbl@allowhyphens\hbox{--}\bbl@allowhyphens\egroup}}

\def\deactivatetilden{%
  \expandafter\let\csname galician@sh@\string~@n@\endcsname\relax
  \expandafter\let\csname galician@sh@\string~@N@\endcsname\relax}
%    \end{macrocode}
%
%    The shorthands for |quoting|.
%
%    \begin{macrocode}
\expandafter\ifx\csname XML@catcodes\endcsname\relax
  \addto\gl@select{%
    \catcode`\<\active\catcode`\>=\active
    \gl@quoting}

  \gl@set@shorthand{<}
  \gl@set@shorthand{>}

  \declare@shorthand{system}{<}{\csname normal@char\string<\endcsname}
  \declare@shorthand{system}{>}{\csname normal@char\string>\endcsname}

  \addto\shorthandsgalician{%
    \gl@activate{<}%
    \gl@activate{>}}
%    \end{macrocode}
%
%    \begin{macrocode}
  \ifes@LaTeXe   %<<<<<<
    \AtBeginDocument{%
      \gl@quoting
      \if@filesw
        \immediate\write\@mainaux{\string\gl@quoting}%
      \fi}%
  \fi            %<<<<<<

  \def\activatequoting{%
    \catcode`>=\active \catcode`<=\active
    \let\gl@quoting\activatequoting}
  \def\deactivatequoting{%
    \catcode`>=12 \catcode`<=12
    \let\gl@quoting\deactivatequoting}

  \declare@shorthand{galician}{<<}{\guillemotleft{}}
  \declare@shorthand{galician}{>>}{\guillemotright{}}
\fi

\let\gl@quoting\relax
\let\deactivatequoting\relax
\let\activatequoting\relax
%    \end{macrocode}
%
%    The acute accents are stored in a macro. If |activeacute| was set
%    as an option it's executed. If not is not deleted for a possible
%    later use in the |cfg| file. In non \LaTeXe{} formats is always
%    executed.
%    
%    \begin{macrocode}
\def\gl@activeacute{%
  \gl@set@shorthand{'}%
  \addto\shorthandsgalician{\gl@activate{'}}%
  \addto\gl@reviveshorthands{\gl@restore{'}}%
  \addto\gl@select{\catcode`'=\active}%
  \declare@shorthand{galician}{'a}{\@tabacckludge'a}%
  \declare@shorthand{galician}{'A}{\@tabacckludge'A}%
  \declare@shorthand{galician}{'e}{\@tabacckludge'e}%
  \declare@shorthand{galician}{'E}{\@tabacckludge'E}%
  \declare@shorthand{galician}{'i}{\@tabacckludge'i}%
  \declare@shorthand{galician}{'I}{\@tabacckludge'I}%
  \declare@shorthand{galician}{'o}{\@tabacckludge'o}%
  \declare@shorthand{galician}{'O}{\@tabacckludge'O}%
  \declare@shorthand{galician}{'u}{\@tabacckludge'u}%
  \declare@shorthand{galician}{'U}{\@tabacckludge'U}%
  \declare@shorthand{galician}{'n}{\~n}%
  \declare@shorthand{galician}{'N}{\~N}%
  \declare@shorthand{galician}{''}{\textquotedblright}%
  \let\gl@activeacute\relax}

\ifes@LaTeXe   %<<<<<<
  \@ifpackagewith{babel}{activeacute}{\gl@activeacute}{}
\else          %<<<<<<
  \gl@activeacute
\fi            %<<<<<<%
%    \end{macrocode}
%
%    And the customization. By default these macros only
%    store the values and do nothing.
%
%    \begin{macrocode}
\def\gl@enumerate#1#2#3#4{%
  \def\gl@enum{{#1}{#2}{#3}{#4}}}

\def\gl@itemize#1#2#3#4{%
  \def\gl@item{{#1}{#2}{#3}{#4}}}
%    \end{macrocode}
%    
%    The part formerly in the |.lld| file comes here. It performs
%    layout adaptation of \LaTeX{} to ``orthodox'' Galician rules.
%    \begin{macrocode}
\ifes@LaTeXe   %<<<<<<

\gl@enumerate{1.}{a)}{1)}{a$'$}
\def\galiciandashitems{\gl@itemize{---}{---}{---}{---}}
\def\galiciansymbitems{%
  \gl@itemize
    {\leavevmode\hbox to 1.2ex
      {\hss\vrule height .9ex width .7ex depth -.2ex\hss}}%
    {\textbullet}%
    {$\m@th\circ$}%
    {$\m@th\diamond$}}
\def\galiciansignitems{%
  \gl@itemize
    {\textbullet}%
    {$\m@th\circ$}%
    {$\m@th\diamond$}%
    {$\m@th\triangleright$}}
\galiciansymbitems

\def\gl@enumdef#1#2#3\@@{%
  \if#21%
    \@namedef{theenum#1}{\arabic{enum#1}}%
  \else\if#2a%
    \@namedef{theenum#1}{\emph{\alph{enum#1}}}%
  \else\if#2A%
    \@namedef{theenum#1}{\Alph{enum#1}}%
  \else\if#2i%
    \@namedef{theenum#1}{\roman{enum#1}}%
  \else\if#2I%
    \@namedef{theenum#1}{\Roman{enum#1}}%
  \else\if#2o%
    \@namedef{theenum#1}{\arabic{enum#1}\protect\gl@sptext{o}}%
  \fi\fi\fi\fi\fi\fi
  \toks@\expandafter{\csname theenum#1\endcsname}
  \expandafter\edef\csname labelenum#1\endcsname
     {\noexpand\gl@listquot\the\toks@#3}}

\addto\layoutgalician{%
  \def\gl@enumerate##1##2##3##4{%
    \gl@enumdef{i}##1\@empty\@empty\@@
    \gl@enumdef{ii}##2\@empty\@empty\@@
    \gl@enumdef{iii}##3\@empty\@empty\@@
    \gl@enumdef{iv}##4\@empty\@empty\@@}%
  \def\gl@itemize##1##2##3##4{%
    \def\labelitemi{\gl@listquot##1}%
    \def\labelitemii{\gl@listquot##2}%
    \def\labelitemiii{\gl@listquot##3}%
    \def\labelitemiv{\gl@listquot##4}}%
  \def\p@enumii{\theenumi}%
  \def\p@enumiii{\theenumi\theenumii}%
  \def\p@enumiv{\p@enumiii\theenumiii}%
  \expandafter\gl@enumerate\gl@enum
  \expandafter\gl@itemize\gl@item
  \DeclareTextCommand{\guillemotleft}{OT1}{%
    \ifmmode\ll
    \else
      \save@sf@q{\penalty\@M
        \leavevmode\hbox{\usefont{U}{lasy}{m}{n}%
          \char40 \kern-0.19em\char40 }}%
    \fi}%
  \DeclareTextCommand{\guillemotright}{OT1}{%
    \ifmmode\gg
    \else
      \save@sf@q{\penalty\@M
          \leavevmode\hbox{\usefont{U}{lasy}{m}{n}%
            \char41 \kern-0.19em\char41 }}%
    \fi}%
  \def\@fnsymbol##1%
    {\ifcase##1\or*\or**\or***\or****\or
     *****\or******\else\@ctrerr\fi}%
  \def\@alph##1%
    {\ifcase##1\or a\or b\or c\or d\or e\or f\or g\or h\or i\or
     l\or m\or n\or \~n\or o\or p\or q\or r\or s\or t\or u\or v\or
     x\or z\else\@ctrerr\fi}%
  \def\@Alph##1%
    {\ifcase##1\or A\or B\or C\or D\or E\or F\or G\or H\or I\or
     L\or M\or N\or \~N\or O\or P\or Q\or R\or S\or T\or U\or V\or
     X\or Z\else\@ctrerr\fi}%
  \let\@afterindentfalse\@afterindenttrue
  \@afterindenttrue
  \def\@seccntformat##1{\csname the##1\endcsname.\quad}%
  \def\numberline##1{\hb@xt@\@tempdima{##1\if&##1&\else.\fi\hfil}}%
  \def\@roman##1{\protect\gl@roman{\number##1}}%
  \def\gl@roman##1{\protect\gl@msc{\romannumeral##1}}%
  \def\glromanindex##1##2{##1{\protect\gl@msc{##2}}}}
%    \end{macrocode}
%    
%    We need to execute the following code when babel has been
%    run, in order to see if |galician| is the main language.
%    
%    \begin{macrocode}
\AtEndOfPackage{%
  \let\gl@activeacute\@undefined
  \def\bbl@tempa{galician}%
  \ifx\bbl@main@language\bbl@tempa
    \AtBeginDocument{\layoutgalician}%
    \addto\gl@select{%
      \@ifstar{\let\layoutgalician\relax}%
              {\layoutgalician\let\layoutgalician\relax}}%
  \fi
  \selectgalician}

\fi            %<<<<<<
%    \end{macrocode}
%    
%    After restoring the catcode of |~| and setting the minimal
%    values for hyphenation, the |.ldf| is finished.
%    
%    \begin{macrocode}
\gl@savedcatcodes

\providehyphenmins{\CurrentOption}{\tw@\tw@}

\ifes@LaTeXe   %<<<<<<
  \ldf@finish{galician}
\else          %<<<<<<
  \gl@select
  \ldf@finish{galician}
  \csname activatequoting\endcsname
\fi            %<<<<<<

%</code>
%    \end{macrocode}
%    That's all in the main file. Now the file with
%    \textsf{custom-bib} macros.
%
%    \begin{macrocode}
%<*bblbst>
\def\bbland{e}
\def\bbleditors{directores}     \def\bbleds{dirs.\@}
\def\bbleditor{director}        \def\bbled{dir.\@}
\def\bbledby{dirixido por}
\def\bbledition{edici\'on}      \def\bbledn{ed.\@}
\def\bbletal{e outros}
\def\bblvolume{volumen}       \def\bblvol{vol.\@}
\def\bblof{de}
\def\bblnumber{n\'umero}        \def\bblno{n\sptext{o}}
\def\bblin{en} 
\def\bblpages{p\'axinas}        \def\bblpp{p\'axs.\@}
\def\bblpage{p\'axina}           \def\bblp{p\'ax.\@}
\def\bblchapter{cap\'itulo}     \def\bblchap{cap.\@}
\def\bbltechreport{informe t\'ecnico}
\def\bbltechrep{inf.\@ t\'ec.\@}
\def\bblmthesis{proxecto de fin de carreira}
\def\bblphdthesis{tesis doutoral}
\def\bblfirst {primeira}        \def\bblfirsto {1\sptext{a}}
\def\bblsecond{segunda}         \def\bblsecondo{2\sptext{a}}
\def\bblthird {terceira}        \def\bblthirdo {3\sptext{a}}
\def\bblfourth{cuarta}          \def\bblfourtho{4\sptext{a}}
\def\bblfifth {quinta}          \def\bblfiftho {5\sptext{a}}
\def\bblth{\sptext{a}}
\let\bblst\bblth   \let\bblnd\bblth   \let\bblrd\bblth
\def\bbljan{xaneiro}  \def\bblfeb{febreiro}  \def\bblmar{marzo}
\def\bblapr{abril}  \def\bblmay{maio}     \def\bbljun{xu\~no}
\def\bbljul{xullo}  \def\bblaug{agosto}   \def\bblsep{setembro}
\def\bbloct{outubro}\def\bblnov{novembro}\def\bbldec{decembro}
%</bblbst>
%    \end{macrocode}
%
%    The |galician| option writes a macro in the page field of
%    \textit{MakeIndex} in entries with medium caps number, and they
%    are rejected. This program is a preprocessor which moves this
%    macro to the entry field.
%
%    \begin{macrocode}
%<*indexgl>
\makeatletter

\newcount\gl@converted
\newcount\gl@processed

\def\gl@encap{`\|}
\def\gl@openrange{`\(}
\def\gl@closerange{`\)}

\def\gl@split@file#1.#2\@@{#1}
\def\gl@split@ext#1.#2\@@{#2}

\typein[\answer]{^^JArchivo que convertir^^J%
   (extension por omision .idx):}
   
\@expandtwoargs\in@{.}{\answer}
\ifin@
  \edef\gl@input@file{\expandafter\gl@split@file\answer\@@}
  \edef\gl@input@ext{\expandafter\gl@split@ext\answer\@@}
\else
  \edef\gl@input@file{\answer}
  \def\gl@input@ext{idx}
\fi

\typein[\answer]{^^JArquivo de destino^^J%
   (arquivo por omision: \gl@input@file.eix,^^J%
    extension por omision .eix):}
\ifx\answer\@empty
  \edef\gl@output{\gl@input@file.eix}
\else
  \@expandtwoargs\in@{.}{\answer}
  \ifin@
     \edef\gl@output{\answer}
  \else
     \edef\gl@output{\answer.eix}
  \fi
\fi

\typein[\answer]{%
 ^^J?Usouse algun esquema especial de controles^^J%
 de MakeIndex para encap, open_range ou close_range?^^J%
 [s/n] (n por omision)}
 
\if s\answer
  \typein[\answer]{^^JCaracter para 'encap'^^J%
    (\string| por omision)}
  \ifx\answer\@empty\else
    \edef\gl@encap{%
      `\expandafter\noexpand\csname\expandafter\string\answer\endcsname}
  \fi
  \typein[\answer]{^^JCaracter para 'open_range'^^J%
    (\string( por omision)}
  \ifx\answer\@empty\else
    \edef\gl@openrange{%
      `\expandafter\noexpand\csname\expandafter\string\answer\endcsname}
  \fi
  \typein[\answer]{^^JCaracter para 'close_range'^^J%
    (\string) por omision)}
  \ifx\answer\@empty\else
    \edef\gl@closerange{%
      `\expandafter\noexpand\csname\expandafter\string\answer\endcsname}
  \fi
\fi
   
\newwrite\gl@indexfile
\immediate\openout\gl@indexfile=\gl@output

\newif\ifgl@encapsulated

\def\gl@roman#1{\romannumeral#1 }
\edef\gl@slash{\expandafter\@gobble\string\\}

\def\indexentry{%
  \begingroup
  \@sanitize
  \gl@indexentry}
  
\begingroup

\catcode`\|=12 \lccode`\|=\gl@encap\relax
\catcode`\(=12 \lccode`\(=\gl@openrange\relax
\catcode`\)=12 \lccode`\)=\gl@closerange\relax

\lowercase{
\gdef\gl@indexentry#1{%
  \endgroup
  \advance\gl@processed\@ne
  \gl@encapsulatedfalse
  \gl@bar@idx#1|\@@
  \gl@idxentry}%
}

\lowercase{
\gdef\gl@idxentry#1{%
  \in@{\gl@roman}{#1}%
  \ifin@
    \advance\gl@converted\@ne
    \immediate\write\gl@indexfile{%
      \string\indexentry{\gl@b|\ifgl@encapsulated\gl@p\fi glromanindex%
        {\ifx\gl@a\@empty\else\gl@slash\gl@a\fi}}{#1}}%
  \else
    \immediate\write\gl@indexfile{%
      \string\indexentry{\gl@b\ifgl@encapsulated|\gl@p\gl@a\fi}{#1}}%
  \fi}
}

\lowercase{
\gdef\gl@bar@idx#1|#2\@@{%
  \def\gl@b{#1}\def\gl@a{#2}%
  \ifx\gl@a\@empty\else\gl@encapsulatedtrue\gl@bar@eat#2\fi}
}

\lowercase{
\gdef\gl@bar@eat#1#2|{\def\gl@p{#1}\def\gl@a{#2}%
  \edef\gl@t{(}\ifx\gl@t\gl@p
  \else\edef\gl@t{)}\ifx\gl@t\gl@p
  \else
    \edef\gl@a{\gl@p\gl@a}\let\gl@p\@empty%
  \fi\fi}
}

\endgroup

\input \gl@input@file.\gl@input@ext

\immediate\closeout\gl@indexfile

\typeout{*****************}
\typeout{procesouse: \gl@input@file.\gl@input@ext }
\typeout{Li'nas lidas: \the\gl@processed}
\typeout{Li'nas convertidas: \the\gl@converted}
\typeout{Resultado en: \gl@output}
\ifnum\gl@converted>\z@
  \typeout{Xenere o 'indice a partir deste arquivo}
\else
  \typeout{Non se realizou ning'un tipo de conversi'on}
  \typeout{P'odese xenerar o arquivo directamente^^J%
           de \gl@input@file.\gl@input@ext}
\fi
\typeout{*****************}
\@@end
%</indexgl>
%    \end{macrocode}
%
% \Finale
%
%%
%% \CharacterTable
%%  {Upper-case    \A\B\C\D\E\F\G\H\I\J\K\L\M\N\O\P\Q\R\S\T\U\V\W\X\Y\Z
%%   Lower-case    \a\b\c\d\e\f\g\h\i\j\k\l\m\n\o\p\q\r\s\t\u\v\w\x\y\z
%%   Digits        \0\1\2\3\4\5\6\7\8\9
%%   Exclamation   \!     Double quote  \"     Hash (number) \#
%%   Dollar        \$     Percent       \%     Ampersand     \&
%%   Acute accent  \'     Left paren    \(     Right paren   \)
%%   Asterisk      \*     Plus          \+     Comma         \,
%%   Minus         \-     Point         \.     Solidus       \/
%%   Colon         \:     Semicolon     \;     Less than     \<
%%   Equals        \=     Greater than  \>     Question mark \?
%%   Commercial at \@     Left bracket  \[     Backslash     \\
%%   Right bracket \]     Circumflex    \^     Underscore    \_
%%   Grave accent  \`     Left brace    \{     Vertical bar  \|
%%   Right brace   \}     Tilde         \~}
%%
\endinput
}
\DeclareOption{german}{% \iffalse meta-comm

% Copyright 1989-2008 Johannes L. Braams and any individual auth
% listed elsewhere in this file.  All rights reserv

% This file is part of the Babel syst
% -----------------------------------

% It may be distributed and/or modified under
% conditions of the LaTeX Project Public License, either version
% of this license or (at your option) any later versi
% The latest version of this license is
%   http://www.latex-project.org/lppl.
% and version 1.3 or later is part of all distributions of La
% version 2003/12/01 or lat

% This work has the LPPL maintenance status "maintaine

% The Current Maintainer of this work is Johannes Braa

% The list of all files belonging to the Babel system
% given in the file `manifest.bbl. See also `legal.bbl' for additio
% informati

% The list of derived (unpacked) files belonging to the distribut
% and covered by LPPL is defined by the unpacking scripts (w
% extension .ins) which are part of the distributi
%
% \CheckSum{3

% \iffa
%    Tell the \LaTeX\ system who we are and write an entry on
%    transcri
%<*d
\ProvidesFile{germanb.d
%</d
%<code>\ProvidesLanguage{germa
%
%\ProvidesFile{germanb.d
        [2008/06/01 v2.6m German support from the babel syst
%\iffa
%% File `germanb.d
%% Babel package for LaTeX version
%% Copyright (C) 1989 - 2
%%           by Johannes Braams, TeXn

%% Germanb Language Definition F
%% Copyright (C) 1989 - 2
%%           by Bernd Raichle raichle at azu.Informatik.Uni-Stuttgart
%%              Johannes Braams, TeXn
% This file is based on german.tex version 2.
%                       by Bernd Raichle, Hubert Partl et.

%% Please report errors to: J.L. Bra
%%                          babel at braams.xs4all

%<*filedriv
\documentclass{ltxd
\font\manual=logo10 % font used for the METAFONT logo, e
\newcommand*\MF{{\manual META}\-{\manual FON
\newcommand*\TeXhax{\TeX h
\newcommand*\babel{\textsf{babe
\newcommand*\langvar{$\langle \it lang \rangl
\newcommand*\note[1
\newcommand*\Lopt[1]{\textsf{#
\newcommand*\file[1]{\texttt{#
\begin{docume
 \DocInput{germanb.d
\end{docume
%</filedriv
%
% \GetFileInfo{germanb.d

% \changes{germanb-1.0a}{1990/05/14}{Incorporated Nico's commen
% \changes{germanb-1.0b}{1990/05/22}{fixed typo in definition
%    austrian language found by Werenfried S
%    \texttt{nspit@fys.ruu.n
% \changes{germanb-1.0c}{1990/07/16}{Fixed some typ
% \changes{germanb-1.1}{1990/07/30}{When using PostScript fonts w
%    the Adobe fontencoding, the dieresis-accent is located elsewhe
%    modified co
% \changes{germanb-1.1a}{1990/08/27}{Modified the documentat
%    somewh
% \changes{germanb-2.0}{1991/04/23}{Modified for babel 3
% \changes{germanb-2.0a}{1991/05/25}{Removed some problems in cha
%    l
% \changes{germanb-2.1}{1991/05/29}{Removed bug found by van der Me
% \changes{germanb-2.2}{1991/06/11}{Removed global assignmen
%    brought uptodate with \file{german.tex} v2.
% \changes{germanb-2.2a}{1991/07/15}{Renamed \file{babel.sty}
%    \file{babel.co
% \changes{germanb-2.3}{1991/11/05}{Rewritten parts of the code to
%    the new features of babel version 3
% \changes{germanb-2.3e}{1991/11/10}{Brought up-to-date w
%    \file{german.tex} v2.3e (plus some bug fixes) [b
% \changes{germanb-2.5}{1994/02/08}{Update or \LaTe
% \changes{germanb-2.5c}{1994/06/26}{Removed the use of \cs{fileda
%    and moved the identification after the loading
%    \file{babel.de
% \changes{germanb-2.6a}{1995/02/15}{Moved the identification to
%    top of the fi
% \changes{germanb-2.6a}{1995/02/15}{Rewrote the code that handles
%    active double quote charact
% \changes{germanb-2.6d}{1996/07/10}{Replaced \cs{undefined} w
%    \cs{@undefined} and \cs{empty} with \cs{@empty} for consiste
%    with \LaTe
% \changes{germanb-2.6d}{1996/10/10}{Moved the definition
%    \cs{atcatcode} right to the beginning

%  \section{The German langua

%    The file \file{\filename}\footnote{The file described in t
%    section has version number \fileversion\ and was last revised
%    \filedate.}  defines all the language definition macros for
%    German language as well as for the Austrian dialect of t
%    language\footnote{This file is a re-implementation of Hub
%    Partl's \file{german.sty} version 2.5b, see~\cite{HP}

%    For this language the character |"| is made active.
%    table~\ref{tab:german-quote} an overview is given of
%    purpose. One of the reasons for this is that in the Ger
%    language some character combinations change when a word is bro
%    between the combination. Also the vertical placement of
%    umlaut can be controlled this w
%    \begin{table}[h
%     \begin{cent
%     \begin{tabular}{lp{8c
%      |"a| & |\"a|, also implemented for the ot
%                  lowercase and uppercase vowels.
%      |"s| & to produce the German \ss{} (like |\ss{}|).
%      |"z| & to produce the German \ss{} (like |\ss{}|).
%      |"ck|& for |ck| to be hyphenated as |k-k|.
%      |"ff|& for |ff| to be hyphenated as |ff-
%                  this is also implemented for l, m, n, p, r and
%      |"S| & for |SS| to be |\uppercase{"s}|.
%      |"Z| & for |SZ| to be |\uppercase{"z}|.
%      \verb="|= & disable ligature at this position.
%      |"-| & an explicit hyphen sign, allowing hyphenat
%             in the rest of the word.
%      |""| & like |"-|, but producing no hyphen s
%             (for compund words with hyphen, e.g.\ |x-""y|).
%      |"~| & for a compound word mark without a breakpoint.
%      |"=| & for a compound word mark with a breakpoint, allow
%             hyphenation in the composing words.
%      |"`| & for German left double quotes (looks like ,,).
%      |"'| & for German right double quotes.
%      |"<| & for French left double quotes (similar to $<<$).
%      |">| & for French right double quotes (similar to $>>$).
%     \end{tabul
%     \caption{The extra definitions m
%              by \file{german.ldf}}\label{tab:german-quo
%     \end{cent
%    \end{tab
%    The quotes in table~\ref{tab:german-quote} can also be typeset
%    using the commands in table~\ref{tab:more-quot
%    \begin{table}[h
%     \begin{cent
%     \begin{tabular}{lp{8c
%      |\glqq| & for German left double quotes (looks like ,,).
%      |\grqq| & for German right double quotes (looks like ``).
%      |\glq|  & for German left single quotes (looks like ,).
%      |\grq|  & for German right single quotes (looks like `).
%      |\flqq| & for French left double quotes (similar to $<<$).
%      |\frqq| & for French right double quotes (similar to $>>$)
%      |\flq|  & for (French) left single quotes (similar to $<$).
%      |\frq|  & for (French) right single quotes (similar to $>$).
%      |\dq|   & the original quotes character (|"|).
%     \end{tabul
%     \caption{More commands which produce quotes, defi
%              by \file{german.ldf}}\label{tab:more-quo
%     \end{cent
%    \end{tab

% \StopEventuall

%    When this file was read through the option \Lopt{germanb} we m
%    it behave as if \Lopt{german} was specifi
% \changes{german-2.6l}{2008/03/17}{Making germanb behave like ger
%    needs some more work besides defining \cs{CurrentOptio
% \changes{germanb-2.6m}{2008/06/01}{Correted a ty
%    \begin{macroco
\def\bbl@tempa{germa
\ifx\CurrentOption\bbl@te
  \def\CurrentOption{germ
  \ifx\l@german\@undefi
    \@nopatterns{Germ
    \adddialect\l@germ

  \let\l@germanb\l@ger
  \AtBeginDocumen
    \let\captionsgermanb\captionsger
    \let\dategermanb\dateger
    \let\extrasgermanb\extrasger
    \let\noextrasgermanb\noextrasger


%    \end{macroco

%    The macro |\LdfInit| takes care of preventing that this file
%    loaded more than once, checking the category code of
%    \texttt{@} sign, e
% \changes{germanb-2.6d}{1996/11/02}{Now use \cs{LdfInit} to perf
%    initial check
%    \begin{macroco
%<*co
\LdfInit\CurrentOption{captions\CurrentOpti
%    \end{macroco

%    When this file is read as an option, i.e., by the |\usepacka
%    command, \texttt{german} will be an `unknown' language, so
%    have to make it known.  So we check for the existence
%    |\l@german| to see whether we have to do something he

% \changes{germanb-2.0}{1991/04/23}{Now use \cs{adddialect}
%    language undefin
% \changes{germanb-2.2d}{1991/10/27}{Removed use of \cs{@ifundefine
% \changes{germanb-2.3e}{1991/11/10}{Added warning, if no ger
%    patterns load
% \changes{germanb-2.5c}{1994/06/26}{Now use \cs{@nopatterns}
%    produce the warni
%    \begin{macroco
\ifx\l@german\@undefi
  \@nopatterns{Germ
  \adddialect\l@germ

%    \end{macroco

%    For the Austrian version of these definitions we just add anot
%    languag
% \changes{germanb-2.0}{1991/04/23}{Now use \cs{adddialect}
%    austri
%    \begin{macroco
\adddialect\l@austrian\l@ger
%    \end{macroco

%    The next step consists of defining commands to switch to (
%    from) the German langua

%  \begin{macro}{\captionsgerm
%  \begin{macro}{\captionsaustri
%    Either the macro |\captionsgerman| or the ma
%    |\captionsaustrian| will define all strings used in the f
%    standard document classes provided with \LaT

% \changes{germanb-2.2}{1991/06/06}{Removed \cs{global} definitio
% \changes{germanb-2.2}{1991/06/06}{\cs{pagename} should
%    \cs{headpagenam
% \changes{germanb-2.3e}{1991/11/10}{Added \cs{prefacenam
%    \cs{seename} and \cs{alsonam
% \changes{germanb-2.4}{1993/07/15}{\cs{headpagename} should
%    \cs{pagenam
% \changes{germanb-2.6b}{1995/07/04}{Added \cs{proofname}
%    AMS-\LaT
% \changes{germanb-2.6d}{1996/07/10}{Construct control sequence on
%    f
% \changes{germanb-2.6j}{2000/09/20}{Added \cs{glossarynam
%    \begin{macroco
\@namedef{captions\CurrentOption
  \def\prefacename{Vorwor
  \def\refname{Literatu
  \def\abstractname{Zusammenfassun
  \def\bibname{Literaturverzeichni
  \def\chaptername{Kapite
  \def\appendixname{Anhan
  \def\contentsname{Inhaltsverzeichnis}%    % oder nur: Inh
  \def\listfigurename{Abbildungsverzeichni
  \def\listtablename{Tabellenverzeichni
  \def\indexname{Inde
  \def\figurename{Abbildun
  \def\tablename{Tabelle}%                  % oder: Ta
  \def\partname{Tei
  \def\enclname{Anlage(n)}%                 % oder: Beilage
  \def\ccname{Verteiler}%                   % oder: Kopien
  \def\headtoname{A
  \def\pagename{Seit
  \def\seename{sieh
  \def\alsoname{siehe auc
  \def\proofname{Bewei
  \def\glossaryname{Glossa

%    \end{macroco
%  \end{mac
%  \end{mac

%  \begin{macro}{\dategerm
%    The macro |\dategerman| redefines the comm
%    |\today| to produce German dat
% \changes{germanb-2.3e}{1991/11/10}{Added \cs{month@germa
% \changes{germanb-2.6f}{1997/10/01}{Use \cs{edef} to def
%    \cs{today} to save memo
% \changes{germanb-2.6f}{1998/03/28}{use \cs{def} instead
%    \cs{ede
%    \begin{macroco
\def\month@german{\ifcase\month
  Januar\or Februar\or M\"arz\or April\or Mai\or Juni
  Juli\or August\or September\or Oktober\or November\or Dezember\
\def\dategerman{\def\today{\number\day.~\month@ger
    \space\number\yea
%    \end{macroco
%  \end{mac

%  \begin{macro}{\dateaustri
%    The macro |\dateaustrian| redefines the comm
%    |\today| to produce Austrian version of the German dat
% \changes{germanb-2.6f}{1997/10/01}{Use \cs{edef} to def
%    \cs{today} to save memo
% \changes{germanb-2.6f}{1998/03/28}{use \cs{def} instead
%    \cs{ede
%    \begin{macroco
\def\dateaustrian{\def\today{\number\day.~\ifnum1=\mo
  J\"anner\else \month@german\fi \space\number\yea
%    \end{macroco
%  \end{mac

%  \begin{macro}{\extrasgerm
%  \begin{macro}{\extrasaustri
% \changes{germanb-2.0b}{1991/05/29}{added some comment chars
%    prevent white spa
% \changes{germanb-2.2}{1991/06/11}{Save all redefined macr
%  \begin{macro}{\noextrasgerm
%  \begin{macro}{\noextrasaustri
% \changes{germanb-1.1}{1990/07/30}{Added \cs{dieresi
% \changes{germanb-2.0b}{1991/05/29}{added some comment chars
%    prevent white spa
% \changes{germanb-2.2}{1991/06/11}{Try to restore everything to
%    former sta
% \changes{germanb-2.6d}{1996/07/10}{Construct control seque
%    \cs{extrasgerman} or \cs{extrasaustrian} on the f

%    Either the macro |\extrasgerman| or the macros |\extrasaustri
%    will perform all the extra definitions needed for the Ger
%    language. The macro |\noextrasgerman| is used to cancel
%    actions of |\extrasgerman

%    For German (as well as for Dutch) the \texttt{"} character
%    made active. This is done once, later on its definition may va
%    \begin{macroco
\initiate@active@char
\@namedef{extras\CurrentOption
  \languageshorthands{germa
\expandafter\addto\csname extras\CurrentOption\endcsnam
  \bbl@activate{
%    \end{macroco
%    Don't forget to turn the shorthands off aga
% \changes{germanb-2.6i}{1999/12/16}{Deactivate shorthands ouside
%    Germ
%    \begin{macroco
\addto\noextrasgerman{\bbl@deactivate{
%    \end{macroco

% \changes{germanb-2.6a}{1995/02/15}{All the code to handle the act
%    double quote has been moved to \file{babel.de

%    In order for \TeX\ to be able to hyphenate German words wh
%    contain `\ss' (in the \texttt{OT1} position |^^Y|) we have
%    give the character a nonzero |\lccode| (see Appendix H, the \
%    boo
% \changes{germanb-2.6c}{1996/04/08}{Use decimal number instead
%    hat-notation as the hat may be activat
%    \begin{macroco
\expandafter\addto\csname extras\CurrentOption\endcsnam
  \babel@savevariable{\lccode2
  \lccode25=
%    \end{macroco
% \changes{germanb-2.6a}{1995/02/15}{Removeed \cs{3} as it is
%    longer in \file{german.ld

%    The umlaut accent macro |\"| is changed to lower the umlaut do
%    The redefinition is done with the help of |\umlautlo
%    \begin{macroco
\expandafter\addto\csname extras\CurrentOption\endcsnam
  \babel@save\"\umlautl
\@namedef{noextras\CurrentOption}{\umlauthi
%    \end{macroco
%    The german hyphenation patterns can be used with |\lefthyphenm
%    and |\righthyphenmin| set to
% \changes{germanb-2.6a}{1995/05/13}{use \cs{germanhyphenmins} to st
%    the correct valu
% \changes{germanb-2.6j}{2000/09/22}{Now use \cs{providehyphenmins}
%    provide a default val
%    \begin{macroco
\providehyphenmins{\CurrentOption}{\tw@\t
%    \end{macroco
%    For German texts we need to make sure that |\frenchspacing|
%    turned
% \changes{germanb-2.6k}{2001/01/26}{Turn frenchspacing on, as
%    \texttt{german.st
%    \begin{macroco
\expandafter\addto\csname extras\CurrentOption\endcsnam
  \bbl@frenchspaci
\expandafter\addto\csname noextras\CurrentOption\endcsnam
  \bbl@nonfrenchspaci
%    \end{macroco
%  \end{mac
%  \end{mac
%  \end{mac
%  \end{mac

% \changes{germanb-2.6a}{1995/02/15}{\cs{umlautlow}
%    \cs{umlauthigh} moved to \file{glyphs.dtx}, as well
%    \cs{newumlaut} (now \cs{lower@umlau

%    The code above is necessary because we need an extra act
%    character. This character is then used as indicated
%    table~\ref{tab:german-quot

%    To be able to define the function of |"|, we first defin
%    couple of `support' macr

% \changes{germanb-2.3e}{1991/11/10}{Added \cs{save@sf@q} macro
%    rewrote all quote macros to use
% \changes{germanb-2.3h}{1991/02/16}{moved definition
%    \cs{allowhyphens}, \cs{set@low@box} and \cs{save@sf@q}
%    \file{babel.co
% \changes{germanb-2.6a}{1995/02/15}{Moved all quotation characters
%    \file{glyphs.dt

%  \begin{macro}{\
%    We save the original double quote character in |\dq| to k
%    it available, the math accent |\"| can now be typed as |
%    \begin{macroco
\begingroup \catcode`\
\def\x{\endgr
  \def\@SS{\mathchar"701
  \def\dq{

%    \end{macroco
%  \end{mac
% \changes{germanb-2.6c}{1996/01/24}{Moved \cs{german@dq@disc}
%    babel.def, calling it \cs{bbl@dis

% \changes{germanb-2.6a}{1995/02/15}{Use \cs{ddot} instead
%    \cs{@MATHUMLAU

%    Now we can define the doublequote macros: the umlau
% \changes{germanb-2.6c}{1996/05/30}{added the \cs{allowhyphen
%    \begin{macroco
\declare@shorthand{german}{"a}{\textormath{\"{a}\allowhyphens}{\ddot
\declare@shorthand{german}{"o}{\textormath{\"{o}\allowhyphens}{\ddot
\declare@shorthand{german}{"u}{\textormath{\"{u}\allowhyphens}{\ddot
\declare@shorthand{german}{"A}{\textormath{\"{A}\allowhyphens}{\ddot
\declare@shorthand{german}{"O}{\textormath{\"{O}\allowhyphens}{\ddot
\declare@shorthand{german}{"U}{\textormath{\"{U}\allowhyphens}{\ddot
%    \end{macroco
%    trem
%    \begin{macroco
\declare@shorthand{german}{"e}{\textormath{\"{e}}{\ddot
\declare@shorthand{german}{"E}{\textormath{\"{E}}{\ddot
\declare@shorthand{german}{"i}{\textormath{\"{\i
                              {\ddot\imat
\declare@shorthand{german}{"I}{\textormath{\"{I}}{\ddot
%    \end{macroco
%    german es-zet (sharp
% \changes{germanb-2.6f}{1997/05/08}{use \cs{SS} instead
%    \texttt{SS}, removed braces after \cs{ss
%    \begin{macroco
\declare@shorthand{german}{"s}{\textormath{\ss}{\@SS{
\declare@shorthand{german}{"S}{\
\declare@shorthand{german}{"z}{\textormath{\ss}{\@SS{
\declare@shorthand{german}{"Z}{
%    \end{macroco
%    german and french quot
%    \begin{macroco
\declare@shorthand{german}{"`}{\gl
\declare@shorthand{german}{"'}{\gr
\declare@shorthand{german}{"<}{\fl
\declare@shorthand{german}{">}{\fr
%    \end{macroco
%    discretionary comma
%    \begin{macroco
\declare@shorthand{german}{"c}{\textormath{\bbl@disc ck}{
\declare@shorthand{german}{"C}{\textormath{\bbl@disc CK}{
\declare@shorthand{german}{"F}{\textormath{\bbl@disc F{FF}}{
\declare@shorthand{german}{"l}{\textormath{\bbl@disc l{ll}}{
\declare@shorthand{german}{"L}{\textormath{\bbl@disc L{LL}}{
\declare@shorthand{german}{"m}{\textormath{\bbl@disc m{mm}}{
\declare@shorthand{german}{"M}{\textormath{\bbl@disc M{MM}}{
\declare@shorthand{german}{"n}{\textormath{\bbl@disc n{nn}}{
\declare@shorthand{german}{"N}{\textormath{\bbl@disc N{NN}}{
\declare@shorthand{german}{"p}{\textormath{\bbl@disc p{pp}}{
\declare@shorthand{german}{"P}{\textormath{\bbl@disc P{PP}}{
\declare@shorthand{german}{"r}{\textormath{\bbl@disc r{rr}}{
\declare@shorthand{german}{"R}{\textormath{\bbl@disc R{RR}}{
\declare@shorthand{german}{"t}{\textormath{\bbl@disc t{tt}}{
\declare@shorthand{german}{"T}{\textormath{\bbl@disc T{TT}}{
%    \end{macroco
%    We need to treat |"f| a bit differently in order to preserve
%    ff-ligatur
% \changes{germanb-2.6f}{1998/06/15}{Copied the coding for \texttt{
%    from german.dtx version 2.5
%    \begin{macroco
\declare@shorthand{german}{"f}{\textormath{\bbl@discff}{
\def\bbl@discff{\penalty
  \afterassignment\bbl@insertff \let\bbl@nextff
\def\bbl@insertf
  \if f\bbl@nex
    \expandafter\@firstoftwo\else\expandafter\@secondoftwo
  {\relax\discretionary{ff-}{f}{ff}\allowhyphens}{f\bbl@nextf
\let\bbl@nextf
%    \end{macroco
%    and some additional comman
%    \begin{macroco
\declare@shorthand{german}{"-}{\nobreak\-\bbl@allowhyphe
\declare@shorthand{german}{"|
  \textormath{\penalty\@M\discretionary{-}{}{\kern.03e
              \allowhyphens}
\declare@shorthand{german}{""}{\hskip\z@sk
\declare@shorthand{german}{"~}{\textormath{\leavevmode\hbox{-}}{
\declare@shorthand{german}{"=}{\penalty\@M-\hskip\z@sk
%    \end{macroco

%  \begin{macro}{\mdq
%  \begin{macro}{\mdqo
%  \begin{macro}{\
%    All that's left to do now is to  define a couple of comma
%    for reasons of compatibility with \file{german.st
% \changes{germanb-2.6f}{1998/06/07}{Now use \cs{shorthandon}
%    \cs{shorthandoff
%    \begin{macroco
\def\mdqon{\shorthandon{
\def\mdqoff{\shorthandoff{
\def\ck{\allowhyphens\discretionary{k-}{k}{ck}\allowhyphe
%    \end{macroco
%  \end{mac
%  \end{mac
%  \end{mac

%    The macro |\ldf@finish| takes care of looking fo
%    configuration file, setting the main language to be switched
%    at |\begin{document}| and resetting the category code
%    \texttt{@} to its original val
% \changes{germanb-2.6d}{1996/11/02}{Now use \cs{ldf@finish} to w
%    u
%    \begin{macroco
\ldf@finish\CurrentOpt
%</co
%    \end{macroco

% \Fin

%% \CharacterTa
%%  {Upper-case    \A\B\C\D\E\F\G\H\I\J\K\L\M\N\O\P\Q\R\S\T\U\V\W\X\
%%   Lower-case    \a\b\c\d\e\f\g\h\i\j\k\l\m\n\o\p\q\r\s\t\u\v\w\x\
%%   Digits        \0\1\2\3\4\5\6\7\
%%   Exclamation   \!     Double quote  \"     Hash (number)
%%   Dollar        \$     Percent       \%     Ampersand
%%   Acute accent  \'     Left paren    \(     Right paren
%%   Asterisk      \*     Plus          \+     Comma
%%   Minus         \-     Point         \.     Solidus
%%   Colon         \:     Semicolon     \;     Less than
%%   Equals        \=     Greater than  \>     Question mark
%%   Commercial at \@     Left bracket  \[     Backslash
%%   Right bracket \]     Circumflex    \^     Underscore
%%   Grave accent  \`     Left brace    \{     Vertical bar
%%   Right brace   \}     Tilde

\endin
}
\DeclareOption{germanb}{% \iffalse meta-comm

% Copyright 1989-2008 Johannes L. Braams and any individual auth
% listed elsewhere in this file.  All rights reserv

% This file is part of the Babel syst
% -----------------------------------

% It may be distributed and/or modified under
% conditions of the LaTeX Project Public License, either version
% of this license or (at your option) any later versi
% The latest version of this license is
%   http://www.latex-project.org/lppl.
% and version 1.3 or later is part of all distributions of La
% version 2003/12/01 or lat

% This work has the LPPL maintenance status "maintaine

% The Current Maintainer of this work is Johannes Braa

% The list of all files belonging to the Babel system
% given in the file `manifest.bbl. See also `legal.bbl' for additio
% informati

% The list of derived (unpacked) files belonging to the distribut
% and covered by LPPL is defined by the unpacking scripts (w
% extension .ins) which are part of the distributi
%
% \CheckSum{3

% \iffa
%    Tell the \LaTeX\ system who we are and write an entry on
%    transcri
%<*d
\ProvidesFile{germanb.d
%</d
%<code>\ProvidesLanguage{germa
%
%\ProvidesFile{germanb.d
        [2008/06/01 v2.6m German support from the babel syst
%\iffa
%% File `germanb.d
%% Babel package for LaTeX version
%% Copyright (C) 1989 - 2
%%           by Johannes Braams, TeXn

%% Germanb Language Definition F
%% Copyright (C) 1989 - 2
%%           by Bernd Raichle raichle at azu.Informatik.Uni-Stuttgart
%%              Johannes Braams, TeXn
% This file is based on german.tex version 2.
%                       by Bernd Raichle, Hubert Partl et.

%% Please report errors to: J.L. Bra
%%                          babel at braams.xs4all

%<*filedriv
\documentclass{ltxd
\font\manual=logo10 % font used for the METAFONT logo, e
\newcommand*\MF{{\manual META}\-{\manual FON
\newcommand*\TeXhax{\TeX h
\newcommand*\babel{\textsf{babe
\newcommand*\langvar{$\langle \it lang \rangl
\newcommand*\note[1
\newcommand*\Lopt[1]{\textsf{#
\newcommand*\file[1]{\texttt{#
\begin{docume
 \DocInput{germanb.d
\end{docume
%</filedriv
%
% \GetFileInfo{germanb.d

% \changes{germanb-1.0a}{1990/05/14}{Incorporated Nico's commen
% \changes{germanb-1.0b}{1990/05/22}{fixed typo in definition
%    austrian language found by Werenfried S
%    \texttt{nspit@fys.ruu.n
% \changes{germanb-1.0c}{1990/07/16}{Fixed some typ
% \changes{germanb-1.1}{1990/07/30}{When using PostScript fonts w
%    the Adobe fontencoding, the dieresis-accent is located elsewhe
%    modified co
% \changes{germanb-1.1a}{1990/08/27}{Modified the documentat
%    somewh
% \changes{germanb-2.0}{1991/04/23}{Modified for babel 3
% \changes{germanb-2.0a}{1991/05/25}{Removed some problems in cha
%    l
% \changes{germanb-2.1}{1991/05/29}{Removed bug found by van der Me
% \changes{germanb-2.2}{1991/06/11}{Removed global assignmen
%    brought uptodate with \file{german.tex} v2.
% \changes{germanb-2.2a}{1991/07/15}{Renamed \file{babel.sty}
%    \file{babel.co
% \changes{germanb-2.3}{1991/11/05}{Rewritten parts of the code to
%    the new features of babel version 3
% \changes{germanb-2.3e}{1991/11/10}{Brought up-to-date w
%    \file{german.tex} v2.3e (plus some bug fixes) [b
% \changes{germanb-2.5}{1994/02/08}{Update or \LaTe
% \changes{germanb-2.5c}{1994/06/26}{Removed the use of \cs{fileda
%    and moved the identification after the loading
%    \file{babel.de
% \changes{germanb-2.6a}{1995/02/15}{Moved the identification to
%    top of the fi
% \changes{germanb-2.6a}{1995/02/15}{Rewrote the code that handles
%    active double quote charact
% \changes{germanb-2.6d}{1996/07/10}{Replaced \cs{undefined} w
%    \cs{@undefined} and \cs{empty} with \cs{@empty} for consiste
%    with \LaTe
% \changes{germanb-2.6d}{1996/10/10}{Moved the definition
%    \cs{atcatcode} right to the beginning

%  \section{The German langua

%    The file \file{\filename}\footnote{The file described in t
%    section has version number \fileversion\ and was last revised
%    \filedate.}  defines all the language definition macros for
%    German language as well as for the Austrian dialect of t
%    language\footnote{This file is a re-implementation of Hub
%    Partl's \file{german.sty} version 2.5b, see~\cite{HP}

%    For this language the character |"| is made active.
%    table~\ref{tab:german-quote} an overview is given of
%    purpose. One of the reasons for this is that in the Ger
%    language some character combinations change when a word is bro
%    between the combination. Also the vertical placement of
%    umlaut can be controlled this w
%    \begin{table}[h
%     \begin{cent
%     \begin{tabular}{lp{8c
%      |"a| & |\"a|, also implemented for the ot
%                  lowercase and uppercase vowels.
%      |"s| & to produce the German \ss{} (like |\ss{}|).
%      |"z| & to produce the German \ss{} (like |\ss{}|).
%      |"ck|& for |ck| to be hyphenated as |k-k|.
%      |"ff|& for |ff| to be hyphenated as |ff-
%                  this is also implemented for l, m, n, p, r and
%      |"S| & for |SS| to be |\uppercase{"s}|.
%      |"Z| & for |SZ| to be |\uppercase{"z}|.
%      \verb="|= & disable ligature at this position.
%      |"-| & an explicit hyphen sign, allowing hyphenat
%             in the rest of the word.
%      |""| & like |"-|, but producing no hyphen s
%             (for compund words with hyphen, e.g.\ |x-""y|).
%      |"~| & for a compound word mark without a breakpoint.
%      |"=| & for a compound word mark with a breakpoint, allow
%             hyphenation in the composing words.
%      |"`| & for German left double quotes (looks like ,,).
%      |"'| & for German right double quotes.
%      |"<| & for French left double quotes (similar to $<<$).
%      |">| & for French right double quotes (similar to $>>$).
%     \end{tabul
%     \caption{The extra definitions m
%              by \file{german.ldf}}\label{tab:german-quo
%     \end{cent
%    \end{tab
%    The quotes in table~\ref{tab:german-quote} can also be typeset
%    using the commands in table~\ref{tab:more-quot
%    \begin{table}[h
%     \begin{cent
%     \begin{tabular}{lp{8c
%      |\glqq| & for German left double quotes (looks like ,,).
%      |\grqq| & for German right double quotes (looks like ``).
%      |\glq|  & for German left single quotes (looks like ,).
%      |\grq|  & for German right single quotes (looks like `).
%      |\flqq| & for French left double quotes (similar to $<<$).
%      |\frqq| & for French right double quotes (similar to $>>$)
%      |\flq|  & for (French) left single quotes (similar to $<$).
%      |\frq|  & for (French) right single quotes (similar to $>$).
%      |\dq|   & the original quotes character (|"|).
%     \end{tabul
%     \caption{More commands which produce quotes, defi
%              by \file{german.ldf}}\label{tab:more-quo
%     \end{cent
%    \end{tab

% \StopEventuall

%    When this file was read through the option \Lopt{germanb} we m
%    it behave as if \Lopt{german} was specifi
% \changes{german-2.6l}{2008/03/17}{Making germanb behave like ger
%    needs some more work besides defining \cs{CurrentOptio
% \changes{germanb-2.6m}{2008/06/01}{Correted a ty
%    \begin{macroco
\def\bbl@tempa{germa
\ifx\CurrentOption\bbl@te
  \def\CurrentOption{germ
  \ifx\l@german\@undefi
    \@nopatterns{Germ
    \adddialect\l@germ

  \let\l@germanb\l@ger
  \AtBeginDocumen
    \let\captionsgermanb\captionsger
    \let\dategermanb\dateger
    \let\extrasgermanb\extrasger
    \let\noextrasgermanb\noextrasger


%    \end{macroco

%    The macro |\LdfInit| takes care of preventing that this file
%    loaded more than once, checking the category code of
%    \texttt{@} sign, e
% \changes{germanb-2.6d}{1996/11/02}{Now use \cs{LdfInit} to perf
%    initial check
%    \begin{macroco
%<*co
\LdfInit\CurrentOption{captions\CurrentOpti
%    \end{macroco

%    When this file is read as an option, i.e., by the |\usepacka
%    command, \texttt{german} will be an `unknown' language, so
%    have to make it known.  So we check for the existence
%    |\l@german| to see whether we have to do something he

% \changes{germanb-2.0}{1991/04/23}{Now use \cs{adddialect}
%    language undefin
% \changes{germanb-2.2d}{1991/10/27}{Removed use of \cs{@ifundefine
% \changes{germanb-2.3e}{1991/11/10}{Added warning, if no ger
%    patterns load
% \changes{germanb-2.5c}{1994/06/26}{Now use \cs{@nopatterns}
%    produce the warni
%    \begin{macroco
\ifx\l@german\@undefi
  \@nopatterns{Germ
  \adddialect\l@germ

%    \end{macroco

%    For the Austrian version of these definitions we just add anot
%    languag
% \changes{germanb-2.0}{1991/04/23}{Now use \cs{adddialect}
%    austri
%    \begin{macroco
\adddialect\l@austrian\l@ger
%    \end{macroco

%    The next step consists of defining commands to switch to (
%    from) the German langua

%  \begin{macro}{\captionsgerm
%  \begin{macro}{\captionsaustri
%    Either the macro |\captionsgerman| or the ma
%    |\captionsaustrian| will define all strings used in the f
%    standard document classes provided with \LaT

% \changes{germanb-2.2}{1991/06/06}{Removed \cs{global} definitio
% \changes{germanb-2.2}{1991/06/06}{\cs{pagename} should
%    \cs{headpagenam
% \changes{germanb-2.3e}{1991/11/10}{Added \cs{prefacenam
%    \cs{seename} and \cs{alsonam
% \changes{germanb-2.4}{1993/07/15}{\cs{headpagename} should
%    \cs{pagenam
% \changes{germanb-2.6b}{1995/07/04}{Added \cs{proofname}
%    AMS-\LaT
% \changes{germanb-2.6d}{1996/07/10}{Construct control sequence on
%    f
% \changes{germanb-2.6j}{2000/09/20}{Added \cs{glossarynam
%    \begin{macroco
\@namedef{captions\CurrentOption
  \def\prefacename{Vorwor
  \def\refname{Literatu
  \def\abstractname{Zusammenfassun
  \def\bibname{Literaturverzeichni
  \def\chaptername{Kapite
  \def\appendixname{Anhan
  \def\contentsname{Inhaltsverzeichnis}%    % oder nur: Inh
  \def\listfigurename{Abbildungsverzeichni
  \def\listtablename{Tabellenverzeichni
  \def\indexname{Inde
  \def\figurename{Abbildun
  \def\tablename{Tabelle}%                  % oder: Ta
  \def\partname{Tei
  \def\enclname{Anlage(n)}%                 % oder: Beilage
  \def\ccname{Verteiler}%                   % oder: Kopien
  \def\headtoname{A
  \def\pagename{Seit
  \def\seename{sieh
  \def\alsoname{siehe auc
  \def\proofname{Bewei
  \def\glossaryname{Glossa

%    \end{macroco
%  \end{mac
%  \end{mac

%  \begin{macro}{\dategerm
%    The macro |\dategerman| redefines the comm
%    |\today| to produce German dat
% \changes{germanb-2.3e}{1991/11/10}{Added \cs{month@germa
% \changes{germanb-2.6f}{1997/10/01}{Use \cs{edef} to def
%    \cs{today} to save memo
% \changes{germanb-2.6f}{1998/03/28}{use \cs{def} instead
%    \cs{ede
%    \begin{macroco
\def\month@german{\ifcase\month
  Januar\or Februar\or M\"arz\or April\or Mai\or Juni
  Juli\or August\or September\or Oktober\or November\or Dezember\
\def\dategerman{\def\today{\number\day.~\month@ger
    \space\number\yea
%    \end{macroco
%  \end{mac

%  \begin{macro}{\dateaustri
%    The macro |\dateaustrian| redefines the comm
%    |\today| to produce Austrian version of the German dat
% \changes{germanb-2.6f}{1997/10/01}{Use \cs{edef} to def
%    \cs{today} to save memo
% \changes{germanb-2.6f}{1998/03/28}{use \cs{def} instead
%    \cs{ede
%    \begin{macroco
\def\dateaustrian{\def\today{\number\day.~\ifnum1=\mo
  J\"anner\else \month@german\fi \space\number\yea
%    \end{macroco
%  \end{mac

%  \begin{macro}{\extrasgerm
%  \begin{macro}{\extrasaustri
% \changes{germanb-2.0b}{1991/05/29}{added some comment chars
%    prevent white spa
% \changes{germanb-2.2}{1991/06/11}{Save all redefined macr
%  \begin{macro}{\noextrasgerm
%  \begin{macro}{\noextrasaustri
% \changes{germanb-1.1}{1990/07/30}{Added \cs{dieresi
% \changes{germanb-2.0b}{1991/05/29}{added some comment chars
%    prevent white spa
% \changes{germanb-2.2}{1991/06/11}{Try to restore everything to
%    former sta
% \changes{germanb-2.6d}{1996/07/10}{Construct control seque
%    \cs{extrasgerman} or \cs{extrasaustrian} on the f

%    Either the macro |\extrasgerman| or the macros |\extrasaustri
%    will perform all the extra definitions needed for the Ger
%    language. The macro |\noextrasgerman| is used to cancel
%    actions of |\extrasgerman

%    For German (as well as for Dutch) the \texttt{"} character
%    made active. This is done once, later on its definition may va
%    \begin{macroco
\initiate@active@char
\@namedef{extras\CurrentOption
  \languageshorthands{germa
\expandafter\addto\csname extras\CurrentOption\endcsnam
  \bbl@activate{
%    \end{macroco
%    Don't forget to turn the shorthands off aga
% \changes{germanb-2.6i}{1999/12/16}{Deactivate shorthands ouside
%    Germ
%    \begin{macroco
\addto\noextrasgerman{\bbl@deactivate{
%    \end{macroco

% \changes{germanb-2.6a}{1995/02/15}{All the code to handle the act
%    double quote has been moved to \file{babel.de

%    In order for \TeX\ to be able to hyphenate German words wh
%    contain `\ss' (in the \texttt{OT1} position |^^Y|) we have
%    give the character a nonzero |\lccode| (see Appendix H, the \
%    boo
% \changes{germanb-2.6c}{1996/04/08}{Use decimal number instead
%    hat-notation as the hat may be activat
%    \begin{macroco
\expandafter\addto\csname extras\CurrentOption\endcsnam
  \babel@savevariable{\lccode2
  \lccode25=
%    \end{macroco
% \changes{germanb-2.6a}{1995/02/15}{Removeed \cs{3} as it is
%    longer in \file{german.ld

%    The umlaut accent macro |\"| is changed to lower the umlaut do
%    The redefinition is done with the help of |\umlautlo
%    \begin{macroco
\expandafter\addto\csname extras\CurrentOption\endcsnam
  \babel@save\"\umlautl
\@namedef{noextras\CurrentOption}{\umlauthi
%    \end{macroco
%    The german hyphenation patterns can be used with |\lefthyphenm
%    and |\righthyphenmin| set to
% \changes{germanb-2.6a}{1995/05/13}{use \cs{germanhyphenmins} to st
%    the correct valu
% \changes{germanb-2.6j}{2000/09/22}{Now use \cs{providehyphenmins}
%    provide a default val
%    \begin{macroco
\providehyphenmins{\CurrentOption}{\tw@\t
%    \end{macroco
%    For German texts we need to make sure that |\frenchspacing|
%    turned
% \changes{germanb-2.6k}{2001/01/26}{Turn frenchspacing on, as
%    \texttt{german.st
%    \begin{macroco
\expandafter\addto\csname extras\CurrentOption\endcsnam
  \bbl@frenchspaci
\expandafter\addto\csname noextras\CurrentOption\endcsnam
  \bbl@nonfrenchspaci
%    \end{macroco
%  \end{mac
%  \end{mac
%  \end{mac
%  \end{mac

% \changes{germanb-2.6a}{1995/02/15}{\cs{umlautlow}
%    \cs{umlauthigh} moved to \file{glyphs.dtx}, as well
%    \cs{newumlaut} (now \cs{lower@umlau

%    The code above is necessary because we need an extra act
%    character. This character is then used as indicated
%    table~\ref{tab:german-quot

%    To be able to define the function of |"|, we first defin
%    couple of `support' macr

% \changes{germanb-2.3e}{1991/11/10}{Added \cs{save@sf@q} macro
%    rewrote all quote macros to use
% \changes{germanb-2.3h}{1991/02/16}{moved definition
%    \cs{allowhyphens}, \cs{set@low@box} and \cs{save@sf@q}
%    \file{babel.co
% \changes{germanb-2.6a}{1995/02/15}{Moved all quotation characters
%    \file{glyphs.dt

%  \begin{macro}{\
%    We save the original double quote character in |\dq| to k
%    it available, the math accent |\"| can now be typed as |
%    \begin{macroco
\begingroup \catcode`\
\def\x{\endgr
  \def\@SS{\mathchar"701
  \def\dq{

%    \end{macroco
%  \end{mac
% \changes{germanb-2.6c}{1996/01/24}{Moved \cs{german@dq@disc}
%    babel.def, calling it \cs{bbl@dis

% \changes{germanb-2.6a}{1995/02/15}{Use \cs{ddot} instead
%    \cs{@MATHUMLAU

%    Now we can define the doublequote macros: the umlau
% \changes{germanb-2.6c}{1996/05/30}{added the \cs{allowhyphen
%    \begin{macroco
\declare@shorthand{german}{"a}{\textormath{\"{a}\allowhyphens}{\ddot
\declare@shorthand{german}{"o}{\textormath{\"{o}\allowhyphens}{\ddot
\declare@shorthand{german}{"u}{\textormath{\"{u}\allowhyphens}{\ddot
\declare@shorthand{german}{"A}{\textormath{\"{A}\allowhyphens}{\ddot
\declare@shorthand{german}{"O}{\textormath{\"{O}\allowhyphens}{\ddot
\declare@shorthand{german}{"U}{\textormath{\"{U}\allowhyphens}{\ddot
%    \end{macroco
%    trem
%    \begin{macroco
\declare@shorthand{german}{"e}{\textormath{\"{e}}{\ddot
\declare@shorthand{german}{"E}{\textormath{\"{E}}{\ddot
\declare@shorthand{german}{"i}{\textormath{\"{\i
                              {\ddot\imat
\declare@shorthand{german}{"I}{\textormath{\"{I}}{\ddot
%    \end{macroco
%    german es-zet (sharp
% \changes{germanb-2.6f}{1997/05/08}{use \cs{SS} instead
%    \texttt{SS}, removed braces after \cs{ss
%    \begin{macroco
\declare@shorthand{german}{"s}{\textormath{\ss}{\@SS{
\declare@shorthand{german}{"S}{\
\declare@shorthand{german}{"z}{\textormath{\ss}{\@SS{
\declare@shorthand{german}{"Z}{
%    \end{macroco
%    german and french quot
%    \begin{macroco
\declare@shorthand{german}{"`}{\gl
\declare@shorthand{german}{"'}{\gr
\declare@shorthand{german}{"<}{\fl
\declare@shorthand{german}{">}{\fr
%    \end{macroco
%    discretionary comma
%    \begin{macroco
\declare@shorthand{german}{"c}{\textormath{\bbl@disc ck}{
\declare@shorthand{german}{"C}{\textormath{\bbl@disc CK}{
\declare@shorthand{german}{"F}{\textormath{\bbl@disc F{FF}}{
\declare@shorthand{german}{"l}{\textormath{\bbl@disc l{ll}}{
\declare@shorthand{german}{"L}{\textormath{\bbl@disc L{LL}}{
\declare@shorthand{german}{"m}{\textormath{\bbl@disc m{mm}}{
\declare@shorthand{german}{"M}{\textormath{\bbl@disc M{MM}}{
\declare@shorthand{german}{"n}{\textormath{\bbl@disc n{nn}}{
\declare@shorthand{german}{"N}{\textormath{\bbl@disc N{NN}}{
\declare@shorthand{german}{"p}{\textormath{\bbl@disc p{pp}}{
\declare@shorthand{german}{"P}{\textormath{\bbl@disc P{PP}}{
\declare@shorthand{german}{"r}{\textormath{\bbl@disc r{rr}}{
\declare@shorthand{german}{"R}{\textormath{\bbl@disc R{RR}}{
\declare@shorthand{german}{"t}{\textormath{\bbl@disc t{tt}}{
\declare@shorthand{german}{"T}{\textormath{\bbl@disc T{TT}}{
%    \end{macroco
%    We need to treat |"f| a bit differently in order to preserve
%    ff-ligatur
% \changes{germanb-2.6f}{1998/06/15}{Copied the coding for \texttt{
%    from german.dtx version 2.5
%    \begin{macroco
\declare@shorthand{german}{"f}{\textormath{\bbl@discff}{
\def\bbl@discff{\penalty
  \afterassignment\bbl@insertff \let\bbl@nextff
\def\bbl@insertf
  \if f\bbl@nex
    \expandafter\@firstoftwo\else\expandafter\@secondoftwo
  {\relax\discretionary{ff-}{f}{ff}\allowhyphens}{f\bbl@nextf
\let\bbl@nextf
%    \end{macroco
%    and some additional comman
%    \begin{macroco
\declare@shorthand{german}{"-}{\nobreak\-\bbl@allowhyphe
\declare@shorthand{german}{"|
  \textormath{\penalty\@M\discretionary{-}{}{\kern.03e
              \allowhyphens}
\declare@shorthand{german}{""}{\hskip\z@sk
\declare@shorthand{german}{"~}{\textormath{\leavevmode\hbox{-}}{
\declare@shorthand{german}{"=}{\penalty\@M-\hskip\z@sk
%    \end{macroco

%  \begin{macro}{\mdq
%  \begin{macro}{\mdqo
%  \begin{macro}{\
%    All that's left to do now is to  define a couple of comma
%    for reasons of compatibility with \file{german.st
% \changes{germanb-2.6f}{1998/06/07}{Now use \cs{shorthandon}
%    \cs{shorthandoff
%    \begin{macroco
\def\mdqon{\shorthandon{
\def\mdqoff{\shorthandoff{
\def\ck{\allowhyphens\discretionary{k-}{k}{ck}\allowhyphe
%    \end{macroco
%  \end{mac
%  \end{mac
%  \end{mac

%    The macro |\ldf@finish| takes care of looking fo
%    configuration file, setting the main language to be switched
%    at |\begin{document}| and resetting the category code
%    \texttt{@} to its original val
% \changes{germanb-2.6d}{1996/11/02}{Now use \cs{ldf@finish} to w
%    u
%    \begin{macroco
\ldf@finish\CurrentOpt
%</co
%    \end{macroco

% \Fin

%% \CharacterTa
%%  {Upper-case    \A\B\C\D\E\F\G\H\I\J\K\L\M\N\O\P\Q\R\S\T\U\V\W\X\
%%   Lower-case    \a\b\c\d\e\f\g\h\i\j\k\l\m\n\o\p\q\r\s\t\u\v\w\x\
%%   Digits        \0\1\2\3\4\5\6\7\
%%   Exclamation   \!     Double quote  \"     Hash (number)
%%   Dollar        \$     Percent       \%     Ampersand
%%   Acute accent  \'     Left paren    \(     Right paren
%%   Asterisk      \*     Plus          \+     Comma
%%   Minus         \-     Point         \.     Solidus
%%   Colon         \:     Semicolon     \;     Less than
%%   Equals        \=     Greater than  \>     Question mark
%%   Commercial at \@     Left bracket  \[     Backslash
%%   Right bracket \]     Circumflex    \^     Underscore
%%   Grave accent  \`     Left brace    \{     Vertical bar
%%   Right brace   \}     Tilde

\endin
}
%    \end{macrocode}
% \changes{babel~3.5f}{1996/05/31}{Added the \Lopt{greek} option}
% \changes{babel~3.7a}{1997/11/13}{Added the \Lopt{polutonikogreek}
%    option}
% \changes{babel~3.7c}{1999/04/22}{set the correct language attribute
%    for polutoniko greek}
%    \begin{macrocode}
\DeclareOption{greek}{%%
%% This file will generate fast loadable files and documentation
%% driver files from the doc files in this package when run through
%% LaTeX or TeX.
%%
%% Copyright 1989-2007 Johannes L. Braams and any individual authors
%% listed elsewhere in this file.  All rights reserved.
%% 
%% This file is part of the Babel system.
%% --------------------------------------
%% 
%% It may be distributed and/or modified under the
%% conditions of the LaTeX Project Public License, either version 1.3
%% of this license or (at your option) any later version.
%% The latest version of this license is in
%%   http://www.latex-project.org/lppl.txt
%% and version 1.3 or later is part of all distributions of LaTeX
%% version 2003/12/01 or later.
%% 
%% This work has the LPPL maintenance status "maintained".
%% 
%% The Current Maintainer of this work is Johannes Braams.
%% 
%% The list of all files belonging to the LaTeX base distribution is
%% given in the file `manifest.bbl. See also `legal.bbl' for additional
%% information.
%% 
%% The list of derived (unpacked) files belonging to the distribution
%% and covered by LPPL is defined by the unpacking scripts (with
%% extension .ins) which are part of the distribution.
%%
%% --------------- start of docstrip commands ------------------
%%
\def\filedate{2007/10/20}
\def\batchfile{greek.ins}
\input docstrip.tex

{\ifx\generate\undefined
\Msg{**********************************************}
\Msg{*}
\Msg{* This installation requires docstrip}
\Msg{* version 2.3c or later.}
\Msg{*}
\Msg{* An older version of docstrip has been input}
\Msg{*}
\Msg{**********************************************}
\errhelp{Move or rename old docstrip.tex.}
\errmessage{Old docstrip in input path}
\batchmode
\csname @@end\endcsname
\fi}

\declarepreamble\mainpreamble
This is a generated file.

Copyright 1989-2007 Johannes L. Braams and any individual authors
listed elsewhere in this file.  All rights reserved.

This file was generated from file(s) of the Babel system.
---------------------------------------------------------

It may be distributed and/or modified under the
conditions of the LaTeX Project Public License, either version 1.3
of this license or (at your option) any later version.
The latest version of this license is in
  http://www.latex-project.org/lppl.txt
and version 1.3 or later is part of all distributions of LaTeX
version 2003/12/01 or later.

This work has the LPPL maintenance status "maintained".

The Current Maintainer of this work is Johannes Braams.

This file may only be distributed together with a copy of the Babel
system. You may however distribute the Babel system without
such generated files.

The list of all files belonging to the Babel distribution is
given in the file `manifest.bbl'. See also `legal.bbl for additional
information.

The list of derived (unpacked) files belonging to the distribution
and covered by LPPL is defined by the unpacking scripts (with
extension .ins) which are part of the distribution.
\endpreamble

\declarepreamble\fdpreamble
This is a generated file.

Copyright 1989-2005 Johannes L. Braams and any individual authors
listed elsewhere in this file.  All rights reserved.

This file was generated from file(s) of the Babel system.
---------------------------------------------------------

It may be distributed and/or modified under the
conditions of the LaTeX Project Public License, either version 1.3
of this license or (at your option) any later version.
The latest version of this license is in
  http://www.latex-project.org/lppl.txt
and version 1.3 or later is part of all distributions of LaTeX
version 2003/12/01 or later.

This work has the LPPL maintenance status "maintained".

The Current Maintainer of this work is Johannes Braams.

This file may only be distributed together with a copy of the Babel
system. You may however distribute the Babel system without
such generated files.

The list of all files belonging to the Babel distribution is
given in the file `manifest.bbl'. See also `legal.bbl for additional
information.

In particular, permission is granted to customize the declarations in
this file to serve the needs of your installation.

However, NO PERMISSION is granted to distribute a modified version
of this file under its original name.

\endpreamble

\keepsilent

\usedir{tex/generic/babel} 

\usepreamble\mainpreamble
\generate{\file{greek.ldf}{\from{greek.dtx}{code}}
          \file{athnum.sty}{\from{athnum.dtx}{package}}
          \file{grmath.sty}{\from{grmath.dtx}{package}}
          \file{grsymb.sty}{\from{grsymb.dtx}{package}}
          }
\usepreamble\fdpreamble
\generate{\file{lgrenc.def}{\from{greek.fdd}{LGRenc}}
          \file{lgrcmr.fd}{\from{greek.fdd}{fd,LGRcmr}}
          \file{lgrcmro.fd}{\from{greek.fdd}{fd,LGRcmro}}
          \file{lgrcmtt.fd}{\from{greek.fdd}{fd,LGRcmtt}}
          \file{lgrcmss.fd}{\from{greek.fdd}{fd,LGRcmss}}
          \file{lgrlcmtt.fd}{\from{greek.fdd}{fd,LGRlcmtt}}
          \file{lgrlcmss.fd}{\from{greek.fdd}{fd,LGRlcmss}}
          }

\ifToplevel{
\Msg{***********************************************************}
\Msg{*}
\Msg{* To finish the installation you have to move the following}
\Msg{* files into a directory searched by TeX:}
\Msg{*}
\Msg{* \space\space All *.def, *.fd, *.ldf, *.sty}
\Msg{*}
\Msg{* To produce the documentation run the files ending with}
\Msg{* '.dtx' and `.fdd' through LaTeX.}
\Msg{*}
\Msg{* Happy TeXing}
\Msg{***********************************************************}
}
 
\endinput
}
\DeclareOption{polutonikogreek}{%
  %%
%% This file will generate fast loadable files and documentation
%% driver files from the doc files in this package when run through
%% LaTeX or TeX.
%%
%% Copyright 1989-2007 Johannes L. Braams and any individual authors
%% listed elsewhere in this file.  All rights reserved.
%% 
%% This file is part of the Babel system.
%% --------------------------------------
%% 
%% It may be distributed and/or modified under the
%% conditions of the LaTeX Project Public License, either version 1.3
%% of this license or (at your option) any later version.
%% The latest version of this license is in
%%   http://www.latex-project.org/lppl.txt
%% and version 1.3 or later is part of all distributions of LaTeX
%% version 2003/12/01 or later.
%% 
%% This work has the LPPL maintenance status "maintained".
%% 
%% The Current Maintainer of this work is Johannes Braams.
%% 
%% The list of all files belonging to the LaTeX base distribution is
%% given in the file `manifest.bbl. See also `legal.bbl' for additional
%% information.
%% 
%% The list of derived (unpacked) files belonging to the distribution
%% and covered by LPPL is defined by the unpacking scripts (with
%% extension .ins) which are part of the distribution.
%%
%% --------------- start of docstrip commands ------------------
%%
\def\filedate{2007/10/20}
\def\batchfile{greek.ins}
\input docstrip.tex

{\ifx\generate\undefined
\Msg{**********************************************}
\Msg{*}
\Msg{* This installation requires docstrip}
\Msg{* version 2.3c or later.}
\Msg{*}
\Msg{* An older version of docstrip has been input}
\Msg{*}
\Msg{**********************************************}
\errhelp{Move or rename old docstrip.tex.}
\errmessage{Old docstrip in input path}
\batchmode
\csname @@end\endcsname
\fi}

\declarepreamble\mainpreamble
This is a generated file.

Copyright 1989-2007 Johannes L. Braams and any individual authors
listed elsewhere in this file.  All rights reserved.

This file was generated from file(s) of the Babel system.
---------------------------------------------------------

It may be distributed and/or modified under the
conditions of the LaTeX Project Public License, either version 1.3
of this license or (at your option) any later version.
The latest version of this license is in
  http://www.latex-project.org/lppl.txt
and version 1.3 or later is part of all distributions of LaTeX
version 2003/12/01 or later.

This work has the LPPL maintenance status "maintained".

The Current Maintainer of this work is Johannes Braams.

This file may only be distributed together with a copy of the Babel
system. You may however distribute the Babel system without
such generated files.

The list of all files belonging to the Babel distribution is
given in the file `manifest.bbl'. See also `legal.bbl for additional
information.

The list of derived (unpacked) files belonging to the distribution
and covered by LPPL is defined by the unpacking scripts (with
extension .ins) which are part of the distribution.
\endpreamble

\declarepreamble\fdpreamble
This is a generated file.

Copyright 1989-2005 Johannes L. Braams and any individual authors
listed elsewhere in this file.  All rights reserved.

This file was generated from file(s) of the Babel system.
---------------------------------------------------------

It may be distributed and/or modified under the
conditions of the LaTeX Project Public License, either version 1.3
of this license or (at your option) any later version.
The latest version of this license is in
  http://www.latex-project.org/lppl.txt
and version 1.3 or later is part of all distributions of LaTeX
version 2003/12/01 or later.

This work has the LPPL maintenance status "maintained".

The Current Maintainer of this work is Johannes Braams.

This file may only be distributed together with a copy of the Babel
system. You may however distribute the Babel system without
such generated files.

The list of all files belonging to the Babel distribution is
given in the file `manifest.bbl'. See also `legal.bbl for additional
information.

In particular, permission is granted to customize the declarations in
this file to serve the needs of your installation.

However, NO PERMISSION is granted to distribute a modified version
of this file under its original name.

\endpreamble

\keepsilent

\usedir{tex/generic/babel} 

\usepreamble\mainpreamble
\generate{\file{greek.ldf}{\from{greek.dtx}{code}}
          \file{athnum.sty}{\from{athnum.dtx}{package}}
          \file{grmath.sty}{\from{grmath.dtx}{package}}
          \file{grsymb.sty}{\from{grsymb.dtx}{package}}
          }
\usepreamble\fdpreamble
\generate{\file{lgrenc.def}{\from{greek.fdd}{LGRenc}}
          \file{lgrcmr.fd}{\from{greek.fdd}{fd,LGRcmr}}
          \file{lgrcmro.fd}{\from{greek.fdd}{fd,LGRcmro}}
          \file{lgrcmtt.fd}{\from{greek.fdd}{fd,LGRcmtt}}
          \file{lgrcmss.fd}{\from{greek.fdd}{fd,LGRcmss}}
          \file{lgrlcmtt.fd}{\from{greek.fdd}{fd,LGRlcmtt}}
          \file{lgrlcmss.fd}{\from{greek.fdd}{fd,LGRlcmss}}
          }

\ifToplevel{
\Msg{***********************************************************}
\Msg{*}
\Msg{* To finish the installation you have to move the following}
\Msg{* files into a directory searched by TeX:}
\Msg{*}
\Msg{* \space\space All *.def, *.fd, *.ldf, *.sty}
\Msg{*}
\Msg{* To produce the documentation run the files ending with}
\Msg{* '.dtx' and `.fdd' through LaTeX.}
\Msg{*}
\Msg{* Happy TeXing}
\Msg{***********************************************************}
}
 
\endinput
%
  \languageattribute{greek}{polutoniko}}
%    \end{macrocode}
% \changes{babel~3.7a}{1998/03/27}{Added the \Lopt{hebrew} option}
%    \begin{macrocode}
\DeclareOption{hebrew}{%
  \input{rlbabel.def}%
  % \iffalse meta-comment
%
% Copyright 1989-2005 Johannes L. Braams and any individual authors
% listed elsewhere in this file.  All rights reserved.
% 
% This file is part of the Babel system.
% --------------------------------------
% 
% It may be distributed and/or modified under the
% conditions of the LaTeX Project Public License, either version 1.3
% of this license or (at your option) any later version.
% The latest version of this license is in
%   http://www.latex-project.org/lppl.txt
% and version 1.3 or later is part of all distributions of LaTeX
% version 2003/12/01 or later.
% 
% This work has the LPPL maintenance status "maintained".
% 
% The Current Maintainer of this work is Johannes Braams.
% 
% The list of all files belonging to the Babel system is
% given in the file `manifest.bbl. See also `legal.bbl' for additional
% information.
% 
% The list of derived (unpacked) files belonging to the distribution
% and covered by LPPL is defined by the unpacking scripts (with
% extension .ins) which are part of the distribution.
% \fi
% \CheckSum{3345}
%
% \iffalse meta-comment
%% Hebrew language definition and additional packages.
%% Copyright (C) 1997 -- 2005 Boris Lavva.
%
%% Babel package for LaTeX version 2e
%% Copyright (C) 1989 -- 2005 by Johannes Braams,
%%                            TeXniek
%%                            All rights reserved.
%<*calendar>
%% TeX & LaTeX macros for computing Hebrew date from Gregorian one
%% Copyright (C) 1991 by Michail Rozman, misha@iop.tartu.ew.su
%%
%</calendar>
% \fi
%
%
% \iffalse
%<hebrew>\ProvidesFile{hebrew.ldf}
%<rightleft>\ProvidesFile{rlbabel.def}
%<calendar>\ProvidesPackage{hebcal}
%<*driver>
\ProvidesFile{hebrew.drv}
%</driver>
% \fi
% \ProvidesFile{hebrew.dtx}
        [2005/03/30 v2.3h %
% \iffalse
%<hebrew>         Hebrew language definition from the babel system
%<rightleft>         Right-to-Left support from the babel system
%<calendar>         Hebrew calendar
%<driver>         Driver file for hebrew support
% \fi
    Hebrew language support from the babel system]
%
% \iffalse
% \subsection{A driver for this document}
%
% The next bit of code contains the documentation driver file for
% \TeX{}, i.e., the file that will produce the documentation you are
% currently reading. It will be extracted from this file by the \dst{}
%  program.
%
%    \begin{macrocode}
%<*driver>
\documentclass{ltxdoc}
\providecommand\babel{\textsf{babel}}
\providecommand\file[1]{\texttt{#1}}
\makeatletter
%    \end{macrocode}
%
%    The code lines are numbered within sections,
%    \begin{macrocode}
\@addtoreset{CodelineNo}{section}
\renewcommand\theCodelineNo{%
  \reset@font\scriptsize\thesection.\arabic{CodelineNo}}
%    \end{macrocode}
%    which should also be visible in the index; hence this
%    redefinition of a macro from \file{doc.sty}.
%    \begin{macrocode}
\renewcommand\codeline@wrindex[1]{\if@filesw
        \immediate\write\@indexfile
            {\string\indexentry{#1}%
            {\number\c@section.\number\c@CodelineNo}}\fi}
%    \end{macrocode}
%
%    The glossary environment is used or the change log, but its
%    definition needs changing for this document.
%    \begin{macrocode}
\renewenvironment{theglossary}{%
    \glossary@prologue%
    \GlossaryParms \let\item\@idxitem \ignorespaces}%
   {}
\makeatother
\DisableCrossrefs
\CodelineIndex
\RecordChanges
\title{Hebrew language support from the \babel\ system}
\author{Boris Lavva}
\date{Printed \today}
\begin{document}
   \maketitle
   \tableofcontents
   \DocInput{hebrew.dtx}
   \DocInput{hebinp.dtx}
   \DocInput{hebrew.fdd}
   \DocInput{heb209.dtx}
   \clearpage
   \def\filename{index}
   \PrintIndex
   \clearpage
   \def\filename{changes}
   \PrintChanges
\end{document}
%</driver>
%    \end{macrocode}
% \fi
%
% \providecommand\babel{\textsf{babel}}
% \providecommand\dst{\textsc{docstrip}}
% \providecommand\file[1]{\texttt{#1}}
% \providecommand\pkg[1]{\texttt{#1}}
% \providecommand\XeT{X\kern-.125em\lower.5ex\hbox{E}\kern-.1667emT\@}
% \providecommand\scrunch{\setlength{\itemsep}{-.05in}}
% \GetFileInfo{hebrew.dtx}
%
% \changes{hebrew~0.1}{??/??/??}{%
%    Preliminary \LaTeX\ Hebrew option (by Sergio Fogel)}
% \changes{hebrew~0.2}{??/??/??}{%
%    Corrections and additions (by Rama Porrat)}
% \changes{hebrew~0.6}{??/??/??}{Additions (by Yael Dubinsky)}
% \changes{hebrew~1.2}{??/??/??}{%
%    Bilingual tables, penalties, documentation and more changes
%    (by Yaniv Bargury)}
% \changes{hebrew~1.30}{1992/05/15}{%
%    Font selection, various (by Alon Ziv)}
% \changes{hebrew~1.31}{1993/02/22}{Bug fixes (by Alon Ziv)}
% \changes{hebrew~1.32}{1993/03/10}{Made font-change command 
%    for numbers `\cs{protect}'ed (by Alon Ziv)}
% \changes{hebrew~1.33}{1993/03/11}{%
%    Made \cs{refstepcounter} work using \cs{@ltor} (by Alon Ziv)}
% \changes{hebrew~1.34}{1993/03/22}{%
%    Moved font loading to another file. Added \cs{mainsec}. 
%    Made all text strings be produced by control codes (similar to
%    \LaTeX 2.09 Mar '92). Fixed \cs{noindent} (by Alon Ziv)}
% \changes{hebrew~1.35}{1993/03/22}{%
%    Moved the texts to a file selected by the current encoding 
%    (by Alon Ziv)}
% \changes{hebrew~1.36}{1993/03/24}{Use \TeX\ tricks to redefine 
%    \cs{theXXXX} without keeping old definitions.
%    Use only \cs{@eng} for direction/font change (removed \cs{@ltor}).
%    Switched from use of \cs{mainsec} to code taken from \babel\
%    system (by Alon Ziv)}
% \changes{hebrew~1.37}{1993/03/28}{%
%    Use \cs{add@around} in defining font size commands. Small bug
%    fixes (by Alon Ziv)}
% \changes{hebrew~1.38}{1993/04/20}{%
%    \cs{everypar} changed so that \cs{noindent} works unmodified 
%    (by Alon Ziv, thanks to Chris Rowley)}
% \changes{hebrew~1.39}{1993/08/10}{%
%    Redefined primitive sectioning commands. Changed \cs{include} so
%    it finds \texttt{.h}, \texttt{.xet}, and \texttt{.ltx} files (no
%    extension needed). Reinstated use of \cs{@ltor} (by Alon Ziv)}
% \changes{hebrew~1.40}{1993/09/01}{Added the \cs{@brackets} hack
%    (by Alon Ziv)}
% \changes{hebrew~1.41}{1993/09/09}{%
%    Reworked towards using NFSS2. Changed some macro names to be more
%    logical: renamed \cs{@ltor} to \cs{@number}, \cs{@eng} to
%    \cs{@latin}, and (in \texttt{hebrew.ldf}) \cs{@heb} to
%    \cs{@hebrew} (by Alon Ziv)}
% \changes{hebrew~1.42}{1993/09/22}{%
%    Made list environments work better. Fixed thebibliography
%    environment (by Alon Ziv)}
% \changes{hebrew~2.0a}{1998/01/01}{%
%    Completely rewritten for \LaTeXe\ and \babel\ support. Various
%    input and font encodings (with NFSS2) are supported too. The
%    original \pkg{hebrew.sty} is divided to a number of packages and
%    definition files for better readability and extensibility. Added
%    some user- and code-level documentation inside the \texttt{.dtx}
%    and \texttt{.fdd} files, and \LaTeX -driven installation with
%    \pkg{hebrew.ins} (by Boris Lavva)}
% \changes{hebrew~2.1}{2000/11/23}{%
%    corrections from Sivan Toledo: sender name in letter, and section name in
%    headings. (by Tzafrir Cohen)}
% \changes{hebrew~2.2}{2000/12/11}{%
%    renamed hebrew letters to heb* (e.g.: alef renamed to hebalef)
%    (by Tzafrir Cohen)}
% \changes{hebrew~2.3}{2001/02/27}{
%    added several \cs{@ifclassloaded}\{slides\} to allow the use of the
%    slides class. (by Tzafrir Cohen)}
% \changes{hebrew~2.3a}{2001/07/09}{
%    The documentation should now be built fine (broken since at least 
%    2.1, and probably 2.0) (by Tzafrir Cohen)}
% \changes{hebrew~2.3b}{2001/08/16}{
%    minor clean-ups. The documentation builds now with no warnings.
%    Also removed \cs{R} from the caption macro (added in 2.1)
%    Added internal \cs{@ensure@L} and \cs{@ensure@R} 
%    (Is there a real need for them? Maybe should they be exposed?)
%    (by Tzafrir Cohen)}
% \changes{hebrew~2.3c}{2001/10/05}{
%    a temporary fix to the \cs{gim} macro. Should be replaced by stuff 
%    from hebcal.
%    (by Tzafrir Cohen)}
% \changes{hebrew~2.3d}{2002/01/04}{
%    Initial support for the prosper class. Added \cs{arabicnorl} .
%    (by Tzafrir Cohen)}
% \changes{hebrew~2.3e}{2002/08/09}{
%    Removing hebtech from this distriution (not relevant to babel),
%    added \cs{HeblatexEncoding}. some docs cleanup
%    (by Tzafrir Cohen)}
% \changes{hebrew~2.3f}{2002/12/26}{
%    redefined \cs{list} instead of redefining every environment 
%    that uses it. some pscolor handling, removed HeblatexEncoding 
%    (don't use 2.3e) (by Tzafrir Cohen)}
% \changes{hebrew~2.3g}{2003/06/05}{
%    Reimplemented the printing of Hebrew numerals and Hebrew
%    counters; modified \pkg{hebcal.sty} to use this implementation
%    when typesetting Hebrew dates; added option |full| to package
%    \pkg{hebcal}; also removed some gratuitous
%    spaces inserted by \pkg{hebcal.sty} by adding comment marks.
%    CAUTION: the changes to \pkg{hebcal.sty} make it dependent on
%    \pkg{babel} and not useable as a stand-alone package. Is this a
%    problem? (by Ron Artstein)}
%
% \section{The Hebrew language}\label{sec:hebrew}
%
%    The file \file{\filename}\footnote{The Hebrew language support
%    files described in this section have version number \fileversion\
%    and were last revised on \filedate.} provides the following
%    packages and files for Hebrew language support:
%    \begin{description}
%    \item[\file{hebrew.ldf}] file defines all the language-specific
%    macros for the Hebrew language. 
%    \item[\file{rlbabel.def}] file is used by |hebrew.ldf| for
%    bidirectional versions of the major \LaTeX{} commands and
%    environments. It is designed to be used with other right-to-left
%    languages, not only with Hebrew.
%    \item[\pkg{hebcal.sty}] package defines a set of macros for
%    computing Hebrew date from Gregorian one.
%    \end{description}
%
%    Additional Hebrew input and font encoding definition files that
%    should be included and used with \file{hebrew.ldf} are:
%    \begin{description}
%    \item[\file{hebinp.dtx}] provides Hebrew input encodings, such as
%          ISO 8859-8, MS Windows codepage 1255 or IBM PC codepage 862
%          (see Section~\ref{sec:hebinp} on page~\pageref{sec:hebinp}).
%    \item[\file{hebrew.fdd}] contains Hebrew font encodings, related
%          font definition files and \pkg{hebfont} package that
%          provides Hebrew font switching commands (see
%          Section~\ref{sec:hebfdd} on page~\pageref{sec:hebfdd} for
%          further details).
%    \end{description}
%
%    \LaTeX~2.09 compatibility files are included with
%    \file{heb209.dtx} and gives possibility to compile existing
%    \LaTeX~2.09 Hebrew documents with small (if any) changes (see
%    Section~\ref{sec:heb209} on page~\pageref{sec:heb209} for
%    details).
%
%    Finally, optional document class \pkg{hebtech} may be useful for
%    writing theses and dissertations in both Hebrew and English (and
%    any other languages included with \babel). It designed to meet
%    requirements of the Graduate School of the Technion --- Israel
%    Institute of Technology. 
%
%    \emph{As of version 2.3e hebtech is no longer distributed together
%    with heblatex. It should be part of a new "hebclasses" package}
%
% \subsection{Acknowledgement}
%
%    The following people have contributed to Hebrew package in one
%    way or another, knowingly or unknowingly. In alphabetical order:
%    Irina Abramovici, Yaniv Bargury, Yael Dubinsky, Sergio Fogel,
%    Dan Haran, Rama Porrat, Michail Rozman, Alon Ziv.
%
%    Tatiana Samoilov and Vitaly Surazhsky found a number of serious
%    bugs in preliminary version of Hebrew package.
%
%    A number of other people have contributed comments and
%    information. Specific contributions are acknowledged within the
%    document.
%
%    I want to thank my wife, Vita, and son, Mishka, for their
%    infinite love and patience.
%
%    If you made a contribution and I haven't mentioned it, don't
%    worry, it was an accident. I'm sorry. Just tell me and I will add
%    you to the next version.
%
% \StopEventually{}
%
% \subsection{The {\normalfont\dst{}} modules}
%
%    The following modules are used in the implementation to direct
%    \dst{} in generating external files:
% \begin{center}
% \begin{tabular}{@{}ll}
%   driver    & produce a documentation driver file \\[4pt]
%   hebrew    & produce Hebrew language support file\\
%   rightleft & create right-to-left support file\\
%   calendar  & create Hebrew calendar package
% \end{tabular}
% \end{center}
%    A typical \dst{} command file would then have entries like:
%    \begin{quote}
%       |\generateFile{hebrew.ldf}{t}{\from{hebrew.dtx}{hebrew}}|
%    \end{quote}
%
% \subsection{Hebrew language definitions}
%
%    The macro |\LdfInit| takes care of preventing that this file is
%    loaded more than once, checking the category code of the |@|
%    sign, etc.
%    \begin{macrocode}
%<*hebrew>
\LdfInit{hebrew}{captionshebrew}
%    \end{macrocode}
%
%    When this file is read as an option, i.e., by the |\usepackage|
%    command, |hebrew| will be an `unknown' language, in which case we
%    have to make it known. So we check for the existence of
%    |\l@hebrew| to see whether we have to do something here. 
%
%    \begin{macrocode}
\ifx\l@hebrew\@undefined
  \@nopatterns{Hebrew}%
  \adddialect\l@hebrew0
\fi
%    \end{macrocode}
%
%  \begin{macro}{\hebrewencoding}
%    \emph{FIX DOCS REGARDING 8BIT}
%
%    Typesetting Hebrew texts implies that a special input and output
%    encoding needs to be used. Generally, the user may choose
%    between different available Hebrew encodings provided. The
%    current support for Hebrew uses all available fonts from the
%    Hebrew University of Jerusalem encoded in `old-code' 7-bit
%    encoding also known as Israeli Standard SI-960. We define for
%    these fonts the Local Hebrew Encoding |LHE| (see the file
%    |hebrew.fdd| for more details), and the |LHE| encoding definition
%    file should be loaded by default.
%
%    Other fonts are available in windows-cp1255 (a superset of ISO-8859-8
%    with nikud). For those, the encoding |HE8| should be used. Such fonts
%    are, e.g., windows' TrueType fonts (once cnverted to Type1 or MetaFont)
%    and IBM's Type1 fonts.
%
%    However, if an user wants to use another font encoding, for
%    example, cyrillic encoding T2 and extended latin encoding T1, ---
%    he/she has to load the corresponding file \emph{before} the
%    \pkg{hebrew} package. This may be done in the following way:
%    \begin{quote}
%      |\usepackage[LHE,T2,T1]{fontenc}|\\
%      |\usepackage[hebrew,russian,english]{babel}|
%    \end{quote}
%    We make sure that the |LHE| encoding is known to \LaTeX{} at end
%    of this package.
%
%    Also note that if you want to use the encoding |HE8| , you should define 
%    the following in your document, \emph{before loading babel}:
%    \begin{quote}
%      |\def\HeblatexEncoding{HE8}|\\
%      |\def\HeblatexEncodingFile{he8enc}|
%    \end{quote}
% \changes{hebrew-2.3h}{2004/02/20}{Make LHE the default encoding for
%    compatibility reasons}
%    \begin{macrocode}
\providecommand{\HeblatexEncoding}{LHE}%
\providecommand{\HeblatexEncodingFile}{lheenc}%
\newcommand{\heblatex@set@encoding}[2]{
}
\AtEndOfPackage{%
  \@ifpackageloaded{fontenc}{%
    \@ifl@aded{def}{%
      \HeblatexEncodingFile}{\def\hebrewencoding{\HeblatexEncoding}}{}%
  }{%
    \input{\HeblatexEncodingFile.def}%
    \def\hebrewencoding{\HeblatexEncoding}%
  }}
%    \end{macrocode}
%  \end{macro}
%
%    We also need to load inputenc package with one of the Hebrew
%    input encodings. By default, we set up the |8859-8| codepage.
%    If an user wants to use many input encodings in the same
%    document, for example, the MS Windows Hebrew codepage |cp1255|
%    and the standard IBM PC Russian codepage |cp866|, he/she has to
%    load the corresponding file \emph{before} the hebrew package
%    too. This may be done in the following way:
%    \begin{quote}
%      |\usepackage[cp1255,cp866]{inputenc}|\\
%      |\usepackage[hebrew,russian,english]{babel}|
%    \end{quote}
%
%    An user can switch input encodings in the document using the
%    command |\inputencoding|, for example, to use the |cp1255|:
%    \begin{quote}
%       |\inputencoding{cp1255}|
%    \end{quote}
%    \begin{macrocode}
\AtEndOfPackage{%
  \@ifpackageloaded{inputenc}{}{\RequirePackage[8859-8]{inputenc}}}
%    \end{macrocode}
%
%    The next step consists of defining commands to switch to (and
%    from) the Hebrew language.
%
%  \begin{macro}{\hebrewhyphenmins}
%    This macro is used to store the correct values of the hyphenation
%    parameters |\lefthyphenmin| and |\righthyphenmin|. They are set
%    to~2.
% \changes{hebrew~2.0b}{2000/09/22}{Now use \cs{providehyphenmins} to
%    provide a default value}
%    \begin{macrocode}
\providehyphenmins{\CurrentOption}{\tw@\tw@}
%    \end{macrocode}
%  \end{macro}
%
% \begin{macro}{\captionshebrew}
%    The macro |\captionshebrew| replaces all captions used in the four
%    standard document classes provided with \LaTeXe with their Hebrew
%    equivalents.
% \changes{hebrew-2.0b}{2000/09/20}{Added \cs{glossaryname}}
%    \begin{macrocode}
\addto\captionshebrew{%
  \def\prefacename{\@ensure@R{\hebmem\hebbet\hebvav\hebalef}}%
  \def\refname{\@ensure@R{\hebresh\hebshin\hebyod\hebmem\hebtav\ %
    \hebmem\hebqof\hebvav\hebresh\hebvav\hebtav}}%
  \def\abstractname{\@ensure@R{\hebtav\hebqof\hebtsadi\hebyod\hebresh}}%
  \def\bibname{\@ensure@R{\hebbet\hebyod\hebbet\heblamed\hebyod\hebvav%
    \hebgimel\hebresh\hebpe\hebyod\hebhe}}%
  \def\chaptername{\@ensure@R{\hebpe\hebresh\hebqof}}%
  \def\appendixname{\@ensure@R{\hebnun\hebsamekh\hebpe\hebhet}}%
  \def\contentsname{\@ensure@R{%
    \hebtav\hebvav\hebkaf\hebfinalnun\ %
    \hebayin\hebnun\hebyod\hebyod\hebnun\hebyod\hebfinalmem}}%
  \def\listfigurename{\@ensure@R{%
    \hebresh\hebshin\hebyod\hebmem\hebtav\ %
    \hebalef\hebyod\hebvav\hebresh\hebyod\hebfinalmem}}%
  \def\listtablename{\@ensure@R{%
    \hebresh\hebshin\hebyod\hebmem\hebtav\
    \hebtet\hebbet\heblamed\hebalef\hebvav\hebtav}}%
  \def\indexname{\@ensure@R{\hebmem\hebpe\hebtav\hebhet}}%
  \def\figurename{\@ensure@R{\hebalef\hebyod\hebvav\hebresh}}%
  \def\tablename{\@ensure@R{\hebtet\hebbet\heblamed\hebhe}}%
  \def\partname{\@ensure@R{\hebhet\heblamed\hebqof}}%
  \def\enclname{\@ensure@R{\hebresh\hebtsadi"\hebbet}}%
  \def\ccname{\@ensure@R{\hebhe\hebayin\hebtav\hebqof\hebyod%
    \hebfinalmem}}%
  \def\headtoname{\@ensure@R{\hebalef\heblamed}}%
  \def\pagename{\@ensure@R{\hebayin\hebmem\hebvav\hebdalet}}%
  \def\psname{\@ensure@R{\hebnun.\hebbet.}}%
  \def\seename{\@ensure@R{\hebresh\hebalef\hebhe}}%
  \def\alsoname{\@ensure@R{\hebresh\hebalef\hebhe \hebgimel%
    \hebmemesof}}%
  \def\proofname{\@ensure@R{\hebhe\hebvav\hebkaf\hebhet\hebhe}}
  \def\glossaryname{\@ensure@L{Glossary}}% <-- Needs translation
}
%    \end{macrocode}
% \end{macro}
%  \begin{macro}{\slidelabel}
%    Here we fix the macro |slidelabel| of the seminar package. Note 
%    that this still won't work well enough when overlays will be 
%    involved
%    \begin{macrocode}
\@ifclassloaded{seminar}{%
  \def\slidelabel{\bf \if@rl\R{\hebshin\hebqof\hebfinalpe{} \theslide}%
                      \else\L{Slide \theslide}%
                      \fi}%
}{}
%    \end{macrocode}
% \end{macro}
%
%    Here we provide an user with translation of Gregorian dates
%    to Hebrew. In addition, the \pkg{hebcal} package can be used
%    to create Hebrew calendar dates.
%
%  \begin{macro}{\hebmonth}
%    The macro |\hebmonth{|\emph{month}|}| produces month names in
%    Hebrew.
%    \begin{macrocode}
\def\hebmonth#1{%
  \ifcase#1\or \hebyod\hebnun\hebvav\hebalef\hebresh\or %
     \hebpe\hebbet\hebresh\hebvav\hebalef\hebresh\or %
     \hebmem\hebresh\hebfinaltsadi\or %
     \hebalef\hebpe\hebresh\hebyod\heblamed\or %
     \hebmem\hebalef\hebyod\or \hebyod\hebvav\hebnun\hebyod\or %
     \hebyod\hebvav\heblamed\hebyod\or %
     \hebalef\hebvav\hebgimel\hebvav\hebsamekh\hebtet\or %
     \hebsamekh\hebpe\hebtet\hebmem\hebbet\hebresh\or %
     \hebalef\hebvav\hebqof\hebtet\hebvav\hebbet\hebresh\or %
     \hebnun\hebvav\hebbet\hebmem\hebbet\hebresh\or %
     \hebdalet\hebtsadi\hebmem\hebbet\hebresh\fi}
%    \end{macrocode}
%  \end{macro}
%
%  \begin{macro}{\hebdate}
%    The macro |\hebdate{|\emph{day}|}{|\emph{month}|}{|\emph{year}|}|
%    translates a given Gregorian date to Hebrew.
%    \begin{macrocode}
\def\hebdate#1#2#3{%
  \beginR\beginL\number#1\endL\ \hebbet\hebmonth{#2}
         \beginL\number#3\endL\endR}
%    \end{macrocode}
%  \end{macro}
%
%  \begin{macro}{\hebday}
%    The macro |\hebday| will replace |\today| command when in Hebrew
%    mode.
%    \begin{macrocode}
\def\hebday{\hebdate{\day}{\month}{\year}}
%    \end{macrocode}
%  \end{macro}
%
% \begin{macro}{\datehebrew}
%    The macro |\datehebrew| redefines the command |\today| to produce
%    Gregorian dates in Hebrew. It uses the macro |\hebday|.
%    \begin{macrocode}
\def\datehebrew{\let\today=\hebday}
%    \end{macrocode}
% \end{macro}
%
%    The macro |\extrashebrew| will perform all the extra definitions
%    needed for the Hebrew language. The macro |\noextrashebrew|
%    is used to cancel the actions of |\extrashebrew|.
%
% \begin{macro}{\extrashebrew}
%    We switch font encoding to Hebrew and direction to
%    right-to-left. We cannot use the regular language switching
%    commands (for example, |\sethebrew| and |\unsethebrew| or
%    |\selectlanguage{hebrew}|), when in restricted horizontal mode,
%    because it will result in \emph{unbalanced} |\beginR| or
%    |\beginL| primitives.
%    Instead, in \TeX 's restricted horizontal mode, the
%    |\L{|\emph{latin text}|}| and |\R{|\emph{hebrew text}|}|, or
%    |\embox{|\emph{latin text}|}| and |\hmbox{|\emph{hebrew text}|}|
%    should be used.
%
%    Hence, we use |\beginR| and |\beginL| switching commands only
%    when not in restricted horizontal mode.
%    \begin{macrocode}
\addto\extrashebrew{%
  \tohebrew%
  \ifhmode\ifinner\else\beginR\fi\fi}
%    \end{macrocode}
% \end{macro}
%
% \begin{macro}{\noextrashebrew}
%    The macro |\noextrashebrew| is used to cancel the actions of
%    |\extrashebrew|. We switch back to the previous font encoding and
%    restore left-to-right direction.
%    \begin{macrocode}
\addto\noextrashebrew{%
  \fromhebrew%
  \ifhmode\ifinner\else\beginL\fi\fi}
%    \end{macrocode}
% \end{macro}
%
%    Generally, we can switch to- and from- Hebrew by means of
%    standard \babel -defined commands, for example,
%    \begin{quote}
%       |\selectlanguage{hebrew}|
%    \end{quote}
%    or
%    \begin{quote}
%       |\begin{otherlanguage}{hebrew}|\\
%       \hspace*{1.5em} some Hebrew text\\
%       |\end{otherlanguage}|
%    \end{quote}
%    Now we define two additional commands that offer the possibility
%    to switch to and from Hebrew language. These commands are
%    backward compatible with the previous versions of
%    \pkg{hebrew.sty}.
%
%  \begin{macro}{\sethebrew}
%  \begin{macro}{\unsethebrew}
%    The command |\sethebrew| will switch from the current font encoding
%    to the hebrew font encoding, and from the current direction of
%    text to the right-to-left mode. The command |\unsethebrew| switches
%    back.
%
%    Both commands use standard right-to-left switching macros
%    |\setrllanguage{|\emph{ r2l language name}|}| and
%    |\unsetrllanguage{|\emph{r2l language name}|}|, that
%    defined in the \file{rlbabel.def} file.
%    \begin{macrocode}
\def\sethebrew{\setrllanguage{hebrew}}
\def\unsethebrew{\unsetrllanguage{hebrew}}
%    \end{macrocode}
%  \end{macro}
%  \end{macro}
%
%  \begin{macro}{\hebrewtext}
%  \begin{macro}{\nohebrewtext}
%    The following two commands are \emph{obsolete} and work only
%    in \LaTeX 2.09 compatibility mode. They are synonyms of
%    |\sethebrew| and |\unsethebrew| defined above.
%    \begin{macrocode}
\if@compatibility
  \let\hebrewtext=\sethebrew
  \let\nohebrewtext=\unsethebrew
\fi
%    \end{macrocode}
%  \end{macro}
%  \end{macro}
%
%  \begin{macro}{\tohebrew}
%  \begin{macro}{\fromhebrew}
%    These two commands change only the current font encoding to- and
%    from- Hebrew encoding. Their implementation uses
%    |\@torl{|\emph{language name}|}| and |\@fromrl| macros defined in
%    \file{rlbabel.def} file. Both commands may be useful \emph{only}
%    for package and class writers, not for regular users.
%    \begin{macrocode}
\def\tohebrew{\@torl{hebrew}}%
\def\fromhebrew{\@fromrl}
%    \end{macrocode}
%  \end{macro}
%  \end{macro}
%
%  \begin{macro}{\@hebrew}
%    Sometimes we need to preserve Hebrew mode without knowing in
%    which environment we are located now. For these cases, the
%    |\@hebrew{|\emph{hebrew text}|}| macro will be useful. Not that
%    this macro is similar to the |\@number| and |\@latin| macros
%    defined in \file{rlbabel.def} file.
%    \begin{macrocode}
\def\@@hebrew#1{\beginR{{\tohebrew#1}}\endR}
\def\@hebrew{\protect\@@hebrew}
%    \end{macrocode}
%  \end{macro}
%
%  \subsubsection{Hebrew numerals}
%
%    We provide commands to print numbers in the traditional
%    notation using Hebrew letters. We need commands that print 
%    a Hebrew number from a decimal input, as well as commands 
%    to print the value of a counter as a Hebrew number.  
%  \begin{macro}{\if@gim@apost}
%  \begin{macro}{\if@gim@final}
%    Hebrew numbers can be written in various styles: with or without
%    apostrophes, and with the letters kaf, mem, nun, pe, tsadi as either
%    final or initial forms when they are the last letters in the
%    sequence. We provide two flags to set the style options.
%    \begin{macrocode}
\newif\if@gim@apost  % whether we print apostrophes
\newif\if@gim@final  % whether we use final or initial letters
%    \end{macrocode}
%  \end{macro}
%  \end{macro}
%  \begin{macro}{\hebrewnumeral}
%  \begin{macro}{\Hebrewnumeral}
%  \begin{macro}{\Hebrewnumeralfinal}
%    The commands that print a Hebrew number 
%    must specify the style locally: relying on a global style
%    option could cause a counter to
%    print in an inconsistent manner---for instance, page numbers
%    might appear in different styles if the global style option
%    changed mid-way through a document.
%    The commands only allow three of the four possible flag
%    combinations (I do not know of a use that requires the
%    combination of final letters and no apostrophes --RA).
%
%    Each command sets the style flags and calls |\@hebrew@numeral|.
%    Double braces are used in order to protect the values of
%    |\@tempcnta| and |\@tempcntb|, which are changed by this call;
%    they also keep the flag assignments local (this is not important
%    because the global values are never used).
%    \begin{macrocode}
\newcommand*{\hebrewnumeral}[1]      % no apostrophe, no final letters
 {{\@gim@finalfalse\@gim@apostfalse\@hebrew@numeral{#1}}}
\newcommand*{\Hebrewnumeral}[1]      % apostrophe, no final letters
 {{\@gim@finalfalse\@gim@aposttrue\@hebrew@numeral{#1}}}
\newcommand*{\Hebrewnumeralfinal}[1] % apostrophe, final letters
 {{\@gim@finaltrue\@gim@aposttrue\@hebrew@numeral{#1}}}
%    \end{macrocode}
%  \end{macro}
%  \end{macro}
%  \end{macro}
%  \begin{macro}{\alph}
%  \begin{macro}{\@alph}
%  \begin{macro}{\Alph}
%  \begin{macro}{\@Alph}
%  \begin{macro}{\Alphfinal}
%  \begin{macro}{\@Alphfinal}
%    Counter-printing commands are based on the above commands. The
%    natural name for the counter-printing commands is |\alph|, because
%    Hebrew numerals are the only way to represent numbers with
%    Hebrew letters (kaf always means~20, never~11). Hebrew has no
%    uppercase letters, hence no need for the familiar meaning of |\Alph|;
%    we therefore define |\alph| to print counters as Hebrew numerals
%    without apostrophes, and |\Alph| to print with apostrophes. A third
%    form, |\Alphfinal|, is provided to print with apostrophes and final
%    letters, as is required for Hebrew year designators. The commands
%    |\alph| and |\Alph| are defined in \pkg{latex.ltx}, and we only
%    need to redefine the internal commands |\@alph| and 
%    |\@Alph|; for |\Alphfinal| we need to provide both a wrapper and
%    an internal command. 
%    The counter printing commands are made semi-robust: without the
%    |\protect|, commands like |\theenumii| break (I'm not quite clear
%    on why this happens, --RA); at the same time, we cannot make the 
%    commands too robust (e.g.~with |\DeclareRobustCommand|) because
%    this would enter the command name rather than its value into
%    files like |.aux|, |.toc| etc\@.
%    The old meanings of meaning of |\@alph| and |\@Alph| are saved
%    upon entering Hebrew mode and restored upon exiting it.
%    \begin{macrocode}
\addto\extrashebrew{%
  \let\saved@alph=\@alph%
  \let\saved@Alph=\@Alph%
  \renewcommand*{\@alph}[1]{\protect\hebrewnumeral{\number#1}}%
  \renewcommand*{\@Alph}[1]{\protect\Hebrewnumeral{\number#1}}%
  \def\Alphfinal#1{\expandafter\@Alphfinal\csname c@#1\endcsname}%
  \providecommand*{\@Alphfinal}[1]{\protect\Hebrewnumeralfinal{\number#1}}}
\addto\noextrashebrew{%
  \let\@alph=\saved@alph%
  \let\@Alph=\saved@Alph}
%    \end{macrocode}
%    Note that |\alph| (without apostrophes) is already the
%    appropriate choice for the second-level enumerate label, and
%    |\Alph| (with apostrophes) is an appropriate choice for appendix;
%    however, the default \LaTeX\ labels need to be redefined for
%    appropriate cross-referencing, see below.
%    \LaTeX\ default class files specify |\Alph| for
%    the fourth-level enumerate level, this should probably be changed.
%    Also, the way labels get flushed left by default looks inappropriate
%    for Hebrew numerals, so we should redefine |\labelenumii| as well
%    as |\labelenumiv| (presently not implemented).
%  \end{macro}
%  \end{macro}
%  \end{macro}
%  \end{macro}
%  \end{macro}
%  \end{macro}
%  \begin{macro}{\theenumii}
%  \begin{macro}{\theenumiv}
%  \begin{macro}{\label}
%    Cross-references to counter labels need to be printed according
%    to the language environment in which a label was issued, not
%    the environment in which it is called: for example, a label~(1b) 
%    issued in a Latin environment should be referred to as~(1b) in a
%    Hebrew text, and label~(2dalet) issued in a Hebrew environment
%    should be referred to as~(2dalet) in a Latin text. This was the
%    unanimous opinion in a poll sent to the Ivri\TeX\ list. 
%    We therefore redefine |\theenumii| and |\theenumiv|, so that an
%    explicit language instruction gets written to the |.aux| file.
%    \begin{macrocode}
\renewcommand{\theenumii}
   {\if@rl\protect\hebrewnumeral{\number\c@enumii}%
    \else\protect\L{\protect\@@alph{\number\c@enumii}}\fi}
\renewcommand{\theenumiv}
   {\if@rl\protect\Hebrewnumeral{\number\c@enumiv}%
    \else\protect\L{\protect\@@Alph{\number\c@enumiv}}\fi}
%    \end{macrocode}
%    We also need to control for the font and direction in which a
%    counter label is printed. Direction is straightforward: a Latin
%    label like~(1b) should be written left-to-right when called in a
%    Hebrew text, and a Hebrew label like~(2dalet) should be written
%    right-to-left when called in a Latin text. The font question is
%    more delicate, because we should decide whether the numerals
%    should be typeset in the font of the language enviroment in which
%    the label was issued, or that of the environment in which it is
%    called. 
%    \begin{itemize}
%     \item
%      A purely numeric label like~(23) looks best if it is set in the
%      font of the surrounding language.
%     \item
%      But a mixed alphanumeric label like~(1b) lookes weird if
%      the~`1' is taken from the Hebrew font; likewise, (2dalet) looks
%      weird if the~`2' is taken from a Latin font.
%     \item
%      Finally, mixing the two possibilities is worst, because a
%      single Hebrew sentence referring to examples~(1b) and~(2) would
%      take the~`1' from the Latin font and the~`2' from the Hebrew
%      font, and this looks really awful. (It is also very hard to
%      implement). 
%    \end{itemize}
%    In light of the conflicting considerations it seems like there's
%    no perfect solution. I have chosen to implement the top option,
%    where numerals are taken from the font of the surrounding
%    language, because it seems to me that reference to purely numeric
%    labels is the most common, so this gives a good solution to the
%    majority of cases and a mediocre solution to the minority.
%
%    We redefine the |\label| command which writes to the
%    |.aux| file. Depending on the language environment we issue
%    appropriate |\beginR/L|$\cdots$|\endR/L| commands to control the
%    direction without affecting the font. Since these commands do not
%    affect the value of |\if@rl|, we cannot use the macro
%    |\@brackets| to determine the correct brackets to be used with
%    |\p@enumiii|; instead, we let the language environment determine an
%    explicit definition.
%    \begin{macrocode}
\def\label#1{\@bsphack
  \if@rl
    \def\p@enumiii{\p@enumii)\theenumii(}%
    \protected@write\@auxout{}%
         {\string\newlabel{#1}{{\beginR\@currentlabel\endR}{\thepage}}}%
  \else
    \def\p@enumiii{\p@enumii(\theenumii)}%
    \protected@write\@auxout{}%
         {\string\newlabel{#1}{{\beginL\@currentlabel\endL}{\thepage}}}%
  \fi
  \@esphack}
%    \end{macrocode}
%    NOTE: it appears that the definition of |\label| is
%    language-independent and thus belongs in \pkg{rlbabel.def}, but
%    this is not the case. The decision to typeset label numerals
%    in the font of the surrounding language is reasonable for Hebrew,
%    because mixed-font (1b) and (2dalet) are somewhat acceptable. The
%    same may not be acceptable for Arabic, whose numeral glyphs are
%    radically different from those in the Latin fonts. The decision
%    about the direction may also be different for Arabic, which is
%    more right-to-left oriented than Hebrew (two examples: dates like
%    15/6/2003 are written left-to-right in Hebrew but right-to-left
%    in Arabic; equations like $1+2=3$ are written left-to-right in
%    Hebrew but right-to-left in Arabic elementary school textbooks
%    using Arabic numeral glyphs). My personal hunch is that a label
%    like~(1b) in an Arabic text would be typeset left-to-right if
%    the~`1' is a Western glyph, but right-to-left if the~`1' is an
%    Arabic glyph. But this is just a guess, I'd have to ask Arab
%    typesetters to find the correct answer. --RA.
%  \end{macro}
%  \end{macro}
%  \end{macro}
%  \begin{macro}{\appendix}
%    The following code provides for the proper printing of appendix
%    numbers in tables of contents. Section and chapter headings are
%    normally bilingual: regardless of the text language, the author
%    supplies each section/chapter with two headings---one for the
%    Hebrew table of contents and one for the Latin table of contents.
%    It makes sense that the label should be a Latin letter in the
%    Latin table of contents and a Hebrew letter in the Hebrew table
%    of contents. The definition is similar to that of |\theenumii|
%    and |\theenumiv| above, but additional |\protect| commands ensure
%    that the entire condition is written the |.aux| file. The
%    appendix number will therefore be typeset according to the
%    environment in which it is used rather than issued: a Hebrew
%    number (with apostrophes) in a Hebrew environment and a Latin
%    capital letter in a Latin environment (the command 
%    |\@@Alph| is set in \pkg{rlbabel.def} to hold the default meaning
%    of \LaTeX\ [latin] |\@Alph|, regardless of the mode in which it is
%    issued). The net result is that
%    the second appendix will be marked with~`B' in the Latin table of
%    contents and with `bet' in the Hebrew table of contents; the mark
%    in the main text will depend on the language of the appendix itself.
%    \begin{macrocode}
\@ifclassloaded{letter}{}{%
\@ifclassloaded{slides}{}{%
  \let\@@appendix=\appendix%
  \@ifclassloaded{article}{%
    \renewcommand\appendix{\@@appendix%
      \renewcommand\thesection
        {\protect\if@rl\protect\Hebrewnumeral{\number\c@section}%
         \protect\else\@@Alph\c@section\protect\fi}}}
   {\renewcommand\appendix{\@@appendix%
      \renewcommand\thechapter
        {\protect\if@rl\protect\Hebrewnumeral{\number\c@chapter}%
         \protect\else\@@Alph\c@chapter\protect\fi}}}}}
%    \end{macrocode}
%    QUESTION: is this also the appropriate way to refer to an
%    appendix in the text, or should we retain the original label the
%    same way we did with |enumerate| labels? 
%    ANOTHER QUESTION: are similar redefinitions needed for other
%    counters that generate texts in bilingual lists like |.lof/.fol|
%    and |.lot/.tol|? --RA.
%  \end{macro}
%  \begin{macro}{\@hebrew@numeral}
%    The command |\@hebrew@numeral| prints a Hebrew number. The groups
%    of thousands, millions, billions are separated by apostrophes and
%    typeset without apostrophes or final letters; the remainder
%    (under 1000) is typeset conventionally, with the selected styles
%    for apostrophes and final letters. 
%    The function calls on |\gim@no@mil| to typeset each
%    three-digit block. The algorithm 
%    is recursive, but the maximum recursion depth is~4 because \TeX\
%    only allows numbers up to $2^{31}-1 = 2{,}147{,}483{,}647$.
%    The typesetting routine is wrapped in |\@hebrew| in order to
%    ensure that numbers are always typeset in Hebrew mode.
%
%    Initialize: |\@tempcnta| holds the value, |\@tempcntb| is used for
%    calculations.
%    \begin{macrocode}
\newcommand*{\@hebrew@numeral}[1]
{\@hebrew{\@tempcnta=#1\@tempcntb=#1\relax
 \divide\@tempcntb by 1000
%    \end{macrocode}
%    If we're under 1000, call |\gim@nomil|
%    \begin{macrocode}
 \ifnum\@tempcntb=0\gim@nomil\@tempcnta\relax
%    \end{macrocode}
%    If we're above 1000 then force no apostrophe and no final letter
%    styles for the value above~1000, recur for the value above~1000,
%    add an apostrophe, and call |\gim@nomil| for the remainder.
%    \begin{macrocode}
 \else{\@gim@apostfalse\@gim@finalfalse\@hebrew@numeral\@tempcntb}'%
      \multiply\@tempcntb by 1000\relax
      \advance\@tempcnta by -\@tempcntb\relax
      \gim@nomil\@tempcnta\relax
 \fi
}}
%    \end{macrocode}
%    NOTE: is it the case that 15,000 and 16,000 are written as
%    yod-he and yod-vav, rather than tet-vav and tet-zayin? This
%    vaguely rings a bell, but I'm not certain. If this is the case,
%    then the current behavior is incorrect and should be changed. --RA.
%  \end{macro}
%  \begin{macro}{\gim@nomil}
%    The command |\gim@nomil| typesets an integer between 0~and~999
%    (for~0 it typesets nothing). The code has been modified from the
%    old |hebcal.sty|
%    (appropriate credits---Boris Lavva and Michail Rozman ?).
%    |\@tempcnta| holds the total value that remains to be typeset.
%    At each stage we find the highest valued letter that is 
%    less than or equal to |\@tempcnta|, and call on |\gim@print| to
%    subtract this value and print the letter.
%
%    Initialize: |\@tempcnta| holds the value, there is no previous
%    letter. 
%    \begin{macrocode}
\newcommand*{\gim@nomil}[1]{\@tempcnta=#1\@gim@prevfalse
%    \end{macrocode}
% Find the hundreds digit.
%    \begin{macrocode}
  \@tempcntb=\@tempcnta\divide\@tempcntb by 100\relax % hundreds digit
  \ifcase\@tempcntb                     % print nothing if no hundreds
     \or\gim@print{100}{\hebqof}%
     \or\gim@print{200}{\hebresh}%
     \or\gim@print{300}{\hebshin}%
     \or\gim@print{400}{\hebtav}%
     \or\hebtav\@gim@prevtrue\gim@print{500}{\hebqof}%
     \or\hebtav\@gim@prevtrue\gim@print{600}{\hebresh}%
     \or\hebtav\@gim@prevtrue\gim@print{700}{\hebshin}%
     \or\hebtav\@gim@prevtrue\gim@print{800}{\hebtav}%
     \or\hebtav\@gim@prevtrue\hebtav\gim@print{900}{\hebqof}%
  \fi
%    \end{macrocode}
%    Find the tens digit. The numbers 15 and 16 are traditionally
%    printed as tet-vav ($9+6$) and tet-zayin ($9+7$) to avoid
%    spelling the Lord's name.
%    \begin{macrocode}
  \@tempcntb=\@tempcnta\divide\@tempcntb by 10\relax      % tens digit
  \ifcase\@tempcntb                         % print nothing if no tens
      \or                                   % number between 10 and 19
              \ifnum\@tempcnta = 16 \gim@print {9}{\hebtet}% tet-zayin
         \else\ifnum\@tempcnta = 15 \gim@print {9}{\hebtet}% tet-vav
         \else                      \gim@print{10}{\hebyod}%
              \fi % \@tempcnta = 15
              \fi % \@tempcnta = 16
%    \end{macrocode}
%    Initial or final forms are selected according to the current
%    style option; |\gim@print| will force a non-final letter in
%    non-final position by means of a local style change.
%    \begin{macrocode}
      \or\gim@print{20}{\if@gim@final\hebfinalkaf\else\hebkaf\fi}%
      \or\gim@print{30}{\heblamed}%
      \or\gim@print{40}{\if@gim@final\hebfinalmem\else\hebmem\fi}%
      \or\gim@print{50}{\if@gim@final\hebfinalnun\else\hebnun\fi}%
      \or\gim@print{60}{\hebsamekh}%
      \or\gim@print{70}{\hebayin}%
      \or\gim@print{80}{\if@gim@final\hebfinalpe\else\hebpe\fi}%
      \or\gim@print{90}{\if@gim@final\hebfinaltsadi\else\hebtsadi\fi}%
  \fi
%    \end{macrocode}
%    Print the ones digit.
%    \begin{macrocode}
  \ifcase\@tempcnta                         % print nothing if no ones
      \or\gim@print{1}{\hebalef}%
      \or\gim@print{2}{\hebbet}%
      \or\gim@print{3}{\hebgimel}%
      \or\gim@print{4}{\hebdalet}%
      \or\gim@print{5}{\hebhe}%
      \or\gim@print{6}{\hebvav}%
      \or\gim@print{7}{\hebzayin}%
      \or\gim@print{8}{\hebhet}%
      \or\gim@print{9}{\hebtet}%
  \fi
}
%    \end{macrocode}
%  \end{macro}
%  \begin{macro}{\gim@print}
%  \begin{macro}{\if@gim@prev}
%    The actual printing routine typesets a digit with the appropriate
%    apostrophes: if a number sequence consists of a
%    single letter then it is followed by a single apostrophe, and if
%    it consists of more than one letter then a double
%    apostrophe is inserted before the last letter.
%    We typeset the letters one at a time, keeping a flag that tells
%    us if any previous letters had been typeset.
%    \begin{macrocode}
\newif\if@gim@prev % flag if a previous letter has been typeset
%    \end{macrocode}
%    For each letter, we
%    first subtract its value from the total. Then, 
%    \begin{itemize}
%     \item
%      if the result is zero then this is the last letter; we check
%      the flag to see if this is the only letter and print it with
%      the appropriate apostrophe;
%     \item
%      if the result is not zero then there remain additional letters
%      to be typeset; we print without an apostrophe and set the
%      `previous letter' flag. 
%    \end{itemize}
%    |\@tempcnta| holds the total value that remains to be typeset.
%    We first deduct the letter's value from |\@tempcnta|,
%    so |\@tempcnta| is zero if and only if this is the last letter.
%    \begin{macrocode}
\newcommand*{\gim@print}[2]{%   #2 is a letter, #1 is its value.
  \advance\@tempcnta by -#1\relax% deduct the value from the remainder
%    \end{macrocode}
%    If this is the last letter, we print with the appropriate
%    apostrophe (depending on the style option):
%    if there is a preceding letter, print |"x| if the style calls for
%    apostrophes, |x| if it doesn't;
%    otherwise, this is the only letter: print |x'| if the style calls
%    for apostrophes, |x| if it doesn't.
%    \begin{macrocode}
  \ifnum\@tempcnta=0% if this is the last letter
     \if@gim@prev\if@gim@apost"\fi#2%
     \else#2\if@gim@apost'\fi\fi%
%    \end{macrocode}
%    If this is not the last letter: print a non-final form (by
%    forcing a local style option) and set the `previous letter' flag.
%    \begin{macrocode}
  \else{\@gim@finalfalse#2}\@gim@prevtrue\fi}
%    \end{macrocode}
%  \end{macro}
%  \end{macro}
%
%  \begin{macro}{\hebr}
%  \begin{macro}{\gim}
%    The older Hebrew counter commands |\hebr| and |\gim| are retained
%    in order to keep older documents from breaking. They are set to
%    be equivalent to |\alph|, and their use is deprecated. Note that
%    |\hebr| gives different results than it had in the past---it
%    now typesets 11 as yod-alef rather than kaf.
%    \begin{macrocode}
\let\hebr=\alph
\let\gim=\alph
%    \end{macrocode}
%  \end{macro}
%  \end{macro}
%
%    For backward compatibility with `older' \pkg{hebrew.sty}
%    packages, we define Hebrew equivalents of some useful \LaTeX\
%    commands. Note, however, that 8-bit macros defined in Hebrew
%    are no longer supported.
%    \begin{macrocode}
\def\hebcopy{\protect\R{\hebhe\hebayin\hebtav\hebqof}}
\def\hebincl{\protect\R{\hebresh\hebtsadi"\hebbet}}
\def\hebpage{\protect\R{\hebayin\hebmem\hebvav\hebdalet}}
\def\hebto{\protect\R{\hebayin\hebdalet}}
%    \end{macrocode}
%    |\hadgesh| produce ``poor man's bold'' (heavy printout), when
%    used with normal font glyphs. It is advisable to use bold font
%    (for example, \emph{Dead Sea}) instead of this macro.
%    \begin{macrocode}
\def\hadgesh#1{\leavevmode\setbox0=\hbox{#1}%
  \kern-.025em\copy0\kern-\wd0
  \kern.05em\copy0\kern-\wd0
  \kern-.025em\raise.0433em\box0 }
%    \end{macrocode}
%    |\piska| and |\piskapiska| sometimes used in `older' hebrew
%    sources, and should not be used in \LaTeXe.
%    \begin{macrocode}
\if@compatibility
  \def\piska#1{\item{#1}\hangindent=-\hangindent}
  \def\piskapiska#1{\itemitem{#1}\hangindent=-\hangindent}
\fi
%    \end{macrocode}
%    The following commands are simply synonyms for the standard ones,
%    provided with \LaTeXe.
%    \begin{macrocode}
\let\makafgadol=\textendash
\let\makafanak=\textemdash
\let\geresh=\textquoteright
\let\opengeresh=\textquoteright
\let\closegeresh=\textquoteleft
\let\openquote=\textquotedblright
\let\closequote=\textquotedblleft
\let\leftquotation=\textquotedblright
\let\rightquotation=\textquotedblleft
%    \end{macrocode}
%
%    We need to ensure that Hebrew is used as the default
%    right-to-left language at |\begin{document}|. The mechanism of
%    defining the |\@rllanguagename| is the same as in \babel 's
%    |\languagename|: the last right-to-left language in the
%    |\usepackage{babel}| line is set as the default right-to-left
%    language at document beginning.
%
%    For example, the following code:
%    \begin{quote}
%       |\usepackage[russian,hebrew,arabic,greek,english]{babel}|
%    \end{quote}
%    will set the Arabic language as the default right-to-left
%    language and the English language as the default language.
%    As a result, the commands |\L{}| and |\embox{}| will use English
%    and |\R{}| and |\hmbox{}| will use Arabic by default. These
%    defaults can be changed with the next |\sethebrew| or
%    |\selectlanguage{|\emph{language name}|}| command.
%    \begin{macrocode}
\AtBeginDocument{\def\@rllanguagename{hebrew}}
%    \end{macrocode}
%    
%    The macro |\ldf@finish| takes care of looking for a configuration
%    file, setting the main language to be switched on at
%    |\begin{document}| and resetting the category code of |@| to its
%    original value.
%    \begin{macrocode}
\ldf@finish{hebrew}
%</hebrew>
%    \end{macrocode}
%
% \subsection{Right to left support}
%
%    This file \pkg{rlbabel.def} defines necessary bidirectional macro
%    support for \LaTeXe. It is designed for use not only with Hebrew,
%    but with any Right-to-Left languages, supported by \babel. The
%    macros provided in this file are language and encoding
%    independent.
%
%    Right-to-left languages will use \TeX\ extensions, namely \TeX\
%    primitives |\beginL|, |\endL| and |\beginR|, |\endR|, currently
%    implemented only in $\varepsilon$-\TeX\ and in \TeX{-}{-}\XeT.
%
%    If $\varepsilon$-\TeX\ is used, we should switch it to the
%    \emph{enhanced} mode:
%    \begin{macrocode}
%<*rightleft>
\ifx\TeXXeTstate\undefined\else%
   \TeXXeTstate=1
\fi
%    \end{macrocode}
%
%    Note, that $\varepsilon$-\TeX 's format file should be created
%    for \emph{extended} mode. Mode can be checked by running
%    $\varepsilon$-\TeX\ on some \TeX{} file, for example:
%    \begin{quote}
%    |This is e-TeX, Version 3.14159-1.1 (Web2c 7.0)|\\
%    |entering extended mode|
%    \end{quote}
%    The second line should be \texttt{entering extended mode}.
%
%    We check if user uses Right-to-Left enabled engine instead of
%    regular Knuth's \TeX:
%    \begin{macrocode}
\ifx\beginL\@undefined%
   \newlinechar`\^^J
   \typeout{^^JTo avoid this error message,^^J%
     run TeX--XeT or e-TeX engine instead of regular TeX.^^J}
   \errmessage{Right-to-Left Support Error: use TeX--XeT or e-TeX
     engine}%
\fi
%    \end{macrocode}
%
% \subsubsection{Switching from LR to RL mode and back}
% 
%    \cs{@torl} and \cs{@fromrl} are called each time the horizontal
%    direction changes. They do all that is necessary besides changing
%    the direction. Currently their task is to change the encoding
%    information and mode (condition \cs{if@rl}). They should not
%    normally be called by users: user-level macros, such as
%    \cs{sethebrew} and \cs{unsethebrew}, as well as \babel 's
%    \cs{selectlanguage} are defined in language-definition files and
%    should be used to change default language (and direction). 
%    
%    Local direction changing commands (for small pieces of text):
%    |\L{}|, |\R{}|, |\embox{}| and |\hmbox{}| are defined below in
%    this file in language-independent manner.
%
% \begin{macro}{\if@rl}
%    \begin{description}\scrunch
%       \item[|\@rltrue|] means that the main mode is currently
%                          Right-to-Left.
%       \item[|\@rlfalse|] means that the main mode is currently
%                          Left-to-Right.
%    \end{description}
%    \begin{macrocode}
\newif\if@rl
%    \end{macrocode}
% \end{macro}
%
% \begin{macro}{\if@rlmain}
%    This is the main direction of the document. Unlike |\if@rl|
%    it is set once and never changes.
%    \begin{description}\scrunch
%       \item[|\@rltrue|]  means that the document is Right-to-Left.
%       \item[|\@rlfalse|] means that the document is Left-to-Right.
%    \end{description}
%    Practically |\if@rlmain| is set according to the value of |\if@rl|
%    in the beginning of the run.
%    \begin{macrocode}
\AtBeginDocument{% Here we set the main document direction
  \newif\if@rlmain% 
  \if@rl% e.g: if the options to babel were [english,hebrew]
    \@rlmaintrue%
  \else%  e.g: if the options to babel were [hebrew,english]
    \@rlmainfalse%
  \fi%
}
%    \end{macrocode}
% \end{macro}
%
% \begin{macro}{\@torl}
%    Switches current direction to Right-to-Left: saves current
%    Left-to-Right encoding in |\lr@encodingdefault|, sets required
%    Right-to-Left language name in |\@rllanguagename| (similar to
%    \babel 's |\languagename|) and changes derection.
%
%    The Right-to-Left language encoding should be defined in |.ldf|
%    file as special macro created by concatenation of the language
%    name and string \texttt{encoding}, for example, for Hebrew it
%    will be |\hebrewencoding|.
%    \begin{macrocode}
\DeclareRobustCommand{\@torl}[1]{%
  \if@rl\else%
     \let\lr@encodingdefault=\encodingdefault%
  \fi%
  \def\@rllanguagename{#1}%
  \def\encodingdefault{\csname#1encoding\endcsname}%
  \fontencoding{\encodingdefault}%
  \selectfont%
  \@rltrue}
%    \end{macrocode}
% \end{macro}
%
% \begin{macro}{\@fromrl}
%    Opposite to |\@torl|, switches current direction to
%    Left-to-Right: restores saved Left-to-Right encoding
%    (|\lr@encodingdefault|) and changes direction.
%    \begin{macrocode}
\DeclareRobustCommand{\@fromrl}{%
  \if@rl%
     \let\encodingdefault=\lr@encodingdefault%
  \fi%
  \fontencoding{\encodingdefault}%
  \selectfont%
  \@rlfalse}
%    \end{macrocode}
% \end{macro}
%
% \begin{macro}{\selectlanguage}
%    This standard \babel 's macro should be redefined to support
%    bidirectional tables. We divide |\selectlanguage| implementation
%    to two parts, and the first part calls the second
%    |\@@selectlanguage|.
%    \begin{macrocode}
\expandafter\def\csname selectlanguage \endcsname#1{%
  \edef\languagename{%
    \ifnum\escapechar=\expandafter`\string#1\@empty
    \else \string#1\@empty\fi}%
  \@@selectlanguage{\languagename}}
%    \end{macrocode}
% \end{macro}
%
% \begin{macro}{\@@selectlanguage}
%    This new internal macro redefines a final part of the standard
%    \babel 's |\select|\-|language| implementation.
%
%    Standard \LaTeX\ provides us with 3 tables: Table of Contents
%    (|.toc|), List of Figures (|.lof|), and List of Tables
%    (|.lot|). In multi-lingual texts mixing Left-to-Right languages
%    with Right-to-Left ones, the use of various directions in one
%    table results in very ugly output. Therefore, these 3 standard
%    tables will be used now only for Left-to-Right languages, and we
%    will add 3 Right-to-Left tables (their extensions are simply
%    reversed ones): RL Table of Contents (|.cot|), RL List of Figures
%    (|.fol|), and RL List of Tables (|.lof|).
%    \begin{macrocode}
\def\@@selectlanguage#1{%
  \select@language{#1}%
  \if@filesw
     \protected@write\@auxout{}{\string\select@language{#1}}%
     \if@rl%
       \addtocontents{cot}{\xstring\select@language{#1}}%
       \addtocontents{fol}{\xstring\select@language{#1}}%
       \addtocontents{tol}{\xstring\select@language{#1}}%     
     \else%
       \addtocontents{toc}{\xstring\select@language{#1}}%
       \addtocontents{lof}{\xstring\select@language{#1}}%
       \addtocontents{lot}{\xstring\select@language{#1}}%
     \fi%
  \fi}
%    \end{macrocode}
% \end{macro}
%
% \begin{macro}{\setrllanguage}
% \begin{macro}{\unsetrllanguage}
%    The |\setrllanguage| and |\unsetrllanguage| pair of macros is
%    proved to very useful in bilingual texts, for example, in
%    Hebrew-English texts. The language-specific commands, for example,
%    |\sethebrew| and |\unsethebrew| use these macros as basis.
%    
%    Implementation saves and restores other language in
%    |\other@languagename| variable, and uses internal macro
%    |\@@selectlanguage|, defined above, to switch between languages.
%    \begin{macrocode}
\let\other@languagename=\languagename
\DeclareRobustCommand{\setrllanguage}[1]{%
   \if@rl\else%
     \let\other@languagename=\languagename%
   \fi%
     \def\languagename{#1}%
     \@@selectlanguage{\languagename}}
%    \end{macrocode}
%
%    \begin{macrocode}
\DeclareRobustCommand{\unsetrllanguage}[1]{%
   \if@rl%
     \let\languagename=\other@languagename%
   \fi
   \@@selectlanguage{\languagename}}
%    \end{macrocode}
% \end{macro}
% \end{macro}
%
% \begin{macro}{\L}
% \begin{macro}{\R}
% \begin{macro}{\HeblatexRedefineL}
%    Macros for changing direction, originally taken from TUGboat.
%    Usage: |\L{|\emph{Left to Right text}|}| and |\R{|\emph{Right to
%    Left text}|}|. Numbers should also be enclosed in |\L{}|, as in
%    |\L{123}|.
%
%    Note, that these macros do not receive language name as
%    parameter. Instead, the saved |\@rllanguagename| will be
%    used. We assume that each Right-to-Left language defines
%    |\to|\emph{languagename} and |\from|\emph{languagename} macros in
%    language definition file, for example, for Hebrew: |\tohebrew|
%    and |\fromhebrew| macros in \pkg{hebrew.ldf} file.
%
%    The macros \cs{L} and \cs{R} include `protect' to to make them robust and
%    allow use, for example, in tables.
%
%    Due to the fact that some packages have different definitions for \cs{L}
%    the macro |\HeblatexRedefineL| is provided to overide them. This may
%    be required with hyperref, for instance.
%    \begin{macrocode}
\let\next=\
\def\HeblatexRedefineL{%
  \def\L{\protect\pL}%
}
\HeblatexRedefineL
\def\pL{\protect\afterassignment\moreL \let\next= }
\def\moreL{\bracetext \aftergroup\endL \beginL\csname
  from\@rllanguagename\endcsname}
%    \end{macrocode}
%
%    \begin{macrocode}
\def\R{\protect\pR}
\def\pR{\protect\afterassignment\moreR \let\next= }
\def\moreR{\bracetext \aftergroup\endR \beginR\csname
  to\@rllanguagename\endcsname}
\def\bracetext{\ifcat\next{\else\ifcat\next}\fi
  \errmessage{Missing left brace has been substituted}\fi \bgroup}
\everydisplay{\if@rl\aftergroup\beginR\fi }
%    \end{macrocode}
% \end{macro}
% \end{macro}
% \end{macro}
%
% \begin{macro}{\@ensure@R}
% \begin{macro}{\@ensure@L}
%    Two small internal macros, a-la |\ensuremath|
%    \begin{macrocode}
\def\@ensure@R#1{\if@rl#1\else\R{#1}\fi}
\def\@ensure@L#1{\if@rl\L{#1}\else#1\fi}
%    \end{macrocode}
% \end{macro}
% \end{macro}
%
%    Take care of Right-to-Left indentation in every paragraph.
%    Originally, \cs{noindent} was redefined for right-to-left by
%    Yaniv Bargury, then the implementation was rewritten by Alon Ziv
%    using an idea by Chris Rowley: \cs{noindent} now works
%    unmodified.
%    \begin{macrocode}
\def\rl@everypar{\if@rl{\setbox\z@\lastbox\beginR\usebox\z@}\fi}
\let\o@everypar=\everypar
\def\everypar#1{\o@everypar{\rl@everypar#1}}
%    \end{macrocode}
%
% \begin{macro}{\hmbox}
% \begin{macro}{\embox}
%    Useful vbox commands. All text in math formulas is best enclosed
%    in these: LR text in |\embox| and RL text in |\hmbox|. |\mbox{}|
%    is useless for both cases, since it typesets in Left-to-Right
%    even for Right-to-Left languages (additions by Yaniv Bargury). 
%    \begin{macrocode}
\newcommand{\hmbox}[1]{\mbox{\R{#1}}}
\newcommand{\embox}[1]{\mbox{\L{#1}}}
%    \end{macrocode}
% \end{macro}
% \end{macro}
%
% \begin{macro}{\@brackets}
%    When in Right-to-Left mode, brackets should be swapped. This
%    macro receives 3 parameters: left bracket, content, right
%    bracket. Brackets can be square brackets, braces, or
%    parentheses.
%    \begin{macrocode}
\def\@brackets#1#2#3{\protect\if@rl #3#2#1\protect\else
  #1#2#3\protect\fi}
%    \end{macrocode}
% \end{macro}
%
% \begin{macro}{\@number}
% \begin{macro}{\@latin}
%    \cs{@number} preserves numbers direction from Left to Right.
%    \cs{@latin} in addition switches current encoding to the latin.
%    \begin{macrocode}
\def\@@number#1{\ifmmode\else\beginL\fi#1\ifmmode\else\endL\fi}
\def\@@latin#1{\@@number{{\@fromrl#1}}}
\def\@number{\protect\@@number}
\def\@latin{\protect\@@latin}
%    \end{macrocode}
% \end{macro}
% \end{macro}
%
% \subsubsection{Counters}
% 
%     To make counter references work in Right to Left text, we need
%     to surround their original definitions with an
%     |\@number{|\ldots|}| or |\@latin{|\ldots|}|. Note, that
%     language-specific counters, such as \cs{hebr} or \cs{gim} are
%     provided with language definition file.
%
%    We start with saving the original definitions:
%    \begin{macrocode}
\let\@@arabic=\@arabic
\let\@@roman=\@roman
\let\@@Roman=\@Roman
\let\@@alph=\@alph
\let\@@Alph=\@Alph
%    \end{macrocode}
%
% \begin{macro}{\@arabic}
% \begin{macro}{\@roman}
% \begin{macro}{\@Roman}
%    Arabic and roman numbers should be from Left to Right. In
%    addition, roman numerals, both lower- and upper-case should be in
%    latin encoding.
%    \begin{macrocode}
\def\@arabic#1{\@number{\@@arabic#1}}
\def\@roman#1{\@latin{\@@roman#1}}
\def\@Roman#1{\@latin{\@@Roman#1}}
%    \end{macrocode}
% \end{macro}
% \end{macro}
% \end{macro}
%
% \begin{macro}{\arabicnorl}
% This macro preserves the original definition of |\arabic|
% (overrides the overriding of |\@arabic|)
%    \begin{macrocode}
\def\arabicnorl#1{\expandafter\@@arabic\csname c@#1\endcsname}
%    \end{macrocode}
% \end{macro}
%
% \begin{macro}{\make@lr}
%    In Right to Left documents all counters defined in the standard
%    document classes \emph{article}, \emph{report} and \emph{book}
%    provided with \LaTeXe, such as |\thesection|, |\thefigure|,
%    |\theequation| should be typed as numbers from left to right. To
%    ensure direction, we use the following
%    |\make@lr{|\emph{counter}|}| macro:
%    \begin{macrocode}
\def\make@lr#1{\begingroup
    \toks@=\expandafter{#1}%
    \edef\x{\endgroup
  \def\noexpand#1{\noexpand\@number{\the\toks@}}}%
  \x}
%    \end{macrocode}
%
%    \begin{macrocode}
\@ifclassloaded{letter}{}{%
  \@ifclassloaded{slides}{}{%
    \make@lr\thesection
    \make@lr\thesubsection
    \make@lr\thesubsubsection
    \make@lr\theparagraph
    \make@lr\thesubparagraph
    \make@lr\thefigure
    \make@lr\thetable
  }
  \make@lr\theequation
}
%    \end{macrocode}
% \end{macro}
%
% \subsubsection{Preserving logos}
% 
%    Preserve \TeX, \LaTeX\ and \LaTeXe\ logos.
% \begin{macro}{\TeX}
%    \begin{macrocode}
\let\@@TeX\TeX
\def\TeX{\@latin{\@@TeX}}
%    \end{macrocode}
% \end{macro}
%
% \begin{macro}{\LaTeX}
%    \begin{macrocode}
\let\@@LaTeX\LaTeX
\def\LaTeX{\@latin{\@@LaTeX}}
%    \end{macrocode}
% \end{macro}
%
% \begin{macro}{\LaTeXe}
%    \begin{macrocode}
\let\@@LaTeXe\LaTeXe
\def\LaTeXe{\@latin{\@@LaTeXe}}
%    \end{macrocode}
% \end{macro}
%
% \subsubsection{List environments}
%
%    List environments in Right-to-Left languages, are ticked and
%    indented from the right instead of from the left. All the
%    definitions that caused indentation are revised for Right-to-Left
%    languages. \LaTeX\ keeps track on the indentation with the
%    \cs{leftmargin} and \cs{rightmargin} values.
%
% \begin{macro}{list}
%    Thus we need to override the definition of the |\list| macro: when
%    in RTL mode, the right margins are the begining of the line.
%    \begin{macrocode}
\def\list#1#2{%
  \ifnum \@listdepth >5\relax
    \@toodeep
  \else
    \global\advance\@listdepth\@ne
  \fi
  \rightmargin\z@
  \listparindent\z@
  \itemindent\z@
  \csname @list\romannumeral\the\@listdepth\endcsname
  \def\@itemlabel{#1}%
  \let\makelabel\@mklab
  \@nmbrlistfalse
  #2\relax
  \@trivlist
  \parskip\parsep
  \parindent\listparindent
  \advance\linewidth -\rightmargin
  \advance\linewidth -\leftmargin
%    \end{macrocode}
%    The only change in the macro is the |\if@rl| case:
%    \begin{macrocode}
  \if@rl
    \advance\@totalleftmargin \rightmargin
  \else
    \advance\@totalleftmargin \leftmargin
  \fi
  \parshape \@ne \@totalleftmargin \linewidth
  \ignorespaces}
%    \end{macrocode}
% \end{macro}
%
% \begin{macro}{\labelenumii}
% \begin{macro}{\p@enumiii}
%    The \cs{labelenumii} and \cs{p@enumiii} commands use
%    \emph{parentheses}. They are revised to work Right-to-Left with
%    the help of \cs{@brackets} macro defined above.
%    \begin{macrocode}
\def\labelenumii{\@brackets(\theenumii)}
\def\p@enumiii{\p@enumii\@brackets(\theenumii)}
%    \end{macrocode}
% \end{macro}
% \end{macro}
%
% \subsubsection{Tables of moving stuff}
%
%    Tables of moving arguments: table of contents (|toc|), list of
%    figures (|lof|) and list of tables (|lot|) are handles here. These
%    three default \LaTeX\ tables will be used now exclusively for
%    Left to Right stuff.
%
%    Three additional Right-to-Left tables: RL table of contents
%    (|cot|), RL list of figures (|fol|), and RL list of tables
%    (|tol|) are added.
%    These three tables will be used exclusively for Right to
%    Left stuff.
%
% \begin{macro}{\@tableofcontents}
% \begin{macro}{\@listoffigures}
% \begin{macro}{\@listoftables}
%    We define 3 new macros similar to the standard \LaTeX\ tables,
%    but with one parameter --- table file extension. These macros
%    will help us to define our additional tables below.
%    \begin{macrocode}
\@ifclassloaded{letter}{}{% other
\@ifclassloaded{slides}{}{% other
  \@ifclassloaded{article}{% article
    \newcommand\@tableofcontents[1]{%
      \section*{\contentsname\@mkboth%
        {\MakeUppercase\contentsname}%
        {\MakeUppercase\contentsname}}%
      \@starttoc{#1}}
    \newcommand\@listoffigures[1]{%
      \section*{\listfigurename\@mkboth%
        {\MakeUppercase\listfigurename}%
        {\MakeUppercase\listfigurename}}%
      \@starttoc{#1}}
    \newcommand\@listoftables[1]{%
      \section*{\listtablename\@mkboth%
        {\MakeUppercase\listtablename}%
        {\MakeUppercase\listtablename}}%
      \@starttoc{#1}}}%
  {% else report or book
    \newcommand\@tableofcontents[1]{%
      \@restonecolfalse\if@twocolumn\@restonecoltrue\onecolumn%
      \fi\chapter*{\contentsname\@mkboth%
        {\MakeUppercase\contentsname}%
        {\MakeUppercase\contentsname}}%
      \@starttoc{#1}\if@restonecol\twocolumn\fi}
    \newcommand\@listoffigures[1]{%
      \@restonecolfalse\if@twocolumn\@restonecoltrue\onecolumn%
      \fi\chapter*{\listfigurename\@mkboth%
        {\MakeUppercase\listfigurename}%
        {\MakeUppercase\listfigurename}}%
      \@starttoc{#1}\if@restonecol\twocolumn\fi}
    \newcommand\@listoftables[1]{%
      \if@twocolumn\@restonecoltrue\onecolumn\else\@restonecolfalse\fi%
      \chapter*{\listtablename\@mkboth%      
        {\MakeUppercase\listtablename}%
        {\MakeUppercase\listtablename}}%
      \@starttoc{#1}\if@restonecol\twocolumn\fi}}%
%    \end{macrocode}
% \end{macro}
% \end{macro}
% \end{macro}
%
% \begin{macro}{\lrtableofcontents}
% \begin{macro}{\lrlistoffigures}
% \begin{macro}{\lrlistoftables}
%    Left-to-Right tables are called now |\lr|\emph{xxx} and defined
%    with the aid of three macros defined above (extensions |toc|,
%    |lof|, and |lot|).
%    \begin{macrocode}
  \newcommand\lrtableofcontents{\@tableofcontents{toc}}%
  \newcommand\lrlistoffigures{\@listoffigures{lof}}%
  \newcommand\lrlistoftables{\@listoftables{lot}}%
%    \end{macrocode}
% \end{macro}
% \end{macro}
% \end{macro}
%
% \begin{macro}{\rltableofcontents}
% \begin{macro}{\rllistoffigures}
% \begin{macro}{\rllistoftables}
%    Right-to-Left tables will be called |\rl|\emph{xxx} and defined
%    with the aid of three macros defined above (extensions |cot|,
%    |fol|, and |tol|).
%    \begin{macrocode}
  \newcommand\rltableofcontents{\@tableofcontents{cot}}%
  \newcommand\rllistoffigures{\@listoffigures{fol}}%
  \newcommand\rllistoftables{\@listoftables{tol}}%
%    \end{macrocode}
% \end{macro}
% \end{macro}
% \end{macro}
%
% \begin{macro}{\tableofcontents}
% \begin{macro}{\listoffigures}
% \begin{macro}{\listoftables}
%    Let |\|\emph{xxx} be |\rl|\emph{xxx} if the current direction is
%    Right-to-Left and |\lr|\emph{xxx} if it is Left-to-Right.
%    \begin{macrocode}
  \renewcommand\tableofcontents{\if@rl\rltableofcontents%
                                \else\lrtableofcontents\fi}
  \renewcommand\listoffigures{\if@rl\rllistoffigures%
                              \else\lrlistoffigures\fi}
  \renewcommand\listoftables{\if@rl\rllistoftables%
                             \else\lrlistoftables\fi}}}
%    \end{macrocode}
% \end{macro}
% \end{macro}
% \end{macro}
%
% \begin{macro}{\@dottedtocline}
%    The following makes problems when making a Right-to-Left tables,
%    since it uses \cs{leftskip} and \cs{rightskip} which are both
%    mode dependent.
%    \begin{macrocode}
\def\@dottedtocline#1#2#3#4#5{%
  \ifnum #1>\c@tocdepth \else
    \vskip \z@ \@plus.2\p@
    {\if@rl\rightskip\else\leftskip\fi #2\relax 
      \if@rl\leftskip\else\rightskip\fi \@tocrmarg \parfillskip
      -\if@rl\leftskip\else\rightskip\fi
     \parindent #2\relax\@afterindenttrue
     \interlinepenalty\@M
     \leavevmode
     \@tempdima #3\relax
     \advance\if@rl\rightskip\else\leftskip\fi \@tempdima 
     \null\nobreak\hskip -\if@rl\rightskip\else\leftskip\fi
     {#4}\nobreak
     \leaders\hbox{$\m@th
        \mkern \@dotsep mu\hbox{.}\mkern \@dotsep
        mu$}\hfill
     \nobreak
     \hb@xt@\@pnumwidth{\hfil\normalfont \normalcolor \beginL#5\endL}%
     \par}%
  \fi}
%    \end{macrocode}
% \end{macro}
%
% \begin{macro}{\l@part}
%    This standard macro was redefined for table of contents since it
%    uses \cs{rightskip} which is mode dependent.
%    \begin{macrocode}
\@ifclassloaded{letter}{}{% other
\@ifclassloaded{slides}{}{% other
\renewcommand*\l@part[2]{%
  \ifnum \c@tocdepth >-2\relax
    \addpenalty{-\@highpenalty}%
    \addvspace{2.25em \@plus\p@}%
    \begingroup
      \setlength\@tempdima{3em}%
      \parindent \z@ \if@rl\leftskip\else\rightskip\fi \@pnumwidth
      \parfillskip -\@pnumwidth
      {\leavevmode
       \large \bfseries #1\hfil \hb@xt@\@pnumwidth{\hss#2}}\par
       \nobreak
         \global\@nobreaktrue
         \everypar{\global\@nobreakfalse\everypar{}}%
    \endgroup
  \fi}}}
%    \end{macrocode}
% \end{macro}
%
% \begin{macro}{\@part}
%    Part is redefined to support new Right-to-Left table of contents
%    (|cot|) as well as the Left-to-Right one (|toc|).
%    \begin{macrocode}
\@ifclassloaded{article}{% article class
  \def\@part[#1]#2{%
    \ifnum \c@secnumdepth >\m@ne
      \refstepcounter{part}%
      \addcontentsline{toc}{part}{\thepart\hspace{1em}#1}%
      \addcontentsline{cot}{part}{\thepart\hspace{1em}#1}%
    \else
      \addcontentsline{toc}{part}{#1}%
      \addcontentsline{cot}{part}{#1}%
    \fi
    {\parindent \z@ \raggedright
     \interlinepenalty \@M
     \normalfont
     \ifnum \c@secnumdepth >\m@ne
       \Large\bfseries \partname~\thepart
       \par\nobreak
     \fi
     \huge \bfseries #2%
     \markboth{}{}\par}%
    \nobreak
    \vskip 3ex
    \@afterheading}%
}{% report and book classes
  \def\@part[#1]#2{%
    \ifnum \c@secnumdepth >-2\relax
      \refstepcounter{part}%
      \addcontentsline{toc}{part}{\thepart\hspace{1em}#1}%
      \addcontentsline{cot}{part}{\thepart\hspace{1em}#1}%
    \else
      \addcontentsline{toc}{part}{#1}%
      \addcontentsline{cot}{part}{#1}%
    \fi
    \markboth{}{}%
    {\centering
     \interlinepenalty \@M
     \normalfont
     \ifnum \c@secnumdepth >-2\relax
       \huge\bfseries \partname~\thepart
       \par
       \vskip 20\p@
     \fi
     \Huge \bfseries #2\par}%
     \@endpart}}
%    \end{macrocode} 
% \end{macro}
%
% \begin{macro}{\@sect}
%    Section was redefined from the \pkg{latex.ltx} file. It is
%    changed to support both Left-to-Right (|toc|) and Right-to-Left
%    (|cot|) table of contents simultaneously.
%    \begin{macrocode}
\def\@sect#1#2#3#4#5#6[#7]#8{%
  \ifnum #2>\c@secnumdepth
    \let\@svsec\@empty
  \else
    \refstepcounter{#1}%
    \protected@edef\@svsec{\@seccntformat{#1}\relax}%
  \fi
  \@tempskipa #5\relax
  \ifdim \@tempskipa>\z@
    \begingroup
      #6{%
        \@hangfrom{\hskip #3\relax\@svsec}%
          \interlinepenalty \@M #8\@@par}%
    \endgroup
    \csname #1mark\endcsname{#7}%
    \addcontentsline{toc}{#1}{%
      \ifnum #2>\c@secnumdepth \else
        \protect\numberline{\csname the#1\endcsname}%
      \fi
      #7}%
    \addcontentsline{cot}{#1}{%
      \ifnum #2>\c@secnumdepth \else
        \protect\numberline{\csname the#1\endcsname}%
      \fi
      #7}%
  \else
    \def\@svsechd{%
      #6{\hskip #3\relax
      \@svsec #8}%
      \csname #1mark\endcsname{#7}%
      \addcontentsline{toc}{#1}{%
        \ifnum #2>\c@secnumdepth \else
          \protect\numberline{\csname the#1\endcsname}%
        \fi
        #7}%
      \addcontentsline{cot}{#1}{%
        \ifnum #2>\c@secnumdepth \else
          \protect\numberline{\csname the#1\endcsname}%
        \fi
        #7}}%
  \fi
  \@xsect{#5}}
%    \end{macrocode} 
% \end{macro}
%
% \begin{macro}{\@caption}
%    Caption was redefined from the \pkg{latex.ltx} file. It is
%    changed to support Left-to-Right list of figures and list of
%    tables (|lof| and |lot|) as well as new Right-to-Left lists 
%    (|fol| and |tol|) simultaneously.
%    \begin{macrocode}
\long\def\@caption#1[#2]#3{%
  \par
  \addcontentsline{\csname ext@#1\endcsname}{#1}%
    {\protect\numberline{\csname the#1\endcsname}%
    {\ignorespaces #2}}%
  \def\@fignm{figure}
  \ifx#1\@fignm\addcontentsline{fol}{#1}%
     {\protect\numberline{\csname the#1\endcsname}%
     {\ignorespaces #2}}\fi%
  \def\@tblnm{table}
  \ifx#1\@tblnm\addcontentsline{tol}{#1}%
     {\protect\numberline{\csname the#1\endcsname}%
     {\ignorespaces #2}}\fi%
  \begingroup
    \@parboxrestore
    \if@minipage
      \@setminipage
    \fi
    \normalsize
    \@makecaption{\csname fnum@#1\endcsname}{\ignorespaces #3}\par
  \endgroup}
%    \end{macrocode} 
% \end{macro}
%
% \begin{macro}{\l@chapter}
%    This standard macro was redefined for table of contents since it
%    uses \cs{rightskip} which is mode dependent.
%    \begin{macrocode}
\@ifclassloaded{letter}{}{%
\@ifclassloaded{slides}{}{%
  \@ifclassloaded{article}{}{%
    \renewcommand*\l@chapter[2]{%
      \ifnum \c@tocdepth >\m@ne
      \addpenalty{-\@highpenalty}%
      \vskip 1.0em \@plus\p@
      \setlength\@tempdima{1.5em}%
      \begingroup
         \parindent \z@ \if@rl\leftskip\else\rightskip\fi \@pnumwidth
         \parfillskip -\@pnumwidth
         \leavevmode \bfseries
         \advance\if@rl\rightskip\else\leftskip\fi\@tempdima
         \hskip -\if@rl\rightskip\else\leftskip\fi
         #1\nobreak\hfil \nobreak\hb@xt@\@pnumwidth{\hss#2}\par
         \penalty\@highpenalty
      \endgroup
      \fi}}}}
%    \end{macrocode} 
% \end{macro}
%
% \begin{macro}{\l@section}
% \begin{macro}{\l@subsection}
% \begin{macro}{\l@subsubsection}
% \begin{macro}{\l@paragraph}
% \begin{macro}{\l@subparagraph}
%    The toc entry for section did not work in article style.
%    Also it does not print dots, which is funny when most of your
%    work is divided into sections.
%
%    It was revised to use |\@dottedtocline| as in \pkg{report.sty}
%    (by Yaniv Bargury) and was updated later for all kinds of
%    sections (by Boris Lavva).
%    \begin{macrocode}
\@ifclassloaded{article}{%
\renewcommand*\l@section{\@dottedtocline{1}{1.5em}{2.3em}}
\renewcommand*\l@subsection{\@dottedtocline{2}{3.8em}{3.2em}}
\renewcommand*\l@subsubsection{\@dottedtocline{3}{7.0em}{4.1em}}
\renewcommand*\l@paragraph{\@dottedtocline{4}{10em}{5em}}
\renewcommand*\l@subparagraph{\@dottedtocline{5}{12em}{6em}}}{}
%    \end{macrocode} 
% \end{macro}
% \end{macro}
% \end{macro}
% \end{macro}
% \end{macro}
%
% \subsubsection{Two-column mode}
%
%    This is the support of \texttt{twocolumn} option for the standard
%    \LaTeXe\ classes.
%    The following code was originally borrowed from the Arab\TeX\
%    package, file \pkg{latexext.sty}, copyright by Klaus Lagally,
%    Institut fuer Informatik, Universitaet Stuttgart. It was updated
%    for this package by Boris Lavva.
%
% \begin{macro}{\@outputdblcol}
% \begin{macro}{\set@outputdblcol}
% \begin{macro}{rl@outputdblcol}
%    First column is \cs{@leftcolumn} will be shown at the right side,
%    Second column is \cs{@outputbox} will be shown at the left side. 
%    
%    |\set@outputdblcol| IS CURRENTLY DISABLED. TODO: REMOVE IT [tzafrir]
%    \begin{macrocode}
\let\@@outputdblcol\@outputdblcol
%\def\set@outputdblcol{%
%  \if@rl\renewcommand{\@outputdblcol}{\rl@outputdblcol}%
%  \else\renewcommand{\@outputdblcol}{\@@outputdblcol}\fi}
\renewcommand{\@outputdblcol}{%
  \if@rlmain%
    \rl@outputdblcol%
  \else%
    \@@outputdblcol%
  \fi%
}
\newcommand{\rl@outputdblcol}{%
  \if@firstcolumn
    \global \@firstcolumnfalse
    \global \setbox\@leftcolumn \box\@outputbox
  \else
    \global \@firstcolumntrue
    \setbox\@outputbox \vbox {\hb@xt@\textwidth {%
                              \hskip\columnwidth%
                              \hfil\vrule\@width\columnseprule\hfil
                              \hb@xt@\columnwidth {%
                                \box\@leftcolumn \hss}%
                              \hb@xt@\columnwidth {%
                                \hskip-\textwidth%
                                \box\@outputbox \hss}%
                              \hskip\columnsep%
                              \hskip\columnwidth}}%
    \@combinedblfloats
    \@outputpage
    \begingroup
      \@dblfloatplacement
      \@startdblcolumn
      \@whilesw\if@fcolmade \fi
        {\@outputpage
         \@startdblcolumn}%
    \endgroup
 \fi}
%    \end{macrocode} 
% \end{macro}
% \end{macro}
% \end{macro}
%
% \subsubsection{Footnotes}
%
% \begin{macro}{\footnoterule}
%    The Right-to-Left footnote rule is simply reversed default
%    Left-to-Right one. Footnotes can be used in RL or LR main 
%    modes, but changing mode while a footnote is pending is still 
%    unsolved.
%    \begin{macrocode}
\let\@@footnoterule=\footnoterule
\def\footnoterule{\if@rl\hb@xt@\hsize{\hss\vbox{\@@footnoterule}}%
                  \else\@@footnoterule\fi} 
%    \end{macrocode} 
% \end{macro}
%
% \subsubsection{Headings and two-side support}
%
%    When using \texttt{headings} or \texttt{myheadings} modes, we
%    have to ensure that the language and direction of heading is the
%    same as the whole chapter/part of the document. This is
%    implementing by setting special variable \cs{headlanguage} when
%    starting new chapter/part.
%
%    In addition, when selecting the \texttt{twoside} option (default in
%    \texttt{book} document class), the LR and RL modes need to be set
%    properly for things on the heading and footing. This is done
%    here too.
%
% \begin{macro}{ps@headings}
% \begin{macro}{ps@myheadings}
% \begin{macro}{headeven}
% \begin{macro}{headodd}
%    First, we will support the standard \pkg{letter} class:
%    \begin{macrocode}
\@ifclassloaded{letter}{%
  \def\headodd{\protect\if@rl\beginR\fi\headtoname{}
               \ignorespaces\toname
               \hfil \@date
               \hfil \pagename{} \thepage\protect\if@rl\endR\fi}
  \if@twoside
     \def\ps@headings{%
         \let\@oddfoot\@empty\let\@evenfoot\@empty
         \def\@oddhead{\select@language{\headlanguage}\headodd}
         \let\@evenhead\@oddhead}
  \else
     \def\ps@headings{%
         \let\@oddfoot\@empty
         \def\@oddhead{\select@language{\headlanguage}\headodd}}
  \fi  
  \def\headfirst{\protect\if@rl\beginR\fi\fromlocation \hfill %
                 \telephonenum\protect\if@rl\endR\fi}
  \def\ps@firstpage{%
     \let\@oddhead\@empty
     \def\@oddfoot{\raisebox{-45\p@}[\z@]{%
        \hb@xt@\textwidth{\hspace*{100\p@}%
          \ifcase \@ptsize\relax
             \normalsize
          \or
             \small
          \or
             \footnotesize
          \fi
        \select@language{\headlanguage}\headfirst}}\hss}}
%
  \renewcommand{\opening}[1]{%
     \let\headlanguage=\languagename%
     \ifx\@empty\fromaddress%
        \thispagestyle{firstpage}%
        {\raggedleft\@date\par}%
     \else  % home address
        \thispagestyle{empty}%
        {\raggedleft
        \if@rl\begin{tabular}{@{\beginR\csname%
          to\@rllanguagename\endcsname}r@{\endR}}\ignorespaces
           \fromaddress \\*[2\parskip]%
           \@date \end{tabular}\par%
        \else\begin{tabular}{l}\ignorespaces
           \fromaddress \\*[2\parskip]%
           \@date \end{tabular}\par%
        \fi}%
     \fi
     \vspace{2\parskip}%
     {\raggedright \toname \\ \toaddress \par}%
     \vspace{2\parskip}%
     #1\par\nobreak}
}
%    \end{macrocode}
%    Then, the \pkg{article}, \pkg{report} and \pkg{book} document
%    classes are supported. Note, that in one-sided mode
%    \cs{markright} was changed to \cs{markboth}.
%    \begin{macrocode}
{% article, report, book
  \def\headeven{\protect\if@rl\beginR\thepage\hfil\rightmark\endR
                \protect\else\thepage\hfil{\slshape\leftmark}
                \protect\fi}
  \def\headodd{\protect\if@rl\beginR\leftmark\hfil\thepage\endR
               \protect\else{\slshape\rightmark}\hfil\thepage
               \protect\fi}
  \@ifclassloaded{article}{% article
    \if@twoside   % two-sided
       \def\ps@headings{%
         \let\@oddfoot\@empty\let\@evenfoot\@empty
         \def\@evenhead{\select@language{\headlanguage}\headeven}%
         \def\@oddhead{\select@language{\headlanguage}\headodd}%
         \let\@mkboth\markboth
         \def\sectionmark##1{%
           \markboth {\MakeUppercase{%
               \ifnum \c@secnumdepth >\z@
                  \thesection\quad
               \fi
               ##1}}{}}%
         \def\subsectionmark##1{%
           \markright{%
             \ifnum \c@secnumdepth >\@ne
                \thesubsection\quad
             \fi
        ##1}}}
    \else          % one-sided
       \def\ps@headings{%
         \let\@oddfoot\@empty
         \def\@oddhead{\headodd}%
         \let\@mkboth\markboth
         \def\sectionmark##1{%
           \markboth{\MakeUppercase{%
               \ifnum \c@secnumdepth >\m@ne
                  \thesection\quad
               \fi
               ##1}}{\MakeUppercase{%
               \ifnum \c@secnumdepth >\m@ne
                  \thesection\quad
               \fi
               ##1}}}}
    \fi
%
    \def\ps@myheadings{%
      \let\@oddfoot\@empty\let\@evenfoot\@empty
      \def\@evenhead{\select@language{\headlanguage}\headeven}%
      \def\@oddhead{\select@language{\headlanguage}\headodd}%
      \let\@mkboth\@gobbletwo
      \let\sectionmark\@gobble
      \let\subsectionmark\@gobble
  }}{% report and book
    \if@twoside  % two-sided
       \def\ps@headings{%
         \let\@oddfoot\@empty\let\@evenfoot\@empty
         \def\@evenhead{\select@language{\headlanguage}\headeven}
         \def\@oddhead{\select@language{\headlanguage}\headodd}
         \let\@mkboth\markboth
         \def\chaptermark##1{%
           \markboth{\MakeUppercase{%
               \ifnum \c@secnumdepth >\m@ne
                  \@chapapp\ \thechapter. \ %
               \fi
               ##1}}{}}%
         \def\sectionmark##1{%
           \markright {\MakeUppercase{%
               \ifnum \c@secnumdepth >\z@
                  \thesection. \ %
               \fi
               ##1}}}}
    \else  % one-sided
       \def\ps@headings{%
         \let\@oddfoot\@empty
         \def\@oddhead{\select@language{\headlanguage}\headodd}
         \let\@mkboth\markboth
         \def\chaptermark##1{%
           \markboth{\MakeUppercase{%
               \ifnum \c@secnumdepth >\m@ne
                  \@chapapp\ \thechapter. \ %
               \fi
               ##1}}{\MakeUppercase{%
               \ifnum \c@secnumdepth >\m@ne
                  \@chapapp\ \thechapter. \ %
               \fi
               ##1}}}}
    \fi
    \def\ps@myheadings{%
      \let\@oddfoot\@empty\let\@evenfoot\@empty
      \def\@evenhead{\select@language{\headlanguage}\headeven}%
      \def\@oddhead{\select@language{\headlanguage}\headodd}%
      \let\@mkboth\@gobbletwo
      \let\chaptermark\@gobble
      \let\sectionmark\@gobble
  }}}
%    \end{macrocode} 
% \end{macro}
% \end{macro}
% \end{macro}
% \end{macro}
%
%    
%    \subsubsection{Postscript Porblems}
%    Any command that is implemented by PostScript directives, e.g
%    commands from the ps-tricks package, needs to be fixed, because the
%    PostScript directives are being interpeted after the document has been
%    converted by \TeX to visual Hebrew (DVI, PostScript and PDF have visual
%    Hebrew). 
%
%    For instance: Suppose you wrote in your document: 
%
%    |\textcolor{cyan}{some ltr text}| 
%
%    This would be interpeted by \TeX to something like:
%
%    |[postscript:make color cyan]some LTR text[postscript:make color black]|
%
%
%    However, with the bidirectionality support we get:
%
%    |\textcolor{cyan}{\hebalef\hebbet}|
%
%    Translated to: 
%
%    |[postscript:make color black]{bet}{alef}[postscript:make color cyan]|
%
%    While we want:
%
%    |[postscript:make color cyan]{bet}{alef}[postscript:make color black]|
%
%    The following code will probably work at least with code that stays in the
%    same line:
%    \begin{macro}{@textcolor}
% \begin{macrocode}
\AtBeginDocument{%
  %I assume that \@textcolor is only defined by the package color
  \ifx\@textcolor\@undefined\else% 
    % If that macro was defined before the beginning of the document,
    % that is: the package was loaded: redefine it with bidi support
    \def\@textcolor#1#2#3{% 
      \if@rl%
        \beginL\protect\leavevmode{\color#1{#2}\beginR#3\endR}\endL%
      \else%
        \protect\leavevmode{\color#1{#2}#3}%
      \fi%
    }%
  \fi%
}
% \end{macrocode}
% \end{macro}
% \begin{macro}{\thetrueSlideCounter}
%    This macro probably needs to be overriden for when using |prosper|,
%    (waiting for feedback. Tzafrir)
%    \begin{macrocode}
\@ifclassloaded{prosper}{%
  \def\thetrueSlideCounter{\arabicnorl{trueSlideCounter}}
}{}
%    \end{macrocode}
% \end{macro}
%
% \subsubsection{Miscellaneous internal \LaTeX\ macros}
%
% \begin{macro}{\raggedright}
% \begin{macro}{\raggedleft}
%    \cs{raggedright} was changed from \pkg{latex.ltx} file to support
%    Right-to-Left mode, because of the bug in its implementation.
%    \begin{macrocode}
\def\raggedright{%
  \let\\\@centercr
  \leftskip\z@skip\rightskip\@flushglue
  \parindent\z@\parfillskip\z@skip}
%    \end{macrocode} 
%    Swap meanings of \cs{raggedright} and \cs{raggedleft} in
%    Right-to-Left mode.
%    \begin{macrocode}
\let\@@raggedleft=\raggedleft
\let\@@raggedright=\raggedright
\renewcommand\raggedleft{\if@rl\@@raggedright%
                         \else\@@raggedleft\fi}
\renewcommand\raggedright{\if@rl\@@raggedleft%
                          \else\@@raggedright\fi}
%    \end{macrocode} 
% \end{macro}
% \end{macro}
%
% \begin{macro}{\author}
%    \cs{author} is inserted with \texttt{tabular} environment, and
%    will be used in restricted horizontal mode. Therefore we have to
%    add explicit direction change command when in Right-to-Left
%    mode.
%    \begin{macrocode}
\let\@@author=\author
\renewcommand{\author}[1]{\@@author{\if@rl\beginR #1\endR\else #1\fi}}
%    \end{macrocode}
% \end{macro}
%
% \begin{macro}{\MakeUppercase}
% \begin{macro}{\MakeLowercase}
%    There are no uppercase and lowercase letters in most
%    Right-to-Left languages, therefore we should redefine
%    \cs{MakeUppercase} and \cs{MakeLowercase} \LaTeXe\ commands.
%    \begin{macrocode}
\let\@@MakeUppercase=\MakeUppercase
\def\MakeUppercase#1{\if@rl#1\else\@@MakeUppercase{#1}\fi}
\let\@@MakeLowercase=\MakeLowercase
\def\MakeLowercase#1{\if@rl#1\else\@@MakeLowercase{#1}\fi}
%    \end{macrocode}
% \end{macro}
% \end{macro}
%
% \begin{macro}{\underline}
%    We should explicitly use \cs{L} and \cs{R} commands in
%    \cs{underline}d text.
%    \begin{macrocode}
\let\@@@underline=\underline
\def\underline#1{\@@@underline{\if@rl\R{#1}\else #1\fi}}
%    \end{macrocode}
% \end{macro}
%
%    \cs{undertext} was added for \LaTeX 2.09 compatibility mode.
%    \begin{macrocode}
\if@compatibility
   \let\undertext=\underline
\fi
%    \end{macrocode} 
%
% \begin{macro}{\@xnthm}
% \begin{macro}{\@opargbegintheorem}
%    The following has been inserted to correct the appearance of the
%    number in \cs{newtheorem} to reorder theorem number components. A
%    similar correction  in the definition of \cs{@opargbegintheorem}
%    was added too.
%    \begin{macrocode}
\def\@xnthm#1#2[#3]{%
  \expandafter\@ifdefinable\csname #1\endcsname
  {\@definecounter{#1}\@addtoreset{#1}{#3}%
    \expandafter\xdef\csname the#1\endcsname{\noexpand\@number
      {\expandafter\noexpand\csname the#3\endcsname \@thmcountersep
        \@thmcounter{#1}}}%
    \global\@namedef{#1}{\@thm{#1}{#2}}%
    \global\@namedef{end#1}{\@endtheorem}}}
%
\def\@opargbegintheorem#1#2#3{%
  \trivlist
      \item[\hskip \labelsep{\bfseries #1\ #2\ 
          \@brackets({#3})}]\itshape}
%    \end{macrocode} 
% \end{macro}
% \end{macro}
%
% \begin{macro}{\@chapter}
% \begin{macro}{\@schapter}
%    The following was added for pretty printing of the chapter
%    numbers, for supporting Right-to-Left tables (\texttt{cot},
%    \texttt{fol}, and \texttt{tol}), to save \cs{headlanguage}
%    for use in running headers, and to start two-column mode
%    depending on chapter's main language.
%    \begin{macrocode}
\@ifclassloaded{article}{}{%
  % For pretty priniting
  \def\@@chapapp{Chapter}
  \def\@@thechapter{\@@arabic\c@chapter}
  \def\@chapter[#1]#2{%
    \let\headlanguage=\languagename%
    %\set@outputdblcol%
    \ifnum \c@secnumdepth >\m@ne
       \refstepcounter{chapter}%
       \typeout{\@@chapapp\space\@@thechapter.}%
       \addcontentsline{toc}{chapter}%
       {\protect\numberline{\thechapter}#1}
       \addcontentsline{cot}{chapter}%
       {\protect\numberline{\thechapter}#1}
    \else
       \addcontentsline{toc}{chapter}{#1}%
       \addcontentsline{cot}{chapter}{#1}%
    \fi
    \chaptermark{#1}
    \addtocontents{lof}{\protect\addvspace{10\p@}}%
    \addtocontents{fol}{\protect\addvspace{10\p@}}%
    \addtocontents{lot}{\protect\addvspace{10\p@}}%
    \addtocontents{tol}{\protect\addvspace{10\p@}}%
    \if@twocolumn
       \@topnewpage[\@makechapterhead{#2}]%
    \else
       \@makechapterhead{#2}%
       \@afterheading
    \fi}
  %
  \def\@schapter#1{%
    \let\headlanguage=\languagename%
    %\set@outputdblcol%    
    \if@twocolumn
       \@topnewpage[\@makeschapterhead{#1}]%
    \else
       \@makeschapterhead{#1}%
       \@afterheading
    \fi}}
%    \end{macrocode} 
% \end{macro}
% \end{macro}
%
% \begin{macro}{\appendix}
%    Changed mainly for pretty printing of appendix numbers, and to
%    start two-column mode with the right language (if needed).
%    \begin{macrocode}
\@ifclassloaded{letter}{}{% other
\@ifclassloaded{slides}{}{% other
  \@ifclassloaded{article}{% article
    \renewcommand\appendix{\par
      \setcounter{section}{0}%
      \setcounter{subsection}{0}%
      \renewcommand\thesection{\@Alph\c@section}}
  }{% report and book
    \renewcommand\appendix{\par
      %\set@outputdblcol%
      \setcounter{chapter}{0}%
      \setcounter{section}{0}%
      \renewcommand\@chapapp{\appendixname}%
      % For pretty priniting
      \def\@@chapapp{Appendix}%
      \def\@@thechapter{\@@Alph\c@chapter}
      \renewcommand\thechapter{\@Alph\c@chapter}}}}}
%    \end{macrocode} 
% \end{macro}
%
% \subsubsection{Bibliography and citations}
%
% \begin{macro}{\@cite}
% \begin{macro}{\@biblabel}
% \begin{macro}{\@lbibitem}
%    Citations are produced by the macro
%    |\@cite{|\emph{LABEL}|}{|\emph{NOTE}|}|. Both the citation label
%    and the note is typeset in the current direction. We have to use
%    \cs{@brackets} macro in \cs{@cite} and \cs{@biblabel} macros. In
%    addition, when using \emph{alpha} or similar bibliography style,
%    the \cs{@lbibitem} is used and have to be update to support bot
%    Right-to-Left and Left-to-Right citations.
%
%    \begin{macrocode}
\def\@cite#1#2{\@brackets[{#1\if@tempswa , #2\fi}]}
\def\@biblabel#1{\@brackets[{#1}]}
\def\@lbibitem[#1]#2{\item[\@biblabel{#1}\hfill]\if@filesw
      {\let\protect\noexpand
       \immediate
       \if@rl\write\@auxout{\string\bibcite{#2}{\R{#1}}}%
       \else\write\@auxout{\string\bibcite{#2}{\L{#1}}}\fi%
      }\fi\ignorespaces}
%    \end{macrocode}
% \end{macro}
% \end{macro}
% \end{macro}
%
% \begin{environment}{thebibliography}
%    Use \cs{rightmargin} instead of \cs{leftmargin} when in RL mode.
%    \begin{macrocode}
\@ifclassloaded{letter}{}{% other
\@ifclassloaded{slides}{}{% other
\@ifclassloaded{article}{%
  \renewenvironment{thebibliography}[1]
  {\section*{\refname\@mkboth%
      {\MakeUppercase\refname}%
      {\MakeUppercase\refname}}%
    \list{\@biblabel{\@arabic\c@enumiv}}%
    {\settowidth\labelwidth{\@biblabel{#1}}%
      \if@rl\leftmargin\else\rightmargin\fi\labelwidth
      \advance\if@rl\leftmargin\else\rightmargin\fi\labelsep
      \@openbib@code
      \usecounter{enumiv}%
      \let\p@enumiv\@empty
      \renewcommand\theenumiv{\@arabic\c@enumiv}}%
    \sloppy
    \clubpenalty4000
    \@clubpenalty \clubpenalty
    \widowpenalty4000%
    \sfcode`\.\@m}
  {\def\@noitemerr
    {\@latex@warning{Empty `thebibliography' environment}}%
     \endlist}}%
{\renewenvironment{thebibliography}[1]{%
    \chapter*{\bibname\@mkboth%
      {\MakeUppercase\bibname}%
      {\MakeUppercase\bibname}}%
    \list{\@biblabel{\@arabic\c@enumiv}}%
    {\settowidth\labelwidth{\@biblabel{#1}}%
      \if@rl\leftmargin\else\rightmargin\fi\labelwidth
      \advance\if@rl\leftmargin\else\rightmargin\fi\labelsep
      \@openbib@code
      \usecounter{enumiv}%
      \let\p@enumiv\@empty
      \renewcommand\theenumiv{\@arabic\c@enumiv}}%
    \sloppy
    \clubpenalty4000
    \@clubpenalty \clubpenalty
    \widowpenalty4000%
    \sfcode`\.\@m}
  {\def\@noitemerr
    {\@latex@warning{Empty `thebibliography' environment}}%
     \endlist}}}}
%    \end{macrocode}
% \end{environment}
%
% \begin{macro}{\@verbatim}
%    All kinds of verbs (\cs{verb},\cs{verb*},\texttt{verbatim} and
%    \texttt{verbatim*}) now can be used in Right-to-Left mode. Errors
%    in latin mode solved too.
%    \begin{macrocode}
\def\@verbatim{%
  \let\do\@makeother \dospecials%
  \obeylines \verbatim@font \@noligs}
%    \end{macrocode}
% \end{macro}
%
% \begin{macro}{\@makecaption}
%    Captions are set always centered. This allows us to use bilingual
%    captions, for example: |\caption{\R{RLtext} \\ \L{LRtext}}|,
%    which will be formatted as:
%    \begin{center}
%    Right to left caption here (RLtext) \\
%    Left to right caption here (LRtext)
%    \end{center}
%    See also \cs{bcaption} command below.
%    \begin{macrocode}
\long\def\@makecaption#1#2{%
  \vskip\abovecaptionskip%
  \begin{center}%
    #1: #2%
  \end{center} \par%
  \vskip\belowcaptionskip}
%    \end{macrocode} 
% \end{macro}
%
% \subsubsection{Additional bidirectional commands}
%
%    \begin{itemize}
%    \item Section headings are typeset with the default global
%    direction.
%    \item Text in section headings in the reverse language \emph{do
%    not} have to be protected for the reflection command, as in:
%    |\protect\L{|\emph{Latin Text}|}|, because \cs{L} and \cs{R} are
%    robust now.
%    \item Table of contents, list of figures and list of tables
%    should be typeset with the \cs{tableofcontents},
%    \cs{listoffigures} and \cs{listoftables} commands respectively.
%    \item The above tables will be typeset in the main direction (and
%    language) in effect where the above commands are placed.
%    \item Only 2 tables of each kind are supported: one for
%    Right-to-Left and another for Left-to-Right directions.
%    \end{itemize}
%
%    How to include line to both tables? One has to use bidirectional
%    sectioning commands as following:
%    \begin{enumerate}
%    \item Use the |\b|\emph{xxx} version of the sectioning commands
%    in the text instead of the |\|\emph{xxx} version (\emph{xxx} is
%    one of: \texttt{part}, \texttt{chapter}, \texttt{section}, 
%    \texttt{subsection}, \texttt{subsubsection}, \texttt{caption}).
%    \item Syntax of the |\b|\emph{xxx} command is
%        |\b|\emph{xxx}|{|\emph{RL text}|}{|\emph{LR text}|}|.
%    Both arguments are typeset in proper direction by default (no
%    need to change direction for the text inside).
%    \item The section header inside the document will be typeset in
%    the global direction in effect at the time. i.e. The |{|\emph{RL
%    text}|}| will be typeset if Right-to-Left mode is in effect and
%    |{|\emph{LR text}|}| otherwise.
%    \end{enumerate}
%
% \begin{macro}{\bpart}
%    \begin{macrocode}
\newcommand{\bpart}[2]{\part{\protect\if@rl%
    #1 \protect\else #2 \protect\fi}}
%    \end{macrocode}
% \end{macro}
%
% \begin{macro}{\bchapter}
%    \begin{macrocode}
\newcommand{\bchapter}[2]{\chapter{\protect\if@rl%
    #1 \protect\else #2 \protect\fi}}
%    \end{macrocode}
% \end{macro}
%
% \begin{macro}{\bsection}
%    \begin{macrocode}
\newcommand{\bsection}[2]{\section{\protect\if@rl%
    #1 \protect\else #2 \protect\fi}}
%    \end{macrocode}
% \end{macro}
%
% \begin{macro}{\bsubsection}
%    \begin{macrocode}
\newcommand{\bsubsection}[2]{\subsection{\protect\if@rl%
    #1 \protect\else #2 \protect\fi}}
%    \end{macrocode}
% \end{macro}
%
% \begin{macro}{\bsubsubsection}
%    \begin{macrocode}
\newcommand{\bsubsubsection}[2]{\subsubsection{\protect\if@rl%
    #1 \protect\else #2 \protect\fi}}
%    \end{macrocode}
% \end{macro}
%
% \begin{macro}{\bcaption}
%    \begin{macrocode}
\newcommand{\bcaption}[2]{%
  \caption[\protect\if@rl \R{#1}\protect\else \L{#2}\protect\fi]{%
    \if@rl\R{#1}\protect\\ \L{#2}
    \else\L{#2}\protect\\ \R{#1}\fi}}
%    \end{macrocode}
% \end{macro}
%
%    The following definition is a modified version of \cs{bchapter}, meant
%    as a bilingual twin for \cs{chapter*} and \cs{section*}
%    (added by Irina Abramovici).
%
% \begin{macro}{\bchapternn}
%    \begin{macrocode}
\newcommand{\bchapternn}[2]{\chapter*{\protect\if@rl% 
    #1 \protect\else #2 \protect\fi}}
%    \end{macrocode}
% \end{macro}
%
% \begin{macro}{\bsectionnn}
%    \begin{macrocode}
\newcommand{\bsectionnn}[2]{\section*{\protect\if@rl%
    #1 \protect\else #2 \protect\fi}}
%    \end{macrocode}
% \end{macro}
%
%    Finally, at end of \babel\ package, the \cs{headlanguage} and
%    two-column mode will be initialized according to the current
%    language.
%    \begin{macrocode}
\AtEndOfPackage{\rlAtEndOfPackage}
%
\def\rlAtEndOfPackage{%
  \global\let\headlanguage=\languagename%\set@outputdblcol%
}
%</rightleft>
%    \end{macrocode}
%
% \subsection{Hebrew calendar}
%
%    The original version of the package \pkg{hebcal.sty}\footnote{The
%    following description of \pkg{hebcal} package is based on the
%    comments included with original source by the author, Michail
%    Rozman.} for \TeX\ and \LaTeX2.09, entitled ``\TeX{} \& \LaTeX{}
%    macros for computing Hebrew date from Gregorian one'' was created
%    by Michail Rozman, |misha@iop.tartu.ew.su|\footnote{Please direct
%    any comments, bug reports, questions, etc. about the package to
%    this address.}
%
%    \begin{tabular}{@{}lr@{}c@{}ll}
%    Released: &Tammuz 12, 5751&--&June 24, 1991   &\\
%    Corrected:&Shebat 10, 5752&--&January 15, 1992&by Rama Porrat\\
%    Corrected:&Adar II 5, 5752&--&March 10, 1992  &by Misha\\
%    Corrected:&Tebeth, 5756   &--&January 1996    &Dan Haran\\
%              &&&&(haran@math.tau.ac.il)
%    \end{tabular}
%
%    The package was adjusted for \babel{} and \LaTeXe{} by Boris
%    Lavva.
%
%    Changes to the printing routine (only) by Ron Artstein, June 1,
%    2003.
%
%    This package should be included \emph{after} the \pkg{babel} with
%    \pkg{hebrew} option, as following:
%    \begin{quote}
%       |\documentclass[|\ldots|]{|\ldots|}|\\
%       |\usepackage[hebrew,|\ldots|,|\emph{other languages}|,|
%                            \ldots|]{babel}|\\
%       |\usepackage{hebcal}|
%    \end{quote}
%
%    Two main user-level commands are provided by this package:
%
%    \DescribeMacro{\Hebrewtoday}
%    Computes today's Hebrew date and prints it. If we are presently
%    in Hebrew mode, the date will be printed in Hebrew, otherwise ---
%    in English (like Shebat 10, 5752).
%
%    \DescribeMacro{\Hebrewdate}
%    Computes the Hebrew date from the given Gregorian date and
%    prints it. If we are presently in Hebrew mode, the date will be
%    printed in Hebrew, otherwise --- in English (like Shebat 10,
%    5752). An example of usage is shown below:
%    \begin{quote}
%       |\newcount\hd \newcount\hm \newcount\hy|\\
%       |\hd=10 \hm=3 \hy=1992|\\
%       |\Hebrewdate{\hd}{\hm}{\hy}|
%    \end{quote}
%
%    \DescribeMacro{full}
%    The package option |full| sets the flag |\@full@hebrew@year|,
%    which causes years from the current millenium to be printed with
%    the thousands digit (he-tav-shin-samekh-gimel). Without this 
%    option, thousands are not printed for the current millenium.
%    NOTE: should this be a command option rather than a package
%    option? --RA.
%
% \subsubsection{Introduction}
%
%    The Hebrew calendar is inherently complicated: it is lunisolar --
%    each year starts close to the autumn equinox, but each month must
%    strictly start at a new moon.  Thus Hebrew calendar must be
%    harmonized simultaneously with both lunar and solar events. In
%    addition, for reasons of the religious practice, the year cannot
%    start on Sunday, Wednesday or Friday.
%
%    For the full description of Hebrew calendar and for the list of
%    references see:
%    \begin{quote}
%      Nachum Dershowitz and Edward M. Reingold,
%      \emph{``Calendarical Calculations''}, Software--Pract.Exper.,
%      vol. 20 (9), pp.899--928 (September 1990).
%    \end{quote}
%    |C| translation of |LISP| programs from the above article
%    available from Mr. Wayne Geiser, |geiser%pictel@uunet.uu.net|.
%
%    The 4\textsuperscript{th} distribution (July 1989) of hdate/hcal
%    (Hebrew calendar programs similar to UNIX date/cal) by Mr. Amos
%    Shapir, |amos@shum.huji.ac.il|, contains short and very clear
%    description of algorithms.
%
% \subsubsection{Registers, Commands, Formatting Macros}
%
%    The command |\Hebrewtoday| produces today's date for Hebrew
%    calendar. It is similar to the standard \LaTeXe{} command
%    |\today|. In addition three numerical registers |\Hebrewday|,
%    |\Hebrewmonth| and |\Hebrewyear| are set.
%    For setting this registers without producing of date string
%    command |\Hebrewsetreg| can be used.
%
%    The command 
%    |\Hebrewdate{|\emph{Gday}|}{|\emph{Gmonth}|}{|\emph{Gyear}|}|
%    produces Hebrew calendar date corresponding to Gregorian date 
%    |Gday.Gmonth.Gyear|. Three numerical registers |\Hebrewday|,
%    |\Hebrewmonth| and |\Hebrewyear| are set.
%
%    For converting arbitrary Gregorian date |Gday.Gmonth.Gyear|
%    to Hebrew date |Hday.Hmonth.Hyear| without producing date string
%    the command:
%    \begin{center}
%      |\HebrewFromGregorian{|\emph{Gday}|}{|\emph{Gmonth}|}{|%
%      \emph{Gyear}|}{|\emph{Hday}|}{|\emph{Hmonth}|}{|\emph{Hyear}|}|
%    \end{center}
%    can be used.
%
%    \begin{macrocode}
%<*calendar>
\newif\if@full@hebrew@year
\@full@hebrew@yearfalse
\DeclareOption{full}{\@full@hebrew@yeartrue}
\ProcessOptions
\newcount\Hebrewday  \newcount\Hebrewmonth   \newcount\Hebrewyear
%    \end{macrocode}
%
% \begin{macro}{\Hebrewdate}
%    Hebrew calendar date corresponding to Gregorian date
%    |Gday.Gmonth.Gyear|. If Hebrew (right-to-left) fonts \& macros
%    are not loaded, we have to use English format.
%    \begin{macrocode}
\def\Hebrewdate#1#2#3{%
    \HebrewFromGregorian{#1}{#2}{#3}
                        {\Hebrewday}{\Hebrewmonth}{\Hebrewyear}%
    \ifundefined{if@rl}% 
       \FormatForEnglish{\Hebrewday}{\Hebrewmonth}{\Hebrewyear}%
    \else%
       \FormatDate{\Hebrewday}{\Hebrewmonth}{\Hebrewyear}%
    \fi}
%    \end{macrocode}
% \end{macro}
%
% \begin{macro}{\Hebrewtoday}
%    Today's date in Hebrew calendar.
%    \begin{macrocode}
\def\Hebrewtoday{\Hebrewdate{\day}{\month}{\year}}
\let\hebrewtoday=\Hebrewtoday
%    \end{macrocode}
% \end{macro}
%
% \begin{macro}{\Hebrewsetreg}
%    Set registers: today's date in hebrew calendar.
%    \begin{macrocode}
\def\Hebrewsetreg{%
    \HebrewFromGregorian{\day}{\month}{\year}
                        {\Hebrewday}{\Hebrewmonth}{\Hebrewyear}}
%    \end{macrocode}
% \end{macro}
%
% \begin{macro}{\FormatDate}
%    Prints a Hebrew calendar date |Hebrewday.Hebrewmonth.Hebrewyear|.
%    \begin{macrocode}
\def\FormatDate#1#2#3{%
        \if@rl%
            \FormatForHebrew{#1}{#2}{#3}%
        \else%
            \FormatForEnglish{#1}{#2}{#3}
        \fi}
%    \end{macrocode}
% \end{macro}
%
%    To prepare another language version of Hebrew calendar commands,
%    one should change or add commands here.
%
%    We start with Hebrew language macros.
% \begin{macro}{\HebrewYearName}
%    Prints Hebrew year as a Hebrew number. Disambiguates strings by
%    adding lamed-pe-gimel to years of the first Jewish millenium and
%    to years divisible by 1000. Suppresses the thousands digit in the
%    current millenium unless the package option |full| is selected.
%    NOTE: should this be provided as a command option rather than a
%    package option? --RA.
%    \begin{macrocode}
\def\HebrewYearName#1{{%
   \@tempcnta=#1\divide\@tempcnta by 1000\multiply\@tempcnta by 1000
   \ifnum#1=\@tempcnta\relax % divisible by 1000: disambiguate
     \Hebrewnumeralfinal{#1}\ )\heblamed\hebpe"\hebgimel(%
   \else % not divisible by 1000
     \ifnum#1<1000\relax     % first millennium: disambiguate
       \Hebrewnumeralfinal{#1}\ )\heblamed\hebpe"\hebgimel(%
     \else 
       \ifnum#1<5000
         \Hebrewnumeralfinal{#1}%
       \else
         \ifnum#1<6000 % current millenium, print without thousands
           \@tempcnta=#1\relax
           \if@full@hebrew@year\else\advance\@tempcnta by -5000\fi
           \Hebrewnumeralfinal{\@tempcnta}%
         \else % #1>6000
           \Hebrewnumeralfinal{#1}%
         \fi
       \fi
     \fi
   \fi}}
%    \end{macrocode}
% \end{macro}
%
% \begin{macro}{\HebrewMonthName}
%    The macro |\HebrewMonthName{|\emph{month}|}{|\emph{year}|}|
%    returns the name of month in the `year'.
%    \begin{macrocode}
\def\HebrewMonthName#1#2{%
    \ifnum #1 = 7 %
    \CheckLeapHebrewYear{#2}%
        \if@HebrewLeap \hebalef\hebdalet\hebresh\ \hebbet'%
           \else \hebalef\hebdalet\hebresh%
        \fi%
    \else%
        \ifcase#1%
           % nothing for 0           
           \or\hebtav\hebshin\hebresh\hebyod%
           \or\hebhet\hebshin\hebvav\hebfinalnun%
           \or\hebkaf\hebsamekh\heblamed\hebvav%
           \or\hebtet\hebbet\hebtav%
           \or\hebshin\hebbet\hebtet%
           \or\hebalef\hebdalet\hebresh\ \hebalef'%
           \or\hebalef\hebdalet\hebresh\ \hebbet'%
           \or\hebnun\hebyod\hebsamekh\hebfinalnun%
           \or\hebalef\hebyod\hebyod\hebresh%
           \or\hebsamekh\hebyod\hebvav\hebfinalnun%
           \or\hebtav\hebmem\hebvav\hebzayin%
           \or\hebalef\hebbet%
           \or\hebalef\heblamed\hebvav\heblamed%
        \fi%
    \fi}
%    \end{macrocode}
% \end{macro}
%
% \begin{macro}{\HebrewDayName}
%    Name of day in Hebrew letters (gimatria).
%    \begin{macrocode}
\def\HebrewDayName#1{\Hebrewnumeral{#1}}
%    \end{macrocode}
% \end{macro}
%
%
% \begin{macro}{\FormatForHebrew}
%    The macro |\FormatForHebrew{|\emph{hday}|}{|\emph{hmonth}
%    |}{|\emph{hyear}|}| returns the formatted Hebrew date in Hebrew
%    language.
%    \begin{macrocode}
\def\FormatForHebrew#1#2#3{%
  \HebrewDayName{#1}~\hebbet\HebrewMonthName{#2}{#3},~%
  \HebrewYearName{#3}}
%    \end{macrocode}
% \end{macro}
%
%    We continue with two English language macros for Hebrew calendar.
% \begin{macro}{\HebrewMonthNameInEnglish}
%    The macro |\HebrewMonthNameInEnglish{|\emph{month}|}{|%
%    \emph{year}|}| is similar to |\Hebrew|\-|Month|\-|Name| described
%    above. It returns the name of month in the Hebrew `year' in
%    English. 
%    \begin{macrocode}
\def\HebrewMonthNameInEnglish#1#2{%
    \ifnum #1 = 7%
    \CheckLeapHebrewYear{#2}%
        \if@HebrewLeap Adar II\else Adar\fi%
    \else%
        \ifcase #1%
            % nothing for 0
            \or Tishrei%
            \or Heshvan%
            \or Kislev%
            \or Tebeth%
            \or Shebat%
            \or Adar I%
            \or Adar II%
            \or Nisan%
            \or Iyar%
            \or Sivan%
            \or Tammuz%
            \or Av%
            \or Elul%
        \fi
    \fi}
%    \end{macrocode}
% \end{macro}
%
% \begin{macro}{\FormatForEnglish}
%    The macro |\FormatForEnglish{|\emph{hday}|}{|\emph{hmonth}
%    |}{|\emph{hyear}|}| is similar to |\Format|\-|For|\-|Hebrew|
%    macro described above and returns the formatted Hebrew date in
%    English.
%    \begin{macrocode}
\def\FormatForEnglish#1#2#3{%
    \HebrewMonthNameInEnglish{#2}{#3} \number#1,\ \number#3}
%    \end{macrocode}
% \end{macro}
%
% \subsubsection{Auxiliary Macros}
%
%    \begin{macrocode}
\newcount\@common
%    \end{macrocode}
% \begin{macro}{\Remainder}
%    |\Remainder{|\emph{a}|}{|\emph{b}|}{|\emph{c}|}| calculates 
%    $c = a\%b == a-b\times\frac{a}{b}$
%    \begin{macrocode}
\def\Remainder#1#2#3{%
    #3 = #1%                   %  c = a
    \divide #3 by #2%          %  c = a/b
    \multiply #3 by -#2%       %  c = -b(a/b)
    \advance #3 by #1}%        %  c = a - b(a/b)
%    \end{macrocode}
% \end{macro}
%    \begin{macrocode}
\newif\if@Divisible
%    \end{macrocode}
% \begin{macro}{\CheckIfDivisible}
%    |\CheckIfDivisible{|\emph{a}|}{|\emph{b}|}| sets
%    |\@Divisibletrue| if $a\%b == 0$
%    \begin{macrocode}
\def\CheckIfDivisible#1#2{%
    {%
      \countdef\tmp = 0% \tmp == \count0 - temporary variable
      \Remainder{#1}{#2}{\tmp}%
      \ifnum \tmp = 0%
          \global\@Divisibletrue%
      \else%
          \global\@Divisiblefalse%
      \fi}}
%    \end{macrocode}
% \end{macro}
%
% \begin{macro}{\ifundefined}
%    From the \TeX book, ex. 7.7: 
%    \begin{quote}
%       |\ifundefined{|\emph{command}|}<true text>\else<false text>\fi|
%    \end{quote}
%    \begin{macrocode}
\def\ifundefined#1{\expandafter\ifx\csname#1\endcsname\relax}
%    \end{macrocode}
% \end{macro}
%
% \subsubsection{Gregorian Part}
%
%    \begin{macrocode}
\newif\if@GregorianLeap
%    \end{macrocode}
% \begin{macro}{\IfGregorianLeap}
%    Conditional which is true if Gregorian `year' is a leap year:
%    $((year\%4==0)\wedge(year\%100\neq 0))\vee(year\%400==0)$
%    \begin{macrocode}
\def\IfGregorianLeap#1{%
    \CheckIfDivisible{#1}{4}%
    \if@Divisible%
        \CheckIfDivisible{#1}{100}%
        \if@Divisible%
            \CheckIfDivisible{#1}{400}%
            \if@Divisible%
                \@GregorianLeaptrue%
            \else%
                \@GregorianLeapfalse%
            \fi%
        \else%
            \@GregorianLeaptrue%
        \fi%
    \else%
        \@GregorianLeapfalse%
    \fi%
    \if@GregorianLeap}
%    \end{macrocode}
% \end{macro}
%
% \begin{macro}{\GregorianDaysInPriorMonths}
%    The macro |\GregorianDaysInPriorMonths{|\emph{month}|}{|^^A
%    \emph{year}|}{|\emph{days}|}| calculates the number of days in
%    months prior to `month' in the `year'.
%    \begin{macrocode}
\def\GregorianDaysInPriorMonths#1#2#3{%
    {%
        #3 = \ifcase #1%
               0 \or%             % no month number 0
               0 \or%
              31 \or%
              59 \or%
              90 \or%
             120 \or%
             151 \or%
             181 \or%
             212 \or%
             243 \or%
             273 \or%
             304 \or%
             334%
        \fi%
        \IfGregorianLeap{#2}%
            \ifnum #1 > 2%        % if month after February
                \advance #3 by 1% % add leap day
            \fi%
        \fi%
        \global\@common = #3}%
    #3 = \@common}
%    \end{macrocode}
% \end{macro}
%
% \begin{macro}{\GregorianDaysInPriorYears}
%    The macro |\GregorianDaysInPriorYears{|\emph{year}|}{|^^A
%    \emph{days}|}| calculates the number of days in years prior to
%    the `year'.
%    \begin{macrocode}
\def\GregorianDaysInPriorYears#1#2{%
     {%
         \countdef\tmpc = 4%      % \tmpc==\count4
         \countdef\tmpb = 2%      % \tmpb==\count2
         \tmpb = #1%              %
         \advance \tmpb by -1%    %
         \tmpc = \tmpb%           % \tmpc = \tmpb = year-1
         \multiply \tmpc by 365%  % Days in prior years =
         #2 = \tmpc%              % = 365*(year-1) ...
         \tmpc = \tmpb%           %
         \divide \tmpc by 4%      % \tmpc = (year-1)/4
         \advance #2 by \tmpc%    % ... plus Julian leap days ...
         \tmpc = \tmpb%           %
         \divide \tmpc by 100%    % \tmpc = (year-1)/100
         \advance #2 by -\tmpc%   % ... minus century years ...
         \tmpc = \tmpb%           %
         \divide \tmpc by 400%    % \tmpc = (year-1)/400
         \advance #2 by \tmpc%    % ... plus 4-century years.
         \global\@common = #2}%
    #2 = \@common}
%    \end{macrocode}
% \end{macro}
%
% \begin{macro}{\AbsoluteFromGregorian}
%    The macro |\AbsoluteFromGregorian{|\emph{day}|}{|\emph{month}^^A
%    |}{|\emph{year}|}{|\emph{absdate}|}| calculates the absolute date
%    (days since $01.01.0001$) from Gregorian date |day.month.year|.
%    \begin{macrocode}
\def\AbsoluteFromGregorian#1#2#3#4{%
    {%
        \countdef\tmpd = 0%       % \tmpd==\count0
        #4 = #1%                  % days so far this month
        \GregorianDaysInPriorMonths{#2}{#3}{\tmpd}%
        \advance #4 by \tmpd%     % add days in prior months
        \GregorianDaysInPriorYears{#3}{\tmpd}%
        \advance #4 by \tmpd%     % add days in prior years
        \global\@common = #4}%
    #4 = \@common}
%    \end{macrocode}
% \end{macro}
%
% \subsubsection{Hebrew Part}
%
%    \begin{macrocode}
\newif\if@HebrewLeap
%    \end{macrocode}
% \begin{macro}{\CheckLeapHebrewYear}
%    Set |\@HebrewLeaptrue| if Hebrew `year' is a leap year, i.e.\ if
%    $(1+7\times year)\%19 < 7$ then \emph{true} else \emph{false}
%    \begin{macrocode}
\def\CheckLeapHebrewYear#1{%
    {%
        \countdef\tmpa = 0%       % \tmpa==\count0
        \countdef\tmpb = 1%       % \tmpb==\count1
%
        \tmpa = #1%
        \multiply \tmpa by 7%
        \advance \tmpa by 1%
        \Remainder{\tmpa}{19}{\tmpb}%
        \ifnum \tmpb < 7%         % \tmpb = (7*year+1)%19
            \global\@HebrewLeaptrue%
        \else%
            \global\@HebrewLeapfalse%
        \fi}}
%    \end{macrocode}
% \end{macro}
%
% \begin{macro}{\HebrewElapsedMonths}
%    The macro |\HebrewElapsedMonths{|\emph{year}|}{|\emph{months}|}|
%    determines the number of months elapsed from the Sunday prior to
%    the start of the Hebrew calendar to the mean conjunction of
%    Tishri of Hebrew `year'.
%    \begin{macrocode}
\def\HebrewElapsedMonths#1#2{%
    {%
        \countdef\tmpa = 0%       % \tmpa==\count0
        \countdef\tmpb = 1%       % \tmpb==\count1
        \countdef\tmpc = 2%       % \tmpc==\count2
%
        \tmpa = #1%               %
        \advance \tmpa by -1%     %
        #2 = \tmpa%               % #2 = \tmpa = year-1
        \divide #2 by 19%         % Number of complete Meton cycles
        \multiply #2 by 235%      % #2 = 235*((year-1)/19)
%
        \Remainder{\tmpa}{19}{\tmpb}% \tmpa = years%19-years this cycle
        \tmpc = \tmpb%            %
        \multiply \tmpb by 12%    %
        \advance #2 by \tmpb%     % add regular months this cycle
%
        \multiply \tmpc by 7%     %
        \advance \tmpc by 1%      %
        \divide \tmpc by 19%      % \tmpc = (1+7*((year-1)%19))/19 -
%                                 %  number of leap months this cycle
        \advance #2 by \tmpc%     %  add leap months
%
        \global\@common = #2}%
    #2 = \@common}
%    \end{macrocode}
% \end{macro}
%
% \begin{macro}{\HebrewElapsedDays}
%    The macro |\HebrewElapsedDays{|\emph{year}|}{|\emph{days}|}|
%    determines the number of days elapsed from the Sunday prior to
%    the start of the Hebrew calendar to the mean conjunction of
%    Tishri of Hebrew `year'.
%    \begin{macrocode}
\def\HebrewElapsedDays#1#2{%
    {%
        \countdef\tmpa = 0%       % \tmpa==\count0
        \countdef\tmpb = 1%       % \tmpb==\count1
        \countdef\tmpc = 2%       % \tmpc==\count2
%
        \HebrewElapsedMonths{#1}{#2}%
        \tmpa = #2%               %
        \multiply \tmpa by 13753% %
        \advance \tmpa by 5604%   % \tmpa=MonthsElapsed*13758 + 5604
        \Remainder{\tmpa}{25920}{\tmpc}% \tmpc == ConjunctionParts
        \divide \tmpa by 25920%
%
        \multiply #2 by 29%
        \advance #2 by 1%
        \advance #2 by \tmpa%     %  #2 = 1 + MonthsElapsed*29 +
%                                 %          PartsElapsed/25920
        \Remainder{#2}{7}{\tmpa}% %  \tmpa == DayOfWeek
        \ifnum \tmpc < 19440%
            \ifnum \tmpc < 9924%
            \else%                % New moon at 9 h. 204 p. or later
                \ifnum \tmpa = 2% % on Tuesday ...
                    \CheckLeapHebrewYear{#1}% of a common year
                    \if@HebrewLeap%
                    \else%
                        \advance #2 by 1%
                    \fi%
                \fi%
            \fi%
            \ifnum \tmpc < 16789%
            \else%                 % New moon at 15 h. 589 p. or later
                \ifnum \tmpa = 1%  % on Monday ...
                    \advance #1 by -1%
                    \CheckLeapHebrewYear{#1}% at the end of leap year
                    \if@HebrewLeap%
                        \advance #2 by 1%
                    \fi%
                \fi%
            \fi%
        \else%
            \advance #2 by 1%      %  new moon at or after midday
        \fi%
%
        \Remainder{#2}{7}{\tmpa}%  %  \tmpa == DayOfWeek
        \ifnum \tmpa = 0%          %  if Sunday ...
            \advance #2 by 1%
        \else%                     %
            \ifnum \tmpa = 3%      %  Wednesday ...
                \advance #2 by 1%
            \else%
                \ifnum \tmpa = 5%  %  or Friday
                     \advance #2 by 1%
                \fi%
            \fi%
        \fi%
        \global\@common = #2}%
    #2 = \@common}
%    \end{macrocode}
% \end{macro}
%
% \begin{macro}{\DaysInHebrewYear}
%    The macro |\DaysInHebrewYear{|\emph{year}|}{|\emph{days}|}|
%    calculates the number of days in Hebrew `year'.
%    \begin{macrocode}
\def\DaysInHebrewYear#1#2{%
    {%
        \countdef\tmpe = 12%   % \tmpe==\count12
%
        \HebrewElapsedDays{#1}{\tmpe}%
        \advance #1 by 1%
        \HebrewElapsedDays{#1}{#2}%
        \advance #2 by -\tmpe%
        \global\@common = #2}%
    #2 = \@common}
%    \end{macrocode}
% \end{macro}
%
% \begin{macro}{\HebrewDaysInPriorMonths}
%    The macro |\HebrewDaysInPriorMonths{|\emph{month}|}{|^^A
%    \emph{year}|}{|\emph{days}|}| calculates the nu\-mber of days in
%    months prior to `month' in the `year'.
%    \begin{macrocode}
\def\HebrewDaysInPriorMonths#1#2#3{%
    {%
        \countdef\tmpf= 14%    % \tmpf==\count14
%
        #3 = \ifcase #1%       % Days in prior month of regular year
               0 \or%          % no month number 0
               0 \or%          % Tishri
              30 \or%          % Heshvan
              59 \or%          % Kislev
              89 \or%          % Tebeth
             118 \or%          % Shebat
             148 \or%          % Adar I
             148 \or%          % Adar II
             177 \or%          % Nisan
             207 \or%          % Iyar
             236 \or%          % Sivan
             266 \or%          % Tammuz
             295 \or%          % Av
             325 \or%          % Elul
             400%              % Dummy
        \fi%
        \CheckLeapHebrewYear{#2}%
        \if@HebrewLeap%            % in leap year
            \ifnum #1 > 6%         % if month after Adar I
                \advance #3 by 30% % add  30 days
            \fi%
        \fi%
        \DaysInHebrewYear{#2}{\tmpf}%
        \ifnum #1 > 3%
            \ifnum \tmpf = 353%    %
                \advance #3 by -1% %
            \fi%                   %  Short Kislev
            \ifnum \tmpf = 383%    %
                \advance #3 by -1% %
            \fi%                   %
        \fi%
%
        \ifnum #1 > 2%
            \ifnum \tmpf = 355%    %
                \advance #3 by 1%  %
            \fi%                   %  Long Heshvan
            \ifnum \tmpf = 385%    %
                \advance #3 by 1%  %
            \fi%                   %
        \fi%
        \global\@common = #3}%
    #3 = \@common}
%    \end{macrocode}
% \end{macro}
%
% \begin{macro}{\AbsoluteFromHebrew}
%    The macro |\AbsoluteFromHebrew{|\emph{day}|}{|\emph{month}^^A
%    |}{|\emph{year}|}{|\emph{absdate}|}| calculates the absolute date
%    of Hebrew date |day.month.year|.
%    \begin{macrocode}
\def\AbsoluteFromHebrew#1#2#3#4{%
    {%
        #4 = #1%
        \HebrewDaysInPriorMonths{#2}{#3}{#1}%
        \advance #4 by #1%         % Add days in prior months this year
        \HebrewElapsedDays{#3}{#1}%
        \advance #4 by #1%         % Add days in prior years
        \advance #4 by -1373429%   % Subtract days before Gregorian
        \global\@common = #4}%     %   01.01.0001
    #4 = \@common}
%    \end{macrocode}
% \end{macro}
%
% \begin{macro}{\HebrewFromGregorian}
%    The macro |\HebrewFromGregorian{|\emph{Gday}|}{|\emph{Gmonth}^^A
%    |}{|\emph{Gyear}|}{|\emph{Hday}|}{|\emph{Hmonth}|}|\-|{|^^A
%    \emph{Hyear}|}| evaluates Hebrew date |Hday|, |Hmonth|, |Hyear|
%    from Gregorian date |Gday|, |Gmonth|, |Gyear|.
%    \begin{macrocode}
\def\HebrewFromGregorian#1#2#3#4#5#6{%
    {%
        \countdef\tmpx= 17%        % \tmpx==\count17
        \countdef\tmpy= 18%        % \tmpy==\count18
        \countdef\tmpz= 19%        % \tmpz==\count19
%
        #6 = #3%                   %
        \global\advance #6 by 3761%  approximation from above
        \AbsoluteFromGregorian{#1}{#2}{#3}{#4}%
        \tmpz = 1  \tmpy = 1%
        \AbsoluteFromHebrew{\tmpz}{\tmpy}{#6}{\tmpx}%
        \ifnum \tmpx > #4%              %
            \global\advance #6 by -1% Hyear = Gyear + 3760
            \AbsoluteFromHebrew{\tmpz}{\tmpy}{#6}{\tmpx}%
        \fi%                            %
        \advance #4 by -\tmpx%     % Days in this year
        \advance #4 by 1%          %
        #5 = #4%                   %
        \divide #5 by 30%          % Approximation for month from below
        \loop%                     % Search for month
            \HebrewDaysInPriorMonths{#5}{#6}{\tmpx}%
            \ifnum \tmpx < #4%
                \advance #5 by 1%
                \tmpy = \tmpx%
        \repeat%
        \global\advance #5 by -1%
        \global\advance #4 by -\tmpy}}
%</calendar>
%    \end{macrocode}
% \end{macro}
%
% \Finale
%%
%% \CharacterTable
%%  {Upper-case    \A\B\C\D\E\F\G\H\I\J\K\L\M\N\O\P\Q\R\S\T\U\V\W\X\Y\Z
%%   Lower-case    \a\b\c\d\e\f\g\h\i\j\k\l\m\n\o\p\q\r\s\t\u\v\w\x\y\z
%%   Digits        \0\1\2\3\4\5\6\7\8\9
%%   Exclamation   \!     Double quote  \"     Hash (number) \#
%%   Dollar        \$     Percent       \%     Ampersand     \&
%%   Acute accent  \'     Left paren    \(     Right paren   \)
%%   Asterisk      \*     Plus          \+     Comma         \,
%%   Minus         \-     Point         \.     Solidus       \/
%%   Colon         \:     Semicolon     \;     Less than     \<
%%   Equals        \=     Greater than  \>     Question mark \?
%%   Commercial at \@     Left bracket  \[     Backslash     \\
%%   Right bracket \]     Circumflex    \^     Underscore    \_
%%   Grave accent  \`     Left brace    \{     Vertical bar  \|
%%   Right brace   \}     Tilde         \~}
%%
\endinput
}
%    \end{macrocode}
%    \Lopt{hungarian} is just a synonym for \Lopt{magyar}
%^^A \changes{babel~3.6i}{1997/02/21}{Added the \Lopt{kannada} option}
% \changes{babel~3.7a}{1997/05/21}{Added the \Lopt{icelandic} option}
%^^A \changes{babel~3.7a}{1998/03/30}{{Added the \Lopt{nagari} option}
% \changes{babel~3.7b}{1998/06/25}{Added the \Lopt{latin} option}
% \changes{babel~3.7m}{2003/11/13}{Added the \Lopt{interlingua}
%    option}
%    \begin{macrocode}
\DeclareOption{hungarian}{%%
%% This file will generate fast loadable files and documentation
%% driver files from the doc files in this package when run through
%% LaTeX or TeX.
%%
%% Copyright 1989-2005 Johannes L. Braams and any individual authors
%% listed elsewhere in this file.  All rights reserved.
%% 
%% This file is part of the Babel system.
%% --------------------------------------
%% 
%% It may be distributed and/or modified under the
%% conditions of the LaTeX Project Public License, either version 1.3
%% of this license or (at your option) any later version.
%% The latest version of this license is in
%%   http://www.latex-project.org/lppl.txt
%% and version 1.3 or later is part of all distributions of LaTeX
%% version 2003/12/01 or later.
%% 
%% This work has the LPPL maintenance status "maintained".
%% 
%% The Current Maintainer of this work is Johannes Braams.
%% 
%% The list of all files belonging to the LaTeX base distribution is
%% given in the file `manifest.bbl. See also `legal.bbl' for additional
%% information.
%% 
%% The list of derived (unpacked) files belonging to the distribution
%% and covered by LPPL is defined by the unpacking scripts (with
%% extension .ins) which are part of the distribution.
%%
%% --------------- start of docstrip commands ------------------
%%
\def\filedate{1999/04/11}
\def\batchfile{magyar.ins}
\input docstrip.tex

{\ifx\generate\undefined
\Msg{**********************************************}
\Msg{*}
\Msg{* This installation requires docstrip}
\Msg{* version 2.3c or later.}
\Msg{*}
\Msg{* An older version of docstrip has been input}
\Msg{*}
\Msg{**********************************************}
\errhelp{Move or rename old docstrip.tex.}
\errmessage{Old docstrip in input path}
\batchmode
\csname @@end\endcsname
\fi}

\declarepreamble\mainpreamble
This is a generated file.

Copyright 1989-2005 Johannes L. Braams and any individual authors
listed elsewhere in this file.  All rights reserved.

This file was generated from file(s) of the Babel system.
---------------------------------------------------------

It may be distributed and/or modified under the
conditions of the LaTeX Project Public License, either version 1.3
of this license or (at your option) any later version.
The latest version of this license is in
  http://www.latex-project.org/lppl.txt
and version 1.3 or later is part of all distributions of LaTeX
version 2003/12/01 or later.

This work has the LPPL maintenance status "maintained".

The Current Maintainer of this work is Johannes Braams.

This file may only be distributed together with a copy of the Babel
system. You may however distribute the Babel system without
such generated files.

The list of all files belonging to the Babel distribution is
given in the file `manifest.bbl'. See also `legal.bbl for additional
information.

The list of derived (unpacked) files belonging to the distribution
and covered by LPPL is defined by the unpacking scripts (with
extension .ins) which are part of the distribution.
\endpreamble

\declarepreamble\fdpreamble
This is a generated file.

Copyright 1989-2005 Johannes L. Braams and any individual authors
listed elsewhere in this file.  All rights reserved.

This file was generated from file(s) of the Babel system.
---------------------------------------------------------

It may be distributed and/or modified under the
conditions of the LaTeX Project Public License, either version 1.3
of this license or (at your option) any later version.
The latest version of this license is in
  http://www.latex-project.org/lppl.txt
and version 1.3 or later is part of all distributions of LaTeX
version 2003/12/01 or later.

This work has the LPPL maintenance status "maintained".

The Current Maintainer of this work is Johannes Braams.

This file may only be distributed together with a copy of the Babel
system. You may however distribute the Babel system without
such generated files.

The list of all files belonging to the Babel distribution is
given in the file `manifest.bbl'. See also `legal.bbl for additional
information.

In particular, permission is granted to customize the declarations in
this file to serve the needs of your installation.

However, NO PERMISSION is granted to distribute a modified version
of this file under its original name.

\endpreamble

\keepsilent

\usedir{tex/generic/babel} 

\usepreamble\mainpreamble
\generate{\file{magyar.ldf}{\from{magyar.dtx}{code}}
          }
\usepreamble\fdpreamble

\ifToplevel{
\Msg{***********************************************************}
\Msg{*}
\Msg{* To finish the installation you have to move the following}
\Msg{* files into a directory searched by TeX:}
\Msg{*}
\Msg{* \space\space All *.def, *.fd, *.ldf, *.sty}
\Msg{*}
\Msg{* To produce the documentation run the files ending with}
\Msg{* '.dtx' and `.fdd' through LaTeX.}
\Msg{*}
\Msg{* Happy TeXing}
\Msg{***********************************************************}
}
 
\endinput
}
\DeclareOption{icelandic}{% \iffalse meta-comment
%
% Copyright 1989-2005 Johannes L. Braams and any individual authors
% listed elsewhere in this file.  All rights reserved.
% 
% This file is part of the Babel system.
% --------------------------------------
% 
% It may be distributed and/or modified under the
% conditions of the LaTeX Project Public License, either version 1.3
% of this license or (at your option) any later version.
% The latest version of this license is in
%   http://www.latex-project.org/lppl.txt
% and version 1.3 or later is part of all distributions of LaTeX
% version 2003/12/01 or later.
% 
% This work has the LPPL maintenance status "maintained".
% 
% The Current Maintainer of this work is Johannes Braams.
% 
% The list of all files belonging to the Babel system is
% given in the file `manifest.bbl. See also `legal.bbl' for additional
% information.
% 
% The list of derived (unpacked) files belonging to the distribution
% and covered by LPPL is defined by the unpacking scripts (with
% extension .ins) which are part of the distribution.
% \fi
% \CheckSum{546}
%
% \iffalse
%    Tell the \LaTeX\ system who we are and write an entry on the
%    transcript.
%<*dtx>
\ProvidesFile{icelandic.dtx}
%</dtx>
%<code>\ProvidesLanguage{icelandic}
%\fi
%\ProvidesFile{icelandic.dtx}
        [2005/03/30 v1.1g Icelandic support from the babel system]
%\iffalse
%% File iceland.dtx
%% Babel package for LaTeX version 2e
%% Copyright (C) 1989--2005
%%           by Johannes Braams, TeXniek
%
%% Icelandic Language Definition File
%% Copyright (C)  1996--2005
%%           by Einar \'Arnason <einar@lif.hi.is>
%%              Johannes Braams, TeXniek
%
%% Please report errors to: Einar \'Arnason
%%                          einar at lif.hi.is
%% or to:                   J.L. Braams
%%                          babel at braams.cistron.nl
%%
%
%<*filedriver>
\documentclass{ltxdoc}
\usepackage{t1enc}
\usepackage[icelandic,english]{babel}
\font\manual=logo10 % font used for the METAFONT logo, etc.
\newcommand*\MF{{\manual META}\-{\manual FONT}}
\newcommand*\TeXhax{\TeX hax}
\newcommand*\babel{\textsf{babel}}
\newcommand*\langvar{$\langle \it lang \rangle$}
\newcommand*\note[1]{}
\newcommand*\Lopt[1]{\textsf{#1}}
\newcommand*\file[1]{\texttt{#1}}
\begin{document}
\begin{center}
  \textbf{\Large A Babel language definition file for 
          Icelandic}\\[3mm]^^A\]
  Einar \'Arnason\\
  \texttt{einar@lif.hi.is}
\end{center}
 \DocInput{icelandic.dtx}
\end{document}
%</filedriver>
%\fi
% \GetFileInfo{iceland.dtx}
%
%
% \changes{icelandic-1.1c}{1997/05/21}{Removed code already present in
%    \file{babel.def}} 
%
%  \section{The Icelandic language}
%
%  \subsection{Overview}
%
%    The file \file{\filename}\footnote{The file described in this
%    section has version number \fileversion\ and was last revised on
%    \filedate. }  defines all the language definition macros for 
%    the Icelandic language
%
%    Customization for the Icelandic language was made following
%    several official and semiofficial publications~\cite{74:_augl,
%    77:_augl,88:_reglur,94:_si,97:_forst_fs}. These publications  do
%    not always agree and we indicate those instances.
%
%    For this language the character |"| is made active. In
%    table~\ref{tab:icelandic-extras} an overview is given of its
%    purpose. 
%    \begin{table}[htb]
%     \begin{center}
%     \begin{otherlanguage*}{icelandic}
%     \begin{tabular}{lp{8cm}}
%      \verb="|= & disable ligature at this position.              \\
%      |"-| & an explicit hyphen sign, allowing hyphenation
%             in the rest of the word.                             \\
%      |""| & like |"-|, but producing no hyphen sign
%             (for compund words with hyphen, e.g.\ |x-""y|).      \\
%      |"~| & for a compound word mark without a breakpoint.       \\
%      |"=| & for a compound word mark with a breakpoint, allowing
%             hyphenation in the composing words.                  \\
%      |"`| & for Icelandic left double quotes (looks like ,,).    \\
%      |"'| & for Icelandic right double quotes.                   \\
%      |">| & for Icelandic `french' left double quotes 
%             (similar to $>>$).\\
%      |"<| & for Icelandic `french' right double quotes
%             (similar to $<<$).\\
%      |"o| & for old Icelandic "o \\
%      |"O| & for old Icelandic "O \\
%      |"�| & for old Icelandic "� \\
%      |"�| & for old Icelandic "� \\
%      |"e| & for old Icelandic "e \\
%      |"E| & for old Icelandic "E \\
%      |"�| & for old Icelandic "� \\
%      |"�| & for old Icelandic "� \\
%      |\tala| & for typesetting numbers\\
%      |\grada| & for the `degree' symbol\\
%      |\gradur| & for `degrees', e.g. 5~\gradur C\\
%      |\upp| & for textsuperscript \\
%     \end{tabular}
%     \end{otherlanguage*}
%     \caption{The shorthands and extra definitions made
%              by \file{icelandic.ldf}}\label{tab:icelandic-extras}
%     \end{center}
%    \end{table}
%    The shorthands in table~\ref{tab:icelandic-extras} can also be 
%    typeset by using the commands in 
%    table~\ref{tab:icelandic-commands}.
%    \begin{table}[htb]
%     \begin{center}
%     \begin{otherlanguage*}{icelandic}
%     \begin{tabular}{lp{8cm}}
%      |\ilqq|  & for Icelandic left double quotes 
%                 (looks like ,,).   \\
%      |\irqq|  & for Icelandic right double quotes 
%                 (looks like ``).  \\
%      |\ilq|   & for Icelandic left single quotes
%                 (looks like ,).    \\
%      |\irq|   & for Icelandic right single quotes 
%                 (looks like `).   \\
%      |\iflqq| & for Icelandic `french' left double quotes 
%                 (similar to $>>$).\\
%      |\ifrqq| & for Icelandic `french' right double quotes 
%                 (similar to $<<$).\\
%      |\ifrq|  & for Icelandic `french' right single quotes 
%                 (similar to $<$).\\
%      |\iflq|  & for Icelandic `french' left single quotes 
%                 (similar to $>$). \\
%      |\dq|    & the original quotes character (|"|).        \\
%      |\oob|   & for old Icelandic "o \\
%      |\Oob|   & for old Icelandic "O \\
%      |\ooob|  & for old Icelandic "� \\
%      |\OOob|  & for old Icelandic "� \\
%      |\eob|   & for old Icelandic "e \\
%      |\Eob|   & for old Icelandic "E \\
%      |\eeob|  & for old Icelandic "� \\
%      |\EEob|  & for old Icelandic "� \\
%     \end{tabular}
%     \end{otherlanguage*}
%     \caption{Commands which produce quotes and old Icelandic
%       diacritics, defined by 
%     \file{icelandic.ldf}}\label{tab:icelandic-commands}
%     \end{center}
%    \end{table}
%
%    \begin{thebibliography}{1}
%    
%    \bibitem{88:_reglur}
%    Al�ingi.
%    \newblock {\em Reglur um fr�gang �ingskjala og prentun umr��na},
%    1988. 
%    
%    \bibitem{74:_augl}
%    Augl�sing um greinarmerkjasetningu.
%    \newblock Stj.t�� B, nr. 133/1974, 1974.
%    
%    \bibitem{77:_augl}
%    Augl�sing um breyting augl�singu nr. 132/1974 um �slenska
%    stafsetningu. 
%    \newblock Stj.t�� B, nr. 261/1977, 1977.
%    
%    \bibitem{unknown72:_first_gramm_treat}
%    Einar Haugen, editor.
%    \newblock {\em First Grammatical Treatise}.
%    \newblock Longman, London, 2 edition, 1972.
%    
%    \bibitem{97:_forst_fs}
%    Sta�lar�� �slands og Fagr�� � uppl�singat�kni, Reykjav�k.
%    \newblock {\em Forsta�all FS 130:1997}, 1997.
%    
%    \bibitem{94:_si}
%    STR� Sta�lar�� �slands.
%    \newblock {\em SI - kerfi�}, 2 edition, 1994.
%    
%    \end{thebibliography}
%     
% \StopEventually{}
%
%
%  \subsection{\TeX{}nical details}
%
%    When this file was read through the option \Lopt{icelandic} we make
%    it behave as if \Lopt{icelandic} was specified.
%
%    \begin{macrocode}
\def\bbl@tempa{icelandic}
\ifx\CurrentOption\bbl@tempa
  \def\CurrentOption{icelandic}
\fi
%    \end{macrocode}
%
%    The macro |\LdfInit| takes care of preventing that this file is
%    loaded more than once, checking the category code of the
%    \texttt{@} sign, etc.
%    \begin{macrocode}
%<*code>
\LdfInit\CurrentOption{captions\CurrentOption}
%    \end{macrocode}
%
%    When this file is read as an option, i.e., by the |\usepackage|
%    command, \texttt{icelandic} will be an `unknown' language, so we
%    have to make it known.  So we check for the existence of
%    |\l@icelandic| to see whether we have to do something here.
%
%    \begin{macrocode}
\ifx\l@icelandic\@undefined
  \@nopatterns{Icelandic}
  \adddialect\l@icelandic0
\fi
%    \end{macrocode}
%
%  \begin{macro}{\if@Two@E}
%    We will need a new `if' : |\if@Two@E| is true if and only if
%    \LaTeXe{} is running \emph{not} in compatibility mode. It is
%    used in the definitions of the command |\tala| and |\upp|.
%    The definition is somewhat complicated, due to the fact that
%    |\if@compatibility| is not recognized as a |\if| in
%    \LaTeX-2.09 based formats.
%    \begin{macrocode}
\newif\if@Two@E \@Two@Etrue
\def\@FI@{\fi}
\ifx\@compatibilitytrue\@undefined
  \@Two@Efalse \def\@FI@{\relax}
\else
  \if@compatibility \@Two@Efalse \fi
\@FI@
%    \end{macrocode}
%  \end{macro}
%
%
%  \begin{macro}{\extrasicelandic}
%  \begin{macro}{\noextrasicelandic}
%
%    The macro |\extrasicelandic| 
%    will perform all the extra definitions needed for the Icelandic
%    language. The macro |\noextrasicelandic| is used to cancel the
%    actions of |\extrasicelandic|. 
%
%    For Icelandic  the \texttt{"} character is
%    made active. This is done once, later on its definition may vary.
%    \begin{macrocode}
\initiate@active@char{"}
\@namedef{extras\CurrentOption}{%
  \languageshorthands{icelandic}}
\expandafter\addto\csname extras\CurrentOption\endcsname{%
  \bbl@activate{"}}
%    \end{macrocode}
%    Don't forget to turn the shorthands off again.
% \changes{icelandic-1.1e}{1999/12/17}{Deactivate shorthands ouside of
%    Icelandic}
%    \begin{macrocode}
\addto\noextrasicelandic{\bbl@deactivate{"}}
%    \end{macrocode}
%
%    The icelandic hyphenation patterns can be used with
%    |\lefthyphenmin| and |\righthyphenmin| set to~2.
% \changes{icelandic-1.1f}{2000/09/22}{Now use \cs{providehyphenmins}
%    to provide a default value}
%    \begin{macrocode}
\providehyphenmins{\CurrentOption}{\tw@\tw@}
%    \end{macrocode}
%  \end{macro}
%  \end{macro}
%
%    The code above is necessary because we need an extra active
%    character. This character is then used as indicated in
%    table~\ref{tab:icelandic-commands}.
%
%    To be able to define the function of |"|, we first define a
%    couple of `support' macros.
%
%  \subsection{Captionnames and date}
%
%    The next step consists of defining the Icelandic equivalents for
%    the \LaTeX{} captionnames.
%
%  \begin{macro}{\captionsicelandic}
%    The macro |\captionsicelandic| will define all strings used
%    used in the four standard document classes provided 
%    with \LaTeX.
% \changes{icelandic-1.1f}{2000/09/20}{Added \cs{glossaryname}}
% \changes{icelandic-1.1g}{2002/01/00}{Only use 7-bit ASCII characters
%    in order to keep the texts input encoding independant}
% \changes{icelandic-1.1g}{2002/01/09}{Added translation for Glossary}
%    \begin{macrocode}
\@namedef{captions\CurrentOption}{%
  \def\prefacename{Form\'{a}li}%
  \def\refname{Heimildir}%
  \def\abstractname{\'{U}tdr\'{a}ttur}%
  \def\bibname{Heimildir}%
  \def\chaptername{Kafli}%
  \def\appendixname{Vi{\dh}auki}%
  \def\contentsname{Efnisyfirlit}%
  \def\listfigurename{Myndaskr\'{a}}%
  \def\listtablename{T\"{o}fluskr\'{a}}%
  \def\indexname{Atri{\dh}isor{\dh}askr\'{a}}%
  \def\figurename{Mynd}%
  \def\tablename{Tafla}%
  \def\partname{Hluti}%
  \def\enclname{Hj\'{a}lagt}%
  \def\ccname{Samrit}% 
  \def\headtoname{Til:}% in letter
  \def\pagename{Bla{\dh}s\'{\i}{\dh}a}%
  \def\seename{Sj\'{a}}%
  \def\alsoname{Sj\'{a} einnig}%
  \def\proofname{S\"{o}nnun}%
  \def\glossaryname{Or{\dh}alisti}%
 }
%    \end{macrocode}
%  \end{macro}
%
% \begin{macro}{\dateicelandic}
%    The macro |\dateicelandic| redefines the command |\today|
%    to produce Icelandic dates.
% \changes{icelandic-1.1c}{1997/10/01}{Use \cs{edef} to define
%    \cs{today} to save memory}
% \changes{icelandic-1.1c}{1998/03/28}{use \cs{def} instead of \cs{edef}}
%    \begin{macrocode}
\def\dateicelandic{%
  \def\today{\number\day.~\ifcase\month\or
    jan\'{u}ar\or febr\'{u}ar\or mars\or apr\'{\i}l\or ma\'{\i}\or 
    j\'{u}n\'{\i}\or j\'{u}l\'{\i}\or \'{a}g\'{u}st\or september\or
    okt\'{o}ber\or n\'{o}vember\or desember\fi
    \space\number\year}}
%    \end{macrocode}
% \end{macro}
%
%
%  \subsection{Icelandic quotation marks}
%
%
%  \begin{macro}{\dq}
%    We save the original double quote character in |\dq| to keep
%    it available, the math accent |\"| can now be typed as |"|.
%    \begin{macrocode}
\begingroup \catcode`\"12
\def\x{\endgroup
  \def\@SS{\mathchar"7019 }
  \def\dq{"}}
\x
%    \end{macrocode}
%  \end{macro}
%
%    Now we can define the icelandic and icelandic `french' quotes.
%    The icelandic `french' guillemets are the reverse of french 
%    guillemets. We define single icelandic `french' quotes for
%    compatibility. Shorthands are provided for a number of different 
%    quotation marks, which make them useable both outside and 
%    inside mathmode.
%
%    \begin{macrocode}
\let\ilq\grq
\let\irq\grq
\let\iflq\frq
\let\ifrq\flq
\let\ilqq\glqq
\let\irqq\grqq
\let\iflqq\frqq
\let\ifrqq\flqq
%    \end{macrocode}
%
%    \begin{macrocode}
\declare@shorthand{icelandic}{"`}{\glqq}
\declare@shorthand{icelandic}{"'}{\grqq}
\declare@shorthand{icelandic}{">}{\frqq}
\declare@shorthand{icelandic}{"<}{\flqq}
%    \end{macrocode}
%    and some additional commands:
%    \begin{macrocode}
\declare@shorthand{icelandic}{"-}{\nobreak\-\bbl@allowhyphens}
\declare@shorthand{icelandic}{"|}{%
  \textormath{\nobreak\discretionary{-}{}{\kern.03em}%
              \bbl@allowhyphens}{}}
\declare@shorthand{icelandic}{""}{\hskip\z@skip}
\declare@shorthand{icelandic}{"~}{\textormath{\leavevmode\hbox{-}}{-}}
\declare@shorthand{icelandic}{"=}{\nobreak-\hskip\z@skip}
%    \end{macrocode}
%
%
%  \subsection{Old Icelandic}
%  \label{oldicelandic}
%
% \changes{icelandic-1.1}{1997/04/24}{Added definitions for old
%    icelandic.}
%
%    In old Icelandic some letters have special diacritical marks, 
%    described for example in \emph{First Grammatical
%    Treatise}~\cite{unknown72:_first_gramm_treat,97:_forst_fs}. We
%    provide these in the \texttt{T1} encoding with the `ogonek'. The
%    ogonek is placed with the letters `o', and `O', `�' and `�', `e'
%    and `E', and `�' and `�'. Shorthands are provided for these as
%    well. 
%
%    The following code by Leszek Holenderski lifted from 
%    \texttt{polish.dtx} is  designed to position the diacritics 
%    correctly for every font in every size. These macros need a 
%    few extra dimension variables.
%
%    \begin{macrocode}
\newdimen\pl@left
\newdimen\pl@down
\newdimen\pl@right
\newdimen\pl@temp
%    \end{macrocode}
%
%  \begin{macro}{\sob}
%    The macro |\sob| is used to put the `ogonek' in the right
%    place.
%
%    \begin{macrocode}
\def\sob#1#2#3#4#5{%parameters: letter and fractions hl,ho,vl,vo
  \setbox0\hbox{#1}\setbox1\hbox{\k{}}\setbox2\hbox{p}%
  \pl@right=#2\wd0 \advance\pl@right by-#3\wd1
  \pl@down=#5\ht1 \advance\pl@down by-#4\ht0
  \pl@left=\pl@right \advance\pl@left by\wd1
  \pl@temp=-\pl@down \advance\pl@temp by\dp2 \dp1=\pl@temp
  \leavevmode
  \kern\pl@right\lower\pl@down\box1\kern-\pl@left #1}
%    \end{macrocode}
%  \end{macro}
%
%  \begin{macro}{\oob}
%  \begin{macro}{\Oob}
%  \begin{macro}{\ooob}
%  \begin{macro}{\OOob}
%  \begin{macro}{\eob}
%  \begin{macro}{\Eob}
%  \begin{macro}{\eeob}
%  \begin{macro}{\EEob}
%    \begin{macrocode}
\DeclareTextCommand{\oob}{T1}{\sob {o}{.85}{0}{.04}{0}}
\DeclareTextCommand{\Oob}{T1}{\sob {O}{.7}{0}{0}{0}}
\DeclareTextCommand{\ooob}{T1}{\sob {�}{.85}{0}{.04}{0}}
\DeclareTextCommand{\OOob}{T1}{\sob {�}{.7}{0}{0}{0}}
\DeclareTextCommand{\eob}{T1}{\sob {e}{1}{0}{.04}{0}}
\DeclareTextCommand{\Eob}{T1}{\sob {E}{1}{0}{.04}{0}}
\DeclareTextCommand{\eeob}{T1}{\sob {�}{1}{0}{.04}{0}}
\DeclareTextCommand{\EEob}{T1}{\sob {�}{1}{0}{.04}{0}}
%    \end{macrocode}
%  \end{macro}
%  \end{macro}
%  \end{macro}
%  \end{macro}
%  \end{macro}
%  \end{macro}
%  \end{macro}
%  \end{macro}
%
%    \begin{macrocode}
\declare@shorthand{icelandic}{"o}{\oob}
\declare@shorthand{icelandic}{"O}{\Oob}
\declare@shorthand{icelandic}{"�}{\ooob}
\declare@shorthand{icelandic}{"�}{\OOob}
\declare@shorthand{icelandic}{"e}{\eob}
\declare@shorthand{icelandic}{"E}{\Eob}
\declare@shorthand{icelandic}{"�}{\eeob}
\declare@shorthand{icelandic}{"�}{\EEob}
%    \end{macrocode}
%
%  \subsection{Formatting numbers}
%  \label{numbers}
%
% \changes{icelandic-1.1a}{1997/04/26}{Added definitions for 
%    formatting numbers in Icelandic and some extra utilities.}
%    This section is lifted from \texttt{frenchb.dtx} by D. Flipo.
%    In English the decimal part starts with a point and thousands
%    should be separated by a comma: an approximation of $1000\pi$
%    should be inputed as |$3{,}141.592{,}653$| in math-mode and
%    as |3,141.592,653| in text.
% \changes{icelandic-1.1b}{1997/05/05}{Added references to various
%    publications used} 
%    
%    In Icelandic the decimal part starts with a comma and thousands
%    should be separated by a space~\cite{88:_reglur} or by a
%    period~\cite{97:_forst_fs}; we have the space. The above 
%    approximation of $1000\pi$ should be inputed as
%    |$3\;141{,}592\;653$| in math-mode and as something like 
%    |3~141,592~653| in text. Braces are mandatory around the comma in
%    math-mode, the reason is mentioned in the \TeX{}book p.~134:
%    the comma is of type |\mathpunct| (thus normally followed by a 
%    space) while the point is of type |\mathord| (no space added).
%
%    Thierry Bouche suggested that a second type of comma, of type
%    |\mathord| would be useful in math-mode, and proposed to
%    introduce a command (named |\decimalsep| in this package),
%    the expansion of which would depend on the current language.
%
%    Vincent Jalby suggested a command |\nombre| to conveniently
%    typeset numbers: inputting |\nombre{3141,592653}| either in
%    text or in math-mode will format this number properly according
%    to the current language (Icelandic or non-Icelandic). We use 
%    |\nombre| to define command |\tala| in Icelandic.
%
%    |\tala| accepts an optional argument which happens to be
%    useful with the extension `dcolumn', it specifies the decimal
%    separator used in the \emph{source code}:
%    |\newcolumntype{d}{D{,}{\decimalsep}{-1}}| \\
%    |\begin{tabular}{|d|}\hline    |    \\
%    |  3,14 \\                     |    \\
%    |  \tala[,]{123,4567} \\     |    \\
%    |  \tala[,]{9876,543}\\\hline|    \\
%    |\end{tabular}                 |    \\
%    will print a column of numbers aligned on the decimal point
%    (comma or point depending on the current language), each slice
%    of 3 digits being separated by a space or a comma according to
%    the current language.
%
%  \begin{macro}{\decimalsep}
%  \begin{macro}{\thousandsep}
%    We need a internal definition, valid in both text and math-mode,
%    for the comma (|\@comma@|) and another one for the unbreakable
%    fixed length space (no glue) used in Icelandic (|\f@thousandsep|).
%
%    The commands |\decimalsep| and |\thousandsep| get default
%    definitions (for the English language) when |icelandic| is loaded;
%    these definitions will be updated when the current language is
%    switched to or from Icelandic.
%    \begin{macrocode}
\mathchardef\m@comma="013B \def\@comma@{\ifmmode\m@comma\else,\fi}
\def\f@thousandsep{\ifmmode\mskip5.5mu\else\penalty\@M\kern.3em\fi}
\newcommand{\decimalsep}{.}  \newcommand{\thousandsep}{\@comma@}
\expandafter\addto\csname extras\CurrentOption\endcsname{%
            \def\decimalsep{\@comma@}%
            \def\thousandsep{\f@thousandsep}}
\expandafter\addto\csname noextras\CurrentOption\endcsname{%
            \def\decimalsep{.}%
            \def\thousandsep{\@comma@}}
%    \end{macrocode}
%  \end{macro}
%  \end{macro}
%
%  \begin{macro}{\tala}
%    The decimal separator used when \emph{inputing} a number
%    with |\tala| \emph{has to be a comma}.
%    |\tala| splits the inputed number into two parts: what
%    comes before the first comma will be formatted by
%    \cs{@integerpart} while the rest (if not  empty) will be
%    formatted by \cs{@decimalpart}. Both parts, once formatted
%    separately will be merged together with between them, either
%    the decimal separator \cs{decimalsep} or (in \LaTeXe{}
%    \emph{only}) the optional argument of |\tala|.
%
%    \begin{macrocode}
\if@Two@E
  \newcommand{\tala}[2][\decimalsep]{%
         \def\@decimalsep{#1}\@tala#2\@empty,\@empty,\@nil}
\else
  \newcommand{\tala}[1]{%
         \def\@decimalsep{\decimalsep}\@tala#1\@empty,\@empty,\@nil}
\fi
\def\@tala#1,#2,#3\@nil{%
       \ifx\@empty#2%
         \@integerpart{#1}%
       \else
         \@integerpart{#1}\@decimalsep\@decimalpart{#2}%
       \fi}
%    \end{macrocode}
%    The easiest bit is the decimal part:
%    We attempt to read the first four digits of the decimal part, if
%    it has less than 4 digits, we just have to print them, otherwise
%    |\thousandsep| has to be appended after the third digit, and the
%    algorithm is applied recursively to the rest of the decimal part.
%    \begin{macrocode}
\def\@decimalpart#1{\@@decimalpart#1\@empty\@empty\@empty}
\def\@@decimalpart#1#2#3#4{#1#2#3%
  \ifx\@empty#4%
  \else
    \thousandsep\expandafter\@@decimalpart\expandafter#4%
  \fi}
%    \end{macrocode}
%    Formatting the integer part is more difficult because the slices
%    of 3 digits start from the \emph{bottom} while the number is
%    read from the top!
%    This (tricky) code is borrowed from David Carlisle's comma.sty.
%    \begin{macrocode}
\def\@integerpart#1{\@@integerpart{}#1\@empty\@empty\@empty}
\def\@@integerpart#1#2#3#4{%
  \ifx\@empty#2%
    \@addthousandsep#1\relax
  \else
    \ifx\@empty#3%
      \@addthousandsep\@empty\@empty#1#2\relax
    \else
      \ifx\@empty#4%
        \@addthousandsep\@empty#1#2#3\relax
      \else
        \@@integerpartafterfi{#1#2#3#4}%
      \fi
    \fi
  \fi}
\def\@@integerpartafterfi#1\fi\fi\fi{\fi\fi\fi\@@integerpart{#1}}
\def\@addthousandsep#1#2#3#4{#1#2#3%
  \if#4\relax
  \else
    \thousandsep\expandafter\@addthousandsep\expandafter#4%
  \fi}
%    \end{macrocode}
%  \end{macro}
%
%  \subsection{Extra utilities}
%
%    We now provide the Icelandic user with some extra utilities.
%
%  \begin{macro}{\upp}
%    |\upp| is for typesetting superscripts.  |\upp| relies on
%  \begin{macro}{\upp@size}
%    The internal macro |\upp@size| holds the size at which the
%    superscript will be typeset. The reason for this is that we have
%    to specify it differently for different formats.
%    \begin{macrocode}
\ifx\sevenrm\@undefined
  \ifx\@ptsize\@undefined
    \let\upp@size\small
  \else
    \ifx\selectfont\@undefined
%    \end{macrocode}
%    In this case the format is the original \LaTeX-2.09:
%    \begin{macrocode}
      \ifcase\@ptsize
        \let\upp@size\ixpt\or
        \let\upp@size\xpt\or
        \let\upp@size\xipt
      \fi
%    \end{macrocode}
%    When |\selectfont| is defined we probably have NFSS available:
%    \begin{macrocode}
    \else
      \ifcase\@ptsize
        \def\upp@size{\fontsize\@ixpt{10pt}\selectfont}\or
        \def\upp@size{\fontsize\@xpt{11pt}\selectfont}\or
        \def\upp@size{\fontsize\@xipt{12pt}\selectfont}
      \fi
    \fi
  \fi
\else
%    \end{macrocode}
%    If we end up here it must be a plain based \TeX{} format, so:
%    \begin{macrocode}
    \let\upp@size\sevenrm
\fi
%    \end{macrocode}
%  \end{macro}
%    Now we can define |\upp|. When \LaTeXe{} runs in
%    compatibility mode (\LaTeX-2.09 emulation), |\textsuperscript| is
%    also defined, but does no good job, so we give two different
%    definitions for |\upp| using |\if@Two@E|.
%    \begin{macrocode}
\if@Two@E
  \DeclareRobustCommand*{\upp}[1]{\textsuperscript{#1}}
\else
  \DeclareRobustCommand*{\upp}[1]{%
    \leavevmode\raise1ex\hbox{\upp@size#1}}
\fi
%    \end{macrocode}
%
%  \end{macro}
%
%    Some definitions for special characters. 
%    |\grada| needs a special treatment: it is |\char6|
%    in T1-encoding and |\char23| in OT1-encoding.
%    \begin{macrocode}
\ifx\fmtname\LaTeXeFmtName
  \DeclareTextSymbol{\grada}{T1}{6}
  \DeclareTextSymbol{\grada}{OT1}{23}
\else
  \def\T@one{T1}
  \ifx\f@encoding\T@one
    \newcommand{\grada}{\char6}
  \else
    \newcommand{\grada}{\char23}
  \fi
\fi
%    \end{macrocode}
%
%  \begin{macro}{\gradur}
%    Macro for typesetting the abbreviation for `degrees' (as in
%    `degrees Celsius'). As the bounding box of the character `degree'
%    has \emph{very} different widths in CMR/DC and PostScript fonts,
%    we fix the width of the bounding box of |\gradur| to 0.3\,em,
%    this lets the symbol `degree' stick to the preceding
%    (e.g., |45\gradur|) or following character (e.g., |20~\gradur C|).
%    \begin{macrocode}
\DeclareRobustCommand*{\gradur}{%
                       \leavevmode\hbox to 0.3em{\hss\grada\hss}}
%    \end{macrocode}
%  \end{macro}
%
%    The macro |\ldf@finish| takes care of looking for a
%    configuration file, setting the main language to be switched on
%    at |\begin{document}| and resetting the category code of
%    \texttt{@} to its original value.
%    \begin{macrocode}
\ldf@finish\CurrentOption
%</code>
%    \end{macrocode}
%
% \Finale
%%
%% \CharacterTable
%%  {Upper-case    \A\B\C\D\E\F\G\H\I\J\K\L\M\N\O\P\Q\R\S\T\U\V\W\X\Y\Z
%%   Lower-case    \a\b\c\d\e\f\g\h\i\j\k\l\m\n\o\p\q\r\s\t\u\v\w\x\y\z
%%   Digits        \0\1\2\3\4\5\6\7\8\9
%%   Exclamation   \!     Double quote  \"     Hash (number) \#
%%   Dollar        \$     Percent       \%     Ampersand     \&
%%   Acute accent  \'     Left paren    \(     Right paren   \)
%%   Asterisk      \*     Plus          \+     Comma         \,
%%   Minus         \-     Point         \.     Solidus       \/
%%   Colon         \:     Semicolon     \;     Less than     \<
%%   Equals        \=     Greater than  \>     Question mark \?
%%   Commercial at \@     Left bracket  \[     Backslash     \\
%%   Right bracket \]     Circumflex    \^     Underscore    \_
%%   Grave accent  \`     Left brace    \{     Vertical bar  \|
%%   Right brace   \}     Tilde         \~}
%%
\endinput
}
\DeclareOption{interlingua}{% \iffalse meta-comment
%
% Copyright 1989-2005 Johannes L. Braams and any individual authors
% listed elsewhere in this file.  All rights reserved.
% 
% This file is part of the Babel system.
% --------------------------------------
% 
% It may be distributed and/or modified under the
% conditions of the LaTeX Project Public License, either version 1.3
% of this license or (at your option) any later version.
% The latest version of this license is in
%   http://www.latex-project.org/lppl.txt
% and version 1.3 or later is part of all distributions of LaTeX
% version 2003/12/01 or later.
% 
% This work has the LPPL maintenance status "maintained".
% 
% The Current Maintainer of this work is Johannes Braams.
% 
% The list of all files belonging to the Babel system is
% given in the file `manifest.bbl. See also `legal.bbl' for additional
% information.
% 
% The list of derived (unpacked) files belonging to the distribution
% and covered by LPPL is defined by the unpacking scripts (with
% extension .ins) which are part of the distribution.
% \fi
% \CheckSum{85}
%
% \iffalse
%    Tell the \LaTeX\ system who we are and write an entry on the
%    transcript.
%<*dtx>
\ProvidesFile{interlingua.dtx}
%</dtx>
%<code>\ProvidesLanguage{interlingua}
%\fi
%\ProvidesFile{interlingua.dtx}
        [2005/03/30 v1.6 Interlingua support from the babel system]
%\iffalse
%% Babel package for LaTeX version 2e
%% Copyright (C) 1989 -- 2005
%%           by Johannes Braams, TeXniek
%
%% Please report errors to: J.L. Braams
%%                          babel at braams.cistron.nl
%
%    This file is part of the babel system, it provides the source code for
%    the Interlingua language definition file.
%<*filedriver>
\documentclass{ltxdoc}
\newcommand*{\TeXhax}{\TeX hax}
\newcommand*{\babel}{\textsf{babel}}
\newcommand*{\langvar}{$\langle \mathit lang \rangle$}
\newcommand*{\note}[1]{}
\newcommand*{\Lopt}[1]{\textsf{#1}}
\newcommand*{\file}[1]{\texttt{#1}}
\usepackage{url}
\begin{document}
 \DocInput{interlingua.dtx}
\end{document}
%</filedriver>
%\fi
% \GetFileInfo{interlingua.dtx}
%
%  \section{The Interlingua language}
%
%    The file \file{\filename}\footnote{The file described in this
%    section has version number \fileversion\ and was last revised on
%    \filedate.}  defines all the language definition macros for the
%    Interlingua language. This file was contributed by Peter
%    Kleiweg, kleiweg at let.rug.nl. 
%
%    Interlingua is an auxiliary language, built from the common
%    vocabulary of Spanish/Portuguese, English, Italian and French,
%    with some normalisation of spelling. The grammar is very easy,
%    more similar to English's than to neolatin languages. The site
%    \url{http://www.interlingua.com} is mostly written in interlingua
%    (as is \url{http://interlingua.altervista.org}), in case you want
%    to read some sample of it. 
% 
%    You can have a look at the grammar at
%    \url{http://www.geocities.com/linguablau} 
%
% \StopEventually{}
%
%    The macro |\LdfInit| takes care of preventing that this file is
%    loaded more than once, checking the category code of the
%    \texttt{@} sign, etc.
%    \begin{macrocode}
%<*code>
\LdfInit{interlingua}{captionsinterlingua}
%    \end{macrocode}
%
%    When this file is read as an option, i.e. by the |\usepackage|
%    command, \texttt{interlingua} could be an `unknown' language in
%    which case we have to make it known.  So we check for the
%    existence of |\l@interlingua| to see whether we have to do
%    something here.
%
%    \begin{macrocode}
\ifx\undefined\l@interlingua
  \@nopatterns{Interlingua}
  \adddialect\l@interlingua0\fi
%    \end{macrocode}

%    The next step consists of defining commands to switch to (and
%    from) the Interlingua language.
%
%
%  \begin{macro}{\interlinguahyphenmins}
%    This macro is used to store the correct values of the hyphenation
%    parameters |\lefthyphenmin| and |\righthyphenmin|.
%    \begin{macrocode}
\providehyphenmins{interlingua}{\tw@\tw@}
%    \end{macrocode}
%  \end{macro}
%
% \begin{macro}{\captionsinterlingua}
%    The macro |\captionsinterlingua| defines all strings used in the
%    four standard documentclasses provided with \LaTeX.
%    \begin{macrocode}
\def\captionsinterlingua{%
  \def\prefacename{Prefacio}%
  \def\refname{Referentias}%
  \def\abstractname{Summario}%
  \def\bibname{Bibliographia}%
  \def\chaptername{Capitulo}%
  \def\appendixname{Appendice}%
  \def\contentsname{Contento}%
  \def\listfigurename{Lista de figuras}%
  \def\listtablename{Lista de tabellas}%
  \def\indexname{Indice}%
  \def\figurename{Figura}%
  \def\tablename{Tabella}%
  \def\partname{Parte}%
  \def\enclname{Incluso}%
  \def\ccname{Copia}%
  \def\headtoname{A}%
  \def\pagename{Pagina}%
  \def\seename{vide}%
  \def\alsoname{vide etiam}%
  \def\proofname{Prova}%
  \def\glossaryname{Glossario}%
  }
%    \end{macrocode}
% \end{macro}


% \begin{macro}{\dateinterlingua}
%    The macro |\dateinterlingua| redefines the command |\today| to
%    produce Interlingua dates.
%    \begin{macrocode}
\def\dateinterlingua{%
  \def\today{le~\number\day\space de \ifcase\month\or
    januario\or februario\or martio\or april\or maio\or junio\or
    julio\or augusto\or septembre\or octobre\or novembre\or
    decembre\fi
    \space \number\year}}
%    \end{macrocode}
% \end{macro}
%

% \begin{macro}{\extrasinterlingua}
% \begin{macro}{\noextrasinterlingua}
%    The macro |\extrasinterlingua| will perform all the extra
%    definitions needed for the Interlingua language. The macro
%    |\noextrasinterlingua| is used to cancel the actions of
%    |\extrasinterlingua|.  For the moment these macros are empty but
%    they are defined for compatibility with the other
%    language definition files.
%
%    \begin{macrocode}
\addto\extrasinterlingua{}
\addto\noextrasinterlingua{}
%    \end{macrocode}
% \end{macro}
% \end{macro}
%
%    The macro |\ldf@finish| takes care of looking for a
%    configuration file, setting the main language to be switched on
%    at |\begin{document}| and resetting the category code of
%    \texttt{@} to its original value.
%    \begin{macrocode}
\ldf@finish{interlingua}
%</code>
%    \end{macrocode}
%
% \Finale
%\endinput
%% \CharacterTable
%%  {Upper-case    \A\B\C\D\E\F\G\H\I\J\K\L\M\N\O\P\Q\R\S\T\U\V\W\X\Y\Z
%%   Lower-case    \a\b\c\d\e\f\g\h\i\j\k\l\m\n\o\p\q\r\s\t\u\v\w\x\y\z
%%   Digits        \0\1\2\3\4\5\6\7\8\9
%%   Exclamation   \!     Double quote  \"     Hash (number) \#
%%   Dollar        \$     Percent       \%     Ampersand     \&
%%   Acute accent  \'     Left paren    \(     Right paren   \)
%%   Asterisk      \*     Plus          \+     Comma         \,
%%   Minus         \-     Point         \.     Solidus       \/
%%   Colon         \:     Semicolon     \;     Less than     \<
%%   Equals        \=     Greater than  \>     Question mark \?
%%   Commercial at \@     Left bracket  \[     Backslash     \\
%%   Right bracket \]     Circumflex    \^     Underscore    \_
%%   Grave accent  \`     Left brace    \{     Vertical bar  \|
%%   Right brace   \}     Tilde         \~}
%%
}
\DeclareOption{irish}{% \iffalse meta-comment
%
% Copyright 1989-2005 Johannes L. Braams and any individual authors
% listed elsewhere in this file.  All rights reserved.
% 
% This file is part of the Babel system.
% --------------------------------------
% 
% It may be distributed and/or modified under the
% conditions of the LaTeX Project Public License, either version 1.3
% of this license or (at your option) any later version.
% The latest version of this license is in
%   http://www.latex-project.org/lppl.txt
% and version 1.3 or later is part of all distributions of LaTeX
% version 2003/12/01 or later.
% 
% This work has the LPPL maintenance status "maintained".
% 
% The Current Maintainer of this work is Johannes Braams.
% 
% The list of all files belonging to the Babel system is
% given in the file `manifest.bbl. See also `legal.bbl' for additional
% information.
% 
% The list of derived (unpacked) files belonging to the distribution
% and covered by LPPL is defined by the unpacking scripts (with
% extension .ins) which are part of the distribution.
% \fi
% \CheckSum{122}
% \iffalse
%    Tell the \LaTeX\ system who we are and write an entry on the
%    transcript.
%<*dtx>
\ProvidesFile{irish.dtx}
%</dtx>
%<code>\ProvidesLanguage{irish}
%\fi
%\ProvidesFile{irish.dtx}
        [2005/03/30 v1.0h Irish support from the babel system]
%\iffalse
%% File `irish.dtx'
%% Babel package for LaTeX version 2e
%% Copyright (C) 1989 -- 2005
%%           by Johannes Braams, TeXniek
%
%% Please report errors to: J.L. Braams
%%                          babel at braams.cistron.nl
%
%    This file is part of the babel system, it provides the source
%    code for the Irish language definition file.
%
%    The Gaeilge or Irish Gaelic terms were tranlated from those
%    provided by Fraser Grant \texttt{FRASER@CERNVM} by Marion Gunn.
%<*filedriver>
\documentclass{ltxdoc}
\newcommand*{\TeXhax}{\TeX hax}
\newcommand*{\babel}{\textsf{babel}}
\newcommand*{\langvar}{$\langle \mathit lang \rangle$}
\newcommand*{\note}[1]{}
\newcommand*{\Lopt}[1]{\textsf{#1}}
\newcommand*{\file}[1]{\texttt{#1}}
\begin{document}
 \DocInput{irish.dtx}
\end{document}
%</filedriver>
%\fi
% \GetFileInfo{irish.dtx}
%
% \changes{irish-1.0b}{1995/06/14}{Corrected typo (PR1652)}
% \changes{irish-1.0e}{1996/10/10}{Replaced \cs{undefined} with
%    \cs{@undefined} and \cs{empty} with \cs{@empty} for consistency
%    with \LaTeX, moved the definition of \cs{atcatcode} right to the
%    beginning.}
%
%  \section{The Irish language}
%
%    The file \file{\filename}\footnote{The file described in this
%    section has version number \fileversion\ and was last revised on
%    \filedate. A contribution was made by Marion Gunn.}  defines all
%    the language definition macros for the Irish language.
%
%    For this language currently no special definitions are needed or
%    available.
%
% \StopEventually{}
%
%    The macro |\LdfInit| takes care of preventing that this file is
%    loaded more than once, checking the category code of the
%    \texttt{@} sign, etc.
% \changes{irish-1.0e}{1996/11/03}{Now use \cs{LdfInit} to perform
%    initial checks} 
%    \begin{macrocode}
%<*code>
\LdfInit{irish}\captionsirish
%    \end{macrocode}
%
%    When this file is read as an option, i.e. by the |\usepackage|
%    command, \texttt{irish} could be an `unknown' language in which
%    case we have to make it known.  So we check for the existence of
%    |\l@irish| to see whether we have to do something here.
%
%    \begin{macrocode}
\ifx\l@irish\@undefined
  \@nopatterns{irish}
  \adddialect\l@irish0\fi
%    \end{macrocode}
%
%    The next step consists of defining commands to switch to (and
%    from) the Irish language.
%
%  \begin{macro}{\irishhyphenmins}
%    This macro is used to store the correct values of the hyphenation
%    parameters |\lefthyphenmin| and |\righthyphenmin|.
% \changes{irish-1.0f}{1998/06/08}{Added definition of
%    \cs{hyphenmins}}
% \changes{irish-1.0h}{2000/09/22}{Now use \cs{providehyphenmins} to
%    provide a default value}
%    \begin{macrocode}
\providehyphenmins{\CurrentOption}{\tw@\thr@@}
%    \end{macrocode}
%  \end{macro}
%
% \begin{macro}{\captionsirish}
%    The macro |\captionsirish| defines all strings used in the
%    four standard documentclasses provided with \LaTeX.
% \changes{irish-1.0c}{1995/07/04}{Added \cs{proofname} for
%    AMS-\LaTeX}
% \changes{irish-1.0f}{1998/06/08}{Added missing translations provided
%    in PR 2719} 
% \changes{irish-1.0h}{2000/09/20}{Added \cs{glossaryname}}
%    \begin{macrocode}
\addto\captionsirish{%
  \def\prefacename{R\'eamhr\'a}%    <-- also "Brollach"
  \def\refname{Tagairt\'{\i}}%
  \def\abstractname{Achoimre}%
  \def\bibname{Leabharliosta}%
  \def\chaptername{Caibidil}%
  \def\appendixname{Aguis\'{\i}n}%
  \def\contentsname{Cl\'ar \'Abhair}%
  \def\listfigurename{L\'ear\'aid\'{\i}}%
  \def\listtablename{T\'abla\'{\i}}%
  \def\indexname{Inn\'eacs}%
  \def\figurename{L\'ear\'aid}%
  \def\tablename{T\'abla}%
  \def\partname{Cuid}%
  \def\enclname{faoi iamh}%
  \def\ccname{cc}%                 abrv. `c\'oip chuig'
  \def\headtoname{Go}%
  \def\pagename{Leathanach}%
  \def\seename{f\'each}%    
  \def\alsoname{f\'each freisin}% 
  \def\proofname{Cruth\'unas}% 
  \def\glossaryname{Glossary}% <-- Needs translation
  }
%    \end{macrocode}
% \end{macro}
%
% \begin{macro}{\dateirish}
%    The macro |\dateirish| redefines the command |\today| to produce
%    Irish dates.
% \changes{irish-1.0f}{1997/10/01}{Use \cs{edef} to define
%    \cs{today} to save memory}
% \changes{irish-1.0f}{1998/03/28}{use \cs{def} instead of \cs{edef}}
%    \begin{macrocode}
\def\dateirish{%
  \def\today{%
    \number\day\space \ifcase\month\or
    Ean\'air\or Feabhra\or M\'arta\or Aibre\'an\or
    Bealtaine\or Meitheamh\or I\'uil\or L\'unasa\or
    Me\'an F\'omhair\or Deireadh F\'omhair\or
    M\'{\i} na Samhna\or M\'{\i} na Nollag\fi
    \space \number\year}}
%    \end{macrocode}
% \end{macro}
%
% \begin{macro}{\extrasirish}
% \begin{macro}{\noextrasirish}
%    The macro |\extrasirish| will perform all the extra definitions
%    needed for the Irish language. The macro |\noextrasirish| is used
%    to cancel the actions of |\extrasirish|.  For the moment these
%    macros are empty but they are defined for compatibility with the
%    other language definition files.
%
%    \begin{macrocode}
\addto\extrasirish{}
\addto\noextrasirish{}
%    \end{macrocode}
% \end{macro}
% \end{macro}
%
%    The macro |\ldf@finish| takes care of looking for a
%    configuration file, setting the main language to be switched on
%    at |\begin{document}| and resetting the category code of
%    \texttt{@} to its original value.
% \changes{irish-1.0e}{1996/11/03}{Now use \cs{ldf@finish} to wrap up}
%    \begin{macrocode}
\ldf@finish{irish}
%</code>
%    \end{macrocode}
%
% \Finale
%\endinput
%% \CharacterTable
%%  {Upper-case    \A\B\C\D\E\F\G\H\I\J\K\L\M\N\O\P\Q\R\S\T\U\V\W\X\Y\Z
%%   Lower-case    \a\b\c\d\e\f\g\h\i\j\k\l\m\n\o\p\q\r\s\t\u\v\w\x\y\z
%%   Digits        \0\1\2\3\4\5\6\7\8\9
%%   Exclamation   \!     Double quote  \"     Hash (number) \#
%%   Dollar        \$     Percent       \%     Ampersand     \&
%%   Acute accent  \'     Left paren    \(     Right paren   \)
%%   Asterisk      \*     Plus          \+     Comma         \,
%%   Minus         \-     Point         \.     Solidus       \/
%%   Colon         \:     Semicolon     \;     Less than     \<
%%   Equals        \=     Greater than  \>     Question mark \?
%%   Commercial at \@     Left bracket  \[     Backslash     \\
%%   Right bracket \]     Circumflex    \^     Underscore    \_
%%   Grave accent  \`     Left brace    \{     Vertical bar  \|
%%   Right brace   \}     Tilde         \~}
%%
}
\DeclareOption{italian}{% \iffalse meta-comme

% Copyright 1989-2008 Johannes L. Braams and any individual autho
% listed elsewhere in this file.  All rights reserve

% This file is part of the Babel syste
% ------------------------------------

% It may be distributed and/or modified under t
% conditions of the LaTeX Project Public License, either version 1
% of this license or (at your option) any later versio
% The latest version of this license is
%   http://www.latex-project.org/lppl.t
% and version 1.3 or later is part of all distributions of LaT
% version 2003/12/01 or late

% This work has the LPPL maintenance status "maintained

% The Current Maintainer of this work is Johannes Braam

% The list of all files belonging to the Babel system
% given in the file `manifest.bbl. See also `legal.bbl' for addition
% informatio

% The list of derived (unpacked) files belonging to the distributi
% and covered by LPPL is defined by the unpacking scripts (wi
% extension .ins) which are part of the distributio
% \
% \CheckSum{42
% \iffal
%    Tell the \LaTeX\ system who we are and write an entry on t
%    transcrip
%<*dt
\ProvidesFile{italian.dt
%</dt
%<code>\ProvidesLanguage{italia
%\
%\ProvidesFile{italian.dt
        [2008/03/14 v1.2t Italian support from the babel syste
%\iffal
%% File `italian.dt
%% Babel package for LaTeX version
%% Copyright (C) 1989 - 20
%%           by Johannes Braams, TeXni

%% Please report errors to: J.L. Braa
%%                          babel at braams.xs4all.
%%                          Claudio Beccar
%%                          claudio.beccari at gmail.

%    This file is part of the babel system, it provides the sour
%    code for the Italian language definition fil
%    The original version of this file was written by Mauriz
%    Codogno, (mau@beatles.cselt.stet.it). Several features were add
%    by Claudio Beccari, (beccari@polito.it
%<*filedrive
\documentclass{ltxdo
\newcommand*\TeXhax{\TeX ha
\newcommand*\babel{\textsf{babel
\newcommand*\langvar{$\langle \it lang \rangle
\newcommand*\note[1]
\newcommand*\Lopt[1]{\textsf{#1
\newcommand*\file[1]{\texttt{#1
\begin{documen
 \DocInput{italian.dt
\end{documen
%</filedrive
%\
% \GetFileInfo{italian.dt

% \changes{italian-0.99}{1990/07/11}{First version, from english.do
% \changes{italian-1.0}{1991/04/23}{Modified for babel 3.
% \changes{italian-1.0a}{1991/05/23}{removed typ
% \changes{italian-1.0b}{1991/05/29}{Removed bug found by van der Mee
% \changes{italian-1.0e}{1991/07/15}{Renamed \file{babel.sty}
%    \file{babel.com
% \changes{italian-1.1}{1992/02/16}{Brought up-to-date with babel 3.2
% \changes{italian-1.2}{1994/02/09}{Update for\ LaTeX
% \changes{italian-1.2e}{1994/06/26}{Removed the use of \cs{filedat
%    and moved identification after the loading of \file{babel.def
% \changes{italian-1.2f}{1995/05/28}{Updated for babel 3.
% \changes{italian-1.2i}{1996/10/10}{Replaced \cs{undefined} wi
%    \cs{@undefined} and \cs{empty} with \cs{@empty} for consisten
%    with \LaTeX, moved the definition of \cs{atcatcode} right to t
%    beginning
% \changes{italian-1.2l}{1999/04/24}{Added \cs{unit}, \cs{ap}, a
%    \cs{ped} macro
% \changes{italian-1.2m}{2000/01/05}{Added support for etymologic
%    hyphenatio
% \changes{italian-1.2n}{2000/02/02}{Completely modified etymologic
%    hyphenation facilit
% \changes{italian-1.2n}{2000/05/28}{Added several commands for t
%   caporali double quotes and for simplifying the accented vowel inpu
% \changes{italian-1.2o}{2000/12/12}{Added \cs{glossaryname
% \changes{italian-1.2p}{2002/07/10}{Removed redefinition
%    \cs{add@acc} since its functionality has been introduced into t
%    kernel of LaTeX 2001/06/0
% \changes{italian-1.2q}{2005/02/05}{Added test for avoiding confli
%    with package units.sty; adjusted caporali functionality, sin
%    the previous one did not work with the standard (although obsolet
%    slides class fil
% \changes{italian-1.2s}{2007/01/12}{Corrected email of C
%  \section{The Italian languag

%    The file \file{\filename}\footnote{The file described in th
%    section has version number \fileversion\ and was last revised
%    \filedate. The original author is Maurizio Codogn
%    (\texttt{mau@beatles.cselt.stet.it}). It has been largely revis
%    by Johannes Braams and Claudio Beccari} defines all t
%    language-specific macros for the Italian languag

%    The features of this language definition file are the followin
%    \begin{enumerat
%    \item The Italian hyphenation is invoked, provided that fi
%      \texttt{ithyph.tex} was loaded when the \LaTeXe\ format w
%      built; in case it was not, read the information coming with yo
%      distribution of the \TeX\ software, and the \babe
%      documentatio
%    \item The language dependent fixed words to be inserted by su
%      commands as |\chapter|, |\caption|, |\tableofcontents
%      etc. are redefined in accordance with the Itali
%      typographical practic
%    \item Since Italian can be easily hyphenated and Italian practi
%      allows to break a word before the last two letters, hyphenati
%      parameters have been set accordingly, but a very high demer
%      value has been set in order to avoid word breaks in t
%      penultimate line of a paragraph. Specifically the |\clubpenalty
%      and the |\widowpenalty| are set to rather high values a
%      |\finalhyphendemerits| is set to such a high value th
%      hyphenation is prohibited between the last two lines of
%      paragraph. In orer to make it consistent, also |\@clubpenalt
%      is set to the same value; actualy the latter value is t
%      reset value after every sectioning command, so that after t
%      first section, |\clubpenalty| is reset to the low default valu
%      Thanks to Enrico Gregorio for spotting this serious bu
%    \item Some language specific shortcuts have been defined so as
%      allow etymological hyphenation, specifically |"| inserts
%      break point in any word boundary that the typesetter choose
%      provided it is not followed by and accented letter (very unlike
%      in Italian, where compulsory accents fall only on the last a
%      ending vowel of a word, but may take place with compound wor
%      that include foreign roots), and \verb="|= when the desired bre
%      point falls before an accented lette
%    \item The shortcut |""| introduces the raised (English) openi
%      double quotes; this shortcut proves its usefulness when o
%      reminds that the Italian keyboard misses the backtick key, a
%      the backtick on a Windows based platform may be obtained only
%      pressing the \texttt{Alt} key while inputting the numerical co
%      0096; very, very annoyin
%    \item The shortcuts |"<| and |">| insert the French guillemot
%      sometimes used in Italian typography; with the T1 font encodi
%      the ligatures |<<| and |>>| should insert such signs directl
%      but not all the  virtual fonts that claim to follow the T1 fo
%      encoding actually contain the guillemots; with the OT1 encodi
%      the guillemots are not available and must be faked in so
%      way. By using the |"<| and |">| shortcuts (even with the
%      encoding) the necessary tests are performed and in case t
%      suitable glyphs are taken from other fonts normally availab
%      with any good, modern \LaTeX\ distributio
%    \item Three new specific commands |\unit|, |\ped|, and |\ap| a
%      introduced so as to  enable the correct composition of technic
%      mathematics according to the ISO~31/XI recommendations. |\uni
%      does not get redefined if the \babel\ package is loaded \emph{afte
%      the package \texttt{units.sty} whose homonymous command pla
%      a different role and follows a different synta
%    \end{enumerat

%    For this language a limited number of shortcuts has been define
%    table~\ref{t:itshrtct}, some of which are used to overco
%    certain limitations of the Italian keyboard;
%    section~\ref{s:itkbd} there are other comments and hints in ord
%    to overcome some other keyboard limitation

%    \begin{table}[htb]\centeri
%    \begin{tabular}{cp{80mm
%    |"|    & inserts a compound word mark where hyphenation is lega
%             it allows etymological hyphenation which is recommend
%             for technical terms, chemical names and the like;
%             does not work if the next character is represented wi
%             a control sequence or is an accented character.
%    \texttt{\string"\string
%           & the same as the above without the limitation
%            characters represented with control sequences or accent
%            ones.
%    |""|   & inserts open quotes ``.\\ %^^A'' emacs matchi
%    |"<|   & inserts open guillemots.
%    |">|   & inserts closed guillemots.
%    |"/|   & equivalent to |\slas
%    \end{tabula
%    \caption{Shortcuts for the Italian language}\label{t:itshrtc
%    \end{tabl

% \StopEventually
%    \begin{thebibliography}{
%    \bibitem{CBec} Beccari C., ``Computer Aided Hyphenation f
%    Italian and Modern Latin'', \textsf{TUGboat} vol.~13, n.~
%    pp.~23-33 (1992
%    \bibitem{Becc2} Beccari C., ``Typesetting mathematics for scien
%    and technology according to ISO\,31/XI'', \textsf{TUGboa
%    vol.~18, n.~1, pp.~39-48 (1997
%    \end{thebibliography

%    The macro |\LdfInit| takes care of preventing that this file
%    loaded more than once, checking the category code of t
%    \texttt{@} sign, et
% \changes{italian-1.2i}{1996/11/03}{Now use \cs{LdfInit} to perfo
%    initial check
% \changes{italian-1.2j}{1996/12/29}{Added braces around second arg
%    \cs{LdfInit
%    \begin{macrocod
%<*cod
\LdfInit{italian}{captionsitalian
%    \end{macrocod

%    When this file is read as an option, i.e. by the |\usepackag
%    command, \texttt{italian} will be an `unknown' language in whi
%    case we have to make it known.  So we check for the existence
%    |\l@italian| to see whether we have to do something her

% \changes{italian-1.0}{1991/04/23}{Now use \cs{adddialect}
%    language undefine
% \changes{italian-1.0h}{1991/10/08}{Removed use of \cs{@ifundefined
% \changes{italian-1.1}{1992/02/16}{Added a warning when
%    hyphenation patterns were loaded
% \changes{italian-1.2e}{1994/06/26}{Now use \cs{@nopatterns}
%    produce the warnin
%    \begin{macrocod
\ifx\l@italian\@undefin
    \@nopatterns{Italian
    \adddialect\l@italian0\
%    \end{macrocod

%    The next step consists of defining commands to switch to (a
%    from) the Italian languag

% \begin{macro}{\captionsitalia
%    The macro |\captionsitalian| defines all strings us
%    in the four standard document classes provided with \LaTe
% \changes{italian-1.0c}{1991/06/06}{Removed \cs{global} definition
% \changes{italian-1.0c}{1991/06/06}{\cs{pagename} should
%    \cs{headpagename
% \changes{italian-1.0d}{1991/07/01}{`contiene' substitued by `Allegat
%    as suggested by Marco Bozzo (\texttt{BOZZO@CERNVM})
% \changes{italian-1.1}{1992/02/16}{Added \cs{seename}, \cs{alsonam
%    and \cs{prefacename
% \changes{italian-1.1}{1993/07/15}{\cs{headpagename} should
%    \cs{pagename
% \changes{italian-1.2b}{1994/05/19}{Changed some of the wor
%    following suggestions from Claudio Beccar
% \changes{italian-1.2g}{1995/07/04}{Added \cs{proofname} f
%    AMS-\LaTe
% \changes{italian-1.2h}{1995/07/27}{Added translation of `Proof
%    \begin{macrocod
\addto\captionsitalian
  \def\prefacename{Prefazione
  \def\refname{Riferimenti bibliografici
  \def\abstractname{Sommario
  \def\bibname{Bibliografia
  \def\chaptername{Capitolo
  \def\appendixname{Appendice
  \def\contentsname{Indice
  \def\listfigurename{Elenco delle figure
  \def\listtablename{Elenco delle tabelle
  \def\indexname{Indice analitico
  \def\figurename{Figura
  \def\tablename{Tabella
  \def\partname{Parte
  \def\enclname{Allegati
  \def\ccname{e~p.~c.
  \def\headtoname{Per
  \def\pagename{Pag.}%    % in Italian the abbreviation is preferr
  \def\seename{vedi
  \def\alsoname{vedi anche
  \def\proofname{Dimostrazione
  \def\glossaryname{Glossario

%    \end{macrocod
% \end{macr

% \begin{macro}{\dateitalia
%    The macro |\dateitalian| redefines the comma
%    |\today| to produce Italian date
% \changes{italian-1.0c}{1991/06/06}{Removed \cs{global} definition
%    \begin{macrocod
\def\dateitalian
  \def\today{\number\day~\ifcase\month\
    gennaio\or febbraio\or marzo\or aprile\or maggio\or giugno\
    luglio\or agosto\or settembre\or ottobre\or novembre\
    dicembre\fi\space \number\year}
%    \end{macrocod
% \end{macr

% \begin{macro}{\italianhyphenmin
% \changes{italian-1.2b}{1994/05/19}{Added setting of left a
%    righthyphenmin according to Claudio Beccari's suggestio

%    The italian hyphenation patterns can be used with bo
%    |\lefthyphenmin| and |\righthyphenmin| set to~
% \changes{italian-1.2m}{2000/09/22}{Now use \cs{providehyphenmins}
%    provide a default valu
%    \begin{macrocod
\providehyphenmins{\CurrentOption}{\tw@\tw
%    \end{macrocod
% \end{macr

% \begin{macro}{\extrasitalia
% \begin{macro}{\noextrasitalia

% \changes{italian-1.2b}{1994/05/19}{Added setting of club- a
%    widowpenalt
%    Lower the chance that clubs or widows occu
% \changes{italian-1.2t}{2007/12/10}{Added \cs{@clubpenalty} to t
%    italian specific settings, otherwise any sectioning command restor
%    it to the default value valid for english and most other language
%    \begin{macrocod
\addto\extrasitalian
  \babel@savevariable\clubpenal
  \babel@savevariable\widowpenal
  \babel@savevariable\@clubpenal
  \clubpenalty3000\widowpenalty3000\@clubpenalty\clubpenalty
%    \end{macrocod

% \changes{italian-1.2b}{1994/05/19}{Added setting
%    finalhyphendemerit

%    Never ever break a word between the last two lines of a paragra
%    in italian text
%    \begin{macrocod
\addto\extrasitalian
  \babel@savevariable\finalhyphendemeri
  \finalhyphendemerits50000000
%    \end{macrocod

% \changes{italian-1.2h}{1995/11/10}{Now give the apostrophe
%    lowercase cod
% \changes{italian-1.2l}{1999/04/5}{Changed example ``begl'italiani
%    (obsolete spelling) with another, ``nell'altezza'', that behav
%    the same wa
%    In order to enable the hyphenation of words such
%    ``nell'altezza'' we give the \texttt{'} a non-zero lower ca
%    code. When we do that \TeX\ finds the following hyphenati
%    points |nel-l'al-tez-za| instead of non
%    \begin{macrocod
\addto\extrasitalian
  \lccode`'=`'
\addto\noextrasitalian
  \lccode`'=0
%    \end{macrocod
% \end{macr
% \end{macr

% \changes{italian-1.2m}{2000/01/05}{Support for etymologic
%    hyphenatio

%   \subsection{Support for etymological hyphenatio

%    In his article on Italian hyphenation \cite{CBec} Beccari point
%    out that the Italian language gets hyphenated on a phonet
%    basis, although etymological hyphenation is allowed; this is
%    contrast with what happens in Latin, for example, whe
%    etymological hyphenation is always used. Since the patterns f
%    both languages would become too complicated in order to cope wi
%    etymological hyphenation, in his paper Beccari proposed t
%    definition of an active character `|_|' such that it could inse
%    a ``soft'' discretionary hyphen at the compound wo
%    boundary. For several reasons that idea and the specific acti
%    character proved to be unpractical and was abandone

%    This problem is so important with the majority of the Europe
%    languages, that \babel\ from the very beginning develop
%    the tradition of making the |"| character active so as to perfo
%    several actions that turned useful with every languag
%    One of these actions consisted in defining the shortcut \verb="
%    that was extensively used in German and in many other languag
%    in order to insert a discretionary hyphen such that hyphenati
%    would not be precluded in the rest of the word as it happens wi
%    the standard \TeX\ command |\-

%    Meanwhile the \texttt{ec} fonts with the double Cork encodi
%    (thus formerly called the \texttt{dc} fonts) have become more
%    less standard and are widely used by virtually all Europeans th
%    write languages with many special national characters; by
%    doing they avoid the use of the |\accent| primitive which wou
%    be required with the standard \texttt{cm} fonts; with the latt
%    fonts the primitive command |\accent| is such that hyphenati
%    becomes almost impossible, in any case strongly impeache

%    The \texttt{ec} fonts contain a special character, nam
%    ``compound word mark'', that occupies position 23 in the fo
%    scheme and may be input with the sequence |^^W|. Up to no
%    apparently, this special character has never been used in
%    practical way for the typesetting of languages rich of compou
%    words; also it has never been inserted in the hyphenation patte
%    files of any language. Beccari modified his pattern fi
%    \file{ithyph.tex v4.8b} for Italian so as to contain five n
%    patterns that involve |^^W|, and he tried to give t
%    \babel\ active character |"|  a new shortcut definitio
%    so as to allow the insertion of the ``compound word mark'' in t
%    proper place within any word where two semantic fragments jo
%    up. With such facility for marking the compound word boundarie
%    etymological hyphenation becomes possible even if the patter
%    know nothing about etymology (but the typesetter hopeful
%    does!
%    In Italian such etymological hyphenation is desirable wi
%    technical terms, chemical names, and the lik

%    Even this solution proved to be inconvenient on certain UN
%    platforms, so Beccari resorted to another approach that uses t
%    \babel\ active character |"| and relies on the catego
%    code of the character that follows |"

%    \changes{italian-1.2n}{2000/02/02}{Completely new etymologic
%    hyphenation facilit

%    \begin{macrocod
\initiate@active@char{"
\addto\extrasitalian{\bbl@activate{"}\languageshorthands{italian}
%    \end{macrocod
%    \begin{macro}{\it@cw
%    The active character |"| is now defined for language |italian|
%    as to perform different actions in math mode compared to te
%    mode; specifically in math mode a double quote is inserted so
%    to produce a double prime sign, while in text mode the tempora
%    macro |\it@next| is defined so as to defer any further acti
%    until the next token category code has been teste
%    \begin{macrocod
\declare@shorthand{italian}{"}
\ifmmo
    \def\it@next{''
\el
    \def\it@next{\futurelet\it@temp\it@cwm
\
\it@ne

%    \end{macrocod
%    \begin{macro}{\it@cw
%    The \cs{it@next} service control sequence is such that upon i
%    execution a temporary variable \cs{it@temp} is made equivalent
%    the next token in the input list without actually removi
%    it. Such temporary token is then tested by the macro \cs{it@cw
%    and if it is found to be a letter token, then it introduces
%    compound word separator control sequence \cs{it@allowhyphen
%    whose expansion introduces a discretionary hyphen and
%    unbreakable space; in case the token is not a letter, then it
%    tested against \verb=|=$_{12}$: if so a compound word separat
%    is inserted and the \verb=|= token is removed, otherwise anoth
%    test is performed so as to see if another double quote si
%    follows: in this case a double open quote mark is inserte
%    otherwise two other tests are performed so as to see
%    guillemets have to be inserted, otherwise nothing is don
%    The double quote shortcut for inserting a double open quote si
%    is useful for people who are inputting Italian text by means
%    an Italian keyboard that unfortunately misses the grave
%    backtick ke
%    By this shortcut |""| becomes equivalent to |``| for inserti
%    raised open high double quote
% \changes{italian-1.2r}{2005/11/17}{Added \cs{nobreak}
%    \cs{it@@cwm} and corrected \cs{it@next
%    \begin{macrocod
\def\it@@cwm{\nobreak\discretionary{-}{}{}\nobreak\hskip\z@skip
\def\it@@ocap#1{\it@ocap}\def\it@@ccap#1{\it@ccap
\DeclareRobustCommand*{\it@cwm}{\let\it@@next\rel
\ifcat\noexpand\it@temp
    \def\it@@next{\it@@cwm
\el
    \if\noexpand\it@temp \string
        \def\it@@next{\it@@cwm\@gobble
    \el
        \if\noexpand\it@temp \string
            \def\it@@next{\it@@ocap
        \el
            \if\noexpand\it@temp \string
                \def\it@@next{\it@@ccap
            \el
                \if\noexpand\it@temp\string
                    \def\it@@next{\slash\@gobble
                \el
                    \ifx\it@temp
                        \def\it@@next{``\@gobble
                    \
                \
            \
        \
    \
\
\it@@next
%    \end{macrocod
%    \end{macr
%    \end{macr


%   \begin{sloppypa
%    By this definition of |"| if one types |macro"istruzione| t
%    possible break points become \textsf{ma-cro-istru-zio-ne}, whi
%    without  the |"| mark they would be \textsf{ma-croi-stru-zio-ne
%    according to the phonetic rules such that the |macro| prefix
%    not taken as a uni
%    A chemical name such as \texttt{des"clor"fenir"amina"cloridrat
%    is breakable as \textsf{des-clor-fe-nir-ami-na-clo-ri-dra-t
%    instead of \textsf{de-sclor-fe-ni-ra-mi-na-..

%    In other language description files a shortcut is defined so
%    to allow a break point without actually inserting any hyph
%    sign; examples are given such as \mbox{entrada/salida}; actual
%    if one wants to allow a breakpoint after the slash, it is mu
%    clearer to type |\slash| instead of |/| and \LaTeX\ do
%    everything by itself; here the shortcut |"/| was introduced
%    stand for |\slash| so that one can type |input"/output| and all
%    a line break after the slas
%    This shortcut works only for the slash, since in Italian su
%    constructs are extremely rar
%   \end{sloppypa

%    Attention: the expansion of |"| takes place before the actu
%    expansion of OT1 or T1 accented sequences such as |\`{a}
%    therefore this etymological hyphenation facility works as
%    should only when the semantic word fragments \textit{do n
%    start} with an accented letter; this in Italian is alwa
%    avoidable, because compulsory accents fall only on the la
%    vowel, but it may be necessary to mark a compound word where o
%    or more components come from a foreign language and conta
%    diacritical marks according to the spelling rules of th
%    language. In this case the special shorthand \verb!"|! may
%    used that performs exactly as |"| normally does, except that t
%    \verb!|! sign is removed from the token input lis
%    \verb!kilo"|{\"o}rsted! gets hyphenated
%    \texttt{ki-lo-\"or-sted

%    \changes{italian-1.2l}{1999/04/05}{Added useful macros f
%    fulfilling ISO 31/XI regulation

%   \subsection{Facilities required by the ISO 31/XI regulation
%    The ISO 31/XI regulations require that units of measure a
%    typeset in upright font in any circumstance, math or text, a
%    that in text mode they are separated from the  numeric
%    indication of the measure with an unbreakable (thin) spac
%    The command |\unit| that was defined for achieving th
%    goal happened to conflict with the homonymous command defined
%    the package \texttt{units.sty}; we therefore need to test if th
%    package has already been loaded so as to avoid conflicts; we assu
%    that if the user loads that package, s/he wants to use that packa
%    facilities and command synta

%    The same regulations require also that super and subscrip
%    (apices and pedices) are in upright font, \emph{not in ma
%    italics}, when they represent ``adjectives'' or appositions
%    mathematical or physical variables that do not represe
%    countable or measurable entities such as, for exampl
%    $V_{\mathrm{max}}$ or $V_{\mathrm{rms}}$ for a maximum or a ro
%    mean square voltage, compared to $V_i$  or $V_T$ as the $i$-
%    voltage in a set, or a voltage that depends on the thermodynam
%    temperature $T$. See \cite{Becc2} for a complete description
%    the ISO regulations in connection with typesettin

%    More rarely it happens to use superscripts that are n
%    mathematical variables, such as the notati
%    $\mathbf{A}^{\!\mathrm{T}}$ to denote the transpose of matr
%    $\mathbf{A}$; text superscripts are useful also as ordinals
%    in old fashioned abbreviations in text mode; for example t
%    feminine ordinal $1^{\mathrm{a}}$ or the  old fashioned obsole
%    abbreviation F$^{\mathrm{lli}}$ for \mbox{Fratelli} in compa
%    names (compare with ``Bros.'' for \underline{bro}ther\underline{
%    in American English); text subscripts are mostly used in logo

%    \begin{macro}{\uni
%    \begin{macro}{\a
%    \begin{macro}{\pe
%    First we define the new (internal) commands |\bbl@unit|, |\bbl@ap
%    and |\bbl@ped| as robust one
% \changes{italian-1.2q}{2005/02/05}{Added testing for avoiding conflic
%     with the units.sty packag
%    \begin{macrocod
\@ifpackageloaded{units}{}
  \DeclareRobustCommand*{\bbl@unit}[1]
    \textormath{\,\mbox{#1}}{\,\mathrm{#1}}

\DeclareRobustCommand*{\bbl@ap}[1]
  \textormath{\textsuperscript{#1}}{^{\mathrm{#1}}}
\DeclareRobustCommand*{\bbl@ped}[1]
  \textormath{$_{\mbox{\fontsize\sf@size\
        \selectfont#1}}$}{_\mathrm{#1}}
%    \end{macrocod
%    Then we can use |\let| to define the user level commands, but
%    case the macros already have a different meaning before enteri
%    in Italian mode typesetting, we first memorize their meaning
%    as to restore them on exi
%    \begin{macrocod
\@ifpackageloaded{units}{}
  \addto\extrasitalian
    \babel@save\unit\let\unit\bbl@unit

\addto\extrasitalian
  \babel@save\ap\let\ap\bbl@
  \babel@save\ped\let\ped\bbl@p

%    \end{macrocod
%    \end{macr
%    \end{macr
%    \end{macr

% \subsection{Accents}\label{s:itkb
%    Most of the other language description files introduce a numb
%    of shortcuts for inserting accents and other language specif
%    diacritical marks in a more comfortable way compared with t
%    lengthy standard \TeX\ conventions. When an Italian keyboard
%    being used on a Windows based platform, it exhibits su
%    limitations that up to now no convenient shortcuts have be
%    developed; the reason lies in the fact that the Italian keyboa
%    lacks the grave accent (also known as ``backtick''), which
%    compulsory on all accented vowels except the `e',  but, on t
%    opposite, it carries the keys with all the accented lowerca
%    vowels; the keyboard lacks also the tie |~| (tilde) key, whi
%    the curly braces require pressing three keys simultaneousl

%    The best solution Italians have found so far is to use a sma
%    editor that accepts shortcut definitions such that, for exampl
%    by striking |"(| one gets directly |{| on the screen and the sa
%    sign is saved into the \file{.tex} file; the same smart edit
%    should be capable of translating the accented characters into t
%    standard \TeX\ sequences when writing a file to disk (for t
%    sake of file portability), and to transform the standard \Te
%    sequences into the corresponding signs when loading a \file{.te
%    file from disk to memory. Such smart editors do exist and can
%    downloaded from the \textsc{ctan} archive

% \changes{italian-1.2p}{2002/07/10}{Removed redefinition of \cs{add@acc} since i
%    functionality has been introduced into the kernel of LaTeX 2001/06/0

%    For what concerns the missing backtick ke
%    which is used also for inputting the open quotes, it must
%    noticed that the shortcut |""| described above completely solv
%    the problem for \textit{double} raised open quotes; according
%    the traditions of particular publishing houses, since there a
%    no  compulsory regulations on the matter, the French guilleme
%    may be used; in this case the T1 font encoding solves the probl
%    by means of its built in ligatures |<<| and |>>|. But\do

%    \subsection{\emph{Caporali} or French double quote
%    Although the T1 font encoding ligatures solve the problem, the
%    are some circumstances where even the T1 font encoding cannot
%    used, either because the author\slash typesetter wants to use t
%    OT1 encoding, or because s/he uses a font set that do
%    not comply completely with the T1 font encoding; some virtu
%    fonts, for example, are supposed to implement the double Co
%    font encoding  but actually miss some glyphs; one such virtu
%    font set is given by the \texttt{ae} virtual fonts, because th
%    are supposed to implement such double font encoding simply usi
%    the \texttt{cm} fonts, of which the type~1 PostScript versi
%    exists  and is freely available. Since guillemets (in Itali
%    \emph{caporali}) do not exist in any \texttt{cm} latin fon
%    their glyphs must be substituted with something else th
%    approaches the

% \changes{italian-1.2q}{2005/02/05}{Redefined the caporali machine
%     so as to avoid incompatibilities with the slides class, as the
%     are no Cyrillic slides fonts as there are for Latins scrip

%    Since in French typesetting guillemets are compulsory, the Fren
%    language definition file resorts to a clever font substitutio
%    such file exploits the \LaTeXe\ font selection machinery so as
%    get the guillemets from the Cyrillic fonts, because it suffic
%    to locally change the default encoding. There are several sets
%    Cyrillic fonts, but the ones that obey the OT2 font encoding a
%    generally distributed with  all recent implementations of t
%    \TeX\ software; they are part of the American Mathematic
%    Society fonts and come both as \textsf{METAFONT} source files a
%    Type~1 PostScript \texttt{.pfb} files. The availability of su
%    fonts should be guaranteed by the presence of t
%    \texttt{OT2cmr.fd} font description file. Actually the presen
%    of this file does not guarantee the completeness of your \Te
%    implementation; should \LaTeX\ complain about a missing Cyrill
%    \texttt{.tfm} file (that kind of file that contains the fo
%    metric parameters) and/or about missing Cyrillic \texttt(.m
%    files, then your \TeX\ system is \emph{incomplete} and you shou
%    download such fonts from the \textsc{ctan} archives. Temporari
%    you may issue the command |\LtxSymbCaporali| so as to approxima
%    the missing glyphs with the \LaTeX\ symbol fonts. In some ca
%    warning messages are issued so as to inform the typesetter abo
%    the necessity of resorting to some \emph{poor man} solutio

%     In spite of these alternate fonts, we must avoid invoking unusu
%     fonts if the available encoding allows to use built in caporal
%     As far as I know (CB) the only T1-encoded font families that mi
%     the guillemets are the AE ones; we therefore first test if t
%     default encoding id the T1 one and in this case if the AE famili
%     are the default ones; in order for this to work properly it
%     necessary to load these optional packages \emph{before} \babe
%     If the T1 encoding is not the default one when the Italian langua
%     is specified, then some substitutions must be don

%    \begin{macro}{\LtxSymbCaporal
%    \begin{macro}{\it@oca
%    \begin{macro}{\it@cca
%     We define some macros for substituting the default guillemets; first t
%     emulation by means of the \LaTeX\ symbols; each one of these macro se
%     actually redefines the control sequences |\it@ocap| and |\it@ccap| th
%     are the ones effectively activated by the shortcuts |"<| and |">
%    \begin{macrocod
\def\LtxSymbCaporali
     \DeclareRobustCommand*{\it@ocap}{\mbox
        \fontencoding{U}\fontfamily{lasy}\selectfont(\kern-0.20em(
        \ignorespaces
     \DeclareRobustCommand*{\it@ccap}{\ifdim\lastskip>\z@\unskip\
     \mbox
        \fontencoding{U}\fontfamily{lasy}\selectfont)\kern-0.20em)}

%    \end{macrocod
%    Then the substitution with any specific font that contains su
%    glyphs; it might be the CBgreek fonts, the Cyrillic one, t
%    super-cm ones, the lm ones, or any other the user might pref
%    (the code is adapted from the one that appears in t
%    \texttt{frenchb.ld} file; thanks to Daniel Flipo)
%    By default if the user did not select the T1 encoding, t
%    existence of the CBgreek fonts is tested; if they exist t
%    guillemets are taken from this font, and since its families are
%    superset of the default CM ones and they apply also to types
%    slides with the standard class \texttt{slides}. If the CBgre
%    fonts are not found, then the existence of the Cyrillic ones
%    tested, although this choice is not suited for typesetti
%    slides; otherwise the poor man solution of the \LaTeX\ speci
%    symbols is used. In any case the user can force the use of t
%    Cyrillic guillemets substitution by issuing the declarati
%    |\CyrillicCaporali| just before the |\begin{document}| statemen
%    in alternative the user can specify wi
%    \begin{flushlef
%    |\CaporaliFrom|\marg{encoding}\marg{family}\marg{opening number}\marg{closing numbe
%    \end{flushlef
%    the encoding and family of the font s/he prefers, and the sl
%    numbers of the opening and closing guillemets respectively. F
%    example if the T1-encoded Latin Modern fonts are desired t
%    specific command should b
%    \begin {flushlef
%    |\CaporaliFrom{T1}{lmr}{19}{20
%    \end{flushlef
%    These user choices might be necessary for assuring the corre
%    typesetting with fonts that contain the required glyphs and a
%    available also in PostScript form so as to use them directly wi
%    \texttt{pdflatex}, for example
%    \begin{macrocod
\def\CaporaliFrom#1#2#3#4
  \DeclareFontEncoding{#1}{}{
  \DeclareTextCommand{\it@ocap}{T1}
    {\fontencoding{#1}\fontfamily{#2}\selectfont\char#3\ignorespaces}
  \DeclareTextCommand{\it@ccap}{T1}{\ifdim\lastskip>\z@\unskip\f
    {\fontencoding{#1}\fontfamily{#2}\selectfont\char#4}
  \DeclareTextCommand{\it@ocap}{OT1}
    {\fontencoding{#1}\fontfamily{#2}\selectfont\char#3\ignorespaces}
  \DeclareTextCommand{\it@ccap}{OT1}{\ifdim\lastskip>\z@\unskip\f
    {\fontencoding{#1}\fontfamily{#2}\selectfont\char#4}
%    \end{macrocod
%    Then we set a boolean variable and test the default famil
%    if such family has a name that starts with the letters ``ae
%    then we have no built in guillemets; of course if the AE fo
%    family is chosen after the \babel\ package is loaded, the te
%    does not perform as required
%    \begin{macrocod
\def\get@ae#1#2#3!{\def\bbl@ae{#1#2}
\def\@ifT@one@noCap{\expandafter\get@ae\f@family
\def\bbl@temp{ae}\ifx\bbl@ae\bbl@temp\expandafter\@firstoftwo\el
    \expandafter\@secondoftwo\fi
%    \end{macrocod
%    We set another couple of boolean variables for testing t
%    existence of the CBgreek or the Cyrillic fon
%    \begin{macrocod
\newif\if@CBgreekEncKno
\IfFileExists{lgrcmr.fd
      {\@CBgreekEncKnowntrue}{\@CBgreekEncKnownfals
\newif\if@CyrEncKno
\IfFileExists{ot2cmr.fd
    {\@CyrEncKnowntrue}{\@CyrEncKnownfalse
%    \end{macrocod
%    \begin {macro}{\CBgreekCaporal
%    \begin {macro}{\CyrillicCaporal
%    \begin {macro}{\T@unoCaporal
%    Next we define the macros |\CBgreekCaporali|, |\T@unoCaporali
%    and |\CyrillicCaporali|; with the latter one we test the load
%    class, and if it's \texttt{slides} nothing gets done. In any ca
%    each one of these declarations, if used, must be specified in t
%    preambl
%    \begin{macrocod
\def\CBgreekCaporali{\@ifclassloaded{slides}
      \IfFileExists{lgrlcmss.fd}{\DeclareFontEncoding{LGR}{}{
            \DeclareRobustCommand*{\it@ccap
                  {\ifdim\lastskip>\z@\unskip\
                        {\fontencoding{LGR}\selectfont))}
            \DeclareRobustCommand*{\it@ocap
                  {{\fontencoding{LGR}\selectfont((}\ignorespaces}
            {\LtxSymbCaporali}
      {\DeclareFontEncoding{LGR}{}{
      \DeclareRobustCommand*{\it@ccap
            {\ifdim\lastskip>\z@\unsk
            \fi{\fontencoding{LGR}\selectfont))}
      \DeclareRobustCommand*{\it@ocap
            {{\fontencoding{LGR}\selectfont((}\ignorespaces}

\def\CyrillicCaporali{\@ifclassloaded{slides}{\relax
      {\DeclareFontEncoding{OT2}{}{
      \DeclareRobustCommand*{\it@ccap
            {\ifdim\lastskip>\z@\unskip\
            {\fontencoding{OT2}\selectfont\char62\relax}
      \DeclareRobustCommand*{\it@ocap
            {{\fontencoding{OT2}\selectfont\char60\relax}\ignorespaces}}
\@onlypreamble{\CBgreekCaporali}\@onlypreamble{\CyrillicCaporali
\def\T@unoCaporali{\DeclareRobustCommand*{\it@ocap}{<<\ignorespaces
     \DeclareRobustCommand*{\it@ccap}{\ifdim\lastskip>\z@\unskip\fi>>}
%    \end{macrocod
%    \end{macr
%    \end{macr
%    \end{macr
%    Now we can do some real setting; first we start testing the encodin
%    if the encoding is T1 we test if the font family is the AE one; if s
%    we further test for other possibiliti
%    \begin{macrocod
\ifx\cf@encoding\bbl@t@o
  \@ifT@one@noCap
     \if@CBgreekEncKno
        \CBgreekCapora
     \el
        \if@CyrEncKno
           \CyrilicCapora
        \el
           \LtxSymbCapora
        \
     \fi
     {\T@unoCaporali
%    \end{macrocod
%     But if the default encoding is not the T1 one, then t
%     substitutions must be performe
%    \begin{macrocod
\el
     \if@CBgreekEncKno
        \CBgreekCapora
     \el
        \if@CyrEncKno
           \CyrilicCapora
        \el
           \LtxSymbCapora
        \
     \
\
%    \end{macrocod
%    \end{macr
%    \end{macr
%    \end{macr

%    \subsection{Finishing command
%    The macro |\ldf@finish| takes care of looking for
%    configuration file, setting the main language to be switched
%    at |\begin{document}| and resetting the category code
%    \texttt{@} to its original valu
% \changes{italian-1.2i}{1996/11/03}{Now use \cs{ldf@finish} to wr
%    u
%    \begin{macrocod
\ldf@finish{italian
%</cod
%    \end{macrocod

% \Fina

%% \CharacterTab
%%  {Upper-case    \A\B\C\D\E\F\G\H\I\J\K\L\M\N\O\P\Q\R\S\T\U\V\W\X\Y
%%   Lower-case    \a\b\c\d\e\f\g\h\i\j\k\l\m\n\o\p\q\r\s\t\u\v\w\x\y
%%   Digits        \0\1\2\3\4\5\6\7\8
%%   Exclamation   \!     Double quote  \"     Hash (number)
%%   Dollar        \$     Percent       \%     Ampersand
%%   Acute accent  \'     Left paren    \(     Right paren
%%   Asterisk      \*     Plus          \+     Comma
%%   Minus         \-     Point         \.     Solidus
%%   Colon         \:     Semicolon     \;     Less than
%%   Equals        \=     Greater than  \>     Question mark
%%   Commercial at \@     Left bracket  \[     Backslash
%%   Right bracket \]     Circumflex    \^     Underscore
%%   Grave accent  \`     Left brace    \{     Vertical bar
%%   Right brace   \}     Tilde         \

\endinp
}
\DeclareOption{latin}{% \iffalse meta-comment
%
% Copyright 1989-2008 Johannes L. Braams and any individual authors
% listed elsewhere in this file.  All rights reserved.
% 
% This file is part of the Babel system.
% --------------------------------------
% 
% It may be distributed and/or modified under the
% conditions of the LaTeX Project Public License, either version 1.3
% of this license or (at your option) any later version.
% The latest version of this license is in
%   http://www.latex-project.org/lppl.txt
% and version 1.3 or later is part of all distributions of LaTeX
% version 2003/12/01 or later.
% 
% This work has the LPPL maintenance status "maintained".
% 
% The Current Maintainer of this work is Johannes Braams.
% 
% The list of all files belonging to the Babel system is
% given in the file `manifest.bbl. See also `legal.bbl' for additional
% information.
% 
% The list of derived (unpacked) files belonging to the distribution
% and covered by LPPL is defined by the unpacking scripts (with
% extension .ins) which are part of the distribution.
% \fi
% \CheckSum{454}
% \iffalse
%    Tell the \LaTeX\ system who we are and write an entry on the
%    transcript.
%<*dtx>
\ProvidesFile{latin.dtx}
%</dtx>
%<code>\ProvidesFile{latin.ldf}
%\fi
%\ProvidesFile{latin.dtx}%
        [2008/07/06 v2.0l Latin support from the babel system]
%\iffalse
%% File `latin.dtx'
%% Babel package for LaTeX version 2e
%% Copyright (C) 1989 - 2008
%%           by Johannes Braams, TeXniek
%
%% Please report errors to: J.L. Braams 
%%                          babel at braams.xs4all.nl
%%                          Claudio Beccari 
%%                          claudio.beccari at gmail.it
%
%    This file is part of the babel system, it provides the source
%    code for the Latin language definition file.
%    The original version of this file was written by
%    Claudio Beccari, (claudio.beccari@polito.it) and includes contributions
%    by Krzysztof Konrad \.Zelechowski, (\texttt{kkz@alfa.mimuw.edu.pl}).
%<*filedriver>
\documentclass{ltxdoc}
\newcommand*\TeXhax{\TeX hax}
\newcommand*\babel{\textsf{babel}}
\newcommand*\langvar{$\langle \it lang \rangle$}
\newcommand*\note[1]{}
\newcommand*\Lopt[1]{\textsf{#1}}
\newcommand*\file[1]{\texttt{#1}}
\providecommand*\pkg[1]{\textsf{#1}}
\begin{document}
 \DocInput{latin.dtx}
\end{document}
%</filedriver>
%\fi
% \GetFileInfo{latin.dtx}
%
% \changes{latin-0.99}{1999/12/06}{First version, from italian.dtx (CB)}
% \changes{latin-0.99}{1999/12/06}{Added shortcuts for breve, macron,
%   and etymological hyphenation (CB)}
% \changes{latin-1.2}{2000/01/31}{Added suggestions from Krzysztof 
%     Konrad \.Zelechowski (CB)}
% \changes{latin-2.0}{2000/02/10}{Completely new etymological
%     hyphenation (CB)}
% \changes{latin-2.0a}{2000/10/15}{Revised by JB}
% \changes{latin-2.0b}{2000/12/13}{Simplified shortcuts for
%     etymological hyphenation; modified breve and macro shortcuts; 
%     language attribute medieval declared}
% \changes{latin-2.0c}{2001/06/04}{Restored caret and equals sign
%     category codes before exiting}
% \changes{latin-2.0d}{2001/06/04}{Restored caret and equals sign
%     category codes before exiting}
% \changes{latin-2.0e}{2003/04/11}{Introduced the language attribute 
%     `withprosodicmarks'; modified use of breve and macron shortcuts
%      in order to avoid possible conflicts with other packages}
% \changes{latin-2.0i}{2008/02/17}{Corrected the \cs{@clubpenalty}
%    problem}
% \changes{latin-2.0j}{2008/03/17}{Added a missing comment char and a
%    missing closing brace}
% \changes{latin-2.0l}{2008/07/06}{Added two missing \cs{end} macro's}
%
%  \section{The Latin language}
%
%    The file \file{\filename}\footnote{The file described in this
%    section has version number \fileversion\ and was last revised on
%    \filedate. The original author is Claudio Beccari with
%    contributions by Krzysztof Konrad \.Zelechowski,
%    (\texttt{kkz@alfa.mimuw.edu.pl})} defines all the
%    language-specific macros for the Latin language both in modern
%    and medieval spelling. 
%
%    For this language the |\clubpenalty|, |\widowpenalty| are set to 
%    rather high values and |\finalhyphendemerits| is set to such a 
%    high value that hyphenation is prohibited between the last two
%    lines of a paragraph.
%
%    For this language two ``styles'' of typesetting are implemented: 
%    ``regular''  or modern-spelling Latin, and medieval Latin.
%    The medieval Latin specific commands can be activated by means of 
%    the language attribute |medieval|; the medieval spelling differs 
%    from the modern one by the systematic use of the lower case `u' 
%    also where in modern spelling the letter `v' is used; when 
%    typesetting with capital letters, on the opposite, the letter 'V' 
%    is used also in place of 'U'.
%    Medieval spelling also includes the ligatures |\ae| (\ae), |\oe|
%    (\oe), |\AE| (\AE), and |\OE| (\OE) that are not used in modern 
%    spelling, nor were used in the classical times.
%
%    Furthermore a third typesetting style |withprosodicmarks| is
%    defined in order to use special shortcuts for inserting breves
%    and macrons when typesetting grammars, dictionaries, teaching
%    texts, and the like, where prosodic marks are important for the
%    complete information on the words or the verses. The shortcuts,
%    listed in table~\ref{t:lashrtct} and described in
%    section~\ref{s:shrtcts}, may interfere with other packages;
%    therefore by default this third style is off and no interference
%    is introduced. If this third style is used and interference is
%    experienced, there are special commands for turning on and off
%    the specific short hand commands of this style.
%
%    For what concerns \textsf{babel} and typesetting with \LaTeX, the
%    differences between the two styles of spelling reveal themselves
%    in the strings used to name for example the ``Preface'' that
%    becomes ``Praefatio'' or ``Pr\ae fatio'' respectively.
%    Hyphenation rules are also different, but the hyphenation pattern
%    file \file{lahyph.tex} takes care of both versions of the 
%    language. Needless to say that such patterns must be loaded in
%    the \LaTeX\ format by running |initex| (or whatever the name
%    if the initializer) on |latex.ltx|.
%
%    The name strings for chapters, figures, tables, etcetera, are 
%    suggested by prof. Raffaella Tabacco, a classicist of the 
%    University of Turin, Italy, to whom we address our warmest 
%    thanks. The names suggested by Krzysztof Konrad \.Zelechowski,
%    when different, are used as the names for the medieval variety, 
%    since he made a word and spelling choice more suited for this 
%    variety.
%
%    For this language some shortcuts are defined according to 
%    table~\ref{t:lashrtct}; all of them are supposed to work with
%    both spelling styles, except where the opposite is explicitly
%    stated.
%    \begin{table}[htb]\centering
%    \begin{tabular}{cp{80mm}}
%    |^i|   & inserts the breve accent as \u{\i}; valid also for the
%             other lowercase vowels, but it does not operate on the
%             medieval ligatures \ae\ and \oe.\\
%    |=a|   & inserts the macron accent as \=a; valid also for the
%             other lowercase vowels, but it does not operate on the
%             medieval ligatures \ae\ and \oe.\\
%    |"|    & inserts a compound word mark where hyphenation is legal;
%             the next character must not be a medieval ligature \ae\
%             or \oe, nor an accented letter (foreign names).\\
%    \texttt{\string"\string|}
%           & same as above, but operates also when the next character
%             is a medieval ligature or an accented letter.
%    \end{tabular}
%    \caption[]{Shortcuts defined for the Latin language. The 
%               characters \texttt{\string^} and \texttt{\string=} are
%               active only when the language attribute
%               \texttt{withprosodicmarks} has been declared,
%               otherwise they are disabled; see 
%               section~\ref{s:shrtcts} for more details.}%
%    \label{t:lashrtct}
%    \end{table}
% \StopEventually{}
%
%    The macro |\LdfInit| takes care of preventing that this file is
%    loaded more than once, checking the category code of the
%    \texttt{@} sign, etc.
%    \begin{macrocode}
%<*code>
\LdfInit{latin}{captionslatin}
%    \end{macrocode}
%
%    When this file is read as an option, i.e. by the |\usepackage|
%    command, \texttt{latin} will be an `unknown' language in which
%    case we have to make it known.  So we check for the existence of
%    |\l@latin| to see whether we have to do something here.
%
%    \begin{macrocode}
\ifx\l@latin\@undefined
    \@nopatterns{Latin}
    \adddialect\l@latin0\fi
%    \end{macrocode}
%
%
%    Now we declare the |medieval| language attribute.
%    \begin{macrocode}
\bbl@declare@ttribute{latin}{medieval}{%
  \addto\captionslatin{\def\prefacename{Pr{\ae}fatio}}%
  \def\november{Nouembris}%
  \expandafter\addto\expandafter\extraslatin
  \expandafter{\extrasmedievallatin}%
  }
%    \end{macrocode}
%
%    The third typesetting style |withprosodicmarks| is defined here
%    \begin{macrocode}
\bbl@declare@ttribute{latin}{withprosodicmarks}{%
  \expandafter\addto\expandafter\extraslatin
  \expandafter{\extraswithprosodicmarks}%
  }
%    \end{macrocode}
%    It must be remembered that the |medieval| and the |withprosodicmarks|
%    styles may be used together.
%
%    The next step consists of defining commands to switch to (and
%    from) the Latin language\footnote{Most of these names were
%    kindly suggested by Raffaella Tabacco.}.
%
% \begin{macro}{\captionslatin}
%    The macro |\captionslatin| defines all strings used
%    in the four standard document classes provided with \LaTeX.
%    \begin{macrocode}
\@namedef{captionslatin}{%
  \def\prefacename{Praefatio}%
  \def\refname{Conspectus librorum}%
  \def\abstractname{Summarium}%
  \def\bibname{Conspectus librorum}%
  \def\chaptername{Caput}%
  \def\appendixname{Additamentum}%
  \def\contentsname{Index}%
  \def\listfigurename{Conspectus descriptionum}%
  \def\listtablename{Conspectus tabularum}%
  \def\indexname{Index rerum notabilium}%
  \def\figurename{Descriptio}%
  \def\tablename{Tabula}%
  \def\partname{Pars}%
  \def\enclname{Adduntur}%   Or " Additur" ? Or simply Add.?
  \def\ccname{Exemplar}%     Use the recipient's dative
  \def\headtoname{\ignorespaces}% Use the recipient's dative
  \def\pagename{Charta}%
  \def\seename{cfr.}%
  \def\alsoname{cfr.}% R.Tabacco never saw "cfr. atque" or similar forms
  \def\proofname{Demonstratio}%
  \def\glossaryname{Glossarium}%
  }
%    \end{macrocode}
% \end{macro}
%    In the above definitions there are some points that might change 
%    in the future or that require a minimum of attention from the
%    typesetter.
%    \begin{enumerate}
%    \item the \cs{enclname} is translated by a passive verb, that 
%      literally means ``(they) are being added''; if just one 
%      enclosure is joined to the document, the plural passive is not 
%      suited any more; nevertheless a generic plural passive might be 
%      incorrect but suited for most circumstances. On the opposite 
%      ``Additur'', the corresponding singular passive, might be more 
%      correct with one enclosure and less suited in general: what 
%      about the abbreviation ``Add.'' that works in both cases, but
%      certainly is less elegant?
%    \item The \cs{headtoname} is empty and gobbles the possible 
%      following space; in practice the typesetter should use the 
%      dative of the recipient's name; since nowadays not all such 
%      names can be translated into Latin, they might result 
%      indeclinable. The clever use of an appellative by the 
%      typesetter such as ``Domino'' or ``Dominae'' might solve the 
%      problem, but the header might get too impressive. The typesetter 
%      must make a decision on his own.
%    \item The same holds true for the copy recipient's name in the 
%      ``Cc'' field of \cs{ccname}.
%    \end{enumerate}
%
%  \begin{macro}{\datelatin}
%    The macro |\datelatin| redefines the command |\today| to produce
%    Latin dates; the choice of faked small caps  Latin numerals is
%    arbitrary and may be changed in the future. For medieval latin
%    the spelling of `Novembris' should be \textit{Nouembris}. This is
%    taken care of by using a control sequence which can be redefined
%    when the attribute `medieval' is selected.
% \changes{latin-2.0f}{2005/03/30}{Added a comment character to
%    prevent unwanted space} 
%    \begin{macrocode}
\def\datelatin{%
  \def\november{Novembris}%
  \def\today{%
    {\check@mathfonts\fontsize\sf@size\z@\math@fontsfalse\selectfont
      \uppercase\expandafter{\romannumeral\day}}~\ifcase\month\or
    Ianuarii\or Februarii\or Martii\or Aprilis\or Maii\or Iunii\or
    Iulii\or Augusti\or Septembris\or Octobris\or \november\or
    Decembris\fi
    \space{\uppercase\expandafter{\romannumeral\year}}}}
%    \end{macrocode}
%  \end{macro}
%
% \begin{macro}{\romandate}
%    Thomas Martin Widmann (\texttt{viralbus@daimi.au.dk}) developed a 
%    macro originally named |\latindate| (but to be renamed 
%    |\romandate| so as not to conflict with the standard \babel\ 
%    conventions) that should compute and translate the current date 
%    into a date \textit{ab urbe condita} with days numbered according 
%    to the kalendae and idus; for the moment this is a placeholder 
%    for Thomas' macro, waiting for a self standing one that keeps 
%    local all the intermediate data, counters, etc. If he succeeds,
%    here is the place to add his macro.
% \end{macro}
%
%  \begin{macro}{\latinhyphenmins}
%    The Latin hyphenation patterns can be used with both
%    |\lefthyphenmin| and |\righthyphenmin| set to~2.
% \changes{latin-2.0a}{2000/10/15}{Now use \cs{providehyphenmins} to
%    provide a default value}
%    \begin{macrocode}
\providehyphenmins{\CurrentOption}{\tw@\tw@}
%    \end{macrocode}
%  \end{macro}
%
% \begin{macro}{\extraslatin}
% \begin{macro}{\noextraslatin}
%    Lower the chance that clubs or widows occur.
%    \begin{macrocode}
\addto\extraslatin{%
  \babel@savevariable\clubpenalty
  \babel@savevariable\@clubpenalty
  \babel@savevariable\widowpenalty
  \clubpenalty3000\@clubpenalty3000\widowpenalty3000}
%    \end{macrocode}
%    Never ever break a word between the last two lines of a paragraph
%    in latin texts.
%    \begin{macrocode}
\addto\extraslatin{%
  \babel@savevariable\finalhyphendemerits
  \finalhyphendemerits50000000}
%    \end{macrocode}
%  \end{macro}
%  \end{macro}
%
%    With medieval Latin we need the suitable correspondence between
%    upper case V and lower case u, since in that spelling there is
%    only one sign, and the u shape is the (uncial) version of the
%    capital V. Everything else is identical with Latin.
%    \begin{macrocode}
\addto\extrasmedievallatin{%
  \babel@savevariable{\lccode`\V}%
  \babel@savevariable{\uccode`\u}%
  \lccode`\V=`\u \uccode`\u=`\V}
%    \end{macrocode}
%
% \begin{macro}{\SetLatinLigatures}
%    We need also the lccodes for \ae\ and \oe; since they occupy
%    different positions in the OT1 \TeX-fontencoding compared to the
%    T1 one, we must save the lc- and the uccodes for both encodings,
%    but we specify the new lc- and uccodes separately as it appears
%    natural not to change encoding while typesetting the same
%    language. The encoding is assumed to be set before starting to
%    use the Latin language, so that if Latin is the default language,
%    the font encoding must be chosen before requiring the \pkg{babel}
%    package with the |latin| option, in any case before any
%    |\selectlanguage| or |\foreignlanguage| command.
%
%    All this fuss is made in order to allow the use of the medieval
%    ligatures \ae\ and \oe\ while typesetting with the medieval
%    spelling; I have my doubts that the medieval spelling should be
%    used at all in modern books, reports, and the like; the uncial
%    `u' shape of the lower case `v' and the above ligatures were
%    fancy styles of the copyists who were able to write faster with
%    those rounded glyphs; with typesetting there is no question of
%    handling a quill penn\dots Since my (CB) opinion may be wrong, I
%    managed to set up the instruments and it is up to the typesetter
%    to use them or not.
%
%    \begin{macrocode}
\addto\extrasmedievallatin{%
  \babel@savevariable{\lccode`\^^e6}% T1   \ae
  \babel@savevariable{\uccode`\^^e6}% T1   \ae
  \babel@savevariable{\lccode`\^^c6}% T1   \AE
  \babel@savevariable{\lccode`\^^f7}% T1   \oe
  \babel@savevariable{\uccode`\^^f7}% T1   \OE
  \babel@savevariable{\lccode`\^^d7}% T1   \OE
  \babel@savevariable{\lccode`\^^1a}% OT1  \ae
  \babel@savevariable{\uccode`\^^1a}% OT1  \ae
  \babel@savevariable{\lccode`\^^1d}% OT1  \AE
  \babel@savevariable{\lccode`\^^1b}% OT1  \oe
  \babel@savevariable{\uccode`\^^1b}% OT1  \OE
  \babel@savevariable{\lccode`\^^1e}% OT1  \OE
  \SetLatinLigatures}
\providecommand\SetLatinLigatures{%
  \def\@tempA{T1}\ifx\@tempA\f@encoding
    \catcode`\^^e6=11 \lccode`\^^e6=`\^^e6 \uccode`\^^e6=`\^^c6 % \ae
    \catcode`\^^c6=11 \lccode`\^^c6=`\^^e6 % \AE
    \catcode`\^^f7=11 \lccode`\^^f7=`\^^f7 \uccode`\^^f7=`\^^d7 % \oe
    \catcode`\^^d7=11 \lccode`\^^d7=`\^^f7 % \OE
  \else
    \catcode`\^^1a=11 \lccode`\^^1a=`\^^1a \uccode`\^^1a=`\^^1d % \ae
    \catcode`\^^1d=11 \lccode`\^^1d=`\^^1a % \AE (^^])
    \catcode`\^^1b=11 \lccode`\^^1b=`\^^1b \uccode`\^^1b=`\^^1e % \oe
    \catcode`\^^1e=11 \lccode`\^^1e=`\^^1b % \OE (^^^)
  \fi
  \let\@tempA\@undefined
  }
%    \end{macrocode}
%    With the above definitions we are sure that |\MakeUppercase| 
%    works properly and |\MakeUppercase{C{\ae}sar}| correctly `yields
%    `C{\AE}SAR''; correspondingly |\MakeUppercase{Heluetia}|
%    correctly yields ``HELVETIA''. 
%    \end{macro}
%
%    \section{Latin shortcuts}\label{s:shrtcts}
%    For writing dictionaries or didactic texts (in modern spelling
%    only) we defined a third language attribute, or a third
%    typesetting style, a couple of other active characters are
%    defined: |^| for marking a vowel with the breve sign, and |=| for
%    marking a vowel with the macro sign. Please take notice that
%    neither the OT1 font encoding, nor the T1 one for most vowels,
%    contain directly the marked vowels, therefore hyphenation of
%    words containing these ``accents'' may become problematic; for
%    this reason the above active characters not only introduce the
%    required accent, but also an unbreakable zero skip that in
%    practice does not introduce a discretionary break, but allows
%    breaks in the rest of the word. 
%
%    It must be remarked that the active characters |^| and |=| may
%    have other meanings in other contexts. For example the equals
%    sign is used by the graphic extensions for specifying keyword
%    options for handling the graphic elements to be included in the
%    document. At the same time, as mentioned in the previous
%    paragraph, diacritical marking in Latin is used only for
%    typesetting certain kind of documents, such as grammars and
%    dictionaries. It is reasonable that the breve and macron active
%    characters are switched on and off at will, and in particular
%    that they are off by default if the attribute |withprosodicmarks|
%    has not been set. 
%
%    \begin{macro}{\ProsodicMarksOn}
%    \begin{macro}{\ProsodicMarksOff}
%    We begin by adding to the normal typesetting style the
%    definitions of the new commands |\ProsodicMarksOn| and
%    |\ProsodicMarksOff| that should produce error messages when the
%    third style is not declared: 
%    \begin{macrocode}
\addto\extraslatin{\def\ProsodicMarksOn{%
    \GenericError{(latin)\@spaces\@spaces\@spaces\@spaces}%
            {Latin language error: \string\ProsodicMarksOn\space
            is defined by setting the\MessageBreak
            language attribute to `withprosodicmarks'\MessageBreak
            If you continue you are likely to encounter\MessageBreak
            fatal errors that I can't recover}%
            {See the Latin language description in the babel
            documentation for explanation}{\@ehd}}}
\addto\extraslatin{\let\ProsodicMarksOff\relax}
%    \end{macrocode}
%
%    Then we temporarily set the caret and the equals sign to active
%    characters so that they can receive their definitions. But first
%    we store their current category codes to restore them later on.
% \changes{latin-2.0k}{2008/03/21}{Save current category codes of
%    equals sign and caret in order to restore them later} 
%    \begin{macrocode}
\@tempcnta=\catcode`\=
\@tempcntb=\catcode`\^
\catcode`\= \active
\catcode`\^ \active
%    \end{macrocode}
%    Now we can add the necessary declarations to the macros that are
%    being activated when the Latin language and its typesetting
%    styles are declared:
%    \begin{macrocode}
\addto\extraslatin{\languageshorthands{latin}}%
\addto\extraswithprosodicmarks{\bbl@activate{^}}%
\addto\extraswithprosodicmarks{\bbl@activate{=}}%
\addto\noextraswithprosodicmarks{\bbl@deactivate{^}}%
\addto\noextraswithprosodicmarks{\bbl@deactivate{=}}%
\addto\extraswithprosodicmarks{\ProsodicMarks}
%    \end{macrocode}
%  \end{macro}
%  \end{macro}
%
%    \begin{macro}{\ProsodicMarks}
%    Next we define the defining macro for the active characters
% \changes{latin-2.0k}{2008/03/21}{Use \cs{active} instead of 13}
%    \begin{macrocode}
\def\ProsodicMarks{%
  \def\ProsodicMarksOn{\catcode`\^ \active\catcode`\= \active}%
  \def\ProsodicMarksOff{\catcode`\^ 7\catcode`\= 12\relax}%
%    \end{macrocode}
%    Notice that with the above redefinitions of the commands
%    |\ProsodicMarksOn| and |\ProsodicMarksOff|, the operation of the
%    newly defined shortcuts may be switched on and off at will, so
%    that even if a picture has to be inserted in the document by
%    means of the commands and keyword options of the |graphicx|
%    package, it suffices to switch them off before invoking the
%    picture including command, and switched on again afterwards; or,
%    even better, since the picture very likely is being inserted
%    within a |figure| environment, it suffices to switch them off
%    within the environment, being conscious that their deactivation
%    remains local to the environment.
% \changes{latin-2.0g}{2005/11/17}{changed \cs{allowhyphens} to
%    \cs{bbl@allowhyphens}} 
%    \begin{macrocode}
  \initiate@active@char{^}%
  \initiate@active@char{=}%
  \declare@shorthand{latin}{^a}{%
    \textormath{\u{a}\bbl@allowhyphens}{\hat{a}}}%
  \declare@shorthand{latin}{^e}{%
    \textormath{\u{e}\bbl@allowhyphens}{\hat{e}}}%
  \declare@shorthand{latin}{^i}{%
    \textormath{\u{\i}\bbl@allowhyphens}{\hat{\imath}}}%
  \declare@shorthand{latin}{^o}{%
    \textormath{\u{o}\bbl@allowhyphens}{\hat{o}}}%
  \declare@shorthand{latin}{^u}{%
    \textormath{\u{u}\bbl@allowhyphens}{\hat{u}}}%
%
  \declare@shorthand{latin}{=a}{%
    \textormath{\={a}\bbl@allowhyphens}{\bar{a}}}%
  \declare@shorthand{latin}{=e}{%
    \textormath{\={e}\bbl@allowhyphens}{\bar{e}}}%
  \declare@shorthand{latin}{=i}{%
    \textormath{\={\i}\bbl@allowhyphens}{\bar{\imath}}}%
  \declare@shorthand{latin}{=o}{%
    \textormath{\={o}\bbl@allowhyphens}{\bar{o}}}%
  \declare@shorthand{latin}{=u}{%
    \textormath{\={u}\bbl@allowhyphens}{\bar{u}}}%
}
%    \end{macrocode}
%    Notice that the short hand definitions are given only for lower
%    case letters; it would not be difficult to extend the set of
%    definitions to upper case letters, but it appears of very little
%    use in view of the kind of documents where prosodic marks are
%    supposed to be used. Nevertheless in those rare cases when it's
%    required to set some uppercase letters with their prosodic
%    marks, it is always possible to use the standard \LaTeX\ commands
%    such as  |\u{I}| for typesetting \u{I}, or |\={A}| for
%    typesetting~\=A.
%
%    Finally we restore the caret and equals sign initial default
%    category codes.
% \changes{latin-2.0k}{2008/03/21}{Restore category codes rather then
%    return them to their \TeX\ default values. \emph{And} do that
%    outside of the command definition}
%    \begin{macrocode}
\catcode`\= \@tempcnta
\catcode`\^ \@tempcntb
%    \end{macrocode}
%    so as to avoid conflicts with other packages or other \babel\
%    options.
%    \end{macro}
%
%    \begin{macro}{\LatinMarksOn}
%    \begin{macro}{\LatinMarksOff}
% \changes{latin-2.0g}{2005/11/17}{Added two commands}
% \changes{latin-2.0h}{2007/10/19}{Added missing backslash}
%    We define two new commands so as to switch on and off the breve 
%    and macron shortcuts.
% \changes{latin-2.0h}{2007/10/20}{Removed the setting of
%    \cs{LatinMarksOff} from \cs{extraslatin}}
% \changes{latin-2.0k}{2008/03/21}{Set the \cs{LatinMarks...} commands
%    equal to the \cs{ProsodicMarks..} commands for compatibility} 
%    \begin{macrocode}
\addto\extraswithprosodicmarks{\let\LatinMarksOn\ProsodicMarksOn}
\addto\extraswithprosodicmarks{\let\LatinMarksOff\ProsodicMarksOff}
%    \end{macrocode}
%    \end{macro}
%    \end{macro}
%
%    It must be understood that by using the above prosodic marks,
%    line breaking is somewhat impeached; since such prosodic marks
%    are used almost exclusively in dictionaries, grammars, and poems
%    (only in school textbooks), this shouldn't be of any importance
%    for what concerns the quality of typesetting.
%
%    \section{Etymological hyphenation}
%    In order to deal in a clean way with prefixes and compound words
%    to be divided etymologically, the active character |"| is given 
%    a special definition so as to behave as a discretionary break
%    with hyphenation allowed after it. 
%    Most of the code for dealing with the active |"| is already
%    contained in the core of \babel, but we are going to use it as a
%    single character shorthand for Latin. 
%    \begin{macrocode}
\initiate@active@char{"}%
\addto\extraslatin{\bbl@activate{"}%
}
%    \end{macrocode}
%
%    A temporary macro is defined so as to take different actions in math
%    mode and text mode: specifically in the former case the macro inserts a
%    double quote as it should in math mode, otherwise another delayed macro
%    comes into action.
%    \begin{macrocode}
\declare@shorthand{latin}{"}{%
  \ifmmode
    \def\lt@@next{''}%
  \else
    \def\lt@@next{\futurelet\lt@temp\lt@cwm}%
  \fi
  \lt@@next
}%
%    \end{macrocode}
%    In text mode the \cs{lt@next} control sequence is such that upon
%    its execution a temporary variable \cs{lt@temp} is made
%    equivalent to the next token in the input list without actually
%    removing it. Such temporary token is then tested by the macro
%    \cs{lt@cwm} and if it is found to be a letter token, then it
%    introduces a compound word separator control sequence
%    \cs{lt@allowhyphens} whose expansion introduces a discretionary
%    hyphen and an unbreakable space; in case the token is not a
%    letter, the token is tested again to find if it is the character
%    \texttt{\string|}, in which case it is gobbled and a
%    discretionary break is introduced. 
% \changes{latin-2.0g}{2005/11/17}{Added a \cs{nobreak}}
%    \begin{macrocode}
\def\lt@allowhyphens{\nobreak\discretionary{-}{}{}\nobreak\hskip\z@skip}
\newcommand*{\lt@cwm}{\let\lt@n@xt\relax
  \ifcat\noexpand\lt@temp a%
    \let\lt@n@xt\lt@allowhyphens
  \else
    \if\noexpand\lt@temp\string|%
        \def\lt@n@xt{\lt@allowhyphens\@gobble}%
    \fi
  \fi
  \lt@n@xt}%
%    \end{macrocode}
%
%    Attention: the category code comparison does not work if the
%    temporary control sequence \cs{lt@temp} has been let equal to an
%    implicit character, such as |\ae|; therefore this etymological
%    hyphenation facility does not work with medieval Latin spelling
%    when |"| immediately precedes a ligature. In order to overcome
%    this drawback the shorthand \verb!"|! may be used in such cases;
%    it behaves exactly as |"|, but it does not test the implicit
%    character control sequence. An input such as
%    \verb!super"|{\ae}quitas!\footnote{This word does not exist in
%    ``regular'' Latin, and it is used just as an example.} gets
%    hyphenated as \texttt{su-per-{\ae}qui-tas} instead of
%    \texttt{su-pe-r\ae-qui-tas}. 
%
%    The macro |\ldf@finish| takes care of looking for a
%    configuration file, setting the main language to be switched on
%    at |\begin{document}| and resetting the category code of
%    \texttt{@} to its original value.
%    \begin{macrocode}
\ldf@finish{latin}
%</code>
%    \end{macrocode}
%
% \Finale
%%
%% \CharacterTable
%%  {Upper-case    \A\B\C\D\E\F\G\H\I\J\K\L\M\N\O\P\Q\R\S\T\U\V\W\X\Y\Z
%%   Lower-case    \a\b\c\d\e\f\g\h\i\j\k\l\m\n\o\p\q\r\s\t\u\v\w\x\y\z
%%   Digits        \0\1\2\3\4\5\6\7\8\9
%%   Exclamation   \!     Double quote  \"     Hash (number) \#
%%   Dollar        \$     Percent       \%     Ampersand     \&
%%   Acute accent  \'     Left paren    \(     Right paren   \)
%%   Asterisk      \*     Plus          \+     Comma         \,
%%   Minus         \-     Point         \.     Solidus       \/
%%   Colon         \:     Semicolon     \;     Less than     \<
%%   Equals        \=     Greater than  \>     Question mark \?
%%   Commercial at \@     Left bracket  \[     Backslash     \\
%%   Right bracket \]     Circumflex    \^     Underscore    \_
%%   Grave accent  \`     Left brace    \{     Vertical bar  \|
%%   Right brace   \}     Tilde         \~}
%%
\endinput
}
\DeclareOption{lowersorbian}{% \iffalse meta-comme

% Copyright 1989-2008 Johannes L. Braams and any individual autho
% listed elsewhere in this file.  All rights reserve

% This file is part of the Babel syste
% ------------------------------------

% It may be distributed and/or modified under t
% conditions of the LaTeX Project Public License, either version 1
% of this license or (at your option) any later versio
% The latest version of this license is
%   http://www.latex-project.org/lppl.t
% and version 1.3 or later is part of all distributions of LaT
% version 2003/12/01 or late

% This work has the LPPL maintenance status "maintained

% The Current Maintainer of this work is Johannes Braam

% The list of all files belonging to the Babel system
% given in the file `manifest.bbl. See also `legal.bbl' for addition
% informatio

% The list of derived (unpacked) files belonging to the distributi
% and covered by LPPL is defined by the unpacking scripts (wi
% extension .ins) which are part of the distributio
% \
% \CheckSum{15
% \iffal

%    Tell the \LaTeX\ system who we are and write an entry on t
%    transcrip
%<*dt
\ProvidesFile{lsorbian.dt
%</dt
%<code>\ProvidesLanguage{lsorbia
%\
%\ProvidesFile{lsorbian.dt
        [2008/03/17 v1.0g Lower Sorbian support from the babel syste
%\iffal
%% File `lsorbian.dt
%% Babel package for LaTeX version
%% Copyright (C) 1989 - 20
%%           by Johannes Braams, TeXni

%% Lower Sorbian Language Definition Fi
%% Copyright (C) 1994 - 20
%%           by Eduard Wern
%           Werner, Eduard
%           Serbski institut z. t
%           Dw\'orni\v{s}\'cowa
%           02625 Budy\v{s}in/Bautz
%           Germany
%           (??)3591 497223
%           edi at kaihh.hanse.de

%% Please report errors to: Eduard Werner edi at kaihh.hanse.

%    This file is part of the babel system, it provides the sour
%    code for the Lower Sorbian definition fil
%<*filedrive
\documentclass{ltxdo
\newcommand*\TeXhax{\TeX ha
\newcommand*\babel{\textsf{babel
\newcommand*\langvar{$\langle \it lang \rangle
\newcommand*\note[1]
\newcommand*\Lopt[1]{\textsf{#1
\newcommand*\file[1]{\texttt{#1
\begin{documen
 \DocInput{lsorbian.dt
\end{documen
%</filedrive
%\

% \GetFileInfo{lsorbian.dt

% \changes{lsorbian-0.1}{1994/10/10}{First versio
% \changes{lsorbian-1.0d}{1996/10/10}{Replaced \cs{undefined} wi
%    \cs{@undefined} and \cs{empty} with \cs{@empty} for consisten
%    with \LaTeX, moved the definition of \cs{atcatcode} right to t
%    beginning

%  \section{The Lower Sorbian languag

%    The file \file{\filename}\footnote{The file described in th
%    section has version number \fileversion\ and was last revised
%    \filedate.  It was written by Eduard Wern
%    (\texttt{edi@kaihh.hanse.de}).}  It defines all t
%    language-specific macros for Lower Sorbia

% \StopEventually

%    The macro |\LdfInit| takes care of preventing that this file
%    loaded more than once, checking the category code of t
%    \texttt{@} sign, et
% \changes{lsorbian-1.0d}{1996/11/03}{Now use \cs{LdfInit} to perfo
%    initial check
% \changes{lsorbian-1.0g}{2007/10/19}{This file can be loaded und
%    more than one name
%    \begin{macrocod
%<*cod
\LdfInit\CurrentOption{date\CurrentOptio
%    \end{macrocod

%    When this file is read as an option, i.e. by the |\usepackag
%    command, \texttt{lsorbian} will be an `unknown' language, in whi
%    case we have to make it known. So we check for the existence
%    |\l@lsorbian| to see whether we have to do something her
% \changes{lsorbian-1.0g}{2007/10/19}{This file can be loaded und
%    more than one name
%
%    \babel\ also knwos the option \Lopt{lowersorbian} we have
%    check that as wel

%    \begin{macrocod
\ifx\l@lowersorbian\@undefin
  \ifx\l@lsorbian\@undefin
    \@nopatterns{Lsorbia
    \adddialect\l@lsorbian\
    \let\l@lowersorbian\l@lsorbi
  \el
    \let\l@lowersorbian\l@lsorbi
  \
\el
  \let\l@lsorbian\l@lowersorbi
\
%    \end{macrocod

%    The next step consists of defining commands to switch to (a
%    from) the Lower Sorbian languag

%  \begin{macro}{\captionslsorbia
%    The macro |\captionslsorbian| defines all strings used in the fo
%    standard documentclasses provided with \LaTe
% \changes{lsorbian-1.0b}{1995/07/04}{Added \cs{proofname} f
%    AMS-\LaTe
% \changes{lsorbian-1.0f}{2000/09/22}{Added \cs{glossaryname
% \changes{lsorbian-1.0g}{2007/10/19}{Make this work for more than o
%    option name
%    \begin{macrocod
\@namedef{captions\CurrentOption}
  \def\prefacename{Zawod
  \def\refname{Referency
  \def\abstractname{Abstrakt
  \def\bibname{Literatura
  \def\chaptername{Kapitl
  \def\appendixname{Dodawki
  \def\contentsname{Wop\'simje\'se
  \def\listfigurename{Zapis wobrazow
  \def\listtablename{Zapis tabulkow
  \def\indexname{Indeks
  \def\figurename{Wobraz
  \def\tablename{Tabulka
  \def\partname{\'Z\v el
  \def\enclname{P\'si\l oga
  \def\ccname{CC
  \def\headtoname{Komu
  \def\pagename{Strona
  \def\seename{gl.
  \def\alsoname{gl.~teke
  \def\proofname{Proof}%  <-- needs translati
  \def\glossaryname{Glossary}% <-- Needs translati

%    \end{macrocod
%  \end{macr

%  \begin{macro}{\newdatelsorbia
%    The macro |\newdatelsorbian| redefines the command |\today|
%    produce Lower Sorbian date
% \changes{lsorbian-1.0e}{1997/10/01}{Use \cs{edef} to defi
%    \cs{today} to save memor
% \changes{lsorbian-1.0e}{1998/03/28}{use \cs{def} instead
%    \cs{edef}
% \changes{lsorbian-1.0g}{2007/10/19}{Make this work for more than o
%    option name
%    \begin{macrocod
\@namedef{newdate\CurrentOption}
  \def\today{\number\day.~\ifcase\month\
    januara\or februara\or m\v erca\or apryla\or maja\
    junija\or julija\or awgusta\or septembra\or oktobra\
    nowembra\or decembra\
    \space \number\year
%    \end{macrocod
%  \end{macr

%  \begin{macro}{\olddatelsorbia
%    The macro |\olddatelsorbian| redefines the command |\today|
%    produce old-style Lower Sorbian date
% \changes{lsorbian-1.0g}{2007/10/19}{Make this work for more than o
%    option name
%    \begin{macrocod
\@namedef{olddate\CurrentOption}
  \def\today{\number\day.~\ifcase\month\
    wjelikego ro\v zka\
    ma\l ego ro\v zka\
    nal\v etnika\
    jat\v sownika\
    ro\v zownika\
    sma\v znika\
    pra\v znika\
    \v znje\'nca\
    po\v znje\'nca\
    winowca\
    nazymnika\o
    godownika\fi \space \number\year
%    \end{macrocod
%  \end{macr

%    The default will be the new-style date
% \changes{lsorbian-1.0g}{2007/10/19}{Make this work for more than o
%    option name
%    \begin{macrocod
\expandafter\let\csname date\CurrentOption\expandafter\endcsna
                \csname newdate\CurrentOption\endcsna
%    \end{macrocod

% \begin{macro}{\extraslsorbia
% \begin{macro}{\noextraslsorbia
%    The macro |\extraslsorbian| will perform all the ext
%    definitions needed for the lsorbian language. The mac
%    |\noextraslsorbian| is used to cancel the actions
%    |\extraslsorbian|.  For the moment these macros are empty b
%    they are defined for compatibility with the other langua
%    definition file

%    \begin{macrocod
\@namedef{extras\CurrentOption}
\@namedef{noextras\CurrentOption}
%    \end{macrocod
% \end{macr
% \end{macr

%    The macro |\ldf@finish| takes care of looking for
%    configuration file, setting the main language to be switched
%    at |\begin{document}| and resetting the category code
%    \texttt{@} to its original valu
% \changes{lsorbian-1.0d}{1996/11/03}{Now use \cs{ldf@finish} to wr
%    up
% \changes{lsorbian-1.0g}{2007/10/19}{Make this work for more than o
%    option nam
%    \begin{macrocod
\ldf@finish\CurrentOpti
%</cod
%    \end{macrocod

% \Fina

%% \CharacterTab
%%  {Upper-case    \A\B\C\D\E\F\G\H\I\J\K\L\M\N\O\P\Q\R\S\T\U\V\W\X\Y
%%   Lower-case    \a\b\c\d\e\f\g\h\i\j\k\l\m\n\o\p\q\r\s\t\u\v\w\x\y
%%   Digits        \0\1\2\3\4\5\6\7\8
%%   Exclamation   \!     Double quote  \"     Hash (number)
%%   Dollar        \$     Percent       \%     Ampersand
%%   Acute accent  \'     Left paren    \(     Right paren
%%   Asterisk      \*     Plus          \+     Comma
%%   Minus         \-     Point         \.     Solidus
%%   Colon         \:     Semicolon     \;     Less than
%%   Equals        \=     Greater than  \>     Question mark
%%   Commercial at \@     Left bracket  \[     Backslash
%%   Right bracket \]     Circumflex    \^     Underscore
%%   Grave accent  \`     Left brace    \{     Vertical bar
%%   Right brace   \}     Tilde         \

\endinp
}
%^^A\DeclareOption{kannada}{\input{kannada.ldf}}
\DeclareOption{magyar}{%%
%% This file will generate fast loadable files and documentation
%% driver files from the doc files in this package when run through
%% LaTeX or TeX.
%%
%% Copyright 1989-2005 Johannes L. Braams and any individual authors
%% listed elsewhere in this file.  All rights reserved.
%% 
%% This file is part of the Babel system.
%% --------------------------------------
%% 
%% It may be distributed and/or modified under the
%% conditions of the LaTeX Project Public License, either version 1.3
%% of this license or (at your option) any later version.
%% The latest version of this license is in
%%   http://www.latex-project.org/lppl.txt
%% and version 1.3 or later is part of all distributions of LaTeX
%% version 2003/12/01 or later.
%% 
%% This work has the LPPL maintenance status "maintained".
%% 
%% The Current Maintainer of this work is Johannes Braams.
%% 
%% The list of all files belonging to the LaTeX base distribution is
%% given in the file `manifest.bbl. See also `legal.bbl' for additional
%% information.
%% 
%% The list of derived (unpacked) files belonging to the distribution
%% and covered by LPPL is defined by the unpacking scripts (with
%% extension .ins) which are part of the distribution.
%%
%% --------------- start of docstrip commands ------------------
%%
\def\filedate{1999/04/11}
\def\batchfile{magyar.ins}
\input docstrip.tex

{\ifx\generate\undefined
\Msg{**********************************************}
\Msg{*}
\Msg{* This installation requires docstrip}
\Msg{* version 2.3c or later.}
\Msg{*}
\Msg{* An older version of docstrip has been input}
\Msg{*}
\Msg{**********************************************}
\errhelp{Move or rename old docstrip.tex.}
\errmessage{Old docstrip in input path}
\batchmode
\csname @@end\endcsname
\fi}

\declarepreamble\mainpreamble
This is a generated file.

Copyright 1989-2005 Johannes L. Braams and any individual authors
listed elsewhere in this file.  All rights reserved.

This file was generated from file(s) of the Babel system.
---------------------------------------------------------

It may be distributed and/or modified under the
conditions of the LaTeX Project Public License, either version 1.3
of this license or (at your option) any later version.
The latest version of this license is in
  http://www.latex-project.org/lppl.txt
and version 1.3 or later is part of all distributions of LaTeX
version 2003/12/01 or later.

This work has the LPPL maintenance status "maintained".

The Current Maintainer of this work is Johannes Braams.

This file may only be distributed together with a copy of the Babel
system. You may however distribute the Babel system without
such generated files.

The list of all files belonging to the Babel distribution is
given in the file `manifest.bbl'. See also `legal.bbl for additional
information.

The list of derived (unpacked) files belonging to the distribution
and covered by LPPL is defined by the unpacking scripts (with
extension .ins) which are part of the distribution.
\endpreamble

\declarepreamble\fdpreamble
This is a generated file.

Copyright 1989-2005 Johannes L. Braams and any individual authors
listed elsewhere in this file.  All rights reserved.

This file was generated from file(s) of the Babel system.
---------------------------------------------------------

It may be distributed and/or modified under the
conditions of the LaTeX Project Public License, either version 1.3
of this license or (at your option) any later version.
The latest version of this license is in
  http://www.latex-project.org/lppl.txt
and version 1.3 or later is part of all distributions of LaTeX
version 2003/12/01 or later.

This work has the LPPL maintenance status "maintained".

The Current Maintainer of this work is Johannes Braams.

This file may only be distributed together with a copy of the Babel
system. You may however distribute the Babel system without
such generated files.

The list of all files belonging to the Babel distribution is
given in the file `manifest.bbl'. See also `legal.bbl for additional
information.

In particular, permission is granted to customize the declarations in
this file to serve the needs of your installation.

However, NO PERMISSION is granted to distribute a modified version
of this file under its original name.

\endpreamble

\keepsilent

\usedir{tex/generic/babel} 

\usepreamble\mainpreamble
\generate{\file{magyar.ldf}{\from{magyar.dtx}{code}}
          }
\usepreamble\fdpreamble

\ifToplevel{
\Msg{***********************************************************}
\Msg{*}
\Msg{* To finish the installation you have to move the following}
\Msg{* files into a directory searched by TeX:}
\Msg{*}
\Msg{* \space\space All *.def, *.fd, *.ldf, *.sty}
\Msg{*}
\Msg{* To produce the documentation run the files ending with}
\Msg{* '.dtx' and `.fdd' through LaTeX.}
\Msg{*}
\Msg{* Happy TeXing}
\Msg{***********************************************************}
}
 
\endinput
}
%^^A\DeclareOption{nagari}{\input{nagari.ldf}}
%    \end{macrocode}
%    `New' German orthography, including Austrian variant:
% \changes{babel~3.6p}{1999/04/10}{Added the \Lopt{ngerman} and
%    \Lopt{naustrian} options}
% \changes{babel~3.7f}{2000/09/26}{Added the \Lopt{samin} option}
% \changes{babel~3.8c}{2004/06/12}{Added the \Lopt{newzealand} option}
%    \begin{macrocode}
\DeclareOption{naustrian}{% \iffalse meta-comment
%
% Copyright 1989-2008 Johannes L. Braams and any individual authors
% listed elsewhere in this file.  All rights reserved.
% 
% This file is part of the Babel system.
% --------------------------------------
% 
% It may be distributed and/or modified under the
% conditions of the LaTeX Project Public License, either version 1.3
% of this license or (at your option) any later version.
% The latest version of this license is in
%   http://www.latex-project.org/lppl.txt
% and version 1.3 or later is part of all distributions of LaTeX
% version 2003/12/01 or later.
% 
% This work has the LPPL maintenance status "maintained".
% 
% The Current Maintainer of this work is Johannes Braams.
% 
% The list of all files belonging to the Babel system is
% given in the file `manifest.bbl. See also `legal.bbl' for additional
% information.
% 
% The list of derived (unpacked) files belonging to the distribution
% and covered by LPPL is defined by the unpacking scripts (with
% extension .ins) which are part of the distribution.
% \fi
% \CheckSum{266}
%
% \iffalse
%    Tell the \LaTeX\ system who we are and write an entry on the
%    transcript.
%<*dtx>
\ProvidesFile{ngermanb.dtx}
%</dtx>
%<code>\ProvidesLanguage{ngermanb}
%\fi
%\ProvidesFile{ngermanb.dtx}
        [2008/03/17 v2.6m new German support from the babel system]
%\iffalse
%% File `ngermanb.dtx'
%% Babel package for LaTeX version 2e
%% Copyright (C) 1989 - 2008
%%           by Johannes Braams, TeXniek
%
%% new Germanb Language Definition File
%% Copyright (C) 1989 - 2008
%%           by Bernd Raichle raichle at azu.Informatik.Uni-Stuttgart.de
%%              Johannes Braams, TeXniek,
%%              Walter Schmidt.
% This file is based on german.tex version 2.5e
%                       by Bernd Raichle, Hubert Partl et.al.
%
%% Please report errors to: J.L. Braams
%%                          babel at braams.xs4all.nl
%
%<*filedriver>
\documentclass{ltxdoc}
\font\manual=logo10 % font used for the METAFONT logo, etc.
\newcommand*\MF{{\manual META}\-{\manual FONT}}
\newcommand*\TeXhax{\TeX hax}
\newcommand*\babel{\textsf{babel}}
\newcommand*\langvar{$\langle \it lang \rangle$}
\newcommand*\note[1]{}
\newcommand*\Lopt[1]{\textsf{#1}}
\newcommand*\file[1]{\texttt{#1}}
\begin{document}
 \DocInput{ngermanb.dtx}
\end{document}
%</filedriver>
%\fi
% \GetFileInfo{ngermanb.dtx}
%
% \changes{ngermanb-2.6f}{1999/03/24}{Renamed from \file{germanb.ldf};
%          language names changed from \texttt{german} and \texttt{austrian}
%          to \texttt{ngerman} and \texttt{naustrian}.}
%
%  \section{The German language -- new orthography}
%
%    The file \file{\filename}\footnote{The file described in this
%    section has version number \fileversion\ and was last revised on
%    \filedate.}  defines all the language definition macros for the
%    German language with the `new orthography' introduced in
%    August 1998.  This includes also the Austrian dialect of this
%    language.
%  
%    As with the `traditional'  German orthography, 
%    the character |"| is made active, and 
%    the commands in  table~\ref{tab:german-quote} can be used, except
%    for |"ck| and |"ff| etc., which are no longer required.
%
%    The internal language names are |ngerman| and |naustrian|.
%
% \StopEventually{}
%
%    When this file was read through the option \Lopt{ngermanb} we make
%    it behave as if \Lopt{ngerman} was specified.
%    \begin{macrocode}
\def\bbl@tempa{ngermanb}
\ifx\CurrentOption\bbl@tempa
  \def\CurrentOption{ngerman}
\fi
%    \end{macrocode}
%
%    The macro |\LdfInit| takes care of preventing that this file is
%    loaded more than once, checking the category code of the
%    \texttt{@} sign, etc.
%    \begin{macrocode}
%<*code>
\LdfInit\CurrentOption{captions\CurrentOption}
%    \end{macrocode}
%
%    When this file is read as an option, i.e., by the |\usepackage|
%    command, \texttt{ngerman} will be an `unknown' language, so we
%    have to make it known.  So we check for the existence of
%    |\l@ngerman| to see whether we have to do something here.
%
%    \begin{macrocode}
\ifx\l@ngerman\@undefined
  \@nopatterns{ngerman}
  \adddialect\l@ngerman0
\fi
%    \end{macrocode}
%
%    For the Austrian version of these definitions we just add another
%    language. 
%    \begin{macrocode}
\adddialect\l@naustrian\l@ngerman
%    \end{macrocode}
%
%    The next step consists of defining commands to switch to (and
%    from) the German language.
%
%  \begin{macro}{\captionsngerman}
%  \begin{macro}{\captionsnaustrian}
%    Either the macro |\captionnsgerman| or the macro
%    |\captionsnaustrian| will define all strings used in the four
%    standard document classes provided with \LaTeX.
%
% \changes{ngermanb-2.6k}{2000/09/20}{Added \cs{glossaryname}}
%    \begin{macrocode}
\@namedef{captions\CurrentOption}{%
  \def\prefacename{Vorwort}%
  \def\refname{Literatur}%
  \def\abstractname{Zusammenfassung}%
  \def\bibname{Literaturverzeichnis}%
  \def\chaptername{Kapitel}%
  \def\appendixname{Anhang}%
  \def\contentsname{Inhaltsverzeichnis}%    % oder nur: Inhalt
  \def\listfigurename{Abbildungsverzeichnis}%
  \def\listtablename{Tabellenverzeichnis}%
  \def\indexname{Index}%
  \def\figurename{Abbildung}%
  \def\tablename{Tabelle}%                  % oder: Tafel
  \def\partname{Teil}%
  \def\enclname{Anlage(n)}%                 % oder: Beilage(n)
  \def\ccname{Verteiler}%                   % oder: Kopien an
  \def\headtoname{An}%
  \def\pagename{Seite}%
  \def\seename{siehe}%
  \def\alsoname{siehe auch}%
  \def\proofname{Beweis}%
  \def\glossaryname{Glossar}%
  }
%    \end{macrocode}
%  \end{macro}
%  \end{macro}
%
%  \begin{macro}{\datengerman}
%    The macro |\datengerman| redefines the command
%    |\today| to produce German dates.
%    \begin{macrocode}
\def\month@ngerman{\ifcase\month\or
  Januar\or Februar\or M\"arz\or April\or Mai\or Juni\or
  Juli\or August\or September\or Oktober\or November\or Dezember\fi}
\def\datengerman{\def\today{\number\day.~\month@ngerman
    \space\number\year}}
%    \end{macrocode}
%  \end{macro}
%
%  \begin{macro}{\dateanustrian}
%    The macro |\datenaustrian| redefines the command
%    |\today| to produce Austrian version of the German dates.
%    \begin{macrocode}
\def\datenaustrian{\def\today{\number\day.~\ifnum1=\month
  J\"anner\else \month@ngerman\fi \space\number\year}}
%    \end{macrocode}
%  \end{macro}
%
%  \begin{macro}{\extrasngerman}
%  \begin{macro}{\extrasnaustrian}
%  \begin{macro}{\noextrasngerman}
%  \begin{macro}{\noextrasnaustrian}
%    Either the macro |\extrasngerman| or the macros |\extrasnaustrian|
%    will perform all the extra definitions needed for the German
%    language. The macro |\noextrasngerman| is used to cancel the
%    actions of |\extrasngerman|. 
%
%    For German (as well as for Dutch) the \texttt{"} character is
%    made active. This is done once, later on its definition may vary.
%    \begin{macrocode}
\initiate@active@char{"}
\@namedef{extras\CurrentOption}{%
  \languageshorthands{ngerman}}
\expandafter\addto\csname extras\CurrentOption\endcsname{%
  \bbl@activate{"}}
%    \end{macrocode}
%    Don't forget to turn the shorthands off again.
% \changes{ngermanb-2.6j}{1999/12/16}{Deactivate shorthands ouside of
%    German}
%    \begin{macrocode}
\addto\noextrasngerman{\bbl@deactivate{"}}
%    \end{macrocode}
%
%
%    In order for \TeX\ to be able to hyphenate German words which
%    contain `\ss' (in the \texttt{OT1} position |^^Y|) we have to
%    give the character a nonzero |\lccode| (see Appendix H, the \TeX
%    book).
%    \begin{macrocode}
\expandafter\addto\csname extras\CurrentOption\endcsname{%
  \babel@savevariable{\lccode25}%
  \lccode25=25}
%    \end{macrocode}
%
%    The umlaut accent macro |\"| is changed to lower the umlaut dots.
%    The redefinition is done with the help of |\umlautlow|.
%    \begin{macrocode}
\expandafter\addto\csname extras\CurrentOption\endcsname{%
  \babel@save\"\umlautlow}
\@namedef{noextras\CurrentOption}{\umlauthigh}
%    \end{macrocode}
%    The current 
%    version of the `new' German hyphenation patterns (\file{dehyphn.tex}
%    is to be used with |\lefthyphenmin| and |\righthyphenmin| set to~2. 
% \changes{ngermanb-2.6k}{2000/09/22}{Now use \cs{providehyphenmins} to
%    provide a default value}
%    \begin{macrocode}
\providehyphenmins{\CurrentOption}{\tw@\tw@}
%    \end{macrocode}
%    For German texts we need to make sure that |\frenchspacing| is
%    turned on.
% \changes{ngermanb-2.6m}{2001/01/26}{Turn frenchspacing on, as in
%    \texttt{german.sty}}
%    \begin{macrocode}
\expandafter\addto\csname extras\CurrentOption\endcsname{%
  \bbl@frenchspacing}
\expandafter\addto\csname noextras\CurrentOption\endcsname{%
  \bbl@nonfrenchspacing}
%    \end{macrocode}
%  \end{macro}
%  \end{macro}
%  \end{macro}
%  \end{macro}
%
%    The code above is necessary because we need an extra active
%    character. This character is then used as indicated in
%    table~\ref{tab:german-quote}.
%
%    To be able to define the function of |"|, we first define a
%    couple of `support' macros.
%
%
%  \begin{macro}{\dq}
%    We save the original double quote character in |\dq| to keep
%    it available, the math accent |\"| can now be typed as |"|.
%    \begin{macrocode}
\begingroup \catcode`\"12
\def\x{\endgroup
  \def\@SS{\mathchar"7019 }
  \def\dq{"}}
\x
%    \end{macrocode}
%  \end{macro}
%
%    Now we can define the doublequote macros: the umlauts,
%    \begin{macrocode}
\declare@shorthand{ngerman}{"a}{\textormath{\"{a}\allowhyphens}{\ddot a}}
\declare@shorthand{ngerman}{"o}{\textormath{\"{o}\allowhyphens}{\ddot o}}
\declare@shorthand{ngerman}{"u}{\textormath{\"{u}\allowhyphens}{\ddot u}}
\declare@shorthand{ngerman}{"A}{\textormath{\"{A}\allowhyphens}{\ddot A}}
\declare@shorthand{ngerman}{"O}{\textormath{\"{O}\allowhyphens}{\ddot O}}
\declare@shorthand{ngerman}{"U}{\textormath{\"{U}\allowhyphens}{\ddot U}}
%    \end{macrocode}
%    tremas,
%    \begin{macrocode}
\declare@shorthand{ngerman}{"e}{\textormath{\"{e}}{\ddot e}}
\declare@shorthand{ngerman}{"E}{\textormath{\"{E}}{\ddot E}}
\declare@shorthand{ngerman}{"i}{\textormath{\"{\i}}%
                              {\ddot\imath}}
\declare@shorthand{ngerman}{"I}{\textormath{\"{I}}{\ddot I}}
%    \end{macrocode}
%    german es-zet (sharp s),
%    \begin{macrocode}
\declare@shorthand{ngerman}{"s}{\textormath{\ss}{\@SS{}}}
\declare@shorthand{ngerman}{"S}{\SS}
\declare@shorthand{ngerman}{"z}{\textormath{\ss}{\@SS{}}}
\declare@shorthand{ngerman}{"Z}{SZ}
%    \end{macrocode}
%    german and french quotes,
%    \begin{macrocode}
\declare@shorthand{ngerman}{"`}{\glqq}
\declare@shorthand{ngerman}{"'}{\grqq}
\declare@shorthand{ngerman}{"<}{\flqq}
\declare@shorthand{ngerman}{">}{\frqq}
%    \end{macrocode}
%    and some additional commands:
%    \begin{macrocode}
\declare@shorthand{ngerman}{"-}{\nobreak\-\bbl@allowhyphens}
\declare@shorthand{ngerman}{"|}{%
  \textormath{\penalty\@M\discretionary{-}{}{\kern.03em}%
              \allowhyphens}{}}
\declare@shorthand{ngerman}{""}{\hskip\z@skip}
\declare@shorthand{ngerman}{"~}{\textormath{\leavevmode\hbox{-}}{-}}
\declare@shorthand{ngerman}{"=}{\penalty\@M-\hskip\z@skip}
%    \end{macrocode}
%
%  \begin{macro}{\mdqon}
%  \begin{macro}{\mdqoff}
%    All that's left to do now is to  define a couple of commands
%    for reasons of compatibility with \file{german.sty}.
%    \begin{macrocode}
\def\mdqon{\shorthandon{"}}
\def\mdqoff{\shorthandoff{"}}
%    \end{macrocode}
%  \end{macro}
%  \end{macro}
%
%    The macro |\ldf@finish| takes care of looking for a
%    configuration file, setting the main language to be switched on
%    at |\begin{document}| and resetting the category code of
%    \texttt{@} to its original value.
%    \begin{macrocode}
\ldf@finish\CurrentOption
%</code>
%    \end{macrocode}
%
% \Finale
%%
%% \CharacterTable
%%  {Upper-case    \A\B\C\D\E\F\G\H\I\J\K\L\M\N\O\P\Q\R\S\T\U\V\W\X\Y\Z
%%   Lower-case    \a\b\c\d\e\f\g\h\i\j\k\l\m\n\o\p\q\r\s\t\u\v\w\x\y\z
%%   Digits        \0\1\2\3\4\5\6\7\8\9
%%   Exclamation   \!     Double quote  \"     Hash (number) \#
%%   Dollar        \$     Percent       \%     Ampersand     \&
%%   Acute accent  \'     Left paren    \(     Right paren   \)
%%   Asterisk      \*     Plus          \+     Comma         \,
%%   Minus         \-     Point         \.     Solidus       \/
%%   Colon         \:     Semicolon     \;     Less than     \<
%%   Equals        \=     Greater than  \>     Question mark \?
%%   Commercial at \@     Left bracket  \[     Backslash     \\
%%   Right bracket \]     Circumflex    \^     Underscore    \_
%%   Grave accent  \`     Left brace    \{     Vertical bar  \|
%%   Right brace   \}     Tilde         \~}
%%
\endinput
}
\DeclareOption{newzealand}{%%
%% This file will generate fast loadable files and documentation
%% driver files from the doc files in this package when run through
%% LaTeX or TeX.
%%
%% Copyright 1989-2005 Johannes L. Braams and any individual authors
%% listed elsewhere in this file.  All rights reserved.
%% 
%% This file is part of the Babel system.
%% --------------------------------------
%% 
%% It may be distributed and/or modified under the
%% conditions of the LaTeX Project Public License, either version 1.3
%% of this license or (at your option) any later version.
%% The latest version of this license is in
%%   http://www.latex-project.org/lppl.txt
%% and version 1.3 or later is part of all distributions of LaTeX
%% version 2003/12/01 or later.
%% 
%% This work has the LPPL maintenance status "maintained".
%% 
%% The Current Maintainer of this work is Johannes Braams.
%% 
%% The list of all files belonging to the LaTeX base distribution is
%% given in the file `manifest.bbl. See also `legal.bbl' for additional
%% information.
%% 
%% The list of derived (unpacked) files belonging to the distribution
%% and covered by LPPL is defined by the unpacking scripts (with
%% extension .ins) which are part of the distribution.
%%
%% --------------- start of docstrip commands ------------------
%%
\def\filedate{1999/04/11}
\def\batchfile{english.ins}
\input docstrip.tex

{\ifx\generate\undefined
\Msg{**********************************************}
\Msg{*}
\Msg{* This installation requires docstrip}
\Msg{* version 2.3c or later.}
\Msg{*}
\Msg{* An older version of docstrip has been input}
\Msg{*}
\Msg{**********************************************}
\errhelp{Move or rename old docstrip.tex.}
\errmessage{Old docstrip in input path}
\batchmode
\csname @@end\endcsname
\fi}

\declarepreamble\mainpreamble
This is a generated file.

Copyright 1989-2005 Johannes L. Braams and any individual authors
listed elsewhere in this file.  All rights reserved.

This file was generated from file(s) of the Babel system.
---------------------------------------------------------

It may be distributed and/or modified under the
conditions of the LaTeX Project Public License, either version 1.3
of this license or (at your option) any later version.
The latest version of this license is in
  http://www.latex-project.org/lppl.txt
and version 1.3 or later is part of all distributions of LaTeX
version 2003/12/01 or later.

This work has the LPPL maintenance status "maintained".

The Current Maintainer of this work is Johannes Braams.

This file may only be distributed together with a copy of the Babel
system. You may however distribute the Babel system without
such generated files.

The list of all files belonging to the Babel distribution is
given in the file `manifest.bbl'. See also `legal.bbl for additional
information.

The list of derived (unpacked) files belonging to the distribution
and covered by LPPL is defined by the unpacking scripts (with
extension .ins) which are part of the distribution.
\endpreamble

\declarepreamble\fdpreamble
This is a generated file.

Copyright 1989-2005 Johannes L. Braams and any individual authors
listed elsewhere in this file.  All rights reserved.

This file was generated from file(s) of the Babel system.
---------------------------------------------------------

It may be distributed and/or modified under the
conditions of the LaTeX Project Public License, either version 1.3
of this license or (at your option) any later version.
The latest version of this license is in
  http://www.latex-project.org/lppl.txt
and version 1.3 or later is part of all distributions of LaTeX
version 2003/12/01 or later.

This work has the LPPL maintenance status "maintained".

The Current Maintainer of this work is Johannes Braams.

This file may only be distributed together with a copy of the Babel
system. You may however distribute the Babel system without
such generated files.

The list of all files belonging to the Babel distribution is
given in the file `manifest.bbl'. See also `legal.bbl for additional
information.

In particular, permission is granted to customize the declarations in
this file to serve the needs of your installation.

However, NO PERMISSION is granted to distribute a modified version
of this file under its original name.

\endpreamble

\keepsilent

\usedir{tex/generic/babel} 

\usepreamble\mainpreamble
\generate{\file{english.ldf}{\from{english.dtx}{code}}
          }
\usepreamble\fdpreamble

\ifToplevel{
\Msg{***********************************************************}
\Msg{*}
\Msg{* To finish the installation you have to move the following}
\Msg{* files into a directory searched by TeX:}
\Msg{*}
\Msg{* \space\space All *.def, *.fd, *.ldf, *.sty}
\Msg{*}
\Msg{* To produce the documentation run the files ending with}
\Msg{* '.dtx' and `.fdd' through LaTeX.}
\Msg{*}
\Msg{* Happy TeXing}
\Msg{***********************************************************}
}
 
\endinput
}
\DeclareOption{ngerman}{% \iffalse meta-comment
%
% Copyright 1989-2008 Johannes L. Braams and any individual authors
% listed elsewhere in this file.  All rights reserved.
% 
% This file is part of the Babel system.
% --------------------------------------
% 
% It may be distributed and/or modified under the
% conditions of the LaTeX Project Public License, either version 1.3
% of this license or (at your option) any later version.
% The latest version of this license is in
%   http://www.latex-project.org/lppl.txt
% and version 1.3 or later is part of all distributions of LaTeX
% version 2003/12/01 or later.
% 
% This work has the LPPL maintenance status "maintained".
% 
% The Current Maintainer of this work is Johannes Braams.
% 
% The list of all files belonging to the Babel system is
% given in the file `manifest.bbl. See also `legal.bbl' for additional
% information.
% 
% The list of derived (unpacked) files belonging to the distribution
% and covered by LPPL is defined by the unpacking scripts (with
% extension .ins) which are part of the distribution.
% \fi
% \CheckSum{266}
%
% \iffalse
%    Tell the \LaTeX\ system who we are and write an entry on the
%    transcript.
%<*dtx>
\ProvidesFile{ngermanb.dtx}
%</dtx>
%<code>\ProvidesLanguage{ngermanb}
%\fi
%\ProvidesFile{ngermanb.dtx}
        [2008/03/17 v2.6m new German support from the babel system]
%\iffalse
%% File `ngermanb.dtx'
%% Babel package for LaTeX version 2e
%% Copyright (C) 1989 - 2008
%%           by Johannes Braams, TeXniek
%
%% new Germanb Language Definition File
%% Copyright (C) 1989 - 2008
%%           by Bernd Raichle raichle at azu.Informatik.Uni-Stuttgart.de
%%              Johannes Braams, TeXniek,
%%              Walter Schmidt.
% This file is based on german.tex version 2.5e
%                       by Bernd Raichle, Hubert Partl et.al.
%
%% Please report errors to: J.L. Braams
%%                          babel at braams.xs4all.nl
%
%<*filedriver>
\documentclass{ltxdoc}
\font\manual=logo10 % font used for the METAFONT logo, etc.
\newcommand*\MF{{\manual META}\-{\manual FONT}}
\newcommand*\TeXhax{\TeX hax}
\newcommand*\babel{\textsf{babel}}
\newcommand*\langvar{$\langle \it lang \rangle$}
\newcommand*\note[1]{}
\newcommand*\Lopt[1]{\textsf{#1}}
\newcommand*\file[1]{\texttt{#1}}
\begin{document}
 \DocInput{ngermanb.dtx}
\end{document}
%</filedriver>
%\fi
% \GetFileInfo{ngermanb.dtx}
%
% \changes{ngermanb-2.6f}{1999/03/24}{Renamed from \file{germanb.ldf};
%          language names changed from \texttt{german} and \texttt{austrian}
%          to \texttt{ngerman} and \texttt{naustrian}.}
%
%  \section{The German language -- new orthography}
%
%    The file \file{\filename}\footnote{The file described in this
%    section has version number \fileversion\ and was last revised on
%    \filedate.}  defines all the language definition macros for the
%    German language with the `new orthography' introduced in
%    August 1998.  This includes also the Austrian dialect of this
%    language.
%  
%    As with the `traditional'  German orthography, 
%    the character |"| is made active, and 
%    the commands in  table~\ref{tab:german-quote} can be used, except
%    for |"ck| and |"ff| etc., which are no longer required.
%
%    The internal language names are |ngerman| and |naustrian|.
%
% \StopEventually{}
%
%    When this file was read through the option \Lopt{ngermanb} we make
%    it behave as if \Lopt{ngerman} was specified.
%    \begin{macrocode}
\def\bbl@tempa{ngermanb}
\ifx\CurrentOption\bbl@tempa
  \def\CurrentOption{ngerman}
\fi
%    \end{macrocode}
%
%    The macro |\LdfInit| takes care of preventing that this file is
%    loaded more than once, checking the category code of the
%    \texttt{@} sign, etc.
%    \begin{macrocode}
%<*code>
\LdfInit\CurrentOption{captions\CurrentOption}
%    \end{macrocode}
%
%    When this file is read as an option, i.e., by the |\usepackage|
%    command, \texttt{ngerman} will be an `unknown' language, so we
%    have to make it known.  So we check for the existence of
%    |\l@ngerman| to see whether we have to do something here.
%
%    \begin{macrocode}
\ifx\l@ngerman\@undefined
  \@nopatterns{ngerman}
  \adddialect\l@ngerman0
\fi
%    \end{macrocode}
%
%    For the Austrian version of these definitions we just add another
%    language. 
%    \begin{macrocode}
\adddialect\l@naustrian\l@ngerman
%    \end{macrocode}
%
%    The next step consists of defining commands to switch to (and
%    from) the German language.
%
%  \begin{macro}{\captionsngerman}
%  \begin{macro}{\captionsnaustrian}
%    Either the macro |\captionnsgerman| or the macro
%    |\captionsnaustrian| will define all strings used in the four
%    standard document classes provided with \LaTeX.
%
% \changes{ngermanb-2.6k}{2000/09/20}{Added \cs{glossaryname}}
%    \begin{macrocode}
\@namedef{captions\CurrentOption}{%
  \def\prefacename{Vorwort}%
  \def\refname{Literatur}%
  \def\abstractname{Zusammenfassung}%
  \def\bibname{Literaturverzeichnis}%
  \def\chaptername{Kapitel}%
  \def\appendixname{Anhang}%
  \def\contentsname{Inhaltsverzeichnis}%    % oder nur: Inhalt
  \def\listfigurename{Abbildungsverzeichnis}%
  \def\listtablename{Tabellenverzeichnis}%
  \def\indexname{Index}%
  \def\figurename{Abbildung}%
  \def\tablename{Tabelle}%                  % oder: Tafel
  \def\partname{Teil}%
  \def\enclname{Anlage(n)}%                 % oder: Beilage(n)
  \def\ccname{Verteiler}%                   % oder: Kopien an
  \def\headtoname{An}%
  \def\pagename{Seite}%
  \def\seename{siehe}%
  \def\alsoname{siehe auch}%
  \def\proofname{Beweis}%
  \def\glossaryname{Glossar}%
  }
%    \end{macrocode}
%  \end{macro}
%  \end{macro}
%
%  \begin{macro}{\datengerman}
%    The macro |\datengerman| redefines the command
%    |\today| to produce German dates.
%    \begin{macrocode}
\def\month@ngerman{\ifcase\month\or
  Januar\or Februar\or M\"arz\or April\or Mai\or Juni\or
  Juli\or August\or September\or Oktober\or November\or Dezember\fi}
\def\datengerman{\def\today{\number\day.~\month@ngerman
    \space\number\year}}
%    \end{macrocode}
%  \end{macro}
%
%  \begin{macro}{\dateanustrian}
%    The macro |\datenaustrian| redefines the command
%    |\today| to produce Austrian version of the German dates.
%    \begin{macrocode}
\def\datenaustrian{\def\today{\number\day.~\ifnum1=\month
  J\"anner\else \month@ngerman\fi \space\number\year}}
%    \end{macrocode}
%  \end{macro}
%
%  \begin{macro}{\extrasngerman}
%  \begin{macro}{\extrasnaustrian}
%  \begin{macro}{\noextrasngerman}
%  \begin{macro}{\noextrasnaustrian}
%    Either the macro |\extrasngerman| or the macros |\extrasnaustrian|
%    will perform all the extra definitions needed for the German
%    language. The macro |\noextrasngerman| is used to cancel the
%    actions of |\extrasngerman|. 
%
%    For German (as well as for Dutch) the \texttt{"} character is
%    made active. This is done once, later on its definition may vary.
%    \begin{macrocode}
\initiate@active@char{"}
\@namedef{extras\CurrentOption}{%
  \languageshorthands{ngerman}}
\expandafter\addto\csname extras\CurrentOption\endcsname{%
  \bbl@activate{"}}
%    \end{macrocode}
%    Don't forget to turn the shorthands off again.
% \changes{ngermanb-2.6j}{1999/12/16}{Deactivate shorthands ouside of
%    German}
%    \begin{macrocode}
\addto\noextrasngerman{\bbl@deactivate{"}}
%    \end{macrocode}
%
%
%    In order for \TeX\ to be able to hyphenate German words which
%    contain `\ss' (in the \texttt{OT1} position |^^Y|) we have to
%    give the character a nonzero |\lccode| (see Appendix H, the \TeX
%    book).
%    \begin{macrocode}
\expandafter\addto\csname extras\CurrentOption\endcsname{%
  \babel@savevariable{\lccode25}%
  \lccode25=25}
%    \end{macrocode}
%
%    The umlaut accent macro |\"| is changed to lower the umlaut dots.
%    The redefinition is done with the help of |\umlautlow|.
%    \begin{macrocode}
\expandafter\addto\csname extras\CurrentOption\endcsname{%
  \babel@save\"\umlautlow}
\@namedef{noextras\CurrentOption}{\umlauthigh}
%    \end{macrocode}
%    The current 
%    version of the `new' German hyphenation patterns (\file{dehyphn.tex}
%    is to be used with |\lefthyphenmin| and |\righthyphenmin| set to~2. 
% \changes{ngermanb-2.6k}{2000/09/22}{Now use \cs{providehyphenmins} to
%    provide a default value}
%    \begin{macrocode}
\providehyphenmins{\CurrentOption}{\tw@\tw@}
%    \end{macrocode}
%    For German texts we need to make sure that |\frenchspacing| is
%    turned on.
% \changes{ngermanb-2.6m}{2001/01/26}{Turn frenchspacing on, as in
%    \texttt{german.sty}}
%    \begin{macrocode}
\expandafter\addto\csname extras\CurrentOption\endcsname{%
  \bbl@frenchspacing}
\expandafter\addto\csname noextras\CurrentOption\endcsname{%
  \bbl@nonfrenchspacing}
%    \end{macrocode}
%  \end{macro}
%  \end{macro}
%  \end{macro}
%  \end{macro}
%
%    The code above is necessary because we need an extra active
%    character. This character is then used as indicated in
%    table~\ref{tab:german-quote}.
%
%    To be able to define the function of |"|, we first define a
%    couple of `support' macros.
%
%
%  \begin{macro}{\dq}
%    We save the original double quote character in |\dq| to keep
%    it available, the math accent |\"| can now be typed as |"|.
%    \begin{macrocode}
\begingroup \catcode`\"12
\def\x{\endgroup
  \def\@SS{\mathchar"7019 }
  \def\dq{"}}
\x
%    \end{macrocode}
%  \end{macro}
%
%    Now we can define the doublequote macros: the umlauts,
%    \begin{macrocode}
\declare@shorthand{ngerman}{"a}{\textormath{\"{a}\allowhyphens}{\ddot a}}
\declare@shorthand{ngerman}{"o}{\textormath{\"{o}\allowhyphens}{\ddot o}}
\declare@shorthand{ngerman}{"u}{\textormath{\"{u}\allowhyphens}{\ddot u}}
\declare@shorthand{ngerman}{"A}{\textormath{\"{A}\allowhyphens}{\ddot A}}
\declare@shorthand{ngerman}{"O}{\textormath{\"{O}\allowhyphens}{\ddot O}}
\declare@shorthand{ngerman}{"U}{\textormath{\"{U}\allowhyphens}{\ddot U}}
%    \end{macrocode}
%    tremas,
%    \begin{macrocode}
\declare@shorthand{ngerman}{"e}{\textormath{\"{e}}{\ddot e}}
\declare@shorthand{ngerman}{"E}{\textormath{\"{E}}{\ddot E}}
\declare@shorthand{ngerman}{"i}{\textormath{\"{\i}}%
                              {\ddot\imath}}
\declare@shorthand{ngerman}{"I}{\textormath{\"{I}}{\ddot I}}
%    \end{macrocode}
%    german es-zet (sharp s),
%    \begin{macrocode}
\declare@shorthand{ngerman}{"s}{\textormath{\ss}{\@SS{}}}
\declare@shorthand{ngerman}{"S}{\SS}
\declare@shorthand{ngerman}{"z}{\textormath{\ss}{\@SS{}}}
\declare@shorthand{ngerman}{"Z}{SZ}
%    \end{macrocode}
%    german and french quotes,
%    \begin{macrocode}
\declare@shorthand{ngerman}{"`}{\glqq}
\declare@shorthand{ngerman}{"'}{\grqq}
\declare@shorthand{ngerman}{"<}{\flqq}
\declare@shorthand{ngerman}{">}{\frqq}
%    \end{macrocode}
%    and some additional commands:
%    \begin{macrocode}
\declare@shorthand{ngerman}{"-}{\nobreak\-\bbl@allowhyphens}
\declare@shorthand{ngerman}{"|}{%
  \textormath{\penalty\@M\discretionary{-}{}{\kern.03em}%
              \allowhyphens}{}}
\declare@shorthand{ngerman}{""}{\hskip\z@skip}
\declare@shorthand{ngerman}{"~}{\textormath{\leavevmode\hbox{-}}{-}}
\declare@shorthand{ngerman}{"=}{\penalty\@M-\hskip\z@skip}
%    \end{macrocode}
%
%  \begin{macro}{\mdqon}
%  \begin{macro}{\mdqoff}
%    All that's left to do now is to  define a couple of commands
%    for reasons of compatibility with \file{german.sty}.
%    \begin{macrocode}
\def\mdqon{\shorthandon{"}}
\def\mdqoff{\shorthandoff{"}}
%    \end{macrocode}
%  \end{macro}
%  \end{macro}
%
%    The macro |\ldf@finish| takes care of looking for a
%    configuration file, setting the main language to be switched on
%    at |\begin{document}| and resetting the category code of
%    \texttt{@} to its original value.
%    \begin{macrocode}
\ldf@finish\CurrentOption
%</code>
%    \end{macrocode}
%
% \Finale
%%
%% \CharacterTable
%%  {Upper-case    \A\B\C\D\E\F\G\H\I\J\K\L\M\N\O\P\Q\R\S\T\U\V\W\X\Y\Z
%%   Lower-case    \a\b\c\d\e\f\g\h\i\j\k\l\m\n\o\p\q\r\s\t\u\v\w\x\y\z
%%   Digits        \0\1\2\3\4\5\6\7\8\9
%%   Exclamation   \!     Double quote  \"     Hash (number) \#
%%   Dollar        \$     Percent       \%     Ampersand     \&
%%   Acute accent  \'     Left paren    \(     Right paren   \)
%%   Asterisk      \*     Plus          \+     Comma         \,
%%   Minus         \-     Point         \.     Solidus       \/
%%   Colon         \:     Semicolon     \;     Less than     \<
%%   Equals        \=     Greater than  \>     Question mark \?
%%   Commercial at \@     Left bracket  \[     Backslash     \\
%%   Right bracket \]     Circumflex    \^     Underscore    \_
%%   Grave accent  \`     Left brace    \{     Vertical bar  \|
%%   Right brace   \}     Tilde         \~}
%%
\endinput
}
\DeclareOption{norsk}{% \iffalse meta-comment
%
% Copyright 1989-2005 Johannes L. Braams and any individual authors
% listed elsewhere in this file.  All rights reserved.
% 
% This file is part of the Babel system.
% --------------------------------------
% 
% It may be distributed and/or modified under the
% conditions of the LaTeX Project Public License, either version 1.3
% of this license or (at your option) any later version.
% The latest version of this license is in
%   http://www.latex-project.org/lppl.txt
% and version 1.3 or later is part of all distributions of LaTeX
% version 2003/12/01 or later.
% 
% This work has the LPPL maintenance status "maintained".
% 
% The Current Maintainer of this work is Johannes Braams.
% 
% The list of all files belonging to the Babel system is
% given in the file `manifest.bbl. See also `legal.bbl' for additional
% information.
% 
% The list of derived (unpacked) files belonging to the distribution
% and covered by LPPL is defined by the unpacking scripts (with
% extension .ins) which are part of the distribution.
% \fi
%\CheckSum{305}
% \iffalse
%    Tell the \LaTeX\ system who we are and write an entry on the
%    transcript.
%<*dtx>
\ProvidesFile{norsk.dtx}
%</dtx>
%<code>\ProvidesLanguage{norsk}
%\fi
%\ProvidesFile{norsk.dtx}
        [2012/08/06 v2.0i Norsk support from the babel system]
%\iffalse
%%File `norsk.dtx'
%% Babel package for LaTeX version 2e
%% Copyright (C) 1989 - 2005
%%           by Johannes Braams, TeXniek
%
%% Please report errors to: J.L. Braams
%%                          babel at braams.cistron.nl
%
%    This file is part of the babel system, it provides the source
%    code for the Norwegian language definition file.  Contributions
%    were made by Haavard Helstrup (HAAVARD@CERNVM) and Alv Kjetil
%    Holme (HOLMEA@CERNVM); the `nynorsk' variant has been supplied by
%    Per Steinar Iversen (iversen@vxcern.cern.ch) and Terje Engeset
%    Petterst (TERJEEP@VSFYS1.FI.UIB.NO)
%
%    Rune Kleveland (runekl at math.uio.no) added the shorthand
%    definitions 
%<*filedriver>
\documentclass{ltxdoc}
\newcommand*\TeXhax{\TeX hax}
\newcommand*\babel{\textsf{babel}}
\newcommand*\langvar{$\langle \it lang \rangle$}
\newcommand*\note[1]{}
\newcommand*\Lopt[1]{\textsf{#1}}
\newcommand*\file[1]{\texttt{#1}}
\begin{document}
 \DocInput{norsk.dtx}
\end{document}
%</filedriver>
%\fi
% \GetFileInfo{norsk.dtx}
%
% \changes{norsk-1.0a}{1991/07/15}{Renamed \file{babel.sty} in
%    \file{babel.com}}
% \changes{norsk-1.1a}{1992/02/16}{Brought up-to-date with babel 3.2a}
% \changes{norsk-1.1c}{1993/11/11}{Added a couple of translations
%    (from Per Norman Oma, TeX@itk.unit.no)}
% \changes{norsk-1.2a}{1994/02/27}{Update for \LaTeXe}
% \changes{norsk-1.2d}{1994/06/26}{Removed the use of \cs{filedate}
%    and moved identification after the loading of \file{babel.def}}
% \changes{norsk-1.2h}{1996/07/12}{Replaced \cs{undefined} with
%    \cs{@undefined} and \cs{empty} with \cs{@empty} for consistency
%    with \LaTeX} 
% \changes{norsk-1.2h}{1996/10/10}{Moved the definition of
%    \cs{atcatcode} right to the beginning.}
%
%
%  \section{The Norwegian language}
%
%    The file \file{\filename}\footnote{The file described in this
%    section has version number \fileversion\ and was last revised on
%    \filedate.  Contributions were made by Haavard Helstrup
%    (\texttt{HAAVARD@CERNVM)} and Alv Kjetil Holme
%    (\texttt{HOLMEA@CERNVM}); the `nynorsk' variant has been supplied
%    by Per Steinar Iversen \texttt{iversen@vxcern.cern.ch}) and Terje
%    Engeset Petterst (\texttt{TERJEEP@VSFYS1.FI.UIB.NO)}; the
%    shorthand definitions were provided by Rune Kleveland
%    (\texttt{runekl@math.uio.no}).} defines all the language definition
%    macros for the Norwegian language as well as for an alternative
%    variant `nynorsk' of this language. 
%
%    For this language the character |"| is made active. In
%    table~\ref{tab:norsk-quote} an overview is given of its purpose.
%    \begin{table}[htb]
%     \begin{center}
%     \begin{tabular}{lp{.7\textwidth}}
%      |"ff|& for |ff| to be hyphenated as |ff-f|,
%             this is also implemented for b, d, f, g, l, m, n,
%             p, r, s, and t. (|o"ppussing|)                        \\
%      |"ee|& Hyphenate |"ee| as |\'e-e|. (|komit"een|)             \\
%      |"-| & an explicit hyphen sign, allowing hyphenation in the
%             composing words. Use this for compound words when the
%             hyphenation patterns fail to hyphenate
%             properly. (|alpin"-anlegg|)                           \\
%      \verb="|= & Like |"-|, but inserts 0.03em space.  Use it if
%             the compound point is spanned by a ligature.
%             (\verb=hoff"|intriger=)                               \\
%      |""| & Like |"-|, but producing no hyphen sign.
%             (|i""g\aa{}r|)                                        \\
%      |"~| & Like |-|, but allows no hyphenation at all. (|E"~cup|)\\
%      |"=| & Like |-|, but allowing hyphenation in the composing
%             words. (|marksistisk"=leninistisk|)                   \\
%      |"<| & for French left double quotes (similar to $<<$).      \\
%      |">| & for French right double quotes (similar to $>>$).     \\
%     \end{tabular}
%     \caption{The extra definitions made
%              by \file{norsk.sty}}\label{tab:norsk-quote}
%     \end{center}
%    \end{table}
% \changes{norsk-2.0a}{1998/06/24}{Describe the use of double quote as
%    active character}
%
%    Rune Kleveland distributes a Norwegian dictionary for ispell
%    (570000 words). It can be found at
%    |http://www.uio.no/~runekl/dictionary.html|. 
%
%    This dictionary supports the spellings |spi"sslede| for
%    `spisslede' (hyphenated spiss-slede) and other such words, and
%    also suggest the spelling |spi"sslede| for `spisslede' and
%    `spissslede'.
%
% \StopEventually{}
%
%    The macro |\LdfInit| takes care of preventing that this file is
%    loaded more than once, checking the category code of the
%    \texttt{@} sign, etc.
% \changes{norsk-1.2h}{1996/11/03}{Now use \cs{LdfInit} to perform
%    initial checks} 
%    \begin{macrocode}
%<*code>
\LdfInit\CurrentOption{captions\CurrentOption}
%    \end{macrocode}
%
%    When this file is read as an option, i.e. by the |\usepackage|
%    command, \texttt{norsk} will be an `unknown' language in which
%    case we have to make it known.  So we check for the existence of
%    |\l@norsk| to see whether we have to do something here.
%
% \changes{norsk-1.0c}{1991/10/29}{Removed use of \cs{@ifundefined}}
% \changes{norsk-1.1a}{1992/02/16}{Added a warning when no hyphenation
%    patterns were loaded.}
% \changes{norsk-1.2d}{1994/06/26}{Now use \cs{@nopatterns} to produce
%    the warning}
%    \begin{macrocode}
\ifx\l@norsk\@undefined
    \@nopatterns{Norsk}
    \adddialect\l@norsk0\fi
%    \end{macrocode}
%
%  \begin{macro}{\norskhyphenmins}
%     Some sets of Norwegian hyphenation patterns can be used with
%     |\lefthyphenmin| set to~1 and |\righthyphenmin| set to~2, but
%     the most common set |nohyph.tex| can't.  So we use
%     |\lefthyphenmin=2| by default.
% \changes{norsk-1.2f}{1995/07/02}{Added setting of hyphenmin
%    parameters}
% \changes{norsk-2.0a}{1998/06/24}{Changed setting of hyphenmin
%    parameters to 2~2} 
% \changes{norsk-2.0e}{2000/09/22}{Now use \cs{providehyphenmins} to
%    provide a default value}
%    \begin{macrocode}
\providehyphenmins{\CurrentOption}{\tw@\tw@}
%    \end{macrocode}
%  \end{macro}
%
%    Now we have to decide which version of the captions should be
%    made available. This can be done by checking the contents of
%    |\CurrentOption|. 
%    \begin{macrocode}
\def\bbl@tempa{norsk}
\ifx\CurrentOption\bbl@tempa
%    \end{macrocode}
%
%    The next step consists of defining commands to switch to (and
%    from) the Norwegian language.
%
% \begin{macro}{\captionsnorsk}
%    The macro |\captionsnorsk| defines all strings used
%    in the four standard documentclasses provided with \LaTeX.
% \changes{norsk-1.1a}{1992/02/16}{Added \cs{seename}, \cs{alsoname} and
%    \cs{prefacename}}
% \changes{norsk-1.1b}{1993/07/15}{\cs{headpagename} should be
%    \cs{pagename}}
% \changes{norsk-1.2f}{1995/07/02}{Added \cs{proofname} for
%    AMS-\LaTeX}
% \changes{norsk-1.2g}{1996/04/01}{Replaced `Proof' by its
%    translation} 
% \changes{norsk-2.0e}{2000/09/20}{Added \cs{glossaryname}}
% \changes{norsk-2.0g}{1996/04/01}{Replaced `Glossary' by its
%    translation} 
%    \begin{macrocode}
  \def\captionsnorsk{%
    \def\prefacename{Forord}%
    \def\refname{Referanser}%
    \def\abstractname{Sammendrag}%
    \def\bibname{Bibliografi}%     or Litteraturoversikt
    %                              or Litteratur or Referanser
    \def\chaptername{Kapittel}%
    \def\appendixname{Tillegg}%    or Appendiks
    \def\contentsname{Innhold}%
    \def\listfigurename{Figurer}%  or Figurliste
    \def\listtablename{Tabeller}%  or Tabelliste
    \def\indexname{Register}%
    \def\figurename{Figur}%
    \def\tablename{Tabell}%
    \def\partname{Del}%
    \def\enclname{Vedlegg}%
    \def\ccname{Kopi sendt}%
    \def\headtoname{Til}% in letter
    \def\pagename{Side}%
    \def\seename{Se}%
    \def\alsoname{Se ogs\aa{}}%
    \def\proofname{Bevis}%
    \def\glossaryname{Ordliste}%
    }
\else
%    \end{macrocode}
% \end{macro}
%
%    For the `nynorsk' version of these definitions we just add a
%    ``dialect''.
%    \begin{macrocode}
  \adddialect\l@nynorsk\l@norsk
%    \end{macrocode}
%
% \begin{macro}{\captionsnynorsk}
%    The macro |\captionsnynorsk| defines all strings used in the four
%    standard documentclasses provided with \LaTeX, but using a
%    different spelling than in the command |\captionsnorsk|.
% \changes{norsk-1.1a}{1992/02/16}{Added \cs{seename}, \cs{alsoname} and
%    \cs{prefacename}}
% \changes{norsk-1.1b}{1993/07/15}{\cs{headpagename} should be
%    \cs{pagename}}
% \changes{norsk-1.2g}{1996/04/01}{Replaced `Proof' by its
%    translation} 
% \changes{norsk-2.0e}{2000/09/20}{Added \cs{glossaryname}}
% \changes{norsk-2.0g}{1996/04/01}{Replaced `Glossary' by its
%    translation} 
% \changes{norks-2.0h}{2001/01/12}{Changed \cs{ccname} and \cs{alsoname}}
%    \begin{macrocode}
  \def\captionsnynorsk{%
    \def\prefacename{Forord}%
    \def\refname{Referansar}%
    \def\abstractname{Samandrag}%
    \def\bibname{Litteratur}%     or Litteraturoversyn
     %                             or Referansar
    \def\chaptername{Kapittel}%
    \def\appendixname{Tillegg}%   or Appendiks
    \def\contentsname{Innhald}%
    \def\listfigurename{Figurar}% or Figurliste
    \def\listtablename{Tabellar}% or Tabelliste
    \def\indexname{Register}%
    \def\figurename{Figur}%
    \def\tablename{Tabell}%
    \def\partname{Del}%
    \def\enclname{Vedlegg}%
    \def\ccname{Kopi til}%
    \def\headtoname{Til}% in letter
    \def\pagename{Side}%
    \def\seename{Sj\aa{}}%
    \def\alsoname{Sj\aa{} \`{o}g}%
    \def\proofname{Bevis}%
    \def\glossaryname{Ordliste}%
    }
\fi
%    \end{macrocode}
% \end{macro}
%
% \begin{macro}{\datenorsk}
%    The macro |\datenorsk| redefines the command |\today| to produce
%    Norwegian dates.
% \changes{norsk-1.2i}{1997/10/01}{Use \cs{edef} to define
%    \cs{today} to save memory}
% \changes{norsk-1.2i}{1998/03/28}{use \cs{def} instead of \cs{edef}}
% \changes{norsk-2.0i}{2012/08/06}{Removed extra space after `desember'}
%    \begin{macrocode}
\@namedef{date\CurrentOption}{%
  \def\today{\number\day.~\ifcase\month\or
    januar\or februar\or mars\or april\or mai\or juni\or
    juli\or august\or september\or oktober\or november\or
    desember\fi
    \space\number\year}}
%    \end{macrocode}
% \end{macro}
%
% \begin{macro}{\extrasnorsk}
% \begin{macro}{\extrasnynorsk}
%    The macro |\extrasnorsk| will perform all the extra definitions
%    needed for the Norwegian language. The macro |\noextrasnorsk| is
%    used to cancel the actions of |\extrasnorsk|.  
%
%    Norwegian typesetting requires |\frencspacing| to be in effect.
%    \begin{macrocode}
\@namedef{extras\CurrentOption}{\bbl@frenchspacing}
\@namedef{noextras\CurrentOption}{\bbl@nonfrenchspacing}
%    \end{macrocode}
% \end{macro}
% \end{macro}
%
%    For Norsk the \texttt{"} character is made active. This is done
%    once, later on its definition may vary.
% \changes{norsk-2.0a}{1998/06/24}{Made double quote character active}
%    \begin{macrocode}
\initiate@active@char{"}
\expandafter\addto\csname extras\CurrentOption\endcsname{%
  \languageshorthands{norsk}}
\expandafter\addto\csname extras\CurrentOption\endcsname{%
  \bbl@activate{"}}
%    \end{macrocode}
%    Don't forget to turn the shorthands off again.
% \changes{norsk-2.0c}{1999/12/17}{Deactivate shorthands ouside of
%    Norsk}
%    \begin{macrocode}
\expandafter\addto\csname noextras\CurrentOption\endcsname{%
  \bbl@deactivate{"}}
%    \end{macrocode}
%
%    The code above is necessary because we need to define a number of
%    shorthand commands. These sharthand commands are then used as
%    indicated in table~\ref{tab:norsk-quote}.
%
%    To be able to define the function of |"|, we first define a
%    couple of `support' macros.
%
%  \begin{macro}{\dq}
%    We save the original double quote character in |\dq| to keep
%    it available, the math accent |\"| can now be typed as |"|.
%    \begin{macrocode}
\begingroup \catcode`\"12
\def\x{\endgroup
  \def\@SS{\mathchar"7019 }
  \def\dq{"}}
\x
%    \end{macrocode}
%  \end{macro}
%
%    Now we can define the discretionary shorthand commands.
%    The number of words where such hyphenation is required is for
%    each character
%    \begin{center}
%      \begin{tabular}{*{11}c}
%        b&d&f &g&k &l &n&p &r&s &t \\
%        4&4&15&3&43&30&8&12&1&33&35
%       \end{tabular}
%    \end{center}
%    taken from a list of 83000 ispell-roots.
%
% \changes{norsk-2.0d}{2000/02/29}{Shorthands are the same for both
%    spelling variants, no need to use \cs{CurrentOption}}
%    \begin{macrocode}
\declare@shorthand{norsk}{"b}{\textormath{\bbl@disc b{bb}}{b}}
\declare@shorthand{norsk}{"B}{\textormath{\bbl@disc B{BB}}{B}}
\declare@shorthand{norsk}{"d}{\textormath{\bbl@disc d{dd}}{d}}
\declare@shorthand{norsk}{"D}{\textormath{\bbl@disc D{DD}}{D}}
\declare@shorthand{norsk}{"e}{\textormath{\bbl@disc e{\'e}}{}}
\declare@shorthand{norsk}{"E}{\textormath{\bbl@disc E{\'E}}{}}
\declare@shorthand{norsk}{"F}{\textormath{\bbl@disc F{FF}}{F}}
\declare@shorthand{norsk}{"g}{\textormath{\bbl@disc g{gg}}{g}}
\declare@shorthand{norsk}{"G}{\textormath{\bbl@disc G{GG}}{G}}
\declare@shorthand{norsk}{"k}{\textormath{\bbl@disc k{kk}}{k}}
\declare@shorthand{norsk}{"K}{\textormath{\bbl@disc K{KK}}{K}}
\declare@shorthand{norsk}{"l}{\textormath{\bbl@disc l{ll}}{l}}
\declare@shorthand{norsk}{"L}{\textormath{\bbl@disc L{LL}}{L}}
\declare@shorthand{norsk}{"n}{\textormath{\bbl@disc n{nn}}{n}}
\declare@shorthand{norsk}{"N}{\textormath{\bbl@disc N{NN}}{N}}
\declare@shorthand{norsk}{"p}{\textormath{\bbl@disc p{pp}}{p}}
\declare@shorthand{norsk}{"P}{\textormath{\bbl@disc P{PP}}{P}}
\declare@shorthand{norsk}{"r}{\textormath{\bbl@disc r{rr}}{r}}
\declare@shorthand{norsk}{"R}{\textormath{\bbl@disc R{RR}}{R}}
\declare@shorthand{norsk}{"s}{\textormath{\bbl@disc s{ss}}{s}}
\declare@shorthand{norsk}{"S}{\textormath{\bbl@disc S{SS}}{S}}
\declare@shorthand{norsk}{"t}{\textormath{\bbl@disc t{tt}}{t}}
\declare@shorthand{norsk}{"T}{\textormath{\bbl@disc T{TT}}{T}}
%    \end{macrocode}
%    We need to treat |"f| a bit differently in order to preserve the
%    ff-ligature. 
% \changes{norsk-2.0b}{1999/11/19}{Copied the coding for \texttt{"f}
%    from germanb.dtx version 2.6g} 
%    \begin{macrocode}
\declare@shorthand{norsk}{"f}{\textormath{\bbl@discff}{f}}
\def\bbl@discff{\penalty\@M
  \afterassignment\bbl@insertff \let\bbl@nextff= }
\def\bbl@insertff{%
  \if f\bbl@nextff
    \expandafter\@firstoftwo\else\expandafter\@secondoftwo\fi
  {\relax\discretionary{ff-}{f}{ff}\allowhyphens}{f\bbl@nextff}}
\let\bbl@nextff=f
%    \end{macrocode}
%    We now  define the French double quotes and some commands 
%    concerning hyphenation:
% \changes{norsk-2.0b}{1999/11/22}{added the french double quotes}
% \changes{norsk-2.0d}{2000/01/28}{Use \cs{bbl@allowhyphens} in
%    \texttt{"-}}
%    \begin{macrocode}
\declare@shorthand{norsk}{"<}{\flqq}
\declare@shorthand{norsk}{">}{\frqq}
\declare@shorthand{norsk}{"-}{\penalty\@M\-\bbl@allowhyphens}
\declare@shorthand{norsk}{"|}{%
  \textormath{\penalty\@M\discretionary{-}{}{\kern.03em}%
              \allowhyphens}{}}
\declare@shorthand{norsk}{""}{\hskip\z@skip}
\declare@shorthand{norsk}{"~}{\textormath{\leavevmode\hbox{-}}{-}}
\declare@shorthand{norsk}{"=}{\penalty\@M-\hskip\z@skip}
%    \end{macrocode}
%
%    The macro |\ldf@finish| takes care of looking for a
%    configuration file, setting the main language to be switched on
%    at |\begin{document}| and resetting the category code of
%    \texttt{@} to its original value.
% \changes{norsk-1.2h}{1996/11/03}{Now use \cs{ldf@finish} to wrap up}
%    \begin{macrocode}
\ldf@finish\CurrentOption
%</code>
%    \end{macrocode}
%
% \Finale
%%
%% \CharacterTable
%%  {Upper-case    \A\B\C\D\E\F\G\H\I\J\K\L\M\N\O\P\Q\R\S\T\U\V\W\X\Y\Z
%%   Lower-case    \a\b\c\d\e\f\g\h\i\j\k\l\m\n\o\p\q\r\s\t\u\v\w\x\y\z
%%   Digits        \0\1\2\3\4\5\6\7\8\9
%%   Exclamation   \!     Double quote  \"     Hash (number) \#
%%   Dollar        \$     Percent       \%     Ampersand     \&
%%   Acute accent  \'     Left paren    \(     Right paren   \)
%%   Asterisk      \*     Plus          \+     Comma         \,
%%   Minus         \-     Point         \.     Solidus       \/
%%   Colon         \:     Semicolon     \;     Less than     \<
%%   Equals        \=     Greater than  \>     Question mark \?
%%   Commercial at \@     Left bracket  \[     Backslash     \\
%%   Right bracket \]     Circumflex    \^     Underscore    \_
%%   Grave accent  \`     Left brace    \{     Vertical bar  \|
%%   Right brace   \}     Tilde         \~}
%%
\endinput
}
\DeclareOption{samin}{% \iffalse meta-comment
%
% Copyright 1989-2005 Johannes L. Braams and any individual authors
% listed elsewhere in this file.  All rights reserved.
% 
% This file is part of the Babel system.
% --------------------------------------
% 
% It may be distributed and/or modified under the
% conditions of the LaTeX Project Public License, either version 1.3
% of this license or (at your option) any later version.
% The latest version of this license is in
%   http://www.latex-project.org/lppl.txt
% and version 1.3 or later is part of all distributions of LaTeX
% version 2003/12/01 or later.
% 
% This work has the LPPL maintenance status "maintained".
% 
% The Current Maintainer of this work is Johannes Braams.
% 
% The list of all files belonging to the Babel system is
% given in the file `manifest.bbl. See also `legal.bbl' for additional
% information.
% 
% The list of derived (unpacked) files belonging to the distribution
% and covered by LPPL is defined by the unpacking scripts (with
% extension .ins) which are part of the distribution.
% \fi
% \CheckSum{120}
%
% \iffalse
%    Tell the \LaTeX\ system who we are and write an entry on the
%    transcript.
%<*dtx>
\ProvidesFile{samin.dtx}
%</dtx>
%<code>\ProvidesLanguage{samin}
%\fi
%\ProvidesFile{samin.dtx}
        [2004/02/20 v1.0c North Sami support from the babel system]
%\iffalse
%% Babel package for LaTeX version 2e
%% Copyright (C) 1989 -- 2004
%%           by Johannes Braams, TeXniek
%
%% Please report errors to: J.L. Braams
%%                          babel at braams.cistron.nl
%
%    This file is part of the babel system, it provides the source code for
%    the North Sami language definition file.
%<*filedriver>
\documentclass{ltxdoc}
\newcommand*{\TeXhax}{\TeX hax}
\newcommand*{\babel}{\textsf{babel}}
\newcommand*{\langvar}{$\langle \mathit lang \rangle$}
\newcommand*{\note}[1]{}
\newcommand*{\Lopt}[1]{\textsf{#1}}
\newcommand*{\file}[1]{\texttt{#1}}
\begin{document}
 \DocInput{samin.dtx}
\end{document}
%</filedriver>
%\fi
% \GetFileInfo{samin.dtx}
%
%  \section{The North Sami language}
%
%    The file \file{\filename}\footnote{The file described in this
%    section has version number \fileversion\ and was last revised on
%    \filedate.  It was written by Regnor Jernsletten,
%    (\texttt{Regnor.Jernsletten@sami.uit.no}) or
%    (\texttt{Regnor.Jernsletten@eunet.no}).}
%    defines all the language definition macros for the North Sami language.
%
%    Several Sami dialects/languages are spoken in Finland, Norway, Sweden
%    and on the Kola Peninsula (Russia).  The alphabets differ, so there
%    will eventually be a need for more \texttt{.dtx} files for e.g. Lule
%    and South Sami.  Hence the name \file{samin.dtx} (and not
%    \file{sami.dtx} or the like) in the North Sami case.
%
%    There are currently no hyphenation patterns available for the North
%    Sami language, but you might consider using the patterns for Finnish
%    (\file{fi8hyph.tex}), Norwegian (\file{nohyph.tex}) or Swedish
%    (\file{sehyph.tex}).  Add a line for the \texttt{samin}  language to
%    the \file{language.dat} file, and rebuild the \LaTeX\ format file.
%    See the documentation for your \LaTeX\ distribution.
%    
%    A note on writing North Sami in \LaTeX:  The TI encoding and EC fonts
%    do not include the T WITH STROKE letter, which you will need a
%    workaround for.  My suggestion is to place the lines\\
%    |\newcommand{\tx}{\mbox{t\hspace{-.35em}-}}|\\
%    |\newcommand{\txx}{\mbox{T\hspace{-.5em}-}}|\\
%    in the preamble of your documents.  They define the commands\\
%    |\txx{}| for LATIN CAPITAL LETTER T WITH STROKE and \\
%    |\tx{}|  for LATIN SMALL   LETTER T WITH STROKE.
%
%  \subsection{The code of \file{samin.dtx}}
%
% \StopEventually{}
%
%    The macro |\LdfInit| takes care of preventing that this file is
%    loaded more than once, checking the category code of the
%    \texttt{@} sign, etc.
%    \begin{macrocode}
%<*code>
\LdfInit{samin}{captionssamin}
%    \end{macrocode}
%
%    When this file is read as an option, i.e. by the |\usepackage|
%    command, \texttt{samin} could be an `unknown' language in
%    which case we have to make it known.  So we check for the
%    existence of |\l@samin| to see whether we have to do
%    something here.
%
%    \begin{macrocode}
\ifx\undefined\l@samin
  \@nopatterns{Samin}
  \adddialect\l@samin0\fi
%    \end{macrocode}
%
%    The next step consists of defining commands to switch to (and
%    from) the North Sami language.
%
%  \begin{macro}{\saminhyphenmins}
%    This macro is used to store the correct values of the hyphenation
%    parameters |\lefthyphenmin| and |\righthyphenmin|.
% \changes{samin-1.0b}{2000/09/26}{use \cs{providehyphenmins}}
%    \begin{macrocode}
\providehyphenmins{samin}{\tw@\tw@}
%    \end{macrocode}
%  \end{macro}
%
% \begin{macro}{\captionssamin}
%    The macro |\captionssamin| defines all strings used in the
%    four standard documentclasses provided with \LaTeX.
% \changes{samin-1.0b}{2000/09/26}{Added \cs{glossaryname}}
% \changes{samin-1.0c}{2000/10/04}{Provided the translation for
%    glossary}
%    \begin{macrocode}
\def\captionssamin{%
  \def\prefacename{Ovdas\'atni}%
  \def\refname{\v Cujuhusat}%
  \def\abstractname{\v Coahkk\'aigeassu}%
  \def\bibname{Girjj\'ala\v svuohta}%
  \def\chaptername{Kapihttal}%
  \def\appendixname{\v Cuovus}%
  \def\contentsname{Sisdoallu}%
  \def\listfigurename{Govvosat}%
  \def\listtablename{Tabeallat}%
  \def\indexname{Registtar}%
  \def\figurename{Govus}%
  \def\tablename{Tabealla}%
  \def\partname{Oassi}%
  \def\enclname{Mielddus}%
  \def\ccname{Kopia s\'addejuvvon}%
  \def\headtoname{Vuost\'aiv\'aldi}%
  \def\pagename{Siidu}%
  \def\seename{geah\v ca}%
  \def\alsoname{geah\v ca maidd\'ai}%
  \def\proofname{Duo\dj{}a\v stus}%
  \def\glossaryname{S\'atnelistu}%
}%
%    \end{macrocode}
% \end{macro}
%
% \begin{macro}{\datesamin}
%    The macro |\datesamin| redefines the command |\today| to
%    produce North Sami dates.
%    \begin{macrocode}
\def\datesamin{%
  \def\today{\ifcase\month\or
    o\dj{}\dj{}ajagem\'anu\or
    guovvam\'anu\or
    njuk\v cam\'anu\or
    cuo\ng{}om\'anu\or
    miessem\'anu\or
    geassem\'anu\or
    suoidnem\'anu\or
    borgem\'anu\or
    \v cak\v cam\'anu\or
    golggotm\'anu\or
    sk\'abmam\'anu\or
    juovlam\'anu\fi
    \space\number\day.~b.\space\number\year}%
}%
%    \end{macrocode}
% \end{macro}
%
%
% \begin{macro}{\extrassamin}
% \begin{macro}{\noextrassamin}
%    The macro |\extrassamin| will perform all the extra
%    definitions needed for the North Sami language. The macro
%    |\noextrassamin| is used to cancel the actions of
%    |\extrassamin|.  For the moment these macros are empty but
%    they are defined for compatibility with the other
%    language definition files.
%
%    \begin{macrocode}
\addto\extrassamin{}
\addto\noextrassamin{}
%    \end{macrocode}
% \end{macro}
% \end{macro}
%
%    The macro |\ldf@finish| takes care of looking for a
%    configuration file, setting the main language to be switched on
%    at |\begin{document}| and resetting the category code of
%    \texttt{@} to its original value.
%    \begin{macrocode}
\ldf@finish{samin}
%</code>
%    \end{macrocode}
%
% \Finale
%\endinput
%% \CharacterTable
%%  {Upper-case    \A\B\C\D\E\F\G\H\I\J\K\L\M\N\O\P\Q\R\S\T\U\V\W\X\Y\Z
%%   Lower-case    \a\b\c\d\e\f\g\h\i\j\k\l\m\n\o\p\q\r\s\t\u\v\w\x\y\z
%%   Digits        \0\1\2\3\4\5\6\7\8\9
%%   Exclamation   \!     Double quote  \"     Hash (number) \#
%%   Dollar        \$     Percent       \%     Ampersand     \&
%%   Acute accent  \'     Left paren    \(     Right paren   \)
%%   Asterisk      \*     Plus          \+     Comma         \,
%%   Minus         \-     Point         \.     Solidus       \/
%%   Colon         \:     Semicolon     \;     Less than     \<
%%   Equals        \=     Greater than  \>     Question mark \?
%%   Commercial at \@     Left bracket  \[     Backslash     \\
%%   Right bracket \]     Circumflex    \^     Underscore    \_
%%   Grave accent  \`     Left brace    \{     Vertical bar  \|
%%   Right brace   \}     Tilde         \~}
%%
}
%    \end{macrocode}
%    For Norwegian two spelling variants are provided.
%    \begin{macrocode}
\DeclareOption{nynorsk}{% \iffalse meta-comment
%
% Copyright 1989-2005 Johannes L. Braams and any individual authors
% listed elsewhere in this file.  All rights reserved.
% 
% This file is part of the Babel system.
% --------------------------------------
% 
% It may be distributed and/or modified under the
% conditions of the LaTeX Project Public License, either version 1.3
% of this license or (at your option) any later version.
% The latest version of this license is in
%   http://www.latex-project.org/lppl.txt
% and version 1.3 or later is part of all distributions of LaTeX
% version 2003/12/01 or later.
% 
% This work has the LPPL maintenance status "maintained".
% 
% The Current Maintainer of this work is Johannes Braams.
% 
% The list of all files belonging to the Babel system is
% given in the file `manifest.bbl. See also `legal.bbl' for additional
% information.
% 
% The list of derived (unpacked) files belonging to the distribution
% and covered by LPPL is defined by the unpacking scripts (with
% extension .ins) which are part of the distribution.
% \fi
%\CheckSum{305}
% \iffalse
%    Tell the \LaTeX\ system who we are and write an entry on the
%    transcript.
%<*dtx>
\ProvidesFile{norsk.dtx}
%</dtx>
%<code>\ProvidesLanguage{norsk}
%\fi
%\ProvidesFile{norsk.dtx}
        [2012/08/06 v2.0i Norsk support from the babel system]
%\iffalse
%%File `norsk.dtx'
%% Babel package for LaTeX version 2e
%% Copyright (C) 1989 - 2005
%%           by Johannes Braams, TeXniek
%
%% Please report errors to: J.L. Braams
%%                          babel at braams.cistron.nl
%
%    This file is part of the babel system, it provides the source
%    code for the Norwegian language definition file.  Contributions
%    were made by Haavard Helstrup (HAAVARD@CERNVM) and Alv Kjetil
%    Holme (HOLMEA@CERNVM); the `nynorsk' variant has been supplied by
%    Per Steinar Iversen (iversen@vxcern.cern.ch) and Terje Engeset
%    Petterst (TERJEEP@VSFYS1.FI.UIB.NO)
%
%    Rune Kleveland (runekl at math.uio.no) added the shorthand
%    definitions 
%<*filedriver>
\documentclass{ltxdoc}
\newcommand*\TeXhax{\TeX hax}
\newcommand*\babel{\textsf{babel}}
\newcommand*\langvar{$\langle \it lang \rangle$}
\newcommand*\note[1]{}
\newcommand*\Lopt[1]{\textsf{#1}}
\newcommand*\file[1]{\texttt{#1}}
\begin{document}
 \DocInput{norsk.dtx}
\end{document}
%</filedriver>
%\fi
% \GetFileInfo{norsk.dtx}
%
% \changes{norsk-1.0a}{1991/07/15}{Renamed \file{babel.sty} in
%    \file{babel.com}}
% \changes{norsk-1.1a}{1992/02/16}{Brought up-to-date with babel 3.2a}
% \changes{norsk-1.1c}{1993/11/11}{Added a couple of translations
%    (from Per Norman Oma, TeX@itk.unit.no)}
% \changes{norsk-1.2a}{1994/02/27}{Update for \LaTeXe}
% \changes{norsk-1.2d}{1994/06/26}{Removed the use of \cs{filedate}
%    and moved identification after the loading of \file{babel.def}}
% \changes{norsk-1.2h}{1996/07/12}{Replaced \cs{undefined} with
%    \cs{@undefined} and \cs{empty} with \cs{@empty} for consistency
%    with \LaTeX} 
% \changes{norsk-1.2h}{1996/10/10}{Moved the definition of
%    \cs{atcatcode} right to the beginning.}
%
%
%  \section{The Norwegian language}
%
%    The file \file{\filename}\footnote{The file described in this
%    section has version number \fileversion\ and was last revised on
%    \filedate.  Contributions were made by Haavard Helstrup
%    (\texttt{HAAVARD@CERNVM)} and Alv Kjetil Holme
%    (\texttt{HOLMEA@CERNVM}); the `nynorsk' variant has been supplied
%    by Per Steinar Iversen \texttt{iversen@vxcern.cern.ch}) and Terje
%    Engeset Petterst (\texttt{TERJEEP@VSFYS1.FI.UIB.NO)}; the
%    shorthand definitions were provided by Rune Kleveland
%    (\texttt{runekl@math.uio.no}).} defines all the language definition
%    macros for the Norwegian language as well as for an alternative
%    variant `nynorsk' of this language. 
%
%    For this language the character |"| is made active. In
%    table~\ref{tab:norsk-quote} an overview is given of its purpose.
%    \begin{table}[htb]
%     \begin{center}
%     \begin{tabular}{lp{.7\textwidth}}
%      |"ff|& for |ff| to be hyphenated as |ff-f|,
%             this is also implemented for b, d, f, g, l, m, n,
%             p, r, s, and t. (|o"ppussing|)                        \\
%      |"ee|& Hyphenate |"ee| as |\'e-e|. (|komit"een|)             \\
%      |"-| & an explicit hyphen sign, allowing hyphenation in the
%             composing words. Use this for compound words when the
%             hyphenation patterns fail to hyphenate
%             properly. (|alpin"-anlegg|)                           \\
%      \verb="|= & Like |"-|, but inserts 0.03em space.  Use it if
%             the compound point is spanned by a ligature.
%             (\verb=hoff"|intriger=)                               \\
%      |""| & Like |"-|, but producing no hyphen sign.
%             (|i""g\aa{}r|)                                        \\
%      |"~| & Like |-|, but allows no hyphenation at all. (|E"~cup|)\\
%      |"=| & Like |-|, but allowing hyphenation in the composing
%             words. (|marksistisk"=leninistisk|)                   \\
%      |"<| & for French left double quotes (similar to $<<$).      \\
%      |">| & for French right double quotes (similar to $>>$).     \\
%     \end{tabular}
%     \caption{The extra definitions made
%              by \file{norsk.sty}}\label{tab:norsk-quote}
%     \end{center}
%    \end{table}
% \changes{norsk-2.0a}{1998/06/24}{Describe the use of double quote as
%    active character}
%
%    Rune Kleveland distributes a Norwegian dictionary for ispell
%    (570000 words). It can be found at
%    |http://www.uio.no/~runekl/dictionary.html|. 
%
%    This dictionary supports the spellings |spi"sslede| for
%    `spisslede' (hyphenated spiss-slede) and other such words, and
%    also suggest the spelling |spi"sslede| for `spisslede' and
%    `spissslede'.
%
% \StopEventually{}
%
%    The macro |\LdfInit| takes care of preventing that this file is
%    loaded more than once, checking the category code of the
%    \texttt{@} sign, etc.
% \changes{norsk-1.2h}{1996/11/03}{Now use \cs{LdfInit} to perform
%    initial checks} 
%    \begin{macrocode}
%<*code>
\LdfInit\CurrentOption{captions\CurrentOption}
%    \end{macrocode}
%
%    When this file is read as an option, i.e. by the |\usepackage|
%    command, \texttt{norsk} will be an `unknown' language in which
%    case we have to make it known.  So we check for the existence of
%    |\l@norsk| to see whether we have to do something here.
%
% \changes{norsk-1.0c}{1991/10/29}{Removed use of \cs{@ifundefined}}
% \changes{norsk-1.1a}{1992/02/16}{Added a warning when no hyphenation
%    patterns were loaded.}
% \changes{norsk-1.2d}{1994/06/26}{Now use \cs{@nopatterns} to produce
%    the warning}
%    \begin{macrocode}
\ifx\l@norsk\@undefined
    \@nopatterns{Norsk}
    \adddialect\l@norsk0\fi
%    \end{macrocode}
%
%  \begin{macro}{\norskhyphenmins}
%     Some sets of Norwegian hyphenation patterns can be used with
%     |\lefthyphenmin| set to~1 and |\righthyphenmin| set to~2, but
%     the most common set |nohyph.tex| can't.  So we use
%     |\lefthyphenmin=2| by default.
% \changes{norsk-1.2f}{1995/07/02}{Added setting of hyphenmin
%    parameters}
% \changes{norsk-2.0a}{1998/06/24}{Changed setting of hyphenmin
%    parameters to 2~2} 
% \changes{norsk-2.0e}{2000/09/22}{Now use \cs{providehyphenmins} to
%    provide a default value}
%    \begin{macrocode}
\providehyphenmins{\CurrentOption}{\tw@\tw@}
%    \end{macrocode}
%  \end{macro}
%
%    Now we have to decide which version of the captions should be
%    made available. This can be done by checking the contents of
%    |\CurrentOption|. 
%    \begin{macrocode}
\def\bbl@tempa{norsk}
\ifx\CurrentOption\bbl@tempa
%    \end{macrocode}
%
%    The next step consists of defining commands to switch to (and
%    from) the Norwegian language.
%
% \begin{macro}{\captionsnorsk}
%    The macro |\captionsnorsk| defines all strings used
%    in the four standard documentclasses provided with \LaTeX.
% \changes{norsk-1.1a}{1992/02/16}{Added \cs{seename}, \cs{alsoname} and
%    \cs{prefacename}}
% \changes{norsk-1.1b}{1993/07/15}{\cs{headpagename} should be
%    \cs{pagename}}
% \changes{norsk-1.2f}{1995/07/02}{Added \cs{proofname} for
%    AMS-\LaTeX}
% \changes{norsk-1.2g}{1996/04/01}{Replaced `Proof' by its
%    translation} 
% \changes{norsk-2.0e}{2000/09/20}{Added \cs{glossaryname}}
% \changes{norsk-2.0g}{1996/04/01}{Replaced `Glossary' by its
%    translation} 
%    \begin{macrocode}
  \def\captionsnorsk{%
    \def\prefacename{Forord}%
    \def\refname{Referanser}%
    \def\abstractname{Sammendrag}%
    \def\bibname{Bibliografi}%     or Litteraturoversikt
    %                              or Litteratur or Referanser
    \def\chaptername{Kapittel}%
    \def\appendixname{Tillegg}%    or Appendiks
    \def\contentsname{Innhold}%
    \def\listfigurename{Figurer}%  or Figurliste
    \def\listtablename{Tabeller}%  or Tabelliste
    \def\indexname{Register}%
    \def\figurename{Figur}%
    \def\tablename{Tabell}%
    \def\partname{Del}%
    \def\enclname{Vedlegg}%
    \def\ccname{Kopi sendt}%
    \def\headtoname{Til}% in letter
    \def\pagename{Side}%
    \def\seename{Se}%
    \def\alsoname{Se ogs\aa{}}%
    \def\proofname{Bevis}%
    \def\glossaryname{Ordliste}%
    }
\else
%    \end{macrocode}
% \end{macro}
%
%    For the `nynorsk' version of these definitions we just add a
%    ``dialect''.
%    \begin{macrocode}
  \adddialect\l@nynorsk\l@norsk
%    \end{macrocode}
%
% \begin{macro}{\captionsnynorsk}
%    The macro |\captionsnynorsk| defines all strings used in the four
%    standard documentclasses provided with \LaTeX, but using a
%    different spelling than in the command |\captionsnorsk|.
% \changes{norsk-1.1a}{1992/02/16}{Added \cs{seename}, \cs{alsoname} and
%    \cs{prefacename}}
% \changes{norsk-1.1b}{1993/07/15}{\cs{headpagename} should be
%    \cs{pagename}}
% \changes{norsk-1.2g}{1996/04/01}{Replaced `Proof' by its
%    translation} 
% \changes{norsk-2.0e}{2000/09/20}{Added \cs{glossaryname}}
% \changes{norsk-2.0g}{1996/04/01}{Replaced `Glossary' by its
%    translation} 
% \changes{norks-2.0h}{2001/01/12}{Changed \cs{ccname} and \cs{alsoname}}
%    \begin{macrocode}
  \def\captionsnynorsk{%
    \def\prefacename{Forord}%
    \def\refname{Referansar}%
    \def\abstractname{Samandrag}%
    \def\bibname{Litteratur}%     or Litteraturoversyn
     %                             or Referansar
    \def\chaptername{Kapittel}%
    \def\appendixname{Tillegg}%   or Appendiks
    \def\contentsname{Innhald}%
    \def\listfigurename{Figurar}% or Figurliste
    \def\listtablename{Tabellar}% or Tabelliste
    \def\indexname{Register}%
    \def\figurename{Figur}%
    \def\tablename{Tabell}%
    \def\partname{Del}%
    \def\enclname{Vedlegg}%
    \def\ccname{Kopi til}%
    \def\headtoname{Til}% in letter
    \def\pagename{Side}%
    \def\seename{Sj\aa{}}%
    \def\alsoname{Sj\aa{} \`{o}g}%
    \def\proofname{Bevis}%
    \def\glossaryname{Ordliste}%
    }
\fi
%    \end{macrocode}
% \end{macro}
%
% \begin{macro}{\datenorsk}
%    The macro |\datenorsk| redefines the command |\today| to produce
%    Norwegian dates.
% \changes{norsk-1.2i}{1997/10/01}{Use \cs{edef} to define
%    \cs{today} to save memory}
% \changes{norsk-1.2i}{1998/03/28}{use \cs{def} instead of \cs{edef}}
% \changes{norsk-2.0i}{2012/08/06}{Removed extra space after `desember'}
%    \begin{macrocode}
\@namedef{date\CurrentOption}{%
  \def\today{\number\day.~\ifcase\month\or
    januar\or februar\or mars\or april\or mai\or juni\or
    juli\or august\or september\or oktober\or november\or
    desember\fi
    \space\number\year}}
%    \end{macrocode}
% \end{macro}
%
% \begin{macro}{\extrasnorsk}
% \begin{macro}{\extrasnynorsk}
%    The macro |\extrasnorsk| will perform all the extra definitions
%    needed for the Norwegian language. The macro |\noextrasnorsk| is
%    used to cancel the actions of |\extrasnorsk|.  
%
%    Norwegian typesetting requires |\frencspacing| to be in effect.
%    \begin{macrocode}
\@namedef{extras\CurrentOption}{\bbl@frenchspacing}
\@namedef{noextras\CurrentOption}{\bbl@nonfrenchspacing}
%    \end{macrocode}
% \end{macro}
% \end{macro}
%
%    For Norsk the \texttt{"} character is made active. This is done
%    once, later on its definition may vary.
% \changes{norsk-2.0a}{1998/06/24}{Made double quote character active}
%    \begin{macrocode}
\initiate@active@char{"}
\expandafter\addto\csname extras\CurrentOption\endcsname{%
  \languageshorthands{norsk}}
\expandafter\addto\csname extras\CurrentOption\endcsname{%
  \bbl@activate{"}}
%    \end{macrocode}
%    Don't forget to turn the shorthands off again.
% \changes{norsk-2.0c}{1999/12/17}{Deactivate shorthands ouside of
%    Norsk}
%    \begin{macrocode}
\expandafter\addto\csname noextras\CurrentOption\endcsname{%
  \bbl@deactivate{"}}
%    \end{macrocode}
%
%    The code above is necessary because we need to define a number of
%    shorthand commands. These sharthand commands are then used as
%    indicated in table~\ref{tab:norsk-quote}.
%
%    To be able to define the function of |"|, we first define a
%    couple of `support' macros.
%
%  \begin{macro}{\dq}
%    We save the original double quote character in |\dq| to keep
%    it available, the math accent |\"| can now be typed as |"|.
%    \begin{macrocode}
\begingroup \catcode`\"12
\def\x{\endgroup
  \def\@SS{\mathchar"7019 }
  \def\dq{"}}
\x
%    \end{macrocode}
%  \end{macro}
%
%    Now we can define the discretionary shorthand commands.
%    The number of words where such hyphenation is required is for
%    each character
%    \begin{center}
%      \begin{tabular}{*{11}c}
%        b&d&f &g&k &l &n&p &r&s &t \\
%        4&4&15&3&43&30&8&12&1&33&35
%       \end{tabular}
%    \end{center}
%    taken from a list of 83000 ispell-roots.
%
% \changes{norsk-2.0d}{2000/02/29}{Shorthands are the same for both
%    spelling variants, no need to use \cs{CurrentOption}}
%    \begin{macrocode}
\declare@shorthand{norsk}{"b}{\textormath{\bbl@disc b{bb}}{b}}
\declare@shorthand{norsk}{"B}{\textormath{\bbl@disc B{BB}}{B}}
\declare@shorthand{norsk}{"d}{\textormath{\bbl@disc d{dd}}{d}}
\declare@shorthand{norsk}{"D}{\textormath{\bbl@disc D{DD}}{D}}
\declare@shorthand{norsk}{"e}{\textormath{\bbl@disc e{\'e}}{}}
\declare@shorthand{norsk}{"E}{\textormath{\bbl@disc E{\'E}}{}}
\declare@shorthand{norsk}{"F}{\textormath{\bbl@disc F{FF}}{F}}
\declare@shorthand{norsk}{"g}{\textormath{\bbl@disc g{gg}}{g}}
\declare@shorthand{norsk}{"G}{\textormath{\bbl@disc G{GG}}{G}}
\declare@shorthand{norsk}{"k}{\textormath{\bbl@disc k{kk}}{k}}
\declare@shorthand{norsk}{"K}{\textormath{\bbl@disc K{KK}}{K}}
\declare@shorthand{norsk}{"l}{\textormath{\bbl@disc l{ll}}{l}}
\declare@shorthand{norsk}{"L}{\textormath{\bbl@disc L{LL}}{L}}
\declare@shorthand{norsk}{"n}{\textormath{\bbl@disc n{nn}}{n}}
\declare@shorthand{norsk}{"N}{\textormath{\bbl@disc N{NN}}{N}}
\declare@shorthand{norsk}{"p}{\textormath{\bbl@disc p{pp}}{p}}
\declare@shorthand{norsk}{"P}{\textormath{\bbl@disc P{PP}}{P}}
\declare@shorthand{norsk}{"r}{\textormath{\bbl@disc r{rr}}{r}}
\declare@shorthand{norsk}{"R}{\textormath{\bbl@disc R{RR}}{R}}
\declare@shorthand{norsk}{"s}{\textormath{\bbl@disc s{ss}}{s}}
\declare@shorthand{norsk}{"S}{\textormath{\bbl@disc S{SS}}{S}}
\declare@shorthand{norsk}{"t}{\textormath{\bbl@disc t{tt}}{t}}
\declare@shorthand{norsk}{"T}{\textormath{\bbl@disc T{TT}}{T}}
%    \end{macrocode}
%    We need to treat |"f| a bit differently in order to preserve the
%    ff-ligature. 
% \changes{norsk-2.0b}{1999/11/19}{Copied the coding for \texttt{"f}
%    from germanb.dtx version 2.6g} 
%    \begin{macrocode}
\declare@shorthand{norsk}{"f}{\textormath{\bbl@discff}{f}}
\def\bbl@discff{\penalty\@M
  \afterassignment\bbl@insertff \let\bbl@nextff= }
\def\bbl@insertff{%
  \if f\bbl@nextff
    \expandafter\@firstoftwo\else\expandafter\@secondoftwo\fi
  {\relax\discretionary{ff-}{f}{ff}\allowhyphens}{f\bbl@nextff}}
\let\bbl@nextff=f
%    \end{macrocode}
%    We now  define the French double quotes and some commands 
%    concerning hyphenation:
% \changes{norsk-2.0b}{1999/11/22}{added the french double quotes}
% \changes{norsk-2.0d}{2000/01/28}{Use \cs{bbl@allowhyphens} in
%    \texttt{"-}}
%    \begin{macrocode}
\declare@shorthand{norsk}{"<}{\flqq}
\declare@shorthand{norsk}{">}{\frqq}
\declare@shorthand{norsk}{"-}{\penalty\@M\-\bbl@allowhyphens}
\declare@shorthand{norsk}{"|}{%
  \textormath{\penalty\@M\discretionary{-}{}{\kern.03em}%
              \allowhyphens}{}}
\declare@shorthand{norsk}{""}{\hskip\z@skip}
\declare@shorthand{norsk}{"~}{\textormath{\leavevmode\hbox{-}}{-}}
\declare@shorthand{norsk}{"=}{\penalty\@M-\hskip\z@skip}
%    \end{macrocode}
%
%    The macro |\ldf@finish| takes care of looking for a
%    configuration file, setting the main language to be switched on
%    at |\begin{document}| and resetting the category code of
%    \texttt{@} to its original value.
% \changes{norsk-1.2h}{1996/11/03}{Now use \cs{ldf@finish} to wrap up}
%    \begin{macrocode}
\ldf@finish\CurrentOption
%</code>
%    \end{macrocode}
%
% \Finale
%%
%% \CharacterTable
%%  {Upper-case    \A\B\C\D\E\F\G\H\I\J\K\L\M\N\O\P\Q\R\S\T\U\V\W\X\Y\Z
%%   Lower-case    \a\b\c\d\e\f\g\h\i\j\k\l\m\n\o\p\q\r\s\t\u\v\w\x\y\z
%%   Digits        \0\1\2\3\4\5\6\7\8\9
%%   Exclamation   \!     Double quote  \"     Hash (number) \#
%%   Dollar        \$     Percent       \%     Ampersand     \&
%%   Acute accent  \'     Left paren    \(     Right paren   \)
%%   Asterisk      \*     Plus          \+     Comma         \,
%%   Minus         \-     Point         \.     Solidus       \/
%%   Colon         \:     Semicolon     \;     Less than     \<
%%   Equals        \=     Greater than  \>     Question mark \?
%%   Commercial at \@     Left bracket  \[     Backslash     \\
%%   Right bracket \]     Circumflex    \^     Underscore    \_
%%   Grave accent  \`     Left brace    \{     Vertical bar  \|
%%   Right brace   \}     Tilde         \~}
%%
\endinput
}
\DeclareOption{polish}{% \iffalse meta-comme

% Copyright 1989-2005 Johannes L. Braams and any individual autho
% listed elsewhere in this file.  All rights reserve

% This file is part of the Babel syste
% ------------------------------------

% It may be distributed and/or modified under t
% conditions of the LaTeX Project Public License, either version 1
% of this license or (at your option) any later versio
% The latest version of this license is
%   http://www.latex-project.org/lppl.t
% and version 1.3 or later is part of all distributions of LaT
% version 2003/12/01 or late

% This work has the LPPL maintenance status "maintained

% The Current Maintainer of this work is Johannes Braam

% The list of all files belonging to the Babel system
% given in the file `manifest.bbl. See also `legal.bbl' for addition
% informatio

% The list of derived (unpacked) files belonging to the distributi
% and covered by LPPL is defined by the unpacking scripts (wi
% extension .ins) which are part of the distributio
% \
% \CheckSum{54

% \iffal
%    Tell the \LaTeX\ system who we are and write an entry on t
%    transcrip
%<*dt
\ProvidesFile{polish.dt
%</dt
%<code>\ProvidesLanguage{polis
%\
%\ProvidesFile{polish.dt
        [2005/03/31 v1.2l Polish support from the babel syste
%\iffal
%% File `polish.dt
%% Babel package for LaTeX version
%% Copyright (C) 1989 -- 20
%%           by Johannes Braams, TeXni

%% Polish Language Definition Fi
%% Copyright (C) 1989 - 20
%%           by Elmar Schalueck, Michael Jani
%              Universitaet-Gesamthochschule Paderbo
%              Warburger Strasse 1
%              4790 Paderbo
%              Germa
%              elmar at uni-paderborn.
%              massa at uni-paderborn.

%% Please report errors to: J.L. Braa
%%                          babel at braams.cistron.

%    This file is part of the babel system, it provides the sour
%    code for the Polish language definition file. It was developp
%    out of Polish.tex, which was written by Elmar Schalueck a
%    Michael Janich. Polish.tex was based on code by Lesz
%    Holenderski, Jerzy Ryll and J. S. Bie\'n from Faculty
%    Mathematics,Informatics and Mechanics of Warsaw University, exe
%    of Jerzy Ryll (Instytut Informatyki Uniwersytetu Warszawskiego
%<*filedrive
\documentclass{ltxdo
\newcommand*\TeXhax{\TeX ha
\newcommand*\babel{\textsf{babel
\newcommand*\langvar{$\langle \it lang \rangle
\newcommand*\note[1]
\newcommand*\Lopt[1]{\textsf{#1
\newcommand*\file[1]{\texttt{#1
\begin{documen
 \DocInput{polish.dt
\end{documen
%</filedrive
%\
% \GetFileInfo{polish.dt

% \changes{polish-1.1c}{1994/06/26}{Removed the use of \cs{filedat
%    and moved identification after the loading of babel.de
% \changes{polish-1.2d}{1996/10/10}{Replaced \cs{undefined} wi
%    \cs{@undefined} and \cs{empty} with \cs{@empty} for consisten
%    with \LaTeX, moved the definition of \cs{atcatcode} right to t
%    beginning

%  \section{The Polish languag

%    The file \file{\filename}\footnote{The file described in th
%    section has version number \fileversion\ and was last revised
%    \filedate.}  defines all the language-specific macros for t
%    Polish languag

%    For this language the character |"| is made active.
%    table~\ref{tab:polish-quote} an overview is given of its purpos
%    \begin{table}[ht
%     \begin{cente
%     \begin{tabular}{lp{8cm
%      |"a| & or |\aob|, for tailed-a (like \c{a})
%      |"A| & or |\Aob|, for tailed-A (like \c{A})
%      |"e| & or |\eob|, for tailed-e (like \c{e})
%      |"E| & or |\Eob|, for tailed-E (like \c{E})
%      |"c| & or |\'c|,  for accented c (like \'c
%                      same with uppercase letters and n,o,s
%      |"l| & or |\lpb{}|, for l with stroke (like \l)
%      |"L| & or |\Lpb{}|, for L with stroke (like \L)
%      |"r| & or |\zkb{}|, for pointed z (like \.z), c
%      pronounciation
%      |"R| & or |\Zkb{}|, for pointed Z (like \.Z)
%      |"z| & or |\'z|,  for accented z
%      |"Z| & or |\'Z|,  for accented Z
%      \verb="|= & disable ligature at this position.
%      |"-| & an explicit hyphen sign, allowing hyphenati
%                  in the rest of the word.
%      |""| & like |"-|, but producing no hyphen si
%                  (for compund words with hyphen, e.g.\ |x-""y|).
%      |"`| & for German left double quotes (looks like ,,).
%      |"'| & for German right double quotes.
%      |"<| & for French left double quotes (similar to $<<$).
%      |">| & for French right double quotes (similar to $>>$).
%     \end{tabula
%     \caption{The extra definitions made by \file{polish.sty
%     \label{tab:polish-quot
%     \end{cente
%    \end{tabl

% \StopEventually

%    The macro |\LdfInit| takes care of preventing that this file
%    loaded more than once, checking the category code of t
%    \texttt{@} sign, et
% \changes{polish-1.2d}{1996/11/03}{Now use \cs{LdfInit} to perfo
%    initial checks
%    \begin{macrocod
%<*cod
\LdfInit{polish}\captionspoli
%    \end{macrocod

%    When this file is read as an option, i.e. by the |\usepackag
%    command, \texttt{polish} could be an `unknown' language in whi
%    case we have to make it known. So we check for the existence
%    |\l@polish| to see whether we have to do something her

% \changes{polish-1.1c}{1994/06/26}{Now use \cs{@nopatterns}
%    produce the warnin
%    \begin{macrocod
\ifx\l@polish\@undefin
  \@nopatterns{Polis
  \adddialect\l@polish0\
%    \end{macrocod

%    The next step consists of defining commands to switch to (a
%    from) the Polish languag

% \begin{macro}{\captionspolis
%    The macro |\captionspolish| defines all strings used in the fo
%    standard documentclasses provided with \LaTe
% \changes{polish-1.2b}{1995/07/04}{Added \cs{proofname} f
%    AMS-\LaTe
% \changes{polish-1.2f}{1997/09/11}{Added translation for Proof a
%    changed translation of Content
% \changes{polish-1.2i}{2000/09/19}{\cs{bibname} and \cs{refname} we
%    swappe
% \changes{polish-1.2i}{2000/09/19}{Added \cs{glossaryname
%    \begin{macrocod
\addto\captionspolish
  \def\prefacename{Przedmowa
  \def\refname{Literatura
  \def\abstractname{Streszczenie
  \def\bibname{Bibliografia
  \def\chaptername{Rozdzia\l
  \def\appendixname{Dodatek
  \def\contentsname{Spis tre\'sci
  \def\listfigurename{Spis rysunk\'ow
  \def\listtablename{Spis tablic
  \def\indexname{Indeks
  \def\figurename{Rysunek
  \def\tablename{Tablica
  \def\partname{Cz\eob{}\'s\'c
  \def\enclname{Za\l\aob{}cznik
  \def\ccname{Kopie:
  \def\headtoname{Do
  \def\pagename{Strona
  \def\seename{Por\'ownaj
  \def\alsoname{Por\'ownaj tak\.ze
  \def\proofname{Dow\'od
  \def\glossaryname{Glossary}% <-- Needs translati

%    \end{macrocod
% \end{macr

% \begin{macro}{\datepolis
%    The macro |\datepolish| redefines the command |\today| to produ
%    Polish date
% \changes{polish-1.2f}{1997/10/01}{Use \cs{edef} to defi
%    \cs{today} to save memor
% \changes{polish-1.2f}{1998/03/28}{use \cs{def} instead of \cs{edef
% \changes{polish-1.2i}{2000/01/08}{A missing comment character caus
%    an unwanted space character in the outpu
%    \begin{macrocod
\def\datepolish
  \def\today{\number\day~\ifcase\month\
  stycznia\or lutego\or marca\or kwietnia\or maja\or czerwca\or lipca\
  sierpnia\or wrze\'snia\or pa\'zdziernika\or listopada\or grudnia\
  \space\number\year

%    \end{macrocod
% \end{macr

% \begin{macro}{\extraspolis
% \begin{macro}{\noextraspolis
%    The macro |\extraspolish| will perform all the extra definitio
%    needed for the Polish language. The macro |\noextraspolish|
%    used to cancel the actions of |\extraspolish

%    For Polish the \texttt{"} character is made active. This
%    done once, later on its definition may vary. Other languages
%    the same document may also use the \texttt{"} character f
%    shorthands; we specify that the polish group of shorthan
%    should be use

%    \begin{macrocod
\initiate@active@char{
\addto\extraspolish{\languageshorthands{polish
\addto\extraspolish{\bbl@activate{"
%    \end{macrocod
%    Don't forget to turn the shorthands off agai
% \changes{polish-1.2h}{1999/12/17}{Deactivate shorthands ouside
%    Polis
%    \begin{macrocod
\addto\noextraspolish{\bbl@deactivate{"
%    \end{macrocod
% \end{macr
% \end{macr

%    The code above is necessary because we need an ext
%    active character. This character is then used as indicated
%    table~\ref{tab:polish-quote

%    If you have problems at the end of a word with a linebreak, u
%    the other version without hyphenation tricks. Some TeX wizard m
%    produce a better solution with forcasting another token to deci
%    whether the character after the double quote is the last in
%    word. Do it and let us kno

%    In Polish texts some letters get special diacritical mark
%    Leszek Holenderski designed the following code to position t
%    diacritics correctly for every font in every size. These macr
%    need a few extra dimension variable

%    \begin{macrocod
\newdimen\pl@le
\newdimen\pl@do
\newdimen\pl@rig
\newdimen\pl@te
%    \end{macrocod

%  \begin{macro}{\so
%    The macro |\sob| is used to put the `ogonek' in the rig
%    plac

% \changes{polish-1.2d}{1996/08/18}{This macro is meant to be used
%    horizontal mode; so leave vertical mode if necessary
%    \begin{macrocod
\def\sob#1#2#3#4#5{%parameters: letter and fractions hl,ho,vl,
  \setbox0\hbox{#1}\setbox1\hbox{$_\mathchar'454$}\setbox2\hbox{p
  \pl@right=#2\wd0 \advance\pl@right by-#3\w
  \pl@down=#5\ht1 \advance\pl@down by-#4\h
  \pl@left=\pl@right \advance\pl@left by\w
  \pl@temp=-\pl@down \advance\pl@temp by\dp2 \dp1=\pl@te
  \leavevmo
  \kern\pl@right\lower\pl@down\box1\kern-\pl@left #
%    \end{macrocod
%  \end{macr

%  \begin{macro}{\ao
%  \begin{macro}{\Ao
%  \begin{macro}{\eo
%  \begin{macro}{\Eo
%    The ogonek is placed with the letters `a', `A', `e', and `E
% \changes{polish-1.2d}{1996/08/18}{Use the constructed version of t
%    characters only in OT1; use proper characters in T1.
%    \begin{macrocod
\DeclareTextCommand{\aob}{OT1}{\sob a{.66}{.20}{0}{.90
\DeclareTextCommand{\Aob}{OT1}{\sob A{.80}{.50}{0}{.90
\DeclareTextCommand{\eob}{OT1}{\sob e{.50}{.35}{0}{.93
\DeclareTextCommand{\Eob}{OT1}{\sob E{.60}{.35}{0}{.90
%    \end{macrocod
%    For the 'new' \texttt{T1} encoding we can provide simpl
%    definitions
%    \begin{macrocod
\DeclareTextCommand{\aob}{T1}{\k
\DeclareTextCommand{\Aob}{T1}{\k
\DeclareTextCommand{\eob}{T1}{\k
\DeclareTextCommand{\Eob}{T1}{\k
%    \end{macrocod
%    Construct the characters by default from the OT1 encodin
%    \begin{macrocod
\ProvideTextCommandDefault{\aob}{\UseTextSymbol{OT1}{\aob
\ProvideTextCommandDefault{\Aob}{\UseTextSymbol{OT1}{\Aob
\ProvideTextCommandDefault{\eob}{\UseTextSymbol{OT1}{\eob
\ProvideTextCommandDefault{\Eob}{\UseTextSymbol{OT1}{\Eob
%    \end{macrocod
%  \end{macr
%  \end{macr
%  \end{macr
%  \end{macr

%  \begin{macro}{\sp
%    The macro |\spb| is used to put the `poprzeczka' in t
%    right plac

% \changes{polish-1.2d}{1996/08/18}{\cs{spb} is meant to be used
%    horizontal mode; so leave vertical mode if necessary
%    \begin{macrocod
\def\spb#1#2#3#4#5
  \setbox0\hbox{#1}\setbox1\hbox{\char'023
  \pl@right=#2\wd0 \advance\pl@right by-#3\w
  \pl@down=#5\ht1 \advance\pl@down by-#4\h
  \pl@left=\pl@right \advance\pl@left by\w
  \ht1=\pl@down \dp1=-\pl@do
  \leavevmo
  \kern\pl@right\lower\pl@down\box1\kern-\pl@left #
%    \end{macrocod
%  \end{macr

%  \begin{macro}{\sk
%    The macro |\skb| is used to put the `kropka' in t
%    right plac

% \changes{polish-1.2d}{1996/08/18}{\cs{skb} is meant to be used
%    horizontal mode; so leave vertical mode if necessary
%    \begin{macrocod
\def\skb#1#2#3#4#5
  \setbox0\hbox{#1}\setbox1\hbox{\char'056
  \pl@right=#2\wd0 \advance\pl@right by-#3\w
  \pl@down=#5\ht1 \advance\pl@down by-#4\h
  \pl@left=\pl@right \advance\pl@left by\w
  \leavevmo
  \kern\pl@right\lower\pl@down\box1\kern-\pl@left #
%    \end{macrocod
%  \end{macr

%  \begin{macro}{\textp
%    For the `poprzeczka' and the `kropka' in text fonts we don't ne
%    any special coding, but we can (almost) use what is alrea
%    availabl

%    \begin{macrocod
\def\textpl
  \def\lpb{\plll
  \def\Lpb{\pLLL
  \def\zkb{\.z
  \def\Zkb{\.Z
%    \end{macrocod
%    Initially we assume that typesetting is done with text font
% \changes{polish-1.0a}{1993/11/05}{Initially execute `textp
%    \begin{macrocod
\text
%    \end{macrocod

%    \begin{macrocod
\let\lll=\l \let\LLL=
\def\plll{\ll
\def\pLLL{\LL
%    \end{macrocod
%  \end{macr

%  \begin{macro}{\telep
%    But for the `teletype' font in `OT1' encoding we have to take so
%    special actions, involving the macros defined abov

%    \begin{macrocod
\def\telepl
  \def\lpb{\spb l{.45}{.5}{.4}{.8}
  \def\Lpb{\spb L{.23}{.5}{.4}{.8}
  \def\zkb{\skb z{.5}{.5}{1.2}{0}
  \def\Zkb{\skb Z{.5}{.5}{1.1}{0}
%    \end{macrocod
%  \end{macr

%    To activate these codes the font changing commands as they a
%    defined in \LaTeX\ are modified. The same is done for pla
%    \TeX's font changing command

%    When |\selectfont| is undefined the current format is spposed to
%    either plain (based) or \LaTeX$\:$2.0
% \changes{polish-1.2a}{1995/06/06}{Don't modify \cs{rm} and friends f
%    \LaTeXe, take \cs{selectfont } instea
%    \begin{macrocod
\ifx\selectfont\@undefin
  \ifx\prm\@undefined \addto\rm{\textpl}\else \addto\prm{\textpl}\
  \ifx\pit\@undefined \addto\it{\textpl}\else \addto\pit{\textpl}\
  \ifx\pbf\@undefined \addto\bf{\textpl}\else \addto\pbf{\textpl}\
  \ifx\psl\@undefined \addto\sl{\textpl}\else \addto\psl{\textpl}\
  \ifx\psf\@undefined                   \else \addto\psf{\textpl}\
  \ifx\psc\@undefined                   \else \addto\psc{\textpl}\
  \ifx\ptt\@undefined \addto\tt{\telepl}\else \addto\ptt{\telepl}\
\el
%    \end{macrocod
%    When |\selectfont| exists we assume \LaTeX
%    \begin{macrocod
  \expandafter\addto\csname selectfont \endcsname
    \csname\f@encoding @pl\endcsnam
\
%    \end{macrocod
%    Currently we support the OT1 and T1 encodings. For T1 we don
%    have to make a difference between typewriter fonts and oth
%    fonts, they all have the same glyph
%    \begin{macrocod
\expandafter\let\csname T1@pl\endcsname\text
%    \end{macrocod
%    For OT1 we need to check the current font family, stored
%    |\f@family|. Unfortunately we need a hack as |\ttdefault|
%    defined as a |\long| macro, while |\f@family| is no
%    \begin{macrocod
\expandafter\def\csname OT1@pl\endcsname
  \long\edef\curr@family{\f@family
  \ifx\curr@family\ttdefau
    \tele
  \el
    \text
  \f
%    \end{macrocod

%  \begin{macro}{\d
%    We save the original double quote character in |\dq| to ke
%    it available, the math accent |\"| can now be typed as |"
%    \begin{macrocod
\begingroup \catcode`\"
\def\x{\endgro
  \def\dq{"

%    \end{macrocod
%  \end{macr

%    Now we can define the doublequote macros for diacritic
% \changes{polish-1.1d}{1995/01/31}{The dqmacro for C used \cs{'c
%    \begin{macrocod
\declare@shorthand{polish}{"a}{\textormath{\aob}{\ddot a
\declare@shorthand{polish}{"A}{\textormath{\Aob}{\ddot A
\declare@shorthand{polish}{"c}{\textormath{\'c}{\acute c
\declare@shorthand{polish}{"C}{\textormath{\'C}{\acute C
\declare@shorthand{polish}{"e}{\textormath{\eob}{\ddot e
\declare@shorthand{polish}{"E}{\textormath{\Eob}{\ddot E
\declare@shorthand{polish}{"l}{\textormath{\lpb}{\ddot l
\declare@shorthand{polish}{"L}{\textormath{\Lpb}{\ddot L
\declare@shorthand{polish}{"n}{\textormath{\'n}{\acute n
\declare@shorthand{polish}{"N}{\textormath{\'N}{\acute N
\declare@shorthand{polish}{"o}{\textormath{\'o}{\acute o
\declare@shorthand{polish}{"O}{\textormath{\'O}{\acute O
\declare@shorthand{polish}{"s}{\textormath{\'s}{\acute s
\declare@shorthand{polish}{"S}{\textormath{\'S}{\acute S
%    \end{macrocod
%  \begin{macro}{\polishr
%  \begin{macro}{\polishz
% \changes{polish-1.2j}{2000/11/11}{Added support for t
%    notationstyles for kropka and accented
%    The command |\polishrz| defines the shorthands |"r|, |"z| a
%    |"x| to produce pointed z, accented z and |"x|. This is t
%    default as these shorthands were defined by this langua
%    definition file for quite some tim
%    \begin{macrocod
\newcommand*{\polishrz}
  \declare@shorthand{polish}{"r}{\textormath{\zkb}{\ddot r}
  \declare@shorthand{polish}{"R}{\textormath{\Zkb}{\ddot R}
  \declare@shorthand{polish}{"z}{\textormath{\'z}{\acute z}
  \declare@shorthand{polish}{"Z}{\textormath{\'Z}{\acute Z}
  \declare@shorthand{polish}{"x}{\dq x
  \declare@shorthand{polish}{"X}{\dq X

\polish
%    \end{macrocod
%    The command |\polishzx| switches to a different set
%    shorthands, |"z|, |"x| and |"r| to produce pointed z, accented
%    and |"r|; a different shorthand notation also in us
% \changes{polish-1.2k}{2003/10/12}{Fixed a typ
%    \begin{macrocod
\newcommand*{\polishzx}
  \declare@shorthand{polish}{"z}{\textormath{\zkb}{\ddot z}
  \declare@shorthand{polish}{"Z}{\textormath{\Zkb}{\ddot Z}
  \declare@shorthand{polish}{"x}{\textormath{\'z}{\acute x}
  \declare@shorthand{polish}{"X}{\textormath{\'Z}{\acute X}
  \declare@shorthand{polish}{"r}{\dq r
  \declare@shorthand{polish}{"R}{\dq R

%    \end{macrocod
%  \end{macr
%  \end{macr

%    Then we define access to two forms of quotation marks, simil
%    to the german and french quotation mark
% \changes{polish-1.2e}{1997/04/03}{Removed empty groups aft
%    double quote and guillemot character
% \changes{polish-1.2l}{2004/02/18}{Changed closing quot
%    \begin{macrocod
\declare@shorthand{polish}{"`}
  \textormath{\quotedblbase}{\mbox{\quotedblbase}
\declare@shorthand{polish}{"'}
  \textormath{\textquotedblright}{\mbox{\textquotedblright}
\declare@shorthand{polish}{"<}
  \textormath{\guillemotleft}{\mbox{\guillemotleft}
\declare@shorthand{polish}{">}
  \textormath{\guillemotright}{\mbox{\guillemotright}
%    \end{macrocod
%    then we define two shorthands to be able to specify hyphenati
%    breakpoints that behavew a little different from |\-
%    \begin{macrocod
\declare@shorthand{polish}{"-}{\nobreak-\bbl@allowhyphen
\declare@shorthand{polish}{""}{\hskip\z@ski
%    \end{macrocod
%    And we want to have a shorthand for disabling a ligatur
%    \begin{macrocod
\declare@shorthand{polish}{"|}
  \textormath{\discretionary{-}{}{\kern.03em}}{
%    \end{macrocod


%  \begin{macro}{\mdqo
%  \begin{macro}{\mdqof
%    All that's left to do now is to  define a couple of comman
%    for reasons of compatibility with \file{polish.tex
% \changes{polish-1.2f}{1998/06/07}{Now use \cs{shorthandon} a
%    \cs{shorthandoff}
%    \begin{macrocod
\def\mdqon{\shorthandon{"
\def\mdqoff{\shorthandoff{"
%    \end{macrocod
%  \end{macr
%  \end{macr

%    The macro |\ldf@finish| takes care of looking for
%    configuration file, setting the main language to be switched
%    at |\begin{document}| and resetting the category code
%    \texttt{@} to its original valu
% \changes{polish-1.2d}{1996/11/03}{Now use \cs{ldf@finish} to wr
%    up
%    \begin{macrocod
\ldf@finish{polis
%</cod
%    \end{macrocod

% \Fina

%% \CharacterTab
%%  {Upper-case    \A\B\C\D\E\F\G\H\I\J\K\L\M\N\O\P\Q\R\S\T\U\V\W\X\Y
%%   Lower-case    \a\b\c\d\e\f\g\h\i\j\k\l\m\n\o\p\q\r\s\t\u\v\w\x\y
%%   Digits        \0\1\2\3\4\5\6\7\8
%%   Exclamation   \!     Double quote  \"     Hash (number)
%%   Dollar        \$     Percent       \%     Ampersand
%%   Acute accent  \'     Left paren    \(     Right paren
%%   Asterisk      \*     Plus          \+     Comma
%%   Minus         \-     Point         \.     Solidus
%%   Colon         \:     Semicolon     \;     Less than
%%   Equals        \=     Greater than  \>     Question mark
%%   Commercial at \@     Left bracket  \[     Backslash
%%   Right bracket \]     Circumflex    \^     Underscore
%%   Grave accent  \`     Left brace    \{     Vertical bar
%%   Right brace   \}     Tilde         \

\endinp
}
\DeclareOption{portuges}{% \iffalse meta-comment
%
% Copyright 1989-2008 Johannes L. Braams and any individual authors
% listed elsewhere in this file.  All rights reserved.
% 
% This file is part of the Babel system.
% --------------------------------------
% 
% It may be distributed and/or modified under the
% conditions of the LaTeX Project Public License, either version 1.3
% of this license or (at your option) any later version.
% The latest version of this license is in
%   http://www.latex-project.org/lppl.txt
% and version 1.3 or later is part of all distributions of LaTeX
% version 2003/12/01 or later.
% 
% This work has the LPPL maintenance status "maintained".
% 
% The Current Maintainer of this work is Johannes Braams.
% 
% The list of all files belonging to the Babel system is
% given in the file `manifest.bbl. See also `legal.bbl' for additional
% information.
% 
% The list of derived (unpacked) files belonging to the distribution
% and covered by LPPL is defined by the unpacking scripts (with
% extension .ins) which are part of the distribution.
% \fi
% \CheckSum{320}
% \iffalse
%    Tell the \LaTeX\ system who we are and write an entry on the
%    transcript.
%<*dtx>
\ProvidesFile{portuges.dtx}
%</dtx>
%<code>\ProvidesLanguage{portuges}
%\fi
%\ProvidesFile{portuges.dtx}
        [2008/03/18 v1.2q Portuguese support from the babel system]
%\iffalse
%% File `portuges.dtx'
%% Babel package for LaTeX version 2e
%% Copyright (C) 1989 - 2008
%%           by Johannes Braams, TeXniek
%
%% Portuguese Language Definition File
%% Copyright (C) 1989 - 2008
%%           by Johannes Braams, TeXniek
%
%% Please report errors to: J.L. Braams
%%                          babel at braams.cistron.nl
%
%    This file is part of the babel system, it provides the source
%    code for the Portuguese language definition file.  The Portuguese
%    words were contributed by Jose Pedro Ramalhete, (JRAMALHE@CERNVM
%    or Jose-Pedro_Ramalhete@MACMAIL).
%
%    Arnaldo Viegas de Lima <arnaldo@VNET.IBM.COM> contributed
%    brasilian translations and suggestions for enhancements.
%<*filedriver>
\documentclass{ltxdoc}
\newcommand*\TeXhax{\TeX hax}
\newcommand*\babel{\textsf{babel}}
\newcommand*\langvar{$\langle \it lang \rangle$}
\newcommand*\note[1]{}
\newcommand*\Lopt[1]{\textsf{#1}}
\newcommand*\file[1]{\texttt{#1}}
\begin{document}
 \DocInput{portuges.dtx}
\end{document}
%</filedriver>
%\fi
%
% \GetFileInfo{portuges.dtx}
%
% \changes{portuges-1.0a}{1991/07/15}{Renamed \file{babel.sty} in
%    \file{babel.com}}
% \changes{portuges-1.1}{1992/02/16}{Brought up-to-date with babel 3.2a}
% \changes{portuges-1.2}{1994/02/26}{Update for \LaTeXe}
% \changes{portuges-1.2d}{1994/06/26}{Removed the use of \cs{filedate}
%    and moved identification after the loading of \file{babel.def}}
% \changes{portuges-1.2g}{1995/06/04}{Enhanced support for brasilian}
% \changes{portuges-1.2j}{1996/07/11}{Replaced \cs{undefined} with
%    \cs{@undefined} and \cs{empty} with \cs{@empty} for consistency
%    with \LaTeX} 
% \changes{portuges-1.2j}{1996/10/10}{Moved the definition of
%    \cs{atcatcode} right to the beginning.}
%
%  \section{The Portuguese language}
%
%    The file \file{\filename}\footnote{The file described in this
%    section has version number \fileversion\ and was last revised on
%    \filedate.  Contributions were made by Jose Pedro Ramalhete
%    (\texttt{JRAMALHE@CERNVM} or
%    \texttt{Jose-Pedro\_Ramalhete@MACMAIL}) and Arnaldo Viegas de
%    Lima \texttt{arnaldo@VNET.IBM.COM}.}  defines all the
%    language-specific macros for the Portuguese language as well as
%    for the Brasilian version of this language.
%
%    For this language the character |"| is made active. In
%    table~\ref{tab:port-quote} an overview is given of its purpose.
%
%    \begin{table}[htb]
%     \centering
%     \begin{tabular}{lp{8cm}}
%       \verb="|= & disable ligature at this position.\\
%        |"-| & an explicit hyphen sign, allowing hyphenation
%               in the rest of the word.\\
%        |""| & like \verb="-=, but producing no hyphen sign (for
%              words that should break at some sign such as
%              ``entrada/salida.''\\
%        |"<| & for French left double quotes (similar to $<<$).\\
%        |">| & for French right double quotes (similar to $>>$).\\
%        |\-| & like the old |\-|, but allowing hyphenation
%               in the rest of the word. \\
%     \end{tabular}
%     \caption{The extra definitions made by \file{portuges.ldf}}
%     \label{tab:port-quote}
%    \end{table}
%
% \StopEventually{}
%
%    The macro |\LdfInit| takes care of preventing that this file is
%    loaded more than once, checking the category code of the
%    \texttt{@} sign, etc.
% \changes{portuges-1.2j}{1996/11/03}{Now use \cs{LdfInit} to perform
%    initial checks} 
%    \begin{macrocode}
%<*code>
\LdfInit\CurrentOption{captions\CurrentOption}
%    \end{macrocode}
%
%    When this file is read as an option, i.e. by the |\usepackage|
%    command, \texttt{portuges} will be an `unknown' language in which
%    case we have to make it known. So we check for the existence of
%    |\l@portuges| to see whether we have to do something here. Since
%    it is possible to load this file with any of the following four
%    options to babel: \Lopt{portuges}, \Lopt{portuguese},
%    \Lopt{brazil} and \Lopt{brazilian} we also allow that the
%    hyphenation patterns are loaded under any of these four names. We
%    just have to find out which one was used.
%
% \changes{portuges-1.0b}{1991/10/29}{Removed use of cs{@ifundefined}}
% \changes{portuges-1.1}{1992/02/16}{Added a warning when no
%    hyphenation patterns were loaded.}
% \changes{portuges-1.2d}{1994/06/26}{Now use \cs{@nopatterns} to
%    produce the warning}
%    \begin{macrocode}
\ifx\l@portuges\@undefined
  \ifx\l@portuguese\@undefined
    \ifx\l@brazil\@undefined
      \ifx\l@brazilian\@undefined
        \@nopatterns{Portuguese}
        \adddialect\l@portuges0
      \else
        \let\l@portuges\l@brazilian
      \fi
    \else
      \let\l@portuges\l@brazil
    \fi
  \else
    \let\l@portuges\l@portuguese
  \fi
\fi
%    \end{macrocode}
%    By now |\l@portuges| is defined. When the language definition
%    file was loaded under a different name we make sure that the
%    hyphenation patterns can be found.
%    \begin{macrocode}
\expandafter\ifx\csname l@\CurrentOption\endcsname\relax
  \expandafter\let\csname l@\CurrentOption\endcsname\l@portuges
\fi
%    \end{macrocode}
%
%    Now we have to decide whether this language definition file was
%    loaded for Portuguese or Brasilian use. This can be done by
%    checking the contents of |\CurrentOption|. When it doesn't
%    contain either `portuges' or `portuguese' we make |\bbl@tempb|
%    empty. 
%    \begin{macrocode}
\def\bbl@tempa{portuguese}
\ifx\CurrentOption\bbl@tempa
  \let\bbl@tempb\@empty
\else
  \def\bbl@tempa{portuges}
  \ifx\CurrentOption\bbl@tempa
    \let\bbl@tempb\@empty
  \else
    \def\bbl@tempb{brazil}
  \fi
\fi
\ifx\bbl@tempb\@empty
%    \end{macrocode}
%
%    The next step consists of defining commands to switch to (and from)
%    the Portuguese language.
%
% \begin{macro}{\captionsportuges}
%    The macro |\captionsportuges| defines all strings used
%    in the four standard documentclasses provided with \LaTeX.
% \changes{portuges-1.1}{1992/02/16}{Added \cs{seename}, \cs{alsoname}
%    and \cs{prefacename}}
% \changes{portuges-1.1}{1993/07/15}{\cs{headpagename} should be
%    \cs{pagename}}
% \changes{portuges-1.2e}{1994/11/09}{Added a few missing
%    translations}
% \changes{portuges-1.2h}{1995/07/04}{Added \cs{proofname} for
%    AMS-\LaTeX}
% \changes{portuges-1.2i}{1995/11/25}{Substituted `Prova' for `Proof'}
%    \begin{macrocode}
  \@namedef{captions\CurrentOption}{%
    \def\prefacename{Pref\'acio}%
    \def\refname{Refer\^encias}%
    \def\abstractname{Resumo}%
    \def\bibname{Bibliografia}%
    \def\chaptername{Cap\'{\i}tulo}%
    \def\appendixname{Ap\^endice}%
%    \end{macrocode}
%    Some discussion took place around the correct translations for
%    `Table of Contents' and `Index'. the translations differ for
%    Portuguese and Brasilian based the following history:
%    \begin{quote}
%      The whole issue is that some books without a real index at the
%      end misused the term `\'Indice' as table of contents. Then,
%      what happens is that some books apeared with `\'Indice' at the
%      begining and a `\'Indice Remissivo' at the end. Remissivo is a
%      redundant word in this case, but was introduced to make up the
%      difference. So in Brasil people started using `Sum\'ario' and
%      `\'Indice Remissivo'. In Portugal this seems not to be very
%      common, therefore we chose `\'Indice' instead of `\'Indice
%      Remissivo'.
%    \end{quote}
%    \begin{macrocode}
    \def\contentsname{Conte\'udo}%
    \def\listfigurename{Lista de Figuras}%
    \def\listtablename{Lista de Tabelas}%
    \def\indexname{\'Indice}%
    \def\figurename{Figura}%
    \def\tablename{Tabela}%
    \def\partname{Parte}%
    \def\enclname{Anexo}%
    \def\ccname{Com c\'opia a}%
    \def\headtoname{Para}%
    \def\pagename{P\'agina}%
    \def\seename{ver}%
    \def\alsoname{ver tamb\'em}%
%    \end{macrocode}
%    An alternate term for `Proof' could be `Prova'.
% \changes{portuges-1.2m}{2000/09/20}{Added \cs{glossaryname}}
% \changes{portuges-1.2p}{2003/05/23}{Substituted `Gloss\'ario' for
%    `Glossary'}
%    \begin{macrocode}
    \def\proofname{Demonstra\c{c}\~ao}%
    \def\glossaryname{Gloss\'ario}%
    }
%    \end{macrocode}
% \end{macro}
%
% \begin{macro}{\dateportuges}
%    The macro |\dateportuges| redefines the command |\today| to
%    produce Portuguese dates.
% \changes{portuges-1.2k}{1997/10/01}{Use \cs{edef} to define
%    \cs{today} to save memory}
% \changes{portuges-1.2k}{1998/03/28}{use \cs{def} instead of
%    \cs{edef}} 
% \changes{portuges-1.2n}{2001/01/27}{Removed spurious space after
%    Dezembro}
%    \begin{macrocode}
  \@namedef{date\CurrentOption}{%
    \def\today{\number\day\space de\space\ifcase\month\or
      Janeiro\or Fevereiro\or Mar\c{c}o\or Abril\or Maio\or Junho\or
      Julho\or Agosto\or Setembro\or Outubro\or Novembro\or Dezembro%
      \fi
      \space de\space\number\year}}
\else
%    \end{macrocode}
% \end{macro}
%
%    For the Brasilian version of these definitions we just add a
%    ``dialect''. 
%    \begin{macrocode}
  \expandafter
    \adddialect\csname l@\CurrentOption\endcsname\l@portuges
%    \end{macrocode}
%
% \begin{macro}{\captionsbrazil}
% \changes{portuges-1.2g}{1995/06/04}{The captions for brasilian and
%    portuguese are different now}
%
%    The ``captions'' are different for both versions of the language,
%    so we define the macro |\captionsbrazil| here.
% \changes{portuges-1.2i}{1995/11/25}{Added \cs{proofname} for
%    AMS-\LaTeX}
% \changes{portuges-1.2m}{2000/09/20}{Added \cs{glossaryname}}
% \changes{portuges-1.2q}{2008/03/18}{Substituted `Gloss\'ario' for
%    `Glossary'}
%    \begin{macrocode}
  \@namedef{captions\CurrentOption}{%
    \def\prefacename{Pref\'acio}%
    \def\refname{Refer\^encias}%
    \def\abstractname{Resumo}%
    \def\bibname{Refer\^encias Bibliogr\'aficas}%
    \def\chaptername{Cap\'{\i}tulo}%
    \def\appendixname{Ap\^endice}%
    \def\contentsname{Sum\'ario}%
    \def\listfigurename{Lista de Figuras}%
    \def\listtablename{Lista de Tabelas}%
    \def\indexname{\'Indice Remissivo}%
    \def\figurename{Figura}%
    \def\tablename{Tabela}%
    \def\partname{Parte}%
    \def\enclname{Anexo}%
    \def\ccname{C\'opia para}%
    \def\headtoname{Para}%
    \def\pagename{P\'agina}%
    \def\seename{veja}%
    \def\alsoname{veja tamb\'em}%
    \def\proofname{Demonstra\c{c}\~ao}%
    \def\glossaryname{Gloss\'ario}%
    }
%    \end{macrocode}
% \end{macro}
%
% \begin{macro}{\datebrazil}
%    The macro |\datebrazil| redefines the command
%    |\today| to produce Brasilian dates, for which the names
%    of the months are not capitalized.
% \changes{portuges-1.2k}{1997/10/01}{Use \cs{edef} to define
%    \cs{today} to save memory}
% \changes{portuges-1.2k}{1998/03/28}{use \cs{def} instead of
%    \cs{edef}} 
% \changes{portuges-1.2n}{2001/01/27}{Removed spurious space after
%    dezembro}
%    \begin{macrocode}
  \@namedef{date\CurrentOption}{%
    \def\today{\number\day\space de\space\ifcase\month\or
      janeiro\or fevereiro\or mar\c{c}o\or abril\or maio\or junho\or
      julho\or agosto\or setembro\or outubro\or novembro\or dezembro%
      \fi
      \space de\space\number\year}}
\fi
%    \end{macrocode}
% \end{macro}
%
%  \begin{macro}{\portugeshyphenmins}
% \changes{portuges-1.2g}{1995/06/04}{Added setting of hyphenmin
%    values}
%    Set correct values for |\lefthyphenmin| and |\righthyphenmin|.
% \changes{portuges-1.2m}{2000/09/22}{Now use \cs{providehyphenmins} to
%    provide a default value}
% \changes{portuges-1.2o}{2001/02/16}{Set \cs{righthyphenmin} to 3 if
%    not provided by the pattern file.}
%    \begin{macrocode}
\providehyphenmins{\CurrentOption}{\tw@\thr@@}
%    \end{macrocode}
%  \end{macro}
%
% \begin{macro}{\extrasportuges}
% \changes{portuges-1.2g}{1995/06/04}{Added the definition of some
%    \texttt{"} shorthands}
% \begin{macro}{\noextrasportuges}
%    The macro |\extrasportuges| will perform all the extra
%    definitions needed for the Portuguese language. The macro
%    |\noextrasportuges| is used to cancel the actions of
%    |\extrasportuges|.
%
%    For Portuguese the \texttt{"} character is made active. This is
%    done once, later on its definition may vary. Other languages in
%    the same document may also use the \texttt{"} character for
%    shorthands; we specify that the portuguese group of shorthands
%    should be used.
%
%    \begin{macrocode}
\initiate@active@char{"}
\@namedef{extras\CurrentOption}{\languageshorthands{portuges}}
\expandafter\addto\csname extras\CurrentOption\endcsname{%
  \bbl@activate{"}}
%    \end{macrocode}
%    Don't forget to turn the shorthands off again.
% \changes{portuges-1.2m}{1999/12/17}{Deactivate shorthands ouside of
%    Basque}
%    \begin{macrocode}
\addto\noextrasportuges{\bbl@deactivate{"}}
%    \end{macrocode}
%    First we define access to the guillemets for quotations,
% \changes{portuges-1.2k}{1997/04/03}{Removed empty groups after
%    guillemot characters}
%    \begin{macrocode}
\declare@shorthand{portuges}{"<}{%
  \textormath{\guillemotleft}{\mbox{\guillemotleft}}}
\declare@shorthand{portuges}{">}{%
  \textormath{\guillemotright}{\mbox{\guillemotright}}}
%    \end{macrocode}
%    then we define two shorthands to be able to specify hyphenation
%    breakpoints that behave a little different from |\-|.
%    \begin{macrocode}
\declare@shorthand{portuges}{"-}{\nobreak-\bbl@allowhyphens}
\declare@shorthand{portuges}{""}{\hskip\z@skip}
%    \end{macrocode}
%    And we want to have a shorthand for disabling a ligature.
%    \begin{macrocode}
\declare@shorthand{portuges}{"|}{%
  \textormath{\discretionary{-}{}{\kern.03em}}{}}
%    \end{macrocode}
% \end{macro}
% \end{macro}
%
%  \begin{macro}{\-}
%
%    All that is left now is the redefinition of |\-|. The new version
%    of |\-| should indicate an extra hyphenation position, while
%    allowing other hyphenation positions to be generated
%    automatically. The standard behaviour of \TeX\ in this respect is
%    very unfortunate for languages such as Dutch and German, where
%    long compound words are quite normal and all one needs is a means
%    to indicate an extra hyphenation position on top of the ones that
%    \TeX\ can generate from the hyphenation patterns.
%    \begin{macrocode}
\expandafter\addto\csname extras\CurrentOption\endcsname{%
  \babel@save\-}
\expandafter\addto\csname extras\CurrentOption\endcsname{%
  \def\-{\allowhyphens\discretionary{-}{}{}\allowhyphens}}
%    \end{macrocode}
%  \end{macro}
%
%  \begin{macro}{\ord}
% \changes{portuges-1.2g}{1995/06/04}{Added macro}
%  \begin{macro}{\ro}
% \changes{portuges-1.2g}{1995/06/04}{Added macro}
%  \begin{macro}{\orda}
% \changes{portuges-1.2g}{1995/06/04}{Added macro}
%  \begin{macro}{\ra}
% \changes{portuges-1.2g}{1995/06/04}{Added macro}
%    We also provide an easy way to typeset ordinals, both in the male
%    (|\ord| or |\ro|) and the female (|orda| or |\ra|) form.
%    \begin{macrocode}
\def\ord{$^{\rm o}$}
\def\orda{$^{\rm a}$}
\let\ro\ord\let\ra\orda
%    \end{macrocode}
%  \end{macro}
%  \end{macro}
%  \end{macro}
%  \end{macro}
%
%    The macro |\ldf@finish| takes care of looking for a
%    configuration file, setting the main language to be switched on
%    at |\begin{document}| and resetting the category code of
%    \texttt{@} to its original value.
% \changes{portuges-1.2j}{1996/11/03}{ow use \cs{ldf@finish} to wrap
%    up} 
%    \begin{macrocode}
\ldf@finish\CurrentOption
%</code>
%    \end{macrocode}
%
% \Finale
%%
%% \CharacterTable
%%  {Upper-case    \A\B\C\D\E\F\G\H\I\J\K\L\M\N\O\P\Q\R\S\T\U\V\W\X\Y\Z
%%   Lower-case    \a\b\c\d\e\f\g\h\i\j\k\l\m\n\o\p\q\r\s\t\u\v\w\x\y\z
%%   Digits        \0\1\2\3\4\5\6\7\8\9
%%   Exclamation   \!     Double quote  \"     Hash (number) \#
%%   Dollar        \$     Percent       \%     Ampersand     \&
%%   Acute accent  \'     Left paren    \(     Right paren   \)
%%   Asterisk      \*     Plus          \+     Comma         \,
%%   Minus         \-     Point         \.     Solidus       \/
%%   Colon         \:     Semicolon     \;     Less than     \<
%%   Equals        \=     Greater than  \>     Question mark \?
%%   Commercial at \@     Left bracket  \[     Backslash     \\
%%   Right bracket \]     Circumflex    \^     Underscore    \_
%%   Grave accent  \`     Left brace    \{     Vertical bar  \|
%%   Right brace   \}     Tilde         \~}
%%
\endinput
}
\DeclareOption{portuguese}{% \iffalse meta-comment
%
% Copyright 1989-2008 Johannes L. Braams and any individual authors
% listed elsewhere in this file.  All rights reserved.
% 
% This file is part of the Babel system.
% --------------------------------------
% 
% It may be distributed and/or modified under the
% conditions of the LaTeX Project Public License, either version 1.3
% of this license or (at your option) any later version.
% The latest version of this license is in
%   http://www.latex-project.org/lppl.txt
% and version 1.3 or later is part of all distributions of LaTeX
% version 2003/12/01 or later.
% 
% This work has the LPPL maintenance status "maintained".
% 
% The Current Maintainer of this work is Johannes Braams.
% 
% The list of all files belonging to the Babel system is
% given in the file `manifest.bbl. See also `legal.bbl' for additional
% information.
% 
% The list of derived (unpacked) files belonging to the distribution
% and covered by LPPL is defined by the unpacking scripts (with
% extension .ins) which are part of the distribution.
% \fi
% \CheckSum{320}
% \iffalse
%    Tell the \LaTeX\ system who we are and write an entry on the
%    transcript.
%<*dtx>
\ProvidesFile{portuges.dtx}
%</dtx>
%<code>\ProvidesLanguage{portuges}
%\fi
%\ProvidesFile{portuges.dtx}
        [2008/03/18 v1.2q Portuguese support from the babel system]
%\iffalse
%% File `portuges.dtx'
%% Babel package for LaTeX version 2e
%% Copyright (C) 1989 - 2008
%%           by Johannes Braams, TeXniek
%
%% Portuguese Language Definition File
%% Copyright (C) 1989 - 2008
%%           by Johannes Braams, TeXniek
%
%% Please report errors to: J.L. Braams
%%                          babel at braams.cistron.nl
%
%    This file is part of the babel system, it provides the source
%    code for the Portuguese language definition file.  The Portuguese
%    words were contributed by Jose Pedro Ramalhete, (JRAMALHE@CERNVM
%    or Jose-Pedro_Ramalhete@MACMAIL).
%
%    Arnaldo Viegas de Lima <arnaldo@VNET.IBM.COM> contributed
%    brasilian translations and suggestions for enhancements.
%<*filedriver>
\documentclass{ltxdoc}
\newcommand*\TeXhax{\TeX hax}
\newcommand*\babel{\textsf{babel}}
\newcommand*\langvar{$\langle \it lang \rangle$}
\newcommand*\note[1]{}
\newcommand*\Lopt[1]{\textsf{#1}}
\newcommand*\file[1]{\texttt{#1}}
\begin{document}
 \DocInput{portuges.dtx}
\end{document}
%</filedriver>
%\fi
%
% \GetFileInfo{portuges.dtx}
%
% \changes{portuges-1.0a}{1991/07/15}{Renamed \file{babel.sty} in
%    \file{babel.com}}
% \changes{portuges-1.1}{1992/02/16}{Brought up-to-date with babel 3.2a}
% \changes{portuges-1.2}{1994/02/26}{Update for \LaTeXe}
% \changes{portuges-1.2d}{1994/06/26}{Removed the use of \cs{filedate}
%    and moved identification after the loading of \file{babel.def}}
% \changes{portuges-1.2g}{1995/06/04}{Enhanced support for brasilian}
% \changes{portuges-1.2j}{1996/07/11}{Replaced \cs{undefined} with
%    \cs{@undefined} and \cs{empty} with \cs{@empty} for consistency
%    with \LaTeX} 
% \changes{portuges-1.2j}{1996/10/10}{Moved the definition of
%    \cs{atcatcode} right to the beginning.}
%
%  \section{The Portuguese language}
%
%    The file \file{\filename}\footnote{The file described in this
%    section has version number \fileversion\ and was last revised on
%    \filedate.  Contributions were made by Jose Pedro Ramalhete
%    (\texttt{JRAMALHE@CERNVM} or
%    \texttt{Jose-Pedro\_Ramalhete@MACMAIL}) and Arnaldo Viegas de
%    Lima \texttt{arnaldo@VNET.IBM.COM}.}  defines all the
%    language-specific macros for the Portuguese language as well as
%    for the Brasilian version of this language.
%
%    For this language the character |"| is made active. In
%    table~\ref{tab:port-quote} an overview is given of its purpose.
%
%    \begin{table}[htb]
%     \centering
%     \begin{tabular}{lp{8cm}}
%       \verb="|= & disable ligature at this position.\\
%        |"-| & an explicit hyphen sign, allowing hyphenation
%               in the rest of the word.\\
%        |""| & like \verb="-=, but producing no hyphen sign (for
%              words that should break at some sign such as
%              ``entrada/salida.''\\
%        |"<| & for French left double quotes (similar to $<<$).\\
%        |">| & for French right double quotes (similar to $>>$).\\
%        |\-| & like the old |\-|, but allowing hyphenation
%               in the rest of the word. \\
%     \end{tabular}
%     \caption{The extra definitions made by \file{portuges.ldf}}
%     \label{tab:port-quote}
%    \end{table}
%
% \StopEventually{}
%
%    The macro |\LdfInit| takes care of preventing that this file is
%    loaded more than once, checking the category code of the
%    \texttt{@} sign, etc.
% \changes{portuges-1.2j}{1996/11/03}{Now use \cs{LdfInit} to perform
%    initial checks} 
%    \begin{macrocode}
%<*code>
\LdfInit\CurrentOption{captions\CurrentOption}
%    \end{macrocode}
%
%    When this file is read as an option, i.e. by the |\usepackage|
%    command, \texttt{portuges} will be an `unknown' language in which
%    case we have to make it known. So we check for the existence of
%    |\l@portuges| to see whether we have to do something here. Since
%    it is possible to load this file with any of the following four
%    options to babel: \Lopt{portuges}, \Lopt{portuguese},
%    \Lopt{brazil} and \Lopt{brazilian} we also allow that the
%    hyphenation patterns are loaded under any of these four names. We
%    just have to find out which one was used.
%
% \changes{portuges-1.0b}{1991/10/29}{Removed use of cs{@ifundefined}}
% \changes{portuges-1.1}{1992/02/16}{Added a warning when no
%    hyphenation patterns were loaded.}
% \changes{portuges-1.2d}{1994/06/26}{Now use \cs{@nopatterns} to
%    produce the warning}
%    \begin{macrocode}
\ifx\l@portuges\@undefined
  \ifx\l@portuguese\@undefined
    \ifx\l@brazil\@undefined
      \ifx\l@brazilian\@undefined
        \@nopatterns{Portuguese}
        \adddialect\l@portuges0
      \else
        \let\l@portuges\l@brazilian
      \fi
    \else
      \let\l@portuges\l@brazil
    \fi
  \else
    \let\l@portuges\l@portuguese
  \fi
\fi
%    \end{macrocode}
%    By now |\l@portuges| is defined. When the language definition
%    file was loaded under a different name we make sure that the
%    hyphenation patterns can be found.
%    \begin{macrocode}
\expandafter\ifx\csname l@\CurrentOption\endcsname\relax
  \expandafter\let\csname l@\CurrentOption\endcsname\l@portuges
\fi
%    \end{macrocode}
%
%    Now we have to decide whether this language definition file was
%    loaded for Portuguese or Brasilian use. This can be done by
%    checking the contents of |\CurrentOption|. When it doesn't
%    contain either `portuges' or `portuguese' we make |\bbl@tempb|
%    empty. 
%    \begin{macrocode}
\def\bbl@tempa{portuguese}
\ifx\CurrentOption\bbl@tempa
  \let\bbl@tempb\@empty
\else
  \def\bbl@tempa{portuges}
  \ifx\CurrentOption\bbl@tempa
    \let\bbl@tempb\@empty
  \else
    \def\bbl@tempb{brazil}
  \fi
\fi
\ifx\bbl@tempb\@empty
%    \end{macrocode}
%
%    The next step consists of defining commands to switch to (and from)
%    the Portuguese language.
%
% \begin{macro}{\captionsportuges}
%    The macro |\captionsportuges| defines all strings used
%    in the four standard documentclasses provided with \LaTeX.
% \changes{portuges-1.1}{1992/02/16}{Added \cs{seename}, \cs{alsoname}
%    and \cs{prefacename}}
% \changes{portuges-1.1}{1993/07/15}{\cs{headpagename} should be
%    \cs{pagename}}
% \changes{portuges-1.2e}{1994/11/09}{Added a few missing
%    translations}
% \changes{portuges-1.2h}{1995/07/04}{Added \cs{proofname} for
%    AMS-\LaTeX}
% \changes{portuges-1.2i}{1995/11/25}{Substituted `Prova' for `Proof'}
%    \begin{macrocode}
  \@namedef{captions\CurrentOption}{%
    \def\prefacename{Pref\'acio}%
    \def\refname{Refer\^encias}%
    \def\abstractname{Resumo}%
    \def\bibname{Bibliografia}%
    \def\chaptername{Cap\'{\i}tulo}%
    \def\appendixname{Ap\^endice}%
%    \end{macrocode}
%    Some discussion took place around the correct translations for
%    `Table of Contents' and `Index'. the translations differ for
%    Portuguese and Brasilian based the following history:
%    \begin{quote}
%      The whole issue is that some books without a real index at the
%      end misused the term `\'Indice' as table of contents. Then,
%      what happens is that some books apeared with `\'Indice' at the
%      begining and a `\'Indice Remissivo' at the end. Remissivo is a
%      redundant word in this case, but was introduced to make up the
%      difference. So in Brasil people started using `Sum\'ario' and
%      `\'Indice Remissivo'. In Portugal this seems not to be very
%      common, therefore we chose `\'Indice' instead of `\'Indice
%      Remissivo'.
%    \end{quote}
%    \begin{macrocode}
    \def\contentsname{Conte\'udo}%
    \def\listfigurename{Lista de Figuras}%
    \def\listtablename{Lista de Tabelas}%
    \def\indexname{\'Indice}%
    \def\figurename{Figura}%
    \def\tablename{Tabela}%
    \def\partname{Parte}%
    \def\enclname{Anexo}%
    \def\ccname{Com c\'opia a}%
    \def\headtoname{Para}%
    \def\pagename{P\'agina}%
    \def\seename{ver}%
    \def\alsoname{ver tamb\'em}%
%    \end{macrocode}
%    An alternate term for `Proof' could be `Prova'.
% \changes{portuges-1.2m}{2000/09/20}{Added \cs{glossaryname}}
% \changes{portuges-1.2p}{2003/05/23}{Substituted `Gloss\'ario' for
%    `Glossary'}
%    \begin{macrocode}
    \def\proofname{Demonstra\c{c}\~ao}%
    \def\glossaryname{Gloss\'ario}%
    }
%    \end{macrocode}
% \end{macro}
%
% \begin{macro}{\dateportuges}
%    The macro |\dateportuges| redefines the command |\today| to
%    produce Portuguese dates.
% \changes{portuges-1.2k}{1997/10/01}{Use \cs{edef} to define
%    \cs{today} to save memory}
% \changes{portuges-1.2k}{1998/03/28}{use \cs{def} instead of
%    \cs{edef}} 
% \changes{portuges-1.2n}{2001/01/27}{Removed spurious space after
%    Dezembro}
%    \begin{macrocode}
  \@namedef{date\CurrentOption}{%
    \def\today{\number\day\space de\space\ifcase\month\or
      Janeiro\or Fevereiro\or Mar\c{c}o\or Abril\or Maio\or Junho\or
      Julho\or Agosto\or Setembro\or Outubro\or Novembro\or Dezembro%
      \fi
      \space de\space\number\year}}
\else
%    \end{macrocode}
% \end{macro}
%
%    For the Brasilian version of these definitions we just add a
%    ``dialect''. 
%    \begin{macrocode}
  \expandafter
    \adddialect\csname l@\CurrentOption\endcsname\l@portuges
%    \end{macrocode}
%
% \begin{macro}{\captionsbrazil}
% \changes{portuges-1.2g}{1995/06/04}{The captions for brasilian and
%    portuguese are different now}
%
%    The ``captions'' are different for both versions of the language,
%    so we define the macro |\captionsbrazil| here.
% \changes{portuges-1.2i}{1995/11/25}{Added \cs{proofname} for
%    AMS-\LaTeX}
% \changes{portuges-1.2m}{2000/09/20}{Added \cs{glossaryname}}
% \changes{portuges-1.2q}{2008/03/18}{Substituted `Gloss\'ario' for
%    `Glossary'}
%    \begin{macrocode}
  \@namedef{captions\CurrentOption}{%
    \def\prefacename{Pref\'acio}%
    \def\refname{Refer\^encias}%
    \def\abstractname{Resumo}%
    \def\bibname{Refer\^encias Bibliogr\'aficas}%
    \def\chaptername{Cap\'{\i}tulo}%
    \def\appendixname{Ap\^endice}%
    \def\contentsname{Sum\'ario}%
    \def\listfigurename{Lista de Figuras}%
    \def\listtablename{Lista de Tabelas}%
    \def\indexname{\'Indice Remissivo}%
    \def\figurename{Figura}%
    \def\tablename{Tabela}%
    \def\partname{Parte}%
    \def\enclname{Anexo}%
    \def\ccname{C\'opia para}%
    \def\headtoname{Para}%
    \def\pagename{P\'agina}%
    \def\seename{veja}%
    \def\alsoname{veja tamb\'em}%
    \def\proofname{Demonstra\c{c}\~ao}%
    \def\glossaryname{Gloss\'ario}%
    }
%    \end{macrocode}
% \end{macro}
%
% \begin{macro}{\datebrazil}
%    The macro |\datebrazil| redefines the command
%    |\today| to produce Brasilian dates, for which the names
%    of the months are not capitalized.
% \changes{portuges-1.2k}{1997/10/01}{Use \cs{edef} to define
%    \cs{today} to save memory}
% \changes{portuges-1.2k}{1998/03/28}{use \cs{def} instead of
%    \cs{edef}} 
% \changes{portuges-1.2n}{2001/01/27}{Removed spurious space after
%    dezembro}
%    \begin{macrocode}
  \@namedef{date\CurrentOption}{%
    \def\today{\number\day\space de\space\ifcase\month\or
      janeiro\or fevereiro\or mar\c{c}o\or abril\or maio\or junho\or
      julho\or agosto\or setembro\or outubro\or novembro\or dezembro%
      \fi
      \space de\space\number\year}}
\fi
%    \end{macrocode}
% \end{macro}
%
%  \begin{macro}{\portugeshyphenmins}
% \changes{portuges-1.2g}{1995/06/04}{Added setting of hyphenmin
%    values}
%    Set correct values for |\lefthyphenmin| and |\righthyphenmin|.
% \changes{portuges-1.2m}{2000/09/22}{Now use \cs{providehyphenmins} to
%    provide a default value}
% \changes{portuges-1.2o}{2001/02/16}{Set \cs{righthyphenmin} to 3 if
%    not provided by the pattern file.}
%    \begin{macrocode}
\providehyphenmins{\CurrentOption}{\tw@\thr@@}
%    \end{macrocode}
%  \end{macro}
%
% \begin{macro}{\extrasportuges}
% \changes{portuges-1.2g}{1995/06/04}{Added the definition of some
%    \texttt{"} shorthands}
% \begin{macro}{\noextrasportuges}
%    The macro |\extrasportuges| will perform all the extra
%    definitions needed for the Portuguese language. The macro
%    |\noextrasportuges| is used to cancel the actions of
%    |\extrasportuges|.
%
%    For Portuguese the \texttt{"} character is made active. This is
%    done once, later on its definition may vary. Other languages in
%    the same document may also use the \texttt{"} character for
%    shorthands; we specify that the portuguese group of shorthands
%    should be used.
%
%    \begin{macrocode}
\initiate@active@char{"}
\@namedef{extras\CurrentOption}{\languageshorthands{portuges}}
\expandafter\addto\csname extras\CurrentOption\endcsname{%
  \bbl@activate{"}}
%    \end{macrocode}
%    Don't forget to turn the shorthands off again.
% \changes{portuges-1.2m}{1999/12/17}{Deactivate shorthands ouside of
%    Basque}
%    \begin{macrocode}
\addto\noextrasportuges{\bbl@deactivate{"}}
%    \end{macrocode}
%    First we define access to the guillemets for quotations,
% \changes{portuges-1.2k}{1997/04/03}{Removed empty groups after
%    guillemot characters}
%    \begin{macrocode}
\declare@shorthand{portuges}{"<}{%
  \textormath{\guillemotleft}{\mbox{\guillemotleft}}}
\declare@shorthand{portuges}{">}{%
  \textormath{\guillemotright}{\mbox{\guillemotright}}}
%    \end{macrocode}
%    then we define two shorthands to be able to specify hyphenation
%    breakpoints that behave a little different from |\-|.
%    \begin{macrocode}
\declare@shorthand{portuges}{"-}{\nobreak-\bbl@allowhyphens}
\declare@shorthand{portuges}{""}{\hskip\z@skip}
%    \end{macrocode}
%    And we want to have a shorthand for disabling a ligature.
%    \begin{macrocode}
\declare@shorthand{portuges}{"|}{%
  \textormath{\discretionary{-}{}{\kern.03em}}{}}
%    \end{macrocode}
% \end{macro}
% \end{macro}
%
%  \begin{macro}{\-}
%
%    All that is left now is the redefinition of |\-|. The new version
%    of |\-| should indicate an extra hyphenation position, while
%    allowing other hyphenation positions to be generated
%    automatically. The standard behaviour of \TeX\ in this respect is
%    very unfortunate for languages such as Dutch and German, where
%    long compound words are quite normal and all one needs is a means
%    to indicate an extra hyphenation position on top of the ones that
%    \TeX\ can generate from the hyphenation patterns.
%    \begin{macrocode}
\expandafter\addto\csname extras\CurrentOption\endcsname{%
  \babel@save\-}
\expandafter\addto\csname extras\CurrentOption\endcsname{%
  \def\-{\allowhyphens\discretionary{-}{}{}\allowhyphens}}
%    \end{macrocode}
%  \end{macro}
%
%  \begin{macro}{\ord}
% \changes{portuges-1.2g}{1995/06/04}{Added macro}
%  \begin{macro}{\ro}
% \changes{portuges-1.2g}{1995/06/04}{Added macro}
%  \begin{macro}{\orda}
% \changes{portuges-1.2g}{1995/06/04}{Added macro}
%  \begin{macro}{\ra}
% \changes{portuges-1.2g}{1995/06/04}{Added macro}
%    We also provide an easy way to typeset ordinals, both in the male
%    (|\ord| or |\ro|) and the female (|orda| or |\ra|) form.
%    \begin{macrocode}
\def\ord{$^{\rm o}$}
\def\orda{$^{\rm a}$}
\let\ro\ord\let\ra\orda
%    \end{macrocode}
%  \end{macro}
%  \end{macro}
%  \end{macro}
%  \end{macro}
%
%    The macro |\ldf@finish| takes care of looking for a
%    configuration file, setting the main language to be switched on
%    at |\begin{document}| and resetting the category code of
%    \texttt{@} to its original value.
% \changes{portuges-1.2j}{1996/11/03}{ow use \cs{ldf@finish} to wrap
%    up} 
%    \begin{macrocode}
\ldf@finish\CurrentOption
%</code>
%    \end{macrocode}
%
% \Finale
%%
%% \CharacterTable
%%  {Upper-case    \A\B\C\D\E\F\G\H\I\J\K\L\M\N\O\P\Q\R\S\T\U\V\W\X\Y\Z
%%   Lower-case    \a\b\c\d\e\f\g\h\i\j\k\l\m\n\o\p\q\r\s\t\u\v\w\x\y\z
%%   Digits        \0\1\2\3\4\5\6\7\8\9
%%   Exclamation   \!     Double quote  \"     Hash (number) \#
%%   Dollar        \$     Percent       \%     Ampersand     \&
%%   Acute accent  \'     Left paren    \(     Right paren   \)
%%   Asterisk      \*     Plus          \+     Comma         \,
%%   Minus         \-     Point         \.     Solidus       \/
%%   Colon         \:     Semicolon     \;     Less than     \<
%%   Equals        \=     Greater than  \>     Question mark \?
%%   Commercial at \@     Left bracket  \[     Backslash     \\
%%   Right bracket \]     Circumflex    \^     Underscore    \_
%%   Grave accent  \`     Left brace    \{     Vertical bar  \|
%%   Right brace   \}     Tilde         \~}
%%
\endinput
}
\DeclareOption{romanian}{% \iffalse meta-comment
%
% Copyright 1989-2005 Johannes L. Braams and any individual authors
% listed elsewhere in this file.  All rights reserved.
% 
% This file is part of the Babel system.
% --------------------------------------
% 
% It may be distributed and/or modified under the
% conditions of the LaTeX Project Public License, either version 1.3
% of this license or (at your option) any later version.
% The latest version of this license is in
%   http://www.latex-project.org/lppl.txt
% and version 1.3 or later is part of all distributions of LaTeX
% version 2003/12/01 or later.
% 
% This work has the LPPL maintenance status "maintained".
% 
% The Current Maintainer of this work is Johannes Braams.
% 
% The list of all files belonging to the Babel system is
% given in the file `manifest.bbl. See also `legal.bbl' for additional
% information.
% 
% The list of derived (unpacked) files belonging to the distribution
% and covered by LPPL is defined by the unpacking scripts (with
% extension .ins) which are part of the distribution.
% \fi
% \CheckSum{89}
% \iffalse
%    Tell the \LaTeX\ system who we are and write an entry on the
%    transcript.
%<*dtx>
\ProvidesFile{romanian.dtx}
%</dtx>
%<code>\ProvidesLanguage{romanian}
%\fi
%\ProvidesFile{romanian.dtx}
        [2005/03/31 v1.2l Romanian support from the babel system]
%\iffalse
%% File `romanian.dtx'
%% Babel package for LaTeX version 2e
%% Copyright (C) 1989 - 2005
%%           by Johannes Braams, TeXniek
%
%% Romanian Language Definition File
%% Copyright (C) 1989 - 2005
%%           by Johannes Braams, TeXniek
%
%% Please report errors to: J.L. Braams
%%                          babel at braams.cistron.nl
%
%    This file is part of the babel system, it provides the source
%    code for the Romanian language definition file. A contribution
%    was made by Umstatter Horst (hhu at cernvm.cern.ch) and Robert
%    Juhasz (robertj at uni-paderborn.de)
%<*filedriver>
\documentclass{ltxdoc}
\newcommand*\TeXhax{\TeX hax}
\newcommand*\babel{\textsf{babel}}
\newcommand*\langvar{$\langle \it lang \rangle$}
\newcommand*\note[1]{}
\newcommand*\Lopt[1]{\textsf{#1}}
\newcommand*\file[1]{\texttt{#1}}
\begin{document}
 \DocInput{romanian.dtx}
\end{document}
%</filedriver>
%\fi
%
% \GetFileInfo{romanian.dtx}
%
% \changes{romanian-1.0a}{1991/07/15}{Renamed babel.sty in babel.com}
% \changes{romanian-1.1}{1992/02/16}{Brought up-to-date with babel 3.2a}
% \changes{romanian-1.2}{1994/02/27}{Update for LaTeX2e}
% \changes{romanian-1.2d}{1994/06/26}{Removed the use of \cs{filedate}
%    and moved identification after the loading of babel.def}
% \changes{romanian-1.2e}{1995/05/25}{Updated for babel release 3.5}
% \changes{romanian-1.2h}{1996/10/10}{Replaced \cs{undefined} with
%    \cs{@undefined} and \cs{empty} with \cs{@empty} for consistency
%    with \LaTeX, moved the definition of \cs{atcatcode} right to the
%    beginning.}
%
%  \section{The Romanian language}
%
%    The file \file{\filename}\footnote{The file described in this
%    section has version number \fileversion\ and was last revised on
%    \filedate.  A contribution was made by Umstatter Horst
%    (\texttt{hhu@cernvm.cern.ch}).}  defines all the
%    language-specific macros for the Romanian language.
%
%    For this language currently no special definitions are needed or
%    available.
%
% \StopEventually{}
%
%    The macro |\LdfInit| takes care of preventing that this file is
%    loaded more than once, checking the category code of the
%    \texttt{@} sign, etc.
% \changes{romanian-1.2h}{1996/11/03}{Now use \cs{LdfInit} to perform
%    initial checks} 
%    \begin{macrocode}
%<*code>
\LdfInit{romanian}\captionsromanian
%    \end{macrocode}
%
%    When this file is read as an option, i.e. by the |\usepackage|
%    command, \texttt{romanian} will be an `unknown' language in which
%    case we have to make it known. So we check for the existence of
%    |\l@romanian| to see whether we have to do something here.
%
% \changes{romanian-1.0b}{1991/10/29}{Removed use of
%    \cs{@ifundefined}}
% \changes{romanian-1.1}{1992/02/16}{Added a warning when no
%    hyphenation patterns were loaded.}
% \changes{romanian-1.2d}{1994/06/26}{Now use \cs{@nopatterns} to
%    produce the warning}
%    \begin{macrocode}
\ifx\l@romanian\@undefined
    \@nopatterns{Romanian}
    \adddialect\l@romanian0\fi
%    \end{macrocode}
%
%    The next step consists of defining commands to switch to (and
%    from) the Romanian language.
%
% \begin{macro}{\captionsromanian}
%    The macro |\captionsromanian| defines all strings used in the
%    four standard documentclasses provided with \LaTeX.
% \changes{romanian-1.1}{1992/02/16}{Added \cs{seename}, \cs{alsoname}
%    and \cs{prefacename}}
% \changes{romanian-1.1}{1992/02/17}{Translation errors found by
%    Robert Juhasz fixed}
% \changes{romanian-1.1}{1993/07/15}{\cs{headpagename} should be
%    \cs{pagename}}
% \changes{romanian-1.2f}{1995/07/04}{Added \cs{proofname} for
%    AMS-\LaTeX}
% \changes{romanian-1.2g}{1995/11/10}{Added translation of `Proof'}
% \changes{romanian-1.2k}{2000/09/20}{Added \cs{glossaryname}}
% \changes{romanian-1.2l}{2003/06/10}{Added translation for Glossary}
%    \begin{macrocode}ls
\addto\captionsromanian{%
  \def\prefacename{Prefa\c{t}\u{a}}%
  \def\refname{Bibliografie}%
  \def\abstractname{Rezumat}%
  \def\bibname{Bibliografie}%
  \def\chaptername{Capitolul}%
  \def\appendixname{Anexa}%
  \def\contentsname{Cuprins}%
  \def\listfigurename{List\u{a} de figuri}%
  \def\listtablename{List\u{a} de tabele}%
  \def\indexname{Glosar}%
  \def\figurename{Figura}%    % sau Plan\c{s}a
  \def\tablename{Tabela}%
  \def\partname{Partea}%
  \def\enclname{Anex\u{a}}%   % sau Anexe
  \def\ccname{Copie}%
  \def\headtoname{Pentru}%
  \def\pagename{Pagina}%
  \def\seename{Vezi}%
  \def\alsoname{Vezi de asemenea}%
  \def\proofname{Demonstra\c{t}ie} %
  \def\glossaryname{Glosar}%
  }%
%    \end{macrocode}
% \end{macro}
%
% \begin{macro}{\dateromanian}
%    The macro |\dateromanian| redefines the command |\today| to
%    produce Romanian dates.
% \changes{romanian-1.1}{1992/02/17}{Translation errors found by Robert
%    Juhasz fixed}
% \changes{romanian-1.2i}{1997/10/01}{Use \cs{edef} to define
%    \cs{today} to save memory}
% \changes{romanian-1.2i}{1998/03/28}{use \cs{def} instead of
%    \cs{edef}} 
%    \begin{macrocode}
\def\dateromanian{%
  \def\today{\number\day~\ifcase\month\or
    ianuarie\or februarie\or martie\or aprilie\or mai\or
    iunie\or iulie\or august\or septembrie\or octombrie\or
    noiembrie\or decembrie\fi
    \space \number\year}}
%    \end{macrocode}
% \end{macro}
%
% \begin{macro}{\extrasromanian}
% \begin{macro}{\noextrasromanian}
%    The macro |\extrasromanian| will perform all the extra
%    definitions needed for the Romanian language. The macro
%    |\noextrasromanian| is used to cancel the actions of
%    |\extrasromanian| For the moment these macros are empty but they
%    are defined for compatibility with the other language definition
%    files.
%
%    \begin{macrocode}
\addto\extrasromanian{}
\addto\noextrasromanian{}
%    \end{macrocode}
% \end{macro}
% \end{macro}
%
%    The macro |\ldf@finish| takes care of looking for a
%    configuration file, setting the main language to be switched on
%    at |\begin{document}| and resetting the category code of
%    \texttt{@} to its original value.
% \changes{romanian-1.2h}{1996/11/03}{Now use \cs{ldf@finish} to wrap
%    up} 
%    \begin{macrocode}
\ldf@finish{romanian}
%</code>
%    \end{macrocode}
%
% \Finale
%%
%% \CharacterTable
%%  {Upper-case    \A\B\C\D\E\F\G\H\I\J\K\L\M\N\O\P\Q\R\S\T\U\V\W\X\Y\Z
%%   Lower-case    \a\b\c\d\e\f\g\h\i\j\k\l\m\n\o\p\q\r\s\t\u\v\w\x\y\z
%%   Digits        \0\1\2\3\4\5\6\7\8\9
%%   Exclamation   \!     Double quote  \"     Hash (number) \#
%%   Dollar        \$     Percent       \%     Ampersand     \&
%%   Acute accent  \'     Left paren    \(     Right paren   \)
%%   Asterisk      \*     Plus          \+     Comma         \,
%%   Minus         \-     Point         \.     Solidus       \/
%%   Colon         \:     Semicolon     \;     Less than     \<
%%   Equals        \=     Greater than  \>     Question mark \?
%%   Commercial at \@     Left bracket  \[     Backslash     \\
%%   Right bracket \]     Circumflex    \^     Underscore    \_
%%   Grave accent  \`     Left brace    \{     Vertical bar  \|
%%   Right brace   \}     Tilde         \~}
%%
\endinput
}
\DeclareOption{russian}{% \iffalse meta-comment
%
% Copyright 1989-2008 Johannes L. Braams and any individual authors
% listed elsewhere in this file.  All rights reserved.
% 
% This file is part of the Babel system.
% --------------------------------------
% 
% It may be distributed and/or modified under the
% conditions of the LaTeX Project Public License, either version 1.3
% of this license or (at your option) any later version.
% The latest version of this license is in
%   http://www.latex-project.org/lppl.txt
% and version 1.3 or later is part of all distributions of LaTeX
% version 2003/12/01 or later.
% 
% This work has the LPPL maintenance status "maintained".
% 
% The Current Maintainer of this work is Johannes Braams.
% 
% The list of all files belonging to the Babel system is
% given in the file `manifest.bbl. See also `legal.bbl' for additional
% information.
% 
% The list of derived (unpacked) files belonging to the distribution
% and covered by LPPL is defined by the unpacking scripts (with
% extension .ins) which are part of the distribution.
% \fi
% \CheckSum{1461}
%
% \iffalse
%    Tell the \LaTeX\ system who we are and write an entry on the
%    transcript.
%<*dtx>
\ProvidesFile{russianb.dtx}
%</dtx>
%<code>\ProvidesLanguage{russianb}
        [2008/03/21 v1.1r Russian support from the babel system]
%
%% File `russianb.dtx'
%% Babel package for LaTeX version 2e
%% Copyright (C) 1989 - 2008
%%           by Johannes Braams, TeXniek
%
%% Russianb Language Definition File
%% Copyright (C) 1995 - 2008
%%           by Olga Lapko cyrtug at mir.msk.su
%%              Johannes Braams, TeXniek
%
%% Adapted to the new T2 and X2 Cyrillic encodings
%%           by Vladimir Volovich TeX at vvv.vsu.ru
%%              Werner Lemberg wl at gnu.org
%
%% Please report errors to: J.L. Braams
%%                          babel at braams dot xs4all dot nl
%
%<*filedriver>
\documentclass{ltxdoc}
\newcommand\TeXhax{\TeX hax}
\newcommand\babel{\textsf{babel}}
\newcommand\langvar{$\langle \it lang \rangle$}
\newcommand\note[1]{}
\newcommand\Lopt[1]{\textsf{#1}}
\newcommand\file[1]{\texttt{#1}}
\newcommand\pkg[1]{\texttt{#1}}
\begin{document}
 \DocInput{russianb.dtx}
\end{document}
%</filedriver>
%\fi
% \GetFileInfo{russianb.dtx}
%
% \changes{russianb-1.1c}{1996/07/11}{Replaced \cs{undefined} with
%    \cs{@undefined} and \cs{empty} with \cs{@empty} for consistency
%    with \LaTeX}
% \changes{russianb-1.1d}{1996/10/10}{Moved the definition of
%    \cs{atcatcode} right to the beginning.}
% \changes{russianb-1.1k}{1999/08/19}{replaced all \cs{penalty}\cs{@M}
%    with \cs{nobreak}}
%
%  \section{The Russian language}
%
%    The file \file{\filename}\footnote{The file described in this section
%    has version number \fileversion\ and was last revised on \filedate.
%    This file was initially derived from the original version of
%    \file{german.sty}, which has some definitions for Russian. Later the
%    definitions from \file{russian.sty} version 1.0b (for \LaTeX\ 2.09),
%    \file{russian.sty} version v2.5c (for \LaTeXe) and \file{francais.sty}
%    version 4.5c and \file{germanb.sty} version 2.5c were added.} defines
%    all the language-specific macros for the Russian language. It needs the
%    file \file{cyrcod} for success documentation with Russian encodings
%    (see below).
%
%    For this language the character |"| is made active. In
%    table~\ref{tab:russian-quote} an overview is given of its purpose.
%
% \changes{russianb-1.1f}{1998/06/26}{%
%    Added definitions of Cyrillic emdash stuff and thinspace}
%
%    \begin{table}[htb]
%      \begin{center}
%      \begin{tabular}{lp{8cm}}
%       \verb="|= & disable ligature at this position.               \\
%       |"-| & an explicit hyphen sign, allowing hyphenation
%                   in the rest of the word.                         \\
%       |"---| & Cyrillic emdash in plain text.                      \\
%       |"--~| & Cyrillic emdash in compound names (surnames).       \\
%       |"--*| & Cyrillic emdash for denoting direct speech.         \\
%       |""| & like |"-|, but producing no hyphen sign
%                   (for compund words with hyphen, e.g.\ |x-""y|
%                   or some other signs  as ``disable/enable'').     \\
%       |"~| & for a compound word mark without a breakpoint.        \\
%       |"=| & for a compound word mark with a breakpoint, allowing
%              hyphenation in the composing words.                   \\
%       |",| & thinspace for initials with a breakpoint
%               in following surname.                                \\
%       |"`| & for German left double quotes
%                   (looks like ,\kern-0.08em,).                     \\
%       |"'| & for German right double quotes (looks like ``).       \\%^^A''
%       |"<| & for French left double quotes (looks like $<\!\!<$).  \\
%       |">| & for French right double quotes (looks like $>\!\!>$). \\
%      \end{tabular}
%      \caption{The extra definitions made
%               by \file{russianb}}\label{tab:russian-quote}
%      \end{center}
%    \end{table}
%
%    The quotes in table~\ref{tab:russian-quote} can also be typeset by
%    using the commands in table~\ref{tab:rmore-quote}.
%
%    \begin{table}[htb]
%      \begin{center}
%      \begin{tabular}{lp{8cm}}
%       |\cdash---| & Cyrillic emdash in plain text.                    \\
%       |\cdash--~| & Cyrillic emdash in compound names (surnames).     \\
%       |\cdash--*| & Cyrillic emdash for denoting direct speech.       \\
%       |\glqq| & for German left double quotes
%                    (looks like ,\kern-0.08em,).                       \\
%       |\grqq| & for German right double quotes (looks like ``).       \\%^^A''
%       |\flqq| & for French left double quotes (looks like $<\!\!<$).  \\
%       |\frqq| & for French right double quotes (looks like $>\!\!>$). \\
%       |\dq|   & the original quotes character (|"|).                  \\
%      \end{tabular}
%      \caption{More commands which produce quotes, defined
%               by \babel}\label{tab:rmore-quote}
%      \end{center}
%    \end{table}
%
%    The French quotes are also available as ligatures `|<<|' and `|>>|' in
%    8-bit Cyrillic font encodings (\texttt{LCY}, \texttt{X2}, \texttt{T2*})
%    and as `|<|' and `|>|' characters in 7-bit Cyrillic font encodings
%    (\texttt{OT2} and \texttt{LWN}).
%
%    The quotation marks traditionally used in Russian were borrowed from
%    other languages (e.g., French and German) so they keep their original
%    names.
%
% \StopEventually{}
%
%    The macro |\LdfInit| takes care of preventing that this file is loaded
%    more than once, checking the category code of the \texttt{@} sign, etc.
%
% \changes{russianb-1.1d}{1996/11/03}{Now use \cs{LdfInit} to perform
%    initial checks}
% \changes{russianb-1.1e}{1996/12/29}{Added closing brace to second
%    argument of \cs{LdfInit}}
%    \begin{macrocode}
%<*code>
\LdfInit{russian}{captionsrussian}
%    \end{macrocode}
%
%    When this file is read as an option, i.e., by the |\usepackage|
%    command, \texttt{russianb} will be an `unknown' language, in which case
%    we have to make it known. So we check for the existence of |\l@russian|
%    to see whether we have to do something here.
%
%    \begin{macrocode}
\ifx\l@russian\@undefined
  \@nopatterns{Russian}
  \adddialect\l@russian0
\fi
%    \end{macrocode}
%
%  \begin{macro}{\latinencoding}
%
%    We need to know the encoding for text that is supposed to be which is
%    active at the end of the \babel\ package. If the \pkg{fontenc} package
%    is loaded later, then\ldots too bad!
%
%    \begin{macrocode}
\let\latinencoding\cf@encoding
%    \end{macrocode}
%
%  \end{macro}
%
%    The user may choose between different available Cyrillic
%    encodings---e.g., \texttt{X2}, \texttt{LCY}, or \texttt{LWN}.\@
%    Hopefully, \texttt{X2} will eventually replace the two latter encodings
%    (\texttt{LCY} and \texttt{LWN}).\@ If the user wants to use another
%    font encoding than the default (\texttt{T2A}), he has to load the
%    corresponding file \emph{before} \file{russianb.sty}. This may be done
%    in the following way:
%
%    \begin{verbatim}
%      % override the default X2 encoding used in Babel
%      \usepackage[LCY,OT1]{fontenc}
%      \usepackage[english,russian]{babel}
%    \end{verbatim}
%    \unskip
%
%    Note: for the Russian language, the \texttt{T2A} encoding is better than
%    \texttt{X2}, because \texttt{X2} does not contain Latin letters, and
%    users should be very careful to switch the language every time they
%    want to typeset a Latin word inside a Russian phrase or vice versa.
%
%    We parse the |\cdp@list| containing the encodings known to \LaTeX\ in
%    the order they were loaded. We set the |\cyrillicencoding| to the
%    \emph{last} loaded encoding in the list of supported Cyrillic
%    encodings: \texttt{OT2}, \texttt{LWN}, \texttt{LCY}, \texttt{X2},
%    \texttt{T2C}, \texttt{T2B}, \texttt{T2A}, if any.
%
%    \begin{macrocode}
\def\reserved@a#1#2{%
   \edef\reserved@b{#1}%
   \edef\reserved@c{#2}%
   \ifx\reserved@b\reserved@c
     \let\cyrillicencoding\reserved@c
   \fi}
\def\cdp@elt#1#2#3#4{%
   \reserved@a{#1}{OT2}%
   \reserved@a{#1}{LWN}%
   \reserved@a{#1}{LCY}%
   \reserved@a{#1}{X2}%
   \reserved@a{#1}{T2C}%
   \reserved@a{#1}{T2B}%
   \reserved@a{#1}{T2A}}
\cdp@list
%    \end{macrocode}
%
%    Now, if |\cyrillicencoding| is undefined, then the user did not load
%    any of supported encodings. So, we have to set |\cyrillicencoding| to
%    some default value. We test the presence of the encoding definition
%    files in the order from less preferable to more preferable encodings.
%    We use the lowercase names (i.e., \file{lcyenc.def} instead of
%    \file{LCYenc.def}).
%
%    \begin{macrocode}
\ifx\cyrillicencoding\undefined
  \IfFileExists{ot2enc.def}{\def\cyrillicencoding{OT2}}\relax
  \IfFileExists{lwnenc.def}{\def\cyrillicencoding{LWN}}\relax
  \IfFileExists{lcyenc.def}{\def\cyrillicencoding{LCY}}\relax
  \IfFileExists{x2enc.def}{\def\cyrillicencoding{X2}}\relax
  \IfFileExists{t2cenc.def}{\def\cyrillicencoding{T2C}}\relax
  \IfFileExists{t2benc.def}{\def\cyrillicencoding{T2B}}\relax
  \IfFileExists{t2aenc.def}{\def\cyrillicencoding{T2A}}\relax
%    \end{macrocode}
%
%    If |\cyrillicencoding| is still undefined, then the user seems not to
%    have a properly installed distribution. A fatal error.
%
%    \begin{macrocode}
  \ifx\cyrillicencoding\undefined
    \PackageError{babel}%
      {No Cyrillic encoding definition files were found}%
      {Your installation is incomplete.\MessageBreak
       You need at least one of the following files:\MessageBreak
       \space\space
       x2enc.def, t2aenc.def, t2benc.def, t2cenc.def,\MessageBreak
       \space\space
       lcyenc.def, lwnenc.def, ot2enc.def.}%
  \else
%    \end{macrocode}
%
%    We avoid |\usepackage[\cyrillicencoding]{fontenc}| because we don't
%    want to force the switch of |\encodingdefault|.
%
%    \begin{macrocode}
    \lowercase
      \expandafter{\expandafter\input\cyrillicencoding enc.def\relax}%
  \fi
\fi
%    \end{macrocode}
%
%    \begin{verbatim}
%      \PackageInfo{babel}
%        {Using `\cyrillicencoding' as a default Cyrillic encoding}%
%    \end{verbatim}
%    \unskip
%
%    \begin{macrocode}
\DeclareRobustCommand{\Russian}{%
  \fontencoding\cyrillicencoding\selectfont
  \let\encodingdefault\cyrillicencoding
  \expandafter\set@hyphenmins\russianhyphenmins
  \language\l@russian}%
\DeclareRobustCommand{\English}{%
  \fontencoding\latinencoding\selectfont
  \let\encodingdefault\latinencoding
  \expandafter\set@hyphenmins\englishhyphenmins
  \language\l@english}%
\let\Rus\Russian
\let\Eng\English
\let\cyrillictext\Russian
\let\cyr\Russian
%    \end{macrocode}
%
%    Since the \texttt{X2} encoding does not contain Latin letters, we
%    should make some redefinitions of \LaTeX\ macros which implicitly
%    produce Latin letters.
%
%    \begin{macrocode}
\expandafter\ifx\csname T@X2\endcsname\relax\else
%    \end{macrocode}
%
%    We put |\latinencoding| in braces to avoid problems with
%    |\@alph| inside minipages (e.g., footnotes inside minipages) where
%    |\@alph| is expanded and we get for example `|\fontencoding OT1|'
%    (|\fontencoding| is robust).
%
%    \begin{macrocode}
  \def\@alph#1{{\fontencoding{\latinencoding}\selectfont
    \ifcase#1\or
      a\or b\or c\or d\or e\or f\or g\or h\or
      i\or j\or k\or l\or m\or n\or o\or p\or
      q\or r\or s\or t\or u\or v\or w\or x\or
      y\or z\else\@ctrerr\fi}}%
  \def\@Alph#1{{\fontencoding{\latinencoding}\selectfont
    \ifcase#1\or
      A\or B\or C\or D\or E\or F\or G\or H\or
      I\or J\or K\or L\or M\or N\or O\or P\or
      Q\or R\or S\or T\or U\or V\or W\or X\or
      Y\or Z\else\@ctrerr\fi}}%
%    \end{macrocode}
%
%    Unfortunately, the commands |\AA| and |\aa| are not encoding dependent
%    in \LaTeX\ (unlike e.g., |\oe| or |\DH|). They are defined as |\r{A}| and
%    |\r{a}|. This leads to unpredictable results when the font encoding
%    does not contain the Latin letters `A' and `a' (like \texttt{X2}).
%
%    \begin{macrocode}
  \DeclareTextSymbolDefault{\AA}{OT1}
  \DeclareTextSymbolDefault{\aa}{OT1}
  \DeclareTextCommand{\aa}{OT1}{\r a}
  \DeclareTextCommand{\AA}{OT1}{\r A}
\fi
%    \end{macrocode}
%
%    The following block redefines the character class of uppercase Greek
%    letters and some accents, if it is equal to 7 (variable family), to
%    avoid incorrect results if the font encoding in some math family does
%    not contain these characters in places of OT1 encoding. The code was
%    taken from |amsmath.dtx|. See comments and further explanation there.
%
% \changes{russianb-1.1n}{2001/02/21}{As this code generates a
%    textfont 7 error it is commented out for now.}
%    \begin{macrocode}
% \begingroup\catcode`\"=12
% % uppercase greek letters:
% \def\@tempa#1{\expandafter\@tempb\meaning#1\relax\relax\relax\relax
%   "0000\@nil#1}
% \def\@tempb#1"#2#3#4#5#6\@nil#7{%
%   \ifnum"#2=7 \count@"1#3#4#5\relax
%     \ifnum\count@<"1000 \else \global\mathchardef#7="0#3#4#5\relax \fi
%   \fi}
% \@tempa\Gamma\@tempa\Delta\@tempa\Theta\@tempa\Lambda\@tempa\Xi
% \@tempa\Pi\@tempa\Sigma\@tempa\Upsilon\@tempa\Phi\@tempa\Psi
% \@tempa\Omega
% % some accents:
% \def\@tempa#1#2\@nil{\def\@tempc{#1}}\def\@tempb{\mathaccent}
% \expandafter\@tempa\hat\relax\relax\@nil
% \ifx\@tempb\@tempc
%   \def\@tempa#1\@nil{#1}%
%   \def\@tempb#1{\afterassignment\@tempa\mathchardef\@tempc=}%
%   \def\do#1"#2{}
%   \def\@tempd#1{\expandafter\@tempb#1\@nil
%     \ifnum\@tempc>"FFF
%       \xdef#1{\mathaccent"\expandafter\do\meaning\@tempc\space}%
%     \fi}
%   \@tempd\hat\@tempd\check\@tempd\tilde\@tempd\acute\@tempd\grave
%   \@tempd\dot\@tempd\ddot\@tempd\breve\@tempd\bar
% \fi
% \endgroup
%    \end{macrocode}
%
%    The user should use the \pkg{inputenc} package when any 8-bit Cyrillic
%    font encoding is used, selecting one of the Cyrillic input encodings.
%    We do not assume any default input encoding, so the user should
%    explicitly call the \pkg{inputenc} package by |\usepackage{inputenc}|.
%    We also removed |\AtBeginDocument|, so \pkg{inputenc} should be used
%    before \babel.
%
% \changes{russianb-1.1l}{1999/08/27}{Made not using inputenc a
%    warning instead of an error} 
%    \begin{macrocode}
\@ifpackageloaded{inputenc}{}{%
  \def\reserved@a{LWN}%
  \ifx\reserved@a\cyrillicencoding\else
    \def\reserved@a{OT2}%
    \ifx\reserved@a\cyrillicencoding\else
      \PackageWarning{babel}%
        {No input encoding specified for Russian language}
  \fi\fi}
%    \end{macrocode}
%
%    Now we define two commands that offer the possibility to switch between
%    Cyrillic and Roman encodings.
%
%  \begin{macro}{\cyrillictext}
%  \begin{macro}{\latintext}
%
%    The command |\cyrillictext| will switch from Latin font encoding to the
%    Cyrillic font encoding, the command |\latintext| switches back. This
%    assumes that the `normal' font encoding is a Latin one. These commands
%    are \emph{declarations}, for shorter peaces of text the commands
%    |\textlatin| and |\textcyrillic| can be used.
%
% \changes{russianb-1.1o}{2003/10/12}{\cs{latintext} is already
%    defined by the core of \babel}
%    \begin{macrocode}
%\DeclareRobustCommand{\latintext}{%
%  \fontencoding{\latinencoding}\selectfont
%  \def\encodingdefault{\latinencoding}}
\let\lat\latintext
%    \end{macrocode}
%
%  \end{macro}
%  \end{macro}
%
%  \begin{macro}{\textcyrillic}
%  \begin{macro}{\textlatin}
%
%    These commands take an argument which is then typeset using the
%    requested font encoding.
% \changes{russianb-1.1o}{2003/10/12}{\cs{textlatin} already defined
%    by the core of \babel}
%    \begin{macrocode}
\DeclareTextFontCommand{\textcyrillic}{\cyrillictext}
%\DeclareTextFontCommand{\textlatin}{\latintext}
%    \end{macrocode}
%
%  \end{macro}
%  \end{macro}
%
%    We make the \TeX
%    \begin{macrocode}
%\ifx\ltxTeX\undefined\let\ltxTeX\TeX\fi
%\ProvideTextCommandDefault{\TeX}{\textlatin{\ltxTeX}}
%    \end{macrocode}
%    and \LaTeX\ logos encoding independent.
%    \begin{macrocode}
%\ifx\ltxLaTeX\undefined\let\ltxLaTeX\LaTeX\fi
%\ProvideTextCommandDefault{\LaTeX}{\textlatin{\ltxLaTeX}}
%    \end{macrocode}
%
%    The next step consists of defining commands to switch to (and
%    from) the Russian language.
%
% \begin{macro}{\captionsrussian}
%
%    The macro |\captionsrussian| defines all strings used in the four
%    standard document classes provided with \LaTeX. The two commands |\cyr|
%    and |\lat| activate Cyrillic resp.\ Latin encoding.
%
%    \begin{macrocode}
\addto\captionsrussian{%
%   FIXME: Where is the \prefacename used?
  \def\prefacename{%
    {\cyr\CYRP\cyrr\cyre\cyrd\cyri\cyrs\cyrl\cyro\cyrv\cyri\cyre}}%
%   {\cyr\CYRV\cyrv\cyre\cyrd\cyre\cyrn\cyri\cyre}}%
  \def\refname{%
    {\cyr\CYRS\cyrp\cyri\cyrs\cyro\cyrk
      \ \cyrl\cyri\cyrt\cyre\cyrr\cyra\cyrt\cyru\cyrr\cyrery}}%
% \def\refname{%
%   {\cyr\CYRL\cyri\cyrt\cyre\cyrr\cyra\cyrt\cyru\cyrr\cyra}}%
  \def\abstractname{%
    {\cyr\CYRA\cyrn\cyrn\cyro\cyrt\cyra\cyrc\cyri\cyrya}}%
  \def\bibname{%
    {\cyr\CYRL\cyri\cyrt\cyre\cyrr\cyra\cyrt\cyru\cyrr\cyra}}%
% \def\bibname{%
%   {\cyr\CYRB\cyri\cyrb\cyrl\cyri\cyro
%    \cyrg\cyrr\cyra\cyrf\cyri\cyrya}}%
% for reports according to GOST:
% \def\bibname{%
%   {\cyr\CYRS\cyrp\cyri\cyrs\cyro\cyrk
%     \ \cyri\cyrs\cyrp\cyro\cyrl\cyrsftsn\cyrz\cyro\cyrv\cyra\cyrn
%     \cyrn\cyrery\cyrh\ \cyri\cyrs\cyrt\cyro\cyrch\cyrn\cyri
%     \cyrk\cyro\cyrv}}%
  \def\chaptername{{\cyr\CYRG\cyrl\cyra\cyrv\cyra}}%
% \@ifundefined{chapter}{}{%
%   \def\chaptername{{\cyr\CYRG\cyrl\cyra\cyrv\cyra}}}%
  \def\appendixname{%
    {\cyr\CYRP\cyrr\cyri\cyrl\cyro\cyrzh\cyre\cyrn\cyri\cyre}}%
%    \end{macrocode}
%
%    There are two names for the Table of Contents that are used in Russian
%    publications. For books (and reports) the second variant is
%    appropriate, but for proceedings the first variant is preferred:
%
%    \begin{macrocode}
  \@ifundefined{thechapter}%
    {\def\contentsname{%
      {\cyr\CYRS\cyro\cyrd\cyre\cyrr\cyrzh\cyra\cyrn\cyri\cyre}}}%
    {\def\contentsname{%
      {\cyr\CYRO\cyrg\cyrl\cyra\cyrv\cyrl\cyre\cyrn\cyri\cyre}}}%
  \def\listfigurename{%
    {\cyr\CYRS\cyrp\cyri\cyrs\cyro\cyrk
      \ \cyri\cyrl\cyrl\cyryu\cyrs\cyrt\cyrr\cyra\cyrc\cyri\cyrishrt}}%
% \def\listfigurename{%
%   {\cyr\CYRS\cyrp\cyri\cyrs\cyro\cyrk
%     \ \cyrr\cyri\cyrs\cyru\cyrn\cyrk\cyro\cyrv}}%
  \def\listtablename{%
    {\cyr\CYRS\cyrp\cyri\cyrs\cyro\cyrk
      \ \cyrt\cyra\cyrb\cyrl\cyri\cyrc}}%
  \def\indexname{%
    {\cyr\CYRP\cyrr\cyre\cyrd\cyrm\cyre\cyrt\cyrn\cyrery\cyrishrt
      \ \cyru\cyrk\cyra\cyrz\cyra\cyrt\cyre\cyrl\cyrsftsn}}%
  \def\authorname{%
    {\cyr\CYRI\cyrm\cyre\cyrn\cyrn\cyro\cyrishrt
      \ \cyru\cyrk\cyra\cyrz\cyra\cyrt\cyre\cyrl\cyrsftsn}}%
  \def\figurename{{\cyr\CYRR\cyri\cyrs.}}%
  \def\tablename{{\cyr\CYRT\cyra\cyrb\cyrl\cyri\cyrc\cyra}}%
  \def\partname{{\cyr\CYRCH\cyra\cyrs\cyrt\cyrsftsn}}%
  \def\enclname{{\cyr\cyrv\cyrk\cyrl.}}%
  \def\ccname{{\cyr\cyri\cyrs\cyrh.}}%
% \def\ccname{{\cyr\cyri\cyrz}}%
  \def\headtoname{{\cyr\cyrv\cyrh.}}%
% \def\headtoname{{\cyr\cyrv}}%
  \def\pagename{{\cyr\cyrs.}}%
% \def\pagename{{\cyr\cyrs\cyrt\cyrr.}}%
  \def\seename{{\cyr\cyrs\cyrm.}}%
  \def\alsoname{{\cyr\cyrs\cyrm.\ \cyrt\cyra\cyrk\cyrzh\cyre}}%
%    \end{macrocode}
% \changes{russianb-1.1m}{2000/09/20}{Added \cs{glossaryname}}
%    \begin{macrocode}
  \def\proofname{{\cyr\CYRD\cyro\cyrk\cyra\cyrz\cyra\cyrt
      \cyre\cyrl\cyrsftsn\cyrs\cyrt\cyrv\cyro}}%
  \def\glossaryname{Glossary}% <-- Needs translation
  }
%    \end{macrocode}
%
% \end{macro}
%
% \begin{macro}{\daterussian}
%
%    The macro |\daterussian| redefines the command |\today| to produce
%    Russian dates.
%
%    \begin{macrocode}
\def\daterussian{%
  \def\today{\number\day~\ifcase\month\or
    \cyrya\cyrn\cyrv\cyra\cyrr\cyrya\or
    \cyrf\cyre\cyrv\cyrr\cyra\cyrl\cyrya\or
    \cyrm\cyra\cyrr\cyrt\cyra\or
    \cyra\cyrp\cyrr\cyre\cyrl\cyrya\or
    \cyrm\cyra\cyrya\or
    \cyri\cyryu\cyrn\cyrya\or
    \cyri\cyryu\cyrl\cyrya\or
    \cyra\cyrv\cyrg\cyru\cyrs\cyrt\cyra\or
    \cyrs\cyre\cyrn\cyrt\cyrya\cyrb\cyrr\cyrya\or
    \cyro\cyrk\cyrt\cyrya\cyrb\cyrr\cyrya\or
    \cyrn\cyro\cyrya\cyrb\cyrr\cyrya\or
    \cyrd\cyre\cyrk\cyra\cyrb\cyrr\cyrya\fi
    \ \number\year~\cyrg.}}
%    \end{macrocode}
%
% \end{macro}
%
% \begin{macro}{\extrasrussian}
%
%    The macro |\extrasrussian| will perform all the extra definitions
%    needed for the Russian language. The macro |\noextrasrussian| is used
%    to cancel the actions of |\extrasrussian|.
%
% \changes{russianb-1.1b}{1996/02/20}{Added switch to \texttt{LWN}
%    encoding}
%
%    The first action we define is to switch on the selected Cyrillic
%    encoding whenever we enter `russian'.
%
%    \begin{macrocode}
\addto\extrasrussian{\cyrillictext}
%    \end{macrocode}
%
%    When the encoding definition file was processed by \LaTeX\ the current
%    font encoding is stored in |\latinencoding|, assuming that \LaTeX\ uses
%    \texttt{T1} or \texttt{OT1} as default. Therefore we switch back to
%    |\latinencoding| whenever the Russian language is no longer `active'.
%
%    \begin{macrocode}
\addto\noextrasrussian{\latintext}
%    \end{macrocode}
%
%  \begin{macro}{\verbatim@font}
%
% \changes{russianb-1.1b}{1996/02/20}{Added changing of
%    \cs{verbatim@font}}
%
%    In order to get both Latin and Cyrillic letters in verbatim text we
%    need to change the definition of an internal \LaTeX\ command somewhat:
%
%    \begin{macrocode}
%\def\verbatim@font{%
%  \let\encodingdefault\latinencoding
%  \normalfont\ttfamily
%  \expandafter\def\csname\cyrillicencoding-cmd\endcsname##1##2{%
%    \ifx\protect\@typeset@protect
%      \begingroup\UseTextSymbol\cyrillicencoding##1\endgroup
%    \else\noexpand##1\fi}}
%    \end{macrocode}
%
%  \end{macro}
%
%    The category code of the characters `\texttt{:}', `\texttt{;}',
%    `\texttt{!}', and `\texttt{?}' is made |\active| to insert a little
%    white space.
%
%    For Russian (as well as for German) the \texttt{"} character also is
%    made active.
%
%    Note: It is \emph{very} questionable whether the Russian typesetting
%    tradition requires additional spacing before those punctuation signs.
%    Therefore, we make the corresponding code optional. If you need it,
%    then define the \texttt{frenchpunct} docstrip option in
%    \file{babel.ins}.
%
% \changes{russianb-1.1f}{1998/06/26}{%
%    Added a hook to insert space
%    or not before `double punctuation' (from frenchb).}
%
%    Borrowed from french.
%    Some users dislike automatic insertion of a space before
%    `double punctuation', and prefer to decide themselves whether a
%    space should be added or not; so a hook |\NoAutoSpaceBeforeFDP|
%    is provided: if this command is added (in file |russianb.cfg|, or
%    anywhere in a document) |russianb| will respect your typing, and
%    introduce a suitable space before `double punctuation' \emph{if
%    and only if} a space is typed in the source file before those
%    signs.
%
%    The command |\AutoSpaceBeforeFDP| switches back to the
%    default behavior of |russianb|.
%
% \changes{russianb-1.1a}{1995/03/07}{Use the new mechanism for dealing
%    with active characters}
%
%    \begin{macrocode}
%<*frenchpunct>
\initiate@active@char{:}
\initiate@active@char{;}
%</frenchpunct>
%<*frenchpunct|spanishligs>
\initiate@active@char{!}
\initiate@active@char{?}
%</frenchpunct|spanishligs>
\initiate@active@char{"}
%    \end{macrocode}
%
%    The code above is necessary because we need extra active characters.
%    The character |"| is used as indicated in
%    table~\ref{tab:russian-quote}.
%
%    We specify that the Russian group of shorthands should be used.
%
%    \begin{macrocode}
\addto\extrasrussian{\languageshorthands{russian}}
%    \end{macrocode}
%
%    These characters are `turned on' once, later their definition may
%    vary.
%
%    \begin{macrocode}
\addto\extrasrussian{%
%<frenchpunct>  \bbl@activate{:}\bbl@activate{;}%
%<frenchpunct|spanishligs>  \bbl@activate{!}\bbl@activate{?}%
  \bbl@activate{"}}
\addto\noextrasrussian{%
%<frenchpunct>  \bbl@deactivate{:}\bbl@deactivate{;}%
%<frenchpunct|spanishligs>  \bbl@deactivate{!}\bbl@deactivate{?}%
  \bbl@deactivate{"}}
%    \end{macrocode}
%
%   The \texttt{X2} and \texttt{T2*} encodings do not contain
%   |spanish_shriek| and |spanish_query| symbols; as a consequence, the
%   ligatures `|?`|' and `|!`|' do not work with them (these characters are
%   useless for Cyrillic texts anyway). But we define the shorthands to
%   emulate these ligatures (optionally).
%
%   We do not use |\latinencoding| here (but instead explicitly use
%   \texttt{OT1}) because the user may choose \texttt{T2A} to be the primary
%   encoding, but it does not contain these characters.
%
%    \begin{macrocode}
%<*spanishligs>
\declare@shorthand{russian}{?`}{\UseTextSymbol{OT1}\textquestiondown}
\declare@shorthand{russian}{!`}{\UseTextSymbol{OT1}\textexclamdown}
%</spanishligs>
%    \end{macrocode}
%
% \begin{macro}{\russian@sh@;@}
% \begin{macro}{\russian@sh@:@}
% \begin{macro}{\russian@sh@!@}
% \begin{macro}{\russian@sh@?@}
%
%    We have to reduce the amount of white space before \texttt{;},
%    \texttt{:} and \texttt{!}. This should only happen in horizontal mode,
%    hence the test with |\ifhmode|.
%
% \changes{russianb-1.1a}{1995/07/04}{Use new \cs{DefineActiveNoArg}}
% \changes{russianb-1.1a}{1995/07/04}{Use the more general mechanism of
%    \cs{declare@shorthand}}
% \changes{russianb-1.1b}{1996/02/08}{Updated to reflect the latest
%    french definitions}
%
%    \begin{macrocode}
%<*frenchpunct>
\declare@shorthand{russian}{;}{%
  \ifhmode
%    \end{macrocode}
%
% \changes{russianb-1.1f}{1998/06/26}{%
%    \thinspace changed to kern.1em (space bit thinner)}
% \changes{russianb-1.1f}{1998/06/26}{%
%    Added a hook to insert space
%    or not before `double punctuation' (from frenchb).}
%
%    In horizontal mode we check for the presence of a `space', `unskip' if
%    it exists and place a |0.1em| kerning.
%
%    \begin{macrocode}
    \ifdim\lastskip>\z@
      \unskip\nobreak\kern.1em
    \else
%    \end{macrocode}
%    If no space has been typed, we add |\FDP@thinspace|
%    which will be
%    defined, up to the user's wishes, as an automatic added
%    thinspace, or as |\@empty|.
%
%    \begin{macrocode}
        \FDP@thinspace
    \fi
  \fi
%    \end{macrocode}
%
%    Now we can insert a `|;|' character.
%
%    \begin{macrocode}
  \string;}
%    \end{macrocode}
%
%    The other definitions are very similar.
%
%    \begin{macrocode}
\declare@shorthand{russian}{:}{%
  \ifhmode
    \ifdim\lastskip>\z@
      \unskip\nobreak\kern.1em
    \else
        \FDP@thinspace
    \fi
  \fi
  \string:}
%    \end{macrocode}
%
%    \begin{macrocode}
\declare@shorthand{russian}{!}{%
  \ifhmode
    \ifdim\lastskip>\z@
      \unskip\nobreak\kern.1em
    \else
        \FDP@thinspace
    \fi
  \fi
  \string!}
%    \end{macrocode}
%
%    \begin{macrocode}
\declare@shorthand{russian}{?}{%
  \ifhmode
    \ifdim\lastskip>\z@
      \unskip\nobreak\kern.1em
    \else
        \FDP@thinspace
    \fi
  \fi
  \string?}
%    \end{macrocode}
%
% \end{macro}
% \end{macro}
% \end{macro}
% \end{macro}
%
%
% \changes{russianb-1.1f}{1998/06/26}{%
%    Added a hook to insert space
%    or not before `double punctuation' (from frenchb).}
%  \begin{macro}{\AutoSpaceBeforeFDP}
%  \begin{macro}{\NoAutoSpaceBeforeFDP}
%  \begin{macro}{\FDP@thinspace}
%    |\FDP@thinspace| is defined as unbreakable
%    spaces if |\AutoSpaceBeforeFDP| is activated or as |\@empty| if
%    |\NoAutoSpaceBeforeFDP| is in use.
%    The default is |\AutoSpaceBeforeFDP|.
%    \begin{macrocode}
\def\AutoSpaceBeforeFDP{%
      \def\FDP@thinspace{\nobreak\kern.1em}}
\def\NoAutoSpaceBeforeFDP{\let\FDP@thinspace\@empty}
\AutoSpaceBeforeFDP
%    \end{macrocode}
%  \end{macro}
%  \end{macro}
%  \end{macro}
%
%  \begin{macro}{\FDPon}
%  \begin{macro}{\FDPoff}
% \changes{russianb-1.1f}{1998/06/26}{One more chance to avoid
%       spaces before double punctuation}
%
%     The next macros allow to switch on/off activeness of double
%     punctuation signs.
%
%    \begin{macrocode}
\def\FDPon{\bbl@activate{:}%
        \bbl@activate{;}%
        \bbl@activate{?}%
        \bbl@activate{!}}
\def\FDPoff{\bbl@deactivate{:}%
        \bbl@deactivate{;}%
        \bbl@deactivate{?}%
        \bbl@deactivate{!}}
%    \end{macrocode}
%  \end{macro}
%  \end{macro}
%
%  \begin{macro}{\system@sh@:@}
%  \begin{macro}{\system@sh@!@}
%  \begin{macro}{\system@sh@?@}
%  \begin{macro}{\system@sh@;@}
%
% \changes{russianb-1.1a}{1995/07/04}{Added system level shorthands}
%
%    When the active characters appear in an environment where their
%    Russian behaviour is not wanted they should give an `expected'
%    result. Therefore we define shorthands at system level as well.
%
%    \begin{macrocode}
\declare@shorthand{system}{:}{\string:}
\declare@shorthand{system}{;}{\string;}
%</frenchpunct>
%<*frenchpunct&!spanishligs>
\declare@shorthand{system}{!}{\string!}
\declare@shorthand{system}{?}{\string?}
%</frenchpunct&!spanishligs>
%    \end{macrocode}
%
%  \end{macro}
%  \end{macro}
%  \end{macro}
%  \end{macro}
%
%    To be able to define the function of `|"|', we first define a couple of
%    `support' macros.
%
%  \begin{macro}{\dq}
%
%    We save the original double quote character in |\dq| to keep it
%    available, the math accent |\"| can now be typed as `|"|'.
%
%    \begin{macrocode}
\begingroup \catcode`\"12
\def\reserved@a{\endgroup
  \def\@SS{\mathchar"7019 }
  \def\dq{"}}
\reserved@a
%    \end{macrocode}
%
%  \end{macro}
%
% \changes{russianb-1.1a}{1995/07/04}{Use \cs{ddot} instead of
%    \cs{@MATHUMLAUT}}
%
%    Now we can define the doublequote macros: german and french quotes.
%    We use definitions of these quotes made in babel.sty.
%    The french quotes are contained in the \texttt{T2*} encodings.
%
%    \begin{macrocode}
\declare@shorthand{russian}{"`}{\glqq}
\declare@shorthand{russian}{"'}{\grqq}
\declare@shorthand{russian}{"<}{\flqq}
\declare@shorthand{russian}{">}{\frqq}
%    \end{macrocode}
%
%    Some additional commands:
%
%    \begin{macrocode}
\declare@shorthand{russian}{""}{\hskip\z@skip}
\declare@shorthand{russian}{"~}{\textormath{\leavevmode\hbox{-}}{-}}
\declare@shorthand{russian}{"=}{\nobreak-\hskip\z@skip}
\declare@shorthand{russian}{"|}{%
  \textormath{\nobreak\discretionary{-}{}{\kern.03em}%
              \allowhyphens}{}}
%    \end{macrocode}
%
%    The next two macros for |"-| and |"---| are somewhat different.
%    We must check whether the second token is a hyphen character:
%
%    \begin{macrocode}
\declare@shorthand{russian}{"-}{%
%    \end{macrocode}
%
%    If the next token is `|-|', we typeset an emdash, otherwise a hyphen
%    sign:
%
%    \begin{macrocode}
  \def\russian@sh@tmp{%
    \if\russian@sh@next-\expandafter\russian@sh@emdash
    \else\expandafter\russian@sh@hyphen\fi
  }%
%    \end{macrocode}
%
%    \TeX\ looks for the next token after the first `|-|': the meaning of
%    this token is written to |\russian@sh@next| and |\russian@sh@tmp| is
%    called.
%
%    \begin{macrocode}
  \futurelet\russian@sh@next\russian@sh@tmp}
%    \end{macrocode}
%
%    Here are the definitions of hyphen and emdash. First the hyphen:
%
%    \begin{macrocode}
\def\russian@sh@hyphen{%
  \nobreak\-\bbl@allowhyphens}
%    \end{macrocode}
%
% \changes{russianb-1.1f}{1998/06/26}{%
%    Rearranging of cyrillic emdash stuff}
%
%    For the emdash definition, there are the two parameters: we must `eat'
%    two last hyphen signs of our emdash\dots :
%    \begin{macrocode}
\def\russian@sh@emdash#1#2{\cdash-#1#2}
%    \end{macrocode}
%  \begin{macro}{\cdash}
%    \dots\ these two parameters are useful for another macro:
%    |\cdash|:
%    \begin{macrocode}
%\ifx\cdash\undefined % should be defined earlier
\def\cdash#1#2#3{\def\tempx@{#3}%
\def\tempa@{-}\def\tempb@{~}\def\tempc@{*}%
 \ifx\tempx@\tempa@\@Acdash\else
  \ifx\tempx@\tempb@\@Bcdash\else
   \ifx\tempx@\tempc@\@Ccdash\else
    \errmessage{Wrong usage of cdash}\fi\fi\fi}
%    \end{macrocode}
%   second parameter (or third for |\cdash|) shows what kind of emdash
%   to create in next step
%      \begin{center}
%      \begin{tabular}{@{}p{.1\hsize}@{}p{.9\hsize}@{}}
%       |"---| & ordinary (plain) Cyrillic emdash inside text:
%       an unbreakable thinspace will be inserted before only in case of
%       a \textit{space} before the dash (it is necessary for dashes after
%       display maths formulae: there could be lists, enumerations etc.\
%       started with ``--- where $a$ is ...'' i.e., the dash starts a line).
%       (Firstly there were planned rather soft rules for user: he may put
%       a space before the dash or not.  But it is difficult to place this
%       thinspace automatically, i.e., by checking modes because after
%       display formulae \TeX{} uses horizontal mode.  Maybe there is a
%       misunderstanding?  Maybe there is another way?)  After a dash
%       a breakable thinspace is always placed; \\
%   \end{tabular}
%   \end{center}
%    \begin{macrocode}
% What is more grammatically: .2em or .2\fontdimen6\font ?
\def\@Acdash{\ifdim\lastskip>\z@\unskip\nobreak\hskip.2em\fi
  \cyrdash\hskip.2em\ignorespaces}%
%    \end{macrocode}
%      \begin{center}
%      \begin{tabular}{@{}p{.1\hsize}@{}p{.9\hsize}@{}}
%       |"--~| & emdash in compound names or surnames
%       (like Mendeleev--Klapeiron); this dash has no space characters
%       around; after the dash some space is added
%       |\exhyphenalty| \\
%   \end{tabular}
%   \end{center}
%    \begin{macrocode}
\def\@Bcdash{\leavevmode\ifdim\lastskip>\z@\unskip\fi
 \nobreak\cyrdash\penalty\exhyphenpenalty\hskip\z@skip\ignorespaces}%
%    \end{macrocode}
%      \begin{center}
%      \begin{tabular}{@{}p{.1\hsize}@{}p{.9\hsize}@{}}
%       |"--*| & for denoting direct speech (a space like |\enskip|
%       must follow the emdash); \\
%   \end{tabular}
%   \end{center}
%    \begin{macrocode}
\def\@Ccdash{\leavevmode
 \nobreak\cyrdash\nobreak\hskip.35em\ignorespaces}%
%\fi
%    \end{macrocode}
%  \end{macro}
%
%  \begin{macro}{\cyrdash}
%   Finally the macro for ``body'' of the Cyrillic emdash.
%   The |\cyrdash| macro will be defined in case this macro hasn't been
%   defined in a fontenc file.  For T2* fonts, cyrdash will be placed in
%   the code of the English emdash thus it uses ligature |---|.
%    \begin{macrocode}
% Is there an IF necessary?
\ifx\cyrdash\undefined
  \def\cyrdash{\hbox to.8em{--\hss--}}
\fi
%    \end{macrocode}
%  \end{macro}
%
% \changes{russianb-1.1f}{1998/06/26}{%
%    Add macro for thinspace between initials}
%
%    Here a really new macro---to place thinspace between initials.
%    This macro used instead of |\,| allows hyphenation in the following
%    surname.
%
% \changes{russianb-1.1r}{2004/11/21}{Removed the commanet character
%    before the next code line, see R3669}
%    \begin{macrocode}
\declare@shorthand{russian}{",}{\nobreak\hskip.2em\ignorespaces}
%    \end{macrocode}
%
% \changes{russianb-1.1f}{1998/06/26}{%
%    Add commands for switch on/off
%    doublequote activeness.  Borrowed from german.}
%
%  \begin{macro}{\mdqon}
%  \begin{macro}{\mdqoff}
%    All that's left to do now is to  define a couple of commands
%    for |"|.
%    \begin{macrocode}
\def\mdqon{\bbl@activate{"}}
\def\mdqoff{\bbl@deactivate{"}}
%    \end{macrocode}
%  \end{macro}
%  \end{macro}
%
%    The Russian hyphenation patterns can be used with |\lefthyphenmin|
%    and |\righthyphenmin| set to~2.
%
% \changes{russianb-1.1a}{1995/07/04}{use \cs{russianhyphenmins} to
%    store the correct values}
% \changes{russianb-1.1m}{2000/09/22}{Now use \cs{providehyphenmins} to
%    provide a default value}
%    \begin{macrocode}
\providehyphenmins{\CurrentOption}{\tw@\tw@}
% temporary hack:
\ifx\englishhyphenmins\undefined
  \def\englishhyphenmins{\tw@\thr@@}
\fi
%    \end{macrocode}
%
%    Now the action |\extrasrussian| has to execute is to make sure that the
%    command |\frenchspacing| is in effect. If this is not the case the
%    execution of |\noextrasrussian| will switch it off again.
%
%    \begin{macrocode}
\addto\extrasrussian{\bbl@frenchspacing}
\addto\noextrasrussian{\bbl@nonfrenchspacing}
%    \end{macrocode}
%
% \end{macro}
%
%    Next we add a new enumeration style for Russian manuscripts with
%    Cyrillic letters, and later on we define some math operator names in
%    accordance with Russian typesetting traditions.
%
%  \begin{macro}{\Asbuk}
%
%    We begin by defining |\Asbuk| which works like |\Alph|, but produces
%    (uppercase) Cyrillic letters intead of Latin ones. The letters YO,
%    ISHRT, HRDSN, ERY, and SFTSN are skipped, as usual for such
%    enumeration.
%
%    \begin{macrocode}
\def\Asbuk#1{\expandafter\@Asbuk\csname c@#1\endcsname}
\def\@Asbuk#1{\ifcase#1\or
  \CYRA\or\CYRB\or\CYRV\or\CYRG\or\CYRD\or\CYRE\or\CYRZH\or
  \CYRZ\or\CYRI\or\CYRK\or\CYRL\or\CYRM\or\CYRN\or\CYRO\or
  \CYRP\or\CYRR\or\CYRS\or\CYRT\or\CYRU\or\CYRF\or\CYRH\or
  \CYRC\or\CYRCH\or\CYRSH\or\CYRSHCH\or\CYREREV\or\CYRYU\or
  \CYRYA\else\@ctrerr\fi}
%    \end{macrocode}
%
%  \end{macro}
%
%  \begin{macro}{\asbuk}
%
%    The macro |\asbuk| is similar to |\alph|; it produces lowercase
%    Russian letters.
%
%    \begin{macrocode}
\def\asbuk#1{\expandafter\@asbuk\csname c@#1\endcsname}
\def\@asbuk#1{\ifcase#1\or
  \cyra\or\cyrb\or\cyrv\or\cyrg\or\cyrd\or\cyre\or\cyrzh\or
  \cyrz\or\cyri\or\cyrk\or\cyrl\or\cyrm\or\cyrn\or\cyro\or
  \cyrp\or\cyrr\or\cyrs\or\cyrt\or\cyru\or\cyrf\or\cyrh\or
  \cyrc\or\cyrch\or\cyrsh\or\cyrshch\or\cyrerev\or\cyryu\or
  \cyrya\else\@ctrerr\fi}
%    \end{macrocode}
%
%  \end{macro}
%
% Set up default Cyrillic math alphabets. To use Cyrillic letters in
% math mode user should load the |textmath| package \emph{before}
% loading fontenc package (or \babel).  Note, that by default Cyrillic
% letters are taken from upright font in math mode (unlike Latin
% letters).
%    \begin{macrocode}
%\RequirePackage{textmath}
\@ifundefined{sym\cyrillicencoding letters}{}{%
\SetSymbolFont{\cyrillicencoding letters}{bold}\cyrillicencoding
  \rmdefault\bfdefault\updefault
\DeclareSymbolFontAlphabet\cyrmathrm{\cyrillicencoding letters}
%    \end{macrocode}
%    And we need a few commands to be able to switch to different variants.
%    \begin{macrocode}
\DeclareMathAlphabet\cyrmathbf\cyrillicencoding
  \rmdefault\bfdefault\updefault
\DeclareMathAlphabet\cyrmathsf\cyrillicencoding
  \sfdefault\mddefault\updefault
\DeclareMathAlphabet\cyrmathit\cyrillicencoding
  \rmdefault\mddefault\itdefault
\DeclareMathAlphabet\cyrmathtt\cyrillicencoding
  \ttdefault\mddefault\updefault
%
\SetMathAlphabet\cyrmathsf{bold}\cyrillicencoding
  \sfdefault\bfdefault\updefault
\SetMathAlphabet\cyrmathit{bold}\cyrillicencoding
  \rmdefault\bfdefault\itdefault
}
%    \end{macrocode}
%
%    Some math functions in Russian math books have other names: e.g.,
%    \texttt{sinh} in Russian is written as \texttt{sh} etc. So we define a
%    number of new math operators.
%
%    |\sinh|:
%    \begin{macrocode}
\def\sh{\mathop{\operator@font sh}\nolimits}
%    \end{macrocode}
%    |\cosh|:
%    \begin{macrocode}
\def\ch{\mathop{\operator@font ch}\nolimits}
%    \end{macrocode}
%    |\tan|:
%    \begin{macrocode}
\def\tg{\mathop{\operator@font tg}\nolimits}
%    \end{macrocode}
%    |\arctan|:
%    \begin{macrocode}
\def\arctg{\mathop{\operator@font arctg}\nolimits}
%    \end{macrocode}
%    arcctg:
%    \begin{macrocode}
\def\arcctg{\mathop{\operator@font arcctg}\nolimits}
%    \end{macrocode}
%    The following macro conflicts with |\th| defined in Latin~1 encoding:
%
%    |\tanh|:
% \changes{russianb-1.1q}{2004/05/21}{Change definition of \cs{th}
%    only for this language}
%    \begin{macrocode}
\addto\extrasrussian{%
  \babel@save{\th}%
  \let\ltx@th\th
  \def\th{\textormath{\ltx@th}%
                     {\mathop{\operator@font th}\nolimits}}%
  }
%    \end{macrocode}
%    |\cot|:
%    \begin{macrocode}
\def\ctg{\mathop{\operator@font ctg}\nolimits}
%    \end{macrocode}
%    |\coth|:
%    \begin{macrocode}
\def\cth{\mathop{\operator@font cth}\nolimits}
%    \end{macrocode}
%    |\csc|:
%    \begin{macrocode}
\def\cosec{\mathop{\operator@font cosec}\nolimits}
%    \end{macrocode}
%
%    And finally some other Russian mathematical symbols:
%    \begin{macrocode}
\def\Prob{\mathop{\kern\z@\mathsf{P}}\nolimits}
\def\Variance{\mathop{\kern\z@\mathsf{D}}\nolimits}
\def\nod{\mathop{\cyrmathrm{\cyrn.\cyro.\cyrd.}}\nolimits}
\def\nok{\mathop{\cyrmathrm{\cyrn.\cyro.\cyrk.}}\nolimits}
\def\NOD{\mathop{\cyrmathrm{\CYRN\CYRO\CYRD}}\nolimits}
\def\NOK{\mathop{\cyrmathrm{\CYRN\CYRO\CYRK}}\nolimits}
\def\Proj{\mathop{\cyrmathrm{\CYRP\cyrr}}\nolimits}
%    \end{macrocode}
%
% This is for compatibility with older Russian packages.
%    \begin{macrocode}
\DeclareRobustCommand{\No}{%
   \ifmmode{\nfss@text{\textnumero}}\else\textnumero\fi}
%    \end{macrocode}
%
%    The macro |\ldf@finish| takes care of looking for a configuration file,
%    setting the main language to be switched on at |\begin{document}| and
%    resetting the category code of \texttt{@} to its original value.
%
% \changes{russianb-1.1d}{1996/11/03}{Now use \cs{ldf@finish} to wrap
%    up}
%
%    \begin{macrocode}
\ldf@finish{russian}
%</code>
%    \end{macrocode}
%
% \Finale
%%
%% \CharacterTable
%%  {Upper-case    \A\B\C\D\E\F\G\H\I\J\K\L\M\N\O\P\Q\R\S\T\U\V\W\X\Y\Z
%%   Lower-case    \a\b\c\d\e\f\g\h\i\j\k\l\m\n\o\p\q\r\s\t\u\v\w\x\y\z
%%   Digits        \0\1\2\3\4\5\6\7\8\9
%%   Exclamation   \!     Double quote  \"     Hash (number) \#
%%   Dollar        \$     Percent       \%     Ampersand     \&
%%   Acute accent  \'     Left paren    \(     Right paren   \)
%%   Asterisk      \*     Plus          \+     Comma         \,
%%   Minus         \-     Point         \.     Solidus       \/
%%   Colon         \:     Semicolon     \;     Less than     \<
%%   Equals        \=     Greater than  \>     Question mark \?
%%   Commercial at \@     Left bracket  \[     Backslash     \\
%%   Right bracket \]     Circumflex    \^     Underscore    \_
%%   Grave accent  \`     Left brace    \{     Vertical bar  \|
%%   Right brace   \}     Tilde         \~}
%%
\endinput
}
%    \end{macrocode}
%^^A \changes{babel~3.7a}{1997/02/07}{Added the \Lopt{sanskrit} option}
% \changes{babel~3.6i}{1997/02/21}{Added the \Lopt{ukrainian} option}
%^^A \changes{babel~3.7a}{1998/03/30}{Added the \Lopt{tamil} option}
%    \begin{macrocode}
%^^A\DeclareOption{sanskrit}{\input{sanskrit.ldf}}
\DeclareOption{scottish}{%%
%% This file will generate fast loadable files and documentation
%% driver files from the doc files in this package when run through
%% LaTeX or TeX.
%%
%% Copyright 1989-2005 Johannes L. Braams and any individual authors
%% listed elsewhere in this file.  All rights reserved.
%% 
%% This file is part of the Babel system.
%% --------------------------------------
%% 
%% It may be distributed and/or modified under the
%% conditions of the LaTeX Project Public License, either version 1.3
%% of this license or (at your option) any later version.
%% The latest version of this license is in
%%   http://www.latex-project.org/lppl.txt
%% and version 1.3 or later is part of all distributions of LaTeX
%% version 2003/12/01 or later.
%% 
%% This work has the LPPL maintenance status "maintained".
%% 
%% The Current Maintainer of this work is Johannes Braams.
%% 
%% The list of all files belonging to the LaTeX base distribution is
%% given in the file `manifest.bbl. See also `legal.bbl' for additional
%% information.
%% 
%% The list of derived (unpacked) files belonging to the distribution
%% and covered by LPPL is defined by the unpacking scripts (with
%% extension .ins) which are part of the distribution.
%%
%% --------------- start of docstrip commands ------------------
%%
\def\filedate{1999/04/11}
\def\batchfile{scottish.ins}
\input docstrip.tex

{\ifx\generate\undefined
\Msg{**********************************************}
\Msg{*}
\Msg{* This installation requires docstrip}
\Msg{* version 2.3c or later.}
\Msg{*}
\Msg{* An older version of docstrip has been input}
\Msg{*}
\Msg{**********************************************}
\errhelp{Move or rename old docstrip.tex.}
\errmessage{Old docstrip in input path}
\batchmode
\csname @@end\endcsname
\fi}

\declarepreamble\mainpreamble
This is a generated file.

Copyright 1989-2005 Johannes L. Braams and any individual authors
listed elsewhere in this file.  All rights reserved.

This file was generated from file(s) of the Babel system.
---------------------------------------------------------

It may be distributed and/or modified under the
conditions of the LaTeX Project Public License, either version 1.3
of this license or (at your option) any later version.
The latest version of this license is in
  http://www.latex-project.org/lppl.txt
and version 1.3 or later is part of all distributions of LaTeX
version 2003/12/01 or later.

This work has the LPPL maintenance status "maintained".

The Current Maintainer of this work is Johannes Braams.

This file may only be distributed together with a copy of the Babel
system. You may however distribute the Babel system without
such generated files.

The list of all files belonging to the Babel distribution is
given in the file `manifest.bbl'. See also `legal.bbl for additional
information.

The list of derived (unpacked) files belonging to the distribution
and covered by LPPL is defined by the unpacking scripts (with
extension .ins) which are part of the distribution.
\endpreamble

\declarepreamble\fdpreamble
This is a generated file.

Copyright 1989-2005 Johannes L. Braams and any individual authors
listed elsewhere in this file.  All rights reserved.

This file was generated from file(s) of the Babel system.
---------------------------------------------------------

It may be distributed and/or modified under the
conditions of the LaTeX Project Public License, either version 1.3
of this license or (at your option) any later version.
The latest version of this license is in
  http://www.latex-project.org/lppl.txt
and version 1.3 or later is part of all distributions of LaTeX
version 2003/12/01 or later.

This work has the LPPL maintenance status "maintained".

The Current Maintainer of this work is Johannes Braams.

This file may only be distributed together with a copy of the Babel
system. You may however distribute the Babel system without
such generated files.

The list of all files belonging to the Babel distribution is
given in the file `manifest.bbl'. See also `legal.bbl for additional
information.

In particular, permission is granted to customize the declarations in
this file to serve the needs of your installation.

However, NO PERMISSION is granted to distribute a modified version
of this file under its original name.

\endpreamble

\keepsilent

\usedir{tex/generic/babel} 

\usepreamble\mainpreamble
\generate{\file{scottish.ldf}{\from{scottish.dtx}{code}}
          }
\usepreamble\fdpreamble

\ifToplevel{
\Msg{***********************************************************}
\Msg{*}
\Msg{* To finish the installation you have to move the following}
\Msg{* files into a directory searched by TeX:}
\Msg{*}
\Msg{* \space\space All *.def, *.fd, *.ldf, *.sty}
\Msg{*}
\Msg{* To produce the documentation run the files ending with}
\Msg{* '.dtx' and `.fdd' through LaTeX.}
\Msg{*}
\Msg{* Happy TeXing}
\Msg{***********************************************************}
}
 
\endinput
}
\DeclareOption{serbian}{%%
%% This file will generate fast loadable files and documentation
%% driver files from the doc files in this package when run through
%% LaTeX or TeX.
%%
%% Copyright 1989-2005 Johannes L. Braams and any individual authors
%% listed elsewhere in this file.  All rights reserved.
%% 
%% This file is part of the Babel system.
%% --------------------------------------
%% 
%% It may be distributed and/or modified under the
%% conditions of the LaTeX Project Public License, either version 1.3
%% of this license or (at your option) any later version.
%% The latest version of this license is in
%%   http://www.latex-project.org/lppl.txt
%% and version 1.3 or later is part of all distributions of LaTeX
%% version 2003/12/01 or later.
%% 
%% This work has the LPPL maintenance status "maintained".
%% 
%% The Current Maintainer of this work is Johannes Braams.
%% 
%% The list of all files belonging to the LaTeX base distribution is
%% given in the file `manifest.bbl. See also `legal.bbl' for additional
%% information.
%% 
%% The list of derived (unpacked) files belonging to the distribution
%% and covered by LPPL is defined by the unpacking scripts (with
%% extension .ins) which are part of the distribution.
%%
%% --------------- start of docstrip commands ------------------
%%
\def\filedate{1999/03/13}
\def\batchfile{serbian.ins}
\input docstrip.tex

{\ifx\generate\undefined
\Msg{**********************************************}
\Msg{*}
\Msg{* This installation requires docstrip}
\Msg{* version 2.3c or later.}
\Msg{*}
\Msg{* An older version of docstrip has been input}
\Msg{*}
\Msg{**********************************************}
\errhelp{Move or rename old docstrip.tex.}
\errmessage{Old docstrip in input path}
\batchmode
\csname @@end\endcsname
\fi}

\declarepreamble\mainpreamble
This is a generated file.

Copyright 1989-2005 Johannes L. Braams and any individual authors
listed elsewhere in this file.  All rights reserved.

This file was generated from file(s) of the Babel system.
---------------------------------------------------------

It may be distributed and/or modified under the
conditions of the LaTeX Project Public License, either version 1.3
of this license or (at your option) any later version.
The latest version of this license is in
  http://www.latex-project.org/lppl.txt
and version 1.3 or later is part of all distributions of LaTeX
version 2003/12/01 or later.

This work has the LPPL maintenance status "maintained".

The Current Maintainer of this work is Johannes Braams.

This file may only be distributed together with a copy of the Babel
system. You may however distribute the Babel system without
such generated files.

The list of all files belonging to the Babel distribution is
given in the file `manifest.bbl'. See also `legal.bbl for additional
information.

The list of derived (unpacked) files belonging to the distribution
and covered by LPPL is defined by the unpacking scripts (with
extension .ins) which are part of the distribution.
\endpreamble

\declarepreamble\fdpreamble
This is a generated file.

Copyright 1989-2005 Johannes L. Braams and any individual authors
listed elsewhere in this file.  All rights reserved.

This file was generated from file(s) of the Babel system.
---------------------------------------------------------

It may be distributed and/or modified under the
conditions of the LaTeX Project Public License, either version 1.3
of this license or (at your option) any later version.
The latest version of this license is in
  http://www.latex-project.org/lppl.txt
and version 1.3 or later is part of all distributions of LaTeX
version 2003/12/01 or later.

This work has the LPPL maintenance status "maintained".

The Current Maintainer of this work is Johannes Braams.

This file may only be distributed together with a copy of the Babel
system. You may however distribute the Babel system without
such generated files.

The list of all files belonging to the Babel distribution is
given in the file `manifest.bbl'. See also `legal.bbl for additional
information.

In particular, permission is granted to customize the declarations in
this file to serve the needs of your installation.

However, NO PERMISSION is granted to distribute a modified version
of this file under its original name.

\endpreamble

\usedir{tex/generic/babel}
\keepsilent
 
\usepreamble\mainpreamble

\generate{\file{serbian.ldf}{\from{serbian.dtx}{code}}}

\usepreamble\fdpreamble
 
\ifToplevel{
\Msg{***********************************************************}
\Msg{*}
\Msg{* To finish the installation you have to move the following}
\Msg{* files into a directory searched by TeX:}
\Msg{*}
\Msg{* \space\space All *.fd}
\Msg{*}
\Msg{* To produce the documentation run the files ending with}
\Msg{* `.fdd' through LaTeX.}
\Msg{*}
\Msg{* Happy TeXing}
\Msg{***********************************************************}
}
 
\endinput
}
\DeclareOption{slovak}{% \iffalse meta-comment
%
% Copyright 1989-2008 Johannes L. Braams and any individual authors
% listed elsewhere in this file.  All rights reserved.
% 
% This file is part of the Babel system.
% --------------------------------------
% 
% It may be distributed and/or modified under the
% conditions of the LaTeX Project Public License, either version 1.3
% of this license or (at your option) any later version.
% The latest version of this license is in
%   http://www.latex-project.org/lppl.txt
% and version 1.3 or later is part of all distributions of LaTeX
% version 2003/12/01 or later.
% 
% This work has the LPPL maintenance status "maintained".
% 
% The Current Maintainer of this work is Johannes Braams.
% 
% The list of all files belonging to the Babel system is
% given in the file `manifest.bbl. See also `legal.bbl' for additional
% information.
% 
% The list of derived (unpacked) files belonging to the distribution
% and covered by LPPL is defined by the unpacking scripts (with
% extension .ins) which are part of the distribution.
% \fi
% \CheckSum{1382}
% \iffalse
%    Tell the \LaTeX\ system who we are and write an entry on the
%    transcript.
%<*dtx>
\ProvidesFile{slovak.dtx}
%</dtx>
%<+code>\ProvidesLanguage{slovak}
%\fi
%\ProvidesFile{slovak.dtx}
        [2008/07/06 v3.1a Slovak support from the babel system]
%\iffalse
%% File `slovak.dtx' 
%% Babel package for LaTeX version 2e
%% Copyright (C) 1989 - 2008
%%           by Johannes Braams, TeXniek
%
%% Slovak Language Definition File
%% Copyright (C) 1989 - 2001
%%           by Jana Chlebikova
%            Department of Artificial Intelligence
%            Faculty of Mathematics and Physics
%            Mlynska dolina
%            84215 Bratislava
%            Slovakia
%            (42)(7) 720003 l. 835
%            (42)(7) 725882
%            chlebikj at mff.uniba.cs (Internet)
%            and Johannes Braams, TeXniek
%
%% Copyright (C) 2002-2005
%%           by Tobias Schlemmer
%            Braunsdorfer Stra\ss e 101
%            01159 Dresden
%            Deutschland
%            Tobias.Schlemmer at web.de
%
%% Copyright (C) 2005-2008
%%           by Petr Tesa\v r\'ik (babel at tesarici.cz)
%
% This file is also based on CSLaTeX
%                       by Ji\v r\'i Zlatu\v ska, Zden\v ek Wagner,
%                          Jaroslav \v Snajdr and Petr Ol\v s\'ak.
%
%% Please report errors to: Petr Tesa\v r\'ik
%%                          babel at tesarici.cz
%
%    This file is part of the babel system, it provides the source
%    code for the Slovak language definition file.
%<*filedriver>
\documentclass{ltxdoc}
\newcommand*\TeXhax{\TeX hax}
\newcommand*\babel{\textsf{babel}}
\newcommand*\langvar{$\langle \it lang \rangle$}
\newcommand*\note[1]{}
\newcommand*\Lopt[1]{\textsf{#1}}
\newcommand*\file[1]{\texttt{#1}}
\begin{document}
 \DocInput{slovak.dtx}
\end{document}
%</filedriver>
%\fi
%
% \def\CS{$\cal C\kern-.1667em
%   \lower.5ex\hbox{$\cal S$}\kern-.075em$}
%
% \GetFileInfo{slovak.dtx}
%
% \changes{slovak-1.0}{1992/07/15}{First version}
% \changes{slovak-1.2}{1994/02/27}{Update for \LaTeXe}
% \changes{slovak-1.2d}{1994/06/26}{Removed the use of \cs{filedate}
%    and moved identification after the loading of \file{babel.def}}
% \changes{slovak-1.2i}{1996/10/10}{Replaced \cs{undefined} with
%    \cs{@undefined} and \cs{empty} with \cs{@empty} for consistency
%    with \LaTeX, moved the definition of \cs{atcatcode} right to the
%    beginning.}
% \changes{slovak-1.3a}{2004/02/20}{Added contributed shorthand
%    definitions} 
% \changes{slovak-3.0}{2005/09/10}{Implemented the functionality of
%    \CS\LaTeX's slovak.sty.  The version number was bumped to 3.0
%    to minimize confusion by being higher than the last version
%    of \CS\LaTeX.}
%
%  \section{The Slovak language}
%
%    The file \file{\filename}\footnote{The file described in this
%    section has version number \fileversion\ and was last revised on
%    \filedate.  It was originally written by Jana Chlebikova
%    (\texttt{chlebik@euromath.dk}) and modified by Tobias Schlemmer
%    (\texttt{Tobias.Schlemmer@web.de}). It was then rewritten by
%    Petr Tesa\v r\'ik (\texttt{babel@tesarici.cz}).} defines all the
%    language-specific macros for the Slovak language.
%
%    For this language the macro |\q| is defined. It was used with the
%    letters (\texttt{t}, \texttt{d}, \texttt{l}, and \texttt{L}) and
%    adds a \texttt{'} to them to simulate a `hook' that should be
%    there.  The result looks like t\kern-2pt\char'47. Since the the T1
%    font encoding has the corresponding characters it is mapped to |\v|.
%    Therefore we recommend using T1 font encoding. If you don't want to
%    use this encoding, please, feel free to redefine |\q| in your file.
%    I think babel will honour this |;-)|.
%
%    For this language the characters |"|, |'| and |^| are made
%    active. In table~\ref{tab:slovak-quote} an overview is given of
%    its purpose. Also the vertical placement of the
%    umlaut can be controlled this way.
%
%    \begin{table}[htb]
%     \begin{center}
%     \begin{tabular}{lp{8cm}}
%      |"a| & |\"a|, also implemented for the other
%                  lowercase and uppercase vowels.                 \\
%      |^d| & |\q d|, also implemented for l, t and L.             \\
%      |^c| & |\v c|, also implemented for C, D, N, n, T, Z and z. \\
%      |^o| & |\^o|, also implemented for O.                       \\
%      |'a| & |\'a|, also implemented for the other lowercase and
%                    uppercase l, r, y and vowels.                 \\
%      \verb="|= & disable ligature at this position.              \\
%      |"-| & an explicit hyphen sign, allowing hyphenation
%             in the rest of the word.                             \\
%      |""| & like |"-|, but producing no hyphen sign
%             (for compund words with hyphen, e.g.\ |x-""y|).      \\
%      |"~| & for a compound word mark without a breakpoint.       \\
%      |"=| & for a compound word mark with a breakpoint, allowing
%             hyphenation in the composing words.                  \\
%      |"`| & for German left double quotes (looks like ,,).       \\
%      |"'| & for German right double quotes.                      \\
%      |"<| & for French left double quotes (similar to $<<$).     \\
%      |">| & for French right double quotes (similar to $>>$).    \\
%     \end{tabular}
%     \caption{The extra definitions made
%              by \file{slovak.ldf}}\label{tab:slovak-quote}
%     \end{center}
%    \end{table}
%
%    The quotes in table~\ref{tab:slovak-quote} can also be typeset by
%    using the commands in table~\ref{tab:smore-quote}.
%    \begin{table}[htb]
%     \begin{center}
%     \begin{tabular}{lp{8cm}}
%      |\glqq| & for German left double quotes (looks like ,,).   \\
%      |\grqq| & for German right double quotes (looks like ``).  \\
%      |\glq|  & for German left single quotes (looks like ,).    \\
%      |\grq|  & for German right single quotes (looks like `).   \\
%      |\flqq| & for French left double quotes (similar to $<<$). \\
%      |\frqq| & for French right double quotes (similar to $>>$).\\
%      |\flq|  & for (French) left single quotes (similar to $<$).  \\
%      |\frq|  & for (French) right single quotes (similar to $>$). \\
%      |\dq|   & the original quotes character (|"|).        \\
%      |\sq|   & the original single quote (|'|).            \\
%     \end{tabular}
%     \caption{More commands which produce quotes, defined
%              by \file{slovak.ldf}}\label{tab:smore-quote}
%     \end{center}
%    \end{table}
%
%  \subsection{Compatibility}
%
%    Great care has been taken to ensure backward compatibility with
%    \CS\LaTeX.  In particular, documents which load this file with
%    |\usepackage{slovak}| should produce identical output with no
%    modifications to the source.  Additionally, all the \CS\LaTeX{}
%    options are recognized:
%
%    \label{tab:cslatex-options}
%    \begin{list}{}
%     {\def\makelabel#1{\sbox0{\Lopt{#1}}%
%        \ifdim\wd0>\labelwidth
%          \setbox0\vbox{\box0\hbox{}} \wd0=0pt \fi
%        \box0\hfil}
%      \setlength{\labelwidth}{2\parindent}
%      \setlength{\leftmargin}{2\parindent}
%      \setlength{\rightmargin}{\parindent}}
%     \item[IL2, T1, OT1]
%       These options set the default font encoding.  Please note
%       that their use is deprecated. You should use the |fontenc|
%       package to select font encoding.
%
%     \item[split, nosplit]
%       These options control whether hyphenated words are
%       automatically split according to Slovak typesetting rules.
%       With the \Lopt{split} option ``je-li'' is hyphenated as
%       ``je-/-li''. The \Lopt{nosplit} option disables this behavior.
%
%       The use of this option is strongly discouraged, as it breaks
%       too many common things---hyphens cannot be used in labels,
%       negative arguments to \TeX{} primitives will not work in
%       horizontal mode (use \cs{minus} as a workaround), and there are
%       a few other peculiarities with using this mode.
%
%     \item[nocaptions]
%
%       This option was used in \CS\LaTeX{} to set up Czech/Slovak
%       typesetting rules, but leave the original captions and dates.
%       The recommended way to achieve this is to use English as the main
%       language of the document and use the environment |otherlanguage*|
%       for Czech text.
%
%     \item[olduv]
%       There are two version of \cs{uv}.  The older one allows the use
%       of \cs{verb} inside the quotes but breaks any respective kerning
%       with the quotes (like that in \CS{} fonts).  The newer one honors
%       the kerning in the font but does not allow \cs{verb} inside the
%       quotes.
%
%       The new version is used by default in \LaTeXe{} and the old version
%       is used with plain \TeX.  You may use \Lopt{olduv} to override the
%       default in \LaTeXe.
%       
%     \item[cstex]
%       This option was used to include the commands \cs{csprimeson} and
%       \cs{csprimesoff}.  Since these commands are always included now,
%       it has been removed and the empty definition lasts for compatibility.
%    \end{list}
%
% \StopEventually{}
%
%  \subsection{Implementation}
%
%    The macro |\LdfInit| takes care of preventing that this file is
%    loaded more than once, checking the category code of the
%    \texttt{@} sign, etc.
% \changes{slovak-1.2i}{1996/11/03}{Now use \cs{LdfInit} to perform
%    initial checks} 
%    \begin{macrocode}
%<*code>
\LdfInit\CurrentOption{date\CurrentOption}
%    \end{macrocode}
%
%    When this file is read as an option, i.e. by the |\usepackage|
%    command, \texttt{slovak} will be an `unknown' language in which
%    case we have to make it known. So we check for the existence of
%    |\l@slovak| to see whether we have to do something here.
%
% \changes{slovak-1.2d}{1994/06/26}{Now use \cs{@nopatterns} to
%    produce the warning}
%    \begin{macrocode}
\ifx\l@slovak\@undefined
    \@nopatterns{Slovak}
    \adddialect\l@slovak0\fi
%    \end{macrocode}
%
%    We need to define these macros early in the process.
%
%    \begin{macrocode}
\def\cs@iltw@{IL2}
\newif\ifcs@splithyphens
\cs@splithyphensfalse
%    \end{macrocode}
%
%    If Babel is not loaded, we provide compatibility with \CS\LaTeX.
%    However, if macro \cs{@ifpackageloaded} is not defined, we assume
%    to be loaded from plain and provide compatibility with csplain.
%    Of course, this does not work well with \LaTeX$\:$2.09, but I
%    doubt anyone will ever want to use this file with \LaTeX$\:$2.09.
%
%    \begin{macrocode}
\ifx\@ifpackageloaded\@undefined
  \let\cs@compat@plain\relax
  \message{csplain compatibility mode}
\else
  \@ifpackageloaded{babel}{}{%
    \let\cs@compat@latex\relax
    \message{cslatex compatibility mode}}
\fi
\ifx\cs@compat@latex\relax
  \ProvidesPackage{slovak}[2008/07/06 v3.1a CSTeX Slovak style]
%    \end{macrocode}
%
%    Declare \CS\LaTeX{} options (see also the descriptions on page
%    \pageref{tab:cslatex-options}).
%
%    \begin{macrocode}
  \DeclareOption{IL2}{\def\encodingdefault{IL2}}
  \DeclareOption {T1}{\def\encodingdefault {T1}}
  \DeclareOption{OT1}{\def\encodingdefault{OT1}}
  \DeclareOption{nosplit}{\cs@splithyphensfalse}
  \DeclareOption{split}{\cs@splithyphenstrue}
  \DeclareOption{nocaptions}{\let\cs@nocaptions=\relax}
  \DeclareOption{olduv}{\let\cs@olduv=\relax}
  \DeclareOption{cstex}{\relax}
%    \end{macrocode}
%
%    Make |IL2| encoding the default.  This can be overriden with
%    the other font encoding options.
%    \begin{macrocode}
  \ExecuteOptions{\cs@iltw@}
%    \end{macrocode}
%
%    Now, process the user-supplied options.
%    \begin{macrocode}
  \ProcessOptions
%    \end{macrocode}
%
%    Standard \LaTeXe{} does not include the IL2 encoding in the format.
%    The encoding can be loaded later using the fontenc package, but
%    \CS\LaTeX{} included IL2 by default.  This means existing documents
%    for \CS\LaTeX{} do not load that package, so load the encoding
%    ourselves in compatibility mode.
%
%    \begin{macrocode}
  \ifx\encodingdefault\cs@iltw@
    \input il2enc.def
  \fi
%    \end{macrocode}
%
%    Restore the definition of \cs{CurrentOption}, clobbered by processing
%    the options.
%
%    \begin{macrocode}
  \def\CurrentOption{slovak}
\fi
%    \end{macrocode}
%
%    The next step consists of defining commands to switch to (and
%    from) the Slovak language.
%
%  \begin{macro}{\captionsslovak}
%    The macro \cs{captionsslovak} defines all strings used in the four
%    standard documentclasses provided with \LaTeX.
% \changes{slovak-1.2g}{1995/07/04}{Added \cs{proofname} for
%    AMS-\LaTeX}
% \changes{slovak-1.2k}{1999/02/28}{Repaired a few spelling mistakes}
% \changes{slovak-1.2l}{2000/09/20}{Added \cs{glossaryname}}
% \changes{slovak-3.0}{2005/12/22}{Updated some translations.  Former
%    translations were: `\'Uvod' for \cs{prefacename}, `Referencie' for
%    \cs{refname}, `Index' for \cs{indexname}, `Obr\'azok' for
%    \cs{figurename}, `Pr\'ilohy' for \cs{enclname}, `CC' for \cs{ccname},
%    `Komu' for \cs{headtoname}, `Strana' for \cs{pagename}}
%    \begin{macrocode}
\@namedef{captions\CurrentOption}{%
  \def\prefacename{Predhovor}%
  \def\refname{Literat\'ura}%
  \def\abstractname{Abstrakt}%
  \def\bibname{Literat\'ura}%
  \def\chaptername{Kapitola}%
  \def\appendixname{Dodatok}%
  \def\contentsname{Obsah}%
  \def\listfigurename{Zoznam obr\'azkov}%
  \def\listtablename{Zoznam tabuliek}%
  \def\indexname{Register}%
  \def\figurename{Obr.}%
  \def\tablename{Tabu\v{l}ka}%
  \def\partname{\v{C}as\v{t}}%
  \def\enclname{Pr\'{\i}loha}%
  \def\ccname{cc.}%
  \def\headtoname{Pre}%
  \def\pagename{Str.}%
  \def\seename{vi\v{d}}%
  \def\alsoname{vi\v{d} tie\v{z}}%
  \def\proofname{D\^okaz}%
  \def\glossaryname{Slovn\'{\i}k}%
  }%
%    \end{macrocode}
%  \end{macro}
%
%  \begin{macro}{\dateslovak}
%    The macro \cs{dateslovak} redefines the command \cs{today}
%    to produce Slovak dates.
% \changes{slovak-1.2j}{1997/10/01}{Use \cs{edef} to define
%    \cs{today} to save memory}
% \changes{slovak-1.2j}{1998/03/28}{use \cs{def} instead of \cs{edef}}
% \changes{slovak-1.2k}{1999/02/28}{Repaired a spelling mistake}
%    \begin{macrocode}
\@namedef{date\CurrentOption}{%
  \def\today{\number\day.~\ifcase\month\or
    janu\'ara\or febru\'ara\or marca\or apr\'{\i}la\or m\'aja\or
    j\'una\or j\'ula\or augusta\or septembra\or okt\'obra\or
    novembra\or decembra\fi
    \space \number\year}}
%    \end{macrocode}
%  \end{macro}
%
%  \begin{macro}{\extrasslovak}
%  \begin{macro}{\noextrasslovak}
%    The macro \cs{extrasslovak} will perform all the extra definitions
%    needed for the Slovak language. The macro \cs{noextrasslovak} is
%    used to cancel the actions of \cs{extrasslovak}.
%
%    For Slovak texts \cs{frenchspacing} should be in effect.  Language
%    group for shorthands is also set here.
% \changes{slovak-3.0}{1995/04/21}{now use \cs{bbl@frenchspacing} and
%    \cs{bbl@nonfrenchspacing}}
% \changes{slovak-3.1}{2006/10/07}{move \cs{languageshorthands} here,
%    so that the language group is always initialized correctly}
%    \begin{macrocode}
\expandafter\addto\csname extras\CurrentOption\endcsname{%
  \bbl@frenchspacing
  \languageshorthands{slovak}}
\expandafter\addto\csname noextras\CurrentOption\endcsname{%
  \bbl@nonfrenchspacing}
%    \end{macrocode}
%
% \changes{slovak-1.2e}{1995/05/28}{Use \LaTeX's \cs{v} accent
%    command}
%    \begin{macrocode}
\expandafter\addto\csname extras\CurrentOption\endcsname{%
  \babel@save\q\let\q\v}
%    \end{macrocode}
%
%    For Slovak three characters are used to define shorthands, they
%    need to be made active.
% \changes{slovak-1.3a}{2004/02/20}{Make three characters available
%    for shorthands}
%    \begin{macrocode}
\ifx\cs@compat@latex\relax\else
  \initiate@active@char{^}
  \addto\extrasslovak{\bbl@activate{^}}
  \addto\noextrasslovak{\bbl@deactivate{^}}
  \initiate@active@char{"}
  \addto\extrasslovak{\bbl@activate{"}\umlautlow}
  \addto\noextrasslovak{\bbl@deactivate{"}\umlauthigh}
  \initiate@active@char{'}
  \@ifpackagewith{babel}{activeacute}{%
    \addto\extrasslovak{\bbl@activate{'}}
    \addto\noextrasslovak{\bbl@deactivate{'}}%
    }{}
\fi
%    \end{macrocode}
%  \end{macro}
%  \end{macro}
%
%  \begin{macro}{\sq}
%  \begin{macro}{\dq}
%    We save the original single and double quote characters in
%    \cs{sq} and \cs{dq} to make them available later.  The math
%    accent |\"| can now be typed as |"|.
%    \begin{macrocode}
\begingroup\catcode`\"=12\catcode`\'=12
\def\x{\endgroup
  \def\sq{'}
  \def\dq{"}}
\x
%    \end{macrocode}
%  \end{macro}
%  \end{macro}
%
% \changes{slovak-1.2b}{1994/06/04}{Added setting of left- and
%    righthyphenmin}
%
%    The slovak hyphenation patterns should be used with
%    |\lefthyphenmin| set to~2 and |\righthyphenmin| set to~3.
% \changes{slovak-1.2e}{1995/05/28}{Now use \cs{slovakhyphenmins}}
% \changes{slovak-1.2l}{2000/09/22}{Now use \cs{providehyphenmins} to
%    provide a default value}
% \changes{slovak-3.0}{2005/12/22}{Changed default \cs{righthyphenmin}
%    to 3 characters.}
%    \begin{macrocode}
\providehyphenmins{\CurrentOption}{\tw@\thr@@}
%    \end{macrocode}
%
%    In order to prevent problems with the active |^| we add a
%    shorthand on system level which expands to a `normal |^|.
%    \begin{macrocode}
\ifx\cs@compat@latex\relax\else
  \declare@shorthand{system}{^}{\csname normal@char\string^\endcsname}
%    \end{macrocode}
%
%    Now we can define the doublequote macros: the umlauts,
%    \begin{macrocode}
  \declare@shorthand{slovak}{"a}{\textormath{\"{a}\allowhyphens}{\ddot a}}
  \declare@shorthand{slovak}{"o}{\textormath{\"{o}\allowhyphens}{\ddot o}}
  \declare@shorthand{slovak}{"u}{\textormath{\"{u}\allowhyphens}{\ddot u}}
  \declare@shorthand{slovak}{"A}{\textormath{\"{A}\allowhyphens}{\ddot A}}
  \declare@shorthand{slovak}{"O}{\textormath{\"{O}\allowhyphens}{\ddot O}}
  \declare@shorthand{slovak}{"U}{\textormath{\"{U}\allowhyphens}{\ddot U}}
%    \end{macrocode}
%    tremas,
%    \begin{macrocode}
  \declare@shorthand{slovak}{"e}{\textormath{\"{e}\allowhyphens}{\ddot e}}
  \declare@shorthand{slovak}{"E}{\textormath{\"{E}\allowhyphens}{\ddot E}}
  \declare@shorthand{slovak}{"i}{\textormath{\"{\i}\allowhyphens}%
                                {\ddot\imath}}
  \declare@shorthand{slovak}{"I}{\textormath{\"{I}\allowhyphens}{\ddot I}}
%    \end{macrocode}
%    other slovak characters
%    \begin{macrocode}
  \declare@shorthand{slovak}{^c}{\textormath{\v{c}\allowhyphens}{\check{c}}}
  \declare@shorthand{slovak}{^d}{\textormath{\q{d}\allowhyphens}{\check{d}}}
  \declare@shorthand{slovak}{^l}{\textormath{\q{l}\allowhyphens}{\check{l}}}
  \declare@shorthand{slovak}{^n}{\textormath{\v{n}\allowhyphens}{\check{n}}}
  \declare@shorthand{slovak}{^o}{\textormath{\^{o}\allowhyphens}{\hat{o}}}
  \declare@shorthand{slovak}{^s}{\textormath{\v{s}\allowhyphens}{\check{s}}}
  \declare@shorthand{slovak}{^t}{\textormath{\q{t}\allowhyphens}{\check{t}}}
  \declare@shorthand{slovak}{^z}{\textormath{\v{z}\allowhyphens}{\check{z}}}
  \declare@shorthand{slovak}{^C}{\textormath{\v{C}\allowhyphens}{\check{C}}}
  \declare@shorthand{slovak}{^D}{\textormath{\v{D}\allowhyphens}{\check{D}}}
  \declare@shorthand{slovak}{^L}{\textormath{\q{L}\allowhyphens}{\check{L}}}
  \declare@shorthand{slovak}{^N}{\textormath{\v{N}\allowhyphens}{\check{N}}}
  \declare@shorthand{slovak}{^O}{\textormath{\^{O}\allowhyphens}{\hat{O}}}
  \declare@shorthand{slovak}{^S}{\textormath{\v{S}\allowhyphens}{\check{S}}}
  \declare@shorthand{slovak}{^T}{\textormath{\v{T}\allowhyphens}{\check{T}}}
  \declare@shorthand{slovak}{^Z}{\textormath{\v{Z}\allowhyphens}{\check{Z}}}
%    \end{macrocode}
%     acute accents,
%    \begin{macrocode}
  \@ifpackagewith{babel}{activeacute}{%
    \declare@shorthand{slovak}{'a}{\textormath{\'a\allowhyphens}{^{\prime}a}}
    \declare@shorthand{slovak}{'e}{\textormath{\'e\allowhyphens}{^{\prime}e}}
    \declare@shorthand{slovak}{'i}{\textormath{\'\i{}\allowhyphens}{^{\prime}i}}
    \declare@shorthand{slovak}{'l}{\textormath{\'l\allowhyphens}{^{\prime}l}}
    \declare@shorthand{slovak}{'o}{\textormath{\'o\allowhyphens}{^{\prime}o}}
    \declare@shorthand{slovak}{'r}{\textormath{\'r\allowhyphens}{^{\prime}r}}
    \declare@shorthand{slovak}{'u}{\textormath{\'u\allowhyphens}{^{\prime}u}}
    \declare@shorthand{slovak}{'y}{\textormath{\'y\allowhyphens}{^{\prime}y}}
    \declare@shorthand{slovak}{'A}{\textormath{\'A\allowhyphens}{^{\prime}A}}
    \declare@shorthand{slovak}{'E}{\textormath{\'E\allowhyphens}{^{\prime}E}}
    \declare@shorthand{slovak}{'I}{\textormath{\'I\allowhyphens}{^{\prime}I}}
    \declare@shorthand{slovak}{'L}{\textormath{\'L\allowhyphens}{^{\prime}l}}
    \declare@shorthand{slovak}{'O}{\textormath{\'O\allowhyphens}{^{\prime}O}}
    \declare@shorthand{slovak}{'R}{\textormath{\'R\allowhyphens}{^{\prime}R}}
    \declare@shorthand{slovak}{'U}{\textormath{\'U\allowhyphens}{^{\prime}U}}
    \declare@shorthand{slovak}{'Y}{\textormath{\'Y\allowhyphens}{^{\prime}Y}}
    \declare@shorthand{slovak}{''}{%
      \textormath{\textquotedblright}{\sp\bgroup\prim@s'}}
    }{}
  %    \end{macrocode}
%    and some additional commands:
%    \begin{macrocode}
  \declare@shorthand{slovak}{"-}{\nobreak\-\bbl@allowhyphens}
  \declare@shorthand{slovak}{"|}{%
    \textormath{\penalty\@M\discretionary{-}{}{\kern.03em}%
                \bbl@allowhyphens}{}}
  \declare@shorthand{slovak}{""}{\hskip\z@skip}
  \declare@shorthand{slovak}{"~}{\textormath{\leavevmode\hbox{-}}{-}}
  \declare@shorthand{slovak}{"=}{\cs@splithyphen}
\fi
%    \end{macrocode}
%
%  \begin{macro}{\v}
%    \LaTeX's normal |\v| accent places a caron over the letter that
%    follows it (\v{o}). This is not what we want for the letters d,
%    t, l and L; for those the accent should change shape. This is
%    acheived by the following.
%    \begin{macrocode}
\AtBeginDocument{%
  \DeclareTextCompositeCommand{\v}{OT1}{t}{%
    t\kern-.23em\raise.24ex\hbox{'}}
  \DeclareTextCompositeCommand{\v}{OT1}{d}{%
    d\kern-.13em\raise.24ex\hbox{'}}
  \DeclareTextCompositeCommand{\v}{OT1}{l}{\lcaron{}}
  \DeclareTextCompositeCommand{\v}{OT1}{L}{\Lcaron{}}}
%    \end{macrocode}
%
%  \begin{macro}{\lcaron}
%  \begin{macro}{\Lcaron}
%    For the letters \texttt{l} and \texttt{L} we want to disinguish
%    between normal fonts and monospaced fonts.
%    \begin{macrocode}
\def\lcaron{%
  \setbox0\hbox{M}\setbox\tw@\hbox{i}%
  \ifdim\wd0>\wd\tw@\relax
    l\kern-.13em\raise.24ex\hbox{'}\kern-.11em%
  \else
    l\raise.45ex\hbox to\z@{\kern-.35em '\hss}%
  \fi}
\def\Lcaron{%
  \setbox0\hbox{M}\setbox\tw@\hbox{i}%
  \ifdim\wd0>\wd\tw@\relax
    L\raise.24ex\hbox to\z@{\kern-.28em'\hss}%
  \else
    L\raise.45ex\hbox to\z@{\kern-.40em '\hss}%
  \fi}
%    \end{macrocode}
%  \end{macro}
%  \end{macro}
%  \end{macro}
%
%    Initialize active quotes.  \CS\LaTeX{} provides a way of
%    converting English-style quotes into Slovak-style ones.  Both
%    single and double quotes are affected, i.e. |``text''| is
%    converted to something like |,,text``| and |`text'| is converted
%    to |,text`|.  This conversion can be switched on and off with
%    \cs{csprimeson} and \cs{csprimesoff}.\footnote{By the way, the
%    names of these macros are misleading, because the handling of
%    primes in math mode is rather marginal, the most important thing
%    being the handling of quotes\ldots}
%
%    These quotes present various troubles, e.g. the kerning is broken,
%    apostrophes are converted to closing single quote, some primitives
%    are broken (most notably the |\catcode`\|\meta{char} syntax will
%    not work any more), and writing them to \file{.aux} files cannot
%    be handled correctly.  For these reasons, these commands are only
%    available in \CS\LaTeX{} compatibility mode.
%
%    \begin{macrocode}
\ifx\cs@compat@latex\relax
  \let\cs@ltxprim@s\prim@s
  \def\csprimeson{%
    \catcode``\active \catcode`'\active \let\prim@s\bbl@prim@s}
  \def\csprimesoff{%
    \catcode``12 \catcode`'12 \let\prim@s\cs@ltxprim@s}
  \begingroup\catcode``\active
  \def\x{\endgroup
    \def`{\futurelet\cs@next\cs@openquote}
    \def\cs@openquote{%
      \ifx`\cs@next \expandafter\cs@opendq
      \else \expandafter\clq
      \fi}%
  }\x
  \begingroup\catcode`'\active
  \def\x{\endgroup
    \def'{\textormath{\futurelet\cs@next\cs@closequote}
                     {^\bgroup\prim@s}}
    \def\cs@closequote{%
      \ifx'\cs@next \expandafter\cs@closedq
      \else \expandafter\crq
      \fi}%
  }\x
  \def\cs@opendq{\clqq\let\cs@next= }
  \def\cs@closedq{\crqq\let\cs@next= }
%    \end{macrocode}
%
%    The way I recommend for typesetting quotes in Slovak documents
%    is to use shorthands similar to those used in German.
%    
% \changes{3.0}{2006/04/21}{|"`| and |"'| changed from German quotes
%   to Slovak quotes}
%    \begin{macrocode}
\else
  \declare@shorthand{slovak}{"`}{\clqq}
  \declare@shorthand{slovak}{"'}{\crqq}
  \declare@shorthand{slovak}{"<}{\flqq}
  \declare@shorthand{slovak}{">}{\frqq}
\fi
%    \end{macrocode}
%
%  \begin{macro}{\clqq}
%    This is the CS opening quote, which is similar to the German
%    quote (\cs{glqq}) but the kerning is different.
%
%    For the OT1 encoding, the quote is constructed from the right
%    double quote (i.e. the ``Opening quotes'' character) by moving
%    it down to the baseline and shifting it to the right, or to the
%    left if italic correction is positive.
%
%    For T1, the ``German Opening quotes'' is used.  It is moved to
%    the right and the total width is enlarged.  This is done in an
%    attempt to minimize the difference between the OT1 and T1
%    versions.
%
% \changes{3.0}{2006/04/20}{Added \cs{leavevmode} to allow an opening
%   quote at the beginning of a paragraph}
%    \begin{macrocode}
\ProvideTextCommand{\clqq}{OT1}{%
  \set@low@box{\textquotedblright}%
  \setbox\@ne=\hbox{l\/}\dimen\@ne=\wd\@ne
  \setbox\@ne=\hbox{l}\advance\dimen\@ne-\wd\@ne
  \leavevmode
  \ifdim\dimen\@ne>\z@\kern-.1em\box\z@\kern.1em
    \else\kern.1em\box\z@\kern-.1em\fi\allowhyphens}
\ProvideTextCommand{\clqq}{T1}
  {\kern.1em\quotedblbase\kern-.0158em\relax}
\ProvideTextCommandDefault{\clqq}{\UseTextSymbol{OT1}\clqq}
%    \end{macrocode}
%  \end{macro}
%
%  \begin{macro}{\crqq}
%    For OT1, the CS closing quote is basically the same as
%    \cs{grqq}, only the \cs{textormath} macro is not used, because
%    as far as I know, \cs{grqq} does not work in math mode anyway.
%
%    For T1, the character is slightly wider and shifted to the
%    right to match its OT1 counterpart.
%
%    \begin{macrocode}
\ProvideTextCommand{\crqq}{OT1}
  {\save@sf@q{\nobreak\kern-.07em\textquotedblleft\kern.07em}}
\ProvideTextCommand{\crqq}{T1}
  {\save@sf@q{\nobreak\kern.06em\textquotedblleft\kern.024em}}
\ProvideTextCommandDefault{\crqq}{\UseTextSymbol{OT1}\crqq}
%    \end{macrocode}
%  \end{macro}
%
%  \begin{macro}{\clq}
%  \begin{macro}{\crq}
%
%    Single CS quotes are similar to double quotes (see the
%    discussion above).
%
%    \begin{macrocode}
\ProvideTextCommand{\clq}{OT1}
  {\set@low@box{\textquoteright}\box\z@\kern.04em\allowhyphens}
\ProvideTextCommand{\clq}{T1}
  {\quotesinglbase\kern-.0428em\relax}
\ProvideTextCommandDefault{\clq}{\UseTextSymbol{OT1}\clq}
\ProvideTextCommand{\crq}{OT1}
  {\save@sf@q{\nobreak\textquoteleft\kern.17em}}
\ProvideTextCommand{\crq}{T1}
  {\save@sf@q{\nobreak\textquoteleft\kern.17em}}
\ProvideTextCommandDefault{\crq}{\UseTextSymbol{OT1}\crq}
%    \end{macrocode}
%  \end{macro}
%  \end{macro}
%
%  \begin{macro}{\uv}
%    There are two versions of \cs{uv}.  The older one opens a group
%    and uses \cs{aftergroup} to typeset the closing quotes.  This
%    version allows using \cs{verb} inside the quotes, because the
%    enclosed text is not passed as an argument, but unfortunately
%    it breaks any kerning with the quotes.  Although the kerning
%    with the opening quote could be fixed, the kerning with the
%    closing quote cannot.
%
%    The newer version is defined as a command with one parameter.
%    It preserves kerning but since the quoted text is passed as an
%    argument, it cannot contain \cs{verb}.
%
%    Decide which version of \cs{uv} should be used.  For sake
%    of compatibility, we use the older version with plain \TeX{}
%    and the newer version with \LaTeXe.
%    \begin{macrocode}
\ifx\cs@compat@plain\@undefined\else\let\cs@olduv=\relax\fi
\ifx\cs@olduv\@undefined
  \DeclareRobustCommand\uv[1]{{\leavevmode\clqq#1\crqq}}
\else
  \DeclareRobustCommand\uv{\bgroup\aftergroup\closequotes
    \leavevmode\clqq\let\cs@next=}
  \def\closequotes{\unskip\crqq\relax}
\fi
%    \end{macrocode}
%  \end{macro}
%
%  \begin{macro}{\cs@wordlen}
%    Declare a counter to hold the length of the word after the
%    hyphen.
%
%    \begin{macrocode}
\newcount\cs@wordlen
%    \end{macrocode}
%  \end{macro}
%
%  \begin{macro}{\cs@hyphen}
%  \begin{macro}{\cs@endash}
%  \begin{macro}{\cs@emdash}
%    Store the original hyphen in a macro. Ditto for the ligatures.
%
% \changes{slovak-3.1}{2006/10/07}{ensure correct catcode for the
%    saved hyphen}
%    \begin{macrocode}
\begingroup\catcode`\-12
\def\x{\endgroup
  \def\cs@hyphen{-}
  \def\cs@endash{--}
  \def\cs@emdash{---}
%    \end{macrocode}
%  \end{macro}
%  \end{macro}
%  \end{macro}
%
%  \begin{macro}{\cs@boxhyphen}
%    Provide a non-breakable hyphen to be used when a compound word
%    is too short to be split, i.e. the second part is shorter than
%    \cs{righthyphenmin}.
%
%    \begin{macrocode}
  \def\cs@boxhyphen{\hbox{-}}
%    \end{macrocode}
%  \end{macro}
%
%  \begin{macro}{\cs@splithyphen}
%    The macro \cs{cs@splithyphen} inserts a split hyphen, while
%    allowing both parts of the compound word to be hyphenated at
%    other places too.
%
%    \begin{macrocode}
  \def\cs@splithyphen{\kern\z@
    \discretionary{-}{\char\hyphenchar\the\font}{-}\nobreak\hskip\z@}
}\x
%    \end{macrocode}
%  \end{macro}
%
%  \begin{macro}{-}
%    To minimize the effects of activating the hyphen character,
%    the active definition expands to the non-active character
%    in all cases where hyphenation cannot occur, i.e. if not
%    typesetting (check \cs{protect}), not in horizontal mode,
%    or in inner horizontal mode.
%
%    \begin{macrocode}
\initiate@active@char{-}
\declare@shorthand{slovak}{-}{%
  \ifx\protect\@typeset@protect
    \ifhmode
      \ifinner
        \bbl@afterelse\bbl@afterelse\bbl@afterelse\cs@hyphen
      \else
        \bbl@afterfi\bbl@afterelse\bbl@afterelse\cs@firsthyphen
      \fi
    \else
      \bbl@afterfi\bbl@afterelse\cs@hyphen
    \fi
  \else
    \bbl@afterfi\cs@hyphen
  \fi}
%    \end{macrocode}
%  \end{macro}
%
%  \begin{macro}{\cs@firsthyphen}
%  \begin{macro}{\cs@firsthyph@n}
%  \begin{macro}{\cs@secondhyphen}
%  \begin{macro}{\cs@secondhyph@n}
%    If we encounter a hyphen, check whether it is followed
%    by a second or a third hyphen and if so, insert the
%    corresponding ligature.
%
%    If we don't find a hyphen, the token found will be placed
%    in \cs{cs@token} for further analysis, and it will also stay
%    in the input.
%
%    \begin{macrocode}
\begingroup\catcode`\-\active
\def\x{\endgroup
  \def\cs@firsthyphen{\futurelet\cs@token\cs@firsthyph@n}
  \def\cs@firsthyph@n{%
    \ifx -\cs@token
      \bbl@afterelse\cs@secondhyphen
    \else
      \bbl@afterfi\cs@checkhyphen
    \fi}
  \def\cs@secondhyphen ##1{%
    \futurelet\cs@token\cs@secondhyph@n}
  \def\cs@secondhyph@n{%
    \ifx -\cs@token
      \bbl@afterelse\cs@emdash\@gobble
    \else
      \bbl@afterfi\cs@endash
    \fi}
}\x
%    \end{macrocode}
%  \end{macro}
%  \end{macro}
%  \end{macro}
%  \end{macro}
%
%  \begin{macro}{\cs@checkhyphen}
%    Check that hyphenation is enabled, and if so, start analyzing
%    the rest of the word, i.e. initialize \cs{cs@word} and \cs{cs@wordlen}
%    and start processing input with \cs{cs@scanword}.
%
%    \begin{macrocode}
\def\cs@checkhyphen{%
  \ifnum\expandafter\hyphenchar\the\font=`\-
    \def\cs@word{}\cs@wordlen\z@
    \bbl@afterelse\cs@scanword
  \else
    \cs@hyphen
  \fi}
%    \end{macrocode}
%  \end{macro}
%
%  \begin{macro}{\cs@scanword}
%  \begin{macro}{\cs@continuescan}
%  \begin{macro}{\cs@gettoken}
%  \begin{macro}{\cs@gett@ken}
%    Each token is first analyzed with \cs{cs@scanword}, which expands
%    the token and passes the first token of the result to
%    \cs{cs@gett@ken}. If the expanded token is not identical to the
%    unexpanded one, presume that it might be expanded further and
%    pass it back to \cs{cs@scanword} until you get an unexpandable
%    token. Then analyze it in \cs{cs@examinetoken}.
%
%    The \cs{cs@continuescan} macro does the same thing as
%    \cs{cs@scanword}, but it does not require the first token to be
%    in \cs{cs@token} already.
%
%    \begin{macrocode}
\def\cs@scanword{\let\cs@lasttoken= \cs@token\expandafter\cs@gettoken}
\def\cs@continuescan{\let\cs@lasttoken\@undefined\expandafter\cs@gettoken}
\def\cs@gettoken{\futurelet\cs@token\cs@gett@ken}
\def\cs@gett@ken{%
  \ifx\cs@token\cs@lasttoken \def\cs@next{\cs@examinetoken}%
  \else \def\cs@next{\cs@scanword}%
  \fi \cs@next}
%    \end{macrocode}
%  \end{macro}
%  \end{macro}
%  \end{macro}
%  \end{macro}
%
%  \begin{macro}{cs@examinetoken}
%    Examine the token in \cs{cs@token}:
%
%    \begin{itemize}
%    \item
%      If it is a letter (catcode 11) or other (catcode 12), add it
%      to \cs{cs@word} with \cs{cs@addparam}.
%
%    \item
%      If it is the \cs{char} primitive, add it with \cs{cs@expandchar}.
%
%    \item
%      If the token starts or ends a group, ignore it with
%      \cs{cs@ignoretoken}.
%
%    \item
%      Otherwise analyze the meaning of the token with
%      \cs{cs@checkchardef} to detect primitives defined with
%      \cs{chardef}.
%
%    \end{itemize}
%
%    \begin{macrocode}
\def\cs@examinetoken{%
  \ifcat A\cs@token
    \def\cs@next{\cs@addparam}%
  \else\ifcat 0\cs@token
    \def\cs@next{\cs@addparam}%
  \else\ifx\char\cs@token
    \def\cs@next{\afterassignment\cs@expandchar\let\cs@token= }%
  \else\ifx\bgroup\cs@token
    \def\cs@next{\cs@ignoretoken\bgroup}%
  \else\ifx\egroup\cs@token
    \def\cs@next{\cs@ignoretoken\egroup}%
  \else\ifx\begingroup\cs@token
    \def\cs@next{\cs@ignoretoken\begingroup}%
  \else\ifx\endgroup\cs@token
    \def\cs@next{\cs@ignoretoken\endgroup}%
  \else
    \def\cs@next{\expandafter\expandafter\expandafter\cs@checkchardef
      \expandafter\meaning\expandafter\cs@token\string\char\end}%
  \fi\fi\fi\fi\fi\fi\fi\cs@next}
%    \end{macrocode}
%  \end{macro}
%
%  \begin{macro}{\cs@checkchardef}
%    Check the meaning of a token and if it is a primitive defined
%    with \cs{chardef}, pass it to \cs{\cs@examinechar} as if it were
%    a \cs{char} sequence. Otherwise, there are no more word characters,
%    so do the final actions in \cs{cs@nosplit}.
%
%    \begin{macrocode}
\expandafter\def\expandafter\cs@checkchardef
  \expandafter#\expandafter1\string\char#2\end{%
    \def\cs@token{#1}%
    \ifx\cs@token\@empty
      \def\cs@next{\afterassignment\cs@examinechar\let\cs@token= }%
    \else
      \def\cs@next{\cs@nosplit}%
    \fi \cs@next}
%    \end{macrocode}
%  \end{macro}
%
%  \begin{macro}{\cs@ignoretoken}
%    Add a token to \cs{cs@word} but do not update the \cs{cs@wordlen}
%    counter. This is mainly useful for group starting and ending
%    primitives, which need to be preserved, but do not affect the word
%    boundary.
%
%    \begin{macrocode}
\def\cs@ignoretoken#1{%
  \edef\cs@word{\cs@word#1}%
  \afterassignment\cs@continuescan\let\cs@token= }
%    \end{macrocode}
%  \end{macro}
%
%  \begin{macro}{cs@addparam}
%    Add a token to \cs{cs@word} and check its lccode. Note that
%    this macro can only be used for tokens which can be passed as
%    a parameter.
%
%    \begin{macrocode}
\def\cs@addparam#1{%
  \edef\cs@word{\cs@word#1}%
  \cs@checkcode{\lccode`#1}}
%    \end{macrocode}
%  \end{macro}
%
%  \begin{macro}{\cs@expandchar}
%  \begin{macro}{\cs@examinechar}
%    Add a \cs{char} sequence to \cs{cs@word} and check its lccode.
%    The charcode is first parsed in \cs{cs@expandchar} and then the
%    resulting \cs{chardef}-defined sequence is analyzed in
%    \cs{cs@examinechar}.
%
%    \begin{macrocode}
\def\cs@expandchar{\afterassignment\cs@examinechar\chardef\cs@token=}
\def\cs@examinechar{%
  \edef\cs@word{\cs@word\char\the\cs@token\space}%
  \cs@checkcode{\lccode\cs@token}}
%    \end{macrocode}
%  \end{macro}
%  \end{macro}
%
%  \begin{macro}{\cs@checkcode}
%    Check the lccode of a character. If it is zero, it does not count
%    to the current word, so finish it with \cs{cs@nosplit}. Otherwise
%    update the \cs{cs@wordlen} counter and go on scanning the word
%    with \cs{cs@continuescan}. When enough characters are gathered in
%    \cs{cs@word} to allow word break, insert the split hyphen and
%    finish.
%
%    \begin{macrocode}
\def\cs@checkcode#1{%
  \ifnum0=#1
    \def\cs@next{\cs@nosplit}%
  \else
    \advance\cs@wordlen\@ne
    \ifnum\righthyphenmin>\the\cs@wordlen
      \def\cs@next{\cs@continuescan}%
    \else
      \cs@splithyphen
      \def\cs@next{\cs@word}%
    \fi
  \fi \cs@next}
%    \end{macrocode}
%  \end{macro}
%
%  \begin{macro}{\cs@nosplit}
%    Insert a non-breakable hyphen followed by the saved word.
%
%    \begin{macrocode}
\def\cs@nosplit{\cs@boxhyphen\cs@word}
%    \end{macrocode}
%  \end{macro}
%
%  \begin{macro}{\cs@hyphen}
%    The \cs{minus} sequence can be used where the active hyphen
%    does not work, e.g. in arguments to \TeX{} primitives in outer
%    horizontal mode.
%
%    \begin{macrocode}
\let\minus\cs@hyphen
%    \end{macrocode}
%  \end{macro}

%  \begin{macro}{\standardhyphens}
%  \begin{macro}{\splithyphens}
%    These macros control whether split hyphens are allowed in Czech
%    and/or Slovak texts. You may use them in any language, but the
%    split hyphen is only activated for Czech and Slovak.
%
% \changes{slovak-3.1}{2006/10/07}{activate with split hyphens and
%    deactivate with standard hyphens, not vice versa}
%    \begin{macrocode}
\def\standardhyphens{\cs@splithyphensfalse\cs@deactivatehyphens}
\def\splithyphens{\cs@splithyphenstrue\cs@activatehyphens}
%    \end{macrocode}
%  \end{macro}
%  \end{macro}
%
%  \begin{macro}{\cs@splitattr}
%    Now we declare the |split| language attribute.  This is
%    similar to the |split| package option of cslatex, but it
%    only affects Slovak, not Czech.
%
% \changes{slovak-3.1}{2006/10/07}{attribute added}
%    \begin{macrocode}
\def\cs@splitattr{\babel@save\ifcs@splithyphens\splithyphens}
\bbl@declare@ttribute{slovak}{split}{%
  \addto\extrasslovak{\cs@splitattr}}
%    \end{macrocode}
%  \end{macro}
%
%  \begin{macro}{\cs@activatehyphens}
%  \begin{macro}{\cs@deactivatehyphens}
%    These macros are defined as \cs{relax} by default to prevent
%    activating/deactivating the hyphen character. They are redefined
%    when the language is switched to Czech/Slovak. At that moment
%    the hyphen is also activated if split hyphens were requested with
%    \cs{splithyphens}.
%
%    When the language is de-activated, de-activate the hyphen and
%    restore the bogus definitions of these macros.
%
%    \begin{macrocode}
\let\cs@activatehyphens\relax
\let\cs@deactivatehyphens\relax
\expandafter\addto\csname extras\CurrentOption\endcsname{%
  \def\cs@activatehyphens{\bbl@activate{-}}%
  \def\cs@deactivatehyphens{\bbl@deactivate{-}}%
  \ifcs@splithyphens\cs@activatehyphens\fi}
\expandafter\addto\csname noextras\CurrentOption\endcsname{%
  \cs@deactivatehyphens
  \let\cs@activatehyphens\relax
  \let\cs@deactivatehyphens\relax}
%    \end{macrocode}
%  \end{macro}
%  \end{macro}
%
%  \begin{macro}{\cs@looseness}
%  \begin{macro}{\looseness}
%    One of the most common situations where an active hyphen will not
%    work properly is the \cs{looseness} primitive. Change its definition
%    so that it deactivates the hyphen if needed.
%
%    \begin{macrocode}
\let\cs@looseness\looseness
\def\looseness{%
  \ifcs@splithyphens
    \cs@deactivatehyphens\afterassignment\cs@activatehyphens \fi
  \cs@looseness}
%    \end{macrocode}
%  \end{macro}
%  \end{macro}
%
%  \begin{macro}{\cs@selectlanguage}
%  \begin{macro}{\cs@main@language}
%    Specifying the |nocaptions| option means that captions and dates
%    are not redefined by default, but they can be switched on later
%    with \cs{captionsslovak} and/or \cs{dateslovak}. 
%
%    We mimic this behavior by redefining \cs{selectlanguage}.  This
%    macro is called once at the beginning of the document to set the
%    main language of the document.  If this is \cs{cs@main@language},
%    it disables the macros for setting captions and date.  In any
%    case, it restores the original definition of \cs{selectlanguage}
%    and expands it.
%
%    The definition of \cs{selectlanguage} can be shared between Czech
%    and Slovak; the actual language is stored in \cs{cs@main@language}.
%
%    \begin{macrocode}
\ifx\cs@nocaptions\@undefined\else
  \edef\cs@main@language{\CurrentOption}
  \ifx\cs@origselect\@undefined
    \let\cs@origselect=\selectlanguage
    \def\selectlanguage{%
      \let\selectlanguage\cs@origselect
      \ifx\bbl@main@language\cs@main@language
        \expandafter\cs@selectlanguage
      \else
        \expandafter\selectlanguage
      \fi}
    \def\cs@selectlanguage{%
      \cs@tempdisable{captions}%
      \cs@tempdisable{date}%
      \selectlanguage}
%    \end{macrocode}
%
%  \begin{macro}{\cs@tempdisable}
%    \cs{cs@tempdisable} disables a language setup macro temporarily,
%    i.e. the macro with the name of \meta{\#1}|\bbl@main@language|
%    just restores the original definition and purges the saved macro
%    from memory.
%
%    \begin{macrocode}
    \def\cs@tempdisable#1{%
      \def\@tempa{cs@#1}%
      \def\@tempb{#1\bbl@main@language}%
      \expandafter\expandafter\expandafter\let
        \expandafter \csname\expandafter \@tempa \expandafter\endcsname
        \csname \@tempb \endcsname
      \expandafter\edef\csname \@tempb \endcsname{%
        \let \expandafter\noexpand \csname \@tempb \endcsname
          \expandafter\noexpand \csname \@tempa \endcsname
        \let \expandafter\noexpand\csname \@tempa \endcsname
          \noexpand\@undefined}}
%    \end{macrocode}
%  \end{macro}
%
%    These macros are not needed, once the initialization is over.
%
%    \begin{macrocode}
    \@onlypreamble\cs@main@language
    \@onlypreamble\cs@origselect
    \@onlypreamble\cs@selectlanguage
    \@onlypreamble\cs@tempdisable
  \fi
\fi
%    \end{macrocode}
%  \end{macro}
%  \end{macro}
%
%    The encoding of mathematical fonts should be changed to IL2.  This
%    allows to use accented letter in some font families.  Besides,
%    documents do not use CM fonts if there are equivalents in CS-fonts,
%    so there is no need to have both bitmaps of CM-font and CS-font.
%
%    \cs{@font@warning} and \cs{@font@info} are temporarily redefined
%    to avoid annoying font warnings.
%
%    \begin{macrocode}
\ifx\cs@compat@plain\@undefined
\ifx\cs@check@enc\@undefined\else
  \def\cs@check@enc{
    \ifx\encodingdefault\cs@iltw@
      \let\cs@warn\@font@warning \let\@font@warning\@gobble
      \let\cs@info\@font@info    \let\@font@info\@gobble
      \SetSymbolFont{operators}{normal}{\cs@iltw@}{cmr}{m}{n}
      \SetSymbolFont{operators}{bold}{\cs@iltw@}{cmr}{bx}{n}
      \SetMathAlphabet\mathbf{normal}{\cs@iltw@}{cmr}{bx}{n}
      \SetMathAlphabet\mathit{normal}{\cs@iltw@}{cmr}{m}{it}
      \SetMathAlphabet\mathrm{normal}{\cs@iltw@}{cmr}{m}{n}
      \SetMathAlphabet\mathsf{normal}{\cs@iltw@}{cmss}{m}{n}
      \SetMathAlphabet\mathtt{normal}{\cs@iltw@}{cmtt}{m}{n}
      \SetMathAlphabet\mathbf{bold}{\cs@iltw@}{cmr}{bx}{n}
      \SetMathAlphabet\mathit{bold}{\cs@iltw@}{cmr}{bx}{it}
      \SetMathAlphabet\mathrm{bold}{\cs@iltw@}{cmr}{bx}{n}
      \SetMathAlphabet\mathsf{bold}{\cs@iltw@}{cmss}{bx}{n}
      \SetMathAlphabet\mathtt{bold}{\cs@iltw@}{cmtt}{m}{n}
      \let\@font@warning\cs@warn \let\cs@warn\@undefined
      \let\@font@info\cs@info    \let\cs@info\@undefined
    \fi
    \let\cs@check@enc\@undefined}
  \AtBeginDocument{\cs@check@enc}
\fi
\fi
%    \end{macrocode}
%
%  \begin{macro}{cs@undoiltw@}
%
%    The thing is that \LaTeXe{} core only supports the T1 encoding
%    and does not bother changing the uc/lc/sfcodes when encoding
%    is switched. :( However, the IL2 encoding \emph{does} change
%    these codes, so if encoding is switched back from IL2, we must
%    also undo the effect of this change to be compatible with
%    \LaTeXe.  OK, this is not the right\textsuperscript{TM} solution
%    but it works.  Cheers to Petr Ol\v s\'ak.
%
%    \begin{macrocode}
\def\cs@undoiltw@{%
  \uccode158=208 \lccode158=158 \sfcode158=1000
  \sfcode159=1000
  \uccode165=133 \lccode165=165 \sfcode165=1000
  \uccode169=137 \lccode169=169 \sfcode169=1000
  \uccode171=139 \lccode171=171 \sfcode171=1000
  \uccode174=142 \lccode174=174 \sfcode174=1000
  \uccode181=149
  \uccode185=153
  \uccode187=155
  \uccode190=0   \lccode190=0
  \uccode254=222 \lccode254=254 \sfcode254=1000
  \uccode255=223 \lccode255=255 \sfcode255=1000}
%    \end{macrocode}
%  \end{macro}
%
%  \begin{macro}{@@enc@update}
%
%    Redefine the \LaTeXe{} internal function \cs{@@enc@update} to
%    change the encodings correctly.
%
%    \begin{macrocode}
\ifx\cs@enc@update\@undefined
\ifx\@@enc@update\@undefined\else
  \let\cs@enc@update\@@enc@update
  \def\@@enc@update{\ifx\cf@encoding\cs@iltw@\cs@undoiltw@\fi
    \cs@enc@update
    \expandafter\ifnum\csname l@\languagename\endcsname=\the\language
      \expandafter\ifx
      \csname l@\languagename:\f@encoding\endcsname\relax
      \else
        \expandafter\expandafter\expandafter\let
          \expandafter\csname
          \expandafter l\expandafter @\expandafter\languagename
          \expandafter\endcsname\csname l@\languagename:\f@encoding\endcsname
      \fi
      \language=\csname l@\languagename\endcsname\relax
    \fi}
\fi\fi
%    \end{macrocode}
%  \end{macro}
%
%    The macro |\ldf@finish| takes care of looking for a
%    configuration file, setting the main language to be switched on
%    at |\begin{document}| and resetting the category code of
%    \texttt{@} to its original value.
% \changes{slovak-1.2i}{1996/11/03}{Now use \cs{ldf@finish} to wrap up}
%    \begin{macrocode}
\ldf@finish\CurrentOption
%</code>
%    \end{macrocode}
%
% \Finale
%%
%% \CharacterTable
%%  {Upper-case    \A\B\C\D\E\F\G\H\I\J\K\L\M\N\O\P\Q\R\S\T\U\V\W\X\Y\Z
%%   Lower-case    \a\b\c\d\e\f\g\h\i\j\k\l\m\n\o\p\q\r\s\t\u\v\w\x\y\z
%%   Digits        \0\1\2\3\4\5\6\7\8\9
%%   Exclamation   \!     Double quote  \"     Hash (number) \#
%%   Dollar        \$     Percent       \%     Ampersand     \&
%%   Acute accent  \'     Left paren    \(     Right paren   \)
%%   Asterisk      \*     Plus          \+     Comma         \,
%%   Minus         \-     Point         \.     Solidus       \/
%%   Colon         \:     Semicolon     \;     Less than     \<
%%   Equals        \=     Greater than  \>     Question mark \?
%%   Commercial at \@     Left bracket  \[     Backslash     \\
%%   Right bracket \]     Circumflex    \^     Underscore    \_
%%   Grave accent  \`     Left brace    \{     Vertical bar  \|
%%   Right brace   \}     Tilde         \~}
%%
\endinput
}
\DeclareOption{slovene}{%%
%% This file will generate fast loadable files and documentation
%% driver files from the doc files in this package when run through
%% LaTeX or TeX.
%%
%% Copyright 1989-2005 Johannes L. Braams and any individual authors
%% listed elsewhere in this file.  All rights reserved.
%% 
%% This file is part of the Babel system.
%% --------------------------------------
%% 
%% It may be distributed and/or modified under the
%% conditions of the LaTeX Project Public License, either version 1.3
%% of this license or (at your option) any later version.
%% The latest version of this license is in
%%   http://www.latex-project.org/lppl.txt
%% and version 1.3 or later is part of all distributions of LaTeX
%% version 2003/12/01 or later.
%% 
%% This work has the LPPL maintenance status "maintained".
%% 
%% The Current Maintainer of this work is Johannes Braams.
%% 
%% The list of all files belonging to the LaTeX base distribution is
%% given in the file `manifest.bbl. See also `legal.bbl' for additional
%% information.
%% 
%% The list of derived (unpacked) files belonging to the distribution
%% and covered by LPPL is defined by the unpacking scripts (with
%% extension .ins) which are part of the distribution.
%%
%% --------------- start of docstrip commands ------------------
%%
\def\filedate{1999/04/11}
\def\batchfile{slovene.ins}
\input docstrip.tex

{\ifx\generate\undefined
\Msg{**********************************************}
\Msg{*}
\Msg{* This installation requires docstrip}
\Msg{* version 2.3c or later.}
\Msg{*}
\Msg{* An older version of docstrip has been input}
\Msg{*}
\Msg{**********************************************}
\errhelp{Move or rename old docstrip.tex.}
\errmessage{Old docstrip in input path}
\batchmode
\csname @@end\endcsname
\fi}

\declarepreamble\mainpreamble
This is a generated file.

Copyright 1989-2005 Johannes L. Braams and any individual authors
listed elsewhere in this file.  All rights reserved.

This file was generated from file(s) of the Babel system.
---------------------------------------------------------

It may be distributed and/or modified under the
conditions of the LaTeX Project Public License, either version 1.3
of this license or (at your option) any later version.
The latest version of this license is in
  http://www.latex-project.org/lppl.txt
and version 1.3 or later is part of all distributions of LaTeX
version 2003/12/01 or later.

This work has the LPPL maintenance status "maintained".

The Current Maintainer of this work is Johannes Braams.

This file may only be distributed together with a copy of the Babel
system. You may however distribute the Babel system without
such generated files.

The list of all files belonging to the Babel distribution is
given in the file `manifest.bbl'. See also `legal.bbl for additional
information.

The list of derived (unpacked) files belonging to the distribution
and covered by LPPL is defined by the unpacking scripts (with
extension .ins) which are part of the distribution.
\endpreamble

\declarepreamble\fdpreamble
This is a generated file.

Copyright 1989-2005 Johannes L. Braams and any individual authors
listed elsewhere in this file.  All rights reserved.

This file was generated from file(s) of the Babel system.
---------------------------------------------------------

It may be distributed and/or modified under the
conditions of the LaTeX Project Public License, either version 1.3
of this license or (at your option) any later version.
The latest version of this license is in
  http://www.latex-project.org/lppl.txt
and version 1.3 or later is part of all distributions of LaTeX
version 2003/12/01 or later.

This work has the LPPL maintenance status "maintained".

The Current Maintainer of this work is Johannes Braams.

This file may only be distributed together with a copy of the Babel
system. You may however distribute the Babel system without
such generated files.

The list of all files belonging to the Babel distribution is
given in the file `manifest.bbl'. See also `legal.bbl for additional
information.

In particular, permission is granted to customize the declarations in
this file to serve the needs of your installation.

However, NO PERMISSION is granted to distribute a modified version
of this file under its original name.

\endpreamble

\keepsilent

\usedir{tex/generic/babel} 

\usepreamble\mainpreamble
\generate{\file{slovene.ldf}{\from{slovene.dtx}{code}}
          }
\usepreamble\fdpreamble

\ifToplevel{
\Msg{***********************************************************}
\Msg{*}
\Msg{* To finish the installation you have to move the following}
\Msg{* files into a directory searched by TeX:}
\Msg{*}
\Msg{* \space\space All *.def, *.fd, *.ldf, *.sty}
\Msg{*}
\Msg{* To produce the documentation run the files ending with}
\Msg{* '.dtx' and `.fdd' through LaTeX.}
\Msg{*}
\Msg{* Happy TeXing}
\Msg{***********************************************************}
}
 
\endinput
}
\DeclareOption{spanish}{% \iffalse meta-comment
%
% Copyright 1989-2008 Johannes L. Braams and any individual authors
% listed elsewhere in this file.  All rights reserved.
% 
% This file is part of the Babel system.
% --------------------------------------
% 
% It may be distributed and/or modified under the
% conditions of the LaTeX Project Public License, either version 1.3
% of this license or (at your option) any later version.
% The latest version of this license is in
%   http://www.latex-project.org/lppl.txt
% and version 1.3 or later is part of all distributions of LaTeX
% version 2003/12/01 or later.
% 
% This work has the LPPL maintenance status "maintained".
% 
% The Current Maintainer of this work is Johannes Braams.
% 
% The list of all files belonging to the Babel system is
% given in the file `manifest.bbl. See also `legal.bbl' for additional
% information.
% 
% The list of derived (unpacked) files belonging to the distribution
% and covered by LPPL is defined by the unpacking scripts (with
% extension .ins) which are part of the distribution.
% \fi
%  \ifx\langdeffile\undefined
%     \CheckSum{2145}
%  \else
%     \CheckSum{2061}
%  \fi
% \ProvidesFile{spanish.dtx}
%       [2008/07/06 v5.0e Spanish support from the babel system] 
%\iffalse
%% File `spanish.dtx'
%% Babel package for LaTeX version 2e
%% Copyright (C) 1989 - 2008
%%           by Johannes Braams, TeXniek
%
%% Spanish Language Definition File
%% Copyright (C) 1997 - 2008
%%        Javier Bezos (www.texytipografia.com)
%%     and
%%        CervanTeX (www.cervantex.org)
%
%% Please report errors to: Javier Bezos (preferably)
%%                          www.texytipografia.com/contact.html
%%                          J.L. Braams
%%                          www.latex-project.org
%
%    This file is part of the babel system, it provides the source
%    code for the Spanish language definition file.
%    The original version of this file was written by Javier Bezos.
%
%<*filedriver>
\let\ooverb\verb
\documentclass[spanish,a4paper]{ltxdoc}
\let\verb\ooverb
\usepackage[activeacute]{babel}
\usepackage{hyperref}

\let\meta\emph

\usepackage{pslatex}
\usepackage{mathptmx}
\usepackage{color}
\usepackage[cp1252]{inputenc}
\usepackage[T1]{fontenc}
\newcommand\act[1]{%
  \\%
  \makebox[1.5pc][l]{\textcolor{green}{$\surd$}}%
  \textsf{#1}\ignorespaces}
\newcommand\deact[1]{%
  \\%
  \makebox[1.5pc][l]{\textcolor{red}{$\times$}}%
  \texttt{#1}\ignorespaces}
\newcommand\txt{\makebox[1.5pc][l]{\textcolor{blue}{$\Rightarrow$}}\ignorespaces}
\newcommand\con{\makebox[1.5pc][l]{\textcolor{magenta}{$\star$}}\ignorespaces}
\newcommand\alw{%
  \\%
  \makebox[1.5pc][l]{\textcolor{green}{$\surd$}}%
  Se define siempre, sin depender de un grupo.}
\newcommand\opp{\qquad Opci�n de paquete}

%\makeindex

\newcommand*\babel{\textsf{babel}}
\newcommand*\file[1]{\texttt{#1}}

\setlength{\arrayrulewidth}{2\arrayrulewidth}
\newcommand\toprule[1]{\cline{1-#1}\\[-2ex]}
\newcommand\botrule[1]{\\[.6ex]\cline{1-#1}}
\newcommand\hmk{$\string|$}

\newenvironment{decl}[1][]%
    {\par\small\addvspace{4.5ex plus 1ex}%
     \vskip -\parskip
     \ifx\relax#1\relax
        \def\@decl@date{}%
     \else
        \def\@decl@date{\NEWfeature{#1}}%
     \fi
     \noindent%\hspace{-\leftmargini}%
     \begin{tabular}{|l|}\hline\ignorespaces}%
    {\\\hline\end{tabular}\nobreak\@decl@date\par\nobreak
     \vspace{2.3ex}\vskip-\parskip}

\newcommand\New[1]{%
  \leavevmode\marginpar{\raggedleft\sffamily Nuevo en #1}}

\newcommand\nm[1]{\unskip\,$^{#1}$}
\newcommand\nt[1]{\quad$^{#1}$\,\ignorespaces}

\makeatletter
  \renewcommand\@biblabel{}
\makeatother

\newcommand\DOT[1]{\lsc{DOT},~#1}
\newcommand\DTL[1]{\lsc{DTL},~#1}
\newcommand\MEA[1]{\lsc{MEA},~#1}

\raggedright
\setlength{\parindent}{0em}
\setlength{\parskip}{3pt}

\addtolength{\oddsidemargin}{-4pc}
\addtolength{\textwidth}{7pc}

\OnlyDescription
\begin{document}
   \DocInput{spanish.dtx}
\end{document}
%</filedriver>
%\fi
%
% \begingroup
% \ifx\langdeffile\undefined
%
%^^A ======= Beginning of text as typeset by spanish.dtx =========
%
%
% \title{Estilo \textsf{spanish}\\
%  para el sistema \babel.\footnote{Este
%  archivo est\'a actualmente en la versi\'on
%  5.0d con fecha 25 de mayo del 2008. ^^A@#
%  Esta copia del manual se compuso el~\today.}}
%
% \author{Javier Bezos\footnote{Por favor, env\'{\i}en comentarios y
% sugerencias en http://www.texytipografia.com/contact.html.  Han
% colaborado de una u otra forma muchas personas, a las cuales
% agradezco sus comentarios y sugerencias; en particular, han sido muy
% activos Juan Luis Varona y Jos� Luis Rivera.  Para m'as informaci'on
% sobre los criterios seguidos, v'ease la referencia: Javier Bezos,
% \textit{Tipograf'ia espa'nola con \TeX.} Para informaci'on sobre
% actualizaciones: http://www.cervantex.org/}}
%
% \date{25 de mayo del 2008} ^^A@#
%
% \maketitle
% 
% {\small\tableofcontents}
%
% \section*{S�mbolos empleados}
%
% \begin{itemize}
% \item[\textcolor{blue}{$\Rightarrow$}] Macros para 
% ser usadas en el texto.
% \item[\textcolor{magenta}{$\star$}] Macros de 
% configuraci�n y preferencias.
% \item[\textcolor{green}{$\surd$}] Grupo que 
% activa la orden.
% \item[\textcolor{red}{$\times$}] Opciones de 
% paquete que anulan la orden. En redonda van las destinadas 
% espec�ficamente a anular ese punto, y en cursiva las que adem�s
% anulan otros aspectos del estilo.
% \end{itemize}
%
% \section{Uso de \textsf{spanish} para babel}
%
% El estilo de \textsf{spanish} para babel tiene con finalidad adaptar una
% serie de elementos de los documentos de \LaTeX\ a la lengua 
% espa�ola, tanto en las traducciones como en la tipograf�a. Para
% usarlo, basta con emplear la opci�n \textsf{spanish} al cargar babel:
%\begin{verbatim}
%\usepackage[spanish]{babel}
%\end{verbatim}
%
% Esto es todo lo que hace falta para conseguir que el documento tenga
% un aspecto espa�ol. En caso de estar en M�xico, v�ase, adem�s, el
% apartado \ref{paises} (<<Opciones por pa�ses>>):\footnote{En pr�ximas versiones se
% a�adir�n m�s pa�ses.}
%\begin{verbatim}
%\usepackage[spanish,mexico]{babel}
%\end{verbatim}
%
% El estilo \textsf{spanish} se puede cargar junto con otras lenguas (v�ase el 
% manual de babel). Si \textsf{spanish} es la �ltima de las lenguas cargadas,
% entonces se considera la lengua principal y se hacen una serie de
% ajustes tipogr�ficos adicionales. En particular, se modifican
% 'ordenes y entornos como:
%\begin{center}
%\begin{tabular}{lll}
%  |enumerate| &  |\roman|    &  |\section| \\
%  |itemize|   &  |\fnsymbol| &  |\subsection|\\
%  |\%|        &  |\alph|     &  |\subsubsection|\\
%                  &  |\Alph|     & \\
%\end{tabular}
%\end{center}
%
% El estilo est� pensado para que sea muy configurable.  Para ello, se
% proporcionan una serie de opciones de paquete, que en caso de
% emplearse deben ir \textit{despu�s} de \textsf{spanish}.  Por
% ejemplo:
% \begin{verbatim}
% \usepackage[french,spanish,es-noindentfirst]{babel} \end{verbatim}
% carga los estilos para el franc�s y el espa�ol, esta �ltima como
% lengua principal; adem�s, evita que \textsf{spanish} sangre el
% primer p�rrafo tras un t�tulo.  Otras opciones se pueden ajustar por
% medio de macros, en particular aquellas que se puede desear cambiar
% en medio del documento (por ejemplo, el formato de la fecha).
%
% Los cambios est�n organizados en una serie de grupos: 
% \textsf{captions, date, text, math} y \textsf{shorthands}.
% Los tres ultimos corresponden a lo que en babel ser�a normalmente
% \textsf{extras}.
%
% \section{\textsf{spanish} como lengua principal}
% 
% Si la lengua principal es \textsf{spanish}, se introducen una serie de
% cambios en el momento de cargar la lengua para adaptar varios
% elementos a los usos tipogr'aficos espa'noles.  Estos cambios
% funcionan con las clases est'andar "+--con otras tal vez alguno de
% ellos no tenga efecto--- y persisten durante todo el documento.
% Ninguno de ellos es necesario para componer el documento, aunque
% naturalmente el resultado ser'a distinto.
%
% \subsection{Listas}
%
% \begin{decl} \txt |\begin{enumerate} ... \end{enumerate}|%
% \deact{es-nolists, \textit{es-nolayout, es-minimal,
% es-sloppy}} \end{decl}
%
% Usa la siguiente secuencia:\\
% \quad 1.\\
% \qquad \emph{a})\\
% \quad\qquad 1)\\
% \qquad\qquad \emph{a$'$})
%
% \begin{decl} \txt |\begin{itemize} ... \end{itemize}|%
% \deact{es-nolists, \textit{es-nolayout, es-minimal,
% es-sloppy}} \end{decl}
%
% Usa la siguiente secuencia:\\
% \quad\leavevmode\hbox to 1.2ex
%     {\hss\vrule height .95ex width .8ex depth -.15ex\hss}\\
% \qquad\textbullet\\
% \quad\qquad $\circ$\\
% \qquad\qquad $\diamond$
%
% \begin{decl}
% \con  |\spanishdashitems    \spanishsignitems|
% \end{decl}
%
% Dos 'ordenes  para cambiar a otros estilos en
% |itemize|: rayas en todos los niveles y \textbullet{} $\circ$
% $\diamond$ $\triangleright$, respectivamente.
%
% \begin{decl}
% \con  |es-nolists|\opp
% \end{decl}
%
% Desactiva los cambios en las listas (aunque  |\es@enumerate| y 
%  |\es@itemize| siguen disponibles).
%
% \subsection{Contadores}
% 
% \begin{decl}
% \txt |\alph    \Alph|\deact{\textit{es-nolayout, es-sloppy}}
% \end{decl}
%
% Incluyen la e'ne.
%
% \begin{decl}
% \txt  |\fnsymbol|\deact{\textit{es-nolayout, es-sloppy}}
% \end{decl}
%
% Se emplean uno, dos, tres... asteriscos (*, **, ***, etc.),
% en lugar de la sucesi'on angloamericana de cruces, barras,
% etc.\footnote{\DOT{162}.}
%
% \begin{decl}
% \txt  |\roman|\deact{es-ucroman, es-lcroman, \textit{es-nolayout, es-minimal, es-sloppy}}
% \end{decl}
%
% Como en castellano no se usan n'umeros romanos en min'uscula,
% |\roman| se redefine para que los d'e en
% versalitas.\footnote{\DTL{197}.} La opci�n de paquete
%  |es-minimal| los desactiva con  |es-ucroman|, y |es-sloppy|
% con  |es-lcroman|.

% \begin{decl}
% \con  |es-ucroman|\opp
% \end{decl}
%
% Opci�n de paquete adicional, que pasa todos los romanos a versales,
% en caso de que no se quiera la versalita o por incompatibilidad con
% alg�n paquete que use de forma indebida  |\roman|.\footnote{En
% el momento de escribir esto, como m�nimo son: \textsf{dramatist,
% epiolmec, flashcards, lipsum, ntheorem, ntheorem-hyper,
% texmate.} Otros paquetes como \textsf{hyperref, easy} y \textsf{exam} 
% ya han sido corregidos.}
%
% \begin{decl}
% \con  |es-lcroman|\opp
% \end{decl}
%
% Como �ltimo recurso, de haber problemas con el valor predeterminado
% o con  |es-ucroman|, con esta opci�n de paquete puede dejarse la
% definici�n de \LaTeX, aunque en espa�ol los romanos en min�scula
% sean una falta ortogr�fica.
%
% \begin{decl}
% \con  |es-preindex|\opp
% \end{decl}
%
% \textit{MakeIndex} no puede entender la forma en que |\roman|
% escribe el n'umero de p'agina, por lo que elimina las l'ineas
% afectadas.  Por ello el archivo |.idx| ha de ser convertido antes de
% procesarlo con \textit{MakeIndex}.  Con este paquete se proporciona
% la utilidad |romanidx.sty| que se encarga de ello.  Simplemente se
% compone ese archivo con \LaTeX{} y a continuaci'on se responde a las
% preguntas que se formulan; el archivo resultante, es decir, el que
% hay que procesar con \textit{MakeIndex,} tiene la extensi�n
% \texttt{eix}.  Este proceso no es necesario si no se introdujo
% ninguna entrada de 'indice en p'aginas numeradas con |\roman| (lo
% cual ser'a lo m'as normal).  Si un s'imbolo propio de
% \emph{MakeIndex} generara problemas, debe encerrarse entre llaves:
% \verb={"|}=.
%
% Con la opci�n de paquete  |es-preindex| se llama desde el
% documento |romanidx.sty|, de forma que no es necesaria su ejecuci�n
% aparte. Tampoco pide ning�n dato, sino que ha de darse en el 
% documento principal con la siguiente orden.
%
% \begin{decl}
% \con  |\spanishindexchars|\marg{encap}\marg{open\_range}\marg{close\_range}
% \end{decl}
%
% De usarse |es-preindex| con un estilo de �ndice que no tiene los
% valores predeterminados de estos tres caracteres especiales, hay que
% darlos con esta orden (es decir, por omisi�n es
% \verb+\spanishindexchars{|}{(}{)}+).
%
% \begin{decl}
% \con  |\spanishscroman    \spanishlcroman    \spanishucroman|
% \end{decl}
%
% Finalmente, tres macros permiten cambios temporales en el
% documento de  |\roman| a versalitas, min�sculas y may�sculas,
% respectivamente.
%
% \subsection{Otros}
%
% \begin{decl}
% \txt  |\guillemotleft    \guillemotright|\deact{\textit{es-nolayout, es-sloppy}}
% \end{decl}
%
% Las comillas latinas para |OT1| son menos angulosas y se generan
% con unas puntas de flecha de |lasy|. En T1 no hay cambios.
%
% \begin{decl}
% \txt  |\section|, |\subsection|, etc., 
% |\tableofcontents|\deact{es-nosectiondot, es-noindentfirst, \textit{es-nolayout, 
% es-mininal, es-sloppy}}
% \end{decl}
%
% Los n'umeros en los t'itulos est'an seguidos de un punto 
% tanto en el texto como en el 'indice. Adem'as,
% el primer p'arrafo tras el t'itulo no elimina la sangr'ia.
% 
% \begin{decl}
% \con  |es-nolayout|\opp
% \end{decl}
% 
% Si no se desea ninguno de estos cambios, basta con usar esta opci�n 
% de paquete.
%
% \section{Traducciones}
%
% \subsection{Nombres}
% 
% \begin{decl}
% \txt  |\refname|, |\tablename|, |\contentsname|, etc.\\
% \con  |\spanishrefname|, |\spanishtablename|, |\spanishcontentsname|, etc.
% \act{captions}
% \end{decl}
%
%    Establecen las traducciones al castellano de algunos t'erminos,
%    tal y como se describe en el cuadro 1.  Para cambiar el texto
%    de ellas, conviene redefinir la forma que empieza con
%     |\spanish...|, ya que, al contrario que las �rdenes
%    |\refname|, |\abstractname|, etc., se pueden redefinir cuando
%    se desee y entran en acci�n al momento y de forma permanente, sin
%    necesidad de |\addto|.
% 
% \begin{table}
% \center\small
% \newcommand\name[2]{%
% \texttt{\textbackslash#1name}&%
% \texttt{\textbackslash spanish#1name}&}
% \caption{Traducciones}
% \vspace{1.5ex}
% \begin{tabular}{l@{\hspace{3em}}l@{\hspace{3em}}l}
% \toprule3
% \name{ref}        & Referencias\\
% \name{abstract}   & Resumen\\
% \name{bib}        & Bibliograf'ia\\
% \name{chapter}    & Cap'itulo\\
% \name{appendix}   & Ap'endice\\
% \name{contents}   & 'Indice general\nm{a}\\
% \name{listfigure} & 'Indice de figuras\\
% \name{listtable}  & 'Indice de cuadros\\
% \name{index}      & 'Indice alfab'etico\\
% \name{figure}     & Figura\\
% \name{table}      & Cuadro\\
% \name{part}       & Parte\\
% \name{encl}       & Adjunto\\
% \name{cc}         & Copia a\\
% \name{headto}     & A\\
% \name{page}       & p'agina\\
% \name{see}        & v'ease\\
% \name{also}       & v'ease tambi'en\\
% \name{proof}     & Demostraci'on
% \botrule3
% \end{tabular}
%
% \vspace{1.5ex}
%
% \begin{minipage}{10cm}\footnotesize
% \nt{a} Pero s'olo <<'Indice>> en \textsf{article}.
% \end{minipage}
% \end{table}
% 
% \begin{decl}
% \con  |es-uppernames|\opp
% \end{decl}
%
% Aunque sea un anglicismo,\footnote{\DOT{197}.} con esta opci�n de
% paquete los sustantivos tienen may�scula inicial.
%
% \begin{decl}
% \con  |es-tabla|\opp
% \end{decl}
%
% En caso de que todos los cuadros sean tablas, esta opci�n permite
% cambiar \textit{cuadro} por \textit{tabla} (en cierto modo,
% \textit{cuadro} es a \textit{tabla} lo que \texttt{table} es a
% \texttt{tabular}).
%
% \subsection{Fechas}
%
% \begin{decl}
% \txt  |\today    \Today|
% \act{date}
% \end{decl}
%
%    Fecha  actual, en la forma \textit{1 de enero de 
%    2004.} Con  |\Today| el mes va en may�scula.
%
% \begin{decl}
% \con  |\spanishdatedel    \spanishdatede|
% \end{decl}
%
%    Con la primera se cambia el formato para que a partir del 2000 se
%    emplee \textit{del} y no \textit{de} (recomendado).  La segunda
%    hace justo lo contrario (predeterminado).
%
% \begin{decl}
% \con  |\spanishreverseddate|
% \end{decl}
%
%   Cambia el formato de  |\today| a la forma
%   \textit{enero 1 del 2004.} Con  |\Today| el mes va en 
%   may�scula.
%
% \section{Abreviaciones (\textit{shorthands})}
% 
%    La lista completa se puede encontrar en el cuadro 2.  En los
%    siguientes apartados se dar'an m'as detalles sobre algunas de
%    ellas.
%
% \begin{table}[!t]
% \center\small
% \caption{Abreviaciones}
% \vspace{1.5ex}
% \begin{tabular}{l@{\hspace{3em}}l@{\hspace{3em}}l}
% \toprule2
% |'a 'e 'i 'o 'u| & 'a 'e 'i 'o 'u\\
% |'A 'E 'I 'O 'U| & 'A 'E 'I 'O 'U\\
% |'n 'N|          & 'n 'N\nm{a}\\
% |"u "U|          & "u "U\\
% |"i "I|          & "i "I\\
% |"a "A "o "O|    & Ordinales: 1"a, 1"A, 1"o, 1"O\\
% |"er "ER|        & Ordinales: 1"er, 1"ER\\
% |"c "C|          & "c "C\\
% |"rr "RR|        & rr, pero -r cuando se divide\\
% |"y|             & El antiguo signo para <<y>>\\
% |"-|             & Como |\-|, pero permite m'as divisiones\\
% |"=|             & Como |-|, pero permite mas divisiones\nm{b}\\
% |"~|             & Gui'on estil'istico\nm{c}\\
% |"+ "+- "+--|    & Como |-|, |--| y |---|, pero sin divisi'on\\
% |~- ~-- ~---|    & Lo mismo que el anterior.\\
% |""|             & Permite mas divisiones antes y despu'es\nm{d}\\
% |"/|             & Una barra algo m'as baja\\
% \verb+"|+        & Divide un logotipo\nm{e}\\
% |"< ">|          & "< ">\\
% |"` "'|          & |\begin{quoting}| |\end{quoting}|\nm{f}\\
% |<< >>|          & Lo mismo que el anterior.\\
% |?` !`|          & ?` !`\nm{g}\\
%|"? "!|           & "? "! alineados con la linea base\nm{h}
% \botrule2
% \end{tabular}
%
% \vspace{1.5ex}
%
% \begin{minipage}{11cm}
% \footnotesize
% \nt{a} La forma |~n| no est� activada por omisi�n a partir de 
% la versi�n 5.
% \nt{b} |"=| viene a ser lo mismo que |""-""|.
% \nt{c} Esta abreviaci'on tiene un uso distinto
% en otras lenguas de babel.
% \nt{d} Como en <<entrada/salida>>.
% \nt{e} Carece de uso en castellano.
% \nt{f} V'ease sec.~2.7. 
% \nt{g} No proporcionadas por este paquete, sino por cada tipo;
% figuran aqu'i como simple recordatorio.
% \nt{h} 'Utiles en r'otulos en may'usculas.
% \end{minipage}
% \end{table}
%
% Los caracteres usados como abreviaciones se comportan
% como otras 'ordenes de \TeX{} y por tanto se hace caso
% omiso de los espacios que le puedan seguir: \verb*|' a| es lo mismo
% que |'a|. Eso tambi'en implica que tras esos caracteres no
% puede ir una llave de cierre y que deber'a escribirse
% |{... '{}}| en lugar de |{... '}|; en modo matem'atico no hay
% ning�n problema y |$x^{a'}$| ($x^{a'}$) es v'alido.
%
% \begin{decl}
% \con  |activeacute|\opp
% \end{decl}
%
% Para poder usar ap'ostrofos como abreviaciones de acentos es
% necesaria esta opci'on en |\usepackage|.  Puede cambiarse este
% comportamiento con la orden |\es@acuteactive| en el archivo de
% configuraci'on |spanish.cfg|; en ese caso los ap'ostrofos se activan
% siempre.
%
% \begin{decl}
% \con  |es-tilden|\opp
% \end{decl}
% 
% Esta orden activa las abreviaciones |~n| y |~N| por compatibilidad
% con versiones anteriores de \textsf{spanish} (y siempre que no se
% empleado tambi�n |es-notilde|). En la versi�n 5 no est�n
% activadas de forma predeterminada.
% 
% \begin{decl}
% \con  |\spanishdeactivate|\marg{caracteres}
% \end{decl}
% 
% Permite desactivar las abreviaciones correspondientes a los
% caracteres dados. Para evitar entrar en conflicto con otras lenguas,
% al salir de \textsf{spanish} se reactivan,\footnote{El punto para
% los decimales no es estrictamente una abreviaci'on y no se
% reactiva.} por lo que si se desea que 
% persistan hay que a'nadir la orden a |\shorthandsspanish| con 
%|\addto|. La orden |\renewcommand\shorthandsspanish{}| es una 
% variante optimizada de
%\begin{verbatim}
% \addto\shorthandsspanish{\spanishdeactivate{.'"~<>}}
%\end{verbatim}
%
% \begin{decl}
% \con  |es-noshorthands|\opp
% \end{decl}
%
% No activa ninguna  abreviaci�n.
% 
% \subsection{Coma decimal}
%
% \begin{decl}
% \txt  |.|\textit{n�mero}\act{shorthands}\deact{es-nodecimaldot, 
% \textit{es-noshorthands, es-minimal, es-sloppy}}
% \end{decl}
%
% En \textsf{spanish}, el punto en matem�ticas sirve como marca decimal
% gen�rica que puede representarse como coma o punto; funciona
% por tanto como marcado l�gico del signo para decimales. Por
% omisi�n, se siguen las normas internacionales ISO y la legislaci�n
% de diversos pa�ses (como de Espa�a y M�xico) de emplear la coma.
% Ya que \TeX\ usa la coma como separador en intervalos o expresiones 
% similares, lo que a'nade un espacio fino, \textsf{spanish}
% interpreta todo punto en modo matem'atico de esta forma siempre
% que est'e seguido de una cifra, pero no en otras circunstancias:
% \begin{quote}\small\begin{tabbing}
% |$1\,234.567\,890$|     \quad \=  $1\,234.567\,890$\\
% |$f(1,2)=12.34.$|        \> $f(1,2)=12.34.$\\
% |$1{.}000$|              \> $1{.}000$, pero\\
% |1.000|                  \> 1.000, pues no es modo matem'atico.
% \end{tabbing}\end{quote}
%
%
% \begin{decl}
% \con  |\decimalcomma    \decimalpoint    \spanishdecimal|\marg{math}
% \end{decl}
%
% Las dos primeras establecen si se usa una coma (predeterminado)
% o un punto, mientras que |\spanishdecimal|\marg{math}
% permite darle una definici'on arbitraria.
%
% \begin{decl}
% \con  |es-nodecimaldot|\opp
% \end{decl}
%
% Cancela por completo el mecanismo del punto decimal.
%
% \subsection{Divisi'on de palabras}
%
% \textsf{Spanish} comprueba la 
% codificaci'on en el momento en que se usa un acento: si la 
% codificaci'on es |OT1| se toman medidas para facilitar
% la divisi'on, que pese a todo nunca ser'a perfecta, mientras que con 
% |T1| se accede directamente al car'acter correspondiente.
%
% \begin{decl}
% \txt  |"-    "=    "~|\act{shorthands}
% \deact{\textit{es-noshorthands, es-sloppy}}
% \end{decl}
%
% Para matizar la divisi'on de palabras hay cuatro posibilidades, dos 
% de ellas con el m'etodo de abreviaciones:
% \begin{itemize}
% \item la orden |\-| es un gui'on opcional que no permite
% m'as divisiones, 
%
% \item |"-| es similar pero permite m'as divisiones,
%
% \item un |-| es un gui'on que no permite m'as divisiones ni
% antes ni despu'es, y
% 
% \item |"=| es el equivalente que s'i las permite,\footnote{No
% es una buena idea usar esta orden, pero en 
% medidas muy cortas puede resultar necesario.}
%
% \end{itemize}
% Por ejemplo (con las posibles divisiones marcadas con \hmk):
% \begin{quote}\small\begin{tabbing}
% |Zaragoza-Barcelona|\qquad \= Zaragoza-\hmk Barcelona\\
% |Zaragoza"=Barcelona| \>
%    Za\hmk ra\hmk go\hmk za-\hmk Bar\hmk ce\hmk lo\hmk na\\
% |semi\-abierto| \> semi\hmk abierto\\
% |semi"-abierto| \> se\hmk mi\hmk abier\hmk to.\footnotemark
% \end{tabbing}\footnotetext{Justo antes y despu'es de
% {\ttfamily\string"\string-} y {\ttfamily\string"\string=} se
% aplican los correspondientes
% valores de {\ttfamily\string\...hyphenmin} lo que implica que la
% divis'on semia\hmk bierto no es posible.
% 'Este es un comportamiento correcto.}
% \end{quote}
%
% La abreviaci'on |"~| se usa cuando se quiere que el gui'on
% tambi'en aparezca al comienzo de la siguiente l'inea. Por ejemplo:
% \begin{quote}\small\begin{tabbing}
% |infra"~rojo|  \quad \= in\hmk fra-ro\hmk jo, pero infra-\hmk-rojo.
% \end{tabbing}\end{quote}
%
% \begin{decl}
%\txt  |"+  "+-  "+--|\act{shorthands}\deact{\emph{no-shorthands, 
% es-sloppy}}
% \end{decl}
%\vskip-1.5pc\vskip0pt
% \begin{decl}
% \txt  |~-  ~--  ~---|\act{shorthands}\deact{es-notilde, \emph{no-shorthands, 
% es-minimal, es-sloppy}}
% \end{decl}

% Evitan divisiones: |~-|, que resulta 'util para expresar una serie
% de n'umeros sin que el gui'on los divida (12~-14, |12~-14|), y
% |~---|, que es la forma que debe usarse para abrir incisos con
% rayas, ya que de lo contrario puede haber una divisi'on entre la
% raya de abrir y la palabra que le sigue:
% \begin{quote}\small\begin{tabbing}
%|Los conciertos ~---o % academias--- que organiz'o...|
% \end{tabbing}\end{quote}
%
% Tambi�n pueden emplearse para esta misma funci�n las abreviaciones
% |"+|, |"+-| y |"+---|.  Mientras que este gui'on evita toda posible
% divisi'on en los elementos que une, la raya (---) y la semirraya
% (--) las permiten en las palabras que le precedan o le sigan.
% 
% Otra abreviaci'on es |"rr| que sirve para el 
% 'unico cambio de escritura del castellano en caso de haber divisi'on.  
% La \lsc{RAE} indica que al a'nadir un prefijo que termina en vocal a 
% una palabra que comienza con \emph{r}, 'esta 'ultima debe 
% doblarse a menos que se unan por un gui'on. Por ejemplo:
% \begin{quote}\small\begin{tabbing}
% |extra"rradio|  \quad \= ex\hmk trarra\hmk dio, pero extra-\hmk 
%    radio.
% \end{tabbing}\end{quote}
% No hay acuerdo sobre si esta regla y otras similares han de 
% aplicarse o no, aunque la opini'on mayoritaria actual est'a en
% contra.
%
% \subsection{Otros}
%
% \begin{decl}
% \txt  |"/|\act{shorthands}\deact{\textit{es-noshorthands, es-sloppy}}
% \end{decl}
%
% Es una utilidad tipogr'afica m'as que espec'ificamente espa'nola.
% En ciertos tipos, como Times, el extremo inferior de la barra est'a
% en la l'inea de base y expresiones como <<am/pm>> resultan poco
% est'eticas.  |"/| produce una barra que, de ser necesario, se baja
% ligeramente.  Computer Modern tiene una barra bien dise'nada y no es
% posible ilustrar aqu'i este punto pero en todo caso se escribir'ia
% |am"/pm|.\footnote{En \MEA{141} se recurre a una soluci'on que es la
% 'unica sencilla en programas de maquetaci'on: usar un cuerpo menor.
% Pero con \TeX{} es mucho m'as f'acil automatizar las tareas.}
%
% \begin{decl}
% \txt  |"y|\act{shorthand}\deact{\textit{es-noshorthands, es-sloppy}}
% \end{decl}
%
% El signo \textit{et tironiano}, que en espa�ol se emple� muy a
% menudo, se puede <<imitar>> con |"y|, siempre que se haya cargado el
% paquete |graphics|; de no ser as'i, se usa la letra $\tau$, aunque
% la variante normal de \TeX{} no es demasiado apropiada.
%
% \section{Funciones de texto y matem�ticas}
%
% \subsection{Abreviaturas}
%
% \begin{decl}
% \txt  |\sptext|\marg{texto}\act{text}\deact{\textit{es-sloppy}}
% \end{decl}
%
% Pone un punto y le sigue el argumento en voladitas.  Para
% abreviaturas como |adm\sptext{'on}| que da adm\sptext{'on}.  Hay seis
% abreviaciones asociadas a ordinales: |"a|, |"A|, |"o|, |"O|, |"er| y
% |"ER| que equivalen a |\sptext{a}|, etc. Muchos tipos
% a'naden un peque'no subrayado que debe evitarse, y por tanto no se
% deben escribir los ordinales con \textsf{inputenc}.
%
% Para ajustar el tama'no lo mejor posible, se usa el de
% 'indices en curso. Esto funciona bien salvo para tama'nos muy 
% grandes o muy peque'nos, donde los resultados son meramente
% aceptables. 
%
% En Plain \TeX{} se ejecuta la orden |\sptextfont| para la
% letra voladita, de forma que |{\bf\let\sptextfont\bf 1"o}| da el
% resultado correcto (|\mit| si es para cursiva). Para usar un tipo
% nuevo con |\sptext| hay que definir tambi'en las variantes 
% matem'aticas con |\newfam|.
%
% \subsection{Espaciado}
% 
% Salvo excepciones, el espaciado espa'nol difiere relativamente poco
% del ingl'es; una de ellas es que en \textsf{spanish} |\frenchspacing|
% est'a activo.
% 
% \begin{decl}
% \txt  |\...|\act{text}\deact{\textit{es-sloppy}}
% \end{decl}
%
% Puntos suspensivos menos espaciados que  |\dots|. El espacio 
% que sigue se conserva:
% \begin{quote}\small\begin{tabbing}
% |\... y solo estaba\... ella.|\quad\=\... y solo estaba\... ella.
% \end{tabbing}\end{quote}
% Tambi'en podr'ian escribirse los tres puntos sin m'as |...|, y en
% la pr'actica no hay diferencia, a menos que se cambie el
% valor del espacio tras punto; en ese caso, la forma con barra
% da los valores apropiados \emph{dentro} de una sentencia, y
% los tres puntos \emph{al final} de ella. Esta orden no 
% interfiere con el valor original de |\.| (un punto suprascrito).
%
% \begin{decl}
% \txt  |\%|\act{text}\deact{\textit{es-minimal, es-sloppy}}
% \end{decl}
%
% Se a�ade un espacio fino antes del signo (que
% m'as exactamente es |\,|, con lo cual se puede "<recuperar"> con
% su opuesto |\!| si |\%| no sigue a una cifra; tambi�n se puede 
% emplear |\percentsign|).
%
% \begin{decl}
% \con  |\spanishplainpercent|
% \end{decl}
%
% Orden para que  |\%| no a�ada el espacio fino. Puede ser �til 
% en cuadros, si  |\%| aparece siempre entre par�ntesis.
%
% \subsection{Fuentes}
%
% \begin{decl}
% \txt  |\lsc|\marg{texto}\act{text}\deact{\textit{es-sloppy}}
% \end{decl}
%
% Pasa \textit{texto} a versalitas:
% \begin{quote}\small\begin{tabbing}
% |\lsc{RAE}| \quad \= \lsc{RAE}\\
% |\lsc{ReNFe}| \quad \= \lsc{ReNFe}.\\
% |siglo \lsc{XVII}| \quad \= siglo \lsc{XVII}\\
% |cap'itulo \lsc{II}| \quad \= cap'itulo \lsc{II}.
% \end{tabbing}\end{quote}
% 
% Para evitar que con un tipo que carece de versalitas acabe
% apareciendo (por substituci'on) un texto de min'usculas se intenta
% usar en estos casos las versales \emph{reales} de un tama'no menor
% (\LaTeX\ tiende a sustituir versalitas por versalitas, pero hay
% excepciones, como con las negritas).
% 
% \begin{decl}
% \txt  |\'i|\alw
% \end{decl}
%
% Lo mismo que  |\'{\i}|.
%
% \subsection{Entrecomillados}
% 
% \begin{decl}
% \txt  |\begin{quoting} ... \end{quoting}|\alw
% \end{decl}
%
% El entorno |quoting| entrecomilla un  
% texto, a'nadiendo comillas de seguir al comienzo de 
% cada p'arrafo en su interior.\footnote{Se puede encontrar
% una detallada exposici'on de las comillas en \DTL{44 ss.} De ah'i
% se ha tomado alg'un ejemplo.}
%
% \begin{decl}
% \txt  |<<    >>|\act{shorthands}\deact{es-noquoting, \textit{es-noshorthands, es-minimal, 
% es-sloppy}}
% \end{decl}
%\vskip-1.5pc\vskip0pt
% \begin{decl}
% \txt  |"`    "'|\act{shorthands}\deact{\textit{es-noshorthands, es-sloppy}}
% \end{decl}
%
% Tambi'en se pueden emplear las abreviaciones |<<| y |>>| (o
% alternativamente |"`| y |"'|) que se limitan a llamar a |quoting|,
% que por ser entorno considera sus cambios internos como locales.
% (Es decir, |<< ...  >>| implica |{<< ...  >>}|.)  Las abreviaciones
% |"<| y |">| contin'uan dando sin m'as los caracteres de comillas de
% abrir y cerrar, respectivamente.
% 
% Por ejemplo:
%\begin{verbatim}
% <<Se llaman <<comillas de seguir>> a las que son de cierre,
% pero se colocan al comienzo de cada p'arrafo cuando se transcribe
% un texto entrecomillado con m'as de un p'arrafo.
% 
% En su interior, como de costumbre, se usan inglesas.>>
%\end{verbatim}
% cuyo resultado es:
% \begin{quotation}\small
% <<Se llaman <<comillas de seguir>> a las que son de cierre,
% pero se colocan al comienzo de cada p'arrafo cuando se transcribe
% un texto entrecomillado con m'as de un p'arrafo.
% 
% En su interior, como de costumbre, se usan inglesas.>>
% \end{quotation}
%
% Tambi�n se a�aden comillas de seguir en listas, excepto con la
% opci�n \texttt{es-nolists} o cualquier otra que las desactive.
% 
% Este entorno se puede redefinir, como por ejemplo:
%\begin{verbatim}
% \renewenvironment{quoting}{\itshape}{}
%\end{verbatim}
% pero en principio no implica un nuevo p'arrafo, ya que 
% est'a pensado para ser usado tambi'en en el texto.
%
% \begin{decl}
% \con  |\lquoti| |\rquoti| |\lquotii| |\rquotii| |\lquotiii| 
% |\rquotiii|
% \end{decl}
% 
% Controlan las comillas en |quoting|, seg'un el
% nivel en que nos encontremos. |\lquoti| son las comillas de abrir
% m'as exteriores, |\lquotii| las de segundo nivel, etc., y lo mismo
% para las de cerrar con |\rquoti|... Para las de seguir siempre se
% usan las de cerrar. Los valores predefinidos est'an en el cuadro 3.
% \begin{table}
% \center\small
% \caption{Entrecomillados}
% \vspace{1.5ex}
% \begin{tabular}{l@{\hspace{5em}}l}
% \toprule2
% |\lquoti|   &|"<|\\
% |\rquoti|   &|">|\\
% |\lquotii|  &|``|\\
% |\rquotii|  &|''|\\
% |\lquotiii| &|`|\\
% |\rquotiii| &|'|
% \botrule2
% \end{tabular}
% \end{table}
% 
% \begin{decl}
% \con  |\activatequoting    \deactivatequoting|
% \end{decl}
% 
% Las incompatibilidades potenciales de estas abreviaciones son
% enormes. Por ejemplo, en \textsf{ifthen} se cancelan las
% comparaciones entre n'umeros;\,\footnote{Y en \texttt{\textbackslash 
% ifnum},
% \texttt{\textbackslash ifdim}, etc., usado por los desarrolladores en
% los paquetes.} tambi'en
% resultan inoperantes |@>>>| y |@<<<| de
% \textsf{amstex}.\footnote{Aunque en 
% este caso cabe usar los sin'onimos |@)))| y |@(((|.}
% Por ello, se da la posibilidad de cancelarlas y reactivarlas con
% estas 'ordenes, aunque si se est'a usando 
% \textsf{xmltex} ya se
% desactivan por completo de forma autom'atica. El entorno
% |quoting| siempre permanece disponible.\footnote{Algunos tipos
% disponen de esta ligadura de forma interna para
% generar los caracteres de comillas, por lo que en ellos tambi'en
% podemos usarlos siempre, aunque los ajustes proporcionados por
% \textsf{spanish} se pueden perder; por otra parte, tampoco se
% usan demasiado a menudo.}
% 
% \subsection{Funciones matem'aticas}
%
% \begin{decl}
% \txt |\lim \limsup \liminf \bmod \pmod \sen \tg| 
% etc.\act{math}\deact{\textit{es-minimal, es-sloppy}}
% \end{decl}
%
% Tradicionalmente, las abreviaciones de lo que en \TeX\ se conocen
% como operadores se han formado a partir del nombre castellano, lo
% que implica la presencia del acento en l'im (en sus tres formas
% |\lim|, |\limsup| y |\liminf|), m'ax, m'in, 'inf y m'od (en sus dos
% formas |\bmod| y |\pmod|).
%
% Con \textsf{spanish} pueden seguirse varias convenciones con ayuda
% de las siguientes 'ordenes:
% \begin{decl} \con  |\accentedoperators| |\unaccentedoperators|
% \end{decl}
% Activa o desactiva los acentos.
% Por omisi'on se acent'uan, como por ejemplo: $\lim_{x\to 0}(1/x)$
% (|$\lim_{x\to 0}(1/x)$|).
% 
% \begin{decl}
% \con  |\spacedoperators| |\unspacedoperators|
% \end{decl}
% Activa o desactiva el espacio entre "<arc"> y la funci'on.
% Lo habitual ha sido con espacio; as'i pues, por omisi'on
% se espacia.
%
% Tambi'en se a'naden |\sen|, |\arcsen|, |\tg| y |\arctg|,
% que dan las funciones respectivas.
% \begin{decl}
% \con  |\spanishoperators|
% \end{decl}
%
% Otras funciones trigonom'etricas se encuentran almacenadas en el
% par'ametro |\spanishoperators|, que inicialmente incluye cotg,
% cosec, senh y tgh.  La raz'on por la que estas funciones se han
% separado es porque su forma no est� normalizada en el 'ambito
% hispanohablante.  De esta forma se puede cambiar por otras con, por
% ejemplo:
%\begin{verbatim}
% \renewcommand{\spanishoperators}{ctg arc\,ctg sh ch th}
%\end{verbatim}
% (separadas con espacio). Cuando se selecciona \textsf{spanish} se crean
% 'ordenes con esos nombres
% y que dan esas funciones (siempre con |\nolimits|). Adem'as de
% las letras sin acentuar se aceptan las 'ordenes |\,| y |\acute|, que
% se pasan por alto para formar el nombre. Por ejemplo, |arc\,ctg|
% se escribe en el documento con
% |\arcctg|, |M\acute{a}x| como |\Max| y |cr\acute{i}t| como |\crit|
% (hay que usar |i| y no |\dotlessi|).
% La orden |\,| responde a |\|(|un|)|spacedoperators|, y |\acute|
% a |\|(|un|)|accentedoperators|.
% 
% Conviene que |\spanishoperators| est'e en el pre'ambulo del
% documento en s'i, antes de |\selectspanish| o de 
% |\begin{document}|.
%
% \begin{decl}
% \txt  |\dotlessi|\act{math}\deact{\textit{es-sloppy}}
% \end{decl}
% 
% La \textit{i} sin punto tambi'en es accesible directamente en modo
% matem'atico con la orden |\dotlessi|, de forma que se puede escribir
% |\acute{\dotlessi}|.  Por ejemplo,
% |$V_{\mathbf{cr\acute{\dotlessi}t}}$| da
% $V_{\mathbf{cr\acute{\dotlessi}t}}$.
%
%
% \section{Opciones generales}
%
% Est�n pensadas principalmente para documentos basados en una clase
% o un estilo editorial muy preciso que no debe tocarse. Para conocer 
% los cambios exactos, v�anse las diferentes entradas que describen 
% las funciones de \textsf{spanish}.
%
% \begin{decl}
% \con  |es-minimal|\opp
% \end{decl}
%
% Anula la mayor�a de los cambios pero deja unas cuantas utilidades
% que pueden resultar utiles en el momento de escribir el texto. 
% 
% \begin{decl}
% \con  |es-sloppy|\opp
% \end{decl}
% 
% Anula, adem�s, todas las ligaduras sin excepci�n, la e�e en listas y los 
% grupos \textsf{text} y \textsf{math}.
%
% \section{Selecci'on}
%
% \begin{decl}
% |\selectspanish|
% \end{decl}
% 
% Por omisi'on, \babel{} deja <<dormidas>> las lenguas hasta que se
% llega a |\begin{document}| con el fin de evitar conflictos por
% las abreviaciones; a cambio,
% se priva de la posibilidad de usar las lenguas en el pre'ambulo 
% en 'ordenes como |\savebox|, |\title|, |\newtheorem|, etc.
% 
% La orden |\selectspanish| permite activar \textsf{spanish} con sus
% extensiones y abreviaciones  antes de
% |\begin{document}|.\footnote{Algunos detalles, que
% apenas afectan a \textsf{spanish}, siguen sin activarse hasta el
% comienzo del documento.}
% De esta forma, podr'iamos decir
%\begin{verbatim}
% \documentclass{book}
% \usepackage[T1]{fontenc}
% \usepackage[cp1252]{inputenc}
% \usepackage[spanish,activeacute,es-notilde]{babel}
% ... % Mas paquetes
% 
% \selectspanish
% 
% \title{T'itulo}
% \author{Autor}
% \newcommand{\pste}{para"-psicol'ogicamente}
% ...   % Mas definiciones
%
% \begin{document}
%\end{verbatim}
%
% \section{Adaptaci'on}
%
% \subsection{Opciones por pa�ses}
% \label{paises}
%
% % \begin{decl} \con |mexico| \quad |mexico-com|
% \end{decl}
%
% La primera cambia \textit{cuadro} a \textit{tabla} y desactiva tanto
%  |<||<>||>| como el punto decimal.  Tambi�n cambia
%  |"`| y  |"'| a ``\,`\,"<\,">\,'\,''. Es decir, aparte de
% redefinir las comillas, equivale a:
% a:
%\begin{verbatim}
%\usepackage[spanish,es-nodecimaldot,es-tabla,es-noquoting]{spanish}
%\end{verbatim}
% La segunda es similar
% pero s� activa el punto decimal.  (Obs�rvese que no van precedidas
% de  |es-|.)
%
% Probablemente, esta opci�n tambi�n sea apropiada en algunos
% pa�ses de Am�rica Central y del Sur.
%
% \subsection{Configuraci'on}
%
% En sus 'ultimas versiones, \babel{} ofrece la posibilidad
% de cargar autom'aticamente un archivo con el mismo nombre que
% el principal, pero con extensi'on |.cfg|. \textsf{Spanish}
% proporciona unas pocas 'ordenes para ser usadas en este archivo:
%
% \begin{decl}
% \con |\es@activeacute|
% \end{decl}
% Activa las abreviaciones con ap'ostrofos, sin que sea
% necesario incluir |activeacute| como opci'on en |\usepackage|.
%
% \begin{decl} \con  |\es@enumerate{<leveli>}|%
%      |{<levelii>}{<leveliii>}{<leveliv>}|\alw
% \end{decl} 
% Cambia los valores preestablecidos por \textsf{spanish} para
% |enumerate|. \textit{leveln} consiste en una letra, que
% indica qu'e formato tendr'a el n'umero, seguida
% de cualquier texto. La letra tiene que ser: |1| (ar'abigo),
% |a| (min'uscula \emph{cursiva}\,\footnote{La letra es cursiva
% pero no los signos que le puedan seguir. M'as bien deber'ia
% decirse destacada, ya que se usa |\string\emph|.
%  V'ease \DTL{11}.}), |A| (versal),
% |i| (romano \emph{versalita}), |I| (romano versal) o
% finalmente |o| (ordinal\,\footnote{Lo normal es no a'nadir ning'un 
% signo tras ordinal.}).
%
% Esta orden no est'a pensada para hacer cambios elaborados, sino
% s�lo meros reajustes. Los valores preestablecidos 
% equivalen a
%\begin{verbatim}
% \es@enumerate{1.}{a)}{1)}{a$'$)}
%\end{verbatim}
%
% \begin{decl} \con  |\es@itemize{<leveli>}|%
%      |{<levelii>}{<leveliii>}{<leveliv>}|\alw
% \end{decl} 
% Lo mismo para |itemize|, s'olo que los argumentos se
% usan de forma literal. Los valores originales de \LaTeX{} son
% similares a
%\begin{verbatim}
% \es@itemize{\textbullet}{\normalfont\bfseries\textendash}
%    {\textasteriskcentered}{\textperiodcentered}
%\end{verbatim}
%
% \begin{decl}\con |\es@operators|\act{math}
% \end{decl}
% Todo lo relativo a operadores se cancela con
%\begin{verbatim}
% \let\es@operators\relax
%\end{verbatim}
% Es buena idea incluirlo si no se van a usar, ya que ahorra memoria.
%
% Otros ajustes 'utiles en este contexto son |\spanishoperators|,
% |\selectspanish| y |\deactivatequoting|.
%
%
% Recordemos que todos los cambios
% operados desde este archivo restan compatibilidad al
% documento, por lo que si se distribuye conviene adjuntarlo
% con el entorno |filecontents|.
%
% \subsection{Pasar opciones desde un paquete o clase}
%
% \begin{decl} \con  |\spanishoptions|
% \end{decl}
%
% Como  |\PassOptionsToPackage| a�ade opciones al comienzo y
% las opciones espec�ficas de \textsf{spanish} han de ir al final, definiendo
% esta macro se puede controlar el comportamiento de \textsf{spanish} antes
% de su carga.
%
% \subsection{Otros cambios}
%
% Las adaptaciones se encuentran organizadas en varios grupos, a los 
% que corresponden sendas macros: 
% |\textspanish|, |\mathspanish|,
% |\shorthandsspanish|, |\datespanish| y |\captionsspanish|.  Pueden
% cancelarse con:
%\begin{verbatim}
%  \renewcommand\textspanish{}
%\end{verbatim}
%
% \section{Plain \TeX}
% 
% Con Plain hay que hacer:
%\begin{verbatim}
% \input spanish.sty
%\end{verbatim}
% 
% Se incluyen: traducciones, casi todas las abreviaciones, coma
% decimal, utilidades para divisi'on de palabras, ordinales en una
% versi'on simplificada (y no muy elegante), funciones matem'aticas,
% |\'i| y espaciado.  La selecci'on de la lengua es inmediata al
% cargar el archivo.
% 
% En cambio no est'an disponibles: entrecomillados, 
% |\lsc| ni las adaptaciones de lengua principal.
%
% \section{Compatibilidad con versiones anteriores}
%
% En versiones de \textsf{babel} bastante antiguas, las abreviaciones 
% con |'| se activaban por omisi�n, mientras que ahora es necesario
% |activeacute|.
%
% En la versi�n 4, la abreviaci�n |~n| se consider� para extinguir.
% En la versi�n 5 sigue estando, pero \textit{no} se activa por
% omisi�n, sino que hay que emplear |es-tilden|.
%
% En la versi�n 5 el grupo \textsf{layout} no se retrasa a
% |\begin{document}|, como en la 4, sino que se ejecuta
% inmediatamente.  Esto permite cambios en el pre�mbulo con otros
% paquetes.  Con ello, adem�s, |\selectspanish*| carece de utilidad.
% La opci�n de paquete |es-delayed| restaura el comportamiento
% anterior, por si hubiera alguna incompatibilidad.
%
% La compatibilidad con la versi�n 2.09 de \LaTeX{} se ha suprimido.
%
% \section*{Referencias}
% \addcontentsline{toc}{section}{Referencias}
%
% \begingroup
% \small
% \leftskip1.5cm \parindent-1.5cm
%
% \makebox[1.5cm][l]{\lsc{DRAE}}\textit{Diccionario de la Academia
%    Espa'nola}, Madrid, Espasa-Calpe, 21"a ed., 1992.
%
% \makebox[1.5cm][l]{\lsc{DOT}}Jos'e Mart'inez de Sousa,
%   \textit{Diccionario de ortograf'ia t'ecnica}, 
%   Madrid, Germ'an S'anchez Ruip'erez/Pir'amide, 1987.
%   (Biblioteca del libro.)
%
% \makebox[1.5cm][l]{\lsc{DTL}}Jos'e Mart'inez de Sousa,
%   \textit{Diccionario de tipograf'ia y del libro}, 
%   Madrid, Paraninfo, 3"a ed., 1992.
%
% \makebox[1.5cm][l]{\lsc{MEA}}Jos'e Mart'inez de Sousa,
%   \textit{Manual de edici'on y autoedici'on}, 
%   Madrid, Pir'amide, 1994.
%
% \leftskip0pt \parindent0pt \vspace{6pt}
%
% {\itshape
% Para otras cuestiones tipogr'aficas, las referencias
% usadas son, entre otras:}
%
% \parindent-1.5pc \leftskip1.5pc \vspace{3pt}
%
% Asociaci�n de Academias de la Lengua Espa�ola,
% \textit{Diccionario panhisp�nico de dudas}, Madrid, Santillana, 2005.
%
% Javier Bezos,
% \textit{Tipograf'ia espa'nola con \TeX}, documento electr'onico
% disponible en 
% \textsf{http://perso.wanadoo.es/jbezos/tipografia.html}. 
%
% Bureau International des Poids et mesures,
% \textit{Le Sist\`{e}me international d'unit�s},
% 8"a ed., Par�s, {\footnotesize BIPM}, 2006, 
% \textsf{http://www.bipm.org/""fr/""si/""si\_brochure/}, 2006-11-10.
%
% Jorge de Buen,
% \textit{Manual de dise�o editorial,} M�xico, Santillana, 2000.
%
%
% \textit{The Chicago Manual of Style}, Chicago, University of
% Chicago Press, 14"a~ed., 1993, esp.~p'ags.~333~-335.
%
% Jos'e Fern'andez Castillo,
% \textit{Normas para correctores y compositores tip'ografos},
% Madrid, Espasa-Calpe, 1959.
%
% IRANOR [AENOR], Normas \lsc{UNE} n'umeros 5010 (<<Signos 
% matem'aticos>>), 5028 (<<S'imbolos 
% geom'etricos>>) y 5029 (<<Impresi'on de los
% s'imbolos de magnitudes y unidades y de los n'umeros>>). 
% [Obsoletas.]
%
% Llerena, Mario,
% \textit{Un manual de estilo,} Miami, Unilit, 1999.
%
% Real Academia Espa'nola,
% \textit{Esbozo de una nueva gram'atica de la
% lengua espa'nola}, Madrid, Espasa-Calpe, 1973.
% 
% V.\ Mart'inez Sicluna,
% \textit{Teor'ia y pr'actica de la tipograf'ia},
% Barcelona, Gustavo Gili, 1945.
%
% Jos'e Mart'inez de Sousa,
% \textit{Diccionario de ortograf'ia de la lengua espa'nola},
% Madrid, Paraninfo, 1996.
%
% Juan Mart'inez Val, \textit{Tipograf'ia pr'actica}, Madrid,
% Laberinto, 2002.
%
% Juan Jos'e Morato, \textit{Gu'ia pr'actica del compositor 
% tipogr'afico}, Madrid, Hernando, 2"a ed., 1908 (1"a ed., 1900,
% 3"a ed., 1933).
%
% Marion Neubauer,
% <<Feinheiten bei wissenschaftlichen Publikationen>>,
% \textit{Die \TeX nisches Kom\"odie},  parte I, vol. 8, n"o 4, 1996,
% p'ags. 23-40; parte II, vol. 9, n"o 1, 1997, p'ags.~25~-44.
%
% Notimex, \textit{Manual de operaci�n y estilo editorial}, M�xico, 
% Notimex, 1999.
%
% Jos'e Polo,
% \textit{Ortograf'ia y ciencia del lenguaje}, Madrid, Paraninfo, 
% 1974.
%
% Siglo 21, \textit{Libro de estilo}, M�xico, Alda, 
% $\mathrm{^s}\!$/$\mathrm{_f}$
% (impr.  1995).
%
% Pedro Valle,
% \textit{C'omo corregir sin ofender}, Buenos Aires, Lumen, 1998.
%
% Hugh C. Wolfe, <<S'imbolos, unidades y nomenclatura>>, 
% \textit{Enciclopedia de F'isica}, dir. Rita G. Lerner y George L. 
% Trigg, Madrid, Alianza, 1987, t.~2, p'ags.~1423~-1451.
%
%\endgroup
%
% \else
%
%^^A  ======= Beginning of text as typeset by user.drv =========
%
% \GetFileInfo{spanish.dtx}
%
% \section{The Spanish language}
%
% The file \file{\filename}\footnote{The file described in this
% section has version number \fileversion\ and was last revised on
% \filedate.  The maintainer from v4.0 on is Javier Bezos
% (http://www.texytipografia.com).  Previous
% versions were made by Julio S\'anchez.  The English documentation
% has been improved by Jos� Luis Rivera; thanks to him it is now a lot
% clearer.} defines all the language-specific macros for the Spanish
% language.
%
% Spanish support is implemented following mainly the guidelines given
% by Jos\'e Mart\'\i nez de Sousa.  You may get the the full
% documentation (more comprehensive, but regrettably only in Spanish)
% by typesetting |spanish.dtx| directly.  There are examples and some
% additional features documented in the Spanish version only.
% Cross-references in this section point to that document.
% 
% \paragraph{Features} This style provides:
%
% \begin{itemize}
% \item Translations following the International \LaTeX{} 
% conventions, as well as |\today|.
% 
% \item Shorthands listed in Table~\ref{tab:spanish-quote-def}.
% Examples in subsection~3.4 are illustrative.  Notice that |"~| has a
% special meaning in \textsf{spanish} different to other languages,
% and is used mainly in linguistic contexts.
% 
% \begin{table}[htb]
% \centering
% \begin{tabular}{lp{8cm}}
% |'a|      & Acute accented a. Works for e, i, o, u, too (both
%             lowercase and uppercase).\\
% |'n|      & \~n (uppercase too).\\
% |"i|      & \"i (uppercase too).\\
% |"u|      & \"u (uppercase too).\\
% |"a| |"o| & Ordinal numbers (uppercase |"A|, |"O| too).\\
% |"er "ER| & Ordinal 1.\textsuperscript{er} 1.\textsuperscript{\textsc{er}}\\
% |"c|      & \c{c} (uppercase too).\\
% |"rr|     & rr, but -r when hyphenated.\\
% |"y|      & An old ligature for ``et'' (like the English \&).\\
% |"-|      & Like |\-|, but allowing hyphenation in the rest 
%             the word.\\
% |"=|      & Like |-|, but allowing hyphenation in the rest 
%             the word.\\
% |"~|      & The hyphen is repeated at the very beginning of 
%             the next line if the word is hyphenated at this
%             point.\\
% |""|      & Like |"-| but producing no hyphen sign.\\
% |~-|      & Like |"-| but with no break after the hyphen. Works for
%             en-dashes (|~--|) and em-dashes (|~---|). |"+|, |"+-| 
%             and |"+--| are synonymous.\\
% |"/|      & A slash slightly lowered, if necessary.\\
% \verb+"|+ & Disable ligatures at this point.\\
% |"<|      & Left guillemets.\\
% |">|      & Right guillemets.\\
% |<<| |>>| & |\begin{quoting}| and |\end{quoting}|. (See below.) 
%             |"`| and |"'| are synonymous.\\
% |"? "!|   & Opening question and exlamation marks (?`!`) 
%             aligned on the baseline, useful for all-caps headings, etc.
% \end{tabular}
% \caption{Extra definitions made by file \file{spanish.ldf}}
% \label{tab:spanish-quote-def}
% \end{table}
% 
% \item |\frenchspacing|.
%
% \item \emph{In math mode}, a dot followed by a digit is replaced
% by a decimal comma.
% 
% \item Spanish ordinals and abbreviations with the |\sptext|\marg{text}
% command as, for instance, |1\sptext{er}|. The preceptive dot is included.
% 
% \item Accented (l\'\i m, m\'ax, m\'\i n, m\'od) and spaced
% (arc\,cos, etc.) functions. 
%
% \item |\dotlessi| is provided for use in math mode.
%
% \item A |quoting| environment and a related pair of shorthands |<<|
% and |>>|. Useful for traditional spanish multi-paragraph quoting.
%
% \item There is a small space before the percent |\%| sign.
% 
% \item |\lsc| provides lowercase small caps. (See subsection~3.10.)
%
% \item Ellipsis is best typed as |...| or, within a sentence, as |\...|
% 
% \end{itemize}
% 
% If \textsf{spanish} is the main language, the command 
% |\layoutspanish| is added to the main group, modifying the standard
% classes throughout the whole document in the following way:
%
% \begin{itemize}
%
% \item Paragraphs are set with |\indentfirst|.
% 
% \item Both |enumerate| and |itemize| are adapted to Spanish rules.
% 
% \item Both |\alph| and |\Alph| include \textit{\~n} after \textit{n}.
% 
% \item Symbol footmarks are one, two, three, etc., asterisks.
% 
% \item |OT1| guillemets are generated with two |lasy| symbols instead
% of small |\ll| and |\gg|.
% 
% \item |\roman| is redefined to write small caps Roman numerals, since
% lowercase Roman numerals are discouraged (see below).
% 
% \item There is a dot after section numbers in titles, headings, and toc.
%
% \end{itemize}
%
% A subset of these features is implemented for Plain \TeX{}
% (accesible with the command |\input spanish.sty|).  Most
% significantly, |\lsc|, the |quoting| environment, and features
% provided by |\layoutspanish| are missing.
%
% \paragraph{Customization}
%
% Beginning with version 5.0, customization is made following two paths:
% via |options| or via |commands|; these options and commands override 
% the layout for Spanish documents at different levels: options are meant for 
% use at the preamble only, while commands may be used in the configuration
% file or at document level.
%
% Global options control the overall appearance of the document, and may 
% be set on the |{babel}| call, right after calling |spanish|, or 
% shortly before the call to |{babel}|, to ensure their proper loading
% at runtime. Thus, the following calls are roughly equivalent:
% 
% \begin{verbatim}
% \usepackage[...,spanish,es-nosectiondot,es-nodecimaldot,...]{babel}
% 
% \def\spanishoptions{es-nosectiondot,es-nodecimaldot}
% \usepackage[...,spanish,...]{babel}
% \end{verbatim}
%
% \begin{table}
% \centering
% \begin{tabular}{cccc}\hline
%        Basic Options       & |es-minimal| & |es-sloppy| & |es-noshorthands| \\\hline
%  |es-noindentfirst|  & X                & X          &   \\
%  |es-nosectiondot|   & X            & X           &   \\
%  |es-nolists|        & X            & X           &   \\
%  |es-noquoting|      & X            & X           & X \\
%  |es-notilde|        & X            & X           & X \\
%  |es-nodecimaldot|   & X            & X           & X \\
%  |es-nolayout|       &              & X           &   \\
%  |es-ucroman|        & X            &             &   \\
%  |es-lcroman|        & X           & X            &   \\\hline
% \end{tabular}
%        \caption{Spanish Customization Options}
%        \label{tab:SpanishCustomizationOptions}
% \end{table}
%
% Some global options are built upon lower level options, and may be
% used as shorthand for more global customizations.
% Table~\ref{tab:SpanishCustomizationOptions} gives an overview of the
% global options constructed this way.  Most of these options are
% self-explanatory: they disable the changes made to the basic \LaTeX\
% layout by |spanish|.  |es-lcroman| however, and a few others, need a
% bit of explanation, and they may be described as follows:
%
% \begin{itemize}
%
% \item Traditional Spanish typography discourages the use of
% lowercase Roman numerals; instead, a smallcaps variant is
% implemented.  However, since |Makeindex| seems to choke on the code
% implementing lowercase Roman numerals (via the |\lsc| macro), two
% workarounds are implemented: the |es-ucroman| option converts all
% Roman numerals to uppercase, and the |es-lcroman| option turns all
% Roman numerals to lowercase; the former should be preferred over the
% latter.  Three macros control local changes to Roman numbers:
% |\spanishscroman|, |\spanishucroman|, and |\spanishlcroman|.
% 
% \item The |es-preindex| option calls the |romanidx.sty| package
% automatically to fix index entries in smallcaps roman form.  An
% additional macro,
% |\spanishindexchars|\marg{encap}\marg{openrange}\marg{closerange}
% determines the characters delimiting index entries.  Defaults are
% \verb=\spanishindexchars{|}{(}{)}=.
%
% \item The |es-tilden| option restores the old tilde |~| shorthand
% for \~n.  This shorthand is however \emph{strongly} deprecated.
%
% \item The |es-nolayout| option disables layout changes in the
% document when |spanish| is the main language.  These changes affect
% enumerated and itemized lists, enumerations (alphabetic order
% excludes \~n), and symbolic footnotes.
%
% \item The |es-noshorthands| disables the shorthand mechanism
% completely: neither |"| nor |'| nor |<| nor |>| nor |~| nor |.| work
% at all.
%
% \item The |es-noquoting| option disables the macros |<<| and |>>|
% calling the |quoting| environment; the alternative macros |"`| and
% |"'| are still available.
%
% \item The |es-uppernames| option makes uppercase versions of
% captions for chapter, tablename, etc.
%
% \item The |es-tabla| option changes ``cuadro'' for ``tabla'' in
% captions.
%
% \end{itemize}
%
% Finally, the Spanish 5 series begins the implementation of national
% variations of Spanish typography, beginning with Mexico.  Thus the
% global options |mexico| and |mexico-com| are adapted to practices
% spread in Mexico, and perhaps Central America, the Caribbean, and
% some countries in South America.\footnote{The main difference is
% that |mexico| disables the |decimaldot| mechanism, while
% |mexico-com| keeps it enabled; both change the |quoting|
% environment, disabling the use of guillemets.}
%
% Many of the global options are implemented via macros, which may be
% included in the configuration file |spanish.cfg|, in the preamble,
% after the call to |babel|, and in the body of the document.  These
% macros are the following.
%
% \begin{itemize}
%
% \item The macros |\spanishdashitems| and |\spanishsignitems| change
% the values of itemized lists to a series of dashes or an alternative
% series of symbols, respectively.
%
% \item The command |\deactivatequoting| deactivates the |<<| and |>>|
% shorthands if you want to use |<| and |>| in numerical comparisons
% and some AMS\TeX\ commands.
% 
% \item You may kill the space in spaced operators with
% |\unspacedoperators|. 
%
% \item You may kill the accents on accented operators with 
% |\unaccentedoperators|.  
%
% \item The command |\decimalpoint| resets the decimal separator to
% its default (dot) value, while |\spanishdecimal|\marg{symbol} allows
% for an arbitrary definition.
%
% \item |\spanishplainpercent| prevents the addition of a thinspace before 
% the percent sign in texts. This might be useful for parenthesized percent
% signs in tables, etc.
%
% \item The macros |\spanishdatedel| and |\spanishdatede| control the 
% if the article is given in years (|del| or |de|). 
%
% \item The macro |\spanishreverseddate| sets the date of the format
% ``Month Day del Year''.
%
% \item The macro |\Today| gives months in uppercase.
%
% \item The macros |\spanish|\textit{caption} change the value of the \emph{caption} 
% automatically (no need to add an |\addto|).
%
% \item The command |\spanishdeactivate|\marg{characters} disables the
% shorthand characters listed in the argument.  Elegible characters
% are the set |.'"~<>|.  These shorthand characters may be globally
% deactivated for Spanish adding this command to |\shorthandsspanish|.
%
% \item Extras are divided in groups controlled by the commands
% |\textspanish|, |\mathspanish|, |\shorthandsspanish| y
% |\layoutspanish|; their values may be cancelled typing
% |\renewcommand|\marg{command}|{}|, or changed at will (check the
% Spanish documentation or the code for details).
% 
% \item The command |\spanishoperators|\marg{operators} defines
% command names for operators in Spanish.  There is no standard name
% for some of them, so they may be created or changed at will.  For
% instance, the command
% |\renewcommand{\spanishoperators}{arc\,ctg m\acute{i}n}|
% creates commands for these functions.  The command
% |\,| adds thinspaces at the appropriate places for spaced operators
% (like |\arcctg| in this case), and the command |\acute|\marg{letter}
% adds an accent to the letter included in the definition (thus,
% |m\acute{i}n| defines the accented function |\min| (m\'\i{}n);
% please notice that |\dotlessi| is not necessary).
% 
% \item The commands
% |\lquoti|\marg{string} |\rquoti|\marg{string} 
% |\lquotii|\marg{string} |\rquotii|\marg{string} 
% |\lquotiii|\marg{string} |\rquotiii|\marg{string} 
% set the quoting signs in the |quoting| environment, 
% nested from outside in. They may be |\renew|ed at will. 
% Default values are shown in table~\ref{tab:spanish-quote-ref}.
% \begin{table}
% \center\small
% \vspace{1.5ex}
% \begin{tabular}{l@{\hspace{5em}}l}
% |\lquoti|   &|"<|\\
% |\rquoti|   &|">|\\
% |\lquotii|  &|``|\\
% |\rquotii|  &|''|\\
% |\lquotiii| &|`|\\
% |\rquotiii| &|'|
% \end{tabular}
% \caption{Default quoting signs set for the \texttt{quoting} environment.}
% \label{tab:spanish-quote-ref}
% \end{table}
%
% \item The command |\selectspanish*| is obsolete: if |spanish| is the
% main language, all its features are available right after loading
% |babel|.  The |es-delayed| option is provided to restore the
% previous behavior and macros for backwards compatibility.
% 
% \end{itemize}
%
% \fi
% \endgroup
% 
%\StopEventually{}
%
%^^A ========== End of manual ===============
%
% \ifx\langdeffile\undefined
% 
% \section{The Code}
% 
% \else
%
% \subsection{The Code}
% 
% \fi
%
% \changes{spanish~5.0a}{2007-02-21}{Reimplemented in full, which some 
% parts rewritten from scratch. Added the es- mechanism and the mexico
% option. Many bug fixes.}
% \changes{spanish~5.0d}{2008-05-25}{Fixed two bugs: misplaced 
% subscripts with lim and the like; problem with \cs{roman} and hyperref.} 
%
%    This file provides definition for both \LaTeXe{} and non
%    \LaTeXe{} formats.
%
%    \begin{macrocode}
%<*code>
\ProvidesLanguage{spanish.ldf}
    [2008/07/06 v5.0e Spanish support from the babel system]
\LdfInit{spanish}\captionsspanish

\edef\es@savedcatcodes{%
 \catcode`\noexpand\~=\the\catcode`\~
 \catcode`\noexpand\"=\the\catcode`\"}
\catcode`\~=\active
\catcode`\"=12

\ifx\undefined\l@spanish
 \@nopatterns{Spanish}
 \adddialect\l@spanish0
\fi

\def\es@sdef#1{\babel@save#1\def#1}

\@ifundefined{documentclass}
 {\let\ifes@latex\iffalse}
 {\let\ifes@latex\iftrue}
%    \end{macrocode}
%
%    Package options for spanish. To avoid error messages dummy
%    options are created on the fly when neccessary.
%
%    \begin{macrocode}
\ifes@latex

\@ifundefined{spanishoptions}{}
{\PassOptionsToPackage{\spanishoptions}{babel}}

\def\es@genoption#1#2#3{%
 \DeclareOption{#1}{}%
 \@ifpackagewith{babel}{#1}%
  {\def\es@a{#1}%
   \expandafter\let\expandafter\es@b\csname opt@babel.sty\endcsname
   \addto\es@b{,#2}%
   \expandafter\let\csname opt@babel.sty\endcsname\es@b
   \AtEndOfPackage{#3}}%
  {}}

\es@genoption{es-minimal}
 {es-ucroman,es-noindentfirst,es-nosectiondot,es-nolists,%
  es-noquoting,es-notilde,es-nodecimaldot}
 {\spanishplainpercent
  \let\es@operators\relax}
\es@genoption{es-sloppy}
 {es-nolayout,es-noshorthands}{}
\es@genoption{es-noshorthands}
 {es-noquoting,es-nodecimaldot,es-notilde}{}
\es@genoption{mexico}
 {mexico-com,es-nodecimaldot}{}
\es@genoption{mexico-com}
 {es-tabla,es-noquoting}
 {\def\lquoti{``}\def\rquoti{''}%
  \def\lquotii{`}\def\rquotii{'}%
  \def\lquotiii{\guillemotleft{}}%
  \def\rquotiii{\guillemotright{}}}

\def\es@ifoption#1#2#3{%
 \DeclareOption{es-#1}{}%
 \@ifpackagewith{babel}{es-#1}{#2}{#3}}%

\def\es@optlayout#1#2{\es@ifoption{#1}{}{\addto\layoutspanish{#2}}}

\else

\def\es@ifoption#1#2#3{\@namedef{spanish#1}{#2}}

\fi

\let\es@uclc\@secondoftwo
\es@ifoption{uppernames}{\let\es@uclc\@firstoftwo}{}

\def\es@tablename{Ccuadro}
\es@ifoption{tabla}{\def\es@tablename{Ttabla}}{}
\es@ifoption{cuadro}{\def\es@tablename{Ccuadro}}{}
%    \end{macrocode}
%
%    Captions follow a two step schema, so that, say, |\refname| is 
%    defined as |\spanishrefname| which in turn contains the string
%    to be printed. The final definition of |\captionsspanish|
%    is built below.
%
%    \begin{macrocode}
\def\captionsspanish{%
 \es@a{preface}{Prefacio}%
 \es@a{ref}{Referencias}%
 \es@a{abstract}{Resumen}%
 \es@a{bib}{Bibliograf\'{\i}a}%
 \es@a{chapter}{Cap\'{\i}tulo}%
 \es@a{appendix}{Ap\'{e}ndice}%
 \es@a{listfigure}{\'{I}ndice de \es@uclc Ffiguras}%
 \es@a{listtable}{\'{I}ndice de \expandafter\es@uclc\es@tablename s}%
 \es@a{index}{\'{I}ndice \es@uclc Aalfab\'{e}tico}%
 \es@a{figure}{Figura}%
 \es@a{table}{\expandafter\@firstoftwo\es@tablename}%
 \es@a{part}{Parte}%
 \es@a{encl}{Adjunto}%
 \es@a{cc}{Copia a}%
 \es@a{headto}{A}%
 \es@a{page}{p\'{a}gina}%
 \es@a{see}{v\'{e}ase}%
 \es@a{also}{v\'{e}ase tambi\'{e}n}%
 \es@a{proof}{Demostraci\'{o}n}%
 \es@a{glossary}{Glosario}%
 \@ifundefined{chapter}
  {\es@a{contents}{\'Indice}}%
  {\es@a{contents}{\'Indice \es@uclc Ggeneral}}}

\def\es@a#1{\@namedef{spanish#1name}}
\captionsspanish
\def\es@a#1#2{%
 \def\expandafter\noexpand\csname#1name\endcsname
 {\expandafter\noexpand\csname spanish#1name\endcsname}}
\edef\captionsspanish{\captionsspanish}
%    \end{macrocode}
%
%    Now two macros for dates (upper and lowercase).
%
%    \begin{macrocode}
\def\es@month#1{%
 \expandafter#1\ifcase\month\or Eenero\or Ffebrero\or
 Mmarzo\or Aabril\or Mmayo\or Jjunio\or Jjulio\or Aagosto\or
 Sseptiembre\or Ooctubre\or Nnoviembre\or Ddiciembre\fi}

\def\es@today#1{%
 \ifcase\es@datefmt
  \the\day~de \es@month#1%
 \else
  \es@month#1~\the\day
 \fi
 \ de\ifnum\year>1999\es@yearl\fi~\the\year}

\def\datespanish{%
 \def\today{\es@today\@secondoftwo}%
 \def\Today{\es@today\@firstoftwo}}
\newcount\es@datefmt
\def\spanishreverseddate{\es@datefmt\@ne}
\def\spanishdatedel{\def\es@yearl{l}}
\def\spanishdatede{\let\es@yearl\@empty}
\spanishdatede
%    \end{macrocode}
%
%    The basic macros to select the language in the preamble or the
%    config file. Use of |\selectlanguage| should be avoided at this
%    early stage because the active chars are not yet
%    active. |\selectspanish| makes them active.
%
%    \begin{macrocode}
\def\selectspanish{%
 \def\selectspanish{%
  \def\selectspanish{%
   \PackageWarning{spanish}{Extra \string\selectspanish ignored}}%
  \es@select}}
\@onlypreamble\selectspanish
\def\es@select{%
 \let\es@select\@undefined
 \selectlanguage{spanish}}

\let\es@shlist\@empty
%    \end{macrocode}
%
%    Instead of joining all the extras directly in |\extrasspanish|,
%    we subdivide them in three further groups.
%
%    \begin{macrocode}
\def\extrasspanish{%
 \textspanish
 \mathspanish
 \ifx\shorthandsspanish\@empty
  \expandafter\spanishdeactivate\expandafter{\es@shlist}%
  \languageshorthands{none}%
 \else
  \shorthandsspanish
 \fi}
\def\noextrasspanish{%
 \ifx\textspanish\@empty\else
  \notextspanish
 \fi
 \ifx\mathspanish\@empty\else
  \nomathspanish
 \fi
 \ifx\shorthandsspanish\@empty\else
  \noshorthandsspanish
 \fi
 \csname es@restorelist\endcsname}

\addto\textspanish{\es@sdef\sptext{\protect\es@sptext}}

\def\es@orddot{.}
%    \end{macrocode}
%
%    The definition of |\sptext| is more elaborated than that of
%     |\textsuperscript|. With uppercase superscript text
%    the scriptscriptsize is used. The mandatory dot is already
%    included. There are two versions, depending on the
%    format.
%
%    \begin{macrocode}
\ifes@latex
 \def\es@sptext#1{%
  {\es@orddot
   \setbox\z@\hbox{8}\dimen@\ht\z@
   \csname S@\f@size\endcsname
   \edef\@tempa{\def\noexpand\@tempc{#1}%
    \lowercase{\def\noexpand\@tempb{#1}}}\@tempa
   \ifx\@tempb\@tempc
    \fontsize\sf@size\z@
    \selectfont
    \advance\dimen@-1.15ex
   \else
    \fontsize\ssf@size\z@
    \selectfont
    \advance\dimen@-1.5ex
   \fi
   \math@fontsfalse\raise\dimen@\hbox{#1}}}
\else
 \let\sptextfont\rm
 \def\es@sptext#1{%
  {\es@orddot
   \setbox\z@\hbox{8}\dimen@\ht\z@
   \edef\@tempa{\def\noexpand\@tempc{#1}%
    \lowercase{\def\noexpand\@tempb{#1}}}\@tempa
   \ifx\@tempb\@tempc
    \advance\dimen@-0.75ex
    \raise\dimen@\hbox{$\scriptstyle\sptextfont#1$}%
   \else
    \advance\dimen@-0.8ex
    \raise\dimen@\hbox{$\scriptscriptstyle\sptextfont#1$}%
   \fi}}
\fi
%    \end{macrocode}
%
%    Now, lowercase small caps. First, we test if there are actual
%    small caps for the current font. If not, faked small caps are
%    used. The \cs{selectfont} in \cs{es@lsc} could seem redundant,
%    but it's not. An intermediate macro allows using an optimized
%    variant for Roman numerals.
%
%    \begin{macrocode}
\ifes@latex
 \addto\textspanish{\es@sdef\lsc{\protect\es@lsc}} 
 \def\es@lsc{\es@xlsc\MakeUppercase\MakeLowercase}
 \def\es@xlsc#1#2#3{%
  \leavevmode
  \hbox{%
   \scshape\selectfont
   \expandafter\ifx\csname\f@encoding/\f@family/\f@series
      /n/\f@size\expandafter\endcsname
    \csname\curr@fontshape/\f@size\endcsname
    \csname S@\f@size\endcsname
    \fontsize\sf@size\z@\selectfont
    \PackageWarning{spanish}{Replacing `\curr@fontshape' by
      \MessageBreak faked small caps}%
    #1{#3}%
   \else
    #2{#3}%
   \fi}}
\fi
%    \end{macrocode}
%
%    The |quoting| environment. This part is not available
%    in Plain. Overriding the default |\everypar| is
%    a bit tricky.
%
%    \begin{macrocode}
\newif\ifes@listquot

\ifes@latex
 \csname newtoks\endcsname\es@quottoks
 \csname newcount\endcsname\es@quotdepth
 \newenvironment{quoting}
  {\leavevmode
  \advance\es@quotdepth\@ne
  \csname lquot\romannumeral\es@quotdepth\endcsname%
  \ifnum\es@quotdepth=\@ne
   \es@listquotfalse
   \let\es@quotpar\everypar
   \let\everypar\es@quottoks
   \everypar\expandafter{\the\es@quotpar}%
   \es@quotpar{\the\everypar
    \ifes@listquot\global\es@listquotfalse\else\es@quotcont\fi}%
  \fi
  \toks@\expandafter{\es@quotcont}%
  \edef\es@quotcont{\the\toks@
   \expandafter\noexpand
   \csname rquot\romannumeral\es@quotdepth\endcsname}}
  {\csname rquot\romannumeral\es@quotdepth\endcsname}
 \def\lquoti{\guillemotleft{}}
 \def\rquoti{\guillemotright{}}
 \def\lquotii{``}
 \def\rquotii{''}
 \def\lquotiii{`}
 \def\rquotiii{'}
 \let\es@quotcont\@empty
%    \end{macrocode}
%
%    If there is a margin par inside quoting, we don't add the
%    quotes. |\es@listqout| stores the quotes to be used before
%    item labels; otherwise they could appear after the labels.
%
%    \begin{macrocode}  
 \addto\@marginparreset{\let\es@quotcont\@empty}
 \DeclareRobustCommand\es@listquot{%
  \csname rquot\romannumeral\es@quotdepth\endcsname
  \global\es@listquottrue}
\fi
%    \end{macrocode}
%
%    Now, the |\frenchspacing|, followed by |\...| and |\%|. 
%
%    \begin{macrocode}
\addto\textspanish{\bbl@frenchspacing}
\addto\notextspanish{\bbl@nonfrenchspacing}
\addto\textspanish{%
 \let\es@save@dot\.%
 \es@sdef\.{\@ifnextchar.{\es@dots}{\es@save@dot}}}
\def\es@dots..{\leavevmode\hbox{...}\spacefactor\@M}
\def\es@sppercent{\unskip\textormath{$\m@th\,$}{\,}}
\def\spanishplainpercent{\let\es@sppercent\@empty}
\addto\textspanish{%
 \let\percentsign\%%
 \es@sdef\%{\es@sppercent\percentsign{}}}
%    \end{macrocode}
%    
%    We follow with the math group.  It's not easy to add an accent to
%    an operator.  The difficulty is that we must avoid using text
%    (that is, |\mbox|) because we have no control on font and size,
%    and at time we should access |\i|, which is a text command
%    forbidden in math mode.  |\dotlessi| must be converted to
%    uppercase if necessary in \LaTeXe.  There are two versions,
%    depending on the format.
%
%    \begin{macrocode}
\addto\mathspanish{\es@sdef\dotlessi{\protect\es@dotlessi}}
\let\nomathspanish\relax

\ifes@latex
 \def\es@texti{\i}
 \addto\@uclclist{\dotlessi\es@texti}
\fi

\ifes@latex
 \def\es@dotlessi{%
  \ifmmode
   {\ifnum\mathgroup=\m@ne
     \imath
    \else
     \count@\escapechar \escapechar=\m@ne
     \expandafter\expandafter\expandafter
      \split@name\expandafter\string\the\textfont\mathgroup\@nil
     \escapechar=\count@
     \@ifundefined{\f@encoding\string\i}%
      {\edef\f@encoding{\string?}}{}%
     \expandafter\count@\the\csname\f@encoding\string\i\endcsname
     \advance\count@"7000
     \mathchar\count@
    \fi}%
  \else
   \i
  \fi}
\else
 \def\es@dotlessi{\textormath{\i}{\mathchar"7010}}
\fi

\def\accentedoperators{%
 \def\es@op@ac##1{\acute{\if i##1\dotlessi\else##1\fi}}}
\def\unaccentedoperators{%
 \def\es@op@ac##1{##1}}
\accentedoperators
\def\spacedoperators{\let\es@op@sp\,}
\def\unspacedoperators{\let\es@op@sp\@empty}
\spacedoperators
\addto\mathspanish{\es@operators}

\ifes@latex\else
 \let\operator@font\rm
\fi
%    \end{macrocode}
%
%    The operators are stored in |\es@operators|, which in turn is
%    included in the math group. Since |\operator@font| is
%    defined in \LaTeXe{} only, we defined in the plain variant.
%
%    \begin{macrocode}
\def\es@operators{%
 \es@sdef\bmod{\nonscript\mskip-\medmuskip\mkern5mu
  \mathbin{\operator@font m\es@op@ac od}\penalty900\mkern5mu
  \nonscript\mskip-\medmuskip}%
 \@ifundefined{@amsmath@err}%
  {\es@sdef\pmod##11{\allowbreak\mkern18mu
    ({\operator@font m\es@op@ac od}\,\,##11)}}%
  {\es@sdef\mod##1{\allowbreak\if@display\mkern18mu
    \else\mkern12mu\fi{\operator@font m\es@op@ac od}\,\,##1}%
   \es@sdef\pmod##1{\pod{{\operator@font m\es@op@ac od}%
    \mkern6mu##1}}}%
 \def\es@a##1 {%
  \if^##1^% empty? continue
   \bbl@afterelse
   \es@a
  \else
   \bbl@afterfi
   {\if&##1% &? finish
   \else
    \bbl@afterfi
    \begingroup
    \let\,\@empty % ignore when def'ing name
    \let\acute\@firstofone % id
    \edef\es@b{\expandafter\noexpand\csname##1\endcsname}%
    \def\,{\noexpand\es@op@sp}%
    \def\acute{\noexpand\es@op@ac}%
    \edef\es@a{\endgroup
     \noexpand\es@sdef\expandafter\noexpand\es@b{%
       \mathop{\noexpand\operator@font##1}\es@c}}%
    \es@a % restores itself
   \es@a
  \fi}%
 \fi}%
 \let\es@b\spanishoperators
 \addto\es@b{ }%
 \let\es@c\@empty
 \expandafter\es@a\es@b l\acute{i}m l\acute{i}m\,sup
  l\acute{i}m\,inf m\acute{a}x \acute{i}nf m\acute{i}n & %
 \def\es@c{\nolimits}%
 \expandafter\es@a\es@b sen tg arc\,sen arc\,cos arc\,tg & }
\def\spanishoperators{cotg cosec senh tgh }
%    \end{macrocode}
%
%    Now comes the text shorthands. They are grouped in
%    |\shorthandsspanish| and this style performs some
%    operations before the babel shortands are called.
%    The aims are to allow espression like |$a^{x'}$|
%    and to deactivate shorthands by making them of
%    category `other.' After providing a |\'i| shorthand,
%    the new macros are defined. 
%    
%    \begin{macrocode}
\DeclareTextCompositeCommand{\'}{OT1}{i}{\@tabacckludge'{\i}}

\def\es@set@shorthand#1{%
 \expandafter\edef\csname es@savecat\string#1\endcsname
  {\the\catcode`#1}%
 \initiate@active@char{#1}%
 \catcode`#1=\csname es@savecat\string#1\endcsname\relax
 \if.#1\else
  \addto\es@restorelist{\es@restore{#1}}%
  \addto\es@select{\shorthandon{#1}}%
  \addto\shorthandsspanish{\es@activate{#1}}%
  \addto\es@shlist{#1}%
 \fi}

\def\es@use@shorthand{%
 \ifx\thepage\relax
  \bbl@afterelse
  \string
 \else
  \bbl@afterfi
  {\ifx\protect\@unexpandable@protect
   \bbl@afterelse
   \noexpand
  \else
   \bbl@afterfi
   \es@use@sh
  \fi}%
 \fi}

\def\es@use@sh#1{%
 \if@safe@actives
  \bbl@afterelse
  \string#1%
 \else%
  \bbl@afterfi
  \textormath
   {\csname active@char\string#1\endcsname}%
   {\csname normal@char\string#1\endcsname}%
 \fi}

\gdef\es@activate#1{%
 \begingroup
  \lccode`\~=`#1
  \lowercase{%
 \endgroup
 \def~{\es@use@shorthand~}}}

\def\spanishdeactivate#1{%
 \@tfor\@tempa:=#1\do{\expandafter\es@spdeactivate\@tempa}}

\def\es@spdeactivate#1{%
 \if.#1%
  \mathcode`\.=\es@period@code
 \else
  \begingroup
   \lccode`\~=`#1
   \lowercase{%
  \endgroup
  \expandafter\let\expandafter~%
   \csname normal@char\string#1\endcsname}%
  \catcode`#1=\csname es@savecat\string#1\endcsname\relax
 \fi}
%    \end{macrocode}
%
%    |\es@restore| is used in the list |\es@restorelist|, which in
%    turn restores all shorthands as defined by \babel. The latter
%    macros also has |\es@quoting|.
%
%    \begin{macrocode}
\def\es@restore#1{%
 \shorthandon{#1}%
 \begingroup
  \lccode`\~=`#1
  \lowercase{%
 \endgroup
 \bbl@deactivate{~}}}
%    \end{macrocode}
%
%    To selectively define the shorthands we have a couple of
%    macros, which defines a certain combination if the first
%    character has been activated as a shorthand. The second
%    one is intended for a few shorthands with an alternative
%    form.                                                      
%
%    \begin{macrocode}
\def\es@declare#1{%
 \@ifundefined{es@savecat\expandafter\string\@firstoftwo#1}%
  {\@gobble}%
  {\declare@shorthand{spanish}{#1}}}
\def\es@declarealt#1#2#3{%
 \es@declare{#1}{#3}%
 \es@declare{#2}{#3}}

\ifes@latex\else
 \def\@tabacckludge#1{\csname\string#1\endcsname}
\fi

\@ifundefined{add@accent}{\def\add@accent#1#2{\accent#1 #2}}{}
%    \end{macrocode}
%
%    Instead of redefining |\'|, we redefine the internal
%    macro for the OT1 encoding.
%
%    \begin{macrocode}
\ifes@latex
 \def\es@accent#1#2#3{%
  \expandafter\@text@composite
  \csname OT1\string#1\endcsname#3\@empty\@text@composite
  {\bbl@allowhyphens\add@accent{#2}{#3}\bbl@allowhyphens
   \setbox\@tempboxa\hbox{#3%
    \global\mathchardef\accent@spacefactor\spacefactor}%
   \spacefactor\accent@spacefactor}}
\else
 \def\es@accent#1#2#3{%
  \bbl@allowhyphens\add@accent{#2}{#3}\bbl@allowhyphens
  \spacefactor\sfcode`#3 }
\fi

\addto\shorthandsspanish{\languageshorthands{spanish}}%
\es@ifoption{noshorthands}{}{\es@set@shorthand{"}}
%    \end{macrocode}
%
%    We override the default |"| of babel, intended for german.
%
%    \begin{macrocode}
\def\es@umlaut#1{%
 \bbl@allowhyphens\add@accent{127}#1\bbl@allowhyphens
 \spacefactor\sfcode`#1 }

\addto\shorthandsspanish{%
 \babel@save\bbl@umlauta
 \let\bbl@umlauta\es@umlaut}
\let\noshorthandsspanish\relax

\ifes@latex
\addto\shorthandsspanish{%
 \expandafter\es@sdef\csname OT1\string\~\endcsname{\es@accent\~{126}}%
 \expandafter\es@sdef\csname OT1\string\'\endcsname{\es@accent\'{19}}}
\else
\addto\shorthandsspanish{%
 \es@sdef\~{\es@accent\~{126}}%
 \es@sdef\'#1{\if#1i\es@accent\'{19}\i\else\es@accent\'{19}{#1}\fi}}
\fi

\def\es@sptext@r#1#2{\es@sptext{#1#2}}
\es@declare{"a}{\sptext{a}}
\es@declare{"A}{\sptext{A}}
\es@declare{"o}{\sptext{o}}
\es@declare{"O}{\sptext{O}}
\es@declare{"e}{\protect\es@sptext@r{e}}
\es@declare{"E}{\protect\es@sptext@r{E}}
\es@declare{"u}{\"u}
\es@declare{"U}{\"U}
\es@declare{"i}{\"{\i}}
\es@declare{"I}{\"I}
\es@declare{"c}{\c{c}}
\es@declare{"C}{\c{C}}
\es@declare{"<}{\guillemotleft{}}
\es@declare{">}{\guillemotright{}}
\def\es@chf{\char\hyphenchar\font}
\es@declare{"-}{\bbl@allowhyphens\-\bbl@allowhyphens}
\es@declare{"=}{\bbl@allowhyphens\es@chf\hskip\z@skip}
\es@declare{"~}
 {\bbl@allowhyphens
  \discretionary{\es@chf}{\es@chf}{\es@chf}%
  \bbl@allowhyphens}
\es@declare{"r}
 {\bbl@allowhyphens
  \discretionary{\es@chf}{}{r}%
  \bbl@allowhyphens}
\es@declare{"R}
 {\bbl@allowhyphens
  \discretionary{\es@chf}{}{R}%
  \bbl@allowhyphens}
\es@declare{"y}
 {\@ifundefined{scalebox}%
   {\ensuremath{\tau}}%
   {\raisebox{1ex}{\scalebox{-1}{\resizebox{.45em}{1ex}{2}}}}}
\es@declare{""}{\hskip\z@skip}
\es@declare{"/}
 {\setbox\z@\hbox{/}%
  \dimen@\ht\z@
  \advance\dimen@-1ex
  \advance\dimen@\dp\z@
  \dimen@.31\dimen@
  \advance\dimen@-\dp\z@
  \ifdim\dimen@>0pt
   \kern.01em\lower\dimen@\box\z@\kern.03em
  \else
   \box\z@
  \fi}
\es@declare{"?}
 {\setbox\z@\hbox{?`}%
  \leavevmode\raise\dp\z@\box\z@}
\es@declare{"!}
 {\setbox\z@\hbox{!`}%
  \leavevmode\raise\dp\z@\box\z@}

\def\spanishdecimal#1{\def\es@decimal{{#1}}}
\def\decimalcomma{\spanishdecimal{,}}
\def\decimalpoint{\spanishdecimal{.}}
\decimalcomma
\es@ifoption{nodecimaldot}{}
 {\AtBeginDocument{\bgroup\@fileswfalse}%
  \es@set@shorthand{.}%
  \AtBeginDocument{\egroup}%
  \@namedef{normal@char\string.}{%
   \@ifnextchar\egroup
    {\mathchar\es@period@code\relax}%
    {\csname active@char\string.\endcsname}}%
  \declare@shorthand{system}{.}{\mathchar\es@period@code\relax}%
  \addto\shorthandsspanish{%
   \mathchardef\es@period@code\the\mathcode`\.%
   \babel@savevariable{\mathcode`\.}%
   \mathcode`\.="8000 %
   \es@activate{.}}%
  \def\es@a#1{\es@declare{.#1}{\es@decimal#1}}%
  \es@a1\es@a2\es@a3\es@a4\es@a5\es@a6\es@a7\es@a8\es@a9\es@a0}

\es@ifoption{notilde}{}{\es@set@shorthand{~}}
\def\deactivatetilden{%
 \expandafter\let\csname spanish@sh@\string~@n@\endcsname\relax
 \expandafter\let\csname spanish@sh@\string~@N@\endcsname\relax}
\es@ifoption{tilden}
 {\es@declare{~n}{\~n}%
  \es@declare{~N}{\~N}}
 {\let\deactivatetilden\relax}
\es@declarealt{~-}{"+}{%
 \leavevmode
 \bgroup
 \let\@sptoken\es@dashes % Changes \@ifnextchar behaviour
 \@ifnextchar-%       
  {\es@dashes}%
  {\hbox{\es@chf}\egroup}}
\def\es@dashes-{%
 \@ifnextchar-%
  {\bbl@allowhyphens\hbox{---}\bbl@allowhyphens\egroup\@gobble}%
  {\bbl@allowhyphens\hbox{--}\bbl@allowhyphens\egroup}}

\es@ifoption{noquoting}%
 {\let\es@quoting\relax
 \let\activatequoting\relax
 \let\deactivatequoting\relax}
 {\@ifundefined{XML@catcodes}%
 {\es@set@shorthand{<}%
  \es@set@shorthand{>}%
  \declare@shorthand{system}{<}{\csname normal@char\string<\endcsname}%
  \declare@shorthand{system}{>}{\csname normal@char\string>\endcsname}%
  \addto\es@restorelist{\es@quoting}%
  \addto\es@select{\es@quoting}%
  \ifes@latex
   \AtBeginDocument{%
    \es@quoting
    \if@filesw
     \immediate\write\@mainaux{\string\@nameuse{es@quoting}}%
    \fi}%
  \fi
  \def\activatequoting{%
   \shorthandon{<>}%
   \let\es@quoting\activatequoting}%
  \def\deactivatequoting{%
   \shorthandoff{<>}%
   \let\es@quoting\deactivatequoting}}{}}

\es@declarealt{<<}{"`}{\begin{quoting}}
\es@declarealt{>>}{"'}{\end{quoting}}
%    \end{macrocode}
%
%    Acute accent shorthands are stored in a macro.  If |activeacute|
%    was set as an option it's executed.  If not is not deleted for a
%    possible later use in the |cfg| file.  In non \LaTeXe{} formats
%    is always executed.
%
% \changes{spanish-5.0e}{2008/07/06}{Two acutes in a row should be
%    turned into a double right quote}
%    \begin{macrocode}
\begingroup
\catcode`\'=12
\gdef\es@activeacute{%
 \es@set@shorthand{'}%
 \def\es@a##1{\es@declare{'##1}{\@tabacckludge'##1}}%
 \es@a a\es@a e\es@a i\es@a o\es@a u%
 \es@a A\es@a E\es@a I\es@a O\es@a U%
 \es@declare{'n}{\~n}%
 \es@declare{'N}{\~N}%
 \es@declare{''}{\textquotedblright}%
%    \end{macrocode}
%
%    But \textsf{spanish} allows two category codes for |'|,
%    so both should be taken into account in \cs{bbl@pr@m@s}.
%    
%    \begin{macrocode}
 \let\es@pr@m@s\bbl@pr@m@s
 \def\bbl@pr@m@s{%
  \ifx'\@let@token
   \bbl@afterelse
   \pr@@@s
  \else
   \bbl@afterfi
   \es@pr@m@s
  \fi}%
 \let\es@activeacute\relax}
\endgroup

\ifes@latex
 \@ifpackagewith{babel}{activeacute}{\es@activeacute}{}
\else
 \es@activeacute
\fi
%    \end{macrocode}
%
%    And the customization. By default these macros only
%    store the values and do nothing.
%
%    \begin{macrocode}
\def\es@enumerate#1#2#3#4{\def\es@enum{{#1}{#2}{#3}{#4}}}
\def\es@itemize#1#2#3#4{\def\es@item{{#1}{#2}{#3}{#4}}}

\ifes@latex
\es@enumerate{1.}{a)}{1)}{a$'$}
\def\spanishdashitems{\es@itemize{---}{---}{---}{---}}
\def\spanishsymbitems{%
 \es@itemize
  {\leavevmode\hbox to 1.2ex
   {\hss\vrule height .9ex width .7ex depth -.2ex\hss}}%
  {\textbullet}%
  {$\m@th\circ$}%
  {$\m@th\diamond$}}
\def\spanishsignitems{%
 \es@itemize{\textbullet}%
  {$\m@th\circ$}%
  {$\m@th\diamond$}%
  {$\m@th\triangleright$}}
\spanishsymbitems
\def\es@enumdef#1#2#3\@@{%
 \if#21%
  \@namedef{theenum#1}{\arabic{enum#1}}%
 \else\if#2a%
  \@namedef{theenum#1}{\emph{\alph{enum#1}}}%
 \else\if#2A%
  \@namedef{theenum#1}{\Alph{enum#1}}%
 \else\if#2i%
  \@namedef{theenum#1}{\roman{enum#1}}%
 \else\if#2I%
  \@namedef{theenum#1}{\Roman{enum#1}}%
 \else\if#2o%
  \@namedef{theenum#1}{\arabic{enum#1}\sptext{o}}%
 \fi\fi\fi\fi\fi\fi
 \toks@\expandafter{\csname theenum#1\endcsname}%
 \expandafter\edef\csname labelenum#1\endcsname
   {\noexpand\es@listquot\the\toks@#3}}
\def\es@guillemot#1#2{%
 \ifmmode#1%
 \else
  \save@sf@q{\penalty\@M
  \leavevmode\hbox{\usefont{U}{lasy}{m}{n}%
   \char#2 \kern-0.19em\char#2 }}%
 \fi}
\def\layoutspanish{%
 \let\layoutspanish\@empty
 \DeclareTextCommand{\guillemotleft}{OT1}{\es@guillemot\ll{40}}%
 \DeclareTextCommand{\guillemotright}{OT1}{\es@guillemot\gg{41}}%
 \def\@fnsymbol##1%
  {\ifcase##1\or*\or**\or***\or****\or
   *****\or******\else\@ctrerr\fi}%
 \def\@alph##1%
  {\ifcase##1\or a\or b\or c\or d\or e\or f\or g\or h\or i\or j\or
   k\or l\or m\or n\or \~n\or o\or p\or q\or r\or s\or t\or u\or v\or
   w\or x\or y\or z\else\@ctrerr\fi}%
 \def\@Alph##1%
  {\ifcase##1\or A\or B\or C\or D\or E\or F\or G\or H\or I\or J\or
   K\or L\or M\or N\or \~N\or O\or P\or Q\or R\or S\or T\or U\or V\or
   W\or X\or Y\or Z\else\@ctrerr\fi}}

\es@optlayout{nolists}{%
 \def\es@enumerate#1#2#3#4{%
  \es@enumdef{i}#1\@empty\@empty\@@
  \es@enumdef{ii}#2\@empty\@empty\@@
  \es@enumdef{iii}#3\@empty\@empty\@@
  \es@enumdef{iv}#4\@empty\@empty\@@}%
 \def\es@itemize#1#2#3#4{%
  \def\labelitemi{\es@listquot#1}%
  \def\labelitemii{\es@listquot#2}%
  \def\labelitemiii{\es@listquot#3}%
  \def\labelitemiv{\es@listquot#4}}%
 \def\p@enumii{\theenumi}%
 \def\p@enumiii{\p@enumii\theenumii}%
 \def\p@enumiv{\p@enumiii\theenumiii}%
 \expandafter\es@enumerate\es@enum
 \expandafter\es@itemize\es@item}
\let\esromanindex\@secondoftwo
\es@ifoption{ucroman}
 {\def\es@romandef{%
   \def\esromanindex##1##2{##1{\uppercase{##2}}}%
   \def\@roman{\@Roman}}}
 {\def\es@romandef{%
   \def\esromanindex##1##2{##1{\protect\es@roman{##2}}}%
   \def\@roman##1{\es@roman{\number##1}}%
   \def\es@roman##1{\es@scroman{\romannumeral##1}}%
   \DeclareRobustCommand\es@scroman{\es@xlsc\uppercase\@firstofone}}}
\es@optlayout{lcroman}{\es@romandef}
\newcommand\spanishlcroman{\def\@roman##1{\romannumeral##1}}
\newcommand\spanishucroman{\def\@roman{\@Roman}}
\newcommand\spanishscroman{\def\@roman##1{\es@roman{\romannumeral##1}}}
\es@optlayout{noindentfirst}{%
 \let\@afterindentfalse\@afterindenttrue
 \@afterindenttrue}
\es@optlayout{nosectiondot}{%
 \def\@seccntformat#1{\csname the#1\endcsname.\quad}%
 \def\numberline#1{\hb@xt@\@tempdima{#1\if&#1&\else.\fi\hfil}}}
\es@ifoption{nolayout}{\let\layoutspanish\relax}{}
\es@ifoption{sloppy}{\let\textspanish\relax\let\mathspanish\relax}{}
\es@ifoption{delayed}{}{\def\es@layoutspanish{\layoutspanish}}
\es@ifoption{preindex}{\AtEndOfPackage{\RequirePackage{romanidx}}}{}
%    \end{macrocode}
%    
%    We need to execute the following code when babel has been
%    run, in order to see if |spanish| is the main language.
%    
%    \begin{macrocode}
\AtEndOfPackage{%
\let\es@activeacute\@undefined
\def\bbl@tempa{spanish}%
\ifx\bbl@main@language\bbl@tempa
 \@nameuse{es@layoutspanish}%
 \addto\es@select{%
  \@ifstar{\PackageError{spanish}%
   {Old syntax--use es-nolayout}%
   {If you don't want changes in layout\MessageBreak
   use the es-nolayout package option}}%
   {}}%
 \AtBeginDocument{\layoutspanish}%
\fi
\selectspanish}
\fi
%    \end{macrocode}
%    
%    After restoring the catcode of |~| and setting the minimal
%    values for hyphenation, the |.ldf| is finished.
%
%    \begin{macrocode}
\es@savedcatcodes
\providehyphenmins{\CurrentOption}{\tw@\tw@}
\ifes@latex\else
 \es@select
\fi
\ldf@finish{spanish}
\csname activatequoting\endcsname
%</code>
%    \end{macrocode}
%    That's all in the main file. 
%
%    The |spanish| option writes a macro in the page field of
%    \textit{MakeIndex} in entries with small caps number, and they
%    are rejected. This program is a preprocessor which moves this
%    macro to the entry field. It can be called from the main 
%    document as a package or with the package option |es-preindex|.
%
%    \begin{macrocode}
%<*indexes>
\makeatletter

\@ifundefined{es@idxfile}
  {\def\spanishindexchars#1#2#3{%
     \edef\es@encap{`\expandafter\noexpand\csname\string#1\endcsname}%
     \edef\es@openrange{`\expandafter\noexpand\csname\string#2\endcsname}%
     \edef\es@closerange{`\expandafter\noexpand\csname\string#3\endcsname}}%
   \spanishindexchars{|}{(}{)}%
   \ifx\documentclass\@twoclasseserror
      \edef\es@idxfile{\jobname}%
      \AtEndDocument{%
        \addto\@defaultsubs{%
          \immediate\closeout\@indexfile
          \input{romanidx.sty}}}%
      \expandafter\endinput
   \fi}{}

\newcount\es@converted
\newcount\es@processed

\def\es@split@file#1.#2\@@{#1}
\def\es@split@ext#1.#2\@@{#2}

\@ifundefined{es@idxfile}
  {\typein[\answer]{^^JArchivo que convertir^^J%
   (extension por omision .idx):}}
  {\let\answer\es@idxfile}

\@expandtwoargs\in@{.}{\answer}
\ifin@
  \edef\es@input@file{\expandafter\es@split@file\answer\@@}
  \edef\es@input@ext{\expandafter\es@split@ext\answer\@@}
\else
  \edef\es@input@file{\answer}
  \def\es@input@ext{idx}
\fi

\@ifundefined{es@idxfile}
  {\typein[\answer]{^^JArchivo de destino^^J%
     (archivo por omision: \es@input@file.eix,^^J%
      extension por omision .eix):}}
  {\let\answer\es@idxfile}
\ifx\answer\@empty
  \edef\es@output{\es@input@file.eix}
\else
  \@expandtwoargs\in@{.}{\answer}
  \ifin@
     \edef\es@output{\answer}
  \else
     \edef\es@output{\answer.eix}
  \fi
\fi

\@ifundefined{es@idxfile}
  {\typein[\answer]{%
   ^^J?Se ha usado algun esquema especial de controles^^J%
   de MakeIndex para encap, open_range o close_range?^^J%
   [s/n] (n por omision)}}
  {\def\answer{n}}

\if s\answer
  \typein[\answer]{^^JCaracter para 'encap'^^J%
    (\string| por omision)}
  \ifx\answer\@empty\else
    \edef\es@encap{%
      `\expandafter\noexpand\csname\expandafter\string\answer\endcsname}
  \fi
  \typein[\answer]{^^JCaracter para 'open_range'^^J%
    (\string( por omision)}
  \ifx\answer\@empty\else
    \edef\es@openrange{%
      `\expandafter\noexpand\csname\expandafter\string\answer\endcsname}
  \fi
  \typein[\answer]{^^JCaracter para 'close_range'^^J%
    (\string) por omision)}
  \ifx\answer\@empty\else
    \edef\es@closerange{%
      `\expandafter\noexpand\csname\expandafter\string\answer\endcsname}
  \fi
\fi

\newwrite\es@indexfile
\immediate\openout\es@indexfile=\es@output

\newif\ifes@encapsulated

\def\es@roman#1{#1}
\edef\es@slash{\expandafter\@gobble\string\\}

\def\indexentry{%
  \begingroup
  \@sanitize
  \es@indexentry}

\begingroup

\catcode`\|=12 \lccode`\|=\es@encap\relax
\catcode`\(=12 \lccode`\(=\es@openrange\relax
\catcode`\)=12 \lccode`\)=\es@closerange\relax

\lowercase{
\gdef\es@indexentry#1{%
  \endgroup
  \advance\es@processed\@ne
  \es@encapsulatedfalse
  \es@bar@idx#1|\@@
  \es@idxentry}%
}

\lowercase{
\gdef\es@idxentry#1{%
  \in@{\es@roman}{#1}%
  \ifin@
    \advance\es@converted\@ne
    \immediate\write\es@indexfile{%
      \string\indexentry{\es@b|\ifes@encapsulated\es@p\fi esromanindex%
        {\ifx\es@a\@empty\else\es@slash\es@a\fi}}{#1}}%
  \else
    \immediate\write\es@indexfile{%
      \string\indexentry{\es@b\ifes@encapsulated|\es@p\es@a\fi}{#1}}%
  \fi}
}

\lowercase{
\gdef\es@bar@idx#1|#2\@@{%
  \def\es@b{#1}\def\es@a{#2}%
  \ifx\es@a\@empty\else\es@encapsulatedtrue\es@bar@eat#2\fi}
}

\lowercase{
\gdef\es@bar@eat#1#2|{\def\es@p{#1}\def\es@a{#2}%
  \edef\es@t{(}\ifx\es@t\es@p
  \else\edef\es@t{)}\ifx\es@t\es@p
  \else
    \edef\es@a{\es@p\es@a}\let\es@p\@empty%
  \fi\fi}
}

\endgroup

\input \es@input@file.\es@input@ext

\immediate\closeout\es@indexfile

\typeout{*****************}
\typeout{Se ha procesado: \es@input@file.\es@input@ext }
\typeout{Lineas leidas: \the\es@processed}
\typeout{Lineas convertidas: \the\es@converted}
\typeout{Resultado en: \es@output}
\ifnum\es@converted>\z@
  \typeout{Genere el indice a partir de ese archivo}
\else
  \typeout{No se ha convertido nada. Se puede generar}
  \typeout{el .ind  directamente de \es@input@file.\es@input@ext}
\fi
\typeout{*****************}

\@ifundefined{es@sdef}{\@@end}{}

\endinput
%</indexes>
%    \end{macrocode}
%
% \Finale
%
%%
%% \CharacterTable
%%  {Upper-case    \A\B\C\D\E\F\G\H\I\J\K\L\M\N\O\P\Q\R\S\T\U\V\W\X\Y\Z
%%   Lower-case    \a\b\c\d\e\f\g\h\i\j\k\l\m\n\o\p\q\r\s\t\u\v\w\x\y\z
%%   Digits        \0\1\2\3\4\5\6\7\8\9
%%   Exclamation   \!     Double quote  \"     Hash (number) \#
%%   Dollar        \$     Percent       \%     Ampersand     \&
%%   Acute accent  \'     Left paren    \(     Right paren   \)
%%   Asterisk      \*     Plus          \+     Comma         \,
%%   Minus         \-     Point         \.     Solidus       \/
%%   Colon         \:     Semicolon     \;     Less than     \<
%%   Equals        \=     Greater than  \>     Question mark \?
%%   Commercial at \@     Left bracket  \[     Backslash     \\
%%   Right bracket \]     Circumflex    \^     Underscore    \_
%%   Grave accent  \`     Left brace    \{     Vertical bar  \|
%%   Right brace   \}     Tilde         \~}
%%
\endinput




}
\DeclareOption{swedish}{% \iffalse meta-comment
%
% Copyright 1989-2005 Johannes L. Braams and any individual authors
% listed elsewhere in this file.  All rights reserved.
% 
% This file is part of the Babel system.
% --------------------------------------
% 
% It may be distributed and/or modified under the
% conditions of the LaTeX Project Public License, either version 1.3
% of this license or (at your option) any later version.
% The latest version of this license is in
%   http://www.latex-project.org/lppl.txt
% and version 1.3 or later is part of all distributions of LaTeX
% version 2003/12/01 or later.
% 
% This work has the LPPL maintenance status "maintained".
% 
% The Current Maintainer of this work is Johannes Braams.
% 
% The list of all files belonging to the Babel system is
% given in the file `manifest.bbl. See also `legal.bbl' for additional
% information.
% 
% The list of derived (unpacked) files belonging to the distribution
% and covered by LPPL is defined by the unpacking scripts (with
% extension .ins) which are part of the distribution.
% \fi
% \CheckSum{272}
% \iffalse
%    Tell the \LaTeX\ system who we are and write an entry on the
%    transcript.
%<*dtx>
\ProvidesFile{swedish.dtx}
%</dtx>
%<code>\ProvidesLanguage{swedish}
%\fi
%\ProvidesFile{swedish.dtx}
        [2005/03/31 v2.3d Swedish support from the babel system]
%\iffalse
%% File `swedish.dtx'
%% Babel package for LaTeX version 2e
%% Copyright (C) 1989 - 2005
%%           by Johannes Braams, TeXniek
%
%% Please report errors to: J.L. Braams
%%                          babel at braams.cistron.nl
%
%    This file is part of the babel system, it provides the source
%    code for the Swedish language definition file.  A contribution
%    was made by Sten Hellman HELLMAN@CERNVM.CERN.CH
%
%    Further enhancements for version 2.0 were provided by 
%    Erik "Osthols <erik\_osthols@yahoo.com>
%<*filedriver>
\documentclass{ltxdoc}
\newcommand*\TeXhax{\TeX hax}
\newcommand*\babel{\textsf{babel}}
\newcommand*\langvar{$\langle \it lang \rangle$}
\newcommand*\note[1]{}
\newcommand*\Lopt[1]{\textsf{#1}}
\newcommand*\file[1]{\texttt{#1}}
\begin{document}
 \DocInput{swedish.dtx}
\end{document}
%</filedriver>
%\fi
% \GetFileInfo{swedish.dtx}
%
% \changes{swedish-1.0a}{1991/07/15}{Renamed \file{babel.sty} in
%    \file{babel.com}}
% \changes{swedish-1.1}{1992/02/16}{Brought up-to-date with babel 3.2a}
% \changes{swedish-1.2}{1994/02/27}{Update for LaTeX2e}
% \changes{swedish-1.3d}{1994/06/26}{Removed the use of \cs{filedate}
%    and moved identification after the loading of \file{babel.def}}
% \changes{swedish-1.3e}{1995/05/28}{Update for release 3.5}
% \changes{swedish-2.0}{1996/01/24}{Introduced active double quote}
% \changes{swedish-2.1}{1996/10/10}{Replaced \cs{undefined} with
%    \cs{@undefined} and \cs{empty} with \cs{@empty} for consistency
%    with \LaTeX, moved the definition of \cs{atcatcode} right to the
%    beginning.}
% \changes{swedish-2.3d}{2001/11/16}{Fixed a \cs{changes} entry}
%
%  \section{The Swedish language}
%
%    The file \file{\filename}\footnote{The file described in this
%    section has version number \fileversion\ and was last revised on
%    \filedate. Contributions were made by Sten Hellman
%    (\texttt{HELLMAN@CERNVM.CERN.CH}) and Erik \"Osthols
%    (\texttt{erik\_osthols@yahoo.com}).} defines all the
%    language-specific macros for the Swedish language. This
%    file has borrowed heavily from |finnish.dtx| and |germanb.dtx|.
%
%    For this language the character |"| is made active. In
%    table~\ref{tab:swedish-quote} an overview is given of its
%    purpose. The vertical placement of the "umlaut" in some letters
%    can be controlled this way.
%    \begin{table}[htb]
%     \begin{center}
%     \begin{tabular}{lp{8cm}}
%      |"a| & Gives \"a, also implemented for |"A|, |"o| and |"O|. \\
%      |"w|, |"W| & gives {\aa} and {\AA}.                         \\
%      |"ff| & for |ff| to be hyphenated as |ff-f|. Used for compound
%              words, such as \texttt{stra|"|ff{\aa}nge}, which
%              should be hyphenated as \texttt{straff-f{\aa}nge}.
%              This is also implemented for b, d, f, g, l, m, n,
%              p, r, s, and t.                                     \\
%      \verb="|= & disable ligature at this position. This should be 
%              used for compound words, such as
%              ``\texttt{stra|"|ffinr\"attning}'',
%              which should not have the ligature ``ffi''.         \\
%      |"-| & an explicit hyphen sign, allowing hyphenation
%             in the rest of the word, such as e. g. in
%             ``x|"-|axeln''.                                      \\
%      |""| & like |"-|, but producing no hyphen sign
%             (for words that should break at some sign such as
%             \texttt{och/|""|eller}).                             \\
%      |"~| & for an explicit hyphen without a breakpoint; useful for
%             expressions such as ``2|"~|3 veckor'' where no linebreak
%             is desirable.                                        \\
%      |"=| & an explicit hyphen sign allowing subsequent hyphenation,
%             for expressions such as ``studiebidrag och \newline
%             -l{\aa}n''.                                          \\
%      |\-| & like the old |\-|, but allowing hyphenation
%             in the rest of the word.                             \\
%     \end{tabular}
%     \caption{The extra definitions made
%              by \file{swedish.sty}}\label{tab:swedish-quote}
%     \end{center}
%    \end{table}
%
%    Two variations for formatting of dates are added. \cs{datesymd}
%    makes \cs{today} output dates formatted as YYYY-MM-DD, which is
%    commonly used in Sweden today. \cs{datesdmy} formats the date as
%    D/M YYYY, which is also very common in Sweden. These commands
%    should be issued after \cs{begin{document}}.
%
% \StopEventually{}
%
%    The macro |\LdfInit| takes care of preventing that this file is
%    loaded more than once, checking the category code of the
%    \texttt{@} sign, etc.
% \changes{swedish-2.1}{1996/11/03}{Now use \cs{LdfInit} to perform
%    initial checks} 
%    \begin{macrocode}
%<*code>
\LdfInit{swedish}\captionsswedish
%    \end{macrocode}
%
%    When this file is read as an option, i.e. by the |\usepackage|
%    command, \texttt{swedish} will be an `unknown' language in which
%    case we have to make it known. So we check for the existence of
%    |\l@swedish| to see whether we have to do something here.
%
% \changes{swedish-1.0c}{1991/10/29}{Removed use of \cs{@ifundefined}}
% \changes{swedish-1.1}{1992/02/16}{Added a warning when no hyphenation
%    patterns were loaded.}
% \changes{swedish-1.3d}{1994/06/26}{Now use \cs{@nopatterns} to
%    producew the warning}
%    \begin{macrocode}
\ifx\l@swedish\@undefined
    \@nopatterns{Swedish}
    \adddialect\l@swedish0\fi
%    \end{macrocode}
%
%    The next step consists of defining commands to switch to the
%    Swedish language. The reason for this is that a user might want
%    to switch back and forth between languages.
%
% \begin{macro}{\captionsswedish}
%    The macro |\captionsswedish| defines all strings used in the four
%    standard documentclasses provided with \LaTeX.
% \changes{swedish-1.0b}{1991/08/21}{removed type in definition of
%    \cs{contentsname}}
% \changes{swedish-1.0b}{1991/08/21}{added definition for \cs{enclname}}
% \changes{swedish-1.0b}{1991/08/21}{made definition of \cs{refname}
%    pluralis}
% \changes{swedish-1.1}{1992/02/16}{Added \cs{seename}, \cs{alsoname}
%    and \cs{prefacename}}
% \changes{swedish-1.1}{1993/07/15}{\cs{headpagename} should be
%    \cs{pagename}}
% \changes{swedish-1.1b}{1993/09/16}{Added translations}
% \changes{swedish-1.3d}{1994/07/27}{Changed \cs{aa} to
%    \cs{csname}\texttt{ aa}\cs{endcsname}, to make \cs{uppercase} do
%    the right thing}
% \changes{swedish-1.3f}{1995/07/04}{Added \cs{proofname} for
%    AMS-\LaTeX}
%    \begin{macrocode}
\addto\captionsswedish{%
  \def\prefacename{F\"orord}%
  \def\refname{Referenser}%
  \def\abstractname{Sammanfattning}%
  \def\bibname{Litteraturf\"orteckning}%
  \def\chaptername{Kapitel}%
  \def\appendixname{Bilaga}%
  \def\contentsname{Inneh\csname aa\endcsname ll}%
  \def\listfigurename{Figurer}%
  \def\listtablename{Tabeller}%
  \def\indexname{Sakregister}%
  \def\figurename{Figur}%
  \def\tablename{Tabell}%
  \def\partname{Del}%
%    \end{macrocode}
% \changes{swedish-2.3a}{2000/01/19}{Added full stop after ``Bil''}
%    \begin{macrocode}
  \def\enclname{Bil.}%
  \def\ccname{Kopia f\"or k\"annedom}%
  \def\headtoname{Till}% in letter
  \def\pagename{Sida}%
  \def\seename{se}%
%    \end{macrocode}
% \changes{swedish-1.3e}{1995/05/29}{Changed \cs{alsoname} from
%    `\texttt{se ocks\aa}'}
%    \begin{macrocode}
  \def\alsoname{se \"aven}%
%    \end{macrocode}
% \changes{swedish-1.3g}{1995/11/03}{Replaced `Proof' by its
%    translation} 
% \changes{swedish-2.3b}{2000/09/26}{Added \cs{glossaryname}}
% \changes{swedish-2.3c}{2001/03/12}{Provided translation for
%    Glossary}
%    \begin{macrocode}
  \def\proofname{Bevis}%
  \def\glossaryname{Ordlista}%
  }%
%    \end{macrocode}
% \end{macro}
%
% \begin{macro}{\dateswedish}
%    The macro |\dateswedish| redefines the command |\today| to
%    produce Swedish dates.
% \changes{swedish-2.2}{1997/10/01}{Use \cs{edef} to define
%    \cs{today} to save memory}
% \changes{swedish-2.2}{1998/03/28}{use \cs{def} instead of
%    \cs{edef}}
%    \begin{macrocode}
\def\dateswedish{%
  \def\today{%
    \number\day~\ifcase\month\or
    januari\or februari\or mars\or april\or maj\or juni\or
    juli\or augusti\or september\or oktober\or november\or
    december\fi
    \space\number\year}}
%    \end{macrocode}
% \end{macro}
%
%  \begin{macro}{\datesymd}
% \changes{swedish-2.3a}{2000/01/20}{Command added}
%    The macro |\datesymd| redefines the command |\today| to
%    produce dates in the format YYYY-MM-DD, common in Sweden.
%    \begin{macrocode}
\def\datesymd{%
  \def\today{\number\year-\two@digits\month-\two@digits\day}%
  }
%    \end{macrocode}
%  \end{macro}
%
%  \begin{macro}{\datesdmy}
% \changes{swedish-2.3a}{2000/01/20}{Command added}
%    The macro |\datesdmy| redefines the command |\today| to
%    produce Swedish dates in the format DD/MM YYYY, also common in
%    Sweden.
%    \begin{macrocode}
\def\datesdmy{%
  \def\today{\number\day/\number\month\space\number\year}%
  }
%    \end{macrocode}
%  \end{macro}
%
%  \begin{macro}{\swedishhyphenmins}
%    The swedish hyphenation patterns can be used with |\lefthyphenmin|
%    set to~2 and |\righthyphenmin| set to~2.
% \changes{swedish-1.3e}{1995/06/02}{use \cs{swedishhyphenmins} to
%    store the correct values}
%    \begin{macrocode}
\providehyphenmins{swedish}{\tw@\tw@}
%    \end{macrocode}
%  \end{macro}
%
% \begin{macro}{\extrasswedish}
% \changes{swedish-1.3e}{1995/05/29}{Added \cs{bbl@frenchspacing}}
% \begin{macro}{\noextrasswedish}
% \changes{swedish-1.3e}{1995/05/29}{Added \cs{bbl@nonfrenchspacing}}
%    The macro |\extrasswedish| performs all the extra definitions
%    needed for the Swedish language. The macro |\noextrasswedish| is
%    used to cancel the actions of |\extrasswedish|.
%
%    For Swedish texts |\frenchspacing| should be in effect.  We make
%    sure this is the case and reset it if necessary.
%
%    \begin{macrocode}
\addto\extrasswedish{\bbl@frenchspacing}
\addto\noextrasswedish{\bbl@nonfrenchspacing}
%    \end{macrocode}
%
%    For Swedish the \texttt{"} character is made active. This is done
%    once, later on its definition may vary.
% \changes{swedish-2.0}{1996/01/24}{Added active double quote
%    character} 
%
%    \begin{macrocode}
\initiate@active@char{"}
\addto\extrasswedish{\languageshorthands{swedish}}
\addto\extrasswedish{\bbl@activate{"}}
%    \end{macrocode}
%    Don't forget to turn the shorthands off again.
% \changes{swedish-2.2b}{1999/12/17}{Deactivate shorthands ouside of
%    Swedish}
%    \begin{macrocode}
\addto\noextrasswedish{\bbl@deactivate{"}}
%    \end{macrocode}
%    The ``umlaut'' accent macro |\"| is changed to lower the
%    ``umlaut'' dots. The redefinition is done with the help of
%    |\umlautlow|.
%^^A    PROBLEM:
%^^A    What happens to ``umlaut''
%^^A    characters entered using e. g. latin input encoding? They should
%^^A    be lowered as well.
%    \begin{macrocode}
\addto\extrasswedish{\babel@save\"\umlautlow}
\addto\noextrasswedish{\umlauthigh}
%    \end{macrocode}
% \end{macro}
% \end{macro}
%
%    The code above is necessary because we need an extra active
%    character. This character is then used as indicated in
%    table~\ref{tab:swedish-quote}.
%
%    To be able to define the function of |"|, we first define a
%    couple of `support' macros.
%
%  \begin{macro}{\dq}
%    We save the original double quote character in |\dq| to keep
%    it available, the math accent |\"| can now be typed as |"|.
%    \begin{macrocode}
\begingroup \catcode`\"12
\def\x{\endgroup
  \def\@SS{\mathchar"7019 }
  \def\dq{"}}
\x
%    \end{macrocode}
%  \end{macro}
%
%    Now we can define the doublequote macros: the umlauts and {\aa}.
% \changes{swedish-2.3a}{2000/01/20}{added \cs{allowhyphens}}
%    \begin{macrocode}
\declare@shorthand{swedish}{"w}{\textormath{{\aa}\allowhyphens}{\ddot w}}
\declare@shorthand{swedish}{"a}{\textormath{\"{a}\allowhyphens}{\ddot a}}
\declare@shorthand{swedish}{"o}{\textormath{\"{o}\allowhyphens}{\ddot o}}
\declare@shorthand{swedish}{"W}{\textormath{{\AA}\allowhyphens}{\ddot W}}
\declare@shorthand{swedish}{"A}{\textormath{\"{A}\allowhyphens}{\ddot A}}
\declare@shorthand{swedish}{"O}{\textormath{\"{O}\allowhyphens}{\ddot O}}
%    \end{macrocode}
%    discretionary commands
%^^A PROBLEM:
%^^A Will these allow subsequent hyphenation? Perhaps we need to
%^^A add \cs{allowhyphens} here as well?
%    \begin{macrocode}
\declare@shorthand{swedish}{"b}{\textormath{\bbl@disc b{bb}}{b}}
\declare@shorthand{swedish}{"B}{\textormath{\bbl@disc B{BB}}{B}}
\declare@shorthand{swedish}{"d}{\textormath{\bbl@disc d{dd}}{d}}
\declare@shorthand{swedish}{"D}{\textormath{\bbl@disc D{DD}}{D}}
\declare@shorthand{swedish}{"f}{\textormath{\bbl@disc f{ff}}{f}}
\declare@shorthand{swedish}{"F}{\textormath{\bbl@disc F{FF}}{F}}
\declare@shorthand{swedish}{"g}{\textormath{\bbl@disc g{gg}}{g}}
\declare@shorthand{swedish}{"G}{\textormath{\bbl@disc G{GG}}{G}}
\declare@shorthand{swedish}{"l}{\textormath{\bbl@disc l{ll}}{l}}
\declare@shorthand{swedish}{"L}{\textormath{\bbl@disc L{LL}}{L}}
\declare@shorthand{swedish}{"m}{\textormath{\bbl@disc m{mm}}{m}}
\declare@shorthand{swedish}{"M}{\textormath{\bbl@disc M{MM}}{M}}
\declare@shorthand{swedish}{"n}{\textormath{\bbl@disc n{nn}}{n}}
\declare@shorthand{swedish}{"N}{\textormath{\bbl@disc N{NN}}{N}}
\declare@shorthand{swedish}{"p}{\textormath{\bbl@disc p{pp}}{p}}
\declare@shorthand{swedish}{"P}{\textormath{\bbl@disc P{PP}}{P}}
\declare@shorthand{swedish}{"r}{\textormath{\bbl@disc r{rr}}{r}}
\declare@shorthand{swedish}{"R}{\textormath{\bbl@disc R{RR}}{R}}
\declare@shorthand{swedish}{"s}{\textormath{\bbl@disc s{ss}}{s}}
\declare@shorthand{swedish}{"S}{\textormath{\bbl@disc S{SS}}{S}}
\declare@shorthand{swedish}{"t}{\textormath{\bbl@disc t{tt}}{t}}
\declare@shorthand{swedish}{"T}{\textormath{\bbl@disc T{TT}}{T}}
%    \end{macrocode}
%    and some additional commands:
% \def\indexeqs{=}
% \changes{swedish-2.3a}{2000/01/28}{changed definition of
%    \texttt{"\protect\indexeqs}, \cs{-} and \texttt{"~}}
%    \begin{macrocode}
\declare@shorthand{swedish}{"-}{\nobreak-\bbl@allowhyphens}
\declare@shorthand{swedish}{"|}{%
  \textormath{\nobreak\discretionary{-}{}{\kern.03em}%
              \bbl@allowhyphens}{}}
\declare@shorthand{swedish}{""}{\hskip\z@skip}
\declare@shorthand{swedish}{"~}{%
  \textormath{\leavevmode\hbox{-}\bbl@allowhyphens}{-}}
\declare@shorthand{swedish}{"=}{\hbox{-}\allowhyphens}
%    \end{macrocode}
%
%  \begin{macro}{\-}
%    Redefinition of |\-|. The new version of |\-| should indicate an
%    extra hyphenation position, while allowing other hyphenation
%    positions to be generated  automatically. The standard behaviour
%    of \TeX\ in this respect is very unfortunate for languages such
%    as Dutch, Finnish, German and Swedish, where long compound words
%    are quite normal and all one needs is a means to indicate an
%    extra hyphenation position on top of the ones that \TeX\ can
%    generate from the hyphenation patterns.
%    \begin{macrocode}
\addto\extrasswedish{\babel@save\-}
\addto\extrasswedish{\def\-{\allowhyphens
                          \discretionary{-}{}{}\allowhyphens}}
%    \end{macrocode}
%  \end{macro}
%
%    The macro |\ldf@finish| takes care of looking for a
%    configuration file, setting the main language to be switched on
%    at |\begin{document}| and resetting the category code of
%    \texttt{@} to its original value.
% \changes{swedish-2.1}{1996/11/03}{Now use \cs{ldf@finish} to wrap up}
%    \begin{macrocode}
\ldf@finish{swedish}
%</code>
%    \end{macrocode}
%
%\Finale
%%
%% \CharacterTable
%%  {Upper-case    \A\B\C\D\E\F\G\H\I\J\K\L\M\N\O\P\Q\R\S\T\U\V\W\X\Y\Z
%%   Lower-case    \a\b\c\d\e\f\g\h\i\j\k\l\m\n\o\p\q\r\s\t\u\v\w\x\y\z
%%   Digits        \0\1\2\3\4\5\6\7\8\9
%%   Exclamation   \!     Double quote  \"     Hash (number) \#
%%   Dollar        \$     Percent       \%     Ampersand     \&
%%   Acute accent  \'     Left paren    \(     Right paren   \)
%%   Asterisk      \*     Plus          \+     Comma         \,
%%   Minus         \-     Point         \.     Solidus       \/
%%   Colon         \:     Semicolon     \;     Less than     \<
%%   Equals        \=     Greater than  \>     Question mark \?
%%   Commercial at \@     Left bracket  \[     Backslash     \\
%%   Right bracket \]     Circumflex    \^     Underscore    \_
%%   Grave accent  \`     Left brace    \{     Vertical bar  \|
%%   Right brace   \}     Tilde         \~}
%%
\endinput
}
%^^A\DeclareOption{tamil}{\input{tamil.ldf}}
\DeclareOption{turkish}{% \iffalse meta-commen

% Copyright 1989-2005 Johannes L. Braams and any individual author
% listed elsewhere in this file.  All rights reserved
%
% This file is part of the Babel system
% -------------------------------------
%
% It may be distributed and/or modified under th
% conditions of the LaTeX Project Public License, either version 1.
% of this license or (at your option) any later version
% The latest version of this license is i
%   http://www.latex-project.org/lppl.tx
% and version 1.3 or later is part of all distributions of LaTe
% version 2003/12/01 or later
%
% This work has the LPPL maintenance status "maintained"
%
% The Current Maintainer of this work is Johannes Braams
%
% The list of all files belonging to the Babel system i
% given in the file `manifest.bbl. See also `legal.bbl' for additiona
% information
%
% The list of derived (unpacked) files belonging to the distributio
% and covered by LPPL is defined by the unpacking scripts (wit
% extension .ins) which are part of the distribution
% \f
% \CheckSum{179
% \iffals
%    Tell the \LaTeX\ system who we are and write an entry on th
%    transcript
%<*dtx
\ProvidesFile{turkish.dtx
%</dtx
%<code>\ProvidesLanguage{turkish
%\f
%\ProvidesFile{turkish.dtx
        [2005/03/31 v1.2m Turkish support from the babel system
%\iffals
%% File `turkish.dtx
%% Babel package for LaTeX version 2
%% Copyright (C) 1989 - 200
%%           by Johannes Braams, TeXnie

%% Please report errors to: J.L. Braam
%%                          babel at braams.cistron.n

%% Turkish Language Definition Fil
%% Copyright (C) 1994 - 200
%%           by Mustafa Bur
%%           rz6001@rziris01.rrz.uni-hamburg.d
%%          (40) 250347

%    This file is part of the babel system, it provides the sourc
%    code for the Turkish language definition file
%<*filedriver
\documentclass{ltxdoc
\newcommand*\TeXhax{\TeX hax
\newcommand*\babel{\textsf{babel}
\newcommand*\langvar{$\langle \it lang \rangle$
\newcommand*\note[1]{
\newcommand*\Lopt[1]{\textsf{#1}
\newcommand*\file[1]{\texttt{#1}
\begin{document
 \DocInput{turkish.dtx
\end{document
%</filedriver
%\f
% \GetFileInfo{turkish.dtx

% \changes{turkish-1.2}{1994/02/27}{Update for \LaTeXe
% \changes{turkish-1.2c}{1994/06/26}{Removed the use of \cs{filedate
%    and moved identification after the loading of \file{babel.def}
% \changes{turkish-1.2h}{1996/07/13}{Replaced \cs{undefined} wit
%    \cs{@undefined} and \cs{empty} with \cs{@empty} for consistenc
%    with \LaTeX
% \changes{turkish-1.2i}{1996/10/10}{Moved the definition o
%    \cs{atcatcode} right to the beginning.

%  \section{The Turkish language

%    The file \file{\filename}\footnote{The file described in thi
%    section has version number \fileversion\ and was last revised o
%    \filedate.}  defines all the language definition macros for th
%    Turkish language\footnote{Mustafa Burc
%    \texttt{z6001@rziris01.rrz.uni-hamburg.de} provided the code fo
%    this file. It is based on the work by Pierre Mackay; Turgut Uyar
%    \texttt{uyar@cs.itu.edu.tr} supplied additional translations i
%    version 1.2j and later}

%    Turkish typographic rules specify that a little `white space
%    should be added before the characters `\texttt{:}', `\texttt{!}
%    and `\texttt{=}'. In order to insert this white spac
%    automatically these characters are made `active'. Als
%    |\frenhspacing| is set

% \StopEventually{

%    The macro |\LdfInit| takes care of preventing that this file i
%    loaded more than once, checking the category code of th
%    \texttt{@} sign, etc
% \changes{turkish-1.2i}{1996/11/03}{ow use \cs{LdfInit} to perfor
%    initial checks}
%    \begin{macrocode
%<*code
\LdfInit{turkish}\captionsturkis
%    \end{macrocode

%    When this file is read as an option, i.e. by the |\usepackage
%    command, \texttt{turkish} could be an `unknown' language in whic
%    case we have to make it known. So we check for the existence o
%    |\l@turkish| to see whether we have to do something here

% \changes{turkish-1.2c}{1994/06/26}{Now use \cs{@nopatterns} t
%    produce the warning
%    \begin{macrocode
\ifx\l@turkish\@undefine
  \@nopatterns{Turkish
  \adddialect\l@turkish0\f
%    \end{macrocode

%    The next step consists of defining commands to switch to (an
%    from) the Turkish language

% \begin{macro}{\captionsturkish
%    The macro |\captionsturkish| defines all strings used in the fou
%    standard documentclasses provided with \LaTeX
% \changes{turkish-1.1}{1993/07/15}{\cs{headpagename} should b
%    \cs{pagename}
% \changes{turkish-1.2b}{1994/06/04}{Added braces behind \cs{i} i
%    strings
% \changes{turkish-1.2f}{1995/07/04}{Added \cs{proofname} fo
%    AMS-\LaTeX}
% \changes{turkish-1.2j}{1997/09/29}{Added and modified translations
% \changes{turkish-1.2k}{1999/04/18}{Incorporated some mor
%    corrections}
% \changes{turkish-1.2m}{2000/09/20}{Added \cs{glossaryname}
%    \begin{macrocode
\addto\captionsturkish{
  \def\prefacename{\"Ons\"oz}
  \def\refname{Kaynaklar}
  \def\abstractname{\"Ozet}
  \def\bibname{Kaynak\c ca}
  \def\chaptername{B\"ol\"um}
  \def\appendixname{Ek}
  \def\contentsname{\.I\c cindekiler}
  \def\listfigurename{\c Sekil Listesi}
  \def\listtablename{Tablo Listesi}
  \def\indexname{Dizin}
  \def\figurename{\c Sekil}
  \def\tablename{Tablo}
  \def\partname{K\i s\i m}
  \def\enclname{\.Ili\c sik}
  \def\ccname{Di\u ger Al\i c\i lar}
  \def\headtoname{Al\i c\i}
  \def\pagename{Sayfa}
  \def\subjectname{\.Ilgili}
  \def\seename{bkz.}
  \def\alsoname{ayr\i ca bkz.}
  \def\proofname{Kan\i t}
  \def\glossaryname{Glossary}% <-- Needs translatio
}
%    \end{macrocode
% \end{macro

% \begin{macro}{\dateturkish
%    The macro |\dateturkish| redefines the command |\today| t
%    produce Turkish dates
% \changes{turkish-1.2b}{1994/06/04}{Added braces behind \cs{i} i
%    strings
% \changes{turkish-1.2d}{1995/01/31}{removed extra closing brace
%    \cs{mont} should be \cs{month}
% \changes{turkish-1.2j}{1997/10/01}{Use \cs{edef} to defin
%    \cs{today} to save memory
% \changes{turkish-1.2j}{1998/03/28}{use \cs{def} instead o
%    \cs{edef}
% \changes{turkish-1.2l}{1999/05/17}{removed dot from the date format
%    \begin{macrocode
\def\dateturkish{
  \def\today{\number\day~\ifcase\month\o
    Ocak\or \c Subat\or Mart\or Nisan\or May\i{}s\or Haziran\o
    Temmuz\or A\u gustos\or Eyl\"ul\or Ekim\or Kas\i{}m\o
    Aral\i{}k\f
    \space\number\year}
%    \end{macrocode
% \end{macro

% \begin{macro}{\extrasturkish
% \changes{turkish-1.2e}{1995/05/15}{Completely rewrote macro
% \begin{macro}{\noextrasturkish
%    The macro |\extrasturkish| will perform all the extra definition
%    needed for the Turkish language. The macro |\noextrasturkish| i
%    used to cancel the actions of |\extrasturkish|

%    Turkish typographic rules specify that a little `white space
%    should be added before the characters `\texttt{:}', `\texttt{!}
%    and `\texttt{=}'. In order to insert this white spac
%    automatically these characters are made |\active|, so they hav
%    to be treated in a special way
%    \begin{macrocode
\initiate@active@char{:
\initiate@active@char{!
\initiate@active@char{=
%    \end{macrocode
%    We specify that the turkish group of shorthands should be used
%    \begin{macrocode
\addto\extrasturkish{\languageshorthands{turkish}
%    \end{macrocode
%    These characters are `turned on' once, later their definition ma
%    vary.
%    \begin{macrocode
\addto\extrasturkish{
  \bbl@activate{:}\bbl@activate{!}\bbl@activate{=}
%    \end{macrocode

%    For Turkish texts |\frenchspacing| should be in effect. W
%    make sure this is the case and reset it if necessary
% \changes{turkish-1.2e}{1995/05/15}{now use \cs{bbl@frenchspacing
%    and \cs{bbl@nonfrenchspacing}
%    \begin{macrocode
\addto\extrasturkish{\bbl@frenchspacing
\addto\noextrasturkish{\bbl@nonfrenchspacing
%    \end{macrocode
% \end{macro
% \end{macro

% \begin{macro}{\turkish@sh@!@
% \begin{macro}{\turkish@sh@=@
% \begin{macro}{\turkish@sh@:@
%    The definitions for the three active characters were made usin
%    intermediate macros. These are defined now. The insertion o
%    extra `white space' should only happen outside math mode, henc
%    the check |\ifmmode| in the macros
% \changes{turkish-1.2d}{1995/01/31}{Added missing \cs{def}
% \changes{turkish-1.2e}{1995/05/15}{Use the new mechanism o
%    \cs{declare@shorthand}
%%^^A JLB:: this code breaks during \begin{document
%%^^A JLB: Needs fixing!
%    \begin{macrocode
\declare@shorthand{turkish}{:}{
  \ifmmod
    \string:
  \else\rela
    \ifhmod
      \ifdim\lastskip>\z
        \unskip\penalty\@M\thinspac
      \f
    \f
    \string:
  \fi
\declare@shorthand{turkish}{!}{
  \ifmmod
    \string!
  \else\rela
    \ifhmod
      \ifdim\lastskip>\z
        \unskip\penalty\@M\thinspac
      \f
    \f
    \string!
  \fi
\declare@shorthand{turkish}{=}{
  \ifmmod
    \string=
  \else\rela
    \ifhmod
      \ifdim\lastskip>\z
        \unskip\kern\fontdimen2\fon
        \kern-1.4\fontdimen3\fon
      \f
    \f
    \string=
  \fi
%    \end{macrocode
% \end{macro
% \end{macro
% \end{macro

%    The macro |\ldf@finish| takes care of looking for
%    configuration file, setting the main language to be switched o
%    at |\begin{document}| and resetting the category code o
%    \texttt{@} to its original value
% \changes{turkish-1.2i}{1996/11/03}{Now use \cs{ldf@finish} to wra
%    up}
%    \begin{macrocode
\ldf@finish{turkish
%</code
%    \end{macrocode

% \Final
%
%% \CharacterTabl
%%  {Upper-case    \A\B\C\D\E\F\G\H\I\J\K\L\M\N\O\P\Q\R\S\T\U\V\W\X\Y\
%%   Lower-case    \a\b\c\d\e\f\g\h\i\j\k\l\m\n\o\p\q\r\s\t\u\v\w\x\y\
%%   Digits        \0\1\2\3\4\5\6\7\8\
%%   Exclamation   \!     Double quote  \"     Hash (number) \
%%   Dollar        \$     Percent       \%     Ampersand     \
%%   Acute accent  \'     Left paren    \(     Right paren   \
%%   Asterisk      \*     Plus          \+     Comma         \
%%   Minus         \-     Point         \.     Solidus       \
%%   Colon         \:     Semicolon     \;     Less than     \
%%   Equals        \=     Greater than  \>     Question mark \
%%   Commercial at \@     Left bracket  \[     Backslash     \
%%   Right bracket \]     Circumflex    \^     Underscore    \
%%   Grave accent  \`     Left brace    \{     Vertical bar  \
%%   Right brace   \}     Tilde         \~
%
\endinpu
}
\DeclareOption{ukrainian}{% \iffalse meta-commen

% Copyright 1989-2008 Johannes L. Braams and any individual author
% listed elsewhere in this file.  All rights reserved
%
% This file is part of the Babel system
% -------------------------------------
%
% It may be distributed and/or modified under th
% conditions of the LaTeX Project Public License, either version 1.
% of this license or (at your option) any later version
% The latest version of this license is i
%   http://www.latex-project.org/lppl.tx
% and version 1.3 or later is part of all distributions of LaTe
% version 2003/12/01 or later
%
% This work has the LPPL maintenance status "maintained"
%
% The Current Maintainer of this work is Johannes Braams
%
% The list of all files belonging to the Babel system i
% given in the file `manifest.bbl. See also `legal.bbl' for additiona
% information
%
% The list of derived (unpacked) files belonging to the distributio
% and covered by LPPL is defined by the unpacking scripts (wit
% extension .ins) which are part of the distribution
% \f
% \CheckSum{1472

% \iffals
%    Tell the \LaTeX\ system who we are and write an entry on th
%    transcript
%<*dtx
\ProvidesFile{ukraineb.dtx
%</dtx
%<code>\ProvidesLanguage{ukraineb
        [2008/03/21 v1.1l Ukrainian support from the babel system

%% File `ukraineb.dtx
%% Babel package for LaTeX version 2
%% Copyright (C) 1989 - 200
%%           by Johannes Braams, TeXnie

%% ukraineb Language Definition Fil
%% Copyright (C) 1997 - 200
%%           by Andrij Shvaika ashv at icmp.lviv.u

%% derived from the Russianb Language Definition Fil
%% Copyright (C) 1995 - 200
%%           by Olga Lapko cyrtug at mir.msk.s
%%              Johannes Braams, TeXnie
% adapted to the new T2 and X2 Cyrillic encoding
%           by Vladimir Volovich TeX at vvv.vsu.r
%              Werner Lemberg wl at gnu.or

%% Please report errors to: J.L. Braam
%%                          babel at braams.xs4all.n

%<*filedriver
\documentclass{ltxdoc
\newcommand\TeXhax{\TeX hax
\newcommand\babel{\textsf{babel}
\newcommand\langvar{$\langle \it lang \rangle$
\newcommand\note[1]{
\newcommand\Lopt[1]{\textsf{#1}
\newcommand\file[1]{\texttt{#1}
\newcommand\pkg[1]{\texttt{#1}
\begin{document
 \DocInput{ukraineb.dtx
\end{document
%</filedriver
%\f
% \GetFileInfo{ukraineb.dtx

% \changes{ukraineb-1.1e}{1999/08/19}{replaced all \cs{penalty}\cs{@M
%    with \cs{nobreak}

%  \section{The Ukrainian language

%    The file \file{\filename}\footnote{The file described in thi
%    section has version number \fileversion
%    This file was derived from the \file{russianb.dtx} version 1.1g.
%    defines all the language-specific macros for the Ukrainia
%    language. It needs the file \file{cyrcod} for success documentatio
%    with Ukrainian encodings (see below)

%    For this language the character |"| is made active. I
%    table~\ref{tab:ukrainian-quote} an overview is given of it
%    purpose.

%    \begin{table}[htb
%      \begin{center
%      \begin{tabular}{lp{8cm}
%       \verb="|= & disable ligature at this position.               \
%       |"-| & an explicit hyphen sign, allowing hyphenatio
%                   in the rest of the word.                         \
%       |"---| & Cyrillic emdash in plain text.                      \
%       |"--~| & Cyrillic emdash in compound names (surnames).       \
%       |"--*| & Cyrillic emdash for denoting direct speech.         \
%       |""| & like |"-|, but producing no hyphen sig
%                   (for compund words with hyphen, e.g.\ |x-""y
%                   or some other signs  as ``disable/enable'').     \
%       |"~| & for a compound word mark without a breakpoint.        \
%       |"=| & for a compound word mark with a breakpoint, allowin
%              hyphenation in the composing words.                   \
%       |",| & thinspace for initials with a breakpoin
%               in following surname.                                \
%       |"`| & for German left double quote
%                   (looks like ,\kern-0.08em,).                     \
%       |"'| & for German right double quotes (looks like ``).       \\%^^A'
%       |"<| & for French left double quotes (looks like $<\!\!<$).  \
%       |">| & for French right double quotes (looks like $>\!\!>$). \
%      \end{tabular
%      \caption{The extra definitions mad
%               by \file{ukraineb}}\label{tab:ukrainian-quote
%      \end{center
%    \end{table

%    The quotes in table~\ref{tab:ukrainian-quote} (see, als
%    table~\ref{tab:russian-quote}) can also be typeset by using the command
%    in table~\ref{tab:umore-quote} (see, also table~\ref{tab:rmore-quote})

%    \begin{table}[htb
%      \begin{center
%      \begin{tabular}{lp{8cm}
%       |\cdash---| & Cyrillic emdash in plain text.                    \
%       |\cdash--~| & Cyrillic emdash in compound names (surnames).     \
%       |\cdash--*| & Cyrillic emdash for denoting direct speech.       \
%       |\glqq| & for German left double quote
%                    (looks like ,\kern-0.08em,).                       \
%       |\grqq| & for German right double quotes (looks like ``).       \\%^^A'
%       |\flqq| & for French left double quotes (looks like $<\!\!<$).  \
%       |\frqq| & for French right double quotes (looks like $>\!\!>$). \
%       |\dq|   & the original quotes character (|"|).                  \
%      \end{tabular
%      \caption{More commands which produce quotes, define
%               by \babel}\label{tab:umore-quote
%      \end{center
%    \end{table

%    The French quotes are also available as ligatures `|<<|' and `|>>|' i
%    8-bit Cyrillic font encodings (\texttt{LCY}, \texttt{X2}, \texttt{T2*}
%    and as `|<|' and `|>|' characters in 7-bit Cyrillic font encoding
%    (\texttt{OT2} and \texttt{LWN})

%    The quotation marks traditionally used in Ukrainian and Russia
%    languages were borrowed from other languages (e.g. French and German
%    so they keep their original names

% \StopEventually{

%    The macro |\LdfInit| takes care of preventing that this file is loade
%    more than once, checking the category code of the \texttt{@} sign, etc

%    \begin{macrocode
%<*code
\LdfInit{ukrainian}{captionsukrainian
%    \end{macrocode

%    When this file is read as an option, i.e., by the |\usepackage
%    command, \texttt{ukraineb} will be an `unknown' language, in which cas
%    we have to make it known. So we check for the existence of |\l@ukrainian
%    to see whether we have to do something here

%    \begin{macrocode
\ifx\l@ukrainian\@undefine
  \@nopatterns{Ukrainian
  \adddialect\l@ukrainian
\f
%    \end{macrocode

%  \begin{macro}{\latinencoding

%    We need to know the encoding for text that is supposed to be which i
%    active at the end of the \babel\ package. If the \pkg{fontenc} packag
%    is loaded later, then\ldots too bad

%    \begin{macrocode
\let\latinencoding\cf@encodin
%    \end{macrocode

%  \end{macro

%    The user may choose between different available Cyrilli
%    encodings---e.g., \texttt{X2}, \texttt{LCY}, or \texttt{LWN}.\
%    Hopefully, \texttt{X2} will eventually replace the two latter encoding
%    (\texttt{LCY} and \texttt{LWN}).\@ If the user wants to use anothe
%    font encoding than the default (\texttt{T2A}), he has to load th
%    corresponding file \emph{before} \file{ukraineb.sty}. This may be don
%    in the following way

%    \begin{verbatim
%      % override the default X2 encoding used in Babe
%      \usepackage[LCY,OT1]{fontenc
%      \usepackage[english,ukrainian]{babel
%    \end{verbatim
%    \unski

%    Note: for the Ukrainian language, the \texttt{T2A} encoding is better tha
%    \texttt{X2}, because \texttt{X2} does not contain Latin letters, an
%    users should be very careful to switch the language every time the
%    want to typeset a Latin word inside a Ukrainian phrase or vice versa

%    We parse the |\cdp@list| containing the encodings known to \LaTeX\ i
%    the order they were loaded. We set the |\cyrillicencoding| to th
%    \emph{last} loaded encoding in the list of supported Cyrilli
%    encodings: \texttt{OT2}, \texttt{LWN}, \texttt{LCY}, \texttt{X2}
%    \texttt{T2C}, \texttt{T2B}, \texttt{T2A}, if any

%    \begin{macrocode
\def\reserved@a#1#2{
   \edef\reserved@b{#1}
   \edef\reserved@c{#2}
   \ifx\reserved@b\reserved@
     \let\cyrillicencoding\reserved@
   \fi
\def\cdp@elt#1#2#3#4{
   \reserved@a{#1}{OT2}
   \reserved@a{#1}{LWN}
   \reserved@a{#1}{LCY}
   \reserved@a{#1}{X2}
   \reserved@a{#1}{T2C}
   \reserved@a{#1}{T2B}
   \reserved@a{#1}{T2A}
\cdp@lis
%    \end{macrocode

%    Now, if |\cyrillicencoding| is undefined, then the user did not loa
%    any of supported encodings. So, we have to set |\cyrillicencoding| t
%    some default value. We test the presence of the encoding definitio
%    files in the order from less preferable to more preferable encodings
%    We use the lowercase names (i.e., \file{lcyenc.def} instead o
%    \file{LCYenc.def})

%    \begin{macrocode
\ifx\cyrillicencoding\undefine
  \IfFileExists{ot2enc.def}{\def\cyrillicencoding{OT2}}\rela
  \IfFileExists{lwnenc.def}{\def\cyrillicencoding{LWN}}\rela
  \IfFileExists{lcyenc.def}{\def\cyrillicencoding{LCY}}\rela
  \IfFileExists{x2enc.def}{\def\cyrillicencoding{X2}}\rela
  \IfFileExists{t2cenc.def}{\def\cyrillicencoding{T2C}}\rela
  \IfFileExists{t2benc.def}{\def\cyrillicencoding{T2B}}\rela
  \IfFileExists{t2aenc.def}{\def\cyrillicencoding{T2A}}\rela
%    \end{macrocode

%    If |\cyrillicencoding| is still undefined, then the user seems not t
%    have a properly installed distribution. A fatal error

%    \begin{macrocode
  \ifx\cyrillicencoding\undefine
    \PackageError{babel}
      {No Cyrillic encoding definition files were found}
      {Your installation is incomplete.\MessageBrea
       You need at least one of the following files:\MessageBrea
       \space\spac
       x2enc.def, t2aenc.def, t2benc.def, t2cenc.def,\MessageBrea
       \space\spac
       lcyenc.def, lwnenc.def, ot2enc.def.}
  \els
%    \end{macrocode

%    We avoid |\usepackage[\cyrillicencoding]{fontenc}| because we don'
%    want to force the switch of |\encodingdefault|

%    \begin{macrocode
    \lowercas
      \expandafter{\expandafter\input\cyrillicencoding enc.def\relax}
  \f
\f
%    \end{macrocode

%    \begin{verbatim
%      \PackageInfo{babel
%        {Using `\cyrillicencoding' as a default Cyrillic encoding}
%    \end{verbatim
%    \unski

%    \begin{macrocode
\DeclareRobustCommand{\Ukrainian}{
  \fontencoding\cyrillicencoding\selectfon
  \let\encodingdefault\cyrillicencodin
  \expandafter\set@hyphenmins\ukrainianhyphenmin
  \language\l@ukrainian}
\DeclareRobustCommand{\English}{
  \fontencoding\latinencoding\selectfon
  \let\encodingdefault\latinencodin
  \expandafter\set@hyphenmins\englishhyphenmin
  \language\l@english}
\let\Ukr\Ukrainia
\let\Eng\Englis
\let\cyrillictext\Ukrainia
\let\cyr\Ukrainia
%    \end{macrocode

%    Since the \texttt{X2} encoding does not contain Latin letters, w
%    should make some redefinitions of \LaTeX\ macros which implicitl
%    produce Latin letters

%    \begin{macrocode
\expandafter\ifx\csname T@X2\endcsname\relax\els
%    \end{macrocode

%    We put |\latinencoding| in braces to avoid problems wit
%    |\@alph| inside minipages (e.g., footnotes inside minipages) wher
%    |\@alph| is expanded and we get for example `|\fontencoding OT1|
%    (|\fontencoding| is robust)

%    \begin{macrocode
  \def\@alph#1{{\fontencoding{\latinencoding}\selectfon
    \ifcase#1\o
      a\or b\or c\or d\or e\or f\or g\or h\o
      i\or j\or k\or l\or m\or n\or o\or p\o
      q\or r\or s\or t\or u\or v\or w\or x\o
      y\or z\else\@ctrerr\fi}}
  \def\@Alph#1{{\fontencoding{\latinencoding}\selectfon
    \ifcase#1\o
      A\or B\or C\or D\or E\or F\or G\or H\o
      I\or J\or K\or L\or M\or N\or O\or P\o
      Q\or R\or S\or T\or U\or V\or W\or X\o
      Y\or Z\else\@ctrerr\fi}}
%    \end{macrocode

%    Unfortunately, the commands |\AA| and |\aa| are not encoding dependen
%    in \LaTeX\ (unlike e.g., |\oe| or |\DH|). They are defined as |\r{A}| an
%    |\r{a}|. This leads to unpredictable results when the font encodin
%    does not contain the Latin letters `A' and `a' (like \texttt{X2})

%    \begin{macrocode
  \DeclareTextSymbolDefault{\AA}{OT1
  \DeclareTextSymbolDefault{\aa}{OT1
  \DeclareTextCommand{\aa}{OT1}{\r a
  \DeclareTextCommand{\AA}{OT1}{\r A
\f
%    \end{macrocode

%    The following block redefines the character class of uppercase Gree
%    letters and some accents, if it is equal to 7 (variable family), t
%    avoid incorrect results if the font encoding in some math family doe
%    not contain these characters in places of OT1 encoding. The code wa
%    taken from |amsmath.dtx|. See comments and further explanation there

% \changes{ukraineb-1.1i}{2001/02/21}{As this code generates
%    textfont 7 error it is commented out for now.
%    \begin{macrocode
% \begingroup\catcode`\"=1
% % uppercase greek letters
% \def\@tempa#1{\expandafter\@tempb\meaning#1\relax\relax\relax\rela
%   "0000\@nil#1
% \def\@tempb#1"#2#3#4#5#6\@nil#7{
%   \ifnum"#2=7 \count@"1#3#4#5\rela
%     \ifnum\count@<"1000 \else \global\mathchardef#7="0#3#4#5\relax \f
%   \fi
% \@tempa\Gamma\@tempa\Delta\@tempa\Theta\@tempa\Lambda\@tempa\X
% \@tempa\Pi\@tempa\Sigma\@tempa\Upsilon\@tempa\Phi\@tempa\Ps
% \@tempa\Omeg
% % some accents
% \def\@tempa#1#2\@nil{\def\@tempc{#1}}\def\@tempb{\mathaccent
% \expandafter\@tempa\hat\relax\relax\@ni
% \ifx\@tempb\@temp
%   \def\@tempa#1\@nil{#1}
%   \def\@tempb#1{\afterassignment\@tempa\mathchardef\@tempc=}
%   \def\do#1"#2{
%   \def\@tempd#1{\expandafter\@tempb#1\@ni
%     \ifnum\@tempc>"FF
%       \xdef#1{\mathaccent"\expandafter\do\meaning\@tempc\space}
%     \fi
%   \@tempd\hat\@tempd\check\@tempd\tilde\@tempd\acute\@tempd\grav
%   \@tempd\dot\@tempd\ddot\@tempd\breve\@tempd\ba
% \f
% \endgrou
%    \end{macrocode

%    The user must use the \pkg{inputenc} package when any 8-bit Cyrilli
%    font encoding is used, selecting one of the Cyrillic input encodings
%    We do not assume any default input encoding, so the user shoul
%    explicitly call the \pkg{inputenc} package by |\usepackage{inputenc}|
%    We also removed |\AtBeginDocument|, so \pkg{inputenc} should be use
%    before \babel

% \changes{ukraineb-1.1f}{1999/08/27}{Made not using inputenc
%    warning instead of an error
%    \begin{macrocode
\@ifpackageloaded{inputenc}{}{
  \def\reserved@a{LWN}
  \ifx\reserved@a\cyrillicencoding\els
    \def\reserved@a{OT2}
    \ifx\reserved@a\cyrillicencoding\els
      \PackageWarning{babel}
        {No input encoding specified for Ukrainian language
  \fi\fi
%    \end{macrocode

%    Now we define two commands that offer the possibility to switch betwee
%    Cyrillic and Roman encodings

%  \begin{macro}{\cyrillictext
%  \begin{macro}{\latintext

%    The command |\cyrillictext| will switch from Latin font encoding to th
%    Cyrillic font encoding, the command |\latintext| switches back. Thi
%    assumes that the `normal' font encoding is a Latin one. These command
%    are \emph{declarations}, for shorter peaces of text the command
%    |\textlatin| and |\textcyrillic| can be used

% \changes{ukraineb-1.1j}{2003/10/12}{\cs{latintext} is alread
%    defined by the core of \babel
%    \begin{macrocode
%\DeclareRobustCommand{\latintext}{
%  \fontencoding{\latinencoding}\selectfon
%  \def\encodingdefault{\latinencoding}
\let\lat\latintex
%    \end{macrocode

%  \end{macro
%  \end{macro

%  \begin{macro}{\textcyrillic
%  \begin{macro}{\textlatin

%    These commands take an argument which is then typeset using th
%    requested font encoding
% \changes{ukraineb-1.1j}{2003/10/12}{\cs{latintext} is alread
%    defined by the core of \babel
%    \begin{macrocode
\DeclareTextFontCommand{\textcyrillic}{\cyrillictext
%\DeclareTextFontCommand{\textlatin}{\latintext
%    \end{macrocode

%  \end{macro
%  \end{macro

%    We make the \Te
%    \begin{macrocode
%\ifx\ltxTeX\undefined\let\ltxTeX\TeX\f
%\ProvideTextCommandDefault{\TeX}{\textlatin{\ltxTeX}
%    \end{macrocode
%    and \LaTeX\ logos encoding independent
%    \begin{macrocode
%\ifx\ltxLaTeX\undefined\let\ltxLaTeX\LaTeX\f
%\ProvideTextCommandDefault{\LaTeX}{\textlatin{\ltxLaTeX}
%    \end{macrocode

%    The next step consists of defining commands to switch to (an
%    from) the Ukrainian language

% \begin{macro}{\captionsukrainian

%    The macro |\captionsukrainian| defines all strings used in the fou
%    standard document classes provided with \LaTeX. The two commands |\cyr
%    and |\lat| activate Cyrillic resp.\ Latin encoding
% \changes{ukraineb-1.1d}{1999/04/03}{replace \cs{CYRUKRI} wit
%    \cs{CYRII} in \cs{authorname}
% \changes{ukraineb-1.1g}{2000/09/20}{Added \cs{glossaryname}
% \changes{ukraineb-1.1h}{2001/02/13}{Added translation fo
%    `Glossary'
%    \begin{macrocode
\addto\captionsukrainian{
  \def\prefacename{{\cyr\CYRV\cyrs\cyrt\cyru\cyrp}}
% \def\prefacename{{\cyr\CYRP\cyre\cyrr\cyre\cyrd\cyrm\cyro\cyrv\cyra}}
  \def\refname{
    {\cyr\CYRL\cyrii\cyrt\cyre\cyrr\cyra\cyrt\cyru\cyrr\cyra}}
%  \def\refname{
%    {\cyr\CYRP\cyre\cyrr\cyre\cyrl\cyrii\cyr
%         \ \cyrp\cyro\cyrs\cyri\cyrl\cyra\cyrn\cyrsftsn}}
  \def\abstractname{
    {\cyr\CYRA\cyrn\cyro\cyrt\cyra\cyrc\cyrii\cyrya}}
%  \def\abstractname{{\cyr\CYRR\cyre\cyrf\cyre\cyrr\cyra\cyrt}}
  \def\bibname{
    {\cyr\CYRB\cyrii\cyrb\cyrl\cyrii\cyro\cyrgup\cyrr\cyra\cyrf\cyrii\cyrya}}
% \def\bibname{{\cyr\CYRL\cyrii\cyrt\cyre\cyrr\cyra\cyrt\cyru\cyrr\cyra}}
  \def\chaptername{{\cyr\CYRR\cyro\cyrz\cyrd\cyrii\cyrl}}
%  \def\chaptername{{\cyr\CYRG\cyrl\cyra\cyrv\cyra}}
  \def\appendixname{{\cyr\CYRD\cyro\cyrd\cyra\cyrt\cyro\cyrk}}
  \def\contentsname{{\cyr\CYRZ\cyrm\cyrii\cyrs\cyrt}}
  \def\listfigurename{{\cyr\CYRP\cyre\cyrr\cyre\cyrl\cyrii\cyr
         \ \cyrii\cyrl\cyryu\cyrs\cyrt\cyrr\cyra\cyrc\cyrii\cyrishrt}}
  \def\listtablename{{\cyr\CYRP\cyre\cyrr\cyre\cyrl\cyrii\cyr
         \ \cyrt\cyra\cyrb\cyrl\cyri\cyrc\cyrsftsn}}
  \def\indexname{{\cyr\CYRP\cyro\cyrk\cyra\cyrzh\cyrch\cyri\cyrk}}
  \def\authorname{{\cyr\CYRII\cyrm\cyre\cyrn\cyrn\cyri\cyrishr
         \ \cyrp\cyro\cyrk\cyra\cyrzh\cyrch\cyri\cyrk}}
  \def\figurename{{\cyr\CYRR\cyri\cyrs.}}
%  \def\figurename{\cyr\CYRR\cyri\cyrs\cyru\cyrn\cyro\cyrk}}
  \def\tablename{{\cyr\CYRT\cyra\cyrb\cyrl.}}
%  \def\tablename{\cyr\CYRT\cyra\cyrb\cyrl\cyri\cyrc\cyrya}}
  \def\partname{{\cyr\CYRCH\cyra\cyrs\cyrt\cyri\cyrn\cyra}}
  \def\enclname{{\cyr\cyrv\cyrk\cyrl\cyra\cyrd\cyrk\cyra}}
  \def\ccname{{\cyr\cyrk\cyro\cyrp\cyrii\cyrya}}
  \def\headtoname{{\cyr\CYRD\cyro}}
  \def\pagename{{\cyr\cyrs.}}
%  \def\pagename{{\cyr\cyrs\cyrt\cyro\cyrr\cyrii\cyrn\cyrk\cyra}}
  \def\seename{{\cyr\cyrd\cyri\cyrv.}}
  \def\alsoname{{\cyr\cyrd\cyri\cyrv.\ \cyrt\cyra\cyrk\cyro\cyrzh}
  \def\proofname{{\cyr\CYRD\cyro\cyrv\cyre\cyrd\cyre\cyrn\cyrn\cyrya}}
  \def\glossaryname{{\cyr\CYRS\cyrl\cyro\cyrv\cyrn\cyri\cyrk\
                   \cyrt\cyre\cyrr\cyrm\cyrii\cyrn\cyrii\cyrv}}

%    \end{macrocode

% \end{macro

% \begin{macro}{\dateukrainian

%    The macro |\dateukrainian| redefines the command |\today| to produc
%    Ukrainian dates

%    \begin{macrocode
\def\dateukrainian{
  \def\today{\number\day~\ifcase\month\o
    \cyrs\cyrii\cyrch\cyrn\cyrya\o
    \cyrl\cyryu\cyrt\cyro\cyrg\cyro\o
    \cyrb\cyre\cyrr\cyre\cyrz\cyrn\cyrya\o
    \cyrk\cyrv\cyrii\cyrt\cyrn\cyrya\o
    \cyrt\cyrr\cyra\cyrv\cyrn\cyrya\o
    \cyrch\cyre\cyrr\cyrv\cyrn\cyrya\o
    \cyrl\cyri\cyrp\cyrn\cyrya\o
    \cyrs\cyre\cyrr\cyrp\cyrn\cyrya\o
    \cyrv\cyre\cyrr\cyre\cyrs\cyrn\cyrya\o
    \cyrzh\cyro\cyrv\cyrt\cyrn\cyrya\o
    \cyrl\cyri\cyrs\cyrt\cyro\cyrp\cyra\cyrd\cyra\o
    \cyrg\cyrr\cyru\cyrd\cyrn\cyrya\f
    \space\number\year~\cyrr.}
%    \end{macrocode

% \end{macro

% \begin{macro}{\extrasukrainian

%    The macro |\extrasukrainian| will perform all the extra definition
%    needed for the Ukrainian language. The macro |\noextrasukrainian
%    is used to cancel the actions of |\extrasukrainian|

%    The first action we define is to switch on the selected Cyrilli
%    encoding whenever we enter `ukrainian'

%    \begin{macrocode
\addto\extrasukrainian{\cyrillictext
%    \end{macrocode

%    When the encoding definition file was processed by \LaTeX\ the curren
%    font encoding is stored in |\latinencoding|, assuming that \LaTeX\ use
%    \texttt{T1} or \texttt{OT1} as default. Therefore we switch back t
%    |\latinencoding| whenever the Ukrainian language is no longer `active'

%    \begin{macrocode
\addto\noextrasukrainian{\latintext
%    \end{macrocode

%    Next we must allow hyphenation in the Ukrainian words with apostroph
%    whenever we enter `ukrainian'. This solution was proposed b
%    Vladimir Volovich <vvv@vvv.vsu.ru

%    \begin{macrocode
\addto\extrasukrainian{\lccode`\'=`\'
\addto\noextrasukrainian{\lccode`\'=0
%    \end{macrocode

%  \begin{macro}{\verbatim@font

%    In order to get both Latin and Cyrillic letters in verbatim text w
%    need to change the definition of an internal \LaTeX\ command somewhat

%    \begin{macrocode
%\def\verbatim@font{
%  \let\encodingdefault\latinencodin
%  \normalfont\ttfamil
%  \expandafter\def\csname\cyrillicencoding-cmd\endcsname##1##2{
%    \ifx\protect\@typeset@protec
%      \begingroup\UseTextSymbol\cyrillicencoding##1\endgrou
%    \else\noexpand##1\fi}
%    \end{macrocode

%  \end{macro

%    The category code of the characters `\texttt{:}', `\texttt{;}'
%    `\texttt{!}', and `\texttt{?}' is made |\active| to insert a littl
%    white space

%    For Ukrainian (as well as for Russian and German) the \texttt{"
%    character also is made active

%    Note: It is \emph{very} questionable whether the Russian typesettin
%    tradition requires additional spacing before those punctuation signs
%    Therefore, we make the corresponding code optional. If you need it
%    then define the \texttt{frenchpunct} docstrip option i
%    \file{babel.ins}

%    Borrowed from french
%    Some users dislike automatic insertion of a space befor
%    `double punctuation', and prefer to decide themselves whether
%    space should be added or not; so a hook |\NoAutoSpaceBeforeFDP
%    is provided: if this command is added (in file |ukraineb.cfg|, o
%    anywhere in a document) |ukraineb| will respect your typing, an
%    introduce a suitable space before `double punctuation' \emph{i
%    and only if} a space is typed in the source file before thos
%    signs

%    The command |\AutoSpaceBeforeFDP| switches back to th
%    default behavior of |ukraineb|

%    \begin{macrocode
%<*frenchpunct
\initiate@active@char{:
\initiate@active@char{;
%</frenchpunct
%<*frenchpunct|spanishligs
\initiate@active@char{!
\initiate@active@char{?
%</frenchpunct|spanishligs
\initiate@active@char{"
%    \end{macrocode

%    The code above is necessary because we need extra active characters
%    The character |"| is used as indicated i
%    table~\ref{tab:ukrainian-quote}

%    We specify that the Ukrainian group of shorthands should be used

%    \begin{macrocode
\addto\extrasukrainian{\languageshorthands{ukrainian}
%    \end{macrocode

%    These characters are `turned on' once, later their definition ma
%    vary

%    \begin{macrocode
\addto\extrasukrainian{
%<frenchpunct>  \bbl@activate{:}\bbl@activate{;}
%<frenchpunct|spanishligs>  \bbl@activate{!}\bbl@activate{?}
  \bbl@activate{"}
\addto\noextrasukrainian{
%<frenchpunct>  \bbl@deactivate{:}\bbl@deactivate{;}
%<frenchpunct|spanishligs>  \bbl@deactivate{!}\bbl@deactivate{?}
  \bbl@deactivate{"}
%    \end{macrocode

%   The \texttt{X2} and \texttt{T2*} encodings do not contai
%   |spanish_shriek| and |spanish_query| symbols; as a consequence, th
%   ligatures `|?`|' and `|!`|' do not work with them (these characters ar
%   useless for Cyrillic texts anyway). But we define the shorthands t
%   emulate these ligatures (optionally)

%   We do not use |\latinencoding| here (but instead explicitly us
%   \texttt{OT1}) because the user may choose \texttt{T2A} to be the primar
%   encoding, but it does not contain these characters

%    \begin{macrocode
%<*spanishligs
\declare@shorthand{ukrainian}{?`}{\UseTextSymbol{OT1}\textquestiondown
\declare@shorthand{ukrainian}{!`}{\UseTextSymbol{OT1}\textexclamdown
%</spanishligs
%    \end{macrocode

% \begin{macro}{\ukrainian@sh@;@
% \begin{macro}{\ukrainian@sh@:@
% \begin{macro}{\ukrainian@sh@!@
% \begin{macro}{\ukrainian@sh@?@

%    We have to reduce the amount of white space before \texttt{;}
%    \texttt{:} and \texttt{!}. This should only happen in horizontal mode
%    hence the test with |\ifhmode|

%    \begin{macrocode
%<*frenchpunct
\declare@shorthand{ukrainian}{;}{
  \ifhmod
%    \end{macrocode

%    In horizontal mode we check for the presence of a `space', `unskip' i
%    it exists and place a |0.1em| kerning

%    \begin{macrocode
    \ifdim\lastskip>\z
      \unskip\nobreak\kern.1e
    \els
%    \end{macrocode
%    If no space has been typed, we add |\FDP@thinspace
%    which will b
%    defined, up to the user's wishes, as an automatic adde
%    thinspace, or as |\@empty|

%    \begin{macrocode
        \FDP@thinspac
    \f
  \f
%    \end{macrocode

%    Now we can insert a `|;|' character

%    \begin{macrocode
  \string;
%    \end{macrocode

%    The other definitions are very similar

%    \begin{macrocode
\declare@shorthand{ukrainian}{:}{
  \ifhmod
    \ifdim\lastskip>\z
      \unskip\nobreak\kern.1e
    \els
        \FDP@thinspac
    \f
  \f
  \string:
%    \end{macrocode

%    \begin{macrocode
\declare@shorthand{ukrainian}{!}{
  \ifhmod
    \ifdim\lastskip>\z
      \unskip\nobreak\kern.1e
    \els
        \FDP@thinspac
    \f
  \f
  \string!
%    \end{macrocode

%    \begin{macrocode
\declare@shorthand{ukrainian}{?}{
  \ifhmod
    \ifdim\lastskip>\z
      \unskip\nobreak\kern.1e
    \els
        \FDP@thinspac
    \f
  \f
  \string?
%    \end{macrocode

% \end{macro
% \end{macro
% \end{macro
% \end{macro


%  \begin{macro}{\AutoSpaceBeforeFDP
%  \begin{macro}{\NoAutoSpaceBeforeFDP
%  \begin{macro}{\FDP@thinspace
%    |\FDP@thinspace| is defined as unbreakabl
%    spaces if |\AutoSpaceBeforeFDP| is activated or as |\@empty| i
%    |\NoAutoSpaceBeforeFDP| is in use
%    The default is |\AutoSpaceBeforeFDP|
%    \begin{macrocode
\def\AutoSpaceBeforeFDP{
      \def\FDP@thinspace{\nobreak\kern.1em}
\def\NoAutoSpaceBeforeFDP{\let\FDP@thinspace\@empty
\AutoSpaceBeforeFD
%    \end{macrocode
%  \end{macro
%  \end{macro
%  \end{macro

%  \begin{macro}{\FDPon
%  \begin{macro}{\FDPoff

%     The next macros allow to switch on/off activeness of doubl
%     punctuation signs

%    \begin{macrocode
\def\FDPon{\bbl@activate{:}
        \bbl@activate{;}
        \bbl@activate{?}
        \bbl@activate{!}
\def\FDPoff{\bbl@deactivate{:}
        \bbl@deactivate{;}
        \bbl@deactivate{?}
        \bbl@deactivate{!}
%    \end{macrocode
%  \end{macro
%  \end{macro

%  \begin{macro}{\system@sh@:@
%  \begin{macro}{\system@sh@!@
%  \begin{macro}{\system@sh@?@
%  \begin{macro}{\system@sh@;@

%    When the active characters appear in an environment where thei
%    Ukrainian behaviour is not wanted they should give an `expected
%    result. Therefore we define shorthands at system level as well

%    \begin{macrocode
\declare@shorthand{system}{:}{\string:
\declare@shorthand{system}{;}{\string;
%</frenchpunct
%<*frenchpunct&!spanishligs
\declare@shorthand{system}{!}{\string!
\declare@shorthand{system}{?}{\string?
%</frenchpunct&!spanishligs
%    \end{macrocode

%  \end{macro
%  \end{macro
%  \end{macro
%  \end{macro

%    To be able to define the function of `|"|', we first define a couple o
%    `support' macros

%  \begin{macro}{\dq

%    We save the original double quote character in |\dq| to keep i
%    available, the math accent |\"| can now be typed as `|"|'

%    \begin{macrocode
\begingroup \catcode`\"1
\def\reserved@a{\endgrou
  \def\@SS{\mathchar"7019
  \def\dq{"}
\reserved@
%    \end{macrocode

%  \end{macro

%    Now we can define the doublequote macros: german and french quotes
%    We use definitions of these quotes made in babel.sty
%    The french quotes are contained in the \texttt{T2*} encodings

%    \begin{macrocode
\declare@shorthand{ukrainian}{"`}{\glqq
\declare@shorthand{ukrainian}{"'}{\grqq
\declare@shorthand{ukrainian}{"<}{\flqq
\declare@shorthand{ukrainian}{">}{\frqq
%    \end{macrocode

%    Some additional commands

%    \begin{macrocode
\declare@shorthand{ukrainian}{""}{\hskip\z@skip
\declare@shorthand{ukrainian}{"~}{\textormath{\leavevmode\hbox{-}}{-}
\declare@shorthand{ukrainian}{"=}{\nobreak-\hskip\z@skip
\declare@shorthand{ukrainian}{"|}{
  \textormath{\nobreak\discretionary{-}{}{\kern.03em}
              \allowhyphens}{}
%    \end{macrocode

%    The next two macros for |"-| and |"---| are somewhat different
%    We must check whether the second token is a hyphen character

%    \begin{macrocode
\declare@shorthand{ukrainian}{"-}{
%    \end{macrocode

%    If the next token is `|-|', we typeset an emdash, otherwise a hyphe
%    sign

%    \begin{macrocode
  \def\ukrainian@sh@tmp{
    \if\ukrainian@sh@next-\expandafter\ukrainian@sh@emdas
    \else\expandafter\ukrainian@sh@hyphen\f
  }
%    \end{macrocode

%    \TeX\ looks for the next token after the first `|-|': the meaning o
%    this token is written to |\ukrainian@sh@next| and |\ukrainian@sh@tmp| i
%    called

%    \begin{macrocode
  \futurelet\ukrainian@sh@next\ukrainian@sh@tmp
%    \end{macrocode

%    Here are the definitions of hyphen and emdash. First the hyphen

%    \begin{macrocode
\def\ukrainian@sh@hyphen{
  \nobreak\-\bbl@allowhyphens
%    \end{macrocode

%    For the emdash definition, there are the two parameters: we must `eat
%    two last hyphen signs of our emdash\dots
%    \begin{macrocode
\def\ukrainian@sh@emdash#1#2{\cdash-#1#2
%    \end{macrocode
%  \begin{macro}{\cdash
%    \dots\ these two parameters are useful for another macro
%    |\cdash|
%    \begin{macrocode
%\ifx\cdash\undefined % should be defined earlie
\def\cdash#1#2#3{\def\tempx@{#3}
\def\tempa@{-}\def\tempb@{~}\def\tempc@{*}
 \ifx\tempx@\tempa@\@Acdash\els
  \ifx\tempx@\tempb@\@Bcdash\els
   \ifx\tempx@\tempc@\@Ccdash\els
    \errmessage{Wrong usage of cdash}\fi\fi\fi
%    \end{macrocode
%   second parameter (or third for |\cdash|) shows what kind of emdas
%   to create in next ste
%      \begin{center
%      \begin{tabular}{@{}p{.1\hsize}@{}p{.9\hsize}@{}
%       |"---| & ordinary (plain) Cyrillic emdash inside text
%       an unbreakable thinspace will be inserted before only in case o
%       a \textit{space} before the dash (it is necessary for dashes afte
%       display maths formulae: there could be lists, enumerations etc.
%       started with ``--- where $a$ is ...'' i.e., the dash starts a line)
%       (Firstly there were planned rather soft rules for user: he may pu
%       a space before the dash or not.  But it is difficult to place thi
%       thinspace automatically, i.e., by checking modes because afte
%       display formulae \TeX{} uses horizontal mode.  Maybe there is
%       misunderstanding?  Maybe there is another way?)  After a das
%       a breakable thinspace is always placed; \
%   \end{tabular
%   \end{center
%    \begin{macrocode
% What is more grammatically: .2em or .2\fontdimen6\font
\def\@Acdash{\ifdim\lastskip>\z@\unskip\nobreak\hskip.2em\f
  \cyrdash\hskip.2em\ignorespaces}
%    \end{macrocode
%      \begin{center
%      \begin{tabular}{@{}p{.1\hsize}@{}p{.9\hsize}@{}
%       |"--~| & emdash in compound names or surname
%       (like Mendeleev--Klapeiron); this dash has no space character
%       around; after the dash some space is adde
%       |\exhyphenalty| \
%   \end{tabular
%   \end{center
%    \begin{macrocode
\def\@Bcdash{\leavevmode\ifdim\lastskip>\z@\unskip\f
 \nobreak\cyrdash\penalty\exhyphenpenalty\hskip\z@skip\ignorespaces}
%    \end{macrocode
%      \begin{center
%      \begin{tabular}{@{}p{.1\hsize}@{}p{.9\hsize}@{}
%       |"--*| & for denoting direct speech (a space like |\enskip
%       must follow the emdash); \
%   \end{tabular
%   \end{center
%    \begin{macrocode
\def\@Ccdash{\leavevmod
 \nobreak\cyrdash\nobreak\hskip.35em\ignorespaces}
%\f
%    \end{macrocode
%  \end{macro

%  \begin{macro}{\cyrdash
%   Finally the macro for ``body'' of the Cyrillic emdash
%   The |\cyrdash| macro will be defined in case this macro hasn't bee
%   defined in a fontenc file.  For T2* fonts, cyrdash will be placed i
%   the code of the English emdash thus it uses ligature |---|
%    \begin{macrocode
% Is there an IF necessary
\ifx\cyrdash\undefine
  \def\cyrdash{\hbox to.8em{--\hss--}
\f
%    \end{macrocode
%  \end{macro

%    Here a really new macro---to place thinspace between initials
%    This macro used instead of |\,| allows hyphenation in the followin
%    surname

%    \begin{macrocode
\declare@shorthand{ukrainian}{",}{\nobreak\hskip.2em\ignorespaces
%    \end{macrocode

%  \begin{macro}{\mdqon
%  \begin{macro}{\mdqoff
%    All that's left to do now is to  define a couple of command
%    for |"|
%    \begin{macrocode
\def\mdqon{\bbl@activate{"}
\def\mdqoff{\bbl@deactivate{"}
%    \end{macrocode
%  \end{macro
%  \end{macro

%    The Ukrainian hyphenation patterns can be used with |\lefthyphenmin
%    and |\righthyphenmin| set to~2
% \changes{ukraineb-1.1g}{2000/09/22}{Now use \cs{providehyphenmins} t
%    provide a default value
%    \begin{macrocode
\providehyphenmins{\CurrentOption}{\tw@\tw@
% temporary hack
\ifx\englishhyphenmins\undefine
  \def\englishhyphenmins{\tw@\thr@@
\f
%    \end{macrocode

%    Now the action |\extrasukrainian| has to execute is to make sure that th
%    command |\frenchspacing| is in effect. If this is not the case th
%    execution of |\noextrasukrainian| will switch it off again

%    \begin{macrocode
\addto\extrasukrainian{\bbl@frenchspacing
\addto\noextrasukrainian{\bbl@nonfrenchspacing
%    \end{macrocode

% \end{macro

%    Next we add a new enumeration style for Ukrainian manuscripts wit
%    Cyrillic letters, and later on we define some math operator names i
%    accordance with Ukrainian and Russian typesetting traditions

%  \begin{macro}{\Asbuk

%    We begin by defining |\Asbuk| which works like |\Alph|, but produce
%    (uppercase) Cyrillic letters intead of Latin ones. The letters CYRGUP
%    and SFTSN are skipped, as usual for such enumeration

%    \begin{macrocode
\def\Asbuk#1{\expandafter\@Asbuk\csname c@#1\endcsname
\def\@Asbuk#1{\ifcase#1\o
  \CYRA\or\CYRB\or\CYRV\or\CYRG\or\CYRD\or\CYRE\or\CYRIE\o
  \CYRZH\or\CYRZ\or\CYRI\or\CYRII\or\CYRYI\or\CYRISHRT\o
  \CYRK\or\CYRL\or\CYRM\or\CYRN\or\CYRO\or\CYRP\or\CYRR\o
  \CYRS\or\CYRT\or\CYRU\or\CYRF\or\CYRH\or\CYRC\or\CYRCH\o
  \CYRSH\or\CYRSHCH\or\CYRYU\or\CYRYA\else\@ctrerr\fi
%    \end{macrocode

%  \end{macro

%  \begin{macro}{\asbuk

%    The macro |\asbuk| is similar to |\alph|; it produces lowercas
%    Ukrainian letters

%    \begin{macrocode
\def\asbuk#1{\expandafter\@asbuk\csname c@#1\endcsname
\def\@asbuk#1{\ifcase#1\o
  \cyra\or\cyrb\or\cyrv\or\cyrg\or\cyrd\or\cyre\or\cyrie\o
  \cyrzh\or\cyrz\or\cyri\or\cyrii\or\cyryi\or\cyrishrt\o
  \cyrk\or\cyrl\or\cyrm\or\cyrn\or\cyro\or\cyrp\or\cyrr\o
  \cyrs\or\cyrt\or\cyru\or\cyrf\or\cyrh\or\cyrc\or\cyrch\o
  \cyrsh\or\cyrshch\or\cyryu\or\cyrya\else\@ctrerr\fi
%    \end{macrocode

%  \end{macro

%    Set up default Cyrillic math alphabets. The math groups for cyrilli
%    letters are defined in the encoding definition files. First, declar
%    a new alphabet for symbols, |\cyrmathrm|, based on the symbol fon
%    for Cyrillic letters defined in the encoding definition file. Note
%    that by default Cyrillic letters are taken from upright font in mat
%    mode (unlike Latin letters)
%    \begin{macrocode
%\RequirePackage{textmath
\@ifundefined{sym\cyrillicencoding letters}{}{
\SetSymbolFont{\cyrillicencoding letters}{bold}\cyrillicencodin
  \rmdefault\bfdefault\updefaul
\DeclareSymbolFontAlphabet\cyrmathrm{\cyrillicencoding letters
%    \end{macrocode
%    And we need a few commands to be able to switch to different variants
%    \begin{macrocode
\DeclareMathAlphabet\cyrmathbf\cyrillicencodin
  \rmdefault\bfdefault\updefaul
\DeclareMathAlphabet\cyrmathsf\cyrillicencodin
  \sfdefault\mddefault\updefaul
\DeclareMathAlphabet\cyrmathit\cyrillicencodin
  \rmdefault\mddefault\itdefaul
\DeclareMathAlphabet\cyrmathtt\cyrillicencodin
  \ttdefault\mddefault\updefaul

\SetMathAlphabet\cyrmathsf{bold}\cyrillicencodin
  \sfdefault\bfdefault\updefaul
\SetMathAlphabet\cyrmathit{bold}\cyrillicencodin
  \rmdefault\bfdefault\itdefaul

%    \end{macrocode

%    Some math functions in Ukrainian and Russian math books have othe
%    names: e.g., \texttt{sinh} in Russian is written as \texttt{sh} etc
%    So we define a number of new math operators

%    |\sinh|
%    \begin{macrocode
\def\sh{\mathop{\operator@font sh}\nolimits
%    \end{macrocode
%    |\cosh|
%    \begin{macrocode
\def\ch{\mathop{\operator@font ch}\nolimits
%    \end{macrocode
%    |\tan|
%    \begin{macrocode
\def\tg{\mathop{\operator@font tg}\nolimits
%    \end{macrocode
%    |\arctan|
%    \begin{macrocode
\def\arctg{\mathop{\operator@font arctg}\nolimits
%    \end{macrocode
%    arcctg
%    \begin{macrocode
\def\arcctg{\mathop{\operator@font arcctg}\nolimits
%    \end{macrocode
%    The following macro conflicts with |\th| defined in Latin~1 encoding

%    |\tanh|
% \changes{ukraineb-1.1k}{2004/05/21}{Change definition of \cs{th
%    only for this language
%    \begin{macrocode
\addto\extrasrussian{
  \babel@save{\th}
  \let\ltx@th\t
  \def\th{\textormath{\ltx@th}
                     {\mathop{\operator@font th}\nolimits}}

%    \end{macrocode
%    |\cot|
%    \begin{macrocode
\def\ctg{\mathop{\operator@font ctg}\nolimits
%    \end{macrocode
%    |\coth|
%    \begin{macrocode
\def\cth{\mathop{\operator@font cth}\nolimits
%    \end{macrocode
%    |\csc|
%    \begin{macrocode
\def\cosec{\mathop{\operator@font cosec}\nolimits
%    \end{macrocode

%    And finally some other Ukrainian and Russian mathematical symbols
%    \begin{macrocode
\def\Prob{\mathop{\kern\z@\mathsf{P}}\nolimits
\def\Variance{\mathop{\kern\z@\mathsf{D}}\nolimits
\def\nsd{\mathop{\cyrmathrm{\cyrn.\cyrs.\cyrd.}}\nolimits
\def\nsk{\mathop{\cyrmathrm{\cyrn.\cyrs.\cyrk.}}\nolimits
\def\NSD{\mathop{\cyrmathrm{\CYRN\CYRS\CYRD}}\nolimits
\def\NSK{\mathop{\cyrmathrm{\CYRN\CYRS\CYRK}}\nolimits
  \def\nod{\mathop{\cyrmathrm{\cyrn.\cyro.\cyrd.}}\nolimits}    % ?????
  \def\nok{\mathop{\cyrmathrm{\cyrn.\cyro.\cyrk.}}\nolimits}    % ?????
  \def\NOD{\mathop{\cyrmathrm{\CYRN\CYRO\CYRD}}\nolimits}       % ?????
  \def\NOK{\mathop{\cyrmathrm{\CYRN\CYRO\CYRK}}\nolimits}       % ?????
\def\Proj{\mathop{\cyrmathrm{\CYRP\cyrr}}\nolimits
%    \end{macrocode

% This is for compatibility with older Ukrainian packages
%    \begin{macrocode
\DeclareRobustCommand{\No}{
   \ifmmode{\nfss@text{\textnumero}}\else\textnumero\fi
%    \end{macrocode

%    The macro |\ldf@finish| takes care of looking for a configuration file
%    setting the main language to be switched on at |\begin{document}| an
%    resetting the category code of \texttt{@} to its original value

%    \begin{macrocode
\ldf@finish{ukrainian
%</code
%    \end{macrocode

% \Final
%
%% \CharacterTabl
%%  {Upper-case    \A\B\C\D\E\F\G\H\I\J\K\L\M\N\O\P\Q\R\S\T\U\V\W\X\Y\
%%   Lower-case    \a\b\c\d\e\f\g\h\i\j\k\l\m\n\o\p\q\r\s\t\u\v\w\x\y\
%%   Digits        \0\1\2\3\4\5\6\7\8\
%%   Exclamation   \!     Double quote  \"     Hash (number) \
%%   Dollar        \$     Percent       \%     Ampersand     \
%%   Acute accent  \'     Left paren    \(     Right paren   \
%%   Asterisk      \*     Plus          \+     Comma         \
%%   Minus         \-     Point         \.     Solidus       \
%%   Colon         \:     Semicolon     \;     Less than     \
%%   Equals        \=     Greater than  \>     Question mark \
%%   Commercial at \@     Left bracket  \[     Backslash     \
%%   Right bracket \]     Circumflex    \^     Underscore    \
%%   Grave accent  \`     Left brace    \{     Vertical bar  \
%%   Right brace   \}     Tilde         \~
%
\endinpu
}
\DeclareOption{uppersorbian}{% \iffalse meta-commen

% Copyright 1989-2008 Johannes L. Braams and any individual author
% listed elsewhere in this file.  All rights reserved
%
% This file is part of the Babel system
% -------------------------------------
%
% It may be distributed and/or modified under th
% conditions of the LaTeX Project Public License, either version 1.
% of this license or (at your option) any later version
% The latest version of this license is i
%   http://www.latex-project.org/lppl.tx
% and version 1.3 or later is part of all distributions of LaTe
% version 2003/12/01 or later
%
% This work has the LPPL maintenance status "maintained"
%
% The Current Maintainer of this work is Johannes Braams
%
% The list of all files belonging to the Babel system i
% given in the file `manifest.bbl. See also `legal.bbl' for additiona
% information
%
% The list of derived (unpacked) files belonging to the distributio
% and covered by LPPL is defined by the unpacking scripts (wit
% extension .ins) which are part of the distribution
% \f
% \CheckSum{344
% \iffals
%    Tell the \LaTeX\ system who we are and write an entry on th
%    transcript
%<*dtx
\ProvidesFile{usorbian.dtx
%</dtx
%<code>\ProvidesLanguage{usorbian
%\f
%\ProvidesFile{usorbian.dtx
        [2008/03/17 v1.0k Upper Sorbian support from the babel system
%\iffals
%% File `usorbian.dtx
%% Babel package for LaTeX version 2
%% Copyright (C) 1989 - 200
%%           by Johannes Braams, TeXnie

%% Upper Sorbian Language Definition Fil
%% Copyright (C) 1994 - 200
%%           by Eduard Werne
%           Werner, Eduard"
%           Serbski institut z. t.
%           Dw\'orni\v{s}\'cowa
%           02625 Budy\v{s}in/Bautze
%           Germany"
%           (??)3591 497223"
%           edi at kaihh.hanse.de"

%% Please report errors to: Eduard Werner edi at kaihh.hanse.d
%
%    This file is part of the babel system, it provides the sourc
%    code for the Upper Sorbian definition file
%<*filedriver
\documentclass{ltxdoc
\newcommand*\TeXhax{\TeX hax
\newcommand*\babel{\textsf{babel}
\newcommand*\langvar{$\langle \it lang \rangle$
\newcommand*\note[1]{
\newcommand*\Lopt[1]{\textsf{#1}
\newcommand*\file[1]{\texttt{#1}
\newfont{\logo}{logo10
\newcommand*\MF{{\logo METAFONT}
\begin{document
 \DocInput{usorbian.dtx
\end{document
%</filedriver
%\f
% \GetFileInfo{usorbian.dtx

% \changes{usorbian-0.1}{1994/10/10}{First version
% \changes{usorbian-0.1b}{1994/10/18}{Made it possible to run throug
%    \LaTeX; added \cs{MF} and removed extra \cs{end{macro}}
% \changes{usorbian-1.0d}{1996/07/13}{Replaced \cs{undefined} wit
%    \cs{@undefined} and \cs{empty} with \cs{@empty} for consistenc
%    with \LaTeX
% \changes{usorbian-1.0e}{1996/10/10}{Moved the definition o
%    \cs{atcatcode} right to the beginning.

%  \section{The Upper Sorbian language

%    The file \file{\filename}\footnote{The file described in thi
%    section has version number \fileversion\ and was last revised o
%    \filedate.  It was written by Eduard Werne
%    (\texttt{edi@kaihh.hanse.de}).}  It defines all th
%    language-specific macros for Upper Sorbian

% \StopEventually{

%    The macro |\LdfInit| takes care of preventing that this file i
%    loaded more than once, checking the category code of th
%    \texttt{@} sign, etc
% \changes{usorbian-1.0e}{1996/11/03}{Now use \cs{LdfInit} to perfor
%    initial checks}
% \changes{usorbian-1.0j}{2007/10/19}{This file can be loaded unde
%    more than one name.
%    \begin{macrocode
%<*code
\LdfInit\CurrentOption{date\CurrentOption
%    \end{macrocode

%    When this file is read as an option, i.e. by the |\usepackage
%    command, \texttt{usorbian} will be an `unknown' language, in whic
%    case we have to make it known. So we check for the existence o
%    |\l@usorbian| to see whether we have to do something here.
% \changes{usorbian-1.0j}{2007/10/19}{Check for the optio
%    lowersorbian
%    A
%    \babel\ also knwos the option \Lopt{uppersorbian} we have t
%    check that as well

%    \begin{macrocode
\ifx\l@uppersorbian\@undefine
  \ifx\l@usorbian\@undefine
    \@nopatterns{Usorbian
    \adddialect\l@usorbian\z
    \let\l@uppersorbian\l@usorbia
  \els
    \let\l@uppersorbian\l@usorbia
  \f
\els
  \let\l@usorbian\l@uppersorbia
\f
%    \end{macrocode

%    The next step consists of defining commands to switch to (an
%    from) the Upper Sorbian language

% \begin{macro}{\captionsusorbian
%    The macro |\captionsusorbian| defines all strings used in the fou
%    standard documentclasses provided with \LaTeX
% \changes{usorbian-0.1c}{1994/11/27}{Removed two typos (Kapitel an
%    Dodatki)
% \changes{usorbian-1.0b}{1995/07/04}{Added \cs{proofname} fo
%    AMS-\LaTeX
% \changes{usorbian-1.0i}{2000/09/22}{Added \cs{glossaryname}
% \changes{usorbian-1.0j}{2007/10/19}{Make this work for more than on
%    option name
%    \begin{macrocode
\@namedef{captions\CurrentOption}{
  \def\prefacename{Zawod}
  \def\refname{Referency}
  \def\abstractname{Abstrakt}
  \def\bibname{Literatura}
  \def\chaptername{Kapitl}
  \def\appendixname{Dodawki}
  \def\contentsname{Wobsah}
  \def\listfigurename{Zapis wobrazow}
  \def\listtablename{Zapis tabulkow}
  \def\indexname{Indeks}
  \def\figurename{Wobraz}
  \def\tablename{Tabulka}
  \def\partname{D\'z\v el}
  \def\enclname{P\v r\l oha}
  \def\ccname{CC}
  \def\headtoname{Komu}
  \def\pagename{Strona}
  \def\seename{hl.}
  \def\alsoname{hl.~te\v z
  \def\proofname{Proof}%  <-- needs translatio
  \def\glossaryname{Glossary}% <-- Needs translatio
  }
%    \end{macrocode
% \end{macro

% \begin{macro}{\newdateusorbian
%    The macro |\newdateusorbian| redefines the command |\today| t
%    produce Upper Sorbian dates
% \changes{usorbian-1.0g}{1997/10/01}{Use \cs{edef} to defin
%    \cs{today} to save memory
% \changes{usorbian-1.0g}{1998/03/28}{use \cs{def} instead o
%    \cs{edef}
% \changes{usorbian-1.0j}{2007/10/19}{Make this work for more than on
%    option name
%    \begin{macrocode
\@namedef{newdate\CurrentOption}{
  \def\today{\number\day.~\ifcase\month\o
    januara\or februara\or m\v erca\or apryla\or meje\or junija\o
    julija\or awgusta\or septembra\or oktobra\o
    nowembra\or decembra\f
    \space \number\year}
%    \end{macrocode
% \end{macro

% \begin{macro}{\olddateusorbian
%    The macro |\olddateusorbian| redefines the command |\today| t
%    produce old-style Upper Sorbian dates
% \changes{usorbian-1.0g}{1997/10/01}{Use \cs{edef} to defin
%    \cs{today} to save memory
% \changes{usorbian-1.0g}{1998/03/28}{use \cs{def} instead o
%    \cs{edef}
% \changes{usorbian-1.0j}{2007/10/19}{Make this work for more than on
%    option name
%    \begin{macrocode
\@namedef{olddate\CurrentOption}{
  \def\today{\number\day.~\ifcase\month\o
    wulkeho r\'o\v zka\or ma\l eho r\'o\v zka\or nal\v etnika\o
    jutrownika\or r\'o\v zownika\or  sma\v znika\or pra\v znika\o
    \v znjenca\or po\v znjenca\or winowca\or nazymnika\o
    hodownika\fi \space \number\year}
%    \end{macrocode
% \end{macro

%    The default will be the new-style dates
% \changes{usorbian-1.0j}{2007/10/19}{Make this work for more than on
%    option name
%    \begin{macrocode
\expandafter\let\csname date\CurrentOption\expandafter\endcsnam
                \csname newdate\CurrentOption\endcsnam
%    \end{macrocode

% \begin{macro}{\extrasusorbian
%    The macro |\extrasusorbian| will perform all the extr
%    definitions needed for the Upper Sorbian language. It's pirate
%    from |germanb.sty|.  The macro |\noextrasusorbian| is used t
%    cancel the actions of |\extrasusorbian|

%    Because for Upper Sorbian (as well as for Dutch) the \texttt{"
%    character is made active. This is done once, later on it
%    definition may vary
% \changes{usorbian-1.0j}{2007/10/19}{Make this work for more than on
%    option name
%    \begin{macrocode
\initiate@active@char{"
\@namedef{extras\CurrentOption}{\languageshorthands{usorbian}
\expandafter\addto\csname extras\CurrentOption\endcsname{
  \bbl@activate{"}
%    \end{macrocode
%    Don't forget to turn the shorthands off again
% \changes{usorbian-1.0h}{1999/12/17}{Deactivate shorthands ouside o
%    Upper Sorbian
%    \begin{macrocode
\expandafter\addto\csname extras\CurrentOption\endcsname{
  \bbl@deactivate{"}
%    \end{macrocode

%    In order for \TeX\ to be able to hyphenate German Upper Sorbia
%    words which contain `\ss' we have to give the character a nonzer
%    |\lccode| (see Appendix H, the \TeX book). As some of the othe
%    language definitions turn the character |^| into a shorthand w
%    need to make sure that it has it's orginial definition here
% \changes{usorbian-1.0k}{2008/03/17}{Make sure the caret has th
%    right \cs{catcdoe}}
%    \begin{macrocode
\begingroup \catcode`\^
\def\x{\endgrou
  \expandafter\addto\csname extras\CurrentOption\endcsname{
    \babel@savevariable{\lccode`\^^Y}
    \lccode`\^^Y`\^^Y}
\
%    \end{macrocode
%    The umlaut accent macro |\"| is changed to lower the umlaut dots
%    The redefinition is done with the help of |\umlautlow|
%    \begin{macrocode
\expandafter\addto\csname extras\CurrentOption\endcsname{
  \babel@save\"\umlautlow
\expandafter\addto\csname noextras\CurrentOption\endcsname{
  \umlauthigh
%    \end{macrocode
%    The Upper Sorbian hyphenation patterns can be used wit
%    |\lefthyphenmin| and |\righthyphenmin| set to~2
% \changes{usorbian-1.0i}{2000/09/22}{Now use \cs{providehyphenmins} t
%    provide a default value
%    \begin{macrocode
\providehyphenmins{\CurrentOption}{\tw@\tw@
%    \end{macrocode
% \end{macro

% \changes{usorbian-1.0a}{1995/05/27}{Removed stuff that has bee
%    moved to \file{babel.def}

%  \begin{macro}{\dq
%    We save the original double quote character in |\dq| to keep i
%    available, the math accent |\"| can now be typed as |"|.  Also w
%    store the original meaning of the command |\"| for future use
%    \begin{macrocode
\begingroup \catcode`\"1
\def\x{\endgrou
  \def\@SS{\mathchar"7019
  \def\dq{"}
\
%    \end{macrocode
% \end{macro

%    Now we can define the doublequote macros: the umlauts
%    \begin{macrocode
\declare@shorthand{usorbian}{"a}{\textormath{\"{a}}{\ddot a}
\declare@shorthand{usorbian}{"o}{\textormath{\"{o}}{\ddot o}
\declare@shorthand{usorbian}{"u}{\textormath{\"{u}}{\ddot u}
\declare@shorthand{usorbian}{"A}{\textormath{\"{A}}{\ddot A}
\declare@shorthand{usorbian}{"O}{\textormath{\"{O}}{\ddot O}
\declare@shorthand{usorbian}{"U}{\textormath{\"{U}}{\ddot U}
%    \end{macrocode
%    tremas
%    \begin{macrocode
\declare@shorthand{usorbian}{"e}{\textormath{\"{e}}{\ddot e}
\declare@shorthand{usorbian}{"E}{\textormath{\"{E}}{\ddot E}
\declare@shorthand{usorbian}{"i}{\textormath{\"{\i}}{\ddot\imath}
\declare@shorthand{usorbian}{"I}{\textormath{\"{I}}{\ddot I}
%    \end{macrocode
%    usorbian es-zet (sharp s)
%    \begin{macrocode
\declare@shorthand{usorbian}{"s}{\textormath{\ss{}}{\@SS{}}
\declare@shorthand{usorbian}{"S}{SS
%    \end{macrocode
%    german and french quotes
% \changes{usorbian-1.0f}{1997/04/03}{Removed empty groups afte
%    double quote and guillemot characters
%    \begin{macrocode
\declare@shorthand{usorbian}{"`}{
  \textormath{\quotedblbase}{\mbox{\quotedblbase}}
\declare@shorthand{usorbian}{"'}{
  \textormath{\textquotedblleft}{\mbox{\textquotedblleft}}
\declare@shorthand{usorbian}{"<}{
  \textormath{\guillemotleft}{\mbox{\guillemotleft}}
\declare@shorthand{usorbian}{">}{
  \textormath{\guillemotright}{\mbox{\guillemotright}}
%    \end{macrocode
%    discretionary command
% \changes{usorbian-1.0c}{1996/01/24}{Now use \cs{bbl@disc}
%    \begin{macrocode
\declare@shorthand{usorbian}{"c}{\textormath{\bbl@disc ck}{c}
\declare@shorthand{usorbian}{"C}{\textormath{\bbl@disc CK}{C}
\declare@shorthand{usorbian}{"f}{\textormath{\bbl@disc f{ff}}{f}
\declare@shorthand{usorbian}{"F}{\textormath{\bbl@disc F{FF}}{F}
\declare@shorthand{usorbian}{"l}{\textormath{\bbl@disc l{ll}}{l}
\declare@shorthand{usorbian}{"L}{\textormath{\bbl@disc L{LL}}{L}
\declare@shorthand{usorbian}{"m}{\textormath{\bbl@disc m{mm}}{m}
\declare@shorthand{usorbian}{"M}{\textormath{\bbl@disc M{MM}}{M}
\declare@shorthand{usorbian}{"n}{\textormath{\bbl@disc n{nn}}{n}
\declare@shorthand{usorbian}{"N}{\textormath{\bbl@disc N{NN}}{N}
\declare@shorthand{usorbian}{"p}{\textormath{\bbl@disc p{pp}}{p}
\declare@shorthand{usorbian}{"P}{\textormath{\bbl@disc P{PP}}{P}
\declare@shorthand{usorbian}{"t}{\textormath{\bbl@disc t{tt}}{t}
\declare@shorthand{usorbian}{"T}{\textormath{\bbl@disc T{TT}}{T}
%    \end{macrocode
%    and some additional commands
%    \begin{macrocode
\declare@shorthand{usorbian}{"-}{\nobreak\-\bbl@allowhyphens
\declare@shorthand{usorbian}{"|}{
  \textormath{\nobreak\discretionary{-}{}{\kern.03em}
              \allowhyphens}{}
\declare@shorthand{usorbian}{""}{\hskip\z@skip
%    \end{macrocode

%  \begin{macro}{\mdqon
%  \begin{macro}{\mdqoff
%  \begin{macro}{\ck
%    All that's left to do now is to  define a couple of command
%    for reasons of compatibility with \file{german.sty}
% \changes{usorbian-1.0g}{1998/06/07}{Now use \cs{shorthandon} an
%    \cs{shorthandoff}}
%    \begin{macrocode
\def\mdqon{\shorthandon{"}
\def\mdqoff{\shorthandoff{"}
\def\ck{\allowhyphens\discretionary{k-}{k}{ck}\allowhyphens
%    \end{macrocode
%  \end{macro
%  \end{macro
%  \end{macro

%    The macro |\ldf@finish| takes care of looking for
%    configuration file, setting the main language to be switched o
%    at |\begin{document}| and resetting the category code o
%    \texttt{@} to its original value
% \changes{usorbian-1.0e}{1996/11/03}{Now use \cs{ldf@finish} to wra
%    up}
% \changes{usorbian-1.0j}{2007/10/19}{Make this work for more than on
%    option name
%    \begin{macrocode
\ldf@finish\CurrentOptio
%</code
%    \end{macrocode

% \Final
%
%% \CharacterTabl
%%  {Upper-case    \A\B\C\D\E\F\G\H\I\J\K\L\M\N\O\P\Q\R\S\T\U\V\W\X\Y\
%%   Lower-case    \a\b\c\d\e\f\g\h\i\j\k\l\m\n\o\p\q\r\s\t\u\v\w\x\y\
%%   Digits        \0\1\2\3\4\5\6\7\8\
%%   Exclamation   \!     Double quote  \"     Hash (number) \
%%   Dollar        \$     Percent       \%     Ampersand     \
%%   Acute accent  \'     Left paren    \(     Right paren   \
%%   Asterisk      \*     Plus          \+     Comma         \
%%   Minus         \-     Point         \.     Solidus       \
%%   Colon         \:     Semicolon     \;     Less than     \
%%   Equals        \=     Greater than  \>     Question mark \
%%   Commercial at \@     Left bracket  \[     Backslash     \
%%   Right bracket \]     Circumflex    \^     Underscore    \
%%   Grave accent  \`     Left brace    \{     Vertical bar  \
%%   Right brace   \}     Tilde         \~
%
\endinpu
}
\DeclareOption{welsh}{%%
%% This file will generate fast loadable files and documentation
%% driver files from the doc files in this package when run through
%% LaTeX or TeX.
%%
%% Copyright 1989-2005 Johannes L. Braams and any individual authors
%% listed elsewhere in this file.  All rights reserved.
%% 
%% This file is part of the Babel system.
%% --------------------------------------
%% 
%% It may be distributed and/or modified under the
%% conditions of the LaTeX Project Public License, either version 1.3
%% of this license or (at your option) any later version.
%% The latest version of this license is in
%%   http://www.latex-project.org/lppl.txt
%% and version 1.3 or later is part of all distributions of LaTeX
%% version 2003/12/01 or later.
%% 
%% This work has the LPPL maintenance status "maintained".
%% 
%% The Current Maintainer of this work is Johannes Braams.
%% 
%% The list of all files belonging to the LaTeX base distribution is
%% given in the file `manifest.bbl. See also `legal.bbl' for additional
%% information.
%% 
%% The list of derived (unpacked) files belonging to the distribution
%% and covered by LPPL is defined by the unpacking scripts (with
%% extension .ins) which are part of the distribution.
%%
%% --------------- start of docstrip commands ------------------
%%
\def\filedate{1999/04/11}
\def\batchfile{welsh.ins}
\input docstrip.tex

{\ifx\generate\undefined
\Msg{**********************************************}
\Msg{*}
\Msg{* This installation requires docstrip}
\Msg{* version 2.3c or later.}
\Msg{*}
\Msg{* An older version of docstrip has been input}
\Msg{*}
\Msg{**********************************************}
\errhelp{Move or rename old docstrip.tex.}
\errmessage{Old docstrip in input path}
\batchmode
\csname @@end\endcsname
\fi}

\declarepreamble\mainpreamble
This is a generated file.

Copyright 1989-2005 Johannes L. Braams and any individual authors
listed elsewhere in this file.  All rights reserved.

This file was generated from file(s) of the Babel system.
---------------------------------------------------------

It may be distributed and/or modified under the
conditions of the LaTeX Project Public License, either version 1.3
of this license or (at your option) any later version.
The latest version of this license is in
  http://www.latex-project.org/lppl.txt
and version 1.3 or later is part of all distributions of LaTeX
version 2003/12/01 or later.

This work has the LPPL maintenance status "maintained".

The Current Maintainer of this work is Johannes Braams.

This file may only be distributed together with a copy of the Babel
system. You may however distribute the Babel system without
such generated files.

The list of all files belonging to the Babel distribution is
given in the file `manifest.bbl'. See also `legal.bbl for additional
information.

The list of derived (unpacked) files belonging to the distribution
and covered by LPPL is defined by the unpacking scripts (with
extension .ins) which are part of the distribution.
\endpreamble

\declarepreamble\fdpreamble
This is a generated file.

Copyright 1989-2005 Johannes L. Braams and any individual authors
listed elsewhere in this file.  All rights reserved.

This file was generated from file(s) of the Babel system.
---------------------------------------------------------

It may be distributed and/or modified under the
conditions of the LaTeX Project Public License, either version 1.3
of this license or (at your option) any later version.
The latest version of this license is in
  http://www.latex-project.org/lppl.txt
and version 1.3 or later is part of all distributions of LaTeX
version 2003/12/01 or later.

This work has the LPPL maintenance status "maintained".

The Current Maintainer of this work is Johannes Braams.

This file may only be distributed together with a copy of the Babel
system. You may however distribute the Babel system without
such generated files.

The list of all files belonging to the Babel distribution is
given in the file `manifest.bbl'. See also `legal.bbl for additional
information.

In particular, permission is granted to customize the declarations in
this file to serve the needs of your installation.

However, NO PERMISSION is granted to distribute a modified version
of this file under its original name.

\endpreamble

\keepsilent

\usedir{tex/generic/babel} 

\usepreamble\mainpreamble
\generate{\file{welsh.ldf}{\from{welsh.dtx}{code}}
          }
\usepreamble\fdpreamble

\ifToplevel{
\Msg{***********************************************************}
\Msg{*}
\Msg{* To finish the installation you have to move the following}
\Msg{* files into a directory searched by TeX:}
\Msg{*}
\Msg{* \space\space All *.def, *.fd, *.ldf, *.sty}
\Msg{*}
\Msg{* To produce the documentation run the files ending with}
\Msg{* '.dtx' and `.fdd' through LaTeX.}
\Msg{*}
\Msg{* Happy TeXing}
\Msg{***********************************************************}
}
 
\endinput
}
%    \end{macrocode}
% \changes{babel~3.6e}{1997/01/08}{Added options \Lopt{UKenglish} and
%    \Lopt{USenglish}}
%    \begin{macrocode}
\DeclareOption{UKenglish}{%%
%% This file will generate fast loadable files and documentation
%% driver files from the doc files in this package when run through
%% LaTeX or TeX.
%%
%% Copyright 1989-2005 Johannes L. Braams and any individual authors
%% listed elsewhere in this file.  All rights reserved.
%% 
%% This file is part of the Babel system.
%% --------------------------------------
%% 
%% It may be distributed and/or modified under the
%% conditions of the LaTeX Project Public License, either version 1.3
%% of this license or (at your option) any later version.
%% The latest version of this license is in
%%   http://www.latex-project.org/lppl.txt
%% and version 1.3 or later is part of all distributions of LaTeX
%% version 2003/12/01 or later.
%% 
%% This work has the LPPL maintenance status "maintained".
%% 
%% The Current Maintainer of this work is Johannes Braams.
%% 
%% The list of all files belonging to the LaTeX base distribution is
%% given in the file `manifest.bbl. See also `legal.bbl' for additional
%% information.
%% 
%% The list of derived (unpacked) files belonging to the distribution
%% and covered by LPPL is defined by the unpacking scripts (with
%% extension .ins) which are part of the distribution.
%%
%% --------------- start of docstrip commands ------------------
%%
\def\filedate{1999/04/11}
\def\batchfile{english.ins}
\input docstrip.tex

{\ifx\generate\undefined
\Msg{**********************************************}
\Msg{*}
\Msg{* This installation requires docstrip}
\Msg{* version 2.3c or later.}
\Msg{*}
\Msg{* An older version of docstrip has been input}
\Msg{*}
\Msg{**********************************************}
\errhelp{Move or rename old docstrip.tex.}
\errmessage{Old docstrip in input path}
\batchmode
\csname @@end\endcsname
\fi}

\declarepreamble\mainpreamble
This is a generated file.

Copyright 1989-2005 Johannes L. Braams and any individual authors
listed elsewhere in this file.  All rights reserved.

This file was generated from file(s) of the Babel system.
---------------------------------------------------------

It may be distributed and/or modified under the
conditions of the LaTeX Project Public License, either version 1.3
of this license or (at your option) any later version.
The latest version of this license is in
  http://www.latex-project.org/lppl.txt
and version 1.3 or later is part of all distributions of LaTeX
version 2003/12/01 or later.

This work has the LPPL maintenance status "maintained".

The Current Maintainer of this work is Johannes Braams.

This file may only be distributed together with a copy of the Babel
system. You may however distribute the Babel system without
such generated files.

The list of all files belonging to the Babel distribution is
given in the file `manifest.bbl'. See also `legal.bbl for additional
information.

The list of derived (unpacked) files belonging to the distribution
and covered by LPPL is defined by the unpacking scripts (with
extension .ins) which are part of the distribution.
\endpreamble

\declarepreamble\fdpreamble
This is a generated file.

Copyright 1989-2005 Johannes L. Braams and any individual authors
listed elsewhere in this file.  All rights reserved.

This file was generated from file(s) of the Babel system.
---------------------------------------------------------

It may be distributed and/or modified under the
conditions of the LaTeX Project Public License, either version 1.3
of this license or (at your option) any later version.
The latest version of this license is in
  http://www.latex-project.org/lppl.txt
and version 1.3 or later is part of all distributions of LaTeX
version 2003/12/01 or later.

This work has the LPPL maintenance status "maintained".

The Current Maintainer of this work is Johannes Braams.

This file may only be distributed together with a copy of the Babel
system. You may however distribute the Babel system without
such generated files.

The list of all files belonging to the Babel distribution is
given in the file `manifest.bbl'. See also `legal.bbl for additional
information.

In particular, permission is granted to customize the declarations in
this file to serve the needs of your installation.

However, NO PERMISSION is granted to distribute a modified version
of this file under its original name.

\endpreamble

\keepsilent

\usedir{tex/generic/babel} 

\usepreamble\mainpreamble
\generate{\file{english.ldf}{\from{english.dtx}{code}}
          }
\usepreamble\fdpreamble

\ifToplevel{
\Msg{***********************************************************}
\Msg{*}
\Msg{* To finish the installation you have to move the following}
\Msg{* files into a directory searched by TeX:}
\Msg{*}
\Msg{* \space\space All *.def, *.fd, *.ldf, *.sty}
\Msg{*}
\Msg{* To produce the documentation run the files ending with}
\Msg{* '.dtx' and `.fdd' through LaTeX.}
\Msg{*}
\Msg{* Happy TeXing}
\Msg{***********************************************************}
}
 
\endinput
}
\DeclareOption{USenglish}{%%
%% This file will generate fast loadable files and documentation
%% driver files from the doc files in this package when run through
%% LaTeX or TeX.
%%
%% Copyright 1989-2005 Johannes L. Braams and any individual authors
%% listed elsewhere in this file.  All rights reserved.
%% 
%% This file is part of the Babel system.
%% --------------------------------------
%% 
%% It may be distributed and/or modified under the
%% conditions of the LaTeX Project Public License, either version 1.3
%% of this license or (at your option) any later version.
%% The latest version of this license is in
%%   http://www.latex-project.org/lppl.txt
%% and version 1.3 or later is part of all distributions of LaTeX
%% version 2003/12/01 or later.
%% 
%% This work has the LPPL maintenance status "maintained".
%% 
%% The Current Maintainer of this work is Johannes Braams.
%% 
%% The list of all files belonging to the LaTeX base distribution is
%% given in the file `manifest.bbl. See also `legal.bbl' for additional
%% information.
%% 
%% The list of derived (unpacked) files belonging to the distribution
%% and covered by LPPL is defined by the unpacking scripts (with
%% extension .ins) which are part of the distribution.
%%
%% --------------- start of docstrip commands ------------------
%%
\def\filedate{1999/04/11}
\def\batchfile{english.ins}
\input docstrip.tex

{\ifx\generate\undefined
\Msg{**********************************************}
\Msg{*}
\Msg{* This installation requires docstrip}
\Msg{* version 2.3c or later.}
\Msg{*}
\Msg{* An older version of docstrip has been input}
\Msg{*}
\Msg{**********************************************}
\errhelp{Move or rename old docstrip.tex.}
\errmessage{Old docstrip in input path}
\batchmode
\csname @@end\endcsname
\fi}

\declarepreamble\mainpreamble
This is a generated file.

Copyright 1989-2005 Johannes L. Braams and any individual authors
listed elsewhere in this file.  All rights reserved.

This file was generated from file(s) of the Babel system.
---------------------------------------------------------

It may be distributed and/or modified under the
conditions of the LaTeX Project Public License, either version 1.3
of this license or (at your option) any later version.
The latest version of this license is in
  http://www.latex-project.org/lppl.txt
and version 1.3 or later is part of all distributions of LaTeX
version 2003/12/01 or later.

This work has the LPPL maintenance status "maintained".

The Current Maintainer of this work is Johannes Braams.

This file may only be distributed together with a copy of the Babel
system. You may however distribute the Babel system without
such generated files.

The list of all files belonging to the Babel distribution is
given in the file `manifest.bbl'. See also `legal.bbl for additional
information.

The list of derived (unpacked) files belonging to the distribution
and covered by LPPL is defined by the unpacking scripts (with
extension .ins) which are part of the distribution.
\endpreamble

\declarepreamble\fdpreamble
This is a generated file.

Copyright 1989-2005 Johannes L. Braams and any individual authors
listed elsewhere in this file.  All rights reserved.

This file was generated from file(s) of the Babel system.
---------------------------------------------------------

It may be distributed and/or modified under the
conditions of the LaTeX Project Public License, either version 1.3
of this license or (at your option) any later version.
The latest version of this license is in
  http://www.latex-project.org/lppl.txt
and version 1.3 or later is part of all distributions of LaTeX
version 2003/12/01 or later.

This work has the LPPL maintenance status "maintained".

The Current Maintainer of this work is Johannes Braams.

This file may only be distributed together with a copy of the Babel
system. You may however distribute the Babel system without
such generated files.

The list of all files belonging to the Babel distribution is
given in the file `manifest.bbl'. See also `legal.bbl for additional
information.

In particular, permission is granted to customize the declarations in
this file to serve the needs of your installation.

However, NO PERMISSION is granted to distribute a modified version
of this file under its original name.

\endpreamble

\keepsilent

\usedir{tex/generic/babel} 

\usepreamble\mainpreamble
\generate{\file{english.ldf}{\from{english.dtx}{code}}
          }
\usepreamble\fdpreamble

\ifToplevel{
\Msg{***********************************************************}
\Msg{*}
\Msg{* To finish the installation you have to move the following}
\Msg{* files into a directory searched by TeX:}
\Msg{*}
\Msg{* \space\space All *.def, *.fd, *.ldf, *.sty}
\Msg{*}
\Msg{* To produce the documentation run the files ending with}
\Msg{* '.dtx' and `.fdd' through LaTeX.}
\Msg{*}
\Msg{* Happy TeXing}
\Msg{***********************************************************}
}
 
\endinput
}
%    \end{macrocode}
%
%    For all those languages for which the option name is the same as
%    the name of the language specific file we specify a default
%    option, which tries to load the file specified. If this doesn't
%    succeed an error is signalled.
% \changes{babel~3.6i}{1997/03/12}{Added default option}
%    \begin{macrocode}
\DeclareOption*{%
  \InputIfFileExists{\CurrentOption.ldf}{}{%
    \PackageError{babel}{%
      Language definition file \CurrentOption.ldf not found}{%
      Maybe you misspelled the language option?}}%
  }
%    \end{macrocode}
%    Another way to extend the list of `known' options for \babel\ is
%    to create the file \file{bblopts.cfg} in which one can add
%    option declarations.
% \changes{babel~3.6i}{1997/03/15}{Added the possibility to have a
%    \file{bblopts.cfg} file with option declarations.}
%    \begin{macrocode}
\InputIfFileExists{bblopts.cfg}{%
  \typeout{*************************************^^J%
           * Local config file bblopts.cfg used^^J%
           *}%
  }{}
%    \end{macrocode}
%
%    Apart from all the language options we also have a few options
%    that influence the behaviour of language definition files.
%
%    The following options don't do anything themselves, they are just
%    defined in order to make it possible for language definition
%    files to check if one of them was specified by the user.
% \changes{babel~3.5d}{1995/07/04}{Added options to influence
%    behaviour of active acute and grave accents}
%    \begin{macrocode}
\DeclareOption{activeacute}{}
\DeclareOption{activegrave}{}
%    \end{macrocode}
%    The next option tells \babel\ to leave shorthand characters
%    active at the end of processing the package. This is \emph{not}
%    the default as it can cause problems with other packages, but for
%    those who want to use the shorthand characters in the preamble of
%    their documents this can help.
% \changes{babel~3.6f}{1997/01/14}{Added option
%    \Lopt{KeepShorthandsActive}}
% \changes{babel~3.7a}{1997/03/21}{No longer define the control
%    sequence \cs{KeepShorthandsActive}}
%    \begin{macrocode}
\DeclareOption{KeepShorthandsActive}{}
%    \end{macrocode}
%
%    The options have to be processed in the order in which the user
%    specified them:
%    \begin{macrocode}
\ProcessOptions*
%    \end{macrocode}
% \changes{babel~3.7c}{1999/03/13}{Added an error message for when no
%    language option was specified}
%    In order to catch the case where the user forgot to specify a
%    language we check whether |\bbl@main@language|, has become
%    defined. If not, no language has been loaded and an error
%    message is displayed.
% \changes{babel~3.7c}{1999/04/09}{No longer us a redefinition of an
%    internal macro, just check \cs{bbl@main@language} and load
%    \file{babel.def}}
%    \begin{macrocode}
\ifx\bbl@main@language\@undefined
  \PackageError{babel}{%
    You haven't specified a language option}{%
    You need to specify a language, either as a global
    option\MessageBreak
    or as an optional argument to the \string\usepackage\space
    command; \MessageBreak
    You shouldn't try to proceed from here, type x to quit.}
%    \end{macrocode}
%    To prevent undefined command errors when the user insists on
%    continuing we load \file{babel.def} here. He should expect more
%    errors though.
%    \begin{macrocode}
  % \iffalse
% This document requires lualatex
%%
%% Copyright (C) 2012-2017 Javier Bezos and Johannes L. Braams.
%% Copyright (C) 1989-2012 Johannes L. Braams and
%%           any individual authors listed elsewhere in this file. 
%% All rights reserved.
%% 
%% This file is part of the Babel system.
%% --------------------------------------
%% 
%% It may be distributed and/or modified under the
%% conditions of the LaTeX Project Public License, either version 1.3
%% of this license or (at your option) any later version.
%% The latest version of this license is in
%%   http://www.latex-project.org/lppl.txt
%% and version 1.3 or later is part of all distributions of LaTeX
%% version 2003/12/01 or later.
%% 
%% This work has the LPPL maintenance status "maintained".
%% 
%% The Current Maintainer of this work is Javier Bezos.
%% 
%% The list of derived (unpacked) files belonging to the distribution
%% and covered by LPPL is defined by the unpacking scripts (with
%% extension |.ins|) which are part of the distribution.
%%
% \fi
%
% \CheckSum{5011}
%
% \iffalse
%<*filedriver>
\ProvidesFile{babel.dtx}[2017/11/03 v3.15 The Babel package]
\documentclass{ltxdoc}
\GetFileInfo{babel.dtx}
\usepackage{fontspec}
\setmainfont[Scale=.88]{Noto Serif}
\setsansfont[Scale=.88]{Noto Sans}
\setmonofont[Scale=.88,FakeStretch=.95]{Noto Mono}
\raggedright
\addtolength{\textwidth}{25pt}
\addtolength{\textheight}{3.5cm}
\addtolength{\topmargin}{-2cm}
\font\manual=logo10 % font used for the METAFONT logo, etc.
\newcommand*\MF{{\manual META}\-{\manual FONT}}
\newcommand*\babel{\textsf{babel}}
\newcommand*\Babel{\textsf{Babel}}
\newcommand*\xetex{\textsf{xetex}}
\newcommand*\luatex{\textsf{luatex}}
\newcommand*\nb[1]{}
\newcommand*\m[1]{\mbox{$\langle$\normalfont\itshape#1\/$\rangle$}}
\newcommand*\langlist{%
  \meta{language}\texttt{,}\meta{language}\texttt{,}...}
\newcommand*\langvar{\m{lang}}
\newcommand*\Lopt[1]{\textsf{#1}}
\newcommand*\Lenv[1]{\texttt{#1}}
\newcommand*\menv[1]{\char`\{#1\char`\}}
\newcommand*\Eenv[1]{%
  \quad\ldots\quad   
  \texttt{\color{thered}\string\end\menv{#1}}}
\newcommand*\file[1]{\texttt{#1}}
\newcommand*\cls[1]{\texttt{#1}}
\newcommand*\pkg[1]{\texttt{#1}}
\addtolength{\oddsidemargin}{1em}
\setlength{\leftmargini}{1.5em}
\usepackage{framed}
\usepackage{multicol}
\usepackage{color,colortbl}
\usepackage[linkbordercolor={.9 .7 .5}]{hyperref}
\newcommand\New[1]{%
  \colorbox[rgb]{.92, .86, .73}{New #1}\enspace\ignorespaces}
\definecolor{thered}  {rgb}{0.65,0.04,0.07}
\definecolor{thegrey} {gray}{0.8}
\definecolor{shadecolor}{rgb}{1,1,0.97}
\definecolor{messages}{rgb}{.66,.13,.27}
\makeatletter
\def\@begintheorem#1#2{%
  \list{}{}%
  \global\advance\@listdepth\m@ne
  \item[{\sffamily\bfseries\color{messages}\hspace*{1.3em}%
        \MakeUppercase{#1}}]}%
\makeatother
\newtheorem{warning}{Warning}
\newtheorem{note}{Note}
\newtheorem{example}{Example}
\let\bblxv\verbatim
\let\bblexv\endverbatim
\def\verbatim{\begin{shaded*}\bblxv\vskip-\baselineskip\vskip2.5\parsep}
\def\endverbatim{\bblexv\vskip-2\baselineskip\end{shaded*}}
\catcode`\_=\active
\def_{\bgroup\let_\egroup\color{thered}}
\def\MacroFont{\fontencoding \encodingdefault \fontfamily\ttdefault
  \fontseries\mddefault \fontshape\updefault \small \catcode`\_=\active}
\definecolor{shadecolor}{rgb}{0.96,0.96,0.93}
\def\PrintDescribeMacro#1{%
  \strut\MacroFont\color{thered}\normalsize\string#1}
\def\Describe#1{%
  \par\penalty-500\vskip3ex\noindent
  \DescribeMacro{#1}\args}
\def\DescribeOther{\vskip-4ex\Describe}
\makeatletter
\let\saved@check@percent\check@percent
\let\check@percent\relax
\def\args#1{%
  \def\bbl@tempa{#1}%
  \ifx\bbl@tempa\@empty\else#1\vskip1ex\fi\ignorespaces}
\begingroup % Changes to ltxdoc
  \catcode`\<\active
  \catcode`\>\active
  \gdef\check@plus@etc{%
    \let\bbl@next\pm@module
    \ifx*\next
      \let\bbl@next\star@module
    \else\ifx/\next
      \let\bbl@next\slash@module
    \else\ifx<\next
      \let\bbl@next\var@module
    \fi\fi\fi
    \bbl@next}
  \gdef\var@module#1#2#3>>{%
    $\langle$\pm@module#2#3>$\rangle$%
    \ifx*#2\ $\equiv$\fi}
\endgroup
\renewcommand*\l@section[2]{%
  \ifnum \c@tocdepth >\z@
    \addpenalty\@secpenalty
    \addvspace{1.0em \@plus\p@}%
    \setlength\@tempdima{2em}%
    \begingroup
      \parindent \z@ \rightskip \@pnumwidth
      \parfillskip -\@pnumwidth
      \leavevmode \bfseries
      \advance\leftskip\@tempdima
      \hskip -\leftskip
      #1\nobreak\hfil \nobreak\hb@xt@\@pnumwidth{\hss #2}\par
    \endgroup
  \fi}
\renewcommand*\l@subsection{\@dottedtocline{2}{2em}{3em}}
\renewcommand*\l@subsubsection{\@dottedtocline{3}{5em}{4em}}
\renewcommand*\l@paragraph{\@dottedtocline{4}{9em}{4.5em}}
\renewcommand\partname{Part}
\def\@pnumwidth{3em}
\makeatother
\begin{document}
\title{Babel, a multilingual package for use with \LaTeX's standard
   document classes\thanks{During the development ideas from Nico
   Poppelier, Piet van Oostrum and many others have been used.
   Bernd Raichle has provided many helpful suggestions.}}
\author{Johannes Braams\\
        Kersengaarde 33\\
        2723 BP Zoetermeer\\
        The Netherlands\\
        \normalsize From version 3.9 on, Javier Bezos\\
        \normalsize \texttt{www.texnia.com}}

\date{Typeset \today}
\begin{titlepage}
\begin{minipage}[t][0pt]{30cm}
\vspace{-3cm}\hspace{-7cm}
\sffamily
\begin{tabular}{p{8cm}p{15cm}}
\cellcolor[rgb]{.86,.73,.67}
&\cellcolor[rgb]{.95,.95,.95}
\vspace{3.6cm}%
\color[rgb]{.55,.4,.35}
\leftskip5mm
\sffamily\fontsize{72}{72}\selectfont Babel
\vspace{1.8cm}
\\
\cellcolor[rgb]{.95,.95,.95}
\vspace{2cm}\hspace{1.5cm}
\begin{minipage}{5cm}
\Large
Version \csname @gobble\expandafter\endcsname\fileversion\newline
\filedate

\vspace{1cm}
\textit{Original author}\newline
Johannes L. Braams

\vspace{.3cm}
\textit{Current maintainer}\newline
Javier Bezos
\end{minipage}
&\cellcolor[rgb]{.92, .86, .73}
\vspace{2cm}
\leftskip5mm
\begin{minipage}{10cm}
\large\setlength\parskip{3mm}\raggedright
  The standard distribution of \LaTeX\ contains a number of document
  classes that are meant to be used, but also serve as examples for
  other users to create their own document classes.  These document
  classes have become very popular among \LaTeX\ users. But it should
  be kept in mind that they were designed for American tastes and
  typography. At one time they even contained a number of hard-wired
  texts.
  
  This manual describes \babel{}, a package that makes use of the
  capabilities of \TeX\ version 3 and, to some extent, \xetex{} and
  \luatex, to provide an environment in which documents can be typeset
  in a language other than US English, or in more than one language or
  script.

  Current development is focused on Unicode engines (Xe\TeX{} and
  Lua\TeX) and the so-called \textit{complex scripts}.  New features
  related to font selection, bidi writing and the like will be added
  incrementally.

  \Babel{} provides support (total or partial) for about 200
  languages, either as a “classical” package option or as an
  |ini| file. Furthermore, new languages can be created from scratch
  easily.

 \vspace{20cm}
\end{minipage}
\end{tabular}
\end{minipage}
\end{titlepage}

\tableofcontents
\clearpage

\part{User guide}

This user guide focuses on \LaTeX. There are also some notes on its
use with Plain \TeX.

If you are interested in the \TeX{} multilingual support, please join the
\textsf{kadingira} list on \texttt{http://tug.org/mailman/listinfo/kadingira}.

\section{The user interface}\label{U-I}

\subsection{Monolingual documents}

In most cases, a single language is required, and then all you need in
\LaTeX{} is to load the package using its standand mechanism for this
purpose, namely, passing that language as an optional argument. In
addition, you may want to set the font and input encodings.

\begin{example}
  Here is a simple full example for “traditional” \TeX{} engines (see
  below for \xetex{} and \luatex{}). The packages |fontenc| and
  |inputenc| do not belong to \babel, but they are included in the
  example because typically you will need them:
\begin{verbatim}
\documentclass{article}

\usepackage[T1]{fontenc}
\usepackage[utf8]{inputenc}

_\usepackage[french]{babel}_

\begin{document}

Plus ça change, plus c'est la même chose!

\end{document}
\end{verbatim}
\end{example}
\begin{warning} 
A common source of trouble is a wrong setting of the input
encoding. Make sure you set the encoding actually used by your editor.
\end{warning}

Another approach is making the language (\Lopt{french} in the example)
a global option in order to let other packages detect and use it:
\begin{verbatim}
_\documentclass[french]{article}_
\usepackage{babel}
\usepackage{varioref}
\end{verbatim}

In this last example, the package \texttt{varioref} will also see
the option and will be able to use it.

\begin{note}
  Because of the way \babel{} has evolved, ``language'' can refer to
  (1) a set of hyphenation patterns as preloaded into the format, (2)
  a package option, (3) an |ldf| file, and (4) a name used in the
  document to select a language or dialect. So, a package option
  refers to a language in a generic way -- sometimes it is the actual
  language name used to select it, sometimes it is a file name loading
  a language with a different name, sometimes it is a file name
  loading several languages. Please, read the documentation for
  specific languages for further info.
\end{note}

\subsection{Multilingual documents}

In multilingual documents, just use several options. The last one is
considered the main language, activated by default. Sometimes, the 
main language changes the document layout (eg, |spanish| and |french|).

\begin{example}
  In \LaTeX, the preamble of the document:
\begin{verbatim}
\documentclass{article}
\usepackage[dutch,english]{babel}
\end{verbatim}
  would tell \LaTeX\ that the document would be written in two
  languages, Dutch and English, and that English would be the first
  language in use, and the main one.
\end{example}

You can also set the main language explicitly:
\begin{verbatim}
\documentclass{article}
\usepackage[_main=english_,dutch]{babel}
\end{verbatim}

\begin{warning}
  Languages may be set as global and as package option at the same
  time, but in such a case you should set explicitly the main language
  with the package option |main|:
\begin{verbatim}
\documentclass[_italian_]{book}
\usepackage[ngerman,_main=italian_]{babel}
\end{verbatim}
\end{warning}

\begin{warning}
  In the preamble the main language has \textit{not} been selected,
  except hyphenation patterns and the name assigned to |\languagename|
  (in particular, shorthands, captions and date are not activated). If
  you need to define boxes and the like in the preamble, you might
  want to use some of the language selectors described below.
\end{warning}

To switch the language there are two basic macros, decribed below in
detail: |\selectlanguage| is used for blocks of text, while
|\foreignlanguage| is for chunks of text inside paragraphs.

\begin{example}
A full bilingual document follows. The main language is |french|,
which is activated when the document begins.
\begin{verbatim}
\documentclass{article}

\usepackage[T1]{fontenc}
\usepackage[utf8]{inputenc}

_\usepackage[english,french]{babel}_

\begin{document}

Plus ça change, plus c'est la même chose!

_\selectlanguage{english}_

And an English paragraph, with a short text in
_\foreignlanguage{french}{français}_.

\end{document}
\end{verbatim}
\end{example}

\subsection{Modifiers}

\New{3.9c} The basic behaviour of some languages can be modified when
loading \babel{} by means of \textit{modifiers}. They are set after
the language name, and are prefixed with a dot (only when the language
is set as package option -- neither global options nor the |main| key
accept them). An example is (spaces are not significant and they can
be added or removed):\footnote{No predefined ``axis'' for modifiers
are provided because languages and their scripts have quite different
needs.}
\begin{verbatim}
\usepackage[latin_.medieval_, spanish_.notilde.lcroman_, danish]{babel}
\end{verbatim}

Attributes (described below) are considered modifiers, ie, you can
set an attribute by including it in the list of modifiers. However,
modifiers is a more general mechanism.

\subsection{\textsf{xelatex} and \textsf{lualatex}}

Many languages are compatible with \textsf{xetex} and \textsf{luatex}.
With them you can use \babel{} to localize the documents.

The Latin script is covered by default in current \LaTeX{} (provided
the document encoding is UTF-8), because the font loader is preloaded
and the font is switched to |lmroman|. Other scripts require loading
\textsf{fontspec}.

\begin{example}
  The following bilingual, single script document in UTF-8 encoding
  just prints a couple of ‘captions’ and |\today| in Danish and
  Vietnamese. No additional packages are required.
\begin{verbatim}
\documentclass{article}

_\usepackage[vietnamese,danish]{babel}_

\begin{document}

\prefacename{} -- \alsoname{} -- \today

\selectlanguage{vietnamese}

\prefacename{} -- \alsoname{} -- \today

\end{document}
\end{verbatim}
\end{example}

\begin{example}
Here is a simple monolingual document in Russian (text from the
Wikipedia). Note neither \textsf{fontenc} nor \textsf{inputenc} are
necessary, but the document should be encoded in UTF-8 and a
so-called Unicode font must be loaded (in this example |\babelfont| is
used, described below).

\begin{verbatim}
\documentclass{article}

_\usepackage[russian]{babel}_

\babelfont{rm}{DejaVu Serif}

\begin{document}

Россия, находящаяся на пересечении множества культур, а также
с учётом многонационального характера её населения, — отличается
высокой степенью этнокультурного многообразия и способностью к
межкультурному диалогу.

\end{document}
\end{verbatim}
\end{example}

\subsection{Troubleshooting}

\begin{itemize}
\item Loading directly |sty| files in \LaTeX{} (ie,
  |\usepackage|\marg{language}) is deprecated and you will get the
  error:\footnote{In old versions the error read ``You have used an
  old interface to call babel'', not very helpful.}
\begin{verbatim}
! Package babel Error: You are loading directly a language style.
(babel)                This syntax is deprecated and you must use
(babel)                \usepackage[language]{babel}.
\end{verbatim}

\item  Another typical error when using \babel{} is the
  following:\footnote{In old versions the error read ``You haven't
  loaded the language LANG yet''.}
\begin{verbatim}
! Package babel Error: Unknown language `LANG'. Either you have misspelled
(babel)                its name, it has not been installed, or you requested
(babel)                it in a previous run. Fix its name, install it or just
(babel)                rerun the file, respectively
\end{verbatim}
  The most frequent reason is, by far, the latest (for example, you
  included |spanish|, but you realized this language is not used after
  all, and therefore you removed it from the option list). In most
  cases, the error vanishes when the document is typeset again, but in
  more severe ones you will need to remove the |aux| file.

\item The following warning is about hyphenation patterns, which are
  not under the direct control of \babel:
\begin{verbatim}
Package babel Warning: No hyphenation patterns were preloaded for
(babel)                the language `LANG' into the format.
(babel)                Please, configure your TeX system to add them and
(babel)                rebuild the format. Now I will use the patterns
(babel)                preloaded for \language=0 instead on input line 57.
\end{verbatim}
  The document will be typeset, but very likely the text will not be
  correctly hyphenated. Some languages may be raising this warning
  wrongly (because they are not hyphenated); it is a bug to be fixed
  -- just ignore it. See the manual of your distribution (Mac\TeX,
  Mik\TeX, \TeX Live, etc.) for further info about how to configure
  it.
\end{itemize}
\subsection{Plain}

In Plain, load languages styles with |\input| and then use
|\begindocument| (the latter is defined by \babel):
\begin{verbatim}
\input estonian.sty
\begindocument
\end{verbatim}

\begin{warning}
  Not all languages provide a |sty| file and some of them are not
  compatible with Plain.\footnote{Even in the \babel{} kernel there
  were some macros not compatible with plain. Hopefully these issues
  will be fixed soon.}
\end{warning}

\subsection{Basic language selectors}

This section describes the commands to be used in the document to
switch the language in multilingual documents. In most cases, only the
two basic macros |\selectlanguage| and |\foreignlanguage| are
necessary. The environments |otherlanguage|, |otherlanguage*| and
|hyphenrules| are auxiliary, and described in the next section.

The main language is selected automatically when the |document|
environment begins. 

\Describe\selectlanguage{\marg{language}}
When a user wants to switch from one language to another he can
do so using the macro |\selectlanguage|. This macro takes the
language, defined previously by a language definition file, as
its argument. It calls several macros that should be defined in
the language definition files to activate the special definitions
for the language chosen:
\begin{verbatim}
\selectlanguage{german}
\end{verbatim}

This command can be used as environment, too.

\begin{note}
  For ``historical reasons'', a macro name is converted to a language
  name without the leading |\|; in other words,
  |\selectlanguage{\german}| is equivalent to |\selectlanguage{german}|.
  Using a macro instead of a ``real'' name is deprecated.
\end{note}

\begin{warning}
  If used inside braces there might be some non-local changes, as this
  would be roughly equivalent to:
\begin{verbatim}
{\selectlanguage{<inner-language>} ...}\selectlanguage{<outer-language>}
\end{verbatim}
  If you want a change which is really local, you must enclose this
  code with an additional grouping level.
\end{warning}

\Describe\foreignlanguage{\marg{language}\marg{text}}
The command |\foreignlanguage| takes two arguments; the second
argument is a phrase to be typeset according to the rules of the
language named in its first one. This command (1) only switches the
extra definitions and the hyphenation rules for the language,
\emph{not} the names and dates, (2) does not send information about
the language to auxiliary files (i.e., the surrounding language is
still in force), and (3) it works even if the language has not been
set as package option (but in such a case it only sets the hyphenation
patterns and a warning is shown).

\subsection{Auxiliary language selectors}

\Describe{\begin\menv{otherlanguage}}{\marg{language}\Eenv{otherlanguage}}

The environment \Lenv{otherlanguage} does basically the same as
|\selectlanguage|, except the language change is (mostly) local to
the environment. 

Actually, there might be some non-local changes, as this environment
is roughly equivalent to:
\begin{verbatim}
\begingroup
\selectlanguage{<inner-language>}
...
\endgroup
\selectlanguage{<outer-language>}
\end{verbatim}
If you want a change which is really local, you must enclose this
environment with an additional grouping, like braces |{}|.

Spaces after the environment are ignored.

\Describe{\begin\menv{otherlanguage*}}%
{\marg{language}\Eenv{otherlanguage*}}

Same as |\foreignlanguage| but as environment. Spaces after the
environment are \textit{not} ignored.

This environment was originally intended for intermixing left-to-right
typesetting with right-to-left typesetting in engines not supporting a
change in the writing direction inside a line. However, by default it never
complied with the documented behaviour and it is just a version as
environment of |\foreignlanguage|.

\Describe{\begin\menv{hyphenrules}}{\marg{language}\Eenv{hyphenrules}}

The environment \Lenv{hyphenrules} can be used to select \emph{only}
the hyphenation rules to be used (it can be used as command,
too). This can for instance be used to select `nohyphenation',
provided that in \file{language.dat} the `language'
\textsf{nohyphenation} is defined by loading \file{zerohyph.tex}. It
deactivates language shorthands, too (but not user shorthands).

Except for these simple uses, |hyphenrules| is discouraged and
|otherlanguage*| (the starred version) is preferred, as the former
does not take into account possible changes in encodings of characters
like, say, |'| done by some languages (eg, \textsf{italian},
\textsf{french}, \textsf{ukraineb}). To set hyphenation exceptions,
use |\babelhyphenation| (see below).

\subsection{More on selection}

\Describe\babeltags{\char`\{\m{tag1} \texttt{=} \m{language1}, \m{tag2}
\texttt{=} \m{language2}, \dots\char`\}} 

\New{3.9i} In multilingual documents with many language switches
the commands above can be cumbersome. With this tool shorter names can
be defined. It adds nothing really new -- it is just syntactical
sugar.

It defines |\text|\m{tag1}\marg{text} to be
|\foreignlanguage|\marg{language1}\marg{text}, and |\begin|\marg{tag1}
to be |\begin{otherlanguage*}|\marg{language1}, and so on. Note
|\|\m{tag1} is also allowed, but remember to set it locally inside a
group. 
\begin{example}
With
\begin{verbatim}
\babeltags{de = german}
\end{verbatim}
you can write
\begin{verbatim}
text \textde{German text} text
\end{verbatim}
and
\begin{verbatim}
text
\begin{de}
  German text
\end{de}
text
\end{verbatim}
\end{example}

\begin{note}
  Something like \verb|\babeltags{finnish = finnish}| is legitimate --
  it defines |\textfinnish| and |\finnish| (and, of course,
  |\begin{finnish}|).
\end{note}

\begin{note}
  Actually, there may be another advantage in the ‘short’ syntax |\text|\m{tag},
  namely, it is not affected by |\MakeUppercase| (while
  |\foreignlanguage| is).
\end{note}

\Describe\babelensure{|[include=|\m{commands}|,exclude=|\m{commands}%
  |,fontenc=|\m{encoding}|]|\marg{language}}

\New{3.9i} Except in a few languages, like \textsf{russian},
captions and dates are just strings, and do not switch the
language. That means you should set it explicitly if you want to use
them, or hyphenation (and in some cases the text itself) will be
wrong. For example:
\begin{verbatim}
\foreignlanguage{russian}{text \foreignlanguage{polish}{\seename} text}
\end{verbatim}

Of course, \TeX{} can do it for you. To avoid switching the language
all the while, |\babelensure| redefines the captions for a given
language to wrap them with a selector:
\begin{verbatim}
\babelensure{polish}
\end{verbatim}
By default only the basic captions and |\today| are redefined, but you
can add further macros with the key |include| in the optional argument
(without commas). Macros not to be modified are listed in
|exclude|. You can also enforce a font encoding with
|fontenc|.\footnote{With it encoded string may not work as expected.}
A couple of examples:
\begin{verbatim}
\babelensure[include=\Today]{spanish}
\babelensure[fontenc=T5]{vietnamese}
\end{verbatim}

They are activated when the language is selected (at the |afterextras|
event), and it makes some assumptions which could not be fulfilled in
some languages. Note also you should include only macros defined by
the language, not global macros (eg, |\TeX| of |\dag|).

With |ini| files (see below), captions are ensured by default.

\subsection{Shorthands}

A \textit{shorthand} is a sequence of one or two characters that
expands to arbitrary \TeX{} code.

Shorthands can be used for different kinds of things, as for example:
(1) in some languages shorthands such as |"a| are defined to be able
to hyphenate the word if the encoding is |OT1|; (2) in some languages
shorthands such as |!| are used to insert the right amount of white
space; (3) several kinds of discretionaries and breaks can be inserted
easily with |"-|, |"=|, etc.

The package \textsf{inputenc} as well as \xetex{} an \luatex{} have
alleviated entering non-ASCII characters, but minority languages and
some kinds of text can still require characters not directly available
on the keyboards (and sometimes not even as separated or precomposed
Unicode characters). As to the point 2, now \textsf{pdfTeX} provides
|\knbccode|, and \luatex{} can manipulate the glyph list. Tools for
point 3 can be still very useful in general.

There are three levels of shorthands: \textit{user},
\textit{language}, and \textit{system} (by order of
precedence). Version 3.9 introduces the \textit{language user} level
on top of the user level, as described below. In most cases, you will
use only shorthands provided by languages.

\begin{note} Note the following:
\begin{enumerate}
\item Activated chars used for two-char shorthands cannot be followed
  by a closing brace |}| and the spaces following are gobbled.  With
  one-char shorthands (eg,~|:|), they are preserved.
\item If on a certain level (system, language, user) there is a
  one-char shorthand, two-char ones starting with that 
  char and on the same level are ignored.
\item Since they are active, a shorthand cannot contain the same
  character in its definition (except if it is deactivated with, eg,
  |string|).
\end{enumerate}
\end{note}

A typical error when using shorthands is the following:
\begin{verbatim}
! Argument of \language@active@arg" has an extra }.
\end{verbatim}
It means there is a closing brace just after a shorthand, which is not
allowed (eg,~|"}|). Just add |{}| after (eg,~|"{}}|). 

\Describe{\shorthandon}{\marg{shorthands-list}}
\DescribeOther{\shorthandoff}{%
\colorbox{thegrey}{\ttfamily\hskip-.2em*\hskip-.2em}%
\marg{shorthands-list}}
It is sometimes necessary to switch a shorthand
character off temporarily, because it must be used in an
entirely different way. For this purpose, the user commands
|\shorthandoff| and |\shorthandon| are provided. They each take a
list of characters as their arguments.

The command |\shorthandoff| sets the |\catcode| for each of the
characters in its argument to other (12); the command |\shorthandon|
sets the |\catcode| to active (13). Both commands only work on `known'
shorthand characters. If a character is not known to be a shorthand
character its category code will be left unchanged.

\New{3.9a} However, |\shorthandoff| does not behave as
you would expect with characters like |~| or |^|, because they
usually are not ``other''. For them |\shorthandoff*| is provided,
so that with
\begin{verbatim}
\shorthandoff*{~^}
\end{verbatim}
|~| is still active, very likely with the meaning of a non-breaking
space, and |^| is the superscript character. The catcodes used are
those when the shorthands are defined, usually when language files are
loaded.

\Describe{\useshorthands}{%
\colorbox{thegrey}{\ttfamily\hskip-.2em*\hskip-.2em}%
\marg{char}}

The command |\useshorthands| initiates the definition of user-defined
shorthand sequences. It has one argument, the character that starts
these personal shorthands.

\New{3.9a} User shorthands are not always alive, as they may
be deactivated by languages (for example, if you use |"| for your user
shorthands and switch from \textsf{german} to \textsf{french}, they
stop working). Therefore, a starred version
|\useshorthands*|\marg{char} is provided, which makes sure shorthands
are always activated.

Currently, if the package option |shorthands| is used, you must include any
character to be activated with |\useshorthands|. This restriction will
be lifted in a future release.

\Describe\defineshorthand{\texttt{[}\langlist\texttt{]}%
     \marg{shorthand}\marg{code}}

The command |\defineshorthand| takes two arguments: the first is
a one- or two-character shorthand sequence, and the second is the
code the shorthand should expand to.

\New{3.9a} An optional argument allows to (re)define language and
system shorthands (some languages do not activate shorthands, so you
may want to add |\languageshorthands|\marg{lang} to the corresponding
|\extras|\m{lang}, as explained below). By default, user shorthands
are (re)defined.

User shorthands override language ones, which in turn override
system shorthands. Language-dependent user shorthands (new in
3.9) take precedence over ``normal'' user shorthands.

\begin{example}
  Let's assume you want a unified set of shorthand for discretionaries
  (languages do not define shorthands consistently, and |"-|, |\-|,
  |"=| have different meanings).  You could start with, say:
\begin{verbatim}
\useshorthands*{"}
\defineshorthand{"*}{\babelhyphen{soft}}
\defineshorthand{"-}{\babelhyphen{hard}}
\end{verbatim}
  However, behaviour of hyphens is language dependent. For example, in
  languages like Polish and Portuguese, a hard hyphen inside compound
  words are repeated at the beginning of the next line. You could then
  set:
\begin{verbatim}
\defineshorthand[*polish,*portugese]{"-}{\babelhyphen{repeat}}
\end{verbatim}
  Here, options with |*| set a language-dependent user shorthand,
  which means the generic one above only applies for the rest of
  languages; without |*| they would (re)define the language shorthands
  instead, which are overriden by user ones.

  Now, you have a single unified shorthand (|"-|), with a
  content-based meaning (`compound word hyphen') whose visual behavior
  is that expected in each context.
\end{example}

\Describe\aliasshorthand{\marg{original}\marg{alias}}

The command |\aliasshorthand| can be used to let another
character perform the same functions as the default shorthand
character. If one prefers for example to use the character |/|
over |"| in typing Polish texts, this can be achieved by entering
|\aliasshorthand{"}{/}|. 

\begin{note}
  The substitute character must \textit{not} have been declared before
  as shorthand (in such a case, |\aliashorthands| is ignored).
\end{note}

\begin{example}
  The following example shows how to replace a shorthand by another
\begin{verbatim}
\aliasshorthand{~}{^}
\AtBeginDocument{\shorthandoff*{~}}
\end{verbatim}
\end{example}

\begin{warning}
  Shorthands remember somehow the original character, and the fallback
  value is that of the latter. So, in this example, if no shorthand if
  found, |^| expands to a non-breaking space, because this is the
  value of |~| (internally, |^| still calls |\active@char~| or
  |\normal@char~|). Furthermore, if you change the |system| value of
  |^| with |\defineshorthand| nothing happens.
\end{warning}

\Describe\languageshorthands{\marg{language}} The command
|\languageshorthands| can be used to switch the shorthands on the
language level. It takes one argument, the name of a language or
|none| (the latter does what its name suggests).\footnote{Actually,
any name not corresponding to a language group does the same as
\texttt{none}. However, follow this convention because it might be
enforced in future releases of \babel{} to catch possible errors.}
Note that for this to work the language should have been specified as
an option when loading the \babel\ package.  For example, you can use
in \textsf{english} the shorthands defined by \textsf{ngerman} with
\begin{verbatim}
\addto\extrasenglish{\languageshorthands{ngerman}}
\end{verbatim}
(You may also need to activate them with, for example,
|\useshorthands|.)

Very often, this is a more convenient way to deactivate shorthands
than |\shorthandoff|, as for example if you want to define a macro
to easy typing phonetic characters with \textsf{tipa}:
\begin{verbatim}
\newcommand{\myipa}[1]{{\languageshorthands{none}\tipaencoding#1}}
\end{verbatim}

\Describe\babelshorthand{\marg{shorthand}}
With this command you can use a shorthand even if (1) not activated in
\texttt{shorthands} (in this case only shorthands for the current
language are taken into account, ie, not user shorthands), (2) turned
off with |\shorthandoff| or (3) deactivated with the internal
|\bbl@deactivate|; for example, \verb|\babelshorthand{"u}| or
\verb|\babelshorthand{:}|.  (You can conveniently define your own
macros, or even you own user shorthands provided they do not ovelap.)

For your records, here is a list of shorthands, but you must double
check them, as they may change:\footnote{Thanks to Enrico Gregorio}

\begin{description}
\itemsep=-\parskip
\item[Languages with no shorthands] Croatian, English (any variety),
  Indonesian, Hebrew, Interlingua, Irish, Lower Sorbian, Malaysian,
  North Sami, Romanian, Scottish, Welsh
\item[Languages with only \texttt{"} as defined shorthand character]
  Albanian, Bulgarian, Danish, Dutch, Finnish, German (old and new
  orthography, also Austrian), Icelandic, Italian, Norwegian, Polish,
  Portuguese (also Brazilian), Russian, Serbian (with Latin script),
  Slovene, Swedish, Ukrainian, Upper Sorbian
\item[Basque] |" ' ~|
\item[Breton] |: ; ? !|
\item[Catalan] |" ' `|
\item[Czech] |" -|
\item[Esperanto] |^|
\item[Estonian] |" ~|
\item[French] (all varieties) |: ; ? !|
\item[Galician] |" . ' ~ < >|
\item[Greek] |~|
\item[Hungarian] |`|
\item[Kurmanji] |^|
\item[Latin] |" ^ =|
\item[Slovak] |" ^ ' -|
\item[Spanish] |" . < > '|
\item[Turkish] |: ! =|
\end{description}
In addition, the \babel{} core declares |~| as a one-char shorthand
which is let, like the standard |~|, to a non breaking
space.\footnote{This declaration serves to nothing, but it is
preserved for backward compatibility.}

\subsection{Package options}

\New{3.9a}
These package options are processed before language options, so
that they are taken into account irrespective of its order. The first
three options have been available in previous versions.

\Describe{KeepShorthandsActive}{}
Tells babel not to deactivate shorthands after loading a language
file, so that they are also availabe in the preamble.

\Describe{activeacute}{} For some languages \babel\ supports this
options to set |'| as a shorthand in case it is not done by default.

\Describe{activegrave}{}
Same for |`|.

\Describe{shorthands=}{\meta{char}\meta{char}...
$\string|$ \texttt{off}}
The only language shorthands activated
are those given, like, eg:
\begin{verbatim}
\usepackage[esperanto,french,shorthands=:;!?]{babel}
\end{verbatim} 
If \verb|'| is included, \texttt{activeacute} is set; if \verb|`| is
included, \texttt{activegrave} is set.  Active characters (like
\verb|~|) should be preceded by \verb|\string| (otherwise they will be
expanded by \LaTeX{} before they are passed to the package and
therefore they will not be recognized); however, |t| is provided for
the common case of |~| (as well as |c| for not so common case of the
comma).

With |shorthands=off| no language shorthands are defined,
As some languages use this mechanism for tools not available
otherwise, a macro \verb|\babelshorthand| is defined, which allows
using them; see above.

\Describe{safe=}{\texttt{none} $\string|$ \texttt{ref} $\string|$
\texttt{bib}} Some \LaTeX{} macros are redefined so that using
shorthands is safe. With \texttt{safe=bib} only |\nocite|, |\bibcite|
and |\bibitem| are redefined. With |safe=ref| only |\newlabel|, |\ref|
and |\pageref| are redefined (as well as a few macros from
\textsf{varioref} and \textsf{ifthen}). With |safe=none| no macro is
redefined. This option is strongly recommended, because a good deal of
incompatibilities and errors are related to these redefinitions -- of
course, in such a case you cannot use shorthands in these macros, but
this is not a real problem (just use ``allowed'' characters). 

\Describe{math=}{\texttt{active} $\string|$ \texttt{normal}}
Shorthands are mainly intended for text, not for math. By setting this
option with the value |normal| they are deactivated in math mode
(default is |active|) and things like |${a'}$| (a closing brace after
a shorthand) are not a source of trouble any more.

\Describe{config=}{\meta{file}} Load \meta{file}\texttt{.cfg} instead
of the default config file |bblopts.cfg| (the file is loaded even
with |noconfigs|).

\Describe{main=}{\meta{language}} Sets the main language, as explained
above, ie, this language is always loaded last. If it is not given as
package or global option, it is added to the list of requested
languages.

\Describe{headfoot=}{\meta{language}} By default, headlines and
footlines are not touched (only marks), and if they contain language
dependent macros (which is not usual) there may be unexpected
results. With this option you may set the language in heads and foots.

\Describe{noconfigs}{} Global and language default config files are
not loaded, so you can make sure your document is not spoilt by an
unexpected \texttt{.cfg} file. However, if the key |config| is set,
this file is loaded.

\Describe{showlanguages}{} Prints to the log the list of languages
loaded when the format was created: number (remember dialects can
share it), name, hyphenation file and exceptions file. 

\Describe{nocase}{} \New{3.9l} Language settings for uppercase and
lowercase mapping (as set by |\SetCase|) are ignored. Use only if there
are incompatibilities with other packages.

\Describe{silent}{} \New{3.9l} No warnings and no \textit{infos} are
written to the log file.\footnote{You can use alternatively the
package \textsf{silence}.}

\Describe{strings=}{\texttt{generic} $\string|$ \texttt{unicode}
$\string|$ \texttt{encoded} $\string|$ \meta{label} $\string|$
\meta{font encoding}} Selects the encoding of strings in languages
supporting this feature. Predefined labels are |generic| (for
traditional \TeX, LICR and ASCII strings), |unicode| (for engines like
\xetex{} and \luatex) and |encoded| (for special cases requiring mixed
encodings). Other allowed values are font encoding codes (|T1|, |T2A|,
|LGR|, |L7X|...), but only in languages supporting them. Be aware with
|encoded| captions are protected, but they work in |\MakeUppercase|
and the like (this feature misuses some internal \LaTeX\ tools, so use
it only as a last resort).

\Describe{hyphenmap=}{\texttt{off} $\string|$ \texttt{main}
$\string|$ \texttt{select} $\string|$ \texttt{other} $\string|$
\texttt{other*}}

\New{3.9g} Sets the behaviour of case mapping for hyphenation,
provided the language defines it.\footnote{Turned off in plain.} It
can take the following values:
\begin{description}
\renewcommand\makelabel[1]{%
  \hspace\labelsep\normalfont\ttfamily\color{thered}#1}
\itemsep=-\parskip
\item[off] deactivates this feature and no case mapping is applied;
\item[first] sets it at the first switching commands in the
  current or parent scope (typically, when the aux file is first read
  and at |\begin{document}|, but also the first |\selectlanguage| in
    the preamble), and it's the default if a single
    language option has been stated;\footnote{Duplicated options count
    as several ones.}
\item[select] sets it only at |\selectlanguage|;
\item[other] also sets it at |otherlanguage|;
\item[other*] also sets it at |otherlanguage*| as well as in heads and
  foots (if the option |headfoot| is used) and in auxiliary files (ie,
  at |\select@language|), and it's the default if several language
  options have been stated. The option |first| can be regarded as an
  optimized version of \texttt{other*} for monolingual
  documents.\footnote{Providing |foreign| is pointless, because the
  case mapping applied is that at the end of paragraph, but if either
  \xetex{} or \luatex{} change this behaviour it might be added. On
  the other hand, |other| is provided even if I [JBL] think it isn't
  really useful, but who knows.}
\end{description}

\Describe{bidi=}{\texttt{default} $\string|$ \texttt{basic-r}}

\New{3.14} Selects the bidi algorithm to be used in \luatex{} and
\xetex{}. With |default| the bidi mechanism is just activated (by default
it is not), but every change must by marked up. In \xetex{} this is
the only option. In \luatex, |basic-r|, provides a simple and fast
method for R text, which handles numbers and unmarked L text within an
R context.

\begin{example}
The following text comes from the Arabic Wikipedia (article about
Arabia). Copy-pasting some text from the Wikipedia is a good way to
test this feature, which will be improved in the future. Remember
|basic-r| is available in \luatex{} only.
  \begingroup
% If you are looking at the code to see how it has been written, you
% will be disappointed :-). The following example is built ad hoc to
% emulate the final result to avoid dependencies, and therefore it's
% not "real" code.
\setmonofont[Scale=.87,Script=Arabic]{DejaVu Sans Mono} \catcode`@=13
\def@#1{\ifcase#1\relax \egroup \or \bgroup\textdir TLT \else
\bgroup\textdir TRT \pardir TRT \fi}
\begin{verbatim}
\documentclass{article}

\usepackage[nil, _bidi=basic-r_]{babel}

_\babelprovide[import=ar, main]{arabic}_

\babelfont{rm}{FreeSerif}

\begin{document}

@9وقد عرفت شبه جزيرة العرب طيلة العصر الهيليني )الاغريقي( بـ
@1Arabia@0 أو @1Aravia@0 )بالاغريقية @1Αραβία@0(، استخدم الرومان ثلاث
بادئات بـ@1“Arabia”@0 على ثلاث مناطق من شبه الجزيرة العربية، إلا أنها
حقيقةً كانت أكبر مما تعرف عليه اليوم.

@0\end{document}
\end{verbatim}
\endgroup
\end{example}

\subsection{The \texttt{base} option}

With this package option \babel{} just loads some basic macros (those
in |switch.def|), defines |\AfterBabelLanguage| and exits. It also
selects the hyphenations patterns for the last language passed as
option (by its name in |language.dat|). There are two main uses:
classes and packages, and as a last resort in case there are, for some
reason, incompatible languages. It can be used if you just want to
select the hyphenations patterns of a single language, too.
% TODO: example

\Describe\AfterBabelLanguage{\marg{option-name}\marg{code}}

This command is currently the only provided by |base|. Executes
\meta{code} when the file loaded by the corresponding package option
is finished (at |\ldf@finish|). The setting is global. So
\begin{verbatim}
\AfterBabelLanguage{french}{...}
\end{verbatim}
does ... at the end of |french.ldf|. It can be used in |ldf| files,
too, but in such a case the code is executed only if
\meta{option-name} is the same as |\CurrentOption| (which could not
be the same as the option name as set in |\usepackage|!). 

\begin{example}
  Consider two languages \textsf{foo} and \textsf{bar} defining the
  same |\macro| with |\newcommand|. An error is raised if you attempt
  to load both. Here is a way to overcome this problem:
\begin{verbatim}
\usepackage[base]{babel}
\AfterBabelLanguage{foo}{%
  \let\macroFoo\macro
  \let\macro\relax}
\usepackage[foo,bar]{babel}
\end{verbatim}
\end{example}

\subsection{\texttt{ini} files}

An alternative approach to define a language is by means of an
\texttt{ini} file. Currently \babel{} provides about 200 of these
files containing the basic data required for a language.

Most of them set the date, and many also the captions (Unicode and
LICR). They will be evolving with the time to add more features
(something to keep in mind if backward compatibility is
important). The following section shows how to make use of them
currently (by means of |\babelprovide|), but a higher interface, based
on package options, in under development.

\begin{example} 
  Although Georgian has its own \texttt{ldf} file, here is how to
  declare this language with an |ini| file in Unicode engines. The
  |nil| language is required, because currently \babel{} raises an
  error if there is no language.
\begingroup
\setmonofont[Scale=.87,Script=Georgian]{DejaVu Sans Mono}
\begin{verbatim}
\documentclass{book}

\usepackage[nil]{babel}
\babelprovide[import=ka, main]{georgian}

\babelfont{rm}{DejaVu Sans}

\begin{document}

\tableofcontents

\chapter{სამზარეულო და სუფრის ტრადიციები}

ქართული ტრადიციული სამზარეულო ერთ-ერთი უმდიდრესია მთელ მსოფლიოში.

\end{document}
\end{verbatim}
\endgroup
\end{example}

Here is the list (u means Unicode captions, and l means LICR
captions):

\begingroup
\bigskip\hrule\nobreak

\makeatletter
\def\tag#1{\par\@hangfrom{\makebox[7em][l]{#1}}\ignorespaces}
\def\hascapu{\textsuperscript{u}}
\def\hascapl{\textsuperscript{l}}

\begin{multicols}{2}

\tag{af} Afrikaans\hascapu\hascapl
\tag{agq} Aghem
\tag{ak} Akan
\tag{am} Amharic\hascapu\hascapl
\tag{ar} Arabic\hascapu\hascapl
\tag{as} Assamese
\tag{asa} Asu
\tag{ast} Asturian\hascapu\hascapl
\tag{az-Cyrl} Azerbaijani
\tag{az-Latn} Azerbaijani
\tag{az} Azerbaijani\hascapu\hascapl
\tag{bas} Basaa
\tag{be} Belarusian\hascapu\hascapl
\tag{bem} Bemba
\tag{bez} Bena
\tag{bg} Bulgarian\hascapu\hascapl
\tag{bm} Bambara
\tag{bn} Bangla\hascapu\hascapl
\tag{bo} Tibetan\hascapu
\tag{brx} Bodo
\tag{bs-Cyrl} Bosnian
\tag{bs-Latn} Bosnian\hascapu\hascapl
\tag{bs} Bosnian\hascapu\hascapl
\tag{ca} Catalan\hascapu\hascapl
\tag{ce} Chechen
\tag{cgg} Chiga
\tag{chr} Cherokee
\tag{ckb} Central Kurdish
\tag{cs} Czech\hascapu\hascapl
\tag{cy} Welsh\hascapu\hascapl
\tag{da} Danish\hascapu\hascapl
\tag{dav} Taita
\tag{de-AT} German\hascapu\hascapl
\tag{de-CH} German\hascapu\hascapl
\tag{de} German\hascapu\hascapl
\tag{dje} Zarma
\tag{dsb} Lower Sorbian\hascapu\hascapl
\tag{dua} Duala
\tag{dyo} Jola-Fonyi
\tag{dz} Dzongkha
\tag{ebu} Embu
\tag{ee} Ewe
\tag{el} Greek\hascapu\hascapl
\tag{en-AU} English\hascapu\hascapl
\tag{en-CA} English\hascapu\hascapl
\tag{en-GB} English\hascapu\hascapl
\tag{en-NZ} English\hascapu\hascapl
\tag{en-US} English\hascapu\hascapl
\tag{en} English\hascapu\hascapl
\tag{eo} Esperanto\hascapu\hascapl
\tag{es-MX} Spanish\hascapu\hascapl
\tag{es} Spanish\hascapu\hascapl
\tag{et} Estonian\hascapu\hascapl
\tag{eu} Basque\hascapu\hascapl
\tag{ewo} Ewondo
\tag{fa} Persian\hascapu\hascapl
\tag{ff} Fulah
\tag{fi} Finnish\hascapu\hascapl
\tag{fil} Filipino
\tag{fo} Faroese
\tag{fur} Friulian\hascapu\hascapl
\tag{fy} Western Frisian
\tag{ga} Irish\hascapu\hascapl
\tag{gd} Scottish Gaelic\hascapu\hascapl
\tag{gl} Galician\hascapu\hascapl
\tag{gsw} Swiss German
\tag{gu} Gujarati
\tag{guz} Gusii
\tag{gv} Manx
\tag{ha-GH} Hausa
\tag{ha-NE} Hausa\hascapl
\tag{ha} Hausa
\tag{haw} Hawaiian
\tag{he} Hebrew\hascapu\hascapl
\tag{hi} Hindi\hascapu
\tag{hr} Croatian\hascapu\hascapl
\tag{hsb} Upper Sorbian\hascapu\hascapl
\tag{hu} Hungarian\hascapu\hascapl
\tag{hy} Armenian
\tag{ia} Interlingua\hascapu\hascapl
\tag{id} Indonesian\hascapu\hascapl
\tag{ig} Igbo
\tag{ii} Sichuan Yi
\tag{is} Icelandic\hascapu\hascapl
\tag{it} Italian\hascapu\hascapl
\tag{ja} Japanese
\tag{jgo} Ngomba
\tag{jmc} Machame
\tag{ka} Georgian\hascapu\hascapl
\tag{kab} Kabyle
\tag{kam} Kamba
\tag{kde} Makonde
\tag{kea} Kabuverdianu
\tag{khq} Koyra Chiini
\tag{ki} Kikuyu
\tag{kk} Kazakh
\tag{kkj} Kako
\tag{kl} Kalaallisut
\tag{kln} Kalenjin
\tag{km} Khmer
\tag{kn} Kannada\hascapu\hascapl
\tag{ko} Korean
\tag{kok} Konkani
\tag{ks} Kashmiri
\tag{ksb} Shambala
\tag{ksf} Bafia
\tag{ksh} Colognian
\tag{kw} Cornish
\tag{ky} Kyrgyz
\tag{lag} Langi
\tag{lb} Luxembourgish
\tag{lg} Ganda
\tag{lkt} Lakota
\tag{ln} Lingala
\tag{lo} Lao\hascapu\hascapl
\tag{lrc} Northern Luri
\tag{lt} Lithuanian\hascapu\hascapl
\tag{lu} Luba-Katanga
\tag{luo} Luo
\tag{luy} Luyia
\tag{lv} Latvian\hascapu\hascapl
\tag{mas} Masai
\tag{mer} Meru
\tag{mfe} Morisyen
\tag{mg} Malagasy
\tag{mgh} Makhuwa-Meetto
\tag{mgo} Metaʼ
\tag{mk} Macedonian\hascapu\hascapl
\tag{ml} Malayalam\hascapu\hascapl
\tag{mn} Mongolian
\tag{mr} Marathi\hascapu\hascapl
\tag{ms-BN} Malay\hascapl
\tag{ms-SG} Malay\hascapl
\tag{ms} Malay\hascapu\hascapl
\tag{mt} Maltese
\tag{mua} Mundang
\tag{my} Burmese
\tag{mzn} Mazanderani
\tag{naq} Nama
\tag{nb} Norwegian Bokmål\hascapu\hascapl
\tag{nd} North Ndebele
\tag{ne} Nepali
\tag{nl} Dutch\hascapu\hascapl
\tag{nmg} Kwasio
\tag{nn} Norwegian Nynorsk\hascapu\hascapl
\tag{nnh} Ngiemboon
\tag{nus} Nuer
\tag{nyn} Nyankole
\tag{om} Oromo
\tag{or} Odia
\tag{os} Ossetic
\tag{pa-Arab} Punjabi
\tag{pa-Guru} Punjabi
\tag{pa} Punjabi
\tag{pl} Polish\hascapu\hascapl
\tag{pms} Piedmontese\hascapu\hascapl
\tag{ps} Pashto
\tag{pt-BR} Portuguese\hascapu\hascapl
\tag{pt-PT} Portuguese\hascapu\hascapl
\tag{pt} Portuguese\hascapu\hascapl
\tag{qu} Quechua
\tag{rm} Romansh\hascapu\hascapl
\tag{rn} Rundi
\tag{ro} Romanian\hascapu\hascapl
\tag{rof} Rombo
\tag{ru} Russian\hascapu\hascapl
\tag{rw} Kinyarwanda
\tag{rwk} Rwa
\tag{sah} Sakha
\tag{saq} Samburu
\tag{sbp} Sangu
\tag{se} Northern Sami\hascapu\hascapl
\tag{seh} Sena
\tag{ses} Koyraboro Senni
\tag{sg} Sango
\tag{shi-Latn} Tachelhit
\tag{shi-Tfng} Tachelhit
\tag{shi} Tachelhit
\tag{si} Sinhala
\tag{sk} Slovak\hascapu\hascapl
\tag{sl} Slovenian\hascapu\hascapl
\tag{smn} Inari Sami
\tag{sn} Shona
\tag{so} Somali
\tag{sq} Albanian\hascapu\hascapl
\tag{sr-Cyrl-BA} Serbian\hascapu\hascapl
\tag{sr-Cyrl-ME} Serbian\hascapu\hascapl
\tag{sr-Cyrl-XK} Serbian\hascapu\hascapl
\tag{sr-Cyrl} Serbian\hascapu\hascapl
\tag{sr-Latn-BA} Serbian\hascapu\hascapl
\tag{sr-Latn-ME} Serbian\hascapu\hascapl
\tag{sr-Latn-XK} Serbian\hascapu\hascapl
\tag{sr-Latn} Serbian\hascapu\hascapl
\tag{sr} Serbian\hascapu\hascapl
\tag{sv} Swedish\hascapu\hascapl
\tag{sw} Swahili
\tag{ta} Tamil\hascapu
\tag{te} Telugu\hascapu\hascapl
\tag{teo} Teso
\tag{th} Thai\hascapu\hascapl
\tag{ti} Tigrinya
\tag{tk} Turkmen\hascapu\hascapl
\tag{to} Tongan
\tag{tr} Turkish\hascapu\hascapl
\tag{twq} Tasawaq
\tag{tzm} Central Atlas Tamazight
\tag{ug} Uyghur
\tag{uk} Ukrainian\hascapu\hascapl
\tag{ur} Urdu\hascapu\hascapl
\tag{uz-Arab} Uzbek
\tag{uz-Cyrl} Uzbek
\tag{uz-Latn} Uzbek
\tag{uz} Uzbek
\tag{vai-Latn} Vai
\tag{vai-Vaii} Vai
\tag{vai} Vai
\tag{vi} Vietnamese\hascapu\hascapl
\tag{vun} Vunjo
\tag{wae} Walser
\tag{xog} Soga
\tag{yav} Yangben
\tag{yi} Yiddish
\tag{yo} Yoruba
\tag{yue} Cantonese
\tag{zgh} Standard Moroccan Tamazight
\tag{zh-Hans-HK} Chinese
\tag{zh-Hans-MO} Chinese
\tag{zh-Hans-SG} Chinese
\tag{zh-Hans} Chinese
\tag{zh-Hant-HK} Chinese
\tag{zh-Hant-MO} Chinese
\tag{zh-Hant} Chinese
\tag{zh} Chinese
\tag{zu} Zulu

\end{multicols}
\endgroup
\hrule
\bigskip

In some context (currently |\babelfont|) an \textsc{ini} file may be
loaded by its name. Here is the list of the names currently
supported. With these languages, |\babelfont| loads (if not done
before) the language and script names (even if the language is defined
as a package option with an \textsf{ldf} file).

\begingroup
\bigskip\hrule\nobreak

\let\\\par

\begin{multicols}{2}

aghem\\
akan\\
albanian\\
american\\
amharic\\
arabic\\
armenian\\
assamese\\
asturian\\
asu\\
australian\\
austrian\\
azerbaijani-cyrillic\\
azerbaijani-cyrl\\
azerbaijani-latin\\
azerbaijani-latn\\
azerbaijani\\
bafia\\
bambara\\
basaa\\
basque\\
belarusian\\
bemba\\
bena\\
bengali\\
bodo\\
bosnian-cyrillic\\
bosnian-cyrl\\
bosnian-latin\\
bosnian-latn\\
bosnian\\
brazilian\\
breton\\
british\\
bulgarian\\
burmese\\
canadian\\
cantonese\\
catalan\\
centralatlastamazight\\
centralkurdish\\
chechen\\
cherokee\\
chiga\\
chinese-hans-hk\\
chinese-hans-mo\\
chinese-hans-sg\\
chinese-hans\\
chinese-hant-hk\\
chinese-hant-mo\\
chinese-hant\\
chinese-simplified-hongkongsarchina\\
chinese-simplified-macausarchina\\
chinese-simplified-singapore\\
chinese-simplified\\
chinese-traditional-hongkongsarchina\\
chinese-traditional-macausarchina\\
chinese-traditional\\
chinese\\
colognian\\
cornish\\
croatian\\
czech\\
danish\\
duala\\
dutch\\
dzongkha\\
embu\\
english-au\\
english-australia\\
english-ca\\
english-canada\\
english-gb\\
english-newzealand\\
english-nz\\
english-unitedkingdom\\
english-unitedstates\\
english-us\\
english\\
esperanto\\
estonian\\
ewe\\
ewondo\\
faroese\\
filipino\\
finnish\\
french-be\\
french-belgium\\
french-ca\\
french-canada\\
french-ch\\
french-lu\\
french-luxembourg\\
french-switzerland\\
french\\
friulian\\
fulah\\
galician\\
ganda\\
georgian\\
german-at\\
german-austria\\
german-ch\\
german-switzerland\\
german\\
greek\\
gujarati\\
gusii\\
hausa-gh\\
hausa-ghana\\
hausa-ne\\
hausa-niger\\
hausa\\
hawaiian\\
hebrew\\
hindi\\
hungarian\\
icelandic\\
igbo\\
inarisami\\
indonesian\\
interlingua\\
irish\\
italian\\
japanese\\
jolafonyi\\
kabuverdianu\\
kabyle\\
kako\\
kalaallisut\\
kalenjin\\
kamba\\
kannada\\
kashmiri\\
kazakh\\
khmer\\
kikuyu\\
kinyarwanda\\
konkani\\
korean\\
koyraborosenni\\
koyrachiini\\
kwasio\\
kyrgyz\\
lakota\\
langi\\
lao\\
latvian\\
lingala\\
lithuanian\\
lowersorbian\\
lsorbian\\
lubakatanga\\
luo\\
luxembourgish\\
luyia\\
macedonian\\
machame\\
makhuwameetto\\
makonde\\
malagasy\\
malay-bn\\
malay-brunei\\
malay-sg\\
malay-singapore\\
malay\\
malayalam\\
maltese\\
manx\\
marathi\\
masai\\
mazanderani\\
meru\\
meta\\
mexican\\
mongolian\\
morisyen\\
mundang\\
nama\\
nepali\\
newzealand\\
ngiemboon\\
ngomba\\
norsk\\
northernluri\\
northernsami\\
northndebele\\
norwegianbokmal\\
norwegiannynorsk\\
nswissgerman\\
nuer\\
nyankole\\
nynorsk\\
occitan\\
oriya\\
oromo\\
ossetic\\
pashto\\
persian\\
piedmontese\\
polish\\
portuguese-br\\
portuguese-brazil\\
portuguese-portugal\\
portuguese-pt\\
portuguese\\
punjabi-arab\\
punjabi-arabic\\
punjabi-gurmukhi\\
punjabi-guru\\
punjabi\\
quechua\\
romanian\\
romansh\\
rombo\\
rundi\\
russian\\
rwa\\
sakha\\
samburu\\
samin\\
sango\\
sangu\\
scottishgaelic\\
sena\\
serbian-cyrillic-bosniaherzegovina\\
serbian-cyrillic-kosovo\\
serbian-cyrillic-montenegro\\
serbian-cyrillic\\
serbian-cyrl-ba\\
serbian-cyrl-me\\
serbian-cyrl-xk\\
serbian-cyrl\\
serbian-latin-bosniaherzegovina\\
serbian-latin-kosovo\\
serbian-latin-montenegro\\
serbian-latin\\
serbian-latn-ba\\
serbian-latn-me\\
serbian-latn-xk\\
serbian-latn\\
serbian\\
shambala\\
shona\\
sichuanyi\\
sinhala\\
slovak\\
slovene\\
slovenian\\
soga\\
somali\\
spanish-mexico\\
spanish-mx\\
spanish\\
standardmoroccantamazight\\
swahili\\
swedish\\
swissgerman\\
tachelhit-latin\\
tachelhit-latn\\
tachelhit-tfng\\
tachelhit-tifinagh\\
tachelhit\\
taita\\
tamil\\
tasawaq\\
telugu\\
teso\\
thai\\
tibetan\\
tigrinya\\
tongan\\
turkish\\
turkmen\\
ukenglish\\
ukrainian\\
uppersorbian\\
urdu\\
usenglish\\
usorbian\\
uyghur\\
uzbek-arab\\
uzbek-arabic\\
uzbek-cyrillic\\
uzbek-cyrl\\
uzbek-latin\\
uzbek-latn\\
uzbek\\
vai-latin\\
vai-latn\\
vai-vai\\
vai-vaii\\
vai\\
vietnam\\
vietnamese\\
vunjo\\
walser\\
welsh\\
westernfrisian\\
yangben\\
yiddish\\
yoruba\\
zarma\\
zulu
afrikaans\\

\end{multicols}
\endgroup
\hrule

\subsection{Selecting fonts}

\New{3.15} Babel provides a high level interface on top of |fontspec| to select
fonts. There is no need to load \textsf{fontspec} explicitly --
\babel{} does it for you with the first |\babelfont|.

\Describe\babelfont{\oarg{language-list}\marg{font-family}%
  \oarg{font-options}\marg{font-name}}

Here \textit{font-family} is |rm|, |sf| or |tt| (or newly defined
ones, as explained below), and \textit{font-name} is the same as in
\textsf{fontspec} and the like.

If no language is given, then it is considered the default font for
the family, activated when a language is selected. On the other hand,
if there is one or more languages in the optional argument, the font
will be assigned to them, overriding the default. Alternatively, you
may set a font for a script -- just precede its name (lowercase) with
a star (eg, |*devanagari|).

\Babel{} takes care the font language and the font script when
languages are selected (as well as the writing direction); see the
recognized languages above. In most cases, you will not need
\textit{font-options}, which is the same as in \textsf{fontspec}, but
you may add further key/value pairs if necessary.

\begin{example}
  Usage in most cases is very simple. Let us assume you are setting up
  a document in Swedish, with some words in Hebrew, with a font suited
  for both languages.
  \begingroup
% If you are looking at the code to see how it has been written, you
% will be disappointed :-). The following example is built ad hoc to
% emulate the final result to avoid dependencies, and therefore it's
% not "real" code.
\setmonofont[Scale=.87,Script=Hebrew]{DejaVu Sans Mono} \catcode`@=13
\def@#1{\ifcase#1\relax \egroup \or \bgroup\textdir TLT \else
\bgroup\textdir TRT \fontspec[Scale=.87,Script=Hebrew]{Liberation
Mono} \fi}
\begin{verbatim}
\documentclass{article}

\usepackage[swedish, bidi=default]{babel}

\babelprovide[import=he]{hebrew}

_\babelfont{rm}{FreeSerif}_

\begin{document}

Svenska \foreignlanguage{hebrew}{@2עִבְרִית@0} svenska.

\end{document}
\end{verbatim}
\endgroup

If on the other hand you have to resort to different fonts, you could
replace the red line above with, say:
\begin{verbatim}
\babelfont{rm}{Iwona}
\babelfont[hebrew]{rm}{FreeSerif}
\end{verbatim}
\end{example}

|\babelfont| can be used to implicitly define a new font family. Just
write its name instead of |rm|, |sf| or |tt|. This is the preferred way
to select fonts in addition to the three basic ones.

\begin{example}
  Here is how to do it:
\begin{verbatim}
\babelfont{kai}{FandolKai}
\end{verbatim}
Now, |\kaifamily| and |\kaidefault| are at your disposal.
\end{example}

\begin{note}
  Directionality is a property affecting margins, intentation, column
  order, etc., not just text. Therefore, it is under the direct
  control of the language, which appplies both the script and the
  direction to the text. As a consequence, there is no need to set
  \texttt{Script} when declaring a font (nor \texttt{Language}). In
  fact, it is even discouraged.
\end{note}

\begin{note}
  |\fontspec| is not touched at all, only the preset font families
  (|rm|, |sf|, |tt|, and the like). If a language is switched when an
  \textit{ad hoc} font is active, or you select the font it with this
  command, neither the script nor the language are passed. You must
  add them by hand. This is by design, for several reasons (for
  example, each font has its own set of features and a generic setting
  for several of them could be problematic, and also a “lower level”
  font selection is useful).
\end{note}

\begin{note}
  The keys |Language| and |Script| just pass these values to the
  \textit{font}, and do \textit{not} set the script for the
  \textit{language} (and therefore the writing direction). In other
  words, the |ini| file or |\babelprovide| provides default values for
  |\babelfont| if omitted, but the opposite is not true. See the note
  above for the reasons of this behaviour.
\end{note}

\begin{warning}
  Do not use |\set|\textit{xxxx}|font| and |\babelfont| at the same
  time.  |\babelfont| follows the standard \LaTeX{} conventions to set
  the basic families -- define |\|\textit{xx}|default|, and activate
  it with |\|\textit{xx}|family|. On the other hand,
  |\set|\textit{xxxx}|font| in \textsf{fontspec} takes a different
  approach, because |\|\textit{xx}|family| is redefined with the
  family name hardcoded (so that |\|\textit{xx}|default| becomes
  no-op). Of course, both methods are incompatible, and if you use
  |\set|\textit{xxxx}|font|, font switching with |\babelfont| just
  does \textit{not} work (nor the standard |\|\textit{xx}|default|,
  for that matter).
\end{warning}

\subsection{Modifying a language}

Modifying the behaviour of a language (say, the chapter “caption”), is
sometimes necessary, but not always trivial.
\begin{itemize}
\item The old way, still valid for many languages, to redefine a
  caption is the following:
\begin{verbatim}
\addto\captionsenglish{%
  \renewcommand\contentsname{Foo}%
}
\end{verbatim}
 As of 3.15, there is no need to hide spaces with \texttt{\%}
(\babel{} removes them), but it is advisable to do it.
\item The new way, which is found in |bulgarian|, |azerbaijani|,
  |spanish|, |french|, |turkish|, |icelandic|, |vietnamese| and a few
  more, as well as in languages created with |\babelprovide| and its
  key |import|, is:
\begin{verbatim}
\renewcommand\spanishchaptername{Foo}
\end{verbatim}
\item Macros to be run when a language is selected can be add to
  |\extras|\m{lang}:
\begin{verbatim}
\addto\extrasrussian{\mymacro}
\end{verbatim}
There is a counterpart for code to be run when a language is
unselected: |\noextras|\m{lang}.
\end{itemize}

\begin{note}
  These macros (|\captions|\m{lang}, |\extras|\m{lang}) may be redefined, but
  must not be used as such -- they just pass information to \babel{},
  which executes them in the proper context.
\end{note}

\subsection{Creating a language}

\New{3.10} And what if there is no style for your language or none
fits your needs? You may then define quickly a language with the
help of the following macro in the preamble.

\Describe\babelprovide{\oarg{options}\marg{language-name}}

Defines the internal structure of the language with some defaults: the
hyphen rules, if not available, are set to the current ones, left and
right hyphen mins are set to 2 and 3, but captions and date are not
defined. Conveniently, \babel{} warns you about what to do. Very
likely you will find alerts like that in the |log| file:
\begin{verbatim}
Package babel Warning: \mylangchaptername not set. Please, define
(babel)                it in the preamble with something like:
(babel)                \renewcommand\maylangchaptername{..}
(babel)                Reported on input line 18.
\end{verbatim}

In most cases, you will only need to define a few macros. 

\begin{example}
  If you need a language named |arhinish|:
\begin{verbatim}
\usepackage[danish]{babel}
\babelprovide{arhinish}
\renewcommand\arhinishchaptername{Chapitula}
\renewcommand\arhinishrefname{Refirenke}
\renewcommand\arhinishhyphenmins{22}
\end{verbatim}
\end{example}

The main language is not changed (\texttt{danish} in this example).
So, you must add |\selectlanguage{arhinish}| or other selectors where
necessary.

If the language has been loaded as an argument in |\documentclass| or
|\usepackage|, then |\babelprovide| redefines the requested data.

\Describe{import=}{\meta{language-tag}}
\New{3.13} Imports data from an |ini| file, including captions, date,
and hyphenmins. For example:
\begin{verbatim}
\babelprovide[import=hu]{hungarian}
\end{verbatim}
Unicode engines load the UTF-8 variants, while 8-bit engines load the
LICR (ie, with macros like |\'| or |\ss|) ones.

There are about 200 |ini| files, with data taken from the |ldf| files
and the CLDR provided by Unicode. Not all languages in the latter are
complete, and therefore neither are the |ini| files. A few languages
will show a warning about the current lack of suitability of the date
format (\textsf{hindi}, \textsf{french}, \textsf{breton}, and
\textsf{occitan}).

Besides |\today|, there is a |\<language>date| macro with three
arguments: year, month and day numbers. In fact, |\today| calls
|\<language>today| which in turn calls
|\<language>date{\year}{\month}{\day}|.

\Describe{captions=}{\meta{language-tag}}
Loads only the strings. For example:
\begin{verbatim}
\babelprovide[captions=hu]{hungarian}
\end{verbatim}

\Describe{hyphenrules=}{\meta{language-list}} With this option, with a
space-separated list of hyphenation rules, \babel{} assigns to the
language the first valid hyphenation rules in the list. For example:
\begin{verbatim}
\babelprovide[hyphenrules=chavacano spanish italian]{chavacano}
\end{verbatim}
If none of the listed hyphenrules exist, the default behaviour
applies. Note in this example we set |chavacano| as first option --
without it, it would select |spanish| even if |chavacano| exists.

A special value is |+|, which allocates a new language (in the \TeX{}
sense). It only makes sense as the last value (or the only one; the
subsequent ones are silently ignored). It is mostly useful with
\luatex, because you can add some patterns with |\babelpatterns|, as
for example:
\begin{verbatim}
\babelprovide[hyphenrules=+]{neo}
\babelpatterns[neo]{a1 e1 i1 o1 u1}
\end{verbatim}
In other engines it just supresses hyphenation (because the pattern
list is empty).

\Describe{main}{} This valueless option makes the language the main
one. Only in newly defined languages.

\Describe{script=}{\meta{script-name}} \New{3.15} Sets the script name
to be used by \textsf{fontspec} (eg, |Devanagari|). Overrides the
value in the |ini| file. This value is particularly important because
it sets the writing direction.

\Describe{language=}{\meta{language-name}} \New{3.15} Sets the
language name to be used by \textsf{fontspec} (eg, |Hindi|). Overrides
the value in the |ini| file. Not so important, but sometimes still
relevant.

\begin{note}
  (1) If you need shorthands, you can use |\useshorthands| and
  |\defineshorthand| as described above. (2) Captions and |\today| are
  ``ensured'' with |\babelensure| (this is be the default in
  |ini|-based languages).
\end{note}

\subsection{Getting the current language name}

\Describe\languagename{}
The control sequence |\languagename| contains the name of the
current language. 

\begin{warning}
  Due to some internal inconsistencies in catcodes, it should
  \textit{not} be used to test its value. Use \textsf{iflang}, by
  Heiko Oberdiek.
\end{warning}

\Describe\iflanguage{\marg{language}\marg{true}\marg{false}}

If more than one language is used, it might be necessary to know which
language is active at a specific time. This can be checked by a call
to |\iflanguage|, but note here ``language'' is used in the \TeX\
sense, as a set of hyphenation patterns, and \textit{not} as its
\textsf{babel} name. This macro takes three arguments.  The first
argument is the name of a language; the second and third arguments are
the actions to take if the result of the test is true or false
respectively.

\begin{warning}
  The advice about |\languagename| also applies here -- use
  \textsf{iflang} instead of |\iflanguage| if possible.
\end{warning}

\subsection{Hooks}

\New{3.9a} A hook is a piece of code to be executed at certain
events. Some hooks are predefined when \luatex{} and \xetex{} are
used.

\Describe\AddBabelHook{\marg{name}\marg{event}\marg{code}}

The same name can be applied to several events.  Hooks may be enabled
and disabled for all defined events with
|\EnableBabelHook|\marg{name}, |\DisableBabelHook|\marg{name}. Names
containing the string |babel| are reserved (they are used, for
example, by |\useshortands*| to add a hook for the event
|afterextras|).

Current events are the following; in some of them you can use one to
three \TeX{} parameters (|#1|, |#2|, |#3|), with the meaning given:
\begin{description}
\renewcommand\makelabel[1]{%
  \hspace\labelsep\normalfont\ttfamily\color{thered}#1}
\itemsep=-\parskip
\item[adddialect] (language name, dialect name) Used by
  \file{luababel.def} to load the patterns if not preloaded.
\item[patterns] (language name, language with encoding) Executed just
  after the |\language| has been set. The second argument has the
  patterns name actually selected (in the form of either |lang:ENC| or
  |lang|).
\item[hyphenation] (language name, language with encoding) Executed
  locally just before exceptions given in |\babelhyphenation| are
  actually set.
\item[defaultcommands] Used (locally) in |\StartBabelCommands|.
\item[encodedcommands] (input, font encodings) Used (locally) in
  |\StartBabelCommands|. Both \xetex{} and \luatex{} make sure the
  encoded text is read correctly.
\item[stopcommands] Used to reset the the above, if necessary.
\item[write] This event comes just after the switching commands are
  written to the |aux| file.
\item[beforeextras] Just before executing |\extras|\m{language}. This
  event and the next one should not contain language-dependent code
  (for that, add it to |\extras|\m{language}).
\item[afterextras] Just after executing |\extras|\m{language}. For
  example, the following deactivates shorthands in all languages:
\begin{verbatim}
\AddBabelHook{noshort}{afterextras}{\languageshorthands{none}}
\end{verbatim}
\item[stringprocess] Instead of a parameter, you can manipulate the
  macro |\BabelString| containing the string to be defined with
  |\SetString|. For example, to use an expanded version of the string
  in the definition, write:
\begin{verbatim}
\AddBabelHook{myhook}{stringprocess}{%
  \protected@edef\BabelString{\BabelString}}
\end{verbatim}
\item[initiateactive] (char as active, char as other, original char)
  \New{3.9i} Executed just after a shorthand has been `initiated'. The three
  parameters are the same character with different catcodes: active,
  other (|\string|'ed) and the original one.
\item[afterreset] \New{3.9i} Executed when selecting a language just after
  |\originalTeX| is run and reset to its base value, before executing
  |\captions|\m{language} and |\date|\m{language}.
\end{description}

Four events are used in \file{hyphen.cfg}, which are handled in a
quite different way for efficiency reasons -- unlike the precedent
ones, they only have a single hook and replace a default definition. 
\begin{description}
\renewcommand\makelabel[1]{%
  \hspace\labelsep\normalfont\ttfamily\color{thered}#1}
\itemsep=-\parskip
\item[everylanguage] (language) Executed before every language patterns are loaded.
\item[loadkernel] (file) By default loads |switch.def|. It can be used
  to load a different version of this files or to load nothing.
\item[loadpatterns] (patterns file) Loads the patterns file. Used by
  \file{luababel.def}.
\item[loadexceptions] (exceptions file) Loads the exceptions
  file. Used by \file{luababel.def}.
\end{description}

\Describe\BabelContentsFiles{}
\New{3.9a} This macro contains a list of ``toc'' types which
require a command to switch the language. Its default value is
|toc,lof,lot|, but you may redefine it with |\renewcommand| (it's up
to you to make sure no toc type is duplicated).

\subsection{Hyphenation tools}

\Describe\babelhyphen{%
  \colorbox{thegrey}{\ttfamily\hskip-.2em*\hskip-.2em}\marg{type}}
\DescribeOther\babelhyphen{%
  \colorbox{thegrey}{\ttfamily\hskip-.2em*\hskip-.2em}\marg{text}}

\New{3.9a} It is customary to classify hyphens in two types: (1)
\textit{explicit} or \textit{hard hyphens}, which in \TeX\ are
entered as \verb|-|, and (2) \textit{optional} or \textit{soft
hyphens}, which are entered as \verb|\-|. Strictly, a \textit{soft
hyphen} is not a hyphen, but just a breaking oportunity or, in
\TeX\ terms, a ``discretionary''; a \textit{hard hyphen} is a hyphen
with a breaking oportunity after it. A further type is a
\textit{non-breaking hyphen}, a hyphen without a breaking
oportunity.

In \TeX, \verb|-| and \verb|\-| forbid further breaking oportunities
in the word. This is the desired behaviour very often, but not
always, and therefore many languages provide shorthands for these
cases. Unfortunately, this has not been done consistently: for
example, in Dutch, Portugese, Catalan or Danish, \verb|"-| is a hard
hyphen, while in German, Spanish, Norwegian, Slovak or Russian, it
is a soft hyphen. Furthermore, some of them even redefine |\-|, so
that you cannot insert a soft hyphen without breaking oportunities
in the rest of the word.

Therefore, some macros are provide with a set of basic ``hyphens''
which can be used by themselves, to define a user shorthand, or even
in language files.
\begin{itemize}
\item |\babelhyphen{soft}| and |\babelhyphen{hard}| are self
  explanatory.
\item |\babelhyphen{repeat}| inserts a hard hyphen which is repeated
  at the beginning of the next line, as done in languages like
  Polish, Portugese and Spanish.
\item |\babelhyphen{nobreak}| inserts a hard hyphen without a break
  after it (even if a space follows).
\item |\babelhyphen{empty}| inserts a break oportunity without
  a hyphen at all.
\item |\babelhyphen|\marg{text} is a hard ``hyphen'' using \m{text}
  instead. A typical case is |\babelhyphen{/}|.
\end{itemize}
With all of them hyphenation in the rest of the word is enabled. If
you don't want enabling it, there is a starred counterpart:
|\babelhyphen*{soft}| (which in most cases is equivalent to the
original |\-|), |\babelhyphen*{hard}|, etc.

Note |hard| is also good for isolated prefixes (eg, \textit{anti-})
and |nobreak| for isolated suffixes (eg, \textit{-ism}), but in both
cases |\babelhyphen*{nobreak}| is usually better. 

There are also some differences with \LaTeX: (1) the character used is
that set for the current font, while in \LaTeX{} it is hardwired to
|-| (a typical value); (2) the hyphen to be used in fonts with a
negative |\hyphenchar| is  |-|, like in \LaTeX, but it can be changed to
another value by redefining |\babelnullhyphen|; (3) a break after the
hyphen is forbidden if preceded by a glue ${>}0$~pt (at the beginning
of a word, provided it is not immediately preceded by, say, a
parenthesis).

\Describe\babelhyphenation{\texttt{[}\langlist\texttt{]}%
    \marg{exceptions}}

\New{3.9a} Sets hyphenation exceptions for the languages given
or, without the optional argument, for \textit{all} languages (eg,
proper nouns or common loan words, and of course monolingual
documents). Language exceptions take precedence over global ones.

It can be used only in the preamble, and exceptions are set when the
language is first selected, thus taking into account changes of
|\lccodes|'s done in |\extras|\m{lang} as well as the language specific
encoding (not set in the preamble by default). Multiple
|\babelhyphenation|'s are allowed. For example:
\begin{verbatim}
\babelhyphenation{Wal-hal-la Dar-bhan-ga}
\end{verbatim}

Listed words are saved expanded and therefore it relies on the
LICR. Of course, it also works without the LICR if the input and the
font encodings are the same, like in Unicode based engines.

\Describe\babelpatterns{\texttt{[}\langlist\texttt{]}%
    \marg{patterns}}

\New{3.9m} \textit{In \textsf{luatex} only},\footnote{With
\textsf{luatex} exceptions and patterns can be modified almost
freely. However, this is very likely a task for a separate package
and \texttt{babel} only provides the most basic tools.} adds or
replaces patterns for the languages given or, without the optional
argument, for \textit{all} languages. If a pattern for a certain
combination already exists, it gets replaced by the new one.

It can be used only in the preamble, and patterns are added when the
language is first selected, thus taking into account changes of
|\lccodes|'s done in |\extras|\m{lang} as well as the language specific
encoding (not set in the preamble by default). Multiple
|\babelpatterns|'s are allowed.

Listed patterns are saved expanded and therefore it relies on the
LICR. Of course, it also works without the LICR if the input and the
font encodings are the same, like in Unicode based engines.

\subsection{Selecting scripts}

Currently \babel{} provides no standard interface to select 
scripts, because they are best selected with either |\fontencoding|
(low level) or a language name (high level). Even the Latin script may
require different encodings (ie, sets of glyphs) depending on the
language, and therefore such a switch would be in a sense
incomplete.\footnote{The so-called Unicode fonts do not improve the
situation either. So, a font suited for Vietnamese is not necessarily
suited for, say, the romanization of Indic languages, and the fact it
contains glyphs for Modern Greek does not mean it includes them for
Classic Greek. As to directionality, it poses special challenges
because it also affects individual characters and layout elements.}

Some languages sharing the same script define macros to switch it (eg,
|\textcyrillic|), but be aware they may also set the language to a
certain default. Even the \babel{} core defined |\textlatin|, but is
was somewhat buggy because in some cases it messed up encodings and
fonts (for example, if the main latin encoding was |LY1|), and
therefore it has been deprecated.\footnote{But still defined for
backwards compatibility.}

No macros to select the writing direction are provided, either --
writing direction is intrinsic to each script and therefore it is best
set by the language (which could be a dummy one). Furthermore, there
are in fact two right-to-left modes, depending on the language, which
differ in the way `weak' numeric characters are ordered (eg, Arabic
\%123 \textit{vs} Hebrew 123\%).

\Describe{\ensureascii}{\marg{text}}

\New{3.9i} This macro makes sure \m{text} is typeset with a
LICR-savvy encoding in the ASCII range. It is used to redefine |\TeX|
and |\LaTeX| so that they are correctly typeset even with |LGR| or
|X2| (the complete list is stored in |\BabelNonASCII|, which by default
is |LGR|, |X2|, |OT2|, |OT3|, |OT6|, |LHE|, |LWN|, |LMA|, |LMC|,
|LMS|, |LMU|, but you can modify it). So, in some sense it fixes the
bug described in the previous paragraph.

If non-ASCII encodings are not loaded (or no encoding at all), it is
no-op (also |\TeX| and |\LaTeX| are not redefined); otherwise,
|\ensureascii| switches to the encoding at the beginning of the
document if ASCII-savvy, or else the last ASCII-savvy encoding
loaded. For example, if you load |LY1,LGR|, then it is set to |LY1|,
but if you load |LY1,T2A| it is set to |T2A|. The symbol encodings
|TS1|, |T3|, and |TS3| are not taken into account, since they are not
used for ``ordinary'' text.

The foregoing rules (which are applied ``at begin document'') cover
most of cases. No asumption is made on characters above
127, which may not follow the LICR conventions -- the goal is just
to ensure most of the ASCII letters and symbols are the right ones.

\subsection{Language attributes}

\DescribeMacro{\languageattribute}
This is a user-level command, to be used in the preamble of a
document (after |\usepackage[...]{babel}|), that declares which
attributes are to be used for a given language. It takes two
arguments: the first is the name of the language; the second,
a (list of) attribute(s) to be used. Attributes must be set in the
preamble and only once -- they cannot be turned on and off.
The command checks whether the language is known in this document
and whether the attribute(s) are known for this language.

Very often, using a \textit{modifier} in a package option is better.

Several language definition files use their own methods to set
options. For example, \textsf{french} uses |\frenchsetup|,
\textsf{magyar} (1.5) uses |\magyarOptions|; modifiers provided by
|spanish| have no attribute counterparts. Macros settting
options are also used (eg, |\ProsodicMarksOn| in \textsf{latin}).

\subsection{Languages supported by \babel}

In the following table most of the languages supported by \babel\ are
listed, together with the names of the option  which you can load
\babel\ with for each language. Note this list is open and the
current options may be different. It does not include |ini| files.

\begin{description}
\itemsep=-\parskip
\sffamily
\item[Afrikaans] afrikaans
\item[Azerbaijani] azerbaijani
\item[Basque] basque
\item[Breton] breton
\item[Bulgarian] bulgarian
\item[Catalan] catalan
\item[Croatian] croatian
\item[Czech] czech
\item[Danish] danish
\item[Dutch] dutch
\item[English] english, USenglish, american, UKenglish,
     british, canadian, australian, newzealand
\item[Esperanto] esperanto
\item[Estonian] estonian
\item[Finnish] finnish
\item[French] french, francais, canadien, acadian
\item[Galician] galician
\item[German] austrian, german, germanb, ngerman, naustrian
\item[Greek] greek, polutonikogreek
\item[Hebrew] hebrew
\item[Icelandic] icelandic
\item[Indonesian] bahasa, indonesian, indon, bahasai
\item[Interlingua] interlingua
\item[Irish Gaelic] irish
\item[Italian] italian
\item[Latin] latin
\item[Lower Sorbian] lowersorbian
\item[Malay] bahasam, malay, melayu
\item[North Sami] samin
\item[Norwegian] norsk, nynorsk
\item[Polish] polish
\item[Portuguese] portuges, portuguese, brazilian, brazil
\item[Romanian] romanian 
\item[Russian] russian
\item[Scottish Gaelic] scottish
\item[Spanish]  spanish
\item[Slovakian] slovak
\item[Slovenian]  slovene
\item[Swedish] swedish
\item[Serbian] serbian
\item[Turkish] turkish
\item[Ukrainian] ukrainian
\item[Upper Sorbian] uppersorbian
\item[Welsh] welsh
\end{description}

There are more languages not listed above, including \textsf{hindi,
thai, thaicjk, latvian, turkmen, magyar, mongolian, romansh,
lithuanian, spanglish, vietnamese, japanese, pinyin, arabic, farsi,
ibygreek, bgreek, serbianc, frenchle, ethiop} and \textsf{friulan}.

Most of them work out of the box, but some may require extra fonts,
encoding files, a preprocessor or even a complete framework (like
CJK).  For example, if you have got the \textsf{velthuis/devnag} package,
you can create a file with extension |.dn|:
\begin{verbatim}
\documentclass{article}
\usepackage[hindi]{babel}
\begin{document}
{\dn devaanaa.m priya.h}
\end{document}
\end{verbatim}
Then you preprocess it with |devnag| \m{file}, which creates
\m{file}|.tex|; you can then typeset the latter with \LaTeX.

\subsection{Tips, workarounds, know issues and notes}

\begin{itemize}
\item If you use the document class \cls{book} \emph{and} you use
  |\ref| inside the argument of |\chapter| (or just use |\ref| inside
  |\MakeUppercase|), \LaTeX\ will keep complaining about an undefined
  label.  To prevent such problems, you could revert to using
  uppercase labels, you can use |\lowercase{\ref{foo}}| inside the
  argument of |\chapter|, or, if you will not use shorthands in
  labels, set the |safe| option to |none| or |bib|.

\item\catcode`\|=12\relax Both \textsf{ltxdoc} and \textsf{babel} use
  \verb|\AtBeginDocument| to change some catcodes, and babel reloads
  \textsf{hhline} to make sure \verb|:| has the right one, so if you
  want to change the catcode of \verb/|/ it has to be done using the
  same method at the proper place, with
\begin{verbatim}
\AtBeginDocument{\DeleteShortVerb{\|}}
\end{verbatim}
  \textit{before} loading babel. This way, when the document begins
  the sequence is (1) make \verb/|/ active (\textsf{ltxdoc}); (2) make
  it unactive (your settings); (3) make babel shorthands active
  (\textsf{babel)}; (4) reload \textsf{hhline} (\textsf{babel}, now
  with the correct catcodes for \verb/|/ and
  \verb|:|).\catcode`\|=\active

\item Documents with several input encodings are not frequent, but
  sometimes are useful. You can set different encodings for different
  languages as the following example shows:
\begin{verbatim}
\addto\extrasfrench{\inputencoding{latin1}}
\addto\extrasrussian{\inputencoding{koi8-r}}
\end{verbatim}
   (A recent version of \textsf{inputenc} is required.)
 \item For the hyphenation to work correctly, lccodes cannot change,
   because \TeX{} only takes into account the values when the
   paragraph is hyphenated, i.e., when it has been
   finished.\footnote{This explains why \LaTeX{} assumes the lowercase
   mapping of T1 and does not provide a tool for multiple
   mappings. Unfortunately, \cs{savinghyphcodes} is not a solution
   either, because lccodes for hyphenation are frozen in the format
   and cannot be changed.} So, if you write a chunk of French text
   with |\foreinglanguage|, the apostrophes might not be taken into
   account. This is a limitation of \TeX, not of
   \babel. Alternatively, you may use |\useshorthands| to activate |'|
   and |\defineshorthand|, or redefine |\textquoteright| (the latter
   is called by the non-ASCII right quote).
\item \verb|\bibitem| is out of sync with \verb|\selectlanguage| in
  the \file{.aux} file. The reason is \verb|\bibitem| uses
  \verb|\immediate| (and others, in fact), while
  \verb|\selectlanguage| doesn't. There is no known workaround.
\item Babel does not take into account |\normalsfcodes| and
  (non-)French spacing is not always properly (un)set by
  languages. However, problems are unlikely to happen and therefore
  this part remains untouched in version 3.9 (but it is in the `to
  do' list). 
\item Using a character mathematically active (ie, with math code
  |"8000|) as a shorthand can make \TeX{} enter in an infinite loop in
  some rare cases. (Another issue in the `to do' list, although there
  is a partial solution.)
\end{itemize}

The following packages can be useful, too (the list is still
far from complete):
\begin{description}
\itemsep=-\parskip
\item[csquotes] Logical markup for quotes.
\item[iflang] Tests correctly the current language.
\item[hyphsubst] Selects a different set of patterns for a language.
\item[translator] An open platform for packages that need to be
  localized.
\item[siunitx] Typesetting of numbers and physical quantities.
\item[biblatex] Programmable bibliographies and citations.
\item[bicaption] Bilingual captions.
\item[babelbib] Multilingual bibliographies.
\item[microtype] Adjusts the typesetting according to
some languages (kerning and spacing). Ligatures can be disabled.
\item[substitutefont] Combines fonts in several encodings.
\item[mkpattern] Generates hyphenation patterns.
\item[tracklang] Tracks which languages have been requested.
\end{description}

\subsection{Current and future work}

Current work is focused on the so-called complex scripts in \luatex{}.
In 8-bit engines, \babel{} provided a basic support for bidi text as part
of the style for Hebrew, but it is somewhat unsatisfactory and
internally replaces some hardwired commands by other hardwired
commands (generic changes would be much better).

It is possible now to typeset Arabic or Hebrew with numbers and L
text.  Next on the roadmap are line breaking in Thai and the like,
as well as “non-European” digits. Also on the roadmap are R layouts
(lists, footnotes, tables, column order), page and section numbering,
and maybe kashida justification.

As to Thai line breaking, here is the basic idea of what \luatex{} can
do for us, with the Thai patterns and a little script (the final
version will not be so little, of course). It replaces each
discretionary by the equivalent to ZWJ.

\begingroup
\catcode`\_=13 \def_{\string_}
\begin{verbatim}
\documentclass{article}

\usepackage[nil]{babel}

\babelprovide[import=th, main]{thai}

\babelfont{rm}{FreeSerif}

\directlua{
local GLYF = node.id'glyph'
function insertsp (head)
  local size = 0
  for item in node.traverse(head) do
    local i = item.id
    if i == GLYF then
      f = font.getfont(item.font)
      size = f.size
    elseif i == 7 then
      local n = node.new(12, 0)
      node.setglue(n, 0, size * 1) % 1 is a factor
      node.insert_before(head, item, n)
      node.remove(head, item)
    end
  end
end

luatexbase.add_to_callback('hyphenate',
  function (head, tail)
    lang.hyphenate(head)
    insertsp(head)
  end, 'insertsp')
}

\begin{document}

(Thai text.)

\end{document}
\end{verbatim}
\endgroup

Useful additions would be, for example, time, currency, addresses and
personal names.\footnote{See for example POSIX, ISO 14652 and the
Unicode Common Locale Data Repository (CLDR). Those system, however,
have limited application to \TeX\ because their aim is just to display
information and not fine typesetting.}. But that is the easy
part, because they don't require modifying the \LaTeX{} internals.

Also interesting are differences in the sentence structure or related
to it. For example, in Basque the number precedes the name (including
chapters), in Hungarian ``from (1)'' is ``(1)-b\H{o}l'', but ``from
(3)'' is ``(3)-b\'{o}l'', in Spanish an item labelled
``3.$^{\textrm{\scriptsize o}}$'' may be referred to as either
``\'{\i}tem 3.$^{\textrm{\scriptsize o}}$'' or
``3.$^{\textrm{\scriptsize er}}$ \'{\i}tem'', and so on.

\subsection{Tentative and experimental code}

Handling of \textbf{``Unicode'' fonts} is problematic. There is
\textsf{fontspec}, but special macros are required (not only the NFSS
ones) and it doesn't provide ``orthogonal axis'' for features,
including those related to the language (mainly language and
script). A couple of tentative macros, were provided by \babel{}
($\ge$3.9g) with a partial solution. These macros are now deprecated
-- use |\babelfont|.
\begin{itemize}
\item |\babelFSstore|\marg{babel-language} sets the current three
  basic families (rm, sf, tt) as the default for the language
  given.
\item |\babelFSdefault|\marg{babel-language}\marg{fontspec-features}
  patches |\fontspec| so that the given features are always passed as
  the optional argument or added to it (not an ideal solution). 
\end{itemize}
So, for example:
\begin{verbatim}
\setmainfont[Language=Turkish]{Minion Pro}
\babelFSstore{turkish}
\setmainfont{Minion Pro}
\babelFSfeatures{turkish}{Language=Turkish}
\end{verbatim}

\textbf{Bidi writing} is taking its \textit{first steps}. 
\textit{First steps} means exactly that. For example, in \luatex{} any
Arabic text must be marked up explicitly in L mode. On the other hand,
\xetex{} poses quite different challenges. Document layout (lists,
footnotes, etc.) is not touched at all.

See the code section for |\foreignlanguage*| (a new starred version of
|\foreignlanguage|).

\xetex{} relies on the font to properly handle these unmarked changes,
so it is not under the control of \TeX.

\section{Loading languages with \file{language.dat}}

\TeX{} and most engines based on it (pdf\TeX, \xetex, $\epsilon$-\TeX,
the main exception being \luatex) require hyphenation patterns to be
preloaded when a format is created (eg, \LaTeX, Xe\LaTeX,
pdf\LaTeX). \babel{} provides a tool which has become standand in many
distributions and based on a ``configuration file'' named
\file{language.dat}. The exact way this file is used depends on the
distribution, so please, read the documentation for the latter (note
also some distributions generate the file with some tool).

\New{3.9q} With \luatex, however, patterns are loaded on the fly when
requested by the language (except the ``0th'' language, typically
\textsf{english}, which is preloaded always).\footnote{This feature
was added to 3.9o, but it was buggy. Both 3.9o and 3.9p are
deprecated.} Until 3.9n, this task was delegated to the package
\textsf{luatex-hyphen}, by Khaled Hosny, \'Elie Roux, and Manuel
P\'egouri\'e-Gonnard, and required an extra file named
|language.dat.lua|, but now a new mechanism has been devised based
solely on |language.dat|. \textbf{You must rebuild the formats} if
upgrading from a previous version.  You may want to have a local
|language.dat| for a particular project (for example, a book on
Chemistry).\footnote{The loader for lua(e)tex is slightly different as
it's not based on \babel{} but on \texttt{etex.src}. Until 3.9p it
just didn't work, but thanks to the new code it works by reloading the
data in the \babel{} way, i.e., with \texttt{language.dat}.}

\subsection{Format}

In that file the person who maintains a \TeX\ environment has to record
for which languages he has hyphenation patterns \emph{and} in which
files these are stored\footnote{This is because different operating
systems sometimes use \emph{very} different file-naming
conventions.}. When hyphenation exceptions are stored in a separate
file this can be indicated by naming that file \emph{after} the file
with the hyphenation patterns.

The file can contain empty lines and comments, as well as lines which
start with an equals (\texttt{=}) sign. Such a line will instruct
\LaTeX\ that the hyphenation patterns just processed have to be known
under an alternative name. Here is an example:
\begin{verbatim}
% File    : language.dat
% Purpose : tell iniTeX what files with patterns to load.
english    english.hyphenations
=british

dutch      hyphen.dutch exceptions.dutch % Nederlands
german hyphen.ger
\end{verbatim}

You may also set the font encoding the patterns are intended for by
following the language name by a colon and the encoding
code.\footnote{This in not a new feature, but in former versions it
didn't work correctly.} For example:
\begin{verbatim}
german:T1 hyphenT1.ger
german hyphen.ger
\end{verbatim}
With the previous settings, if the enconding when the language is
selected is |T1| then the patterns in \file{hyphenT1.ger} are
used, but otherwise use those in \file{hyphen.ger} (note the encoding
could be set in |\extras|\m{lang}).

A typical error when using \babel{} is the following:
\begin{verbatim}
No hyphenation patterns were preloaded for
the language `<lang>' into the format.
Please, configure your TeX system to add them and
rebuild the format. Now I will use the patterns
preloaded for english instead}}
\end{verbatim}
It simply means you must reconfigure \file{language.dat}, either by 
hand or with the tools provided by your distribution.

\section{The interface between the core of \babel{} and the language
definition files}

The \textit{language definition files} (ldf) must conform to a
number of conventions, because these files have to fill in the gaps
left by the common code in \file{babel.def}, i.\,e., the
definitions of the macros that produce texts.  Also the
language-switching possibility which has been built into the
\babel{} system has its implications.

The following assumptions are made:
\begin{itemize}
\item Some of the language-specific definitions might be used by plain
  \TeX\ users, so the files have to be coded so that they can be read
  by both \LaTeX\ and plain \TeX. The current format can be checked by
  looking at the value of the macro |\fmtname|.
\item The common part of the \babel{} system redefines a number of
  macros and environments (defined previously in the document style)
  to put in the names of macros that replace the previously hard-wired
  texts.  These macros have to be defined in the language definition
  files.
\item The language definition files must define five macros, used to
  activate and deactivate the language-specific definitions.  These
  macros are |\|\langvar|hyphenmins|, |\captions|\langvar,
  |\date|\langvar, |\extras|\langvar\ and |\noextras|\langvar (the
  last two may be left empty); where \langvar\ is either the name of
  the language definition file or the name of the \LaTeX\ option that
  is to be used. These macros and their functions are discussed
  below. You must define all or none for a language (or a dialect);
  defining, say, |\date|\langvar\ but not |\captions|\langvar\ does
  not raise an error but can lead to unexpected results.
\item When a language definition file is loaded, it can define
  |\l@|\langvar\ to be a dialect of |\language0| when |\l@|\langvar\
  is undefined.
\item Language names must be all lowercase. If an unknow language is
  selected, \babel{} will attempt setting it after lowercasing its
  name.
\item The semantics of modifiers is not defined (on purpose). In
  most cases, they will just be simple separated options (eg,
  \texttt{spanish}), but a language might require, say, a set of
  options organized as a tree with suboptions (in such a case, the
  recommended separator is \verb|/|).
\end{itemize}

Some recommendations:
\begin{itemize}
\item The preferred shorthand is |"|, which is not used in \LaTeX{}
  (quotes are entered as |``| and |''|). Other good choices are
  characters which are not used in a certain context (eg, |=| in an
  ancient language). Note however |=|, |<|, |>|, |:| and the like
  can be dangerous, because they may be used as part of the syntax
  of some elements (numeric expressions, key/value pairs, etc.).
\item Captions should not contain shorthands or encoding dependent
  commands (the latter is not always possible, but should be clearly
  documented). They should be defined using the LICR. You may
  also use the new tools for encoded strings, described below.
\item Avoid adding things to |\noextras|\m{lang} except for umlauthigh
  and friends, |\bbl@deactivate|, |\bbl@(non)frenchspacing|, and
  language specific macros. Use always, if possible, |\bbl@save| and
  |\bbl@savevariable| (except if you still want to have access to the
  previous value). Do not reset a macro or a setting to a hardcoded
  value. Never. Instead save its value in |\extras|\m{lang}.
\item Do not switch scripts. If you want to make sure a set of glyphs
  is used, switch either the font encoding (low level) or the language
  (high level, which in turn may switch the font encoding). Usage of things
  like |\latintext| is deprecated.\footnote{But not removed, for backward
  compatibility.}
\item Please, for ``private'' internal macros do not use the |\bbl@|
  prefix. It is used by \babel{} and it can lead to incompatibilities.
\end{itemize}

There are no special requirements for documenting your language
files. Now they are not included in the base \babel{} manual, so
provide a standalone document suited for your needs, as well as other
files you think can be useful. A PDF and a ``readme'' are strongly
recommended.

\subsection{Basic macros}

In the core of the \babel{} system, several macros are defined for use
in language definition files. Their purpose is to make a new language
known. The first two are related to hyphenation patterns.

\DescribeMacro{\addlanguage}
The macro |\addlanguage| is a non-outer version of the macro
|\newlanguage|, defined in \file{plain.tex} version~3.x. For older
versions of \file{plain.tex} and \file{lplain.tex} a substitute
definition is used. Here ``language'' is used in the \TeX{} sense of
set of hyphenation patterns.

\DescribeMacro{\adddialect}
The macro |\adddialect| can be used when two languages can (or
must) use the same hyphenation patterns. This can also be useful
for languages for which no patterns are preloaded in the
format. In such cases the default behaviour of the \babel{}
system is to define this language as a `dialect' of the language
for which the patterns were loaded as |\language0|.  Here
``language'' is used in the \TeX{} sense of set of hyphenation
patterns.

\DescribeMacro{\<lang>hyphenmins}
The macro |\|\langvar|hyphenmins| is used to store the values of
the |\lefthyphenmin| and |\righthyphenmin|. Redefine this macro
to set your own values, with two numbers corresponding to these
two parameters. For example:
\begin{verbatim}
\renewcommand\spanishhyphenmins{34}
\end{verbatim}
(Assigning |\lefthyphenmin| and |\righthyphenmin| directly in
|\extras<lang>| has no effect.)

\DescribeMacro{\providehyphenmins}
The macro |\providehyphenmins| should be used in the language
definition files to set |\lefthyphenmin| and
|\righthyphenmin|. This macro will check whether these parameters
were provided by the hyphenation file before it takes any action.
If these values have been already set, this command is ignored
(currenty, default pattern files do \textit{not} set them).

\DescribeMacro{\captions\langvar}
The macro |\captions|\langvar\ defines the macros that
hold the texts to replace the original hard-wired texts.

\DescribeMacro{\date\langvar}
The macro |\date|\langvar\ defines |\today|.

\DescribeMacro{\extras\langvar}
The macro |\extras|\langvar\ contains all the extra definitions needed
for a specific language. This macro, like the following, is a hook --
you can add things to it, but it must not be used directly.

\DescribeMacro{\noextras\langvar}
Because we want to let the user switch
between languages, but we do not know what state \TeX\ might be in
after the execution of |\extras|\langvar, a macro that brings
\TeX\ into a predefined state is needed. It will be no surprise
that the name of this macro is |\noextras|\langvar.

\DescribeMacro{\bbl@declare@ttribute}
This is a command to be used in the language definition files for
declaring a language attribute. It takes three arguments: the
name of the language, the attribute to be defined, and the code
to be executed when the attribute is to be used.

\DescribeMacro{\main@language}
To postpone the activation of the definitions needed for a
language until the beginning of a document, all language
definition files should use |\main@language| instead of
|\selectlanguage|. This will just store the name of the language,
and the proper language will be activated at the start of the
document.

\DescribeMacro{\ProvidesLanguage}
The macro |\ProvidesLanguage| should be used to identify the
language definition files. Its syntax is similar to the syntax
of the \LaTeX\ command |\ProvidesPackage|.

\DescribeMacro{\LdfInit}
The macro |\LdfInit| performs a couple of standard checks that
must be made at the beginning of a language definition file,
such as checking the category code of the @-sign, preventing
the \file{.ldf} file from being processed twice, etc.

\DescribeMacro{\ldf@quit}
The macro |\ldf@quit| does work needed
if a \file{.ldf} file was processed
earlier. This includes resetting the category code
of the @-sign, preparing the language to be activated at
|\begin{document}| time, and ending the input stream.

\DescribeMacro{\ldf@finish}
The macro |\ldf@finish| does work needed
at the end of each \file{.ldf} file. This
includes resetting the category code of the @-sign,
loading a local configuration file, and preparing the language
to be activated at |\begin{document}| time.

\DescribeMacro{\loadlocalcfg}
After processing a language definition file,
\LaTeX\ can be instructed to load a local configuration
file. This file can, for instance, be used to add strings to
|\captions|\langvar\ to support local document
classes. The user will be informed that this
configuration file has been loaded. This macro is called by
|\ldf@finish|.

\DescribeMacro{\substitutefontfamily}
(Deprecated.) This command takes three arguments, a font encoding and
two font family names. It creates a font description file for the
first font in the given encoding. This \file{.fd} file will instruct
\LaTeX\ to use a font from the second family when a font from the
first family in the given encoding seems to be needed.

\subsection{Skeleton}

Here is the basic structure of an |ldf| file, with a language, a
dialect and an attribute. Strings are best defined using the method
explained in in sec. \ref{s:strings} (\babel{} 3.9 and later). 

\begin{verbatim}
\ProvidesLanguage{<language>}
     [2016/04/23 v0.0 <Language> support from the babel system]
\LdfInit{<language>}{captions<language>}

\ifx\undefined\l@<language>
  \@nopatterns{<Language>}
  \adddialect\l@<language>0
\fi

\adddialect\l@<dialect>\l@<language>

\bbl@declare@ttribute{<language>}{<attrib>}{%
  \expandafter\addto\expandafter\extras<language>
  \expandafter{\extras<attrib><language>}%
  \let\captions<language>\captions<attrib><language>}

\providehyphenmins{<language>}{\tw@\thr@@}

\StartBabelCommands*{<language>}{captions}
\SetString\chaptername{<chapter name>}
% More strings

\StartBabelCommands*{<language>}{date}
\SetString\monthiname{<name of first month>}
% More strings

\StartBabelCommands*{<dialect>}{captions}
\SetString\chaptername{<chapter name>}
% More strings

\StartBabelCommands*{<dialect>}{date}
\SetString\monthiname{<name of first month>}
% More strings

\EndBabelCommands

\addto\extras<language>{}
\addto\noextras<language>{}
\let\extras<dialect>\extras<language>
\let\noextras<dialect>\noextras<language>

\ldf@finish{<language>}
\end{verbatim}

\subsection{Support for active characters}

In quite a number of language definition files, active characters are
introduced. To facilitate this, some support macros are provided.

\DescribeMacro{\initiate@active@char}
The internal macro |\initiate@active@char| is used in language
definition files to instruct \LaTeX\ to give a character the category
code `active'. When a character has been made active it will remain
that way until the end of the document. Its definition may vary.

\DescribeMacro{\bbl@activate}
\DescribeMacro{\bbl@deactivate}
The command |\bbl@activate| is used to change the way an active
character expands. |\bbl@activate| `switches on' the active behaviour
of the character. |\bbl@deactivate| lets the active character expand
to its former (mostly) non-active self.

\DescribeMacro{\declare@shorthand}
The macro |\declare@shorthand| is used to define the various
shorthands. It takes three arguments: the name for the collection of
shorthands this definition belongs to; the character (sequence) that
makes up the shorthand, i.e.\ |~| or |"a|; and the code to be executed
when the shorthand is encountered. (It does \textit{not} raise an
error if the shorthand character has not been ``initiated''.)

\DescribeMacro{\bbl@add@special}
\DescribeMacro{\bbl@remove@special}
The \TeX book states: ``Plain \TeX\ includes a macro called
|\dospecials| that is essentially a set macro, representing the set of
all characters that have a special category code.'' \cite[p.~380]{DEK}
It is used to set text `verbatim'.  To make this work if more
characters get a special category code, you have to add this character
to the macro |\dospecial|.  \LaTeX\ adds another macro called
|\@sanitize| representing the same character set, but without the
curly braces.  The macros |\bbl@add@special|\meta{char} and
|\bbl@remove@special|\meta{char} add and remove the character
\meta{char} to these two sets.

\subsection{Support for saving macro definitions}

Language definition files may want to \emph{re}define macros that
already exist. Therefore a mechanism for saving (and restoring) the
original definition of those macros is provided. We provide two macros
for this\footnote{This mechanism was introduced by Bernd Raichle.}.

\DescribeMacro{\babel@save}
To save the current meaning of any control sequence, the macro
|\babel@save| is provided. It takes one argument, \meta{csname}, the
control sequence for which the meaning has to be saved.

\DescribeMacro{\babel@savevariable}
A second macro is provided to save the current value of a variable.
In this context, anything that is allowed after the |\the| primitive
is considered to be a variable. The macro takes one argument, the
\meta{variable}.

The effect of the preceding macros is to append a piece of code to
the current definition of |\originalTeX|. When |\originalTeX| is
expanded, this code restores the previous definition of the control
sequence or the previous value of the variable.

\subsection{Support for extending macros}

\DescribeMacro{\addto}
The macro |\addto{|\meta{control sequence}|}{|\meta{\TeX\ code}|}| can
be used to extend the definition of a macro. The macro need not be
defined (ie, it can be undefined or |\relax|). This macro can, for
instance, be used in adding instructions to a macro like
|\extrasenglish|.

Be careful when using this macro, because depending on the case the
assignment could be either global (usually) or local (sometimes). That
does not seem very consistent, but this behaviour is preserved for
backward compatibility. If you are using \pkg{etoolbox}, by Philipp
Lehman, consider using the tools provided by this package instead of
|\addto|.

\subsection{Macros common to a number of languages}

\DescribeMacro{\bbl@allowhyphens}
In several languages compound words are used. This means that when
\TeX\ has to hyphenate such a compound word, it only does so at the
`\texttt{-}' that is used in such words. To allow hyphenation in the
rest of such a compound word, the macro |\bbl@allowhyphens| can be
used.

\DescribeMacro{\allowhyphens}
Same as |\bbl@allowhyphens|, but does nothing if the encoding is
|T1|. It is intended mainly for characters provided as real glyphs by
this encoding but constructed with |\accent| in |OT1|.

Note the previous command (|\bbl@allowhyphens|) has different
applications (hyphens and discretionaries) than this one (composite
chars). Note also prior to version 3.7, |\allowhyphens| had the
behaviour of |\bbl@allowhyphens|.

\DescribeMacro{\set@low@box}
For some languages, quotes need to be lowered to the baseline. For
this purpose the macro |\set@low@box| is available. It takes one
argument and puts that argument in an |\hbox|, at the baseline. The
result is available in |\box0| for further processing.

\DescribeMacro{\save@sf@q}
Sometimes it is necessary to preserve the |\spacefactor|.  For this
purpose the macro |\save@sf@q| is available. It takes one argument,
saves the current spacefactor, executes the argument, and restores the
spacefactor.

\DescribeMacro{\bbl@frenchspacing}
\DescribeMacro{\bbl@nonfrenchspacing}
The commands |\bbl@frenchspacing| and |\bbl@nonfrenchspacing| can be
used to properly switch French spacing on and off.

\subsection{Encoding-dependent strings}
\label{s:strings}

\New{3.9a} Babel 3.9 provides a way of defining strings in several
encodings, intended mainly for \luatex{} and \xetex. This is the only new
feature requiring changes in language files if you want to make use of
it.

Furthermore, it must be activated explicitly, with the package option
|strings|. If there is no |strings|, these blocks are ignored, except
|\SetCase|s (and except if forced as described below). In other words,
the old way of defining/switching strings still works and it's used by
default.

It consist is a series of blocks started with
|\StartBabelCommands|. The last block is closed with
|\EndBabelCommands|. Each block is a single group (ie, local
declarations apply until the next |\StartBabelCommands| or
|\EndBabelCommands|). An |ldf| may contain several series of this
kind.

Thanks to this new feature, string values and string language
switching are not mixed any more. No need of |\addto|. If the language
is |french|, just redefine |\frenchchaptername|.

\Describe\StartBabelCommands
  {\marg{language-list}\marg{category}\oarg{selector}}

The \m{language-list} specifies which languages the block is
intended for. A block is taken into account only if the
|\CurrentOption| is listed here. Alternatively, you can define
|\BabelLanguages| to a comma-separated list of languages to be
defined (if undefined, |\StartBabelCommands| sets it to
|\CurrentOption|). You may write |\CurrentOption| as the language,
but this is discouraged -- a explicit name (or names) is much better
and clearer.

A ``selector'' is a name to be used as value in package option
|strings|, optionally followed by extra info about the encodings to be
used. The name |unicode| must be used for \xetex{} and \luatex{} (the
key |strings| has also other two special values: |generic| and
|encoded|).

If a string is set several times (because several blocks are read),
the first one take precedence (ie, it works much like
|\providecommand|).

Encoding info is |charset=| followed by a charset, which if given sets
how the strings should be traslated to the internal representation
used by the engine, typically |utf8|, which is the only value
supported currently (default is no traslations). Note |charset| is
applied by \luatex{} and \xetex{} when reading the file, not when the
macro or string is used in the document.

A list of font encodings which the strings are expected to work with
can be given after |fontenc=| (separated with spaces, if two or more) --
recommended, but not mandatory, although blocks without this key are
not taken into account if you have requested |strings=encoded|.

Blocks without a selector are read always if the key |strings| has
been used.  They provide fallback values, and therefore must be the
last blocks; they should be provided always if possible and all
strings should be defined somehow inside it; they can be the only
blocks (mainly LGC scripts using the LICR). Blocks without a selector
can be activated explicitly with |strings=generic| (no block is taken
into account except those). With |strings=encoded|, strings in those
blocks are set as default (internally, |?|). With |strings=encoded|
strings are protected, but they are correctly expanded in
|\MakeUppercase| and the like. If there is no key |strings|, string
definitions are ignored, but |\SetCase|s are still honoured (in a
|encoded| way).

The \m{category} is either |captions|, |date| or |extras|. You must
stick to these three categories, even if no error is raised when using
other name.\footnote{In future releases further categories may be
added.\nb{like `monetary', `time', `address', `name', `case' or
`numeric'}} It may be empty, too, but in such a case using
|\SetString| is an error (but not |\SetCase|).

\begin{verbatim}
\StartBabelCommands{language}{captions}
  [unicode, fontenc=TU EU1 EU2, charset=utf8]
\SetString{\chaptername}{utf8-string}

\StartBabelCommands{language}{captions}
\SetString{\chaptername}{ascii-maybe-LICR-string}

\EndBabelCommands
\end{verbatim}

A real example is:
\begin{verbatim}
\StartBabelCommands{austrian}{date}
  [unicode, fontenc=TU EU1 EU2, charset=utf8]
  \SetString\monthiname{Jänner}

\StartBabelCommands{german,austrian}{date}
  [unicode, fontenc=TU EU1 EU2, charset=utf8]
  \SetString\monthiiiname{März}

\StartBabelCommands{austrian}{date}
  \SetString\monthiname{J\"{a}nner}

\StartBabelCommands{german}{date}
  \SetString\monthiname{Januar}

\StartBabelCommands{german,austrian}{date}
  \SetString\monthiiname{Februar}
  \SetString\monthiiiname{M\"{a}rz}
  \SetString\monthivname{April}
  \SetString\monthvname{Mai}
  \SetString\monthviname{Juni}
  \SetString\monthviiname{Juli}
  \SetString\monthviiiname{August}
  \SetString\monthixname{September}
  \SetString\monthxname{Oktober}
  \SetString\monthxiname{November}
  \SetString\monthxiiname{Dezenber}
  \SetString\today{\number\day.~%
    \csname month\romannumeral\month name\endcsname\space
    \number\year}

\StartBabelCommands{german,austrian}{captions}
  \SetString\prefacename{Vorwort}
  [etc.]

\EndBabelCommands
\end{verbatim}

When used in |ldf| files, previous values of |\|\m{category}\m{language}
are overriden, which means the old way to define strings still works
and used by default (to be precise, is first set to undefined and then
strings are added). However, when used in the preamble or in a
package, new settings are added to the previous ones, if the language
exists (in the \babel{} sense, ie, if |\date|\m{language} exists).

\Describe\StartBabelCommands{%
  \colorbox{thegrey}{\ttfamily\hskip-.2em*\hskip-.2em}%
  \marg{language-list}\marg{category}\oarg{selector}}
The starred version just forces |strings| to take a value -- if not set
as package option, then the default for the engine is used. This is
not done by default to prevent backward incompatibilities, but if you
are creating a new language this version is better. It's up to
the maintainers of the current languages to decide if using it is
appropiate.\footnote{This replaces in 3.9g a short-lived
\texttt{\string\UseStrings} which has been removed because it did
not work.}

\Describe\EndBabelCommands{}
Marks the end of the series of blocks.

\Describe\AfterBabelCommands{\marg{code}}
The code is delayed and executed at the global scope just after |\EndBabelCommands|.

\Describe\SetString{\marg{macro-name}\marg{string}}
Adds \meta{macro-name} to the current category, and defines globally
\meta{lang-macro-name} to \meta{code} (after applying the
transformation corresponding to the current charset or defined with
the hook |stringprocess|).

Use this command to define strings, without including any ``logic'' if
possible, which should be a separated macro. See the example above for
the date.

\Describe\SetStringLoop{\marg{macro-name}\marg{string-list}}
A convenient way to define several ordered names at once. For example,
to define |\abmoniname|, |\abmoniiname|, etc. (and similarly with
|abday|):
\begin{verbatim}
\SetStringLoop{abmon#1name}{en,fb,mr,ab,my,jn,jl,ag,sp,oc,nv,dc}
\SetStringLoop{abday#1name}{lu,ma,mi,ju,vi,sa,do}
\end{verbatim}
|#1| is replaced by the roman numeral.

\Describe\SetCase{\oarg{map-list}\marg{toupper-code}\marg{tolower-code}}
Sets globally code to be executed at |\MakeUppercase| and
|\MakeLowercase|. The code would be typically things like |\let\BB\bb|
and |\uccode| or |\lccode| (although for the reasons explained above,
changes in lc/uc codes may not work). A \meta{map-list} is a series of
macros using the internal format of |\@uclclist| (eg,
|\bb\BB\cc\CC|). The mandatory arguments take precedence over the
optional one. This command, unlike |\SetString|, is executed always
(even without |strings|), and it is intented for minor readjustments
only.

For example, as |T1| is the default case mapping in \LaTeX, we could
set for Turkish: % :-( Seem to be a bug in listings. Fixed with &&.
\begin{verbatim}
\StartBabelCommands{turkish}{}[ot1enc, fontenc=OT1]
\SetCase
  {\uccode"10=`I\relax}
  {\lccode`I="10\relax}

\StartBabelCommands{turkish}{}[unicode, fontenc=TU EU1 EU2, charset=utf8]
\SetCase
  {\uccode`i=`İ\relax
   \uccode`ı=`I\relax}
  {\lccode`İ=`i\relax
   \lccode`I=`ı\relax}

\StartBabelCommands{turkish}{}
\SetCase
  {\uccode`i="9D\relax
   \uccode"19=`I\relax}
  {\lccode"9D=`i\relax
   \lccode`I="19\relax}

\EndBabelCommands
\end{verbatim}
(Note the mapping for |OT1| is not complete.)

\Describe\SetHyphenMap{\marg{to-lower-macros}}
\New{3.9g} Case mapping serves in \TeX{} for two unrelated purposes: case
transforms (upper/lower) and hyphenation. |\SetCase| handles the
former, while hyphenation is handled by |\SetHyphenMap| and controlled
with the package option |hyphenmap|. So, even if internally they are based
on the same \TeX{} primitive (|\lccode|), \babel{} sets them separately.

There are three helper macros to
be used inside |\SetHyphenMap|:
\begin{itemize}
\item |\BabelLower|\marg{uccode}\marg{lccode} is
  similar to |\lccode| but it's ignored if the char has been set and
  saves the original lccode to restore it when switching the language
  (except with |hyphenmap=first|).
\item |\BabelLowerMM|\marg{uccode-from}\marg{uccode-to}%
  \marg{step}\marg{lccode-from} loops though the given uppercase
  codes, using the step, and assigns them the lccode, which is also
  increased (|MM| stands for \textit{many-to-many}).
\item |\BabelLowerMO|\marg{uccode-from}\marg{uccode-to}%
  \marg{step}\marg{lccode} loops though the given uppercase
  codes, using the step, and assigns them the lccode, which is fixed
  (|MO| stands for \textit{many-to-one}).
\end{itemize}
An example is (which is redundant, because these assignments are done
by both \luatex{} and \xetex{}):
\begin{verbatim}
\SetHyphenMap{\BabelLowerMM{"100}{"11F}{2}{"101}}
\end{verbatim}

This macro is not intended to fix wrong mappings done by Unicode
(which are the default in both \xetex{} and \luatex{}) -- if an
assignment is wrong, fix it directly.


\section{Changes}

\subsection{Changes in \babel\ version 3.9}

Most of changes in version 3.9 are related to bugs, either to fix them
(there were lots), or to provide some alternatives. Even new features
like |\babelhyphen| are intended to solve a certain problem (in this
case, the lacking of a uniform syntax and behaviour for shorthands
across languages). These changes are described in this manual in the
corresponding place. A selective list follows:
\begin{itemize}
\item |\select@language| did not set |\languagename|. This meant the
  language in force when auxiliary files were loaded was the one used
  in, for example, shorthands -- if the language was |german|, a
  |\select@language{spanish}| had no effect.

\item |\foreignlanguage| and |otherlanguage*| messed up
  |\extras<language>|. Scripts, encodings and many other things were
  not switched correctly.

\item The |:ENC| mechanism for hyphenation patterns used the encoding
  of the \textit{previous} language, not that of the language being
  selected.

\item |'| (with |activeacute|) had the original value when writing to an
  auxiliary file, and things like an infinite loop could happen. It
  worked incorrectly with |^| (if activated) and also if deactivated.

\item Active chars where not reset at the end of language options, and
  that lead to incompatibilities between languages.

\item |\textormath| raised and error with a conditional.

\item |\aliasshorthand| didn't work (or only in a few and very specific
  cases).

\item |\l@english| was defined incorrectly (using |\let| instead of
  |\chardef|).

\item |ldf| files not bundled with babel were not recognized when
  called as global options.
\end{itemize}

\subsection{Changes in \babel\ version 3.7}

In \babel\ version 3.7 a number of bugs that were found in
version~3.6 are fixed. Also a number of changes and additions
have occurred:
\begin{itemize}
\item Shorthands are expandable again. The disadvantage is that
  one has to type |'{}a| when the acute accent is used as a
  shorthand character. The advantage is that a number of other
  problems (such as the breaking of ligatures, etc.) have
  vanished.
\item Two new commands, |\shorthandon| and |\shorthandoff| have
  been introduced to enable to temporarily switch off one or more
  shorthands.
\item Support for typesetting Hebrew (and potential support for
  typesetting other right-to-left written languages) is now
  available thanks to Rama Porrat and Boris Lavva.
\item A language attribute has been added to the |\mark...|
  commands in order to make sure that a Greek header line comes
  out right on the last page before a language switch.
\item Hyphenation pattern files are now read \emph{inside a
  group}; therefore any changes a pattern file needs to make to
  lowercase codes, uppercase codes, and category codes are kept
  local to that group. If they are needed for the language, these
  changes will need to be repeated and stored in |\extras...|
\item The concept of language attributes is introduced. It is intended
  to give the user some control over the features a
  language-definition file provides. Its first use is for the Greek
  language, where the user can choose the πολυτονικό (``polytonikó'' or
  multi-accented) Greek way of typesetting texts.
\item The environment \Lenv{hyphenrules} is introduced.
\item The syntax of the file \file{language.dat} has been
  extended to allow (optionally) specifying the font
  encoding to be used while processing the patterns file.
\item The command |\providehyphenmins| should now be used in
  language definition files in order to be able to keep any
  settings provided by the pattern file.
\end{itemize}

\DocInput{babel.dtx}

\section{Acknowledgements}

I would like to thank all who volunteered as $\beta$-testers for their
time. Michel Goossens supplied contributions for most of the other
languages.  Nico Poppelier helped polish the text of the documentation
and supplied parts of the macros for the Dutch language.  Paul Wackers
and Werenfried Spit helped find and repair bugs.

During the further development of the babel system I received much
help from Bernd Raichle, for which I am grateful.

\begin{thebibliography}{9}
 \bibitem{AT} Huda Smitshuijzen Abifares, \textit{Arabic Typography},
   Saqi, 2001.
 \bibitem{DEK} Donald E. Knuth,
   \emph{The \TeX book}, Addison-Wesley, 1986.
 \bibitem{LLbook} Leslie Lamport,
    \emph{\LaTeX, A document preparation System}, Addison-Wesley,
    1986.
 \bibitem{treebus} K.F. Treebus.
    \emph{Tekstwijzer, een gids voor het grafisch verwerken van
    tekst.}
    SDU Uitgeverij ('s-Gravenhage, 1988). 
 \bibitem{HP} Hubert Partl,
   \emph{German \TeX}, \emph{TUGboat} 9 (1988) \#1, p.~70--72.
  \bibitem{LLth} Leslie Lamport,
    in: \TeX hax Digest, Volume 89, \#13, 17 February 1989.
 \bibitem{BEP} Johannes Braams, Victor Eijkhout and Nico Poppelier,
   \emph{The development of national \LaTeX\ styles},
   \emph{TUGboat} 10 (1989) \#3, p.~401--406.
 \bibitem{FE} Yannis Haralambous,
    \emph{Fonts \& Encodings}, O'Reilly, 2007.
 \bibitem{ilatex} Joachim Schrod,
   \emph{International \LaTeX\ is ready to use},
   \emph{TUGboat} 11 (1990) \#1, p.~87--90.
 \bibitem{STS} Apostolos Syropoulos, Antonis Tsolomitis and Nick
   Sofroniu,  
   \emph{Digital typography using \LaTeX},
   Springer, 2002, p.~301--373.
\end{thebibliography}
\end{document}
%</filedriver>
%
% \fi
%
% \changes{babel~3.8e}{2005/03/24}{Many enhancements to the text by
%    Andrew Young} 
% \changes{babel~3.9c}{2013/04/04}{Added the ``modifiers'' mechanism}
% \changes{babel~3.9g}{2013/06/01}{bbplain merged}
% \changes{babel~3.9k}{2014/03/23}{Code and doc reorganized, and some
%    minor enhancements}
%
%\begingroup
%  \catcode`<=\active%
%  \catcode`>=\active
%  \makeatletter
%  \gdef\MakePrivateLetters{%
%    \catcode`@=11\relax
%    \gdef<##1{\ifx##1@$\langle\langle$\bgroup\itshape\rmfamily
%      \expandafter\bblref
%      \else\string<##1\fi}%
%    \gdef\bblref##1@>{##1\/\egroup$\rangle\rangle$}}%
%  \global\let\check@percent\saved@check@percent
%\endgroup
%
% \part{The code}
%
% \babel{} is being developed incrementally, which means parts of the
% code are under development and therefore incomplete. Only documented
% features are considered complete. In other words, use \babel{} only
% as documented (except, of course, if you want to explore and test
% them -- you can post suggestions about multilingual issues to
% |kadingira@tug.org| on |http://tug.org/mailman/listinfo/kadingira|). 
%
% \section{Identification and loading of required files}
%
%    \textit{Code documentation is still under revision.}
%
%    The \babel{} package after unpacking consists of the following files:
%    \begin{description}
%    \itemsep=-\parskip
%    \item[switch.def] defines macros to set and switch languages.
%    \item[babel.def] defines the rest of macros. It has tow parts: a
%    generic one and a second one only for LaTeX{}.
%    \item[babel.sty] is the \LaTeX{} package, which set options and
%    load language styles.
%  \item[plain.def] defines some \LaTeX{} macros required by
%    \file{babel.def} and provides a few tools for Plain.
%   \item[hyphen.cfg] is the file to be used when generating the
%    formats to load hyphenation patterns. By default it also loads
%    \file{switch.def}.
%    \end{description}
%
%    The \babel{} installer extends \textsf{docstrip} with a few
%    ``pseudo-guards'' to set ``variables'' used at installation time.
%    They are used with |<||@name@>| at the appropiated places in the
%    source code and shown below with
%    $\langle\langle$\textit{name}$\rangle\rangle$. That brings a
%    little bit of literate programming.
%
%    \begin{macrocode}
%<<version=3.15>>
%<<date=2017/11/03>>
%    \end{macrocode}
%
% \section{Tools}
%
% \textbf{Do not use the following macros in ldf files. They may
% change in the future}. This applies mainly to those recently added
% for replacing, trimming and looping. The older ones, like
% |\bbl@afterfi|, will not change.
%
% \changes{babel~3.9t}{2017/04/22}{Added new helper macros. Not all are
%   currently used, but will be in 3.10 -- \cs{bbl@trim},
%   \cs{bbl@ifunset}, \cs{bbl@exp}, \cs{bbl@stripslash}}
%
%    We define some basic macros which just make the code cleaner.
%    |\bbl@add| is now used internally instead of |\addto| because of
%    the unpredictable behaviour of the latter. Used in
%    \file{babel.def} and in \file{babel.sty}, which means in \LaTeX{}
%    is executed twice, but we need them when defining options and
%    \file{babel.def} cannot be load until options have been defined.
%    This does not hurt, but should be fixed somehow.
%
% \changes{babel~3.9i}{2014/02/16}{\cs{@for} didn't work with
%    Plain. Added \cs{bbl@loop}}
% \changes{babel~3.15}{2017/10/30}{New convenience macros
%   \cs{bbl@xin@} and \cs{bbl@cs}}
%
%    \begin{macrocode}
%<<*Basic macros>>
\def\bbl@stripslash{\expandafter\@gobble\string}
\def\bbl@add#1#2{%
  \bbl@ifunset{\bbl@stripslash#1}%
    {\def#1{#2}}%
    {\expandafter\def\expandafter#1\expandafter{#1#2}}}
\def\bbl@xin@{\@expandtwoargs\in@}
\def\bbl@csarg#1#2{\expandafter#1\csname bbl@#2\endcsname}%
\def\bbl@cs#1{\csname bbl@#1\endcsname}
\def\bbl@loop#1#2#3{\bbl@@loop#1{#3}#2,\@nnil,}
\def\bbl@loopx#1#2{\expandafter\bbl@loop\expandafter#1\expandafter{#2}}
\def\bbl@@loop#1#2#3,{%
  \ifx\@nnil#3\relax\else
    \def#1{#3}#2\bbl@afterfi\bbl@@loop#1{#2}%
  \fi}
\def\bbl@for#1#2#3{\bbl@loopx#1{#2}{\ifx#1\@empty\else#3\fi}}
%    \end{macrocode}
%
% \changes{babel~3.9t}{2017/04/22}{Use \cs{bbl@ifunset} instead of
%    \cs{@ifundefined}.}
%
%  \begin{macro}{\bbl@add@list}
%    This internal macro adds its second argument to a comma
%    separated list in its first argument. When the list is not
%    defined yet (or empty), it will be initiated. It presumes
%    expandable character strings.
%
% \changes{babel~3.9t}{2017/04/22}{Redefined to avoid infinite loops
%    if the macro is \cs{relax}.}
%
%    \begin{macrocode}
\def\bbl@add@list#1#2{%
  \edef#1{%
    \bbl@ifunset{\bbl@stripslash#1}%
      {}%
      {\ifx#1\@empty\else#1,\fi}%
    #2}}
%    \end{macrocode}
%
%  \end{macro}
%
%  \begin{macro}{\bbl@afterelse}
%  \begin{macro}{\bbl@afterfi}
%    Because the code that is used in the handling of active
%    characters may need to look ahead, we take extra care to `throw'
%    it over the |\else| and |\fi| parts of an
%    |\if|-statement\footnote{This code is based on code presented in
%    TUGboat vol. 12, no2, June 1991 in ``An expansion Power Lemma''
%    by Sonja Maus.}. These macros will break if another |\if...\fi|
%    statement appears in one of the arguments and it is not enclosed
%    in braces.
%
%    \begin{macrocode}
\long\def\bbl@afterelse#1\else#2\fi{\fi#1}
\long\def\bbl@afterfi#1\fi{\fi#1}
%    \end{macrocode}
%
%  \end{macro}
%  \end{macro}
%
% \begin{macro}{\bbl@trim}
%   The following piece of code is stolen (with some changes) from
%   \textsf{keyval}, by David Carlisle. It defines two macros:
%   |\bbl@trim| and |\bbl@trim@def|. The first one strips the leading
%   and trailing spaces from the second argument and then applies the
%   first argument (a macro, |\toks@| and the like). The second one,
%   as its name suggests, defines the first argument as the stripped
%   second argument.
%
%    \begin{macrocode}
\def\bbl@tempa#1{%
  \long\def\bbl@trim##1##2{%
    \futurelet\bbl@trim@a\bbl@trim@c##2\@nil\@nil#1\@nil\relax{##1}}%
  \def\bbl@trim@c{%
    \ifx\bbl@trim@a\@sptoken
      \expandafter\bbl@trim@b
    \else
      \expandafter\bbl@trim@b\expandafter#1%
    \fi}%
  \long\def\bbl@trim@b#1##1 \@nil{\bbl@trim@i##1}}
\bbl@tempa{ }
\long\def\bbl@trim@i#1\@nil#2\relax#3{#3{#1}}
\long\def\bbl@trim@def#1{\bbl@trim{\def#1}}
%    \end{macrocode}
% \end{macro}
% 
%
% \begin{macro}{\bbl@ifunset}
%   To check if a macro is defined, we create a new macro, which does
%   the same as |\@ifundefined|. However, in an $\epsilon$-tex engine,
%   it is based on |\ifcsname|, which is more efficient, and do not
%   waste memory.
%
%    \begin{macrocode}
\def\bbl@ifunset#1{%
  \expandafter\ifx\csname#1\endcsname\relax
    \expandafter\@firstoftwo
  \else
    \expandafter\@secondoftwo
  \fi}
\bbl@ifunset{ifcsname}%
  {}%
  {\def\bbl@ifunset#1{%
     \ifcsname#1\endcsname
       \expandafter\ifx\csname#1\endcsname\relax
         \bbl@afterelse\expandafter\@firstoftwo
       \else
         \bbl@afterfi\expandafter\@secondoftwo
       \fi
     \else
       \expandafter\@firstoftwo
     \fi}}
%    \end{macrocode}
% \end{macro}
%    
% \begin{macro}{\bbl@ifblank}
%   A tool from \textsf{url}, by Donald Arseneau, which tests if a
%   string is empty or space.
%
%    \begin{macrocode}
\def\bbl@ifblank#1{%
  \bbl@ifblank@i#1\@nil\@nil\@secondoftwo\@firstoftwo\@nil}
\long\def\bbl@ifblank@i#1#2\@nil#3#4#5\@nil{#4}
%    \end{macrocode}
% \end{macro}
%
% For each element in the comma separated <key>|=|<value> list,
% execute <code> with |#1| and |#2| as the key and the value of
% current item (trimmed). In addition, the item is passed verbatim as
% |#3|. With the <key> alone, it passes |\@empty| (ie, the macro thus
% named, not an empty argument, which is what you get with <key>|=|
% and no value).
%
%    \begin{macrocode}
\def\bbl@forkv#1#2{%
  \def\bbl@kvcmd##1##2##3{#2}%
  \bbl@kvnext#1,\@nil,}
\def\bbl@kvnext#1,{%
  \ifx\@nil#1\relax\else
    \bbl@ifblank{#1}{}{\bbl@forkv@eq#1=\@empty=\@nil{#1}}%
    \expandafter\bbl@kvnext
  \fi}
\def\bbl@forkv@eq#1=#2=#3\@nil#4{%
  \bbl@trim@def\bbl@forkv@a{#1}%
  \bbl@trim{\expandafter\bbl@kvcmd\expandafter{\bbl@forkv@a}}{#2}{#4}}
%    \end{macrocode}
%
% A \textit{for} loop. Each item (trimmed), is |#1|.  It cannot be
% nested (it's doable, but we don't need it).
%
%    \begin{macrocode}
\def\bbl@vforeach#1#2{%
  \def\bbl@forcmd##1{#2}%
  \bbl@fornext#1,\@nil,}
\def\bbl@fornext#1,{%
  \ifx\@nil#1\relax\else
    \bbl@ifblank{#1}{}{\bbl@trim\bbl@forcmd{#1}}%
    \expandafter\bbl@fornext
  \fi}
\def\bbl@foreach#1{\expandafter\bbl@vforeach\expandafter{#1}}
%    \end{macrocode}
%
% \begin{macro}{\bbl@replace}
%
%    \begin{macrocode}
\def\bbl@replace#1#2#3{% in #1 -> repl #2 by #3
  \toks@{}%
  \def\bbl@replace@aux##1#2##2#2{%
    \ifx\bbl@nil##2%
      \toks@\expandafter{\the\toks@##1}%
    \else
      \toks@\expandafter{\the\toks@##1#3}%
      \bbl@afterfi
      \bbl@replace@aux##2#2%
    \fi}%
  \expandafter\bbl@replace@aux#1#2\bbl@nil#2% 
  \edef#1{\the\toks@}}
%    \end{macrocode}
% \end{macro}
%
% \begin{macro}{\bbl@exp}
%
%   Now, just syntactical sugar, but it makes partial expansion of
%   some code a lot more simple and readable. Here |\\| stands for
%   |\noexpand| and |\<..>| for |\noexpand| applied to a built macro
%   name (the latter does not define the macro if undefined to
%   |\relax|, because it is created locally). The result may be
%   followed by extra arguments, if necessary.
%
%    \begin{macrocode}
\def\bbl@exp#1{%
  \begingroup
    \let\\\noexpand
    \def\<##1>{\expandafter\noexpand\csname##1\endcsname}%
    \edef\bbl@exp@aux{\endgroup#1}%
  \bbl@exp@aux}
%    \end{macrocode}
% \end{macro}
% 
% Two further tools.  |\bbl@samestring| first expand its arguments and
% then compare their expansion (sanitized, so that the catcodes do not
% matter). |\bbl@engine| takes the following values: 0 is pdf\TeX, 1
% is \luatex, and 2 is \xetex. You may use the latter it in your
% language style if you want.
%
%    \begin{macrocode}
\def\bbl@ifsamestring#1#2{%
  \begingroup
    \protected@edef\bbl@tempb{#1}%
    \edef\bbl@tempb{\expandafter\strip@prefix\meaning\bbl@tempb}%
    \protected@edef\bbl@tempc{#2}%
    \edef\bbl@tempc{\expandafter\strip@prefix\meaning\bbl@tempc}%
    \ifx\bbl@tempb\bbl@tempc
      \aftergroup\@firstoftwo
    \else
      \aftergroup\@secondoftwo
    \fi
  \endgroup}
\chardef\bbl@engine=%
  \ifx\directlua\@undefined
    \ifx\XeTeXinputencoding\@undefined
      \z@
    \else
      \tw@
    \fi
  \else
    \@ne
  \fi
%<</Basic macros>>
%    \end{macrocode}
%
% Some files identify themselves with a \LaTeX{} macro. The following
% code is placed before them to define (and then undefine) if not in
% \LaTeX.
%
%    \begin{macrocode}
%<<*Make sure ProvidesFile is defined>>
\ifx\ProvidesFile\@undefined
  \def\ProvidesFile#1[#2 #3 #4]{%
    \wlog{File: #1 #4 #3 <#2>}%
    \let\ProvidesFile\@undefined}
\fi
%<</Make sure ProvidesFile is defined>>
%    \end{macrocode}
%
% The following code is used in \file{babel.sty} and
% \file{babel.def}, and loads (only once) the data in |language.dat|.
%
%    \begin{macrocode}
%<<*Load patterns in luatex>>
\ifx\directlua\@undefined\else
  \ifx\bbl@luapatterns\@undefined
    \input luababel.def
  \fi
\fi
%<</Load patterns in luatex>>
%    \end{macrocode}
%
%    The following code is used in \file{babel.def} and
%    \file{switch.def}.
%
%    \begin{macrocode}
%<<*Load macros for plain if not LaTeX>>
\ifx\AtBeginDocument\@undefined
  \input plain.def\relax
\fi
%<</Load macros for plain if not LaTeX>>
%    \end{macrocode}
%
%  \subsection{Multiple languages}
%
%  \begin{macro}{\language}
%    Plain \TeX\ version~3.0 provides the primitive |\language| that
%    is used to store the current language. When used with a pre-3.0
%    version this function has to be implemented by allocating a
%    counter. The following block is used in \file{switch.def} and
%    \file{hyphen.cfg}; the latter may seem redundant, but remember
%    \babel{} doesn't requires loading \file{switch.def} in the format.
%
%    \begin{macrocode}
%<<*Define core switching macros>>
\ifx\language\@undefined
  \csname newcount\endcsname\language
\fi
%<</Define core switching macros>>
%    \end{macrocode}
%
%  \end{macro}
%
%  \begin{macro}{\last@language}
%    Another counter is used to store the last language defined.  For
%    pre-3.0 formats an extra counter has to be allocated.
%
%  \begin{macro}{\addlanguage}
%
%    To add languages to \TeX's memory plain \TeX\ version~3.0
%    supplies |\newlanguage|, in a pre-3.0 environment a similar macro
%    has to be provided. For both cases a new macro is defined here,
%    because the original |\newlanguage| was defined to be |\outer|.
%
%    For a format based on plain version~2.x, the definition of
%    |\newlanguage| can not be copied because |\count 19| is used for
%    other purposes in these formats. Therefore |\addlanguage| is
%    defined using a definition based on the macros used to define
%    |\newlanguage| in plain \TeX\ version~3.0.
%
%    For formats based on plain version~3.0 the definition of
%    |\newlanguage| can be simply copied, removing |\outer|.
%    Plain \TeX\ version 3.0 uses |\count 19| for this purpose.
%
%    \begin{macrocode}
%<<*Define core switching macros>>
\ifx\newlanguage\@undefined
  \csname newcount\endcsname\last@language
  \def\addlanguage#1{%
    \global\advance\last@language\@ne
    \ifnum\last@language<\@cclvi
    \else
      \errmessage{No room for a new \string\language!}%
    \fi
    \global\chardef#1\last@language
    \wlog{\string#1 = \string\language\the\last@language}}
\else
  \countdef\last@language=19
  \def\addlanguage{\alloc@9\language\chardef\@cclvi}
\fi
%<</Define core switching macros>>
%    \end{macrocode}
%
%  \end{macro}
%  \end{macro}
%
%    Now we make sure all required files are loaded.  When the command
%    |\AtBeginDocument| doesn't exist we assume that we are dealing
%    with a plain-based format or \LaTeX2.09. In that case the file
%    \file{plain.def} is needed (which also defines
%    |\AtBeginDocument|, and therefore it is not loaded twice). We
%    need the first part when the format is created, and |\orig@dump|
%    is used as a flag. Otherwise, we need to use the second part, so
%    |\orig@dump| is not defined (\file{plain.def} undefines it).
%
% \changes{babel~3.9a}{2012/12/21}{Use \cs{orig@dump} as flag instead
%    of \cs{adddialect}}
%
%    Check if the current version of \file{switch.def} has been
%    previously loaded (mainly, \file{hyphen.cfg}). If not, load it
%    now. We cannot load |babel.def| here because we first need to
%    declare and process the package options.
%
% \changes{babel~3.9a}{2012/08/11}{Now switch.def is loaded always, so
%    that there is no need to rebuild formats just to update babel}
% \changes{babel~3.9a}{2012/12/13}{But switch.def is loaded only if
%    loaded in a different version (or not loaded)}
% \changes{babel~3.9a}{2013/01/14}{Added the debug option}
% \changes{babel~3.9a}{2013/02/05}{Added \cs{bbl@add}}
%
%    \section{The Package File (\LaTeX, \texttt{babel.sty})}
%
%    In order to make use of the features of \LaTeXe, the \babel\
%    system contains a package file, \file{babel.sty}. This file is
%    loaded by the |\usepackage| command and defines all the language
%    options whose name is different from that of the |.ldf| file
%    (like variant spellings). It also takes care of a number of
%    compatibility issues with other packages an defines a few
%    aditional package options.
%
%
%    Apart from all the language options below we also have a few options
%    that influence the behaviour of language definition files.
%
%    Many of the following options don't do anything themselves, they
%    are just defined in order to make it possible for babel and
%    language definition files to check if one of them was specified
%    by the user.
%
%   \subsection{\texttt{base}}
%
%    The first option to be processed is |base|, which set the
%    hyphenation patterns then resets |ver@babel.sty| so that
%    \LaTeX forgets about the first loading. After |switch.def| has
%    been loaded (above) and |\AfterBabelLanguage| defined, exits.
%
% \changes{babel~3.9a}{2012/10/05}{preset option started,
%    party stolen from fontenc}
% \changes{babel~3.9a}{2012/10/17}{Hooks started}
% \changes{babel~3.9a}{2013/02/07}{Rejected preset, and replaced by
%   base}
% \changes{babel~3.9q}{2016/02/11}{Load patterns with option base.
%    To be improved. Moved showlanguages before base}
%
%    \begin{macrocode}
%<*package>
\NeedsTeXFormat{LaTeX2e}[2005/12/01]
\ProvidesPackage{babel}[<@date@> <@version@> The Babel package]
\@ifpackagewith{babel}{debug}
  {\let\bbl@debug\@firstofone}
  {\let\bbl@debug\@gobble}
\input switch.def\relax
<@Load patterns in luatex@>
<@Basic macros@>
\def\AfterBabelLanguage#1{%
  \global\expandafter\bbl@add\csname#1.ldf-h@@k\endcsname}%
%    \end{macrocode}
%
% If the format created a list of loaded languages (in
% |\bbl@languages|), get the name of the 0-th to show the actual
% language used.
%
%    \begin{macrocode}
\ifx\bbl@languages\@undefined\else
  \begingroup
    \catcode`\^^I=12
    \@ifpackagewith{babel}{showlanguages}{%
      \begingroup
        \def\bbl@elt#1#2#3#4{\wlog{#2^^I#1^^I#3^^I#4}}%
        \wlog{<*languages>}%
        \bbl@languages
        \wlog{</languages>}%
      \endgroup}{}
  \endgroup
  \def\bbl@elt#1#2#3#4{%
    \ifnum#2=\z@
      \gdef\bbl@nulllanguage{#1}%
      \def\bbl@elt##1##2##3##4{}%
    \fi}%
  \bbl@languages
\fi
\@ifpackagewith{babel}{bidi=basic-r}{% must go before any \DeclareOption
  \let\bbl@beforeforeign\leavevmode
  \AtEndOfPackage{\EnableBabelHook{babel-bidi}}%
  \RequirePackage{luatexbase}%
  \directlua{
    require('babel-bidi.lua')
    require('babel-bidi-basic-r.lua')
    luatexbase.add_to_callback('pre_linebreak_filter',
      Babel.pre_otfload,
      'Babel.pre_otfload',
      luatexbase.priority_in_callback('pre_linebreak_filter',
        'luaotfload.node_processor') or nil)
    luatexbase.add_to_callback('hpack_filter',
      Babel.pre_otfload,
      'Babel.pre_otfload',
      luatexbase.priority_in_callback('hpack_filter',
        'luaotfload.node_processor') or nil)}}{}
%    \end{macrocode}
%
% Now the \texttt{base} option. With it we can define (and load, with
% \luatex) hyphenation patterns, even if we are not interesed in the
% rest of babel. Useful for old versions of \textsf{polyglossia}, too.
%
%    \begin{macrocode}
\@ifpackagewith{babel}{base}{%
  \ifx\directlua\@undefined
    \DeclareOption*{\bbl@patterns{\CurrentOption}}%
  \else
    \DeclareOption*{\bbl@patterns@lua{\CurrentOption}}%
  \fi
  \DeclareOption{base}{}%
  \DeclareOption{showlanguages}{}%
  \ProcessOptions
  \global\expandafter\let\csname opt@babel.sty\endcsname\relax
  \global\expandafter\let\csname ver@babel.sty\endcsname\relax
  \global\let\@ifl@ter@@\@ifl@ter
  \def\@ifl@ter#1#2#3#4#5{\global\let\@ifl@ter\@ifl@ter@@}%
  \endinput}{}%
%    \end{macrocode}
%
%    \subsection{\texttt{key=value} options and other general option}
%
%    The following macros extract language modifiers, and only real
%    package options are kept in the option list. Modifiers are saved
%    and assigned to |\BabelModifiers| at |\bbl@load@language|; when
%    no modifiers have been given, the former is |\relax|. How
%    modifiers are handled are left to language styles; they can use
%    |\in@|, loop them with |\@for| or load |keyval|, for example).
%
% \changes{babel~3.9e}{2013/04/15}{Bug fixed - a dot was added in
%    key=value pairs}
%
%    \begin{macrocode}
\bbl@csarg\let{tempa\expandafter}\csname opt@babel.sty\endcsname
\def\bbl@tempb#1.#2{%
   #1\ifx\@empty#2\else,\bbl@afterfi\bbl@tempb#2\fi}%
\def\bbl@tempd#1.#2\@nnil{%
  \ifx\@empty#2%
    \edef\bbl@tempc{\ifx\bbl@tempc\@empty\else\bbl@tempc,\fi#1}%
  \else
    \in@{=}{#1}\ifin@
      \edef\bbl@tempc{\ifx\bbl@tempc\@empty\else\bbl@tempc,\fi#1.#2}%
    \else
      \edef\bbl@tempc{\ifx\bbl@tempc\@empty\else\bbl@tempc,\fi#1}%
      \bbl@csarg\edef{mod@#1}{\bbl@tempb#2}%
    \fi
  \fi}
\let\bbl@tempc\@empty
\bbl@foreach\bbl@tempa{\bbl@tempd#1.\@empty\@nnil}
\expandafter\let\csname opt@babel.sty\endcsname\bbl@tempc
%    \end{macrocode}
%
%    The next option tells \babel\ to leave shorthand characters
%    active at the end of processing the package. This is \emph{not}
%    the default as it can cause problems with other packages, but for
%    those who want to use the shorthand characters in the preamble of
%    their documents this can help.
%
% \changes{babel~3.9a}{2012/08/14}{Implemented the \texttt{noconfigs}
%    option}
% \changes{babel~3.9a}{2012/09/26}{Implemented the
%    \texttt{showlanguages} option}
% \changes{babel~3.9g}{2013/08/07}{Options for hyphenmap}
% \changes{babel~3.9l}{2014/07/29}{Option \texttt{silent}}
%
%    \begin{macrocode}
\DeclareOption{KeepShorthandsActive}{}
\DeclareOption{activeacute}{}
\DeclareOption{activegrave}{}
\DeclareOption{debug}{}
\DeclareOption{noconfigs}{}
\DeclareOption{showlanguages}{}
\DeclareOption{silent}{}
\DeclareOption{shorthands=off}{\bbl@tempa shorthands=\bbl@tempa}
<@More package options@>
%    \end{macrocode}
%
% Handling of package options is done in three passes. (I [JBL] am not
% very happy with the idea, anyway.) The first one processes options
% which has been declared above or follow the syntax |<key>=<value>|,
% the second one loads the requested languages, except the main one if
% set with the key |main|, and the third one loads the latter. First,
% we ``flag'' valid keys with a nil value.
%
% \changes{babel~3.9a}{2012/08/10}{Added the `safe' key, including code
%    below for selecting the redefined macros}
%
%    \begin{macrocode}
\let\bbl@opt@shorthands\@nnil
\let\bbl@opt@config\@nnil
\let\bbl@opt@main\@nnil
\let\bbl@opt@headfoot\@nnil
%    \end{macrocode}
%
% The following tool is defined temporarily to store the values of
% options.
%
%    \begin{macrocode}
\def\bbl@tempa#1=#2\bbl@tempa{%
  \bbl@csarg\ifx{opt@#1}\@nnil
    \bbl@csarg\edef{opt@#1}{#2}%
  \else
    \bbl@error{%
      Bad option `#1=#2'. Either you have misspelled the\\%
      key or there is a previous setting of `#1'}{%
      Valid keys are `shorthands', `config', `strings', `main',\\%
      `headfoot', `safe', `math'}
  \fi}
%    \end{macrocode}
%
%    Now the option list is processed, taking into account only
%    currently declared options (including those declared with a |=|),
%    and |<key>=<value>| options (the former take precedence). 
%    Unrecognized options are saved in |\bbl@language@opts|, because
%    they are language options. 
%
%    \begin{macrocode}
\let\bbl@language@opts\@empty
\DeclareOption*{%
  \bbl@xin@{\string=}{\CurrentOption}%
  \ifin@
    \expandafter\bbl@tempa\CurrentOption\bbl@tempa
  \else
    \bbl@add@list\bbl@language@opts{\CurrentOption}%
  \fi}
%    \end{macrocode}
%
%    Now we finish the first pass (and start over).
%
%    \begin{macrocode}
\ProcessOptions*
%    \end{macrocode}
%
%    \subsection{Conditional loading of shorthands}
%
%    If there is no |shorthands=<chars>|, the original \textsf{babel}
%    macros are left untouched, but if there is, these macros are
%    wrapped (in |babel.def|) to define only those given.
%
%    A bit of optimization: if there is no |shorthands=|, then
%    |\bbl@ifshorthands| is always true, and it is always false if
%    |shorthands| is empty. Also, some code makes sense only with
%    |shorthands=...|.
%
% \changes{babel~3.9c}{2013/04/07}{Added t and c for tilde and comma}
%
%    \begin{macrocode}
\def\bbl@sh@string#1{%
  \ifx#1\@empty\else
    \ifx#1t\string~%
    \else\ifx#1c\string,%
    \else\string#1%
    \fi\fi
    \expandafter\bbl@sh@string
  \fi}
\ifx\bbl@opt@shorthands\@nnil
  \def\bbl@ifshorthand#1#2#3{#2}%
\else\ifx\bbl@opt@shorthands\@empty
  \def\bbl@ifshorthand#1#2#3{#3}%
\else
%    \end{macrocode}
%
%    The following macro tests if a shortand is one of the allowed
%    ones.
%
%    \begin{macrocode}
  \def\bbl@ifshorthand#1{%
    \bbl@xin@{\string#1}{\bbl@opt@shorthands}%
    \ifin@
      \expandafter\@firstoftwo
    \else
      \expandafter\@secondoftwo
    \fi}
%    \end{macrocode}
%
%    We make sure all chars in the string are `other', with the help
%    of an auxiliary macro defined above (which also zaps spaces).
%
%    \begin{macrocode}
  \edef\bbl@opt@shorthands{%
    \expandafter\bbl@sh@string\bbl@opt@shorthands\@empty}%
%    \end{macrocode}
%
%    The following is ignored with |shorthands=off|, since it is
%    intended to take some aditional actions for certain chars.
%
%    \begin{macrocode}
  \bbl@ifshorthand{'}%
    {\PassOptionsToPackage{activeacute}{babel}}{}
  \bbl@ifshorthand{`}%
    {\PassOptionsToPackage{activegrave}{babel}}{}
\fi\fi
%    \end{macrocode}
%
% \changes{babel~3.9a}{2012/07/30}{Code setting language in
%    head/foots.  Related to babel/3796}
%
%    With |headfoot=lang| we can set the language used in heads/foots.
%    For example, in babel/3796 just adds |headfoot=english|.  It
%    misuses \cs{@resetactivechars} but seems to work.
%
%    \begin{macrocode}
\ifx\bbl@opt@headfoot\@nnil\else
  \g@addto@macro\@resetactivechars{%
    \set@typeset@protect                  
    \expandafter\select@language@x\expandafter{\bbl@opt@headfoot}%
    \let\protect\noexpand}
\fi
%    \end{macrocode}
%
% For the option safe we use a different approach -- |\bbl@opt@safe|
% says which macros are redefined (B for bibs and R for refs). By
% default, both are set.
%
%    \begin{macrocode}
\ifx\bbl@opt@safe\@undefined
  \def\bbl@opt@safe{BR}
\fi
\ifx\bbl@opt@main\@nnil\else
  \edef\bbl@language@opts{%
    \ifx\bbl@language@opts\@empty\else\bbl@language@opts,\fi
      \bbl@opt@main}
\fi
%    \end{macrocode}
%
% \subsection{Language options}
%
% \changes{babel~3.9a}{2012/06/15}{Rewritten the loading mechanism, so
%    that languages not declared are also correctly recognized, even
%    if given as global options} 
% \changes{babel~3.9a}{2012/08/12}{Revised the loading mechanism}
% \changes{babel~3.9i}{2014/03/01}{Removed German options, because
%    they are now loaded directly}
%
%    Languages are loaded when processing the corresponding option
%    \textit{except} if a |main| language has been set. In such a
%    case, it is not loaded until all options has been processed.
%    The following macro inputs the ldf file and does some additional
%    checks (|\input| works, too, but possible errors are not catched).
%
%    \begin{macrocode}
\let\bbl@afterlang\relax
\let\BabelModifiers\relax
\let\bbl@loaded\@empty
\def\bbl@load@language#1{%
  \InputIfFileExists{#1.ldf}%
    {\edef\bbl@loaded{\CurrentOption
       \ifx\bbl@loaded\@empty\else,\bbl@loaded\fi}%
     \expandafter\let\expandafter\bbl@afterlang
        \csname\CurrentOption.ldf-h@@k\endcsname
     \expandafter\let\expandafter\BabelModifiers
        \csname bbl@mod@\CurrentOption\endcsname}%
    {\bbl@error{%
       Unknown option `\CurrentOption'. Either you misspelled it\\%
       or the language definition file \CurrentOption.ldf was not found}{%
       Valid options are: shorthands=, KeepShorthandsActive,\\%
       activeacute, activegrave, noconfigs, safe=, main=, math=\\%
       headfoot=, strings=, config=, hyphenmap=, or a language name.}}}
%    \end{macrocode}
%
%    Now, we set language options whose names are different from |ldf| files.
% 
% \changes{babel~3.9t}{2017/04/23}{Removed options for English, Indonesian and
%    Malay, now handled with proxy files}
% \changes{babel~3.13}{2017/08/24}{Removed options for French,
%    too. ldf files now takes priority if exist, except Hebrew (to do)}
%
%    \begin{macrocode}
\def\bbl@try@load@lang#1#2#3{%
    \IfFileExists{\CurrentOption.ldf}%
      {\bbl@load@language{\CurrentOption}}%
      {#1\bbl@load@language{#2}#3}}
\DeclareOption{afrikaans}{\bbl@try@load@lang{}{dutch}{}}
\DeclareOption{brazil}{\bbl@try@load@lang{}{portuges}{}}
\DeclareOption{brazilian}{\bbl@try@load@lang{}{portuges}{}}
\DeclareOption{hebrew}{%
  \input{rlbabel.def}%
  \bbl@load@language{hebrew}}
\DeclareOption{hungarian}{\bbl@try@load@lang{}{magyar}{}}
\DeclareOption{lowersorbian}{\bbl@try@load@lang{}{lsorbian}{}}
\DeclareOption{nynorsk}{\bbl@try@load@lang{}{norsk}{}}
\DeclareOption{polutonikogreek}{%
  \bbl@try@load@lang{}{greek}{\languageattribute{greek}{polutoniko}}}
\DeclareOption{portuguese}{\bbl@try@load@lang{}{portuges}{}}
\DeclareOption{russian}{\bbl@try@load@lang{}{russianb}{}}
\DeclareOption{ukrainian}{\bbl@try@load@lang{}{ukraineb}{}}
\DeclareOption{uppersorbian}{\bbl@try@load@lang{}{usorbian}{}}
%    \end{macrocode}
%
%    Another way to extend the list of `known' options for \babel\ was
%    to create the file \file{bblopts.cfg} in which one can add option
%    declarations. However, this mechanism is deprecated -- if you
%    want an alternative name for a language, just create a new |.ldf|
%    file loading the actual one. You can also set the name
%    of the file with the package option |config=<name>|, which will
%    load |<name>.cfg| instead.
%
% \changes{babel~3.9a}{2012/06/28}{Added the \cs{AfterBabelLanguage}
%    mechanism, to be used mainly with the local cfg file.}
% \changes{babel~3.9a}{2012/06/31}{Now you can set the name of the
%    local cfg file.}
%
%    \begin{macrocode}
\ifx\bbl@opt@config\@nnil
  \@ifpackagewith{babel}{noconfigs}{}%
    {\InputIfFileExists{bblopts.cfg}%
      {\typeout{*************************************^^J%
               * Local config file bblopts.cfg used^^J%
               *}}%
      {}}%
\else
  \InputIfFileExists{\bbl@opt@config.cfg}%
    {\typeout{*************************************^^J%
             * Local config file \bbl@opt@config.cfg used^^J%
             *}}%
    {\bbl@error{%
       Local config file `\bbl@opt@config.cfg' not found}{%
       Perhaps you misspelled it.}}%
\fi
%    \end{macrocode}
%
%    Recognizing global options in packages not having a closed set of
%    them is not trivial, as for them to be processed they must be
%    defined explicitly. So, package options not yet taken into
%    account and stored in |bbl@language@opts| are assumed to be
%    languages (note this list also contains the language given with
%    |main|). If not declared above, the name of the option and the
%    file are the same.
%
%    \begin{macrocode}
\bbl@for\bbl@tempa\bbl@language@opts{%
  \bbl@ifunset{ds@\bbl@tempa}%
    {\edef\bbl@tempb{%
       \noexpand\DeclareOption
         {\bbl@tempa}%
         {\noexpand\bbl@load@language{\bbl@tempa}}}%
     \bbl@tempb}%
     \@empty}
%    \end{macrocode}
%
%    Now, we make sure an option is explicitly declared for any
%    language set as global option, by checking if an |ldf|
%    exists. The previous step was, in fact, somewhat redundant, but
%    that way we minimize accesing the file system just to see if the
%    option could be a language.
%
%    \begin{macrocode}
\bbl@foreach\@classoptionslist{%
  \bbl@ifunset{ds@#1}%
    {\IfFileExists{#1.ldf}%
      {\DeclareOption{#1}{\bbl@load@language{#1}}}%
      {}}%
    {}}
%    \end{macrocode}
%
%    If a main language has been set, store it for the third pass.
%
%    \begin{macrocode}
\ifx\bbl@opt@main\@nnil\else
  \expandafter
  \let\expandafter\bbl@loadmain\csname ds@\bbl@opt@main\endcsname
  \DeclareOption{\bbl@opt@main}{}
\fi
%    \end{macrocode}
%
%    And we are done, because all options for this pass has been
%    declared. Those already processed in the first pass are just
%    ignored.
%
% \changes{babel~3.9a}{2012/12/22}{Default option does nothing}
%
%    The options have to be processed in the order in which the user
%    specified them (except, of course, global options, which \LaTeX{}
%    processes before):
%
%    \begin{macrocode}
\def\AfterBabelLanguage#1{%
  \bbl@ifsamestring\CurrentOption{#1}{\global\bbl@add\bbl@afterlang}{}}
\DeclareOption*{}
\ProcessOptions*
%    \end{macrocode}
%
%    This finished the second pass. Now the third one begins, which
%    loads the main language set with the key |main|. A warning is
%    raised if the main language is not the same as the last named
%    one, or if the value of the key |main| is not a language. Then
%    execute directly the option (because it could be used only in
%    |main|). After loading all languages, we deactivate
%    |\AfterBabelLanguage|.
%
%    \begin{macrocode}
\ifx\bbl@opt@main\@nnil
  \edef\bbl@tempa{\@classoptionslist,\bbl@language@opts}
  \let\bbl@tempc\@empty
  \bbl@for\bbl@tempb\bbl@tempa{%
    \bbl@xin@{,\bbl@tempb,}{,\bbl@loaded,}%
    \ifin@\edef\bbl@tempc{\bbl@tempb}\fi}
  \def\bbl@tempa#1,#2\@nnil{\def\bbl@tempb{#1}}
  \expandafter\bbl@tempa\bbl@loaded,\@nnil
  \ifx\bbl@tempb\bbl@tempc\else
    \bbl@warning{%
      Last declared language option is `\bbl@tempc',\\%
      but the last processed one was `\bbl@tempb'.\\%
      The main language cannot be set as both a global\\%
      and a package option. Use `main=\bbl@tempc' as\\%
      option. Reported}%
  \fi
\else
  \DeclareOption{\bbl@opt@main}{\bbl@loadmain}
  \ExecuteOptions{\bbl@opt@main}
  \DeclareOption*{}
  \ProcessOptions*
\fi
\def\AfterBabelLanguage{%
  \bbl@error
    {Too late for \string\AfterBabelLanguage}%
    {Languages have been loaded, so I can do nothing}}
%    \end{macrocode}
%
%    In order to catch the case where the user forgot to specify a
%    language we check whether |\bbl@main@language|, has become
%    defined. If not, no language has been loaded and an error
%    message is displayed.
%
% \changes{babel~3.9a}{2012/06/24}{Now babel is not loaded to prevent
%    the document from raising errors after fixing it}
%
%    \begin{macrocode}
\ifx\bbl@main@language\@undefined
  \bbl@error{%
    You haven't specified a language option}{%
    You need to specify a language, either as a global option\\%
    or as an optional argument to the \string\usepackage\space
    command;\\%
    You shouldn't try to proceed from here, type x to quit.}
\fi
%</package>
%    \end{macrocode}
%
% \section{The kernel of Babel (\texttt{babel.def}, common)}
%
% The kernel of the \babel\ system is stored in either
% \file{hyphen.cfg} or \file{switch.def} and \file{babel.def}.  The
% file \file{babel.def} contains most of the code, while
% \file{switch.def} defines the language switching commands; both can
% be read at run time. The file \file{hyphen.cfg} is a file that can
% be loaded into the format, which is necessary when you want to be
% able to switch hyphenation patterns (by default, it also inputs
% \file{switch.def}, for ``historical reasons'', but it is not
% necessary). When \file{babel.def} is loaded it checks if the current
% version of \file{switch.def} is in the format; if not it is
% loaded. A further file, \file{babel.sty}, contains \LaTeX-specific
% stuff.
%
% Because plain \TeX\ users might want to use some of the features of
% the \babel{} system too, care has to be taken that plain \TeX\ can
% process the files. For this reason the current format will have to
% be checked in a number of places. Some of the code below is common
% to plain \TeX\ and \LaTeX, some of it is for the \LaTeX\ case only.
%
% Plain formats based on etex (etex, xetex, luatex) don't load
% |hyphen.cfg| but |etex.src|, which follows a different naming
% convention, so we need to define the babel names. It presumes
% |language.def| exists and it is the same file used when formats were
% created.
%
% \changes{babel~3.9a}{2013/01/11}{Added \cs{bbl@for} for loops
%    ignoring empties}
% \changes{babel~3.9c}{2013/04/06}{Normalize \cs{bbl@afterlang} to
%    relax}
% \changes{babel~3.9i}{2014/03/10}{Make sure \cs{bbl@language@opts}
%    is defined.}
% \changes{babel~3.9i}{2014/03/11}{Define \cs{l@} values from
%   \cs{lang@} values set in Plain etex/xetex/luatex}
%
%    \subsection{Tools}
%
% \changes{babel~3.9k}{2014/03/24}{Added definition for
%    \cs{uselanguage}}
% \changes{babel~3.9n}{2015/12/21}{Define a few macros for 2.09}
% \changes{babel~3.9p}{2016/02/05}{Added a test for lua(e)tex.}
% \changes{babel~3.9q}{2016/02/12}{Load lua patterns if not lualatex.}
%
%    \begin{macrocode}
%<*core>
\ifx\ldf@quit\@undefined
\else
  \expandafter\endinput
\fi
<@Make sure ProvidesFile is defined@>
\ProvidesFile{babel.def}[<@date@> <@version@> Babel common definitions]
<@Load macros for plain if not LaTeX@>
\ifx\bbl@ifshorthand\@undefined
  \def\bbl@ifshorthand#1#2#3{#2}%
  \def\bbl@opt@safe{BR}
  \def\AfterBabelLanguage#1#2{}
  \let\bbl@afterlang\relax
  \let\bbl@language@opts\@empty
\fi
\input switch.def\relax
\ifx\bbl@languages\@undefined
  \ifx\directlua\@undefined
    \openin1 = language.def
    \ifeof1
      \closein1
      \message{I couldn't find the file language.def}
    \else
      \closein1
      \begingroup
        \def\addlanguage#1#2#3#4#5{%
          \expandafter\ifx\csname lang@#1\endcsname\relax\else
            \global\expandafter\let\csname l@#1\expandafter\endcsname
              \csname lang@#1\endcsname
          \fi}%
        \def\uselanguage#1{}%
        \input language.def
      \endgroup
    \fi
  \fi
  \chardef\l@english\z@
\fi
<@Load patterns in luatex@>
<@Basic macros@>
%    \end{macrocode}
%
%  \begin{macro}{\addto}
%    For each language four control sequences have to be defined that
%    control the language-specific definitions. To be able to add
%    something to these macro once they have been defined the macro
%    |\addto| is introduced. It takes two arguments, a \meta{control
%    sequence} and \TeX-code to be added to the \meta{control
%    sequence}.
%
%    If the \meta{control sequence} has not been defined before it is
%    defined now.  The control sequence could also expand to |\relax|,
%    in which case a circular definition results. The net result is a
%    stack overflow.  Otherwise the replacement text for the
%    \meta{control sequence} is expanded and stored in a token
%    register, together with the \TeX-code to be added.  Finally the
%    \meta{control sequence} is \emph{re}defined, using the contents
%    of the token register.
%
%    \begin{macrocode}
\def\addto#1#2{%
  \ifx#1\@undefined
    \def#1{#2}%
  \else
    \ifx#1\relax
      \def#1{#2}%
    \else
      {\toks@\expandafter{#1#2}%
       \xdef#1{\the\toks@}}%
    \fi
  \fi}
%    \end{macrocode}
%
%  \end{macro}
%
% \changes{babel~3.9a}{2012/08/10}{Removed the \cs{peek@token} and
%    \textsc{test@token} stuff}
%
%    The macro |\initiate@active@char| takes all the necessary actions
%    to make its argument a shorthand character. The real work is
%    performed once for each character.
%
% \changes{babel~3.9a}{1999/04/30}{Added \cs{bbl@withactive}}
%
%    \begin{macrocode}
\def\bbl@withactive#1#2{%
  \begingroup
    \lccode`~=`#2\relax
    \lowercase{\endgroup#1~}}
%    \end{macrocode}
%
%
%  \begin{macro}{\bbl@redefine}
%    To redefine a command, we save the old meaning of the macro.
%    Then we redefine it to call the original macro with the
%    `sanitized' argument.  The reason why we do it this way is that
%    we don't want to redefine the \LaTeX\ macros completely in case
%    their definitions change (they have changed in the past).
%
%    Because we need to redefine a number of commands we define the
%    command |\bbl@redefine| which takes care of this. It creates a
%    new control sequence, |\org@...|
%
%    \begin{macrocode}
\def\bbl@redefine#1{%
  \edef\bbl@tempa{\bbl@stripslash#1}%
  \expandafter\let\csname org@\bbl@tempa\endcsname#1%
  \expandafter\def\csname\bbl@tempa\endcsname}
%    \end{macrocode}
%
%    This command should only be used in the preamble of the document.
%
%    \begin{macrocode}
\@onlypreamble\bbl@redefine
%    \end{macrocode}
%
%  \end{macro}
%
%  \begin{macro}{\bbl@redefine@long}
%    This version of |\babel@redefine| can be used to redefine |\long|
%    commands such as |\ifthenelse|.
%
%    \begin{macrocode}
\def\bbl@redefine@long#1{%
  \edef\bbl@tempa{\bbl@stripslash#1}%
  \expandafter\let\csname org@\bbl@tempa\endcsname#1%
  \expandafter\long\expandafter\def\csname\bbl@tempa\endcsname}
\@onlypreamble\bbl@redefine@long
%    \end{macrocode}
%
%  \end{macro}
%
%  \begin{macro}{\bbl@redefinerobust}
%    For commands that are redefined, but which \textit{might} be
%    robust we need a slightly more intelligent macro. A robust
%    command |foo| is defined to expand to |\protect|\verb*|\foo |. So
%    it is necessary to check whether \verb*|\foo | exists. The result
%    is that the command that is being redefined is always robust
%    afterwards.  Therefore all we need to do now is define \verb*|\foo |.
%
%    \begin{macrocode}
\def\bbl@redefinerobust#1{%
  \edef\bbl@tempa{\bbl@stripslash#1}%
  \bbl@ifunset{\bbl@tempa\space}%
    {\expandafter\let\csname org@\bbl@tempa\endcsname#1%
     \bbl@exp{\def\\#1{\\\protect\<\bbl@tempa\space>}}}%
    {\bbl@exp{\let\<org@\bbl@tempa>\<\bbl@tempa\space>}}%
    \@namedef{\bbl@tempa\space}}
%    \end{macrocode}
%
%    This command should only be used in the preamble of the document.
%
%    \begin{macrocode}
\@onlypreamble\bbl@redefinerobust
%    \end{macrocode}
%
%  \end{macro}
%
%  \subsection{Hooks}
%
%  Note they are loaded in babel.def. switch.def only provides a
%  ``hook'' for hooks (with a default value which is a no-op,
%  below). Admittedly, the current implementation is a somewhat
%  simplistic and does vety little to catch errors, but it is intended
%  for developpers, after all. |\bbl@usehooks| is the commands used by
%  babel to execute hooks defined for an event.
%
% \changes{babel~3.9k}{2014/03/23}{Removed a \cs{newcommand},
%    undefined in Plain}
%
%    \begin{macrocode}
\def\AddBabelHook#1#2{%
  \bbl@ifunset{bbl@hk@#1}{\EnableBabelHook{#1}}{}%
  \def\bbl@tempa##1,#2=##2,##3\@empty{\def\bbl@tempb{##2}}%
  \expandafter\bbl@tempa\bbl@evargs,#2=,\@empty
  \bbl@ifunset{bbl@ev@#1@#2}%
    {\bbl@csarg\bbl@add{ev@#2}{\bbl@elt{#1}}%
     \bbl@csarg\newcommand}%
    {\bbl@csarg\let{ev@#1@#2}\relax
     \bbl@csarg\newcommand}%
  {ev@#1@#2}[\bbl@tempb]}
\def\EnableBabelHook#1{\bbl@csarg\let{hk@#1}\@firstofone}
\def\DisableBabelHook#1{\bbl@csarg\let{hk@#1}\@gobble}
\def\bbl@usehooks#1#2{%
  \def\bbl@elt##1{%
    \@nameuse{bbl@hk@##1}{\@nameuse{bbl@ev@##1@#1}#2}}%
  \@nameuse{bbl@ev@#1}}
%    \end{macrocode}
%
%    To ensure forward compatibility, arguments in hooks are set
%    implicitly. So, if a further argument is added in the future,
%    there is no need to change the existing code. Note events
%    intended for \textsf{hyphen.cfg} are also loaded (just in
%    case you need them for some reason). 
%
%    \begin{macrocode}
\def\bbl@evargs{,% don't delete the comma
  everylanguage=1,loadkernel=1,loadpatterns=1,loadexceptions=1,%
  adddialect=2,patterns=2,defaultcommands=0,encodedcommands=2,write=0,%
  beforeextras=0,afterextras=0,stopcommands=0,stringprocess=0,%
  hyphenation=2,initiateactive=3,afterreset=0,foreign=0,foreign*=0}
%    \end{macrocode}
%
% \begin{macro}{\babelensure}
%
% The user command just parses the optional argument and creates a
% new macro named |\bbl@e@|\m{language}. We register a hook at the
% |afterextras| event which just executes this macro in a
% ``complete'' selection (which, if undefined, is |\relax| and does
% nothing). This part is somewhat involved because we have to make
% sure things are expanded the correct number of times.
%
% The macro |\bbl@e@|\m{language} contains
% |\bbl@ensure|\marg{include}\marg{exclude}\marg{fontenc}, which in in
% turn loops over the macros names in |\bbl@captionslist|, excluding
% (with the help of |\in@|) those in the |exclude| list. If the
% |fontenc| is given (and not |\relax|), the |\fontencoding| is also
% added. Then we loop over the |include| list, but if the macro
% already contains |\foreignlanguage|, nothing is done.  Note this
% macro (1) is not restricted to the preamble, and (2) changes are
% local.
%
% \changes{babel~3.9i}{2014/02/14}{Macro \cs{babelensure} added}
% \changes{babel~3.9k}{2014/03/23}{Encapsulate \cs{foreignlanguage} in
%   \cs{bbl@ensure@}language, to "protect" strings}
% \changes{babel~3.9s}{2017/04/10}{Bug fix - extra spaces because a
%   missing percent}
% \changes{babel~3.9s}{2017/04/10}{\cs{bbl@ensure@lang} defined only
%   once}
% \changes{babel~3.10}{2017/05/06}{\cs{bbl@ensured} renamed to
%   \cs{bbl@captionslist} (for \cs{babelprovide}), which means
%   \cs{today} must be given explicitly in \cs{babelensure}}
%
%    \begin{macrocode}
\newcommand\babelensure[2][]{%  TODO - revise test files
  \AddBabelHook{babel-ensure}{afterextras}{%
    \ifcase\bbl@select@type
      \@nameuse{bbl@e@\languagename}%
    \fi}%
  \begingroup
    \let\bbl@ens@include\@empty
    \let\bbl@ens@exclude\@empty
    \def\bbl@ens@fontenc{\relax}%
    \def\bbl@tempb##1{%
      \ifx\@empty##1\else\noexpand##1\expandafter\bbl@tempb\fi}%
    \edef\bbl@tempa{\bbl@tempb#1\@empty}%
    \def\bbl@tempb##1=##2\@@{\@namedef{bbl@ens@##1}{##2}}%
    \bbl@foreach\bbl@tempa{\bbl@tempb##1\@@}%
    \def\bbl@tempc{\bbl@ensure}%
    \expandafter\bbl@add\expandafter\bbl@tempc\expandafter{%
      \expandafter{\bbl@ens@include}}%
    \expandafter\bbl@add\expandafter\bbl@tempc\expandafter{%
      \expandafter{\bbl@ens@exclude}}%
    \toks@\expandafter{\bbl@tempc}%
    \bbl@exp{%
  \endgroup
  \def\<bbl@e@#2>{\the\toks@{\bbl@ens@fontenc}}}}
\def\bbl@ensure#1#2#3{% 1: include 2: exclude 3: fontenc
  \def\bbl@tempb##1{% elt for (excluding) \bbl@captionslist list
    \ifx##1\@empty\else
      \in@{##1}{#2}%
      \ifin@\else
        \bbl@ifunset{bbl@ensure@\languagename}%
          {\bbl@exp{%
            \\\DeclareRobustCommand\<bbl@ensure@\languagename>[1]{%
              \\\foreignlanguage{\languagename}%
              {\ifx\relax#3\else
                \\\fontencoding{#3}\\\selectfont
               \fi
               ########1}}}}%
          {}%
        \toks@\expandafter{##1}%
        \edef##1{%
           \bbl@csarg\noexpand{ensure@\languagename}%
           {\the\toks@}}%
      \fi
      \expandafter\bbl@tempb
    \fi}%
  \expandafter\bbl@tempb\bbl@captionslist\today\@empty
  \def\bbl@tempa##1{% elt for include list
    \ifx##1\@empty\else
      \bbl@csarg\in@{ensure@\languagename\expandafter}\expandafter{##1}%
      \ifin@\else
        \bbl@tempb##1\@empty
      \fi
      \expandafter\bbl@tempa
    \fi}%
  \bbl@tempa#1\@empty} 
\def\bbl@captionslist{%
  \prefacename\refname\abstractname\bibname\chaptername\appendixname
  \contentsname\listfigurename\listtablename\indexname\figurename
  \tablename\partname\enclname\ccname\headtoname\pagename\seename
  \alsoname\proofname\glossaryname}   
%    \end{macrocode}
%
%    \end{macro}
%
%  \subsection{Setting up language files}
%
% \begin{macro}{\LdfInit}
%    The second version of |\LdfInit| macro takes two arguments. The first
%    argument is the name of the language that will be defined in the
%    language definition file; the second argument is either a control
%    sequence or a string from which a control sequence should be
%    constructed. The existence of the control sequence indicates that
%    the file has been processed before.
%
%    At the start of processing a language definition file we always
%    check the category code of the at-sign. We make sure that it is
%    a `letter' during the processing of the file. We also save its
%    name as the last called option, even if not loaded.
%
%    Another character that needs to have the correct category code
%    during processing of language definition files is the equals sign,
%    `=', because it is sometimes used in constructions with the
%    |\let| primitive. Therefore we store its current catcode and
%    restore it later on.
%
%    Now we check whether we should perhaps stop the processing of
%    this file. To do this we first need to check whether the second
%    argument that is passed to |\LdfInit| is a control sequence. We
%    do that by looking at the first token after passing |#2| through
%    |string|. When it is equal to |\@backslashchar| we are dealing
%    with a control sequence which we can compare with |\@undefined|.
%
% \changes{babel~3.9a}{2012/08/11}{\cs{ldf@quit} is not delayed any
%   more after \cs{fi} , since \cs{endinput} is not executed
%   immediately}
% \changes{babel~3.9g}{2012/08/11}{Preset the ``family'' of macros
%   \cs{Babel}...}
%
%   If so, we call |\ldf@quit| to set the main language, restore the
%   category code of the @-sign and call |\endinput|
%
%    When |#2| was \emph{not} a control sequence we construct one and
%    compare it with |\relax|.
%
%    Finally we check |\originalTeX|.
%
%    \begin{macrocode}
\def\bbl@ldfinit{%
  \let\bbl@screset\@empty
  \let\BabelStrings\bbl@opt@string
  \let\BabelOptions\@empty
  \let\BabelLanguages\relax
  \ifx\originalTeX\@undefined
    \let\originalTeX\@empty
  \else
    \originalTeX
  \fi}
\def\LdfInit#1#2{%
  \chardef\atcatcode=\catcode`\@
  \catcode`\@=11\relax
  \chardef\eqcatcode=\catcode`\=
  \catcode`\==12\relax
  \expandafter\if\expandafter\@backslashchar
                 \expandafter\@car\string#2\@nil
    \ifx#2\@undefined\else
      \ldf@quit{#1}%
    \fi
  \else
    \expandafter\ifx\csname#2\endcsname\relax\else
      \ldf@quit{#1}%
    \fi
  \fi
  \bbl@ldfinit}
%    \end{macrocode}
%
%  \end{macro}
%
%  \begin{macro}{\ldf@quit}
%    This macro interrupts the processing of a language definition file.
%
%    \begin{macrocode}
\def\ldf@quit#1{%
  \expandafter\main@language\expandafter{#1}%
  \catcode`\@=\atcatcode \let\atcatcode\relax
  \catcode`\==\eqcatcode \let\eqcatcode\relax
  \endinput}
%    \end{macrocode}
%
%  \end{macro}
%
%  \begin{macro}{\ldf@finish}
%    This macro takes one argument. It is the name of the language
%    that was defined in the language definition file.
%
%    We load the local configuration file if one is present, we set
%    the main language (taking into account that the argument might be
%    a control sequence that needs to be expanded) and reset the
%    category code of the @-sign.
%
% \changes{babel~3.9a}{2012/10/01}{Added \cs{bbl@afterlang} which
%    executes the code delayed with \cs{AfterBabelLanguage}}
%
%    \begin{macrocode}
\def\bbl@afterldf#1{%
  \bbl@afterlang
  \let\bbl@afterlang\relax
  \let\BabelModifiers\relax
  \let\bbl@screset\relax}%
\def\ldf@finish#1{%
  \loadlocalcfg{#1}%
  \bbl@afterldf{#1}%
  \expandafter\main@language\expandafter{#1}%
  \catcode`\@=\atcatcode \let\atcatcode\relax
  \catcode`\==\eqcatcode \let\eqcatcode\relax}
%    \end{macrocode}
%
%  \end{macro}
%
%    After the preamble of the document the commands |\LdfInit|,
%    |\ldf@quit| and |\ldf@finish| are no longer needed. Therefore
%    they are turned into warning messages in \LaTeX.
%
%    \begin{macrocode}
\@onlypreamble\LdfInit
\@onlypreamble\ldf@quit
\@onlypreamble\ldf@finish
%    \end{macrocode}
%
%  \begin{macro}{\main@language}
%  \begin{macro}{\bbl@main@language}
%    This command should be used in the various language definition
%    files. It stores its argument in |\bbl@main@language|; to be used
%    to switch to the correct language at the beginning of the
%    document.
%
% \changes{babel~3.8l}{2008/07/06}{Use \cs{bbl@patterns}}
%
%    \begin{macrocode}
\def\main@language#1{%
  \def\bbl@main@language{#1}%
  \let\languagename\bbl@main@language
  \bbl@patterns{\languagename}}
%    \end{macrocode}
%
%    We also have to make sure that some code gets executed at the
%    beginning of the document.
%
%    \begin{macrocode}
\AtBeginDocument{%
  \expandafter\selectlanguage\expandafter{\bbl@main@language}}
%    \end{macrocode}
%
%  \end{macro}
%  \end{macro}
%   
% A bit of optimization. Select in heads/foots the language only if
% necessary. 
%
%    \begin{macrocode}
\def\select@language@x#1{%
  \ifcase\bbl@select@type
    \bbl@ifsamestring\languagename{#1}{}{\select@language{#1}}%
  \else
    \select@language{#1}%
  \fi}
%    \end{macrocode}
%
% \subsection{Shorthands}
%
%  \begin{macro}{\bbl@add@special}
%    The macro |\bbl@add@special| is used to add a new character (or
%    single character control sequence) to the macro |\dospecials|
%    (and |\@sanitize| if \LaTeX\ is used). It is used only at one
%    place, namely when |\initiate@active@char| is called (which is
%    ignored if the char has been made active before).  Because
%    |\@sanitize| can be undefined, we put the definition inside a
%    conditional.
%
%    Items are added to the lists without checking its existence or
%    the original catcode. It does not hurt, but should be fixed. It's
%    already done with |\nfss@catcodes|, added in 3.10.
%
% \changes{3.10}{2017/05/14}{Refactored. Add to \cs{nfss@catcodes} too.}
%
%    \begin{macrocode}
\def\bbl@add@special#1{% 1:a macro like \", \?, etc.
  \bbl@add\dospecials{\do#1}% test @sanitize = \relax, for back. compat.
  \bbl@ifunset{@sanitize}{}{\bbl@add\@sanitize{\@makeother#1}}%
  \ifx\nfss@catcodes\@undefined\else % TODO - same for above
    \begingroup
      \catcode`#1\active
      \nfss@catcodes
      \ifnum\catcode`#1=\active
        \endgroup
        \bbl@add\nfss@catcodes{\@makeother#1}%
      \else
        \endgroup
      \fi
  \fi}
%    \end{macrocode}
%
%  \end{macro}
%
%  \begin{macro}{\bbl@remove@special}
%
%    The companion of the former macro is |\bbl@remove@special|.  It
%    removes a character from the set macros |\dospecials| and
%    |\@sanitize|, but it is not used at all in the \babel{} core.
%
%    \begin{macrocode}
\def\bbl@remove@special#1{%
  \begingroup
    \def\x##1##2{\ifnum`#1=`##2\noexpand\@empty
                 \else\noexpand##1\noexpand##2\fi}%
    \def\do{\x\do}%
    \def\@makeother{\x\@makeother}%
  \edef\x{\endgroup
    \def\noexpand\dospecials{\dospecials}%
    \expandafter\ifx\csname @sanitize\endcsname\relax\else
      \def\noexpand\@sanitize{\@sanitize}%
    \fi}%
  \x}
%    \end{macrocode}
%
%  \end{macro}
%
%  \begin{macro}{\initiate@active@char}
%    A language definition file can call this macro to make a
%    character active. This macro takes one argument, the character
%    that is to be made active. When the character was already active
%    this macro does nothing. Otherwise, this macro defines the
%    control sequence |\normal@char|\m{char} to expand to the
%    character in its `normal state' and it defines the active
%    character to expand to |\normal@char|\m{char} by default
%    (\m{char} being the character to be made active). Later its
%    definition can be changed to expand to |\active@char|\m{char}
%    by calling |\bbl@activate{|\m{char}|}|.
%
%    For example, to make the double quote character active one could
%    have |\initiate@active@char{"}| in a language definition file.
%    This defines |"| as |\active@prefix "\active@char"| (where the
%    first |"| is the character with its original catcode, when the
%    shorthand is created, and |\active@char"| is a single token). In
%    protected contexts, it expands to |\protect "| or |\noexpand "|
%    (ie, with the original |"|); otherwise |\active@char"| is
%    executed. This macro in turn expands to |\normal@char"| in
%    ``safe'' contexts (eg, |\label|), but |\user@active"| in normal
%    ``unsafe'' ones. The latter search a definition in the user,
%    language and system levels, in this order, but if none is found,
%    |\normal@char"| is used.  However, a deactivated shorthand (with
%    |\bbl@deactivate| is defined as |\active@prefix "\normal@char"|.
%
%    The following macro is used to define shorthands in the three
%    levels. It takes 4 arguments: the (string'ed) character,
%    |\<level>@group|, |<level>@active| and |<next-level>@active|
%    (except in |system|).
%
% \changes{babel~3.9a}{2012/08/18}{New macro, with code from
%    \cs{@initiate@active@char}}
%
%    \begin{macrocode}
\def\bbl@active@def#1#2#3#4{%
  \@namedef{#3#1}{%
    \expandafter\ifx\csname#2@sh@#1@\endcsname\relax
      \bbl@afterelse\bbl@sh@select#2#1{#3@arg#1}{#4#1}%
    \else
      \bbl@afterfi\csname#2@sh@#1@\endcsname
    \fi}%
%    \end{macrocode}
%
%    When there is also no current-level shorthand with an argument we
%    will check whether there is a next-level  defined shorthand for
%    this active character. 
%
%    \begin{macrocode}
  \long\@namedef{#3@arg#1}##1{%
    \expandafter\ifx\csname#2@sh@#1@\string##1@\endcsname\relax
      \bbl@afterelse\csname#4#1\endcsname##1%
    \else
      \bbl@afterfi\csname#2@sh@#1@\string##1@\endcsname
    \fi}}%
%    \end{macrocode}
%
% \changes{babel~3.9a}{2012/08/18}{Removed an extra hash. Now calls
%    \cs{@initiate@active@char} with 3 arguments.}
%
%    |\initiate@active@char| calls |\@initiate@active@char| with 3
%    arguments. All of them are the same character with different
%    catcodes: active, other (|\string|'ed) and the original one.
%    This trick simplifies the code a lot.
%
%    \begin{macrocode}
\def\initiate@active@char#1{%
  \bbl@ifunset{active@char\string#1}%
    {\bbl@withactive
      {\expandafter\@initiate@active@char\expandafter}#1\string#1#1}%
    {}}
%    \end{macrocode}
%
% \changes{babel~3.9e}{2012/08/18}{Introduced the 3-argument
%   \cs{@initiate@active@char}, with different catcodes: active,
%   string'ed, and original. Reorganized}
% \changes{babel~3.9a}{2012/08/19}{The catcode is saved}
% \changes{babel~3.9a}{2012/09/09}{The original definition is saved,
% too}
% \changes{babel~3.9a}{2012/12/27}{Take into account mathematically
%   active chars, to avoid infinite loops}
%
%   The very first thing to do is saving the original catcode and the
%   original definition, even if not active, which is possible
%   (undefined characters require a special treatement to avoid
%   making them |\relax|).  
%
%    \begin{macrocode}
\def\@initiate@active@char#1#2#3{%
  \bbl@csarg\edef{oricat@#2}{\catcode`#2=\the\catcode`#2\relax}%
  \ifx#1\@undefined
    \bbl@csarg\edef{oridef@#2}{\let\noexpand#1\noexpand\@undefined}%
  \else
    \bbl@csarg\let{oridef@@#2}#1%
    \bbl@csarg\edef{oridef@#2}{%
      \let\noexpand#1%
      \expandafter\noexpand\csname bbl@oridef@@#2\endcsname}%
  \fi
%    \end{macrocode}
%
%    If the character is already active we provide the default
%    expansion under this shorthand mechanism. Otherwise we write a
%    message in the transcript file, and define |\normal@char|\m{char}
%    to expand to the character in its default state. If the character
%    is mathematically active when \babel{} is loaded (for example
%    |'|) the normal expansion is somewhat different to avoid an
%    infinite loop (but it does not prevent the loop if the mathcode
%    is set to |"8000| \textit{a posteriori}).
%
%    \begin{macrocode}
  \ifx#1#3\relax
    \expandafter\let\csname normal@char#2\endcsname#3%
  \else
    \bbl@info{Making #2 an active character}%
    \ifnum\mathcode`#2="8000
      \@namedef{normal@char#2}{%
        \textormath{#3}{\csname bbl@oridef@@#2\endcsname}}%
    \else
      \@namedef{normal@char#2}{#3}%
    \fi
%    \end{macrocode}
%
%    To prevent problems with the loading of other packages after
%    \babel\ we reset the catcode of the character to the original one
%    at the end of the package and of each language file (except with
%    \textsf{KeepShorthandsActive}). It is re-activate again at
%    |\begin{document}|.  We also need to make sure that the
%    shorthands are active during the processing of the \file{.aux}
%    file. Otherwise some citations may give unexpected results in
%    the printout when a shorthand was used in the optional argument
%    of |\bibitem| for example. Then we make it active (not strictly
%    necessary, but done for backward compatibility).
%
%    \begin{macrocode}
    \bbl@restoreactive{#2}%
    \AtBeginDocument{%
      \catcode`#2\active
      \if@filesw
        \immediate\write\@mainaux{\catcode`\string#2\active}%
      \fi}%
    \expandafter\bbl@add@special\csname#2\endcsname
    \catcode`#2\active
  \fi
%    \end{macrocode}
%
%    Now we have set |\normal@char|\m{char}, we must define
%    |\active@char|\m{char}, to be executed when the character is
%    activated.  We define the first level expansion of
%    |\active@char|\m{char} to check the status of the |@safe@actives|
%    flag. If it is set to true we expand to the `normal' version of
%    this character, otherwise we call |\user@active|\m{char} to start
%    the search of a definition in the user, language and system
%    levels (or eventually |normal@char|\m{char}).
%
% \changes{babel~3.9a}{2012/12/27}{Added code for option math=normal}
% \changes{babel~3.9i}{2014/02/03}{Don't call directly
%    \cs{user@active}, but with an intermediate step}
%
%    \begin{macrocode}
  \let\bbl@tempa\@firstoftwo
  \if\string^#2%
    \def\bbl@tempa{\noexpand\textormath}%
  \else
    \ifx\bbl@mathnormal\@undefined\else
      \let\bbl@tempa\bbl@mathnormal
    \fi
  \fi
  \expandafter\edef\csname active@char#2\endcsname{%
    \bbl@tempa
      {\noexpand\if@safe@actives
         \noexpand\expandafter
         \expandafter\noexpand\csname normal@char#2\endcsname
       \noexpand\else
         \noexpand\expandafter
         \expandafter\noexpand\csname bbl@doactive#2\endcsname
       \noexpand\fi}%
     {\expandafter\noexpand\csname normal@char#2\endcsname}}%
  \bbl@csarg\edef{doactive#2}{%
    \expandafter\noexpand\csname user@active#2\endcsname}%
%    \end{macrocode}
%
% \changes{babel~3.9a}{2012/12/27}{Shorthands are not defined
%    directly, but with a couple of intermediate macros}
% 
%    We now define the default values which the shorthand is set to
%    when activated or deactivated. It is set to the deactivated form
%    (globally), so that the character expands to
%    \begin{center}
%    |\active@prefix| \m{char} |\normal@char|\m{char}
%    \end{center}
%    (where |\active@char|\m{char} is \emph{one} control sequence!).
%
%    \begin{macrocode}
  \bbl@csarg\edef{active@#2}{%
    \noexpand\active@prefix\noexpand#1%
    \expandafter\noexpand\csname active@char#2\endcsname}%
  \bbl@csarg\edef{normal@#2}{%
    \noexpand\active@prefix\noexpand#1%
    \expandafter\noexpand\csname normal@char#2\endcsname}%
  \expandafter\let\expandafter#1\csname bbl@normal@#2\endcsname
%    \end{macrocode}
%
%    The next level of the code checks whether a user has defined a
%    shorthand for himself with this character. First we check for a
%    single character shorthand. If that doesn't exist we check for a
%    shorthand with an argument.
%
% \changes{babel~3.8b}{2004/04/19}{Now use \cs{bbl@sh@select}}
% \changes{babel~3.9a}{2012/08/18}{Instead of the ``copy-paste pattern''
% a new macro is used}
%
%    \begin{macrocode}
  \bbl@active@def#2\user@group{user@active}{language@active}%
  \bbl@active@def#2\language@group{language@active}{system@active}%
  \bbl@active@def#2\system@group{system@active}{normal@char}%
%    \end{macrocode}
%
%    In order to do the right thing when a shorthand with an argument
%    is used by itself at the end of the line we provide a definition
%    for the case of an empty argument. For that case we let the
%    shorthand character expand to its non-active self. Also, When a
%    shorthand combination such as |''| ends up in a heading \TeX\
%    would see |\protect'\protect'|. To prevent this from happening a
%    couple of shorthand needs to be defined at user level.
%
% \changes{babel~3.9a}{2012/8/18}{Use \cs{user@group}, as above,
%    instead of the hardwired \texttt{user}}
%
%    \begin{macrocode}
  \expandafter\edef\csname\user@group @sh@#2@@\endcsname
    {\expandafter\noexpand\csname normal@char#2\endcsname}%
  \expandafter\edef\csname\user@group @sh@#2@\string\protect@\endcsname
    {\expandafter\noexpand\csname user@active#2\endcsname}%
%    \end{macrocode}
%
%    Finally, a couple of special cases are taken care of.  (1) If we
%    are making the right quote (|'|) active we need to change |\pr@m@s| as
%    well.  Also, make sure that a single |'| in math mode `does the
%    right thing'.  (2) If we are using the caret (|^|) as a shorthand
%    character special care should be taken to make sure math still
%    works. Therefore an extra level of expansion is introduced with a
%    check for math mode on the upper level.
%
% \changes{babel~3.9a}{2012/09/11}{The output routine resets the quote
%   to \cs{active@math@prime}, so we redefine the latter with the new
%  ``normal'' value}
% \changes{babel~3.9a}{2012/06/20}{Added a couple of missing
%    comment characters (PR 4146)}
% \changes{babel~3.9a}{2012/07/29}{Use \cs{textormath} instead of
%    \cs{ifmath}}
% \changes{babel~3.9a}{2012/11/26}{Compare the char, irrespective of
%    its catcode.}
% \changes{babel~3.9a}{2012/12/27}{Removed the redeclaration of
%   \cs{normal@char'} because it is handled in a generic way above}
% \changes{babel~3.9a}{2012/12/29}{Removed the intermediate step of
%   \cs{bbl@act@caret} and moved above} 
% \changes{babel~3.9i}{2012/12/29}{Added the event \cs{initiateactive}} 
%
%    \begin{macrocode}
  \if\string'#2%
    \let\prim@s\bbl@prim@s
    \let\active@math@prime#1%
  \fi
  \bbl@usehooks{initiateactive}{{#1}{#2}{#3}}}
%    \end{macrocode}
%
%    The following package options control the behaviour of shorthands
%    in math mode.
%
%    \begin{macrocode}
%<<*More package options>>
\DeclareOption{math=active}{}
\DeclareOption{math=normal}{\def\bbl@mathnormal{\noexpand\textormath}}
%<</More package options>>
%    \end{macrocode}
%
%    Initiating a shorthand makes active the char. That is not
%    strictly necessary but it is still done for backward
%    compatibility. So we need to restore the original catcode at the
%    end of package \textit{and} and the end of the |ldf|.
%
% \changes{babel~3.9a}{2012/07/04}{Catcodes are also restored after
%    each language, to prevent incompatibilities. Use \cs{string} instead
%    of \cs{noexpand} and add \cs{relax}}
% \changes{babel~3.9a}{2012/10/18}{Catcodes are deactivated in a separate
%    macro, which is made no-op when babel exits}
%
%    \begin{macrocode}
\@ifpackagewith{babel}{KeepShorthandsActive}%
  {\let\bbl@restoreactive\@gobble}%
  {\def\bbl@restoreactive#1{%
     \bbl@exp{%
       \\\AfterBabelLanguage\\\CurrentOption
         {\catcode`#1=\the\catcode`#1\relax}%
       \\\AtEndOfPackage
         {\catcode`#1=\the\catcode`#1\relax}}}%
   \AtEndOfPackage{\let\bbl@restoreactive\@gobble}}
%    \end{macrocode}
%
%  \end{macro}
%
%  \begin{macro}{\bbl@sh@select}
%    This command helps the shorthand supporting macros to select how
%    to proceed. Note that this macro needs to be expandable as do all
%    the shorthand macros in order for them to work in expansion-only
%    environments such as the argument of |\hyphenation|.
%
%    This macro expects the name of a group of shorthands in its first
%    argument and a shorthand character in its second argument. It
%    will expand to either |\bbl@firstcs| or |\bbl@scndcs|. Hence two
%    more arguments need to follow it.
%
% \changes{babel~3.8b}{2004/04/19}{Added command}
% \changes{babel~3.9a}{2012/08/18}{Removed \cs{string}s, because the
%   char are already string'ed}
%
%    \begin{macrocode}
\def\bbl@sh@select#1#2{%
  \expandafter\ifx\csname#1@sh@#2@sel\endcsname\relax
    \bbl@afterelse\bbl@scndcs
  \else
    \bbl@afterfi\csname#1@sh@#2@sel\endcsname
  \fi}
%    \end{macrocode}
%
%  \end{macro}
%
%  \begin{macro}{\active@prefix}
%    The command |\active@prefix| which is used in the expansion of
%    active characters has a function similar to |\OT1-cmd| in that it
%    |\protect|s the active character whenever |\protect| is
%    \emph{not} |\@typeset@protect|.
%
%    \begin{macrocode}
\def\active@prefix#1{%
  \ifx\protect\@typeset@protect
  \else
%    \end{macrocode}
%
%    When |\protect| is set to |\@unexpandable@protect| we make sure
%    that the active character is als \emph{not} expanded by inserting
%    |\noexpand| in front of it. The |\@gobble| is needed to remove
%    a token such as |\activechar:| (when the double colon was the
%    active character to be dealt with).
%
%    \begin{macrocode}
    \ifx\protect\@unexpandable@protect
      \noexpand#1%
    \else
      \protect#1%
    \fi
    \expandafter\@gobble
  \fi}
%    \end{macrocode}
%
%  \end{macro}
%
%  \begin{macro}{\if@safe@actives}
%    In some circumstances it is necessary to be able to change the
%    expansion of an active character on the fly. For this purpose the
%    switch |@safe@actives| is available. The setting of this switch
%    should be checked in the first level expansion of
%    |\active@char|\m{char}.
%
%    \begin{macrocode}
\newif\if@safe@actives
\@safe@activesfalse
%    \end{macrocode}
%
%  \end{macro}
%
%  \begin{macro}{\bbl@restore@actives}
%    When the output routine kicks in while the
%    active characters were made ``safe'' this must be undone in
%    the headers to prevent unexpected typeset results. For this
%    situation we define a command to make them ``unsafe'' again.
%
%    \begin{macrocode}
\def\bbl@restore@actives{\if@safe@actives\@safe@activesfalse\fi}
%    \end{macrocode}
%
%  \end{macro}
%
%  \begin{macro}{\bbl@activate}
%  \begin{macro}{\bbl@deactivate}
%
%  \changes{babel~3.9a}{2013/01/11}{\cs{bbl@withactive} makes sure the
%    catcode is active}
%
%    Both macros take one argument, like |\initiate@active@char|. The
%    macro is used to change the definition of an active character to
%    expand to |\active@char|\m{char} in the case of |\bbl@activate|,
%    or |\normal@char|\m{char} in the case of
%    |\bbl@deactivate|.
%
%    \begin{macrocode}
\def\bbl@activate#1{%
  \bbl@withactive{\expandafter\let\expandafter}#1%
    \csname bbl@active@\string#1\endcsname}
\def\bbl@deactivate#1{%
  \bbl@withactive{\expandafter\let\expandafter}#1%
    \csname bbl@normal@\string#1\endcsname}
%    \end{macrocode}
%
%  \end{macro}
%  \end{macro}
%
%  \begin{macro}{\bbl@firstcs}
%  \begin{macro}{\bbl@scndcs}
%    These macros have two arguments. They use one of their arguments
%    to build a control sequence from.
%
%    \begin{macrocode}
\def\bbl@firstcs#1#2{\csname#1\endcsname}
\def\bbl@scndcs#1#2{\csname#2\endcsname}
%    \end{macrocode}
%
%  \end{macro}
%  \end{macro}
%
%  \begin{macro}{\declare@shorthand}
%    The command |\declare@shorthand| is used to declare a shorthand
%    on a certain level. It takes three arguments:
%    \begin{enumerate}
%    \item a name for the collection of shorthands, i.e. `system', or
%      `dutch';
%    \item the character (sequence) that makes up the shorthand,
%      i.e. |~| or |"a|;
%    \item the code to be executed when the shorthand is encountered.
%    \end{enumerate}
%
% \changes{babel~3.8b}{2004/04/19}{We need to support shorthands with
%    and without argument in different groups; added the name of the
%    group to the storage macro}
% \changes{babel~3.9a}{2012/07/03}{Check if shorthands are redefined}
%
%    \begin{macrocode}
\def\declare@shorthand#1#2{\@decl@short{#1}#2\@nil}
\def\@decl@short#1#2#3\@nil#4{%
  \def\bbl@tempa{#3}%
  \ifx\bbl@tempa\@empty
    \expandafter\let\csname #1@sh@\string#2@sel\endcsname\bbl@scndcs
    \bbl@ifunset{#1@sh@\string#2@}{}%
      {\def\bbl@tempa{#4}%
       \expandafter\ifx\csname#1@sh@\string#2@\endcsname\bbl@tempa
       \else
         \bbl@info
           {Redefining #1 shorthand \string#2\\%
            in language \CurrentOption}%
       \fi}%
    \@namedef{#1@sh@\string#2@}{#4}%
  \else
    \expandafter\let\csname #1@sh@\string#2@sel\endcsname\bbl@firstcs
    \bbl@ifunset{#1@sh@\string#2@\string#3@}{}%
      {\def\bbl@tempa{#4}%
       \expandafter\ifx\csname#1@sh@\string#2@\string#3@\endcsname\bbl@tempa
       \else
         \bbl@info
           {Redefining #1 shorthand \string#2\string#3\\%
            in language \CurrentOption}%
       \fi}%
    \@namedef{#1@sh@\string#2@\string#3@}{#4}%
  \fi}
%    \end{macrocode}
%
%  \end{macro}
%
%  \begin{macro}{\textormath}
%    Some of the shorthands that will be declared by the language
%    definition files have to be usable in both text and mathmode. To
%    achieve this the helper macro |\textormath| is provided.
%
% \changes{babel~3.9a}{2012/12/29}{Failed if an argument had a
%    condicional. Use the more robust mechanism of \cs{XXXoftwo}} 
%
%    \begin{macrocode}
\def\textormath{%
  \ifmmode
    \expandafter\@secondoftwo
  \else
    \expandafter\@firstoftwo
  \fi}
%    \end{macrocode}
%
%  \end{macro}
%
%  \begin{macro}{\user@group}
%  \begin{macro}{\language@group}
%  \begin{macro}{\system@group}
%    The current concept of `shorthands' supports three levels or
%    groups of shorthands. For each level the name of the level or
%    group is stored in a macro. The default is to have a user group;
%    use language group `english' and have a system group called
%    `system'.
%
%    \begin{macrocode}
\def\user@group{user}
\def\language@group{english}
\def\system@group{system}
%    \end{macrocode}
%
%  \end{macro}
%  \end{macro}
%  \end{macro}
%
%  \begin{macro}{\useshorthands}
%    This is the user level command to tell \LaTeX\ that user level
%    shorthands will be used in the document. It takes one argument,
%    the character that starts a shorthand. First note that this is
%    user level, and then initialize and activate the character for
%    use as a shorthand character (ie, it's active in the
%    preamble). Languages can deactivate shorthands, so a starred
%    version is also provided which activates them always after the
%    language has been switched.
%
% \changes{babel~3.9a}{2012/08/05}{Now \cs{bbl@activate} makes sure
%    the catcode is active, so this part is simplified}
% \changes{babel~3.9a}{2012/08/12}{User shorhands can be
%   defined even with shorthands=off}
%
%    \begin{macrocode}
\def\useshorthands{%
  \@ifstar\bbl@usesh@s{\bbl@usesh@x{}}}
\def\bbl@usesh@s#1{%
  \bbl@usesh@x
    {\AddBabelHook{babel-sh-\string#1}{afterextras}{\bbl@activate{#1}}}%
    {#1}}
\def\bbl@usesh@x#1#2{%
  \bbl@ifshorthand{#2}%
    {\def\user@group{user}%
     \initiate@active@char{#2}%
     #1%
     \bbl@activate{#2}}%
    {\bbl@error
       {Cannot declare a shorthand turned off (\string#2)}
       {Sorry, but you cannot use shorthands which have been\\%
        turned off in the package options}}}
%    \end{macrocode}
%
%  \end{macro}
%
%  \begin{macro}{\defineshorthand}
%
% \changes{babel~3.9a}{2012/08/05}{Added optional argument, to provide
%    a way to (re)define language shorthands} 
% \changes{babel~3.9a}{2012/08/25}{Extended for language-dependent
%    user macros, with two new auxiliary macros}
%
%    Currently we only support two groups of user level shorthands,
%    named internally |user| and |user@<lang>| (language-dependent
%    user shorthands). By default, only the first one is taken into
%    account, but if the former is also used (in the optional argument
%    of |\defineshorthand|) a new level is inserted for it
%    (|user@generic|, done by |\bbl@set@user@generic|); we make also
%    sure |{}| and |\protect| are taken into account in this new top
%    level.
%
%    \begin{macrocode}
\def\user@language@group{user@\language@group}
\def\bbl@set@user@generic#1#2{%
  \bbl@ifunset{user@generic@active#1}%
    {\bbl@active@def#1\user@language@group{user@active}{user@generic@active}%
     \bbl@active@def#1\user@group{user@generic@active}{language@active}%
     \expandafter\edef\csname#2@sh@#1@@\endcsname{%
       \expandafter\noexpand\csname normal@char#1\endcsname}%
     \expandafter\edef\csname#2@sh@#1@\string\protect@\endcsname{%
       \expandafter\noexpand\csname user@active#1\endcsname}}%
  \@empty}
\newcommand\defineshorthand[3][user]{%
  \edef\bbl@tempa{\zap@space#1 \@empty}%
  \bbl@for\bbl@tempb\bbl@tempa{%
    \if*\expandafter\@car\bbl@tempb\@nil
      \edef\bbl@tempb{user@\expandafter\@gobble\bbl@tempb}%
      \@expandtwoargs
        \bbl@set@user@generic{\expandafter\string\@car#2\@nil}\bbl@tempb
    \fi
    \declare@shorthand{\bbl@tempb}{#2}{#3}}}
%    \end{macrocode}
%
%  \end{macro}
%
%  \begin{macro}{\languageshorthands}
%    A user level command to change the language from which shorthands
%    are used. Unfortunately, \babel{} currently does not keep track
%    of defined groups, and therefore there is no way to catch a
%    possible change in casing.
%
%    \begin{macrocode}
\def\languageshorthands#1{\def\language@group{#1}}
%    \end{macrocode}
%
%  \end{macro}
%
%  \begin{macro}{\aliasshorthand}
%    First the new shorthand needs to be initialized,
%
%    \begin{macrocode}
\def\aliasshorthand#1#2{%
  \bbl@ifshorthand{#2}%
    {\expandafter\ifx\csname active@char\string#2\endcsname\relax
       \ifx\document\@notprerr
         \@notshorthand{#2}%
       \else
         \initiate@active@char{#2}%
%    \end{macrocode}
%
% \changes{babel~3.9a}{2012/08/06}{Instead of letting the new shorthand to
%    the original char, which very often didn't work, we define it
%    directly}
% \changes{babel~3.9a}{2012/08/20}{Make sure both characters (old an
%    new) are active}
%
%    Then, we define the new shorthand in terms of the original
%    one, but note with |\aliasshorthands{"}{/}| is
%    |\active@prefix /\active@char/|, so we still need to let the
%    lattest to |\active@char"|.
%
%    \begin{macrocode}
         \expandafter\let\csname active@char\string#2\expandafter\endcsname
           \csname active@char\string#1\endcsname
         \expandafter\let\csname normal@char\string#2\expandafter\endcsname
           \csname normal@char\string#1\endcsname
         \bbl@activate{#2}%
       \fi
     \fi}%
    {\bbl@error
       {Cannot declare a shorthand turned off (\string#2)}
       {Sorry, but you cannot use shorthands which have been\\%
        turned off in the package options}}}
%    \end{macrocode}
%
%  \end{macro}
%
%  \begin{macro}{\@notshorthand}
%
% \changes{v3.8d}{2004/11/20}{Error message added}
%    
%    \begin{macrocode}
\def\@notshorthand#1{%
  \bbl@error{%
    The character `\string #1' should be made a shorthand character;\\%
    add the command \string\useshorthands\string{#1\string} to
    the preamble.\\%
    I will ignore your instruction}%
   {You may proceed, but expect unexpected results}}
%    \end{macrocode}
%
%  \end{macro}
%
%  \begin{macro}{\shorthandon}
%  \begin{macro}{\shorthandoff}
%    The first level definition of these macros just passes the
%    argument on to |\bbl@switch@sh|, adding |\@nil| at the end to
%    denote the end of the list of characters.
%
%    \begin{macrocode}
\newcommand*\shorthandon[1]{\bbl@switch@sh\@ne#1\@nnil}
\DeclareRobustCommand*\shorthandoff{%
  \@ifstar{\bbl@shorthandoff\tw@}{\bbl@shorthandoff\z@}}
\def\bbl@shorthandoff#1#2{\bbl@switch@sh#1#2\@nnil}
%    \end{macrocode}
%
%  \begin{macro}{\bbl@switch@sh}
%
%   \changes{babel~3.9a}{2013/02/21}{Code revised}
%
%    The macro |\bbl@switch@sh| takes the list of characters apart one
%    by  one and subsequently switches the category code of the
%    shorthand character according to the first argument of
%    |\bbl@switch@sh|.
%
%    But before any of this switching takes place we make sure that
%    the character we are dealing with is known as a shorthand
%    character. If it is, a macro such as |\active@char"| should
%    exist.
%
%    Switching off and on is easy -- we just set the category
%    code to `other' (12) and |\active|. With the starred version, the
%    original catcode and the original definition, saved
%    in |@initiate@active@char|, are restored. 
%
%    \begin{macrocode}
\def\bbl@switch@sh#1#2{%
  \ifx#2\@nnil\else
    \bbl@ifunset{bbl@active@\string#2}%
      {\bbl@error
         {I cannot switch `\string#2' on or off--not a shorthand}%
         {This character is not a shorthand. Maybe you made\\%
          a typing mistake? I will ignore your instruction}}%
      {\ifcase#1%
         \catcode`#212\relax
       \or
         \catcode`#2\active
       \or
         \csname bbl@oricat@\string#2\endcsname
         \csname bbl@oridef@\string#2\endcsname
       \fi}%
    \bbl@afterfi\bbl@switch@sh#1%
  \fi}
%    \end{macrocode}
%
%  \end{macro}
%  \end{macro}
%  \end{macro}
%
% \changes{babel~3.9a}{2012/06/16}{Added code}
%
% Note the value is that at the expansion time, eg, in the preample
% shorhands are usually deactivated.
%
%    \begin{macrocode}
\def\babelshorthand{\active@prefix\babelshorthand\bbl@putsh}
\def\bbl@putsh#1{%
  \bbl@ifunset{bbl@active@\string#1}%
     {\bbl@putsh@i#1\@empty\@nnil}%
     {\csname bbl@active@\string#1\endcsname}}
\def\bbl@putsh@i#1#2\@nnil{%
  \csname\languagename @sh@\string#1@%
    \ifx\@empty#2\else\string#2@\fi\endcsname}
\ifx\bbl@opt@shorthands\@nnil\else
  \let\bbl@s@initiate@active@char\initiate@active@char
  \def\initiate@active@char#1{%
    \bbl@ifshorthand{#1}{\bbl@s@initiate@active@char{#1}}{}}
  \let\bbl@s@switch@sh\bbl@switch@sh
  \def\bbl@switch@sh#1#2{%
    \ifx#2\@nnil\else
      \bbl@afterfi
      \bbl@ifshorthand{#2}{\bbl@s@switch@sh#1{#2}}{\bbl@switch@sh#1}%
    \fi}
  \let\bbl@s@activate\bbl@activate
  \def\bbl@activate#1{%
    \bbl@ifshorthand{#1}{\bbl@s@activate{#1}}{}}
  \let\bbl@s@deactivate\bbl@deactivate
  \def\bbl@deactivate#1{%
    \bbl@ifshorthand{#1}{\bbl@s@deactivate{#1}}{}}
\fi
%    \end{macrocode}
%
% \changes{babel~3.9a}{2012/12/27}{Removed redundant system declarations}
%
%  \begin{macro}{\bbl@prim@s}
%  \begin{macro}{\bbl@pr@m@s}
%
% \changes{babel~3.9a}{2012/07/29}{\cs{bbl@pr@m@s} rewritten to
%    take into account catcodes for both the quote and the hat}
%
%    One of the internal macros that are involved in substituting
%    |\prime| for each right quote in mathmode is |\prim@s|. This
%    checks if the next character is a right quote. When the right
%    quote is active, the definition of this macro needs to be adapted
%    to look also for an active right quote; the hat could be active,
%    too.
%
%    \begin{macrocode}
\def\bbl@prim@s{%
  \prime\futurelet\@let@token\bbl@pr@m@s}
\def\bbl@if@primes#1#2{%
  \ifx#1\@let@token
    \expandafter\@firstoftwo
  \else\ifx#2\@let@token
    \bbl@afterelse\expandafter\@firstoftwo
  \else
    \bbl@afterfi\expandafter\@secondoftwo
  \fi\fi}
\begingroup
  \catcode`\^=7  \catcode`\*=\active  \lccode`\*=`\^
  \catcode`\'=12 \catcode`\"=\active  \lccode`\"=`\' 
  \lowercase{%
    \gdef\bbl@pr@m@s{%
      \bbl@if@primes"'%
        \pr@@@s
        {\bbl@if@primes*^\pr@@@t\egroup}}}
\endgroup
%    \end{macrocode}
%
%  \end{macro}
%  \end{macro}
%
%    Usually the |~| is active and expands to \verb*=\penalty\@M\ =.
%    When it is written to the \file{.aux} file it is written
%    expanded. To prevent that and to be able to use the character |~|
%    as a start character for a shorthand, it is redefined here as a
%    one character shorthand on system level. The system declaration
%    is in most cases redundant (when |~| is still a non-break space),
%    and in some cases is inconvenient (if |~| has been redefined);
%    however, for backward compatibility it is maintained (some
%    existing documents may rely on the \babel{} value).
%
% \changes{babel~3.9i}{2014/02/06}{Moved from above, after
%    \cs{bbl@usehook} has been defined}
% \changes{babel~3.9k}{2014/02/06}{Moved again at the original place}
%
%    \begin{macrocode}
\initiate@active@char{~}
\declare@shorthand{system}{~}{\leavevmode\nobreak\ }
\bbl@activate{~}
%    \end{macrocode}
%
%  \begin{macro}{\OT1dqpos}
%  \begin{macro}{\T1dqpos}
%    The position of the double quote character is different for the
%    OT1 and T1 encodings. It will later be selected using the
%    |\f@encoding| macro. Therefore we define two macros here to store
%    the position of the character in these encodings.
%
%    \begin{macrocode}
\expandafter\def\csname OT1dqpos\endcsname{127}
\expandafter\def\csname T1dqpos\endcsname{4}
%    \end{macrocode}
%
%    When the macro |\f@encoding| is undefined (as it is in plain
%    \TeX) we define it here to expand to \texttt{OT1}
%
%    \begin{macrocode}
\ifx\f@encoding\@undefined
  \def\f@encoding{OT1}
\fi
%    \end{macrocode}
%
%  \end{macro}
%  \end{macro}
%
%  \subsection{Language attributes}
%
%    Language attributes provide a means to give the user control over
%    which features of the language definition files he wants to
%    enable.
%  \begin{macro}{\languageattribute}
%
% \changes{babel~3.9a}{2012/09/07}{Use \cs{@expandtwoargs} with
%   \cs{in@}}
%
%    The macro |\languageattribute| checks whether its arguments are
%    valid and then activates the selected language attribute.
%    First check whether the language is known, and then process each
%    attribute in the list.
%
%    \begin{macrocode}
\newcommand\languageattribute[2]{%
  \def\bbl@tempc{#1}%
  \bbl@fixname\bbl@tempc
  \bbl@iflanguage\bbl@tempc{%
    \bbl@vforeach{#2}{%
%    \end{macrocode}
%
%    We want to make sure that each attribute is selected only once;
%    therefore we store the already selected attributes in
%    |\bbl@known@attribs|. When that control sequence is not yet
%    defined this attribute is certainly not selected before.
%
%    \begin{macrocode}
      \ifx\bbl@known@attribs\@undefined
        \in@false
      \else
%    \end{macrocode}
%
%    Now we need to see if the attribute occurs in the list of
%    already selected attributes.
%
%    \begin{macrocode}
        \bbl@xin@{,\bbl@tempc-##1,}{,\bbl@known@attribs,}%
      \fi
%    \end{macrocode}
%
%    When the attribute was in the list we issue a warning; this might
%    not be the users intention.
%
%    \begin{macrocode}
      \ifin@
        \bbl@warning{%
          You have more than once selected the attribute '##1'\\%
          for language #1}%
      \else
%    \end{macrocode}
%
%    When we end up here the attribute is not selected before. So, we
%    add it to the list of selected attributes and execute the
%    associated \TeX-code.
%
%    \begin{macrocode}
        \bbl@exp{%
          \\\bbl@add@list\\\bbl@known@attribs{\bbl@tempc-##1}}%
        \edef\bbl@tempa{\bbl@tempc-##1}%
        \expandafter\bbl@ifknown@ttrib\expandafter{\bbl@tempa}\bbl@attributes%
        {\csname\bbl@tempc @attr@##1\endcsname}%
        {\@attrerr{\bbl@tempc}{##1}}%
     \fi}}}
%    \end{macrocode}
%
%    This command should only be used in the preamble of a document.
%
%    \begin{macrocode}
\@onlypreamble\languageattribute
%    \end{macrocode}
%
%    The error text to be issued when an unknown attribute is
%    selected.
%
%    \begin{macrocode}
\newcommand*{\@attrerr}[2]{%
  \bbl@error
    {The attribute #2 is unknown for language #1.}%
    {Your command will be ignored, type <return> to proceed}}
%    \end{macrocode}
%
%  \end{macro}
%
%  \begin{macro}{\bbl@declare@ttribute}
%    This command adds the new language/attribute combination to the
%    list of known attributes.
%
%    Then it defines a control sequence to be executed when the
%    attribute is used in a document. The result of this should be
%    that the macro |\extras...| for the current language is extended,
%    otherwise the attribute will not work as its code is removed from
%    memory at |\begin{document}|.
%
%    \begin{macrocode}
\def\bbl@declare@ttribute#1#2#3{%
  \bbl@xin@{,#2,}{,\BabelModifiers,}%
  \ifin@
    \AfterBabelLanguage{#1}{\languageattribute{#1}{#2}}%
  \fi
  \bbl@add@list\bbl@attributes{#1-#2}%
  \expandafter\def\csname#1@attr@#2\endcsname{#3}}
%    \end{macrocode}
%
%  \end{macro}
%
%  \begin{macro}{\bbl@ifattributeset}
%    This internal macro has 4 arguments. It can be used to interpret
%    \TeX\ code based on whether a certain attribute was set. This
%    command should appear inside the argument to |\AtBeginDocument|
%    because the attributes are set in the document preamble,
%    \emph{after} \babel\ is loaded.
%
%    The first argument is the language, the second argument the
%    attribute being checked, and the third and fourth arguments are
%    the true and false clauses.
%
%    \begin{macrocode}
\def\bbl@ifattributeset#1#2#3#4{%
%    \end{macrocode}
%
%    First we need to find out if any attributes were set; if not
%    we're done.
%
%    \begin{macrocode}
  \ifx\bbl@known@attribs\@undefined
    \in@false
  \else
%    \end{macrocode}
%
%    The we need to check the list of known attributes.
%
%    \begin{macrocode}
    \bbl@xin@{,#1-#2,}{,\bbl@known@attribs,}%
  \fi
%    \end{macrocode}
%
%    When we're this far |\ifin@| has a value indicating if the
%    attribute in question was set or not. Just to be safe the code to
%    be executed is `thrown over the |\fi|'.
%
%    \begin{macrocode}
  \ifin@
    \bbl@afterelse#3%
  \else
    \bbl@afterfi#4%
  \fi
  }
%    \end{macrocode}
%
%  \end{macro}
%
%  \begin{macro}{\bbl@ifknown@ttrib}
%    An internal macro to check whether a given language/attribute is
%    known. The macro takes 4 arguments, the language/attribute, the
%    attribute list, the \TeX-code to be executed when the attribute
%    is known and the \TeX-code to be executed otherwise.
%
%    \begin{macrocode}
\def\bbl@ifknown@ttrib#1#2{%
%    \end{macrocode}
%
%    We first assume the attribute is unknown.
%
%    \begin{macrocode}
  \let\bbl@tempa\@secondoftwo
%    \end{macrocode}
%
%    Then we loop over the list of known attributes, trying to find a
%    match.
%
%    \begin{macrocode}
  \bbl@loopx\bbl@tempb{#2}{%
    \expandafter\in@\expandafter{\expandafter,\bbl@tempb,}{,#1,}%
    \ifin@
%    \end{macrocode}
%
%    When a match is found the definition of |\bbl@tempa| is changed.
%
%    \begin{macrocode}
      \let\bbl@tempa\@firstoftwo
    \else
    \fi}%
%    \end{macrocode}
%
%    Finally we execute |\bbl@tempa|.
%
%    \begin{macrocode}
  \bbl@tempa
}
%    \end{macrocode}
%
%  \end{macro}
%
%  \begin{macro}{\bbl@clear@ttribs}
%    This macro removes all the attribute code from \LaTeX's memory at
%    |\begin{document}| time (if any is present).
%
%    \begin{macrocode}
\def\bbl@clear@ttribs{%
  \ifx\bbl@attributes\@undefined\else
    \bbl@loopx\bbl@tempa{\bbl@attributes}{%
      \expandafter\bbl@clear@ttrib\bbl@tempa.
      }%
    \let\bbl@attributes\@undefined
  \fi}
\def\bbl@clear@ttrib#1-#2.{%
  \expandafter\let\csname#1@attr@#2\endcsname\@undefined}
\AtBeginDocument{\bbl@clear@ttribs}
%    \end{macrocode}
%
%  \end{macro}
%
%  \subsection{Support for saving macro definitions}
%
%  To save the meaning of control sequences using |\babel@save|, we
%  use temporary control sequences.  To save hash table entries for
%  these control sequences, we don't use the name of the control
%  sequence to be saved to construct the temporary name.  Instead we
%  simply use the value of a counter, which is reset to zero each time
%  we begin to save new values.  This works well because we release
%  the saved meanings before we begin to save a new set of control
%  sequence meanings (see |\selectlanguage| and |\originalTeX|). Note
%  undefined macros are not undefined any more when saved -- they are
%  |\relax|'ed.
%
%  \begin{macro}{\babel@savecnt}
%  \begin{macro}{\babel@beginsave}
%    The initialization of a new save cycle: reset the counter to
%    zero.
%
%    \begin{macrocode}
\def\babel@beginsave{\babel@savecnt\z@}
%    \end{macrocode}
%
%    Before it's forgotten, allocate the counter and initialize all.
%
%    \begin{macrocode}
\newcount\babel@savecnt
\babel@beginsave
%    \end{macrocode}
%
%  \end{macro}
%  \end{macro}
%
%  \begin{macro}{\babel@save}
%    The macro |\babel@save|\meta{csname} saves the current meaning of
%    the control sequence \meta{csname} to
%    |\originalTeX|\footnote{\cs{originalTeX} has to be
%    expandable, i.\,e.\ you shouldn't let it to \cs{relax}.}.
%    To do this, we let the current meaning to a temporary control
%    sequence, the restore commands are appended to |\originalTeX| and
%    the counter is incremented.
%
%    \begin{macrocode}
\def\babel@save#1{%
  \expandafter\let\csname babel@\number\babel@savecnt\endcsname#1\relax
  \toks@\expandafter{\originalTeX\let#1=}%
  \bbl@exp{%
    \def\\\originalTeX{\the\toks@\<babel@\number\babel@savecnt>\relax}}%
  \advance\babel@savecnt\@ne}
%    \end{macrocode}
%
%  \end{macro}
%
%  \begin{macro}{\babel@savevariable}
%
%    The macro |\babel@savevariable|\meta{variable} saves the value of
%    the variable.  \meta{variable} can be anything allowed after the
%    |\the| primitive.
%
%    \begin{macrocode}
\def\babel@savevariable#1{%
  \toks@\expandafter{\originalTeX #1=}%
  \bbl@exp{\def\\\originalTeX{\the\toks@\the#1\relax}}}
%    \end{macrocode}
%
%  \end{macro}
%
%  \begin{macro}{\bbl@frenchspacing}
%  \begin{macro}{\bbl@nonfrenchspacing}
%    Some languages need to have |\frenchspacing| in effect. Others
%    don't want that. The command |\bbl@frenchspacing| switches it on
%    when it isn't already in effect and |\bbl@nonfrenchspacing|
%    switches it off if necessary.
%
%    \begin{macrocode}
\def\bbl@frenchspacing{%
  \ifnum\the\sfcode`\.=\@m
    \let\bbl@nonfrenchspacing\relax
  \else
    \frenchspacing
    \let\bbl@nonfrenchspacing\nonfrenchspacing
  \fi}
\let\bbl@nonfrenchspacing\nonfrenchspacing
%    \end{macrocode}
%
%  \end{macro}
%  \end{macro}
%
% \subsection{Short tags}
%
% \begin{macro}{\babeltags}
% This macro is straightforward. After zapping spaces, we
% loop over the list and define the macros |\text|\m{tag} and
% |\|\m{tag}. Definitions are first expanded so that they don't
% contain |\csname| but the actual macro.
%
% \changes{babel~3.9i}{2014/02/21}{Macro \cs{babeltags} added}
%
%   \begin{macrocode}
\def\babeltags#1{%
  \edef\bbl@tempa{\zap@space#1 \@empty}%
  \def\bbl@tempb##1=##2\@@{%
    \edef\bbl@tempc{%
      \noexpand\newcommand
      \expandafter\noexpand\csname ##1\endcsname{%
        \noexpand\protect
        \expandafter\noexpand\csname otherlanguage*\endcsname{##2}}
      \noexpand\newcommand
      \expandafter\noexpand\csname text##1\endcsname{%
        \noexpand\foreignlanguage{##2}}}
    \bbl@tempc}%
  \bbl@for\bbl@tempa\bbl@tempa{%
    \expandafter\bbl@tempb\bbl@tempa\@@}}
%    \end{macrocode}
%
% \end{macro}
%
%   \subsection{Hyphens}
%
%  \begin{macro}{\babelhyphenation}
%
%     This macro saves hyphenation exceptions. Two macros are used to
%     store them: |\bbl@hyphenation@| for the global ones and
%     |\bbl@hyphenation<lang>| for language ones. See |\bbl@patterns|
%     above for further details. We make sure there is a space between
%     words when multiple commands are used.
%
%     \changes{babel~3.9a}{2012/08/28}{Macro added}
%
%    \begin{macrocode}
\@onlypreamble\babelhyphenation
\AtEndOfPackage{%
  \newcommand\babelhyphenation[2][\@empty]{%
    \ifx\bbl@hyphenation@\relax
      \let\bbl@hyphenation@\@empty
    \fi
    \ifx\bbl@hyphlist\@empty\else
      \bbl@warning{%
        You must not intermingle \string\selectlanguage\space and\\%
        \string\babelhyphenation\space or some exceptions will not\\%
        be taken into account. Reported}%
    \fi
    \ifx\@empty#1%
      \protected@edef\bbl@hyphenation@{\bbl@hyphenation@\space#2}%
    \else
      \bbl@vforeach{#1}{%
        \def\bbl@tempa{##1}%
        \bbl@fixname\bbl@tempa
        \bbl@iflanguage\bbl@tempa{%
          \bbl@csarg\protected@edef{hyphenation@\bbl@tempa}{%
            \bbl@ifunset{bbl@hyphenation@\bbl@tempa}%
              \@empty
              {\csname bbl@hyphenation@\bbl@tempa\endcsname\space}%
            #2}}}%
    \fi}}
%    \end{macrocode}
%
%  \end{macro}
%
%  \begin{macro}{\bbl@allowhyphens}
%
% \changes{babel-3.9a}{2012/07/28}{Replaced many \cs{allowhyphens} by
%    \cs{bbl@allowhyphen}. They were either no-op or executed always.}
% \changes{babel-3.9i}{2014/01/29}{\cs{bbl@allowhyphens} must be
%    ignored at the beginning of a paragraph or table cell.}
% \changes{babel-3.9t}{2017/04/26}{Fixed misplaced \cs{nobreak} -
%    sx366454 - soft hyphens could vanish.}
%
%    This macro makes hyphenation possible. Basically its definition
%    is nothing more than |\nobreak| |\hskip| \texttt{0pt plus
%    0pt}\footnote{\TeX\ begins and ends a word for hyphenation at a
%    glue node. The penalty prevents a linebreak at this glue node.}.
%
%    \begin{macrocode}
\def\bbl@allowhyphens{\ifvmode\else\nobreak\hskip\z@skip\fi}
\def\bbl@t@one{T1}
\def\allowhyphens{\ifx\cf@encoding\bbl@t@one\else\bbl@allowhyphens\fi}
%    \end{macrocode}
%
%  \end{macro}
%
% \changes{babel-3.9a}{2012/08/27}{Added \cs{babelhyphen} and related
%    macros}
%
% \begin{macro}{\babelhyphen}
%
%    Macros to insert common hyphens. Note the space before |@| in
%    |\babelhyphen|. Instead of protecting it with
%    |\DeclareRobustCommand|, which could insert a |\relax|, we use
%    the same procedure as shorthands, with |\active@prefix|.
%
%    \begin{macrocode}
\newcommand\babelnullhyphen{\char\hyphenchar\font}
\def\babelhyphen{\active@prefix\babelhyphen\bbl@hyphen}
\def\bbl@hyphen{%
  \@ifstar{\bbl@hyphen@i @}{\bbl@hyphen@i\@empty}}
\def\bbl@hyphen@i#1#2{%
  \bbl@ifunset{bbl@hy@#1#2\@empty}%
    {\csname bbl@#1usehyphen\endcsname{\discretionary{#2}{}{#2}}}%
    {\csname bbl@hy@#1#2\@empty\endcsname}}
%    \end{macrocode}
%
%    The following two commands are used to wrap the ``hyphen'' and
%    set the behaviour of the rest of the word -- the version with a
%    single |@| is used when further hyphenation is allowed, while
%    that with |@@| if no more hyphen are allowed. In both cases, if
%    the hyphen is preceded by a positive space, breaking after the
%    hyphen is disallowed.
%
%    There should not be a discretionaty after a hyphen at the
%    beginning of a word, so it is prevented if preceded by a
%    skip. Unfortunately, this does handle cases like ``(-suffix)''.
%    |\nobreak| is always preceded by |\leavevmode|, in case the
%    shorthand starts a paragraph.
%
%    \begin{macrocode}
\def\bbl@usehyphen#1{%
  \leavevmode
  \ifdim\lastskip>\z@\mbox{#1}\else\nobreak#1\fi
  \nobreak\hskip\z@skip}
\def\bbl@@usehyphen#1{%
  \leavevmode\ifdim\lastskip>\z@\mbox{#1}\else#1\fi}
%    \end{macrocode}
%
%    The following macro inserts the hyphen char.
%
%    \begin{macrocode}
\def\bbl@hyphenchar{%
  \ifnum\hyphenchar\font=\m@ne
    \babelnullhyphen
  \else
    \char\hyphenchar\font
  \fi}
%    \end{macrocode}
%
%    Finally, we define the hyphen ``types''. Their names will not
%    change, so you may use them in |ldf|'s. After a space, the
%    |\mbox| in |\bbl@hy@nobreak| is redundant.
%
%    \begin{macrocode}
\def\bbl@hy@soft{\bbl@usehyphen{\discretionary{\bbl@hyphenchar}{}{}}}
\def\bbl@hy@@soft{\bbl@@usehyphen{\discretionary{\bbl@hyphenchar}{}{}}}
\def\bbl@hy@hard{\bbl@usehyphen\bbl@hyphenchar}
\def\bbl@hy@@hard{\bbl@@usehyphen\bbl@hyphenchar}
\def\bbl@hy@nobreak{\bbl@usehyphen{\mbox{\bbl@hyphenchar}}}
\def\bbl@hy@@nobreak{\mbox{\bbl@hyphenchar}}
\def\bbl@hy@repeat{%
  \bbl@usehyphen{%
    \discretionary{\bbl@hyphenchar}{\bbl@hyphenchar}{\bbl@hyphenchar}}}
\def\bbl@hy@@repeat{%
  \bbl@@usehyphen{%
    \discretionary{\bbl@hyphenchar}{\bbl@hyphenchar}{\bbl@hyphenchar}}}
\def\bbl@hy@empty{\hskip\z@skip}
\def\bbl@hy@@empty{\discretionary{}{}{}}
%    \end{macrocode}
%
%  \end{macro}
%
%  \begin{macro}{\bbl@disc}
%    For some languages the macro |\bbl@disc| is used to ease the
%    insertion of discretionaries for letters that behave `abnormally'
%    at a breakpoint.
%
%    \begin{macrocode}
\def\bbl@disc#1#2{\nobreak\discretionary{#2-}{}{#1}\bbl@allowhyphens}
%    \end{macrocode}
%
%  \end{macro}
%  \subsection{Multiencoding strings}
%
% \changes{babel~3.9a}{2012/09/05}{Added tentative code for string
%  declarations}
% \changes{babel~3.9a}{2012/12/24}{Added hooks}
%
% The aim following commands is to provide a commom interface for
% strings in several encodings. They also contains several hooks which
% can be ued by \luatex{} and \xetex. The code is organized here with
% pseudo-guards, so we start with the basic commands.
%
%  \paragraph{Tools}
%
% But first, a couple of tools. The first one makes global a local
% variable. This is not the best solution, but it works.
%
%    \begin{macrocode}
\def\bbl@toglobal#1{\global\let#1#1}
\def\bbl@recatcode#1{%
  \@tempcnta="7F
  \def\bbl@tempa{%
    \ifnum\@tempcnta>"FF\else
      \catcode\@tempcnta=#1\relax
      \advance\@tempcnta\@ne
      \expandafter\bbl@tempa
    \fi}%
  \bbl@tempa}
%    \end{macrocode}
%
% The second one. We need to patch |\@uclclist|, but it is done once
% and only if |\SetCase| is used or if strings are encoded.  The code
% is far from satisfactory for several reasons, including the fact
% |\@uclclist| is not a list any more. Therefore a package option is
% added to ignore it. Instead of gobbling the macro
% getting the next two elements (usually |\reserved@a|), we pass it as
% argument to |\bbl@uclc|. The parser is restarted inside
% |\|\m{lang}|@bbl@uclc| because we do not know how many expansions
% are necessary (depends on whether strings are encoded). The last
% part is tricky -- when uppercasing, we have:
%\begin{verbatim}
% \let\bbl@tolower\@empty\bbl@toupper\@empty
%\end{verbatim}
% and starts over (and similarly when lowercasing).
%
%    \changes{babel~3.9l}{2014/07/29}{Now tries to catch the parsing
%    macro. Removed some redundant code. Option |nocase|.}
%
%    \begin{macrocode}
\@ifpackagewith{babel}{nocase}%
  {\let\bbl@patchuclc\relax}%
  {\def\bbl@patchuclc{%
    \global\let\bbl@patchuclc\relax
    \g@addto@macro\@uclclist{\reserved@b{\reserved@b\bbl@uclc}}%
    \gdef\bbl@uclc##1{%
      \let\bbl@encoded\bbl@encoded@uclc
      \bbl@ifunset{\languagename @bbl@uclc}% and resumes it
        {##1}%
        {\let\bbl@tempa##1\relax % Used by LANG@bbl@uclc
         \csname\languagename @bbl@uclc\endcsname}%
      {\bbl@tolower\@empty}{\bbl@toupper\@empty}}%
    \gdef\bbl@tolower{\csname\languagename @bbl@lc\endcsname}%
    \gdef\bbl@toupper{\csname\languagename @bbl@uc\endcsname}}}
%    \end{macrocode}
%
%    \begin{macrocode}
%<<*More package options>>
\DeclareOption{nocase}{}
%<</More package options>>
%    \end{macrocode}
%
% The following package options control the behaviour of |\SetString|.
%
%    \begin{macrocode}
%<<*More package options>>
\let\bbl@opt@strings\@nnil % accept strings=value
\DeclareOption{strings}{\def\bbl@opt@strings{\BabelStringsDefault}}
\DeclareOption{strings=encoded}{\let\bbl@opt@strings\relax}
\def\BabelStringsDefault{generic}
%<</More package options>>
%    \end{macrocode}
%
%    \paragraph{Main command} This is the main command.  With the
%    first use it is redefined to omit the basic setup in subsequent
%    blocks. We make sure strings contain actual letters in the range
%    128-255, not active characters.
%
%    \changes{babel~3.9g}{2013/07/29}{Added starred variant. A bit of
%    clean up. Removed \cs{UseString}, which didn't work.}
%    \changes{babel~3.9g}{2013/08/01}{Now several languages can be
%    processed with \cs{BabelLanguages}, if set in the ldf.}
%
%    \begin{macrocode}
\@onlypreamble\StartBabelCommands
\def\StartBabelCommands{%
  \begingroup   
  \bbl@recatcode{11}%
  <@Macros local to BabelCommands@>
  \def\bbl@provstring##1##2{%
    \providecommand##1{##2}%
    \bbl@toglobal##1}%
  \global\let\bbl@scafter\@empty
  \let\StartBabelCommands\bbl@startcmds
  \ifx\BabelLanguages\relax
     \let\BabelLanguages\CurrentOption
  \fi
  \begingroup
  \let\bbl@screset\@nnil % local flag - disable 1st stopcommands
  \StartBabelCommands}
\def\bbl@startcmds{%
  \ifx\bbl@screset\@nnil\else
    \bbl@usehooks{stopcommands}{}%
  \fi
  \endgroup
  \begingroup
  \@ifstar
    {\ifx\bbl@opt@strings\@nnil
       \let\bbl@opt@strings\BabelStringsDefault
     \fi
     \bbl@startcmds@i}%
    \bbl@startcmds@i}
\def\bbl@startcmds@i#1#2{%
  \edef\bbl@L{\zap@space#1 \@empty}%
  \edef\bbl@G{\zap@space#2 \@empty}%
  \bbl@startcmds@ii}
%    \end{macrocode}
%
%    Parse the encoding info to get the label, input, and font parts.
%
%    Select the behaviour of |\SetString|. Thre are two main cases,
%    depending of if there is an optional argument: without it and
%    |strings=encoded|, strings are defined
%    always; otherwise, they are set only if they are still undefined
%    (ie, fallback values). With labelled blocks and
%    |strings=encoded|, define the strings, but with another value,
%    define strings only if the current label or font encoding is the
%    value of |strings|; otherwise (ie, no |strings| or a block whose
%    label is not in |strings=|) do nothing.
%
%    We presume the current block is not loaded, and therefore set
%    (above) a couple of default values to gobble the arguments. Then,
%    these macros are redefined if necessary according to several
%    parameters.
%
% \changes{babel~3.9g}{2013/08/04}{Use \cs{ProvideTextCommand}, which
%   does with encoded strings what the manual says.}
% \changes{babel~3.9h}{2013/11/08}{Tidied up code related to
%  \cs{bbl@scswitch}}
% 
%    \begin{macrocode}
\newcommand\bbl@startcmds@ii[1][\@empty]{%
  \let\SetString\@gobbletwo
  \let\bbl@stringdef\@gobbletwo
  \let\AfterBabelCommands\@gobble
  \ifx\@empty#1%
    \def\bbl@sc@label{generic}%
    \def\bbl@encstring##1##2{%
      \ProvideTextCommandDefault##1{##2}%
      \bbl@toglobal##1%
      \expandafter\bbl@toglobal\csname\string?\string##1\endcsname}%
    \let\bbl@sctest\in@true
  \else
    \let\bbl@sc@charset\space % <- zapped below
    \let\bbl@sc@fontenc\space % <-   "      "
    \def\bbl@tempa##1=##2\@nil{%
      \bbl@csarg\edef{sc@\zap@space##1 \@empty}{##2 }}%
    \bbl@vforeach{label=#1}{\bbl@tempa##1\@nil}%
    \def\bbl@tempa##1 ##2{% space -> comma
      ##1%
      \ifx\@empty##2\else\ifx,##1,\else,\fi\bbl@afterfi\bbl@tempa##2\fi}%
    \edef\bbl@sc@fontenc{\expandafter\bbl@tempa\bbl@sc@fontenc\@empty}%
    \edef\bbl@sc@label{\expandafter\zap@space\bbl@sc@label\@empty}%
    \edef\bbl@sc@charset{\expandafter\zap@space\bbl@sc@charset\@empty}%
    \def\bbl@encstring##1##2{%
      \bbl@foreach\bbl@sc@fontenc{%
        \bbl@ifunset{T@####1}%
          {}%
          {\ProvideTextCommand##1{####1}{##2}%
           \bbl@toglobal##1%
           \expandafter
           \bbl@toglobal\csname####1\string##1\endcsname}}}%
    \def\bbl@sctest{%
      \bbl@xin@{,\bbl@opt@strings,}{,\bbl@sc@label,\bbl@sc@fontenc,}}%
  \fi
  \ifx\bbl@opt@strings\@nnil         % ie, no strings key -> defaults
  \else\ifx\bbl@opt@strings\relax    % ie, strings=encoded
    \let\AfterBabelCommands\bbl@aftercmds
    \let\SetString\bbl@setstring
    \let\bbl@stringdef\bbl@encstring
  \else       % ie, strings=value
  \bbl@sctest
  \ifin@
    \let\AfterBabelCommands\bbl@aftercmds
    \let\SetString\bbl@setstring
    \let\bbl@stringdef\bbl@provstring
  \fi\fi\fi
  \bbl@scswitch
  \ifx\bbl@G\@empty
    \def\SetString##1##2{%
      \bbl@error{Missing group for string \string##1}%
        {You must assign strings to some category, typically\\%
         captions or extras, but you set none}}%
  \fi
  \ifx\@empty#1%
    \bbl@usehooks{defaultcommands}{}%
  \else
    \@expandtwoargs
    \bbl@usehooks{encodedcommands}{{\bbl@sc@charset}{\bbl@sc@fontenc}}%
  \fi}
%    \end{macrocode}
%
%    There are two versions of |\bbl@scswitch|. The first version is
%    used when |ldf|s are read, and it makes sure
%    |\|\m{group}\m{language} is reset, but only once (|\bbl@screset|
%    is used to keep track of this). The second version is used in the
%    preamble and packages loaded after \babel{} and does nothing. The
%    macro |\bbl@forlang| loops |\bbl@L| but its body is executed only
%    if the value is in |\BabelLanguages| (inside \babel) or
%    |\date|\m{language} is defined (after \babel{} has been loaded).
%    There are also two version of |\bbl@forlang|. The first one skips
%    the current iteration if the language is not in |\BabelLanguages|
%    (used in |ldf|s), and the second one skips undefined languages
%    (after \babel{} has been loaded) . 
%
%   \begin{macrocode}
\def\bbl@forlang#1#2{%
  \bbl@for#1\bbl@L{%
    \bbl@xin@{,#1,}{,\BabelLanguages,}%
    \ifin@#2\relax\fi}}
\def\bbl@scswitch{%
  \bbl@forlang\bbl@tempa{%
    \ifx\bbl@G\@empty\else
      \ifx\SetString\@gobbletwo\else
        \edef\bbl@GL{\bbl@G\bbl@tempa}%
        \bbl@xin@{,\bbl@GL,}{,\bbl@screset,}%
        \ifin@\else
          \global\expandafter\let\csname\bbl@GL\endcsname\@undefined
          \xdef\bbl@screset{\bbl@screset,\bbl@GL}%
        \fi
      \fi
    \fi}}
\AtEndOfPackage{%
  \def\bbl@forlang#1#2{\bbl@for#1\bbl@L{\bbl@ifunset{date#1}{}{#2}}}%
  \let\bbl@scswitch\relax}
\@onlypreamble\EndBabelCommands
\def\EndBabelCommands{%
  \bbl@usehooks{stopcommands}{}%
  \endgroup
  \endgroup
  \bbl@scafter}
%    \end{macrocode}
%    
%    Now we define commands to be used inside |\StartBabelCommands|.
%   
%    \paragraph{Strings} The following macro is the actual definition
%    of |\SetString| when it is ``active''
%
%    First save the ``switcher''. Create it if undefined. Strings are
%    defined only if undefined (ie, like |\providescommmand|). With
%    the event |stringprocess| you can preprocess the string by
%    manipulating the value of |\BabelString|. If there are several
%    hooks assigned to this event, preprocessing is done in the same
%    order as defined.  Finally, the string is set.
%
% \changes{babel~3.9g}{2013/07/29}{Added \cs{bbl@forlang} to ignore in
%   the preamble unknown languages, as described in the doc.}
% 
%    \begin{macrocode}
\def\bbl@setstring#1#2{%
  \bbl@forlang\bbl@tempa{%
    \edef\bbl@LC{\bbl@tempa\bbl@stripslash#1}%
    \bbl@ifunset{\bbl@LC}% eg, \germanchaptername
      {\global\expandafter  % TODO - con \bbl@exp ?
       \bbl@add\csname\bbl@G\bbl@tempa\expandafter\endcsname\expandafter
         {\expandafter\bbl@scset\expandafter#1\csname\bbl@LC\endcsname}}%
      {}%
    \def\BabelString{#2}%
    \bbl@usehooks{stringprocess}{}%
    \expandafter\bbl@stringdef
      \csname\bbl@LC\expandafter\endcsname\expandafter{\BabelString}}}
%    \end{macrocode}
%
%     Now, some addtional stuff to be used when encoded strings are
%     used. Captions then include |\bbl@encoded| for string to be
%     expanded in case transformations. It is |\relax| by default, but
%     in |\MakeUppercase| and |\MakeLowercase| its value is a modified
%     expandable |\@changed@cmd|.
% 
% \changes{babel~3.9i}{2014/03/13}{Added code to expand captions in
%     case transformations.}
%
%    \begin{macrocode}
\ifx\bbl@opt@strings\relax
  \def\bbl@scset#1#2{\def#1{\bbl@encoded#2}}
  \bbl@patchuclc
  \let\bbl@encoded\relax    
  \def\bbl@encoded@uclc#1{%
    \@inmathwarn#1%
    \expandafter\ifx\csname\cf@encoding\string#1\endcsname\relax
      \expandafter\ifx\csname ?\string#1\endcsname\relax
        \TextSymbolUnavailable#1%
      \else
        \csname ?\string#1\endcsname
      \fi
    \else
      \csname\cf@encoding\string#1\endcsname
    \fi}
\else
  \def\bbl@scset#1#2{\def#1{#2}}
\fi
%    \end{macrocode}
%
%    Define |\SetStringLoop|, which is actually set inside
%    |\StartBabelCommands|. The current definition is
%    somewhat complicated because we need a count, but |\count@| is
%    not under our control (remember |\SetString| may call hooks).
%    Instead of defining a dedicated count, we just ``pre-expand''
%    its value.
%
% \changes{babel~3.9h}{2013/10/16}{Tidied up and bug fixed - first
%    element expanded prematurely.}
% 
%    \begin{macrocode}
%<<*Macros local to BabelCommands>>
\def\SetStringLoop##1##2{%
    \def\bbl@templ####1{\expandafter\noexpand\csname##1\endcsname}%
    \count@\z@
    \bbl@loop\bbl@tempa{##2}{% empty items and spaces are ok
      \advance\count@\@ne
      \toks@\expandafter{\bbl@tempa}%
      \bbl@exp{%
        \\\SetString\bbl@templ{\romannumeral\count@}{\the\toks@}%
        \count@=\the\count@\relax}}}%
%<</Macros local to BabelCommands>>
%    \end{macrocode}
%
%    \paragraph{Delaying code} Now the definition of
%    |\AfterBabelCommands| when it is activated.
%
%    \begin{macrocode}
\def\bbl@aftercmds#1{%
  \toks@\expandafter{\bbl@scafter#1}%
  \xdef\bbl@scafter{\the\toks@}}
%    \end{macrocode}
%
% \paragraph{Case mapping}
%
% The command |\SetCase| provides a way to change the behaviour of
% |\MakeUppercase| and |\MakeLowercase|. |\bbl@tempa| is set by the
% patched |\@uclclist| to the parsing command.
%
% \changes{babel~3.9h}{2013/11/08}{Use \cs{bbl@encstrings} - they
% should be defined always, even if no `strings'}
%
%    \begin{macrocode}
%<<*Macros local to BabelCommands>>
  \newcommand\SetCase[3][]{%
    \bbl@patchuclc
    \bbl@forlang\bbl@tempa{%
      \expandafter\bbl@encstring
        \csname\bbl@tempa @bbl@uclc\endcsname{\bbl@tempa##1}%
      \expandafter\bbl@encstring
        \csname\bbl@tempa @bbl@uc\endcsname{##2}% 
      \expandafter\bbl@encstring
        \csname\bbl@tempa @bbl@lc\endcsname{##3}}}%
%<</Macros local to BabelCommands>>
%    \end{macrocode}
%
% Macros to deal with case mapping for hyphenation. To decide if the
% document is monolingual or multilingual, we make a rough guess --
% just see if there is a comma in the languages list, built in the
% first pass of the package options. 
%
%    \begin{macrocode}
%<<*Macros local to BabelCommands>>
  \newcommand\SetHyphenMap[1]{%
    \bbl@forlang\bbl@tempa{%
      \expandafter\bbl@stringdef
        \csname\bbl@tempa @bbl@hyphenmap\endcsname{##1}}}
%<</Macros local to BabelCommands>>
%    \end{macrocode}
%
%   There are 3 helper macros which do most of the work for you.
%
%   \begin{macrocode}
\newcommand\BabelLower[2]{% one to one.
  \ifnum\lccode#1=#2\else
    \babel@savevariable{\lccode#1}%
    \lccode#1=#2\relax
  \fi}
\newcommand\BabelLowerMM[4]{% many-to-many
  \@tempcnta=#1\relax
  \@tempcntb=#4\relax
  \def\bbl@tempa{%
    \ifnum\@tempcnta>#2\else
      \@expandtwoargs\BabelLower{\the\@tempcnta}{\the\@tempcntb}%
      \advance\@tempcnta#3\relax
      \advance\@tempcntb#3\relax
      \expandafter\bbl@tempa
    \fi}%
  \bbl@tempa}
\newcommand\BabelLowerMO[4]{% many-to-one
  \@tempcnta=#1\relax
  \def\bbl@tempa{%
    \ifnum\@tempcnta>#2\else
      \@expandtwoargs\BabelLower{\the\@tempcnta}{#4}%
      \advance\@tempcnta#3
      \expandafter\bbl@tempa
    \fi}%
  \bbl@tempa}
%    \end{macrocode}
%
%    The following package options control the behaviour of
%    hyphenation mapping.
%
% \changes{babel~3.9t}{2017/04/28}{Renamed \cs{bbl@hymapopt} to
%    \cs{bbl@opt@hyphenmap} for consistency}
%
%    \begin{macrocode}
%<<*More package options>>    
\DeclareOption{hyphenmap=off}{\chardef\bbl@opt@hyphenmap\z@}
\DeclareOption{hyphenmap=first}{\chardef\bbl@opt@hyphenmap\@ne}
\DeclareOption{hyphenmap=select}{\chardef\bbl@opt@hyphenmap\tw@}      
\DeclareOption{hyphenmap=other}{\chardef\bbl@opt@hyphenmap\thr@@}
\DeclareOption{hyphenmap=other*}{\chardef\bbl@opt@hyphenmap4\relax}
%<</More package options>>    
%    \end{macrocode}
%
%    Initial setup to provide a default behaviour if |hypenmap|
%    is not set.
%
%    \begin{macrocode}
\AtEndOfPackage{%
  \ifx\bbl@opt@hyphenmap\@undefined
    \bbl@xin@{,}{\bbl@language@opts}%
    \chardef\bbl@opt@hyphenmap\ifin@4\else\@ne\fi
  \fi}
%    \end{macrocode}
%
% \subsection{Macros common to a number of languages}
%
%  \begin{macro}{\set@low@box}
%    The following macro is used to lower quotes to the same level as
%    the comma.  It prepares its argument in box register~0.
%
%    \begin{macrocode}
\def\set@low@box#1{\setbox\tw@\hbox{,}\setbox\z@\hbox{#1}%
    \dimen\z@\ht\z@ \advance\dimen\z@ -\ht\tw@%
    \setbox\z@\hbox{\lower\dimen\z@ \box\z@}\ht\z@\ht\tw@ \dp\z@\dp\tw@}
%    \end{macrocode}
%
%  \end{macro}
%
%  \begin{macro}{\save@sf@q}
%    The macro |\save@sf@q| is used to save and reset the current
%    space factor.
%
%    \begin{macrocode}
\def\save@sf@q#1{\leavevmode
  \begingroup 
    \edef\@SF{\spacefactor\the\spacefactor}#1\@SF
  \endgroup}
%    \end{macrocode}
%
%  \end{macro}
%
%  \subsection{Making glyphs available}
%
%    This section makes a number of glyphs available that either do not
%    exist in the \texttt{OT1} encoding and have to be `faked', or
%    that are not accessible through \file{T1enc.def}.
%
%  \subsubsection{Quotation marks}
%
%  \begin{macro}{\quotedblbase}
%    In the \texttt{T1} encoding the opening double quote at the
%    baseline is available as a separate character, accessible via
%    |\quotedblbase|. In the \texttt{OT1} encoding it is not
%    available, therefore we make it available by lowering the normal
%    open quote character to the baseline.
%
%    \begin{macrocode}
\ProvideTextCommand{\quotedblbase}{OT1}{%
  \save@sf@q{\set@low@box{\textquotedblright\/}%
    \box\z@\kern-.04em\bbl@allowhyphens}}
%    \end{macrocode}
%
%    Make sure that when an encoding other than \texttt{OT1} or
%    \texttt{T1} is used this glyph can still be typeset.
%
%    \begin{macrocode}
\ProvideTextCommandDefault{\quotedblbase}{%
  \UseTextSymbol{OT1}{\quotedblbase}}
%    \end{macrocode}
%
%  \end{macro}
%
%  \begin{macro}{\quotesinglbase}
%    We also need the single quote character at the baseline.
%
%    \begin{macrocode}
\ProvideTextCommand{\quotesinglbase}{OT1}{%
  \save@sf@q{\set@low@box{\textquoteright\/}%
    \box\z@\kern-.04em\bbl@allowhyphens}}
%    \end{macrocode}
%
%    Make sure that when an encoding other than \texttt{OT1} or
%    \texttt{T1} is used this glyph can still be typeset.
%
%    \begin{macrocode}
\ProvideTextCommandDefault{\quotesinglbase}{%
  \UseTextSymbol{OT1}{\quotesinglbase}}
%    \end{macrocode}
%
%  \end{macro}
%
%  \begin{macro}{\guillemotleft}
%  \begin{macro}{\guillemotright}
%    The guillemet characters are not available in \texttt{OT1}
%    encoding. They are faked.
%
%    \begin{macrocode}
\ProvideTextCommand{\guillemotleft}{OT1}{%
  \ifmmode
    \ll
  \else
    \save@sf@q{\nobreak
      \raise.2ex\hbox{$\scriptscriptstyle\ll$}\bbl@allowhyphens}%
  \fi}
\ProvideTextCommand{\guillemotright}{OT1}{%
  \ifmmode
    \gg
  \else
    \save@sf@q{\nobreak
      \raise.2ex\hbox{$\scriptscriptstyle\gg$}\bbl@allowhyphens}%
  \fi}
%    \end{macrocode}
%
%    Make sure that when an encoding other than \texttt{OT1} or
%    \texttt{T1} is used these glyphs can still be typeset.
%
%    \begin{macrocode}
\ProvideTextCommandDefault{\guillemotleft}{%
  \UseTextSymbol{OT1}{\guillemotleft}}
\ProvideTextCommandDefault{\guillemotright}{%
  \UseTextSymbol{OT1}{\guillemotright}}
%    \end{macrocode}
%
%  \end{macro}
%  \end{macro}
%
%  \begin{macro}{\guilsinglleft}
%  \begin{macro}{\guilsinglright}
%    The single guillemets are not available in \texttt{OT1}
%    encoding. They are faked.
%
%    \begin{macrocode}
\ProvideTextCommand{\guilsinglleft}{OT1}{%
  \ifmmode
    <%
  \else
    \save@sf@q{\nobreak
      \raise.2ex\hbox{$\scriptscriptstyle<$}\bbl@allowhyphens}%
  \fi}
\ProvideTextCommand{\guilsinglright}{OT1}{%
  \ifmmode
    >%
  \else
    \save@sf@q{\nobreak
      \raise.2ex\hbox{$\scriptscriptstyle>$}\bbl@allowhyphens}%
  \fi}
%    \end{macrocode}
%
%    Make sure that when an encoding other than \texttt{OT1} or
%    \texttt{T1} is used these glyphs can still be typeset.
%
%    \begin{macrocode}
\ProvideTextCommandDefault{\guilsinglleft}{%
  \UseTextSymbol{OT1}{\guilsinglleft}}
\ProvideTextCommandDefault{\guilsinglright}{%
  \UseTextSymbol{OT1}{\guilsinglright}}
%    \end{macrocode}
%
%  \end{macro}
%  \end{macro}
%
%  \subsubsection{Letters}
%
%  \begin{macro}{\ij}
%  \begin{macro}{\IJ}
%    The dutch language uses the letter `ij'. It is available in
%    \texttt{T1} encoded fonts, but not in the \texttt{OT1} encoded
%    fonts. Therefore we fake it for the \texttt{OT1} encoding.
%
% \changes{babel~3.9a}{2012/07/28}{Removed the first \cs{allowhyphens}.
%    Moved the second one just after the kern.}
%
%    \begin{macrocode}
\DeclareTextCommand{\ij}{OT1}{%
  i\kern-0.02em\bbl@allowhyphens j}
\DeclareTextCommand{\IJ}{OT1}{%
  I\kern-0.02em\bbl@allowhyphens J}
\DeclareTextCommand{\ij}{T1}{\char188}
\DeclareTextCommand{\IJ}{T1}{\char156}
%    \end{macrocode}
%
%    Make sure that when an encoding other than \texttt{OT1} or
%    \texttt{T1} is used these glyphs can still be typeset.
%
%    \begin{macrocode}
\ProvideTextCommandDefault{\ij}{%
  \UseTextSymbol{OT1}{\ij}}
\ProvideTextCommandDefault{\IJ}{%
  \UseTextSymbol{OT1}{\IJ}}
%    \end{macrocode}
%
%  \end{macro}
%  \end{macro}
%
%  \begin{macro}{\dj}
%  \begin{macro}{\DJ}
%    The croatian language needs the letters |\dj| and |\DJ|; they are
%    available in the \texttt{T1} encoding, but not in the
%    \texttt{OT1} encoding by default.
%
%    Some code to construct these glyphs for the \texttt{OT1} encoding
%    was made available to me by Stipcevic Mario,
%    (\texttt{stipcevic@olimp.irb.hr}).
%
%    \begin{macrocode}
\def\crrtic@{\hrule height0.1ex width0.3em}
\def\crttic@{\hrule height0.1ex width0.33em}
\def\ddj@{%
  \setbox0\hbox{d}\dimen@=\ht0
  \advance\dimen@1ex
  \dimen@.45\dimen@
  \dimen@ii\expandafter\rem@pt\the\fontdimen\@ne\font\dimen@
  \advance\dimen@ii.5ex
  \leavevmode\rlap{\raise\dimen@\hbox{\kern\dimen@ii\vbox{\crrtic@}}}}
\def\DDJ@{%
  \setbox0\hbox{D}\dimen@=.55\ht0
  \dimen@ii\expandafter\rem@pt\the\fontdimen\@ne\font\dimen@
  \advance\dimen@ii.15ex %            correction for the dash position
  \advance\dimen@ii-.15\fontdimen7\font %     correction for cmtt font
  \dimen\thr@@\expandafter\rem@pt\the\fontdimen7\font\dimen@
  \leavevmode\rlap{\raise\dimen@\hbox{\kern\dimen@ii\vbox{\crttic@}}}}
%
\DeclareTextCommand{\dj}{OT1}{\ddj@ d}
\DeclareTextCommand{\DJ}{OT1}{\DDJ@ D}
%    \end{macrocode}
%
%    Make sure that when an encoding other than \texttt{OT1} or
%    \texttt{T1} is used these glyphs can still be typeset.
%
%    \begin{macrocode}
\ProvideTextCommandDefault{\dj}{%
  \UseTextSymbol{OT1}{\dj}}
\ProvideTextCommandDefault{\DJ}{%
  \UseTextSymbol{OT1}{\DJ}}
%    \end{macrocode}
%
%  \end{macro}
%  \end{macro}
%
%  \begin{macro}{\SS}
%    For the \texttt{T1} encoding |\SS| is defined and selects a
%    specific glyph from the font, but for other encodings it is not
%    available. Therefore we make it available here.
%
%    \begin{macrocode}
\DeclareTextCommand{\SS}{OT1}{SS}
\ProvideTextCommandDefault{\SS}{\UseTextSymbol{OT1}{\SS}}
%    \end{macrocode}
%
%  \end{macro}
%
% \subsubsection{Shorthands for quotation marks}
%
%    Shorthands are provided for a number of different quotation
%    marks, which make them usable both outside and inside mathmode.
%    They are defined with |\ProvideTextCommandDefault|, but this is
%    very likely not required because their definitions are based on
%    encoding dependent macros.
%
%  \begin{macro}{\glq}
%  \begin{macro}{\grq}
%
% \changes{babel~3.8b}{2004/05/02}{Made \cs{glq} fontencoding
%    dependent as well}
%
%    The `german' single quotes.
%
%    \begin{macrocode}
\ProvideTextCommandDefault{\glq}{%
  \textormath{\quotesinglbase}{\mbox{\quotesinglbase}}}
%    \end{macrocode}
%
%    The definition of |\grq| depends on the fontencoding. With
%    \texttt{T1} encoding no extra kerning is needed.
%
%    \begin{macrocode}
\ProvideTextCommand{\grq}{T1}{%
  \textormath{\textquoteleft}{\mbox{\textquoteleft}}}
\ProvideTextCommand{\grq}{TU}{%
  \textormath{\textquoteleft}{\mbox{\textquoteleft}}}
\ProvideTextCommand{\grq}{OT1}{%
  \save@sf@q{\kern-.0125em
    \textormath{\textquoteleft}{\mbox{\textquoteleft}}%
    \kern.07em\relax}}
\ProvideTextCommandDefault{\grq}{\UseTextSymbol{OT1}\grq}
%    \end{macrocode}
%
%  \end{macro}
%  \end{macro}
%
%  \begin{macro}{\glqq}
%  \begin{macro}{\grqq}
%
% \changes{babel~3.8b}{2004/05/02}{Made \cs{grqq} fontencoding
%    dependent as well}
%
%    The `german' double quotes.
%
%    \begin{macrocode}
\ProvideTextCommandDefault{\glqq}{%
  \textormath{\quotedblbase}{\mbox{\quotedblbase}}}
%    \end{macrocode}
%
%    The definition of |\grqq| depends on the fontencoding. With
%    \texttt{T1} encoding no extra kerning is needed.
%
%    \begin{macrocode}
\ProvideTextCommand{\grqq}{T1}{%
  \textormath{\textquotedblleft}{\mbox{\textquotedblleft}}}
\ProvideTextCommand{\grqq}{TU}{%
  \textormath{\textquotedblleft}{\mbox{\textquotedblleft}}}
\ProvideTextCommand{\grqq}{OT1}{%
  \save@sf@q{\kern-.07em
    \textormath{\textquotedblleft}{\mbox{\textquotedblleft}}%
    \kern.07em\relax}}
\ProvideTextCommandDefault{\grqq}{\UseTextSymbol{OT1}\grqq}
%    \end{macrocode}
%
%  \end{macro}
%  \end{macro}
%
%  \begin{macro}{\flq}
%  \begin{macro}{\frq}
%
% \changes{babel~3.8b}{2004/05/02}{Made \cs{flq} and \cs{frq}
%    fontencoding dependent}
%
%    The `french' single guillemets.
%
%    \begin{macrocode}
\ProvideTextCommandDefault{\flq}{%
  \textormath{\guilsinglleft}{\mbox{\guilsinglleft}}}
\ProvideTextCommandDefault{\frq}{%
  \textormath{\guilsinglright}{\mbox{\guilsinglright}}}
%    \end{macrocode}
%
%  \end{macro}
%  \end{macro}
%
%  \begin{macro}{\flqq}
%  \begin{macro}{\frqq}
%
% \changes{babel~3.8b}{2004/05/02}{Made \cs{flqq} and \cs{frqq}
%    fontencoding dependent}
%
%    The `french' double guillemets.
%
%    \begin{macrocode}
\ProvideTextCommandDefault{\flqq}{%
  \textormath{\guillemotleft}{\mbox{\guillemotleft}}}
\ProvideTextCommandDefault{\frqq}{%
  \textormath{\guillemotright}{\mbox{\guillemotright}}}
%    \end{macrocode}
%
%  \end{macro}
%  \end{macro}
%
%  \subsubsection{Umlauts and tremas}
%
%    The command |\"| needs to have a different effect for different
%    languages. For German for instance, the `umlaut' should be
%    positioned lower than the default position for placing it over
%    the letters a, o, u, A, O and U. When placed over an e, i, E or I
%    it can retain its normal position. For Dutch the same glyph is
%    always placed in the lower position.
%
%  \begin{macro}{\umlauthigh}
%
% \changes{v3.8a}{2004/02/19}{Use \cs{leavevmode}\cs{bgroup} to
%    prevent problems when this command occurs in vertical mode.}
%
%  \begin{macro}{\umlautlow}
%    To be able to provide both positions of |\"| we provide two
%    commands to switch the positioning, the default will be
%    |\umlauthigh| (the normal positioning).
%
%    \begin{macrocode}
\def\umlauthigh{%
  \def\bbl@umlauta##1{\leavevmode\bgroup%
      \expandafter\accent\csname\f@encoding dqpos\endcsname
      ##1\bbl@allowhyphens\egroup}%
  \let\bbl@umlaute\bbl@umlauta}
\def\umlautlow{%
  \def\bbl@umlauta{\protect\lower@umlaut}}
\def\umlautelow{%
  \def\bbl@umlaute{\protect\lower@umlaut}}
\umlauthigh
%    \end{macrocode}
%
%  \end{macro}
%  \end{macro}
%
%  \begin{macro}{\lower@umlaut}
%    The command |\lower@umlaut| is used to position the |\"| closer
%    to the letter.
%
%    We want the umlaut character lowered, nearer to the letter. To do
%    this we need an extra \meta{dimen} register.
%
%    \begin{macrocode}
\expandafter\ifx\csname U@D\endcsname\relax
  \csname newdimen\endcsname\U@D
\fi
%    \end{macrocode}
%
%    The following code fools \TeX's \texttt{make\_accent} procedure
%    about the current x-height of the font to force another placement
%    of the umlaut character.
%    First we have to save the current x-height of the font, because
%    we'll change this font dimension and this is always done
%    globally.
%
% \changes{v3.8a}{2004/02/19}{Use \cs{leavevmode}\cs{bgroup} to
%    prevent problems when this command occurs in vertical mode.}
%
%    Then we compute the new x-height in such a way that the umlaut
%    character is lowered to the base character.  The value of
%    \texttt{.45ex} depends on the \MF\ parameters with which the
%    fonts were built.  (Just try out, which value will look best.)
%    If the new x-height is too low, it is not changed.
%    Finally we call the |\accent| primitive, reset the old x-height
%    and insert the base character in the argument.
%
%    \begin{macrocode}
\def\lower@umlaut#1{%
  \leavevmode\bgroup
    \U@D 1ex%
    {\setbox\z@\hbox{%
      \expandafter\char\csname\f@encoding dqpos\endcsname}%
      \dimen@ -.45ex\advance\dimen@\ht\z@
      \ifdim 1ex<\dimen@ \fontdimen5\font\dimen@ \fi}%
    \expandafter\accent\csname\f@encoding dqpos\endcsname
    \fontdimen5\font\U@D #1%
  \egroup}
%    \end{macrocode}
%
%  \end{macro}
%
%    For all vowels we declare |\"| to be a composite command which
%    uses |\bbl@umlauta| or |\bbl@umlaute| to position the umlaut
%    character. We need to be sure that these definitions override the
%    ones that are provided when the package \pkg{fontenc} with
%    option \Lopt{OT1} is used. Therefore these declarations are
%    postponed until the beginning of the document. Note these
%    definitions only apply to some languages, but 
%    babel sets them for \textit{all} languages -- you may want to
%    redefine |\bbl@umlauta| and/or |\bbl@umlaute| for a language
%    in the corresponding |ldf| (using the babel switching mechanism,
%    of course).
%
%    \begin{macrocode}
\AtBeginDocument{%
  \DeclareTextCompositeCommand{\"}{OT1}{a}{\bbl@umlauta{a}}%
  \DeclareTextCompositeCommand{\"}{OT1}{e}{\bbl@umlaute{e}}%
  \DeclareTextCompositeCommand{\"}{OT1}{i}{\bbl@umlaute{\i}}%
  \DeclareTextCompositeCommand{\"}{OT1}{\i}{\bbl@umlaute{\i}}%
  \DeclareTextCompositeCommand{\"}{OT1}{o}{\bbl@umlauta{o}}%
  \DeclareTextCompositeCommand{\"}{OT1}{u}{\bbl@umlauta{u}}%
  \DeclareTextCompositeCommand{\"}{OT1}{A}{\bbl@umlauta{A}}%
  \DeclareTextCompositeCommand{\"}{OT1}{E}{\bbl@umlaute{E}}%
  \DeclareTextCompositeCommand{\"}{OT1}{I}{\bbl@umlaute{I}}%
  \DeclareTextCompositeCommand{\"}{OT1}{O}{\bbl@umlauta{O}}%
  \DeclareTextCompositeCommand{\"}{OT1}{U}{\bbl@umlauta{U}}%
}
%    \end{macrocode}
%
%    Finally, the default is to use English as the main language.
%
% \changes{babel~3.9a}{2012-05-17}{Languages are best assigned with
%    \cs{chardef}, not \cs{let}}
%
%    \begin{macrocode}
\ifx\l@english\@undefined
  \chardef\l@english\z@
\fi
\main@language{english}
%    \end{macrocode}
%
%    Now we load definition files for engines.
%
%    \begin{macrocode}
\ifcase\bbl@engine\or
  \input luababel.def
\or
  \input xebabel.def
\fi
%    \end{macrocode}
%
% \section{The kernel of Babel (\texttt{babel.def}, only \LaTeX)}
%
% \subsection{The redefinition of the style commands}
%
%    The rest of the code in this file can only be processed by
%    \LaTeX, so we check the current format. If it is plain \TeX,
%    processing should stop here. But, because of the need to limit
%    the scope of the definition of |\format|, a macro that is used
%    locally in the following |\if|~statement, this comparison is done
%    inside a group. To prevent \TeX\ from complaining about an
%    unclosed group, the processing of the command |\endinput| is
%    deferred until after the group is closed. This is accomplished by
%    the command |\aftergroup|.
%
%    \begin{macrocode}
{\def\format{lplain}
\ifx\fmtname\format
\else
  \def\format{LaTeX2e}
  \ifx\fmtname\format
  \else
    \aftergroup\endinput
  \fi
\fi}
%    \end{macrocode}
%
% \subsection{Creating languages}
%
% |\babelprovide| is a general purpose tool for creating
% languages. Currently it just creates the language infrastructure,
% but in the future it will be able to read data from |ini| files, as
% well as to create variants. Unlike the \textsf{nil} pseudo-language,
% captions are defined, but with a warning to invite the user to
% provide the real string.
%
% \changes{3.10}{2017/05/19}{Added \cs{babelprovide}}
% \changes{3.13}{2017/08/30}{Added \cs{import}, which also reads
% dates. Some refactoring in the ini reader.}
% \changes{3.15}{2017/10/30}{New keys script, language}
%
%
%    \begin{macrocode}
\newcommand\babelprovide[2][]{%  
  \let\bbl@savelangname\languagename
  \def\languagename{#2}%
  \let\bbl@KVP@captions\@nil
  \let\bbl@KVP@import\@nil
  \let\bbl@KVP@main\@nil
  \let\bbl@KVP@script\@nil
  \let\bbl@KVP@language\@nil
  \let\bbl@KVP@dir\@nil
  \let\bbl@KVP@hyphenrules\@nil
  \bbl@forkv{#1}{\bbl@csarg\def{KVP@##1}{##2}}%  TODO - error handling
  \ifx\bbl@KVP@captions\@nil
    \let\bbl@KVP@captions\bbl@KVP@import
  \fi
  \bbl@ifunset{date#2}%
    {\bbl@provide@new{#2}}%
    {\bbl@ifblank{#1}%
      {\bbl@error
        {If you want to modify `#2' you must tell how in\\%
         the optional argument. Currently there are three\\%
         options: captions=lang-tag, hyphenrules=lang-list\\%
         import=lang-tag}%
        {Use this macro as documented}}%
      {\bbl@provide@renew{#2}}}%
  \bbl@exp{\\\babelensure[exclude=\\\today]{#2}}%
  \ifx\bbl@KVP@script\@nil\else
    \bbl@csarg\edef{sname@#2}{\bbl@KVP@script}%
  \fi
  \ifx\bbl@KVP@language\@nil\else
    \bbl@csarg\edef{lname@#2}{\bbl@KVP@language}%
  \fi
  \let\languagename\bbl@savelangname}
%    \end{macrocode}
%
% Depending on whether or not the language exists, we define two macros.
%
%    \begin{macrocode}
\def\bbl@provide@new#1{%
  \@namedef{date#1}{}% marks lang exists - required by \StartBabelCommands
  \@namedef{extras#1}{}%
  \@namedef{noextras#1}{}%
  \StartBabelCommands*{#1}{captions}%
    \ifx\bbl@KVP@captions\@nil %      and also if import, implicit
      \def\bbl@tempb##1{%             elt for \bbl@captionslist
        \ifx##1\@empty\else
          \bbl@exp{%
            \\\SetString\\##1{%
              \\\bbl@nocaption{\bbl@stripslash##1}{\<#1\bbl@stripslash##1>}}}%
          \expandafter\bbl@tempb
        \fi}%  
      \expandafter\bbl@tempb\bbl@captionslist\@empty
    \else
      \bbl@read@ini{\bbl@KVP@captions}%  Here all letters cat = 11
      \bbl@after@ini
      \bbl@savestrings
    \fi
  \StartBabelCommands*{#1}{date}%
    \ifx\bbl@KVP@import\@nil
      \bbl@exp{%
        \\\SetString\\\today{\\\bbl@nocaption{today}{\<#1today>}}}%
    \else
      \bbl@savetoday
      \bbl@savedate
    \fi
  \EndBabelCommands
  \bbl@exp{%
    \def\<#1hyphenmins>{%
      {\bbl@ifunset{bbl@lfthm@#1}{2}{\@nameuse{bbl@lfthm@#1}}}%
      {\bbl@ifunset{bbl@rgthm@#1}{3}{\@nameuse{bbl@rgthm@#1}}}}}%
  \bbl@provide@hyphens{#1}%
  \ifx\bbl@KVP@main\@nil\else
     \expandafter\main@language\expandafter{#1}%
  \fi}
\def\bbl@provide@renew#1{%
  \ifx\bbl@KVP@captions\@nil\else
    \StartBabelCommands*{#1}{captions}%
      \bbl@read@ini{\bbl@KVP@captions}%   Here all letters cat = 11
      \bbl@after@ini
      \bbl@savestrings
    \EndBabelCommands
 \fi
 \ifx\bbl@KVP@import\@nil\else
   \StartBabelCommands*{#1}{date}%
     \bbl@savetoday
     \bbl@savedate
   \EndBabelCommands
  \fi
  \bbl@provide@hyphens{#1}}
%    \end{macrocode}
%
% The |hyphenrules| option is handled with an auxiliary macro.
%
% \changes{3.14}{2017/10/03}{Take into account ini settings for
% hyphenrules if `import'.}
%
%    \begin{macrocode}
\def\bbl@provide@hyphens#1{%
  \let\bbl@tempa\relax
  \ifx\bbl@KVP@hyphenrules\@nil\else
    \bbl@replace\bbl@KVP@hyphenrules{ }{,}%
    \bbl@foreach\bbl@KVP@hyphenrules{%
      \ifx\bbl@tempa\relax    %  if not yet found
        \bbl@ifsamestring{##1}{+}%
          {{\bbl@exp{\\\addlanguage\<l@##1>}}}%
          {}%
        \bbl@ifunset{l@##1}%
           {}%
           {\bbl@exp{\let\bbl@tempa\<l@##1>}}%
      \fi}%
  \fi
  \ifx\bbl@tempa\relax %        if no opt or no language in opt found
    \ifx\bbl@KVP@import\@nil\else % if importing
      \bbl@exp{%                and hyphenrules is not empty
        \\\bbl@ifblank{\@nameuse{bbl@hyphr@#1}}%
          {}%
          {\let\\\bbl@tempa\<l@\@nameuse{bbl@hyphr@\languagename}>}}%
    \fi
  \fi
  \bbl@ifunset{bbl@tempa}%       ie, relax or undefined
    {\bbl@ifunset{l@#1}%         no hyphenrules found - fallback
       {\bbl@exp{\\\adddialect\<l@#1>\language}}%
       {}}%                      so, l@<lang> is ok - nothing to do
    {\bbl@exp{\\\adddialect\<l@#1>\bbl@tempa}}}%  found in opt list or ini
%    \end{macrocode}
%
% The reader of |ini| files. There are 3 possible cases: a section name
% (in the form |[...]|), a comment (starting with |;|) and a
% key/value pair. \textit{TODO - Work in progress.}
%
%    \begin{macrocode}
\def\bbl@read@ini#1{%
  \openin1=babel-#1.ini
  \ifeof1
    \bbl@error
      {There is no ini file for the requested language\\%
       (#1). Perhaps you misspelled it or your installation\\%
       is not complete.}%
      {Fix the name or reinstall babel.}%
  \else
    \let\bbl@section\@empty
    \let\bbl@savestrings\@empty
    \let\bbl@savetoday\@empty
    \let\bbl@savedate\@empty
    \let\bbl@inireader\bbl@iniskip
    \bbl@info{Importing data from babel-#1.ini for \languagename}%
    \loop
      \endlinechar\m@ne
      \read1 to \bbl@line
      \endlinechar`\^^M
    \if T\ifeof1F\fi T\relax % Trick, because inside \loop
      \ifx\bbl@line\@empty\else
        \expandafter\bbl@iniline\bbl@line\bbl@iniline
      \fi
    \repeat
  \fi}
\def\bbl@iniline#1\bbl@iniline{%
  \@ifnextchar[\bbl@inisec{\@ifnextchar;\bbl@iniskip\bbl@inireader}#1\@@}% ]
%    \end{macrocode}
%
% The special cases for comment lines and sections are handled by the
% two following commands. In sections, we provide the posibility to
% take extra actions at the end or at the start (TODO - but note the last
% section is not ended). By default, key=val pairs are ignored. 
%      
%    \begin{macrocode}
\def\bbl@iniskip#1\@@{}%      if starts with ;
\def\bbl@inisec[#1]#2\@@{%    if starts with opening bracket
  \@nameuse{bbl@secpost@\bbl@section}%  ends previous section
  \def\bbl@section{#1}%
  \@nameuse{bbl@secpre@\bbl@section}%   starts current section
  \bbl@ifunset{bbl@secline@#1}%
    {\let\bbl@inireader\bbl@iniskip}%
    {\bbl@exp{\let\\\bbl@inireader\<bbl@secline@#1>}}}
%    \end{macrocode}
%      
% Reads a key=val line and stores the trimmed val in
% |\bbl@@kv@<section>.<key>|.
%
%    \begin{macrocode}
\def\bbl@inikv#1=#2\@@{%     key=value
  \bbl@trim@def\bbl@tempa{#1}%
  \bbl@trim\toks@{#2}%
  \bbl@csarg\edef{@kv@\bbl@section.\bbl@tempa}{\the\toks@}}
%    \end{macrocode}
%  
% The previous assignments are local, so we need to export them. If
% the value is empty, we can provide a default value.
%
%    \begin{macrocode}
\def\bbl@exportkey#1#2#3{%
  \bbl@ifunset{bbl@@kv@#2}%
    {\bbl@csarg\gdef{#1@\languagename}{#3}}%
    {\expandafter\ifx\csname bbl@@kv@#2\endcsname\@empty
       \bbl@csarg\gdef{#1@\languagename}{#3}%
     \else
       \bbl@exp{\global\let\<bbl@#1@\languagename>\<bbl@@kv@#2>}%
     \fi}}
%    \end{macrocode}
%
% Key-value pairs are treated differently depending on the section in
% the |ini| file.  The following macros are the readers for
% |identification| and |typography|.
%
%    \begin{macrocode}
\let\bbl@secline@identification\bbl@inikv
\def\bbl@secpost@identification{%
  \bbl@exportkey{lname}{identification.name.english}{}%
  \bbl@exportkey{lbcp}{identification.tag.bcp47}{}%
  \bbl@exportkey{lotf}{identification.tag.opentype}{dflt}%
  \bbl@exportkey{sname}{identification.script.name}{}%
  \bbl@exportkey{sbcp}{identification.script.tag.bcp47}{}%
  \bbl@exportkey{sotf}{identification.script.tag.opentype}{DFLT}}
\let\bbl@secline@typography\bbl@inikv
\def\bbl@after@ini{%
  \bbl@exportkey{lfthm}{typography.lefthyphenmin}{2}%
  \bbl@exportkey{rgthm}{typography.righthyphenmin}{3}%
  \bbl@exportkey{hyphr}{typography.hyphenrules}{}%
  \def\bbl@tempa{0.9}%
  \bbl@csarg\ifx{@kv@identification.version}\bbl@tempa
    \bbl@warning{%
      The `\languagename' date format may not be suitable\\%
      for proper typesetting, and therefore it very likely will\\%
      change in a future release. Reported}%
  \fi
  \bbl@toglobal\bbl@savetoday
  \bbl@toglobal\bbl@savedate}
%    \end{macrocode}
%      
% Now |captions| and |captions.licr|, depending on the engine. And
% also for dates. They rely on a few auxilary macros.
%
%    \begin{macrocode}
\ifcase\bbl@engine
  \bbl@csarg\def{secline@captions.licr}#1=#2\@@{%
    \bbl@ini@captions@aux{#1}{#2}}
  \bbl@csarg\def{secline@date.gregorian}#1=#2\@@{%       for defaults
    \bbl@ini@dategreg#1...\relax{#2}}
  \bbl@csarg\def{secline@date.gregorian.licr}#1=#2\@@{%  override
    \bbl@ini@dategreg#1...\relax{#2}}
\else
  \def\bbl@secline@captions#1=#2\@@{%
    \bbl@ini@captions@aux{#1}{#2}}
  \bbl@csarg\def{secline@date.gregorian}#1=#2\@@{%
    \bbl@ini@dategreg#1...\relax{#2}}
\fi
%    \end{macrocode}
%      
% The auxiliary macro for captions define |\<caption>name|.
%
%    \begin{macrocode}
\def\bbl@ini@captions@aux#1#2{%
  \bbl@trim@def\bbl@tempa{#1}%
  \bbl@ifblank{#2}%
    {\bbl@exp{%
       \toks@{\\\bbl@nocaption{\bbl@tempa}\<\languagename\bbl@tempa name>}}}%
    {\bbl@trim\toks@{#2}}%
  \bbl@exp{%
    \\\bbl@add\\\bbl@savestrings{%
      \\\SetString\<\bbl@tempa name>{\the\toks@}}}}
%    \end{macrocode}
%
% But dates are more complex. The full date format is stores in
% |date.gregorian|, so we must read it in non-Unicode engines, too.
%      
%    \begin{macrocode}
\bbl@csarg\def{secpre@date.gregorian.licr}{%
  \ifcase\bbl@engine\let\bbl@savedate\@empty\fi}
\def\bbl@ini@dategreg#1.#2.#3.#4\relax#5{% TODO - ignore with 'captions'
  \bbl@trim@def\bbl@tempa{#1.#2}%
  \bbl@ifsamestring{\bbl@tempa}{months.wide}%
    {\bbl@trim@def\bbl@tempa{#3}%
     \bbl@trim\toks@{#5}%
     \bbl@exp{%
      \\\bbl@add\\\bbl@savedate{%
        \\\SetString\<month\romannumeral\bbl@tempa name>{\the\toks@}}}}%
    {\bbl@ifsamestring{\bbl@tempa}{date.long}% 
      {\bbl@trim@def\bbl@toreplace{#5}%
       \bbl@TG@@date
       \global\bbl@csarg\let{date@\languagename}\bbl@toreplace
       \bbl@exp{%
         \gdef\<\languagename date>####1####2####3{%
           \<bbl@ensure@\languagename>{%
             \<bbl@date@\languagename>{####1}{####2}{####3}}}%
         \\\bbl@add\\\bbl@savetoday{%
           \\\SetString\\\today{%
             \<\languagename date>{\year}{\month}{\day}}}}}}%
      {}}
%    \end{macrocode}
%
% Dates will require some macros for the basic formatting. They may be
% redefined by language, so ``semi-public'' names (camel case) are
% used. Oddly enough, the CLDR places particles like “de”
% inconsistenly in either in the date or in the month name.
%
%    \begin{macrocode}
\newcommand\BabelDateSpace{\nobreakspace{}}
\newcommand\BabelDateDot{.\@}
\newcommand\BabelDated[1]{{\number#1}}
\newcommand\BabelDatedd[1]{{\ifnum#1<10 0\fi\number#1}}
\newcommand\BabelDateM[1]{{\number#1}}
\newcommand\BabelDateMM[1]{{\ifnum#1<10 0\fi\number#1}}
\newcommand\BabelDateMMMM[1]{{%
  \csname month\romannumeral\month name\endcsname}}%
\newcommand\BabelDatey[1]{{\number#1}}%
\newcommand\BabelDateyy[1]{{%
  \ifnum#1<10 0\number#1 %
  \else\ifnum#1<100 \number#1 %
  \else\ifnum#1<1000 \expandafter\@gobble\number#1 %
  \else\ifnum#1<10000 \expandafter\@gobbletwo\number#1 %
  \else
    \bbl@error
      {Currently two-digit years are restricted to the\\
       range 0-9999.}%
      {There is little you can do. Sorry.}%
  \fi\fi\fi\fi}}
\newcommand\BabelDateyyyy[1]{{\number#1}}
\def\bbl@replace@finish@iii#1{%
  \bbl@exp{\def\\#1####1####2####3{\the\toks@}}}
\def\bbl@TG@@date{%
  \bbl@replace\bbl@toreplace{[ ]}{\BabelDateSpace{}}%
  \bbl@replace\bbl@toreplace{[.]}{\BabelDateDot{}}%
  \bbl@replace\bbl@toreplace{[d]}{\BabelDated{####3}}%
  \bbl@replace\bbl@toreplace{[dd]}{\BabelDatedd{####3}}%
  \bbl@replace\bbl@toreplace{[M]}{\BabelDateM{####2}}%
  \bbl@replace\bbl@toreplace{[MM]}{\BabelDateMM{####2}}%
  \bbl@replace\bbl@toreplace{[MMMM]}{\BabelDateMMMM{####2}}%
  \bbl@replace\bbl@toreplace{[y]}{\BabelDatey{####1}}%
  \bbl@replace\bbl@toreplace{[yy]}{\BabelDateyy{####1}}%
  \bbl@replace\bbl@toreplace{[yyyy]}{\BabelDateyyyy{####1}}%
% Note after \bbl@replace \toks@ contains the resulting string.
% TODO - Using this implicit behavior doesn't seem a good idea.
  \bbl@replace@finish@iii\bbl@toreplace}
%    \end{macrocode}
%
% Language and Script values to be used when defining a font or
% setting the direction are set with the following macros.
%
%    \begin{macrocode}
\def\bbl@provide@lsys#1{%
  \bbl@ifunset{bbl@lname@#1}%
    {\bbl@ini@ids{#1}}%
    {}%
  \bbl@csarg\let{lsys@#1}\@empty
  \bbl@ifunset{bbl@sname@#1}{\bbl@csarg\gdef{sname@#1}{Default}}{}%
  \bbl@ifunset{bbl@sotf#1}{\bbl@csarg\gdef{sotf@#1}{DFLT}}{}%
  \bbl@csarg\bbl@add@list{lsys@#1}{Script=\bbl@cs{sname@#1}}%
  \bbl@ifunset{bbl@lname@#1}{}%
    {\bbl@csarg\bbl@add@list{lsys@#1}{Language=\bbl@cs{lname@#1}}}%
  \bbl@csarg\bbl@toglobal{lsys@#1}}%
 %  \bbl@exp{% TODO - should be global
 %    \<keys_if_exist:nnF>{fontspec-opentype/Script}{\bbl@cs{sname@#1}}%
 %      {\\\newfontscript{\bbl@cs{sname@#1}}{\bbl@cs{sotf@#1}}}% 
 %    \<keys_if_exist:nnF>{fontspec-opentype/Language}{\bbl@cs{lname@#1}}%
 %      {\\\newfontlanguage{\bbl@cs{lname@#1}}{\bbl@cs{lotf@#1}}}}}
%    \end{macrocode}
% 
% The following |ini| reader ignores everything but the
% |identification| section. It is called when a font is defined (ie,
% when the language is first selected) to know which script/language
% must be enabled. This means we must make sure a few characters are
% not active. The |ini| is not read directly, but with a proxy |tex|
% file named as the language.
%
%    \begin{macrocode}
\def\bbl@ini@ids#1{%
  \def\BabelBeforeIni##1##2{%
    \begingroup
      \bbl@add\bbl@secpost@identification{%
        \def\bbl@iniline########1\bbl@iniline{}}%
      \catcode`\[=12 \catcode`\]=12 \catcode`\==12 
      \bbl@read@ini{##1}%
    \endgroup}
  \InputIfFileExists{babel-#1.tex}{}{}}
%    \end{macrocode}
%
% \subsection{Cross referencing macros}
%
%    The \LaTeX\ book states:
%  \begin{quote}
%    The \emph{key} argument is any sequence of letters, digits, and
%    punctuation symbols; upper- and lowercase letters are regarded as
%    different.
%  \end{quote}
%    When the above quote should still be true when a document is
%    typeset in a language that has active characters, special care
%    has to be taken of the category codes of these characters when
%    they appear in an argument of the cross referencing macros.
%
%    When a cross referencing command processes its argument, all
%    tokens in this argument should be character tokens with category
%    `letter' or `other'.
%
%    The only way to accomplish this in most cases is to use the trick
%    described in the \TeX book~\cite{DEK} (Appendix~D, page~382).
%    The primitive |\meaning| applied to a token expands to the
%    current meaning of this token.  For example, `|\meaning\A|' with
%    |\A| defined as `|\def\A#1{\B}|' expands to the characters
%    `|macro:#1->\B|' with all category codes set to `other' or
%    `space'.
%  \begin{macro}{\newlabel}
%    The macro |\label| writes a line with a |\newlabel| command
%    into the |.aux| file to define labels.
%
%    \begin{macrocode}
%\bbl@redefine\newlabel#1#2{%
%  \@safe@activestrue\org@newlabel{#1}{#2}\@safe@activesfalse}
%    \end{macrocode}
%
%  \end{macro}
%
%  \begin{macro}{\@newl@bel}
%    We need to change the definition of the \LaTeX-internal macro
%    |\@newl@bel|. This is needed because we need to make sure that
%    shorthand characters expand to their non-active version.
%
%    The following package options control which macros are to be
%    redefined.
%
%    \begin{macrocode}
%<<*More package options>>
\DeclareOption{safe=none}{\let\bbl@opt@safe\@empty}
\DeclareOption{safe=bib}{\def\bbl@opt@safe{B}}
\DeclareOption{safe=ref}{\def\bbl@opt@safe{R}}
%<</More package options>>
%    \end{macrocode}
%
%    First we open a new group to keep the changed setting of
%    |\protect| local and then we set the |@safe@actives| switch to
%    true to make sure that any shorthand that appears in any of the
%    arguments immediately expands to its non-active self.
%
%    \begin{macrocode}
\ifx\bbl@opt@safe\@empty\else
  \def\@newl@bel#1#2#3{%
   {\@safe@activestrue
    \bbl@ifunset{#1@#2}%
       \relax
       {\gdef\@multiplelabels{%
          \@latex@warning@no@line{There were multiply-defined labels}}%
        \@latex@warning@no@line{Label `#2' multiply defined}}%
    \global\@namedef{#1@#2}{#3}}}
%    \end{macrocode}
%
%  \end{macro}
%
%  \begin{macro}{\@testdef}
%    An internal \LaTeX\ macro used to test if the labels that have
%    been written on the |.aux| file have changed.  It is called by
%    the |\enddocument| macro. This macro needs to be completely
%    rewritten, using |\meaning|. The reason for this is that in some
%    cases the expansion of |\#1@#2| contains the same characters as
%    the |#3|; but the character codes differ. Therefore \LaTeX\ keeps
%    reporting that the labels may have changed.
%
%    \begin{macrocode}
  \CheckCommand*\@testdef[3]{%
    \def\reserved@a{#3}%
    \expandafter\ifx\csname#1@#2\endcsname\reserved@a
    \else
      \@tempswatrue
    \fi}
%    \end{macrocode}
%
%    Now that we made sure that |\@testdef| still has the same
%    definition we can rewrite it. First we make the shorthands
%    `safe'.
%
%    \begin{macrocode}
  \def\@testdef#1#2#3{%
    \@safe@activestrue
%    \end{macrocode}
%
%    Then we use |\bbl@tempa| as an `alias' for the macro that
%    contains the label which is being checked.
%
%    \begin{macrocode}
    \expandafter\let\expandafter\bbl@tempa\csname #1@#2\endcsname
%    \end{macrocode}
%
%    Then we define |\bbl@tempb| just as |\@newl@bel| does it.
%
%    \begin{macrocode}
    \def\bbl@tempb{#3}%
    \@safe@activesfalse
%    \end{macrocode}
%
%    When the label is defined we replace the definition of
%    |\bbl@tempa| by its meaning.
%
%    \begin{macrocode}
    \ifx\bbl@tempa\relax
    \else
      \edef\bbl@tempa{\expandafter\strip@prefix\meaning\bbl@tempa}%
    \fi
%    \end{macrocode}
%
%    We do the same for |\bbl@tempb|.
%
%    \begin{macrocode}
    \edef\bbl@tempb{\expandafter\strip@prefix\meaning\bbl@tempb}%
%    \end{macrocode}
%
%    If the label didn't change, |\bbl@tempa| and |\bbl@tempb| should
%    be identical macros.
%
%    \begin{macrocode}
    \ifx\bbl@tempa\bbl@tempb
    \else
      \@tempswatrue
    \fi}
\fi
%    \end{macrocode}
%
%  \end{macro}
%
%  \begin{macro}{\ref}
%  \begin{macro}{\pageref}
%    The same holds for the macro |\ref| that references a label
%    and |\pageref| to reference a page. So we redefine |\ref| and
%    |\pageref|. While we change these macros, we make them robust as
%    well (if they weren't already) to prevent problems if they should
%    become expanded at the wrong moment.
%
%    \begin{macrocode}
\bbl@xin@{R}\bbl@opt@safe
\ifin@
  \bbl@redefinerobust\ref#1{%
    \@safe@activestrue\org@ref{#1}\@safe@activesfalse}
  \bbl@redefinerobust\pageref#1{%
    \@safe@activestrue\org@pageref{#1}\@safe@activesfalse}
\else
  \let\org@ref\ref
  \let\org@pageref\pageref
\fi
%    \end{macrocode}
%
%  \end{macro}
%  \end{macro}
%
%  \begin{macro}{\@citex}
%    The macro used to cite from a bibliography, |\cite|, uses an
%    internal macro, |\@citex|.
%    It is this internal macro that picks up the argument(s),
%    so we redefine this internal macro and leave |\cite| alone. The
%    first argument is used for typesetting, so the shorthands need
%    only be deactivated in the second argument.
%
%    \begin{macrocode}
\bbl@xin@{B}\bbl@opt@safe
\ifin@
  \bbl@redefine\@citex[#1]#2{%
    \@safe@activestrue\edef\@tempa{#2}\@safe@activesfalse
    \org@@citex[#1]{\@tempa}}
%    \end{macrocode}
%
%    Unfortunately, the packages \pkg{natbib} and \pkg{cite} need a
%    different definition of |\@citex|...
%    To begin with, \pkg{natbib} has a definition for |\@citex| with
%    \emph{three} arguments... We only know that a package is loaded
%    when |\begin{document}| is executed, so we need to postpone the
%    different redefinition.
%
%    \begin{macrocode}
  \AtBeginDocument{%
    \@ifpackageloaded{natbib}{%
%    \end{macrocode}
%
%    Notice that we use |\def| here instead of |\bbl@redefine| because
%    |\org@@citex| is already defined and we don't want to overwrite
%    that definition (it would result in parameter stack overflow
%    because of a circular definition).
%
%   (Recent versions of natbib change dynamically |\@citex|, so PR4087
%     doesn't seem fixable in a simple way. Just load natbib before.)
%
%    \begin{macrocode}
    \def\@citex[#1][#2]#3{%
      \@safe@activestrue\edef\@tempa{#3}\@safe@activesfalse
      \org@@citex[#1][#2]{\@tempa}}%
    }{}}
%    \end{macrocode}
%
%    The package \pkg{cite} has a definition of |\@citex| where the
%    shorthands need to be turned off in both arguments.
%
%    \begin{macrocode}
  \AtBeginDocument{%
    \@ifpackageloaded{cite}{%
      \def\@citex[#1]#2{%
        \@safe@activestrue\org@@citex[#1]{#2}\@safe@activesfalse}%
      }{}}
%    \end{macrocode}
%
%  \end{macro}
%
%  \begin{macro}{\nocite}
%    The macro |\nocite| which is used to instruct BiB\TeX\ to
%    extract uncited references from the database.
%
%    \begin{macrocode}
  \bbl@redefine\nocite#1{%
    \@safe@activestrue\org@nocite{#1}\@safe@activesfalse}
%    \end{macrocode}
%
%  \end{macro}
%
%  \begin{macro}{\bibcite}
%    The macro that is used in the |.aux| file to define citation
%    labels. When packages such as \pkg{natbib} or \pkg{cite} are not
%    loaded its second argument is used to typeset the citation
%    label. In that case, this second argument can contain active
%    characters but is used in an environment where
%    |\@safe@activestrue| is in effect. This switch needs to be reset
%    inside the |\hbox| which contains the citation label. In order to
%    determine during \file{.aux} file processing which definition of
%    |\bibcite| is needed we define |\bibcite| in such a way that it
%    redefines itself with the proper definition.
%
%    \begin{macrocode}
  \bbl@redefine\bibcite{%
%    \end{macrocode}
%
%    We call |\bbl@cite@choice| to select the proper definition for
%    |\bibcite|. This new definition is then activated.
%
%    \begin{macrocode}
    \bbl@cite@choice
    \bibcite}
%    \end{macrocode}
%
%  \end{macro}
%
%  \begin{macro}{\bbl@bibcite}
%    The macro |\bbl@bibcite| holds the definition of |\bibcite|
%    needed when neither \pkg{natbib} nor \pkg{cite} is loaded.
%
%    \begin{macrocode}
  \def\bbl@bibcite#1#2{%
    \org@bibcite{#1}{\@safe@activesfalse#2}}
%    \end{macrocode}
%
%  \end{macro}
%
%  \begin{macro}{\bbl@cite@choice}
%    The macro |\bbl@cite@choice| determines which definition of
%    |\bibcite| is needed.
%
%    \begin{macrocode}
  \def\bbl@cite@choice{%
%    \end{macrocode}
%
%    First we give |\bibcite| its default definition.
%
%    \begin{macrocode}
    \global\let\bibcite\bbl@bibcite
%    \end{macrocode}
%
%    Then, when \pkg{natbib} is loaded we restore the original
%    definition of |\bibcite|.
%
%    \begin{macrocode}
    \@ifpackageloaded{natbib}{\global\let\bibcite\org@bibcite}{}%
%    \end{macrocode}
%
%    For \pkg{cite} we do the same.
%
%    \begin{macrocode}
    \@ifpackageloaded{cite}{\global\let\bibcite\org@bibcite}{}%
%    \end{macrocode}
%
%    Make sure this only happens once.
%
%    \begin{macrocode}
    \global\let\bbl@cite@choice\relax}
%    \end{macrocode}
%
%    When a document is run for the first time, no \file{.aux} file is
%    available, and |\bibcite| will not yet be properly defined. In
%    this case, this has to happen before the document starts.
%
%    \begin{macrocode}
  \AtBeginDocument{\bbl@cite@choice}
%    \end{macrocode}
%
%  \end{macro}
%
%  \begin{macro}{\@bibitem}
%    One of the two internal \LaTeX\ macros called by |\bibitem|
%    that write the citation label on the |.aux| file.
%
%    \begin{macrocode}
  \bbl@redefine\@bibitem#1{%
    \@safe@activestrue\org@@bibitem{#1}\@safe@activesfalse}
\else
  \let\org@nocite\nocite
  \let\org@@citex\@citex
  \let\org@bibcite\bibcite
  \let\org@@bibitem\@bibitem
\fi
%    \end{macrocode}
%
%  \end{macro}
%
%  \subsection{Marks}
%
%  \begin{macro}{\markright}
%    Because the output routine is asynchronous, we must
%    pass the current language attribute to the head lines, together
%    with the text that is put into them. To achieve this we need to
%    adapt the definition of |\markright| and |\markboth| somewhat.
%
% \changes{babel~3.8c}{2004/05/26}{No need to add \emph{anything} to
%    an empty mark; prevented this by checking the contents of the
%    argument}
% \changes{babel~3.8f}{2005/05/15}{Make the definition independent of
%    the original definition; expand \cs{languagename} before passing
%    it into the token registers} 
% \changes{babel~3.9t}{2017/04/23}{Refactored \cs{markright} and
%    \cs{markboth}}  
%
%    We check whether the argument is empty; if it is, we just make
%    sure the scratch token register is empty.  Next, we store the
%    argument to |\markright| in the scratch token register. This way
%    these commands will not be expanded later, and we make sure that
%    the text is typeset using the correct language settings. While
%    doing so, we make sure that active characters that may end up in
%    the mark are not disabled by the output routine kicking in while
%    \cs{@safe@activestrue} is in effect.
%
%    \begin{macrocode}
\bbl@redefine\markright#1{%
  \bbl@ifblank{#1}%
    {\org@markright{}}%
    {\toks@{#1}%
     \bbl@exp{%
       \\\org@markright{\\\protect\\\foreignlanguage{\languagename}%
         {\\\protect\\\bbl@restore@actives\the\toks@}}}}}
%    \end{macrocode}
%
%  \end{macro}
%
%  \begin{macro}{\markboth}
%  \begin{macro}{\@mkboth}
%    The definition of |\markboth| is equivalent to that of
%    |\markright|, except that we need two token registers. The
%    documentclasses \cls{report} and \cls{book} define and set the
%    headings for the page. While doing so they also store a copy of
%    |\markboth| in |\@mkboth|. Therefore we need to check whether
%    |\@mkboth| has already been set. If so we neeed to do that again
%    with the new definition of |\markboth|.
%
% \changes{babel~3.8c}{2004/05/26}{No need to add \emph{anything} to
%    an empty mark, prevented this by checking the contents of the
%    arguments} 
% \changes{babel~3.8f}{2005/05/15}{Make the definition independent of
%    the original definition; expand \cs{languagename} before passing
%    it into the token registers} 
% \changes{babel~3.8j}{2008/03/21}{Added setting of \cs{@mkboth} (PR
%    3826)} 
%
%    \begin{macrocode}
\ifx\@mkboth\markboth
  \def\bbl@tempc{\let\@mkboth\markboth}
\else
  \def\bbl@tempc{}
\fi
%    \end{macrocode}
%
%    Now we can start the new definition of |\markboth|
%
%    \begin{macrocode}
\bbl@redefine\markboth#1#2{%
  \protected@edef\bbl@tempb##1{%
    \protect\foreignlanguage{\languagename}{\protect\bbl@restore@actives##1}}%
  \bbl@ifblank{#1}%
    {\toks@{}}%
    {\toks@\expandafter{\bbl@tempb{#1}}}%
  \bbl@ifblank{#2}%
    {\@temptokena{}}%
    {\@temptokena\expandafter{\bbl@tempb{#2}}}%
  \bbl@exp{\\\org@markboth{\the\toks@}{\the\@temptokena}}}
%    \end{macrocode}
%
%    and copy it to |\@mkboth| if necessary.
%
%    \begin{macrocode}
\bbl@tempc
%    \end{macrocode}
%
%  \end{macro}
%  \end{macro}
%
%  \subsection{Preventing clashes with other packages}
%
%  \subsubsection{\pkg{ifthen}}
%
%  \begin{macro}{\ifthenelse}
%
% \changes{babel~3.9a}{2012/09/07}{Redefine only if `ref' is `safe'}
% \changes{babel~3.9a}{2013/01/03}{Moved to babel.def}
%
%    Sometimes a document writer wants to create a special effect
%    depending on the page a certain fragment of text appears on. This
%    can be achieved by the following piece of code:
%\begin{verbatim}
%    \ifthenelse{\isodd{\pageref{some:label}}}
%               {code for odd pages}
%               {code for even pages}
%\end{verbatim}
%    In order for this to work the argument of |\isodd| needs to be
%    fully expandable. With the above redefinition of |\pageref| it is
%    not in the case of this example. To overcome that, we add some
%    code to the definition of |\ifthenelse| to make things work.
%
%    The first thing we need to do is check if the package
%    \pkg{ifthen} is loaded. This should be done at |\begin{document}|
%    time.
%
%    \begin{macrocode}
\bbl@xin@{R}\bbl@opt@safe
\ifin@
  \AtBeginDocument{%
    \@ifpackageloaded{ifthen}{%
%    \end{macrocode}
%
%    Then we can redefine |\ifthenelse|:
%
% \changes{babel~3.9a}{2012/06/22}{\cs{ref} is also taken into account}
% \changes{babel~3.9n}{2015/12/14}{Don't use generic temp
%    macros. babel/4441}
%
%    \begin{macrocode}
      \bbl@redefine@long\ifthenelse#1#2#3{%
%    \end{macrocode}
%
%    We want to revert the definition of |\pageref| and |\ref| to
%    their original definition for the first argument of |\ifthenelse|,
%    so we first need to store their current meanings.
%
%    \begin{macrocode}
        \let\bbl@temp@pref\pageref
        \let\pageref\org@pageref
        \let\bbl@temp@ref\ref
        \let\ref\org@ref
%    \end{macrocode}
%
%    Then we can set the |\@safe@actives| switch and call the original
%    |\ifthenelse|. In order to be able to use shorthands in the
%    second and third arguments of |\ifthenelse| the resetting of the
%    switch \emph{and} the definition of |\pageref| happens inside
%    those arguments.  When the package wasn't loaded we do nothing.
%
%    \begin{macrocode}
        \@safe@activestrue
        \org@ifthenelse{#1}%
          {\let\pageref\bbl@temp@pref
           \let\ref\bbl@temp@ref
           \@safe@activesfalse
           #2}%
          {\let\pageref\bbl@temp@pref
           \let\ref\bbl@temp@ref
           \@safe@activesfalse
           #3}%
        }%
      }{}%
    }
%    \end{macrocode}
%
%  \end{macro}
%
%  \subsubsection{\pkg{varioref}}
%
%  \begin{macro}{\@@vpageref}
%  \begin{macro}{\vrefpagenum}
%  \begin{macro}{\Ref}
%
% \changes{babel~3.8g}{2005/05/21}{We also need to adapt \cs{Ref}
%    which needs to be able to uppercase the first letter of the
%    expansion of \cs{ref}}
%
%    When the package varioref is in use we need to modify its
%    internal command |\@@vpageref| in order to prevent problems when
%    an active character ends up in the argument of |\vref|.
%
%    \begin{macrocode}
  \AtBeginDocument{%
    \@ifpackageloaded{varioref}{%
      \bbl@redefine\@@vpageref#1[#2]#3{%
        \@safe@activestrue
        \org@@@vpageref{#1}[#2]{#3}%
        \@safe@activesfalse}%
%    \end{macrocode}
%
%    The same needs to happen for |\vrefpagenum|.
%
%    \begin{macrocode}
      \bbl@redefine\vrefpagenum#1#2{%
        \@safe@activestrue
        \org@vrefpagenum{#1}{#2}%
        \@safe@activesfalse}%
%    \end{macrocode}
%
%    The package \pkg{varioref} defines |\Ref| to be a robust command
%    wich uppercases the first character of the reference text. In
%    order to be able to do that it needs to access the exandable form
%    of |\ref|. So we employ a little trick here. We redefine the
%    (internal) command \verb*|\Ref | to call |\org@ref| instead of
%    |\ref|. The disadvantgage of this solution is that whenever the
%    derfinition of |\Ref| changes, this definition needs to be updated
%    as well.
%
%    \begin{macrocode}
      \expandafter\def\csname Ref \endcsname#1{%
        \protected@edef\@tempa{\org@ref{#1}}\expandafter\MakeUppercase\@tempa}
      }{}%
    }
\fi
%    \end{macrocode}
%
%  \end{macro}
%  \end{macro}
%  \end{macro}
%
%  \subsubsection{\pkg{hhline}}
%
%  \begin{macro}{\hhline}
%    Delaying the activation of the shorthand characters has introduced
%    a problem with the \pkg{hhline} package. The reason is that it
%    uses the `:' character which is made active by the french support
%    in \babel. Therefore we need to \emph{reload} the package when
%    the `:' is an active character.
%
%    So at |\begin{document}| we check whether \pkg{hhline} is loaded.
%
%    \begin{macrocode}
\AtEndOfPackage{%
  \AtBeginDocument{%
    \@ifpackageloaded{hhline}%
%    \end{macrocode}
%
%    Then we check whether the expansion of |\normal@char:| is not
%    equal to |\relax|.
%
% \changes{babel~3.8b}{2004/04/19}{added \cs{string} to prevent
%    unwanted expansion of the colon}
%
%    \begin{macrocode}
      {\expandafter\ifx\csname normal@char\string:\endcsname\relax
       \else
%    \end{macrocode}
%
%    In that case we simply reload the package. Note that this happens
%    \emph{after} the category code of the @-sign has been changed to
%    other, so we need to temporarily change it to letter again.
%
%    \begin{macrocode}
         \makeatletter
         \def\@currname{hhline}% \iffalse meta-comment
%
% Copyright 1993-2014
%
% The LaTeX3 Project and any individual authors listed elsewhere
% in this file.
%
% This file is part of the Standard LaTeX `Tools Bundle'.
% -------------------------------------------------------
%
% It may be distributed and/or modified under the
% conditions of the LaTeX Project Public License, either version 1.3c
% of this license or (at your option) any later version.
% The latest version of this license is in
%    http://www.latex-project.org/lppl.txt
% and version 1.3c or later is part of all distributions of LaTeX
% version 2005/12/01 or later.
%
% The list of all files belonging to the LaTeX `Tools Bundle' is
% given in the file `manifest.txt'.
%
% \fi
% \iffalse
%% File: hhline.dtx Copyright (C) 1991-1994 David Carlisle
%
%<package>\NeedsTeXFormat{LaTeX2e}
%<package>\ProvidesPackage{hhline}
%<package>         [2014/10/28 v2.03 Table rule package (DPC)]
%
%<*driver>
\documentclass{ltxdoc}
\usepackage{hhline}
\GetFileInfo{hhline.sty}
\begin{document}
\title{The \textsf{hhline} package\thanks{This file
        has version number \fileversion, last
        revised \filedate.}}
\author{David Carlisle\\carlisle@cs.man.ac.uk}
\date{\filedate}
 \maketitle
 \DeleteShortVerb{\|}
 \DocInput{hhline.dtx}
\end{document}
%</driver>
% \fi
%
%
% \changes{v1.00}{1991/06/04}{Initial Version}
% \changes{v2.00}{1991/11/06}
%     {Add tilde which allows \cmd\cline-like constructions.}
% \changes{v2.01}{1992/06/26}
%    {Re-issue for the new  doc and docstrip.}
% \changes{v2.02}{1994/03/14}
%    {Update for LaTeX2e.}
% \changes{v2.03}{1994/05/23}
%    {New style warning.}
%
%
% \CheckSum{244}
%
% \MakeShortVerb{\"}
%
% \begin{abstract}
% "\hhline" produces a line like "\hline", or a double line like
% "\hline\hline", except for its interaction with vertical lines.
% \end{abstract}
%
% \arrayrulewidth=1pt
% \doublerulesep=3pt
%
% \section{Introduction}
% The argument to "\hhline" is similar to the preamble of an {\tt
% array} or {\tt tabular}. It consists of a list of tokens with the
% following meanings:
% \[
% \begin{tabular}{cl}
%   "="   & A double hline the width of a column.\\
%   "-"   & A single hline the width of a column.\\[10pt]
%   "~"   & A column with no hline.\\[10pt]
%
%   "|"   & A vline which `cuts' through a double (or single) hline.\\
%   ":"   & A vline which is broken by a double hline.\\[10pt]
%
%   "#"   & A double hline segment between two vlines.\\
%   "t"   & The top half of a double hline segment.\\
%   "b"   & The bottom half of a double hline segment.\\
%
%   "*"   & "*{3}{==#}" expands to "==#==#==#",
%                   as in the {\tt*}-form for the preamble.
% \end{tabular}
% \]
% If a double vline is specified ("||" or "::") then the hlines
% produced by "\hhline" are broken. To obtain the effect of an hline
% `cutting through' the double vline, use a "#" or omit the vline
% specifiers, depending on whether or not you wish the double vline to
% break.
%
% The tokens {\tt t} and {\tt b} must be used between two vertical
% rules. "|tb|" produces the same lines  as "#", but is much less
% efficient. The main use for these are to make constructions like
% "|t:" (top left corner) and ":b|" (bottom right corner).
%
% If "\hhline" is used to make a single hline, then the argument
% should only contain the tokens "-", "~"  and "|" (and
% {\tt*}-expressions).
%
% An example using most of these features is:
% \[
% \vcenter{\hsize=2in\begin{verbatim}
% \begin{tabular}{||cc||c|c||}
% \hhline{|t:==:t:==:t|}
% a&b&c&d\\
% \hhline{|:==:|~|~||}
% 1&2&3&4\\
% \hhline{#==#~|=#}
% i&j&k&l\\
% \hhline{||--||--||}
% w&x&y&z\\
% \hhline{|b:==:b:==:b|}
% \end{tabular}
% \end{verbatim}
% }
% \qquad
% \begin{tabular}{||cc||c|c||}
% \hhline{|t:==:t:==:t|}
% a&b&c&d\\
% \hhline{|:==:|~|~||}
% 1&2&3&4\\
% \hhline{#==#~|=#}
% i&j&k&l\\
% \hhline{||--||--||}
% w&x&y&z\\
% \hhline{|b:==:b:==:b|}
% \end{tabular}
% \]
%
% The lines produced by \LaTeX's "\hline" consist of a single (\TeX\
% primitive) "\hrule". The lines produced by "\hhline" are made
% up of lots of small line segments. \TeX\ will place these very
% accurately in the {\tt .dvi} file, but the program that you use to
% print the {\tt .dvi} file may not line up these segments exactly. (A
% similar problem can occur with diagonal lines in the {\tt picture}
% environment.)
%
% If this effect causes a problem, you could try a different driver
% program, or if this is not possible, increasing "\arrayrulewidth"
% may help to reduce the effect.
%
% \StopEventually{}
%
% \section{The Macros}
%
%    \begin{macrocode}
%<*package>
%    \end{macrocode}
%
% \begin{macro}{\HH@box}
% Makes a box containing a double hline segment. The most common case,
% both rules of length "\doublerulesep" will be stored in "\box1", this
% is not initialised until "\hhline" is called as the user may change
% the parameters "\doublerulesep" and "\arrayrulewidth". The two
% arguments to "\HH@box" are the widths (ie lengths) of the top and
% bottom rules.
%    \begin{macrocode}
\def\HH@box#1#2{\vbox{%
  \hrule \@height \arrayrulewidth \@width #1
  \vskip \doublerulesep
  \hrule \@height \arrayrulewidth \@width #2}}
%    \end{macrocode}
% \end{macro}
%
% \begin{macro}{\HH@add}
% Build up the preamble in the register "\toks@".
%    \begin{macrocode}
\def\HH@add#1{\toks@\expandafter{\the\toks@#1}}
%    \end{macrocode}
% \end{macro}

% \begin{macro}{\HH@xexpast}
% \begin{macro}{\HH@xexnoop}
% We `borrow' the version of "\@xexpast" from Mittelbach's array.sty,
% as this allows "#" to appear in the argument list.
%    \begin{macrocode}
\def\HH@xexpast#1*#2#3#4\@@{%
   \@tempcnta #2
   \toks@={#1}\@temptokena={#3}%
   \let\the@toksz\relax \let\the@toks\relax
   \def\@tempa{\the@toksz}%
   \ifnum\@tempcnta >0 \@whilenum\@tempcnta >0\do
     {\edef\@tempa{\@tempa\the@toks}\advance \@tempcnta \m@ne}%
       \let \@tempb \HH@xexpast \else
       \let \@tempb \HH@xexnoop \fi
   \def\the@toksz{\the\toks@}\def\the@toks{\the\@temptokena}%
   \edef\@tempa{\@tempa}%
   \expandafter \@tempb \@tempa #4\@@}

\def\HH@xexnoop#1\@@{}
%    \end{macrocode}
% \end{macro}
% \end{macro}
%
% \begin{macro}{\hhline}
% Use a simplified version of "\@mkpream" to break apart the argument
% to "\hhline". Actually it is oversimplified, It assumes that the
% vertical rules are at the end of the column. If you were to specify
% "c|@{xx}|" in the array argument, then "\hhline" would not be
% able to access the first vertical rule. (It ought to have an "@"
% option, and add "\leaders" up to the width of a box containing the
% "@"-expression. We use a loop made with "\futurelet" rather
% than "\@tfor" so that we can use "#" to denote the crossing of
% a double hline with a double vline.\\
% "\if@firstamp" is true in the first column and false otherwise.\\
% "\if@tempswa"  is true if the previous entry was a vline
%                     (":", "|" or "#").
%    \begin{macrocode}
\def\hhline#1{\omit\@firstamptrue\@tempswafalse
%    \end{macrocode}
% Put two rules of width "\doublerulesep" in "\box1"
%    \begin{macrocode}
\global\setbox\@ne\HH@box\doublerulesep\doublerulesep
%    \end{macrocode}
% If Mittelbach's {\tt array.sty} is loaded, we do not need the negative
% "\hskip"'s around vertical rules.
%    \begin{macrocode}
  \xdef\@tempc{\ifx\extrarowheight\HH@undef\hskip-.5\arrayrulewidth\fi}%
%    \end{macrocode}
% Now expand the {\tt*}-forms and add dummy tokens ( "\relax" and
% "`" ) to either end of the token list. Call "\HH@let" to start
% processing the token list.
%    \begin{macrocode}
    \HH@xexpast\relax#1*0x\@@\toks@{}\expandafter\HH@let\@tempa`}
%    \end{macrocode}
% \end{macro}

% \begin{macro}{\HH@let}
% Discard the last token, look at the next one.
%    \begin{macrocode}
\def\HH@let#1{\futurelet\@tempb\HH@loop}
%    \end{macrocode}
% \end{macro}

% \begin{macro}{\HH@loop}
% The main loop. Note we use "\ifx" rather than "\if" in
% version~2 as the new token "~" is active.
%    \begin{macrocode}
\def\HH@loop{%
%    \end{macrocode}
% If next token is "`", stop the loop and put the lines into this row
% of the alignment.
%    \begin{macrocode}
  \ifx\@tempb`\def\next##1{\the\toks@\cr}\else\let\next\HH@let
%    \end{macrocode}
% "|", add a vertical rule (across either a double or
% single hline).
%    \begin{macrocode}
  \ifx\@tempb|\if@tempswa\HH@add{\hskip\doublerulesep}\fi\@tempswatrue
          \HH@add{\@tempc\vline\@tempc}\else
%    \end{macrocode}
% ":", add a broken vertical rule (across a double hline).
%    \begin{macrocode}
  \ifx\@tempb:\if@tempswa\HH@add{\hskip\doublerulesep}\fi\@tempswatrue
      \HH@add{\@tempc\HH@box\arrayrulewidth\arrayrulewidth\@tempc}\else
%    \end{macrocode}
% "#", add a double hline segment between two vlines.
%    \begin{macrocode}
  \ifx\@tempb##\if@tempswa\HH@add{\hskip\doublerulesep}\fi\@tempswatrue
         \HH@add{\@tempc\vline\@tempc\copy\@ne\@tempc\vline\@tempc}\else
%    \end{macrocode}
% "~", A column with no hline (this gives an effect similar to
% \verb+\cline+).
%    \begin{macrocode}
  \ifx\@tempb~\@tempswafalse
           \if@firstamp\@firstampfalse\else\HH@add{&\omit}\fi
              \HH@add{\hfil}\else
%    \end{macrocode}
% "-", add a single hline across the column.
%    \begin{macrocode}
  \ifx\@tempb-\@tempswafalse
           \if@firstamp\@firstampfalse\else\HH@add{&\omit}\fi
              \HH@add{\leaders\hrule\@height\arrayrulewidth\hfil}\else
%    \end{macrocode}
% "=", add a double hline across the column.
%    \begin{macrocode}
  \ifx\@tempb=\@tempswafalse
       \if@firstamp\@firstampfalse\else\HH@add{&\omit}\fi
%    \end{macrocode}
%     Put in as many copies of "\box1" as possible with
%     "\leaders", this may leave gaps at the ends, so put an extra box
%     at each end, overlapping the "\leaders".
%    \begin{macrocode}
       \HH@add
          {\rlap{\copy\@ne}\leaders\copy\@ne\hfil\llap{\copy\@ne}}\else
%    \end{macrocode}
% "t", add the top half of a double hline segment, in a "\rlap"
% so that it may be used with {\tt b}.
%    \begin{macrocode}
  \ifx\@tempb t\HH@add{\rlap{\HH@box\doublerulesep\z@}}\else
%    \end{macrocode}
% "b", add the bottom half of a double hline segment in a "\rlap"
% so that it may be used with {\tt t}.
%    \begin{macrocode}
  \ifx\@tempb b\HH@add{\rlap{\HH@box\z@\doublerulesep}}\else
%    \end{macrocode}
% Otherwise ignore the token, with a warning.
%    \begin{macrocode}
  \PackageWarning{hhline}%
      {\meaning\@tempb\space ignored in \noexpand\hhline argument%
       \MessageBreak}%
  \fi\fi\fi\fi\fi\fi\fi\fi\fi
%    \end{macrocode}
% Go around the loop again.
%    \begin{macrocode}
  \next}
%    \end{macrocode}
% \end{macro}
%
%    \begin{macrocode}
%</package>
%    \end{macrocode}
%
% \Finale
\endinput
\makeatother
       \fi}%
      {}}}
%    \end{macrocode}
%
%  \end{macro}
%
%  \subsubsection{\pkg{hyperref}}
%
%  \begin{macro}{\pdfstringdefDisableCommands}
%
% \changes{babel~3.8j}{2008/03/16}{Inform \pkg{hyperref} to use
%    shorthands at system level (PR4006)}
%
%    A number of interworking problems between \pkg{babel} and
%    \pkg{hyperref} are tackled by \pkg{hyperref} itself. The
%    following code was introduced to prevent some annoying warnings
%    but it broke bookmarks. This was quickly fixed in \pkg{hyperref},
%    which essentially made it no-op. However, it will not removed for
%    the moment because \pkg{hyperref} is expecting it.
%    
%    \begin{macrocode}
\AtBeginDocument{%
  \ifx\pdfstringdefDisableCommands\@undefined\else
    \pdfstringdefDisableCommands{\languageshorthands{system}}%
  \fi}
%    \end{macrocode}
%
%  \end{macro}
%
%  \subsubsection{\pkg{fancyhdr}}
%
%  \begin{macro}{\FOREIGNLANGUAGE}
%    The package \pkg{fancyhdr} treats the running head and fout lines
%    somewhat differently as the standard classes. A symptom of this is
%    that the command |\foreignlanguage| which \babel\ adds to the
%    marks can end up inside the argument of |\MakeUppercase|. To
%    prevent unexpected results we need to define |\FOREIGNLANGUAGE|
%    here.
%
%    \begin{macrocode}
\DeclareRobustCommand{\FOREIGNLANGUAGE}[1]{%
  \lowercase{\foreignlanguage{#1}}}
%    \end{macrocode}
%
%  \end{macro}
%
%  \begin{macro}{\substitutefontfamily}
%    The command |\substitutefontfamily| creates an
%    \file{.fd} file on the fly. The first argument is an encoding
%    mnemonic, the second and third arguments are font family names.
%
%    \begin{macrocode}
\def\substitutefontfamily#1#2#3{%
  \lowercase{\immediate\openout15=#1#2.fd\relax}%
  \immediate\write15{%
    \string\ProvidesFile{#1#2.fd}%
    [\the\year/\two@digits{\the\month}/\two@digits{\the\day}
     \space generated font description file]^^J
    \string\DeclareFontFamily{#1}{#2}{}^^J
    \string\DeclareFontShape{#1}{#2}{m}{n}{<->ssub * #3/m/n}{}^^J
    \string\DeclareFontShape{#1}{#2}{m}{it}{<->ssub * #3/m/it}{}^^J
    \string\DeclareFontShape{#1}{#2}{m}{sl}{<->ssub * #3/m/sl}{}^^J
    \string\DeclareFontShape{#1}{#2}{m}{sc}{<->ssub * #3/m/sc}{}^^J
    \string\DeclareFontShape{#1}{#2}{b}{n}{<->ssub * #3/bx/n}{}^^J
    \string\DeclareFontShape{#1}{#2}{b}{it}{<->ssub * #3/bx/it}{}^^J
    \string\DeclareFontShape{#1}{#2}{b}{sl}{<->ssub * #3/bx/sl}{}^^J
    \string\DeclareFontShape{#1}{#2}{b}{sc}{<->ssub * #3/bx/sc}{}^^J
    }%
  \closeout15
  }
%    \end{macrocode}
%
%    This command should only be used in the preamble of a document.
%
%    \begin{macrocode}
\@onlypreamble\substitutefontfamily
%    \end{macrocode}
%
%  \end{macro}
%
%  \subsection{Encoding and fonts}
%
%  Because documents may use non-ASCII font encodings, we make sure
%  that the logos of \TeX\ and \LaTeX\ always come out in the right
%  encoding. There is a list of non-ASCII encodings. Unfortunately,
%  \textsf{fontenc} deletes its package options, so we must guess
%  which encodings has been loaded by traversing |\@filelist| to
%  search for \m{enc}|enc.def|. If a non-ASCII has been loaded, we
%  define versions of |\TeX| and |\LaTeX| for them using
%  |\ensureascii|. The default ASCII encoding is set, too (in reverse
%  order): the ``main'' encoding (when the document begins), the last
%  loaded, or |OT1|.
%
%  \begin{macro}{\ensureascii}
%
% \changes{babel~3.9i}{2014/02/14}{Macro added, to replace
%    \cs{textlatin} and friends}
% \changes{babel~3.9j}{2014/03/17}{Moved misplaced code - it should be
%    executed only with LaTeX}
%
%    \begin{macrocode}
\newcommand\BabelNonASCII{LGR,X2,OT2,OT3,OT6,LHE,LWN,LMA,LMC,LMS,LMU,}
\let\org@TeX\TeX
\let\org@LaTeX\LaTeX
\let\ensureascii\@firstofone
\AtBeginDocument{%
  \in@false
  \bbl@foreach\BabelNonASCII{% is there a non-ascii enc?
    \ifin@\else
      \lowercase{\bbl@xin@{,#1enc.def,}{,\@filelist,}}%
    \fi}%
  \ifin@ % if a non-ascii has been loaded
    \def\ensureascii#1{{\fontencoding{OT1}\selectfont#1}}%
    \DeclareTextCommandDefault{\TeX}{\org@TeX}%
    \DeclareTextCommandDefault{\LaTeX}{\org@LaTeX}%
    \def\bbl@tempb#1\@@{\uppercase{\bbl@tempc#1}ENC.DEF\@empty\@@}%
    \def\bbl@tempc#1ENC.DEF#2\@@{%
      \ifx\@empty#2\else
        \bbl@ifunset{T@#1}% 
          {}%
          {\bbl@xin@{,#1,}{,\BabelNonASCII,}%
           \ifin@
             \DeclareTextCommand{\TeX}{#1}{\ensureascii{\org@TeX}}%
             \DeclareTextCommand{\LaTeX}{#1}{\ensureascii{\org@LaTeX}}%
           \else
             \def\ensureascii##1{{\fontencoding{#1}\selectfont##1}}%
           \fi}%
      \fi}%
    \bbl@foreach\@filelist{\bbl@tempb#1\@@}%  TODO - \@@ de mas??
    \bbl@xin@{,\cf@encoding,}{,\BabelNonASCII,}%
    \ifin@\else
      \edef\ensureascii#1{{%
        \noexpand\fontencoding{\cf@encoding}\noexpand\selectfont#1}}%
    \fi
  \fi}
%    \end{macrocode}
%
%  \end{macro}
%
%  Now comes the old deprecated stuff (with a little change in 3.9l,
%  for \textsf{fontspec}).  The first thing we need to do is to
%  determine, at |\begin{document}|, which latin fontencoding to use.
%
%  \begin{macro}{\latinencoding}
%    When text is being typeset in an encoding other than `latin'
%    (\texttt{OT1} or \texttt{T1}), it would be nice to still have
%    Roman numerals come out in the Latin encoding.
%    So we first assume that the current encoding at the end
%    of processing the package is the Latin encoding.
%
%    \begin{macrocode}
\AtEndOfPackage{\edef\latinencoding{\cf@encoding}}
%    \end{macrocode}
%
%    But this might be overruled with a later loading of the package
%    \pkg{fontenc}. Therefore we check at the execution of
%    |\begin{document}| whether it was loaded with the \Lopt{T1}
%    option. The normal way to do this (using |\@ifpackageloaded|) is
%    disabled for this package. Now we have to revert to parsing the
%    internal macro |\@filelist| which contains all the filenames
%    loaded.
%
% \changes{babel~3.9l}{2014/08/02}{fontspec used to set
%    \cs{latinencoding} to EUx, but now it doesn't. So, it's done
%    here.}
% \changes{babel~3.9o}{2016/01/27}{With fontspec, first check if
%   \cs{UTFencname} exists.}
%
%    \begin{macrocode}
\AtBeginDocument{%
  \@ifpackageloaded{fontspec}%
    {\xdef\latinencoding{%
       \ifx\UTFencname\@undefined
         EU\ifcase\bbl@engine\or2\or1\fi
       \else
         \UTFencname
       \fi}}%
    {\gdef\latinencoding{OT1}%
     \ifx\cf@encoding\bbl@t@one
       \xdef\latinencoding{\bbl@t@one}%
     \else
       \@ifl@aded{def}{t1enc}{\xdef\latinencoding{\bbl@t@one}}{}%
     \fi}}
%    \end{macrocode}
%
%  \end{macro}
%
%  \begin{macro}{\latintext}
%    Then we can define the command |\latintext| which is a
%    declarative switch to a latin font-encoding. Usage of this macro
%    is deprecated.
%
%    \begin{macrocode}
\DeclareRobustCommand{\latintext}{%
  \fontencoding{\latinencoding}\selectfont
  \def\encodingdefault{\latinencoding}}
%    \end{macrocode}
%
%  \end{macro}
%
%  \begin{macro}{\textlatin}
%    This command takes an argument which is then typeset using the
%    requested font encoding. In order to avoid many encoding switches
%    it operates in a local scope.
%
%    \begin{macrocode}
\ifx\@undefined\DeclareTextFontCommand
  \DeclareRobustCommand{\textlatin}[1]{\leavevmode{\latintext #1}}
\else
  \DeclareTextFontCommand{\textlatin}{\latintext}
\fi
%    \end{macrocode}
%
% \end{macro}
%   
% \subsection{Basic bidi support}
%
%    \textbf{Work in progress.} This code is currently placed here for
%    practical reasons. 
%
%    \begin{itemize}
%    \item pdftex provides a minimal support for bidi text, and it
%      must be done by hand. Vertical typesetting is not possible.
%    \item \xetex{} is somewhat better, thanks to its font engine
%      (even if not always reliable) and a few additional tools. However,
%      very little is done at the paragraph level. Another challenging
%      problem is text direction does not honour \TeX{} grouping.
%    \item \luatex{} can provide the most complete solution, as we can
%      manipulate almost freely the node list, the generated lines,
%      and so on, but bidi text does not work out of the box and some
%      development is necessary. It also provides tools to properly
%      set left-to-right and right-to-left page layouts. As Lua\TeX-ja
%      shows, vertical typesetting is posible, too. Its main drawback
%      is font handling is often considered to be less mature than
%      \xetex.\footnote{Although in my [JBL] experience problems are
%      in fact minimal.}
%    \end{itemize}
%
% \changes{3.15}{2017/10/30}{Use an attribute instead of tex language
%    (reserved for hyphenation).}
% \changes{3.15}{2017/10/30}{Store direction in @wdir@<lang>.}
%
%    \begin{macrocode}
\def\bbl@alscripts{,Arabic,Syriac,Thaana,}
\def\bbl@rscripts{%
  ,Imperial Aramaic,Avestan,Cypriot,Hatran,Hebrew,%
  Old Hungarian,Old Hungarian,Lydian,Mandaean,Manichaean,%
  Manichaean,Meroitic Cursive,Meroitic,Old North Arabian,%
  Nabataean,N'Ko,Orkhon,Palmyrene,Inscriptional Pahlavi,%
  Psalter Pahlavi,Phoenician,Inscriptional Parthian,Samaritan,%
  Old South Arabian,}%
\def\bbl@provide@dirs#1{%
  \bbl@xin@{\csname bbl@sname@#1\endcsname}{\bbl@alscripts\bbl@rscripts}%
  \ifin@
    \global\bbl@csarg\chardef{wdir@#1}\@ne
    \bbl@xin@{\csname bbl@sname@#1\endcsname}{\bbl@alscripts}%
    \ifin@
      \global\bbl@csarg\chardef{wdir@#1}\tw@  % useless in xetex
    \fi
  \else
    \global\bbl@csarg\chardef{wdir@#1}\z@
  \fi}
\def\bbl@switchdir{%
  \bbl@ifunset{bbl@lsys@\languagename}{\bbl@provide@lsys{\languagename}}{}%
  \bbl@ifunset{bbl@wdir@\languagename}{\bbl@provide@dirs{\languagename}}{}%
  \bbl@exp{\\\bbl@setdirs\bbl@cs{wdir@\languagename}}}
\def\bbl@setdirs#1{% TODO - math
  \ifcase\bbl@select@type % TODO - strictly, not the right test
    \bbl@pagedir{#1}%
    \bbl@bodydir{#1}%
    \bbl@pardir{#1}%
  \fi
  \bbl@textdir{#1}}
\ifodd\bbl@engine
  \AddBabelHook{babel-bidi}{afterextras}{\bbl@switchdir}
  \DisableBabelHook{babel-bidi}
  \def\bbl@getluadir#1{%
    \directlua{
      if tex.#1dir == 'TLT' then
        tex.sprint('0')
      elseif tex.#1dir == 'TRT' then
        tex.sprint('1')
      end}}
  \def\bbl@setdir#1#2#3{% 1=text/par.. 2=\textdir.. 3=0 lr/1 rl
    \ifcase#3\relax
      \ifcase\bbl@getluadir{#1}\relax\else
        #2 TLT\relax
      \fi
    \else
      \ifcase\bbl@getluadir{#1}\relax
        #2 TRT\relax
      \fi
    \fi}
  \def\bbl@textdir#1{%
    \bbl@setdir{text}\textdir{#1}% TODO - ?\linedir
    \setattribute\bbl@attr@dir{#1}}
  \def\bbl@pardir{\bbl@setdir{par}\pardir}
  \def\bbl@bodydir{\bbl@setdir{body}\bodydir}
  \def\bbl@pagedir{\bbl@setdir{page}\pagedir}
  \def\bbl@dirparastext{\pardir\the\textdir\relax}%   %%%%
\else
  \AddBabelHook{babel-bidi}{afterextras}{\bbl@switchdir}
  \DisableBabelHook{babel-bidi}
  \newcount\bbl@dirlevel
  \chardef\bbl@thetextdir\z@
  \chardef\bbl@thepardir\z@
  \def\bbl@textdir#1{%
    \ifcase#1\relax
       \chardef\bbl@thetextdir\z@
       \bbl@textdir@i\beginL\endL
     \else
       \chardef\bbl@thetextdir\@ne
       \bbl@textdir@i\beginR\endR
    \fi}
  \def\bbl@textdir@i#1#2{%
    \ifhmode
      \ifnum\currentgrouplevel>\z@
        \ifnum\currentgrouplevel=\bbl@dirlevel
          \bbl@error{Multiple bidi settings inside a group}%
            {I'll insert a new group, but expect wrong results.}%
          \bgroup\aftergroup#2\aftergroup\egroup
        \else
          \ifcase\currentgrouptype\or % 0 bottom 
            \aftergroup#2% 1 simple {}
          \or
            \bgroup\aftergroup#2\aftergroup\egroup % 2 hbox
          \or
            \bgroup\aftergroup#2\aftergroup\egroup % 3 adj hbox
          \or\or\or % vbox vtop align
          \or
            \bgroup\aftergroup#2\aftergroup\egroup % 7 noalign
          \or\or\or\or\or\or % output math disc insert vcent mathchoice
          \or
            \aftergroup#2% 14 \begingroup
          \else
            \bgroup\aftergroup#2\aftergroup\egroup % 15 adj
          \fi
        \fi
        \bbl@dirlevel\currentgrouplevel
      \fi
      #1%
    \fi}
  \def\bbl@pardir#1{\chardef\bbl@thepardir#1\relax}
  \let\bbl@bodydir\@gobble
  \let\bbl@pagedir\@gobble
  \def\bbl@dirparastext{\chardef\bbl@thepardir\bbl@thetextdir}
%    \end{macrocode}
%  
% The following command is executed only if there is a right-to-left
% script (once). It activates the |\everypar| hack for \xetex, to
% properly handle the par direction. Note text and par dirs are
% decoupled.
%
%    \begin{macrocode}
  \def\bbl@xebidipar{%
    \let\bbl@xebidipar\relax 
    \TeXXeTstate\@ne
    \def\bbl@xeeverypar{%
      \ifcase\bbl@thepardir\else
        {\setbox\z@\lastbox\beginR\box\z@}%
      \fi
      \ifcase\bbl@thetextdir\else\beginR\fi}%
    \let\bbl@severypar\everypar
    \newtoks\everypar
    \everypar=\bbl@severypar
    \bbl@severypar{\bbl@xeeverypar\the\everypar}}
\fi
%    \end{macrocode}
%
% \subsection{Local Language Configuration}
%
%  \begin{macro}{\loadlocalcfg}
%    At some sites it may be necessary to add site-specific actions to
%    a language definition file. This can be done by creating a file
%    with the same name as the language definition file, but with the
%    extension \file{.cfg}. For instance the file \file{norsk.cfg}
%    will be loaded when the language definition file \file{norsk.ldf}
%    is loaded.
%
%    For plain-based formats we don't want to override the definition
%    of |\loadlocalcfg| from \file{plain.def}.
%
%    \begin{macrocode}
\ifx\loadlocalcfg\@undefined
  \@ifpackagewith{babel}{noconfigs}%
    {\let\loadlocalcfg\@gobble}%
    {\def\loadlocalcfg#1{%
      \InputIfFileExists{#1.cfg}%
        {\typeout{*************************************^^J%
                       * Local config file #1.cfg used^^J%
                       *}}%
        \@empty}}
\fi
%    \end{macrocode}
%
%    Just to be compatible with \LaTeX$\:$2.09 we add a few more lines
%    of code:
%
%    \begin{macrocode}
\ifx\@unexpandable@protect\@undefined
  \def\@unexpandable@protect{\noexpand\protect\noexpand}
  \long\def\protected@write#1#2#3{%
    \begingroup
      \let\thepage\relax
      #2%
      \let\protect\@unexpandable@protect
      \edef\reserved@a{\write#1{#3}}%
      \reserved@a
    \endgroup
    \if@nobreak\ifvmode\nobreak\fi\fi}
\fi
%</core>
%    \end{macrocode}%
%    \end{macro}
%
% \section{Multiple languages (\texttt{switch.def)}}
%
%    Plain \TeX\ version~3.0 provides the primitive |\language| that
%    is used to store the current language. When used with a pre-3.0
%    version this function has to be implemented by allocating a
%    counter.
%
%    \begin{macrocode}
%<*kernel>
<@Make sure ProvidesFile is defined@>
\ProvidesFile{switch.def}[<@date@> <@version@> Babel switching mechanism]
<@Load macros for plain if not LaTeX@>
<@Define core switching macros@>
%    \end{macrocode}
%
%  \begin{macro}{\adddialect}
%    The macro |\adddialect| can be used to add the name of a dialect
%    or variant language, for which an already defined hyphenation
%    table can be used.
%
%    \begin{macrocode}
\def\bbl@version{<@version@>}
\def\bbl@date{<@date@>}
\def\adddialect#1#2{%
  \global\chardef#1#2\relax
  \bbl@usehooks{adddialect}{{#1}{#2}}%
  \wlog{\string#1 = a dialect from \string\language#2}}
%    \end{macrocode}
%
%  \end{macro}
%
% \changes{babel~3.9a}{2012/09/07}{Added macro}
% \changes{babel~3.9a}{2013/01/23}{New macro to normalize 
%    a macro (eg, \cs{languagename}) to lowercase if necessary}
%
%    |\bbl@iflanguage| executes code only if the language |l@|
%    exists. Otherwise raises and error.
%
%    The argument of |\bbl@fixname| has to be a macro name, as it may get
%    ``fixed'' if casing (lc/uc) is wrong. It's intented to fix a
%    long-standing bug when |\foreignlanguage| and the like appear in
%    a |\MakeXXXcase|. However, a lowercase form is not imposed to
%    improve backward compatibility (perhaps you defined a language
%    named |MYLANG|, but unfortunately mixed case names cannot be
%    trapped). Note |l@| is encapsulated, so that its case does not
%    change.
%
%    \begin{macrocode}
\def\bbl@fixname#1{%
  \begingroup
    \def\bbl@tempe{l@}%
    \edef\bbl@tempd{\noexpand\@ifundefined{\noexpand\bbl@tempe#1}}%
    \bbl@tempd
      {\lowercase\expandafter{\bbl@tempd}%
         {\uppercase\expandafter{\bbl@tempd}%
           \@empty
           {\edef\bbl@tempd{\def\noexpand#1{#1}}%
            \uppercase\expandafter{\bbl@tempd}}}%
         {\edef\bbl@tempd{\def\noexpand#1{#1}}%
          \lowercase\expandafter{\bbl@tempd}}}%
      \@empty
    \edef\bbl@tempd{\endgroup\def\noexpand#1{#1}}%
  \bbl@tempd}
\def\bbl@iflanguage#1{%
  \@ifundefined{l@#1}{\@nolanerr{#1}\@gobble}\@firstofone}
%    \end{macrocode}
%
%  \begin{macro}{\iflanguage}
%    Users might want to test (in a private package for instance)
%    which language is currently active. For this we provide a test
%    macro, |\iflanguage|, that has three arguments.  It checks
%    whether the first argument is a known language. If so, it
%    compares the first argument with the value of |\language|. Then,
%    depending on the result of the comparison, it executes either the
%    second or the third argument.
%
%    \begin{macrocode}
\def\iflanguage#1{%
  \bbl@iflanguage{#1}{%
    \ifnum\csname l@#1\endcsname=\language
      \expandafter\@firstoftwo
    \else
      \expandafter\@secondoftwo
    \fi}}
%    \end{macrocode}
%
%  \end{macro}
%
%   \subsection{Selecting the language}
%
%  \begin{macro}{\selectlanguage}
%    The macro |\selectlanguage| checks whether the language is
%    already defined before it performs its actual task, which is to
%    update |\language| and activate language-specific definitions.
%
%    To allow the call of |\selectlanguage| either with a control
%    sequence name or with a simple string as argument, we have to use
%    a trick to delete the optional escape character.
%
%    To convert a control sequence to a string, we use the |\string|
%    primitive.  Next we have to look at the first character of this
%    string and compare it with the escape character.  Because this
%    escape character can be changed by setting the internal integer
%    |\escapechar| to a character number, we have to compare this
%    number with the character of the string.  To do this we have to
%    use \TeX's backquote notation to specify the character as a
%    number.
%
%    If the first character of the |\string|'ed argument is the
%    current escape character, the comparison has stripped this
%    character and the rest in the `then' part consists of the rest of
%    the control sequence name.  Otherwise we know that either the
%    argument is not a control sequence or |\escapechar| is set to a
%    value outside of the character range~$0$--$255$.
%
%    If the user gives an empty argument, we provide a default
%    argument for |\string|.  This argument should expand to nothing.
%
% \changes{babel~3.9a}{2012/11/16}{\cs{bbl@select@type} keep tracks of
%    the selection method: 0 is select, 1 is foreign}
%
%    \begin{macrocode}
\let\bbl@select@type\z@
\edef\selectlanguage{%
  \noexpand\protect
  \expandafter\noexpand\csname selectlanguage \endcsname}
%    \end{macrocode}
%
%    Because the command |\selectlanguage| could be used in a moving
%    argument it expands to \verb*=\protect\selectlanguage =.
%    Therefore, we have to make sure that a macro |\protect| exists.
%    If it doesn't it is |\let| to |\relax|.
%
%    \begin{macrocode}
\ifx\@undefined\protect\let\protect\relax\fi
%    \end{macrocode}
%
%    As \LaTeX$\:$2.09 writes to files \textit{expanded} whereas
%    \LaTeXe\ takes care \textit{not} to expand the arguments of
%    |\write| statements we need to be a bit clever about the way we
%    add information to \file{.aux} files. Therefore we introduce the
%    macro |\xstring| which should expand to the right amount of
%    |\string|'s.
%
%    \begin{macrocode}
\ifx\documentclass\@undefined
  \def\xstring{\string\string\string}
\else
  \let\xstring\string
\fi
%    \end{macrocode}
%
% \end{macro}
%
%    Since version 3.5 \babel\ writes entries to the auxiliary files in
%    order to typeset table of contents etc. in the correct language
%    environment.
%  \begin{macro}{\bbl@pop@language}
%    \emph{But} when the language change happens \emph{inside} a group
%    the end of the group doesn't write anything to the auxiliary
%    files. Therefore we need \TeX's |aftergroup| mechanism to help
%    us. The command |\aftergroup| stores the token immediately
%    following it to be executed when the current group is closed. So
%    we define a temporary control sequence |\bbl@pop@language| to be
%    executed at the end of the group. It calls |\bbl@set@language|
%    with the name of the current language as its argument.
%
%  \begin{macro}{\bbl@language@stack}
%    The previous solution works for one level of nesting groups, but
%    as soon as more levels are used it is no longer adequate. For
%    that case we need to keep track of the nested languages using a
%    stack mechanism. This stack is called |\bbl@language@stack| and
%    initially empty.
%
%    \begin{macrocode}
\def\bbl@language@stack{}
%    \end{macrocode}
%
%    When using a stack we need a mechanism to push an element on the
%    stack and to retrieve the information afterwards.
%  \begin{macro}{\bbl@push@language}
%  \begin{macro}{\bbl@pop@language}
%    The stack is simply a list of languagenames, separated with a `+'
%    sign; the push function can be simple:
%
%    \begin{macrocode}
\def\bbl@push@language{%
  \xdef\bbl@language@stack{\languagename+\bbl@language@stack}}
%    \end{macrocode}
%
%    Retrieving information from the stack is a little bit less simple,
%    as we need to remove the element from the stack while storing it
%    in the macro |\languagename|. For this we first define a helper function.
%  \begin{macro}{\bbl@pop@lang}
%    This macro stores its first element (which is delimited by the
%    `+'-sign) in |\languagename| and stores the rest of the string
%    (delimited by `-') in its third argument.
%
%    \begin{macrocode}
\def\bbl@pop@lang#1+#2-#3{%
  \edef\languagename{#1}\xdef#3{#2}}
%    \end{macrocode}
%
%  \end{macro}
%    The reason for the somewhat weird arrangement of arguments to the
%    helper function is the fact it is called in the following way.
%    This means that before |\bbl@pop@lang| is executed \TeX\ first
%    \emph{expands} the stack, stored in |\bbl@language@stack|. The
%    result of that is that the argument string of |\bbl@pop@lang|
%    contains one or more language names, each followed by a `+'-sign
%    (zero language names won't occur as this macro will only be
%    called after something has been pushed on the stack) followed by
%    the `-'-sign and finally the reference to the stack.
%
%    \begin{macrocode}
\let\bbl@ifrestoring\@secondoftwo
\def\bbl@pop@language{%
  \expandafter\bbl@pop@lang\bbl@language@stack-\bbl@language@stack
  \let\bbl@ifrestoring\@firstoftwo
  \expandafter\bbl@set@language\expandafter{\languagename}%
  \let\bbl@ifrestoring\@secondoftwo}
%    \end{macrocode}
%
%    Once the name of the previous language is retrieved from the stack,
%    it is fed to |\bbl@set@language| to do the actual work of
%    switching everything that needs switching.
%  \end{macro}
%  \end{macro}
%  \end{macro}
%
%    \begin{macrocode}
\expandafter\def\csname selectlanguage \endcsname#1{%
  \ifnum\bbl@hymapsel=\@cclv\let\bbl@hymapsel\tw@\fi
  \bbl@push@language
  \aftergroup\bbl@pop@language
  \bbl@set@language{#1}}
%    \end{macrocode}
%
%  \end{macro}
%
%  \begin{macro}{\bbl@set@language}
% 
%    The macro |\bbl@set@language| takes care of switching the
%    language environment \emph{and} of writing entries on the
%    auxiliary files.  For historial reasons, language names can be
%    either |language| of |\language|. To catch either form a trick is
%    used, but unfortunately as a side effect the catcodes of letters
%    in |\languagename| are not well defined. The list of auxiliary
%    files can be extended by redefining |\BabelContentsFiles|, but
%    make sure they are loaded inside a group (as |aux|, |toc|,
%    |lof|, and |lot| do) or the last language of the document will
%    remain active afterwards.
%
%    We also write a command to change the current language in the
%    auxiliary files.
%
% \changes{babel~3.9a}{2012/09/09}{Added hook}
% \changes{babel~3.9a}{2012/11/07}{Use a loop for contents files, with
%    the help of \cs{BabelContentsFiles}}
% \changes{babel~3.9a}{2013/03/08}{Don't write to aux if language is
%    unknown}
% \changes{babel~3.9h}{2013/11/20}{Error with a more helpful text }
%
%    \begin{macrocode}
\def\BabelContentsFiles{toc,lof,lot}
\def\bbl@set@language#1{%
  \edef\languagename{%
    \ifnum\escapechar=\expandafter`\string#1\@empty
    \else\string#1\@empty\fi}%
  \select@language{\languagename}%
  \expandafter\ifx\csname date\languagename\endcsname\relax\else
    \if@filesw
      \protected@write\@auxout{}{\string\select@language{\languagename}}%
      \bbl@foreach\BabelContentsFiles{%
        \addtocontents{##1}{\xstring\select@language{\languagename}}}%
      \bbl@usehooks{write}{}%
    \fi
  \fi}
\def\select@language#1{%
  \ifnum\bbl@hymapsel=\@cclv\chardef\bbl@hymapsel4\relax\fi
  \edef\languagename{#1}%
  \bbl@fixname\languagename
  \bbl@iflanguage\languagename{%
    \expandafter\ifx\csname date\languagename\endcsname\relax
      \bbl@error
        {Unknown language `#1'. Either you have\\%
         misspelled its name, it has not been installed,\\%
         or you requested it in a previous run. Fix its name,\\%
         install it or just rerun the file, respectively}%
        {You may proceed, but expect wrong results}%
    \else
      \let\bbl@select@type\z@
      \expandafter\bbl@switch\expandafter{\languagename}%
    \fi}}
%    \end{macrocode}
%    
% A bit of optimization. Select in heads/foots the language only if
% necessary. The real thing is in \texttt{babel.def}.
%
%    \begin{macrocode}
\let\select@language@x\select@language
%    \end{macrocode}
%
%  \end{macro}
%
%    First, check if the user asks for a known language. If so,
%    update the value of |\language| and call |\originalTeX|
%    to bring \TeX\ in a certain pre-defined state.
%
% \changes{babel~3.8l}{2008/07/06}{Use \cs{bbl@patterns}}
% \changes{babel~3.9a}{2012/07/27}{Moved \cs{bbl@patterns} to the
%    correct place, after setting the extras for the current
%    language}
% \changes{babel~3.9a}{2012/08/01}{Created \cs{bbl@switch} with code
%    shared by \cs{select@language} and \cs{foreing@language}}
%
%    The name of the language is stored in the control sequence
%    |\languagename|.
%
%    Then we have to \emph{re}define |\originalTeX| to compensate for
%    the things that have been activated.  To save memory space for
%    the macro definition of |\originalTeX|, we construct the control
%    sequence name for the |\noextras|\langvar\ command at definition
%    time by expanding the |\csname| primitive.
%
%    Now activate the language-specific definitions. This is done by
%    constructing the names of three macros by concatenating three
%    words with the argument of |\selectlanguage|, and calling these
%    macros. \nb{What if \cs{hyphenation} was used in |extras|? Patch
%    temporarily |\set@hyphenmins| and hyphenation. It can be done in
%    hooks if necessary.}
%
%
%    The switching of the values of |\lefthyphenmin| and
%    |\righthyphenmin| is somewhat different. First we save their
%    current values, then we check if |\|\langvar|hyphenmins| is
%    defined. If it is not, we set default values (2 and 3), otherwise
%    the values in |\|\langvar|hyphenmins| will be used.
%
% \changes{babel~3.9a}{2012/08/01}{Adddd \cs{bbl@iflanguagename} and
%   \cs{select@language@x}, which is no-op if the language is the same}
% \changes{babel~3.9a}{2013/01/23}{\cs{select@language} sets 
%   \textsc{languagename} so that it has the correct value in the aux
%   file (eg, shorthand expansion was wrong)}
% \changes{babel~3.9a}{2012/08/14}{Make sure the save counter is reset
%    even if \cs{originalTeX} is used in other contexts}
% \changes{babel~3.9c}{2013/04/08}{Removed an extra empty line}
% \changes{babel~3.9h}{2013/11/29}{Use \cs{def} instead of
%    \cs{renewcommand} for \cs{BabelLower}}
% \changes{babel~3.9i}{2014/03/04}{Added `afterreset' hook}
% \changes{3.15}{2017/10/30}{Remove spaces inside captions and date.}
%
%    \begin{macrocode}
\def\bbl@switch#1{%
  \originalTeX
  \expandafter\def\expandafter\originalTeX\expandafter{%
    \csname noextras#1\endcsname
    \let\originalTeX\@empty
    \babel@beginsave}%
  \bbl@usehooks{afterreset}{}%
  \languageshorthands{none}%
  \ifcase\bbl@select@type
    \ifhmode
      \hskip\z@skip % trick to ignore spaces
      \csname captions#1\endcsname\relax
      \csname date#1\endcsname\relax
      \loop\ifdim\lastskip>\z@\unskip\repeat\unskip
    \else
      \csname captions#1\endcsname\relax
      \csname date#1\endcsname\relax
    \fi
  \fi
  \bbl@usehooks{beforeextras}{}%
  \csname extras#1\endcsname\relax
  \bbl@usehooks{afterextras}{}%
  \ifcase\bbl@opt@hyphenmap\or
    \def\BabelLower##1##2{\lccode##1=##2\relax}%
    \ifnum\bbl@hymapsel>4\else
      \csname\languagename @bbl@hyphenmap\endcsname
    \fi
    \chardef\bbl@opt@hyphenmap\z@
  \else
    \ifnum\bbl@hymapsel>\bbl@opt@hyphenmap\else
      \csname\languagename @bbl@hyphenmap\endcsname
    \fi
  \fi
  \global\let\bbl@hymapsel\@cclv
  \bbl@patterns{#1}%
  \babel@savevariable\lefthyphenmin
  \babel@savevariable\righthyphenmin
  \expandafter\ifx\csname #1hyphenmins\endcsname\relax
    \set@hyphenmins\tw@\thr@@\relax
  \else
    \expandafter\expandafter\expandafter\set@hyphenmins
      \csname #1hyphenmins\endcsname\relax
  \fi}
%    \end{macrocode}
%
%  \begin{environment}{otherlanguage}
%    The \Lenv{otherlanguage} environment can be used as an
%    alternative to using the |\selectlanguage| declarative
%    command. When you are typesetting a document which mixes
%    left-to-right and right-to-left typesetting you have to use this
%    environment in order to let things work as you expect them to.
%
%    The |\ignorespaces| command is necessary to hide the environment
%    when it is entered in horizontal mode.
%
% \changes{babel~3.9a}{2012/07/31}{Removed \cs{originalTeX}}
%
%    \begin{macrocode}
\long\def\otherlanguage#1{%
  \ifnum\bbl@hymapsel=\@cclv\let\bbl@hymapsel\thr@@\fi
  \csname selectlanguage \endcsname{#1}%
  \ignorespaces}
%    \end{macrocode}
%
%    The |\endotherlanguage| part of the environment tries to hide
%    itself when it is called in horizontal mode.
%
%    \begin{macrocode}
\long\def\endotherlanguage{%
  \global\@ignoretrue\ignorespaces}
%    \end{macrocode}
%
%  \end{environment}
%
%  \begin{environment}{otherlanguage*}
%    The \Lenv{otherlanguage} environment is meant to be used when a
%    large part of text from a different language needs to be typeset,
%    but without changing the translation of words such as `figure'.
%    This environment makes use of |\foreign@language|.
%
%    \begin{macrocode}
\expandafter\def\csname otherlanguage*\endcsname#1{%
  \ifnum\bbl@hymapsel=\@cclv\chardef\bbl@hymapsel4\relax\fi
  \foreign@language{#1}}
%    \end{macrocode}
%
%    At the end of the environment we need to switch off the extra
%    definitions. The grouping mechanism of the environment will take
%    care of resetting the correct hyphenation rules and ``extras''.
%
%    \begin{macrocode} 
\expandafter\let\csname endotherlanguage*\endcsname\relax
%    \end{macrocode}
%
%  \end{environment}
%
%  \begin{macro}{\foreignlanguage}
%    The |\foreignlanguage| command is another substitute for the
%    |\selectlanguage| command. This command takes two arguments, the
%    first argument is the name of the language to use for typesetting
%    the text specified in the second argument.
%
%    Unlike |\selectlanguage| this command doesn't switch
%    \emph{everything}, it only switches the hyphenation rules and the
%    extra definitions for the language specified. It does this within
%    a group and assumes the |\extras|\langvar\ command doesn't make
%    any |\global| changes. The coding is very similar to part of
%    |\selectlanguage|.
%
%    |\bbl@beforeforeign| is a trick to fix a bug in bidi
%    texts. |\foreignlanguage| is supposed to be a `text' command, and
%    therefore it must emit a |\leavevmode|, but it does not, and
%    therefore the indent is placed on the opposite margin. For
%    backward compatibility, however, it is done only if a
%    right-to-left script is requested; otherwise, it is no-op.
%
%    (3.11) |\foreignlanguage*| is a temporary, experimental macro for
%    a few lines with a different script direction, while preserving
%    the paragraph format (thank the braces around |\par|, things like
%    |\hangindent| are not reset). Do not use it in production,
%    because its semantics and its syntax may change (and very likely
%    will, or even it could be removed altogether). Currently it
%    enters in vmode and then selects the language (which in turn sets the
%    paragraph direction).
%
%    (3.11) Also experimental are the hook |foreign| and
%    |foreign*|. With them you can redefine |\BabelText| which by
%    default does nothing. Its
%    behaviour is not well defined yet. So, use it in
%    horizontal mode only if you do not want surprises.
%
%    In other words, at the beginning of a paragraph |\foreignlanguage|
%    enters into hmode with the surrounding lang, and with
%    |\foreignlanguage*| with the new lang.
%
% \changes{babel~3.9a}{2012/07/30}{Removed unnecesary \cs{noextras}
%    just before closing the group}
% \changes{babel~3.9a}{2012/07/31}{Moved \cs{originalTeX} to
%    \cs{foreing@language} so that it's also used in
%    \texttt{otherlanguage*}}
% \changes{babel~3.9a}{2012/12/24}{\cs{foreignlanguage} defined
%    similarly to \cs{selectlanguage}, protecting the whole macro}
% \changes{babel~3.11}{2017/03/04}{\cs{foreignlanguage*},
%    \cs{bbl@beforeforeign} and hooks}
%
%    \begin{macrocode}
\let\bbl@beforeforeign\@empty
\edef\foreignlanguage{%
  \noexpand\protect
  \expandafter\noexpand\csname foreignlanguage \endcsname}
\expandafter\def\csname foreignlanguage \endcsname{%
  \@ifstar\bbl@foreign@s\bbl@foreign@x}
\def\bbl@foreign@x#1#2{%
  \begingroup
    \let\BabelText\@firstofone
    \bbl@beforeforeign
    \foreign@language{#1}%
    \bbl@usehooks{foreign}{}%
    \BabelText{#2}% Now in horizontal mode!
  \endgroup}
\def\bbl@foreign@s#1#2{% TODO - \shapemode, \@setpar, ?\@@par
  \begingroup
    {\par}%
    \let\BabelText\@firstofone
    \foreign@language{#1}%
    \bbl@usehooks{foreign*}{}%
    \bbl@dirparastext
    \BabelText{#2}% Still in vertical mode!
    {\par}%
  \endgroup}
%    \end{macrocode}
%
%  \end{macro}
%
%  \begin{macro}{\foreign@language}
%    This macro does the
%    work for |\foreignlanguage| and the \Lenv{otherlanguage*}
%    environment. First we need to store the name of the language and
%    check that it is a known language. Then it just calls
%    |bbl@switch|.
%
% \changes{babel~3.9h}{2013/11/29}{The warning shows the language
%    actually selected (with fixed case)}
%
%    \begin{macrocode}
\def\foreign@language#1{%
  \edef\languagename{#1}%
  \bbl@fixname\languagename
  \bbl@iflanguage\languagename{%
    \expandafter\ifx\csname date\languagename\endcsname\relax
      \bbl@warning
        {Unknown language `#1'. Either you have\\%
         misspelled its name, it has not been installed,\\%
         or you requested it in a previous run. Fix its name,\\%
         install it or just rerun the file, respectively.\\%
         I'll proceed, but expect wrong results.\\%
         Reported}%
    \fi
    \let\bbl@select@type\@ne
    \expandafter\bbl@switch\expandafter{\languagename}}}
%    \end{macrocode}
%
%  \end{macro}
%
%  \begin{macro}{\bbl@patterns}
%
% \changes{babel~3.8l}{2008/07/06}{Macro added}
% \changes{babel~3.9a}{2012/08/28}{Extended to set hyphenation
%    exceptions as defined with \cs{babelhyphenation}}
% \changes{babel~3.9m}{2015/07/25}{Preset \cs{bbl@pttnlist} and
%    \cs{bbl@patterns@} to relax, for luatex.}
%
%    This macro selects the hyphenation patterns by changing the
%    \cs{language} register.  If special hyphenation patterns
%    are available specifically for the current font encoding,
%    use them instead of the default.
%
%    It also sets hyphenation exceptions, but only once, because they
%    are global (here language |\lccode|'s has been set,
%    too). |\bbl@hyphenation@| is set to relax until the very first
%    |\babelhyphenation|, so do nothing with this value. If the
%    exceptions for a language (by its number, not its name, so that
%    |:ENC| is taken into account) has been set, then use
%    |\hyphenation| with both global and language exceptions and empty
%    the latter to mark they must not be set again.
%
%    \begin{macrocode}
\let\bbl@hyphlist\@empty
\let\bbl@hyphenation@\relax
\let\bbl@pttnlist\@empty
\let\bbl@patterns@\relax
\let\bbl@hymapsel=\@cclv
\def\bbl@patterns#1{%
  \language=\expandafter\ifx\csname l@#1:\f@encoding\endcsname\relax
      \csname l@#1\endcsname
      \edef\bbl@tempa{#1}%
    \else
      \csname l@#1:\f@encoding\endcsname
      \edef\bbl@tempa{#1:\f@encoding}%
    \fi
  \@expandtwoargs\bbl@usehooks{patterns}{{#1}{\bbl@tempa}}%
  \@ifundefined{bbl@hyphenation@}{}{% Can be \relax!
    \begingroup
      \bbl@xin@{,\number\language,}{,\bbl@hyphlist}%
      \ifin@\else
        \@expandtwoargs\bbl@usehooks{hyphenation}{{#1}{\bbl@tempa}}%
        \hyphenation{%
          \bbl@hyphenation@
          \@ifundefined{bbl@hyphenation@#1}%
            \@empty
            {\space\csname bbl@hyphenation@#1\endcsname}}%
        \xdef\bbl@hyphlist{\bbl@hyphlist\number\language,}%
      \fi
    \endgroup}}
%    \end{macrocode}
%
%  \end{macro}
%
%  \begin{environment}{hyphenrules}
%    The environment \Lenv{hyphenrules} can be used to select
%    \emph{just} the hyphenation rules. This environment does
%    \emph{not} change |\languagename| and when the hyphenation rules
%    specified were not loaded it has no effect. Note however,
%    |\lccode|'s and font encodings are not set at all, so in most
%    cases you should use |otherlanguage*|.
%
% \changes{babel~3.8j}{2008/03/16}{Also set the hyphenmin parameters to
%    the correct value (PR3997)} 
% \changes{babel~3.8l}{2008/07/06}{Use \cs{bbl@patterns}}
% \changes{3.15}{2017/10/30}{Don't set language name. Use temp macro.}
%
%    \begin{macrocode}
\def\hyphenrules#1{%
  \edef\bbl@tempf{#1}%
  \bbl@fixname\bbl@tempf
  \bbl@iflanguage\bbl@tempf{%
    \expandafter\bbl@patterns\expandafter{\bbl@tempf}%
    \languageshorthands{none}%
    \bbl@ifunset{\bbl@tempf hyphenmins}%
      {\set@hyphenmins\tw@\thr@@\relax}%
      {\bbl@exp{\\\set@hyphenmins\@nameuse{\bbl@tempf hyphenmins}}}}}
\let\endhyphenrules\@empty
%    \end{macrocode}
%
%  \end{environment}
%
%  \begin{macro}{\providehyphenmins}
%    The macro |\providehyphenmins| should be used in the language
%    definition files to provide a \emph{default} setting for the
%    hyphenation parameters |\lefthyphenmin| and |\righthyphenmin|. If
%    the macro |\|\langvar|hyphenmins| is already defined this command
%    has no effect.
%
%    \begin{macrocode}
\def\providehyphenmins#1#2{%
  \expandafter\ifx\csname #1hyphenmins\endcsname\relax
    \@namedef{#1hyphenmins}{#2}%
  \fi}
%    \end{macrocode}
%
%  \end{macro}
%
%  \begin{macro}{\set@hyphenmins}
%    This macro sets the values of |\lefthyphenmin| and
%    |\righthyphenmin|. It expects two values as its argument.
%
%    \begin{macrocode}
\def\set@hyphenmins#1#2{%
  \lefthyphenmin#1\relax
  \righthyphenmin#2\relax}
%    \end{macrocode}
%
%  \end{macro}
%
%  \begin{macro}{\ProvidesLanguage}
%    The identification code for each file is something that was
%    introduced in \LaTeXe. When the command |\ProvidesFile| does not
%    exist, a dummy definition is provided temporarily. For use in the
%    language definition file the command |\ProvidesLanguage| is
%    defined by \babel.
%
%    Depending on the format, ie, on if the former is defined, we use
%    a similar definition or not.
%
%    \changes{babel~3.9a}{2012/12/09}{Save info about the babel
%    version in the format (switch.def) so that it can be checked
%    later if necessary}
%
%    \begin{macrocode}
\ifx\ProvidesFile\@undefined
  \def\ProvidesLanguage#1[#2 #3 #4]{%
    \wlog{Language: #1 #4 #3 <#2>}%
    }
\else
  \def\ProvidesLanguage#1{%
    \begingroup
      \catcode`\ 10 %
      \@makeother\/%
      \@ifnextchar[%]
        {\@provideslanguage{#1}}{\@provideslanguage{#1}[]}}
  \def\@provideslanguage#1[#2]{%
    \wlog{Language: #1 #2}%
    \expandafter\xdef\csname ver@#1.ldf\endcsname{#2}%
    \endgroup}
\fi
%    \end{macrocode}
%
%  \end{macro}
%
%  \begin{macro}{\LdfInit}
%    This macro is defined in two versions. The first version is to be
%    part of the `kernel' of \babel, ie. the part that is loaded in
%    the format; the second version is defined in \file{babel.def}.
%    The version in the format just checks the category code of the
%    ampersand and then loads \file{babel.def}.
%
%    The category code of the ampersand is restored and the macro
%    calls itself again with the new definition from
%    \file{babel.def}
%
%    \begin{macrocode}
\def\LdfInit{%
  \chardef\atcatcode=\catcode`\@
  \catcode`\@=11\relax
  \input babel.def\relax
  \catcode`\@=\atcatcode \let\atcatcode\relax
  \LdfInit}
%    \end{macrocode}
%
%  \end{macro}
%
%  \begin{macro}{\originalTeX}
%    The macro|\originalTeX| should be known to \TeX\ at this moment.
%    As it has to be expandable we |\let| it to |\@empty| instead of
%    |\relax|.
%
%    \begin{macrocode}
\ifx\originalTeX\@undefined\let\originalTeX\@empty\fi
%    \end{macrocode}
%
%    Because this part of the code can be included in a format, we
%    make sure that the macro which initialises the save mechanism,
%    |\babel@beginsave|, is not considered to be undefined.
%
%    \begin{macrocode}
\ifx\babel@beginsave\@undefined\let\babel@beginsave\relax\fi
%    \end{macrocode}
%
%  \end{macro}
%
% A few macro names are reserved for future releases of \babel, which
% will use the concept of `locale':
%
% \changes{babel~3.9s}{2017/04/13}{Reserved macro names for `locale'}
%
%    \begin{macrocode}
\providecommand\setlocale{%
  \bbl@error
    {Not yet available}%
    {Find an armchair, sit down and wait}}
\let\uselocale\setlocale
\let\locale\setlocale
\let\selectlocale\setlocale
\let\textlocale\setlocale
\let\textlanguage\setlocale
\let\languagetext\setlocale
%    \end{macrocode}
%
%    \subsection{Errors}
%
%  \begin{macro}{\@nolanerr}
%  \begin{macro}{\@nopatterns}
%    The \babel\ package will signal an error when a documents tries
%    to select a language that hasn't been defined earlier. When a
%    user selects a language for which no hyphenation patterns were
%    loaded into the format he will be given a warning about that
%    fact. We revert to the patterns for |\language|=0 in that case.
%    In most formats that will be (US)english, but it might also be
%    empty.
%  \begin{macro}{\@noopterr}
%    When the package was loaded without options not everything will
%    work as expected. An error message is issued in that case.
%
%    When the format knows about |\PackageError| it must be \LaTeXe,
%    so we can safely use its error handling interface. Otherwise
%    we'll have to `keep it simple'.
%
% \changes{babel~3.9a}{2012/07/30}{\cs{newcommand}s replaced by
%    \cs{def}'s, so that the file can be loaded twice}
% \changes{babel~3.9a}{2013/01/26}{Define generic variants instead of
%    duplicating each predefined message}    
%
%    \begin{macrocode}
\edef\bbl@nulllanguage{\string\language=0}
\ifx\PackageError\@undefined
  \def\bbl@error#1#2{%
    \begingroup
      \newlinechar=`\^^J
      \def\\{^^J(babel) }%
      \errhelp{#2}\errmessage{\\#1}%
    \endgroup}
  \def\bbl@warning#1{%
    \begingroup
      \newlinechar=`\^^J
      \def\\{^^J(babel) }%
      \message{\\#1}%
    \endgroup}
  \def\bbl@info#1{%
    \begingroup
      \newlinechar=`\^^J
      \def\\{^^J}%
      \wlog{#1}%
    \endgroup}
\else
  \def\bbl@error#1#2{%
    \begingroup
      \def\\{\MessageBreak}%
      \PackageError{babel}{#1}{#2}%
    \endgroup}
  \def\bbl@warning#1{%
    \begingroup
      \def\\{\MessageBreak}%
      \PackageWarning{babel}{#1}%
    \endgroup}
  \def\bbl@info#1{%
    \begingroup
      \def\\{\MessageBreak}%
      \PackageInfo{babel}{#1}%
    \endgroup}
\fi
\@ifpackagewith{babel}{silent}
  {\let\bbl@info\@gobble
   \let\bbl@warning\@gobble}
  {}
\def\bbl@nocaption#1#2{% 1: text to be printed 2: caption macro \langXname
  \gdef#2{\textbf{?#1?}}%
  #2%
  \bbl@warning{%
    \string#2 not set. Please, define\\%
    it in the preamble with something like:\\%
    \string\renewcommand\string#2{..}\\%
    Reported}}
\def\@nolanerr#1{%
  \bbl@error
    {You haven't defined the language #1\space yet}%
    {Your command will be ignored, type <return> to proceed}}
\def\@nopatterns#1{%
  \bbl@warning
    {No hyphenation patterns were preloaded for\\%
     the language `#1' into the format.\\%
     Please, configure your TeX system to add them and\\%
     rebuild the format. Now I will use the patterns\\%
     preloaded for \bbl@nulllanguage\space instead}}
\let\bbl@usehooks\@gobbletwo
%</kernel>
%    \end{macrocode}
%
%  \end{macro}
%  \end{macro}
%  \end{macro}
%
%    \section{Loading hyphenation patterns}
%
%  The following code is meant to be read by ini\TeX\ because it
%  should instruct \TeX\ to read hyphenation patterns. To this end the
%  \texttt{docstrip} option \texttt{patterns} can be used to include
%  this code in the file \file{hyphen.cfg}. Code is written with lower
%  level macros.
%
%    |toks8| stores info to be shown when the program is run.
%
% \changes{babel~3.9g}{2013/05/30}{Code moved from plain.def}
%
%    We want to add a message to the message \LaTeX$\:$2.09 puts in
%    the |\everyjob| register. This could be done by the following
%    code: 
%\begin{verbatim}
%    \let\orgeveryjob\everyjob
%    \def\everyjob#1{%
%      \orgeveryjob{#1}%
%      \orgeveryjob\expandafter{\the\orgeveryjob\immediate\write16{%
%          hyphenation patterns for \the\loaded@patterns loaded.}}%
%      \let\everyjob\orgeveryjob\let\orgeveryjob\@undefined}
%\end{verbatim}
%    The code above redefines the control sequence \cs{everyjob}
%    in order to be able to add something to the current contents of
%    the register. This is necessary because the processing of
%    hyphenation patterns happens long before \LaTeX\ fills the
%    register.
%
%    There are some problems with this approach though.
%  \begin{itemize}
%    \item When someone wants to use several hyphenation patterns with
%    \SliTeX\ the above scheme won't work. The reason is that \SliTeX\
%    overwrites the contents of the |\everyjob| register with its own
%    message.
%    \item Plain \TeX\ does not use the |\everyjob| register so the
%    message would not be displayed.
%  \end{itemize}
%    To circumvent this a `dirty trick' can be used. As this code is
%    only processed when creating a new format file there is one
%    command that is sure to be used, |\dump|. Therefore the original
%    |\dump| is saved in |\org@dump| and a new definition is supplied.
%
%    To make sure that \LaTeX$\:$2.09 executes the
%    |\@begindocumenthook| we would want to alter |\begin{document}|,
%    but as this done too often already, we add the new code at the
%    front of |\@preamblecmds|. But we can only do that after it has
%    been defined, so we add this piece of code to |\dump|.
%
%    This new definition starts by adding an instruction to write a
%    message on the terminal and in the transcript file to inform the
%    user of the preloaded hyphenation patterns.
%
%    Then everything is restored to the old situation and the format
%    is dumped.
%   
%    \begin{macrocode}
%<*patterns>
<@Make sure ProvidesFile is defined@>
\ProvidesFile{hyphen.cfg}[<@date@> <@version@> Babel hyphens]
\xdef\bbl@format{\jobname}
\ifx\AtBeginDocument\@undefined
  \def\@empty{}
  \let\orig@dump\dump
  \def\dump{%
    \ifx\@ztryfc\@undefined
    \else
      \toks0=\expandafter{\@preamblecmds}%
      \edef\@preamblecmds{\noexpand\@begindocumenthook\the\toks0}%
      \def\@begindocumenthook{}%
    \fi
    \let\dump\orig@dump\let\orig@dump\@undefined\dump}
\fi
<@Define core switching macros@>
\toks8{Babel <<@version@>> and hyphenation patterns for }%
%    \end{macrocode}
%
%
%  \begin{macro}{\process@line}
%    Each line in the file \file{language.dat} is processed by
%    |\process@line| after it is read. The first thing this macro does
%    is to check whether the line starts with \texttt{=}.
%    When the first token of a line is an \texttt{=}, the macro
%    |\process@synonym| is called; otherwise the macro
%    |\process@language| will continue.
%
% \changes{babel~3.9a}{2012/12/12}{Use spaces as delimiters, to avoid
%    extra spaces. Once parsed, pass them in the traditional way}
%
%    \begin{macrocode}   
\def\process@line#1#2 #3 #4 {%
  \ifx=#1%
    \process@synonym{#2}%
  \else
    \process@language{#1#2}{#3}{#4}%
  \fi
  \ignorespaces}
%    \end{macrocode}
%
%  \end{macro}
%
%  \begin{macro}{\process@synonym}
%
%    This macro takes care of the lines which start with an
%    \texttt{=}. It needs an empty token register to begin with.
%    |\bbl@languages| is also set to empty. 
%
%    \begin{macrocode}
\toks@{}
\def\bbl@languages{}
%    \end{macrocode}
%
%    When no languages have been loaded yet, the name following the
%    \texttt{=} will be a synonym for hyphenation register 0. So, it is stored
%    in a token register and executed when the first pattern file has
%    been processed. (The |\relax| just helps to the |\if| below
%    catching synonyms without a language.)
%
%    Otherwise the name will be a synonym for the language loaded last.
%
%    We also need to copy the hyphenmin parameters for the synonym.
%
% \changes{babel~3.9a}{2012/06/25}{Added \cs{bbl@languages}}
%
%    \begin{macrocode}
\def\process@synonym#1{%
  \ifnum\last@language=\m@ne
    \toks@\expandafter{\the\toks@\relax\process@synonym{#1}}%
  \else
    \expandafter\chardef\csname l@#1\endcsname\last@language
    \wlog{\string\l@#1=\string\language\the\last@language}%
    \expandafter\let\csname #1hyphenmins\expandafter\endcsname
      \csname\languagename hyphenmins\endcsname
    \let\bbl@elt\relax
    \edef\bbl@languages{\bbl@languages\bbl@elt{#1}{\the\last@language}{}{}}%
  \fi}
%    \end{macrocode}
%
%  \end{macro}
%
%  \begin{macro}{\process@language}
%    The macro |\process@language| is used to process a non-empty line
%    from the `configuration file'. It has three arguments, each
%    delimited by white space. The first argument is the `name' of a
%    language; the second is the name of the file that contains the
%    patterns. The optional third argument is the name of a file
%    containing hyphenation exceptions.
%
%    The first thing to do is call |\addlanguage| to allocate a
%    pattern register and to make that register `active'.
%    Then the `name' of the language that will be loaded now is
%    added to the token register |\toks8|. and finally
%    the pattern file is read.
%
% \changes{babel~3.9a}{2012/12/10}{Removed \cs{selectfont} (I presume
%    it was intended to catch wrong encoding codes, but I don't think
%    this is necessary and as a side effect it might preload fonts)}
%
%    For some hyphenation patterns it is needed to load them with a
%    specific font encoding selected. This can be specified in the
%    file \file{language.dat} by adding for instance `\texttt{:T1}' to
%    the name of the language. The macro |\bbl@get@enc| extracts the
%    font encoding from the language name and stores it in
%    |\bbl@hyph@enc|. The latter can be used in hyphenation files if
%    you need to set a behaviour depending on the given encoding (it
%    is set to empty if no encoding is given).
%
%    Pattern files may contain assignments to |\lefthyphenmin| and
%    |\righthyphenmin|. \TeX\ does not keep track of these
%    assignments. Therefore we try to detect such assignments and
%    store them in the |\|\langvar|hyphenmins| macro. When no
%    assignments were made we provide a default setting.
%
%    Some pattern files contain changes to the |\lccode| en |\uccode|
%    arrays. Such changes should remain local to the language;
%    therefore we process the pattern file in a group; the |\patterns|
%    command acts globally so its effect will be remembered.
%
%    Then we globally store the settings of |\lefthyphenmin| and
%    |\righthyphenmin| and close the group.
%
%    When the hyphenation patterns have been processed we need to see
%    if a file with hyphenation exceptions needs to be read. This is
%    the case when the third argument is not empty and when it does
%    not contain a space token. (Note however there is no need to save
%    hyphenation exceptions into the format.)
%
% \changes{babel~3.9a}{2012/06/25}{Added \cs{bbl@languages}}
% \changes{babel~3.9f}{2013/05/16}{Restored code to set default
%    hyphenmins, which was deleted mistakenly}
%
%    \cs{bbl@languages} saves a snapshot of the loaded languagues in the
%    form  \cs{bbl@elt}\marg{language-name}\marg{number}%
%    \marg{patterns-file}\marg{exceptions-file}. Note the last 2
%    arguments are empty in `dialects' defined in |language.dat| with
%    |=|. Note also the language name can have encoding info.
%
%    Finally, if the counter |\language| is equal to zero we execute the
%    synonyms stored.
%
%    \begin{macrocode}
\def\process@language#1#2#3{%
  \expandafter\addlanguage\csname l@#1\endcsname
  \expandafter\language\csname l@#1\endcsname
  \edef\languagename{#1}%
  \bbl@hook@everylanguage{#1}%
  \bbl@get@enc#1::\@@@
  \begingroup
    \lefthyphenmin\m@ne
    \bbl@hook@loadpatterns{#2}%
    \ifnum\lefthyphenmin=\m@ne
    \else
      \expandafter\xdef\csname #1hyphenmins\endcsname{%
        \the\lefthyphenmin\the\righthyphenmin}%
    \fi
  \endgroup
  \def\bbl@tempa{#3}%
  \ifx\bbl@tempa\@empty\else
    \bbl@hook@loadexceptions{#3}%
  \fi
  \let\bbl@elt\relax
  \edef\bbl@languages{%
    \bbl@languages\bbl@elt{#1}{\the\language}{#2}{\bbl@tempa}}%
  \ifnum\the\language=\z@
    \expandafter\ifx\csname #1hyphenmins\endcsname\relax
      \set@hyphenmins\tw@\thr@@\relax
    \else
      \expandafter\expandafter\expandafter\set@hyphenmins
        \csname #1hyphenmins\endcsname
    \fi
    \the\toks@
    \toks@{}%
  \fi}
%    \end{macrocode}
%
%  \begin{macro}{\bbl@get@enc}
%
%   \changes{babel~3.9a}{2012/12/11}{Code much simplified}
%
%  \begin{macro}{\bbl@hyph@enc}
%    The macro |\bbl@get@enc| extracts the font encoding from the
%    language name and stores it in |\bbl@hyph@enc|. It uses delimited
%    arguments to achieve this.
%
%    \begin{macrocode}
\def\bbl@get@enc#1:#2:#3\@@@{\def\bbl@hyph@enc{#2}}
%    \end{macrocode}
%
%  \end{macro}
%  \end{macro}
%  \end{macro}
%
%    Now, hooks are defined. For efficiency reasons, they are dealt
%    here in a special way. Besides \luatex, format specific
%    configuration files are taken into account.
%
% \changes{babel~3.9b}{2013/03/25}{Fixed an idiot slip: \cs{def}
%   intead of \cs{let}}
%
%    \begin{macrocode}
\def\bbl@hook@everylanguage#1{}
\def\bbl@hook@loadpatterns#1{\input #1\relax}
\let\bbl@hook@loadexceptions\bbl@hook@loadpatterns
\let\bbl@hook@loadkernel\bbl@hook@loadpatterns
\begingroup
  \def\AddBabelHook#1#2{%
    \expandafter\ifx\csname bbl@hook@#2\endcsname\relax
      \def\next{\toks1}%
    \else
      \def\next{\expandafter\gdef\csname bbl@hook@#2\endcsname####1}%
    \fi
    \next}
  \ifx\directlua\@undefined
    \ifx\XeTeXinputencoding\@undefined\else
      \input xebabel.def
    \fi
  \else
    \input luababel.def
  \fi
  \openin1 = babel-\bbl@format.cfg
  \ifeof1
  \else
    \input babel-\bbl@format.cfg\relax
  \fi
  \closein1
\endgroup
\bbl@hook@loadkernel{switch.def}
%    \end{macrocode}
%
%  \begin{macro}{\readconfigfile}
%    The configuration file can now be opened for reading.
%
%    \begin{macrocode}
\openin1 = language.dat
%    \end{macrocode}
%
%    See if the file exists, if not, use the default hyphenation file
%    \file{hyphen.tex}. The user will be informed about this.
%
%    \begin{macrocode}
\def\languagename{english}%
\ifeof1
  \message{I couldn't find the file language.dat,\space
           I will try the file hyphen.tex}
  \input hyphen.tex\relax
  \chardef\l@english\z@
\else
%    \end{macrocode}
%
%    Pattern registers are allocated using count register
%    |\last@language|. Its initial value is~0. The definition of the
%    macro |\newlanguage| is such that it first increments the count
%    register and then defines the language. In order to have the
%    first patterns loaded in pattern register number~0 we initialize
%    |\last@language| with the value~$-1$.
%
%    \begin{macrocode}
  \last@language\m@ne
%    \end{macrocode}
%
%    We now read lines from the file until the end is found
%
%    \begin{macrocode}
  \loop
%    \end{macrocode}
%
%    While reading from the input, it is useful to switch off
%    recognition of the end-of-line character. This saves us stripping
%    off spaces from the contents of the control sequence.
%
%    \begin{macrocode}
    \endlinechar\m@ne
    \read1 to \bbl@line
    \endlinechar`\^^M
%    \end{macrocode}
%
% \changes{babel~3.9a}{2012/12/14}{Test simplified and moved}
% \changes{babel~3.9a}{2012/12/12}{Use only spaces as delimiters and
%    not /, as previouly done} 
%
%    If the file has reached its end, exit from the loop here. If not,
%    empty lines are skipped. Add 3 space characters to the end of
%    |\bbl@line|. This is needed to be able to recognize the arguments
%    of |\process@line| later on. The default language should be the
%    very first one.
%
%    \begin{macrocode}
    \if T\ifeof1F\fi T\relax
      \ifx\bbl@line\@empty\else
        \edef\bbl@line{\bbl@line\space\space\space}%
        \expandafter\process@line\bbl@line\relax
      \fi
  \repeat
%    \end{macrocode}
%
%    Check for the end of the file. We must reverse the test for
%    |\ifeof| without |\else|. Then reactivate the default patterns.
%
% \changes{babel~3.8m}{2008/07/08}{Also restore the name of the
%    language in \cs{languagename} (PR 4039)} 
%
%    \begin{macrocode}
  \begingroup
    \def\bbl@elt#1#2#3#4{%
      \global\language=#2\relax
      \gdef\languagename{#1}%
      \def\bbl@elt##1##2##3##4{}}%
    \bbl@languages
  \endgroup
\fi
%    \end{macrocode}
%
%    and close the configuration file.
%
%    \begin{macrocode}
\closein1
%    \end{macrocode}
%
%    We add a message about the fact that babel is loaded in the
%    format and with which language patterns to the \cs{everyjob}
%    register.
%
% \changes{babel~3.9a}{2012/09/25}{The list of languages is not
%    printed every job any more (it is saved in \cs{bbl@languages}).} 
% \changes{babel~3.9g}{2013/07/28}{In non-LaTeX formats the number of
%    languages were not printed. Moved from \cs{dump} and cleaned up:
%    now \cs{toks}8 is expanded here.} 
% \changes{babel~3.9o}{2016/01/25}{The number of languages loaded was
%    off by 1.} 
%
%    \begin{macrocode}
\if/\the\toks@/\else
  \errhelp{language.dat loads no language, only synonyms}
  \errmessage{Orphan language synonym}
\fi
\advance\last@language\@ne
\edef\bbl@tempa{%
  \everyjob{%
    \the\everyjob
    \ifx\typeout\@undefined
      \immediate\write16%
    \else
      \noexpand\typeout
    \fi
    {\the\toks8 \the\last@language\space language(s) loaded.}}}
\advance\last@language\m@ne
\bbl@tempa
%    \end{macrocode}
%
%    Also remove some macros from memory and raise an error
%    if |\toks@| is not empty. Finally load \file{switch.def}, but the
%    latter is not required and the line inputting it may be commented
%    out.
%
% \changes{babel~3.9a}{2012/12/11}{Raise error if there are synonyms
%    without languages}
%
%    \begin{macrocode}
\let\bbl@line\@undefined
\let\process@line\@undefined
\let\process@synonym\@undefined
\let\process@language\@undefined
\let\bbl@get@enc\@undefined
\let\bbl@hyph@enc\@undefined
\let\bbl@tempa\@undefined
\let\bbl@hook@loadkernel\@undefined
\let\bbl@hook@everylanguage\@undefined
\let\bbl@hook@loadpatterns\@undefined
\let\bbl@hook@loadexceptions\@undefined
%</patterns>
%    \end{macrocode}
%
%    Here the code for ini\TeX\ ends.
%  \end{macro}
%
% \section{Font handling with fontspec}
%
% \changes{3.15}{2017/10/30}{New way to select fonts, with \cs{babelfont}}
%
% Add the bidi handler just before luaoftload, which is loaded by default
% by LaTeX. Just in case, consider the possibility it has not been
% loaded. First, a couple of definitions related to bidi [misplaced].
%
%    \begin{macrocode}
%<<*More package options>>
\DeclareOption{bidi=basic-r}%
  {\newattribute\bbl@attr@dir
   \let\bbl@beforeforeign\leavevmode
   \AtEndOfPackage{\EnableBabelHook{babel-bidi}}}
\DeclareOption{bidi=default}%
  {\let\bbl@beforeforeign\leavevmode
   \ifodd\bbl@engine
     \newattribute\bbl@attr@dir
   \fi
   \AtEndOfPackage{%
     \EnableBabelHook{babel-bidi}%
     \ifodd\bbl@engine\else
       \bbl@xebidipar
     \fi}}
%<</More package options>>
%    \end{macrocode}
%      
% With explicit languages, we could define the font at once, but we
% don't. Just wait and see if the language is actually activated.
%
%    \begin{macrocode}
%<<*Font selection>> 
\@onlypreamble\babelfont
\newcommand\babelfont[2][]{%  1=langs/scripts 2=fam
  \edef\bbl@tempa{#1}%
  \def\bbl@tempb{#2}%
  \ifx\fontspec\@undefined
    \usepackage{fontspec}%
  \fi
  \EnableBabelHook{babel-fontspec}%
  \bbl@bblfont}
\newcommand\bbl@bblfont[2][]{% 1=features 2=fontname
  \bbl@ifunset{\bbl@tempb family}{\bbl@providefam{\bbl@tempb}}{}%
  \bbl@ifunset{bbl@lsys@\languagename}{\bbl@provide@lsys{\languagename}}{}%
  \expandafter\bbl@ifblank\expandafter{\bbl@tempa}%
    {\bbl@csarg\edef{\bbl@tempb dflt@}{<>{#1}{#2}}% save bbl@rmdflt@
     \bbl@exp{%
       \let\<bbl@\bbl@tempb dflt@\languagename>\<bbl@\bbl@tempb dflt@>%
       \\\bbl@font@set\<bbl@\bbl@tempb dflt@\languagename>%
                      \<\bbl@tempb default>\<\bbl@tempb family>}}%
    {\bbl@foreach\bbl@tempa{% ie bbl@rmdflt@lang / *scrt
       \bbl@csarg\def{\bbl@tempb dflt@##1}{<>{#1}{#2}}}}}% 
%    \end{macrocode}
%
% If the family in the previous command does not exist, it must be
% defined. Here is how:  
%
%    \begin{macrocode}
\def\bbl@providefam#1{%
  \bbl@exp{%
    \\\newcommand\<#1default>{}% Just define it
    \\\bbl@add@list\\\bbl@font@fams{#1}%
    \\\DeclareRobustCommand\<#1family>{%
      \\\not@math@alphabet\<#1family>\relax
      \\\fontfamily\<#1default>\\\selectfont}%
    \\\DeclareTextFontCommand{\<text#1>}{\<#1family>}}}
%    \end{macrocode}
%      
% The following macro is activated when the hook |babel-fontspec| is
% enabled.
%
%    \begin{macrocode}
\def\bbl@switchfont{%
  \bbl@ifunset{bbl@lsys@\languagename}{\bbl@provide@lsys{\languagename}}{}%
  \bbl@exp{%  eg Arabic -> arabic
    \lowercase{\edef\\\bbl@tempa{\bbl@cs{sname@\languagename}}}}%
  \bbl@foreach\bbl@font@fams{%
    \bbl@ifunset{bbl@##1dflt@\languagename}%    (1) language?
      {\bbl@ifunset{bbl@##1dflt@*\bbl@tempa}%   (2) from script?
         {\bbl@ifunset{bbl@##1dflt@}%           2=F - (3) from generic?
           {}%                                  123=F - nothing!
           {\bbl@exp{%                          3=T - from generic
              \global\let\<bbl@##1dflt@\languagename>%
                         \<bbl@##1dflt@>}}}%
         {\bbl@exp{%                            2=T - from script
            \global\let\<bbl@##1dflt@\languagename>% 
                       \<bbl@##1dflt@*\bbl@tempa>}}}%
      {}}%                               1=T - language, already defined
  \def\bbl@tempa{%
    \bbl@warning{The current font is not a standard family.\\%
      Script and Language are not applied. Consider defining\\%
      a new family with \string\babelfont,}}%
  \bbl@foreach\bbl@font@fams{%     don't gather with prev for
    \bbl@ifunset{bbl@##1dflt@\languagename}%
      {\bbl@cs{famrst@##1}%
       \global\bbl@csarg\let{famrst@##1}\relax}%
      {\bbl@exp{% order is relevant
         \\\bbl@add\\\originalTeX{%
           \\\bbl@font@rst{\bbl@cs{##1dflt@\languagename}}%
                          \<##1default>\<##1family>{##1}}%
         \\\bbl@font@set\<bbl@##1dflt@\languagename>% the main part!
                        \<##1default>\<##1family>}}}%
  \bbl@ifrestoring{}{\bbl@tempa}}%
%    \end{macrocode}
%
% Now the macros defining the font with \textsf{fontspec}.
%
% When there are repeated keys in \textsf{fontspec}, the last value
% wins. So, we just place the ini settings at the beginning, and user
% settings will take precedence.
%
%    \begin{macrocode}
\def\bbl@font@set#1#2#3{%
  \bbl@xin@{<>}{#1}%
  \ifin@
    \bbl@exp{\\\bbl@fontspec@set\\#1\expandafter\@gobbletwo#1}%
  \fi
  \bbl@exp{%
    \def\\#2{#1}%        eg, \rmdefault{\bbl@rm1dflt@lang}
    \\\bbl@ifsamestring{#2}{\f@family}{\\#3\let\\\bbl@tempa\relax}{}}}
\def\bbl@fontspec@set#1#2#3{%
  \bbl@exp{\<fontspec_set_family:Nnn>\\#1%
    {\bbl@cs{lsys@\languagename},#2}}{#3}%
  \bbl@toglobal#1}%
%    \end{macrocode}
% 
% font@rst and famrst are only used when there is no global settings,
% to save and restore de previous families. Not really necessary, but
% done for optimization.
%
%    \begin{macrocode}
\def\bbl@font@rst#1#2#3#4{%
  \bbl@csarg\def{famrst@#4}{\bbl@font@set{#1}#2#3}}
%    \end{macrocode}
%
% The default font families. They are eurocentric, but the list can be
% expanded easily with |\babelfont|.
% 
%    \begin{macrocode}
\def\bbl@font@fams{rm,sf,tt}
%    \end{macrocode}
%
% The old tentative way. Short and preverved for compatibility, but
% deprecated. Note there is no direct alternative for
% |\babelFSfeatures|. The reason in explained in the user guide, but
% essentially -- that was not the way to go :-).
%
%    \begin{macrocode}
\newcommand\babelFSstore[2][]{%
  \bbl@ifblank{#1}%
    {\bbl@csarg\def{sname@#2}{Latin}}%
    {\bbl@csarg\def{sname@#2}{#1}}%
  \bbl@provide@dirs{#2}%
  \bbl@csarg\ifnum{wdir@#2}>\z@
    \let\bbl@beforeforeign\leavevmode
    \EnableBabelHook{babel-bidi}%
  \fi
  \bbl@foreach{#2}{%
    \bbl@FSstore{##1}{rm}\rmdefault\bbl@save@rmdefault
    \bbl@FSstore{##1}{sf}\sfdefault\bbl@save@sfdefault
    \bbl@FSstore{##1}{tt}\ttdefault\bbl@save@ttdefault}}
\def\bbl@FSstore#1#2#3#4{%
  \bbl@csarg\edef{#2default#1}{#3}%
  \expandafter\addto\csname extras#1\endcsname{%
    \let#4#3%
    \ifx#3\f@family
      \edef#3{\csname bbl@#2default#1\endcsname}%
      \fontfamily{#3}\selectfont
    \else
      \edef#3{\csname bbl@#2default#1\endcsname}%
    \fi}%
  \expandafter\addto\csname noextras#1\endcsname{%
    \ifx#3\f@family
      \fontfamily{#4}\selectfont
    \fi
    \let#3#4}}
\let\bbl@langfeatures\@empty
\def\babelFSfeatures{% make sure \fontspec is redefined once
  \let\bbl@ori@fontspec\fontspec
  \renewcommand\fontspec[1][]{%
    \bbl@ori@fontspec[\bbl@langfeatures##1]}
  \let\babelFSfeatures\bbl@FSfeatures
  \babelFSfeatures}
\def\bbl@FSfeatures#1#2{%
  \expandafter\addto\csname extras#1\endcsname{%
    \babel@save\bbl@langfeatures
    \edef\bbl@langfeatures{#2,}}}
%<</Font selection>>
%    \end{macrocode}
%      
%
%    \section{Hooks for XeTeX and LuaTeX}
%
%    \subsection{XeTeX}
%
%    Unfortunately, the current encoding cannot be retrieved and
%    therefore it is reset always to |utf8|, which seems a sensible
%    default.
%
%    \LaTeX{} sets many ``codes'' just before loading
%    \verb|hyphen.cfg|. That is not a problem in luatex, but in xetex
%    they must be reset to the proper value. Most of the work is done in
%    \textsf{xe(la)tex.ini}, so here we just ``undo'' some of the
%    changes done by \LaTeX. Anyway, for consistency Lua\TeX{} also
%    resets the catcodes. 
%
% \changes{bbunicode~1.0c}{2014/03/10}{Reset ``codes'' set by \cs{LaTeX}
%    to what xetex expects. Used also in luatex.}
% \changes{bbunicode~1.0f}{2015/12/06}{This block was assigned to
%    xetex, even in luatex. Fixed here and below.}
%
%    \begin{macrocode}
%<<*Restore Unicode catcodes before loading patterns>>
  \begingroup
      % Reset chars "80-"C0 to category "other", no case mapping:
    \catcode`\@=11 \count@=128
    \loop\ifnum\count@<192
      \global\uccode\count@=0 \global\lccode\count@=0
      \global\catcode\count@=12 \global\sfcode\count@=1000
      \advance\count@ by 1 \repeat
      % Other:
    \def\O ##1 {%
      \global\uccode"##1=0 \global\lccode"##1=0
      \global\catcode"##1=12 \global\sfcode"##1=1000 }%
      % Letter:
    \def\L ##1 ##2 ##3 {\global\catcode"##1=11
      \global\uccode"##1="##2
      \global\lccode"##1="##3
      % Uppercase letters have sfcode=999:
      \ifnum"##1="##3 \else \global\sfcode"##1=999 \fi }%
      % Letter without case mappings:
    \def\l ##1 {\L ##1 ##1 ##1 }%
    \l 00AA
    \L 00B5 039C 00B5
    \l 00BA
    \O 00D7
    \l 00DF
    \O 00F7
    \L 00FF 0178 00FF
  \endgroup
  \input #1\relax
%<</Restore Unicode catcodes before loading patterns>>
%    \end{macrocode}
%
% Now, the code.
%
%    \begin{macrocode}
%<*xetex>
\def\BabelStringsDefault{unicode}
\let\xebbl@stop\relax
\AddBabelHook{xetex}{encodedcommands}{%
  \def\bbl@tempa{#1}%
  \ifx\bbl@tempa\@empty
    \XeTeXinputencoding"bytes"%
  \else
    \XeTeXinputencoding"#1"%
  \fi
  \def\xebbl@stop{\XeTeXinputencoding"utf8"}}
\AddBabelHook{xetex}{stopcommands}{%
  \xebbl@stop
  \let\xebbl@stop\relax}
\AddBabelHook{xetex}{loadkernel}{%
<@Restore Unicode catcodes before loading patterns@>}
\ifx\DisableBabelHook\@undefined\endinput\fi
\AddBabelHook{babel-fontspec}{afterextras}{\bbl@switchfont}
\DisableBabelHook{babel-fontspec}
<@Font selection@>
%</xetex>
%    \end{macrocode}
%
% \subsection{LuaTeX}
%
% The new loader for luatex is based solely on |language.dat|, which
% is read on the fly. The code shouldn't be executed when the format
% is build, so we check if |\AddBabelHook| is defined. Then comes a
% modified version of the loader in |hyphen.cfg| (without the
% hyphenmins stuff, which is under the direct control of \babel).
%
% The names |\l@<language>| are defined and take some value from the
% beginning because all \texttt{ldf} files assume this for the
% corresponding language to be considered valid, but patterns are not
% loaded (except the first one). This is done later, when the language
% is first selected (which usually means when the \texttt{ldf}
% finishes). If a language has been loaded, |\bbl@hyphendata@<num>|
% exists (with the names of the files read).
%
% The default setup preloads the first language into the format. This
% is intended mainly for `english', so that it's available without
% further intervention from the user.  To avoid duplicating it, the
% following rule applies: if the ``0th'' language and the first
% language in |language.dat| have the same name then just ignore the
% latter. If there are new synonymous, the are added, but note if the
% language patterns have not been preloaded they won't at run time.
%
% Other preloaded languages could be read twice, if they has been
% preloaded into the format. This is not optimal, but it shouldn't
% happen very often -- with \luatex{} patterns are best loaded when
% the document is typeset, and the ``0th'' language is preloaded just
% for backwards compatibility.
%
% As of 1.1b, lua(e)tex is taken into account. Formerly, loading of
% patterns on the fly didn't work in this format, but with the new
% loader it does.  Unfortunately, the format is not based on \babel,
% and data could be duplicated, because languages are reassigned above
% those in the format (nothing serious, anyway). Note even with this
% format |language.dat| is used (under the principle of a single
% source), instead of |language.def|.
%
% Of course, there is room for improvements, like tools to read and
% reassign languages, which would require modifying the language list,
% and better error handling.
%
% We need catcode tables, but no format (targeted by \babel) provide a
% command to allocate them (although there are packages like
% \textsf{ctablestack}). For the moment, a dangerous approach is used
% -- just allocate a high random number and cross the fingers. To
% complicate things, \textsf{etex.sty} changes the way languages are
% allocated.
%
% \changes{bbunicode~1.0b}{2013/04/22}{luatex-hyphen is loaded with
%   require. Changes supplied by \'{E}lie Roux.}
% \changes{bbunicode~1.0c}{2014/03/10}{Defined hook for
%   `initiateactive', to fetch the next token and continue only if
%   letter or other.}
% \changes{bbunicode~1.0d}{2014/03/21}{Removed the `misfeature' for
%   `initiateactive'.}
% \changes{bbunicode~1.0e}{2015/05/10}{Use brackets instead of
%   \cs{luaescapestring}.}
% \changes{bbunicode~1.0e}{2015/07/26}{Added function addpattern
%   and modified the patterns hook.}
% \changes{bbunicode~1.1a}{2016/01/26}{New hyphenation loader for
%   luatex.}
% \changes{bbunicode~1.1b}{2016/02/05}{Also lua(e)tex.}
% \changes{bbunicode~1.1c}{2016/02/08}{Base reading of patterns on
%   number, not in name.}
% \changes{bbunicode~1.1c}{2016/02/08}{Some hacks for polyglossia. To
%   be improved.}
% \changes{bbunicode~1.1c}{2016/02/23}{Thoroughly revised.}
% \changes{bbunicode~1.1d}{2016/4/22}{Lua: Fixed a line break at
%   \cs{foreignlanguage} with unloaded patterns. Added
%   \cs{babelcatcodetablenum}, just in case.}
%
%    \begin{macrocode}
%<*luatex>
\ifx\AddBabelHook\@undefined
\begingroup
  \toks@{}
  \count@\z@ % 0=start, 1=0th, 2=normal
  \def\bbl@process@line#1#2 #3 #4 {%
    \ifx=#1%
      \bbl@process@synonym{#2}%
    \else
      \bbl@process@language{#1#2}{#3}{#4}%
    \fi
    \ignorespaces}
  \def\bbl@manylang{%
    \ifnum\bbl@last>\@ne
      \bbl@info{Non-standard hyphenation setup}%
    \fi
    \let\bbl@manylang\relax}
  \def\bbl@process@language#1#2#3{%
    \ifcase\count@
      \@ifundefined{zth@#1}{\count@\tw@}{\count@\@ne}%
    \or
      \count@\tw@
    \fi
    \ifnum\count@=\tw@
      \expandafter\addlanguage\csname l@#1\endcsname
      \language\allocationnumber
      \chardef\bbl@last\allocationnumber
      \bbl@manylang
      \let\bbl@elt\relax
      \xdef\bbl@languages{%
        \bbl@languages\bbl@elt{#1}{\the\language}{#2}{#3}}%
    \fi
    \the\toks@
    \toks@{}}
  \def\bbl@process@synonym@aux#1#2{%
    \global\expandafter\chardef\csname l@#1\endcsname#2\relax
    \let\bbl@elt\relax
    \xdef\bbl@languages{%
      \bbl@languages\bbl@elt{#1}{#2}{}{}}}%
  \def\bbl@process@synonym#1{%
    \ifcase\count@
      \toks@\expandafter{\the\toks@\relax\bbl@process@synonym{#1}}%
    \or
      \@ifundefined{zth@#1}{\bbl@process@synonym@aux{#1}{0}}{}%
    \else
      \bbl@process@synonym@aux{#1}{\the\bbl@last}%
    \fi}
  \ifx\bbl@languages\@undefined % Just a (sensible?) guess
    \chardef\l@english\z@
    \chardef\l@USenglish\z@
    \chardef\bbl@last\z@
    \global\@namedef{bbl@hyphendata@0}{{hyphen.tex}{}}
    \gdef\bbl@languages{%
      \bbl@elt{english}{0}{hyphen.tex}{}%
      \bbl@elt{USenglish}{0}{}{}}
  \else
    \global\let\bbl@languages@format\bbl@languages
    \def\bbl@elt#1#2#3#4{% Remove all except language 0
      \ifnum#2>\z@\else
        \noexpand\bbl@elt{#1}{#2}{#3}{#4}%
      \fi}%
    \xdef\bbl@languages{\bbl@languages}%
  \fi
  \def\bbl@elt#1#2#3#4{\@namedef{zth@#1}{}} % Define flags
  \bbl@languages
  \openin1=language.dat
  \ifeof1
    \bbl@warning{I couldn't find language.dat. No additional\\%
                 patterns loaded. Reported}%
  \else
    \loop
      \endlinechar\m@ne
      \read1 to \bbl@line
      \endlinechar`\^^M
      \if T\ifeof1F\fi T\relax
        \ifx\bbl@line\@empty\else
          \edef\bbl@line{\bbl@line\space\space\space}%
          \expandafter\bbl@process@line\bbl@line\relax
        \fi
    \repeat
  \fi
\endgroup
\def\bbl@get@enc#1:#2:#3\@@@{\def\bbl@hyph@enc{#2}}
\ifx\babelcatcodetablenum\@undefined
  \def\babelcatcodetablenum{5211}
\fi
\def\bbl@luapatterns#1#2{%
  \bbl@get@enc#1::\@@@
  \setbox\z@\hbox\bgroup
    \begingroup
      \ifx\catcodetable\@undefined
        \let\savecatcodetable\luatexsavecatcodetable
        \let\initcatcodetable\luatexinitcatcodetable
        \let\catcodetable\luatexcatcodetable
      \fi
      \savecatcodetable\babelcatcodetablenum\relax
      \initcatcodetable\numexpr\babelcatcodetablenum+1\relax
      \catcodetable\numexpr\babelcatcodetablenum+1\relax
      \catcode`\#=6  \catcode`\$=3 \catcode`\&=4 \catcode`\^=7
      \catcode`\_=8  \catcode`\{=1 \catcode`\}=2 \catcode`\~=13
      \catcode`\@=11 \catcode`\^^I=10 \catcode`\^^J=12
      \catcode`\<=12 \catcode`\>=12 \catcode`\*=12 \catcode`\.=12
      \catcode`\-=12 \catcode`\/=12 \catcode`\[=12 \catcode`\]=12
      \catcode`\`=12 \catcode`\'=12 \catcode`\"=12
      \input #1\relax
      \catcodetable\babelcatcodetablenum\relax
    \endgroup
    \def\bbl@tempa{#2}%
    \ifx\bbl@tempa\@empty\else
      \input #2\relax
    \fi
  \egroup}%
\def\bbl@patterns@lua#1{%
  \language=\expandafter\ifx\csname l@#1:\f@encoding\endcsname\relax
    \csname l@#1\endcsname
    \edef\bbl@tempa{#1}%
  \else
    \csname l@#1:\f@encoding\endcsname
    \edef\bbl@tempa{#1:\f@encoding}%
  \fi\relax
  \@namedef{lu@texhyphen@loaded@\the\language}{}% Temp
  \@ifundefined{bbl@hyphendata@\the\language}%
    {\def\bbl@elt##1##2##3##4{%
       \ifnum##2=\csname l@\bbl@tempa\endcsname % #2=spanish, dutch:OT1...
         \def\bbl@tempb{##3}%
         \ifx\bbl@tempb\@empty\else % if not a synonymous
           \def\bbl@tempc{{##3}{##4}}%
         \fi
         \bbl@csarg\xdef{hyphendata@##2}{\bbl@tempc}%
       \fi}%
     \bbl@languages
     \@ifundefined{bbl@hyphendata@\the\language}%
       {\bbl@info{No hyphenation patterns were set for\\%
                  language ‘\bbl@tempa’. Reported}}%
       {\expandafter\expandafter\expandafter\bbl@luapatterns
          \csname bbl@hyphendata@\the\language\endcsname}}{}}
\endinput\fi
\begingroup
\catcode`\%=12
\catcode`\'=12
\catcode`\"=12
\catcode`\:=12
\directlua{
  Babel = Babel or {}
  function Babel.bytes(line)
    return line:gsub("(.)",
      function (chr) return unicode.utf8.char(string.byte(chr)) end)
  end
  function Babel.begin_process_input()
    if luatexbase and luatexbase.add_to_callback then
      luatexbase.add_to_callback('process_input_buffer',
                                 Babel.bytes,'Babel.bytes')
    else
      Babel.callback = callback.find('process_input_buffer')
      callback.register('process_input_buffer',Babel.bytes)
    end
  end
  function Babel.end_process_input ()
    if luatexbase and luatexbase.remove_from_callback then
      luatexbase.remove_from_callback('process_input_buffer','Babel.bytes')
    else
      callback.register('process_input_buffer',Babel.callback)
    end
  end
  function Babel.addpatterns(pp, lg)
    local lg = lang.new(lg)
    local pats = lang.patterns(lg) or ''
    lang.clear_patterns(lg)
    for p in pp:gmatch('[^%s]+') do
      ss = ''
      for i in string.utfcharacters(p:gsub('%d', '')) do
         ss = ss .. '%d?' .. i
      end
      ss = ss:gsub('^%%d%?%.', '%%.') .. '%d?'
      ss = ss:gsub('%.%%d%?$', '%%.')
      pats, n = pats:gsub('%s' .. ss .. '%s', ' ' .. p .. ' ')
      if n == 0 then
        tex.sprint(
          [[\string\csname\space bbl@info\endcsname{New pattern: ]]
          .. p .. [[}]])
        pats = pats .. ' ' .. p
      else
        tex.sprint(
          [[\string\csname\space bbl@info\endcsname{Renew pattern: ]]
          .. p .. [[}]])
      end
    end
    lang.patterns(lg, pats)
  end
}
\endgroup
\def\BabelStringsDefault{unicode}
\let\luabbl@stop\relax
\AddBabelHook{luatex}{encodedcommands}{%
  \def\bbl@tempa{utf8}\def\bbl@tempb{#1}%
  \ifx\bbl@tempa\bbl@tempb\else
    \directlua{Babel.begin_process_input()}%
    \def\luabbl@stop{%
      \directlua{Babel.end_process_input()}}%
  \fi}%
\AddBabelHook{luatex}{stopcommands}{%
  \luabbl@stop
  \let\luabbl@stop\relax}
\AddBabelHook{luatex}{patterns}{%
  \@ifundefined{bbl@hyphendata@\the\language}%
    {\def\bbl@elt##1##2##3##4{%
       \ifnum##2=\csname l@#2\endcsname % #2=spanish, dutch:OT1...
         \def\bbl@tempb{##3}%
         \ifx\bbl@tempb\@empty\else % if not a synonymous
           \def\bbl@tempc{{##3}{##4}}%
         \fi
         \bbl@csarg\xdef{hyphendata@##2}{\bbl@tempc}%
       \fi}%
     \bbl@languages
     \@ifundefined{bbl@hyphendata@\the\language}%
       {\bbl@info{No hyphenation patterns were set for\\%
                  language ‘#2’. Reported}}%
       {\expandafter\expandafter\expandafter\bbl@luapatterns
          \csname bbl@hyphendata@\the\language\endcsname}}{}%
  \@ifundefined{bbl@patterns@}{}{%
    \begingroup
      \bbl@xin@{,\number\language,}{,\bbl@pttnlist}%
      \ifin@\else
        \ifx\bbl@patterns@\@empty\else
           \directlua{ Babel.addpatterns(
             [[\bbl@patterns@]], \number\language) }%
        \fi
        \@ifundefined{bbl@patterns@#1}%
          \@empty
          {\directlua{ Babel.addpatterns(
               [[\space\csname bbl@patterns@#1\endcsname]],
               \number\language) }}%
        \xdef\bbl@pttnlist{\bbl@pttnlist\number\language,}%
      \fi
    \endgroup}}
\AddBabelHook{luatex}{everylanguage}{%
  \def\process@language##1##2##3{%
    \def\process@line####1####2 ####3 ####4 {}}}
\AddBabelHook{luatex}{loadpatterns}{%
   \input #1\relax
   \expandafter\gdef\csname bbl@hyphendata@\the\language\endcsname
     {{#1}{}}}
\AddBabelHook{luatex}{loadexceptions}{%
   \input #1\relax
   \def\bbl@tempb##1##2{{##1}{#1}}%
   \expandafter\xdef\csname bbl@hyphendata@\the\language\endcsname
     {\expandafter\expandafter\expandafter\bbl@tempb
      \csname bbl@hyphendata@\the\language\endcsname}}
%    \end{macrocode}
%
%  \begin{macro}{\babelpatterns}
%
%    This macro adds patterns. Two macros are used to store them:
%    |\bbl@patterns@| for the global ones and |\bbl@patterns@<lang>|
%    for language ones. We make sure there is a space between words
%    when multiple commands are used.
% \changes{bbunicode~1.0e}{2015/07/26}{Macro \cs{babelpatterns} added}
%
%    \begin{macrocode}
\@onlypreamble\babelpatterns
\AtEndOfPackage{%
  \newcommand\babelpatterns[2][\@empty]{%
    \ifx\bbl@patterns@\relax
      \let\bbl@patterns@\@empty
    \fi
    \ifx\bbl@pttnlist\@empty\else
      \bbl@warning{%
        You must not intermingle \string\selectlanguage\space and\\%
        \string\babelpatterns\space or some patterns will not\\%
        be taken into account. Reported}%
    \fi
    \ifx\@empty#1%
      \protected@edef\bbl@patterns@{\bbl@patterns@\space#2}%
    \else
      \edef\bbl@tempb{\zap@space#1 \@empty}%
      \bbl@for\bbl@tempa\bbl@tempb{%
        \bbl@fixname\bbl@tempa
        \bbl@iflanguage\bbl@tempa{%
          \bbl@csarg\protected@edef{patterns@\bbl@tempa}{%
            \@ifundefined{bbl@patterns@\bbl@tempa}%
              \@empty
              {\csname bbl@patterns@\bbl@tempa\endcsname\space}%
            #2}}}%
    \fi}}
%    \end{macrocode}
%  \end{macro}
%
% Common stuff.
%
%    \begin{macrocode}
\AddBabelHook{luatex}{loadkernel}{%
<@Restore Unicode catcodes before loading patterns@>}
\ifx\DisableBabelHook\@undefined\endinput\fi
\AddBabelHook{babel-fontspec}{afterextras}{\bbl@switchfont}
\DisableBabelHook{babel-fontspec}
<@Font selection@>
%</luatex>
%    \end{macrocode}
%
%    \section{Bidi support in \luatex}
%
% \changes{3.14}{2017/09/30}{LuaTeX - support for R/AL texts - basic-r}
%
% \textbf{Work in progress}. The file \textsf{babel-bidi.lua}
% currently only contains data. It's a large and boring file and it's
% not shown here. See the generated file.
%\iffalse
%    \begin{macrocode}
%<*bidi> 
-- Data from Unicode and ConTeXt

Babel = Babel or {}

Babel.ranges={ 
 {0x0590, 0x05FF, 'r'},
 {0x0600, 0x07BF, 'al'},
 {0x07C0, 0x085F, 'r'},
 {0x0860, 0x086F, 'al'},
 {0x08A0, 0x08FF, 'al'},
 {0xFB1D, 0xFB4F, 'r'},
 {0xFB50, 0xFDFF, 'al'},
 {0xFE70, 0xFEFF, 'al'},
 {0x10800, 0x10C4F, 'r'},
 {0x1E800, 0x1E8DF, 'r'},
 {0x1E900, 0x1E95F, 'r'},
 {0x1F300, 0x1F9FF, 'on'}
}

Babel.characters={
 [0x0]={d='bn'},
 [0x1]={d='bn'},
 [0x2]={d='bn'},
 [0x3]={d='bn'},
 [0x4]={d='bn'},
 [0x5]={d='bn'},
 [0x6]={d='bn'},
 [0x7]={d='bn'},
 [0x8]={d='bn'},
 [0x9]={d='s'},
 [0xA]={d='b'},
 [0xB]={d='s'},
 [0xC]={d='ws'},
 [0xD]={d='b'},
 [0xE]={d='bn'},
 [0xF]={d='bn'},
 [0x10]={d='bn'},
 [0x11]={d='bn'},
 [0x12]={d='bn'},
 [0x13]={d='bn'},
 [0x14]={d='bn'},
 [0x15]={d='bn'},
 [0x16]={d='bn'},
 [0x17]={d='bn'},
 [0x18]={d='bn'},
 [0x19]={d='bn'},
 [0x1A]={d='bn'},
 [0x1B]={d='bn'},
 [0x1C]={d='b'},
 [0x1D]={d='b'},
 [0x1E]={d='b'},
 [0x1F]={d='s'},
 [0x20]={d='ws'},
 [0x21]={d='on'},
 [0x22]={d='on'},
 [0x23]={d='et'},
 [0x24]={d='et'},
 [0x25]={d='et'},
 [0x26]={d='on'},
 [0x27]={d='on'},
 [0x28]={d='on', m=0x29},
 [0x29]={d='on', m=0x28},
 [0x2A]={d='on'},
 [0x2B]={d='es'},
 [0x2C]={d='cs'},
 [0x2D]={d='es'},
 [0x2E]={d='cs'},
 [0x2F]={d='cs'},
 [0x30]={d='en'},
 [0x31]={d='en'},
 [0x32]={d='en'},
 [0x33]={d='en'},
 [0x34]={d='en'},
 [0x35]={d='en'},
 [0x36]={d='en'},
 [0x37]={d='en'},
 [0x38]={d='en'},
 [0x39]={d='en'},
 [0x3A]={d='cs'},
 [0x3B]={d='on'},
 [0x3C]={d='on', m=0x3E},
 [0x3D]={d='on'},
 [0x3E]={d='on', m=0x3C},
 [0x3F]={d='on'},
 [0x40]={d='on'},
 [0x5B]={d='on', m=0x5D},
 [0x5C]={d='on'},
 [0x5D]={d='on', m=0x5B},
 [0x5E]={d='on'},
 [0x5F]={d='on'},
 [0x60]={d='on'},
 [0x7B]={d='on', m=0x7D},
 [0x7C]={d='on'},
 [0x7D]={d='on', m=0x7B},
 [0x7E]={d='on'},
 [0x7F]={d='bn'},
 [0x80]={d='bn'},
 [0x81]={d='bn'},
 [0x82]={d='bn'},
 [0x83]={d='bn'},
 [0x84]={d='bn'},
 [0x85]={d='b'},
 [0x86]={d='bn'},
 [0x87]={d='bn'},
 [0x88]={d='bn'},
 [0x89]={d='bn'},
 [0x8A]={d='bn'},
 [0x8B]={d='bn'},
 [0x8C]={d='bn'},
 [0x8D]={d='bn'},
 [0x8E]={d='bn'},
 [0x8F]={d='bn'},
 [0x90]={d='bn'},
 [0x91]={d='bn'},
 [0x92]={d='bn'},
 [0x93]={d='bn'},
 [0x94]={d='bn'},
 [0x95]={d='bn'},
 [0x96]={d='bn'},
 [0x97]={d='bn'},
 [0x98]={d='bn'},
 [0x99]={d='bn'},
 [0x9A]={d='bn'},
 [0x9B]={d='bn'},
 [0x9C]={d='bn'},
 [0x9D]={d='bn'},
 [0x9E]={d='bn'},
 [0x9F]={d='bn'},
 [0xA0]={d='cs'},
 [0xA1]={d='on'},
 [0xA2]={d='et'},
 [0xA3]={d='et'},
 [0xA4]={d='et'},
 [0xA5]={d='et'},
 [0xA6]={d='on'},
 [0xA7]={d='on'},
 [0xA8]={d='on'},
 [0xA9]={d='on'},
 [0xAB]={d='on', m=0xBB},
 [0xAC]={d='on'},
 [0xAD]={d='bn'},
 [0xAE]={d='on'},
 [0xAF]={d='on'},
 [0xB0]={d='et'},
 [0xB1]={d='et'},
 [0xB2]={d='en'},
 [0xB3]={d='en'},
 [0xB4]={d='on'},
 [0xB6]={d='on'},
 [0xB7]={d='on'},
 [0xB8]={d='on'},
 [0xB9]={d='en'},
 [0xBB]={d='on', m=0xAB},
 [0xBC]={d='on'},
 [0xBD]={d='on'},
 [0xBE]={d='on'},
 [0xBF]={d='on'},
 [0xD7]={d='on'},
 [0xF7]={d='on'},
 [0x2B9]={d='on'},
 [0x2BA]={d='on'},
 [0x2C2]={d='on'},
 [0x2C3]={d='on'},
 [0x2C4]={d='on'},
 [0x2C5]={d='on'},
 [0x2C6]={d='on'},
 [0x2C7]={d='on'},
 [0x2C8]={d='on'},
 [0x2C9]={d='on'},
 [0x2CA]={d='on'},
 [0x2CB]={d='on'},
 [0x2CC]={d='on'},
 [0x2CD]={d='on'},
 [0x2CE]={d='on'},
 [0x2CF]={d='on'},
 [0x2D2]={d='on'},
 [0x2D3]={d='on'},
 [0x2D4]={d='on'},
 [0x2D5]={d='on'},
 [0x2D6]={d='on'},
 [0x2D7]={d='on'},
 [0x2D8]={d='on'},
 [0x2D9]={d='on'},
 [0x2DA]={d='on'},
 [0x2DB]={d='on'},
 [0x2DC]={d='on'},
 [0x2DD]={d='on'},
 [0x2DE]={d='on'},
 [0x2DF]={d='on'},
 [0x2E5]={d='on'},
 [0x2E6]={d='on'},
 [0x2E7]={d='on'},
 [0x2E8]={d='on'},
 [0x2E9]={d='on'},
 [0x2EA]={d='on'},
 [0x2EB]={d='on'},
 [0x2EC]={d='on'},
 [0x2ED]={d='on'},
 [0x2EF]={d='on'},
 [0x2F0]={d='on'},
 [0x2F1]={d='on'},
 [0x2F2]={d='on'},
 [0x2F3]={d='on'},
 [0x2F4]={d='on'},
 [0x2F5]={d='on'},
 [0x2F6]={d='on'},
 [0x2F7]={d='on'},
 [0x2F8]={d='on'},
 [0x2F9]={d='on'},
 [0x2FA]={d='on'},
 [0x2FB]={d='on'},
 [0x2FC]={d='on'},
 [0x2FD]={d='on'},
 [0x2FE]={d='on'},
 [0x2FF]={d='on'},
 [0x300]={d='nsm'},
 [0x301]={d='nsm'},
 [0x302]={d='nsm'},
 [0x303]={d='nsm'},
 [0x304]={d='nsm'},
 [0x305]={d='nsm'},
 [0x306]={d='nsm'},
 [0x307]={d='nsm'},
 [0x308]={d='nsm'},
 [0x309]={d='nsm'},
 [0x30A]={d='nsm'},
 [0x30B]={d='nsm'},
 [0x30C]={d='nsm'},
 [0x30D]={d='nsm'},
 [0x30E]={d='nsm'},
 [0x30F]={d='nsm'},
 [0x310]={d='nsm'},
 [0x311]={d='nsm'},
 [0x312]={d='nsm'},
 [0x313]={d='nsm'},
 [0x314]={d='nsm'},
 [0x315]={d='nsm'},
 [0x316]={d='nsm'},
 [0x317]={d='nsm'},
 [0x318]={d='nsm'},
 [0x319]={d='nsm'},
 [0x31A]={d='nsm'},
 [0x31B]={d='nsm'},
 [0x31C]={d='nsm'},
 [0x31D]={d='nsm'},
 [0x31E]={d='nsm'},
 [0x31F]={d='nsm'},
 [0x320]={d='nsm'},
 [0x321]={d='nsm'},
 [0x322]={d='nsm'},
 [0x323]={d='nsm'},
 [0x324]={d='nsm'},
 [0x325]={d='nsm'},
 [0x326]={d='nsm'},
 [0x327]={d='nsm'},
 [0x328]={d='nsm'},
 [0x329]={d='nsm'},
 [0x32A]={d='nsm'},
 [0x32B]={d='nsm'},
 [0x32C]={d='nsm'},
 [0x32D]={d='nsm'},
 [0x32E]={d='nsm'},
 [0x32F]={d='nsm'},
 [0x330]={d='nsm'},
 [0x331]={d='nsm'},
 [0x332]={d='nsm'},
 [0x333]={d='nsm'},
 [0x334]={d='nsm'},
 [0x335]={d='nsm'},
 [0x336]={d='nsm'},
 [0x337]={d='nsm'},
 [0x338]={d='nsm'},
 [0x339]={d='nsm'},
 [0x33A]={d='nsm'},
 [0x33B]={d='nsm'},
 [0x33C]={d='nsm'},
 [0x33D]={d='nsm'},
 [0x33E]={d='nsm'},
 [0x33F]={d='nsm'},
 [0x340]={d='nsm'},
 [0x341]={d='nsm'},
 [0x342]={d='nsm'},
 [0x343]={d='nsm'},
 [0x344]={d='nsm'},
 [0x345]={d='nsm'},
 [0x346]={d='nsm'},
 [0x347]={d='nsm'},
 [0x348]={d='nsm'},
 [0x349]={d='nsm'},
 [0x34A]={d='nsm'},
 [0x34B]={d='nsm'},
 [0x34C]={d='nsm'},
 [0x34D]={d='nsm'},
 [0x34E]={d='nsm'},
 [0x34F]={d='nsm'},
 [0x350]={d='nsm'},
 [0x351]={d='nsm'},
 [0x352]={d='nsm'},
 [0x353]={d='nsm'},
 [0x354]={d='nsm'},
 [0x355]={d='nsm'},
 [0x356]={d='nsm'},
 [0x357]={d='nsm'},
 [0x358]={d='nsm'},
 [0x359]={d='nsm'},
 [0x35A]={d='nsm'},
 [0x35B]={d='nsm'},
 [0x35C]={d='nsm'},
 [0x35D]={d='nsm'},
 [0x35E]={d='nsm'},
 [0x35F]={d='nsm'},
 [0x360]={d='nsm'},
 [0x361]={d='nsm'},
 [0x362]={d='nsm'},
 [0x363]={d='nsm'},
 [0x364]={d='nsm'},
 [0x365]={d='nsm'},
 [0x366]={d='nsm'},
 [0x367]={d='nsm'},
 [0x368]={d='nsm'},
 [0x369]={d='nsm'},
 [0x36A]={d='nsm'},
 [0x36B]={d='nsm'},
 [0x36C]={d='nsm'},
 [0x36D]={d='nsm'},
 [0x36E]={d='nsm'},
 [0x36F]={d='nsm'},
 [0x374]={d='on'},
 [0x375]={d='on'},
 [0x37E]={d='on'},
 [0x384]={d='on'},
 [0x385]={d='on'},
 [0x387]={d='on'},
 [0x3F6]={d='on'},
 [0x483]={d='nsm'},
 [0x484]={d='nsm'},
 [0x485]={d='nsm'},
 [0x486]={d='nsm'},
 [0x487]={d='nsm'},
 [0x488]={d='nsm'},
 [0x489]={d='nsm'},
 [0x58A]={d='on'},
 [0x58D]={d='on'},
 [0x58E]={d='on'},
 [0x58F]={d='et'},
 [0x591]={d='nsm'},
 [0x592]={d='nsm'},
 [0x593]={d='nsm'},
 [0x594]={d='nsm'},
 [0x595]={d='nsm'},
 [0x596]={d='nsm'},
 [0x597]={d='nsm'},
 [0x598]={d='nsm'},
 [0x599]={d='nsm'},
 [0x59A]={d='nsm'},
 [0x59B]={d='nsm'},
 [0x59C]={d='nsm'},
 [0x59D]={d='nsm'},
 [0x59E]={d='nsm'},
 [0x59F]={d='nsm'},
 [0x5A0]={d='nsm'},
 [0x5A1]={d='nsm'},
 [0x5A2]={d='nsm'},
 [0x5A3]={d='nsm'},
 [0x5A4]={d='nsm'},
 [0x5A5]={d='nsm'},
 [0x5A6]={d='nsm'},
 [0x5A7]={d='nsm'},
 [0x5A8]={d='nsm'},
 [0x5A9]={d='nsm'},
 [0x5AA]={d='nsm'},
 [0x5AB]={d='nsm'},
 [0x5AC]={d='nsm'},
 [0x5AD]={d='nsm'},
 [0x5AE]={d='nsm'},
 [0x5AF]={d='nsm'},
 [0x5B0]={d='nsm'},
 [0x5B1]={d='nsm'},
 [0x5B2]={d='nsm'},
 [0x5B3]={d='nsm'},
 [0x5B4]={d='nsm'},
 [0x5B5]={d='nsm'},
 [0x5B6]={d='nsm'},
 [0x5B7]={d='nsm'},
 [0x5B8]={d='nsm'},
 [0x5B9]={d='nsm'},
 [0x5BA]={d='nsm'},
 [0x5BB]={d='nsm'},
 [0x5BC]={d='nsm'},
 [0x5BD]={d='nsm'},
 [0x5BF]={d='nsm'},
 [0x5C1]={d='nsm'},
 [0x5C2]={d='nsm'},
 [0x5C4]={d='nsm'},
 [0x5C5]={d='nsm'},
 [0x5C7]={d='nsm'},
 [0x600]={d='an'},
 [0x601]={d='an'},
 [0x602]={d='an'},
 [0x603]={d='an'},
 [0x604]={d='an'},
 [0x605]={d='an'},
 [0x606]={d='on'},
 [0x607]={d='on'},
 [0x608]={d='al'},
 [0x609]={d='et'},
 [0x60A]={d='et'},
 [0x60B]={d='al'},
 [0x60C]={d='cs'},
 [0x60D]={d='al'},
 [0x60E]={d='on'},
 [0x60F]={d='on'},
 [0x610]={d='nsm'},
 [0x611]={d='nsm'},
 [0x612]={d='nsm'},
 [0x613]={d='nsm'},
 [0x614]={d='nsm'},
 [0x615]={d='nsm'},
 [0x616]={d='nsm'},
 [0x617]={d='nsm'},
 [0x618]={d='nsm'},
 [0x619]={d='nsm'},
 [0x61A]={d='nsm'},
 [0x64B]={d='nsm'},
 [0x64C]={d='nsm'},
 [0x64D]={d='nsm'},
 [0x64E]={d='nsm'},
 [0x64F]={d='nsm'},
 [0x650]={d='nsm'},
 [0x651]={d='nsm'},
 [0x652]={d='nsm'},
 [0x653]={d='nsm'},
 [0x654]={d='nsm'},
 [0x655]={d='nsm'},
 [0x656]={d='nsm'},
 [0x657]={d='nsm'},
 [0x658]={d='nsm'},
 [0x659]={d='nsm'},
 [0x65A]={d='nsm'},
 [0x65B]={d='nsm'},
 [0x65C]={d='nsm'},
 [0x65D]={d='nsm'},
 [0x65E]={d='nsm'},
 [0x65F]={d='nsm'},
 [0x660]={d='an'},
 [0x661]={d='an'},
 [0x662]={d='an'},
 [0x663]={d='an'},
 [0x664]={d='an'},
 [0x665]={d='an'},
 [0x666]={d='an'},
 [0x667]={d='an'},
 [0x668]={d='an'},
 [0x669]={d='an'},
 [0x66A]={d='et'},
 [0x66B]={d='an'},
 [0x66C]={d='an'},
 [0x670]={d='nsm'},
 [0x6D6]={d='nsm'},
 [0x6D7]={d='nsm'},
 [0x6D8]={d='nsm'},
 [0x6D9]={d='nsm'},
 [0x6DA]={d='nsm'},
 [0x6DB]={d='nsm'},
 [0x6DC]={d='nsm'},
 [0x6DD]={d='an'},
 [0x6DE]={d='on'},
 [0x6DF]={d='nsm'},
 [0x6E0]={d='nsm'},
 [0x6E1]={d='nsm'},
 [0x6E2]={d='nsm'},
 [0x6E3]={d='nsm'},
 [0x6E4]={d='nsm'},
 [0x6E7]={d='nsm'},
 [0x6E8]={d='nsm'},
 [0x6E9]={d='on'},
 [0x6EA]={d='nsm'},
 [0x6EB]={d='nsm'},
 [0x6EC]={d='nsm'},
 [0x6ED]={d='nsm'},
 [0x6F0]={d='en'},
 [0x6F1]={d='en'},
 [0x6F2]={d='en'},
 [0x6F3]={d='en'},
 [0x6F4]={d='en'},
 [0x6F5]={d='en'},
 [0x6F6]={d='en'},
 [0x6F7]={d='en'},
 [0x6F8]={d='en'},
 [0x6F9]={d='en'},
 [0x711]={d='nsm'},
 [0x730]={d='nsm'},
 [0x731]={d='nsm'},
 [0x732]={d='nsm'},
 [0x733]={d='nsm'},
 [0x734]={d='nsm'},
 [0x735]={d='nsm'},
 [0x736]={d='nsm'},
 [0x737]={d='nsm'},
 [0x738]={d='nsm'},
 [0x739]={d='nsm'},
 [0x73A]={d='nsm'},
 [0x73B]={d='nsm'},
 [0x73C]={d='nsm'},
 [0x73D]={d='nsm'},
 [0x73E]={d='nsm'},
 [0x73F]={d='nsm'},
 [0x740]={d='nsm'},
 [0x741]={d='nsm'},
 [0x742]={d='nsm'},
 [0x743]={d='nsm'},
 [0x744]={d='nsm'},
 [0x745]={d='nsm'},
 [0x746]={d='nsm'},
 [0x747]={d='nsm'},
 [0x748]={d='nsm'},
 [0x749]={d='nsm'},
 [0x74A]={d='nsm'},
 [0x7A6]={d='nsm'},
 [0x7A7]={d='nsm'},
 [0x7A8]={d='nsm'},
 [0x7A9]={d='nsm'},
 [0x7AA]={d='nsm'},
 [0x7AB]={d='nsm'},
 [0x7AC]={d='nsm'},
 [0x7AD]={d='nsm'},
 [0x7AE]={d='nsm'},
 [0x7AF]={d='nsm'},
 [0x7B0]={d='nsm'},
 [0x7EB]={d='nsm'},
 [0x7EC]={d='nsm'},
 [0x7ED]={d='nsm'},
 [0x7EE]={d='nsm'},
 [0x7EF]={d='nsm'},
 [0x7F0]={d='nsm'},
 [0x7F1]={d='nsm'},
 [0x7F2]={d='nsm'},
 [0x7F3]={d='nsm'},
 [0x7F6]={d='on'},
 [0x7F7]={d='on'},
 [0x7F8]={d='on'},
 [0x7F9]={d='on'},
 [0x816]={d='nsm'},
 [0x817]={d='nsm'},
 [0x818]={d='nsm'},
 [0x819]={d='nsm'},
 [0x81B]={d='nsm'},
 [0x81C]={d='nsm'},
 [0x81D]={d='nsm'},
 [0x81E]={d='nsm'},
 [0x81F]={d='nsm'},
 [0x820]={d='nsm'},
 [0x821]={d='nsm'},
 [0x822]={d='nsm'},
 [0x823]={d='nsm'},
 [0x825]={d='nsm'},
 [0x826]={d='nsm'},
 [0x827]={d='nsm'},
 [0x829]={d='nsm'},
 [0x82A]={d='nsm'},
 [0x82B]={d='nsm'},
 [0x82C]={d='nsm'},
 [0x82D]={d='nsm'},
 [0x859]={d='nsm'},
 [0x85A]={d='nsm'},
 [0x85B]={d='nsm'},
 [0x8D4]={d='nsm'},
 [0x8D5]={d='nsm'},
 [0x8D6]={d='nsm'},
 [0x8D7]={d='nsm'},
 [0x8D8]={d='nsm'},
 [0x8D9]={d='nsm'},
 [0x8DA]={d='nsm'},
 [0x8DB]={d='nsm'},
 [0x8DC]={d='nsm'},
 [0x8DD]={d='nsm'},
 [0x8DE]={d='nsm'},
 [0x8DF]={d='nsm'},
 [0x8E0]={d='nsm'},
 [0x8E1]={d='nsm'},
 [0x8E2]={d='an'},
 [0x8E3]={d='nsm'},
 [0x8E4]={d='nsm'},
 [0x8E5]={d='nsm'},
 [0x8E6]={d='nsm'},
 [0x8E7]={d='nsm'},
 [0x8E8]={d='nsm'},
 [0x8E9]={d='nsm'},
 [0x8EA]={d='nsm'},
 [0x8EB]={d='nsm'},
 [0x8EC]={d='nsm'},
 [0x8ED]={d='nsm'},
 [0x8EE]={d='nsm'},
 [0x8EF]={d='nsm'},
 [0x8F0]={d='nsm'},
 [0x8F1]={d='nsm'},
 [0x8F2]={d='nsm'},
 [0x8F3]={d='nsm'},
 [0x8F4]={d='nsm'},
 [0x8F5]={d='nsm'},
 [0x8F6]={d='nsm'},
 [0x8F7]={d='nsm'},
 [0x8F8]={d='nsm'},
 [0x8F9]={d='nsm'},
 [0x8FA]={d='nsm'},
 [0x8FB]={d='nsm'},
 [0x8FC]={d='nsm'},
 [0x8FD]={d='nsm'},
 [0x8FE]={d='nsm'},
 [0x8FF]={d='nsm'},
 [0x900]={d='nsm'},
 [0x901]={d='nsm'},
 [0x902]={d='nsm'},
 [0x93A]={d='nsm'},
 [0x93C]={d='nsm'},
 [0x941]={d='nsm'},
 [0x942]={d='nsm'},
 [0x943]={d='nsm'},
 [0x944]={d='nsm'},
 [0x945]={d='nsm'},
 [0x946]={d='nsm'},
 [0x947]={d='nsm'},
 [0x948]={d='nsm'},
 [0x94D]={d='nsm'},
 [0x951]={d='nsm'},
 [0x952]={d='nsm'},
 [0x953]={d='nsm'},
 [0x954]={d='nsm'},
 [0x955]={d='nsm'},
 [0x956]={d='nsm'},
 [0x957]={d='nsm'},
 [0x962]={d='nsm'},
 [0x963]={d='nsm'},
 [0x981]={d='nsm'},
 [0x9BC]={d='nsm'},
 [0x9C1]={d='nsm'},
 [0x9C2]={d='nsm'},
 [0x9C3]={d='nsm'},
 [0x9C4]={d='nsm'},
 [0x9CD]={d='nsm'},
 [0x9E2]={d='nsm'},
 [0x9E3]={d='nsm'},
 [0x9F2]={d='et'},
 [0x9F3]={d='et'},
 [0x9FB]={d='et'},
 [0xA01]={d='nsm'},
 [0xA02]={d='nsm'},
 [0xA3C]={d='nsm'},
 [0xA41]={d='nsm'},
 [0xA42]={d='nsm'},
 [0xA47]={d='nsm'},
 [0xA48]={d='nsm'},
 [0xA4B]={d='nsm'},
 [0xA4C]={d='nsm'},
 [0xA4D]={d='nsm'},
 [0xA51]={d='nsm'},
 [0xA70]={d='nsm'},
 [0xA71]={d='nsm'},
 [0xA75]={d='nsm'},
 [0xA81]={d='nsm'},
 [0xA82]={d='nsm'},
 [0xABC]={d='nsm'},
 [0xAC1]={d='nsm'},
 [0xAC2]={d='nsm'},
 [0xAC3]={d='nsm'},
 [0xAC4]={d='nsm'},
 [0xAC5]={d='nsm'},
 [0xAC7]={d='nsm'},
 [0xAC8]={d='nsm'},
 [0xACD]={d='nsm'},
 [0xAE2]={d='nsm'},
 [0xAE3]={d='nsm'},
 [0xAF1]={d='et'},
 [0xB01]={d='nsm'},
 [0xB3C]={d='nsm'},
 [0xB3F]={d='nsm'},
 [0xB41]={d='nsm'},
 [0xB42]={d='nsm'},
 [0xB43]={d='nsm'},
 [0xB44]={d='nsm'},
 [0xB4D]={d='nsm'},
 [0xB56]={d='nsm'},
 [0xB62]={d='nsm'},
 [0xB63]={d='nsm'},
 [0xB82]={d='nsm'},
 [0xBC0]={d='nsm'},
 [0xBCD]={d='nsm'},
 [0xBF3]={d='on'},
 [0xBF4]={d='on'},
 [0xBF5]={d='on'},
 [0xBF6]={d='on'},
 [0xBF7]={d='on'},
 [0xBF8]={d='on'},
 [0xBF9]={d='et'},
 [0xBFA]={d='on'},
 [0xC00]={d='nsm'},
 [0xC3E]={d='nsm'},
 [0xC3F]={d='nsm'},
 [0xC40]={d='nsm'},
 [0xC46]={d='nsm'},
 [0xC47]={d='nsm'},
 [0xC48]={d='nsm'},
 [0xC4A]={d='nsm'},
 [0xC4B]={d='nsm'},
 [0xC4C]={d='nsm'},
 [0xC4D]={d='nsm'},
 [0xC55]={d='nsm'},
 [0xC56]={d='nsm'},
 [0xC62]={d='nsm'},
 [0xC63]={d='nsm'},
 [0xC78]={d='on'},
 [0xC79]={d='on'},
 [0xC7A]={d='on'},
 [0xC7B]={d='on'},
 [0xC7C]={d='on'},
 [0xC7D]={d='on'},
 [0xC7E]={d='on'},
 [0xC81]={d='nsm'},
 [0xCBC]={d='nsm'},
 [0xCCC]={d='nsm'},
 [0xCCD]={d='nsm'},
 [0xCE2]={d='nsm'},
 [0xCE3]={d='nsm'},
 [0xD01]={d='nsm'},
 [0xD41]={d='nsm'},
 [0xD42]={d='nsm'},
 [0xD43]={d='nsm'},
 [0xD44]={d='nsm'},
 [0xD4D]={d='nsm'},
 [0xD62]={d='nsm'},
 [0xD63]={d='nsm'},
 [0xDCA]={d='nsm'},
 [0xDD2]={d='nsm'},
 [0xDD3]={d='nsm'},
 [0xDD4]={d='nsm'},
 [0xDD6]={d='nsm'},
 [0xE31]={d='nsm'},
 [0xE34]={d='nsm'},
 [0xE35]={d='nsm'},
 [0xE36]={d='nsm'},
 [0xE37]={d='nsm'},
 [0xE38]={d='nsm'},
 [0xE39]={d='nsm'},
 [0xE3A]={d='nsm'},
 [0xE3F]={d='et'},
 [0xE47]={d='nsm'},
 [0xE48]={d='nsm'},
 [0xE49]={d='nsm'},
 [0xE4A]={d='nsm'},
 [0xE4B]={d='nsm'},
 [0xE4C]={d='nsm'},
 [0xE4D]={d='nsm'},
 [0xE4E]={d='nsm'},
 [0xEB1]={d='nsm'},
 [0xEB4]={d='nsm'},
 [0xEB5]={d='nsm'},
 [0xEB6]={d='nsm'},
 [0xEB7]={d='nsm'},
 [0xEB8]={d='nsm'},
 [0xEB9]={d='nsm'},
 [0xEBB]={d='nsm'},
 [0xEBC]={d='nsm'},
 [0xEC8]={d='nsm'},
 [0xEC9]={d='nsm'},
 [0xECA]={d='nsm'},
 [0xECB]={d='nsm'},
 [0xECC]={d='nsm'},
 [0xECD]={d='nsm'},
 [0xF18]={d='nsm'},
 [0xF19]={d='nsm'},
 [0xF35]={d='nsm'},
 [0xF37]={d='nsm'},
 [0xF39]={d='nsm'},
 [0xF3A]={d='on', m=0xF3B},
 [0xF3B]={d='on', m=0xF3A},
 [0xF3C]={d='on', m=0xF3D},
 [0xF3D]={d='on', m=0xF3C},
 [0xF71]={d='nsm'},
 [0xF72]={d='nsm'},
 [0xF73]={d='nsm'},
 [0xF74]={d='nsm'},
 [0xF75]={d='nsm'},
 [0xF76]={d='nsm'},
 [0xF77]={d='nsm'},
 [0xF78]={d='nsm'},
 [0xF79]={d='nsm'},
 [0xF7A]={d='nsm'},
 [0xF7B]={d='nsm'},
 [0xF7C]={d='nsm'},
 [0xF7D]={d='nsm'},
 [0xF7E]={d='nsm'},
 [0xF80]={d='nsm'},
 [0xF81]={d='nsm'},
 [0xF82]={d='nsm'},
 [0xF83]={d='nsm'},
 [0xF84]={d='nsm'},
 [0xF86]={d='nsm'},
 [0xF87]={d='nsm'},
 [0xF8D]={d='nsm'},
 [0xF8E]={d='nsm'},
 [0xF8F]={d='nsm'},
 [0xF90]={d='nsm'},
 [0xF91]={d='nsm'},
 [0xF92]={d='nsm'},
 [0xF93]={d='nsm'},
 [0xF94]={d='nsm'},
 [0xF95]={d='nsm'},
 [0xF96]={d='nsm'},
 [0xF97]={d='nsm'},
 [0xF99]={d='nsm'},
 [0xF9A]={d='nsm'},
 [0xF9B]={d='nsm'},
 [0xF9C]={d='nsm'},
 [0xF9D]={d='nsm'},
 [0xF9E]={d='nsm'},
 [0xF9F]={d='nsm'},
 [0xFA0]={d='nsm'},
 [0xFA1]={d='nsm'},
 [0xFA2]={d='nsm'},
 [0xFA3]={d='nsm'},
 [0xFA4]={d='nsm'},
 [0xFA5]={d='nsm'},
 [0xFA6]={d='nsm'},
 [0xFA7]={d='nsm'},
 [0xFA8]={d='nsm'},
 [0xFA9]={d='nsm'},
 [0xFAA]={d='nsm'},
 [0xFAB]={d='nsm'},
 [0xFAC]={d='nsm'},
 [0xFAD]={d='nsm'},
 [0xFAE]={d='nsm'},
 [0xFAF]={d='nsm'},
 [0xFB0]={d='nsm'},
 [0xFB1]={d='nsm'},
 [0xFB2]={d='nsm'},
 [0xFB3]={d='nsm'},
 [0xFB4]={d='nsm'},
 [0xFB5]={d='nsm'},
 [0xFB6]={d='nsm'},
 [0xFB7]={d='nsm'},
 [0xFB8]={d='nsm'},
 [0xFB9]={d='nsm'},
 [0xFBA]={d='nsm'},
 [0xFBB]={d='nsm'},
 [0xFBC]={d='nsm'},
 [0xFC6]={d='nsm'},
 [0x102D]={d='nsm'},
 [0x102E]={d='nsm'},
 [0x102F]={d='nsm'},
 [0x1030]={d='nsm'},
 [0x1032]={d='nsm'},
 [0x1033]={d='nsm'},
 [0x1034]={d='nsm'},
 [0x1035]={d='nsm'},
 [0x1036]={d='nsm'},
 [0x1037]={d='nsm'},
 [0x1039]={d='nsm'},
 [0x103A]={d='nsm'},
 [0x103D]={d='nsm'},
 [0x103E]={d='nsm'},
 [0x1058]={d='nsm'},
 [0x1059]={d='nsm'},
 [0x105E]={d='nsm'},
 [0x105F]={d='nsm'},
 [0x1060]={d='nsm'},
 [0x1071]={d='nsm'},
 [0x1072]={d='nsm'},
 [0x1073]={d='nsm'},
 [0x1074]={d='nsm'},
 [0x1082]={d='nsm'},
 [0x1085]={d='nsm'},
 [0x1086]={d='nsm'},
 [0x108D]={d='nsm'},
 [0x109D]={d='nsm'},
 [0x135D]={d='nsm'},
 [0x135E]={d='nsm'},
 [0x135F]={d='nsm'},
 [0x1390]={d='on'},
 [0x1391]={d='on'},
 [0x1392]={d='on'},
 [0x1393]={d='on'},
 [0x1394]={d='on'},
 [0x1395]={d='on'},
 [0x1396]={d='on'},
 [0x1397]={d='on'},
 [0x1398]={d='on'},
 [0x1399]={d='on'},
 [0x1400]={d='on'},
 [0x1680]={d='ws'},
 [0x169B]={d='on', m=0x169C},
 [0x169C]={d='on', m=0x169B},
 [0x1712]={d='nsm'},
 [0x1713]={d='nsm'},
 [0x1714]={d='nsm'},
 [0x1732]={d='nsm'},
 [0x1733]={d='nsm'},
 [0x1734]={d='nsm'},
 [0x1752]={d='nsm'},
 [0x1753]={d='nsm'},
 [0x1772]={d='nsm'},
 [0x1773]={d='nsm'},
 [0x17B4]={d='nsm'},
 [0x17B5]={d='nsm'},
 [0x17B7]={d='nsm'},
 [0x17B8]={d='nsm'},
 [0x17B9]={d='nsm'},
 [0x17BA]={d='nsm'},
 [0x17BB]={d='nsm'},
 [0x17BC]={d='nsm'},
 [0x17BD]={d='nsm'},
 [0x17C6]={d='nsm'},
 [0x17C9]={d='nsm'},
 [0x17CA]={d='nsm'},
 [0x17CB]={d='nsm'},
 [0x17CC]={d='nsm'},
 [0x17CD]={d='nsm'},
 [0x17CE]={d='nsm'},
 [0x17CF]={d='nsm'},
 [0x17D0]={d='nsm'},
 [0x17D1]={d='nsm'},
 [0x17D2]={d='nsm'},
 [0x17D3]={d='nsm'},
 [0x17DB]={d='et'},
 [0x17DD]={d='nsm'},
 [0x17F0]={d='on'},
 [0x17F1]={d='on'},
 [0x17F2]={d='on'},
 [0x17F3]={d='on'},
 [0x17F4]={d='on'},
 [0x17F5]={d='on'},
 [0x17F6]={d='on'},
 [0x17F7]={d='on'},
 [0x17F8]={d='on'},
 [0x17F9]={d='on'},
 [0x1800]={d='on'},
 [0x1801]={d='on'},
 [0x1802]={d='on'},
 [0x1803]={d='on'},
 [0x1804]={d='on'},
 [0x1805]={d='on'},
 [0x1806]={d='on'},
 [0x1807]={d='on'},
 [0x1808]={d='on'},
 [0x1809]={d='on'},
 [0x180A]={d='on'},
 [0x180B]={d='nsm'},
 [0x180C]={d='nsm'},
 [0x180D]={d='nsm'},
 [0x180E]={d='bn'},
 [0x1885]={d='nsm'},
 [0x1886]={d='nsm'},
 [0x18A9]={d='nsm'},
 [0x1920]={d='nsm'},
 [0x1921]={d='nsm'},
 [0x1922]={d='nsm'},
 [0x1927]={d='nsm'},
 [0x1928]={d='nsm'},
 [0x1932]={d='nsm'},
 [0x1939]={d='nsm'},
 [0x193A]={d='nsm'},
 [0x193B]={d='nsm'},
 [0x1940]={d='on'},
 [0x1944]={d='on'},
 [0x1945]={d='on'},
 [0x19DE]={d='on'},
 [0x19DF]={d='on'},
 [0x19E0]={d='on'},
 [0x19E1]={d='on'},
 [0x19E2]={d='on'},
 [0x19E3]={d='on'},
 [0x19E4]={d='on'},
 [0x19E5]={d='on'},
 [0x19E6]={d='on'},
 [0x19E7]={d='on'},
 [0x19E8]={d='on'},
 [0x19E9]={d='on'},
 [0x19EA]={d='on'},
 [0x19EB]={d='on'},
 [0x19EC]={d='on'},
 [0x19ED]={d='on'},
 [0x19EE]={d='on'},
 [0x19EF]={d='on'},
 [0x19F0]={d='on'},
 [0x19F1]={d='on'},
 [0x19F2]={d='on'},
 [0x19F3]={d='on'},
 [0x19F4]={d='on'},
 [0x19F5]={d='on'},
 [0x19F6]={d='on'},
 [0x19F7]={d='on'},
 [0x19F8]={d='on'},
 [0x19F9]={d='on'},
 [0x19FA]={d='on'},
 [0x19FB]={d='on'},
 [0x19FC]={d='on'},
 [0x19FD]={d='on'},
 [0x19FE]={d='on'},
 [0x19FF]={d='on'},
 [0x1A17]={d='nsm'},
 [0x1A18]={d='nsm'},
 [0x1A1B]={d='nsm'},
 [0x1A56]={d='nsm'},
 [0x1A58]={d='nsm'},
 [0x1A59]={d='nsm'},
 [0x1A5A]={d='nsm'},
 [0x1A5B]={d='nsm'},
 [0x1A5C]={d='nsm'},
 [0x1A5D]={d='nsm'},
 [0x1A5E]={d='nsm'},
 [0x1A60]={d='nsm'},
 [0x1A62]={d='nsm'},
 [0x1A65]={d='nsm'},
 [0x1A66]={d='nsm'},
 [0x1A67]={d='nsm'},
 [0x1A68]={d='nsm'},
 [0x1A69]={d='nsm'},
 [0x1A6A]={d='nsm'},
 [0x1A6B]={d='nsm'},
 [0x1A6C]={d='nsm'},
 [0x1A73]={d='nsm'},
 [0x1A74]={d='nsm'},
 [0x1A75]={d='nsm'},
 [0x1A76]={d='nsm'},
 [0x1A77]={d='nsm'},
 [0x1A78]={d='nsm'},
 [0x1A79]={d='nsm'},
 [0x1A7A]={d='nsm'},
 [0x1A7B]={d='nsm'},
 [0x1A7C]={d='nsm'},
 [0x1A7F]={d='nsm'},
 [0x1AB0]={d='nsm'},
 [0x1AB1]={d='nsm'},
 [0x1AB2]={d='nsm'},
 [0x1AB3]={d='nsm'},
 [0x1AB4]={d='nsm'},
 [0x1AB5]={d='nsm'},
 [0x1AB6]={d='nsm'},
 [0x1AB7]={d='nsm'},
 [0x1AB8]={d='nsm'},
 [0x1AB9]={d='nsm'},
 [0x1ABA]={d='nsm'},
 [0x1ABB]={d='nsm'},
 [0x1ABC]={d='nsm'},
 [0x1ABD]={d='nsm'},
 [0x1ABE]={d='nsm'},
 [0x1B00]={d='nsm'},
 [0x1B01]={d='nsm'},
 [0x1B02]={d='nsm'},
 [0x1B03]={d='nsm'},
 [0x1B34]={d='nsm'},
 [0x1B36]={d='nsm'},
 [0x1B37]={d='nsm'},
 [0x1B38]={d='nsm'},
 [0x1B39]={d='nsm'},
 [0x1B3A]={d='nsm'},
 [0x1B3C]={d='nsm'},
 [0x1B42]={d='nsm'},
 [0x1B6B]={d='nsm'},
 [0x1B6C]={d='nsm'},
 [0x1B6D]={d='nsm'},
 [0x1B6E]={d='nsm'},
 [0x1B6F]={d='nsm'},
 [0x1B70]={d='nsm'},
 [0x1B71]={d='nsm'},
 [0x1B72]={d='nsm'},
 [0x1B73]={d='nsm'},
 [0x1B80]={d='nsm'},
 [0x1B81]={d='nsm'},
 [0x1BA2]={d='nsm'},
 [0x1BA3]={d='nsm'},
 [0x1BA4]={d='nsm'},
 [0x1BA5]={d='nsm'},
 [0x1BA8]={d='nsm'},
 [0x1BA9]={d='nsm'},
 [0x1BAB]={d='nsm'},
 [0x1BAC]={d='nsm'},
 [0x1BAD]={d='nsm'},
 [0x1BE6]={d='nsm'},
 [0x1BE8]={d='nsm'},
 [0x1BE9]={d='nsm'},
 [0x1BED]={d='nsm'},
 [0x1BEF]={d='nsm'},
 [0x1BF0]={d='nsm'},
 [0x1BF1]={d='nsm'},
 [0x1C2C]={d='nsm'},
 [0x1C2D]={d='nsm'},
 [0x1C2E]={d='nsm'},
 [0x1C2F]={d='nsm'},
 [0x1C30]={d='nsm'},
 [0x1C31]={d='nsm'},
 [0x1C32]={d='nsm'},
 [0x1C33]={d='nsm'},
 [0x1C36]={d='nsm'},
 [0x1C37]={d='nsm'},
 [0x1CD0]={d='nsm'},
 [0x1CD1]={d='nsm'},
 [0x1CD2]={d='nsm'},
 [0x1CD4]={d='nsm'},
 [0x1CD5]={d='nsm'},
 [0x1CD6]={d='nsm'},
 [0x1CD7]={d='nsm'},
 [0x1CD8]={d='nsm'},
 [0x1CD9]={d='nsm'},
 [0x1CDA]={d='nsm'},
 [0x1CDB]={d='nsm'},
 [0x1CDC]={d='nsm'},
 [0x1CDD]={d='nsm'},
 [0x1CDE]={d='nsm'},
 [0x1CDF]={d='nsm'},
 [0x1CE0]={d='nsm'},
 [0x1CE2]={d='nsm'},
 [0x1CE3]={d='nsm'},
 [0x1CE4]={d='nsm'},
 [0x1CE5]={d='nsm'},
 [0x1CE6]={d='nsm'},
 [0x1CE7]={d='nsm'},
 [0x1CE8]={d='nsm'},
 [0x1CED]={d='nsm'},
 [0x1CF4]={d='nsm'},
 [0x1CF8]={d='nsm'},
 [0x1CF9]={d='nsm'},
 [0x1DC0]={d='nsm'},
 [0x1DC1]={d='nsm'},
 [0x1DC2]={d='nsm'},
 [0x1DC3]={d='nsm'},
 [0x1DC4]={d='nsm'},
 [0x1DC5]={d='nsm'},
 [0x1DC6]={d='nsm'},
 [0x1DC7]={d='nsm'},
 [0x1DC8]={d='nsm'},
 [0x1DC9]={d='nsm'},
 [0x1DCA]={d='nsm'},
 [0x1DCB]={d='nsm'},
 [0x1DCC]={d='nsm'},
 [0x1DCD]={d='nsm'},
 [0x1DCE]={d='nsm'},
 [0x1DCF]={d='nsm'},
 [0x1DD0]={d='nsm'},
 [0x1DD1]={d='nsm'},
 [0x1DD2]={d='nsm'},
 [0x1DD3]={d='nsm'},
 [0x1DD4]={d='nsm'},
 [0x1DD5]={d='nsm'},
 [0x1DD6]={d='nsm'},
 [0x1DD7]={d='nsm'},
 [0x1DD8]={d='nsm'},
 [0x1DD9]={d='nsm'},
 [0x1DDA]={d='nsm'},
 [0x1DDB]={d='nsm'},
 [0x1DDC]={d='nsm'},
 [0x1DDD]={d='nsm'},
 [0x1DDE]={d='nsm'},
 [0x1DDF]={d='nsm'},
 [0x1DE0]={d='nsm'},
 [0x1DE1]={d='nsm'},
 [0x1DE2]={d='nsm'},
 [0x1DE3]={d='nsm'},
 [0x1DE4]={d='nsm'},
 [0x1DE5]={d='nsm'},
 [0x1DE6]={d='nsm'},
 [0x1DE7]={d='nsm'},
 [0x1DE8]={d='nsm'},
 [0x1DE9]={d='nsm'},
 [0x1DEA]={d='nsm'},
 [0x1DEB]={d='nsm'},
 [0x1DEC]={d='nsm'},
 [0x1DED]={d='nsm'},
 [0x1DEE]={d='nsm'},
 [0x1DEF]={d='nsm'},
 [0x1DF0]={d='nsm'},
 [0x1DF1]={d='nsm'},
 [0x1DF2]={d='nsm'},
 [0x1DF3]={d='nsm'},
 [0x1DF4]={d='nsm'},
 [0x1DF5]={d='nsm'},
 [0x1DFB]={d='nsm'},
 [0x1DFC]={d='nsm'},
 [0x1DFD]={d='nsm'},
 [0x1DFE]={d='nsm'},
 [0x1DFF]={d='nsm'},
 [0x1FBD]={d='on'},
 [0x1FBF]={d='on'},
 [0x1FC0]={d='on'},
 [0x1FC1]={d='on'},
 [0x1FCD]={d='on'},
 [0x1FCE]={d='on'},
 [0x1FCF]={d='on'},
 [0x1FDD]={d='on'},
 [0x1FDE]={d='on'},
 [0x1FDF]={d='on'},
 [0x1FED]={d='on'},
 [0x1FEE]={d='on'},
 [0x1FEF]={d='on'},
 [0x1FFD]={d='on'},
 [0x1FFE]={d='on'},
 [0x2000]={d='ws'},
 [0x2001]={d='ws'},
 [0x2002]={d='ws'},
 [0x2003]={d='ws'},
 [0x2004]={d='ws'},
 [0x2005]={d='ws'},
 [0x2006]={d='ws'},
 [0x2007]={d='ws'},
 [0x2008]={d='ws'},
 [0x2009]={d='ws'},
 [0x200A]={d='ws'},
 [0x200B]={d='bn'},
 [0x200C]={d='bn'},
 [0x200D]={d='bn'},
 [0x200F]={d='r'},
 [0x2010]={d='on'},
 [0x2011]={d='on'},
 [0x2012]={d='on'},
 [0x2013]={d='on'},
 [0x2014]={d='on'},
 [0x2015]={d='on'},
 [0x2016]={d='on'},
 [0x2017]={d='on'},
 [0x2018]={d='on'},
 [0x2019]={d='on'},
 [0x201A]={d='on'},
 [0x201B]={d='on'},
 [0x201C]={d='on'},
 [0x201D]={d='on'},
 [0x201E]={d='on'},
 [0x201F]={d='on'},
 [0x2020]={d='on'},
 [0x2021]={d='on'},
 [0x2022]={d='on'},
 [0x2023]={d='on'},
 [0x2024]={d='on'},
 [0x2025]={d='on'},
 [0x2026]={d='on'},
 [0x2027]={d='on'},
 [0x2028]={d='ws'},
 [0x2029]={d='b'},
 [0x202A]={d='lre'},
 [0x202B]={d='rle'},
 [0x202C]={d='pdf'},
 [0x202D]={d='lro'},
 [0x202E]={d='rlo'},
 [0x202F]={d='cs'},
 [0x2030]={d='et'},
 [0x2031]={d='et'},
 [0x2032]={d='et'},
 [0x2033]={d='et'},
 [0x2034]={d='et'},
 [0x2035]={d='on'},
 [0x2036]={d='on'},
 [0x2037]={d='on'},
 [0x2038]={d='on'},
 [0x2039]={d='on', m=0x203A},
 [0x203A]={d='on', m=0x2039},
 [0x203B]={d='on'},
 [0x203C]={d='on'},
 [0x203D]={d='on'},
 [0x203E]={d='on'},
 [0x203F]={d='on'},
 [0x2040]={d='on'},
 [0x2041]={d='on'},
 [0x2042]={d='on'},
 [0x2043]={d='on'},
 [0x2044]={d='cs'},
 [0x2045]={d='on', m=0x2046},
 [0x2046]={d='on', m=0x2045},
 [0x2047]={d='on'},
 [0x2048]={d='on'},
 [0x2049]={d='on'},
 [0x204A]={d='on'},
 [0x204B]={d='on'},
 [0x204C]={d='on'},
 [0x204D]={d='on'},
 [0x204E]={d='on'},
 [0x204F]={d='on'},
 [0x2050]={d='on'},
 [0x2051]={d='on'},
 [0x2052]={d='on'},
 [0x2053]={d='on'},
 [0x2054]={d='on'},
 [0x2055]={d='on'},
 [0x2056]={d='on'},
 [0x2057]={d='on'},
 [0x2058]={d='on'},
 [0x2059]={d='on'},
 [0x205A]={d='on'},
 [0x205B]={d='on'},
 [0x205C]={d='on'},
 [0x205D]={d='on'},
 [0x205E]={d='on'},
 [0x205F]={d='ws'},
 [0x2060]={d='bn'},
 [0x2061]={d='bn'},
 [0x2062]={d='bn'},
 [0x2063]={d='bn'},
 [0x2064]={d='bn'},
 [0x2066]={d='lri'},
 [0x2067]={d='rli'},
 [0x2068]={d='fsi'},
 [0x2069]={d='pdi'},
 [0x206A]={d='bn'},
 [0x206B]={d='bn'},
 [0x206C]={d='bn'},
 [0x206D]={d='bn'},
 [0x206E]={d='bn'},
 [0x206F]={d='bn'},
 [0x2070]={d='en'},
 [0x2074]={d='en'},
 [0x2075]={d='en'},
 [0x2076]={d='en'},
 [0x2077]={d='en'},
 [0x2078]={d='en'},
 [0x2079]={d='en'},
 [0x207A]={d='es'},
 [0x207B]={d='es'},
 [0x207C]={d='on'},
 [0x207D]={d='on', m=0x207E},
 [0x207E]={d='on', m=0x207D},
 [0x2080]={d='en'},
 [0x2081]={d='en'},
 [0x2082]={d='en'},
 [0x2083]={d='en'},
 [0x2084]={d='en'},
 [0x2085]={d='en'},
 [0x2086]={d='en'},
 [0x2087]={d='en'},
 [0x2088]={d='en'},
 [0x2089]={d='en'},
 [0x208A]={d='es'},
 [0x208B]={d='es'},
 [0x208C]={d='on'},
 [0x208D]={d='on', m=0x208E},
 [0x208E]={d='on', m=0x208D},
 [0x20A0]={d='et'},
 [0x20A1]={d='et'},
 [0x20A2]={d='et'},
 [0x20A3]={d='et'},
 [0x20A4]={d='et'},
 [0x20A5]={d='et'},
 [0x20A6]={d='et'},
 [0x20A7]={d='et'},
 [0x20A8]={d='et'},
 [0x20A9]={d='et'},
 [0x20AA]={d='et'},
 [0x20AB]={d='et'},
 [0x20AC]={d='et'},
 [0x20AD]={d='et'},
 [0x20AE]={d='et'},
 [0x20AF]={d='et'},
 [0x20B0]={d='et'},
 [0x20B1]={d='et'},
 [0x20B2]={d='et'},
 [0x20B3]={d='et'},
 [0x20B4]={d='et'},
 [0x20B5]={d='et'},
 [0x20B6]={d='et'},
 [0x20B7]={d='et'},
 [0x20B8]={d='et'},
 [0x20B9]={d='et'},
 [0x20BA]={d='et'},
 [0x20BB]={d='et'},
 [0x20BC]={d='et'},
 [0x20BD]={d='et'},
 [0x20BE]={d='et'},
 [0x20D0]={d='nsm'},
 [0x20D1]={d='nsm'},
 [0x20D2]={d='nsm'},
 [0x20D3]={d='nsm'},
 [0x20D4]={d='nsm'},
 [0x20D5]={d='nsm'},
 [0x20D6]={d='nsm'},
 [0x20D7]={d='nsm'},
 [0x20D8]={d='nsm'},
 [0x20D9]={d='nsm'},
 [0x20DA]={d='nsm'},
 [0x20DB]={d='nsm'},
 [0x20DC]={d='nsm'},
 [0x20DD]={d='nsm'},
 [0x20DE]={d='nsm'},
 [0x20DF]={d='nsm'},
 [0x20E0]={d='nsm'},
 [0x20E1]={d='nsm'},
 [0x20E2]={d='nsm'},
 [0x20E3]={d='nsm'},
 [0x20E4]={d='nsm'},
 [0x20E5]={d='nsm'},
 [0x20E6]={d='nsm'},
 [0x20E7]={d='nsm'},
 [0x20E8]={d='nsm'},
 [0x20E9]={d='nsm'},
 [0x20EA]={d='nsm'},
 [0x20EB]={d='nsm'},
 [0x20EC]={d='nsm'},
 [0x20ED]={d='nsm'},
 [0x20EE]={d='nsm'},
 [0x20EF]={d='nsm'},
 [0x20F0]={d='nsm'},
 [0x2100]={d='on'},
 [0x2101]={d='on'},
 [0x2103]={d='on'},
 [0x2104]={d='on'},
 [0x2105]={d='on'},
 [0x2106]={d='on'},
 [0x2108]={d='on'},
 [0x2109]={d='on'},
 [0x2114]={d='on'},
 [0x2116]={d='on'},
 [0x2117]={d='on'},
 [0x2118]={d='on'},
 [0x211E]={d='on'},
 [0x211F]={d='on'},
 [0x2120]={d='on'},
 [0x2121]={d='on'},
 [0x2122]={d='on'},
 [0x2123]={d='on'},
 [0x2125]={d='on'},
 [0x2127]={d='on'},
 [0x2129]={d='on'},
 [0x212E]={d='et'},
 [0x213A]={d='on'},
 [0x213B]={d='on'},
 [0x2140]={d='on'},
 [0x2141]={d='on'},
 [0x2142]={d='on'},
 [0x2143]={d='on'},
 [0x2144]={d='on'},
 [0x214A]={d='on'},
 [0x214B]={d='on'},
 [0x214C]={d='on'},
 [0x214D]={d='on'},
 [0x2150]={d='on'},
 [0x2151]={d='on'},
 [0x2152]={d='on'},
 [0x2153]={d='on'},
 [0x2154]={d='on'},
 [0x2155]={d='on'},
 [0x2156]={d='on'},
 [0x2157]={d='on'},
 [0x2158]={d='on'},
 [0x2159]={d='on'},
 [0x215A]={d='on'},
 [0x215B]={d='on'},
 [0x215C]={d='on'},
 [0x215D]={d='on'},
 [0x215E]={d='on'},
 [0x215F]={d='on'},
 [0x2189]={d='on'},
 [0x218A]={d='on'},
 [0x218B]={d='on'},
 [0x2190]={d='on'},
 [0x2191]={d='on'},
 [0x2192]={d='on'},
 [0x2193]={d='on'},
 [0x2194]={d='on'},
 [0x2195]={d='on'},
 [0x2196]={d='on'},
 [0x2197]={d='on'},
 [0x2198]={d='on'},
 [0x2199]={d='on'},
 [0x219A]={d='on'},
 [0x219B]={d='on'},
 [0x219C]={d='on'},
 [0x219D]={d='on'},
 [0x219E]={d='on'},
 [0x219F]={d='on'},
 [0x21A0]={d='on'},
 [0x21A1]={d='on'},
 [0x21A2]={d='on'},
 [0x21A3]={d='on'},
 [0x21A4]={d='on'},
 [0x21A5]={d='on'},
 [0x21A6]={d='on'},
 [0x21A7]={d='on'},
 [0x21A8]={d='on'},
 [0x21A9]={d='on'},
 [0x21AA]={d='on'},
 [0x21AB]={d='on'},
 [0x21AC]={d='on'},
 [0x21AD]={d='on'},
 [0x21AE]={d='on'},
 [0x21AF]={d='on'},
 [0x21B0]={d='on'},
 [0x21B1]={d='on'},
 [0x21B2]={d='on'},
 [0x21B3]={d='on'},
 [0x21B4]={d='on'},
 [0x21B5]={d='on'},
 [0x21B6]={d='on'},
 [0x21B7]={d='on'},
 [0x21B8]={d='on'},
 [0x21B9]={d='on'},
 [0x21BA]={d='on'},
 [0x21BB]={d='on'},
 [0x21BC]={d='on'},
 [0x21BD]={d='on'},
 [0x21BE]={d='on'},
 [0x21BF]={d='on'},
 [0x21C0]={d='on'},
 [0x21C1]={d='on'},
 [0x21C2]={d='on'},
 [0x21C3]={d='on'},
 [0x21C4]={d='on'},
 [0x21C5]={d='on'},
 [0x21C6]={d='on'},
 [0x21C7]={d='on'},
 [0x21C8]={d='on'},
 [0x21C9]={d='on'},
 [0x21CA]={d='on'},
 [0x21CB]={d='on'},
 [0x21CC]={d='on'},
 [0x21CD]={d='on'},
 [0x21CE]={d='on'},
 [0x21CF]={d='on'},
 [0x21D0]={d='on'},
 [0x21D1]={d='on'},
 [0x21D2]={d='on'},
 [0x21D3]={d='on'},
 [0x21D4]={d='on'},
 [0x21D5]={d='on'},
 [0x21D6]={d='on'},
 [0x21D7]={d='on'},
 [0x21D8]={d='on'},
 [0x21D9]={d='on'},
 [0x21DA]={d='on'},
 [0x21DB]={d='on'},
 [0x21DC]={d='on'},
 [0x21DD]={d='on'},
 [0x21DE]={d='on'},
 [0x21DF]={d='on'},
 [0x21E0]={d='on'},
 [0x21E1]={d='on'},
 [0x21E2]={d='on'},
 [0x21E3]={d='on'},
 [0x21E4]={d='on'},
 [0x21E5]={d='on'},
 [0x21E6]={d='on'},
 [0x21E7]={d='on'},
 [0x21E8]={d='on'},
 [0x21E9]={d='on'},
 [0x21EA]={d='on'},
 [0x21EB]={d='on'},
 [0x21EC]={d='on'},
 [0x21ED]={d='on'},
 [0x21EE]={d='on'},
 [0x21EF]={d='on'},
 [0x21F0]={d='on'},
 [0x21F1]={d='on'},
 [0x21F2]={d='on'},
 [0x21F3]={d='on'},
 [0x21F4]={d='on'},
 [0x21F5]={d='on'},
 [0x21F6]={d='on'},
 [0x21F7]={d='on'},
 [0x21F8]={d='on'},
 [0x21F9]={d='on'},
 [0x21FA]={d='on'},
 [0x21FB]={d='on'},
 [0x21FC]={d='on'},
 [0x21FD]={d='on'},
 [0x21FE]={d='on'},
 [0x21FF]={d='on'},
 [0x2200]={d='on'},
 [0x2201]={d='on'},
 [0x2202]={d='on'},
 [0x2203]={d='on'},
 [0x2204]={d='on'},
 [0x2205]={d='on'},
 [0x2206]={d='on'},
 [0x2207]={d='on'},
 [0x2208]={d='on', m=0x220B},
 [0x2209]={d='on', m=0x220C},
 [0x220A]={d='on', m=0x220D},
 [0x220B]={d='on', m=0x2208},
 [0x220C]={d='on', m=0x2209},
 [0x220D]={d='on', m=0x220A},
 [0x220E]={d='on'},
 [0x220F]={d='on'},
 [0x2210]={d='on'},
 [0x2211]={d='on'},
 [0x2212]={d='es'},
 [0x2213]={d='et'},
 [0x2214]={d='on'},
 [0x2215]={d='on', m=0x29F5},
 [0x2216]={d='on'},
 [0x2217]={d='on'},
 [0x2218]={d='on'},
 [0x2219]={d='on'},
 [0x221A]={d='on'},
 [0x221B]={d='on'},
 [0x221C]={d='on'},
 [0x221D]={d='on'},
 [0x221E]={d='on'},
 [0x221F]={d='on'},
 [0x2220]={d='on'},
 [0x2221]={d='on'},
 [0x2222]={d='on'},
 [0x2223]={d='on'},
 [0x2224]={d='on'},
 [0x2225]={d='on'},
 [0x2226]={d='on'},
 [0x2227]={d='on'},
 [0x2228]={d='on'},
 [0x2229]={d='on'},
 [0x222A]={d='on'},
 [0x222B]={d='on'},
 [0x222C]={d='on'},
 [0x222D]={d='on'},
 [0x222E]={d='on'},
 [0x222F]={d='on'},
 [0x2230]={d='on'},
 [0x2231]={d='on'},
 [0x2232]={d='on'},
 [0x2233]={d='on'},
 [0x2234]={d='on'},
 [0x2235]={d='on'},
 [0x2236]={d='on'},
 [0x2237]={d='on'},
 [0x2238]={d='on'},
 [0x2239]={d='on'},
 [0x223A]={d='on'},
 [0x223B]={d='on'},
 [0x223C]={d='on', m=0x223D},
 [0x223D]={d='on', m=0x223C},
 [0x223E]={d='on'},
 [0x223F]={d='on'},
 [0x2240]={d='on'},
 [0x2241]={d='on'},
 [0x2242]={d='on'},
 [0x2243]={d='on', m=0x22CD},
 [0x2244]={d='on'},
 [0x2245]={d='on'},
 [0x2246]={d='on'},
 [0x2247]={d='on'},
 [0x2248]={d='on'},
 [0x2249]={d='on'},
 [0x224A]={d='on'},
 [0x224B]={d='on'},
 [0x224C]={d='on'},
 [0x224D]={d='on'},
 [0x224E]={d='on'},
 [0x224F]={d='on'},
 [0x2250]={d='on'},
 [0x2251]={d='on'},
 [0x2252]={d='on', m=0x2253},
 [0x2253]={d='on', m=0x2252},
 [0x2254]={d='on', m=0x2255},
 [0x2255]={d='on', m=0x2254},
 [0x2256]={d='on'},
 [0x2257]={d='on'},
 [0x2258]={d='on'},
 [0x2259]={d='on'},
 [0x225A]={d='on'},
 [0x225B]={d='on'},
 [0x225C]={d='on'},
 [0x225D]={d='on'},
 [0x225E]={d='on'},
 [0x225F]={d='on'},
 [0x2260]={d='on'},
 [0x2261]={d='on'},
 [0x2262]={d='on'},
 [0x2263]={d='on'},
 [0x2264]={d='on', m=0x2265},
 [0x2265]={d='on', m=0x2264},
 [0x2266]={d='on', m=0x2267},
 [0x2267]={d='on', m=0x2266},
 [0x2268]={d='on', m=0x2269},
 [0x2269]={d='on', m=0x2268},
 [0x226A]={d='on', m=0x226B},
 [0x226B]={d='on', m=0x226A},
 [0x226C]={d='on'},
 [0x226D]={d='on'},
 [0x226E]={d='on', m=0x226F},
 [0x226F]={d='on', m=0x226E},
 [0x2270]={d='on', m=0x2271},
 [0x2271]={d='on', m=0x2270},
 [0x2272]={d='on', m=0x2273},
 [0x2273]={d='on', m=0x2272},
 [0x2274]={d='on', m=0x2275},
 [0x2275]={d='on', m=0x2274},
 [0x2276]={d='on', m=0x2277},
 [0x2277]={d='on', m=0x2276},
 [0x2278]={d='on', m=0x2279},
 [0x2279]={d='on', m=0x2278},
 [0x227A]={d='on', m=0x227B},
 [0x227B]={d='on', m=0x227A},
 [0x227C]={d='on', m=0x227D},
 [0x227D]={d='on', m=0x227C},
 [0x227E]={d='on', m=0x227F},
 [0x227F]={d='on', m=0x227E},
 [0x2280]={d='on', m=0x2281},
 [0x2281]={d='on', m=0x2280},
 [0x2282]={d='on', m=0x2283},
 [0x2283]={d='on', m=0x2282},
 [0x2284]={d='on', m=0x2285},
 [0x2285]={d='on', m=0x2284},
 [0x2286]={d='on', m=0x2287},
 [0x2287]={d='on', m=0x2286},
 [0x2288]={d='on', m=0x2289},
 [0x2289]={d='on', m=0x2288},
 [0x228A]={d='on', m=0x228B},
 [0x228B]={d='on', m=0x228A},
 [0x228C]={d='on'},
 [0x228D]={d='on'},
 [0x228E]={d='on'},
 [0x228F]={d='on', m=0x2290},
 [0x2290]={d='on', m=0x228F},
 [0x2291]={d='on', m=0x2292},
 [0x2292]={d='on', m=0x2291},
 [0x2293]={d='on'},
 [0x2294]={d='on'},
 [0x2295]={d='on'},
 [0x2296]={d='on'},
 [0x2297]={d='on'},
 [0x2298]={d='on', m=0x29B8},
 [0x2299]={d='on'},
 [0x229A]={d='on'},
 [0x229B]={d='on'},
 [0x229C]={d='on'},
 [0x229D]={d='on'},
 [0x229E]={d='on'},
 [0x229F]={d='on'},
 [0x22A0]={d='on'},
 [0x22A1]={d='on'},
 [0x22A2]={d='on', m=0x22A3},
 [0x22A3]={d='on', m=0x22A2},
 [0x22A4]={d='on'},
 [0x22A5]={d='on'},
 [0x22A6]={d='on', m=0x2ADE},
 [0x22A7]={d='on'},
 [0x22A8]={d='on', m=0x2AE4},
 [0x22A9]={d='on', m=0x2AE3},
 [0x22AA]={d='on'},
 [0x22AB]={d='on', m=0x2AE5},
 [0x22AC]={d='on'},
 [0x22AD]={d='on'},
 [0x22AE]={d='on'},
 [0x22AF]={d='on'},
 [0x22B0]={d='on', m=0x22B1},
 [0x22B1]={d='on', m=0x22B0},
 [0x22B2]={d='on', m=0x22B3},
 [0x22B3]={d='on', m=0x22B2},
 [0x22B4]={d='on', m=0x22B5},
 [0x22B5]={d='on', m=0x22B4},
 [0x22B6]={d='on', m=0x22B7},
 [0x22B7]={d='on', m=0x22B6},
 [0x22B8]={d='on'},
 [0x22B9]={d='on'},
 [0x22BA]={d='on'},
 [0x22BB]={d='on'},
 [0x22BC]={d='on'},
 [0x22BD]={d='on'},
 [0x22BE]={d='on'},
 [0x22BF]={d='on'},
 [0x22C0]={d='on'},
 [0x22C1]={d='on'},
 [0x22C2]={d='on'},
 [0x22C3]={d='on'},
 [0x22C4]={d='on'},
 [0x22C5]={d='on'},
 [0x22C6]={d='on'},
 [0x22C7]={d='on'},
 [0x22C8]={d='on'},
 [0x22C9]={d='on', m=0x22CA},
 [0x22CA]={d='on', m=0x22C9},
 [0x22CB]={d='on', m=0x22CC},
 [0x22CC]={d='on', m=0x22CB},
 [0x22CD]={d='on', m=0x2243},
 [0x22CE]={d='on'},
 [0x22CF]={d='on'},
 [0x22D0]={d='on', m=0x22D1},
 [0x22D1]={d='on', m=0x22D0},
 [0x22D2]={d='on'},
 [0x22D3]={d='on'},
 [0x22D4]={d='on'},
 [0x22D5]={d='on'},
 [0x22D6]={d='on', m=0x22D7},
 [0x22D7]={d='on', m=0x22D6},
 [0x22D8]={d='on', m=0x22D9},
 [0x22D9]={d='on', m=0x22D8},
 [0x22DA]={d='on', m=0x22DB},
 [0x22DB]={d='on', m=0x22DA},
 [0x22DC]={d='on', m=0x22DD},
 [0x22DD]={d='on', m=0x22DC},
 [0x22DE]={d='on', m=0x22DF},
 [0x22DF]={d='on', m=0x22DE},
 [0x22E0]={d='on', m=0x22E1},
 [0x22E1]={d='on', m=0x22E0},
 [0x22E2]={d='on', m=0x22E3},
 [0x22E3]={d='on', m=0x22E2},
 [0x22E4]={d='on', m=0x22E5},
 [0x22E5]={d='on', m=0x22E4},
 [0x22E6]={d='on', m=0x22E7},
 [0x22E7]={d='on', m=0x22E6},
 [0x22E8]={d='on', m=0x22E9},
 [0x22E9]={d='on', m=0x22E8},
 [0x22EA]={d='on', m=0x22EB},
 [0x22EB]={d='on', m=0x22EA},
 [0x22EC]={d='on', m=0x22ED},
 [0x22ED]={d='on', m=0x22EC},
 [0x22EE]={d='on'},
 [0x22EF]={d='on'},
 [0x22F0]={d='on', m=0x22F1},
 [0x22F1]={d='on', m=0x22F0},
 [0x22F2]={d='on', m=0x22FA},
 [0x22F3]={d='on', m=0x22FB},
 [0x22F4]={d='on', m=0x22FC},
 [0x22F5]={d='on'},
 [0x22F6]={d='on', m=0x22FD},
 [0x22F7]={d='on', m=0x22FE},
 [0x22F8]={d='on'},
 [0x22F9]={d='on'},
 [0x22FA]={d='on', m=0x22F2},
 [0x22FB]={d='on', m=0x22F3},
 [0x22FC]={d='on', m=0x22F4},
 [0x22FD]={d='on', m=0x22F6},
 [0x22FE]={d='on', m=0x22F7},
 [0x22FF]={d='on'},
 [0x2300]={d='on'},
 [0x2301]={d='on'},
 [0x2302]={d='on'},
 [0x2303]={d='on'},
 [0x2304]={d='on'},
 [0x2305]={d='on'},
 [0x2306]={d='on'},
 [0x2307]={d='on'},
 [0x2308]={d='on', m=0x2309},
 [0x2309]={d='on', m=0x2308},
 [0x230A]={d='on', m=0x230B},
 [0x230B]={d='on', m=0x230A},
 [0x230C]={d='on'},
 [0x230D]={d='on'},
 [0x230E]={d='on'},
 [0x230F]={d='on'},
 [0x2310]={d='on'},
 [0x2311]={d='on'},
 [0x2312]={d='on'},
 [0x2313]={d='on'},
 [0x2314]={d='on'},
 [0x2315]={d='on'},
 [0x2316]={d='on'},
 [0x2317]={d='on'},
 [0x2318]={d='on'},
 [0x2319]={d='on'},
 [0x231A]={d='on'},
 [0x231B]={d='on'},
 [0x231C]={d='on'},
 [0x231D]={d='on'},
 [0x231E]={d='on'},
 [0x231F]={d='on'},
 [0x2320]={d='on'},
 [0x2321]={d='on'},
 [0x2322]={d='on'},
 [0x2323]={d='on'},
 [0x2324]={d='on'},
 [0x2325]={d='on'},
 [0x2326]={d='on'},
 [0x2327]={d='on'},
 [0x2328]={d='on'},
 [0x2329]={d='on', m=0x232A},
 [0x232A]={d='on', m=0x2329},
 [0x232B]={d='on'},
 [0x232C]={d='on'},
 [0x232D]={d='on'},
 [0x232E]={d='on'},
 [0x232F]={d='on'},
 [0x2330]={d='on'},
 [0x2331]={d='on'},
 [0x2332]={d='on'},
 [0x2333]={d='on'},
 [0x2334]={d='on'},
 [0x2335]={d='on'},
 [0x237B]={d='on'},
 [0x237C]={d='on'},
 [0x237D]={d='on'},
 [0x237E]={d='on'},
 [0x237F]={d='on'},
 [0x2380]={d='on'},
 [0x2381]={d='on'},
 [0x2382]={d='on'},
 [0x2383]={d='on'},
 [0x2384]={d='on'},
 [0x2385]={d='on'},
 [0x2386]={d='on'},
 [0x2387]={d='on'},
 [0x2388]={d='on'},
 [0x2389]={d='on'},
 [0x238A]={d='on'},
 [0x238B]={d='on'},
 [0x238C]={d='on'},
 [0x238D]={d='on'},
 [0x238E]={d='on'},
 [0x238F]={d='on'},
 [0x2390]={d='on'},
 [0x2391]={d='on'},
 [0x2392]={d='on'},
 [0x2393]={d='on'},
 [0x2394]={d='on'},
 [0x2396]={d='on'},
 [0x2397]={d='on'},
 [0x2398]={d='on'},
 [0x2399]={d='on'},
 [0x239A]={d='on'},
 [0x239B]={d='on'},
 [0x239C]={d='on'},
 [0x239D]={d='on'},
 [0x239E]={d='on'},
 [0x239F]={d='on'},
 [0x23A0]={d='on'},
 [0x23A1]={d='on'},
 [0x23A2]={d='on'},
 [0x23A3]={d='on'},
 [0x23A4]={d='on'},
 [0x23A5]={d='on'},
 [0x23A6]={d='on'},
 [0x23A7]={d='on'},
 [0x23A8]={d='on'},
 [0x23A9]={d='on'},
 [0x23AA]={d='on'},
 [0x23AB]={d='on'},
 [0x23AC]={d='on'},
 [0x23AD]={d='on'},
 [0x23AE]={d='on'},
 [0x23AF]={d='on'},
 [0x23B0]={d='on'},
 [0x23B1]={d='on'},
 [0x23B2]={d='on'},
 [0x23B3]={d='on'},
 [0x23B4]={d='on'},
 [0x23B5]={d='on'},
 [0x23B6]={d='on'},
 [0x23B7]={d='on'},
 [0x23B8]={d='on'},
 [0x23B9]={d='on'},
 [0x23BA]={d='on'},
 [0x23BB]={d='on'},
 [0x23BC]={d='on'},
 [0x23BD]={d='on'},
 [0x23BE]={d='on'},
 [0x23BF]={d='on'},
 [0x23C0]={d='on'},
 [0x23C1]={d='on'},
 [0x23C2]={d='on'},
 [0x23C3]={d='on'},
 [0x23C4]={d='on'},
 [0x23C5]={d='on'},
 [0x23C6]={d='on'},
 [0x23C7]={d='on'},
 [0x23C8]={d='on'},
 [0x23C9]={d='on'},
 [0x23CA]={d='on'},
 [0x23CB]={d='on'},
 [0x23CC]={d='on'},
 [0x23CD]={d='on'},
 [0x23CE]={d='on'},
 [0x23CF]={d='on'},
 [0x23D0]={d='on'},
 [0x23D1]={d='on'},
 [0x23D2]={d='on'},
 [0x23D3]={d='on'},
 [0x23D4]={d='on'},
 [0x23D5]={d='on'},
 [0x23D6]={d='on'},
 [0x23D7]={d='on'},
 [0x23D8]={d='on'},
 [0x23D9]={d='on'},
 [0x23DA]={d='on'},
 [0x23DB]={d='on'},
 [0x23DC]={d='on'},
 [0x23DD]={d='on'},
 [0x23DE]={d='on'},
 [0x23DF]={d='on'},
 [0x23E0]={d='on'},
 [0x23E1]={d='on'},
 [0x23E2]={d='on'},
 [0x23E3]={d='on'},
 [0x23E4]={d='on'},
 [0x23E5]={d='on'},
 [0x23E6]={d='on'},
 [0x23E7]={d='on'},
 [0x23E8]={d='on'},
 [0x23E9]={d='on'},
 [0x23EA]={d='on'},
 [0x23EB]={d='on'},
 [0x23EC]={d='on'},
 [0x23ED]={d='on'},
 [0x23EE]={d='on'},
 [0x23EF]={d='on'},
 [0x23F0]={d='on'},
 [0x23F1]={d='on'},
 [0x23F2]={d='on'},
 [0x23F3]={d='on'},
 [0x23F4]={d='on'},
 [0x23F5]={d='on'},
 [0x23F6]={d='on'},
 [0x23F7]={d='on'},
 [0x23F8]={d='on'},
 [0x23F9]={d='on'},
 [0x23FA]={d='on'},
 [0x23FB]={d='on'},
 [0x23FC]={d='on'},
 [0x23FD]={d='on'},
 [0x23FE]={d='on'},
 [0x2400]={d='on'},
 [0x2401]={d='on'},
 [0x2402]={d='on'},
 [0x2403]={d='on'},
 [0x2404]={d='on'},
 [0x2405]={d='on'},
 [0x2406]={d='on'},
 [0x2407]={d='on'},
 [0x2408]={d='on'},
 [0x2409]={d='on'},
 [0x240A]={d='on'},
 [0x240B]={d='on'},
 [0x240C]={d='on'},
 [0x240D]={d='on'},
 [0x240E]={d='on'},
 [0x240F]={d='on'},
 [0x2410]={d='on'},
 [0x2411]={d='on'},
 [0x2412]={d='on'},
 [0x2413]={d='on'},
 [0x2414]={d='on'},
 [0x2415]={d='on'},
 [0x2416]={d='on'},
 [0x2417]={d='on'},
 [0x2418]={d='on'},
 [0x2419]={d='on'},
 [0x241A]={d='on'},
 [0x241B]={d='on'},
 [0x241C]={d='on'},
 [0x241D]={d='on'},
 [0x241E]={d='on'},
 [0x241F]={d='on'},
 [0x2420]={d='on'},
 [0x2421]={d='on'},
 [0x2422]={d='on'},
 [0x2423]={d='on'},
 [0x2424]={d='on'},
 [0x2425]={d='on'},
 [0x2426]={d='on'},
 [0x2440]={d='on'},
 [0x2441]={d='on'},
 [0x2442]={d='on'},
 [0x2443]={d='on'},
 [0x2444]={d='on'},
 [0x2445]={d='on'},
 [0x2446]={d='on'},
 [0x2447]={d='on'},
 [0x2448]={d='on'},
 [0x2449]={d='on'},
 [0x244A]={d='on'},
 [0x2460]={d='on'},
 [0x2461]={d='on'},
 [0x2462]={d='on'},
 [0x2463]={d='on'},
 [0x2464]={d='on'},
 [0x2465]={d='on'},
 [0x2466]={d='on'},
 [0x2467]={d='on'},
 [0x2468]={d='on'},
 [0x2469]={d='on'},
 [0x246A]={d='on'},
 [0x246B]={d='on'},
 [0x246C]={d='on'},
 [0x246D]={d='on'},
 [0x246E]={d='on'},
 [0x246F]={d='on'},
 [0x2470]={d='on'},
 [0x2471]={d='on'},
 [0x2472]={d='on'},
 [0x2473]={d='on'},
 [0x2474]={d='on'},
 [0x2475]={d='on'},
 [0x2476]={d='on'},
 [0x2477]={d='on'},
 [0x2478]={d='on'},
 [0x2479]={d='on'},
 [0x247A]={d='on'},
 [0x247B]={d='on'},
 [0x247C]={d='on'},
 [0x247D]={d='on'},
 [0x247E]={d='on'},
 [0x247F]={d='on'},
 [0x2480]={d='on'},
 [0x2481]={d='on'},
 [0x2482]={d='on'},
 [0x2483]={d='on'},
 [0x2484]={d='on'},
 [0x2485]={d='on'},
 [0x2486]={d='on'},
 [0x2487]={d='on'},
 [0x2488]={d='en'},
 [0x2489]={d='en'},
 [0x248A]={d='en'},
 [0x248B]={d='en'},
 [0x248C]={d='en'},
 [0x248D]={d='en'},
 [0x248E]={d='en'},
 [0x248F]={d='en'},
 [0x2490]={d='en'},
 [0x2491]={d='en'},
 [0x2492]={d='en'},
 [0x2493]={d='en'},
 [0x2494]={d='en'},
 [0x2495]={d='en'},
 [0x2496]={d='en'},
 [0x2497]={d='en'},
 [0x2498]={d='en'},
 [0x2499]={d='en'},
 [0x249A]={d='en'},
 [0x249B]={d='en'},
 [0x24EA]={d='on'},
 [0x24EB]={d='on'},
 [0x24EC]={d='on'},
 [0x24ED]={d='on'},
 [0x24EE]={d='on'},
 [0x24EF]={d='on'},
 [0x24F0]={d='on'},
 [0x24F1]={d='on'},
 [0x24F2]={d='on'},
 [0x24F3]={d='on'},
 [0x24F4]={d='on'},
 [0x24F5]={d='on'},
 [0x24F6]={d='on'},
 [0x24F7]={d='on'},
 [0x24F8]={d='on'},
 [0x24F9]={d='on'},
 [0x24FA]={d='on'},
 [0x24FB]={d='on'},
 [0x24FC]={d='on'},
 [0x24FD]={d='on'},
 [0x24FE]={d='on'},
 [0x24FF]={d='on'},
 [0x2500]={d='on'},
 [0x2501]={d='on'},
 [0x2502]={d='on'},
 [0x2503]={d='on'},
 [0x2504]={d='on'},
 [0x2505]={d='on'},
 [0x2506]={d='on'},
 [0x2507]={d='on'},
 [0x2508]={d='on'},
 [0x2509]={d='on'},
 [0x250A]={d='on'},
 [0x250B]={d='on'},
 [0x250C]={d='on'},
 [0x250D]={d='on'},
 [0x250E]={d='on'},
 [0x250F]={d='on'},
 [0x2510]={d='on'},
 [0x2511]={d='on'},
 [0x2512]={d='on'},
 [0x2513]={d='on'},
 [0x2514]={d='on'},
 [0x2515]={d='on'},
 [0x2516]={d='on'},
 [0x2517]={d='on'},
 [0x2518]={d='on'},
 [0x2519]={d='on'},
 [0x251A]={d='on'},
 [0x251B]={d='on'},
 [0x251C]={d='on'},
 [0x251D]={d='on'},
 [0x251E]={d='on'},
 [0x251F]={d='on'},
 [0x2520]={d='on'},
 [0x2521]={d='on'},
 [0x2522]={d='on'},
 [0x2523]={d='on'},
 [0x2524]={d='on'},
 [0x2525]={d='on'},
 [0x2526]={d='on'},
 [0x2527]={d='on'},
 [0x2528]={d='on'},
 [0x2529]={d='on'},
 [0x252A]={d='on'},
 [0x252B]={d='on'},
 [0x252C]={d='on'},
 [0x252D]={d='on'},
 [0x252E]={d='on'},
 [0x252F]={d='on'},
 [0x2530]={d='on'},
 [0x2531]={d='on'},
 [0x2532]={d='on'},
 [0x2533]={d='on'},
 [0x2534]={d='on'},
 [0x2535]={d='on'},
 [0x2536]={d='on'},
 [0x2537]={d='on'},
 [0x2538]={d='on'},
 [0x2539]={d='on'},
 [0x253A]={d='on'},
 [0x253B]={d='on'},
 [0x253C]={d='on'},
 [0x253D]={d='on'},
 [0x253E]={d='on'},
 [0x253F]={d='on'},
 [0x2540]={d='on'},
 [0x2541]={d='on'},
 [0x2542]={d='on'},
 [0x2543]={d='on'},
 [0x2544]={d='on'},
 [0x2545]={d='on'},
 [0x2546]={d='on'},
 [0x2547]={d='on'},
 [0x2548]={d='on'},
 [0x2549]={d='on'},
 [0x254A]={d='on'},
 [0x254B]={d='on'},
 [0x254C]={d='on'},
 [0x254D]={d='on'},
 [0x254E]={d='on'},
 [0x254F]={d='on'},
 [0x2550]={d='on'},
 [0x2551]={d='on'},
 [0x2552]={d='on'},
 [0x2553]={d='on'},
 [0x2554]={d='on'},
 [0x2555]={d='on'},
 [0x2556]={d='on'},
 [0x2557]={d='on'},
 [0x2558]={d='on'},
 [0x2559]={d='on'},
 [0x255A]={d='on'},
 [0x255B]={d='on'},
 [0x255C]={d='on'},
 [0x255D]={d='on'},
 [0x255E]={d='on'},
 [0x255F]={d='on'},
 [0x2560]={d='on'},
 [0x2561]={d='on'},
 [0x2562]={d='on'},
 [0x2563]={d='on'},
 [0x2564]={d='on'},
 [0x2565]={d='on'},
 [0x2566]={d='on'},
 [0x2567]={d='on'},
 [0x2568]={d='on'},
 [0x2569]={d='on'},
 [0x256A]={d='on'},
 [0x256B]={d='on'},
 [0x256C]={d='on'},
 [0x256D]={d='on'},
 [0x256E]={d='on'},
 [0x256F]={d='on'},
 [0x2570]={d='on'},
 [0x2571]={d='on'},
 [0x2572]={d='on'},
 [0x2573]={d='on'},
 [0x2574]={d='on'},
 [0x2575]={d='on'},
 [0x2576]={d='on'},
 [0x2577]={d='on'},
 [0x2578]={d='on'},
 [0x2579]={d='on'},
 [0x257A]={d='on'},
 [0x257B]={d='on'},
 [0x257C]={d='on'},
 [0x257D]={d='on'},
 [0x257E]={d='on'},
 [0x257F]={d='on'},
 [0x2580]={d='on'},
 [0x2581]={d='on'},
 [0x2582]={d='on'},
 [0x2583]={d='on'},
 [0x2584]={d='on'},
 [0x2585]={d='on'},
 [0x2586]={d='on'},
 [0x2587]={d='on'},
 [0x2588]={d='on'},
 [0x2589]={d='on'},
 [0x258A]={d='on'},
 [0x258B]={d='on'},
 [0x258C]={d='on'},
 [0x258D]={d='on'},
 [0x258E]={d='on'},
 [0x258F]={d='on'},
 [0x2590]={d='on'},
 [0x2591]={d='on'},
 [0x2592]={d='on'},
 [0x2593]={d='on'},
 [0x2594]={d='on'},
 [0x2595]={d='on'},
 [0x2596]={d='on'},
 [0x2597]={d='on'},
 [0x2598]={d='on'},
 [0x2599]={d='on'},
 [0x259A]={d='on'},
 [0x259B]={d='on'},
 [0x259C]={d='on'},
 [0x259D]={d='on'},
 [0x259E]={d='on'},
 [0x259F]={d='on'},
 [0x25A0]={d='on'},
 [0x25A1]={d='on'},
 [0x25A2]={d='on'},
 [0x25A3]={d='on'},
 [0x25A4]={d='on'},
 [0x25A5]={d='on'},
 [0x25A6]={d='on'},
 [0x25A7]={d='on'},
 [0x25A8]={d='on'},
 [0x25A9]={d='on'},
 [0x25AA]={d='on'},
 [0x25AB]={d='on'},
 [0x25AC]={d='on'},
 [0x25AD]={d='on'},
 [0x25AE]={d='on'},
 [0x25AF]={d='on'},
 [0x25B0]={d='on'},
 [0x25B1]={d='on'},
 [0x25B2]={d='on'},
 [0x25B3]={d='on'},
 [0x25B4]={d='on'},
 [0x25B5]={d='on'},
 [0x25B6]={d='on'},
 [0x25B7]={d='on'},
 [0x25B8]={d='on'},
 [0x25B9]={d='on'},
 [0x25BA]={d='on'},
 [0x25BB]={d='on'},
 [0x25BC]={d='on'},
 [0x25BD]={d='on'},
 [0x25BE]={d='on'},
 [0x25BF]={d='on'},
 [0x25C0]={d='on'},
 [0x25C1]={d='on'},
 [0x25C2]={d='on'},
 [0x25C3]={d='on'},
 [0x25C4]={d='on'},
 [0x25C5]={d='on'},
 [0x25C6]={d='on'},
 [0x25C7]={d='on'},
 [0x25C8]={d='on'},
 [0x25C9]={d='on'},
 [0x25CA]={d='on'},
 [0x25CB]={d='on'},
 [0x25CC]={d='on'},
 [0x25CD]={d='on'},
 [0x25CE]={d='on'},
 [0x25CF]={d='on'},
 [0x25D0]={d='on'},
 [0x25D1]={d='on'},
 [0x25D2]={d='on'},
 [0x25D3]={d='on'},
 [0x25D4]={d='on'},
 [0x25D5]={d='on'},
 [0x25D6]={d='on'},
 [0x25D7]={d='on'},
 [0x25D8]={d='on'},
 [0x25D9]={d='on'},
 [0x25DA]={d='on'},
 [0x25DB]={d='on'},
 [0x25DC]={d='on'},
 [0x25DD]={d='on'},
 [0x25DE]={d='on'},
 [0x25DF]={d='on'},
 [0x25E0]={d='on'},
 [0x25E1]={d='on'},
 [0x25E2]={d='on'},
 [0x25E3]={d='on'},
 [0x25E4]={d='on'},
 [0x25E5]={d='on'},
 [0x25E6]={d='on'},
 [0x25E7]={d='on'},
 [0x25E8]={d='on'},
 [0x25E9]={d='on'},
 [0x25EA]={d='on'},
 [0x25EB]={d='on'},
 [0x25EC]={d='on'},
 [0x25ED]={d='on'},
 [0x25EE]={d='on'},
 [0x25EF]={d='on'},
 [0x25F0]={d='on'},
 [0x25F1]={d='on'},
 [0x25F2]={d='on'},
 [0x25F3]={d='on'},
 [0x25F4]={d='on'},
 [0x25F5]={d='on'},
 [0x25F6]={d='on'},
 [0x25F7]={d='on'},
 [0x25F8]={d='on'},
 [0x25F9]={d='on'},
 [0x25FA]={d='on'},
 [0x25FB]={d='on'},
 [0x25FC]={d='on'},
 [0x25FD]={d='on'},
 [0x25FE]={d='on'},
 [0x25FF]={d='on'},
 [0x2600]={d='on'},
 [0x2601]={d='on'},
 [0x2602]={d='on'},
 [0x2603]={d='on'},
 [0x2604]={d='on'},
 [0x2605]={d='on'},
 [0x2606]={d='on'},
 [0x2607]={d='on'},
 [0x2608]={d='on'},
 [0x2609]={d='on'},
 [0x260A]={d='on'},
 [0x260B]={d='on'},
 [0x260C]={d='on'},
 [0x260D]={d='on'},
 [0x260E]={d='on'},
 [0x260F]={d='on'},
 [0x2610]={d='on'},
 [0x2611]={d='on'},
 [0x2612]={d='on'},
 [0x2613]={d='on'},
 [0x2614]={d='on'},
 [0x2615]={d='on'},
 [0x2616]={d='on'},
 [0x2617]={d='on'},
 [0x2618]={d='on'},
 [0x2619]={d='on'},
 [0x261A]={d='on'},
 [0x261B]={d='on'},
 [0x261C]={d='on'},
 [0x261D]={d='on'},
 [0x261E]={d='on'},
 [0x261F]={d='on'},
 [0x2620]={d='on'},
 [0x2621]={d='on'},
 [0x2622]={d='on'},
 [0x2623]={d='on'},
 [0x2624]={d='on'},
 [0x2625]={d='on'},
 [0x2626]={d='on'},
 [0x2627]={d='on'},
 [0x2628]={d='on'},
 [0x2629]={d='on'},
 [0x262A]={d='on'},
 [0x262B]={d='on'},
 [0x262C]={d='on'},
 [0x262D]={d='on'},
 [0x262E]={d='on'},
 [0x262F]={d='on'},
 [0x2630]={d='on'},
 [0x2631]={d='on'},
 [0x2632]={d='on'},
 [0x2633]={d='on'},
 [0x2634]={d='on'},
 [0x2635]={d='on'},
 [0x2636]={d='on'},
 [0x2637]={d='on'},
 [0x2638]={d='on'},
 [0x2639]={d='on'},
 [0x263A]={d='on'},
 [0x263B]={d='on'},
 [0x263C]={d='on'},
 [0x263D]={d='on'},
 [0x263E]={d='on'},
 [0x263F]={d='on'},
 [0x2640]={d='on'},
 [0x2641]={d='on'},
 [0x2642]={d='on'},
 [0x2643]={d='on'},
 [0x2644]={d='on'},
 [0x2645]={d='on'},
 [0x2646]={d='on'},
 [0x2647]={d='on'},
 [0x2648]={d='on'},
 [0x2649]={d='on'},
 [0x264A]={d='on'},
 [0x264B]={d='on'},
 [0x264C]={d='on'},
 [0x264D]={d='on'},
 [0x264E]={d='on'},
 [0x264F]={d='on'},
 [0x2650]={d='on'},
 [0x2651]={d='on'},
 [0x2652]={d='on'},
 [0x2653]={d='on'},
 [0x2654]={d='on'},
 [0x2655]={d='on'},
 [0x2656]={d='on'},
 [0x2657]={d='on'},
 [0x2658]={d='on'},
 [0x2659]={d='on'},
 [0x265A]={d='on'},
 [0x265B]={d='on'},
 [0x265C]={d='on'},
 [0x265D]={d='on'},
 [0x265E]={d='on'},
 [0x265F]={d='on'},
 [0x2660]={d='on'},
 [0x2661]={d='on'},
 [0x2662]={d='on'},
 [0x2663]={d='on'},
 [0x2664]={d='on'},
 [0x2665]={d='on'},
 [0x2666]={d='on'},
 [0x2667]={d='on'},
 [0x2668]={d='on'},
 [0x2669]={d='on'},
 [0x266A]={d='on'},
 [0x266B]={d='on'},
 [0x266C]={d='on'},
 [0x266D]={d='on'},
 [0x266E]={d='on'},
 [0x266F]={d='on'},
 [0x2670]={d='on'},
 [0x2671]={d='on'},
 [0x2672]={d='on'},
 [0x2673]={d='on'},
 [0x2674]={d='on'},
 [0x2675]={d='on'},
 [0x2676]={d='on'},
 [0x2677]={d='on'},
 [0x2678]={d='on'},
 [0x2679]={d='on'},
 [0x267A]={d='on'},
 [0x267B]={d='on'},
 [0x267C]={d='on'},
 [0x267D]={d='on'},
 [0x267E]={d='on'},
 [0x267F]={d='on'},
 [0x2680]={d='on'},
 [0x2681]={d='on'},
 [0x2682]={d='on'},
 [0x2683]={d='on'},
 [0x2684]={d='on'},
 [0x2685]={d='on'},
 [0x2686]={d='on'},
 [0x2687]={d='on'},
 [0x2688]={d='on'},
 [0x2689]={d='on'},
 [0x268A]={d='on'},
 [0x268B]={d='on'},
 [0x268C]={d='on'},
 [0x268D]={d='on'},
 [0x268E]={d='on'},
 [0x268F]={d='on'},
 [0x2690]={d='on'},
 [0x2691]={d='on'},
 [0x2692]={d='on'},
 [0x2693]={d='on'},
 [0x2694]={d='on'},
 [0x2695]={d='on'},
 [0x2696]={d='on'},
 [0x2697]={d='on'},
 [0x2698]={d='on'},
 [0x2699]={d='on'},
 [0x269A]={d='on'},
 [0x269B]={d='on'},
 [0x269C]={d='on'},
 [0x269D]={d='on'},
 [0x269E]={d='on'},
 [0x269F]={d='on'},
 [0x26A0]={d='on'},
 [0x26A1]={d='on'},
 [0x26A2]={d='on'},
 [0x26A3]={d='on'},
 [0x26A4]={d='on'},
 [0x26A5]={d='on'},
 [0x26A6]={d='on'},
 [0x26A7]={d='on'},
 [0x26A8]={d='on'},
 [0x26A9]={d='on'},
 [0x26AA]={d='on'},
 [0x26AB]={d='on'},
 [0x26AD]={d='on'},
 [0x26AE]={d='on'},
 [0x26AF]={d='on'},
 [0x26B0]={d='on'},
 [0x26B1]={d='on'},
 [0x26B2]={d='on'},
 [0x26B3]={d='on'},
 [0x26B4]={d='on'},
 [0x26B5]={d='on'},
 [0x26B6]={d='on'},
 [0x26B7]={d='on'},
 [0x26B8]={d='on'},
 [0x26B9]={d='on'},
 [0x26BA]={d='on'},
 [0x26BB]={d='on'},
 [0x26BC]={d='on'},
 [0x26BD]={d='on'},
 [0x26BE]={d='on'},
 [0x26BF]={d='on'},
 [0x26C0]={d='on'},
 [0x26C1]={d='on'},
 [0x26C2]={d='on'},
 [0x26C3]={d='on'},
 [0x26C4]={d='on'},
 [0x26C5]={d='on'},
 [0x26C6]={d='on'},
 [0x26C7]={d='on'},
 [0x26C8]={d='on'},
 [0x26C9]={d='on'},
 [0x26CA]={d='on'},
 [0x26CB]={d='on'},
 [0x26CC]={d='on'},
 [0x26CD]={d='on'},
 [0x26CE]={d='on'},
 [0x26CF]={d='on'},
 [0x26D0]={d='on'},
 [0x26D1]={d='on'},
 [0x26D2]={d='on'},
 [0x26D3]={d='on'},
 [0x26D4]={d='on'},
 [0x26D5]={d='on'},
 [0x26D6]={d='on'},
 [0x26D7]={d='on'},
 [0x26D8]={d='on'},
 [0x26D9]={d='on'},
 [0x26DA]={d='on'},
 [0x26DB]={d='on'},
 [0x26DC]={d='on'},
 [0x26DD]={d='on'},
 [0x26DE]={d='on'},
 [0x26DF]={d='on'},
 [0x26E0]={d='on'},
 [0x26E1]={d='on'},
 [0x26E2]={d='on'},
 [0x26E3]={d='on'},
 [0x26E4]={d='on'},
 [0x26E5]={d='on'},
 [0x26E6]={d='on'},
 [0x26E7]={d='on'},
 [0x26E8]={d='on'},
 [0x26E9]={d='on'},
 [0x26EA]={d='on'},
 [0x26EB]={d='on'},
 [0x26EC]={d='on'},
 [0x26ED]={d='on'},
 [0x26EE]={d='on'},
 [0x26EF]={d='on'},
 [0x26F0]={d='on'},
 [0x26F1]={d='on'},
 [0x26F2]={d='on'},
 [0x26F3]={d='on'},
 [0x26F4]={d='on'},
 [0x26F5]={d='on'},
 [0x26F6]={d='on'},
 [0x26F7]={d='on'},
 [0x26F8]={d='on'},
 [0x26F9]={d='on'},
 [0x26FA]={d='on'},
 [0x26FB]={d='on'},
 [0x26FC]={d='on'},
 [0x26FD]={d='on'},
 [0x26FE]={d='on'},
 [0x26FF]={d='on'},
 [0x2700]={d='on'},
 [0x2701]={d='on'},
 [0x2702]={d='on'},
 [0x2703]={d='on'},
 [0x2704]={d='on'},
 [0x2705]={d='on'},
 [0x2706]={d='on'},
 [0x2707]={d='on'},
 [0x2708]={d='on'},
 [0x2709]={d='on'},
 [0x270A]={d='on'},
 [0x270B]={d='on'},
 [0x270C]={d='on'},
 [0x270D]={d='on'},
 [0x270E]={d='on'},
 [0x270F]={d='on'},
 [0x2710]={d='on'},
 [0x2711]={d='on'},
 [0x2712]={d='on'},
 [0x2713]={d='on'},
 [0x2714]={d='on'},
 [0x2715]={d='on'},
 [0x2716]={d='on'},
 [0x2717]={d='on'},
 [0x2718]={d='on'},
 [0x2719]={d='on'},
 [0x271A]={d='on'},
 [0x271B]={d='on'},
 [0x271C]={d='on'},
 [0x271D]={d='on'},
 [0x271E]={d='on'},
 [0x271F]={d='on'},
 [0x2720]={d='on'},
 [0x2721]={d='on'},
 [0x2722]={d='on'},
 [0x2723]={d='on'},
 [0x2724]={d='on'},
 [0x2725]={d='on'},
 [0x2726]={d='on'},
 [0x2727]={d='on'},
 [0x2728]={d='on'},
 [0x2729]={d='on'},
 [0x272A]={d='on'},
 [0x272B]={d='on'},
 [0x272C]={d='on'},
 [0x272D]={d='on'},
 [0x272E]={d='on'},
 [0x272F]={d='on'},
 [0x2730]={d='on'},
 [0x2731]={d='on'},
 [0x2732]={d='on'},
 [0x2733]={d='on'},
 [0x2734]={d='on'},
 [0x2735]={d='on'},
 [0x2736]={d='on'},
 [0x2737]={d='on'},
 [0x2738]={d='on'},
 [0x2739]={d='on'},
 [0x273A]={d='on'},
 [0x273B]={d='on'},
 [0x273C]={d='on'},
 [0x273D]={d='on'},
 [0x273E]={d='on'},
 [0x273F]={d='on'},
 [0x2740]={d='on'},
 [0x2741]={d='on'},
 [0x2742]={d='on'},
 [0x2743]={d='on'},
 [0x2744]={d='on'},
 [0x2745]={d='on'},
 [0x2746]={d='on'},
 [0x2747]={d='on'},
 [0x2748]={d='on'},
 [0x2749]={d='on'},
 [0x274A]={d='on'},
 [0x274B]={d='on'},
 [0x274C]={d='on'},
 [0x274D]={d='on'},
 [0x274E]={d='on'},
 [0x274F]={d='on'},
 [0x2750]={d='on'},
 [0x2751]={d='on'},
 [0x2752]={d='on'},
 [0x2753]={d='on'},
 [0x2754]={d='on'},
 [0x2755]={d='on'},
 [0x2756]={d='on'},
 [0x2757]={d='on'},
 [0x2758]={d='on'},
 [0x2759]={d='on'},
 [0x275A]={d='on'},
 [0x275B]={d='on'},
 [0x275C]={d='on'},
 [0x275D]={d='on'},
 [0x275E]={d='on'},
 [0x275F]={d='on'},
 [0x2760]={d='on'},
 [0x2761]={d='on'},
 [0x2762]={d='on'},
 [0x2763]={d='on'},
 [0x2764]={d='on'},
 [0x2765]={d='on'},
 [0x2766]={d='on'},
 [0x2767]={d='on'},
 [0x2768]={d='on', m=0x2769},
 [0x2769]={d='on', m=0x2768},
 [0x276A]={d='on', m=0x276B},
 [0x276B]={d='on', m=0x276A},
 [0x276C]={d='on', m=0x276D},
 [0x276D]={d='on', m=0x276C},
 [0x276E]={d='on', m=0x276F},
 [0x276F]={d='on', m=0x276E},
 [0x2770]={d='on', m=0x2771},
 [0x2771]={d='on', m=0x2770},
 [0x2772]={d='on', m=0x2773},
 [0x2773]={d='on', m=0x2772},
 [0x2774]={d='on', m=0x2775},
 [0x2775]={d='on', m=0x2774},
 [0x2776]={d='on'},
 [0x2777]={d='on'},
 [0x2778]={d='on'},
 [0x2779]={d='on'},
 [0x277A]={d='on'},
 [0x277B]={d='on'},
 [0x277C]={d='on'},
 [0x277D]={d='on'},
 [0x277E]={d='on'},
 [0x277F]={d='on'},
 [0x2780]={d='on'},
 [0x2781]={d='on'},
 [0x2782]={d='on'},
 [0x2783]={d='on'},
 [0x2784]={d='on'},
 [0x2785]={d='on'},
 [0x2786]={d='on'},
 [0x2787]={d='on'},
 [0x2788]={d='on'},
 [0x2789]={d='on'},
 [0x278A]={d='on'},
 [0x278B]={d='on'},
 [0x278C]={d='on'},
 [0x278D]={d='on'},
 [0x278E]={d='on'},
 [0x278F]={d='on'},
 [0x2790]={d='on'},
 [0x2791]={d='on'},
 [0x2792]={d='on'},
 [0x2793]={d='on'},
 [0x2794]={d='on'},
 [0x2795]={d='on'},
 [0x2796]={d='on'},
 [0x2797]={d='on'},
 [0x2798]={d='on'},
 [0x2799]={d='on'},
 [0x279A]={d='on'},
 [0x279B]={d='on'},
 [0x279C]={d='on'},
 [0x279D]={d='on'},
 [0x279E]={d='on'},
 [0x279F]={d='on'},
 [0x27A0]={d='on'},
 [0x27A1]={d='on'},
 [0x27A2]={d='on'},
 [0x27A3]={d='on'},
 [0x27A4]={d='on'},
 [0x27A5]={d='on'},
 [0x27A6]={d='on'},
 [0x27A7]={d='on'},
 [0x27A8]={d='on'},
 [0x27A9]={d='on'},
 [0x27AA]={d='on'},
 [0x27AB]={d='on'},
 [0x27AC]={d='on'},
 [0x27AD]={d='on'},
 [0x27AE]={d='on'},
 [0x27AF]={d='on'},
 [0x27B0]={d='on'},
 [0x27B1]={d='on'},
 [0x27B2]={d='on'},
 [0x27B3]={d='on'},
 [0x27B4]={d='on'},
 [0x27B5]={d='on'},
 [0x27B6]={d='on'},
 [0x27B7]={d='on'},
 [0x27B8]={d='on'},
 [0x27B9]={d='on'},
 [0x27BA]={d='on'},
 [0x27BB]={d='on'},
 [0x27BC]={d='on'},
 [0x27BD]={d='on'},
 [0x27BE]={d='on'},
 [0x27BF]={d='on'},
 [0x27C0]={d='on'},
 [0x27C1]={d='on'},
 [0x27C2]={d='on'},
 [0x27C3]={d='on', m=0x27C4},
 [0x27C4]={d='on', m=0x27C3},
 [0x27C5]={d='on', m=0x27C6},
 [0x27C6]={d='on', m=0x27C5},
 [0x27C7]={d='on'},
 [0x27C8]={d='on', m=0x27C9},
 [0x27C9]={d='on', m=0x27C8},
 [0x27CA]={d='on'},
 [0x27CB]={d='on', m=0x27CD},
 [0x27CC]={d='on'},
 [0x27CD]={d='on', m=0x27CB},
 [0x27CE]={d='on'},
 [0x27CF]={d='on'},
 [0x27D0]={d='on'},
 [0x27D1]={d='on'},
 [0x27D2]={d='on'},
 [0x27D3]={d='on'},
 [0x27D4]={d='on'},
 [0x27D5]={d='on', m=0x27D6},
 [0x27D6]={d='on', m=0x27D5},
 [0x27D7]={d='on'},
 [0x27D8]={d='on'},
 [0x27D9]={d='on'},
 [0x27DA]={d='on'},
 [0x27DB]={d='on'},
 [0x27DC]={d='on'},
 [0x27DD]={d='on', m=0x27DE},
 [0x27DE]={d='on', m=0x27DD},
 [0x27DF]={d='on'},
 [0x27E0]={d='on'},
 [0x27E1]={d='on'},
 [0x27E2]={d='on', m=0x27E3},
 [0x27E3]={d='on', m=0x27E2},
 [0x27E4]={d='on', m=0x27E5},
 [0x27E5]={d='on', m=0x27E4},
 [0x27E6]={d='on', m=0x27E7},
 [0x27E7]={d='on', m=0x27E6},
 [0x27E8]={d='on', m=0x27E9},
 [0x27E9]={d='on', m=0x27E8},
 [0x27EA]={d='on', m=0x27EB},
 [0x27EB]={d='on', m=0x27EA},
 [0x27EC]={d='on', m=0x27ED},
 [0x27ED]={d='on', m=0x27EC},
 [0x27EE]={d='on', m=0x27EF},
 [0x27EF]={d='on', m=0x27EE},
 [0x27F0]={d='on'},
 [0x27F1]={d='on'},
 [0x27F2]={d='on'},
 [0x27F3]={d='on'},
 [0x27F4]={d='on'},
 [0x27F5]={d='on'},
 [0x27F6]={d='on'},
 [0x27F7]={d='on'},
 [0x27F8]={d='on'},
 [0x27F9]={d='on'},
 [0x27FA]={d='on'},
 [0x27FB]={d='on'},
 [0x27FC]={d='on'},
 [0x27FD]={d='on'},
 [0x27FE]={d='on'},
 [0x27FF]={d='on'},
 [0x2900]={d='on'},
 [0x2901]={d='on'},
 [0x2902]={d='on'},
 [0x2903]={d='on'},
 [0x2904]={d='on'},
 [0x2905]={d='on'},
 [0x2906]={d='on'},
 [0x2907]={d='on'},
 [0x2908]={d='on'},
 [0x2909]={d='on'},
 [0x290A]={d='on'},
 [0x290B]={d='on'},
 [0x290C]={d='on'},
 [0x290D]={d='on'},
 [0x290E]={d='on'},
 [0x290F]={d='on'},
 [0x2910]={d='on'},
 [0x2911]={d='on'},
 [0x2912]={d='on'},
 [0x2913]={d='on'},
 [0x2914]={d='on'},
 [0x2915]={d='on'},
 [0x2916]={d='on'},
 [0x2917]={d='on'},
 [0x2918]={d='on'},
 [0x2919]={d='on'},
 [0x291A]={d='on'},
 [0x291B]={d='on'},
 [0x291C]={d='on'},
 [0x291D]={d='on'},
 [0x291E]={d='on'},
 [0x291F]={d='on'},
 [0x2920]={d='on'},
 [0x2921]={d='on'},
 [0x2922]={d='on'},
 [0x2923]={d='on'},
 [0x2924]={d='on'},
 [0x2925]={d='on'},
 [0x2926]={d='on'},
 [0x2927]={d='on'},
 [0x2928]={d='on'},
 [0x2929]={d='on'},
 [0x292A]={d='on'},
 [0x292B]={d='on'},
 [0x292C]={d='on'},
 [0x292D]={d='on'},
 [0x292E]={d='on'},
 [0x292F]={d='on'},
 [0x2930]={d='on'},
 [0x2931]={d='on'},
 [0x2932]={d='on'},
 [0x2933]={d='on'},
 [0x2934]={d='on'},
 [0x2935]={d='on'},
 [0x2936]={d='on'},
 [0x2937]={d='on'},
 [0x2938]={d='on'},
 [0x2939]={d='on'},
 [0x293A]={d='on'},
 [0x293B]={d='on'},
 [0x293C]={d='on'},
 [0x293D]={d='on'},
 [0x293E]={d='on'},
 [0x293F]={d='on'},
 [0x2940]={d='on'},
 [0x2941]={d='on'},
 [0x2942]={d='on'},
 [0x2943]={d='on'},
 [0x2944]={d='on'},
 [0x2945]={d='on'},
 [0x2946]={d='on'},
 [0x2947]={d='on'},
 [0x2948]={d='on'},
 [0x2949]={d='on'},
 [0x294A]={d='on'},
 [0x294B]={d='on'},
 [0x294C]={d='on'},
 [0x294D]={d='on'},
 [0x294E]={d='on'},
 [0x294F]={d='on'},
 [0x2950]={d='on'},
 [0x2951]={d='on'},
 [0x2952]={d='on'},
 [0x2953]={d='on'},
 [0x2954]={d='on'},
 [0x2955]={d='on'},
 [0x2956]={d='on'},
 [0x2957]={d='on'},
 [0x2958]={d='on'},
 [0x2959]={d='on'},
 [0x295A]={d='on'},
 [0x295B]={d='on'},
 [0x295C]={d='on'},
 [0x295D]={d='on'},
 [0x295E]={d='on'},
 [0x295F]={d='on'},
 [0x2960]={d='on'},
 [0x2961]={d='on'},
 [0x2962]={d='on'},
 [0x2963]={d='on'},
 [0x2964]={d='on'},
 [0x2965]={d='on'},
 [0x2966]={d='on'},
 [0x2967]={d='on'},
 [0x2968]={d='on'},
 [0x2969]={d='on'},
 [0x296A]={d='on'},
 [0x296B]={d='on'},
 [0x296C]={d='on'},
 [0x296D]={d='on'},
 [0x296E]={d='on'},
 [0x296F]={d='on'},
 [0x2970]={d='on'},
 [0x2971]={d='on'},
 [0x2972]={d='on'},
 [0x2973]={d='on'},
 [0x2974]={d='on'},
 [0x2975]={d='on'},
 [0x2976]={d='on'},
 [0x2977]={d='on'},
 [0x2978]={d='on'},
 [0x2979]={d='on'},
 [0x297A]={d='on'},
 [0x297B]={d='on'},
 [0x297C]={d='on'},
 [0x297D]={d='on'},
 [0x297E]={d='on'},
 [0x297F]={d='on'},
 [0x2980]={d='on'},
 [0x2981]={d='on'},
 [0x2982]={d='on'},
 [0x2983]={d='on', m=0x2984},
 [0x2984]={d='on', m=0x2983},
 [0x2985]={d='on', m=0x2986},
 [0x2986]={d='on', m=0x2985},
 [0x2987]={d='on', m=0x2988},
 [0x2988]={d='on', m=0x2987},
 [0x2989]={d='on', m=0x298A},
 [0x298A]={d='on', m=0x2989},
 [0x298B]={d='on', m=0x298C},
 [0x298C]={d='on', m=0x298B},
 [0x298D]={d='on', m=0x2990},
 [0x298E]={d='on', m=0x298F},
 [0x298F]={d='on', m=0x298E},
 [0x2990]={d='on', m=0x298D},
 [0x2991]={d='on', m=0x2992},
 [0x2992]={d='on', m=0x2991},
 [0x2993]={d='on', m=0x2994},
 [0x2994]={d='on', m=0x2993},
 [0x2995]={d='on', m=0x2996},
 [0x2996]={d='on', m=0x2995},
 [0x2997]={d='on', m=0x2998},
 [0x2998]={d='on', m=0x2997},
 [0x2999]={d='on'},
 [0x299A]={d='on'},
 [0x299B]={d='on'},
 [0x299C]={d='on'},
 [0x299D]={d='on'},
 [0x299E]={d='on'},
 [0x299F]={d='on'},
 [0x29A0]={d='on'},
 [0x29A1]={d='on'},
 [0x29A2]={d='on'},
 [0x29A3]={d='on'},
 [0x29A4]={d='on'},
 [0x29A5]={d='on'},
 [0x29A6]={d='on'},
 [0x29A7]={d='on'},
 [0x29A8]={d='on'},
 [0x29A9]={d='on'},
 [0x29AA]={d='on'},
 [0x29AB]={d='on'},
 [0x29AC]={d='on'},
 [0x29AD]={d='on'},
 [0x29AE]={d='on'},
 [0x29AF]={d='on'},
 [0x29B0]={d='on'},
 [0x29B1]={d='on'},
 [0x29B2]={d='on'},
 [0x29B3]={d='on'},
 [0x29B4]={d='on'},
 [0x29B5]={d='on'},
 [0x29B6]={d='on'},
 [0x29B7]={d='on'},
 [0x29B8]={d='on', m=0x2298},
 [0x29B9]={d='on'},
 [0x29BA]={d='on'},
 [0x29BB]={d='on'},
 [0x29BC]={d='on'},
 [0x29BD]={d='on'},
 [0x29BE]={d='on'},
 [0x29BF]={d='on'},
 [0x29C0]={d='on', m=0x29C1},
 [0x29C1]={d='on', m=0x29C0},
 [0x29C2]={d='on'},
 [0x29C3]={d='on'},
 [0x29C4]={d='on', m=0x29C5},
 [0x29C5]={d='on', m=0x29C4},
 [0x29C6]={d='on'},
 [0x29C7]={d='on'},
 [0x29C8]={d='on'},
 [0x29C9]={d='on'},
 [0x29CA]={d='on'},
 [0x29CB]={d='on'},
 [0x29CC]={d='on'},
 [0x29CD]={d='on'},
 [0x29CE]={d='on'},
 [0x29CF]={d='on', m=0x29D0},
 [0x29D0]={d='on', m=0x29CF},
 [0x29D1]={d='on', m=0x29D2},
 [0x29D2]={d='on', m=0x29D1},
 [0x29D3]={d='on'},
 [0x29D4]={d='on', m=0x29D5},
 [0x29D5]={d='on', m=0x29D4},
 [0x29D6]={d='on'},
 [0x29D7]={d='on'},
 [0x29D8]={d='on', m=0x29D9},
 [0x29D9]={d='on', m=0x29D8},
 [0x29DA]={d='on', m=0x29DB},
 [0x29DB]={d='on', m=0x29DA},
 [0x29DC]={d='on'},
 [0x29DD]={d='on'},
 [0x29DE]={d='on'},
 [0x29DF]={d='on'},
 [0x29E0]={d='on'},
 [0x29E1]={d='on'},
 [0x29E2]={d='on'},
 [0x29E3]={d='on'},
 [0x29E4]={d='on'},
 [0x29E5]={d='on'},
 [0x29E6]={d='on'},
 [0x29E7]={d='on'},
 [0x29E8]={d='on'},
 [0x29E9]={d='on'},
 [0x29EA]={d='on'},
 [0x29EB]={d='on'},
 [0x29EC]={d='on'},
 [0x29ED]={d='on'},
 [0x29EE]={d='on'},
 [0x29EF]={d='on'},
 [0x29F0]={d='on'},
 [0x29F1]={d='on'},
 [0x29F2]={d='on'},
 [0x29F3]={d='on'},
 [0x29F4]={d='on'},
 [0x29F5]={d='on', m=0x2215},
 [0x29F6]={d='on'},
 [0x29F7]={d='on'},
 [0x29F8]={d='on', m=0x29F9},
 [0x29F9]={d='on', m=0x29F8},
 [0x29FA]={d='on'},
 [0x29FB]={d='on'},
 [0x29FC]={d='on', m=0x29FD},
 [0x29FD]={d='on', m=0x29FC},
 [0x29FE]={d='on'},
 [0x29FF]={d='on'},
 [0x2A00]={d='on'},
 [0x2A01]={d='on'},
 [0x2A02]={d='on'},
 [0x2A03]={d='on'},
 [0x2A04]={d='on'},
 [0x2A05]={d='on'},
 [0x2A06]={d='on'},
 [0x2A07]={d='on'},
 [0x2A08]={d='on'},
 [0x2A09]={d='on'},
 [0x2A0A]={d='on'},
 [0x2A0B]={d='on'},
 [0x2A0C]={d='on'},
 [0x2A0D]={d='on'},
 [0x2A0E]={d='on'},
 [0x2A0F]={d='on'},
 [0x2A10]={d='on'},
 [0x2A11]={d='on'},
 [0x2A12]={d='on'},
 [0x2A13]={d='on'},
 [0x2A14]={d='on'},
 [0x2A15]={d='on'},
 [0x2A16]={d='on'},
 [0x2A17]={d='on'},
 [0x2A18]={d='on'},
 [0x2A19]={d='on'},
 [0x2A1A]={d='on'},
 [0x2A1B]={d='on'},
 [0x2A1C]={d='on'},
 [0x2A1D]={d='on'},
 [0x2A1E]={d='on'},
 [0x2A1F]={d='on'},
 [0x2A20]={d='on'},
 [0x2A21]={d='on'},
 [0x2A22]={d='on'},
 [0x2A23]={d='on'},
 [0x2A24]={d='on'},
 [0x2A25]={d='on'},
 [0x2A26]={d='on'},
 [0x2A27]={d='on'},
 [0x2A28]={d='on'},
 [0x2A29]={d='on'},
 [0x2A2A]={d='on'},
 [0x2A2B]={d='on', m=0x2A2C},
 [0x2A2C]={d='on', m=0x2A2B},
 [0x2A2D]={d='on', m=0x2A2E},
 [0x2A2E]={d='on', m=0x2A2D},
 [0x2A2F]={d='on'},
 [0x2A30]={d='on'},
 [0x2A31]={d='on'},
 [0x2A32]={d='on'},
 [0x2A33]={d='on'},
 [0x2A34]={d='on', m=0x2A35},
 [0x2A35]={d='on', m=0x2A34},
 [0x2A36]={d='on'},
 [0x2A37]={d='on'},
 [0x2A38]={d='on'},
 [0x2A39]={d='on'},
 [0x2A3A]={d='on'},
 [0x2A3B]={d='on'},
 [0x2A3C]={d='on', m=0x2A3D},
 [0x2A3D]={d='on', m=0x2A3C},
 [0x2A3E]={d='on'},
 [0x2A3F]={d='on'},
 [0x2A40]={d='on'},
 [0x2A41]={d='on'},
 [0x2A42]={d='on'},
 [0x2A43]={d='on'},
 [0x2A44]={d='on'},
 [0x2A45]={d='on'},
 [0x2A46]={d='on'},
 [0x2A47]={d='on'},
 [0x2A48]={d='on'},
 [0x2A49]={d='on'},
 [0x2A4A]={d='on'},
 [0x2A4B]={d='on'},
 [0x2A4C]={d='on'},
 [0x2A4D]={d='on'},
 [0x2A4E]={d='on'},
 [0x2A4F]={d='on'},
 [0x2A50]={d='on'},
 [0x2A51]={d='on'},
 [0x2A52]={d='on'},
 [0x2A53]={d='on'},
 [0x2A54]={d='on'},
 [0x2A55]={d='on'},
 [0x2A56]={d='on'},
 [0x2A57]={d='on'},
 [0x2A58]={d='on'},
 [0x2A59]={d='on'},
 [0x2A5A]={d='on'},
 [0x2A5B]={d='on'},
 [0x2A5C]={d='on'},
 [0x2A5D]={d='on'},
 [0x2A5E]={d='on'},
 [0x2A5F]={d='on'},
 [0x2A60]={d='on'},
 [0x2A61]={d='on'},
 [0x2A62]={d='on'},
 [0x2A63]={d='on'},
 [0x2A64]={d='on', m=0x2A65},
 [0x2A65]={d='on', m=0x2A64},
 [0x2A66]={d='on'},
 [0x2A67]={d='on'},
 [0x2A68]={d='on'},
 [0x2A69]={d='on'},
 [0x2A6A]={d='on'},
 [0x2A6B]={d='on'},
 [0x2A6C]={d='on'},
 [0x2A6D]={d='on'},
 [0x2A6E]={d='on'},
 [0x2A6F]={d='on'},
 [0x2A70]={d='on'},
 [0x2A71]={d='on'},
 [0x2A72]={d='on'},
 [0x2A73]={d='on'},
 [0x2A74]={d='on'},
 [0x2A75]={d='on'},
 [0x2A76]={d='on'},
 [0x2A77]={d='on'},
 [0x2A78]={d='on'},
 [0x2A79]={d='on', m=0x2A7A},
 [0x2A7A]={d='on', m=0x2A79},
 [0x2A7B]={d='on'},
 [0x2A7C]={d='on'},
 [0x2A7D]={d='on', m=0x2A7E},
 [0x2A7E]={d='on', m=0x2A7D},
 [0x2A7F]={d='on', m=0x2A80},
 [0x2A80]={d='on', m=0x2A7F},
 [0x2A81]={d='on', m=0x2A82},
 [0x2A82]={d='on', m=0x2A81},
 [0x2A83]={d='on', m=0x2A84},
 [0x2A84]={d='on', m=0x2A83},
 [0x2A85]={d='on'},
 [0x2A86]={d='on'},
 [0x2A87]={d='on'},
 [0x2A88]={d='on'},
 [0x2A89]={d='on'},
 [0x2A8A]={d='on'},
 [0x2A8B]={d='on', m=0x2A8C},
 [0x2A8C]={d='on', m=0x2A8B},
 [0x2A8D]={d='on'},
 [0x2A8E]={d='on'},
 [0x2A8F]={d='on'},
 [0x2A90]={d='on'},
 [0x2A91]={d='on', m=0x2A92},
 [0x2A92]={d='on', m=0x2A91},
 [0x2A93]={d='on', m=0x2A94},
 [0x2A94]={d='on', m=0x2A93},
 [0x2A95]={d='on', m=0x2A96},
 [0x2A96]={d='on', m=0x2A95},
 [0x2A97]={d='on', m=0x2A98},
 [0x2A98]={d='on', m=0x2A97},
 [0x2A99]={d='on', m=0x2A9A},
 [0x2A9A]={d='on', m=0x2A99},
 [0x2A9B]={d='on', m=0x2A9C},
 [0x2A9C]={d='on', m=0x2A9B},
 [0x2A9D]={d='on'},
 [0x2A9E]={d='on'},
 [0x2A9F]={d='on'},
 [0x2AA0]={d='on'},
 [0x2AA1]={d='on', m=0x2AA2},
 [0x2AA2]={d='on', m=0x2AA1},
 [0x2AA3]={d='on'},
 [0x2AA4]={d='on'},
 [0x2AA5]={d='on'},
 [0x2AA6]={d='on', m=0x2AA7},
 [0x2AA7]={d='on', m=0x2AA6},
 [0x2AA8]={d='on', m=0x2AA9},
 [0x2AA9]={d='on', m=0x2AA8},
 [0x2AAA]={d='on', m=0x2AAB},
 [0x2AAB]={d='on', m=0x2AAA},
 [0x2AAC]={d='on', m=0x2AAD},
 [0x2AAD]={d='on', m=0x2AAC},
 [0x2AAE]={d='on'},
 [0x2AAF]={d='on', m=0x2AB0},
 [0x2AB0]={d='on', m=0x2AAF},
 [0x2AB1]={d='on'},
 [0x2AB2]={d='on'},
 [0x2AB3]={d='on', m=0x2AB4},
 [0x2AB4]={d='on', m=0x2AB3},
 [0x2AB5]={d='on'},
 [0x2AB6]={d='on'},
 [0x2AB7]={d='on'},
 [0x2AB8]={d='on'},
 [0x2AB9]={d='on'},
 [0x2ABA]={d='on'},
 [0x2ABB]={d='on', m=0x2ABC},
 [0x2ABC]={d='on', m=0x2ABB},
 [0x2ABD]={d='on', m=0x2ABE},
 [0x2ABE]={d='on', m=0x2ABD},
 [0x2ABF]={d='on', m=0x2AC0},
 [0x2AC0]={d='on', m=0x2ABF},
 [0x2AC1]={d='on', m=0x2AC2},
 [0x2AC2]={d='on', m=0x2AC1},
 [0x2AC3]={d='on', m=0x2AC4},
 [0x2AC4]={d='on', m=0x2AC3},
 [0x2AC5]={d='on', m=0x2AC6},
 [0x2AC6]={d='on', m=0x2AC5},
 [0x2AC7]={d='on'},
 [0x2AC8]={d='on'},
 [0x2AC9]={d='on'},
 [0x2ACA]={d='on'},
 [0x2ACB]={d='on'},
 [0x2ACC]={d='on'},
 [0x2ACD]={d='on', m=0x2ACE},
 [0x2ACE]={d='on', m=0x2ACD},
 [0x2ACF]={d='on', m=0x2AD0},
 [0x2AD0]={d='on', m=0x2ACF},
 [0x2AD1]={d='on', m=0x2AD2},
 [0x2AD2]={d='on', m=0x2AD1},
 [0x2AD3]={d='on', m=0x2AD4},
 [0x2AD4]={d='on', m=0x2AD3},
 [0x2AD5]={d='on', m=0x2AD6},
 [0x2AD6]={d='on', m=0x2AD5},
 [0x2AD7]={d='on'},
 [0x2AD8]={d='on'},
 [0x2AD9]={d='on'},
 [0x2ADA]={d='on'},
 [0x2ADB]={d='on'},
 [0x2ADC]={d='on'},
 [0x2ADD]={d='on'},
 [0x2ADE]={d='on', m=0x22A6},
 [0x2ADF]={d='on'},
 [0x2AE0]={d='on'},
 [0x2AE1]={d='on'},
 [0x2AE2]={d='on'},
 [0x2AE3]={d='on', m=0x22A9},
 [0x2AE4]={d='on', m=0x22A8},
 [0x2AE5]={d='on', m=0x22AB},
 [0x2AE6]={d='on'},
 [0x2AE7]={d='on'},
 [0x2AE8]={d='on'},
 [0x2AE9]={d='on'},
 [0x2AEA]={d='on'},
 [0x2AEB]={d='on'},
 [0x2AEC]={d='on', m=0x2AED},
 [0x2AED]={d='on', m=0x2AEC},
 [0x2AEE]={d='on'},
 [0x2AEF]={d='on'},
 [0x2AF0]={d='on'},
 [0x2AF1]={d='on'},
 [0x2AF2]={d='on'},
 [0x2AF3]={d='on'},
 [0x2AF4]={d='on'},
 [0x2AF5]={d='on'},
 [0x2AF6]={d='on'},
 [0x2AF7]={d='on', m=0x2AF8},
 [0x2AF8]={d='on', m=0x2AF7},
 [0x2AF9]={d='on', m=0x2AFA},
 [0x2AFA]={d='on', m=0x2AF9},
 [0x2AFB]={d='on'},
 [0x2AFC]={d='on'},
 [0x2AFD]={d='on'},
 [0x2AFE]={d='on'},
 [0x2AFF]={d='on'},
 [0x2B00]={d='on'},
 [0x2B01]={d='on'},
 [0x2B02]={d='on'},
 [0x2B03]={d='on'},
 [0x2B04]={d='on'},
 [0x2B05]={d='on'},
 [0x2B06]={d='on'},
 [0x2B07]={d='on'},
 [0x2B08]={d='on'},
 [0x2B09]={d='on'},
 [0x2B0A]={d='on'},
 [0x2B0B]={d='on'},
 [0x2B0C]={d='on'},
 [0x2B0D]={d='on'},
 [0x2B0E]={d='on'},
 [0x2B0F]={d='on'},
 [0x2B10]={d='on'},
 [0x2B11]={d='on'},
 [0x2B12]={d='on'},
 [0x2B13]={d='on'},
 [0x2B14]={d='on'},
 [0x2B15]={d='on'},
 [0x2B16]={d='on'},
 [0x2B17]={d='on'},
 [0x2B18]={d='on'},
 [0x2B19]={d='on'},
 [0x2B1A]={d='on'},
 [0x2B1B]={d='on'},
 [0x2B1C]={d='on'},
 [0x2B1D]={d='on'},
 [0x2B1E]={d='on'},
 [0x2B1F]={d='on'},
 [0x2B20]={d='on'},
 [0x2B21]={d='on'},
 [0x2B22]={d='on'},
 [0x2B23]={d='on'},
 [0x2B24]={d='on'},
 [0x2B25]={d='on'},
 [0x2B26]={d='on'},
 [0x2B27]={d='on'},
 [0x2B28]={d='on'},
 [0x2B29]={d='on'},
 [0x2B2A]={d='on'},
 [0x2B2B]={d='on'},
 [0x2B2C]={d='on'},
 [0x2B2D]={d='on'},
 [0x2B2E]={d='on'},
 [0x2B2F]={d='on'},
 [0x2B30]={d='on'},
 [0x2B31]={d='on'},
 [0x2B32]={d='on'},
 [0x2B33]={d='on'},
 [0x2B34]={d='on'},
 [0x2B35]={d='on'},
 [0x2B36]={d='on'},
 [0x2B37]={d='on'},
 [0x2B38]={d='on'},
 [0x2B39]={d='on'},
 [0x2B3A]={d='on'},
 [0x2B3B]={d='on'},
 [0x2B3C]={d='on'},
 [0x2B3D]={d='on'},
 [0x2B3E]={d='on'},
 [0x2B3F]={d='on'},
 [0x2B40]={d='on'},
 [0x2B41]={d='on'},
 [0x2B42]={d='on'},
 [0x2B43]={d='on'},
 [0x2B44]={d='on'},
 [0x2B45]={d='on'},
 [0x2B46]={d='on'},
 [0x2B47]={d='on'},
 [0x2B48]={d='on'},
 [0x2B49]={d='on'},
 [0x2B4A]={d='on'},
 [0x2B4B]={d='on'},
 [0x2B4C]={d='on'},
 [0x2B4D]={d='on'},
 [0x2B4E]={d='on'},
 [0x2B4F]={d='on'},
 [0x2B50]={d='on'},
 [0x2B51]={d='on'},
 [0x2B52]={d='on'},
 [0x2B53]={d='on'},
 [0x2B54]={d='on'},
 [0x2B55]={d='on'},
 [0x2B56]={d='on'},
 [0x2B57]={d='on'},
 [0x2B58]={d='on'},
 [0x2B59]={d='on'},
 [0x2B5A]={d='on'},
 [0x2B5B]={d='on'},
 [0x2B5C]={d='on'},
 [0x2B5D]={d='on'},
 [0x2B5E]={d='on'},
 [0x2B5F]={d='on'},
 [0x2B60]={d='on'},
 [0x2B61]={d='on'},
 [0x2B62]={d='on'},
 [0x2B63]={d='on'},
 [0x2B64]={d='on'},
 [0x2B65]={d='on'},
 [0x2B66]={d='on'},
 [0x2B67]={d='on'},
 [0x2B68]={d='on'},
 [0x2B69]={d='on'},
 [0x2B6A]={d='on'},
 [0x2B6B]={d='on'},
 [0x2B6C]={d='on'},
 [0x2B6D]={d='on'},
 [0x2B6E]={d='on'},
 [0x2B6F]={d='on'},
 [0x2B70]={d='on'},
 [0x2B71]={d='on'},
 [0x2B72]={d='on'},
 [0x2B73]={d='on'},
 [0x2B76]={d='on'},
 [0x2B77]={d='on'},
 [0x2B78]={d='on'},
 [0x2B79]={d='on'},
 [0x2B7A]={d='on'},
 [0x2B7B]={d='on'},
 [0x2B7C]={d='on'},
 [0x2B7D]={d='on'},
 [0x2B7E]={d='on'},
 [0x2B7F]={d='on'},
 [0x2B80]={d='on'},
 [0x2B81]={d='on'},
 [0x2B82]={d='on'},
 [0x2B83]={d='on'},
 [0x2B84]={d='on'},
 [0x2B85]={d='on'},
 [0x2B86]={d='on'},
 [0x2B87]={d='on'},
 [0x2B88]={d='on'},
 [0x2B89]={d='on'},
 [0x2B8A]={d='on'},
 [0x2B8B]={d='on'},
 [0x2B8C]={d='on'},
 [0x2B8D]={d='on'},
 [0x2B8E]={d='on'},
 [0x2B8F]={d='on'},
 [0x2B90]={d='on'},
 [0x2B91]={d='on'},
 [0x2B92]={d='on'},
 [0x2B93]={d='on'},
 [0x2B94]={d='on'},
 [0x2B95]={d='on'},
 [0x2B98]={d='on'},
 [0x2B99]={d='on'},
 [0x2B9A]={d='on'},
 [0x2B9B]={d='on'},
 [0x2B9C]={d='on'},
 [0x2B9D]={d='on'},
 [0x2B9E]={d='on'},
 [0x2B9F]={d='on'},
 [0x2BA0]={d='on'},
 [0x2BA1]={d='on'},
 [0x2BA2]={d='on'},
 [0x2BA3]={d='on'},
 [0x2BA4]={d='on'},
 [0x2BA5]={d='on'},
 [0x2BA6]={d='on'},
 [0x2BA7]={d='on'},
 [0x2BA8]={d='on'},
 [0x2BA9]={d='on'},
 [0x2BAA]={d='on'},
 [0x2BAB]={d='on'},
 [0x2BAC]={d='on'},
 [0x2BAD]={d='on'},
 [0x2BAE]={d='on'},
 [0x2BAF]={d='on'},
 [0x2BB0]={d='on'},
 [0x2BB1]={d='on'},
 [0x2BB2]={d='on'},
 [0x2BB3]={d='on'},
 [0x2BB4]={d='on'},
 [0x2BB5]={d='on'},
 [0x2BB6]={d='on'},
 [0x2BB7]={d='on'},
 [0x2BB8]={d='on'},
 [0x2BB9]={d='on'},
 [0x2BBD]={d='on'},
 [0x2BBE]={d='on'},
 [0x2BBF]={d='on'},
 [0x2BC0]={d='on'},
 [0x2BC1]={d='on'},
 [0x2BC2]={d='on'},
 [0x2BC3]={d='on'},
 [0x2BC4]={d='on'},
 [0x2BC5]={d='on'},
 [0x2BC6]={d='on'},
 [0x2BC7]={d='on'},
 [0x2BC8]={d='on'},
 [0x2BCA]={d='on'},
 [0x2BCB]={d='on'},
 [0x2BCC]={d='on'},
 [0x2BCD]={d='on'},
 [0x2BCE]={d='on'},
 [0x2BCF]={d='on'},
 [0x2BD0]={d='on'},
 [0x2BD1]={d='on'},
 [0x2BEC]={d='on'},
 [0x2BED]={d='on'},
 [0x2BEE]={d='on'},
 [0x2BEF]={d='on'},
 [0x2CE5]={d='on'},
 [0x2CE6]={d='on'},
 [0x2CE7]={d='on'},
 [0x2CE8]={d='on'},
 [0x2CE9]={d='on'},
 [0x2CEA]={d='on'},
 [0x2CEF]={d='nsm'},
 [0x2CF0]={d='nsm'},
 [0x2CF1]={d='nsm'},
 [0x2CF9]={d='on'},
 [0x2CFA]={d='on'},
 [0x2CFB]={d='on'},
 [0x2CFC]={d='on'},
 [0x2CFD]={d='on'},
 [0x2CFE]={d='on'},
 [0x2CFF]={d='on'},
 [0x2D7F]={d='nsm'},
 [0x2DE0]={d='nsm'},
 [0x2DE1]={d='nsm'},
 [0x2DE2]={d='nsm'},
 [0x2DE3]={d='nsm'},
 [0x2DE4]={d='nsm'},
 [0x2DE5]={d='nsm'},
 [0x2DE6]={d='nsm'},
 [0x2DE7]={d='nsm'},
 [0x2DE8]={d='nsm'},
 [0x2DE9]={d='nsm'},
 [0x2DEA]={d='nsm'},
 [0x2DEB]={d='nsm'},
 [0x2DEC]={d='nsm'},
 [0x2DED]={d='nsm'},
 [0x2DEE]={d='nsm'},
 [0x2DEF]={d='nsm'},
 [0x2DF0]={d='nsm'},
 [0x2DF1]={d='nsm'},
 [0x2DF2]={d='nsm'},
 [0x2DF3]={d='nsm'},
 [0x2DF4]={d='nsm'},
 [0x2DF5]={d='nsm'},
 [0x2DF6]={d='nsm'},
 [0x2DF7]={d='nsm'},
 [0x2DF8]={d='nsm'},
 [0x2DF9]={d='nsm'},
 [0x2DFA]={d='nsm'},
 [0x2DFB]={d='nsm'},
 [0x2DFC]={d='nsm'},
 [0x2DFD]={d='nsm'},
 [0x2DFE]={d='nsm'},
 [0x2DFF]={d='nsm'},
 [0x2E00]={d='on'},
 [0x2E01]={d='on'},
 [0x2E02]={d='on', m=0x2E03},
 [0x2E03]={d='on', m=0x2E02},
 [0x2E04]={d='on', m=0x2E05},
 [0x2E05]={d='on', m=0x2E04},
 [0x2E06]={d='on'},
 [0x2E07]={d='on'},
 [0x2E08]={d='on'},
 [0x2E09]={d='on', m=0x2E0A},
 [0x2E0A]={d='on', m=0x2E09},
 [0x2E0B]={d='on'},
 [0x2E0C]={d='on', m=0x2E0D},
 [0x2E0D]={d='on', m=0x2E0C},
 [0x2E0E]={d='on'},
 [0x2E0F]={d='on'},
 [0x2E10]={d='on'},
 [0x2E11]={d='on'},
 [0x2E12]={d='on'},
 [0x2E13]={d='on'},
 [0x2E14]={d='on'},
 [0x2E15]={d='on'},
 [0x2E16]={d='on'},
 [0x2E17]={d='on'},
 [0x2E18]={d='on'},
 [0x2E19]={d='on'},
 [0x2E1A]={d='on'},
 [0x2E1B]={d='on'},
 [0x2E1C]={d='on', m=0x2E1D},
 [0x2E1D]={d='on', m=0x2E1C},
 [0x2E1E]={d='on'},
 [0x2E1F]={d='on'},
 [0x2E20]={d='on', m=0x2E21},
 [0x2E21]={d='on', m=0x2E20},
 [0x2E22]={d='on', m=0x2E23},
 [0x2E23]={d='on', m=0x2E22},
 [0x2E24]={d='on', m=0x2E25},
 [0x2E25]={d='on', m=0x2E24},
 [0x2E26]={d='on', m=0x2E27},
 [0x2E27]={d='on', m=0x2E26},
 [0x2E28]={d='on', m=0x2E29},
 [0x2E29]={d='on', m=0x2E28},
 [0x2E2A]={d='on'},
 [0x2E2B]={d='on'},
 [0x2E2C]={d='on'},
 [0x2E2D]={d='on'},
 [0x2E2E]={d='on'},
 [0x2E2F]={d='on'},
 [0x2E30]={d='on'},
 [0x2E31]={d='on'},
 [0x2E32]={d='on'},
 [0x2E33]={d='on'},
 [0x2E34]={d='on'},
 [0x2E35]={d='on'},
 [0x2E36]={d='on'},
 [0x2E37]={d='on'},
 [0x2E38]={d='on'},
 [0x2E39]={d='on'},
 [0x2E3A]={d='on'},
 [0x2E3B]={d='on'},
 [0x2E3C]={d='on'},
 [0x2E3D]={d='on'},
 [0x2E3E]={d='on'},
 [0x2E3F]={d='on'},
 [0x2E40]={d='on'},
 [0x2E41]={d='on'},
 [0x2E42]={d='on'},
 [0x2E43]={d='on'},
 [0x2E44]={d='on'},
 [0x2E80]={d='on'},
 [0x2E81]={d='on'},
 [0x2E82]={d='on'},
 [0x2E83]={d='on'},
 [0x2E84]={d='on'},
 [0x2E85]={d='on'},
 [0x2E86]={d='on'},
 [0x2E87]={d='on'},
 [0x2E88]={d='on'},
 [0x2E89]={d='on'},
 [0x2E8A]={d='on'},
 [0x2E8B]={d='on'},
 [0x2E8C]={d='on'},
 [0x2E8D]={d='on'},
 [0x2E8E]={d='on'},
 [0x2E8F]={d='on'},
 [0x2E90]={d='on'},
 [0x2E91]={d='on'},
 [0x2E92]={d='on'},
 [0x2E93]={d='on'},
 [0x2E94]={d='on'},
 [0x2E95]={d='on'},
 [0x2E96]={d='on'},
 [0x2E97]={d='on'},
 [0x2E98]={d='on'},
 [0x2E99]={d='on'},
 [0x2E9B]={d='on'},
 [0x2E9C]={d='on'},
 [0x2E9D]={d='on'},
 [0x2E9E]={d='on'},
 [0x2E9F]={d='on'},
 [0x2EA0]={d='on'},
 [0x2EA1]={d='on'},
 [0x2EA2]={d='on'},
 [0x2EA3]={d='on'},
 [0x2EA4]={d='on'},
 [0x2EA5]={d='on'},
 [0x2EA6]={d='on'},
 [0x2EA7]={d='on'},
 [0x2EA8]={d='on'},
 [0x2EA9]={d='on'},
 [0x2EAA]={d='on'},
 [0x2EAB]={d='on'},
 [0x2EAC]={d='on'},
 [0x2EAD]={d='on'},
 [0x2EAE]={d='on'},
 [0x2EAF]={d='on'},
 [0x2EB0]={d='on'},
 [0x2EB1]={d='on'},
 [0x2EB2]={d='on'},
 [0x2EB3]={d='on'},
 [0x2EB4]={d='on'},
 [0x2EB5]={d='on'},
 [0x2EB6]={d='on'},
 [0x2EB7]={d='on'},
 [0x2EB8]={d='on'},
 [0x2EB9]={d='on'},
 [0x2EBA]={d='on'},
 [0x2EBB]={d='on'},
 [0x2EBC]={d='on'},
 [0x2EBD]={d='on'},
 [0x2EBE]={d='on'},
 [0x2EBF]={d='on'},
 [0x2EC0]={d='on'},
 [0x2EC1]={d='on'},
 [0x2EC2]={d='on'},
 [0x2EC3]={d='on'},
 [0x2EC4]={d='on'},
 [0x2EC5]={d='on'},
 [0x2EC6]={d='on'},
 [0x2EC7]={d='on'},
 [0x2EC8]={d='on'},
 [0x2EC9]={d='on'},
 [0x2ECA]={d='on'},
 [0x2ECB]={d='on'},
 [0x2ECC]={d='on'},
 [0x2ECD]={d='on'},
 [0x2ECE]={d='on'},
 [0x2ECF]={d='on'},
 [0x2ED0]={d='on'},
 [0x2ED1]={d='on'},
 [0x2ED2]={d='on'},
 [0x2ED3]={d='on'},
 [0x2ED4]={d='on'},
 [0x2ED5]={d='on'},
 [0x2ED6]={d='on'},
 [0x2ED7]={d='on'},
 [0x2ED8]={d='on'},
 [0x2ED9]={d='on'},
 [0x2EDA]={d='on'},
 [0x2EDB]={d='on'},
 [0x2EDC]={d='on'},
 [0x2EDD]={d='on'},
 [0x2EDE]={d='on'},
 [0x2EDF]={d='on'},
 [0x2EE0]={d='on'},
 [0x2EE1]={d='on'},
 [0x2EE2]={d='on'},
 [0x2EE3]={d='on'},
 [0x2EE4]={d='on'},
 [0x2EE5]={d='on'},
 [0x2EE6]={d='on'},
 [0x2EE7]={d='on'},
 [0x2EE8]={d='on'},
 [0x2EE9]={d='on'},
 [0x2EEA]={d='on'},
 [0x2EEB]={d='on'},
 [0x2EEC]={d='on'},
 [0x2EED]={d='on'},
 [0x2EEE]={d='on'},
 [0x2EEF]={d='on'},
 [0x2EF0]={d='on'},
 [0x2EF1]={d='on'},
 [0x2EF2]={d='on'},
 [0x2EF3]={d='on'},
 [0x2F00]={d='on'},
 [0x2F01]={d='on'},
 [0x2F02]={d='on'},
 [0x2F03]={d='on'},
 [0x2F04]={d='on'},
 [0x2F05]={d='on'},
 [0x2F06]={d='on'},
 [0x2F07]={d='on'},
 [0x2F08]={d='on'},
 [0x2F09]={d='on'},
 [0x2F0A]={d='on'},
 [0x2F0B]={d='on'},
 [0x2F0C]={d='on'},
 [0x2F0D]={d='on'},
 [0x2F0E]={d='on'},
 [0x2F0F]={d='on'},
 [0x2F10]={d='on'},
 [0x2F11]={d='on'},
 [0x2F12]={d='on'},
 [0x2F13]={d='on'},
 [0x2F14]={d='on'},
 [0x2F15]={d='on'},
 [0x2F16]={d='on'},
 [0x2F17]={d='on'},
 [0x2F18]={d='on'},
 [0x2F19]={d='on'},
 [0x2F1A]={d='on'},
 [0x2F1B]={d='on'},
 [0x2F1C]={d='on'},
 [0x2F1D]={d='on'},
 [0x2F1E]={d='on'},
 [0x2F1F]={d='on'},
 [0x2F20]={d='on'},
 [0x2F21]={d='on'},
 [0x2F22]={d='on'},
 [0x2F23]={d='on'},
 [0x2F24]={d='on'},
 [0x2F25]={d='on'},
 [0x2F26]={d='on'},
 [0x2F27]={d='on'},
 [0x2F28]={d='on'},
 [0x2F29]={d='on'},
 [0x2F2A]={d='on'},
 [0x2F2B]={d='on'},
 [0x2F2C]={d='on'},
 [0x2F2D]={d='on'},
 [0x2F2E]={d='on'},
 [0x2F2F]={d='on'},
 [0x2F30]={d='on'},
 [0x2F31]={d='on'},
 [0x2F32]={d='on'},
 [0x2F33]={d='on'},
 [0x2F34]={d='on'},
 [0x2F35]={d='on'},
 [0x2F36]={d='on'},
 [0x2F37]={d='on'},
 [0x2F38]={d='on'},
 [0x2F39]={d='on'},
 [0x2F3A]={d='on'},
 [0x2F3B]={d='on'},
 [0x2F3C]={d='on'},
 [0x2F3D]={d='on'},
 [0x2F3E]={d='on'},
 [0x2F3F]={d='on'},
 [0x2F40]={d='on'},
 [0x2F41]={d='on'},
 [0x2F42]={d='on'},
 [0x2F43]={d='on'},
 [0x2F44]={d='on'},
 [0x2F45]={d='on'},
 [0x2F46]={d='on'},
 [0x2F47]={d='on'},
 [0x2F48]={d='on'},
 [0x2F49]={d='on'},
 [0x2F4A]={d='on'},
 [0x2F4B]={d='on'},
 [0x2F4C]={d='on'},
 [0x2F4D]={d='on'},
 [0x2F4E]={d='on'},
 [0x2F4F]={d='on'},
 [0x2F50]={d='on'},
 [0x2F51]={d='on'},
 [0x2F52]={d='on'},
 [0x2F53]={d='on'},
 [0x2F54]={d='on'},
 [0x2F55]={d='on'},
 [0x2F56]={d='on'},
 [0x2F57]={d='on'},
 [0x2F58]={d='on'},
 [0x2F59]={d='on'},
 [0x2F5A]={d='on'},
 [0x2F5B]={d='on'},
 [0x2F5C]={d='on'},
 [0x2F5D]={d='on'},
 [0x2F5E]={d='on'},
 [0x2F5F]={d='on'},
 [0x2F60]={d='on'},
 [0x2F61]={d='on'},
 [0x2F62]={d='on'},
 [0x2F63]={d='on'},
 [0x2F64]={d='on'},
 [0x2F65]={d='on'},
 [0x2F66]={d='on'},
 [0x2F67]={d='on'},
 [0x2F68]={d='on'},
 [0x2F69]={d='on'},
 [0x2F6A]={d='on'},
 [0x2F6B]={d='on'},
 [0x2F6C]={d='on'},
 [0x2F6D]={d='on'},
 [0x2F6E]={d='on'},
 [0x2F6F]={d='on'},
 [0x2F70]={d='on'},
 [0x2F71]={d='on'},
 [0x2F72]={d='on'},
 [0x2F73]={d='on'},
 [0x2F74]={d='on'},
 [0x2F75]={d='on'},
 [0x2F76]={d='on'},
 [0x2F77]={d='on'},
 [0x2F78]={d='on'},
 [0x2F79]={d='on'},
 [0x2F7A]={d='on'},
 [0x2F7B]={d='on'},
 [0x2F7C]={d='on'},
 [0x2F7D]={d='on'},
 [0x2F7E]={d='on'},
 [0x2F7F]={d='on'},
 [0x2F80]={d='on'},
 [0x2F81]={d='on'},
 [0x2F82]={d='on'},
 [0x2F83]={d='on'},
 [0x2F84]={d='on'},
 [0x2F85]={d='on'},
 [0x2F86]={d='on'},
 [0x2F87]={d='on'},
 [0x2F88]={d='on'},
 [0x2F89]={d='on'},
 [0x2F8A]={d='on'},
 [0x2F8B]={d='on'},
 [0x2F8C]={d='on'},
 [0x2F8D]={d='on'},
 [0x2F8E]={d='on'},
 [0x2F8F]={d='on'},
 [0x2F90]={d='on'},
 [0x2F91]={d='on'},
 [0x2F92]={d='on'},
 [0x2F93]={d='on'},
 [0x2F94]={d='on'},
 [0x2F95]={d='on'},
 [0x2F96]={d='on'},
 [0x2F97]={d='on'},
 [0x2F98]={d='on'},
 [0x2F99]={d='on'},
 [0x2F9A]={d='on'},
 [0x2F9B]={d='on'},
 [0x2F9C]={d='on'},
 [0x2F9D]={d='on'},
 [0x2F9E]={d='on'},
 [0x2F9F]={d='on'},
 [0x2FA0]={d='on'},
 [0x2FA1]={d='on'},
 [0x2FA2]={d='on'},
 [0x2FA3]={d='on'},
 [0x2FA4]={d='on'},
 [0x2FA5]={d='on'},
 [0x2FA6]={d='on'},
 [0x2FA7]={d='on'},
 [0x2FA8]={d='on'},
 [0x2FA9]={d='on'},
 [0x2FAA]={d='on'},
 [0x2FAB]={d='on'},
 [0x2FAC]={d='on'},
 [0x2FAD]={d='on'},
 [0x2FAE]={d='on'},
 [0x2FAF]={d='on'},
 [0x2FB0]={d='on'},
 [0x2FB1]={d='on'},
 [0x2FB2]={d='on'},
 [0x2FB3]={d='on'},
 [0x2FB4]={d='on'},
 [0x2FB5]={d='on'},
 [0x2FB6]={d='on'},
 [0x2FB7]={d='on'},
 [0x2FB8]={d='on'},
 [0x2FB9]={d='on'},
 [0x2FBA]={d='on'},
 [0x2FBB]={d='on'},
 [0x2FBC]={d='on'},
 [0x2FBD]={d='on'},
 [0x2FBE]={d='on'},
 [0x2FBF]={d='on'},
 [0x2FC0]={d='on'},
 [0x2FC1]={d='on'},
 [0x2FC2]={d='on'},
 [0x2FC3]={d='on'},
 [0x2FC4]={d='on'},
 [0x2FC5]={d='on'},
 [0x2FC6]={d='on'},
 [0x2FC7]={d='on'},
 [0x2FC8]={d='on'},
 [0x2FC9]={d='on'},
 [0x2FCA]={d='on'},
 [0x2FCB]={d='on'},
 [0x2FCC]={d='on'},
 [0x2FCD]={d='on'},
 [0x2FCE]={d='on'},
 [0x2FCF]={d='on'},
 [0x2FD0]={d='on'},
 [0x2FD1]={d='on'},
 [0x2FD2]={d='on'},
 [0x2FD3]={d='on'},
 [0x2FD4]={d='on'},
 [0x2FD5]={d='on'},
 [0x2FF0]={d='on'},
 [0x2FF1]={d='on'},
 [0x2FF2]={d='on'},
 [0x2FF3]={d='on'},
 [0x2FF4]={d='on'},
 [0x2FF5]={d='on'},
 [0x2FF6]={d='on'},
 [0x2FF7]={d='on'},
 [0x2FF8]={d='on'},
 [0x2FF9]={d='on'},
 [0x2FFA]={d='on'},
 [0x2FFB]={d='on'},
 [0x3000]={d='ws'},
 [0x3001]={d='on'},
 [0x3002]={d='on'},
 [0x3003]={d='on'},
 [0x3004]={d='on'},
 [0x3008]={d='on', m=0x3009},
 [0x3009]={d='on', m=0x3008},
 [0x300A]={d='on', m=0x300B},
 [0x300B]={d='on', m=0x300A},
 [0x300C]={d='on', m=0x300D},
 [0x300D]={d='on', m=0x300C},
 [0x300E]={d='on', m=0x300F},
 [0x300F]={d='on', m=0x300E},
 [0x3010]={d='on', m=0x3011},
 [0x3011]={d='on', m=0x3010},
 [0x3012]={d='on'},
 [0x3013]={d='on'},
 [0x3014]={d='on', m=0x3015},
 [0x3015]={d='on', m=0x3014},
 [0x3016]={d='on', m=0x3017},
 [0x3017]={d='on', m=0x3016},
 [0x3018]={d='on', m=0x3019},
 [0x3019]={d='on', m=0x3018},
 [0x301A]={d='on', m=0x301B},
 [0x301B]={d='on', m=0x301A},
 [0x301C]={d='on'},
 [0x301D]={d='on'},
 [0x301E]={d='on'},
 [0x301F]={d='on'},
 [0x3020]={d='on'},
 [0x302A]={d='nsm'},
 [0x302B]={d='nsm'},
 [0x302C]={d='nsm'},
 [0x302D]={d='nsm'},
 [0x3030]={d='on'},
 [0x3036]={d='on'},
 [0x3037]={d='on'},
 [0x303D]={d='on'},
 [0x303E]={d='on'},
 [0x303F]={d='on'},
 [0x3099]={d='nsm'},
 [0x309A]={d='nsm'},
 [0x309B]={d='on'},
 [0x309C]={d='on'},
 [0x30A0]={d='on'},
 [0x30FB]={d='on'},
 [0x31C0]={d='on'},
 [0x31C1]={d='on'},
 [0x31C2]={d='on'},
 [0x31C3]={d='on'},
 [0x31C4]={d='on'},
 [0x31C5]={d='on'},
 [0x31C6]={d='on'},
 [0x31C7]={d='on'},
 [0x31C8]={d='on'},
 [0x31C9]={d='on'},
 [0x31CA]={d='on'},
 [0x31CB]={d='on'},
 [0x31CC]={d='on'},
 [0x31CD]={d='on'},
 [0x31CE]={d='on'},
 [0x31CF]={d='on'},
 [0x31D0]={d='on'},
 [0x31D1]={d='on'},
 [0x31D2]={d='on'},
 [0x31D3]={d='on'},
 [0x31D4]={d='on'},
 [0x31D5]={d='on'},
 [0x31D6]={d='on'},
 [0x31D7]={d='on'},
 [0x31D8]={d='on'},
 [0x31D9]={d='on'},
 [0x31DA]={d='on'},
 [0x31DB]={d='on'},
 [0x31DC]={d='on'},
 [0x31DD]={d='on'},
 [0x31DE]={d='on'},
 [0x31DF]={d='on'},
 [0x31E0]={d='on'},
 [0x31E1]={d='on'},
 [0x31E2]={d='on'},
 [0x31E3]={d='on'},
 [0x321D]={d='on'},
 [0x321E]={d='on'},
 [0x3250]={d='on'},
 [0x3251]={d='on'},
 [0x3252]={d='on'},
 [0x3253]={d='on'},
 [0x3254]={d='on'},
 [0x3255]={d='on'},
 [0x3256]={d='on'},
 [0x3257]={d='on'},
 [0x3258]={d='on'},
 [0x3259]={d='on'},
 [0x325A]={d='on'},
 [0x325B]={d='on'},
 [0x325C]={d='on'},
 [0x325D]={d='on'},
 [0x325E]={d='on'},
 [0x325F]={d='on'},
 [0x327C]={d='on'},
 [0x327D]={d='on'},
 [0x327E]={d='on'},
 [0x32B1]={d='on'},
 [0x32B2]={d='on'},
 [0x32B3]={d='on'},
 [0x32B4]={d='on'},
 [0x32B5]={d='on'},
 [0x32B6]={d='on'},
 [0x32B7]={d='on'},
 [0x32B8]={d='on'},
 [0x32B9]={d='on'},
 [0x32BA]={d='on'},
 [0x32BB]={d='on'},
 [0x32BC]={d='on'},
 [0x32BD]={d='on'},
 [0x32BE]={d='on'},
 [0x32BF]={d='on'},
 [0x32CC]={d='on'},
 [0x32CD]={d='on'},
 [0x32CE]={d='on'},
 [0x32CF]={d='on'},
 [0x3377]={d='on'},
 [0x3378]={d='on'},
 [0x3379]={d='on'},
 [0x337A]={d='on'},
 [0x33DE]={d='on'},
 [0x33DF]={d='on'},
 [0x33FF]={d='on'},
 [0x4DC0]={d='on'},
 [0x4DC1]={d='on'},
 [0x4DC2]={d='on'},
 [0x4DC3]={d='on'},
 [0x4DC4]={d='on'},
 [0x4DC5]={d='on'},
 [0x4DC6]={d='on'},
 [0x4DC7]={d='on'},
 [0x4DC8]={d='on'},
 [0x4DC9]={d='on'},
 [0x4DCA]={d='on'},
 [0x4DCB]={d='on'},
 [0x4DCC]={d='on'},
 [0x4DCD]={d='on'},
 [0x4DCE]={d='on'},
 [0x4DCF]={d='on'},
 [0x4DD0]={d='on'},
 [0x4DD1]={d='on'},
 [0x4DD2]={d='on'},
 [0x4DD3]={d='on'},
 [0x4DD4]={d='on'},
 [0x4DD5]={d='on'},
 [0x4DD6]={d='on'},
 [0x4DD7]={d='on'},
 [0x4DD8]={d='on'},
 [0x4DD9]={d='on'},
 [0x4DDA]={d='on'},
 [0x4DDB]={d='on'},
 [0x4DDC]={d='on'},
 [0x4DDD]={d='on'},
 [0x4DDE]={d='on'},
 [0x4DDF]={d='on'},
 [0x4DE0]={d='on'},
 [0x4DE1]={d='on'},
 [0x4DE2]={d='on'},
 [0x4DE3]={d='on'},
 [0x4DE4]={d='on'},
 [0x4DE5]={d='on'},
 [0x4DE6]={d='on'},
 [0x4DE7]={d='on'},
 [0x4DE8]={d='on'},
 [0x4DE9]={d='on'},
 [0x4DEA]={d='on'},
 [0x4DEB]={d='on'},
 [0x4DEC]={d='on'},
 [0x4DED]={d='on'},
 [0x4DEE]={d='on'},
 [0x4DEF]={d='on'},
 [0x4DF0]={d='on'},
 [0x4DF1]={d='on'},
 [0x4DF2]={d='on'},
 [0x4DF3]={d='on'},
 [0x4DF4]={d='on'},
 [0x4DF5]={d='on'},
 [0x4DF6]={d='on'},
 [0x4DF7]={d='on'},
 [0x4DF8]={d='on'},
 [0x4DF9]={d='on'},
 [0x4DFA]={d='on'},
 [0x4DFB]={d='on'},
 [0x4DFC]={d='on'},
 [0x4DFD]={d='on'},
 [0x4DFE]={d='on'},
 [0x4DFF]={d='on'},
 [0xA490]={d='on'},
 [0xA491]={d='on'},
 [0xA492]={d='on'},
 [0xA493]={d='on'},
 [0xA494]={d='on'},
 [0xA495]={d='on'},
 [0xA496]={d='on'},
 [0xA497]={d='on'},
 [0xA498]={d='on'},
 [0xA499]={d='on'},
 [0xA49A]={d='on'},
 [0xA49B]={d='on'},
 [0xA49C]={d='on'},
 [0xA49D]={d='on'},
 [0xA49E]={d='on'},
 [0xA49F]={d='on'},
 [0xA4A0]={d='on'},
 [0xA4A1]={d='on'},
 [0xA4A2]={d='on'},
 [0xA4A3]={d='on'},
 [0xA4A4]={d='on'},
 [0xA4A5]={d='on'},
 [0xA4A6]={d='on'},
 [0xA4A7]={d='on'},
 [0xA4A8]={d='on'},
 [0xA4A9]={d='on'},
 [0xA4AA]={d='on'},
 [0xA4AB]={d='on'},
 [0xA4AC]={d='on'},
 [0xA4AD]={d='on'},
 [0xA4AE]={d='on'},
 [0xA4AF]={d='on'},
 [0xA4B0]={d='on'},
 [0xA4B1]={d='on'},
 [0xA4B2]={d='on'},
 [0xA4B3]={d='on'},
 [0xA4B4]={d='on'},
 [0xA4B5]={d='on'},
 [0xA4B6]={d='on'},
 [0xA4B7]={d='on'},
 [0xA4B8]={d='on'},
 [0xA4B9]={d='on'},
 [0xA4BA]={d='on'},
 [0xA4BB]={d='on'},
 [0xA4BC]={d='on'},
 [0xA4BD]={d='on'},
 [0xA4BE]={d='on'},
 [0xA4BF]={d='on'},
 [0xA4C0]={d='on'},
 [0xA4C1]={d='on'},
 [0xA4C2]={d='on'},
 [0xA4C3]={d='on'},
 [0xA4C4]={d='on'},
 [0xA4C5]={d='on'},
 [0xA4C6]={d='on'},
 [0xA60D]={d='on'},
 [0xA60E]={d='on'},
 [0xA60F]={d='on'},
 [0xA66F]={d='nsm'},
 [0xA670]={d='nsm'},
 [0xA671]={d='nsm'},
 [0xA672]={d='nsm'},
 [0xA673]={d='on'},
 [0xA674]={d='nsm'},
 [0xA675]={d='nsm'},
 [0xA676]={d='nsm'},
 [0xA677]={d='nsm'},
 [0xA678]={d='nsm'},
 [0xA679]={d='nsm'},
 [0xA67A]={d='nsm'},
 [0xA67B]={d='nsm'},
 [0xA67C]={d='nsm'},
 [0xA67D]={d='nsm'},
 [0xA67E]={d='on'},
 [0xA67F]={d='on'},
 [0xA69E]={d='nsm'},
 [0xA69F]={d='nsm'},
 [0xA6F0]={d='nsm'},
 [0xA6F1]={d='nsm'},
 [0xA700]={d='on'},
 [0xA701]={d='on'},
 [0xA702]={d='on'},
 [0xA703]={d='on'},
 [0xA704]={d='on'},
 [0xA705]={d='on'},
 [0xA706]={d='on'},
 [0xA707]={d='on'},
 [0xA708]={d='on'},
 [0xA709]={d='on'},
 [0xA70A]={d='on'},
 [0xA70B]={d='on'},
 [0xA70C]={d='on'},
 [0xA70D]={d='on'},
 [0xA70E]={d='on'},
 [0xA70F]={d='on'},
 [0xA710]={d='on'},
 [0xA711]={d='on'},
 [0xA712]={d='on'},
 [0xA713]={d='on'},
 [0xA714]={d='on'},
 [0xA715]={d='on'},
 [0xA716]={d='on'},
 [0xA717]={d='on'},
 [0xA718]={d='on'},
 [0xA719]={d='on'},
 [0xA71A]={d='on'},
 [0xA71B]={d='on'},
 [0xA71C]={d='on'},
 [0xA71D]={d='on'},
 [0xA71E]={d='on'},
 [0xA71F]={d='on'},
 [0xA720]={d='on'},
 [0xA721]={d='on'},
 [0xA788]={d='on'},
 [0xA802]={d='nsm'},
 [0xA806]={d='nsm'},
 [0xA80B]={d='nsm'},
 [0xA825]={d='nsm'},
 [0xA826]={d='nsm'},
 [0xA828]={d='on'},
 [0xA829]={d='on'},
 [0xA82A]={d='on'},
 [0xA82B]={d='on'},
 [0xA838]={d='et'},
 [0xA839]={d='et'},
 [0xA874]={d='on'},
 [0xA875]={d='on'},
 [0xA876]={d='on'},
 [0xA877]={d='on'},
 [0xA8C4]={d='nsm'},
 [0xA8C5]={d='nsm'},
 [0xA8E0]={d='nsm'},
 [0xA8E1]={d='nsm'},
 [0xA8E2]={d='nsm'},
 [0xA8E3]={d='nsm'},
 [0xA8E4]={d='nsm'},
 [0xA8E5]={d='nsm'},
 [0xA8E6]={d='nsm'},
 [0xA8E7]={d='nsm'},
 [0xA8E8]={d='nsm'},
 [0xA8E9]={d='nsm'},
 [0xA8EA]={d='nsm'},
 [0xA8EB]={d='nsm'},
 [0xA8EC]={d='nsm'},
 [0xA8ED]={d='nsm'},
 [0xA8EE]={d='nsm'},
 [0xA8EF]={d='nsm'},
 [0xA8F0]={d='nsm'},
 [0xA8F1]={d='nsm'},
 [0xA926]={d='nsm'},
 [0xA927]={d='nsm'},
 [0xA928]={d='nsm'},
 [0xA929]={d='nsm'},
 [0xA92A]={d='nsm'},
 [0xA92B]={d='nsm'},
 [0xA92C]={d='nsm'},
 [0xA92D]={d='nsm'},
 [0xA947]={d='nsm'},
 [0xA948]={d='nsm'},
 [0xA949]={d='nsm'},
 [0xA94A]={d='nsm'},
 [0xA94B]={d='nsm'},
 [0xA94C]={d='nsm'},
 [0xA94D]={d='nsm'},
 [0xA94E]={d='nsm'},
 [0xA94F]={d='nsm'},
 [0xA950]={d='nsm'},
 [0xA951]={d='nsm'},
 [0xA980]={d='nsm'},
 [0xA981]={d='nsm'},
 [0xA982]={d='nsm'},
 [0xA9B3]={d='nsm'},
 [0xA9B6]={d='nsm'},
 [0xA9B7]={d='nsm'},
 [0xA9B8]={d='nsm'},
 [0xA9B9]={d='nsm'},
 [0xA9BC]={d='nsm'},
 [0xA9E5]={d='nsm'},
 [0xAA29]={d='nsm'},
 [0xAA2A]={d='nsm'},
 [0xAA2B]={d='nsm'},
 [0xAA2C]={d='nsm'},
 [0xAA2D]={d='nsm'},
 [0xAA2E]={d='nsm'},
 [0xAA31]={d='nsm'},
 [0xAA32]={d='nsm'},
 [0xAA35]={d='nsm'},
 [0xAA36]={d='nsm'},
 [0xAA43]={d='nsm'},
 [0xAA4C]={d='nsm'},
 [0xAA7C]={d='nsm'},
 [0xAAB0]={d='nsm'},
 [0xAAB2]={d='nsm'},
 [0xAAB3]={d='nsm'},
 [0xAAB4]={d='nsm'},
 [0xAAB7]={d='nsm'},
 [0xAAB8]={d='nsm'},
 [0xAABE]={d='nsm'},
 [0xAABF]={d='nsm'},
 [0xAAC1]={d='nsm'},
 [0xAAEC]={d='nsm'},
 [0xAAED]={d='nsm'},
 [0xAAF6]={d='nsm'},
 [0xABE5]={d='nsm'},
 [0xABE8]={d='nsm'},
 [0xABED]={d='nsm'},
 [0xFB1E]={d='nsm'},
 [0xFB29]={d='es'},
 [0xFD3E]={d='on'},
 [0xFD3F]={d='on'},
 [0xFDFD]={d='on'},
 [0xFE10]={d='on'},
 [0xFE11]={d='on'},
 [0xFE12]={d='on'},
 [0xFE13]={d='on'},
 [0xFE14]={d='on'},
 [0xFE15]={d='on'},
 [0xFE16]={d='on'},
 [0xFE17]={d='on'},
 [0xFE18]={d='on'},
 [0xFE19]={d='on'},
 [0xFE20]={d='nsm'},
 [0xFE21]={d='nsm'},
 [0xFE22]={d='nsm'},
 [0xFE23]={d='nsm'},
 [0xFE24]={d='nsm'},
 [0xFE25]={d='nsm'},
 [0xFE26]={d='nsm'},
 [0xFE27]={d='nsm'},
 [0xFE28]={d='nsm'},
 [0xFE29]={d='nsm'},
 [0xFE2A]={d='nsm'},
 [0xFE2B]={d='nsm'},
 [0xFE2C]={d='nsm'},
 [0xFE2D]={d='nsm'},
 [0xFE2E]={d='nsm'},
 [0xFE2F]={d='nsm'},
 [0xFE30]={d='on'},
 [0xFE31]={d='on'},
 [0xFE32]={d='on'},
 [0xFE33]={d='on'},
 [0xFE34]={d='on'},
 [0xFE35]={d='on'},
 [0xFE36]={d='on'},
 [0xFE37]={d='on'},
 [0xFE38]={d='on'},
 [0xFE39]={d='on'},
 [0xFE3A]={d='on'},
 [0xFE3B]={d='on'},
 [0xFE3C]={d='on'},
 [0xFE3D]={d='on'},
 [0xFE3E]={d='on'},
 [0xFE3F]={d='on'},
 [0xFE40]={d='on'},
 [0xFE41]={d='on'},
 [0xFE42]={d='on'},
 [0xFE43]={d='on'},
 [0xFE44]={d='on'},
 [0xFE45]={d='on'},
 [0xFE46]={d='on'},
 [0xFE47]={d='on'},
 [0xFE48]={d='on'},
 [0xFE49]={d='on'},
 [0xFE4A]={d='on'},
 [0xFE4B]={d='on'},
 [0xFE4C]={d='on'},
 [0xFE4D]={d='on'},
 [0xFE4E]={d='on'},
 [0xFE4F]={d='on'},
 [0xFE50]={d='cs'},
 [0xFE51]={d='on'},
 [0xFE52]={d='cs'},
 [0xFE54]={d='on'},
 [0xFE55]={d='cs'},
 [0xFE56]={d='on'},
 [0xFE57]={d='on'},
 [0xFE58]={d='on'},
 [0xFE59]={d='on', m=0xFE5A},
 [0xFE5A]={d='on', m=0xFE59},
 [0xFE5B]={d='on', m=0xFE5C},
 [0xFE5C]={d='on', m=0xFE5B},
 [0xFE5D]={d='on', m=0xFE5E},
 [0xFE5E]={d='on', m=0xFE5D},
 [0xFE5F]={d='et'},
 [0xFE60]={d='on'},
 [0xFE61]={d='on'},
 [0xFE62]={d='es'},
 [0xFE63]={d='es'},
 [0xFE64]={d='on', m=0xFE65},
 [0xFE65]={d='on', m=0xFE64},
 [0xFE66]={d='on'},
 [0xFE68]={d='on'},
 [0xFE69]={d='et'},
 [0xFE6A]={d='et'},
 [0xFE6B]={d='on'},
 [0xFEFF]={d='bn'},
 [0xFF01]={d='on'},
 [0xFF02]={d='on'},
 [0xFF03]={d='et'},
 [0xFF04]={d='et'},
 [0xFF05]={d='et'},
 [0xFF06]={d='on'},
 [0xFF07]={d='on'},
 [0xFF08]={d='on', m=0xFF09},
 [0xFF09]={d='on', m=0xFF08},
 [0xFF0A]={d='on'},
 [0xFF0B]={d='es'},
 [0xFF0C]={d='cs'},
 [0xFF0D]={d='es'},
 [0xFF0E]={d='cs'},
 [0xFF0F]={d='cs'},
 [0xFF10]={d='en'},
 [0xFF11]={d='en'},
 [0xFF12]={d='en'},
 [0xFF13]={d='en'},
 [0xFF14]={d='en'},
 [0xFF15]={d='en'},
 [0xFF16]={d='en'},
 [0xFF17]={d='en'},
 [0xFF18]={d='en'},
 [0xFF19]={d='en'},
 [0xFF1A]={d='cs'},
 [0xFF1B]={d='on'},
 [0xFF1C]={d='on', m=0xFF1E},
 [0xFF1D]={d='on'},
 [0xFF1E]={d='on', m=0xFF1C},
 [0xFF1F]={d='on'},
 [0xFF20]={d='on'},
 [0xFF3B]={d='on', m=0xFF3D},
 [0xFF3C]={d='on'},
 [0xFF3D]={d='on', m=0xFF3B},
 [0xFF3E]={d='on'},
 [0xFF3F]={d='on'},
 [0xFF40]={d='on'},
 [0xFF5B]={d='on', m=0xFF5D},
 [0xFF5C]={d='on'},
 [0xFF5D]={d='on', m=0xFF5B},
 [0xFF5E]={d='on'},
 [0xFF5F]={d='on', m=0xFF60},
 [0xFF60]={d='on', m=0xFF5F},
 [0xFF61]={d='on'},
 [0xFF62]={d='on', m=0xFF63},
 [0xFF63]={d='on', m=0xFF62},
 [0xFF64]={d='on'},
 [0xFF65]={d='on'},
 [0xFFE0]={d='et'},
 [0xFFE1]={d='et'},
 [0xFFE2]={d='on'},
 [0xFFE3]={d='on'},
 [0xFFE4]={d='on'},
 [0xFFE5]={d='et'},
 [0xFFE6]={d='et'},
 [0xFFE8]={d='on'},
 [0xFFE9]={d='on'},
 [0xFFEA]={d='on'},
 [0xFFEB]={d='on'},
 [0xFFEC]={d='on'},
 [0xFFED]={d='on'},
 [0xFFEE]={d='on'},
 [0xFFF9]={d='on'},
 [0xFFFA]={d='on'},
 [0xFFFB]={d='on'},
 [0xFFFC]={d='on'},
 [0xFFFD]={d='on'},
 [0x10101]={d='on'},
 [0x10140]={d='on'},
 [0x10141]={d='on'},
 [0x10142]={d='on'},
 [0x10143]={d='on'},
 [0x10144]={d='on'},
 [0x10145]={d='on'},
 [0x10146]={d='on'},
 [0x10147]={d='on'},
 [0x10148]={d='on'},
 [0x10149]={d='on'},
 [0x1014A]={d='on'},
 [0x1014B]={d='on'},
 [0x1014C]={d='on'},
 [0x1014D]={d='on'},
 [0x1014E]={d='on'},
 [0x1014F]={d='on'},
 [0x10150]={d='on'},
 [0x10151]={d='on'},
 [0x10152]={d='on'},
 [0x10153]={d='on'},
 [0x10154]={d='on'},
 [0x10155]={d='on'},
 [0x10156]={d='on'},
 [0x10157]={d='on'},
 [0x10158]={d='on'},
 [0x10159]={d='on'},
 [0x1015A]={d='on'},
 [0x1015B]={d='on'},
 [0x1015C]={d='on'},
 [0x1015D]={d='on'},
 [0x1015E]={d='on'},
 [0x1015F]={d='on'},
 [0x10160]={d='on'},
 [0x10161]={d='on'},
 [0x10162]={d='on'},
 [0x10163]={d='on'},
 [0x10164]={d='on'},
 [0x10165]={d='on'},
 [0x10166]={d='on'},
 [0x10167]={d='on'},
 [0x10168]={d='on'},
 [0x10169]={d='on'},
 [0x1016A]={d='on'},
 [0x1016B]={d='on'},
 [0x1016C]={d='on'},
 [0x1016D]={d='on'},
 [0x1016E]={d='on'},
 [0x1016F]={d='on'},
 [0x10170]={d='on'},
 [0x10171]={d='on'},
 [0x10172]={d='on'},
 [0x10173]={d='on'},
 [0x10174]={d='on'},
 [0x10175]={d='on'},
 [0x10176]={d='on'},
 [0x10177]={d='on'},
 [0x10178]={d='on'},
 [0x10179]={d='on'},
 [0x1017A]={d='on'},
 [0x1017B]={d='on'},
 [0x1017C]={d='on'},
 [0x1017D]={d='on'},
 [0x1017E]={d='on'},
 [0x1017F]={d='on'},
 [0x10180]={d='on'},
 [0x10181]={d='on'},
 [0x10182]={d='on'},
 [0x10183]={d='on'},
 [0x10184]={d='on'},
 [0x10185]={d='on'},
 [0x10186]={d='on'},
 [0x10187]={d='on'},
 [0x10188]={d='on'},
 [0x10189]={d='on'},
 [0x1018A]={d='on'},
 [0x1018B]={d='on'},
 [0x1018C]={d='on'},
 [0x10190]={d='on'},
 [0x10191]={d='on'},
 [0x10192]={d='on'},
 [0x10193]={d='on'},
 [0x10194]={d='on'},
 [0x10195]={d='on'},
 [0x10196]={d='on'},
 [0x10197]={d='on'},
 [0x10198]={d='on'},
 [0x10199]={d='on'},
 [0x1019A]={d='on'},
 [0x1019B]={d='on'},
 [0x101A0]={d='on'},
 [0x101FD]={d='nsm'},
 [0x102E0]={d='nsm'},
 [0x102E1]={d='en'},
 [0x102E2]={d='en'},
 [0x102E3]={d='en'},
 [0x102E4]={d='en'},
 [0x102E5]={d='en'},
 [0x102E6]={d='en'},
 [0x102E7]={d='en'},
 [0x102E8]={d='en'},
 [0x102E9]={d='en'},
 [0x102EA]={d='en'},
 [0x102EB]={d='en'},
 [0x102EC]={d='en'},
 [0x102ED]={d='en'},
 [0x102EE]={d='en'},
 [0x102EF]={d='en'},
 [0x102F0]={d='en'},
 [0x102F1]={d='en'},
 [0x102F2]={d='en'},
 [0x102F3]={d='en'},
 [0x102F4]={d='en'},
 [0x102F5]={d='en'},
 [0x102F6]={d='en'},
 [0x102F7]={d='en'},
 [0x102F8]={d='en'},
 [0x102F9]={d='en'},
 [0x102FA]={d='en'},
 [0x102FB]={d='en'},
 [0x10376]={d='nsm'},
 [0x10377]={d='nsm'},
 [0x10378]={d='nsm'},
 [0x10379]={d='nsm'},
 [0x1037A]={d='nsm'},
 [0x1091F]={d='on'},
 [0x10A01]={d='nsm'},
 [0x10A02]={d='nsm'},
 [0x10A03]={d='nsm'},
 [0x10A05]={d='nsm'},
 [0x10A06]={d='nsm'},
 [0x10A0C]={d='nsm'},
 [0x10A0D]={d='nsm'},
 [0x10A0E]={d='nsm'},
 [0x10A0F]={d='nsm'},
 [0x10A38]={d='nsm'},
 [0x10A39]={d='nsm'},
 [0x10A3A]={d='nsm'},
 [0x10A3F]={d='nsm'},
 [0x10AE5]={d='nsm'},
 [0x10AE6]={d='nsm'},
 [0x10B39]={d='on'},
 [0x10B3A]={d='on'},
 [0x10B3B]={d='on'},
 [0x10B3C]={d='on'},
 [0x10B3D]={d='on'},
 [0x10B3E]={d='on'},
 [0x10B3F]={d='on'},
 [0x10C80]={d='r'},
 [0x10C81]={d='r'},
 [0x10C82]={d='r'},
 [0x10C83]={d='r'},
 [0x10C84]={d='r'},
 [0x10C85]={d='r'},
 [0x10C86]={d='r'},
 [0x10C87]={d='r'},
 [0x10C88]={d='r'},
 [0x10C89]={d='r'},
 [0x10C8A]={d='r'},
 [0x10C8B]={d='r'},
 [0x10C8C]={d='r'},
 [0x10C8D]={d='r'},
 [0x10C8E]={d='r'},
 [0x10C8F]={d='r'},
 [0x10C90]={d='r'},
 [0x10C91]={d='r'},
 [0x10C92]={d='r'},
 [0x10C93]={d='r'},
 [0x10C94]={d='r'},
 [0x10C95]={d='r'},
 [0x10C96]={d='r'},
 [0x10C97]={d='r'},
 [0x10C98]={d='r'},
 [0x10C99]={d='r'},
 [0x10C9A]={d='r'},
 [0x10C9B]={d='r'},
 [0x10C9C]={d='r'},
 [0x10C9D]={d='r'},
 [0x10C9E]={d='r'},
 [0x10C9F]={d='r'},
 [0x10CA0]={d='r'},
 [0x10CA1]={d='r'},
 [0x10CA2]={d='r'},
 [0x10CA3]={d='r'},
 [0x10CA4]={d='r'},
 [0x10CA5]={d='r'},
 [0x10CA6]={d='r'},
 [0x10CA7]={d='r'},
 [0x10CA8]={d='r'},
 [0x10CA9]={d='r'},
 [0x10CAA]={d='r'},
 [0x10CAB]={d='r'},
 [0x10CAC]={d='r'},
 [0x10CAD]={d='r'},
 [0x10CAE]={d='r'},
 [0x10CAF]={d='r'},
 [0x10CB0]={d='r'},
 [0x10CB1]={d='r'},
 [0x10CB2]={d='r'},
 [0x10CC0]={d='r'},
 [0x10CC1]={d='r'},
 [0x10CC2]={d='r'},
 [0x10CC3]={d='r'},
 [0x10CC4]={d='r'},
 [0x10CC5]={d='r'},
 [0x10CC6]={d='r'},
 [0x10CC7]={d='r'},
 [0x10CC8]={d='r'},
 [0x10CC9]={d='r'},
 [0x10CCA]={d='r'},
 [0x10CCB]={d='r'},
 [0x10CCC]={d='r'},
 [0x10CCD]={d='r'},
 [0x10CCE]={d='r'},
 [0x10CCF]={d='r'},
 [0x10CD0]={d='r'},
 [0x10CD1]={d='r'},
 [0x10CD2]={d='r'},
 [0x10CD3]={d='r'},
 [0x10CD4]={d='r'},
 [0x10CD5]={d='r'},
 [0x10CD6]={d='r'},
 [0x10CD7]={d='r'},
 [0x10CD8]={d='r'},
 [0x10CD9]={d='r'},
 [0x10CDA]={d='r'},
 [0x10CDB]={d='r'},
 [0x10CDC]={d='r'},
 [0x10CDD]={d='r'},
 [0x10CDE]={d='r'},
 [0x10CDF]={d='r'},
 [0x10CE0]={d='r'},
 [0x10CE1]={d='r'},
 [0x10CE2]={d='r'},
 [0x10CE3]={d='r'},
 [0x10CE4]={d='r'},
 [0x10CE5]={d='r'},
 [0x10CE6]={d='r'},
 [0x10CE7]={d='r'},
 [0x10CE8]={d='r'},
 [0x10CE9]={d='r'},
 [0x10CEA]={d='r'},
 [0x10CEB]={d='r'},
 [0x10CEC]={d='r'},
 [0x10CED]={d='r'},
 [0x10CEE]={d='r'},
 [0x10CEF]={d='r'},
 [0x10CF0]={d='r'},
 [0x10CF1]={d='r'},
 [0x10CF2]={d='r'},
 [0x10CFA]={d='r'},
 [0x10CFB]={d='r'},
 [0x10CFC]={d='r'},
 [0x10CFD]={d='r'},
 [0x10CFE]={d='r'},
 [0x10CFF]={d='r'},
 [0x10E60]={d='an'},
 [0x10E61]={d='an'},
 [0x10E62]={d='an'},
 [0x10E63]={d='an'},
 [0x10E64]={d='an'},
 [0x10E65]={d='an'},
 [0x10E66]={d='an'},
 [0x10E67]={d='an'},
 [0x10E68]={d='an'},
 [0x10E69]={d='an'},
 [0x10E6A]={d='an'},
 [0x10E6B]={d='an'},
 [0x10E6C]={d='an'},
 [0x10E6D]={d='an'},
 [0x10E6E]={d='an'},
 [0x10E6F]={d='an'},
 [0x10E70]={d='an'},
 [0x10E71]={d='an'},
 [0x10E72]={d='an'},
 [0x10E73]={d='an'},
 [0x10E74]={d='an'},
 [0x10E75]={d='an'},
 [0x10E76]={d='an'},
 [0x10E77]={d='an'},
 [0x10E78]={d='an'},
 [0x10E79]={d='an'},
 [0x10E7A]={d='an'},
 [0x10E7B]={d='an'},
 [0x10E7C]={d='an'},
 [0x10E7D]={d='an'},
 [0x10E7E]={d='an'},
 [0x11001]={d='nsm'},
 [0x11038]={d='nsm'},
 [0x11039]={d='nsm'},
 [0x1103A]={d='nsm'},
 [0x1103B]={d='nsm'},
 [0x1103C]={d='nsm'},
 [0x1103D]={d='nsm'},
 [0x1103E]={d='nsm'},
 [0x1103F]={d='nsm'},
 [0x11040]={d='nsm'},
 [0x11041]={d='nsm'},
 [0x11042]={d='nsm'},
 [0x11043]={d='nsm'},
 [0x11044]={d='nsm'},
 [0x11045]={d='nsm'},
 [0x11046]={d='nsm'},
 [0x11052]={d='on'},
 [0x11053]={d='on'},
 [0x11054]={d='on'},
 [0x11055]={d='on'},
 [0x11056]={d='on'},
 [0x11057]={d='on'},
 [0x11058]={d='on'},
 [0x11059]={d='on'},
 [0x1105A]={d='on'},
 [0x1105B]={d='on'},
 [0x1105C]={d='on'},
 [0x1105D]={d='on'},
 [0x1105E]={d='on'},
 [0x1105F]={d='on'},
 [0x11060]={d='on'},
 [0x11061]={d='on'},
 [0x11062]={d='on'},
 [0x11063]={d='on'},
 [0x11064]={d='on'},
 [0x11065]={d='on'},
 [0x1107F]={d='nsm'},
 [0x11080]={d='nsm'},
 [0x11081]={d='nsm'},
 [0x110B3]={d='nsm'},
 [0x110B4]={d='nsm'},
 [0x110B5]={d='nsm'},
 [0x110B6]={d='nsm'},
 [0x110B9]={d='nsm'},
 [0x110BA]={d='nsm'},
 [0x11100]={d='nsm'},
 [0x11101]={d='nsm'},
 [0x11102]={d='nsm'},
 [0x11127]={d='nsm'},
 [0x11128]={d='nsm'},
 [0x11129]={d='nsm'},
 [0x1112A]={d='nsm'},
 [0x1112B]={d='nsm'},
 [0x1112D]={d='nsm'},
 [0x1112E]={d='nsm'},
 [0x1112F]={d='nsm'},
 [0x11130]={d='nsm'},
 [0x11131]={d='nsm'},
 [0x11132]={d='nsm'},
 [0x11133]={d='nsm'},
 [0x11134]={d='nsm'},
 [0x11173]={d='nsm'},
 [0x11180]={d='nsm'},
 [0x11181]={d='nsm'},
 [0x111B6]={d='nsm'},
 [0x111B7]={d='nsm'},
 [0x111B8]={d='nsm'},
 [0x111B9]={d='nsm'},
 [0x111BA]={d='nsm'},
 [0x111BB]={d='nsm'},
 [0x111BC]={d='nsm'},
 [0x111BD]={d='nsm'},
 [0x111BE]={d='nsm'},
 [0x111CA]={d='nsm'},
 [0x111CB]={d='nsm'},
 [0x111CC]={d='nsm'},
 [0x1122F]={d='nsm'},
 [0x11230]={d='nsm'},
 [0x11231]={d='nsm'},
 [0x11234]={d='nsm'},
 [0x11236]={d='nsm'},
 [0x11237]={d='nsm'},
 [0x1123E]={d='nsm'},
 [0x112DF]={d='nsm'},
 [0x112E3]={d='nsm'},
 [0x112E4]={d='nsm'},
 [0x112E5]={d='nsm'},
 [0x112E6]={d='nsm'},
 [0x112E7]={d='nsm'},
 [0x112E8]={d='nsm'},
 [0x112E9]={d='nsm'},
 [0x112EA]={d='nsm'},
 [0x11300]={d='nsm'},
 [0x11301]={d='nsm'},
 [0x1133C]={d='nsm'},
 [0x11340]={d='nsm'},
 [0x11366]={d='nsm'},
 [0x11367]={d='nsm'},
 [0x11368]={d='nsm'},
 [0x11369]={d='nsm'},
 [0x1136A]={d='nsm'},
 [0x1136B]={d='nsm'},
 [0x1136C]={d='nsm'},
 [0x11370]={d='nsm'},
 [0x11371]={d='nsm'},
 [0x11372]={d='nsm'},
 [0x11373]={d='nsm'},
 [0x11374]={d='nsm'},
 [0x11438]={d='nsm'},
 [0x11439]={d='nsm'},
 [0x1143A]={d='nsm'},
 [0x1143B]={d='nsm'},
 [0x1143C]={d='nsm'},
 [0x1143D]={d='nsm'},
 [0x1143E]={d='nsm'},
 [0x1143F]={d='nsm'},
 [0x11442]={d='nsm'},
 [0x11443]={d='nsm'},
 [0x11444]={d='nsm'},
 [0x11446]={d='nsm'},
 [0x114B3]={d='nsm'},
 [0x114B4]={d='nsm'},
 [0x114B5]={d='nsm'},
 [0x114B6]={d='nsm'},
 [0x114B7]={d='nsm'},
 [0x114B8]={d='nsm'},
 [0x114BA]={d='nsm'},
 [0x114BF]={d='nsm'},
 [0x114C0]={d='nsm'},
 [0x114C2]={d='nsm'},
 [0x114C3]={d='nsm'},
 [0x115B2]={d='nsm'},
 [0x115B3]={d='nsm'},
 [0x115B4]={d='nsm'},
 [0x115B5]={d='nsm'},
 [0x115BC]={d='nsm'},
 [0x115BD]={d='nsm'},
 [0x115BF]={d='nsm'},
 [0x115C0]={d='nsm'},
 [0x115DC]={d='nsm'},
 [0x115DD]={d='nsm'},
 [0x11633]={d='nsm'},
 [0x11634]={d='nsm'},
 [0x11635]={d='nsm'},
 [0x11636]={d='nsm'},
 [0x11637]={d='nsm'},
 [0x11638]={d='nsm'},
 [0x11639]={d='nsm'},
 [0x1163A]={d='nsm'},
 [0x1163D]={d='nsm'},
 [0x1163F]={d='nsm'},
 [0x11640]={d='nsm'},
 [0x11660]={d='on'},
 [0x11661]={d='on'},
 [0x11662]={d='on'},
 [0x11663]={d='on'},
 [0x11664]={d='on'},
 [0x11665]={d='on'},
 [0x11666]={d='on'},
 [0x11667]={d='on'},
 [0x11668]={d='on'},
 [0x11669]={d='on'},
 [0x1166A]={d='on'},
 [0x1166B]={d='on'},
 [0x1166C]={d='on'},
 [0x116AB]={d='nsm'},
 [0x116AD]={d='nsm'},
 [0x116B0]={d='nsm'},
 [0x116B1]={d='nsm'},
 [0x116B2]={d='nsm'},
 [0x116B3]={d='nsm'},
 [0x116B4]={d='nsm'},
 [0x116B5]={d='nsm'},
 [0x116B7]={d='nsm'},
 [0x1171D]={d='nsm'},
 [0x1171E]={d='nsm'},
 [0x1171F]={d='nsm'},
 [0x11722]={d='nsm'},
 [0x11723]={d='nsm'},
 [0x11724]={d='nsm'},
 [0x11725]={d='nsm'},
 [0x11727]={d='nsm'},
 [0x11728]={d='nsm'},
 [0x11729]={d='nsm'},
 [0x1172A]={d='nsm'},
 [0x1172B]={d='nsm'},
 [0x11C30]={d='nsm'},
 [0x11C31]={d='nsm'},
 [0x11C32]={d='nsm'},
 [0x11C33]={d='nsm'},
 [0x11C34]={d='nsm'},
 [0x11C35]={d='nsm'},
 [0x11C36]={d='nsm'},
 [0x11C38]={d='nsm'},
 [0x11C39]={d='nsm'},
 [0x11C3A]={d='nsm'},
 [0x11C3B]={d='nsm'},
 [0x11C3C]={d='nsm'},
 [0x11C3D]={d='nsm'},
 [0x11C92]={d='nsm'},
 [0x11C93]={d='nsm'},
 [0x11C94]={d='nsm'},
 [0x11C95]={d='nsm'},
 [0x11C96]={d='nsm'},
 [0x11C97]={d='nsm'},
 [0x11C98]={d='nsm'},
 [0x11C99]={d='nsm'},
 [0x11C9A]={d='nsm'},
 [0x11C9B]={d='nsm'},
 [0x11C9C]={d='nsm'},
 [0x11C9D]={d='nsm'},
 [0x11C9E]={d='nsm'},
 [0x11C9F]={d='nsm'},
 [0x11CA0]={d='nsm'},
 [0x11CA1]={d='nsm'},
 [0x11CA2]={d='nsm'},
 [0x11CA3]={d='nsm'},
 [0x11CA4]={d='nsm'},
 [0x11CA5]={d='nsm'},
 [0x11CA6]={d='nsm'},
 [0x11CA7]={d='nsm'},
 [0x11CAA]={d='nsm'},
 [0x11CAB]={d='nsm'},
 [0x11CAC]={d='nsm'},
 [0x11CAD]={d='nsm'},
 [0x11CAE]={d='nsm'},
 [0x11CAF]={d='nsm'},
 [0x11CB0]={d='nsm'},
 [0x11CB2]={d='nsm'},
 [0x11CB3]={d='nsm'},
 [0x11CB5]={d='nsm'},
 [0x11CB6]={d='nsm'},
 [0x16AF0]={d='nsm'},
 [0x16AF1]={d='nsm'},
 [0x16AF2]={d='nsm'},
 [0x16AF3]={d='nsm'},
 [0x16AF4]={d='nsm'},
 [0x16B30]={d='nsm'},
 [0x16B31]={d='nsm'},
 [0x16B32]={d='nsm'},
 [0x16B33]={d='nsm'},
 [0x16B34]={d='nsm'},
 [0x16B35]={d='nsm'},
 [0x16B36]={d='nsm'},
 [0x16F8F]={d='nsm'},
 [0x16F90]={d='nsm'},
 [0x16F91]={d='nsm'},
 [0x16F92]={d='nsm'},
 [0x1BC9D]={d='nsm'},
 [0x1BC9E]={d='nsm'},
 [0x1BCA0]={d='bn'},
 [0x1BCA1]={d='bn'},
 [0x1BCA2]={d='bn'},
 [0x1BCA3]={d='bn'},
 [0x1D167]={d='nsm'},
 [0x1D168]={d='nsm'},
 [0x1D169]={d='nsm'},
 [0x1D173]={d='bn'},
 [0x1D174]={d='bn'},
 [0x1D175]={d='bn'},
 [0x1D176]={d='bn'},
 [0x1D177]={d='bn'},
 [0x1D178]={d='bn'},
 [0x1D179]={d='bn'},
 [0x1D17A]={d='bn'},
 [0x1D17B]={d='nsm'},
 [0x1D17C]={d='nsm'},
 [0x1D17D]={d='nsm'},
 [0x1D17E]={d='nsm'},
 [0x1D17F]={d='nsm'},
 [0x1D180]={d='nsm'},
 [0x1D181]={d='nsm'},
 [0x1D182]={d='nsm'},
 [0x1D185]={d='nsm'},
 [0x1D186]={d='nsm'},
 [0x1D187]={d='nsm'},
 [0x1D188]={d='nsm'},
 [0x1D189]={d='nsm'},
 [0x1D18A]={d='nsm'},
 [0x1D18B]={d='nsm'},
 [0x1D1AA]={d='nsm'},
 [0x1D1AB]={d='nsm'},
 [0x1D1AC]={d='nsm'},
 [0x1D1AD]={d='nsm'},
 [0x1D200]={d='on'},
 [0x1D201]={d='on'},
 [0x1D202]={d='on'},
 [0x1D203]={d='on'},
 [0x1D204]={d='on'},
 [0x1D205]={d='on'},
 [0x1D206]={d='on'},
 [0x1D207]={d='on'},
 [0x1D208]={d='on'},
 [0x1D209]={d='on'},
 [0x1D20A]={d='on'},
 [0x1D20B]={d='on'},
 [0x1D20C]={d='on'},
 [0x1D20D]={d='on'},
 [0x1D20E]={d='on'},
 [0x1D20F]={d='on'},
 [0x1D210]={d='on'},
 [0x1D211]={d='on'},
 [0x1D212]={d='on'},
 [0x1D213]={d='on'},
 [0x1D214]={d='on'},
 [0x1D215]={d='on'},
 [0x1D216]={d='on'},
 [0x1D217]={d='on'},
 [0x1D218]={d='on'},
 [0x1D219]={d='on'},
 [0x1D21A]={d='on'},
 [0x1D21B]={d='on'},
 [0x1D21C]={d='on'},
 [0x1D21D]={d='on'},
 [0x1D21E]={d='on'},
 [0x1D21F]={d='on'},
 [0x1D220]={d='on'},
 [0x1D221]={d='on'},
 [0x1D222]={d='on'},
 [0x1D223]={d='on'},
 [0x1D224]={d='on'},
 [0x1D225]={d='on'},
 [0x1D226]={d='on'},
 [0x1D227]={d='on'},
 [0x1D228]={d='on'},
 [0x1D229]={d='on'},
 [0x1D22A]={d='on'},
 [0x1D22B]={d='on'},
 [0x1D22C]={d='on'},
 [0x1D22D]={d='on'},
 [0x1D22E]={d='on'},
 [0x1D22F]={d='on'},
 [0x1D230]={d='on'},
 [0x1D231]={d='on'},
 [0x1D232]={d='on'},
 [0x1D233]={d='on'},
 [0x1D234]={d='on'},
 [0x1D235]={d='on'},
 [0x1D236]={d='on'},
 [0x1D237]={d='on'},
 [0x1D238]={d='on'},
 [0x1D239]={d='on'},
 [0x1D23A]={d='on'},
 [0x1D23B]={d='on'},
 [0x1D23C]={d='on'},
 [0x1D23D]={d='on'},
 [0x1D23E]={d='on'},
 [0x1D23F]={d='on'},
 [0x1D240]={d='on'},
 [0x1D241]={d='on'},
 [0x1D242]={d='nsm'},
 [0x1D243]={d='nsm'},
 [0x1D244]={d='nsm'},
 [0x1D245]={d='on'},
 [0x1D300]={d='on'},
 [0x1D301]={d='on'},
 [0x1D302]={d='on'},
 [0x1D303]={d='on'},
 [0x1D304]={d='on'},
 [0x1D305]={d='on'},
 [0x1D306]={d='on'},
 [0x1D307]={d='on'},
 [0x1D308]={d='on'},
 [0x1D309]={d='on'},
 [0x1D30A]={d='on'},
 [0x1D30B]={d='on'},
 [0x1D30C]={d='on'},
 [0x1D30D]={d='on'},
 [0x1D30E]={d='on'},
 [0x1D30F]={d='on'},
 [0x1D310]={d='on'},
 [0x1D311]={d='on'},
 [0x1D312]={d='on'},
 [0x1D313]={d='on'},
 [0x1D314]={d='on'},
 [0x1D315]={d='on'},
 [0x1D316]={d='on'},
 [0x1D317]={d='on'},
 [0x1D318]={d='on'},
 [0x1D319]={d='on'},
 [0x1D31A]={d='on'},
 [0x1D31B]={d='on'},
 [0x1D31C]={d='on'},
 [0x1D31D]={d='on'},
 [0x1D31E]={d='on'},
 [0x1D31F]={d='on'},
 [0x1D320]={d='on'},
 [0x1D321]={d='on'},
 [0x1D322]={d='on'},
 [0x1D323]={d='on'},
 [0x1D324]={d='on'},
 [0x1D325]={d='on'},
 [0x1D326]={d='on'},
 [0x1D327]={d='on'},
 [0x1D328]={d='on'},
 [0x1D329]={d='on'},
 [0x1D32A]={d='on'},
 [0x1D32B]={d='on'},
 [0x1D32C]={d='on'},
 [0x1D32D]={d='on'},
 [0x1D32E]={d='on'},
 [0x1D32F]={d='on'},
 [0x1D330]={d='on'},
 [0x1D331]={d='on'},
 [0x1D332]={d='on'},
 [0x1D333]={d='on'},
 [0x1D334]={d='on'},
 [0x1D335]={d='on'},
 [0x1D336]={d='on'},
 [0x1D337]={d='on'},
 [0x1D338]={d='on'},
 [0x1D339]={d='on'},
 [0x1D33A]={d='on'},
 [0x1D33B]={d='on'},
 [0x1D33C]={d='on'},
 [0x1D33D]={d='on'},
 [0x1D33E]={d='on'},
 [0x1D33F]={d='on'},
 [0x1D340]={d='on'},
 [0x1D341]={d='on'},
 [0x1D342]={d='on'},
 [0x1D343]={d='on'},
 [0x1D344]={d='on'},
 [0x1D345]={d='on'},
 [0x1D346]={d='on'},
 [0x1D347]={d='on'},
 [0x1D348]={d='on'},
 [0x1D349]={d='on'},
 [0x1D34A]={d='on'},
 [0x1D34B]={d='on'},
 [0x1D34C]={d='on'},
 [0x1D34D]={d='on'},
 [0x1D34E]={d='on'},
 [0x1D34F]={d='on'},
 [0x1D350]={d='on'},
 [0x1D351]={d='on'},
 [0x1D352]={d='on'},
 [0x1D353]={d='on'},
 [0x1D354]={d='on'},
 [0x1D355]={d='on'},
 [0x1D356]={d='on'},
 [0x1D6DB]={d='on'},
 [0x1D715]={d='on'},
 [0x1D74F]={d='on'},
 [0x1D789]={d='on'},
 [0x1D7C3]={d='on'},
 [0x1D7CE]={d='en'},
 [0x1D7CF]={d='en'},
 [0x1D7D0]={d='en'},
 [0x1D7D1]={d='en'},
 [0x1D7D2]={d='en'},
 [0x1D7D3]={d='en'},
 [0x1D7D4]={d='en'},
 [0x1D7D5]={d='en'},
 [0x1D7D6]={d='en'},
 [0x1D7D7]={d='en'},
 [0x1D7D8]={d='en'},
 [0x1D7D9]={d='en'},
 [0x1D7DA]={d='en'},
 [0x1D7DB]={d='en'},
 [0x1D7DC]={d='en'},
 [0x1D7DD]={d='en'},
 [0x1D7DE]={d='en'},
 [0x1D7DF]={d='en'},
 [0x1D7E0]={d='en'},
 [0x1D7E1]={d='en'},
 [0x1D7E2]={d='en'},
 [0x1D7E3]={d='en'},
 [0x1D7E4]={d='en'},
 [0x1D7E5]={d='en'},
 [0x1D7E6]={d='en'},
 [0x1D7E7]={d='en'},
 [0x1D7E8]={d='en'},
 [0x1D7E9]={d='en'},
 [0x1D7EA]={d='en'},
 [0x1D7EB]={d='en'},
 [0x1D7EC]={d='en'},
 [0x1D7ED]={d='en'},
 [0x1D7EE]={d='en'},
 [0x1D7EF]={d='en'},
 [0x1D7F0]={d='en'},
 [0x1D7F1]={d='en'},
 [0x1D7F2]={d='en'},
 [0x1D7F3]={d='en'},
 [0x1D7F4]={d='en'},
 [0x1D7F5]={d='en'},
 [0x1D7F6]={d='en'},
 [0x1D7F7]={d='en'},
 [0x1D7F8]={d='en'},
 [0x1D7F9]={d='en'},
 [0x1D7FA]={d='en'},
 [0x1D7FB]={d='en'},
 [0x1D7FC]={d='en'},
 [0x1D7FD]={d='en'},
 [0x1D7FE]={d='en'},
 [0x1D7FF]={d='en'},
 [0x1DA00]={d='nsm'},
 [0x1DA01]={d='nsm'},
 [0x1DA02]={d='nsm'},
 [0x1DA03]={d='nsm'},
 [0x1DA04]={d='nsm'},
 [0x1DA05]={d='nsm'},
 [0x1DA06]={d='nsm'},
 [0x1DA07]={d='nsm'},
 [0x1DA08]={d='nsm'},
 [0x1DA09]={d='nsm'},
 [0x1DA0A]={d='nsm'},
 [0x1DA0B]={d='nsm'},
 [0x1DA0C]={d='nsm'},
 [0x1DA0D]={d='nsm'},
 [0x1DA0E]={d='nsm'},
 [0x1DA0F]={d='nsm'},
 [0x1DA10]={d='nsm'},
 [0x1DA11]={d='nsm'},
 [0x1DA12]={d='nsm'},
 [0x1DA13]={d='nsm'},
 [0x1DA14]={d='nsm'},
 [0x1DA15]={d='nsm'},
 [0x1DA16]={d='nsm'},
 [0x1DA17]={d='nsm'},
 [0x1DA18]={d='nsm'},
 [0x1DA19]={d='nsm'},
 [0x1DA1A]={d='nsm'},
 [0x1DA1B]={d='nsm'},
 [0x1DA1C]={d='nsm'},
 [0x1DA1D]={d='nsm'},
 [0x1DA1E]={d='nsm'},
 [0x1DA1F]={d='nsm'},
 [0x1DA20]={d='nsm'},
 [0x1DA21]={d='nsm'},
 [0x1DA22]={d='nsm'},
 [0x1DA23]={d='nsm'},
 [0x1DA24]={d='nsm'},
 [0x1DA25]={d='nsm'},
 [0x1DA26]={d='nsm'},
 [0x1DA27]={d='nsm'},
 [0x1DA28]={d='nsm'},
 [0x1DA29]={d='nsm'},
 [0x1DA2A]={d='nsm'},
 [0x1DA2B]={d='nsm'},
 [0x1DA2C]={d='nsm'},
 [0x1DA2D]={d='nsm'},
 [0x1DA2E]={d='nsm'},
 [0x1DA2F]={d='nsm'},
 [0x1DA30]={d='nsm'},
 [0x1DA31]={d='nsm'},
 [0x1DA32]={d='nsm'},
 [0x1DA33]={d='nsm'},
 [0x1DA34]={d='nsm'},
 [0x1DA35]={d='nsm'},
 [0x1DA36]={d='nsm'},
 [0x1DA3B]={d='nsm'},
 [0x1DA3C]={d='nsm'},
 [0x1DA3D]={d='nsm'},
 [0x1DA3E]={d='nsm'},
 [0x1DA3F]={d='nsm'},
 [0x1DA40]={d='nsm'},
 [0x1DA41]={d='nsm'},
 [0x1DA42]={d='nsm'},
 [0x1DA43]={d='nsm'},
 [0x1DA44]={d='nsm'},
 [0x1DA45]={d='nsm'},
 [0x1DA46]={d='nsm'},
 [0x1DA47]={d='nsm'},
 [0x1DA48]={d='nsm'},
 [0x1DA49]={d='nsm'},
 [0x1DA4A]={d='nsm'},
 [0x1DA4B]={d='nsm'},
 [0x1DA4C]={d='nsm'},
 [0x1DA4D]={d='nsm'},
 [0x1DA4E]={d='nsm'},
 [0x1DA4F]={d='nsm'},
 [0x1DA50]={d='nsm'},
 [0x1DA51]={d='nsm'},
 [0x1DA52]={d='nsm'},
 [0x1DA53]={d='nsm'},
 [0x1DA54]={d='nsm'},
 [0x1DA55]={d='nsm'},
 [0x1DA56]={d='nsm'},
 [0x1DA57]={d='nsm'},
 [0x1DA58]={d='nsm'},
 [0x1DA59]={d='nsm'},
 [0x1DA5A]={d='nsm'},
 [0x1DA5B]={d='nsm'},
 [0x1DA5C]={d='nsm'},
 [0x1DA5D]={d='nsm'},
 [0x1DA5E]={d='nsm'},
 [0x1DA5F]={d='nsm'},
 [0x1DA60]={d='nsm'},
 [0x1DA61]={d='nsm'},
 [0x1DA62]={d='nsm'},
 [0x1DA63]={d='nsm'},
 [0x1DA64]={d='nsm'},
 [0x1DA65]={d='nsm'},
 [0x1DA66]={d='nsm'},
 [0x1DA67]={d='nsm'},
 [0x1DA68]={d='nsm'},
 [0x1DA69]={d='nsm'},
 [0x1DA6A]={d='nsm'},
 [0x1DA6B]={d='nsm'},
 [0x1DA6C]={d='nsm'},
 [0x1DA75]={d='nsm'},
 [0x1DA84]={d='nsm'},
 [0x1DA9B]={d='nsm'},
 [0x1DA9C]={d='nsm'},
 [0x1DA9D]={d='nsm'},
 [0x1DA9E]={d='nsm'},
 [0x1DA9F]={d='nsm'},
 [0x1DAA1]={d='nsm'},
 [0x1DAA2]={d='nsm'},
 [0x1DAA3]={d='nsm'},
 [0x1DAA4]={d='nsm'},
 [0x1DAA5]={d='nsm'},
 [0x1DAA6]={d='nsm'},
 [0x1DAA7]={d='nsm'},
 [0x1DAA8]={d='nsm'},
 [0x1DAA9]={d='nsm'},
 [0x1DAAA]={d='nsm'},
 [0x1DAAB]={d='nsm'},
 [0x1DAAC]={d='nsm'},
 [0x1DAAD]={d='nsm'},
 [0x1DAAE]={d='nsm'},
 [0x1DAAF]={d='nsm'},
 [0x1E000]={d='nsm'},
 [0x1E001]={d='nsm'},
 [0x1E002]={d='nsm'},
 [0x1E003]={d='nsm'},
 [0x1E004]={d='nsm'},
 [0x1E005]={d='nsm'},
 [0x1E006]={d='nsm'},
 [0x1E008]={d='nsm'},
 [0x1E009]={d='nsm'},
 [0x1E00A]={d='nsm'},
 [0x1E00B]={d='nsm'},
 [0x1E00C]={d='nsm'},
 [0x1E00D]={d='nsm'},
 [0x1E00E]={d='nsm'},
 [0x1E00F]={d='nsm'},
 [0x1E010]={d='nsm'},
 [0x1E011]={d='nsm'},
 [0x1E012]={d='nsm'},
 [0x1E013]={d='nsm'},
 [0x1E014]={d='nsm'},
 [0x1E015]={d='nsm'},
 [0x1E016]={d='nsm'},
 [0x1E017]={d='nsm'},
 [0x1E018]={d='nsm'},
 [0x1E01B]={d='nsm'},
 [0x1E01C]={d='nsm'},
 [0x1E01D]={d='nsm'},
 [0x1E01E]={d='nsm'},
 [0x1E01F]={d='nsm'},
 [0x1E020]={d='nsm'},
 [0x1E021]={d='nsm'},
 [0x1E023]={d='nsm'},
 [0x1E024]={d='nsm'},
 [0x1E026]={d='nsm'},
 [0x1E027]={d='nsm'},
 [0x1E028]={d='nsm'},
 [0x1E029]={d='nsm'},
 [0x1E02A]={d='nsm'},
 [0x1E8D0]={d='nsm'},
 [0x1E8D1]={d='nsm'},
 [0x1E8D2]={d='nsm'},
 [0x1E8D3]={d='nsm'},
 [0x1E8D4]={d='nsm'},
 [0x1E8D5]={d='nsm'},
 [0x1E8D6]={d='nsm'},
 [0x1E944]={d='nsm'},
 [0x1E945]={d='nsm'},
 [0x1E946]={d='nsm'},
 [0x1E947]={d='nsm'},
 [0x1E948]={d='nsm'},
 [0x1E949]={d='nsm'},
 [0x1E94A]={d='nsm'},
 [0x1EE00]={d='al'},
 [0x1EE01]={d='al'},
 [0x1EE02]={d='al'},
 [0x1EE03]={d='al'},
 [0x1EE05]={d='al'},
 [0x1EE06]={d='al'},
 [0x1EE07]={d='al'},
 [0x1EE08]={d='al'},
 [0x1EE09]={d='al'},
 [0x1EE0A]={d='al'},
 [0x1EE0B]={d='al'},
 [0x1EE0C]={d='al'},
 [0x1EE0D]={d='al'},
 [0x1EE0E]={d='al'},
 [0x1EE0F]={d='al'},
 [0x1EE10]={d='al'},
 [0x1EE11]={d='al'},
 [0x1EE12]={d='al'},
 [0x1EE13]={d='al'},
 [0x1EE14]={d='al'},
 [0x1EE15]={d='al'},
 [0x1EE16]={d='al'},
 [0x1EE17]={d='al'},
 [0x1EE18]={d='al'},
 [0x1EE19]={d='al'},
 [0x1EE1A]={d='al'},
 [0x1EE1B]={d='al'},
 [0x1EE1C]={d='al'},
 [0x1EE1D]={d='al'},
 [0x1EE1E]={d='al'},
 [0x1EE1F]={d='al'},
 [0x1EE21]={d='al'},
 [0x1EE22]={d='al'},
 [0x1EE24]={d='al'},
 [0x1EE27]={d='al'},
 [0x1EE29]={d='al'},
 [0x1EE2A]={d='al'},
 [0x1EE2B]={d='al'},
 [0x1EE2C]={d='al'},
 [0x1EE2D]={d='al'},
 [0x1EE2E]={d='al'},
 [0x1EE2F]={d='al'},
 [0x1EE30]={d='al'},
 [0x1EE31]={d='al'},
 [0x1EE32]={d='al'},
 [0x1EE34]={d='al'},
 [0x1EE35]={d='al'},
 [0x1EE36]={d='al'},
 [0x1EE37]={d='al'},
 [0x1EE39]={d='al'},
 [0x1EE3B]={d='al'},
 [0x1EE42]={d='al'},
 [0x1EE47]={d='al'},
 [0x1EE49]={d='al'},
 [0x1EE4B]={d='al'},
 [0x1EE4D]={d='al'},
 [0x1EE4E]={d='al'},
 [0x1EE4F]={d='al'},
 [0x1EE51]={d='al'},
 [0x1EE52]={d='al'},
 [0x1EE54]={d='al'},
 [0x1EE57]={d='al'},
 [0x1EE59]={d='al'},
 [0x1EE5B]={d='al'},
 [0x1EE5D]={d='al'},
 [0x1EE5F]={d='al'},
 [0x1EE61]={d='al'},
 [0x1EE62]={d='al'},
 [0x1EE64]={d='al'},
 [0x1EE67]={d='al'},
 [0x1EE68]={d='al'},
 [0x1EE69]={d='al'},
 [0x1EE6A]={d='al'},
 [0x1EE6C]={d='al'},
 [0x1EE6D]={d='al'},
 [0x1EE6E]={d='al'},
 [0x1EE6F]={d='al'},
 [0x1EE70]={d='al'},
 [0x1EE71]={d='al'},
 [0x1EE72]={d='al'},
 [0x1EE74]={d='al'},
 [0x1EE75]={d='al'},
 [0x1EE76]={d='al'},
 [0x1EE77]={d='al'},
 [0x1EE79]={d='al'},
 [0x1EE7A]={d='al'},
 [0x1EE7B]={d='al'},
 [0x1EE7C]={d='al'},
 [0x1EE7E]={d='al'},
 [0x1EE80]={d='al'},
 [0x1EE81]={d='al'},
 [0x1EE82]={d='al'},
 [0x1EE83]={d='al'},
 [0x1EE84]={d='al'},
 [0x1EE85]={d='al'},
 [0x1EE86]={d='al'},
 [0x1EE87]={d='al'},
 [0x1EE88]={d='al'},
 [0x1EE89]={d='al'},
 [0x1EE8B]={d='al'},
 [0x1EE8C]={d='al'},
 [0x1EE8D]={d='al'},
 [0x1EE8E]={d='al'},
 [0x1EE8F]={d='al'},
 [0x1EE90]={d='al'},
 [0x1EE91]={d='al'},
 [0x1EE92]={d='al'},
 [0x1EE93]={d='al'},
 [0x1EE94]={d='al'},
 [0x1EE95]={d='al'},
 [0x1EE96]={d='al'},
 [0x1EE97]={d='al'},
 [0x1EE98]={d='al'},
 [0x1EE99]={d='al'},
 [0x1EE9A]={d='al'},
 [0x1EE9B]={d='al'},
 [0x1EEA1]={d='al'},
 [0x1EEA2]={d='al'},
 [0x1EEA3]={d='al'},
 [0x1EEA5]={d='al'},
 [0x1EEA6]={d='al'},
 [0x1EEA7]={d='al'},
 [0x1EEA8]={d='al'},
 [0x1EEA9]={d='al'},
 [0x1EEAB]={d='al'},
 [0x1EEAC]={d='al'},
 [0x1EEAD]={d='al'},
 [0x1EEAE]={d='al'},
 [0x1EEAF]={d='al'},
 [0x1EEB0]={d='al'},
 [0x1EEB1]={d='al'},
 [0x1EEB2]={d='al'},
 [0x1EEB3]={d='al'},
 [0x1EEB4]={d='al'},
 [0x1EEB5]={d='al'},
 [0x1EEB6]={d='al'},
 [0x1EEB7]={d='al'},
 [0x1EEB8]={d='al'},
 [0x1EEB9]={d='al'},
 [0x1EEBA]={d='al'},
 [0x1EEBB]={d='al'},
 [0x1EEF0]={d='on'},
 [0x1EEF1]={d='on'},
 [0x1F000]={d='on'},
 [0x1F001]={d='on'},
 [0x1F002]={d='on'},
 [0x1F003]={d='on'},
 [0x1F004]={d='on'},
 [0x1F005]={d='on'},
 [0x1F006]={d='on'},
 [0x1F007]={d='on'},
 [0x1F008]={d='on'},
 [0x1F009]={d='on'},
 [0x1F00A]={d='on'},
 [0x1F00B]={d='on'},
 [0x1F00C]={d='on'},
 [0x1F00D]={d='on'},
 [0x1F00E]={d='on'},
 [0x1F00F]={d='on'},
 [0x1F010]={d='on'},
 [0x1F011]={d='on'},
 [0x1F012]={d='on'},
 [0x1F013]={d='on'},
 [0x1F014]={d='on'},
 [0x1F015]={d='on'},
 [0x1F016]={d='on'},
 [0x1F017]={d='on'},
 [0x1F018]={d='on'},
 [0x1F019]={d='on'},
 [0x1F01A]={d='on'},
 [0x1F01B]={d='on'},
 [0x1F01C]={d='on'},
 [0x1F01D]={d='on'},
 [0x1F01E]={d='on'},
 [0x1F01F]={d='on'},
 [0x1F020]={d='on'},
 [0x1F021]={d='on'},
 [0x1F022]={d='on'},
 [0x1F023]={d='on'},
 [0x1F024]={d='on'},
 [0x1F025]={d='on'},
 [0x1F026]={d='on'},
 [0x1F027]={d='on'},
 [0x1F028]={d='on'},
 [0x1F029]={d='on'},
 [0x1F02A]={d='on'},
 [0x1F02B]={d='on'},
 [0x1F030]={d='on'},
 [0x1F031]={d='on'},
 [0x1F032]={d='on'},
 [0x1F033]={d='on'},
 [0x1F034]={d='on'},
 [0x1F035]={d='on'},
 [0x1F036]={d='on'},
 [0x1F037]={d='on'},
 [0x1F038]={d='on'},
 [0x1F039]={d='on'},
 [0x1F03A]={d='on'},
 [0x1F03B]={d='on'},
 [0x1F03C]={d='on'},
 [0x1F03D]={d='on'},
 [0x1F03E]={d='on'},
 [0x1F03F]={d='on'},
 [0x1F040]={d='on'},
 [0x1F041]={d='on'},
 [0x1F042]={d='on'},
 [0x1F043]={d='on'},
 [0x1F044]={d='on'},
 [0x1F045]={d='on'},
 [0x1F046]={d='on'},
 [0x1F047]={d='on'},
 [0x1F048]={d='on'},
 [0x1F049]={d='on'},
 [0x1F04A]={d='on'},
 [0x1F04B]={d='on'},
 [0x1F04C]={d='on'},
 [0x1F04D]={d='on'},
 [0x1F04E]={d='on'},
 [0x1F04F]={d='on'},
 [0x1F050]={d='on'},
 [0x1F051]={d='on'},
 [0x1F052]={d='on'},
 [0x1F053]={d='on'},
 [0x1F054]={d='on'},
 [0x1F055]={d='on'},
 [0x1F056]={d='on'},
 [0x1F057]={d='on'},
 [0x1F058]={d='on'},
 [0x1F059]={d='on'},
 [0x1F05A]={d='on'},
 [0x1F05B]={d='on'},
 [0x1F05C]={d='on'},
 [0x1F05D]={d='on'},
 [0x1F05E]={d='on'},
 [0x1F05F]={d='on'},
 [0x1F060]={d='on'},
 [0x1F061]={d='on'},
 [0x1F062]={d='on'},
 [0x1F063]={d='on'},
 [0x1F064]={d='on'},
 [0x1F065]={d='on'},
 [0x1F066]={d='on'},
 [0x1F067]={d='on'},
 [0x1F068]={d='on'},
 [0x1F069]={d='on'},
 [0x1F06A]={d='on'},
 [0x1F06B]={d='on'},
 [0x1F06C]={d='on'},
 [0x1F06D]={d='on'},
 [0x1F06E]={d='on'},
 [0x1F06F]={d='on'},
 [0x1F070]={d='on'},
 [0x1F071]={d='on'},
 [0x1F072]={d='on'},
 [0x1F073]={d='on'},
 [0x1F074]={d='on'},
 [0x1F075]={d='on'},
 [0x1F076]={d='on'},
 [0x1F077]={d='on'},
 [0x1F078]={d='on'},
 [0x1F079]={d='on'},
 [0x1F07A]={d='on'},
 [0x1F07B]={d='on'},
 [0x1F07C]={d='on'},
 [0x1F07D]={d='on'},
 [0x1F07E]={d='on'},
 [0x1F07F]={d='on'},
 [0x1F080]={d='on'},
 [0x1F081]={d='on'},
 [0x1F082]={d='on'},
 [0x1F083]={d='on'},
 [0x1F084]={d='on'},
 [0x1F085]={d='on'},
 [0x1F086]={d='on'},
 [0x1F087]={d='on'},
 [0x1F088]={d='on'},
 [0x1F089]={d='on'},
 [0x1F08A]={d='on'},
 [0x1F08B]={d='on'},
 [0x1F08C]={d='on'},
 [0x1F08D]={d='on'},
 [0x1F08E]={d='on'},
 [0x1F08F]={d='on'},
 [0x1F090]={d='on'},
 [0x1F091]={d='on'},
 [0x1F092]={d='on'},
 [0x1F093]={d='on'},
 [0x1F0A0]={d='on'},
 [0x1F0A1]={d='on'},
 [0x1F0A2]={d='on'},
 [0x1F0A3]={d='on'},
 [0x1F0A4]={d='on'},
 [0x1F0A5]={d='on'},
 [0x1F0A6]={d='on'},
 [0x1F0A7]={d='on'},
 [0x1F0A8]={d='on'},
 [0x1F0A9]={d='on'},
 [0x1F0AA]={d='on'},
 [0x1F0AB]={d='on'},
 [0x1F0AC]={d='on'},
 [0x1F0AD]={d='on'},
 [0x1F0AE]={d='on'},
 [0x1F0B1]={d='on'},
 [0x1F0B2]={d='on'},
 [0x1F0B3]={d='on'},
 [0x1F0B4]={d='on'},
 [0x1F0B5]={d='on'},
 [0x1F0B6]={d='on'},
 [0x1F0B7]={d='on'},
 [0x1F0B8]={d='on'},
 [0x1F0B9]={d='on'},
 [0x1F0BA]={d='on'},
 [0x1F0BB]={d='on'},
 [0x1F0BC]={d='on'},
 [0x1F0BD]={d='on'},
 [0x1F0BE]={d='on'},
 [0x1F0BF]={d='on'},
 [0x1F0C1]={d='on'},
 [0x1F0C2]={d='on'},
 [0x1F0C3]={d='on'},
 [0x1F0C4]={d='on'},
 [0x1F0C5]={d='on'},
 [0x1F0C6]={d='on'},
 [0x1F0C7]={d='on'},
 [0x1F0C8]={d='on'},
 [0x1F0C9]={d='on'},
 [0x1F0CA]={d='on'},
 [0x1F0CB]={d='on'},
 [0x1F0CC]={d='on'},
 [0x1F0CD]={d='on'},
 [0x1F0CE]={d='on'},
 [0x1F0CF]={d='on'},
 [0x1F0D1]={d='on'},
 [0x1F0D2]={d='on'},
 [0x1F0D3]={d='on'},
 [0x1F0D4]={d='on'},
 [0x1F0D5]={d='on'},
 [0x1F0D6]={d='on'},
 [0x1F0D7]={d='on'},
 [0x1F0D8]={d='on'},
 [0x1F0D9]={d='on'},
 [0x1F0DA]={d='on'},
 [0x1F0DB]={d='on'},
 [0x1F0DC]={d='on'},
 [0x1F0DD]={d='on'},
 [0x1F0DE]={d='on'},
 [0x1F0DF]={d='on'},
 [0x1F0E0]={d='on'},
 [0x1F0E1]={d='on'},
 [0x1F0E2]={d='on'},
 [0x1F0E3]={d='on'},
 [0x1F0E4]={d='on'},
 [0x1F0E5]={d='on'},
 [0x1F0E6]={d='on'},
 [0x1F0E7]={d='on'},
 [0x1F0E8]={d='on'},
 [0x1F0E9]={d='on'},
 [0x1F0EA]={d='on'},
 [0x1F0EB]={d='on'},
 [0x1F0EC]={d='on'},
 [0x1F0ED]={d='on'},
 [0x1F0EE]={d='on'},
 [0x1F0EF]={d='on'},
 [0x1F0F0]={d='on'},
 [0x1F0F1]={d='on'},
 [0x1F0F2]={d='on'},
 [0x1F0F3]={d='on'},
 [0x1F0F4]={d='on'},
 [0x1F0F5]={d='on'},
 [0x1F100]={d='en'},
 [0x1F101]={d='en'},
 [0x1F102]={d='en'},
 [0x1F103]={d='en'},
 [0x1F104]={d='en'},
 [0x1F105]={d='en'},
 [0x1F106]={d='en'},
 [0x1F107]={d='en'},
 [0x1F108]={d='en'},
 [0x1F109]={d='en'},
 [0x1F10A]={d='en'},
 [0x1F10B]={d='on'},
 [0x1F10C]={d='on'},
 [0x1F16A]={d='on'},
 [0x1F16B]={d='on'},
 [0xE0001]={d='bn'},
 [0xE0020]={d='bn'},
 [0xE0021]={d='bn'},
 [0xE0022]={d='bn'},
 [0xE0023]={d='bn'},
 [0xE0024]={d='bn'},
 [0xE0025]={d='bn'},
 [0xE0026]={d='bn'},
 [0xE0027]={d='bn'},
 [0xE0028]={d='bn'},
 [0xE0029]={d='bn'},
 [0xE002A]={d='bn'},
 [0xE002B]={d='bn'},
 [0xE002C]={d='bn'},
 [0xE002D]={d='bn'},
 [0xE002E]={d='bn'},
 [0xE002F]={d='bn'},
 [0xE0030]={d='bn'},
 [0xE0031]={d='bn'},
 [0xE0032]={d='bn'},
 [0xE0033]={d='bn'},
 [0xE0034]={d='bn'},
 [0xE0035]={d='bn'},
 [0xE0036]={d='bn'},
 [0xE0037]={d='bn'},
 [0xE0038]={d='bn'},
 [0xE0039]={d='bn'},
 [0xE003A]={d='bn'},
 [0xE003B]={d='bn'},
 [0xE003C]={d='bn'},
 [0xE003D]={d='bn'},
 [0xE003E]={d='bn'},
 [0xE003F]={d='bn'},
 [0xE0040]={d='bn'},
 [0xE0041]={d='bn'},
 [0xE0042]={d='bn'},
 [0xE0043]={d='bn'},
 [0xE0044]={d='bn'},
 [0xE0045]={d='bn'},
 [0xE0046]={d='bn'},
 [0xE0047]={d='bn'},
 [0xE0048]={d='bn'},
 [0xE0049]={d='bn'},
 [0xE004A]={d='bn'},
 [0xE004B]={d='bn'},
 [0xE004C]={d='bn'},
 [0xE004D]={d='bn'},
 [0xE004E]={d='bn'},
 [0xE004F]={d='bn'},
 [0xE0050]={d='bn'},
 [0xE0051]={d='bn'},
 [0xE0052]={d='bn'},
 [0xE0053]={d='bn'},
 [0xE0054]={d='bn'},
 [0xE0055]={d='bn'},
 [0xE0056]={d='bn'},
 [0xE0057]={d='bn'},
 [0xE0058]={d='bn'},
 [0xE0059]={d='bn'},
 [0xE005A]={d='bn'},
 [0xE005B]={d='bn'},
 [0xE005C]={d='bn'},
 [0xE005D]={d='bn'},
 [0xE005E]={d='bn'},
 [0xE005F]={d='bn'},
 [0xE0060]={d='bn'},
 [0xE0061]={d='bn'},
 [0xE0062]={d='bn'},
 [0xE0063]={d='bn'},
 [0xE0064]={d='bn'},
 [0xE0065]={d='bn'},
 [0xE0066]={d='bn'},
 [0xE0067]={d='bn'},
 [0xE0068]={d='bn'},
 [0xE0069]={d='bn'},
 [0xE006A]={d='bn'},
 [0xE006B]={d='bn'},
 [0xE006C]={d='bn'},
 [0xE006D]={d='bn'},
 [0xE006E]={d='bn'},
 [0xE006F]={d='bn'},
 [0xE0070]={d='bn'},
 [0xE0071]={d='bn'},
 [0xE0072]={d='bn'},
 [0xE0073]={d='bn'},
 [0xE0074]={d='bn'},
 [0xE0075]={d='bn'},
 [0xE0076]={d='bn'},
 [0xE0077]={d='bn'},
 [0xE0078]={d='bn'},
 [0xE0079]={d='bn'},
 [0xE007A]={d='bn'},
 [0xE007B]={d='bn'},
 [0xE007C]={d='bn'},
 [0xE007D]={d='bn'},
 [0xE007E]={d='bn'},
 [0xE007F]={d='bn'}
}
%</bidi>
%    \end{macrocode}
%\fi
%
% Now the |basic-r| bidi mode. One of the aims is to implement a fast
% and simple bidi algorithm, with a single loop. I managed to do it
% for R texts, with a second smaller loop for a special case. The code
% is still somewhat chaotic, but its behavior is essentially
% correct. I cannot resist copying the following text from
% \textsf{Emacs} |bidi.c| (which also attempts to implement the bidi
% algorithm with a single loop):
%
% \begin{quote}
%   Arrrgh!! The UAX\#9 algorithm is too deeply entrenched in the
%   assumption of batch-style processing [...]. May the fleas of a
%   thousand camels infest the armpits of those who design supposedly
%   general-purpose algorithms by looking at their own
%   implementations, and fail to consider other possible
%   implementations!
% \end{quote}
% 
% Well, it took me some time to guess what the batch rules in UAX\#9
% actually mean (in other word, \textit{what} they do and\textit{why},
% and not only \textit{how}), but I think (or I hope) I've managed to
% understand them.
%
% In some sense, there are two bidi modes, one for numbers, and the
% other for text.  Furthermore, setting just the direction in R text
% is not enough, because there are actually \textit{two} R modes (set
% explicitly in Unicode with RLM and ALM). In \babel{} the dir is set
% by a higher protocol based on the language/script, which in turn
% sets the correct dir (<l>, <r> or <al>).
%
% From UAX\#9: “Where available, markup should be used instead of the
% explicit formatting characters”. So, this simple version just
% ignores formatting characters. Actually, most of that annex is
% devoted to how to handle them. 
%
% BD14-BD16 are not implemented. Unicode (and the W3C) are making a
% great effort to deal with some special problematic cases in
% “streamed” plain text. I don't think this is the way to go --
% particular issues should be fixed by a high level interface taking
% into account the needs of the document. And here is where \luatex{}
% excels, because everything related to bidi writing is under our
% control.
%
%    \begin{macrocode}
%<*basic-r>
Babel = Babel or {}

require('babel-bidi.lua')

local characters = Babel.characters
local ranges = Babel.ranges

local DIR = node.id("dir")

local function dir_mark(head, from, to, outer)
  dir = (outer == 'r') and 'TLT' or 'TRT' -- ie, reverse
  local d = node.new(DIR)
  d.dir = '+' .. dir
  node.insert_before(head, from, d)
  d = node.new(DIR)
  d.dir = '-' .. dir
  node.insert_after(head, to, d)
end

function Babel.pre_otfload(head)
  local first_n, last_n            -- first and last char with nums
  local last_es                    -- an auxiliary 'last' used with nums
  local first_d, last_d            -- first and last char in L/R block
  local dir, dir_real
%    \end{macrocode}
%
%   Next also depends on script/lang (<al>/<r>). To be set by
%   babel.  |tex.pardir| is dangerous, could be (re)set but it
%   should be changed only in vmode. There are two strong's --
%   |strong| = l/al/r and |strong_lr| = l/r (there must be a better
%   way):
%
%    \begin{macrocode}
  local strong = ('TRT' == tex.pardir) and 'r' or 'l'  
  local strong_lr = (strong == 'l') and 'l' or 'r' 
  local outer = strong

  local new_dir = false
  local first_dir = false

  local last_lr

  local type_n = ''

  for item in node.traverse(head) do

    -- three cases: glyph, dir, otherwise
    if item.id == node.id'glyph' then
       
      local chardata = characters[item.char]
      dir = chardata and chardata.d or nil
      if not dir then
        for nn, et in ipairs(ranges) do
          if item.char < et[1] then
            break
          elseif item.char <= et[2] then
            dir = et[3]
            break
          end
        end
      end
      dir = dir or 'l'
%    \end{macrocode}
%
%      Next is based on the assumption babel sets the language AND
%      switches the script with its dir. We treat a language block as
%      a separate Unicode sequence. The following piece of code is
%      executed at the first glyph after a `dir' node. We don't know
%      the current language until then.
%
%    \begin{macrocode}
      if new_dir then
        attr_dir = 0
        for at in node.traverse(item.attr) do
          if at.number == luatexbase.registernumber'bbl@attr@dir' then
            attr_dir = at.value
          end
        end
        texio.write_nl(attr_dir)
        if attr_dir == 1 then
          strong = 'r'
        elseif attr_dir == 2 then
          strong = 'al'
        else
          strong = 'l'
        end
        strong_lr = (strong == 'l') and 'l' or 'r' 
        outer = strong_lr
        new_dir = false
      end
      if dir == 'nsm' then dir = strong end             -- W1
%    \end{macrocode}
%
% \textbf{Numbers.} The dual <al>/<r> system for R is somewhat
% cumbersome.
%
%    \begin{macrocode}
      dir_real = dir               -- We need dir_real to set strong below
      if dir == 'al' then dir = 'r' end -- W3
%    \end{macrocode}
%
% By W2, there are no <en> <et> <es> if |strong == <al>|, only
% <an>. Therefore, there are not <et en> nor <en et>, W5 can be
% ignored, and W6 applied:
%
%    \begin{macrocode}
      if strong == 'al' then
        if dir == 'en' then dir = 'an' end                -- W2
        if dir == 'et' or dir == 'es' then dir = 'on' end -- W6
        strong_lr = 'r'                                   -- W3
      end
%    \end{macrocode}
%
% Once finished the basic setup for glyphs, consider the two other
% cases: dir node and the rest.
%
%    \begin{macrocode}
    elseif item.id == node.id'dir' then
      new_dir = true 
      dir = nil
    else
      dir = nil          -- Not a char
    end
%    \end{macrocode}
%
% Numbers in R mode. A sequence of <en>, <et>, <an>, <es> and <cs> is
% typeset (with some rules) in L mode. We store the starting and
% ending points, and only when anything different is found (including
% nil, ie, a non-char), the textdir is set. This means you cannot
% insert, say, a whatsit, but this is what I would expect (with
% \textsf{luacolor} you may colorize some digits). Anyway, this
% behaviour could be changed with a switch in the future.  Note in the
% first branch only <an> is relevant if <al>.
%
%    \begin{macrocode}
    if dir == 'en' or dir == 'an' or dir == 'et' then
      if dir ~= 'et' then
        type_n = dir
      end
      first_n = first_n or item
      last_n = last_es or item
      last_es = nil
    elseif dir == 'es' and last_n then -- W3+W6
      last_es = item
    elseif dir == 'cs' then            -- it's right - do nothing
    elseif first_n then -- & if dir = any but en, et, an, es, cs, inc nil
      if strong_lr == 'r' and type_n ~= '' then
        dir_mark(head, first_n, last_n, 'r') 
      elseif strong_lr == 'l' and first_d and type_n == 'an' then
        dir_mark(head, first_n, last_n, 'r') 
        dir_mark(head, first_d, last_d, outer)
        first_d, last_d = nil, nil
      elseif strong_lr == 'l' and type_n ~= '' then
        last_d = last_n
      end
      type_n = ''
      first_n, last_n = nil, nil
    end
%    \end{macrocode}
%
% R text in L, or L text in R. Order of |dir_ mark|'s are relevant: d
% goes outside n, and therefore it's emitted after. See |dir_mark| to
% understand why (but is the nesting actually necessary or is a flat
% dir structure enough?). Only L, R (and AL) chars are taken into
% account -- everything else, including spaces, whatsits, etc., are
% ignored:
%
%    \begin{macrocode}
    if dir == 'l' or dir == 'r' then
      if dir ~= outer then
        first_d = first_d or item
        last_d = item
      elseif first_d and dir ~= strong_lr then
        dir_mark(head, first_d, last_d, outer)
        first_d, last_d = nil, nil
     end
    end  
%    \end{macrocode}
%
% \textbf{Mirroring.}  Each chunk of text in a certain language is
% considered a ``closed'' sequence.  If <r on r> and <l on l>, it's
% clearly <r> and <l>, resptly, but with other combinations depends on
% outer. From all these, we select only those resolving <on> $\to$
% <r>. At the beginning (when |last_lr| is nil) of an R text, they are
% mirrored directly.
%
% TODO - numbers in R mode are processed. It doesn't hurt, but should
% not be done.
%
%    \begin{macrocode}
    if dir and not last_lr and dir ~= 'l' and outer == 'r' then
      item.char = characters[item.char] and
                  characters[item.char].m or item.char
    elseif (dir or new_dir) and last_lr ~= item then
      local mir = outer .. strong_lr .. (dir or outer)
      if mir == 'rrr' or mir == 'lrr' or mir == 'rrl' or mir == 'rlr' then
        for ch in node.traverse(node.next(last_lr)) do
          if ch == item then break end
          if ch.id == node.id'glyph' then
            ch.char = characters[ch.char].m or ch.char
          end
        end
      end
    end
%    \end{macrocode}
%
% Save some values for the next iteration. If the current node is
% `dir', open a new sequence. Since dir could be changed, strong is
% set with its real value (|dir_real|).
%
%    \begin{macrocode}
    if dir == 'l' or dir == 'r' then
      last_lr = item
      strong = dir_real            -- Don't search back - best save now
      strong_lr = (strong == 'l') and 'l' or 'r'
    elseif new_dir then
      last_lr = nil
    end
  end 
%    \end{macrocode}
%
% Mirror the last chars if they are no directed. And make sure any
% open block is closed, too.
%
%    \begin{macrocode}
  if last_lr and outer == 'r' then
    for ch in node.traverse_id(node.id'glyph', node.next(last_lr)) do
      ch.char = characters[ch.char].m or ch.char
    end
  end
  if first_n then 
    dir_mark(head, first_n, last_n, outer)
  end
  if first_d then 
    dir_mark(head, first_d, last_d, outer)
  end
%    \end{macrocode}
%
% In boxes, the dir node could be added before the original head, so
% the actual head is the previous node.
%
%    \begin{macrocode}
  return node.prev(head) or head
end
%</basic-r>
%    \end{macrocode}
%
%    \section{The `nil' language}
%
%    This `language' does nothing, except setting the hyphenation patterns to
%    nohyphenation.
%
%    For this language currently no special definitions are needed or
%    available.
%
%    The macro |\LdfInit| takes care of preventing that this file is
%    loaded more than once, checking the category code of the
%    \texttt{@} sign, etc.
%
%    \begin{macrocode}
%<*nil>
\ProvidesLanguage{nil}[<@date@> <@version@> Nil language]
\LdfInit{nil}{datenil}
%    \end{macrocode}
%
%    When this file is read as an option, i.e. by the |\usepackage|
%    command, \texttt{nil} could be an `unknown' language in which
%    case we have to make it known. 
%
%    \begin{macrocode}
\ifx\l@nohyphenation\@undefined
   \@nopatterns{nil}
   \adddialect\l@nil0
\else
   \let\l@nil\l@nohyphenation
\fi
%    \end{macrocode}
%
%    This macro is used to store the values of the hyphenation
%    parameters |\lefthyphenmin| and |\righthyphenmin|.
%
%    \begin{macrocode}
\providehyphenmins{\CurrentOption}{\m@ne\m@ne}
%    \end{macrocode}
%
%    The next step consists of defining commands to switch to (and
%    from) the `nil' language.
% \begin{macro}{\captionnil}
% \begin{macro}{\datenil}
%
%    \begin{macrocode}
\let\captionsnil\@empty
\let\datenil\@empty
%    \end{macrocode}
%
% \end{macro}
% \end{macro}
%  
%    The macro |\ldf@finish| takes care of looking for a
%    configuration file, setting the main language to be switched on
%    at |\begin{document}| and resetting the category code of
%    \texttt{@} to its original value.
%
%    \begin{macrocode}
\ldf@finish{nil}
%</nil>
%    \end{macrocode}
%
%
%  \section{Support for Plain \TeX\ (\texttt{plain.def})}
%
%  \subsection{Not renaming \file{hyphen.tex}}
%    As Don Knuth has declared that the filename \file{hyphen.tex} may
%    only be used to designate \emph{his} version of the american
%    English hyphenation patterns, a new solution has to be found in
%    order to be able to load hyphenation patterns for other languages
%    in a plain-based \TeX-format. 
%    When asked he responded:
%    \begin{quote}
%      That file name is ``sacred'', and if anybody changes it they will
%      cause severe upward/downward compatibility headaches.
%
%      People can have a file localhyphen.tex or whatever they like,
%      but they mustn't diddle with hyphen.tex (or plain.tex except to
%      preload additional fonts). 
%    \end{quote}
%
%    The files \file{bplain.tex} and \file{blplain.tex} can be used as
%    replacement wrappers around \file{plain.tex} and
%    \file{lplain.tex} to acheive the desired effect, based on the
%    \pkg{babel} package. If you load each of them with ini\TeX, you
%    will get a file called either \file{bplain.fmt} or
%    \file{blplain.fmt}, which you can use as replacements for
%    \file{plain.fmt} and \file{lplain.fmt}.
%
%    As these files are going to be read as the first thing ini\TeX\
%    sees, we need to set some category codes just to be able to
%    change the definition of |\input|
%
%    \begin{macrocode}
%<*bplain|blplain>
\catcode`\{=1 % left brace is begin-group character
\catcode`\}=2 % right brace is end-group character
\catcode`\#=6 % hash mark is macro parameter character
%    \end{macrocode}
%
%    Now let's see if a file called \file{hyphen.cfg} can be found
%    somewhere on \TeX's input path by trying to open it for
%    reading... 
%
%    \begin{macrocode}
\openin 0 hyphen.cfg
%    \end{macrocode}
%
%    If the file wasn't found the following test turns out true.
%
%    \begin{macrocode}
\ifeof0
\else
%    \end{macrocode}
%
%    When \file{hyphen.cfg} could be opened we make sure that
%    \emph{it} will be read instead of the file \file{hyphen.tex}
%    which should (according to Don Knuth's ruling) contain the
%    american English hyphenation patterns and nothing else.
%
%    We do this by first saving the original meaning of |\input| (and
%    I use a one letter control sequence for that so as not to waste
%    multi-letter control sequence on this in the format).
%
%    \begin{macrocode}
  \let\a\input
%    \end{macrocode}
%
%    Then |\input| is defined to forget about its argument and load
%    \file{hyphen.cfg} instead.
%
%    \begin{macrocode}
  \def\input #1 {%
    \let\input\a
    \a hyphen.cfg
%    \end{macrocode}
%
%    Once that's done the original meaning of |\input| can be restored
%    and the definition of |\a| can be forgotten.
%
%    \begin{macrocode}
    \let\a\undefined
  }
\fi
%</bplain|blplain>
%    \end{macrocode}
%
%    Now that we have made sure that \file{hyphen.cfg} will be loaded
%    at the right moment it is time to load \file{plain.tex}.
%
%    \begin{macrocode}
%<bplain>\a plain.tex
%<blplain>\a lplain.tex
%    \end{macrocode}
%
%    Finally we change the contents of |\fmtname| to indicate that
%    this is \emph{not} the plain format, but a format based on plain
%    with the \pkg{babel} package preloaded.
%
%    \begin{macrocode}
%<bplain>\def\fmtname{babel-plain}
%<blplain>\def\fmtname{babel-lplain}
%    \end{macrocode}
%
%    When you are using a different format, based on plain.tex you can
%    make a copy of blplain.tex, rename it and replace \file{plain.tex}
%    with the name of your format file.
%
%  \subsection{Emulating some \LaTeX{} features}
%
%    The following code duplicates or emulates parts of \LaTeXe\ that
%    are needed for \babel.
%
% \changes{bbplain-1.0s}{2012/12/21}{\cs{loadlocalcfg} not loaded in
%    the format} 
%
%    \begin{macrocode}
%<*plain>
\def\@empty{}
\def\loadlocalcfg#1{%
  \openin0#1.cfg
  \ifeof0
    \closein0
  \else
    \closein0
    {\immediate\write16{*************************************}%
     \immediate\write16{* Local config file #1.cfg used}%
     \immediate\write16{*}%
     }
    \input #1.cfg\relax
  \fi
  \@endofldf}
%    \end{macrocode}
%
% \subsection{General tools}
%
%    A number of \LaTeX\ macro's that are needed later on.
%
% \changes{bbplain-1.0t}{2013/04/10}{Added \cs{@expandtwoargs}}
% \changes{babel~3.9h}{2013/12/02}{Added \cs{zap@space}}
% \changes{babel~3.9k}{2014/03/22}{Added \cs{@nnil}}
% \changes{babel~3.9k}{2014/03/22}{Added \cs{@gobbletwo}}
% \changes{babel~3.9k}{2014/03/22}{Added \cs{protected@edef}}
%
%    \begin{macrocode}
\long\def\@firstofone#1{#1}
\long\def\@firstoftwo#1#2{#1}
\long\def\@secondoftwo#1#2{#2}
\def\@nnil{\@nil}
\def\@gobbletwo#1#2{}
\def\@ifstar#1{\@ifnextchar *{\@firstoftwo{#1}}}
\def\@star@or@long#1{%
  \@ifstar
  {\let\l@ngrel@x\relax#1}%
  {\let\l@ngrel@x\long#1}}
\let\l@ngrel@x\relax
\def\@car#1#2\@nil{#1}
\def\@cdr#1#2\@nil{#2}
\let\@typeset@protect\relax
\let\protected@edef\edef
\long\def\@gobble#1{}
\edef\@backslashchar{\expandafter\@gobble\string\\}
\def\strip@prefix#1>{}
\def\g@addto@macro#1#2{{%
    \toks@\expandafter{#1#2}%
    \xdef#1{\the\toks@}}}
\def\@namedef#1{\expandafter\def\csname #1\endcsname}
\def\@nameuse#1{\csname #1\endcsname}
\def\@ifundefined#1{%
  \expandafter\ifx\csname#1\endcsname\relax
    \expandafter\@firstoftwo
  \else
    \expandafter\@secondoftwo
  \fi}
\def\@expandtwoargs#1#2#3{%
  \edef\reserved@a{\noexpand#1{#2}{#3}}\reserved@a}
\def\zap@space#1 #2{%
  #1%
  \ifx#2\@empty\else\expandafter\zap@space\fi
  #2}
%    \end{macrocode}
%
%    \LaTeXe\ has the command |\@onlypreamble| which adds commands to
%    a list of commands that are no longer needed after
%    |\begin{document}|.
%
%    \begin{macrocode}
\ifx\@preamblecmds\@undefined
  \def\@preamblecmds{}
\fi
\def\@onlypreamble#1{%
  \expandafter\gdef\expandafter\@preamblecmds\expandafter{%
    \@preamblecmds\do#1}}
\@onlypreamble\@onlypreamble
%    \end{macrocode}
%
%    Mimick \LaTeX's |\AtBeginDocument|; for this to work the user
%    needs to add |\begindocument| to his file.
%
%    \begin{macrocode}
\def\begindocument{%
  \@begindocumenthook
  \global\let\@begindocumenthook\@undefined
  \def\do##1{\global\let##1\@undefined}%
  \@preamblecmds
  \global\let\do\noexpand}
%    \end{macrocode}
%    
%    \begin{macrocode}
\ifx\@begindocumenthook\@undefined
  \def\@begindocumenthook{}
\fi
\@onlypreamble\@begindocumenthook
\def\AtBeginDocument{\g@addto@macro\@begindocumenthook}
%    \end{macrocode}
%
%    We also have to mimick \LaTeX's |\AtEndOfPackage|. Our
%    replacement macro is much simpler; it stores its argument in
%    |\@endofldf|.
%
%  \changes{babel~3.9h}{2013/11/28}{Set \cs{bbl@opt@hyphenmap} to 0 - we
%     presume hyphenmap=off in plain}
%
%    \begin{macrocode}
\def\AtEndOfPackage#1{\g@addto@macro\@endofldf{#1}}
\@onlypreamble\AtEndOfPackage
\def\@endofldf{}
\@onlypreamble\@endofldf
\let\bbl@afterlang\@empty
\chardef\bbl@opt@hyphenmap\z@
%    \end{macrocode}
%
%    \LaTeX\ needs to be able to switch off writing to its auxiliary
%    files; plain doesn't have them by default.
%
%    \begin{macrocode}
\ifx\if@filesw\@undefined
  \expandafter\let\csname if@filesw\expandafter\endcsname
    \csname iffalse\endcsname
\fi
%    \end{macrocode}
%
%    Mimick \LaTeX's commands to define control sequences.
%
%    \begin{macrocode}
\def\newcommand{\@star@or@long\new@command}
\def\new@command#1{%
  \@testopt{\@newcommand#1}0}
\def\@newcommand#1[#2]{%
  \@ifnextchar [{\@xargdef#1[#2]}%
                {\@argdef#1[#2]}}
\long\def\@argdef#1[#2]#3{%
  \@yargdef#1\@ne{#2}{#3}}
\long\def\@xargdef#1[#2][#3]#4{%
  \expandafter\def\expandafter#1\expandafter{%
    \expandafter\@protected@testopt\expandafter #1%
    \csname\string#1\expandafter\endcsname{#3}}%
  \expandafter\@yargdef \csname\string#1\endcsname
  \tw@{#2}{#4}}
\long\def\@yargdef#1#2#3{%
  \@tempcnta#3\relax
  \advance \@tempcnta \@ne
  \let\@hash@\relax
  \edef\reserved@a{\ifx#2\tw@ [\@hash@1]\fi}%
  \@tempcntb #2%
  \@whilenum\@tempcntb <\@tempcnta
  \do{%
    \edef\reserved@a{\reserved@a\@hash@\the\@tempcntb}%
    \advance\@tempcntb \@ne}%
  \let\@hash@##%
  \l@ngrel@x\expandafter\def\expandafter#1\reserved@a}
\def\providecommand{\@star@or@long\provide@command}
\def\provide@command#1{%
  \begingroup
    \escapechar\m@ne\xdef\@gtempa{{\string#1}}%
  \endgroup
  \expandafter\@ifundefined\@gtempa
    {\def\reserved@a{\new@command#1}}%
    {\let\reserved@a\relax
     \def\reserved@a{\new@command\reserved@a}}%
   \reserved@a}%
%    \end{macrocode}
%    
%    \begin{macrocode}
\def\DeclareRobustCommand{\@star@or@long\declare@robustcommand}
\def\declare@robustcommand#1{%
   \edef\reserved@a{\string#1}%
   \def\reserved@b{#1}%
   \edef\reserved@b{\expandafter\strip@prefix\meaning\reserved@b}%
   \edef#1{%
      \ifx\reserved@a\reserved@b
         \noexpand\x@protect
         \noexpand#1%
      \fi
      \noexpand\protect
      \expandafter\noexpand\csname\bbl@stripslash#1 \endcsname
   }%
   \expandafter\new@command\csname\bbl@stripslash#1 \endcsname
}
\def\x@protect#1{%
   \ifx\protect\@typeset@protect\else
      \@x@protect#1%
   \fi
}
\def\@x@protect#1\fi#2#3{%
   \fi\protect#1%
}
%    \end{macrocode}
%
%    The following little macro |\in@| is taken from \file{latex.ltx};
%    it checks whether its first argument is part of its second
%    argument. It uses the boolean |\in@|; allocating a new boolean
%    inside conditionally executed code is not possible, hence the
%    construct with the temporary definition of |\bbl@tempa|.
%
% \changes{bbplain-1.0s}{2013/01/15}{Use \cs{bbl@tempa} as
%    documented}
%
%    \begin{macrocode}
\def\bbl@tempa{\csname newif\endcsname\ifin@}
\ifx\in@\@undefined
  \def\in@#1#2{%
    \def\in@@##1#1##2##3\in@@{%
      \ifx\in@##2\in@false\else\in@true\fi}%
    \in@@#2#1\in@\in@@}
\else
  \let\bbl@tempa\@empty
\fi
\bbl@tempa
%    \end{macrocode}
%
%    \LaTeX\ has a macro to check whether a certain package was loaded
%    with specific options. The command has two extra arguments which
%    are code to be executed in either the true or false case.
%    This is used to detect whether the document needs one of the
%    accents to be activated (\Lopt{activegrave} and
%    \Lopt{activeacute}). For plain \TeX\ we assume that the user
%    wants them to be active by default. Therefore the only thing we
%    do is execute the third argument (the code for the true case).
% 
%    \begin{macrocode}
\def\@ifpackagewith#1#2#3#4{#3}
%    \end{macrocode}
%
%    The \LaTeX\ macro |\@ifl@aded| checks whether a file was
%    loaded. This functionality is not needed for plain \TeX\ but we
%    need the macro to be defined as a no-op.
%
%    \begin{macrocode}
\def\@ifl@aded#1#2#3#4{}
%    \end{macrocode}
%
%    For the following code we need to make sure that the commands
%    |\newcommand| and |\providecommand| exist with some sensible
%    definition. They are not fully equivalent to their \LaTeXe\
%    versions; just enough to make things work in plain~\TeX
%    environments.
%
%    \begin{macrocode}
\ifx\@tempcnta\@undefined
  \csname newcount\endcsname\@tempcnta\relax
\fi
\ifx\@tempcntb\@undefined
  \csname newcount\endcsname\@tempcntb\relax
\fi
%    \end{macrocode}
%
%    To prevent wasting two counters in \LaTeX$\:$2.09 (because
%    counters with the same name are allocated later by it) we reset
%    the counter that holds the next free counter (|\count10|).
%
%    \begin{macrocode}
\ifx\bye\@undefined
  \advance\count10 by -2\relax
\fi
\ifx\@ifnextchar\@undefined
  \def\@ifnextchar#1#2#3{%
    \let\reserved@d=#1%
    \def\reserved@a{#2}\def\reserved@b{#3}%
    \futurelet\@let@token\@ifnch}
  \def\@ifnch{%
    \ifx\@let@token\@sptoken
      \let\reserved@c\@xifnch
    \else
      \ifx\@let@token\reserved@d
        \let\reserved@c\reserved@a
      \else
        \let\reserved@c\reserved@b
      \fi
    \fi
    \reserved@c}
  \def\:{\let\@sptoken= } \:  % this makes \@sptoken a space token
  \def\:{\@xifnch} \expandafter\def\: {\futurelet\@let@token\@ifnch}
\fi
\def\@testopt#1#2{%
  \@ifnextchar[{#1}{#1[#2]}}
\def\@protected@testopt#1{%
  \ifx\protect\@typeset@protect
    \expandafter\@testopt
  \else
    \@x@protect#1%
  \fi}
\long\def\@whilenum#1\do #2{\ifnum #1\relax #2\relax\@iwhilenum{#1\relax
     #2\relax}\fi}
\long\def\@iwhilenum#1{\ifnum #1\expandafter\@iwhilenum
         \else\expandafter\@gobble\fi{#1}}
%    \end{macrocode}
%
%  \subsection{Encoding related macros}
%
%    Code from \file{ltoutenc.dtx}, adapted for use in the plain \TeX\
%    environment. 
%
%    \begin{macrocode}
\def\DeclareTextCommand{%
   \@dec@text@cmd\providecommand
}
\def\ProvideTextCommand{%
   \@dec@text@cmd\providecommand
}
\def\DeclareTextSymbol#1#2#3{%
   \@dec@text@cmd\chardef#1{#2}#3\relax
}
\def\@dec@text@cmd#1#2#3{%
   \expandafter\def\expandafter#2%
      \expandafter{%
         \csname#3-cmd\expandafter\endcsname
         \expandafter#2%
         \csname#3\string#2\endcsname
      }%
%   \let\@ifdefinable\@rc@ifdefinable
   \expandafter#1\csname#3\string#2\endcsname
}
\def\@current@cmd#1{%
  \ifx\protect\@typeset@protect\else
      \noexpand#1\expandafter\@gobble
  \fi
}
\def\@changed@cmd#1#2{%
   \ifx\protect\@typeset@protect
      \expandafter\ifx\csname\cf@encoding\string#1\endcsname\relax
         \expandafter\ifx\csname ?\string#1\endcsname\relax
            \expandafter\def\csname ?\string#1\endcsname{%
               \@changed@x@err{#1}%
            }%
         \fi
         \global\expandafter\let
           \csname\cf@encoding \string#1\expandafter\endcsname
           \csname ?\string#1\endcsname
      \fi
      \csname\cf@encoding\string#1%
        \expandafter\endcsname
   \else
      \noexpand#1%
   \fi
}
\def\@changed@x@err#1{%
    \errhelp{Your command will be ignored, type <return> to proceed}%
    \errmessage{Command \protect#1 undefined in encoding \cf@encoding}}
\def\DeclareTextCommandDefault#1{%
   \DeclareTextCommand#1?%
}
\def\ProvideTextCommandDefault#1{%
   \ProvideTextCommand#1?%
}
\expandafter\let\csname OT1-cmd\endcsname\@current@cmd
\expandafter\let\csname?-cmd\endcsname\@changed@cmd
\def\DeclareTextAccent#1#2#3{%
  \DeclareTextCommand#1{#2}[1]{\accent#3 ##1}
}
\def\DeclareTextCompositeCommand#1#2#3#4{%
   \expandafter\let\expandafter\reserved@a\csname#2\string#1\endcsname
   \edef\reserved@b{\string##1}%
   \edef\reserved@c{%
     \expandafter\@strip@args\meaning\reserved@a:-\@strip@args}%
   \ifx\reserved@b\reserved@c
      \expandafter\expandafter\expandafter\ifx
         \expandafter\@car\reserved@a\relax\relax\@nil
         \@text@composite
      \else
         \edef\reserved@b##1{%
            \def\expandafter\noexpand
               \csname#2\string#1\endcsname####1{%
               \noexpand\@text@composite
                  \expandafter\noexpand\csname#2\string#1\endcsname
                  ####1\noexpand\@empty\noexpand\@text@composite
                  {##1}%
            }%
         }%
         \expandafter\reserved@b\expandafter{\reserved@a{##1}}%
      \fi
      \expandafter\def\csname\expandafter\string\csname
         #2\endcsname\string#1-\string#3\endcsname{#4}
   \else
     \errhelp{Your command will be ignored, type <return> to proceed}%
     \errmessage{\string\DeclareTextCompositeCommand\space used on
         inappropriate command \protect#1}
   \fi
}
\def\@text@composite#1#2#3\@text@composite{%
   \expandafter\@text@composite@x
      \csname\string#1-\string#2\endcsname
}
\def\@text@composite@x#1#2{%
   \ifx#1\relax
      #2%
   \else
      #1%
   \fi
}
%
\def\@strip@args#1:#2-#3\@strip@args{#2}
\def\DeclareTextComposite#1#2#3#4{%
   \def\reserved@a{\DeclareTextCompositeCommand#1{#2}{#3}}%
   \bgroup
      \lccode`\@=#4%
      \lowercase{%
   \egroup
      \reserved@a @%
   }%
}
%
\def\UseTextSymbol#1#2{%
%   \let\@curr@enc\cf@encoding
%   \@use@text@encoding{#1}%
   #2%
%   \@use@text@encoding\@curr@enc
}
\def\UseTextAccent#1#2#3{%
%   \let\@curr@enc\cf@encoding
%   \@use@text@encoding{#1}%
%   #2{\@use@text@encoding\@curr@enc\selectfont#3}%
%   \@use@text@encoding\@curr@enc
}
\def\@use@text@encoding#1{%
%   \edef\f@encoding{#1}%
%   \xdef\font@name{%
%      \csname\curr@fontshape/\f@size\endcsname
%   }%
%   \pickup@font
%   \font@name
%   \@@enc@update
}
\def\DeclareTextSymbolDefault#1#2{%
   \DeclareTextCommandDefault#1{\UseTextSymbol{#2}#1}%
}
\def\DeclareTextAccentDefault#1#2{%
   \DeclareTextCommandDefault#1{\UseTextAccent{#2}#1}%
}
\def\cf@encoding{OT1}
%    \end{macrocode}
%
%    Currently we only use the \LaTeXe\ method for accents for those
%    that are known to be made active in \emph{some} language
%    definition file.
%
%    \begin{macrocode}
\DeclareTextAccent{\"}{OT1}{127}
\DeclareTextAccent{\'}{OT1}{19}
\DeclareTextAccent{\^}{OT1}{94}
\DeclareTextAccent{\`}{OT1}{18}
\DeclareTextAccent{\~}{OT1}{126}
%    \end{macrocode}
%
%    The following control sequences are used in \file{babel.def}
%    but are not defined for \textsc{plain} \TeX.
%
%    \begin{macrocode}
\DeclareTextSymbol{\textquotedblleft}{OT1}{92}
\DeclareTextSymbol{\textquotedblright}{OT1}{`\"}
\DeclareTextSymbol{\textquoteleft}{OT1}{`\`}
\DeclareTextSymbol{\textquoteright}{OT1}{`\'}
\DeclareTextSymbol{\i}{OT1}{16}
\DeclareTextSymbol{\ss}{OT1}{25}
%    \end{macrocode}
%
%    For a couple of languages we need the \LaTeX-control sequence
%    |\scriptsize| to be available. Because plain \TeX\ doesn't have
%    such a sofisticated font mechanism as \LaTeX\ has, we just |\let|
%    it to |\sevenrm|.
%
%    \begin{macrocode}
\ifx\scriptsize\@undefined
  \let\scriptsize\sevenrm
\fi
%    \end{macrocode}
%
%  \subsection{Babel options}
% 
% \changes{babel~3.9k}{2014/03/22}{Moved code from babel.def, and add
%    some new tools (not yet documented)} 
%
% The file |babel.def| expects some definitions made in the \LaTeX{}
% style file. So we must provide them at least some predefined values as
% well some tools to set them (even if not all options are
% available). There in no package options, and therefore and alternative
% mechanism is provided. For the moment, only |\babeloptionstrings| and
% |\babeloptionmath| are provided, which can be defined before loading
% \babel. |\BabelModifiers| can be set too (but not sure it works).
%
% \begin{macrocode}
\let\bbl@opt@shorthands\@nnil
\def\bbl@ifshorthand#1#2#3{#2}%
\ifx\babeloptionstrings\@undefined
  \let\bbl@opt@strings\@nnil
\else
  \let\bbl@opt@strings\babeloptionstrings
\fi
\def\bbl@tempa{normal}
\ifx\babeloptionmath\bbl@tempa
  \def\bbl@mathnormal{\noexpand\textormath}
\fi
\def\BabelStringsDefault{generic}
\ifx\BabelModifiers\@undefined\let\BabelModifiers\relax\fi
\let\bbl@afterlang\relax
\let\bbl@language@opts\@empty
\ifx\@uclclist\@undefined\let\@uclclist\@empty\fi
\def\AfterBabelLanguage#1#2{}
%</plain>
%    \end{macrocode}
%
% \Finale
%
\endinput
%
% Local Variables: 
% mode: doctex
% coding: utf-8
% TeX-engine: luatex
% End: 


\fi
%    \end{macrocode}
%
%  \begin{macro}{\substitutefontfamily}
%    The command |\substitutefontfamily| creates an \file{.fd} file on
%    the fly. The first argument is an encoding mnemonic, the second
%    and third arguments are font family names.
% \changes{babel~3.7j}{2003/06/15}{create file with lowercase name}
%    \begin{macrocode}
\def\substitutefontfamily#1#2#3{%
  \lowercase{\immediate\openout15=#1#2.fd\relax}%
  \immediate\write15{%
    \string\ProvidesFile{#1#2.fd}%
    [\the\year/\two@digits{\the\month}/\two@digits{\the\day}
     \space generated font description file]^^J
    \string\DeclareFontFamily{#1}{#2}{}^^J
    \string\DeclareFontShape{#1}{#2}{m}{n}{<->ssub * #3/m/n}{}^^J
    \string\DeclareFontShape{#1}{#2}{m}{it}{<->ssub * #3/m/it}{}^^J
    \string\DeclareFontShape{#1}{#2}{m}{sl}{<->ssub * #3/m/sl}{}^^J
    \string\DeclareFontShape{#1}{#2}{m}{sc}{<->ssub * #3/m/sc}{}^^J
    \string\DeclareFontShape{#1}{#2}{b}{n}{<->ssub * #3/bx/n}{}^^J
    \string\DeclareFontShape{#1}{#2}{b}{it}{<->ssub * #3/bx/it}{}^^J
    \string\DeclareFontShape{#1}{#2}{b}{sl}{<->ssub * #3/bx/sl}{}^^J
    \string\DeclareFontShape{#1}{#2}{b}{sc}{<->ssub * #3/bx/sc}{}^^J
    }%
  \closeout15
  }
%    \end{macrocode}
%    This command should only be used in the preamble of a document.
%    \begin{macrocode}
\@onlypreamble\substitutefontfamily
%    \end{macrocode}
%  \end{macro}
%
%    \begin{macrocode}
%</package>
%    \end{macrocode}
%
% \section{The Kernel of Babel}
%
%    The kernel of the \babel\ system is stored in either
%    \file{hyphen.cfg} or \file{switch.def} and \file{babel.def}. The
%    file \file{hyphen.cfg} is a file that can be loaded into the
%    format, which is necessary when you want to be able to switch
%    hyphenation patterns. The file \file{babel.def} contains some
%    \TeX\ code that can be read in at run time. When \file{babel.def}
%    is loaded it checks if \file{hyphen.cfg} is in the format; if
%    not the file \file{switch.def} is loaded.
%
%    Because plain \TeX\ users might want to use some of the features
%    of the \babel{} system too, care has to be taken that plain \TeX\
%    can process the files. For this reason the current format will
%    have to be checked in a number of places. Some of the code below
%    is common to plain \TeX\ and \LaTeX, some of it is for the
%    \LaTeX\ case only.
%
%    When the command |\AtBeginDocument| doesn't exist we assume that
%    we are dealing with a plain-based format. In that case the file
%    \file{plain.def} is needed.
%
%    \begin{macrocode}
%<*kernel|core>
\ifx\AtBeginDocument\@undefined
%    \end{macrocode}
%    But we need to use the second part of \file{plain.def} (when we
%    load it from \file{switch.def}) which we can do by defining
%    |\adddialect|.
% \changes{babel~3.7c}{1999/04/20}{define \cs{adddialect} before
%    loading \file{plain.def} here}
%    \begin{macrocode}
%<kernel&!patterns>  \def\adddialect{}
  \input plain.def\relax
\fi
%</kernel|core>
%    \end{macrocode}
%
%    Check the presence of the command |\iflanguage|, if it is
%    undefined read the file \file{switch.def}.
% \changes{babel~3.0d}{1991/10/29}{Removed use of \cs{@ifundefined}}
%    \begin{macrocode}
%<*core>
\ifx\iflanguage\@undefined
  \input switch.def\relax
\fi
%</core>
%    \end{macrocode}
% \changes{babel~3.6a}{1996/11/02}{Removed \cs{babel@core@loaded}, no
%    longer needed with the advent of \cs{LdfInit}}
%
%  \subsection{Encoding issues (part 1)}
%
%    The first thing we need to do is to determine, at
%    |\begin{document}|, which latin fontencoding to use.
%
%  \begin{macro}{\latinencoding}
% \changes{babel~3.6i}{1997/03/15}{Macro added, moved from
%    \file{.ldf} files}
%    When text is being typeset in an encoding other than `latin'
%    (\texttt{OT1} or \texttt{T1}), it would be nice to still have
%    Roman numerals come out in the Latin encoding.
%    So we first assume that the current encoding at the end
%    of processing the package is the Latin encoding.
%    \begin{macrocode}
%<*core>
\AtEndOfPackage{\edef\latinencoding{\cf@encoding}}
%    \end{macrocode}
%    But this might be overruled with a later loading of the package
%    \pkg{fontenc}. Therefor we check at the execution of
%    |\begin{document}| whether it was loaded with the \Lopt{T1}
%    option. The normal way to do this (using |\@ifpackageloaded|) is
%    disabled for this package. Now we have to revert to parsing the
%    internal macro |\@filelist| which contains all the filenames
%    loaded.
% \changes{babel~3.6k}{1999/03/15}{Use T1 encoding when it is a known
%    encoding}
% \changes{babel~3.6m}{1999/04/06}{Can't use \cs{@ifpackageloaded}
%    need to parse \cs{@filelist}}
% \changes{babel~3.6n}{1999/04/07}{moved checking for fontenc right to
%    the top of \file{babel.sty}}
% \changes{babel~3.6n}{1999/04/07}{Added a check for `manual' selection
%    of \texttt{T1} encoding, without loading \pkg{fontenc}}
% \changes{babel~3.6q}{1999/04/12}{Better solution then parsing
%    \cs{@filelist}, use \cs{@ifl@aded}}
% \changes{babel~3.6u}{1999/04/20}{Moved this code to
%    \file{babel.def}}
%    \begin{macrocode}
\AtBeginDocument{%
  \gdef\latinencoding{OT1}%
  \ifx\cf@encoding\bbl@t@one
    \xdef\latinencoding{\bbl@t@one}%
  \else
    \@ifl@aded{def}{t1enc}{\xdef\latinencoding{\bbl@t@one}}{}%
  \fi
  }
%    \end{macrocode}
%  \end{macro}
%
%  \begin{macro}{\latintext}
% \changes{babel~3.6i}{1997/03/15}{Macro added, moved from
%    \file{.ldf} files}
%    Then we can define the command |\latintext| which is a
%    declarative switch to a latin font-encoding.
%    \begin{macrocode}
\DeclareRobustCommand{\latintext}{%
  \fontencoding{\latinencoding}\selectfont
  \def\encodingdefault{\latinencoding}}
%    \end{macrocode}
%  \end{macro}
%
%  \begin{macro}{\textlatin}
% \changes{babel~3.6i}{1997/03/15}{Macro added, moved from
%    \file{.ldf} files}
% \changes{babel~3.7j}{2003/03/19}{added \cs{leavevmode} to prevent a
%    paragraph starting \emph{inside} the group}
% \changes{babel~3.7k}{2003/10/12}{Use \cs{DeclareTextFontComand}}
%    This command takes an argument which is then typeset using the
%    requested font encoding. In order to avoid many encoding switches
%    it operates in a local scope.
%    \begin{macrocode}
\ifx\@undefined\DeclareTextFontCommand
  \DeclareRobustCommand{\textlatin}[1]{\leavevmode{\latintext #1}}
\else
  \DeclareTextFontCommand{\textlatin}{\latintext}
\fi
%</core>
%    \end{macrocode}
%  \end{macro}
%
%    We also need to redefine a number of commands to ensure that the
%    right font encoding is used, but this can't be done before
%    \file{babel.def} is loaded.
% \changes{babel~3.6o}{1999/04/07}{Moved the rest of the font encoding
%    related definitions to their original place}
%
% \subsection{Multiple languages}
%
%    With \TeX\ version~3.0 it has become possible to load hyphenation
%    patterns for more than one language. This means that some extra
%    administration has to be taken care of.  The user has to know for
%    which languages patterns have been loaded, and what values of
%    |\language| have been used.
%
%    Some discussion has been going on in the \TeX\ world about how to
%    use |\language|. Some have suggested to set a fixed standard,
%    i.\,e., patterns for each language should \emph{always} be loaded
%    in the same location. It has also been suggested to use the
%    \textsc{iso} list for this purpose. Others have pointed out that
%    the \textsc{iso} list contains more than 256~languages, which
%    have \emph{not} been numbered consecutively.
%
%    I think the best way to use |\language|, is to use it
%    dynamically.  This code implements an algorithm to do so. It uses
%    an external file in which the person who maintains a \TeX\
%    environment has to record for which languages he has hyphenation
%    patterns \emph{and} in which files these are stored\footnote{This
%    is because different operating systems sometimes use \emph{very}
%    different file-naming conventions.}. When hyphenation exceptions
%    are stored in a separate file this can be indicated by naming
%    that file \emph{after} the file with the hyphenation patterns.
%
%    This ``configuration file'' can contain empty lines and comments,
%    as well as lines which start with an equals (\texttt{=})
%    sign. Such a line will instruct \LaTeX\ that the hyphenation
%    patterns just processed have to be known under an alternative
%    name. Here is an example:
%  \begin{verbatim}
%    % File    : language.dat
%    % Purpose : tell iniTeX what files with patterns to load.
%    english    english.hyphenations
%    =british
%
%    dutch      hyphen.dutch exceptions.dutch % Nederlands
%    german hyphen.ger
%  \end{verbatim}
%
%    As the file \file{switch.def} needs to be read only once, we
%    check whether it was read before.  If it was, the command
%    |\iflanguage| is already defined, so we can stop processing.
%    \begin{macrocode}
%<*kernel>
%<*!patterns>
\expandafter\ifx\csname iflanguage\endcsname\relax \else
\expandafter\endinput
\fi
%</!patterns>
%    \end{macrocode}
%
%  \begin{macro}{\language}
%    Plain \TeX\ version~3.0 provides the primitive |\language| that
%    is used to store the current language. When used with a pre-3.0
%    version this function has to be implemented by allocating a
%    counter.
%    \begin{macrocode}
\ifx\language\@undefined
  \csname newcount\endcsname\language
\fi
%    \end{macrocode}
%  \end{macro}
%
%  \begin{macro}{\last@language}
%    Another counter is used to store the last language defined.  For
%    pre-3.0 formats an extra counter has to be allocated,
%    \begin{macrocode}
\ifx\newlanguage\@undefined
  \csname newcount\endcsname\last@language
%    \end{macrocode}
%    plain \TeX\ version 3.0 uses |\count 19| for this purpose.
%    \begin{macrocode}
\else
  \countdef\last@language=19
\fi
%    \end{macrocode}
%  \end{macro}
%
%  \begin{macro}{\addlanguage}
%
%    To add languages to \TeX's memory plain \TeX\ version~3.0
%    supplies |\newlanguage|, in a pre-3.0 environment a similar macro
%    has to be provided. For both cases a new macro is defined here,
%    because the original |\newlanguage| was defined to be |\outer|.
%
%    For a format based on plain version~2.x, the definition of
%    |\newlanguage| can not be copied because |\count 19| is used for
%    other purposes in these formats. Therefor |\addlanguage| is
%    defined using a definition based on the macros used to define
%    |\newlanguage| in plain \TeX\ version~3.0.
% \changes{babel~3.2}{1991/11/11}{Added a \texttt{\%}, removed
%    \texttt{by}}
%    \begin{macrocode}
\ifx\newlanguage\@undefined
  \def\addlanguage#1{%
    \global\advance\last@language \@ne
    \ifnum\last@language<\@cclvi
    \else
        \errmessage{No room for a new \string\language!}%
    \fi
    \global\chardef#1\last@language
    \wlog{\string#1 = \string\language\the\last@language}}
%    \end{macrocode}
%
%    For formats based on plain version~3.0 the definition of
%    |\newlanguage| can be simply copied, removing |\outer|.
%
%    \begin{macrocode}
\else
  \def\addlanguage{\alloc@9\language\chardef\@cclvi}
\fi
%    \end{macrocode}
%  \end{macro}
%
%  \begin{macro}{\adddialect}
%    The macro |\adddialect| can be used to add the name of a dialect
%    or variant language, for which an already defined hyphenation
%    table can be used.
% \changes{babel~3.2}{1991/11/11}{Added \cs{relax}}
%    \begin{macrocode}
\def\adddialect#1#2{%
    \global\chardef#1#2\relax
    \wlog{\string#1 = a dialect from \string\language#2}}
%    \end{macrocode}
%  \end{macro}
%
%  \begin{macro}{\iflanguage}
%    Users might want to test (in a private package for instance)
%    which language is currently active. For this we provide a test
%    macro, |\iflanguage|, that has three arguments.  It checks
%    whether the first argument is a known language. If so, it
%    compares the first argument with the value of |\language|. Then,
%    depending on the result of the comparison, it executes either the
%    second or the third argument.
% \changes{babel~3.0a}{1991/05/29}{Added \cs{@bsphack} and
%    \cs{@esphack}}
% \changes{babel~3.0c}{1991/07/21}{Added comment character after
%    \texttt{\#2}}
% \changes{babel~3.0d}{1991/08/08}{Removed superfluous
%    \cs{expandafter}}
% \changes{babel~3.0d}{1991/10/07}{Removed space hacks and use of
%    \cs{@ifundefined}}
% \changes{babel~3.2}{1991/11/11}{Rephrased \cs{ifnum} test}
% \changes{babel~3.7a}{1998/06/10}{Now evaluate the \cs{ifnum} test
%    \emph{after} the \cs{fi} from the \cs{ifx} test and use
%    \cs{@firstoftwo} and \cs{@secondoftwo}}
% \changes{babel~3.7b}{1998/06/29}{Slight enhancement: added braces
%    around first argument of \cs{bbl@afterfi}}
%    \begin{macrocode}
\def\iflanguage#1{%
  \expandafter\ifx\csname l@#1\endcsname\relax
    \@nolanerr{#1}%
  \else
    \bbl@afterfi{\ifnum\csname l@#1\endcsname=\language
      \expandafter\@firstoftwo
    \else
      \expandafter\@secondoftwo
    \fi}%
  \fi}
%    \end{macrocode}
%  \end{macro}
%
%  \begin{macro}{\selectlanguage}
%    The macro |\selectlanguage| checks whether the language is
%    already defined before it performs its actual task, which is to
%    update |\language| and activate language-specific definitions.
%
%    To allow the call of |\selectlanguage| either with a control
%    sequence name or with a simple string as argument, we have to use
%    a trick to delete the optional escape character.
%
%    To convert a control sequence to a string, we use the |\string|
%    primitive.  Next we have to look at the first character of this
%    string and compare it with the escape character.  Because this
%    escape character can be changed by setting the internal integer
%    |\escapechar| to a character number, we have to compare this
%    number with the character of the string.  To do this we have to
%    use \TeX's backquote notation to specify the character as a
%    number.
%
%    If the first character of the |\string|'ed argument is the
%    current escape character, the comparison has stripped this
%    character and the rest in the `then' part consists of the rest of
%    the control sequence name.  Otherwise we know that either the
%    argument is not a control sequence or |\escapechar| is set to a
%    value outside of the character range~$0$--$255$.
%
%    If the user gives an empty argument, we provide a default
%    argument for |\string|.  This argument should expand to nothing.
%
% \changes{babel~3.0c}{1991/06/06}{Made \cs{selectlanguage}
%    robust}
% \changes{babel~3.2}{1991/11/11}{Modified to allow arguments that
%    start with an escape character}
% \changes{babel~3.2a}{1991/11/17}{Simplified the modification to
%    allow the use in a \cs{write} command}
% \changes{babel~3.5b}{1995/05/13}{Store the name of the current
%    language in a control sequence instead of passing the whole macro
%    construct to strip the escape character in the argument of
%    \cs{selectlanguage }.}
% \changes{babel~3.5f}{1995/08/30}{Added a missing percent character}
% \changes{babel~3.5f}{1995/11/16}{Moved check for escape character
%    one level down in the expansion}
%    \begin{macrocode}
\edef\selectlanguage{%
  \noexpand\protect
  \expandafter\noexpand\csname selectlanguage \endcsname
  }
%    \end{macrocode}
%    Because the command |\selectlanguage| could be used in a moving
%    argument it expands to \verb*=\protect\selectlanguage =.
%    Therefor, we have to make sure that a macro |\protect| exists.
%    If it doesn't it is |\let| to |\relax|.
%    \begin{macrocode}
\ifx\@undefined\protect\let\protect\relax\fi
%    \end{macrocode}
%    As \LaTeX$\:$2.09 writes to files \textit{expanded} whereas
%    \LaTeXe\ takes care \textit{not} to expand the arguments of
%    |\write| statements we need to be a bit clever about the way we
%    add information to \file{.aux} files. Therefor we introduce the
%    macro |\xstring| which should expand to the right amount of
%    |\string|'s.
%    \begin{macrocode}
\ifx\documentclass\@undefined
  \def\xstring{\string\string\string}
\else
  \let\xstring\string
\fi
%    \end{macrocode}
%
% \changes{babel~3.5b}{1995/03/04}{Changed the name of the internal
%    macro to \cs{selectlanguage }.}
% \changes{babel~3.5b}{1995/03/05}{Added an extra level of expansion to
%    separate the switching mechanism from writing to aux files}
% \changes{babel~3.7f}{2000/09/25}{Use \cs{aftergroup} to keep the
%    language grouping correct in auxiliary files {PR3091}}
%    Since version 3.5 \babel\ writes entries to the auxiliary files in
%    order to typeset table of contents etc. in the correct language
%    environment.
%  \begin{macro}{\bbl@pop@language}
%    \emph{But} when the language change happens \emph{inside} a group
%    the end of the group doesn't write anything to the auxiliary
%    files. Therefor we need \TeX's |aftergroup| mechanism to help
%    us. The command |\aftergroup| stores the token immediately
%    following it to be executed when the current group is closed. So
%    we define a temporary control sequence |\bbl@pop@language| to be
%    executed at the end of the group. It calls |\bbl@set@language|
%    with the name of the current language as its argument.
%
% \changes{babel~3.7j}{2003/03/18}{Introduce the language stack
%    mechanism}
%  \begin{macro}{\bbl@language@stack}
%    The previous solution works for one level of nesting groups, but
%    as soon as more levels are used it is no longer adequate. For
%    that case we need to keep track of the nested languages using a
%    stack mechanism. This stack is called |\bbl@language@stack| and
%    initially empty.
%    \begin{macrocode}
\xdef\bbl@language@stack{}
%    \end{macrocode}
%    When using a stack we need a mechanism to push an element on the
%    stack and to retrieve the information afterwards.
%  \begin{macro}{\bbl@push@language}
%  \begin{macro}{\bbl@pop@language}
%    The stack is simply a list of languagenames, separated with a `+'
%    sign; the push function can be simple:
%    \begin{macrocode}
\def\bbl@push@language{%
  \xdef\bbl@language@stack{\languagename+\bbl@language@stack}%
  }
%    \end{macrocode}
%    Retrieving information from the stack is a little bit less simple,
%    as we need to remove the element from the stack while storing it
%    in the macro |\languagename|. For this we first define a helper function.
%  \begin{macro}{\bbl@pop@lang}
%    This macro stores its first element (which is delimited by the
%    `+'-sign) in |\languagename| and stores the rest of the string
%    (delimited by `-') in its third argument.
%    \begin{macrocode}
\def\bbl@pop@lang#1+#2-#3{%
  \def\languagename{#1}\xdef#3{#2}%
  }
%    \end{macrocode}
%  \end{macro}
%    The reason for the somewhat weird arrangement of arguments to the
%    helper function is the fact it is called in the following way:
%    \begin{macrocode}
\def\bbl@pop@language{%
  \expandafter\bbl@pop@lang\bbl@language@stack-\bbl@language@stack
%    \end{macrocode}
%    This means that before |\bbl@pop@lang| is executed \TeX\ first
%    \emph{expands} the stack, stored in |\bbl@language@stack|. The
%    result of that is that the argument string of |\bbl@pop@lang|
%    contains one or more language names, each followed by a `+'-sign
%    (zero language names won't occur as this macro will only be
%    called after something has been pushed on the stack) followed by
%    the `-'-sign and finally the reference to the stack.
%    \begin{macrocode}$$
  \expandafter\bbl@set@language\expandafter{\languagename}%
  }
%    \end{macrocode}
%    Once the name of the previous language is retrieved from the stack,
%    it is fed to |\bbl@set@language| to do the actual work of
%    switching everything that needs switching.
%  \end{macro}
%  \end{macro}
%  \end{macro}
%
% \changes{babel~3.7j}{2003/03/18}{Now use the language stack mechanism}
%    \begin{macrocode}
\expandafter\def\csname selectlanguage \endcsname#1{%
  \bbl@push@language
  \aftergroup\bbl@pop@language
  \bbl@set@language{#1}}
%    \end{macrocode}
% \changes{babel~3.7m}{2003/11/12}{Removed the superfluous empty
%    definition of \cs{bbl@pop@language}}
%  \end{macro}
%
%  \begin{macro}{\bbl@set@language}
% \changes{babel~3.5f}{1995/11/16}{Now also define \cs{languagename}
%    at this level}
% \changes{babel~3.7f}{2000/09/25}{Macro \cs{bbl@set@language}
%    introduced}
%    The macro |\bbl@set@language| takes care of switching the language
%    environment \emph{and} of writing entries on the auxiliary files.
%    \begin{macrocode}
\def\bbl@set@language#1{%
  \edef\languagename{%
    \ifnum\escapechar=\expandafter`\string#1\@empty
    \else \string#1\@empty\fi}%
  \select@language{\languagename}%
%    \end{macrocode}
%    We also write a command to change the current language in the
%    auxiliary files.
% \changes{babel~3.5a}{1995/02/17}{write the language change to the
%    auxiliary files}
%    \begin{macrocode}
  \if@filesw
    \protected@write\@auxout{}{\string\select@language{\languagename}}%
    \addtocontents{toc}{\xstring\select@language{\languagename}}%
    \addtocontents{lof}{\xstring\select@language{\languagename}}%
    \addtocontents{lot}{\xstring\select@language{\languagename}}%
  \fi}
%    \end{macrocode}
%  \end{macro}
%
%    First, check if the user asks for a known language. If so,
%    update the value of |\language| and call |\originalTeX|
%    to bring \TeX\ in a certain pre-defined state.
% \changes{babel~3.0a}{1991/05/29}{Added \cs{@bsphack} and
%    \cs{@esphack}}
% \changes{babel~3.0d}{1991/08/08}{Removed superfluous
%    \cs{expandafter}}
% \changes{babel~3.0d}{1991/10/07}{Removed space hacks and use of
%    \cs{@ifundefined}}
% \changes{babel~3.2a}{1991/11/17}{Added \cs{relax} as first command
%    to stop an expansion if \cs{protect} is empty}
% \changes{babel~3.6a}{1996/11/07}{Check for the existence of
%    \cs{date...} instead of \cs{l@...}}
% \changes{babel~3.7m}{2003/11/16}{Check for the existence of both
%    \cs{l@...} and \cs{date...}}
%    \begin{macrocode}
\def\select@language#1{%
  \expandafter\ifx\csname l@#1\endcsname\relax
    \@nolanerr{#1}%
  \else
    \expandafter\ifx\csname date#1\endcsname\relax
      \@noopterr{#1}%
    \else
      \language=\csname l@#1\endcsname\relax
      \originalTeX
%    \end{macrocode}
%    The name of the language is stored in the control sequence
%    |\languagename|. The contents of this control sequence could be
%    tested in the following way:
%  \begin{verbatim}
%    \edef\tmp{\string english}
%    \ifx\languagename\tmp
%        ...
%    \else
%        ...
%    \fi
%  \end{verbatim}
%    The construction with |\string| is necessary because
%    |\languagename| returns the name with characters of category code
%    \texttt{12} (other).  Then we have to \emph{re}define
%    |\originalTeX| to compensate for the things that have been
%    activated.  To save memory space for the macro definition of
%    |\originalTeX|, we construct the control sequence name for the
%    |\noextras|\langvar\ command at definition time by expanding the
%    |\csname| primitive.
% \changes{babel~3.0a}{1991/06/06}{Replaced \cs{gdef} with \cs{def}}
% \changes{babel~3.1}{1991/10/31}{\cs{originalTeX} should only be
%    executed once}
% \changes{babel~3.2a}{1991/11/17}{Added three \cs{expandafter}s
%    to save macro space for \cs{originalTeX}}
% \changes{babel~3.2a}{1991/11/20}{Moved definition of
%    \cs{originalTeX} before \cs{extras\langvar}}
% \changes{babel~3.2a}{1991/11/24}{Set \cs{originalTeX} to
%    \cs{empty}, because it should be expandable.}
%    \begin{macrocode}
      \expandafter\def\expandafter\originalTeX
          \expandafter{\csname noextras#1\endcsname
                       \let\originalTeX\@empty}%
%    \end{macrocode}
% \changes{babel~3.6d}{1997/01/07}{set the language shorthands to
%    `none' before switching on the extras}
%    \begin{macrocode}
      \languageshorthands{none}%
      \babel@beginsave
%    \end{macrocode}
%    Now activate the language-specific definitions. This is done by
%    constructing the names of three macros by concatenating three
%    words with the argument of |\selectlanguage|, and calling these
%    macros.
% \changes{babel~3.5b}{1995/05/13}{Separated the setting of the
%    hyphenmin values}
%    \begin{macrocode}
      \csname captions#1\endcsname
      \csname date#1\endcsname
      \csname extras#1\endcsname\relax
%    \end{macrocode}
%    The switching of the values of |\lefthyphenmin| and
%    |\righthyphenmin| is somewhat different. First we save their
%    current values, then we check if |\|\langvar|hyphenmins| is
%    defined. If it is not, we set default values (2 and 3), otherwise
%    the values in |\|\langvar|hyphenmins| will be used.
% \changes{babel~3.5b}{1995/06/05}{Addedd default setting of hyphenmin
%    parameters}
%    \begin{macrocode}
      \babel@savevariable\lefthyphenmin
      \babel@savevariable\righthyphenmin
      \expandafter\ifx\csname #1hyphenmins\endcsname\relax
        \set@hyphenmins\tw@\thr@@\relax
      \else
        \expandafter\expandafter\expandafter\set@hyphenmins
          \csname #1hyphenmins\endcsname\relax
      \fi
    \fi
  \fi}
%    \end{macrocode}
%  \end{macro}
%
%  \begin{environment}{otherlanguage}
%    The \Lenv{otherlanguage} environment can be used as an
%    alternative to using the |\selectlanguage| declarative
%    command. When you are typesetting a document which mixes
%    left-to-right and right-to-left typesetting you have to use this
%    environment in order to let things work as you expect them to.
%
%    The first thing this environment does is store the name of the
%    language in |\languagename|; it then calls
%    \verb*=\selectlanguage = to switch on everything that is needed for
%    this language The |\ignorespaces| command is necessary to hide
%    the environment when it is entered in horizontal mode.
% \changes{babel~3.5d}{1995/06/22}{environment added}
% \changes{babel~3.5e}{1995/07/07}{changed name}
% \changes{babel~3.7j}{2003/03/18}{rely on \cs{selectlanguage } to
%    keep track of the nesting}
%    \begin{macrocode}
\long\def\otherlanguage#1{%
  \csname selectlanguage \endcsname{#1}%
  \ignorespaces
  }
%    \end{macrocode}
%    The |\endotherlanguage| part of the environment calls
%    |\originalTeX| to restore (most of) the settings and tries to
%    hide itself when it is called in horizontal mode.
%    \begin{macrocode}
\long\def\endotherlanguage{%
  \originalTeX
  \global\@ignoretrue\ignorespaces
  }
%    \end{macrocode}
%  \end{environment}
%
%
%  \begin{environment}{otherlanguage*}
%    The \Lenv{otherlanguage} environment is meant to be used when a
%    large part of text from a different language needs to be typeset,
%    but without changing the translation of words such as `figure'.
%
%    This environment makes use of |\foreign@language|.
% \changes{babel~3.5f}{1996/05/29}{environment added}
% \changes{babel~3.6d}{1997/01/07}{Introduced \cs{foreign@language}}
%    \begin{macrocode}
\expandafter\def\csname otherlanguage*\endcsname#1{%
  \foreign@language{#1}%
  }
%    \end{macrocode}
%    At the end of the environment we need to switch off the extra
%    definitions. The grouping mechanism of the environment will take
%    care of resetting the correct hyphenation rules.
%    \begin{macrocode}
\expandafter\def\csname endotherlanguage*\endcsname{%
  \csname noextras\languagename\endcsname
  }
%    \end{macrocode}
%  \end{environment}
%
%  \begin{macro}{\foreignlanguage}
%    The |\foreignlanguage| command is another substitute for the
%    |\selectlanguage| command. This command takes two arguments, the
%    first argument is the name of the language to use for typesetting
%    the text specified in the second argument.
%
%    Unlike |\selectlanguage| this command doesn't switch
%    \emph{everything}, it only switches the hyphenation rules and the
%    extra definitions for the language specified. It does this within
%    a group and assumes the |\extras|\langvar\ command doesn't make
%    any |\global| changes. The coding is very similar to part of
%    |\selectlanguage|.
% \changes{babel~3.5d}{1995/06/22}{Macro added}
% \changes{babel~3.6d}{1997/01/07}{Introduced \cs{foreign@language}}
% \changes{babel~3.7a}{1998/03/12}{Added executing \cs{originalTeX}}
%    \begin{macrocode}
\def\foreignlanguage{\protect\csname foreignlanguage \endcsname}
\expandafter\def\csname foreignlanguage \endcsname#1#2{%
  \begingroup
    \originalTeX
    \foreign@language{#1}%
    #2%
    \csname noextras#1\endcsname
  \endgroup
  }
%    \end{macrocode}
%  \end{macro}
%
%  \begin{macro}{\foreign@language}
% \changes{babel~3.6d}{1997/01/07}{New macro}
%    This macro does the work for |\foreignlanguage| and the
%    \Lenv{otherlanguage*} environment.
%    \begin{macrocode}
\def\foreign@language#1{%
%    \end{macrocode}
%    First we need to store the name of the language and check that it
%    is a known language.
%    \begin{macrocode}
  \def\languagename{#1}%
  \expandafter\ifx\csname l@#1\endcsname\relax
    \@nolanerr{#1}%
  \else
%    \end{macrocode}
%    If it is we can select the proper hyphenation table and switch on
%    the extra definitions for this language.
% \changes{babel~3.6d}{1997/01/07}{set the language shorthands to
%    `none' before switching on the extras}
%    \begin{macrocode}
    \language=\csname l@#1\endcsname\relax
    \languageshorthands{none}%
%    \end{macrocode}
%    Then we set the left- and right hyphenmin variables.
% \changes{babel~3.6d}{1997/01/07}{Added \cs{relax} to prevent
%    disappearance of the first token after this command.}
%    \begin{macrocode}
    \csname extras#1\endcsname
    \expandafter\ifx\csname #1hyphenmins\endcsname\relax
      \set@hyphenmins\tw@\thr@@\relax
    \else
      \expandafter\expandafter\expandafter\set@hyphenmins
        \csname #1hyphenmins\endcsname\relax
    \fi
  \fi
  }
%    \end{macrocode}
%  \end{macro}
%
%  \begin{environment}{hyphenrules}
% \changes{babel~3.7e}{2000/01/28}{Added environment hyphenrules}
%    The environment \Lenv{hyphenrules} can be used to select
%    \emph{just} the hyphenation rules. This environment does
%    \emph{not} change |\languagename| and when the hyphenation rules
%    specified were not loaded it has no effect.
% \changes{babel~3.8j}{2008/03/16}{Also set the hyphenmin paramters to
%    the correct value (PR3997)} 
%    \begin{macrocode}
\def\hyphenrules#1{%
  \expandafter\ifx\csname l@#1\endcsname\@undefined
    \@nolanerr{#1}%
  \else
    \language=\csname l@#1\endcsname\relax
    \languageshorthands{none}%
       \expandafter\ifx\csname #1hyphenmins\endcsname\relax
         \set@hyphenmins\tw@\thr@@\relax
       \else
         \expandafter\expandafter\expandafter\set@hyphenmins
         \csname #1hyphenmins\endcsname\relax
       \fi
  \fi
  }
\def\endhyphenrules{}
%    \end{macrocode}
%  \end{environment}
%
%  \begin{macro}{\providehyphenmins}
% \changes{babel~3.7f}{2000/02/18}{added macro}
%    The macro |\providehyphenmins| should be used in the language
%    definition files to provide a \emph{default} setting for the
%    hyphenation parameters |\lefthyphenmin| and |\righthyphenmin|. If
%    the macro |\|\langvar|hyphenmins| is already defined this command
%    has no effect.
%    \begin{macrocode}
\def\providehyphenmins#1#2{%
  \expandafter\ifx\csname #1hyphenmins\endcsname\relax
    \@namedef{#1hyphenmins}{#2}%
  \fi}
%    \end{macrocode}
%  \end{macro}
%
%  \begin{macro}{\set@hyphenmins}
%    This macro sets the values of |\lefthyphenmin| and
%    |\righthyphenmin|. It expects two values as its argument.
%    \begin{macrocode}
\def\set@hyphenmins#1#2{\lefthyphenmin#1\righthyphenmin#2}
%    \end{macrocode}
%  \end{macro}
%
%  \begin{macro}{\LdfInit}
% \changes{babel~3.6a}{1996/10/16}{Macro added}
%    This macro is defined in two versions. The first version is to be
%    part of the `kernel' of \babel, ie. the part that is loaded in
%    the format; the second version is defined in \file{babel.def}.
%    The version in the format just checks the category code of the
%    ampersand and then loads \file{babel.def}.
%    \begin{macrocode}
\def\LdfInit{%
  \chardef\atcatcode=\catcode`\@
  \catcode`\@=11\relax
  \input babel.def\relax
%    \end{macrocode}
%    The category code of the ampersand is restored and the macro
%    calls itself again with the new definition from
%    \file{babel.def}
%    \begin{macrocode}
  \catcode`\@=\atcatcode \let\atcatcode\relax
  \LdfInit}
%</kernel>
%    \end{macrocode}
%    The second version of this macro takes two arguments. The first
%    argument is the name of the language that will be defined in the
%    language definition file; the second argument is either a control
%    sequence or a string from which a control sequence should be
%    constructed. The existence of the control sequence indicates that
%    the file has been processed before.
%
%    At the start of processing a language definition file we always
%    check the category code of the ampersand. We make sure that it is
%    a `letter' during the processing of the file.
%    \begin{macrocode}
%<*core>
\def\LdfInit#1#2{%
  \chardef\atcatcode=\catcode`\@
  \catcode`\@=11\relax
%    \end{macrocode}
%    Another character that needs to have the correct category code
%    during processing of language definition files is the equals sign,
%    `=', because it is sometimes used in constructions with the
%    |\let| primitive. Therefor we store its current catcode and
%    restore it later on.
% \changes{babel~3.7o}{2003/11/26}{make sure the equals sign has its
%    default category code}
%    \begin{macrocode}
  \chardef\eqcatcode=\catcode`\=
  \catcode`\==12\relax
%    \end{macrocode}
%    Now we check whether we should perhaps stop the processing of
%    this file. To do this we first need to check whether the second
%    argument that is passed to |\LdfInit| is a control sequence. We
%    do that by looking at the first token after passing |#2| through
%    |string|. When it is equal to |\@backslashchar| we are dealing
%    with a control sequence which we can compare with |\@undefined|.
%    \begin{macrocode}
  \let\bbl@tempa\relax
  \expandafter\if\expandafter\@backslashchar
                 \expandafter\@car\string#2\@nil
    \ifx#2\@undefined
    \else
%    \end{macrocode}
%    If so, we call |\ldf@quit| (but after the end of this |\if|
%    construction) to set the main language, restore the category code
%    of the @-sign and call |\endinput|.
%    \begin{macrocode}
      \def\bbl@tempa{\ldf@quit{#1}}
    \fi
  \else
%    \end{macrocode}
%    When |#2| was \emph{not} a control sequence we construct one and
%    compare it with |\relax|.
%    \begin{macrocode}
    \expandafter\ifx\csname#2\endcsname\relax
    \else
      \def\bbl@tempa{\ldf@quit{#1}}
    \fi
  \fi
  \bbl@tempa
%    \end{macrocode}
%    Finally we check |\originalTeX|.
%    \begin{macrocode}
  \ifx\originalTeX\@undefined
    \let\originalTeX\@empty
  \else
    \originalTeX
  \fi}
%    \end{macrocode}
%  \end{macro}
%
%  \begin{macro}{\ldf@quit}
% \changes{babel~3.6a}{1996/10/29}{Macro added}
%    This macro interrupts the processing of a language definition file.
% \changes{babel~3.7o}{2003/11/26}{Also restore the category code of
%    the equals sign}
%    \begin{macrocode}
\def\ldf@quit#1{%
  \expandafter\main@language\expandafter{#1}%
  \catcode`\@=\atcatcode \let\atcatcode\relax
  \catcode`\==\eqcatcode \let\eqcatcode\relax
  \endinput
}
%    \end{macrocode}
%  \end{macro}
%
%  \begin{macro}{\ldf@finish}
% \changes{babel~3.6a}{1996/10/16}{Macro added}
%    This macro takes one argument. It is the name of the language
%    that was defined in the language definition file.
%
%    We load the local configuration file if one is present, we set
%    the main language (taking into account that the argument might be
%    a control sequence that needs to be expanded) and reset the
%    category code of the @-sign.
% \changes{babel~3.7o}{2003/11/26}{Also restore the category code of
%    the equals sign}
%    \begin{macrocode}
\def\ldf@finish#1{%
  \loadlocalcfg{#1}%
  \expandafter\main@language\expandafter{#1}%
  \catcode`\@=\atcatcode \let\atcatcode\relax
  \catcode`\==\eqcatcode \let\eqcatcode\relax
  }
%    \end{macrocode}
%  \end{macro}
%
%    After the preamble of the document the commands |\LdfInit|,
%    |\ldf@quit| and |\ldf@finish| are no longer needed. Therefor
%    they are turned into warning messages in \LaTeX.
%    \begin{macrocode}
\@onlypreamble\LdfInit
\@onlypreamble\ldf@quit
\@onlypreamble\ldf@finish
%    \end{macrocode}
%
%  \begin{macro}{\main@language}
% \changes{babel~3.5a}{1995/02/17}{Macro added}
% \changes{babel~3.6a}{1996/10/16}{\cs{main@language} now also sets
%    \cs{languagename} and \cs{l@languagename} for use by other
%    packages in the preamble of a document}
%  \begin{macro}{\bbl@main@language}
% \changes{babel~3.5a}{1995/02/17}{Macro added}
%    This command should be used in the various language definition
%    files. It stores its argument in |\bbl@main@language|; to be used
%    to switch to the correct language at the beginning of the
%    document.
%    \begin{macrocode}
\def\main@language#1{%
  \def\bbl@main@language{#1}%
  \let\languagename\bbl@main@language
  \language=\csname l@\languagename\endcsname\relax
  }
%    \end{macrocode}
%    The default is to use English as the main language.
% \changes{babel~3.6c}{1997/01/05}{When \file{hyphen.cfg} is not
%    loaded in the format \cs{l@english} might not be defined; assume
%    english is language 0}
%    \begin{macrocode}
\ifx\l@english\@undefined
  \let\l@english\z@
\fi
\main@language{english}
%    \end{macrocode}
%    We also have to make sure that some code gets executed at the
%    beginning of the document.
%    \begin{macrocode}
\AtBeginDocument{%
  \expandafter\selectlanguage\expandafter{\bbl@main@language}}
%</core>
%    \end{macrocode}
%  \end{macro}
%  \end{macro}
%
%  \begin{macro}{\originalTeX}
%    The macro|\originalTeX| should be known to \TeX\ at this moment.
%    As it has to be expandable we |\let| it to |\@empty| instead of
%    |\relax|.
% \changes{babel~3.2a}{1991/11/24}{Set \cs{originalTeX} to
%    \cs{empty}, because it should be expandable.}
%    \begin{macrocode}
%<*kernel>
\ifx\originalTeX\@undefined\let\originalTeX\@empty\fi
%    \end{macrocode}
%    Because this part of the code can be included in a format, we
%    make sure that the macro which initialises the save mechanism,
%    |\babel@beginsave|, is not considered to be undefined.
%    \begin{macrocode}
\ifx\babel@beginsave\@undefined\let\babel@beginsave\relax\fi
%    \end{macrocode}
%  \end{macro}
%
%  \begin{macro}{\@nolanerr}
% \changes{babel~3.4e}{1994/06/25}{Use \cs{PackageError} in \LaTeXe\
%    mode}
%  \begin{macro}{\@nopatterns}
% \changes{babel~3.4e}{1994/06/25}{Macro added}
%    The \babel\ package will signal an error when a documents tries
%    to select a language that hasn't been defined earlier. When a
%    user selects a language for which no hyphenation patterns were
%    loaded into the format he will be given a warning about that
%    fact. We revert to the patterns for |\language|=0 in that case.
%    In most formats that will be (US)english, but it might also be
%    empty.
%  \begin{macro}{\@noopterr}
% \changes{babel~3.7m}{2003/11/16}{Macro added}
%    When the package was loaded without options not everything will
%    work as expected. An error message is issued in that case.
%
%    When the format knows about |\PackageError| it must be \LaTeXe,
%    so we can safely use its error handling interface. Otherwise
%    we'll have to `keep it simple'.
% \changes{babel~3.0d}{1991/10/07}{Added a percent sign to remove
%    unwanted white space}
% \changes{babel~3.5a}{1995/02/15}{Added \cs{@activated} to log active
%    characters}
% \changes{babel~3.5c}{1995/06/19}{Added missing closing brace}
%    \begin{macrocode}
\ifx\PackageError\@undefined
  \def\@nolanerr#1{%
    \errhelp{Your command will be ignored, type <return> to proceed}%
    \errmessage{You haven't defined the language #1\space yet}}
  \def\@nopatterns#1{%
    \message{No hyphenation patterns were loaded for}%
    \message{the language `#1'}%
    \message{I will use the patterns loaded for \string\language=0
          instead}}
  \def\@noopterr#1{%
    \errmessage{The option #1 was not specified in \string\usepackage}
    \errhelp{You may continue, but expect unexpected results}}
  \def\@activated#1{%
    \wlog{Package babel Info: Making #1 an active character}}
\else
  \newcommand*{\@nolanerr}[1]{%
    \PackageError{babel}%
                 {You haven't defined the language #1\space yet}%
        {Your command will be ignored, type <return> to proceed}}
  \newcommand*{\@nopatterns}[1]{%
    \PackageWarningNoLine{babel}%
        {No hyphenation patterns were loaded for\MessageBreak
          the language `#1'\MessageBreak
          I will use the patterns loaded for \string\language=0
          instead}}
  \newcommand*{\@noopterr}[1]{%
    \PackageError{babel}%
                 {You haven't loaded the option #1\space yet}%
             {You may proceed, but expect unexpected results}}
  \newcommand*{\@activated}[1]{%
    \PackageInfo{babel}{%
      Making #1 an active character}}
\fi
%    \end{macrocode}
%  \end{macro}
%  \end{macro}
%  \end{macro}
%
%    The following code is meant to be read by ini\TeX\ because it
%    should instruct \TeX\ to read hyphenation patterns. To this end
%    the \texttt{docstrip} option \texttt{patterns} can be used to
%    include this code in the file \file{hyphen.cfg}.
%    \begin{macrocode}
%<*patterns>
%    \end{macrocode}
%
% \changes{babel~3.5g}{1996/07/09}{Removed the use of
%    \cs{patterns@loaded} altogether}
%
%  \begin{macro}{\process@line}
% \changes{babel~3.5b}{1995/04/28}{added macro}
%    Each line in the file \file{language.dat} is processed by
%    |\process@line| after it is read. The first thing this macro does
%    is to check whether the line starts with \texttt{=}.
%    When the first token of a line is an \texttt{=}, the macro
%    |\process@synonym| is called; otherwise the macro
%    |\process@language| will continue.
% \changes{babel~3.5g}{1996/07/09}{Simplified code, removing
%    \cs{bbl@eq@}}
% \changes{babel~3.7c}{1999/04/09}{added an extra argument in order to
%    prevent a trailing space from becoming part of the control
%    sequence when defining a synonym (PR 2851)}
%    \begin{macrocode}
\def\process@line#1#2 #3/{%
  \ifx=#1
    \process@synonym#2 /
  \else
    \process@language#1#2 #3/%
  \fi
  }
%    \end{macrocode}
%  \end{macro}
%
%  \begin{macro}{\process@synonym}
% \changes{babel~3.5b}{1995/04/28}{added macro}
%    This macro takes care of the lines which start with an
%    \texttt{=}. It needs an empty token register to begin with.
%    \begin{macrocode}
\toks@{}
\def\process@synonym#1 /{%
  \ifnum\last@language=\m@ne
%    \end{macrocode}
%    When no languages have been loaded yet, the name following the
%    \texttt{=} will be a synonym for hyphenation register 0.
%    \begin{macrocode}
    \expandafter\chardef\csname l@#1\endcsname0\relax
    \wlog{\string\l@#1=\string\language0}
%    \end{macrocode}
%    As no hyphenation patterns are read in yet, we can not yet set
%    the hyphenmin parameters. Therefor a commands to do so is stored
%    in a token register and executed when the first pattern file has
%    been processed.
% \changes{babel~3.7c}{1999/04/27}{Use a token register to temporarily
%    store a command to set hyphenmin parameters for the synonym which
%    is defined \emph{before} the first pattern file is processed}
%    \begin{macrocode}
    \toks@\expandafter{\the\toks@
      \expandafter\let\csname #1hyphenmins\expandafter\endcsname
      \csname\languagename hyphenmins\endcsname}%
  \else
%    \end{macrocode}
%    Otherwise the name will be a synonym for the language loaded last.
%    \begin{macrocode}
    \expandafter\chardef\csname l@#1\endcsname\last@language
    \wlog{\string\l@#1=\string\language\the\last@language}
%    \end{macrocode}
%    We also need to copy the hyphenmin parameters for the synonym.
% \changes{babel~3.7c}{1999/04/22}{Now also store hyphenmin parameters
%    for language synonyms}
%    \begin{macrocode}
    \expandafter\let\csname #1hyphenmins\expandafter\endcsname
    \csname\languagename hyphenmins\endcsname
  \fi
  }
%    \end{macrocode}
%  \end{macro}
%
%  \begin{macro}{\process@language}
%    The macro |\process@language| is used to process a non-empty line
%    from the `configuration file'. It has three arguments, each
%    delimited by white space. The third argument is optional,
%    so a |/| character is expected to delimit the last
%    argument.  The first argument is the `name' of a language; the
%    second is the name of the file that contains the patterns. The
%    optional third argument is the name of a file containing
%    hyphenation exceptions.
%
%    The first thing to do is call |\addlanguage| to allocate a
%    pattern register and to make that register `active'.
% \changes{babel~3.0d}{1991/08/08}{Removed superfluous
%    \cs{expandafter}}
% \changes{babel~3.0d}{1991/08/21}{Reinserted \cs{expandafter}}
% \changes{babel~3.0d}{1991/10/27}{Added the collection of pattern
%    names.}
% \changes{babel~3.7c}{1999/04/22}{Also store \cs{languagename} for
%    possible later use in \cs{process@synonym}}
%    \begin{macrocode}
\def\process@language#1 #2 #3/{%
  \expandafter\addlanguage\csname l@#1\endcsname
  \expandafter\language\csname l@#1\endcsname
  \def\languagename{#1}%
%    \end{macrocode}
%    Then the `name' of the language that will be loaded now is
%    added to the token register |\toks8|. and finally
%    the pattern file is read.
%    \begin{macrocode}
  \global\toks8\expandafter{\the\toks8#1, }%
%    \end{macrocode}
% \changes{babel~3.7f}{2000/02/18}{Allow for the encoding to be used
%    as part of the language name} 
%    For some hyphenation patterns it is needed to load them with a
%    specific font encoding selected. This can be specified in the
%    file \file{language.dat} by adding for instance `\texttt{:T1}' to
%    the name of the language. The macro |\bbl@get@enc| extracts the
%    font encoding from the language name and stores it in
%    |\bbl@hyph@enc|.
%    \begin{macrocode}
  \begingroup
    \bbl@get@enc#1:\@@@
    \ifx\bbl@hyph@enc\@empty
    \else
      \fontencoding{\bbl@hyph@enc}\selectfont
    \fi
%    \end{macrocode}
%
% \changes{babel~3.4e}{1994/06/24}{Added code to detect assignments to
%    left- and righthyphenmin in the patternfile.}
%    Some pattern files contain assignments to |\lefthyphenmin| and
%    |\righthyphenmin|. \TeX\ does not keep track of these
%    assignments. Therefor we try to detect such assignments and
%    store them in the |\|\langvar|hyphenmins| macro. When no
%    assignments were made we provide a default setting.
%    \begin{macrocode}
    \lefthyphenmin\m@ne
%    \end{macrocode}
%    Some pattern files contain changes to the |\lccode| en |\uccode|
%    arrays. Such changes should remain local to the language;
%    therefor we process the pattern file in a group; the |\patterns|
%    command acts globally so its effect will be remembered.
% \changes{babel~3.7a}{1998/03/27}{Read pattern files in a group}
% \changes{babel~3.7c}{1999/04/05}{need to set hyphenmin values
%    globally}
% \changes{babel~3.7c}{1999/04/22}{Set \cs{lefthyphenmin} to \cs{m@ne}
%    \emph{inside} the group; explicitly set the hyphenmin parameters
%    for language 0}
%    \begin{macrocode}
    \input #2\relax
%    \end{macrocode}
%    Now we globally store the settings of |\lefthyphenmin| and
%    |\righthyphenmin| and close the group.
% \changes{babel~3.7c}{1999/04/25}{Only set hyphenmin values when the
%    pattern file changed them}
%    \begin{macrocode}
    \ifnum\lefthyphenmin=\m@ne
    \else
      \expandafter\xdef\csname #1hyphenmins\endcsname{%
        \the\lefthyphenmin\the\righthyphenmin}%
    \fi
  \endgroup
%    \end{macrocode}
%    If the counter |\language| is still equal to zero we set the
%    hyphenmin parameters to the values for the language loaded on
%    pattern register 0.
%    \begin{macrocode}
  \ifnum\the\language=\z@
    \expandafter\ifx\csname #1hyphenmins\endcsname\relax
      \set@hyphenmins\tw@\thr@@\relax
    \else
      \expandafter\expandafter\expandafter\set@hyphenmins
        \csname #1hyphenmins\endcsname
    \fi
%    \end{macrocode}
%    Now execute the contents of token register zero as it may
%    contain commands which set the hyphenmin parameters for synonyms
%    that were defined before the first pattern file is read in.
% \changes{babel~3.7c}{1999/04/27}{Added the execution of the contents
%    of \cs{toks@}}
%    \begin{macrocode}
    \the\toks@
  \fi
%    \end{macrocode}
%    Empty the token register after use.
%    \begin{macrocode}
  \toks@{}%
%    \end{macrocode}
%    When the hyphenation patterns have been processed we need to see
%    if a file with hyphenation exceptions needs to be read. This is
%    the case when the third argument is not empty and when it does
%    not contain a space token.
% \changes{babel~3.5b}{1995/04/28}{Added optional reading of file with
%    hyphenation exceptions}
% \changes{babel~3.5f}{1995/07/25}{Use \cs{empty} instead of
%    \cs{@empty} as the latter is unknown in plain}
%    \begin{macrocode}
  \def\bbl@tempa{#3}%
  \ifx\bbl@tempa\@empty
  \else
    \ifx\bbl@tempa\space
    \else
      \input #3\relax
    \fi
  \fi
  }
%    \end{macrocode}
%
%  \begin{macro}{\bbl@get@enc}
% \changes{babel~3.7f}{2000/02/18}{Added macro}
%  \begin{macro}{\bbl@hyph@enc}
%    The macro |\bbl@get@enc| extracts the font encoding from the
%    language name and stores it in |\bbl@hyph@enc|. It uses delimited
%    arguments to achieve this.
%    \begin{macrocode}
\def\bbl@get@enc#1:#2\@@@{%
%    \end{macrocode}
%    First store both arguments in temporary macros,
%    \begin{macrocode}
  \def\bbl@tempa{#1}%
  \def\bbl@tempb{#2}%
%    \end{macrocode}
%    then, if the second argument was empty, no font encoding was
%    specified and we're done.
%    \begin{macrocode}
  \ifx\bbl@tempb\@empty
    \let\bbl@hyph@enc\@empty
  \else
%    \end{macrocode}
%    But if the second argument was \emph{not} empty it will now have
%    a superfluous colon attached to it which we need to remove. This
%    done by feeding it to |\bbl@get@enc|. The string that we are
%    after will then be in the first argument and be stored in
%    |\bbl@tempa|.
%    \begin{macrocode}
    \bbl@get@enc#2\@@@
    \edef\bbl@hyph@enc{\bbl@tempa}%
  \fi}
%    \end{macrocode}
%  \end{macro}
%  \end{macro}
%  \end{macro}
%
%  \begin{macro}{\readconfigfile}
%    The configuration file can now be opened for reading.
%    \begin{macrocode}
\openin1 = language.dat
%    \end{macrocode}
%
%    See if the file exists, if not, use the default hyphenation file
%    \file{hyphen.tex}. The user will be informed about this.
%
%    \begin{macrocode}
\ifeof1
  \message{I couldn't find the file language.dat,\space
           I will try the file hyphen.tex}
  \input hyphen.tex\relax
\else
%    \end{macrocode}
%
%    Pattern registers are allocated using count register
%    |\last@language|. Its initial value is~0. The definition of the
%    macro |\newlanguage| is such that it first increments the count
%    register and then defines the language. In order to have the
%    first patterns loaded in pattern register number~0 we initialize
%    |\last@language| with the value~$-1$.
%
% \changes{babel~3.1}{1991/05/21}{Removed use of \cs{toks0}}
%    \begin{macrocode}
  \last@language\m@ne
%    \end{macrocode}
%
%    We now read lines from the file until the end is found
%
%    \begin{macrocode}
  \loop
%    \end{macrocode}
%
%    While reading from the input, it is useful to switch off
%    recognition of the end-of-line character. This saves us stripping
%    off spaces from the contents of the control sequence.
%
%    \begin{macrocode}
    \endlinechar\m@ne
    \read1 to \bbl@line
    \endlinechar`\^^M
%    \end{macrocode}
%
%    Empty lines are skipped.
%    \begin{macrocode}
    \ifx\bbl@line\@empty
    \else
%    \end{macrocode}
%
%    Now we add a space and a |/| character to the end of
%    |\bbl@line|. This is needed to be able to recognize the third,
%    optional, argument of |\process@language| later on.
% \changes{babel~3.5b}{1995/04/28}{Now add a \cs{space} and a /
%    character}
%    \begin{macrocode}
      \edef\bbl@line{\bbl@line\space/}%
      \expandafter\process@line\bbl@line
    \fi
%    \end{macrocode}
%
%    Check for the end of the file.  To avoid a new \texttt{if}
%    control sequence we create the necessary |\iftrue| or |\iffalse|
%    with the help of |\csname|.  But there is one complication with
%    this approach: when skipping the \texttt{loop...repeat} \TeX\ has
%    to read |\if|/|\fi| pairs.  So we have to insert a `dummy'
%    |\iftrue|.
% \changes{babel~3.1}{1991/10/31}{Removed the extra \texttt{if}
%    control sequence}
%    \begin{macrocode}
    \iftrue \csname fi\endcsname
    \csname if\ifeof1 false\else true\fi\endcsname
  \repeat
%    \end{macrocode}
%
%    Reactivate the default patterns,
%    \begin{macrocode}
  \language=0
\fi
%    \end{macrocode}
%    and close the configuration file.
% \changes{babel~3.2a}{1991/11/20}{Free macro space for
%    \cs{process@language}}
%    \begin{macrocode}
\closein1
%    \end{macrocode}
%    Also remove some macros from memory
%    \begin{macrocode}
\let\process@language\@undefined
\let\process@synonym\@undefined
\let\process@line\@undefined
\let\bbl@tempa\@undefined
\let\bbl@tempb\@undefined
\let\bbl@eq@\@undefined
\let\bbl@line\@undefined
\let\bbl@get@enc\@undefined
%    \end{macrocode}
%
% \changes{babel~3.5f}{1995/11/08}{Moved the fiddling with \cs{dump}
%     to \file{bbplain.dtx} as it is no longer needed for \LaTeX}
%    We add a message about the fact that babel is loaded in the
%    format and with which language patterns to the \cs{everyjob}
%    register.
% \changes{babel~3.6h}{1997/01/23}{Added a couple of \cs{expandafter}s
%    to copy the contents of \cs{toks8} into \cs{everyjob} instead of
%    the reference}
%    \begin{macrocode}
\ifx\addto@hook\@undefined
\else
  \expandafter\addto@hook\expandafter\everyjob\expandafter{%
    \expandafter\typeout\expandafter{\the\toks8 loaded.}}
\fi
%    \end{macrocode}
%    Here the code for ini\TeX\ ends.
%    \begin{macrocode}
%</patterns>
%</kernel>
%    \end{macrocode}
%  \end{macro}
%
% \subsection{Support for active characters}
%
%  \begin{macro}{\bbl@add@special}
% \changes{babel~3.2}{1991/11/10}{Added macro}
%    The macro |\bbl@add@special| is used to add a new character (or
%    single character control sequence) to the macro |\dospecials|
%    (and |\@sanitize| if \LaTeX\ is used).
%
%    To keep all changes local, we begin a new group.  Then we
%    redefine the macros |\do| and |\@makeother| to add themselves and
%    the given character without expansion.
%    \begin{macrocode}
%<*core|shorthands>
\def\bbl@add@special#1{\begingroup
    \def\do{\noexpand\do\noexpand}%
    \def\@makeother{\noexpand\@makeother\noexpand}%
%    \end{macrocode}
%    To add the character to the macros, we expand the original macros
%    with the additional character inside the redefinition of the
%    macros.  Because |\@sanitize| can be undefined, we put the
%    definition inside a conditional.
%    \begin{macrocode}
    \edef\x{\endgroup
      \def\noexpand\dospecials{\dospecials\do#1}%
      \expandafter\ifx\csname @sanitize\endcsname\relax \else
        \def\noexpand\@sanitize{\@sanitize\@makeother#1}%
      \fi}%
%    \end{macrocode}
%    The macro |\x| contains at this moment the following:\\
%    |\endgroup\def\dospecials{|\textit{old contents}%
%    |\do|\meta{char}|}|.\\
%    If |\@sanitize| is defined, it contains an additional definition
%    of this macro.  The last thing we have to do, is the expansion of
%    |\x|.  Then |\endgroup| is executed, which restores the old
%    meaning of |\x|, |\do| and |\@makeother|.  After the group is
%    closed, the new definition of |\dospecials| (and |\@sanitize|) is
%    assigned.
%    \begin{macrocode}
  \x}
%    \end{macrocode}
%  \end{macro}
%
%  \begin{macro}{\bbl@remove@special}
% \changes{babel~3.2}{1991/11/10}{Added macro}
%    The companion of the former macro is |\bbl@remove@special|.  It
%    is used to remove a character from the set macros |\dospecials|
%    and |\@sanitize|.
%
%    To keep all changes local, we begin a new group.  Then we define
%    a help macro |\x|, which expands to empty if the characters
%    match, otherwise it expands to its nonexpandable input.  Because
%    \TeX\ inserts a |\relax|, if the corresponding |\else| or |\fi|
%    is scanned before the comparison is evaluated, we provide a `stop
%    sign' which should expand to nothing.
%    \begin{macrocode}
\def\bbl@remove@special#1{\begingroup
    \def\x##1##2{\ifnum`#1=`##2\noexpand\@empty
                 \else\noexpand##1\noexpand##2\fi}%
%    \end{macrocode}
%    With the help of this macro we define |\do| and |\make@other|.
%    \begin{macrocode}
    \def\do{\x\do}%
    \def\@makeother{\x\@makeother}%
%    \end{macrocode}
%    The rest of the work is similar to |\bbl@add@special|.
%    \begin{macrocode}
    \edef\x{\endgroup
      \def\noexpand\dospecials{\dospecials}%
      \expandafter\ifx\csname @sanitize\endcsname\relax \else
        \def\noexpand\@sanitize{\@sanitize}%
      \fi}%
  \x}
%    \end{macrocode}
%  \end{macro}
%
%  \subsection{Shorthands}
%
%  \begin{macro}{\initiate@active@char}
% \changes{babel~3.5a}{1995/02/11}{Added macro}
% \changes{babel~3.5b}{1995/03/03}{Renamed macro}
%    A language definition file can call this macro to make a
%    character active. This macro takes one argument, the character
%    that is to be made active. When the character was already active
%    this macro does nothing. Otherwise, this macro defines the
%    control sequence |\normal@char|\m{char} to expand to the
%    character in its `normal state' and it defines the active
%    character to expand to |\normal@char|\m{char} by default
%    (\m{char} being the character to be made active). Later its
%    definition can be changed to expand to |\active@char|\m{char}
%    by calling |\bbl@activate{|\m{char}|}|.
%
%    For example, to make the double quote character active one could
%    have the following line in a language definition file:
%  \begin{verbatim}
%    \initiate@active@char{"}
%  \end{verbatim}
%
%  \begin{macro}{\bbl@afterelse}
%  \begin{macro}{\bbl@afterfi}
%    Because the code that is used in the handling of active
%    characters may need to look ahead, we take extra care to `throw'
%    it over the |\else| and |\fi| parts of an
%    |\if|-statement\footnote{This code is based on code presented in
%    TUGboat vol. 12, no2, June 1991 in ``An expansion Power Lemma''
%    by Sonja Maus.}. These macros will break if another |\if...\fi|
%    statement appears in one of the arguments.
% \changes{babel~3.6i}{1997/02/20}{Made \cs{bbl@afterelse} and
%    \cs{bbl@afterfi} \cs{long}}
%    \begin{macrocode}
\long\def\bbl@afterelse#1\else#2\fi{\fi#1}
\long\def\bbl@afterfi#1\fi{\fi#1}
%    \end{macrocode}
%  \end{macro}
%  \end{macro}
%
% \changes{babel~3.7a}{1997/02/23}{Commented out \c{peek@token} and
%    \cs{test@token} as shorthands are made expandable again}
%
%  \begin{macro}{\peek@token}
% \changes{babel~3.5f}{1995/12/06}{macro added}
% \changes{babel~3.6i}{1998/03/10}{Renamed \cs{test@token} to
%    \cs{bbl@test@token} to prevent a clash with Arab\TeX}
%    To prevent error messages when a shorthand, which
%    normally takes an argument, sees a |\par|, or |}|, or similar
%    tokens, we need to be able to `peek' at what is coming up next in
%    the input stream. Depending on the category code of the token
%    that is seen, we need to either continue the code for the active
%    character, or insert the non-active version of that character in
%    the output. The macro |\peek@token| therefore takes two
%    arguments, with which it constructs the control sequence to
%    expand next. It |\let|'s |\bbl@nexta| and |\bbl@nextb| to the two
%    possible macros. This is necessary for |\bbl@test@token| to take
%    the right decision.
%    \begin{macrocode}
%\def\peek@token#1#2{%
%  \expandafter\let\expandafter\bbl@nexta\csname #1\string#2\endcsname
%  \expandafter\let\expandafter\bbl@nextb
%    \csname system@active\string#2\endcsname
%  \futurelet\bbl@token\bbl@test@token}
%    \end{macrocode}
%
%  \begin{macro}{\bbl@test@token}
% \changes{babel~3.5f}{1995/12/06}{macro added}
% \changes{babel~3.6i}{1998/03/10}{renamed \cs{bbl@token} to
%    \cs{bbl@test@token} to prevent a clash with Arab\TeX}
%    When the result of peeking at the next token has yielded a token
%    with category `letter', `other' or `active' it is safe to proceed
%    with evaluating the code for the shorthand. When a token is found
%    with any other category code proceeding is unsafe and therefor
%    the original shorthand character is inserted in the output. The
%    macro that calls |\bbl@test@token| needs to setup |\bbl@nexta|
%    and |\bbl@nextb| in order to achieve this.
%    \begin{macrocode}
%\def\bbl@test@token{%
%  \let\bbl@next\bbl@nexta
%  \ifcat\noexpand\bbl@token a%
%  \else
%    \ifcat\noexpand\bbl@token=%
%    \else
%      \ifcat\noexpand\bbl@token\noexpand\bbl@next
%      \else
%        \let\bbl@next\bbl@nextb
%      \fi
%    \fi
%  \fi
%  \bbl@next}
%    \end{macrocode}
%  \end{macro}
%  \end{macro}
%
%^^A
%^^A Tekens met mathcode >"8000 zorgen voor problemen.
%^^A hier kan op getest worden door ze catcode13 te geven 
%^^A en te vragen of er een undefined macro ontstaat:
%^^A \ifx#1\undefined{matchcode<"8000}\else get active definition 
%^^A using \let \fi
%^^A
%    The macro |\initiate@active@char| takes all the necessary actions
%    to make its argument a shorthand character. The real work is
%    performed once for each character.
% \changes{babel~3.7c}{1999/04/30}{Only execute
%    \cs{initiate@active@char} once for each character}
%    \begin{macrocode}
\def\initiate@active@char#1{%
  \expandafter\ifx\csname active@char\string##1\endcsname\relax
    \bbl@afterfi{\@initiate@active@char{#1}}%
  \fi}
%    \end{macrocode}
%    Note that the definition of |\@initiate@active@char| needs an
%    active character, for this the |~| is used. Some of the changes
%    we need, do not have to become available later on, so we do it
%    inside a group.
%    \begin{macrocode}
\begingroup
  \catcode`\~\active
  \def\x{\endgroup
    \def\@initiate@active@char##1{%
%    \end{macrocode}
%    If the character is already active we provide the default
%    expansion under this shorthand mechanism.
% \changes{babel~3.5f}{1996/01/09}{Deal correctly with already active
%    characters, provide top level expansion and define all lower
%    level expansion macros outside of the \cs{else} branch.}
% \changes{babel~3.5g}{1996/08/13}{Top level expansion of
%    \cs{normal@char char} where char is already active, should be the
%    expansion of the active character, not the active character
%    itself as this causes an endless loop}
% \changes{babel~3.7d}{1999/08/19}{Make sure the active character
%    doesn't get expanded more then once by the \cs{edef} by adding
%    \cs{expandafter}\cs{strip@prefix}\cs{meaning}}
% \changes{babel~3.7e}{1999/09/06}{previous change was rubbish; use
%    \cs{let} instead of \cs{edef}}
%    \begin{macrocode}
      \ifcat\noexpand##1\noexpand~\relax
        \@ifundefined{normal@char\string##1}{%
          \expandafter\let\csname normal@char\string##1\endcsname##1%
          \expandafter\gdef
            \expandafter##1%
            \expandafter{%
              \expandafter\active@prefix\expandafter##1%
              \csname normal@char\string##1\endcsname}}{}%
      \else
%    \end{macrocode}
%    Otherwise we write a message in the transcript file,
%    \begin{macrocode}
        \@activated{##1}%
%    \end{macrocode}
%    and define |\normal@char|\m{char} to expand to the character in
%    its default state.
%    \begin{macrocode}
        \@namedef{normal@char\string##1}{##1}%
%    \end{macrocode}
%    If we are making the right quote active we need to change
%    |\pr@m@s| as well.
% \changes{babel~3.5a}{1995/03/10}{Added a check for right quote and
%    adapt \cs{pr@m@s} if necessary}
% \changes{babel~3.7f}{1999/12/18}{The redefinition needs to take
%    place one level higher, \cs{prim@s} needs to be redefined.}
%    \begin{macrocode}
        \ifx##1'%
          \let\prim@s\bbl@prim@s
%    \end{macrocode}
%    Also, make sure that a single |'| in math mode `does the right
%    thing'.
% \changes{babel~3.7f}{1999/12/18}{Insert a check for math mode in the
%    definition of \cs{normal@char'}}
% \changes{babel~3.7g}{2000/10/02}{use \cs{textormath} to get rid of
%    the \cs{fi} (PR 3266)}
%    \begin{macrocode}
          \@namedef{normal@char\string##1}{%
            \textormath{##1}{^\bgroup\prim@s}}%
        \fi
%    \end{macrocode}
%    If we are using the caret as a shorthand character special care
%    should be taken to make sure math still works. Therefor an extra
%    level of expansion is introduced with a check for math mode on
%    the upper level.
% \changes{babel~3.7f}{1999/12/18}{Introduced an extra level of
%    expansion in the definition of an active caret}
%    \begin{macrocode}
        \ifx##1^%
          \gdef\bbl@act@caret{%
            \ifmmode
              \csname normal@char\string^\endcsname
            \else
              \bbl@afterfi
              {\if@safe@actives
                \bbl@afterelse\csname normal@char\string##1\endcsname
               \else
                \bbl@afterfi\csname user@active\string##1\endcsname
               \fi}%
            \fi}
        \fi
%    \end{macrocode}
%    To prevent problems with the loading of other packages after
%    \babel\ we reset the catcode of the character at the end of the
%    package.
% \changes{babel~3.5f}{1995/12/01}{Restore the category code of a
%    shorthand char at end of package}
% \changes{babel~3.6f}{1997/01/14}{Made restoring of the category code
%    of shorthand characters optional}
% \changes{babel~3.7a}{1997/03/21}{Use \cs{@ifpackagewith} to
%    determine whether shorthand characters need to remain active}
%    \begin{macrocode}
        \@ifpackagewith{babel}{KeepShorthandsActive}{}{%
          \edef\bbl@tempa{\catcode`\noexpand##1\the\catcode`##1}%
          \expandafter\AtEndOfPackage\expandafter{\bbl@tempa}}%
%    \end{macrocode}
%    Now we set the lowercase code of the |~| equal to that of the
%    character to be made active and execute the rest of the code
%    inside a |\lowercase| `environment'.
% \changes{babel~3.5f}{1996/01/25}{store the \cs{lccode} of the tie
%    before changing it}
%    \begin{macrocode}
        \@tempcnta=\lccode`\~
        \lccode`~=`##1%
        \lowercase{%
%    \end{macrocode}
%    Make the character active and add it to |\dospecials| and
%    |\@sanitize|.
%    \begin{macrocode}
          \catcode`~\active
          \expandafter\bbl@add@special
            \csname \string##1\endcsname
%    \end{macrocode}
%    Also re-activate it again at |\begin{document}|.
%    \begin{macrocode}
            \AtBeginDocument{%
              \catcode`##1\active
%    \end{macrocode}
%    We also need to make sure that the shorthands are active during
%    the processing of the \file{.aux} file. Otherwise some citations
%    may give unexpected results in the printout when a shorthand was
%    used in the optional argument of |\bibitem| for example.
% \changes{babel~3.6i}{1997/03/01}{Make shorthands active during
%    \file{.aux} file processing}
%    \begin{macrocode}
              \if@filesw
                \immediate\write\@mainaux{%
                  \string\catcode`##1\string\active}%
              \fi}%
%    \end{macrocode}
%    Define the character to expand to
%    \begin{center}
%    |\active@prefix| \m{char} |\normal@char|\m{char}
%    \end{center}
%    (where |\active@char|\m{char} is \emph{one} control sequence!).
% \changes{babel~3.5f}{1996/01/25}{restore the \cs{lccode} of the tie}
%    \begin{macrocode}
          \expandafter\gdef
            \expandafter~%
            \expandafter{%
            \expandafter\active@prefix\expandafter##1%
            \csname normal@char\string##1\endcsname}}%
        \lccode`\~\@tempcnta
      \fi
%    \end{macrocode}
%    For the active caret we first expand to |\bbl@act@caret| in order
%    to be able to handle math mode correctly.
% \changes{babel~3.7f}{2000/09/25}{Make an exception for the active
%    caret which needs an extra level of expansion}
%    \begin{macrocode}
      \ifx##1^%
        \@namedef{active@char\string##1}{\bbl@act@caret}%
      \else
%    \end{macrocode}
%    We define the first level expansion of |\active@char|\m{char} to
%    check the status of the |@safe@actives| flag. If it is set to
%    true we expand to the `normal' version of this character,
%    otherwise we call |\@active@char|\m{char}.
%    \begin{macrocode}
        \@namedef{active@char\string##1}{%
          \if@safe@actives
            \bbl@afterelse\csname normal@char\string##1\endcsname
          \else
            \bbl@afterfi\csname user@active\string##1\endcsname
          \fi}%
      \fi
%    \end{macrocode}
%    The next level of the code checks whether a user has defined a
%    shorthand for himself with this character. First we check for a
%    single character shorthand. If that doesn't exist we check for a
%    shorthand with an argument.
% \changes{babel~3.5d}{1995/07/02}{Skip the user-level active char
%    with argument if no shorthands with arguments were defined}
% \changes{babel~3.8b}{2004/04/19}{Now use \cs{bbl@sh@select}}
%    \begin{macrocode}
      \@namedef{user@active\string##1}{%
        \expandafter\ifx
        \csname \user@group @sh@\string##1@\endcsname
        \relax
          \bbl@afterelse\bbl@sh@select\user@group##1%
        {user@active@arg\string##1}{language@active\string##1}%
        \else
          \bbl@afterfi\csname \user@group @sh@\string##1@\endcsname
        \fi}%
%    \end{macrocode}
%    When there is also no user-level shorthand with an argument we
%    will check whether there is a language defined shorthand for
%    this active character. Before the next token is absorbed as
%    argument we need to make sure that this is safe. Therefor
%    |\peek@token| is called to decide that.
% \changes{babel~3.5f}{1995/12/07}{use \cs{peek@token} to check whether
%    it is safe to proceed}
% \changes{babel~3.6i}{1997/02/20}{Remove the use of \cs{peek@token}
%    again and make the \cs{...active@arg...} commands \cs{long}}
% \changes{babel~3.7e}{1999/09/24}{pass the argument on with braces in
%    order to prevent it from breaking up}
% \changes{babel~3.7f}{2000/02/18}{remove the braces again}
%    \begin{macrocode}
      \long\@namedef{user@active@arg\string##1}####1{%
        \expandafter\ifx
        \csname \user@group @sh@\string##1@\string####1@\endcsname
        \relax
          \bbl@afterelse
          \csname language@active\string##1\endcsname####1%
        \else
          \bbl@afterfi
          \csname \user@group @sh@\string##1@\string####1@%
          \endcsname
        \fi}%
%    \end{macrocode}
%    In order to do the right thing when a shorthand with an argument
%    is used by itself at the end of the line we provide a definition
%    for the case of an empty argument. For that case we let the
%    shorthand character expand to its non-active self.
%    \begin{macrocode}
      \@namedef{\user@group @sh@\string##1@@}{%
        \csname normal@char\string##1\endcsname}
%    \end{macrocode}
%
%    Like the shorthands that can be defined by the user, a language
%    definition file can also define shorthands with and without an
%    argument, so we need two more macros to check if they exist.
% \changes{babel~3.5d}{1995/07/02}{Skip the language-level active char
%    with argument if no shorthands with arguments were defined}
% \changes{babel~3.8b}{2004/04/19}{Now use \cs{bbl@sh@select}}
%    \begin{macrocode}
      \@namedef{language@active\string##1}{%
        \expandafter\ifx
        \csname \language@group @sh@\string##1@\endcsname
        \relax
          \bbl@afterelse\bbl@sh@select\language@group##1%
          {language@active@arg\string##1}{system@active\string##1}%
        \else
          \bbl@afterfi
          \csname \language@group @sh@\string##1@\endcsname
        \fi}%
%    \end{macrocode}
% \changes{babel~3.5f}{1995/12/07}{use \cs{peek@token} to check whether
%    it is safe to proceed}
% \changes{babel~3.6i}{1997/02/20}{Remove the use of \cs{peek@token}
%    again}
% \changes{babel~3.7e}{1999/09/24}{pass the argument on with braces in
%    order to prevent it from breaking up}
% \changes{babel~3.7f}{2000/02/18}{remove the braces again}
%    \begin{macrocode}
      \long\@namedef{language@active@arg\string##1}####1{%
        \expandafter\ifx
        \csname \language@group @sh@\string##1@\string####1@\endcsname
        \relax
          \bbl@afterelse
          \csname system@active\string##1\endcsname####1%
        \else
          \bbl@afterfi
          \csname \language@group @sh@\string##1@\string####1@%
          \endcsname
        \fi}%
%    \end{macrocode}
%    And the same goes for the system level.
% \changes{babel~3.8b}{2004/04/19}{Now use \cs{bbl@sh@select}}
%    \begin{macrocode}
      \@namedef{system@active\string##1}{%
        \expandafter\ifx
        \csname \system@group @sh@\string##1@\endcsname
        \relax
          \bbl@afterelse\bbl@sh@select\system@group##1%
          {system@active@arg\string##1}{normal@char\string##1}%
        \else
          \bbl@afterfi\csname \system@group @sh@\string##1@\endcsname
        \fi}%
%    \end{macrocode}
%    When no shorthands were found the `normal' version of the active
%    character is inserted.
% \changes{babel~3.5f}{1995/12/07}{use \cs{peek@token} to check whether
%    it is safe to proceed}
% \changes{babel~3.6i}{1997/02/20}{Remove the use of \cs{peek@token}
%    again}
%    \begin{macrocode}
      \long\@namedef{system@active@arg\string##1}####1{%
        \expandafter\ifx
        \csname \system@group @sh@\string##1@\string####1@\endcsname
        \relax
          \bbl@afterelse\csname normal@char\string##1\endcsname####1%
        \else
          \bbl@afterfi
          \csname \system@group @sh@\string##1@\string####1@\endcsname
        \fi}%
%    \end{macrocode}
%    When a shorthand combination such as |''| ends up in a heading
%    \TeX\ would see |\protect'\protect'|. To prevent this from
%    happening a shorthand needs to be defined at user level.
% \changes{babel~3.7f}{1999/12/09}{Added an extra shorthand
%    combination on user level to catch an interfering \cs{protect}}
%    \begin{macrocode}
      \@namedef{user@sh@\string##1@\string\protect@}{%
        \csname user@active\string##1\endcsname}%
      }%
    }\x
%    \end{macrocode}
%  \end{macro}
%
%  \begin{macro}{\bbl@sh@select}
%    This command helps the shorthand supporting macros to select how
%    to proceed. Note that this macro needs to be expandable as do all
%    the shorthand macros in order for them to work in expansion-only
%    environments such as the argument of |\hyphenation|.
%
%    This macro expects the name of a group of shorthands in its first
%    argument and a shorthand character in its second argument. It
%    will expand to either |\bbl@firstcs| or |\bbl@scndcs|. Hence two
%    more arguments need to follow it.
% \changes{babel~3.8b}{2004/04/19}{Added command}
%    \begin{macrocode}
\def\bbl@sh@select#1#2{%
  \expandafter\ifx\csname#1@sh@\string#2@sel\endcsname\relax
    \bbl@afterelse\bbl@scndcs
  \else
    \bbl@afterfi\csname#1@sh@\string#2@sel\endcsname
  \fi
}
%    \end{macrocode}
%  \end{macro}
%
%  \begin{macro}{\active@prefix}
%    The command |\active@prefix| which is used in the expansion of
%    active characters has a function similar to |\OT1-cmd| in that it
%    |\protect|s the active character whenever |\protect| is
%    \emph{not} |\@typeset@protect|.
% \changes{babel~3.5d}{1995/07/02}{\cs{@protected@cmd} has vanished
%    from \file{ltoutenc.dtx}}
% \changes{babel~3.7o}{2003/11/17}{Added handling of the situation
%    where \cs{protect} is set to \cs{@unexpandable@protect}}
%    \begin{macrocode}
\def\active@prefix#1{%
  \ifx\protect\@typeset@protect
  \else
%    \end{macrocode}
%    When |\protect| is set to |\@unexpandable@protect| we make sure
%    that the active character is als \emph{not} expanded by inserting
%    |\noexpand| in front of it. The |\@gobble| is needed to remove
%    a token such as |\activechar:| (when the double colon was the
%    active character to be dealt with).
%    \begin{macrocode}
    \ifx\protect\@unexpandable@protect
      \bbl@afterelse\bbl@afterfi\noexpand#1\@gobble
    \else
      \bbl@afterfi\bbl@afterfi\protect#1\@gobble
    \fi
  \fi}
%    \end{macrocode}
%  \end{macro}
%
%  \begin{macro}{\if@safe@actives}
%    In some circumstances it is necessary to be able to change the
%    expansion of an active character on the fly. For this purpose the
%    switch |@safe@actives| is available. The setting of this switch
%    should be checked in the first level expansion of
%    |\active@char|\m{char}.
%    \begin{macrocode}
\newif\if@safe@actives
\@safe@activesfalse
%    \end{macrocode}
%  \end{macro}
%
%  \begin{macro}{\bbl@restore@actives}
% \changes{babel~3.7m}{2003/11/15}{New macro added}
%    When the output routine kicks in while the
%    active characters were made ``safe'' this must be undone in
%    the headers to prevent unexpected typeset results. For this
%    situation we define a command to make them ``unsafe'' again.
%    \begin{macrocode}
\def\bbl@restore@actives{\if@safe@actives\@safe@activesfalse\fi}
%    \end{macrocode}
%  \end{macro}
%
%  \begin{macro}{\bbl@activate}
% \changes{babel~3.5a}{1995/02/11}{Added macro}
%
%    This macro takes one argument, like |\initiate@active@char|. The
%    macro is used to change the definition of an active character to
%    expand to |\active@char|\m{char} instead of
%    |\normal@char|\m{char}.
%    \begin{macrocode}
\def\bbl@activate#1{%
  \expandafter\def
  \expandafter#1\expandafter{%
    \expandafter\active@prefix
    \expandafter#1\csname active@char\string#1\endcsname}%
}
%    \end{macrocode}
%  \end{macro}
%
%  \begin{macro}{\bbl@deactivate}
% \changes{babel~3.5a}{1995/02/11}{Added macro}
%    This macro takes one argument, like |\bbl@activate|. The macro
%    doesn't really make a character non-active; it changes its
%    definition to expand to |\normal@char|\m{char}.
%    \begin{macrocode}
\def\bbl@deactivate#1{%
  \expandafter\def
  \expandafter#1\expandafter{%
    \expandafter\active@prefix
    \expandafter#1\csname normal@char\string#1\endcsname}%
}
%    \end{macrocode}
%  \end{macro}
%
%  \begin{macro}{\bbl@firstcs}
%  \begin{macro}{\bbl@scndcs}
%    These macros have two arguments. They use one of their arguments
%    to build a control sequence from.
%    \begin{macrocode}
\def\bbl@firstcs#1#2{\csname#1\endcsname}
\def\bbl@scndcs#1#2{\csname#2\endcsname}
%    \end{macrocode}
%  \end{macro}
%  \end{macro}
%
%  \begin{macro}{\declare@shorthand}
%    The command |\declare@shorthand| is used to declare a shorthand
%    on a certain level. It takes three arguments:
%    \begin{enumerate}
%    \item a name for the collection of shorthands, i.e. `system', or
%      `dutch';
%    \item the character (sequence) that makes up the shorthand,
%      i.e. |~| or |"a|;
%    \item the code to be executed when the shorthand is encountered.
%    \end{enumerate}
% \changes{babel~3.5d}{1995/07/02}{Make a `note' when a shorthand with
%    an argument is defined.}
% \changes{babel~3.6i}{1997/02/23}{Make it possible to distinguish the
%    constructed control sequences for the case with argument}
% \changes{babel~3.8b}{2004/04/19}{We need to support shorthands with
%    and without argument in different groups; added the name of the
%    group to the storage macro}
%    \begin{macrocode}
\def\declare@shorthand#1#2{\@decl@short{#1}#2\@nil}
\def\@decl@short#1#2#3\@nil#4{%
  \def\bbl@tempa{#3}%
  \ifx\bbl@tempa\@empty
    \expandafter\let\csname #1@sh@\string#2@sel\endcsname\bbl@scndcs
    \@namedef{#1@sh@\string#2@}{#4}%
  \else
    \expandafter\let\csname #1@sh@\string#2@sel\endcsname\bbl@firstcs
    \@namedef{#1@sh@\string#2@\string#3@}{#4}%
  \fi}
%    \end{macrocode}
%  \end{macro}
%
%  \begin{macro}{\textormath}
%    Some of the shorthands that will be declared by the language
%    definition files have to be usable in both text and mathmode. To
%    achieve this the helper macro |\textormath| is provided.
%    \begin{macrocode}
\def\textormath#1#2{%
  \ifmmode
    \bbl@afterelse#2%
  \else
    \bbl@afterfi#1%
  \fi}
%    \end{macrocode}
%  \end{macro}
%
%  \begin{macro}{\user@group}
%  \begin{macro}{\language@group}
%  \begin{macro}{\system@group}
%    The current concept of `shorthands' supports three levels or
%    groups of shorthands. For each level the name of the level or
%    group is stored in a macro. The default is to have a user group;
%    use language group `english' and have a system group called
%    `system'.
% \changes{babel~3.6i}{1997/02/24}{Have a user group called `user' by
%    default}
%    \begin{macrocode}
\def\user@group{user}
\def\language@group{english}
\def\system@group{system}
%    \end{macrocode}
%  \end{macro}
%  \end{macro}
%  \end{macro}
%
%  \begin{macro}{\useshorthands}
%    This is the user level command to tell \LaTeX\ that user level
%    shorthands will be used in the document. It takes one argument,
%    the character that starts a shorthand.
% \changes{babel~3.7j}{2001/11/11}{When \TeX\ has seen a character
%    its category code is fixed; need to use a `stand-in' for the
%    call of \cs{bbl@activate}} 
%    \begin{macrocode}
\def\useshorthands#1{%
%    \end{macrocode}
%    First note that this is user level.
%    \begin{macrocode}
  \def\user@group{user}%
%    \end{macrocode}
%    Then initialize the character for use as a shorthand character.
%    \begin{macrocode}
  \initiate@active@char{#1}%
%    \end{macrocode}
%    Now that \TeX\ has seen the character its category code is
%    fixed, but for the actions of |\bbl@activate| to succeed we need
%    it to be active. Hence the trick with the |\lccode| to circumvent
%    this.
% \changes{babel~3.7j}{2003/09/11}{The change from 11/112001 was
%    incomplete} 
%    \begin{macrocode}
  \@tempcnta\lccode`\~
  \lccode`~=`#1%
  \lowercase{\catcode`~\active\bbl@activate{~}}%
  \lccode`\~\@tempcnta}
%    \end{macrocode}
%  \end{macro}
%
%  \begin{macro}{\defineshorthand}
%    Currently we only support one group of user level shorthands,
%    called `user'.
%    \begin{macrocode}
\def\defineshorthand{\declare@shorthand{user}}
%    \end{macrocode}
%  \end{macro}
%
%  \begin{macro}{\languageshorthands}
%    A user level command to change the language from which shorthands
%    are used.
%    \begin{macrocode}
\def\languageshorthands#1{\def\language@group{#1}}
%    \end{macrocode}
%  \end{macro}
%
%  \begin{macro}{\aliasshorthand}
% \changes{babel~3.5f}{1996/01/25}{New command}
%    \begin{macrocode}
\def\aliasshorthand#1#2{%
%    \end{macrocode}
%    First the new shorthand needs to be initialized,
%    \begin{macrocode}
  \expandafter\ifx\csname active@char\string#2\endcsname\relax
     \ifx\document\@notprerr
       \@notshorthand{#2}
     \else
       \initiate@active@char{#2}%
%    \end{macrocode}
%    Then we need to use the |\lccode| trick to make the new shorthand
%    behave like the old one. Therefore we save the current |\lccode|
%    of the |~|-character and restore it later. Then we |\let| the new
%    shorthand character be equal to the original. 
%    \begin{macrocode}
       \@tempcnta\lccode`\~
       \lccode`~=`#2%
       \lowercase{\let~#1}%
       \lccode`\~\@tempcnta
     \fi
   \fi
}
%    \end{macrocode}
%  \end{macro}
%
%  \begin{macro}{\@notshorthand}
% \changes{v3.8d}{2004/11/20}{Error message added}
%    
%    \begin{macrocode}
\def\@notshorthand#1{%
       \PackageError{babel}{%
         The character `\string #1' should be made
         a shorthand character;\MessageBreak
         add the command \string\useshorthands\string{#1\string} to
         the preamble.\MessageBreak
         I will ignore your instruction}{}%
   }
%    \end{macrocode}
%  \end{macro}
%
%  \begin{macro}{\shorthandon}
% \changes{babel~3.7a}{1998/06/07}{Added command}
%  \begin{macro}{\shorthandoff}
% \changes{babel~3.7a}{1998/06/07}{Added command}
%    The first level definition of these macros just passes the
%    argument on to |\bbl@switch@sh|, adding |\@nil| at the end to
%    denote the end of the list of characters.
%    \begin{macrocode}
\newcommand*\shorthandon[1]{\bbl@switch@sh{on}#1\@nil}
\newcommand*\shorthandoff[1]{\bbl@switch@sh{off}#1\@nil}
%    \end{macrocode}
%
%  \begin{macro}{\bbl@switch@sh}
% \changes{babel~3.7a}{1998/06/07}{Added command}
%    The macro |\bbl@switch@sh| takes the list of characters apart one
%    by  one and subsequently switches the category code of the
%    shorthand character according to the first argument of
%    |\bbl@switch@sh|.
%    \begin{macrocode}
\def\bbl@switch@sh#1#2#3\@nil{%
%    \end{macrocode}
%    But before any of this switching takes place we make sure that
%    the character we are dealing with is known as a shorthand
%    character. If it is, a macro such as |\active@char"| should
%    exist.
%    \begin{macrocode}
  \@ifundefined{active@char\string#2}{%
    \PackageError{babel}{%
      The character '\string #2' is not a shorthand character
      in \languagename}{%
      Maybe you made a typing mistake?\MessageBreak
      I will ignore your instruction}}{%
    \csname bbl@switch@sh@#1\endcsname#2}%
%    \end{macrocode}
%    Now that, as the first character in the list has been taken care
%    of, we pass the rest of the list back to |\bbl@switch@sh|.
%    \begin{macrocode}
  \ifx#3\@empty\else
    \bbl@afterfi\bbl@switch@sh{#1}#3\@nil
  \fi}
%    \end{macrocode}
%  \end{macro}
%
%  \begin{macro}{\bbl@switch@sh@off}
%    All that is left to do is define the actual switching
%    macros. Switching off is easy, we just set the category code to
%    `other' (12).
%    \begin{macrocode}
\def\bbl@switch@sh@off#1{\catcode`#112\relax}
%    \end{macrocode}
%  \end{macro}
%
%  \begin{macro}{\bbl@switch@sh@on}
%    But switching the shorthand character back on is a bit more
%    tricky. It involves making sure that we have an active character
%    to begin with when the macro is being defined. It also needs the
%    use of |\lowercase| and |\lccode| trickery to get everything to
%    work out as expected. And to keep things local that need to
%    remain local a group is opened, which is closed as soon as |\x|
%    gets executed.
% \changes{babel~3.8j}{2008/03/21}{Added a group in order to protect
%    the current lowercase code of the tilde (PR 3851)} 
%    \begin{macrocode}
\begingroup
  \catcode`\~\active
  \def\x{\endgroup
    \def\bbl@switch@sh@on##1{%
      \begingroup
      \lccode`~=`##1%
      \lowercase{\endgroup
        \catcode`~\active
        }%
      }%
    }
%    \end{macrocode}
%  \end{macro}
%    The next operation makes the above definition effective.
%    \begin{macrocode}
\x
%
%    \end{macrocode}
%  \end{macro}
%  \end{macro}
%
%    To prevent problems with constructs such as |\char"01A| when the
%    double quote is made active, we define a shorthand on
%    system level.
% \changes{babel~3.5a}{1995/03/10}{Replaced 16 system shorthands to
%    deal with hex numbers by one}
%    \begin{macrocode}
\declare@shorthand{system}{"}{\csname normal@char\string"\endcsname}
%    \end{macrocode}
%
%    When the right quote is made active we need to take care of
%    handling it correctly in mathmode. Therefore we define a
%    shorthand at system level to make it expand to a non-active right
%    quote in textmode, but expand to its original definition in
%    mathmode. (Note that the right quote is `active' in mathmode
%    because of its mathcode.)
% \changes{babel~3.5a}{1995/03/10}{Added a system shorthand for the
%    right quote}
%    \begin{macrocode}
\declare@shorthand{system}{'}{%
  \textormath{\csname normal@char\string'\endcsname}%
             {\sp\bgroup\prim@s}}
%    \end{macrocode}
%
%    When the left quote is  made active we need to take care of
%    handling it correctly when it is followed by for instance an open
%    brace token. Therefore we define a shorthand at system level to
%    make it expand to a non-active left quote.
% \changes{babel~3.5f}{1996/03/06}{Added a system shorthand for the
%    left quote}
%    \begin{macrocode}
\declare@shorthand{system}{`}{\csname normal@char\string`\endcsname}
%    \end{macrocode}
%
%  \begin{macro}{\bbl@prim@s}
% \changes{babel~3.7f}{1999/12/01}{Need to redefine \cs{prim@s} as
%    well as plain \TeX's definition uses \cs{next}}
%  \begin{macro}{\bbl@pr@m@s}
% \changes{babel~3.5a}{1995/03/10}{Added macro}
%    One of the internal macros that are involved in substituting
%    |\prime| for each right quote in mathmode is |\prim@s|. This
%    checks if the next character is a right quote. When the right
%    quote is active, the definition of this macro needs to be adapted
%    to look for an active right quote.
%    \begin{macrocode}
\def\bbl@prim@s{%
  \prime\futurelet\@let@token\bbl@pr@m@s}
\begingroup
  \catcode`\'\active\let'\relax
  \def\x{\endgroup
    \def\bbl@pr@m@s{%
      \ifx'\@let@token
        \expandafter\pr@@@s
      \else
        \ifx^\@let@token
          \expandafter\expandafter\expandafter\pr@@@t
        \else
          \egroup
        \fi
      \fi}%
    }
\x
%    \end{macrocode}
%  \end{macro}
%  \end{macro}
%
%    \begin{macrocode}
%</core|shorthands>
%    \end{macrocode}
%
%    Normally the |~| is active and expands to \verb*=\penalty\@M\ =.
%    When it is written to the \file{.aux} file it is written
%    expanded. To prevent that and to be able to use the character |~|
%    as a start character for a shorthand, it is redefined here as a
%    one character shorthand on system level.
% \changes{babel~3.5f}{1996/04/02}{No need to reset the category code
%    of the tilde as \cs{initiate@active@char} now correctly deals
%    with active characters}
%    \begin{macrocode}
%<*core>
\initiate@active@char{~}
\declare@shorthand{system}{~}{\leavevmode\nobreak\ }
\bbl@activate{~}
%    \end{macrocode}
%
%  \begin{macro}{\OT1dqpos}
%  \begin{macro}{\T1dqpos}
%    The position of the double quote character is different for the
%    OT1 and T1 encodings. It will later be selected using the
%    |\f@encoding| macro. Therefor we define two macros here to store
%    the position of the character in these encodings.
%    \begin{macrocode}
\expandafter\def\csname OT1dqpos\endcsname{127}
\expandafter\def\csname T1dqpos\endcsname{4}
%    \end{macrocode}
%    When the macro |\f@encoding| is undefined (as it is in plain
%    \TeX) we define it here to expand to \texttt{OT1}
%    \begin{macrocode}
\ifx\f@encoding\@undefined
  \def\f@encoding{OT1}
\fi
%    \end{macrocode}
%  \end{macro}
%  \end{macro}
%
%  \subsection{Language attributes}
%
%    Language attributes provide a means to give the user control over
%    which features of the language definition files he wants to
%    enable.
% \changes{babel~3.7c}{1998/07/02}{Added support for language
%    attributes}
%  \begin{macro}{\languageattribute}
%    The macro |\languageattribute| checks whether its arguments are
%    valid and then activates the selected language attribute.
%    \begin{macrocode}
\newcommand\languageattribute[2]{%
%    \end{macrocode}
%    First check whether the language is known.
%    \begin{macrocode}
  \expandafter\ifx\csname l@#1\endcsname\relax
    \@nolanerr{#1}%
  \else
%    \end{macrocode}
%    Than process each attribute in the list.
%    \begin{macrocode}
    \@for\bbl@attr:=#2\do{%
%    \end{macrocode}
%    We want to make sure that each attribute is selected only once;
%    therefor we store the already selected attributes in
%    |\bbl@known@attribs|. When that control sequence is not yet
%    defined this attribute is certainly not selected before.
%    \begin{macrocode}
      \ifx\bbl@known@attribs\@undefined
        \in@false
      \else
%    \end{macrocode}
%    Now we need to see if the attribute occurs in the list of
%    already selected attributes.
%    \begin{macrocode}
        \edef\bbl@tempa{\noexpand\in@{,#1-\bbl@attr,}%
          {,\bbl@known@attribs,}}%
        \bbl@tempa
      \fi
%    \end{macrocode}
%    When the attribute was in the list we issue a warning; this might
%    not be the users intention.
%    \begin{macrocode}
      \ifin@
        \PackageWarning{Babel}{%
          You have more than once selected the attribute
          '\bbl@attr'\MessageBreak for language #1}%
      \else
%    \end{macrocode}
%    When we end up here the attribute is not selected before. So, we
%    add it to the list of selected attributes and execute the
%    associated \TeX-code.
%    \begin{macrocode}
        \edef\bbl@tempa{%
          \noexpand\bbl@add@list\noexpand\bbl@known@attribs{#1-\bbl@attr}}%
        \bbl@tempa
        \edef\bbl@tempa{#1-\bbl@attr}%
        \expandafter\bbl@ifknown@ttrib\expandafter{\bbl@tempa}\bbl@attributes%
        {\csname#1@attr@\bbl@attr\endcsname}%
        {\@attrerr{#1}{\bbl@attr}}%
     \fi
      }
  \fi}
%    \end{macrocode}
%    This command should only be used in the preamble of a document.
%    \begin{macrocode}
\@onlypreamble\languageattribute
%    \end{macrocode}
%    The error text to be issued when an unknown attribute is
%    selected.
%    \begin{macrocode}
  \newcommand*{\@attrerr}[2]{%
    \PackageError{babel}%
                 {The attribute #2 is unknown for language #1.}%
        {Your command will be ignored, type <return> to proceed}}
%    \end{macrocode}
%  \end{macro}
%
%  \begin{macro}{\bbl@declare@ttribute}
%    This command adds the new language/attribute combination to the
%    list of known attributes.
%    \begin{macrocode}
\def\bbl@declare@ttribute#1#2#3{%
  \bbl@add@list\bbl@attributes{#1-#2}%
%    \end{macrocode}
%    Then it defines a control sequence to be executed when the
%    attribute is used in a document. The result of this should be
%    that the macro |\extras...| for the current language is extended,
%    otherwise the attribute will not work as its code is removed from
%    memory at |\begin{document}|.
%    \begin{macrocode}
  \expandafter\def\csname#1@attr@#2\endcsname{#3}%
  }
%    \end{macrocode}
%  \end{macro}
%
%  \begin{macro}{\bbl@ifattributeset}
% \changes{babel~3.7f}{2000/02/12}{macro added}
%    This internal macro has 4 arguments. It can be used to interpret
%    \TeX\ code based on whether a certain attribute was set. This
%    command should appear inside the argument to |\AtBeginDocument|
%    because the attributes are set in the document preamble,
%    \emph{after} \babel\ is loaded.
%
%    The first argument is the language, the second argument the
%    attribute being checked, and the third and fourth arguments are
%    the true and false clauses.
%    \begin{macrocode}
\def\bbl@ifattributeset#1#2#3#4{%
%    \end{macrocode}
%    First we need to find out if any attributes were set; if not
%    we're done.
%    \begin{macrocode}
  \ifx\bbl@known@attribs\@undefined
    \in@false
  \else
%    \end{macrocode}
%    The we need to check the list of known attributes.
%    \begin{macrocode}
    \edef\bbl@tempa{\noexpand\in@{,#1-#2,}%
      {,\bbl@known@attribs,}}%
    \bbl@tempa
  \fi
%    \end{macrocode}
%    When we're this far |\ifin@| has a value indicating if the
%    attribute in question was set or not. Just to be safe the code to
%    be executed is `thrown over the |\fi|'.
%    \begin{macrocode}
  \ifin@
    \bbl@afterelse#3%
  \else
    \bbl@afterfi#4%
  \fi
  }
%    \end{macrocode}
%  \end{macro}
%
%  \begin{macro}{\bbl@add@list}
%    This internal macro adds its second argument to a comma
%    separated list in its first argument. When the list is not
%    defined yet (or empty), it will be initiated
%    \begin{macrocode}
\def\bbl@add@list#1#2{%
  \ifx#1\@undefined
    \def#1{#2}%
  \else
    \ifx#1\@empty
      \def#1{#2}%
    \else
      \edef#1{#1,#2}%
    \fi
  \fi
  }
%    \end{macrocode}
%  \end{macro}
%
%  \begin{macro}{\bbl@ifknown@ttrib}
%    An internal macro to check whether a given language/attribute is
%    known. The macro takes 4 arguments, the language/attribute, the
%    attribute list, the \TeX-code to be executed when the attribute
%    is known and the \TeX-code to be executed otherwise.
%    \begin{macrocode}
\def\bbl@ifknown@ttrib#1#2{%
%    \end{macrocode}
%    We first assume the attribute is unknown.
%    \begin{macrocode}
  \let\bbl@tempa\@secondoftwo
%    \end{macrocode}
%    Then we loop over the list of known attributes, trying to find a
%    match.
%    \begin{macrocode}
  \@for\bbl@tempb:=#2\do{%
    \expandafter\in@\expandafter{\expandafter,\bbl@tempb,}{,#1,}%
    \ifin@
%    \end{macrocode}
%    When a match is found the definition of |\bbl@tempa| is changed.
%    \begin{macrocode}
      \let\bbl@tempa\@firstoftwo
    \else
    \fi}%
%    \end{macrocode}
%    Finally we execute |\bbl@tempa|.
%    \begin{macrocode}
  \bbl@tempa
}
%    \end{macrocode}
%  \end{macro}
%
%  \begin{macro}{\bbl@clear@ttribs}
%    This macro removes all the attribute code from \LaTeX's memory at
%    |\begin{document}| time (if any is present).
% \changes{babel~3.7e}{1999/09/24}{When \cs{bbl@attributes} is
%    undefined this should be a no-op} 
%    \begin{macrocode}
\def\bbl@clear@ttribs{%
  \ifx\bbl@attributes\@undefined\else
    \@for\bbl@tempa:=\bbl@attributes\do{%
      \expandafter\bbl@clear@ttrib\bbl@tempa.
      }%
    \let\bbl@attributes\@undefined
  \fi
  }
\def\bbl@clear@ttrib#1-#2.{%
  \expandafter\let\csname#1@attr@#2\endcsname\@undefined}
\AtBeginDocument{\bbl@clear@ttribs}
%    \end{macrocode}
%  \end{macro}
%
%  \subsection{Support for saving macro definitions}
%
%    To save the meaning of control sequences using |\babel@save|, we
%    use temporary control sequences.  To save hash table entries for
%    these control sequences, we don't use the name of the control
%    sequence to be saved to construct the temporary name.  Instead we
%    simply use the value of a counter, which is reset to zero each
%    time we begin to save new values.  This works well because we
%    release the saved meanings before we begin to save a new set of
%    control sequence meanings (see |\selectlanguage| and
%    |\originalTeX|).
%
%  \begin{macro}{\babel@savecnt}
% \changes{babel~3.2}{1991/11/10}{Added macro}
%  \begin{macro}{\babel@beginsave}
% \changes{babel~3.2}{1991/11/10}{Added macro}
%    The initialization of a new save cycle: reset the counter to
%    zero.
%    \begin{macrocode}
\def\babel@beginsave{\babel@savecnt\z@}
%    \end{macrocode}
%    Before it's forgotten, allocate the counter and initialize all.
%    \begin{macrocode}
\newcount\babel@savecnt
\babel@beginsave
%    \end{macrocode}
%  \end{macro}
%  \end{macro}
%
%  \begin{macro}{\babel@save}
% \changes{babel~3.2}{1991/11/10}{Added macro}
%    The macro |\babel@save|\meta{csname} saves the current meaning of
%    the control sequence \meta{csname} to
%    |\originalTeX|\footnote{\cs{originalTeX} has to be
%    expandable, i.\,e.\ you shouldn't let it to \cs{relax}.}.
%    To do this, we let the current meaning to a temporary control
%    sequence, the restore commands are appended to |\originalTeX| and
%    the counter is incremented.
% \changes{babel~3.2c}{1992/03/17}{missing backslash led to errors
%    when executing \cs{originalTeX}}
% \changes{babel~3.2d}{1992/07/02}{saving in \cs{babel@i} and
%    restoring from \cs{@babel@i} doesn't work very well...}
%    \begin{macrocode}
\def\babel@save#1{%
  \expandafter\let\csname babel@\number\babel@savecnt\endcsname #1\relax
  \begingroup
    \toks@\expandafter{\originalTeX \let#1=}%
    \edef\x{\endgroup
      \def\noexpand\originalTeX{\the\toks@ \expandafter\noexpand
         \csname babel@\number\babel@savecnt\endcsname\relax}}%
  \x
  \advance\babel@savecnt\@ne}
%    \end{macrocode}
%  \end{macro}
%
%  \begin{macro}{\babel@savevariable}
% \changes{babel~3.2}{1991/11/10}{Added macro}
%    The macro |\babel@savevariable|\meta{variable} saves the value of
%    the variable.  \meta{variable} can be anything allowed after the
%    |\the| primitive.
%    \begin{macrocode}
\def\babel@savevariable#1{\begingroup
    \toks@\expandafter{\originalTeX #1=}%
    \edef\x{\endgroup
      \def\noexpand\originalTeX{\the\toks@ \the#1\relax}}%
  \x}
%    \end{macrocode}
%  \end{macro}
%
%  \begin{macro}{\bbl@frenchspacing}
%  \begin{macro}{\bbl@nonfrenchspacing}
%    Some languages need to have |\frenchspacing| in effect. Others
%    don't want that. The command |\bbl@frenchspacing| switches it on
%    when it isn't already in effect and |\bbl@nonfrenchspacing|
%    switches it off if necessary.
%    \begin{macrocode}
\def\bbl@frenchspacing{%
  \ifnum\the\sfcode`\.=\@m
    \let\bbl@nonfrenchspacing\relax
  \else
    \frenchspacing
    \let\bbl@nonfrenchspacing\nonfrenchspacing
  \fi}
\let\bbl@nonfrenchspacing\nonfrenchspacing
%    \end{macrocode}
%  \end{macro}
%  \end{macro}
%
% \subsection{Support for extending macros}
%
%  \begin{macro}{\addto}
%    For each language four control sequences have to be defined that
%    control the language-specific definitions. To be able to add
%    something to these macro once they have been defined the macro
%    |\addto| is introduced. It takes two arguments, a \meta{control
%    sequence} and \TeX-code to be added to the \meta{control
%    sequence}.
%
%    If the \meta{control sequence} has not been defined before it is
%    defined now.
% \changes{babel~3.1}{1991/11/05}{Added macro}
% \changes{babel~3.4}{1994/02/04}{Changed to use toks register}
% \changes{babel~3.6b}{1996/12/30}{Also check if control sequence
%    expands to \cs{relax}}
%    \begin{macrocode}
\def\addto#1#2{%
  \ifx#1\@undefined
    \def#1{#2}%
  \else
%    \end{macrocode}
%    The control sequence could also expand to |\relax|, in which case
%    a circular definition results. The net result is a stack overflow.
%    \begin{macrocode}
    \ifx#1\relax
      \def#1{#2}%
    \else
%    \end{macrocode}
%    Otherwise the replacement text for the \meta{control sequence} is
%    expanded and stored in a token register, together with the
%    \TeX-code to be added.  Finally the \meta{control sequence} is
%    \emph{re}defined, using the contents of the token register.
%    \begin{macrocode}
      {\toks@\expandafter{#1#2}%
        \xdef#1{\the\toks@}}%
    \fi
  \fi
}
%    \end{macrocode}
%  \end{macro}
%
% \subsection{Macros common to a number of languages}
%
%  \begin{macro}{\allowhyphens}
% \changes{babel~3.2b}{1992/02/16}{Moved macro from language
%    definition files}
% \changes{babel~3.7a}{1998/03/12}{Make \cs{allowhyphens} a no-op for
%    T1 fontencoding}
%    This macro makes hyphenation possible. Basically its definition
%    is nothing more than |\nobreak| |\hskip| \texttt{0pt plus
%    0pt}\footnote{\TeX\ begins and ends a word for hyphenation at a
%    glue node. The penalty prevents a linebreak at this glue node.}.
%    \begin{macrocode}
\def\bbl@t@one{T1}
\def\allowhyphens{%
  \ifx\cf@encoding\bbl@t@one\else\bbl@allowhyphens\fi}
\def\bbl@allowhyphens{\nobreak\hskip\z@skip}
%    \end{macrocode}
%  \end{macro}
%
%  \begin{macro}{\set@low@box}
% \changes{babel~3.2b}{1992/02/16}{Moved macro from language
%    definition files}
%    The following macro is used to lower quotes to the same level as
%    the comma.  It prepares its argument in box register~0.
%    \begin{macrocode}
\def\set@low@box#1{\setbox\tw@\hbox{,}\setbox\z@\hbox{#1}%
    \dimen\z@\ht\z@ \advance\dimen\z@ -\ht\tw@%
    \setbox\z@\hbox{\lower\dimen\z@ \box\z@}\ht\z@\ht\tw@ \dp\z@\dp\tw@}
%    \end{macrocode}
%  \end{macro}
%
%  \begin{macro}{\save@sf@q}
% \changes{babel~3.2b}{1992/02/16}{Moved macro from language
%    definition files}
%    The macro |\save@sf@q| is used to save and reset the current
%    space factor.
% \changes{babel~3.7f}{2000/09/19}{PR3119, don't start a paragraph in
%    a local group}
%    \begin{macrocode}
\def\save@sf@q #1{\leavevmode
 \begingroup 
  \edef\@SF{\spacefactor \the\spacefactor}#1\@SF
 \endgroup
}
%    \end{macrocode}
%  \end{macro}
%
%  \begin{macro}{\bbl@disc}
% \changes{babel~3.5f}{1996/01/24}{Macro moved from language
%    definition files}
%    For some languages the macro |\bbl@disc| is used to ease the
%    insertion of discretionaries for letters that behave `abnormally'
%    at a breakpoint.
%    \begin{macrocode}
\def\bbl@disc#1#2{%
  \nobreak\discretionary{#2-}{}{#1}\allowhyphens}
%    \end{macrocode}
%  \end{macro}
%
% \changes{babel~3.5c}{1995/06/14}{Repaired a typo (itlaic, PR1652)}
%
%  \subsection{Making glyphs available}
%
%    The file \file{\filename}\footnote{The file described in this
%    section has version number \fileversion, and was last revised on
%    \filedate.} makes a number of glyphs available that either do not
%    exist in the \texttt{OT1} encoding and have to be `faked', or
%    that are not accessible through \file{T1enc.def}.
%
%  \subsection{Quotation marks}
%
%  \begin{macro}{\quotedblbase}
%    In the \texttt{T1} encoding the opening double quote at the
%    baseline is available as a separate character, accessible via
%    |\quotedblbase|. In the \texttt{OT1} encoding it is not
%    available, therefor we make it available by lowering the normal
%    open quote character to the baseline.
%    \begin{macrocode}
\ProvideTextCommand{\quotedblbase}{OT1}{%
  \save@sf@q{\set@low@box{\textquotedblright\/}%
    \box\z@\kern-.04em\allowhyphens}}
%    \end{macrocode}
%    Make sure that when an encoding other than \texttt{OT1} or
%    \texttt{T1} is used this glyph can still be typeset.
%    \begin{macrocode}
\ProvideTextCommandDefault{\quotedblbase}{%
  \UseTextSymbol{OT1}{\quotedblbase}}
%    \end{macrocode}
%  \end{macro}
%
%  \begin{macro}{\quotesinglbase}
%    We also need the single quote character at the baseline.
%    \begin{macrocode}
\ProvideTextCommand{\quotesinglbase}{OT1}{%
  \save@sf@q{\set@low@box{\textquoteright\/}%
    \box\z@\kern-.04em\allowhyphens}}
%    \end{macrocode}
%    Make sure that when an encoding other than \texttt{OT1} or
%    \texttt{T1} is used this glyph can still be typeset.
%    \begin{macrocode}
\ProvideTextCommandDefault{\quotesinglbase}{%
  \UseTextSymbol{OT1}{\quotesinglbase}}
%    \end{macrocode}
%  \end{macro}
%
%  \begin{macro}{\guillemotleft}
%  \begin{macro}{\guillemotright}
%    The guillemet characters are not available in \texttt{OT1}
%    encoding. They are faked.
%    \begin{macrocode}
\ProvideTextCommand{\guillemotleft}{OT1}{%
  \ifmmode
    \ll
  \else
    \save@sf@q{\nobreak
      \raise.2ex\hbox{$\scriptscriptstyle\ll$}\allowhyphens}%
  \fi}
\ProvideTextCommand{\guillemotright}{OT1}{%
  \ifmmode
    \gg
  \else
    \save@sf@q{\nobreak
      \raise.2ex\hbox{$\scriptscriptstyle\gg$}\allowhyphens}%
  \fi}
%    \end{macrocode}
%    Make sure that when an encoding other than \texttt{OT1} or
%    \texttt{T1} is used these glyphs can still be typeset.
%    \begin{macrocode}
\ProvideTextCommandDefault{\guillemotleft}{%
  \UseTextSymbol{OT1}{\guillemotleft}}
\ProvideTextCommandDefault{\guillemotright}{%
  \UseTextSymbol{OT1}{\guillemotright}}
%    \end{macrocode}
%  \end{macro}
%  \end{macro}
%
%  \begin{macro}{\guilsinglleft}
%  \begin{macro}{\guilsinglright}
%    The single guillemets are not available in \texttt{OT1}
%    encoding. They are faked.
%    \begin{macrocode}
\ProvideTextCommand{\guilsinglleft}{OT1}{%
  \ifmmode
    <%
  \else
    \save@sf@q{\nobreak
      \raise.2ex\hbox{$\scriptscriptstyle<$}\allowhyphens}%
  \fi}
\ProvideTextCommand{\guilsinglright}{OT1}{%
  \ifmmode
    >%
  \else
    \save@sf@q{\nobreak
      \raise.2ex\hbox{$\scriptscriptstyle>$}\allowhyphens}%
  \fi}
%    \end{macrocode}
%    Make sure that when an encoding other than \texttt{OT1} or
%    \texttt{T1} is used these glyphs can still be typeset.
%    \begin{macrocode}
\ProvideTextCommandDefault{\guilsinglleft}{%
  \UseTextSymbol{OT1}{\guilsinglleft}}
\ProvideTextCommandDefault{\guilsinglright}{%
  \UseTextSymbol{OT1}{\guilsinglright}}
%    \end{macrocode}
%  \end{macro}
%  \end{macro}
%
%
%  \subsection{Letters}
%
%  \begin{macro}{\ij}
%  \begin{macro}{\IJ}
%    The dutch language uses the letter `ij'. It is available in
%    \texttt{T1} encoded fonts, but not in the \texttt{OT1} encoded
%    fonts. Therefor we fake it for the \texttt{OT1} encoding.
% \changes{dutch-3.7a}{1995/02/04}{Changed the kerning in the faked ij
%    to match the dc-version of it}
%    \begin{macrocode}
\DeclareTextCommand{\ij}{OT1}{%
  \allowhyphens i\kern-0.02em j\allowhyphens}
\DeclareTextCommand{\IJ}{OT1}{%
  \allowhyphens I\kern-0.02em J\allowhyphens}
\DeclareTextCommand{\ij}{T1}{\char188}
\DeclareTextCommand{\IJ}{T1}{\char156}
%    \end{macrocode}
%    Make sure that when an encoding other than \texttt{OT1} or
%    \texttt{T1} is used these glyphs can still be typeset.
%    \begin{macrocode}
\ProvideTextCommandDefault{\ij}{%
  \UseTextSymbol{OT1}{\ij}}
\ProvideTextCommandDefault{\IJ}{%
  \UseTextSymbol{OT1}{\IJ}}
%    \end{macrocode}
%  \end{macro}
%  \end{macro}
%
%  \begin{macro}{\dj}
%  \begin{macro}{\DJ}
%    The croatian language needs the letters |\dj| and |\DJ|; they are
%    available in the \texttt{T1} encoding, but not in the
%    \texttt{OT1} encoding by default.
%
%    Some code to construct these glyphs for the \texttt{OT1} encoding
%    was made available to me by Stipcevic Mario,
%    (\texttt{stipcevic@olimp.irb.hr}).
% \changes{babel~3.5f}{1996/03/28}{New definition of \cs{dj}, see PR
%    2058}
%    \begin{macrocode}
\def\crrtic@{\hrule height0.1ex width0.3em}
\def\crttic@{\hrule height0.1ex width0.33em}
%
\def\ddj@{%
  \setbox0\hbox{d}\dimen@=\ht0
  \advance\dimen@1ex
  \dimen@.45\dimen@
  \dimen@ii\expandafter\rem@pt\the\fontdimen\@ne\font\dimen@
  \advance\dimen@ii.5ex
  \leavevmode\rlap{\raise\dimen@\hbox{\kern\dimen@ii\vbox{\crrtic@}}}}
\def\DDJ@{%
  \setbox0\hbox{D}\dimen@=.55\ht0
  \dimen@ii\expandafter\rem@pt\the\fontdimen\@ne\font\dimen@
  \advance\dimen@ii.15ex %            correction for the dash position
  \advance\dimen@ii-.15\fontdimen7\font %     correction for cmtt font
  \dimen\thr@@\expandafter\rem@pt\the\fontdimen7\font\dimen@
  \leavevmode\rlap{\raise\dimen@\hbox{\kern\dimen@ii\vbox{\crttic@}}}}
%
\DeclareTextCommand{\dj}{OT1}{\ddj@ d}
\DeclareTextCommand{\DJ}{OT1}{\DDJ@ D}
%    \end{macrocode}
%    Make sure that when an encoding other than \texttt{OT1} or
%    \texttt{T1} is used these glyphs can still be typeset.
%    \begin{macrocode}
\ProvideTextCommandDefault{\dj}{%
  \UseTextSymbol{OT1}{\dj}}
\ProvideTextCommandDefault{\DJ}{%
  \UseTextSymbol{OT1}{\DJ}}
%    \end{macrocode}
%  \end{macro}
%  \end{macro}
%
%  \begin{macro}{\SS}
%    For the \texttt{T1} encoding |\SS| is defined and selects a
%    specific glyph from the font, but for other encodings it is not
%    available. Therefor we make it available here.
%    \begin{macrocode}
\DeclareTextCommand{\SS}{OT1}{SS}
\ProvideTextCommandDefault{\SS}{\UseTextSymbol{OT1}{\SS}}
%    \end{macrocode}
%  \end{macro}
%
% \subsection{Shorthands for quotation marks}
%
%    Shorthands are provided for a number of different quotation
%    marks, which make them usable both outside and inside mathmode.
%
%  \begin{macro}{\glq}
%  \begin{macro}{\grq}
% \changes{babel~3.7a}{1997/04/25}{Make the definition of \cs{grq}
%    dependent on the font encoding}
% \changes{babel~3.8b}{2004/05/02}{Made \cs{glq} fontencoding
%    dependent as well} 
%    The `german' single quotes.
%    \begin{macrocode}
\ProvideTextCommand{\glq}{OT1}{%
  \textormath{\quotesinglbase}{\mbox{\quotesinglbase}}}
\ProvideTextCommand{\glq}{T1}{%
  \textormath{\quotesinglbase}{\mbox{\quotesinglbase}}}
\ProvideTextCommandDefault{\glq}{\UseTextSymbol{OT1}\glq}
%    \end{macrocode}
%    The definition of |\grq| depends on the fontencoding. With
%    \texttt{T1} encoding no extra kerning is needed.
%    \begin{macrocode}
\ProvideTextCommand{\grq}{T1}{%
  \textormath{\textquoteleft}{\mbox{\textquoteleft}}}
\ProvideTextCommand{\grq}{OT1}{%
  \save@sf@q{\kern-.0125em%
  \textormath{\textquoteleft}{\mbox{\textquoteleft}}%
  \kern.07em\relax}}
\ProvideTextCommandDefault{\grq}{\UseTextSymbol{OT1}\grq}
%    \end{macrocode}
%  \end{macro}
%  \end{macro}
%
%  \begin{macro}{\glqq}
%  \begin{macro}{\grqq}
% \changes{babel~3.7a}{1997/04/25}{Make the definition of \cs{grqq}
%    dependent on the font encoding}
% \changes{babel~3.8b}{2004/05/02}{Made \cs{grqq} fontencoding
%    dependent as well} 
%    The `german' double quotes.
%    \begin{macrocode}
\ProvideTextCommand{\glqq}{OT1}{%
  \textormath{\quotedblbase}{\mbox{\quotedblbase}}}
\ProvideTextCommand{\glqq}{T1}{%
  \textormath{\quotedblbase}{\mbox{\quotedblbase}}}
\ProvideTextCommandDefault{\glqq}{\UseTextSymbol{OT1}\glqq}
%    \end{macrocode}
%    The definition of |\grqq| depends on the fontencoding. With
%    \texttt{T1} encoding no extra kerning is needed.
%    \begin{macrocode}
\ProvideTextCommand{\grqq}{T1}{%
  \textormath{\textquotedblleft}{\mbox{\textquotedblleft}}}
\ProvideTextCommand{\grqq}{OT1}{%
  \save@sf@q{\kern-.07em%
  \textormath{\textquotedblleft}{\mbox{\textquotedblleft}}%
  \kern.07em\relax}}
\ProvideTextCommandDefault{\grqq}{\UseTextSymbol{OT1}\grqq}
%    \end{macrocode}
%  \end{macro}
%  \end{macro}
%
%  \begin{macro}{\flq}
%  \begin{macro}{\frq}
% \changes{babel~3.5f}{1995/08/07}{corrected spelling of
%    \cs{quilsingl...}}
% \changes{babel~3.5f}{1995/09/05}{now use \cs{textormath} in these
%    definitions}
% \changes{babel~3.8b}{2004/05/02}{Made \cs{flq} and \cs{frq}
%    fontencoding dependent} 
%    The `french' single guillemets.
%    \begin{macrocode}
\ProvideTextCommand{\flq}{OT1}{%
  \textormath{\guilsinglleft}{\mbox{\guilsinglleft}}}
\ProvideTextCommand{\flq}{T1}{%
  \textormath{\guilsinglleft}{\mbox{\guilsinglleft}}}
\ProvideTextCommandDefault{\flq}{\UseTextSymbol{OT1}\flq}
%    \end{macrocode}
%    
%    \begin{macrocode}
\ProvideTextCommand{\frq}{OT1}{%
  \textormath{\guilsinglright}{\mbox{\guilsinglright}}}
\ProvideTextCommand{\frq}{T1}{%
  \textormath{\guilsinglright}{\mbox{\guilsinglright}}}
\ProvideTextCommandDefault{\frq}{\UseTextSymbol{OT1}\frq}
%    \end{macrocode}
%  \end{macro}
%  \end{macro}
%
%  \begin{macro}{\flqq}
%  \begin{macro}{\frqq}
% \changes{babel~3.5f}{1995/08/07}{corrected spelling of
%    \cs{quillemot...}}
% \changes{babel~3.5f}{1995/09/05}{now use \cs{textormath} in these
%    definitions}
% \changes{babel~3.8b}{2004/05/02}{Made \cs{flqq} and \cs{frqq}
%    fontencoding dependent} 
%    The `french' double guillemets.
%    \begin{macrocode}
\ProvideTextCommand{\flqq}{OT1}{%
  \textormath{\guillemotleft}{\mbox{\guillemotleft}}}
\ProvideTextCommand{\flqq}{T1}{%
  \textormath{\guillemotleft}{\mbox{\guillemotleft}}}
\ProvideTextCommandDefault{\flqq}{\UseTextSymbol{OT1}\flqq}
%    \end{macrocode}
%    
%    \begin{macrocode}
\ProvideTextCommand{\frqq}{OT1}{%
  \textormath{\guillemotright}{\mbox{\guillemotright}}}
\ProvideTextCommand{\frqq}{T1}{%
  \textormath{\guillemotright}{\mbox{\guillemotright}}}
\ProvideTextCommandDefault{\frqq}{\UseTextSymbol{OT1}\frqq}
%    \end{macrocode}
%  \end{macro}
%  \end{macro}
%
%  \subsection{Umlauts and trema's}
%
%    The command |\"| needs to have a different effect for different
%    languages. For German for instance, the `umlaut' should be
%    positioned lower than the default position for placing it over
%    the letters a, o, u, A, O and U. When placed over an e, i, E or I
%    it can retain its normal position. For Dutch the same glyph is
%    always placed in the lower position.
%
%  \begin{macro}{\umlauthigh}
% \changes{v3.8a}{2004/02/19}{Use \cs{leavevmode}\cs{bgroup} to
%    prevent problems when this command occurs in vertical mode.}
%  \begin{macro}{\umlautlow}
%    To be able to provide both positions of |\"| we provide two
%    commands to switch the positioning, the default will be
%    |\umlauthigh| (the normal positioning).
%    \begin{macrocode}
\def\umlauthigh{%
  \def\bbl@umlauta##1{\leavevmode\bgroup%
      \expandafter\accent\csname\f@encoding dqpos\endcsname
      ##1\allowhyphens\egroup}%
  \let\bbl@umlaute\bbl@umlauta}
\def\umlautlow{%
  \def\bbl@umlauta{\protect\lower@umlaut}}
\def\umlautelow{%
  \def\bbl@umlaute{\protect\lower@umlaut}}
\umlauthigh
%    \end{macrocode}
%  \end{macro}
%  \end{macro}
%
%  \begin{macro}{\lower@umlaut}
%    The command |\lower@umlaut| is used to position the |\"| closer
%    the the letter.
%
%    We want the umlaut character lowered, nearer to the letter. To do
%    this we need an extra \meta{dimen} register.
%    \begin{macrocode}
\expandafter\ifx\csname U@D\endcsname\relax
  \csname newdimen\endcsname\U@D
\fi
%    \end{macrocode}
%    The following code fools \TeX's \texttt{make\_accent} procedure
%    about the current x-height of the font to force another placement
%    of the umlaut character.
%    \begin{macrocode}
\def\lower@umlaut#1{%
%    \end{macrocode}
%    First we have to save the current x-height of the font, because
%    we'll change this font dimension and this is always done
%    globally.
% \changes{v3.8a}{2004/02/19}{Use \cs{leavevmode}\cs{bgroup} to
%    prevent problems when this command occurs in vertical mode.}
%    \begin{macrocode}
  \leavevmode\bgroup
    \U@D 1ex%
%    \end{macrocode}
%    Then we compute the new x-height in such a way that the umlaut
%    character is lowered to the base character.  The value of
%    \texttt{.45ex} depends on the \MF\ parameters with which the
%    fonts were built.  (Just try out, which value will look best.)
%    \begin{macrocode}
    {\setbox\z@\hbox{%
      \expandafter\char\csname\f@encoding dqpos\endcsname}%
      \dimen@ -.45ex\advance\dimen@\ht\z@
%    \end{macrocode}
%    If the new x-height is too low, it is not changed.
%    \begin{macrocode}
      \ifdim 1ex<\dimen@ \fontdimen5\font\dimen@ \fi}%
%    \end{macrocode}
%    Finally we call the |\accent| primitive, reset the old x-height
%    and insert the base character in the argument.
% \changes{babel~3.5f}{1996/04/02}{Added a \cs{allowhyphens}}
% \changes{babel~3.5f}{1996/06/25}{removed \cs{allowhyphens}}
%    \begin{macrocode}
    \expandafter\accent\csname\f@encoding dqpos\endcsname
    \fontdimen5\font\U@D #1%
  \egroup}
%    \end{macrocode}
%  \end{macro}
%
%    For all vowels we declare |\"| to be a composite command which
%    uses |\bbl@umlauta| or |\bbl@umlaute| to position the umlaut
%    character. We need to be sure that these definitions override the
%    ones that are provided when the package \pkg{fontenc} with
%    option \Lopt{OT1} is used. Therefor these declarations are
%    postponed until the beginning of the document.
%    \begin{macrocode}
\AtBeginDocument{%
  \DeclareTextCompositeCommand{\"}{OT1}{a}{\bbl@umlauta{a}}%
  \DeclareTextCompositeCommand{\"}{OT1}{e}{\bbl@umlaute{e}}%
  \DeclareTextCompositeCommand{\"}{OT1}{i}{\bbl@umlaute{\i}}%
  \DeclareTextCompositeCommand{\"}{OT1}{\i}{\bbl@umlaute{\i}}%
  \DeclareTextCompositeCommand{\"}{OT1}{o}{\bbl@umlauta{o}}%
  \DeclareTextCompositeCommand{\"}{OT1}{u}{\bbl@umlauta{u}}%
  \DeclareTextCompositeCommand{\"}{OT1}{A}{\bbl@umlauta{A}}%
  \DeclareTextCompositeCommand{\"}{OT1}{E}{\bbl@umlaute{E}}%
  \DeclareTextCompositeCommand{\"}{OT1}{I}{\bbl@umlaute{I}}%
  \DeclareTextCompositeCommand{\"}{OT1}{O}{\bbl@umlauta{O}}%
  \DeclareTextCompositeCommand{\"}{OT1}{U}{\bbl@umlauta{U}}%
}
%    \end{macrocode}
%
% \subsection{The redefinition of the style commands}
%
%    The rest of the code in this file can only be processed by
%    \LaTeX, so we check the current format. If it is plain \TeX,
%    processing should stop here. But, because of the need to limit
%    the scope of the definition of |\format|, a macro that is used
%    locally in the following |\if|~statement, this comparison is done
%    inside a group. To prevent \TeX\ from complaining about an
%    unclosed group, the processing of the command |\endinput| is
%    deferred until after the group is closed. This is accomplished by
%    the command |\aftergroup|.
%    \begin{macrocode}
{\def\format{lplain}
\ifx\fmtname\format
\else
  \def\format{LaTeX2e}
  \ifx\fmtname\format
  \else
    \aftergroup\endinput
  \fi
\fi}
%    \end{macrocode}
%
%    Now that we're sure that the code is seen by \LaTeX\ only, we
%    have to find out what the main (primary) document style is
%    because we want to redefine some macros.  This is only necessary
%    for releases of \LaTeX\ dated before December~1991. Therefor
%    this part of the code can optionally be included in
%    \file{babel.def} by specifying the \texttt{docstrip} option
%    \texttt{names}.
%    \begin{macrocode}
%<*names>
%    \end{macrocode}
%
%    The standard styles can be distinguished by checking whether some
%    macros are defined. In table~\ref{styles} an overview is given of
%    the macros that can be used for this purpose.
%  \begin{table}[htb]
%  \begin{center}
% \DeleteShortVerb{\|}
%  \begin{tabular}{|lcp{8cm}|}
%   \hline
%   article         & : & both the \verb+\chapter+ and \verb+\opening+
%                         macros are undefined\\
%   report and book & : & the \verb+\chapter+ macro is defined and
%                         the \verb+\opening+ is undefined\\
%   letter          & : & the \verb+\chapter+ macro is undefined and
%                         the \verb+\opening+ is defined\\
%   \hline
%  \end{tabular}
% \caption{How to determine the main document style}\label{styles}
% \MakeShortVerb{\|}
%  \end{center}
%  \end{table}
%
%    \noindent The macros that have to be redefined for the
%    \texttt{report} and \texttt{book} document styles happen to be
%    the same, so there is no need to distinguish between those two
%    styles.
%
%  \begin{macro}{\doc@style}
%    First a parameter |\doc@style| is defined to identify the current
%    document style. This parameter might have been defined by a
%    document style that already uses macros instead of hard-wired
%    texts, such as \file{artikel1.sty}~\cite{BEP}, so the existence of
%    |\doc@style| is checked. If this macro is undefined, i.\,e., if
%    the document style is unknown and could therefore contain
%    hard-wired texts, |\doc@style| is defined to the default
%    value~`0'.
% \changes{babel~3.0d}{1991/10/29}{Removed use of \cs{@ifundefined}}
%    \begin{macrocode}
\ifx\@undefined\doc@style
  \def\doc@style{0}%
%    \end{macrocode}
%    This parameter is defined in the following \texttt{if}
%    construction (see table~\ref{styles}):
%
%    \begin{macrocode}
  \ifx\@undefined\opening
    \ifx\@undefined\chapter
      \def\doc@style{1}%
    \else
      \def\doc@style{2}%
    \fi
  \else
    \def\doc@style{3}%
  \fi%
\fi%
%    \end{macrocode}
%  \end{macro}
%
% \changes{babel~3.1}{1991/11/05}{Removed definition of
%    \cs{if@restonecol}}
%
%    \subsubsection{Redefinition of macros}
%
%    Now here comes the real work: we start to redefine things and
%    replace hard-wired texts by macros. These redefinitions should be
%    carried out conditionally, in case it has already been done.
%
%    For the \texttt{figure} and \texttt{table} environments we have
%    in all styles:
%    \begin{macrocode}
\@ifundefined{figurename}{\def\fnum@figure{\figurename{} \thefigure}}{}
\@ifundefined{tablename}{\def\fnum@table{\tablename{} \thetable}}{}
%    \end{macrocode}
%
%    The rest of the macros have to be treated differently for each
%    style.  When |\doc@style| still has its default value nothing
%    needs to be done.
%    \begin{macrocode}
\ifcase \doc@style\relax
\or
%    \end{macrocode}
%
%    This means that \file{babel.def} is read after the
%    \texttt{article} style, where no |\chapter| and |\opening|
%    commands are defined\footnote{A fact that was pointed out to me
%    by Nico Poppelier and was already used in Piet van Oostrum's
%    document style option~\texttt{nl}.}.
%
%    First we have the |\tableofcontents|,
%    |\listoffigures| and |\listoftables|:
%    \begin{macrocode}
\@ifundefined{contentsname}%
    {\def\tableofcontents{\section*{\contentsname\@mkboth
          {\uppercase{\contentsname}}{\uppercase{\contentsname}}}%
      \@starttoc{toc}}}{}

\@ifundefined{listfigurename}%
    {\def\listoffigures{\section*{\listfigurename\@mkboth
          {\uppercase{\listfigurename}}{\uppercase{\listfigurename}}}%
     \@starttoc{lof}}}{}

\@ifundefined{listtablename}%
    {\def\listoftables{\section*{\listtablename\@mkboth
          {\uppercase{\listtablename}}{\uppercase{\listtablename}}}%
      \@starttoc{lot}}}{}
%    \end{macrocode}
%
% Then the |\thebibliography| and |\theindex| environments.
%
%    \begin{macrocode}
\@ifundefined{refname}%
    {\def\thebibliography#1{\section*{\refname
      \@mkboth{\uppercase{\refname}}{\uppercase{\refname}}}%
      \list{[\arabic{enumi}]}{\settowidth\labelwidth{[#1]}%
        \leftmargin\labelwidth
        \advance\leftmargin\labelsep
        \usecounter{enumi}}%
        \def\newblock{\hskip.11em plus.33em minus.07em}%
        \sloppy\clubpenalty4000\widowpenalty\clubpenalty
        \sfcode`\.=1000\relax}}{}

\@ifundefined{indexname}%
    {\def\theindex{\@restonecoltrue\if@twocolumn\@restonecolfalse\fi
     \columnseprule \z@
     \columnsep 35pt\twocolumn[\section*{\indexname}]%
       \@mkboth{\uppercase{\indexname}}{\uppercase{\indexname}}%
       \thispagestyle{plain}%
       \parskip\z@ plus.3pt\parindent\z@\let\item\@idxitem}}{}
%    \end{macrocode}
%
% The |abstract| environment:
%
%    \begin{macrocode}
\@ifundefined{abstractname}%
    {\def\abstract{\if@twocolumn
    \section*{\abstractname}%
    \else \small
    \begin{center}%
    {\bf \abstractname\vspace{-.5em}\vspace{\z@}}%
    \end{center}%
    \quotation
    \fi}}{}
%    \end{macrocode}
%
% And last but not least, the macro |\part|:
%
%    \begin{macrocode}
\@ifundefined{partname}%
{\def\@part[#1]#2{\ifnum \c@secnumdepth >\m@ne
        \refstepcounter{part}%
        \addcontentsline{toc}{part}{\thepart
        \hspace{1em}#1}\else
      \addcontentsline{toc}{part}{#1}\fi
   {\parindent\z@ \raggedright
    \ifnum \c@secnumdepth >\m@ne
      \Large \bf \partname{} \thepart
      \par \nobreak
    \fi
    \huge \bf
    #2\markboth{}{}\par}%
    \nobreak
    \vskip 3ex\@afterheading}%
}{}
%    \end{macrocode}
%
%    This is all that needs to be done for the \texttt{article} style.
%
%    \begin{macrocode}
\or
%    \end{macrocode}
%
%    The next case is formed by the two styles \texttt{book} and
%    \texttt{report}.  Basically we have to do the same as for the
%    \texttt{article} style, except now we must also change the
%    |\chapter| command.
%
%    The tables of contents, figures and tables:
%    \begin{macrocode}
\@ifundefined{contentsname}%
    {\def\tableofcontents{\@restonecolfalse
      \if@twocolumn\@restonecoltrue\onecolumn
      \fi\chapter*{\contentsname\@mkboth
          {\uppercase{\contentsname}}{\uppercase{\contentsname}}}%
      \@starttoc{toc}%
      \csname if@restonecol\endcsname\twocolumn
      \csname fi\endcsname}}{}

\@ifundefined{listfigurename}%
    {\def\listoffigures{\@restonecolfalse
      \if@twocolumn\@restonecoltrue\onecolumn
      \fi\chapter*{\listfigurename\@mkboth
          {\uppercase{\listfigurename}}{\uppercase{\listfigurename}}}%
      \@starttoc{lof}%
      \csname if@restonecol\endcsname\twocolumn
      \csname fi\endcsname}}{}

\@ifundefined{listtablename}%
    {\def\listoftables{\@restonecolfalse
      \if@twocolumn\@restonecoltrue\onecolumn
      \fi\chapter*{\listtablename\@mkboth
          {\uppercase{\listtablename}}{\uppercase{\listtablename}}}%
      \@starttoc{lot}%
      \csname if@restonecol\endcsname\twocolumn
      \csname fi\endcsname}}{}
%    \end{macrocode}
%
%    Again, the |bibliography| and |index| environments; notice that
%    in this case we use |\bibname| instead of |\refname| as in the
%    definitions for the \texttt{article} style.  The reason for this
%    is that in the \texttt{article} document style the term
%    `References' is used in the definition of |\thebibliography|. In
%    the \texttt{report} and \texttt{book} document styles the term
%    `Bibliography' is used.
%    \begin{macrocode}
\@ifundefined{bibname}%
    {\def\thebibliography#1{\chapter*{\bibname
     \@mkboth{\uppercase{\bibname}}{\uppercase{\bibname}}}%
     \list{[\arabic{enumi}]}{\settowidth\labelwidth{[#1]}%
     \leftmargin\labelwidth \advance\leftmargin\labelsep
     \usecounter{enumi}}%
     \def\newblock{\hskip.11em plus.33em minus.07em}%
     \sloppy\clubpenalty4000\widowpenalty\clubpenalty
     \sfcode`\.=1000\relax}}{}

\@ifundefined{indexname}%
    {\def\theindex{\@restonecoltrue\if@twocolumn\@restonecolfalse\fi
    \columnseprule \z@
    \columnsep 35pt\twocolumn[\@makeschapterhead{\indexname}]%
      \@mkboth{\uppercase{\indexname}}{\uppercase{\indexname}}%
    \thispagestyle{plain}%
    \parskip\z@ plus.3pt\parindent\z@ \let\item\@idxitem}}{}
%    \end{macrocode}
%
% Here is the |abstract| environment:
%    \begin{macrocode}
\@ifundefined{abstractname}%
    {\def\abstract{\titlepage
    \null\vfil
    \begin{center}%
    {\bf \abstractname}%
    \end{center}}}{}
%    \end{macrocode}
%
%     And last but not least the |\chapter|, |\appendix| and
%    |\part| macros.
%    \begin{macrocode}
\@ifundefined{chaptername}{\def\@chapapp{\chaptername}}{}
%
\@ifundefined{appendixname}%
    {\def\appendix{\par
      \setcounter{chapter}{0}%
      \setcounter{section}{0}%
      \def\@chapapp{\appendixname}%
      \def\thechapter{\Alph{chapter}}}}{}
%
\@ifundefined{partname}%
    {\def\@part[#1]#2{\ifnum \c@secnumdepth >-2\relax
            \refstepcounter{part}%
            \addcontentsline{toc}{part}{\thepart
            \hspace{1em}#1}\else
            \addcontentsline{toc}{part}{#1}\fi
       \markboth{}{}%
       {\centering
        \ifnum \c@secnumdepth >-2\relax
          \huge\bf \partname{} \thepart
        \par
        \vskip 20pt \fi
        \Huge \bf
        #1\par}\@endpart}}{}%
%    \end{macrocode}
%
%    \begin{macrocode}
\or
%    \end{macrocode}
%
%    Now we address the case where \file{babel.def} is read after the
%    \texttt{letter} style. The \texttt{letter} document style
%    defines the macro |\opening| and some other macros that are
%    specific to \texttt{letter}. This means that we have to redefine
%    other macros, compared to the previous two cases.
%
%    First two macros for the material at the end of a letter, the
%    |\cc| and |\encl| macros.
%    \begin{macrocode}
\@ifundefined{ccname}%
    {\def\cc#1{\par\noindent
     \parbox[t]{\textwidth}%
     {\@hangfrom{\rm \ccname : }\ignorespaces #1\strut}\par}}{}

\@ifundefined{enclname}%
    {\def\encl#1{\par\noindent
     \parbox[t]{\textwidth}%
     {\@hangfrom{\rm \enclname : }\ignorespaces #1\strut}\par}}{}
%    \end{macrocode}
%
%    The last thing we have to do here is to redefine the
%    \texttt{headings} pagestyle:
% \changes{babel~3.3}{1993/07/11}{\cs{headpagename} should be
%    \cs{pagename}}
%    \begin{macrocode}
\@ifundefined{headtoname}%
    {\def\ps@headings{%
        \def\@oddhead{\sl \headtoname{} \ignorespaces\toname \hfil
                      \@date \hfil \pagename{} \thepage}%
        \def\@oddfoot{}}}{}
%    \end{macrocode}
%
%    This was the last of the four standard document styles, so if
%    |\doc@style| has another value we do nothing and just close the
%    \texttt{if} construction.
%    \begin{macrocode}
\fi
%    \end{macrocode}
%    Here ends the code that can be optionally included when a version
%    of \LaTeX\ is in use that is dated \emph{before} December~1991.
%    \begin{macrocode}
%</names>
%</core>
%    \end{macrocode}
%
% \subsection{Cross referencing macros}
%
%    The \LaTeX\ book states:
%  \begin{quote}
%    The \emph{key} argument is any sequence of letters, digits, and
%    punctuation symbols; upper- and lowercase letters are regarded as
%    different.
%  \end{quote}
%    When the above quote should still be true when a document is
%    typeset in a language that has active characters, special care
%    has to be taken of the category codes of these characters when
%    they appear in an argument of the cross referencing macros.
%
%    When a cross referencing command processes its argument, all
%    tokens in this argument should be character tokens with category
%    `letter' or `other'.
%
%    The only way to accomplish this in most cases is to use the trick
%    described in the \TeX book~\cite{DEK} (Appendix~D, page~382).
%    The primitive |\meaning| applied to a token expands to the
%    current meaning of this token.  For example, `|\meaning\A|' with
%    |\A| defined as `|\def\A#1{\B}|' expands to the characters
%    `|macro:#1->\B|' with all category codes set to `other' or
%    `space'.
%
%  \begin{macro}{\bbl@redefine}
% \changes{babel~3.5f}{1995/11/15}{Macro added}
%    To redefine a command, we save the old meaning of the macro.
%    Then we redefine it to call the original macro with the
%    `sanitized' argument.  The reason why we do it this way is that
%    we don't want to redefine the \LaTeX\ macros completely in case
%    their definitions change (they have changed in the past).
%
%    Because we need to redefine a number of commands we define the
%    command |\bbl@redefine| which takes care of this. It creates a
%    new control sequence, |\org@...|
%    \begin{macrocode}
%<*core|shorthands>
\def\bbl@redefine#1{%
  \edef\bbl@tempa{\expandafter\@gobble\string#1}%
  \expandafter\let\csname org@\bbl@tempa\endcsname#1
  \expandafter\def\csname\bbl@tempa\endcsname}
%    \end{macrocode}
%
%    This command should only be used in the preamble of the document.
%    \begin{macrocode}
\@onlypreamble\bbl@redefine
%    \end{macrocode}
%  \end{macro}
%
%  \begin{macro}{\bbl@redefine@long}
% \changes{babel~3.6f}{1997/01/14}{Macro added}
%    This version of |\babel@redefine| can be used to redefine |\long|
%    commands such as |\ifthenelse|.
%    \begin{macrocode}
\def\bbl@redefine@long#1{%
  \edef\bbl@tempa{\expandafter\@gobble\string#1}%
  \expandafter\let\csname org@\bbl@tempa\endcsname#1
  \expandafter\long\expandafter\def\csname\bbl@tempa\endcsname}
\@onlypreamble\bbl@redefine@long
%    \end{macrocode}
%  \end{macro}
%
%  \begin{macro}{\bbl@redefinerobust}
% \changes{babel~3.5f}{1995/11/15}{Macro added}
%    For commands that are redefined, but which \textit{might} be
%    robust we need a slightly more intelligent macro. A robust
%    command |foo| is defined to expand to |\protect|\verb*|\foo |. So
%    it is necessary to check whether \verb*|\foo | exists.
%    \begin{macrocode}
\def\bbl@redefinerobust#1{%
  \edef\bbl@tempa{\expandafter\@gobble\string#1}%
  \expandafter\ifx\csname \bbl@tempa\space\endcsname\relax
    \expandafter\let\csname org@\bbl@tempa\endcsname#1
    \expandafter\edef\csname\bbl@tempa\endcsname{\noexpand\protect
      \expandafter\noexpand\csname\bbl@tempa\space\endcsname}%
  \else
    \expandafter\let\csname org@\bbl@tempa\expandafter\endcsname
                    \csname\bbl@tempa\space\endcsname
  \fi
%    \end{macrocode}
%    The result of the code above is that the command that is being
%    redefined is always robust afterwards. Therefor all we need to do
%    now is define \verb*|\foo |.
% \changes{babel~3.5f}{1996/04/09}{Define \cs*{foo } instead of
%    \cs{foo}}
%    \begin{macrocode}
  \expandafter\def\csname\bbl@tempa\space\endcsname}
%    \end{macrocode}
%
%    This command should only be used in the preamble of the document.
%    \begin{macrocode}
\@onlypreamble\bbl@redefinerobust
%    \end{macrocode}
%  \end{macro}
%
%  \begin{macro}{\newlabel}
% \changes{babel~3.5f}{1995/11/15}{Now use \cs{bbl@redefine}}
%    The macro |\label| writes a line with a |\newlabel| command
%    into the |.aux| file to define labels.
%    \begin{macrocode}
%\bbl@redefine\newlabel#1#2{%
%  \@safe@activestrue\org@newlabel{#1}{#2}\@safe@activesfalse}
%    \end{macrocode}
%  \end{macro}
%
%  \begin{macro}{\@newl@bel}
% \changes{babel~3.6i}{1997/03/01}{Now redefine \cs{@newl@bel} instead
%    of \cs{@lbibitem} and \cs{newlabel}}
%    We need to change the definition of the \LaTeX-internal macro
%    |\@newl@bel|. This is needed because we need to make sure that
%    shorthand characters expand to their non-active version.
%
%^^A The following lines commented out in preparation for a change
%^^A in the LaTeX definition of \@enwl@bel... JLB 2000/10/01
%^^A
%^^A    To play it safe when redefining a \LaTeX-internal command we
%^^A    first check whether its definition didn't change.
%^^A    \begin{macrocode}
%^^A\CheckCommand*\@newl@bel[3]{%
%^^A  \@ifundefined{#1@#2}%
%^^A    \relax
%^^A    {\gdef \@multiplelabels {%
%^^A       \@latex@warning@no@line{There were multiply-defined labels}}%
%^^A     \@latex@warning@no@line{Label `#2' multiply defined}}%
%^^A  \global\@namedef{#1@#2}{#3}}
%^^A    \end{macrocode}
%^^A    Then we give the new definition.
%    \begin{macrocode}
\def\@newl@bel#1#2#3{%
%    \end{macrocode}
%    First we open a new group to keep the changed setting of
%    |\protect| local and then we set the |@safe@actives| switch to
%    true to make sure that any shorthand that appears in any of the
%    arguments immediately expands to its non-active self.
% \changes{babel~3.7a}{1997/12/19}{Call \cs{@safe@activestrue}
%    directly}
%    \begin{macrocode}
  {%
    \@safe@activestrue
    \@ifundefined{#1@#2}%
      \relax
      {%
        \gdef \@multiplelabels {%
          \@latex@warning@no@line{There were multiply-defined labels}}%
        \@latex@warning@no@line{Label `#2' multiply defined}%
      }%
    \global\@namedef{#1@#2}{#3}%
    }%
  }
%    \end{macrocode}
%  \end{macro}
%
%  \begin{macro}{\@testdef}
%    An internal \LaTeX\ macro used to test if the labels that have
%    been written on the |.aux| file have changed.  It is called by
%    the |\enddocument| macro. This macro needs to be completely
%    rewritten, using |\meaning|. The reason for this is that in some
%    cases the expansion of |\#1@#2| contains the same characters as
%    the |#3|; but the character codes differ. Therefor \LaTeX\ keeps
%    reporting that the labels may have changed.
% \changes{babel~3.4g}{1994/08/30}{Moved the \cs{def} inside the
%    macrocode environment}
% \changes{babel~3.5f}{1995/11/15}{Now use \cs{bbl@redefine}}
% \changes{babel~3.5f}{1996/01/09}{Complete rewrite of this macro as
%    the same character ended up with different category codes in the
%    labels that are being compared. Now use \cs{meaning}}
% \changes{babel~3.5f}{1996/01/16}{Use \cs{strip@prefix} only on
%    \cs{bbl@tempa} when it is not \cs{relax}}
% \changes{babel~3.6i}{1997/02/28}{Make sure that shorthands don't get
%    expanded at the wrong moment.}
% \changes{babel~3.6i}{1997/03/01}{\cs{@safe@activesfalse} is now
%    part of the label definition}
% \changes{babel~3.7a}{1998/03/13}{Removed \cs{@safe@activesfalse}
%    from the label definition}
%    \begin{macrocode}
\CheckCommand*\@testdef[3]{%
  \def\reserved@a{#3}%
  \expandafter \ifx \csname #1@#2\endcsname \reserved@a
  \else
    \@tempswatrue
  \fi}
%    \end{macrocode}
%    Now that we made sure that |\@testdef| still has the same
%    definition we can rewrite it. First we make the shorthands
%    `safe'.
%    \begin{macrocode}
\def\@testdef #1#2#3{%
  \@safe@activestrue
%    \end{macrocode}
%    Then we use |\bbl@tempa| as an `alias' for the macro that
%    contains the label which is being checked.
%    \begin{macrocode}
  \expandafter\let\expandafter\bbl@tempa\csname #1@#2\endcsname
%    \end{macrocode}
%    Then we define |\bbl@tempb| just as |\@newl@bel| does it.
%    \begin{macrocode}
  \def\bbl@tempb{#3}%
  \@safe@activesfalse
%    \end{macrocode}
%    When the label is defined we replace the definition of
%    |\bbl@tempa| by its meaning.
%    \begin{macrocode}
  \ifx\bbl@tempa\relax
  \else
    \edef\bbl@tempa{\expandafter\strip@prefix\meaning\bbl@tempa}%
  \fi
%    \end{macrocode}
%    We do the same for |\bbl@tempb|.
%    \begin{macrocode}
  \edef\bbl@tempb{\expandafter\strip@prefix\meaning\bbl@tempb}%
%    \end{macrocode}
%    If the label didn't change, |\bbl@tempa| and |\bbl@tempb| should
%    be identical macros.
%    \begin{macrocode}
  \ifx \bbl@tempa \bbl@tempb
  \else
    \@tempswatrue
  \fi}
%    \end{macrocode}
%  \end{macro}
%
%  \begin{macro}{\ref}
%  \begin{macro}{\pageref}
%    The same holds for the macro |\ref| that references a label
%    and |\pageref| to reference a page. So we redefine |\ref| and
%    |\pageref|. While we change these macros, we make them robust as
%    well (if they weren't already) to prevent problems if they should
%    become expanded at the wrong moment.
% \changes{babel~3.5b}{1995/03/07}{Made \cs{ref} and \cs{pageref}
%    robust (PR1353)}
% \changes{babel~3.5d}{1995/07/04}{use a different control sequence
%    while making \cs{ref} and \cs{pageref} robust}
% \changes{babel~3.5f}{1995/11/06}{redefine \cs*{ref } if it exists
%    instead of \cs{ref}}
% \changes{babel~3.5f}{1995/11/15}{Now use \cs{bbl@redefinerobust}}
% \changes{babel~3.5f}{1996/01/19}{redefine \cs{\@setref} instead of
%    \cs{ref} and \cs{pageref} in \LaTeXe.}
% \changes{babel~3.5f}{1996/01/21}{Reverse the previous change as it
%    inhibits the use of active characters in labels}
%    \begin{macrocode}
\bbl@redefinerobust\ref#1{%
  \@safe@activestrue\org@ref{#1}\@safe@activesfalse}
\bbl@redefinerobust\pageref#1{%
  \@safe@activestrue\org@pageref{#1}\@safe@activesfalse}
%    \end{macrocode}
%  \end{macro}
%  \end{macro}
%
%  \begin{macro}{\@citex}
% \changes{babel~3.5f}{1995/11/15}{Now use \cs{bbl@redefine}}
%    The macro used to cite from a bibliography, |\cite|, uses an
%    internal macro, |\@citex|.
%    It is this internal macro that picks up the argument(s),
%    so we redefine this internal macro and leave |\cite| alone. The
%    first argument is used for typesetting, so the shorthands need
%    only be deactivated in the second argument.
% \changes{babel~3.7g}{2000/10/01}{The shorthands need to be
%    deactivated for the second argument of \cs{@citex} only.}
%    \begin{macrocode}
\bbl@redefine\@citex[#1]#2{%
  \@safe@activestrue\edef\@tempa{#2}\@safe@activesfalse
  \org@@citex[#1]{\@tempa}}
%    \end{macrocode}
%    Unfortunately, the packages \pkg{natbib} and \pkg{cite} need a
%    different definition of |\@citex|...
%    To begin with, \pkg{natbib} has a definition for |\@citex| with
%    \emph{three} arguments... We only know that a package is loaded
%    when |\begin{document}| is executed, so we need to postpone the
%    different redefinition.
%    \begin{macrocode}
\AtBeginDocument{%
  \@ifpackageloaded{natbib}{%
%    \end{macrocode}
%    Notice that we use |\def| here instead of |\bbl@redefine| because
%    |\org@@citex| is already defined and we don't want to overwrite
%    that definition (it would result in parameter stack overflow
%    because of a circular definition).
%    \begin{macrocode}
    \def\@citex[#1][#2]#3{%
      \@safe@activestrue\edef\@tempa{#3}\@safe@activesfalse
      \org@@citex[#1][#2]{\@tempa}}%
  }{}}
%    \end{macrocode}
%    The package \pkg{cite} has a definition of |\@citex| where the
%    shorthands need to be turned off in both arguments.
%    \begin{macrocode}
\AtBeginDocument{%
  \@ifpackageloaded{cite}{%
    \def\@citex[#1]#2{%
      \@safe@activestrue\org@@citex[#1]{#2}\@safe@activesfalse}%
    }{}}
%    \end{macrocode}
%  \end{macro}
%
%  \begin{macro}{\nocite}
% \changes{babel~3.5f}{1995/11/15}{Now use \cs{bbl@redefine}}
%    The macro |\nocite| which is used to instruct BiB\TeX\ to
%    extract uncited references from the database.
%    \begin{macrocode}
\bbl@redefine\nocite#1{%
  \@safe@activestrue\org@nocite{#1}\@safe@activesfalse}
%    \end{macrocode}
%  \end{macro}
%
%  \begin{macro}{\bibcite}
% \changes{babel~3.5f}{1995/11/15}{Now use \cs{bbl@redefine}}
%    The macro that is used in the |.aux| file to define citation
%    labels. When packages such as \pkg{natbib} or \pkg{cite} are not
%    loaded its second argument is used to typeset the citation
%    label. In that case, this second argument can contain active
%    characters but is used in an environment where
%    |\@safe@activestrue| is in effect. This switch needs to be reset
%    inside the |\hbox| which contains the citation label. In order to
%    determine during \file{.aux} file processing which definition of
%    |\bibcite| is needed we define |\bibcite| in such a way that it
%    redefines itself with the proper definition.
% \changes{babel~3.6s}{1999/04/13}{Need to determine `online' which
%    definition of \cs{bibcite} is needed}
% \changes{babel~3.6v}{1999/04/21}{Also check for \pkg{cite} it can't
%    handle \cs{@safe@activesfalse} in its second argument}
%    \begin{macrocode}
\bbl@redefine\bibcite{%
%    \end{macrocode}
%    We call |\bbl@cite@choice| to select the proper definition for
%    |\bibcite|. This new definition is then activated.
%    \begin{macrocode}
  \bbl@cite@choice
  \bibcite}
%    \end{macrocode}
%  \end{macro}
%
%  \begin{macro}{\bbl@bibcite}
% \changes{babel~3.6v}{1999/04/21}{Macro \cs{bbl@bibcite} added}
%    The macro |\bbl@bibcite| holds the definition of |\bibcite|
%    needed when neither \pkg{natbib} nor \pkg{cite} is loaded.
%    \begin{macrocode}
\def\bbl@bibcite#1#2{%
  \org@bibcite{#1}{\@safe@activesfalse#2}}
%    \end{macrocode}
%  \end{macro}
%
%  \begin{macro}{\bbl@cite@choice}
% \changes{babel~3.6v}{1999/04/21}{Macro \cs{bbl@cite@choice} added}
%    The macro |\bbl@cite@choice| determines which definition of
%    |\bibcite| is needed.
%    \begin{macrocode}
\def\bbl@cite@choice{%
%    \end{macrocode}
%    First we give |\bibcite| its default definition.
%    \begin{macrocode}
  \global\let\bibcite\bbl@bibcite
%    \end{macrocode}
%    Then, when \pkg{natbib} is loaded we restore the original
%    definition of |\bibcite| .
%    \begin{macrocode}
  \@ifpackageloaded{natbib}{\global\let\bibcite\org@bibcite}{}%
%    \end{macrocode}
%    For \pkg{cite} we do the same.
%    \begin{macrocode}
  \@ifpackageloaded{cite}{\global\let\bibcite\org@bibcite}{}%
%    \end{macrocode}
%    Make sure this only happens once.
%    \begin{macrocode}
  \global\let\bbl@cite@choice\relax
  }
%    \end{macrocode}
%
%    When a document is run for the first time, no \file{.aux} file is
%    available, and |\bibcite| will not yet be properly defined. In
%    this case, this has to happen before the document starts.
%    \begin{macrocode}
\AtBeginDocument{\bbl@cite@choice}
%    \end{macrocode}
%  \end{macro}
%
%  \begin{macro}{\@bibitem}
% \changes{babel~3.5f}{1995/11/15}{Now use \cs{bbl@redefine}}
%    One of the two internal \LaTeX\ macros called by |\bibitem|
%    that write the citation label on the |.aux| file.
%    \begin{macrocode}
\bbl@redefine\@bibitem#1{%
  \@safe@activestrue\org@@bibitem{#1}\@safe@activesfalse}
%    \end{macrocode}
%  \end{macro}
%
%  \subsection{marks}
%
%  \begin{macro}{\markright}
% \changes{babel~3.6i}{1997/03/15}{Added redefinition of \cs{mark...}
%    commands}
%    Because the output routine is asynchronous, we must
%    pass the current language attribute to the head lines, together
%    with the text that is put into them. To achieve this we need to
%    adapt the definition of |\markright| and |\markboth| somewhat.
% \changes{babel~3.7c}{1999/04/08}{Removed the use of \cs{head@lang}
%    (PR 2990)}
% \changes{babel~3.7c}{1999/04/09}{Avoid expanding the arguments by
%    storing them in token registers}
% \changes{babel~3.7m}{2003/11/15}{added \cs{bbl@restore@actives} to
%    the mark}
% \changes{babel~3.8c}{2004/05/26}{No need to add \emph{anything} to
%    an empty mark; prevented this by checking the contents of the
%    argument}
% \changes{babel~3.8f}{2005/05/15}{Make the definition independent of
%    the original definition; expand \cs{languagename} before passing
%    it into the token registers} 
%    \begin{macrocode}
\bbl@redefine\markright#1{%
%    \end{macrocode}
%    First of all we temporarily store the language switching command,
%    using an expanded definition in order to get the current value of
%    |\languagename|. 
%    \begin{macrocode}
  \edef\bbl@tempb{\noexpand\protect
    \noexpand\foreignlanguage{\languagename}}%
%    \end{macrocode}
%    Then, we check whether the argument is empty; if it is, we
%    just make sure the scratch token register is empty.
%    \begin{macrocode}
  \def\bbl@arg{#1}%
  \ifx\bbl@arg\@empty
    \toks@{}%
  \else
%    \end{macrocode}
%    Next, we store the argument to |\markright| in the scratch token
%    register, together with the expansion of |\bbl@tempb| (containing
%    the language switching command) as defined before. This way
%    these commands will not be expanded by using |\edef| later
%    on, and we make sure that the text is typeset using the
%    correct language settings. While doing so, we make sure that
%    active characters that may end up in the mark are not disabled by
%    the output routine kicking in while \cs{@safe@activestrue} is in
%    effect.
%    \begin{macrocode}
    \expandafter\toks@\expandafter{%
             \bbl@tempb{\protect\bbl@restore@actives#1}}%
  \fi
%    \end{macrocode}
%    Then we define a temporary control sequence using |\edef|.
%    \begin{macrocode}
  \edef\bbl@tempa{%
%    \end{macrocode}
%     When |\bbl@tempa| is executed, only |\languagename| will be
%    expanded, because of the way the token register was filled.
%    \begin{macrocode}
    \noexpand\org@markright{\the\toks@}}%
  \bbl@tempa
}
%    \end{macrocode}
%  \end{macro}
%
%  \begin{macro}{\markboth}
%  \begin{macro}{\@mkboth}
%    The definition of |\markboth| is equivalent to that of
%    |\markright|, except that we need two token registers. The
%    documentclasses \cls{report} and \cls{book} define and set the
%    headings for the page. While doing so they also store a copy of
%    |\markboth| in |\@mkboth|. Therefor we need to check whether
%    |\@mkboth| has already been set. If so we neeed to do that again
%    with the new definition of |\makrboth|.
% \changes{babel~3.7m}{2003/11/15}{added \cs{bbl@restore@actives} to
%    the mark}
% \changes{babel~3.8c}{2004/05/26}{No need to add \emph{anything} to
%    an empty mark, prevented this by checking the contents of the
%    arguments} 
% \changes{babel~3.8f}{2005/05/15}{Make the definition independent of
%    the original definition; expand \cs{languagename} before passing
%    it into the token registers} 
% \changes{babel~3.8j}{2008/03/21}{Added setting of \cs{@mkboth} (PR
%    3826)} 
%    \begin{macrocode}
\ifx\@mkboth\markboth
  \def\bbl@tempc{\let\@mkboth\markboth}
\else
  \def\bbl@tempc{}
\fi
%    \end{macrocode}
%    Now we can start the new definition of |\markboth|
%    \begin{macrocode}
\bbl@redefine\markboth#1#2{%
  \edef\bbl@tempb{\noexpand\protect
    \noexpand\foreignlanguage{\languagename}}%
  \def\bbl@arg{#1}%
  \ifx\bbl@arg\@empty
    \toks@{}%
  \else
   \expandafter\toks@\expandafter{%
             \bbl@tempb{\protect\bbl@restore@actives#1}}%
  \fi
  \def\bbl@arg{#2}%
  \ifx\bbl@arg\@empty
    \toks8{}%
  \else
    \expandafter\toks8\expandafter{%
             \bbl@tempb{\protect\bbl@restore@actives#2}}%
  \fi
  \edef\bbl@tempa{%
    \noexpand\org@markboth{\the\toks@}{\the\toks8}}%
  \bbl@tempa
}
%    \end{macrocode}
%    and copy it to |\@mkboth| if necesary.
%    \begin{macrocode}
\bbl@tempc
%</core|shorthands>
%    \end{macrocode}
%  \end{macro}
%
%  \subsection{Encoding issues (part 2)}
%
% \changes{babel~3.7c}{1999/04/16}{Removed redefinition of \cs{@roman}
%    and \cs{@Roman}}
%
%  \begin{macro}{\TeX}
%  \begin{macro}{\LaTeX}
% \changes{babel~3.7a}{1998/03/12}{Make \TeX\ and \LaTeX\ logos
%    encoding-independent}
%    Because documents may use font encodings other than one of the
%    latin encodings, we make sure that the logos of \TeX\ and
%    \LaTeX\ always come out in the right encoding.
%    \begin{macrocode}
%<*core>
\bbl@redefine\TeX{\textlatin{\org@TeX}}
\bbl@redefine\LaTeX{\textlatin{\org@LaTeX}}
%</core>
%    \end{macrocode}
%  \end{macro}
%  \end{macro}
%
%  \subsection{Preventing clashes with other packages}
%
%  \subsubsection{\pkg{ifthen}}
%
%  \begin{macro}{\ifthenelse}
% \changes{babel~3.5g}{1996/08/11}{Redefinition of \cs{ifthenelse}
%    added to circumvent problems with \cs{pageref} in the argument of
%    \cs{isodd}}
%    Sometimes a document writer wants to create a special effect
%    depending on the page a certain fragment of text appears on. This
%    can be achieved by the following piece of code:
% \begin{verbatim}
%    \ifthenelse{\isodd{\pageref{some:label}}}
%               {code for odd pages}
%               {code for even pages}
% \end{verbatim}
%    In order for this to work the argument of |\isodd| needs to be
%    fully expandable. With the above redefinition of |\pageref| it is
%    not in the case of this example. To overcome that, we add some
%    code to the definition of |\ifthenelse| to make things work.
%
%    The first thing we need to do is check if the package
%    \pkg{ifthen} is loaded. This should be done at |\begin{document}|
%    time.
%    \begin{macrocode}
%<*package>
\AtBeginDocument{%
  \@ifpackageloaded{ifthen}{%
%    \end{macrocode}
%    Then we can redefine |\ifthenelse|:
% \changes{babel~3.6f}{1997/01/14}{\cs{ifthenelse} needs to be long}
%    \begin{macrocode}
    \bbl@redefine@long\ifthenelse#1#2#3{%
%    \end{macrocode}
%    We want to revert the definition of |\pageref| to its original
%    definition for the duration of |\ifthenelse|, so we first need to
%    store its current meaning.
%    \begin{macrocode}
      \let\bbl@tempa\pageref
      \let\pageref\org@pageref
%    \end{macrocode}
%    Then we can set the |\@safe@actives| switch and call the original
%    |\ifthenelse|. In order to be able to use shorthands in the
%    second and third arguments of |\ifthenelse| the resetting of the
%    switch \emph{and} the definition of |\pageref| happens inside
%    those arguments. 
% \changes{babel~3.6i}{1997/02/25}{Now reset the @safe@actives switch
%    inside the 2nd and 3rd arguments of \cs{ifthenelse}}
% \changes{babel~3.7f}{2000/06/29}{\cs{pageref} needs to have its
%    babel definition reinstated in the second and third arguments}
%    \begin{macrocode}
      \@safe@activestrue
      \org@ifthenelse{#1}{%
        \let\pageref\bbl@tempa
        \@safe@activesfalse
        #2}{%
        \let\pageref\bbl@tempa
        \@safe@activesfalse
        #3}%
      }%
%    \end{macrocode}
%    When the package wasn't loaded we do nothing.
%    \begin{macrocode}
    }{}%
  }
%    \end{macrocode}
%  \end{macro}
%
%  \subsubsection{\pkg{varioref}}
%
%  \begin{macro}{\@@vpageref}
% \changes{babel~3.6a}{1996/10/29}{Redefinition of \cs{@@vpageref}
%    added to circumvent problems with active \texttt{:} in the
%    argument of \cs{vref} when \pkg{varioref} is used}
%  \begin{macro}{\vrefpagenum}
% \changes{babel~3.7o}{2003/11/18}{Added redefinition of
%    \cs{vrefpagenum} which deals with ranges of pages}
%  \begin{macro}{\Ref}
% \changes{babel~3.8g}{2005/05/21}{We also need to adapt \cs{Ref}
%    which needs to be able to uppercase the first letter of the
%    expansion of \cs{ref}} 
%    When the package varioref is in use we need to modify its
%    internal command |\@@vpageref| in order to prevent problems when
%    an active character ends up in the argument of |\vref|.
%    \begin{macrocode}
\AtBeginDocument{%
  \@ifpackageloaded{varioref}{%
    \bbl@redefine\@@vpageref#1[#2]#3{%
      \@safe@activestrue
      \org@@@vpageref{#1}[#2]{#3}%
      \@safe@activesfalse}%
%    \end{macrocode}
%    The same needs to happen for |\vrefpagenum|.
%    \begin{macrocode}
    \bbl@redefine\vrefpagenum#1#2{%
      \@safe@activestrue
      \org@vrefpagenum{#1}{#2}%
      \@safe@activesfalse}%
%    \end{macrocode}
%    The package \pkg{varioref} defines |\Ref| to be a robust command
%    wich uppercases the first character of the reference text. In
%    order to be able to do that it needs to access the exandable form
%    of |\ref|. So we employ a little trick here. We redefine the
%    (internal) command \verb*|\Ref | to call |\org@ref| instead of
%    |\ref|. The disadvantgage of this solution is that whenever the
%    derfinition of |\Ref| changes, this definition needs to be updated
%    as well.
%    \begin{macrocode}
    \expandafter\def\csname Ref \endcsname#1{%
      \protected@edef\@tempa{\org@ref{#1}}\expandafter\MakeUppercase\@tempa}
    }{}%
  }
%    \end{macrocode}
%  \end{macro}
%  \end{macro}
%  \end{macro}
%
%  \subsubsection{\pkg{hhline}}
%
%  \begin{macro}{\hhline}
%    Delaying the activation of the shorthand characters has introduced
%    a problem with the \pkg{hhline} package. The reason is that it
%    uses the `:' character which is made active by the french support
%    in \babel. Therefor we need to \emph{reload} the package when
%    the `:' is an active character.
%
%    So at |\begin{document}| we check whether \pkg{hhline} is loaded.
%    \begin{macrocode}
\AtBeginDocument{%
  \@ifpackageloaded{hhline}%
%    \end{macrocode}
%    Then we check whether the expansion of |\normal@char:| is not
%    equal to |\relax|.
% \changes{babel~3.8b}{2004/04/19}{added \cs{string} to prevent
%    unwanted expansion of the colon}
%    \begin{macrocode}
    {\expandafter\ifx\csname normal@char\string:\endcsname\relax
     \else
%    \end{macrocode}
%    In that case we simply reload the package. Note that this happens
%    \emph{after} the category code of the @-sign has been changed to
%    other, so we need to temporarily change it to letter again.
%    \begin{macrocode}
       \makeatletter
       \def\@currname{hhline}% \iffalse meta-comment
%
% Copyright 1993-2014
%
% The LaTeX3 Project and any individual authors listed elsewhere
% in this file.
%
% This file is part of the Standard LaTeX `Tools Bundle'.
% -------------------------------------------------------
%
% It may be distributed and/or modified under the
% conditions of the LaTeX Project Public License, either version 1.3c
% of this license or (at your option) any later version.
% The latest version of this license is in
%    http://www.latex-project.org/lppl.txt
% and version 1.3c or later is part of all distributions of LaTeX
% version 2005/12/01 or later.
%
% The list of all files belonging to the LaTeX `Tools Bundle' is
% given in the file `manifest.txt'.
%
% \fi
% \iffalse
%% File: hhline.dtx Copyright (C) 1991-1994 David Carlisle
%
%<package>\NeedsTeXFormat{LaTeX2e}
%<package>\ProvidesPackage{hhline}
%<package>         [2014/10/28 v2.03 Table rule package (DPC)]
%
%<*driver>
\documentclass{ltxdoc}
\usepackage{hhline}
\GetFileInfo{hhline.sty}
\begin{document}
\title{The \textsf{hhline} package\thanks{This file
        has version number \fileversion, last
        revised \filedate.}}
\author{David Carlisle\\carlisle@cs.man.ac.uk}
\date{\filedate}
 \maketitle
 \DeleteShortVerb{\|}
 \DocInput{hhline.dtx}
\end{document}
%</driver>
% \fi
%
%
% \changes{v1.00}{1991/06/04}{Initial Version}
% \changes{v2.00}{1991/11/06}
%     {Add tilde which allows \cmd\cline-like constructions.}
% \changes{v2.01}{1992/06/26}
%    {Re-issue for the new  doc and docstrip.}
% \changes{v2.02}{1994/03/14}
%    {Update for LaTeX2e.}
% \changes{v2.03}{1994/05/23}
%    {New style warning.}
%
%
% \CheckSum{244}
%
% \MakeShortVerb{\"}
%
% \begin{abstract}
% "\hhline" produces a line like "\hline", or a double line like
% "\hline\hline", except for its interaction with vertical lines.
% \end{abstract}
%
% \arrayrulewidth=1pt
% \doublerulesep=3pt
%
% \section{Introduction}
% The argument to "\hhline" is similar to the preamble of an {\tt
% array} or {\tt tabular}. It consists of a list of tokens with the
% following meanings:
% \[
% \begin{tabular}{cl}
%   "="   & A double hline the width of a column.\\
%   "-"   & A single hline the width of a column.\\[10pt]
%   "~"   & A column with no hline.\\[10pt]
%
%   "|"   & A vline which `cuts' through a double (or single) hline.\\
%   ":"   & A vline which is broken by a double hline.\\[10pt]
%
%   "#"   & A double hline segment between two vlines.\\
%   "t"   & The top half of a double hline segment.\\
%   "b"   & The bottom half of a double hline segment.\\
%
%   "*"   & "*{3}{==#}" expands to "==#==#==#",
%                   as in the {\tt*}-form for the preamble.
% \end{tabular}
% \]
% If a double vline is specified ("||" or "::") then the hlines
% produced by "\hhline" are broken. To obtain the effect of an hline
% `cutting through' the double vline, use a "#" or omit the vline
% specifiers, depending on whether or not you wish the double vline to
% break.
%
% The tokens {\tt t} and {\tt b} must be used between two vertical
% rules. "|tb|" produces the same lines  as "#", but is much less
% efficient. The main use for these are to make constructions like
% "|t:" (top left corner) and ":b|" (bottom right corner).
%
% If "\hhline" is used to make a single hline, then the argument
% should only contain the tokens "-", "~"  and "|" (and
% {\tt*}-expressions).
%
% An example using most of these features is:
% \[
% \vcenter{\hsize=2in\begin{verbatim}
% \begin{tabular}{||cc||c|c||}
% \hhline{|t:==:t:==:t|}
% a&b&c&d\\
% \hhline{|:==:|~|~||}
% 1&2&3&4\\
% \hhline{#==#~|=#}
% i&j&k&l\\
% \hhline{||--||--||}
% w&x&y&z\\
% \hhline{|b:==:b:==:b|}
% \end{tabular}
% \end{verbatim}
% }
% \qquad
% \begin{tabular}{||cc||c|c||}
% \hhline{|t:==:t:==:t|}
% a&b&c&d\\
% \hhline{|:==:|~|~||}
% 1&2&3&4\\
% \hhline{#==#~|=#}
% i&j&k&l\\
% \hhline{||--||--||}
% w&x&y&z\\
% \hhline{|b:==:b:==:b|}
% \end{tabular}
% \]
%
% The lines produced by \LaTeX's "\hline" consist of a single (\TeX\
% primitive) "\hrule". The lines produced by "\hhline" are made
% up of lots of small line segments. \TeX\ will place these very
% accurately in the {\tt .dvi} file, but the program that you use to
% print the {\tt .dvi} file may not line up these segments exactly. (A
% similar problem can occur with diagonal lines in the {\tt picture}
% environment.)
%
% If this effect causes a problem, you could try a different driver
% program, or if this is not possible, increasing "\arrayrulewidth"
% may help to reduce the effect.
%
% \StopEventually{}
%
% \section{The Macros}
%
%    \begin{macrocode}
%<*package>
%    \end{macrocode}
%
% \begin{macro}{\HH@box}
% Makes a box containing a double hline segment. The most common case,
% both rules of length "\doublerulesep" will be stored in "\box1", this
% is not initialised until "\hhline" is called as the user may change
% the parameters "\doublerulesep" and "\arrayrulewidth". The two
% arguments to "\HH@box" are the widths (ie lengths) of the top and
% bottom rules.
%    \begin{macrocode}
\def\HH@box#1#2{\vbox{%
  \hrule \@height \arrayrulewidth \@width #1
  \vskip \doublerulesep
  \hrule \@height \arrayrulewidth \@width #2}}
%    \end{macrocode}
% \end{macro}
%
% \begin{macro}{\HH@add}
% Build up the preamble in the register "\toks@".
%    \begin{macrocode}
\def\HH@add#1{\toks@\expandafter{\the\toks@#1}}
%    \end{macrocode}
% \end{macro}

% \begin{macro}{\HH@xexpast}
% \begin{macro}{\HH@xexnoop}
% We `borrow' the version of "\@xexpast" from Mittelbach's array.sty,
% as this allows "#" to appear in the argument list.
%    \begin{macrocode}
\def\HH@xexpast#1*#2#3#4\@@{%
   \@tempcnta #2
   \toks@={#1}\@temptokena={#3}%
   \let\the@toksz\relax \let\the@toks\relax
   \def\@tempa{\the@toksz}%
   \ifnum\@tempcnta >0 \@whilenum\@tempcnta >0\do
     {\edef\@tempa{\@tempa\the@toks}\advance \@tempcnta \m@ne}%
       \let \@tempb \HH@xexpast \else
       \let \@tempb \HH@xexnoop \fi
   \def\the@toksz{\the\toks@}\def\the@toks{\the\@temptokena}%
   \edef\@tempa{\@tempa}%
   \expandafter \@tempb \@tempa #4\@@}

\def\HH@xexnoop#1\@@{}
%    \end{macrocode}
% \end{macro}
% \end{macro}
%
% \begin{macro}{\hhline}
% Use a simplified version of "\@mkpream" to break apart the argument
% to "\hhline". Actually it is oversimplified, It assumes that the
% vertical rules are at the end of the column. If you were to specify
% "c|@{xx}|" in the array argument, then "\hhline" would not be
% able to access the first vertical rule. (It ought to have an "@"
% option, and add "\leaders" up to the width of a box containing the
% "@"-expression. We use a loop made with "\futurelet" rather
% than "\@tfor" so that we can use "#" to denote the crossing of
% a double hline with a double vline.\\
% "\if@firstamp" is true in the first column and false otherwise.\\
% "\if@tempswa"  is true if the previous entry was a vline
%                     (":", "|" or "#").
%    \begin{macrocode}
\def\hhline#1{\omit\@firstamptrue\@tempswafalse
%    \end{macrocode}
% Put two rules of width "\doublerulesep" in "\box1"
%    \begin{macrocode}
\global\setbox\@ne\HH@box\doublerulesep\doublerulesep
%    \end{macrocode}
% If Mittelbach's {\tt array.sty} is loaded, we do not need the negative
% "\hskip"'s around vertical rules.
%    \begin{macrocode}
  \xdef\@tempc{\ifx\extrarowheight\HH@undef\hskip-.5\arrayrulewidth\fi}%
%    \end{macrocode}
% Now expand the {\tt*}-forms and add dummy tokens ( "\relax" and
% "`" ) to either end of the token list. Call "\HH@let" to start
% processing the token list.
%    \begin{macrocode}
    \HH@xexpast\relax#1*0x\@@\toks@{}\expandafter\HH@let\@tempa`}
%    \end{macrocode}
% \end{macro}

% \begin{macro}{\HH@let}
% Discard the last token, look at the next one.
%    \begin{macrocode}
\def\HH@let#1{\futurelet\@tempb\HH@loop}
%    \end{macrocode}
% \end{macro}

% \begin{macro}{\HH@loop}
% The main loop. Note we use "\ifx" rather than "\if" in
% version~2 as the new token "~" is active.
%    \begin{macrocode}
\def\HH@loop{%
%    \end{macrocode}
% If next token is "`", stop the loop and put the lines into this row
% of the alignment.
%    \begin{macrocode}
  \ifx\@tempb`\def\next##1{\the\toks@\cr}\else\let\next\HH@let
%    \end{macrocode}
% "|", add a vertical rule (across either a double or
% single hline).
%    \begin{macrocode}
  \ifx\@tempb|\if@tempswa\HH@add{\hskip\doublerulesep}\fi\@tempswatrue
          \HH@add{\@tempc\vline\@tempc}\else
%    \end{macrocode}
% ":", add a broken vertical rule (across a double hline).
%    \begin{macrocode}
  \ifx\@tempb:\if@tempswa\HH@add{\hskip\doublerulesep}\fi\@tempswatrue
      \HH@add{\@tempc\HH@box\arrayrulewidth\arrayrulewidth\@tempc}\else
%    \end{macrocode}
% "#", add a double hline segment between two vlines.
%    \begin{macrocode}
  \ifx\@tempb##\if@tempswa\HH@add{\hskip\doublerulesep}\fi\@tempswatrue
         \HH@add{\@tempc\vline\@tempc\copy\@ne\@tempc\vline\@tempc}\else
%    \end{macrocode}
% "~", A column with no hline (this gives an effect similar to
% \verb+\cline+).
%    \begin{macrocode}
  \ifx\@tempb~\@tempswafalse
           \if@firstamp\@firstampfalse\else\HH@add{&\omit}\fi
              \HH@add{\hfil}\else
%    \end{macrocode}
% "-", add a single hline across the column.
%    \begin{macrocode}
  \ifx\@tempb-\@tempswafalse
           \if@firstamp\@firstampfalse\else\HH@add{&\omit}\fi
              \HH@add{\leaders\hrule\@height\arrayrulewidth\hfil}\else
%    \end{macrocode}
% "=", add a double hline across the column.
%    \begin{macrocode}
  \ifx\@tempb=\@tempswafalse
       \if@firstamp\@firstampfalse\else\HH@add{&\omit}\fi
%    \end{macrocode}
%     Put in as many copies of "\box1" as possible with
%     "\leaders", this may leave gaps at the ends, so put an extra box
%     at each end, overlapping the "\leaders".
%    \begin{macrocode}
       \HH@add
          {\rlap{\copy\@ne}\leaders\copy\@ne\hfil\llap{\copy\@ne}}\else
%    \end{macrocode}
% "t", add the top half of a double hline segment, in a "\rlap"
% so that it may be used with {\tt b}.
%    \begin{macrocode}
  \ifx\@tempb t\HH@add{\rlap{\HH@box\doublerulesep\z@}}\else
%    \end{macrocode}
% "b", add the bottom half of a double hline segment in a "\rlap"
% so that it may be used with {\tt t}.
%    \begin{macrocode}
  \ifx\@tempb b\HH@add{\rlap{\HH@box\z@\doublerulesep}}\else
%    \end{macrocode}
% Otherwise ignore the token, with a warning.
%    \begin{macrocode}
  \PackageWarning{hhline}%
      {\meaning\@tempb\space ignored in \noexpand\hhline argument%
       \MessageBreak}%
  \fi\fi\fi\fi\fi\fi\fi\fi\fi
%    \end{macrocode}
% Go around the loop again.
%    \begin{macrocode}
  \next}
%    \end{macrocode}
% \end{macro}
%
%    \begin{macrocode}
%</package>
%    \end{macrocode}
%
% \Finale
\endinput
\makeatother
     \fi}%
    {}}
%    \end{macrocode}
%  \end{macro}
%
%  \subsubsection{\pkg{hyperref}}
%
%  \begin{macro}{\pdfstringdefDisableCommands}
% \changes{babel~3.8j}{2008/03/16}{Inform \pkg{hyperref} to use
%    shorthands at system level (PR4006)}
%    Although a number of interworking problems between \pkg{babel}
%    and \pkg{hyperref} are tackled by \pkg{hyperref} itself we need
%    to take care of correctly handling the shorthand characters.
%    When they get expanded inside a bookmark a warning will appear in
%    the log file which can be prevented. This is done by informing
%    \pkg{hyperref} that it should the shorthands as defined on the
%    system level rather than at the user level.
%    
%    \begin{macrocode}
\AtBeginDocument{%
  \@ifundefined{pdfstringdefDisableCommands}%
    {}%
    {\pdfstringdefDisableCommands{%
       \languageshorthands{system}}%
    }%
}
%    \end{macrocode}
%  \end{macro}
%
%
%  \subsubsection{General}
%
%  \begin{macro}{\FOREIGNLANGUAGE}
%    The package \pkg{fancyhdr} treats the running head and fout lines
%    somewhat differently as the standard classes. A symptom of this is
%    that the command |\foreignlanguage| which \babel\ adds to the
%    marks can end up inside the argument of |\MakeUppercase|. To
%    prevent unexpected results we need to define |\FOREIGNLANGUAGE|
%    here.
% \changes{babel~3.7j}{2003/05/23}{Define \cs{FOREIGNLANGUAGE}
%    unconditionally}
%    \begin{macrocode}
\DeclareRobustCommand{\FOREIGNLANGUAGE}[1]{%
  \lowercase{\foreignlanguage{#1}}}
%</package>
%    \end{macrocode}
%  \end{macro}
%
%  \begin{macro}{\nfss@catcodes}
% \changes{babel~3.5g}{1996/08/18}{Need to add the double quote and
%    acute characters to \cs{nfss@catcodes} to prevent problems when
%    reading in .fd files}
%    \LaTeX's font selection scheme sometimes wants to read font
%    definition files in the middle of processing the document. In
%    order to guard against any characters having the wrong
%    |\catcode|s it always calls |\nfss@catcodes| before loading a
%    file. Unfortunately, the characters |"| and |'| are not dealt
%    with. Therefor we have to add them until \LaTeX\ does that
%    herself.
%    \begin{macrocode}
%<*core|shorthands>
\ifx\nfss@catcodes\@undefined
\else
  \addto\nfss@catcodes{%
    \@makeother\'%
    \@makeother\"%
    }
\fi
%    \end{macrocode}
%  \end{macro}
%
%    \begin{macrocode}
%</core|shorthands>
%    \end{macrocode}
%
% \section{Local Language Configuration}
%
%  \begin{macro}{\loadlocalcfg}
%    At some sites it may be necessary to add site-specific actions to
%    a language definition file. This can be done by creating a file
%    with the same name as the language definition file, but with the
%    extension \file{.cfg}. For instance the file \file{norsk.cfg}
%    will be loaded when the language definition file \file{norsk.ldf}
%    is loaded.
%
% \changes{babel~3.5d}{1995/06/22}{Added macro}
%    \begin{macrocode}
%<*core>
%    \end{macrocode}
%    For plain-based formats we don't want to override the definition
%    of |\loadlocalcfg| from \file{plain.def}.
%    \begin{macrocode}
\ifx\loadlocalcfg\@undefined
  \def\loadlocalcfg#1{%
    \InputIfFileExists{#1.cfg}
           {\typeout{*************************************^^J%
                     * Local config file #1.cfg used^^J%
                     *}%
            }
           {}}
\fi
%    \end{macrocode}
%    Just to be compatible with \LaTeX$\:$2.09 we add a few more lines
%    of code:
%    \begin{macrocode}
\ifx\@unexpandable@protect\@undefined
  \def\@unexpandable@protect{\noexpand\protect\noexpand}
  \long\def \protected@write#1#2#3{%
        \begingroup
         \let\thepage\relax
         #2%
         \let\protect\@unexpandable@protect
         \edef\reserved@a{\write#1{#3}}%
         \reserved@a
        \endgroup
        \if@nobreak\ifvmode\nobreak\fi\fi
  }
\fi
%</core>
%    \end{macrocode}
%  \end{macro}
%
%
% \clearpage
% \section{Driver files for the documented source code}
%
%    Since \babel\ version 3.4 all source files that are part of the
%    \babel\ system can be typeset separately. But to typeset
%    them all in one document, the file \file{babel.drv} can be used.
%    If you only want the information on how to use the \babel\ system
%    and what goodies are provided by the language-specific files, you
%    can run the file \file{user.drv} through \LaTeX\ to get a user
%    guide.
%
% \changes{babel~3.4b}{1994/05/18}{Use the ltxdoc class instead of
%    article}
% \changes{babel~3.7a}{1997/05/21}{Now need packages t1enc and
%    supertabular to be loaded; the documentation for icelandic needs
%    its \file{.ldf} file to be present}
% \changes{babel~3.8a}{2004/02/20}{Also load package url}
%    \begin{macrocode}
%<*driver>
\documentclass{ltxdoc}
\usepackage{url,t1enc,supertabular}
\usepackage[icelandic,english]{babel}
\DoNotIndex{\!,\',\,,\.,\-,\:,\;,\?,\/,\^,\`,\@M}
\DoNotIndex{\@,\@ne,\@m,\@afterheading,\@date,\@endpart}
\DoNotIndex{\@hangfrom,\@idxitem,\@makeschapterhead,\@mkboth}
\DoNotIndex{\@oddfoot,\@oddhead,\@restonecolfalse,\@restonecoltrue}
\DoNotIndex{\@starttoc,\@unused}
\DoNotIndex{\accent,\active}
\DoNotIndex{\addcontentsline,\advance,\Alph,\arabic}
\DoNotIndex{\baselineskip,\begin,\begingroup,\bf,\box,\c@secnumdepth}
\DoNotIndex{\catcode,\centering,\char,\chardef,\clubpenalty}
\DoNotIndex{\columnsep,\columnseprule,\crcr,\csname}
\DoNotIndex{\day,\def,\dimen,\discretionary,\divide,\dp,\do}
\DoNotIndex{\edef,\else,\@empty,\end,\endgroup,\endcsname,\endinput}
\DoNotIndex{\errhelp,\errmessage,\expandafter,\fi,\filedate}
\DoNotIndex{\fileversion,\fmtname,\fnum@figure,\fnum@table,\fontdimen}
\DoNotIndex{\gdef,\global}
\DoNotIndex{\hbox,\hidewidth,\hfil,\hskip,\hspace,\ht,\Huge,\huge}
\DoNotIndex{\ialign,\if@twocolumn,\ifcase,\ifcat,\ifhmode,\ifmmode}
\DoNotIndex{\ifnum,\ifx,\immediate,\ignorespaces,\input,\item}
\DoNotIndex{\kern}
\DoNotIndex{\labelsep,\Large,\large,\labelwidth,\lccode,\leftmargin}
\DoNotIndex{\lineskip,\leavevmode,\let,\list,\ll,\long,\lower}
\DoNotIndex{\m@ne,\mathchar,\mathaccent,\markboth,\month,\multiply}
\DoNotIndex{\newblock,\newbox,\newcount,\newdimen,\newif,\newwrite}
\DoNotIndex{\nobreak,\noexpand,\noindent,\null,\number}
\DoNotIndex{\onecolumn,\or}
\DoNotIndex{\p@,par, \parbox,\parindent,\parskip,\penalty}
\DoNotIndex{\protect,\ps@headings}
\DoNotIndex{\quotation}
\DoNotIndex{\raggedright,\raise,\refstepcounter,\relax,\rm,\setbox}
\DoNotIndex{\section,\setcounter,\settowidth,\scriptscriptstyle}
\DoNotIndex{\sfcode,\sl,\sloppy,\small,\space,\spacefactor,\strut}
\DoNotIndex{\string}
\DoNotIndex{\textwidth,\the,\thechapter,\thefigure,\thepage,\thepart}
\DoNotIndex{\thetable,\thispagestyle,\titlepage,\tracingmacros}
\DoNotIndex{\tw@,\twocolumn,\typeout,\uppercase,\usecounter}
\DoNotIndex{\vbox,\vfil,\vskip,\vspace,\vss}
\DoNotIndex{\widowpenalty,\write,\xdef,\year,\z@,\z@skip}
%    \end{macrocode}
%
%     Here |\dlqq| is defined so that  an example of |"'| can be
%     given.
%    \begin{macrocode}
\makeatletter
\gdef\dlqq{{\setbox\tw@=\hbox{,}\setbox\z@=\hbox{''}%
  \dimen\z@=\ht\z@ \advance\dimen\z@-\ht\tw@
  \setbox\z@=\hbox{\lower\dimen\z@\box\z@}\ht\z@=\ht\tw@
  \dp\z@=\dp\tw@ \box\z@\kern-.04em}}
%    \end{macrocode}
%
%    The code lines are numbered within sections,
%    \begin{macrocode}
%<*!user>
\@addtoreset{CodelineNo}{section}
\renewcommand\theCodelineNo{%
  \reset@font\scriptsize\thesection.\arabic{CodelineNo}}
%    \end{macrocode}
%    which should also be visible in the index; hence this
%    redefinition of a macro from \file{doc.sty}.
%    \begin{macrocode}
\renewcommand\codeline@wrindex[1]{\if@filesw
        \immediate\write\@indexfile
            {\string\indexentry{#1}%
            {\number\c@section.\number\c@CodelineNo}}\fi}
%    \end{macrocode}
%
%    The glossary environment is used or the change log, but its
%    definition needs changing for this document.
%    \begin{macrocode}
\renewenvironment{theglossary}{%
    \glossary@prologue%
    \GlossaryParms \let\item\@idxitem \ignorespaces}%
   {}
%</!user>
\makeatother
%    \end{macrocode}
%
%    A few shorthands used in the documentation
% \changes{babel~3.5g}{1996/07/06}{Added definition of \cs{Babel}}
%    \begin{macrocode}
\font\manual=logo10 % font used for the METAFONT logo, etc.
\newcommand*\MF{{\manual META}\-{\manual FONT}}
\newcommand*\TeXhax{\TeX hax}
\newcommand*\babel{\textsf{babel}}
\newcommand*\Babel{\textsf{Babel}}
\newcommand*\m[1]{\mbox{$\langle$\it#1\/$\rangle$}}
\newcommand*\langvar{\m{lang}}
%    \end{macrocode}
%
%     Some more definitions needed in the documentation.
%    \begin{macrocode}
%\newcommand*\note[1]{\textbf{#1}}
\newcommand*\note[1]{}
\newcommand*\bsl{\protect\bslash}
\newcommand*\Lopt[1]{\textsf{#1}}
\newcommand*\Lenv[1]{\textsf{#1}}
\newcommand*\file[1]{\texttt{#1}}
\newcommand*\cls[1]{\texttt{#1}}
\newcommand*\pkg[1]{\texttt{#1}}
\newcommand*\langdeffile[1]{%
%<-user>  \clearpage
  \DocInput{#1}}
%    \end{macrocode}
%
%    When a full index should be generated uncomment the line with
%    |\EnableCrossrefs|. Beware, processing may take some time.
%    Use |\DisableCrossrefs| when the index is ready.
%    \begin{macrocode}
%  \EnableCrossrefs
\DisableCrossrefs
%    \end{macrocode}
%
%    Inlude the change log.
%    \begin{macrocode}
%<-user>\RecordChanges
%    \end{macrocode}
%    The index should use the linenumbers of the code.
%    \begin{macrocode}
%<-user>\CodelineIndex
%    \end{macrocode}
%
% Set everything in |\MacroFont| instead of |\AltMacroFont|
%    \begin{macrocode}
\setcounter{StandardModuleDepth}{1}
%    \end{macrocode}
%
%    For the user guide we only want the description parts of all the
%    files.
%    \begin{macrocode}
%<+user>\OnlyDescription
%    \end{macrocode}
%    Here starts the document
%    \begin{macrocode}
\begin{document}
\DocInput{babel.dtx}
%    \end{macrocode}
%
%    All the language definition files.
% \changes{babel~3.2e}{1992/07/07}{Added slovak}
% \changes{babel~3.3}{1993/07/11}{Added catalan and galician}
% \changes{babel~3.3}{1993/07/11}{Added turkish}
% \changes{babel~3.4}{1994/02/28}{Added bahasa}
% \changes{babel~3.5a}{1995/02/16}{Added breton, irish, scottish}
% \changes{babel~3.5b}{1995/05/19}{Added lsorbian, usorbian}
% \changes{babel~3.5c}{1995/06/14}{Changed the order of including the
%    language files somewhat (PR1652)}
% \changes{babel~3.5g}{1996/07/06}{Added greek}
% \changes{babel~3.6a}{1996/12/14}{Added welsh}
%^^A \changes{babel~3.6i}{1997/02/07}{Added sanskrit}
% \changes{babel~3.6i}{1997/02/22}{Added basque}
^^A% \changes{babel~3.6i}{1997/02/22}{Added kannada}
% \changes{babel~3.7a}{1997/05/21}{Added icelandic}
% \changes{babel~3.7b}{1998/06/25}{Added Latin}
% \changes{babel~3.7c}{1999/03/09}{Added ukrainian}
% \changes{babel~3.7c}{1999/05/09}{Added hebrew and serbian}
% \changes{babel~3.7e}{1999/11/22}{Added missing hebrew files}
% \changes{babel~3.7f}{2000/09/21}{Added bulgarian}
% \changes{babel~3.7f}{2000/09/26}{Added samin}
% \changes{babel~3.8a}{2004/02/20}{Added interlingua}
% \changes{babel~3.8h}{2005/11/23}{Added albanian and bahasam}
%    \begin{macrocode}
%<+user>\clearpage
\langdeffile{esperanto.dtx}
\langdeffile{interlingua.dtx}
%
\langdeffile{dutch.dtx}
\langdeffile{english.dtx}
\langdeffile{germanb.dtx}
\langdeffile{ngermanb.dtx}
%
\langdeffile{breton.dtx}
\langdeffile{welsh.dtx}
\langdeffile{irish.dtx}
\langdeffile{scottish.dtx}
%
\langdeffile{greek.dtx}
%
\langdeffile{frenchb.dtx}
\langdeffile{italian.dtx}
\langdeffile{latin.dtx}
\langdeffile{portuges.dtx}
\langdeffile{spanish.dtx}
\langdeffile{catalan.dtx}
\langdeffile{galician.dtx}
\langdeffile{basque.dtx}
\langdeffile{romanian.dtx}
%
\langdeffile{danish.dtx}
\langdeffile{icelandic.dtx}
\langdeffile{norsk.dtx}
\langdeffile{swedish.dtx}
\langdeffile{samin.dtx}
%
\langdeffile{finnish.dtx}
\langdeffile{magyar.dtx}
\langdeffile{estonian.dtx}
%
\langdeffile{albanian.dtx}
\langdeffile{croatian.dtx}
\langdeffile{czech.dtx}
\langdeffile{polish.dtx}
\langdeffile{serbian.dtx}
\langdeffile{slovak.dtx}
\langdeffile{slovene.dtx}
\langdeffile{russianb.dtx}
\langdeffile{bulgarian.dtx}
\langdeffile{ukraineb.dtx}
%
\langdeffile{lsorbian.dtx}
\langdeffile{usorbian.dtx}
\langdeffile{turkish.dtx}
%
\langdeffile{hebrew.dtx}
\DocInput{hebinp.dtx}
\DocInput{hebrew.fdd}
\DocInput{heb209.dtx}
\langdeffile{bahasa.dtx}
\langdeffile{bahasam.dtx}
%\langdeffile{sanskrit.dtx}
%\langdeffile{kannada.dtx}
%\langdeffile{nagari.dtx}
%\langdeffile{tamil.dtx}
\clearpage
\DocInput{bbplain.dtx}
%    \end{macrocode}
%    Finally print the index and change log (not for the user guide).
%    \begin{macrocode}
%<*!user>
\clearpage
\def\filename{index}
\PrintIndex
\clearpage
\def\filename{changes}
\PrintChanges
%</!user>
\end{document}
%</driver>
%    \end{macrocode}
%
% \Finale
%
%%
%% \CharacterTable
%%  {Upper-case    \A\B\C\D\E\F\G\H\I\J\K\L\M\N\O\P\Q\R\S\T\U\V\W\X\Y\Z
%%   Lower-case    \a\b\c\d\e\f\g\h\i\j\k\l\m\n\o\p\q\r\s\t\u\v\w\x\y\z
%%   Digits        \0\1\2\3\4\5\6\7\8\9
%%   Exclamation   \!     Double quote  \"     Hash (number) \#
%%   Dollar        \$     Percent       \%     Ampersand     \&
%%   Acute accent  \'     Left paren    \(     Right paren   \)
%%   Asterisk      \*     Plus          \+     Comma         \,
%%   Minus         \-     Point         \.     Solidus       \/
%%   Colon         \:     Semicolon     \;     Less than     \<
%%   Equals        \=     Greater than  \>     Question mark \?
%%   Commercial at \@     Left bracket  \[     Backslash     \\
%%   Right bracket \]     Circumflex    \^     Underscore    \_
%%   Grave accent  \`     Left brace    \{     Vertical bar  \|
%%   Right brace   \}     Tilde         \~}
\endinput


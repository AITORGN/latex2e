% \iffalse meta-comment
%
% Copyright 1989-2008 Johannes L. Braams and any individual authors
% listed elsewhere in this file.  All rights reserved.
% 
% This file is part of the Babel system.
% --------------------------------------
% 
% It may be distributed and/or modified under the
% conditions of the LaTeX Project Public License, either version 1.3
% of this license or (at your option) any later version.
% The latest version of this license is in
%   http://www.latex-project.org/lppl.txt
% and version 1.3 or later is part of all distributions of LaTeX
% version 2003/12/01 or later.
% 
% This work has the LPPL maintenance status "maintained".
% 
% The Current Maintainer of this work is Johannes Braams.
% 
% The list of all files belonging to the Babel system is
% given in the file `manifest.bbl. See also `legal.bbl' for additional
% information.
% 
% The list of derived (unpacked) files belonging to the distribution
% and covered by LPPL is defined by the unpacking scripts (with
% extension .ins) which are part of the distribution.
% \fi
% \CheckSum{4211}
%%
% \def\filename{babel.dtx}
% \let\thisfilename\filename
%
%\iffalse
% \changes{babel~3.5g}{1996/10/10}{We need at least \LaTeX\ from
%    December 1994}
% \changes{babel~3.6k}{1999/03/18}{We need at least \LaTeX\ from
%    June 1998}
%    \begin{macrocode}
%<package>\NeedsTeXFormat{LaTeX2e}[2005/12/01]
%    \end{macrocode}
%
%% File 'babel.dtx'
%\fi
%%\ProvidesFile{babel.dtx}[2012/06/16 v3.9a-alpha-2 The Babel package]
%\iffalse
%
% Babel DOCUMENT-STYLE option for LaTeX version 2.09 or plain TeX;
%% Babel package for LaTeX2e.
%
%% Copyright (C) 1989 -- 2008 by Johannes Braams,
%%                            TeXniek
%%                            all rights reserved.
%% Copyright (C) 2012         by Johannes Braams
%%                            TeXniek
%%                            by Javier Bezos
%%                            all rights reserved.
%
%% Please report errors to: J.L. Braams
%%                          babel at braams.xs4all.nl
%<*filedriver>
\documentclass{ltxdoc}
\usepackage{supertabular}
\font\manual=logo10 % font used for the METAFONT logo, etc.
\newcommand*\MF{{\manual META}\-{\manual FONT}}
\newcommand*\TeXhax{\TeX hax}
\newcommand*\babel{\textsf{babel}}
\newcommand*\Babel{\textsf{Babel}}
\newcommand*\m[1]{\mbox{$\langle$\it#1\/$\rangle$}}
\newcommand*\langvar{\m{lang}}
\newcommand*\note[1]{}
\newcommand*\bsl{\protect\bslash}
\newcommand*\Lopt[1]{\textsf{#1}}
\newcommand*\Lenv[1]{\textsf{#1}}
\newcommand*\file[1]{\texttt{#1}}
\newcommand*\cls[1]{\texttt{#1}}
\newcommand*\pkg[1]{\texttt{#1}}
\begin{document}
 \DocInput{babel.dtx}
\end{document}
%</filedriver>
% \changes{babel~3.7a}{1997/04/16}{Make multiple loading of
%    \file{babel.def} impossible} 
% \changes{babel~3.9a}{2012/05/16}{Now using ldf@quit for the test} 
%    \begin{macrocode}
%<*core>
\ifx\ldf@quit\@undefined
\else
  \expandafter\endinput
\fi
%</core>
%    \end{macrocode}
%<*dtx>
\ProvidesFile{babel.dtx}
%</dtx>
%\fi
%
% \GetFileInfo{babel.dtx}
%
% \changes{babel~2.0a}{1990/04/02}{Added text about \file{german.sty}}
% \changes{babel~2.0b}{1990/04/18}{Changed order of code to prevent
%    plain \TeX from seeing all of it}
% \changes{babel~2.1}{1990/04/24}{Modified user interface,
%    \cs{langTeX} no longer necessary}
% \changes{babel~2.1a}{1990/05/01}{Incorporated Nico's comments}
% \changes{babel~2.1b}{1990/05/01}{rename \cs{language} to
%    \cs{current@language}}
% \changes{babel~2.1c}{1990/05/22}{abstract for report fixed, missing
%    \texttt{\}}, found by Nicolas Brouard}
% \changes{babel~2.1d}{1990/07/04}{Missing right brace in definition of
%    abstract environment, found by Werenfried Spit}
% \changes{babel~2.1e}{1990/07/16}{Incorporated more comments from
%    Nico}
% \changes{babel~2.2}{1990/07/17}{Renamed \cs{newlanguage} to
%    \cs{addlanguage}}
% \changes{babel~2.2a}{1990/08/27}{Modified the documentation
%    somewhat}
% \changes{babel~3.0}{1991/04/23}{Moved part of the code to hyphen.doc
%    in preparation for \TeX~3.0}
% \changes{babel~3.0a}{1991/05/21}{Updated comments in various places}
% \changes{babel~3.0b}{1991/05/25}{Removed some problems in change log}
% \changes{babel~3.0c}{1991/07/15}{Renamed \file{babel.sty} and
%    \file{latexhax.sty} to \file{.com}}
% \changes{babel~3.1}{1991/10/31}{Added the support for active
%    characters and for extending a macro}
% \changes{babel~3.1}{1991/11/05}{Removed the need for
%    \file{latexhax}}
% \changes{babel~3.2}{1991/11/10}{Some Changes by br}
% \changes{babel~3.2a}{1992/02/15}{Fixups of the code and
%    documentation}
% \changes{babel~3.3}{1993/07/06}{Included driver file, and prepared
%    for distribution}
% \changes{babel~3.4}{1994/01/30}{Updated for \LaTeXe}
% \changes{babel~3.4}{1994/02/28}{Added language definition file for
%    bahasa}
% \changes{babel~3.4b}{1994/05/18}{Added a small driver to be able to
%    process just this file}
% \changes{babel~3.5a}{1995/02/03}{Provided common code to handle the
%    active double quote}
% \changes{babel~3.5c}{1995/06/14}{corrected a few typos (PR1652)}
% \changes{babel~3.5d}{1995/07/02}{Merged glyphs.dtx into this file}
% \changes{babel~3.5f}{1995/07/13}{repaired a typo}
% \changes{babel~3.5f}{1996/01/09}{replaced \cs{tmp}, \cs{bbl@tmp} and
%    \cs{bbl@temp} with \cs{bbl@tempa}}
% \changes{babel~3.5g}{1996/07/09}{replaced \cs{undefined} with
%    \cs{@undefined} to be consistent with \LaTeX}
% \changes{babel~3.7d}{1999/05/05}{Fixed a few typos in \cs{changes}
%    entries which made typesetting the code impossible}
% \changes{babel~3.7h}{2001/03/01}{Added a number of missing comment
%    characters which caused spurious white space}
% \changes{babel~3.8e}{2005/03/24}{Many enhancements to the text by
%    Andrew Young} 
%
% \title {Babel, a multilingual package for use with \LaTeX's standard
%    document classes\thanks{During the development ideas from Nico
%    Poppelier, Piet van Oostrum and many others have been used.
%    Bernd Raichle has provided many helpful suggestions.}}
%
% \author{Johannes Braams\\
%         Kersengaarde 33\\
%         2723 BP Zoetermeer\\
%         The Netherlands\\
%         \texttt{babel\char64 braams.xs4all.nl}\\
%         \normalsize For version 3.9, Javier Bezos}
%
% \date{Printed \today}
%
% \maketitle
%
%  \begin{abstract}
%    The standard distribution of \LaTeX\ contains a number of
%    document classes that are meant to be used, but also serve as
%    examples for other users to create their own document classes.
%    These document classes have become very popular among \LaTeX\
%    users. But it should be kept in mind that they were designed for
%    American tastes and typography. At one time they contained a
%    number of hard-wired texts. This report describes \babel{}, a
%    package that makes use of the new capabilities of \TeX\ version 3
%    to provide an environment in which documents can be typeset in
%    a language other than US English, or in more than one language.
%  \end{abstract}
%
%  \tableofcontents
%
% \section{The user interface}\label{U-I}
%
%    The user interface of this package is quite simple. It consists
%    of a set of commands that switch from one language to another, and
%    a set of commands that deal with shorthands. It is also possible
%    to find out what the current language is.
%
%  \DescribeMacro{\selectlanguage}
%    When a user wants to switch from one language to another he can
%    do so using the macro |\selectlanguage|. This macro takes the
%    language, defined previously by a language definition file, as
%    its argument. It calls several macros that should be defined in
%    the language definition files to activate the special definitions
%    for the language chosen.
%
%  \DescribeEnv{otherlanguage}
%    The environment \Lenv{otherlanguage} does basically the same as
%    |\selectlanguage|, except the language change is local to the
%    environment. This environment is required for intermixing
%    left-to-right typesetting with right-to-left typesetting.
%    The language to switch to is specified as an
%    argument to |\begin{otherlanguage}|.
%
%  \DescribeMacro{\foreignlanguage}
%    The command |\foreignlanguage| takes two arguments; the second
%    argument is a phrase to be typeset according to the rules of the
%    language named in its first argument. This command only switches
%    the extra definitions and the hyphenation rules for the language,
%    \emph{not} the names and dates.
%
%  \DescribeEnv{otherlanguage*}
%    In the environment \Lenv{otherlanguage*} only the typesetting
%    is done according to the rules of the other language, but the
%    text-strings such as `figure', `table', etc. are left as they
%    were set outside this environment.
%
%  \DescribeEnv{hyphenrules}
%    The environment \Lenv{hyphenrules} can be used to select
%    \emph{only} the hyphenation rules to be used. This can for
%    instance be used to select `nohyphenation', provided that in
%    \file{language.dat} the `language' nohyphenation is defined by
%    loading \file{serohyph.tex}.
%
%  \DescribeMacro{\languagename}
%    The control sequence |\languagename| contains the name of the
%    current language.
%
%  \DescribeMacro{\iflanguage} If more than one language is used, it
%    might be necessary to know which language is active at a specific
%    time. This can be checked by a call to |\iflanguage|. This macro
%    takes three arguments.  The first argument is the name of a
%    language; the second and third arguments are the actions to take
%    if the result of the test is \texttt{true} or \texttt{false}
%    respectively. Here ``language'' is used in the \TeX\ sense, as a
%    set of hyphenation patterns, and not as its \textsf{babel} name
%    (for the latter, use \textsf{iflang}, by Heiko Oberdiek, or check
%    the value of |\languagename| with the help of \textsf{ifthen}).
%
%  \DescribeMacro{\useshorthands}
%    The command |\useshorthands| initiates the definition of
%    user-defined shorthand sequences. It has one argument, the
%    character that starts these personal shorthands.
%
%  \DescribeMacro{\defineshorthand}
%     The command |\defineshorthand| takes two arguments: the first
%     is a one- or two-character shorthand sequence, and the second is
%     the code the shorthand should expand to.
%
%  \DescribeMacro{\aliasshorthand}
%    The command |\aliasshorthand| can be used to let another
%    character perform the same functions as the default shorthand
%    character. If one prefers for example to use the character |/|
%    over |"| in typing polish texts, this can be achieved by entering
%    |\aliasshorthand{"}{/}|. \emph{Please note} that the substitute
%    shorthand character must have been declared in the preamble of
%    your document, using a command such as |\useshorthands{/}| in this
%    example.
%
%  \DescribeMacro{\languageshorthands}
%     The command |\languageshorthands| can be used to switch the
%     shorthands on the language level. It takes one argument, the
%     name of a language. Note that for this to work the language
%     should have been specified as an option when loading the \babel\
%     package.
%
%  \DescribeMacro{\shorthandon}
%  \DescribeMacro{\shorthandoff}
%    It is sometimes necessary to switch a shorthand
%    character off temporarily, because it must be used in an
%    entirely different way. For this purpose, the user commands
%    |\shorthandoff| and |\shorthandon| are provided. They each take a
%    list of characters as their arguments. The command |\shorthandoff|
%    sets the |\catcode| for each of the characters in its argument to
%    other (12); the command |\shorthandon| sets the |\catcode| to
%    active (13). Both commands only work on `known'
%    shorthand characters. If a character is not known to be a
%    shorthand character its category code will be left unchanged.
%
%  \DescribeMacro{\languageattribute}
%    This is a user-level command, to be used in the preamble of a
%    document (after |\usepackage[...]{babel}|), that declares which
%    attributes are to be used for a given language. It takes two
%    arguments: the first is the name of the language; the second,
%    a (list of) attribute(s) to used.
%    The command checks whether the language is known in this document
%    and whether the attribute(s) are known for this language.
%
% \subsection{Languages supported by \Babel}
%
%    In the following table all the languages supported by \Babel\ are
%    listed, together with the names of the options with which you can
%    load \babel\ for each language.
%
%    \begin{center}
%      \tablehead{Language & Option(s)\\\hline}
%      \tabletail{\hline}
%      \begin{supertabular}{l p{8cm}}
%        Afrikaans  & afrikaans\\
%        Bahasa     & bahasa, indonesian, indon, bahasai,
%                     bahasam, malay, meyalu\\
%        Basque     & basque\\
%        Breton     & breton\\
%        Bulgarian  & bulgarian\\
%        Catalan    & catalan\\
%        Croatian   & croatian\\
%        Czech      & czech\\
%        Danish     & danish\\
%        Dutch      & dutch\\
%        English    & english, USenglish, american, UKenglish,
%                     british, canadian, australian, newzealand\\
%        Esperanto  & esperanto\\
%        Estonian   & estonian\\
%        Finnish    & finnish\\
%        French     & french, francais, canadien, acadian\\
%        Galician   & galician\\
%        German     & austrian, german, germanb, ngerman, naustrian\\
%        Greek      & greek, polutonikogreek \\
%        Hebrew     & hebrew \\
%        Hungarian  & magyar, hungarian\\
%        Icelandic  & icelandic \\
%        Interlingua & interlingua \\
%        Irish Gaelic & irish\\
%        Italian    & italian\\
%^^A        Kannada    & kannada \\
%        Latin      & latin \\
%        Lower Sorbian & lowersorbian\\
%^^A        Devnagari  & nagari \\
%        North Sami & samin \\
%        Norwegian  & norsk, nynorsk\\
%        Polish     & polish\\
%        Portuguese & portuges, portuguese, brazilian, brazil\\
%        Romanian   & romanian\\
%        Russian    & russian\\
%^^A        Sanskrit   & sanskrit\\
%        Scottish Gaelic & scottish\\
%        Spanish    & spanish\\
%        Slovakian  & slovak\\
%        Slovenian  & slovene\\
%        Swedish    & swedish\\
%        Serbian    & serbian\\
%^^A        Tamil      & tamil \\
%        Turkish    & turkish\\
%        Ukrainian  & ukrainian\\
%        Upper Sorbian & uppersorbian\\
%        Welsh      & welsh\\
%      \end{supertabular}
%    \end{center}
%
%    For some languages \babel\ supports the options
%    \Lopt{activeacute} and \Lopt{activegrave}; for typestting Russian
%    texts, \babel\ knows about the options \Lopt{LWN} and \Lopt{LCY}
%    to specify the fontencoding of the cyrillic font used. Currently
%    only \Lopt{LWN} is supported.
%
% \subsection{Workarounds}
%
%    If you use the document class \cls{book} \emph{and} you use
%    |\ref| inside the argument of |\chapter|,
%    \LaTeX\ will keep complaining about an undefined
%    label. The reason is that the argument of |\ref| is passed through
%    |\uppercase| at some time during processing. To prevent such
%    problems, you could revert to using uppercase labels, or you can
%    use |\lowercase{\ref{foo}}| inside the argument of |\chapter|.
%
% \section{Changes for \LaTeXe}
%
%    With the advent of \LaTeXe\ the interface to \babel\ in the
%    preamble of the document has changed. With \LaTeX2.09 one used to
%    call up the \babel\ system with a line such as:
%
%\begin{verbatim}
%\documentstyle[dutch,english]{article}
%\end{verbatim}
%
%    which would tell \LaTeX\ that the document would be written in
%    two languages, Dutch and English, and that English would be the
%    first language in use.
%
%    The \LaTeXe\ way of providing the same information is:
%
%\begin{verbatim}
%\documentclass{article}
%\usepackage[dutch,english]{babel}
%\end{verbatim}
%
%    or, making \Lopt{dutch} and \Lopt{english} global options in
%    order to let other packages detect and use them:
%
%\begin{verbatim}
%\documentclass[dutch,english]{article}
%\usepackage{babel}
%\usepackage{varioref}
%\end{verbatim}
%
%    In this last example, the package \texttt{varioref} will also see
%    the options and will be able to use them.
%
% \section{Changes in \Babel\ version 3.7}
%
%    In \Babel\ version 3.7 a number of bugs that were found in
%    version~3.6 are fixed. Also a number of changes and additions
%    have occurred:
%    \begin{itemize}
%    \item Shorthands are expandable again. The disadvantage is that
%      one has to type |'{}a| when the acute accent is used as a
%      shorthand character. The advantage is that a number of other
%      problems (such as the breaking of ligatures, etc.) have
%      vanished.
%    \item Two new commands, |\shorthandon| and |\shorthandoff| have
%      been introduced to enable to temporarily switch off one or more
%      shorthands.
%^^A    \item Support for typesetting Sanskrit in transliteration is now
%^^A      available, thanks to Jun Takashima.
%^^A    \item Support for typesetting Kannada, Devnagari and Tamil is now
%^^A      available thanks to Jun Takashima.
%    \item Support for typesetting Greek has been enhanced. Code from
%      the \pkg{kdgreek} package (suggested by the author) was added
%      and |\greeknumeral| has been added.
%    \item Support for typesetting Basque is now available thanks to
%      Juan Aguirregabiria.
%    \item Support for typesetting Serbian with Latin script is now
%      available thanks to Dejan Muhamedagi\'{c} and Jankovic
%      Slobodan.
%    \item Support for typesetting Hebrew (and potential support for
%      typesetting other right-to-left written languages) is now
%      available thanks to Rama Porrat and Boris Lavva.
%    \item Support for typesetting Bulgarian is now available thanks to
%      Georgi Boshnakov.
%    \item Support for typesetting Latin is now available, thanks to
%      Claudio Beccari and Krzysztof Konrad \.Zelechowski.
%    \item Support for typesetting North Sami is now available, thanks
%      to Regnor Jernsletten.
%    \item The options \Lopt{canadian}, \Lopt{canadien} and
%      \Lopt{acadien} have been added for Canadian English and French
%      use.
%    \item A language attribute has been added to the |\mark...|
%      commands in order to make sure that a Greek header line comes
%      out right on the last page before a language switch.
%    \item Hyphenation pattern files are now read \emph{inside a
%      group}; therefore any changes a pattern file needs to make to
%      lowercase codes, uppercase codes, and category codes are kept
%      local to that group. If they are needed for the language, these
%      changes will need to be repeated and stored in |\extras...|
%    \item The concept of language attributes is introduced. It is
%      intended to give the user some control over the
%      features a language-definition file provides. Its
%      first use is for the Greek language, where the user can choose
%      the  $\pi o\lambda\upsilon\tau o\nu\kappa\acute{o}$
%      (``Polutoniko'' or multi-accented) Greek way of typesetting
%      texts. These attributes will possibly find wider use in future
%      releases.
%    \item The environment \Lenv{hyphenrules} is introduced.
%    \item The syntax of the file \file{language.dat} has been
%      extended to allow (optionally) specifying the font
%      encoding to be used while processing the patterns file.
%    \item The command |\providehyphenmins| should now be used in
%      language definition files in order to be able to keep any
%      settings provided by the pattern file.
%    \end{itemize}
%
% \section{Changes in \Babel\ version 3.6}
%
%    In \Babel\ version 3.6 a number of bugs that were found in
%    version~3.5 are fixed. Also a number of changes and additions
%    have occurred:
%    \begin{itemize}
%    \item A new environment \Lenv{otherlanguage*} is introduced. it
%      only switches the `specials', but leaves the `captions'
%      untouched.
%    \item The shorthands are no longer fully expandable. Some
%      problems could only be solved by peeking at the token following
%      an active character. The advantage is that |'{}a| works as
%      expected for languages that have the |'| active.
%    \item Support for typesetting french texts is much enhanced; the
%      file \file{francais.ldf} is now replaced by \file{frenchb.ldf}
%      which is maintained by Daniel Flipo.
%    \item Support for typesetting the russian language is again
%      available. The language definition file was originally
%      developed by Olga Lapko from CyrTUG. The fonts needed to
%      typeset the russian language are now part of the \babel\
%      distribution. The support is not yet up to the level which is
%      needed according to Olga, but this is a start.
%    \item Support for typesetting greek texts is now also
%      available. What is offered in this release is a first attempt;
%      it will be enhanced later on by Yannis Haralambous.
%    \item in \babel\ 3.6j some hooks have been added for the
%      development of support for Hebrew typesetting.
%    \item Support for typesetting texts in Afrikaans (a variant of
%      Dutch, spoken in South Africa) has been added to
%      \file{dutch.ldf}.
%    \item Support for typesetting Welsh texts is now available.
%    \item A new command |\aliasshorthand| is introduced. It seems
%      that in Poland various conventions are used to type the
%      necessary Polish letters. It is now possible to use the
%      character~|/| as a shorthand character instead of the
%      character~|"|, by issuing the command |\aliasshorthand{"}{/}|.
%    \item The shorthand mechanism now deals correctly with characters
%      that are already active.
%    \item Shorthand characters are made active at |\begin{document}|,
%      not earlier. This is to prevent problems with other packages.
%    \item A \emph{preambleonly} command |\substitutefontfamily| has
%      been added to create \file{.fd} files on the fly when the font
%      families of the Latin text differ from the families used for
%      the Cyrillic or Greek parts of the text.
%    \item Three new commands |\LdfInit|, |\ldf@quit| and
%      |\ldf@finish| are introduced that perform a number of standard
%      tasks.
%    \item In babel 3.6k the language Ukrainian has been added and the
%      support for Russian typesetting has been adapted to the package
%      'cyrillic' to be released with the December 1998 release of
%      \LaTeXe.
%    \end{itemize}
%
% \section{Changes in \Babel\ version 3.5}
%
%    In \Babel\ version 3.5 a lot of changes have been made when
%    compared with the previous release. Here is a list of the most
%    important ones:
%    \begin{itemize}
%    \item the selection of the language is delayed until
%      |\begin{document}|, which means you must
%      add appropriate |\selectlanguage| commands if you include
%      |\hyphenation| lists in the preamble of your document.
%    \item \babel\ now has a \Lenv{language} environment and a new
%      command |\foreignlanguage|;
%    \item the way active characters are dealt with is completely
%      changed. They are called `shorthands'; one can have three
%      levels of shorthands: on the user level, the language level,
%      and on `system level'. A consequence of the new way of handling
%      active characters is that they are now written to auxiliary
%      files `verbatim';
%    \item A language change now also writes information in the
%      \file{.aux} file, as the change might also affect typesetting
%      the table of contents. The consequence is that an .aux file
%      generated by a LaTeX format with babel preloaded gives errors
%      when read with a LaTeX format without babel; but I think this
%      probably doesn't occur;
%    \item \babel\ is now compatible with the \pkg{inputenc} and
%      \pkg{fontenc} packages;
%    \item the language definition files now have a new extension,
%      \file{ldf};
%    \item the syntax of the file \file{language.dat} is extended to
%      be compatible with the \pkg{french} package by Bernard Gaulle;
%    \item each language definition file looks for a configuration
%      file which has the same name, but the extension \file{.cfg}. It
%    can contain any valid \LaTeX\ code.
%    \end{itemize}
%
% \section{The interface between the core of \babel{} and the language
%    definition files}
%
%    In the core of the \babel{} system, several macros are defined
%    for use in language definition files. Their purpose
%    is to make a new language known.
%
%  \DescribeMacro{\addlanguage}
%    The macro |\addlanguage| is a non-outer version of the macro
%    |\newlanguage|, defined in \file{plain.tex} version~3.x. For
%    older versions of \file{plain.tex} and \file{lplain.tex} a
%    substitute definition is used.
%
%  \DescribeMacro{\adddialect}
%    The macro |\adddialect| can be used when two
%    languages can (or must) use the same hyphenation
%    patterns. This can also be useful for
%    languages for which no patterns are preloaded in the format. In
%    such cases the default behaviour of the \babel{} system is to
%    define this language as a `dialect' of the language for which the
%    patterns were loaded as |\language0|.
%
%    The language definition files must conform to a number of
%    conventions, because these files have to fill
%    in the gaps left by the common code in \file{babel.def}, i.\,e.,
%    the definitions of the macros that produce texts.  Also the
%    language-switching possibility which has been built into the
%    \babel{} system has its implications.
%
%    The following assumptions are made:
%   \begin{itemize}
%    \item Some of the language-specific definitions might be used by
%    plain \TeX\ users, so the files have to be coded so that they
%    can be read by both \LaTeX\ and plain \TeX. The current
%    format can be checked by looking at the value of the macro
%    |\fmtname|.
%
%    \item The common part of the \babel{} system redefines a number
%    of macros and environments (defined previously in the document
%    style) to put in the names of macros that replace the previously
%    hard-wired texts.  These macros have to be defined in the
%    language definition files.
%
%    \item The language definition files define five macros, used to
%    activate and deactivate the language-specific definitions.  These
%    macros are |\|\langvar|hyphenmins|, |\captions|\langvar,
%    |\date|\langvar, |\extras|\langvar\ and |\noextras|\langvar; where
%    \langvar\ is either the name of the language definition file or
%    the name of the \LaTeX\ option that is to be used. These
%    macros and their functions are discussed below.
%
%    \item When a language definition file is loaded, it can define
%    |\l@|\langvar\ to be a dialect of |\language0| when
%    |\l@|\langvar\ is undefined.
%
%    \item The language definition files can be read in the preamble of
%    the document, but also in the middle of document processing. This
%    means that they have to function independently of the current
%    |\catcode| of the \texttt{@}~sign.
%   \end{itemize}
%
%  \DescribeMacro{\providehyphenmins}
%    The macro |\providehyphenmins| should be used in the language
%    definition files to set the |\lefthyphenmin| and
%    |\righthyphenmin|. This macro will check whether these parameters
%    were provided by the hyphenation file before it takes any action.
%
%  \DescribeMacro{\langhyphenmins}
%    The macro |\|\langvar|hyphenmins| is used to store the values of
%    the |\lefthyphenmin| and |\righthyphenmin|.
%
%  \DescribeMacro{\captionslang}
%    The macro |\captions|\langvar\ defines the macros that
%    hold the texts to replace the original hard-wired texts.
%
%  \DescribeMacro{\datelang}
%    The macro |\date|\langvar\ defines |\today| and
%
%  \DescribeMacro{\extraslang}
%    The macro |\extras|\langvar\ contains all the extra definitions
%    needed for a specific language. This macro, like the following,
%    is a hook -- it must not never used directly.
%
%  \DescribeMacro{\noextraslang}
%    Because we want to let the user switch
%    between languages, but we do not know what state \TeX\ might be in
%    after the execution of |\extras|\langvar, a macro that brings
%    \TeX\ into a predefined state is needed. It will be no surprise
%    that the name of this macro is |\noextras|\langvar.
%
%  \DescribeMacro{\bbl@declare@ttribute}
%    This is a command to be used in the language definition files for
%    declaring a language attribute. It takes three arguments: the
%    name of the language, the attribute to be defined, and the code
%    to be executed when the attribute is to be used.
%
%  \DescribeMacro{\main@language}
%    To postpone the activation of the definitions needed for a
%    language until the beginning of a document, all language
%    definition files should use |\main@language| instead of
%    |\selectlanguage|. This will just store the name of the language,
%    and the proper language will be activated at the start of the
%    document.
%
%  \DescribeMacro{\ProvidesLanguage}
%    The macro |\ProvidesLanguage| should be used to identify the
%    language definition files. Its syntax is similar to the syntax
%    of the \LaTeX\ command |\ProvidesPackage|.
%
%  \DescribeMacro{\LdfInit}
%    The macro |\LdfInit| performs a couple of standard checks that
%    must be made at the beginning of a language definition file,
%    such as checking the category code of the @-sign, preventing
%    the \file{.ldf} file from being processed twice, etc.
%
%  \DescribeMacro{\ldf@quit}
%    The macro |\ldf@quit| does work needed
%    if a \file{.ldf} file was processed
%    earlier. This includes resetting the category code
%    of the @-sign, preparing the language to be activated at
%    |\begin{document}| time, and ending the input stream.
%
%  \DescribeMacro{\ldf@finish}
%    The macro |\ldf@finish| does work needed
%    at the end of each \file{.ldf} file. This
%    includes resetting the category code of the @-sign,
%    loading a local configuration file, and preparing the language
%    to be activated at |\begin{document}| time.
%
%  \DescribeMacro{\loadlocalcfg}
%    After processing a language definition file,
%    \LaTeX\ can be instructed to load a local configuration
%    file. This file can, for instance, be used to add strings to
%    |\captions|\langvar\ to support local document
%    classes. The user will be informed that this
%    configuration file has been loaded. This macro is called by
%    |\ldf@finish|.
%
%  \DescribeMacro{\substitutefontfamily}
%    This command takes three arguments, a font encoding and two font
%    family names. It creates a font description file for the first
%    font in the given encoding. This \file{.fd} file will instruct
%    \LaTeX\ to use a font from the second family when a font from the
%    first family in the given encoding seems to be needed.
%
% \subsection{Support for active characters}
%
%    In quite a number of language definition files, active characters
%    are introduced. To facilitate this, some support macros are
%    provided.
%
% \DescribeMacro{\initiate@active@char}
%    The internal macro |\initiate@active@char| is used in language
%    definition files to instruct \LaTeX\ to give a character the
%    category code `active'. When a character has been made active it
%    will remain that way until the end of the document. Its
%    definition may vary.
%
% \DescribeMacro{\bbl@activate}
% \DescribeMacro{\bbl@deactivate}
%    The command |\bbl@activate| is used to change the way an active
%    character expands. |\bbl@activate| `switches on' the active
%    behaviour of the character. |\bbl@deactivate| lets the active
%    character expand to its former (mostly) non-active self.
%
% \DescribeMacro{\declare@shorthand}
%    The macro |\declare@shorthand| is used to define the various
%    shorthands. It takes three arguments: the name for the collection
%    of shorthands this definition belongs to; the character
%    (sequence) that makes up the shorthand, i.e.\ |~| or |"a|; and the
%    code to be executed when the shorthand is encountered.
%
% \DescribeMacro{\bbl@add@special}
% \DescribeMacro{\bbl@remove@special}
%    The \TeX book states: ``Plain \TeX\ includes a macro called
%    |\dospecials| that is
%    essentially a set macro, representing the set of all characters
%    that have a special category code.'' \cite[p.~380]{DEK} It is
%    used to set text `verbatim'.  To make this work if more
%    characters get a special category code, you have to add this
%    character to the macro |\dospecial|.  \LaTeX\ adds another macro
%    called |\@sanitize| representing the same character set, but
%    without the curly braces.  The macros
%    |\bbl@add@special|\meta{char} and
%    |\bbl@remove@special|\meta{char} add and remove the character
%    \meta{char} to these two sets.
%
% \subsection{Support for saving macro definitions}
%
%    Language definition files may want to \emph{re}define macros that
%    already exist. Therefor a mechanism for saving (and restoring)
%    the original definition of those macros is provided. We provide
%    two macros for this\footnote{This mechanism was introduced by
%    Bernd Raichle.}.
%
% \DescribeMacro{\babel@save} To save the current meaning of any
%    control sequence, the macro |\babel@save| is provided. It takes
%    one argument, \meta{csname}, the control sequence for which the
%    meaning has to be saved.
%
% \DescribeMacro{\babel@savevariable} A second macro is provided to
%    save the current value of a variable.  In this context, anything
%    that is allowed after the |\the| primitive is considered to be a
%    variable. The macro takes one argument, the \meta{variable}.
%
%    The effect of the preceding macros is to append a piece of code
%    to the current definition of |\originalTeX|. When
%    |\originalTeX| is expanded, this code restores the previous
%    definition of the control sequence or the previous value of the
%    variable.
%
% \subsection{Support for extending macros}
%
% \DescribeMacro{\addto}
%    The macro |\addto{|\meta{control sequence}|}{|\meta{\TeX\
%    code}|}| can be used to extend the definition of a macro. The
%    macro need not be defined. This macro can, for instance, be used
%    in adding instructions to a macro like |\extrasenglish|.
%
% \subsection{Macros common to a number of languages}
%
% \DescribeMacro{\allowhyphens}
%    In a couple of European languages compound words are used. This
%    means that when \TeX\ has to hyphenate such a compound word, it
%    only does so at the `\texttt{-}' that is used in such words. To
%    allow hyphenation in the rest of such a compound word, the macro
%    |\allowhyphens| can be used.
%
% \DescribeMacro{\set@low@box}
%    For some languages, quotes need to be lowered to the baseline. For
%    this purpose the macro |\set@low@box| is available. It takes one
%    argument and puts that argument in an |\hbox|, at the
%    baseline. The result is available in |\box0| for further
%    processing.
%
% \DescribeMacro{\save@sf@q}
%    Sometimes it is necessary to preserve the |\spacefactor|.  For
%    this purpose the macro |\save@sf@q| is available. It takes one
%    argument, saves the current spacefactor, executes the argument,
%    and restores the spacefactor.
%
% \DescribeMacro{\bbl@frenchspacing}
% \DescribeMacro{\bbl@nonfrenchspacing}
%    The commands |\bbl@frenchspacing| and |\bbl@nonfrenchspacing| can
%    be used to properly switch French spacing on and off.
%
% \section{Compatibility with \file{german.sty}}\label{l-h}
%
%    The file \file{german.sty} has been
%    one of the sources of inspiration for the \babel{}
%    system. Because of this I wanted to include \file{german.sty} in
%    the \babel{} system.  To be able to do that I had to allow for
%    one incompatibility: in the definition of the macro
%    |\selectlanguage| in \file{german.sty} the argument is used as the
%    {$\langle \it number \rangle$} for an |\ifcase|. So in this case
%    a call to |\selectlanguage| might look like
%    |\selectlanguage{\german}|.
%
%    In the definition of the macro |\selectlanguage| in
%    \file{babel.def} the argument is used as a part of other
%    macronames, so a call to |\selectlanguage| now looks like
%    |\selectlanguage{german}|.  Notice the absence of the escape
%    character.  As of version~3.1a of \babel{} both syntaxes are
%    allowed.
%
%    All other features of the original \file{german.sty} have been
%    copied into a new file, called \file{germanb.sty}\footnote{The
%    `b' is added to the name to distinguish the file from Partls'
%    file.}.
%
%    Although the \babel{} system was developed to be used with
%    \LaTeX, some of the features implemented in the language
%    definition files might be needed by plain \TeX\ users. Care has
%    been taken that all files in the system can be processed by plain
%    \TeX.
%
% \section{Compatibility with \file{ngerman.sty}}
%
%    When used with the options \Lopt{ngerman} or \Lopt{naustrian},
%    \babel{} will provide all features of the package \pkg{ngerman}.
%    There is however one exception:  The commands for special
%    hyphenation of double consonants (|"ff| etc.) and ck (|"ck|),
%    which are no longer required with the new German orthography, are
%    undefined. With the \pkg{ngerman} package, however, these
%    commands will generate appropriate warning messages only.
%
% \section{Compatibility with the \pkg{french} package}
%
%    It has been reported to me that the package \pkg{french} by
%    Bernard Gaulle (\texttt{gaulle@idris.fr}) works
%    together with \babel. On the other hand, it seems \emph{not} to
%    work well together with a lot of other packages. Therefore I have
%    decided to no longer load \file{french.ldf} by default. Instead,
%    when you want to use the package by Bernard Gaulle, you will have
%    to request it specifically, by passing either \Lopt{frenchle} or
%    \Lopt{frenchpro} as an option to \babel.
%
%\StopEventually{%
% \clearpage
% \let\filename\thisfilename
% \section{Conclusion}
%
%    A system of document options has been presented that enable the
%    user of \LaTeX\ to adapt the standard document classes of \LaTeX\
%    to the language he or she prefers to use. These options offer the
%    possibility of switching between languages in one document. The
%    basic interface consists of using one option, which is the same
%    for \emph{all} standard document classes.
%
%    In some cases the language definition files provide macros that
%    can be useful to plain \TeX\ users as well as to \LaTeX\ users.
%    The \babel{} system has been implemented so that it
%    can be used by both groups of users.
%
% \section{Acknowledgements}
%
%    I would like to thank all who volunteered as $\beta$-testers for
%    their time. I would like to mention Julio Sanchez who supplied
%    the option file for the Spanish language and Maurizio Codogno who
%    supplied the option file for the Italian language. Michel Goossens
%    supplied contributions for most of the other languages.  Nico
%    Poppelier helped polish the text of the documentation and
%    supplied parts of the macros for the Dutch language.  Paul
%    Wackers and Werenfried Spit helped find and repair bugs.
%
%    During the further development of the babel system I received
%    much help from Bernd Raichle, for which I am grateful.
%
%  \begin{thebibliography}{9}
%    \bibitem{DEK} Donald E. Knuth,
%      \emph{The \TeX book}, Addison-Wesley, 1986.
%    \bibitem{LLbook} Leslie Lamport,
%       \emph{\LaTeX, A document preparation System}, Addison-Wesley,
%       1986.
%    \bibitem{treebus} K.F. Treebus.
%       \emph{Tekstwijzer, een gids voor het grafisch verwerken van
%       tekst.}
%       SDU Uitgeverij ('s-Gravenhage, 1988). A Dutch book on layout
%       design and typography.
%    \bibitem{HP} Hubert Partl,
%      \emph{German \TeX}, \emph{TUGboat} 9 (1988) \#1, p.~70--72.
%     \bibitem{LLth} Leslie Lamport,
%       in: \TeXhax\ Digest, Volume 89, \#13, 17 February 1989.
%    \bibitem{BEP} Johannes Braams, Victor Eijkhout and Nico Poppelier,
%      \emph{The development of national \LaTeX\ styles},
%      \emph{TUGboat} 10 (1989) \#3, p.~401--406.
%    \bibitem{ilatex} Joachim Schrod,
%      \emph{International \LaTeX\ is ready to use},
%      \emph{TUGboat} 11 (1990) \#1, p.~87--90.
%  \end{thebibliography}
% }
%
% \section{Identification}
%
%    The file \file{babel.sty}\footnote{The file described in this
%    section is called \texttt{\filename}, has version
%    number~\fileversion\ and was last revised on~\filedate.} is meant
%    for \LaTeXe, therefor we make sure that the format file used is
%    the right one.
%
%  \begin{macro}{\ProvidesLanguage}
% \changes{babel~3.7a}{1997/03/18}{Added macro to prevent problems
%    with unexpected \cs{ProvidesFile} in plain formats because of
%    \babel.}
%    The identification code for each file is something that was
%    introduced in \LaTeXe. When the command |\ProvidesFile| does not
%    exist, a dummy definition is provided temporarily. For use in the
%    language definition file the command |\ProvidesLanguage| is
%    defined by \babel.
% \changes{babel~3.4e}{1994/06/24}{Redid the identification code,
%    provided dummy definition of \cs{ProvidesFile} for plain \TeX}
% \changes{babel~3.5f}{1995/07/26}{Store version in \cs{fileversion}}
% \changes{babel~3.5f}{1995/12/18}{Need to temporarily change the
%    definition of \cs{ProvidesFile} for December 1995 release}
% \changes{babel~3.5g}{1996/07/09}{Save a few csnames; use
%    \cs{bbl@tempa} instead of \cs{\@ProvidesFile} and store message
%    in \cs{toks8}}
%    \begin{macrocode}
%<*!package>
\ifx\ProvidesFile\@undefined
  \def\ProvidesFile#1[#2 #3 #4]{%
    \wlog{File: #1 #4 #3 <#2>}%
%<*kernel&patterns>
    \toks8{Babel <#3> and hyphenation patterns for }%
%</kernel&patterns>
    \let\ProvidesFile\@undefined
    }
%    \end{macrocode}
%    As an alternative for |\ProvidesFile| we define
%    |\ProvidesLanguage| here to be used in the language definition
%    files.
%    \begin{macrocode}
%<*kernel>
  \def\ProvidesLanguage#1[#2 #3 #4]{%
    \wlog{Language: #1 #4 #3 <#2>}%
    }
\else
%    \end{macrocode}
%    In this case we save the original definition of |\ProvidesFile| in
%    |\bbl@tempa| and restore it after we have stored the version of
%    the file in |\toks8|.
% \changes{babel~3.7a}{1997/11/04}{Removed superfluous braces}
%    \begin{macrocode}
%<*kernel&patterns>
  \let\bbl@tempa\ProvidesFile
  \def\ProvidesFile#1[#2 #3 #4]{%
    \toks8{Babel <#3> and hyphenation patterns for }%
    \bbl@tempa#1[#2 #3 #4]%
    \let\ProvidesFile\bbl@tempa}
%</kernel&patterns>
%    \end{macrocode}
%    When |\ProvidesFile| is defined we give |\ProvidesLanguage| a
%    similar definition.
%    \begin{macrocode}
  \def\ProvidesLanguage#1{%
    \begingroup
      \catcode`\ 10 %
      \@makeother\/%
      \@ifnextchar[%]
        {\@provideslanguage{#1}}{\@provideslanguage{#1}[]}}
  \def\@provideslanguage#1[#2]{%
    \wlog{Language: #1 #2}%
    \expandafter\xdef\csname ver@#1.ldf\endcsname{#2}%
    \endgroup}
%</kernel>
\fi
%</!package>
%    \end{macrocode}
%  \end{macro}
%
%    Identify each file that is produced from this source file.
% \changes{babel~3.4c}{1995/04/28}{lhyphen.cfg has become
%    lthyphen.cfg}
% \changes{babel~3.5b}{1995/01/25}{lthyphen.cfg has become hyphen.cfg}
%    \begin{macrocode}
%<package>\ProvidesPackage{babel}
%<core>\ProvidesFile{babel.def}
%<kernel&patterns>\ProvidesFile{hyphen.cfg}
%<kernel&!patterns>\ProvidesFile{switch.def}
%<driver&!user>\ProvidesFile{babel.drv}
%<driver&user>\ProvidesFile{user.drv}
                [2012/06/28 v3.9a alpha 2 %
%<package>     The Babel package]
%<core>         Babel common definitions]
%<kernel>      Babel language switching mechanism]
%<driver>]
%    \end{macrocode}
%
%    \section{The Package File}
%
%    In order to make use of the features of \LaTeXe, the \babel\
%    system contains a package file, \file{babel.sty}. This file is
%    loaded by the |\usepackage| command and defines all the language
%    options whose name is different from that of the |.ldf| file
%    (like variant spellings). It also takes care of a number of
%    compatibility issues with other packages an defines a few
%    aditional package options.
%
%    \subsection{key=value options}
%
%    Handling of package options is done in three passes. [!!! Not
%    very happy with the idea, anyway.] The first one processes
%    options which follow the syntax |<key>=<value>|, the second one
%    loads the requested languages, except the main one if set with
%    the key |main|, and the third one loads the latter. First, we
%    ``flag'' valid options with a nil value.
%    \begin{macrocode}
%<*package>
\let\bbl@opt@shorthands\@nnil
\let\bbl@opt@config\@nnil
\let\bbl@opt@main\@nnil
%    \end{macrocode}
%    The following tool is defined temporarily to store the values of
%    options.
%    \begin{macrocode}
\def\bbl@a#1=#2\bbl@a{%
  \expandafter\ifx\csname bbl@opt@#1\endcsname\@nnil
    \expandafter\edef\csname bbl@opt@#1\endcsname{#2}%
  \else
    \PackageError{babel}{%
      Bad option `#1=#2'. Either you have misspelled the\MessageBreak
      key or there is a previous setting of `#1'}{%
      Valid keys are `shorthands' and `main'.}%
  \fi}
%    \end{macrocode}
%    Now the option list is processed, taking into account only
%    |<key>=<value>| options. |shorthand=off| is set separately.
%    \begin{macrocode}
\DeclareOption{shorthands=off}{\bbl@a shorthands=\bbl@a}
\DeclareOption*{%
  \@expandtwoargs\in@{\string=}{\CurrentOption}%
  \ifin@
    \expandafter\bbl@a\CurrentOption\bbl@a
  \fi}
%    \end{macrocode}
%    Now we finish the first pass (and start over).
%    \begin{macrocode}
\ProcessOptions*
%    \end{macrocode}
%
%    \subsection{Conditional loading of shorthands}
%
%    If there is no |shorthands=<chars>|, the original \textsf{babel}
%    macros are left untouched, but if there is, these macros are
%    wrapped (in |babel.def|) to define only those given. In this
%    mode, some macros are removed and one is added
%    (|\babelshorthand|).
%    \begin{macrocode}
\long\def\bbl@afterelse#1\else#2\fi{\fi#1}
\long\def\bbl@afterfi#1\fi{\fi#1}
%    \begin{macrocode}
%      A bit of optimization. Some code makes sense only with
%      |shorthands=...|.
\ifx\bbl@opt@shorthands\@nnil
  \def\bbl@ifshorthand#1#2#3{#3}%
\else
%    \end{macrocode}
%    We make sure all chars are `other', with the help of an auxiliary
%    macro.
%    \begin{macrocode}
  \def\bbl@sh@string#1{%
    \ifx#1\@empty\else
      \string#1%
      \expandafter\bbl@sh@string
    \fi}
  \edef\bbl@opt@shorthands{%
    \expandafter\bbl@sh@string\bbl@opt@shorthands\@empty}%
%    \end{macrocode}
%    The following macros tests if a shortand is one of the allowed
%    ones.
%    \begin{macrocode}
  \edef\bbl@ifshorthand#1{%
    \noexpand\expandafter 
    \noexpand\bbl@ifsh@i
    \noexpand\string
    #1\bbl@opt@shorthands
    \noexpand\@empty\noexpand\@secondoftwo}
  \def\bbl@aux@ifsh#1\@secondoftwo{\@firstoftwo}
  \def\bbl@ifsh@i#1#2{%
    \ifx#1#2%
      \expandafter\bbl@aux@ifsh
    \else
      \ifx#2\@empty
        \bbl@afterelse{\expandafter\@gobble}%
      \else
        \bbl@afterfi{\expandafter\bbl@ifsh@i}%
      \fi
    \fi
    #1}
%    \end{macrocode}
%    The following is ignored with |shorthands=off|, since it is
%    intended to take some aditional actions for certain chars.
%   !!!  2012/07/04 Code for bbl@languages, to be moved.
%    \begin{macrocode}
  \ifx\bbl@opt@shorthands\@empty
    \def\bbl@ifshorthand#1#2#3{#3}%
  \else
    \bbl@ifshorthand{'}%
      {\PassOptionsToPackage{activeacute}{babel}}{}
    \bbl@ifshorthand{`}%
      {\PassOptionsToPackage{activegrave}{babel}}{}
    % \bbl@ifshorthand{\string:}{}%
    %   {\g@addto@macro\bbl@ignorepackages{,hhline,}}
  \fi
\fi
\ifx\bbl@languages\@undefined\else
  \def\bbl@tempa#1/0/#2\@nnil{#1}%
  \edef\bbl@nulllanguage{\expandafter\bbl@tempa\bbl@languages\@nnil}
  \def\@nopatterns#1{%
    \PackageWarningNoLine{babel}%
      {No hyphenation patterns were loaded for\MessageBreak
        the language `#1'\MessageBreak
        I will use the patterns loaded for \bbl@nulllanguage\space
        instead}}
\fi
%    \end{macrocode}
%
%    Apart from all the language options below we also have a few options
%    that influence the behaviour of language definition files.
%
%    The following options don't do anything themselves, they are just
%    defined in order to make it possible for language definition
%    files to check if one of them was specified by the user.
% \changes{babel~3.5d}{1995/07/04}{Added options to influence
%    behaviour of active acute and grave accents}
%    \begin{macrocode}
\DeclareOption{activeacute}{}
\DeclareOption{activegrave}{}
%    \end{macrocode}
%    The next option tells \babel\ to leave shorthand characters
%    active at the end of processing the package. This is \emph{not}
%    the default as it can cause problems with other packages, but for
%    those who want to use the shorthand characters in the preamble of
%    their documents this can help.
% \changes{babel~3.6f}{1997/01/14}{Added option
%    \Lopt{KeepShorthandsActive}}
% \changes{babel~3.7a}{1997/03/21}{No longer define the control
%    sequence \cs{KeepShorthandsActive}}
%    \begin{macrocode}
\DeclareOption{KeepShorthandsActive}{}
%    \end{macrocode}
% !!!! Other options. In this pass, |shorthands=off| does nothing.
%    \begin{macrocode}
\DeclareOption{nocrossrefs}{}
\DeclareOption{nocitations}{}
\DeclareOption{noconfig}{}
\DeclareOption{nomarks}{}
\DeclareOption{delay}{}
%    \end{macrocode}
%
%  \subsection{Language options}
%
% \changes{babel~3.6c}{1997/01/05}{When \cs{LdfInit} is undefined we
%    need to load \file{babel.def} from \file{babel.sty}}
% \changes{babel~3.6l}{1999/04/03}{Don't load \file{babel.def} now,
%    but rather define \cs{LdfInit} temporarily in order to load
%    \file{babel.def} at the right time, preventing problems with the
%    temporary definition of \cs{bbl@redefine}}
% \changes{babel~3.6r}{1999/04/12}{We \textbf{do} need to load
%    \file{babel.def} right now as \cs{ProvidesLanguage} needs to be
%    defined before the \file{.ldf} files are read and the reason for
%    for 3.6l has been removed}
% \changes{babel~3.9a}{2012/06/15}{Rewritten the loading mechanism, so
%    that languages not declared are also correctly recognized, even
%    if given as global options} 
% \changes{babel~3.5a}{1995/03/14}{Changed extension of language
%    definition files to \texttt{ldf}}
% \changes{babel~3.5d}{1995/07/02}{Load language definition files
%    \emph{after} the check for the hyphenation patterns}
% \changes{babel~3.5g}{1996/10/04}{Added option \Lopt{afrikaans}}
% \changes{babel~3.7g}{2001/02/09}{Added option \Lopt{acadian}}
% \changes{babel~3.8c}{2004/06/12}{Added option \Lopt{australian}}
% \changes{babel~3.8h}{2005/11/23}{Added option \Lopt{albanian}}
% \changes{babel~3.6i}{1997/02/20}{Added the \Lopt{Basque} option}
% \changes{babel~3.8h}{2005/11/23}{added synonyms \Lopt{indonesian},
%    \Lopt{indon} and \Lopt{bahasai} for the original bahasa
%    (indonesia) support}
% \changes{babel~3.8h}{2005/11/23}{added \Lopt{malay}, \Lopt{meyaluy}
%    and \Lopt{bahasam} for the Bahasa Malaysia support}
% \changes{babel~3.5b}{1995/05/25}{Added \Lopt{brazilian} as
%    alternative for \Lopt{brazil}}
% \changes{babel~3.5d}{1995/07/02}{Added \Lopt{british} as an
%    alternative for \Lopt{english} with a preference for british
%    hyphenation}
% \changes{babel~3.7f}{2000/09/21}{Added the \Lopt{bulgarian} option}
% \changes{babel~3.7g}{2001/02/07}{Added option \Lopt{canadian}}
% \changes{babel~3.7g}{2001/02/09}{Added option \Lopt{canadien}}
% \changes{babel~3.5b}{1995/06/06}{Added the \Lopt{estonian} option}
% \changes{babel~3.5f}{1996/01/10}{Now use the file \file{frenchb.ldf}
%    from Daniel Flipo for french support}
% \changes{babel~3.6e}{1997/01/08}{Added option \Lopt{frenchb} an
%    alias for \Lopt{francais}}
% \changes{babel~3.5d}{1995/07/02}{Load \file{french.ldf} when it is
%    found instead of \file{frenchb.ldf}}
% \changes{babel~3.7j}{2003/06/07}{\emph{only} load
%    \file{frenchb.ldf}}
% \changes{babel~3.5f}{1996/05/31}{Added the \Lopt{greek} option}
% \changes{babel~3.7a}{1997/11/13}{Added the \Lopt{polutonikogreek}
%    option}
% \changes{babel~3.7c}{1999/04/22}{set the correct language attribute
%    for polutoniko greek}
% \changes{babel~3.7a}{1998/03/27}{Added the \Lopt{hebrew} option}
% \changes{babel~3.7b}{1998/06/25}{Added the \Lopt{latin} option}
% \changes{babel~3.7m}{2003/11/13}{Added the \Lopt{interlingua}
%    option}
% \changes{babel~3.6p}{1999/04/10}{Added the \Lopt{ngerman} and
%    \Lopt{naustrian} options}
% \changes{babel~3.7f}{2000/09/26}{Added the \Lopt{samin} option}
% \changes{babel~3.8c}{2004/06/12}{Added the \Lopt{newzealand} option}
% \changes{babel~3.6e}{1997/01/08}{Added options \Lopt{UKenglish} and
%    \Lopt{USenglish}}
%
%    Languages are loaded when processing the corresponding option
%    \textit{except} if a |main| language has been set. In such a
%    case, it is not loaded until all options has been processed.
%    \begin{macrocode}
\def\bbl@tempa#1#2{%
  \def\bbl@a{#1}%
  \ifx\bbl@a\bbl@opt@main
    \def\bbl@loadmain{%
      \DeclareOption{#1}{#2\csname #1.ldf-h@@k\endcsname}}%
    \DeclareOption{#1}\@empty
  \else
    \DeclareOption{#1}{#2\csname #1.ldf-h@@k\endcsname}%
  \fi}%
%    \end{macrocode}
%    Now, we set language options, but first make sure |\LdfInit| is defined.
%    \begin{macrocode}
\ifx\LdfInit\@undefined\input babel.def\relax\fi
\bbl@tempa{acadian}{% \iffalse meta-comment
%
% Copyright 1989-2009 Johannes L. Braams and any individual authors
% listed elsewhere in this file.  All rights reserved.
% 
% This file is part of the Babel system.
% --------------------------------------
% 
% It may be distributed and/or modified under the
% conditions of the LaTeX Project Public License, either version 1.3
% of this license or (at your option) any later version.
% The latest version of this license is in
%   http://www.latex-project.org/lppl.txt
% and version 1.3 or later is part of all distributions of LaTeX
% version 2003/12/01 or later.
% 
% This work has the LPPL maintenance status "maintained".
% 
% The Current Maintainer of this work is Johannes Braams.
% 
% The list of all files belonging to the Babel system is
% given in the file `manifest.bbl. See also `legal.bbl' for additional
% information.
% 
% The list of derived (unpacked) files belonging to the distribution
% and covered by LPPL is defined by the unpacking scripts (with
% extension .ins) which are part of the distribution.
% \fi
% \CheckSum{2135}
%
% \iffalse
%    Tell the \LaTeX\ system who we are and write an entry on the
%    transcript. Nothing to write to the .cfg file, if generated.
%<*dtx>
\ProvidesFile{frenchb.dtx}
%</dtx>
% \changes{v2.1d}{2008/05/04}{Argument of \cs{ProvidesLanguage} changed
%     from `french' to `frenchb', otherwise \cs{listfiles} prints
%     no date/version information.  The bug with \cs{listfiles}
%     (introduced in v.1.5!), was pointed out by Ulrike Fischer.}
%<code>\ProvidesLanguage{frenchb}
%\ProvidesFile{frenchb.dtx}
%<*!cfg>
        [2009/03/16 v2.3d French support from the babel system]
%</!cfg>
%<*cfg>
%% frenchb.cfg: configuration file for frenchb.ldf
%% Daniel Flipo Daniel.Flipo at univ-lille1.fr
%</cfg>
%%    File `frenchb.dtx'
%%    Babel package for LaTeX version 2e
%%    Copyright (C) 1989 - 2009
%%              by Johannes Braams, TeXniek
%
%<*!cfg>
%%    Frenchb language Definition File
%%    Copyright (C) 1989 - 2009
%%              by Johannes Braams, TeXniek
%%                 Daniel Flipo, GUTenberg
%
%%    Please report errors to: Daniel Flipo, GUTenberg
%%                             Daniel.Flipo at univ-lille1.fr
%</!cfg>
%
%    This file is part of the babel system, it provides the source
%    code for the French language definition file.
%
%<*filedriver>
\documentclass[a4paper]{ltxdoc}
\DeclareFontEncoding{T1}{}{}
\DeclareFontSubstitution{T1}{lmr}{m}{n}
\DeclareTextCommand{\guillemotleft}{OT1}{%
  {\fontencoding{T1}\fontfamily{lmr}\selectfont\char19}}
\DeclareTextCommand{\guillemotright}{OT1}{%
  {\fontencoding{T1}\fontfamily{lmr}\selectfont\char20}}
\newcommand*\TeXhax{\TeX hax}
\newcommand*\babel{\textsf{babel}}
\newcommand*\langvar{$\langle \mathit lang \rangle$}
\newcommand*\note[1]{}
\newcommand*\Lopt[1]{\textsf{#1}}
\newcommand*\file[1]{\texttt{#1}}
\begin{document}
\setlength{\parindent}{0pt}
\begin{center}
  \textbf{\Large A Babel language definition file for French}\\[3mm]^^A\]
  Daniel \textsc{Flipo}\\
  \texttt{Daniel.Flipo@univ-lille1.fr}
\end{center}
 \RecordChanges
 \DocInput{frenchb.dtx}
\end{document}
%</filedriver>
% \fi
% \GetFileInfo{frenchb.dtx}
%
%  \section{The French language}
%
%    The file \file{\filename}\footnote{The file described in this
%    section has version number \fileversion\ and was last revised on
%    \filedate.}, defines all the language definition macros for the
%    French language.
%
%    Customisation for the French language is achieved following the
%    book ``Lexique des r\`egles typographiques en usage \`a
%    l'Imprimerie nationale'' troisi\`eme \'edition (1994),
%    ISBN-2-11-081075-0.
%
%    First version released: 1.1 (1996/05/31) as part of
%    \babel-3.6beta.
%
%    |frenchb| has been improved using helpful suggestions from many
%    people, mainly from Jacques Andr\'e, Michel Bovani, Thierry Bouche,
%    and Vincent Jalby.  Thanks to all of them!
%
%    This new version (2.x) has been designed to be used with \LaTeXe{}
%    and Plain\TeX{} formats only. \LaTeX-2.09 is no longer supported.
%    Changes between version 1.6 and \fileversion{} are listed in
%    subsection~\ref{ssec-changes} p.~\pageref{ssec-changes}.
%
%    An extensive documentation is available in French here:\\
%    |http://daniel.flipo.free.fr/frenchb|
%
%  \subsection{Basic interface}
%
%    In a multilingual document, some typographic rules are language
%    dependent, i.e. spaces before `double punctuation' (|:| |;| |!|
%    |?|) in French, others concern the general layout (i.e. layout of
%    lists, footnotes, indentation of first paragraphs of sections) and
%    should apply to the whole document.
%
%    Starting with version~2.2, |frenchb| behaves differently according
%    to \babel's \emph{main language} defined as the \emph{last}
%    option\footnote{Its name is kept in \texttt{\textbackslash
%           bbl@main@language}.} at \babel's loading.  When French is
%    not \babel's main language, |frenchb| no longer alters the global
%    layout of the document (even in parts where French is the current
%    language): the layout of lists, footnotes, indentation of first
%    paragraphs of sections are not customised by |frenchb|.
%
%    When French is loaded as the last option of \babel, |frenchb|
%    makes the following changes to the global layout, \emph{both in
%    French and in all other languages}\footnote{%
%       For each item, hooks are provided to reset standard
%       \LaTeX{} settings or to emulate the behavior of former versions
%       of \texttt{frenchb} (see command
%       \texttt{\textbackslash frenchbsetup\{\}},
%       section~\ref{ssec-custom}).}:
%    \begin{enumerate}
%    \item the first paragraph of each section is indented
%          (\LaTeX{} only);
%    \item the default items in itemize environment are set to `--'
%          instead of `\textbullet', and all vertical spacing and glue
%          is deleted; it is possible to change `--' to something else
%          (`---' for instance) using |\frenchbsetup{}|;
%    \item vertical spacing in general \LaTeX{} lists is
%          shortened;
%    \item footnotes are displayed ``\`a la fran\c{c}aise''.
%    \end{enumerate}
%
%    Regarding local typography, the command |\selectlanguage{french}|
%    switches to the French language\footnote{%
%      \texttt{\textbackslash selectlanguage\{francais\}}
%      and \texttt{\textbackslash selectlanguage\{frenchb\}} are kept
%      for backward compatibility but should no longer be used.},
%    with the following effects:
%    \begin{enumerate}
%    \item French hyphenation patterns are made active;
%    \item `double punctuation' (|:| |;| |!| |?|) is made
%           active%\footnote{Actually, they are active in the whole
%           document, only their expansions differ in French and
%           outside French} for correct spacing in French;
%    \item |\today| prints the date in French;
%    \item the caption names are translated into French
%          (\LaTeX{} only);
%    \item the space after |\dots| is removed in French.
%    \end{enumerate}
%
%    Some commands are provided in |frenchb| to make typesetting
%    easier:
%    \begin{enumerate}
%    \item French quotation marks can be entered using the commands
%          |\og| and |\fg| which work in \LaTeXe and Plain\TeX,
%          their appearance depending on what is available to draw
%          them; even if you use \LaTeXe{} \emph{and} |T1|-encoding,
%          you should refrain from entering them as
%          |<<~French quotation marks~>>|: |\og| and |\fg| provide
%          better horizontal spacing.
%          |\og| and |\fg| can be used outside French, they typeset
%          then English quotes `` and ''.
%    \item A command |\up| is provided to typeset superscripts like
%          |M\up{me}| (abbreviation for ``Madame''), |1\up{er}| (for
%          ``premier'').  Other commands are also provided for
%          ordinals: |\ier|, |\iere|, |\iers|, |\ieres|, |\ieme|,
%          |\iemes| (|3\iemes| prints 3\textsuperscript{es}).
%    \item Family names should be typeset in small capitals and never
%          be hyphenated, the macro |\bsc| (boxed small caps) does
%          this, e.g., |Leslie~\bsc{Lamport}| will produce
%          Leslie~\mbox{\textsc{Lamport}}. Note that composed names
%          (such as Dupont-Durant) may now be hyphenated on explicit
%          hyphens, this differs from |frenchb|~v.1.x.
%    \item Commands |\primo|, |\secundo|, |\tertio| and |\quarto|
%          print 1\textsuperscript{o}, 2\textsuperscript{o},
%          3\textsuperscript{o}, 4\textsuperscript{o}.
%          |\FrenchEnumerate{6}| prints 6\textsuperscript{o}.
%    \item Abbreviations for ``Num\'ero(s)'' and ``num\'ero(s)''
%          (N\textsuperscript{o} N\textsuperscript{os}
%          n\textsuperscript{o} and n\textsuperscript{os}~)
%          are obtained via the commands |\No|, |\Nos|, |\no|, |\nos|.
%    \item Two commands are provided to typeset the symbol for
%          ``degr\'e'': |\degre| prints the raw character and
%          |\degres| should be used to typeset temperatures (e.g.,
%          ``|20~\degres C|'' with an unbreakable space), or for
%          alcohols' strengths (e.g., ``|45\degres|'' with \emph{no}
%          space in French).
%    \item In math mode the comma has to be surrounded with
%          braces to avoid a spurious space being inserted after it,
%          in decimal numbers for instance (see the \TeX{}book p.~134).
%          The command |\DecimalMathComma| makes the comma be an
%          ordinary character \emph{in French only} (no space added);
%          as a counterpart, if |\DecimalMathComma| is active, an
%          explicit space has to be added in lists and intervals:
%          |$[0,\ 1]$|, |$(x,\ y)$|. |\StandardMathComma| switches back
%          to the standard behaviour of the comma.
%    \item A command |\nombre| was provided in 1.x versions to easily
%          format numbers in slices of three digits separated either
%          by a comma in English or with a space in French; |\nombre|
%          is now mapped to |\numprint| from \file{numprint.sty}, see
%          \file{numprint.pdf} for more information.
%    \item |frenchb| has been designed to take advantage of the |xspace|
%          package if present: adding |\usepackage{xspace}| in the
%          preamble will force macros like |\fg|, |\ier|, |\ieme|,
%          |\dots|, \dots, to respect the spaces you type after them,
%          for instance typing `|1\ier juin|' will print
%          `1\textsuperscript{er} juin' (no need for a forced space
%          after |1\ier|).
%    \end{enumerate}
%
%  \subsection{Customisation}
%  \label{ssec-custom}
%
%     Up to version 1.6, customisation of |frenchb| was achieved
%     by entering commands in \file{frenchb.cfg}.  This possibility
%     remains for compatibility, but \emph{should not longer be used}.
%     Version 2.0 introduces a new command |\frenchbsetup{}| using
%     the \file{keyval} syntax which should make it easier to choose
%     among the many options available. The command |\frenchbsetup{}|
%     is to appear in the preamble only (after loading \babel).
%
%     \vspace{.5\baselineskip}
%     |\frenchbsetup{ShowOptions}| prints all available options to
%     the \file{.log} file, it is just meant as a remainder of the
%     list of offered options. As usual with \file{keyval} syntax,
%     boolean options (as |ShowOptions|) can be entered as
%     |ShowOptions=true| or just |ShowOptions|, the `|=true|' part
%     can be omitted.
%
%     \vspace{.5\baselineskip}
%     The other options are listed below. Their default value is shown
%     between brackets, sometimes followed be a `\texttt{*}'.
%     The `\texttt{*}' means that the default shown applies when
%     |frenchb| is loaded as the \emph{last} option of \babel{}
%     ---\babel's \emph{main language}---, and is toggled otherwise:
%     \begin{itemize}
%     \item |StandardLayout=true [false*]| forces |frenchb| not to
%       interfere with the layout: no action on any kind of lists,
%       first paragraphs of sections are not indented (as in English),
%       no action on footnotes. This option replaces the former
%       command |\StandardLayout|.  It can be used to avoid conflicts
%       with classes or packages which customise lists or footnotes.
%     \item |GlobalLayoutFrench=false [true*]| can be used, when French
%       is the main language, to emulate what prior versions of
%       |frenchb| (pre-2.2) did: lists, and first paragraphs
%       of sections will be displayed the standard way in other
%       languages than French, and ``\`a la fran\c{c}aise'' in French.
%       Note that the layout of footnotes is language independent
%       anyway (see below |FrenchFootnotes| and |AutoSpaceFootnotes|).
%       This option replaces the former command |\FrenchLayout|.
%     \item |ReduceListSpacing=false [true*]|; |frenchb| normally
%       reduces the values of the vertical spaces used in the
%       environment |list| in French; setting this option to |false|
%       reverts to the standard settings of |list|.  This option
%       replaces the former command |\FrenchListSpacingfalse|.
%     \item |CompactItemize=false [true*]|; |frenchb| normally
%       suppresses any vertical space between items of |itemize| lists
%       in French; setting this option to |false| reverts to the
%       standard settings of |itemize| lists.  This option replaces
%       the former command |\FrenchItemizeSpacingfalse|.
%     \item |StandardItemLabels=true [false*]| when set to |true| this
%       option stops |frenchb| from changing the labels in |itemize|
%       lists in French.
%     \item |ItemLabels=\textemdash|, |\textbullet|, |\ding{43}|,
%       \dots, |[\textendash*]|; when |StandardItemLabels=false| (the
%       default), this option enables to choose the label used in
%       |itemize| lists for all levels.  The next three options do
%       the same but each one for one level only. Note that the
%       example |\ding{43}| requires |\usepackage{pifont}|.
%     \item |ItemLabeli=\textemdash|, |\textbullet|, |\ding{43}|,
%       \dots,|[\textendash*]|
%     \item |ItemLabelii=\textemdash|, |\textbullet|, |\ding{43}|,
%       \dots, |[\textendash*]|
%     \item |ItemLabeliii=\textemdash|, |\textbullet|, |\ding{43}|,
%       \dots, |[\textendash*]|
%     \item |ItemLabeliv=\textemdash|, |\textbullet|, |\ding{43}|,
%       \dots, |[\textendash*]|
%     \item |StandardLists=true [false*]| forbids |frenchb| to
%       customise any kind of list. Do activate the option
%       |StandardLists| when using classes or packages that customise
%       lists too (|enumitem|, |paralist|, \dots{}) to avoid conflicts.
%       This option is just a shorthand for |ReduceListSpacing=false|
%       and |CompactItemize=false| and |StandardItemLabels=true|.
%     \item |IndentFirst=false [true*]|; |frenchb| normally forces
%       indentation of the first paragraph of sections.
%       When this option is set to |false|, the first paragraph of
%       will look the same in French and in English (not indented).
%     \item |FrenchFootnotes=false [true*]| reverts to the standard
%       layout of footnotes. By default |frenchb| typesets leading
%       numbers as `1.\hspace{.5em}' instead of `$\hbox{}^1$', but
%       has no effect on footnotes numbered with symbols (as in the
%       |\thanks| command).  The former commands |\StandardFootnotes|
%       and |\FrenchFootnotes| are still there, |\StandardFootnotes|
%       can be useful when some footnotes are numbered with letters
%       (inside minipages for instance).
%     \item |AutoSpaceFootnotes=false [true*]| ; by default |frenchb|
%       adds a thin space in the running text before the number or
%       symbol calling the footnote.  Making this option |false|
%       reverts to the standard setting (no space added).
%     \item |FrenchSuperscripts=false [true]| ; then
%       |\up=\textsuperscript| (option added in version 2.1).
%       Should only be made |false| to recompile older documents.
%       By default |\up| now relies on |\fup| designed to produce
%       better looking superscripts.
%     \item |AutoSpacePunctuation=false [true]|; in French, the user
%       \emph{should} input a space before the four characters `|:;!?|'
%       but as many people forget about it (even among native French
%       writers!), the default behaviour of |frenchb| is to
%       automatically add a |\thinspace| before `|;|' `|!|' `|?|' and a
%       normal (unbreakable) space before~`|:|' (this is recommended by
%       the French Imprimerie nationale).  This is convenient in most
%       cases but can lead to addition of spurious spaces in URLs or in
%       MS-DOS paths but only if they are no typed using |\texttt| or
%       verbatim mode. When the current font is a monospaced
%       (typewriter) font, |AutoSpacePunctuation| is locally switched
%       to |false|, no spurious space is added in that case, so the
%       default behaviour of of |frenchb| in that area should be fine
%       in most circumstances.
%
%       Choosing |AutoSpacePunctuation=false| will ensure that
%       a proper space will be added before `|:;!?|' \emph{if and only
%       if} a (normal) space has been typed in. Those who are unsure
%       about their typing in this area should stick to the default
%       option and type |\string;| |\string:| |\string!| |\string?|
%       instead of |;| |:| |!| |?| in case an unwanted space is
%       added by |frenchb|.
%     \item |ThinColonSpace=true [false]| changes the normal
%       (unbreakable) space added before the colon `:' to a thin space,
%       so that the same amount of space is added before any of the
%       four double punctuation characters. The default setting is
%       supported by the French Imprimerie nationale.
%     \item |LowercaseSuperscripts=false [true]| ; by default |frenchb|
%       inhibits the uppercasing of superscripts (for instance when they
%       are moved to page headers). Making this option |false|
%       will disable this behaviour (not recommended).
%     \item |PartNameFull=false [true]|; when true, |frenchb| numbers
%       the title of |\part{}| commands as ``Premi\`ere partie'',
%       ``Deuxi\`eme partie'' and so on. With some classes which change
%       the|\part{}| command (AMS and SMF classes do so), you will get
%       ``Premi\`ere partie~I'', ``Deuxi\`eme partie~II'' instead;
%       when this occurs, this option should be set to |false|,
%       part titles will then be printed as ``Partie I'', ``Partie II''.
%     \item |og=|\texttt{\guillemotleft}, |fg=|\texttt{\guillemotright};
%       when guillemets characters are available on the keyboard
%       (through a compose key for instance), it is nice to use them
%       instead of typing |\og| and |\fg|. This option tells |frenchb|
%       which characters are opening and closing French guillemets
%       (they depend on the input encoding), then you can type either
%       \texttt{\guillemotleft{} guillemets \guillemotright}, or
%       \texttt{\guillemotleft{}guillemets\guillemotright} (with or
%       without spaces), to get properly typeset French quotes.
%       This option requires \file{inputenc} to be loaded with the
%       proper encoding, it works with 8-bits encodings (latin1,
%       latin9, ansinew,  applemac,\dots) and multi-byte encodings
%       (utf8 and utf8x).
%     \end{itemize}
%
%  \subsection{Hyphenation checks}
%  \label{ssec-hyphen}
%
%    Once you have built your format, a good precaution would be to
%    perform some basic tests about hyphenation in French. For
%    \LaTeXe{} I suggest this:
%    \begin{itemize}
%    \item run the following file, with the encoding suitable for
%      your machine (\textit{my-encoding} will be |latin1| for
%      \textsc{unix} machines, |ansinew| for PCs running~Windows,
%      |applemac| or |latin1| for Macintoshs, or |utf8|\dots\\[3mm]^^A\]
%      |%%% Test file for French hyphenation.|\\
%      |\documentclass{article}|\\
%      |\usepackage[|\textit{my-encoding}|]{inputenc}|\\
%      |\usepackage[T1]{fontenc} % Use LM fonts|\\
%      |\usepackage{lmodern}     % for French|\\
%      |\usepackage[frenchb]{babel}|\\
%      |\begin{document}|\\
%      |\showhyphens{signal container \'ev\'enement alg\`ebre}|\\
%      |\showhyphens{|\texttt{signal container \'ev\'enement
%                     alg\`ebre}|}|\\
%      |\end{document}|
%    \item check the hyphenations proposed by \TeX{} in your log-file;
%      in French you should get with both 7-bit and 8-bit encodings\\
%      \texttt{si-gnal contai-ner \'ev\'e-ne-ment al-g\`ebre}.\\
%      Do not care about how accented characters are displayed in the
%      log-file, what matters is the position of the `|-|' hyphen
%      signs \emph{only}.
%    \end{itemize}
%    If they are all correct, your installation (probably) works fine,
%    if one (or more) is (are) wrong, ask a local wizard to see what's
%    going wrong and perform the test again (or e-mail me about what
%    happens).\\
%    Frequent mismatches:
%    \begin{itemize}
%    \item you get |sig-nal con-tainer|, this probably means that the
%    hyphenation patterns you are using are for US-English, not for
%    French;
%    \item you get no hyphen at all in \texttt{\'ev\'e-ne-ment}, this
%    probably means that you are using CM fonts and the macro
%    |\accent| to produce accented characters.
%    Using 8-bits fonts with built-in accented characters avoids
%    this kind of mismatch.
%    \end{itemize}
%
%    \textbf{Options' order} -- Please remember that options are read
%    in the order they appear inside the |\frenchbsetup| command.
%    Someone wishing that |frenchb| leaves the layout of lists
%    and footnotes untouched but caring for indentation of first
%    paragraph of sections could choose
%    |\frenchbsetup{StandardLayout,IndentFirst}| and get the expected
%    layout. Choosing |\frenchbsetup{IndentFirst,StandardLayout}|
%    would not lead to the expected result: option |IndentFirst| would
%    be overwritten by |StandardLayout|.
%
%  \subsection{Changes}
%  \label{ssec-changes}
%
%  \subsubsection*{What's new in version 2.0?}
%
%    Here is the list of all changes:
%    \begin{itemize}
%    \item Support for \LaTeX-2.09 and for \LaTeXe{} in compatibility
%      mode has been dropped. This version is meant for \LaTeXe{} and
%      Plain based formats (like \file{bplain}). \LaTeXe{} formats
%      based on ml\TeX{} are no longer supported either (plenty of
%      good 8-bits fonts are available now, so T1 encoding should
%      be preferred for typesetting in French). A warning is issued
%      when OT1 encoding is in use at the |\begin{document}|.
%    \item Customisation should now be handled by command
%      |\frenchbsetup{}|, \file{frenchb.cfg} (kept for compatibility)
%      should no longer be used. See section~\ref{ssec-custom} for
%      the list of available options.
%    \item Captions in figures and table have changed in French: former
%      abbreviations ``Fig.'' and ``Tab.'' have been replaced by full
%      names ``Figure'' and ``Table''.  If this leads to formatting
%      problems in captions, you can add the following two commands to
%      your preamble (after loading \babel) to get the former captions\\
%      |\addto\captionsfrench{\def\figurename{{\scshape Fig.}}}|\\
%      |\addto\captionsfrench{\def\tablename{{\scshape Tab.}}}|.
%    \item The |\nombre| command is now provided by the \file{numprint}
%      package which has to be loaded \emph{after} \babel{} with the
%      option |autolanguage| if number formatting should depend on the
%      current language.
%    \item The |\bsc| command no longer uses an |\hbox| to stop
%      hyphenation of names but a |\kern0pt| instead. This change
%      enables \file{microtype} to fine tune the length of the
%      argument of |\bsc|; as a side-effect, compound names like
%      Dupont-Durand can now be hyphenated on  explicit hyphens.
%      You can get back to the former behaviour of |\bsc| by adding\\
%      |\renewcommand*{\bsc}[1]{\leavevmode\hbox{\scshape #1}}|\\
%      to the preamble of your document.
%    \item Footnotes are now displayed ``\`a la fran\c caise'' for the
%      whole document, except with an explicit\\
%      |\frenchbsetup{AutoSpaceFootnotes=false,FrenchFootnotes=false}|.\\
%      Add this command if you want standard footnotes. It is still
%      possible to revert locally to the standard layout of footnotes
%      by adding |\StandardFootnotes| (inside a |minipage| environment
%      for instance).
%    \end{itemize}
%
%  \subsubsection*{What's new in version 2.1?}
%
%      New command |\fup| to typeset better looking superscripts.
%      Former command |\up| is now defined as |\fup|, but an option
%      |\frenchbsetup{FrenchSuperscripts=false}| is provided for
%      backward compatibility.  |\fup| was designed using ideas from
%      Jacques Andr\'e, Thierry Bouche and Ren\'e Fritz, thanks to them!
%
%  \subsubsection*{What's new in version 2.2?}
%
%      Starting with version~2.2a, |frenchb| alters the layout of
%      lists, footnotes, and the indentation of first paragraphs of
%      sections) \emph{only if} French is the ``main language''
%      (i.e. babel's last language option). The layout is global for
%      the whole document: lists, etc. look the same in French and in
%      other languages, everything is typeset ``\`a la fran\c caise''
%      if French is the ``main language'', otherwise |frenchb| doesn't
%      change anything regarding lists, footnotes, and indentation of
%      paragraphs.
%
%  \subsubsection*{What's new in version 2.3?}
%
%      Starting with version~2.3a, |frenchb| no longer inserts spaces
%      automatically before `|:;!?|' when a typewriter font is in use;
%      this was suggested by Yannis Haralambous to prevent
%      spurious spaces in computer source code or expressions like
%      \texttt{C\string:/foo}, \texttt{http\string://foo.bar},
%      etc.  An option (|OriginalTypewriter|) is provided to get back
%      to the former behaviour of |frenchb|.
%
%      Another probably invisible change: lowercase conversion in
%      |\up{}| is now achieved by the \LaTeX{} command |\MakeLowercase|
%      instead of \TeX's |\lowercase| command.  This prevents error
%      messages when diacritics are used inside |\up{}| (diacritics
%      should \emph{never} be used in superscripts though!).
%
% \StopEventually{}
%
%  \subsection{File frenchb.cfg}
%  \label{sec-cfg}
%
%    \file{frenchb.cfg} is now a dummy file just kept for compatibility
%    with previous versions.
%
% \iffalse
%<*cfg>
% \fi
%    \begin{macrocode}
%%%%%%%%%%%%%%%%%%%%%%%%%%%%%%%%%%%%%%%%%%%%%%%%%%%%%%%%%%%%%%%%%%%%%%
%%%%%%%%%  WARNING: THIS  FILE SHOULD  NO  LONGER  BE  USED  %%%%%%%%%
%% If you want to customise frenchb, please DO NOT hack into the code!
%% Do no put any code in this file either, please use the new command
%% \frenchbsetup{} with the proper options to customise frenchb.
%% 
%% Add \frenchbsetup{ShowOptions} to your preamble to see the list of
%% available options and/or read the documentation.
%%%%%%%%%%%%%%%%%%%%%%%%%%%%%%%%%%%%%%%%%%%%%%%%%%%%%%%%%%%%%%%%%%%%%%
%    \end{macrocode}
% \iffalse
%</cfg>
% \fi
%
%  \section{\TeX{}nical details}
%
%  \subsection{Initial setup}
%
% \changes{v2.1d}{2008/05/02}{Argument of \cs{ProvidesLanguage} changed
%     above from `french' to `frenchb' (otherwise \cs{listfiles} prints
%     no date/version information).  The real name of current language
%     (french) as to be corrected before calling \cs{LdfInit}.}
%
% \iffalse
%<*code>
% \fi
%
%    While this file was read through the option \Lopt{frenchb} we make
%    it behave as if \Lopt{french} was specified.
%    \begin{macrocode}
\def\CurrentOption{french}
%    \end{macrocode}
%
%    The macro |\LdfInit| takes care of preventing that this file is
%    loaded more than once, checking the category code of the
%    \texttt{@} sign, etc.
%
%    \begin{macrocode}
\LdfInit\CurrentOption\datefrench
%    \end{macrocode}
%
% \changes{v2.1d}{2008/05/04}{Avoid warning ``\cs{end} occurred
%   when \cs{ifx} ... incomplete'' with LaTeX-2.09.}
%
%  \begin{macro}{\ifLaTeXe}
%    No support is provided for late \LaTeX-2.09: issue a warning
%    and exit if \LaTeX-2.09 is in use. Plain is still supported.
%    \begin{macrocode}
\newif\ifLaTeXe
\let\bbl@tempa\relax
\ifx\magnification\@undefined
   \ifx\@compatibilitytrue\@undefined
     \PackageError{frenchb.ldf}
        {LaTeX-2.09 format is no longer supported.\MessageBreak
         Aborting here}
        {Please upgrade to LaTeX2e!}
     \let\bbl@tempa\endinput
   \else
     \LaTeXetrue
   \fi
\fi
\bbl@tempa
%    \end{macrocode}
%  \end{macro}
%
%    Check if hyphenation patterns for the French language have been
%    loaded in language.dat; we allow for the names `french',
%    `francais', `canadien' or `acadian'. The latter two are both
%    names used in Canada for variants of French that are in use in
%    that country.
%
%    \begin{macrocode}
\ifx\l@french\@undefined
  \ifx\l@francais\@undefined
    \ifx\l@canadien\@undefined
      \ifx\l@acadian\@undefined
        \@nopatterns{French}
        \adddialect\l@french0
      \else
        \let\l@french\l@acadian
      \fi
    \else
      \let\l@french\l@canadien
    \fi
  \else
    \let\l@french\l@francais
  \fi
\fi
%    \end{macrocode}
%    Now |\l@french| is always defined.
%
%    The internal name for the French language is |french|;
%    |francais| and |frenchb| are synonymous for |french|:
%    first let both names use the same hyphenation patterns.
%    Later we will have to set aliases for |\captionsfrench|,
%    |\datefrench|, |\extrasfrench| and |\noextrasfrench|.
%    As French uses the standard values of |\lefthyphenmin| (2)
%    and |\righthyphenmin| (3), no special setting is required here.
%
%    \begin{macrocode}
\ifx\l@francais\@undefined
  \let\l@francais\l@french
\fi
\ifx\l@frenchb\@undefined
  \let\l@frenchb\l@french
\fi
%    \end{macrocode}
%    When this language definition file was loaded for one of the
%    Canadian versions of French we need to make sure that a suitable
%    hyphenation pattern register will be found by \TeX.
%    \begin{macrocode}
\ifx\l@canadien\@undefined
  \let\l@canadien\l@french
\fi
\ifx\l@acadian\@undefined
  \let\l@acadian\l@french
\fi
%    \end{macrocode}
%
%    This language definition can be loaded for different variants of
%    the French language. The `key' \babel\ macros are only defined
%    once, using `french' as the language name, but |frenchb| and
%    |francais| are synonymous.
%    \begin{macrocode}
\def\datefrancais{\datefrench}
\def\datefrenchb{\datefrench}
\def\extrasfrancais{\extrasfrench}
\def\extrasfrenchb{\extrasfrench}
\def\noextrasfrancais{\noextrasfrench}
\def\noextrasfrenchb{\noextrasfrench}
%    \end{macrocode}
%
% \begin{macro}{\extrasfrench}
% \begin{macro}{\noextrasfrench}
%    The macro |\extrasfrench| will perform all the extra
%    definitions needed for the French language.
%    The macro |\noextrasfrench| is used to cancel the actions of
%    |\extrasfrench|.\\
%    In French, character ``apostrophe'' is a letter in expressions
%    like |l'ambulance| (French  hyphenation patterns provide entries
%    for this kind of words).  This means that the |\lccode| of
%    ``apostrophe'' has to be non null in French for proper hyphenation
%    of those expressions, and has to be reset to null when exiting
%    French.
%
%    \begin{macrocode}
\@namedef{extras\CurrentOption}{\lccode`\'=`\'}
\@namedef{noextras\CurrentOption}{\lccode`\'=0}
%    \end{macrocode}
% \end{macro}
% \end{macro}
%
%    One more thing |\extrasfrench| needs to do is to make sure that
%    |\frenchspacing| is in effect.  |\noextrasfrench| will switch
%    |\frenchspacing| off again.
%    \begin{macrocode}
  \expandafter\addto\csname extras\CurrentOption\endcsname{%
    \bbl@frenchspacing}
  \expandafter\addto\csname noextras\CurrentOption\endcsname{%
    \bbl@nonfrenchspacing}
%    \end{macrocode}
%
%  \subsection{Punctuation}
%  \label{sec-punct}
%
%    As long as no better solution is available%
%    \footnote{Lua\TeX, or pdf\TeX{} might provide alternatives in
%       the future\dots},
%    the `double punctuation' characters (|;| |!| |?| and |:|) have to
%    be made |\active| for an automatic control of the amount of space
%    to insert before them. Before doing so, we have to save the
%    standard definition of |\@makecaption| (which includes two ':')
%    to compare it later to its definition at the |\begin{document}|.
%    \begin{macrocode}
\long\def\STD@makecaption#1#2{%
  \vskip\abovecaptionskip
  \sbox\@tempboxa{#1: #2}%
  \ifdim \wd\@tempboxa >\hsize
    #1: #2\par
  \else
    \global \@minipagefalse
    \hb@xt@\hsize{\hfil\box\@tempboxa\hfil}%
  \fi
  \vskip\belowcaptionskip}%
%    \end{macrocode}
%
%    We define a new `if' |\FBpunct@active| which will be made false
%    whenever a better alternative will be available. Another `if'
%    |\FBAutoSpacePunctuation| needs to be defined now.
%    \begin{macrocode}
\newif\ifFBpunct@active          \FBpunct@activetrue
\newif\ifFBAutoSpacePunctuation  \FBAutoSpacePunctuationtrue
%    \end{macrocode}
%    The following code makes the four characters |;| |!| |?| and |:|
%    `active' and provides their definitions.
%    \begin{macrocode}
\ifFBpunct@active
  \initiate@active@char{:}
  \initiate@active@char{;}
  \initiate@active@char{!}
  \initiate@active@char{?}
%    \end{macrocode}
%    We first tune the amount of space before \texttt{;}
%    \texttt{!}  \texttt{?} and \texttt{:}.  This should only happen
%    in horizontal mode, hence the test |\ifhmode|.
%
%    In horizontal mode, if a space has been typed before `;' we
%    remove it and put an unbreakable |\thinspace| instead. If no
%    space has been typed, we add |\FDP@thinspace| which will be
%    defined, up to the user's wishes, as an automatic added
%    thin space, or as |\@empty|.
%    \begin{macrocode}
  \declare@shorthand{french}{;}{%
      \ifhmode
      \ifdim\lastskip>\z@
          \unskip\penalty\@M\thinspace
          \else
            \FDP@thinspace
        \fi
      \fi
%    \end{macrocode}
%    Now we can insert a |;| character.
%    \begin{macrocode}
      \string;}
%    \end{macrocode}
%    The next three definitions are very similar.
%    \begin{macrocode}
  \declare@shorthand{french}{!}{%
      \ifhmode
        \ifdim\lastskip>\z@
          \unskip\penalty\@M\thinspace
        \else
          \FDP@thinspace
        \fi
      \fi
      \string!}
  \declare@shorthand{french}{?}{%
      \ifhmode
        \ifdim\lastskip>\z@
          \unskip\penalty\@M\thinspace
        \else
          \FDP@thinspace
        \fi
      \fi
      \string?}
%    \end{macrocode}
%    According to the I.N. specifications, the `:' requires a normal
%    space before it, but some people prefer a |\thinspace| (just
%    like the other three). We define |\Fcolonspace| to hold the
%    required amount of space (user customisable).
%    \begin{macrocode}
  \newcommand*{\Fcolonspace}{\space}
  \declare@shorthand{french}{:}{%
      \ifhmode
        \ifdim\lastskip>\z@
          \unskip\penalty\@M\Fcolonspace
        \else
          \FDP@colonspace
        \fi
      \fi
      \string:}
%    \end{macrocode}
%
% \changes{v2.3a}{2008/10/10}{\cs{NoAutoSpaceBeforeFDP} and
%    \cs{AutoSpaceBeforeFDP} now set the flag
%    \cs{ifFBAutoSpacePunctuation} accordingly (LaTeX only).}
%
%  \begin{macro}{\AutoSpaceBeforeFDP}
%  \begin{macro}{\NoAutoSpaceBeforeFDP}
%    |\FDP@thinspace| and |\FDP@space| are defined as unbreakable
%    spaces by |\autospace@beforeFDP| or as |\@empty| by
%    |\noautospace@beforeFDP| (internal commands), user commands
%    |\AutoSpaceBeforeFDP| and |\NoAutoSpaceBeforeFDP| do the same and
%    take care of the flag |\ifFBAutoSpacePunctuation| in \LaTeX{}.
%    Set the default now for Plain (done later for \LaTeX).
%    \begin{macrocode}
  \def\autospace@beforeFDP{%
          \def\FDP@thinspace{\penalty\@M\thinspace}%
          \def\FDP@colonspace{\penalty\@M\Fcolonspace}}
  \def\noautospace@beforeFDP{\let\FDP@thinspace\@empty
                            \let\FDP@colonspace\@empty}
  \ifLaTeXe
    \def\AutoSpaceBeforeFDP{\autospace@beforeFDP
                            \FBAutoSpacePunctuationtrue}
    \def\NoAutoSpaceBeforeFDP{\noautospace@beforeFDP
                              \FBAutoSpacePunctuationfalse}
  \else
    \let\AutoSpaceBeforeFDP\autospace@beforeFDP
    \let\NoAutoSpaceBeforeFDP\noautospace@beforeFDP
    \AutoSpaceBeforeFDP
  \fi
%    \end{macrocode}
% \end{macro}
% \end{macro}
%
% \changes{v2.3a}{2008/10/10}{In LaTeX, frenchb no longer adds spaces
%     before `double punctuation' characters in computer code.
%     Suggested by Yannis Haralambous.}
%
% \changes{v2.3c}{2009/02/07}{Commands \cs{ttfamily}, \cs{rmfamily}
%    and \cs{sffamily} have to be robust.  Bug introduced in 2.3a,
%    pointed out by Manuel P\'egouri\'e-Gonnard.}
%
%    In \LaTeXe{} |\ttfamily| (and hence |\texttt|) will be redefined
%    `AtBeginDocument' as |\ttfamilyFB| so that no space
%    is added before the four |; : ! ?| characters, even if
%    |AutoSpacePunctuation| is true.  |\rmfamily| and |\sffamily| need
%    to be redefined also (|\ttfamily| is not always used inside a
%    group, its effect can be cancelled by |\rmfamily| or |\sffamily|).
%
%    These redefinitions can be canceled if necessary, for instance to
%    recompile older documents, see option |OriginalTypewriter| below.
%    \begin{macrocode}
  \ifLaTeXe
    \let\ttfamilyORI\ttfamily
    \let\rmfamilyORI\rmfamily
    \let\sffamilyORI\sffamily
    \DeclareRobustCommand\ttfamilyFB{%
         \noautospace@beforeFDP\ttfamilyORI}%
    \DeclareRobustCommand\rmfamilyFB{%
         \ifFBAutoSpacePunctuation
            \autospace@beforeFDP\rmfamilyORI
         \else
            \noautospace@beforeFDP\rmfamilyORI
         \fi}%
    \DeclareRobustCommand\sffamilyFB{%
         \ifFBAutoSpacePunctuation
            \autospace@beforeFDP\sffamilyORI
         \else
            \noautospace@beforeFDP\sffamilyORI
         \fi}%
  \fi
%    \end{macrocode}
%
%    When the active characters appear in an environment where their
%    French behaviour is not wanted they should give an `expected'
%    result. Therefore we define shorthands at system level as well.
%    \begin{macrocode}
  \declare@shorthand{system}{:}{\string:}
  \declare@shorthand{system}{!}{\string!}
  \declare@shorthand{system}{?}{\string?}
  \declare@shorthand{system}{;}{\string;}
%    \end{macrocode}
%    We specify that the French group of shorthands should be used.
%    \begin{macrocode}
  \addto\extrasfrench{%
    \languageshorthands{french}%
%    \end{macrocode}
%    These characters are `turned on' once, later their definition may
%    vary. Don't misunderstand the following code: they keep being
%    active all along the document, even when leaving French.
%    \begin{macrocode}
    \bbl@activate{:}\bbl@activate{;}%
    \bbl@activate{!}\bbl@activate{?}%
  }
  \addto\noextrasfrench{%
  \bbl@deactivate{:}\bbl@deactivate{;}%
  \bbl@deactivate{!}\bbl@deactivate{?}}
\fi
%    \end{macrocode}
%
%  \subsection{French quotation marks}
%
%  \begin{macro}{\og}
%  \begin{macro}{\fg}
%    The top macros for quotation marks will be called |\og|
%    (``\underline{o}uvrez \underline{g}uillemets'') and |\fg|
%    (``\underline{f}ermez \underline{g}uillemets'').
%    Another option for typesetting quotes in multilingual texts
%    is to use the package |csquotes.sty| and its command |\enquote|.
%
%    \begin{macrocode}
\newcommand*{\og}{\@empty}
\newcommand*{\fg}{\@empty}
%    \end{macrocode}
%  \end{macro}
%  \end{macro}
%
%  \begin{macro}{\guillemotleft}
%  \begin{macro}{\guillemotright}
%    \LaTeX{} users are supposed to use 8-bit output encodings (T1,
%    LY1,\dots) to typeset French, those who still stick to OT1 should
%    call |aeguill.sty| or a similar package. In both cases the
%    commands |\guillemotleft| and |\guillemotright| will print the
%    French opening and closing quote characters from the output font.
%    For XeLaTeX, |\guillemotleft| and |\guillemotright| are defined
%    by package \file{xunicode.sty}.
%    We will check `AtBeginDocument' that the proper output encodings
%    are in use (see end of section~\ref{sec-keyval}).
%
%    We give the following definitions for Plain users only as a (poor)
%    fall-back, they are welcome to change them for anything better.
%    \begin{macrocode}
\ifLaTeXe
\else
  \ifx\guillemotleft\@undefined
    \def\guillemotleft{\leavevmode\raise0.25ex
                       \hbox{$\scriptscriptstyle\ll$}}
  \fi
  \ifx\guillemotright\@undefined
    \def\guillemotright{\raise0.25ex
                        \hbox{$\scriptscriptstyle\gg$}}
  \fi
  \let\xspace\relax
\fi
%    \end{macrocode}
%  \end{macro}
%  \end{macro}
%
%    The next step is to provide correct spacing after |\guillemotleft|
%    and before |\guillemotright|: a space precedes and follows
%    quotation marks but no line break is allowed neither \emph{after}
%    the opening one, nor \emph{before} the closing one.
%    |\FBguill@spacing| which does the spacing, has been fine tuned by
%    Thierry Bouche.  French quotes (including spacing) are printed by
%    |\FB@og| and |\FB@fg|, the expansion of the top level commands
%    |\og| and |\og| is different in and outside French.
%    We'll try to be smart to users of David~Carlisle's |xspace|
%    package: if this package is loaded there will be no need for |{}|
%    or |\ | to get a space after |\fg|, otherwise |\xspace| will be
%    defined as |\relax| (done at the end of this file).
%
%    \begin{macrocode}
\newcommand*{\FBguill@spacing}{\penalty\@M\hskip.8\fontdimen2\font
                                            plus.3\fontdimen3\font
                                           minus.8\fontdimen4\font}
\DeclareRobustCommand*{\FB@og}{\leavevmode
                               \guillemotleft\FBguill@spacing}
\DeclareRobustCommand*{\FB@fg}{\ifdim\lastskip>\z@\unskip\fi
                               \FBguill@spacing\guillemotright\xspace}
%    \end{macrocode}
%
%    The top level definitions for French quotation marks are switched
%    on and off through the |\extrasfrench| |\noextrasfrench|
%    mechanism. Outside French, |\og| and |\fg| will typeset standard
%    English opening and closing double quotes.
%
%    \begin{macrocode}
\ifLaTeXe
  \def\bbl@frenchguillemets{\renewcommand*{\og}{\FB@og}%
                            \renewcommand*{\fg}{\FB@fg}}
  \def\bbl@nonfrenchguillemets{\renewcommand*{\og}{\textquotedblleft}%
            \renewcommand*{\fg}{\ifdim\lastskip>\z@\unskip\fi
                                   \textquotedblright}}
\else
   \def\bbl@frenchguillemets{\let\og\FB@og
                             \let\fg\FB@fg}
   \def\bbl@nonfrenchguillemets{\def\og{``}%
                     \def\fg{\ifdim\lastskip>\z@\unskip\fi ''}}
\fi
\expandafter\addto\csname extras\CurrentOption\endcsname{%
  \bbl@frenchguillemets}
\expandafter\addto\csname noextras\CurrentOption\endcsname{%
  \bbl@nonfrenchguillemets}
%    \end{macrocode}
%
%  \subsection{Date in French}
%
% \begin{macro}{\datefrench}
%    The macro |\datefrench| redefines the command |\today| to
%    produce French dates.
%
% \changes{v2.0}{2006/11/06}{2 '\cs{relax}' added in
%    \cs{today}'s definition.}
%
% \changes{v2.1a}{2008/03/25}{\cs{today} changed (correction in 2.0
%    was wrong: \cs{today} was printed without spaces in toc).}
%
%    \begin{macrocode}
\@namedef{date\CurrentOption}{%
  \def\today{{\number\day}\ifnum1=\day {\ier}\fi \space
    \ifcase\month
      \or janvier\or f\'evrier\or mars\or avril\or mai\or juin\or
      juillet\or ao\^ut\or septembre\or octobre\or novembre\or
      d\'ecembre\fi
    \space \number\year}}
%    \end{macrocode}
% \end{macro}
%
%  \subsection{Extra utilities}
%
%    Let's provide the French user with some extra utilities.
%
% \changes{v2.1a}{2008/03/24}{Command \cs{fup} added to produce
%    better superscripts than \cs{textsuperscript}.}
%
%  \begin{macro}{\up}
%
% \changes{v2.1c}{2008/04/29}{Provide a temporary definition
%    (hyperref safe) of \cs{up} in case it has to be expanded in
%    the preamble (by beamer's \cs{title} command for instance).}
%
%  \begin{macro}{\fup}
%
% \changes{v2.1b}{2008/04/02}{Command \cs{fup} changed to use
%    real superscripts from fourier v. 1.6.}
%
% \changes{v2.2a}{2008/05/08}{\cs{newif} and \cs{newdimen} moved
%    before \cs{ifLaTeXe} to avoid an error with plainTeX.}
%
% \changes{v2.3a}{2008/09/30}{\cs{lowercase} changed to
%    \cs{MakeLowercase} as the former doesn't work for non ASCII
%    characters in encodings like applemac, utf-8,\dots}
%
%    |\up| eases the typesetting of superscripts like
%    `1\textsuperscript{er}'.  Up to version 2.0 of |frenchb| |\up| was
%    just a shortcut for |\textsuperscript| in \LaTeXe, but several
%    users complained that |\textsuperscript| typesets superscripts
%    too high and too big, so we now define |\fup| as an attempt to
%    produce better looking superscripts.  |\up| is defined as |\fup|
%    but can be redefined by |\frenchbsetup{FrenchSuperscripts=false}|
%    as |\textsuperscript| for compatibility with previous versions.
%
%    When a font has built-in superscripts, the best thing to do is
%    to just use them, otherwise |\fup| has to simulate superscripts
%    by scaling and raising ordinary letters.  Scaling is done using
%    package \file{scalefnt} which will be loaded at the end of
%    \babel's loading (|frenchb| being an option of babel, it cannot
%    load a package while being read).
%
%    \begin{macrocode}
\newif\ifFB@poorman
\newdimen\FB@Mht
\ifLaTeXe
  \AtEndOfPackage{\RequirePackage{scalefnt}}
%    \end{macrocode}
%    |\FB@up@fake| holds the definition of fake superscripts.
%    The scaling ratio is 0.65, raising is computed to put the top of
%    lower case letters (like `m') just under the top  of upper case
%    letters (like `M'), precisely 12\% down.  The chosen settings
%    look correct for most fonts, but can be tuned by the end-user
%    if necessary by changing |\FBsupR| and |\FBsupS| commands.
%
%    |\FB@lc| is defined as |\MakeLowercase| to inhibit the uppercasing
%    of superscripts (this may happen in page headers with the standard
%    classes but is wrong); |\FB@lc| can be redefined to do nothing
%    by option |LowercaseSuperscripts=false| of |\frenchbsetup{}|.
%    \begin{macrocode}
  \newcommand*{\FBsupR}{-0.12}
  \newcommand*{\FBsupS}{0.65}
  \newcommand*{\FB@lc}[1]{\MakeLowercase{#1}}
  \DeclareRobustCommand*{\FB@up@fake}[1]{%
    \settoheight{\FB@Mht}{M}%
    \addtolength{\FB@Mht}{\FBsupR \FB@Mht}%
    \addtolength{\FB@Mht}{-\FBsupS ex}%
    \raisebox{\FB@Mht}{\scalefont{\FBsupS}{\FB@lc{#1}}}%
    }
%    \end{macrocode}
%    The only packages I currently know to take advantage of real
%    superscripts are a) \file{xltxtra} used in conjunction with
%    XeLaTeX and OpenType fonts having the font feature
%    'VerticalPosition=Superior' (\file{xltxtra} defines
%    |\realsuperscript| and |\fakesuperscript|) and b) \file{fourier}
%    (from version 1.6) when Expert Utopia fonts are available.
%
%    |\FB@up| checks whether the current font is a Type1 `Expert'
%    (or `Pro') font with real superscripts or not (the code works
%    currently only with \file{fourier-1.6} but could work with any
%    Expert Type1 font with built-in superscripts, see below), and
%    decides to use real or fake superscripts.
%    It works as follows: the content of |\f@family| (family name of
%    the current font) is split by |\FB@split| into two pieces, the
%    first three characters (`|fut|' for Fourier, `|ppl|' for Adobe's
%    Palatino, \dots) stored in |\FB@firstthree| and the rest stored
%    in |\FB@suffix| which is expected to be `|x|' or `|j|' for expert
%    fonts.
%    \begin{macrocode}
  \def\FB@split#1#2#3#4\@nil{\def\FB@firstthree{#1#2#3}%
                             \def\FB@suffix{#4}}
  \def\FB@x{x}
  \def\FB@j{j}
  \DeclareRobustCommand*{\FB@up}[1]{%
    \bgroup \FB@poormantrue
      \expandafter\FB@split\f@family\@nil
%    \end{macrocode}
%    Then |\FB@up| looks for a \file{.fd} file named \file{t1fut-sup.fd}
%    (Fourier) or \file{t1ppl-sup.fd} (Palatino), etc. supposed to
%    define the subfamily (|fut-sup| or |ppl-sup|, etc.) giving access
%    to the built-in superscripts.  If the \file{.fd} file is not found
%    by |\IfFileExists|, |\FB@up| falls back on fake superscripts,
%    otherwise |\FB@suffix| is checked to decide whether to use fake or
%    real superscripts.
%    \begin{macrocode}
      \edef\reserved@a{\lowercase{%
         \noexpand\IfFileExists{\f@encoding\FB@firstthree -sup.fd}}}%
      \reserved@a
        {\ifx\FB@suffix\FB@x \FB@poormanfalse\fi
         \ifx\FB@suffix\FB@j \FB@poormanfalse\fi
         \ifFB@poorman \FB@up@fake{#1}%
         \else         \FB@up@real{#1}%
         \fi}%
        {\FB@up@fake{#1}}%
    \egroup}
%    \end{macrocode}
%    |\FB@up@real| just picks up the superscripts from the subfamily
%    (and forces lowercase).
%    \begin{macrocode}
  \newcommand*{\FB@up@real}[1]{\bgroup
       \fontfamily{\FB@firstthree -sup}\selectfont \FB@lc{#1}\egroup}
%    \end{macrocode}
%    |\fup| is now defined as |\FB@up| unless |\realsuperscript| is
%    defined (occurs with XeLaTeX calling \file{xltxtra.sty}).
%    \begin{macrocode}
  \DeclareRobustCommand*{\fup}[1]{%
    \@ifundefined{realsuperscript}%
      {\FB@up{#1}}%
      {\bgroup\let\fakesuperscript\FB@up@fake
            \realsuperscript{\FB@lc{#1}}\egroup}}
%    \end{macrocode}
%    Temporary definition of |up| (redefined `AtBeginDocument').
%    \begin{macrocode}
  \newcommand*{\up}{\relax}
%    \end{macrocode}
%    Poor man's definition of |\up| for Plain. In \LaTeXe,
%    |\up| will be defined as |\fup| or |\textsuperscript| later on
%    while processing the options of |\frenchbsetup{}|.
%    \begin{macrocode}
\else
  \newcommand*{\up}[1]{\leavevmode\raise1ex\hbox{\sevenrm #1}}
\fi
%    \end{macrocode}
%  \end{macro}
%  \end{macro}
%
%  \begin{macro}{\ieme}
%  \begin{macro}{\ier}
%  \begin{macro}{\iere}
%  \begin{macro}{\iemes}
%  \begin{macro}{\iers}
%  \begin{macro}{\ieres}
%  Some handy macros for those who don't know how to abbreviate ordinals:
%    \begin{macrocode}
\def\ieme{\up{\lowercase{e}}\xspace}
\def\iemes{\up{\lowercase{es}}\xspace}
\def\ier{\up{\lowercase{er}}\xspace}
\def\iers{\up{\lowercase{ers}}\xspace}
\def\iere{\up{\lowercase{re}}\xspace}
\def\ieres{\up{\lowercase{res}}\xspace}
%    \end{macrocode}
%  \end{macro}
%  \end{macro}
%  \end{macro}
%  \end{macro}
%  \end{macro}
%  \end{macro}
%
% \changes{v2.1c}{2008/04/29}{Added commands \cs{Nos} and \cs{nos}.}
%
%  \begin{macro}{\No}
%  \begin{macro}{\no}
%  \begin{macro}{\Nos}
%  \begin{macro}{\nos}
%  \begin{macro}{\primo}
%  \begin{macro}{\fprimo)}
%    And some more macros relying on |\up| for numbering,
%    first two support macros.
%    \begin{macrocode}
\newcommand*{\FrenchEnumerate}[1]{%
                       #1\up{\lowercase{o}}\kern+.3em}
\newcommand*{\FrenchPopularEnumerate}[1]{%
                       #1\up{\lowercase{o}})\kern+.3em}
%    \end{macrocode}
%
%    Typing |\primo| should result in `$1^{\rm o}$\kern+.3em',
%    \begin{macrocode}
\def\primo{\FrenchEnumerate1}
\def\secundo{\FrenchEnumerate2}
\def\tertio{\FrenchEnumerate3}
\def\quarto{\FrenchEnumerate4}
%    \end{macrocode}
%    while typing |\fprimo)| gives `1$^{\rm o}$)\kern+.3em.
%    \begin{macrocode}
\def\fprimo){\FrenchPopularEnumerate1}
\def\fsecundo){\FrenchPopularEnumerate2}
\def\ftertio){\FrenchPopularEnumerate3}
\def\fquarto){\FrenchPopularEnumerate4}
%    \end{macrocode}
%
%    Let's provide four macros for the common abbreviations
%    of ``Num\'ero''.
%    \begin{macrocode}
\DeclareRobustCommand*{\No}{N\up{\lowercase{o}}\kern+.2em}
\DeclareRobustCommand*{\no}{n\up{\lowercase{o}}\kern+.2em}
\DeclareRobustCommand*{\Nos}{N\up{\lowercase{os}}\kern+.2em}
\DeclareRobustCommand*{\nos}{n\up{\lowercase{os}}\kern+.2em}
%    \end{macrocode}
%  \end{macro}
%  \end{macro}
%  \end{macro}
%  \end{macro}
%  \end{macro}
%  \end{macro}
%
%  \begin{macro}{\bsc}
%    As family names should be written in small capitals and never be
%    hyphenated, we provide a command (its name comes from Boxed Small
%    Caps) to input them easily.  Note that this command has changed
%    with version~2 of |frenchb|: a |\kern0pt| is used instead of |\hbox|
%    because |\hbox| would break microtype's font expansion; as a
%    (positive?) side effect, composed names (such as Dupont-Durand)
%    can now be hyphenated on explicit hyphens.
%    Usage: |Jean~\bsc{Duchemin}|.
%
% \changes{v2.0}{2006/11/06}{\cs{hbox} dropped, replaced by
%    \cs{kern0pt}.}
%
%    \begin{macrocode}
\DeclareRobustCommand*{\bsc}[1]{\leavevmode\begingroup\kern0pt
                                           \scshape #1\endgroup}
\ifLaTeXe\else\let\scshape\relax\fi
%    \end{macrocode}
%  \end{macro}
%
%    Some definitions for special characters.  We won't define |\tilde|
%    as a Text Symbol not to conflict with the macro |\tilde| for math
%    mode and use the name |\tild| instead. Note that |\boi| may
%    \emph{not} be used in math mode, its name in math mode is
%    |\backslash|.  |\degre|  can be accessed by the command |\r{}|
%    for ring accent.
%
%    \begin{macrocode}
\ifLaTeXe
  \DeclareTextSymbol{\at}{T1}{64}
  \DeclareTextSymbol{\circonflexe}{T1}{94}
  \DeclareTextSymbol{\tild}{T1}{126}
  \DeclareTextSymbolDefault{\at}{T1}
  \DeclareTextSymbolDefault{\circonflexe}{T1}
  \DeclareTextSymbolDefault{\tild}{T1}
  \DeclareRobustCommand*{\boi}{\textbackslash}
  \DeclareRobustCommand*{\degre}{\r{}}
\else
  \def\T@one{T1}
  \ifx\f@encoding\T@one
    \newcommand*{\degre}{\char6}
  \else
    \newcommand*{\degre}{\char23}
  \fi
  \newcommand*{\at}{\char64}
  \newcommand*{\circonflexe}{\char94}
  \newcommand*{\tild}{\char126}
  \newcommand*{\boi}{$\backslash$}
\fi
%    \end{macrocode}
%
%  \begin{macro}{\degres}
%    We now define a macro |\degres| for typesetting the abbreviation
%    for `degrees' (as in `degrees Celsius'). As the bounding box of
%    the character `degree' has \emph{very} different widths in CM/EC
%    and PostScript fonts, we fix the width of the bounding box of
%    |\degres| to 0.3\,em, this lets the symbol `degree' stick to the
%    preceding (e.g., |45\degres|) or following character
%    (e.g., |20~\degres C|).
%
%    If the \TeX{} Companion fonts are available (\file{textcomp.sty}),
%    we pick up |\textdegree| from them instead of using emulating
%    `degrees' from the |\r{}| accent. Otherwise we overwrite the
%    (poor) definition of |\textdegree| given in \file{latin1.def},
%    \file{applemac.def} etc. (called by  \file{inputenc.sty}) by
%    our definition of |\degres|. We also advice the user (once only)
%    to use TS1-encoding.
%
% \changes{v2.1c}{2008/04/29}{Provide a temporary definition (hyperref
%    safe) of \cs{degres} in case it has to be expanded in the preamble
%    (by beamer's \cs{title} command for instance).}
%
%    \begin{macrocode}
\ifLaTeXe
  \newcommand*{\degres}{\degre}
  \def\Warning@degree@TSone{%
        \PackageWarning{frenchb.ldf}{%
           Degrees would look better in TS1-encoding:
           \MessageBreak add \protect
           \usepackage{textcomp} to the preamble.
           \MessageBreak Degrees used}}
  \AtBeginDocument{\expandafter\ifx\csname M@TS1\endcsname\relax
                     \DeclareRobustCommand*{\degres}{%
                       \leavevmode\hbox to 0.3em{\hss\degre\hss}%
                       \Warning@degree@TSone
                       \global\let\Warning@degree@TSone\relax}%
                      \let\textdegree\degres
                   \else
                     \DeclareRobustCommand*{\degres}{%
                         \hbox{\UseTextSymbol{TS1}{\textdegree}}}%
                   \fi}
\else
  \newcommand*{\degres}{%
    \leavevmode\hbox to 0.3em{\hss\degre\hss}}
\fi
%    \end{macrocode}
%  \end{macro}
%
%  \subsection{Formatting numbers}
%  \label{sec-numbers}
%
%  \begin{macro}{\DecimalMathComma}
%  \begin{macro}{\StandardMathComma}
%    As mentioned in the \TeX{}book p.~134, the comma is of type
%    |\mathpunct| in math mode: it is automatically followed by a
%    space. This is convenient in lists and intervals but
%    unpleasant when the comma is used as a decimal separator
%    in French: it has to be entered as |{,}|.
%    |\DecimalMathComma| makes the comma be an ordinary character
%    (of type |\mathord|) in French \emph{only} (no space added);
%    |\StandardMathComma| switches back to the standard behaviour
%    of the comma.
%    \begin{macrocode}
\newcount\std@mcc
\newcount\dec@mcc
\std@mcc=\mathcode`\,
\dec@mcc=\std@mcc
\@tempcnta=\std@mcc
\divide\@tempcnta by "1000
\multiply\@tempcnta by "1000
\advance\dec@mcc by -\@tempcnta
\newcommand*{\DecimalMathComma}{\iflanguage{french}%
                                 {\mathcode`\,=\dec@mcc}{}%
              \addto\extrasfrench{\mathcode`\,=\dec@mcc}}
\newcommand*{\StandardMathComma}{\mathcode`\,=\std@mcc
             \addto\extrasfrench{\mathcode`\,=\std@mcc}}
\expandafter\addto\csname noextras\CurrentOption\endcsname{%
   \mathcode`\,=\std@mcc}
%    \end{macrocode}
%  \end{macro}
%  \end{macro}
%
%  \begin{macro}{\nombre}
%
% \changes{v2.0}{2006/11/06}{\cs{nombre} requires now numprint.sty.}
%
%    The command |\nombre| is now borrowed from |numprint.sty| for
%    \LaTeXe.  There is no point to maintain the former tricky code
%    when a package is dedicated to do the same job and more.
%    For Plain based formats, |\nombre| no longer formats numbers,
%    it prints them as is and issues a warning about the change.
%
%    Fake command |\nombre| for Plain based formats, warning users of
%    |frenchb| v.1.x. of the change.
%    \begin{macrocode}
\newcommand*{\nombre}[1]{{#1}\message{%
     *** \noexpand\nombre no longer formats numbers\string! ***}}%
%    \end{macrocode}
%  \end{macro}
%
%    The next definitions only make sense for \LaTeXe. Let's cleanup
%    and exit if the format in Plain based.
%
%    \begin{macrocode}
\let\FBstop@here\relax
\def\FBclean@on@exit{\let\ifLaTeXe\@undefined
                     \let\LaTeXetrue\@undefined
                     \let\LaTeXefalse\@undefined}
\ifx\magnification\@undefined
\else
   \def\FBstop@here{\let\STD@makecaption\relax
                    \FBclean@on@exit
                    \ldf@quit\CurrentOption\endinput}
\fi
\FBstop@here
%    \end{macrocode}
%
%    What follows now is for \LaTeXe{} \emph{only}.
%    We redefine |\nombre| for \LaTeXe. A warning is issued
%    at the first call of |\nombre| if |\numprint| is not
%    defined, suggesting what to do.  The package |numprint|
%    is \emph{not} loaded automatically by |frenchb| because of
%    possible options conflict.
%
%    \begin{macrocode}
\renewcommand*{\nombre}[1]{\Warning@nombre\numprint{#1}}
\newcommand*{\Warning@nombre}{%
   \@ifundefined{numprint}%
      {\PackageWarning{frenchb.ldf}{%
         \protect\nombre\space now relies on package numprint.sty,
         \MessageBreak add \protect
         \usepackage[autolanguage]{numprint}\MessageBreak
         to your preamble *after* loading babel, \MessageBreak
         see file numprint.pdf for other options.\MessageBreak
         \protect\nombre\space called}%
       \global\let\Warning@nombre\relax
       \global\let\numprint\relax
      }{}%
}
%    \end{macrocode}
%
% \changes{v2.0c}{2007/06/25}{There is no need to define here
%    numprint's command \cs{npstylefrench}, it will be redefined
%    `AtBeginDocument' by \cs{FBprocess@options}.}
%
% \changes{v2.0c}{2007/06/25}{\cs{ThinSpaceInFrenchNumbers} added
%     for compatibility with frenchb-1.x.}
%
%    \begin{macrocode}
\newcommand*{\ThinSpaceInFrenchNumbers}{%
   \PackageWarning{frenchb.ldf}{%
         Type \protect\frenchbsetup{ThinSpaceInFrenchNumbers}
         \MessageBreak Command \protect\ThinSpaceInFrenchNumbers\space
         is no longer\MessageBreak  defined in frenchb v.2,}}
%    \end{macrocode}
%
%  \subsection{Caption names}
%
%    The next step consists of defining the French equivalents for
%    the \LaTeX{} caption names.
%
% \begin{macro}{\captionsfrench}
%    Let's first define  |\captionsfrench| which sets all strings used
%    in the four standard document classes provided with \LaTeX.
%
% \changes{v2.0}{2006/11/06}{`Fig.' changed to `Figure' and
%     `Tab.' to `Table'.}
%
% \changes{v2.0}{2006/12/15}{Set \cs{CaptionSeparator} in
%     \cs{extrasfrench} now instead of \cs{captionsfrench}
%     because it has to be reset when leaving French.}
%
%    \begin{macrocode}
\@namedef{captions\CurrentOption}{%
   \def\refname{R\'ef\'erences}%
   \def\abstractname{R\'esum\'e}%
   \def\bibname{Bibliographie}%
   \def\prefacename{Pr\'eface}%
   \def\chaptername{Chapitre}%
   \def\appendixname{Annexe}%
   \def\contentsname{Table des mati\`eres}%
   \def\listfigurename{Table des figures}%
   \def\listtablename{Liste des tableaux}%
   \def\indexname{Index}%
   \def\figurename{{\scshape Figure}}%
   \def\tablename{{\scshape Table}}%
%    \end{macrocode}
%   ``Premi\`ere partie'' instead of ``Part I''.
%    \begin{macrocode}
   \def\partname{\protect\@Fpt partie}%
   \def\@Fpt{{\ifcase\value{part}\or Premi\`ere\or Deuxi\`eme\or
   Troisi\`eme\or Quatri\`eme\or Cinqui\`eme\or Sixi\`eme\or
   Septi\`eme\or Huiti\`eme\or Neuvi\`eme\or Dixi\`eme\or Onzi\`eme\or
   Douzi\`eme\or Treizi\`eme\or Quatorzi\`eme\or Quinzi\`eme\or
   Seizi\`eme\or Dix-septi\`eme\or Dix-huiti\`eme\or Dix-neuvi\`eme\or
   Vingti\`eme\fi}\space\def\thepart{}}%
   \def\pagename{page}%
   \def\seename{{\emph{voir}}}%
   \def\alsoname{{\emph{voir aussi}}}%
   \def\enclname{P.~J. }%
   \def\ccname{Copie \`a }%
   \def\headtoname{}%
   \def\proofname{D\'emonstration}%
   \def\glossaryname{Glossaire}%
   }
%    \end{macrocode}
% \end{macro}
%
%    As some users who choose |frenchb| or |francais| as option of
%    \babel, might customise |\captionsfrenchb| or |\captionsfrancais|
%    in the preamble, we merge their changes at the |\begin{document}|
%    when they do so.
%    The other variants of French (canadien, acadian) are defined by
%    checking if the relevant option was used and then adding one extra
%    level of expansion.
%
%    \begin{macrocode}
\AtBeginDocument{\let\captions@French\captionsfrench
                 \@ifundefined{captionsfrenchb}%
                    {\let\captions@Frenchb\relax}%
                    {\let\captions@Frenchb\captionsfrenchb}%
                 \@ifundefined{captionsfrancais}%
                    {\let\captions@Francais\relax}%
                    {\let\captions@Francais\captionsfrancais}%
                 \def\captionsfrench{\captions@French
                        \captions@Francais\captions@Frenchb}%
                 \def\captionsfrancais{\captionsfrench}%
                 \def\captionsfrenchb{\captionsfrench}%
                 \iflanguage{french}{\captionsfrench}{}%
                }
\@ifpackagewith{babel}{canadien}{%
  \def\captionscanadien{\captionsfrench}%
  \def\datecanadien{\datefrench}%
  \def\extrascanadien{\extrasfrench}%
  \def\noextrascanadien{\noextrasfrench}%
  }{}
\@ifpackagewith{babel}{acadian}{%
  \def\captionsacadian{\captionsfrench}%
  \def\dateacadian{\datefrench}%
  \def\extrasacadian{\extrasfrench}%
  \def\noextrasacadian{\noextrasfrench}%
  }{}
%    \end{macrocode}
%
% \begin{macro}{\CaptionSeparator}
%    Let's consider now captions in figures and tables.
%    In French, captions in figures and tables should be printed with
%    endash (`--') instead of the standard `:'.
%
%    The standard definition of |\@makecaption| (e.g., the one provided
%    in article.cls, report.cls, book.cls which is frozen for \LaTeXe{}
%    according to Frank Mittelbach), has been saved in
%    |\STD@makecaption| before making `:' active
%    (see section~\ref{sec-punct}). `AtBeginDocument' we compare it to
%    its current definition (some classes like koma-script classes,
%    AMS classes, ua-thesis.cls\dots change it).
%    If they are identical, |frenchb| just adds a hook called
%    |\CaptionSeparator| to |\@makecaption|, |\CaptionSeparator|
%    defaults to `: ' as in the standard |\@makecaption|, and will be
%    changed to ` -- ' in French.
%    If the definitions differ, |frenchb| doesn't overwrite the changes,
%    but prints a message in the .log file.
%
%    \begin{macrocode}
\def\CaptionSeparator{\string:\space}
\long\def\FB@makecaption#1#2{%
  \vskip\abovecaptionskip
  \sbox\@tempboxa{#1\CaptionSeparator #2}%
  \ifdim \wd\@tempboxa >\hsize
    #1\CaptionSeparator #2\par
  \else
    \global \@minipagefalse
    \hb@xt@\hsize{\hfil\box\@tempboxa\hfil}%
  \fi
  \vskip\belowcaptionskip}
\AtBeginDocument{%
  \ifx\@makecaption\STD@makecaption
      \global\let\@makecaption\FB@makecaption
  \else
    \@ifundefined{@makecaption}{}%
       {\PackageWarning{frenchb.ldf}%
        {The definition of \protect\@makecaption\space
         has been changed,\MessageBreak
         frenchb will NOT customise it;\MessageBreak reported}%
       }%
  \fi
  \let\FB@makecaption\relax
  \let\STD@makecaption\relax
}
\expandafter\addto\csname extras\CurrentOption\endcsname{%
   \def\CaptionSeparator{\space\textendash\space}}
\expandafter\addto\csname noextras\CurrentOption\endcsname{%
    \def\CaptionSeparator{\string:\space}}
%    \end{macrocode}
% \end{macro}
%
%  \subsection{French lists}
%  \label{sec-lists}
%
%  \begin{macro}{\listFB}
%  \begin{macro}{\listORI}
%    Vertical spacing in general lists should be shorter in French
%    texts than the defaults provided by \LaTeX.
%    Note that the easy way, just changing values of vertical spacing
%    parameters when entering French and restoring them to their
%    defaults on exit would not work; as most lists are based on
%    |\list| we will define a variant of |\list| (|\listFB|) to
%    be used in French.
%
%    The amount of vertical space before and after a list is given by
%    |\topsep| + |\parskip| (+ |\partopsep| if the list starts a new
%    paragraph). IMHO, |\parskip| should be added \emph{only} when
%    the list starts a new paragraph, so I subtract |\parskip| from
%    |\topsep| and add it back to |\partopsep|; this will normally
%    make no difference because |\parskip|'s default value is 0pt, but
%    will be noticeable when |\parskip| is \emph{not} null.
%
%    |\endlist| is not redefined, but |\endlistORI| is provided for
%    the users who prefer to define their own lists from the original
%    command, they can code: |\begin{listORI}{}{} \end{listORI}|.
%    \begin{macrocode}
\let\listORI\list
\let\endlistORI\endlist
\def\FB@listsettings{%
      \setlength{\itemsep}{0.4ex plus 0.2ex minus 0.2ex}%
      \setlength{\parsep}{0.4ex plus 0.2ex minus 0.2ex}%
      \setlength{\topsep}{0.8ex plus 0.4ex minus 0.4ex}%
      \setlength{\partopsep}{0.4ex plus 0.2ex minus 0.2ex}%
%    \end{macrocode}
%    |\parskip| is of type `skip', its mean value only (\emph{not
%    the glue}) should be subtracted from |\topsep| and added to
%    |\partopsep|, so convert |\parskip| to a `dimen' using
%    |\@tempdima|.
%    \begin{macrocode}
      \@tempdima=\parskip
      \addtolength{\topsep}{-\@tempdima}%
      \addtolength{\partopsep}{\@tempdima}}%
\def\listFB#1#2{\listORI{#1}{\FB@listsettings #2}}%
\let\endlistFB\endlist
%    \end{macrocode}
%  \end{macro}
%  \end{macro}
%
%  \begin{macro}{\itemizeFB}
%  \begin{macro}{\itemizeORI}
%  \begin{macro}{\bbl@frenchlabelitems}
%  \begin{macro}{\bbl@nonfrenchlabelitems}
%    Let's now consider French itemize lists.  They differ from those
%    provided by the standard \LaTeXe{} classes:
%    \begin{itemize}
%      \item vertical spacing between items, before and after
%         the list, should be \emph{null} with \emph{no glue} added;
%      \item the item labels of a first level list should be vertically
%          aligned on the paragraph's first character (i.e. at
%          |\parindent| from the left margin);
%      \item the `\textbullet' is never used in French itemize-lists,
%          a long dash `--' is preferred for all levels. The item label
%          used in French is stored in |\FrenchLabelItem}|, it defaults
%          to `--' and can be changed using |\frenchbsetup{}| (see
%          section~\ref{sec-keyval}).
%    \end{itemize}
%
%    \begin{macrocode}
\newcommand*{\FrenchLabelItem}{\textendash}
\newcommand*{\Frlabelitemi}{\FrenchLabelItem}
\newcommand*{\Frlabelitemii}{\FrenchLabelItem}
\newcommand*{\Frlabelitemiii}{\FrenchLabelItem}
\newcommand*{\Frlabelitemiv}{\FrenchLabelItem}
%    \end{macrocode}
%    |\bbl@frenchlabelitems| saves current itemize labels and changes
%    them to their value in French. This code should never be executed
%    twice in a row, so we need a new flag that will be set and reset
%    by |\bbl@nonfrenchlabelitems| and |\bbl@frenchlabelitems|.
%    \begin{macrocode}
\newif\ifFB@enterFrench  \FB@enterFrenchtrue
\def\bbl@frenchlabelitems{%
  \ifFB@enterFrench
    \let\@ltiORI\labelitemi
    \let\@ltiiORI\labelitemii
    \let\@ltiiiORI\labelitemiii
    \let\@ltivORI\labelitemiv
    \let\labelitemi\Frlabelitemi
    \let\labelitemii\Frlabelitemii
    \let\labelitemiii\Frlabelitemiii
    \let\labelitemiv\Frlabelitemiv
    \FB@enterFrenchfalse
  \fi
}
\let\itemizeORI\itemize
\let\enditemizeORI\enditemize
\let\enditemizeFB\enditemize
\def\itemizeFB{%
    \ifnum \@itemdepth >\thr@@\@toodeep\else
      \advance\@itemdepth\@ne
      \edef\@itemitem{labelitem\romannumeral\the\@itemdepth}%
      \expandafter
      \listORI
      \csname\@itemitem\endcsname
      {\settowidth{\labelwidth}{\csname\@itemitem\endcsname}%
       \setlength{\leftmargin}{\labelwidth}%
       \addtolength{\leftmargin}{\labelsep}%
       \ifnum\@listdepth=0
         \setlength{\itemindent}{\parindent}%
       \else
         \addtolength{\leftmargin}{\parindent}%
       \fi
       \setlength{\itemsep}{\z@}%
       \setlength{\parsep}{\z@}%
       \setlength{\topsep}{\z@}%
       \setlength{\partopsep}{\z@}%
%    \end{macrocode}
%    |\parskip| is of type `skip', its mean value only (\emph{not
%    the glue}) should be subtracted from |\topsep| and added to
%    |\partopsep|, so convert |\parskip| to a `dimen' using
%    |\@tempdima|.
%    \begin{macrocode}
       \@tempdima=\parskip
       \addtolength{\topsep}{-\@tempdima}%
       \addtolength{\partopsep}{\@tempdima}}%
    \fi}
%    \end{macrocode}
%    The user's changes in labelitems are saved when leaving French for
%    further use when switching back to French.  This code should never
%    be executed twice in a row (toggle with |\bbl@frenchlabelitems|).
%    \begin{macrocode}
\def\bbl@nonfrenchlabelitems{%
  \ifFB@enterFrench
  \else
      \let\Frlabelitemi\labelitemi
      \let\Frlabelitemii\labelitemii
      \let\Frlabelitemiii\labelitemiii
      \let\Frlabelitemiv\labelitemiv
      \let\labelitemi\@ltiORI
      \let\labelitemii\@ltiiORI
      \let\labelitemiii\@ltiiiORI
      \let\labelitemiv\@ltivORI
      \FB@enterFrenchtrue
  \fi
}
%    \end{macrocode}
%  \end{macro}
%  \end{macro}
%  \end{macro}
%  \end{macro}
%
%  \subsection{French indentation of sections}
%  \label{sec-indent}
%
%  \begin{macro}{\bbl@frenchindent}
%  \begin{macro}{\bbl@nonfrenchindent}
%    In French the first paragraph of each section should be indented,
%    this is another difference with US-English. This is controlled by
%    the flag |\if@afterindent|.
%
% \changes{v2.3d}{2009/03/16}{Bug correction: previous versions of
%    frenchb set the flag \cs{if@afterindent} to false outside
%    French which is correct for English but wrong for some languages
%    like Spanish.  Pointed out by Juan Jos\'e Torrens.}
%
%    We need to save the value of the flag |\if@afterindent|
%    `AtBeginDocument' before eventually changing its value.
%
%    \begin{macrocode}
\AtBeginDocument{\ifx\@afterindentfalse\@afterindenttrue
                       \let\@aifORI\@afterindenttrue
                 \else \let\@aifORI\@afterindentfalse
                 \fi
}
\def\bbl@frenchindent{\let\@afterindentfalse\@afterindenttrue
                      \@afterindenttrue}
\def\bbl@nonfrenchindent{\let\@afterindentfalse\@aifORI
                         \@afterindentfalse}
%    \end{macrocode}
%  \end{macro}
%  \end{macro}
%
%  \subsection{Formatting footnotes}
%  \label{sec-footnotes}
%
% \changes{v2.0}{2006/11/06}{Footnotes are now printed
%     by default `\`a la fran\c caise' for the whole document.}
%
% \changes{v2.0b}{2007/04/18}{Footnotes: Just do nothing
%    (except warning) when the bigfoot package is loaded.}
%
%    The |bigfoot| package deeply changes the way footnotes are
%    handled. When |bigfoot| is loaded, we just warn the user that
%    |frenchb| will drop the customisation of footnotes.
%
%    The layout of footnotes is controlled by two flags
%    |\ifFBAutoSpaceFootnotes| and |\ifFBFrenchFootnotes| which are
%    set by options of |\frenchbsetup{}| (see section~\ref{sec-keyval}).
%    Notice that the layout of footnotes \emph{does not depend} on the
%    current language (just think of two footnotes on the same page
%    looking different because one was called in a French part, the
%    other one in English!).
%
%    When |\ifFBAutoSpaceFootnotes| is true, |\@footnotemark| (whose
%    definition is saved at the |\begin{document}| in order to include
%    any customisation that packages might have done) is redefined to
%    add a thin space before the number or symbol calling a footnote
%    (any space typed in is removed first).  This has no effect on
%    the layout of the footnote itself.
%
%    \begin{macrocode}
\AtBeginDocument{\@ifpackageloaded{bigfoot}%
                   {\PackageWarning{frenchb.ldf}%
                     {bigfoot package in use.\MessageBreak
                      frenchb will NOT customise footnotes;\MessageBreak
                      reported}}%
                   {\let\@footnotemarkORI\@footnotemark
                    \def\@footnotemarkFB{\leavevmode\unskip\unkern
                                         \,\@footnotemarkORI}%
                    \ifFBAutoSpaceFootnotes
                      \let\@footnotemark\@footnotemarkFB
                    \fi}%
                }
%    \end{macrocode}
%
%    We then define |\@makefntextFB|, a variant of |\@makefntext|
%    which is responsible for the layout of footnotes, to match the
%    specifications of the French `Imprimerie Nationale':  footnotes
%    will be indented by |\parindentFFN|, numbers (if any) typeset on
%    the baseline (instead of superscripts) and followed by a dot
%    and an half quad space. Whenever symbols are used to number
%    footnotes (as in |\thanks| for instance), we switch back to the
%    standard layout (the French layout of footnotes is meant for
%    footnotes numbered by Arabic or Roman digits).
%
% \changes{v2.0}{2006/11/06}{\cs{parindentFFN} not changed if
%    already defined (required by JA for cah-gut.cls).}
%
% \changes{v2.3b}{2008/12/06}{New commands \cs{dotFFN} and
%    \cs{kernFFN} for more flexibility (suggested by JA).}
%
%    The value of |\parindentFFN| will be redefined at the
%    |\begin{document}|, as the maximum of |\parindent| and 1.5em
%    \emph{unless} it has been set in the preamble (the weird value
%    10in is just for testing whether |\parindentFFN| has been set
%    or not).
%
%    \begin{macrocode}
\newcommand*{\dotFFN}{.}
\newcommand*{\kernFFN}{\kern .5em}
\newdimen\parindentFFN
\parindentFFN=10in
\def\ftnISsymbol{\@fnsymbol\c@footnote}
\long\def\@makefntextFB#1{\ifx\thefootnote\ftnISsymbol
                            \@makefntextORI{#1}%
                          \else
                            \parindent=\parindentFFN
                            \rule\z@\footnotesep
                            \setbox\@tempboxa\hbox{\@thefnmark}%
                            \ifdim\wd\@tempboxa>\z@
                              \llap{\@thefnmark}\dotFFN\kernFFN
                            \fi #1
                          \fi}%
%    \end{macrocode}
%
%    We save the standard definition of |\@makefntext| at the
%    |\begin{document}|, and then redefine |\@makefntext| according to
%    the value of flag |\ifFBFrenchFootnotes| (true or false).
%
%    \begin{macrocode}
\AtBeginDocument{\@ifpackageloaded{bigfoot}{}%
                  {\ifdim\parindentFFN<10in
                   \else
                      \parindentFFN=\parindent
                      \ifdim\parindentFFN<1.5em\parindentFFN=1.5em\fi
                   \fi
                   \let\@makefntextORI\@makefntext
                   \long\def\@makefntext#1{%
                      \ifFBFrenchFootnotes
                         \@makefntextFB{#1}%
                      \else
                         \@makefntextORI{#1}%
                      \fi}%
                  }%
                }
%    \end{macrocode}
%
%    For compatibility reasons, we provide definitions for the commands
%    dealing with the layout of footnotes in |frenchb| version~1.6.
%    |\frenchbsetup{}| (see in section \ref{sec-keyval}) should be
%    preferred for setting these options.  |\StandardFootnotes| may
%    still be used locally (in minipages for instance), that's why the
%    test |\ifFBFrenchFootnotes| is done inside |\@makefntext|.
%    \begin{macrocode}
\newcommand*{\AddThinSpaceBeforeFootnotes}{\FBAutoSpaceFootnotestrue}
\newcommand*{\FrenchFootnotes}{\FBFrenchFootnotestrue}
\newcommand*{\StandardFootnotes}{\FBFrenchFootnotesfalse}
%    \end{macrocode}
%
%  \subsection{Global layout}
%  \label{sec-global}
%
%    In multilingual documents, some typographic rules must depend
%    on the current language (e.g., hyphenation, typesetting of
%    numbers, spacing before double punctuation\dots), others should,
%    IMHO, be kept global to the document: especially the layout of
%    lists (see~\ref{sec-lists}) and footnotes
%    (see~\ref{sec-footnotes}), and the indentation of the first
%    paragraph of sections (see~\ref{sec-indent}).
%
%    From version 2.2 on, if |frenchb| is \babel's ``main language''
%    (i.e. last language option at \babel's loading), |frenchb|
%    customises the layout (i.e. lists, indentation of the first
%    paragraphs of sections and footnotes) in the whole document
%    regardless the current language.   On the other hand, if |frenchb|
%    is \emph{not} \babel's ``main language'', it leaves the layout
%    unchanged both in French and in other languages.
%
%  \begin{macro}{\FrenchLayout}
%  \begin{macro}{\StandardLayout}
%    The former commands |\FrenchLayout| and |\StandardLayout| are kept
%    for compatibility reasons but should no longer be used.
%
% \changes{v2.0g}{2008/03/23}{Flag \cs{ifFBStandardLayout} not checked
%     by \cs{FBprocess@options}, low-level flags have to be set
%     one by one.}
%
%    \begin{macrocode}
\newcommand*{\FrenchLayout}{%
    \FBGlobalLayoutFrenchtrue
    \PackageWarning{frenchb.ldf}%
    {\protect\FrenchLayout\space is obsolete.  Please use\MessageBreak
     \protect\frenchbsetup{GlobalLayoutFrench} instead.}%
}
\newcommand*{\StandardLayout}{%
  \FBReduceListSpacingfalse
  \FBCompactItemizefalse
  \FBStandardItemLabelstrue
  \FBIndentFirstfalse
  \FBFrenchFootnotesfalse
  \FBAutoSpaceFootnotesfalse
  \PackageWarning{frenchb.ldf}%
    {\protect\StandardLayout\space is obsolete.  Please use\MessageBreak
    \protect\frenchbsetup{StandardLayout} instead.}%
}
\@onlypreamble\FrenchLayout
\@onlypreamble\StandardLayout
%    \end{macrocode}
%  \end{macro}
%  \end{macro}
%
%  \subsection{Dots\dots}
%  \label{sec-dots}
%
%  \begin{macro}{\FBtextellipsis}
%    \LaTeXe's standard definition of |\dots| in text-mode is
%    |\textellipsis| which includes a |\kern| at the end;
%    this space is not wanted in some cases (before a closing brace
%    for instance) and |\kern| breaks hyphenation of the next word.
%    We define |\FBtextellipsis| for French (in \LaTeXe{} only).
%
%    The |\if| construction in the \LaTeXe{} definition of |\dots|
%    doesn't allow the use of |xspace| (|xspace| is always followed
%    by a |\fi|), so we use the AMS-\LaTeX{} construction of |\dots|;
%    this has to be done `AtBeginDocument' not to be overwritten
%    when \file{amsmath.sty} is loaded after \babel.
%
% \changes{v2.0}{2006/11/06}{Added special case for LY1 encoding,
%    see  bug report from Bruno Voisin (2004/05/18).}
%
%    LY1 has a ready made character for |\textellipsis|, it should be
%    used in French too (pointed out by Bruno Voisin).
%
%    \begin{macrocode}
\DeclareTextSymbol{\FBtextellipsis}{LY1}{133}
\DeclareTextCommandDefault{\FBtextellipsis}{%
    .\kern\fontdimen3\font.\kern\fontdimen3\font.\xspace}
%    \end{macrocode}
%    |\Mdots@| and |\Tdots@ORI| hold the definitions of |\dots| in
%    Math and Text mode. They default to those of amsmath-2.0, and
%    will revert to standard \LaTeX{} definitions `AtBeginDocument',
%    if amsmath has not been loaded. |\Mdots@| doesn't change when
%    switching from/to French, while |\Tdots@| is |\FBtextellipsis|
%    in French and |\Tdots@ORI| otherwise.
%    \begin{macrocode}
\newcommand*{\Tdots@ORI}{\@xp\textellipsis}
\newcommand*{\Tdots@}{\Tdots@ORI}
\newcommand*{\Mdots@}{\@xp\mdots@}
\AtBeginDocument{\DeclareRobustCommand*{\dots}{\relax
                 \csname\ifmmode M\else T\fi dots@\endcsname}%
                 \@ifundefined{@xp}{\let\@xp\relax}{}%
                 \@ifundefined{mdots@}{\let\Tdots@ORI\textellipsis
                                       \let\Mdots@\mathellipsis}{}}
\def\bbl@frenchdots{\let\Tdots@\FBtextellipsis}
\def\bbl@nonfrenchdots{\let\Tdots@\Tdots@ORI}
\expandafter\addto\csname extras\CurrentOption\endcsname{%
    \bbl@frenchdots}
\expandafter\addto\csname noextras\CurrentOption\endcsname{%
    \bbl@nonfrenchdots}
%    \end{macrocode}
%  \end{macro}
%
%  \subsection{Setup options: keyval stuff}
%  \label{sec-keyval}
%
% \changes{v2.0}{2006/11/06}{New command \cs{frenchbsetup} added
%     for global customisation.}
%
% \changes{v2.0c}{2007/06/25}{Option ThinSpaceInFrenchNumbers added.}
%
% \changes{v2.0d}{2007/07/15}{Options og and fg changed: limit
%     the definition to French so that quote characters can be used
%     in German.}
%
% \changes{v2.0e}{2007/10/05}{New option: StandardLists.}
%
% \changes{v2.0f}{2008/03/23}{Two typos corrected in
%    option StandardLists: [false] $\to$ [true] and
%    StandardLayout $\to$ StandardLists.}
%
% \changes{v2.0f}{2008/03/23}{StandardLayout option had no
%     effect on lists.  Test moved to \cs{FBprocess@options}.}
%
% \changes{v2.0g}{2008/03/23}{Revert previous change to
%     StandardLayout. This option must set the three flags
%     \cs{FBReduceListSpacingfalse}, \cs{FBCompactItemizefalse},
%     and \cs{FBStandardItemLabeltrue} instead of
%     \cs{FBStandardListstrue}, so that later options can still
%     change their value before executing \cs{FBprocess@options}.
%     Same thing for option StandardLists.}
%
% \changes{v2.1a}{2008/03/24}{New option: FrenchSuperscripts
%     to define \cs{up} as \cs{fup} or as \cs{textsuperscript}.}
%
% \changes{v2.1a}{2008/03/30}{New option: LowercaseSuperscripts.}
%
% \changes{v2.2a}{2008/05/08}{The global layout of the document is
%     no longer changed when frenchb is not the last option of babel
%     (\cs{bbl@main@language}). Suggested by Ulrike Fischer.}
%
% \changes{v2.2a}{2008/05/08}{Values of flags
%     \cs{ifFBReduceListSpacing}, \cs{ifFBCompactItemize},
%     \cs{ifFBStandardItemLabels}, \cs{ifFBIndentFirst},
%     \cs{ifFBFrenchFootnotes}, \cs{ifFBAutoSpaceFootnotes} changed:
%     default now means `StandardLayout', it will be changed to
%     `FrenchLayout' AtEndOfPackage only if french is
%     \cs{bbl@main@language}.}
%
% \changes{v2.2a}{2008/05/08}{When frenchb is babel's last option,
%     French becomes the document's main language, so
%     GlobalLayoutFrench applies.}
%
% \changes{v2.3a}{2008/10/10}{New option: OriginalTypewriter. Now
%    frenchb switches to \cs{noautospace@beforeFDP} when a tt-font is
%    in use.  When OriginalTypewriter is set to true, frenchb behaves
%    as in pre-2.3 versions.}
%
%    We first define a collection of conditionals with their defaults
%    (true or false).
%
%    \begin{macrocode}
\newif\ifFBStandardLayout           \FBStandardLayouttrue
\newif\ifFBGlobalLayoutFrench       \FBGlobalLayoutFrenchfalse
\newif\ifFBReduceListSpacing        \FBReduceListSpacingfalse
\newif\ifFBCompactItemize           \FBCompactItemizefalse
\newif\ifFBStandardItemLabels       \FBStandardItemLabelstrue
\newif\ifFBStandardLists            \FBStandardListstrue
\newif\ifFBIndentFirst              \FBIndentFirstfalse
\newif\ifFBFrenchFootnotes          \FBFrenchFootnotesfalse
\newif\ifFBAutoSpaceFootnotes       \FBAutoSpaceFootnotesfalse
\newif\ifFBOriginalTypewriter       \FBOriginalTypewriterfalse
\newif\ifFBThinColonSpace           \FBThinColonSpacefalse
\newif\ifFBThinSpaceInFrenchNumbers \FBThinSpaceInFrenchNumbersfalse
\newif\ifFBFrenchSuperscripts       \FBFrenchSuperscriptstrue
\newif\ifFBLowercaseSuperscripts    \FBLowercaseSuperscriptstrue
\newif\ifFBPartNameFull             \FBPartNameFulltrue
\newif\ifFBShowOptions              \FBShowOptionsfalse
%    \end{macrocode}
%
%    The defaults values of these flags have been set so that |frenchb|
%    does not change anything regarding the global layout.
%    |\bbl@main@language| (set by the last option of babel) controls
%    the global layout of the document.  We check the current language
%    `AtEndOfPackage' (it is |\bbl@main@language|); if it is French,
%    the values of some flags have to be changed to ensure a French
%    looking layout for the whole document (even in parts written in
%    languages other than French); the end-user will then be able to
%    customise the values of all these flags with |\frenchbsetup{}|.
%    \begin{macrocode}
\AtEndOfPackage{%
   \iflanguage{french}{\FBReduceListSpacingtrue
                       \FBCompactItemizetrue
                       \FBStandardItemLabelsfalse
                       \FBIndentFirsttrue
                       \FBFrenchFootnotestrue
                       \FBAutoSpaceFootnotestrue
                       \FBGlobalLayoutFrenchtrue}%
                      {}%
}
%    \end{macrocode}
%
%  \begin{macro}{\frenchbsetup}
%    From version 2.0 on, all setup options are handled by \emph{one}
%    command |\frenchbsetup| using the keyval syntax.
%    Let's now define this command which reads and sets the options
%    to be processed later (at |\begin{document}|) by
%    |\FBprocess@options|. It  can only be called in the preamble.
%    \begin{macrocode}
\newcommand*{\frenchbsetup}[1]{%
  \setkeys{FB}{#1}%
}%
\@onlypreamble\frenchbsetup
%    \end{macrocode}
%    |frenchb| being an option of babel, it cannot load a package
%    (keyval) while |frenchb.ldf| is read, so we defer the loading of
%    \file{keyval} and the options setup at the end of \babel's loading.
%
%    |StandardLayout| resets the layout in French to the standard layout
%    defined par the \LaTeX{} class and packages loaded. It deals with
%    lists, indentation of first paragraphs of sections and footnotes.
%    Other keys, entered \emph{after} |StandardLayout| in
%    |\frenchbsetup|, can overrule some of the |StandardLayout|
%     settings.
%
%    |GlobalLayoutFrench| forces the layout in French and (as far as
%    possible) outside French to meet the French typographic standards.
%
% \changes{v2.3d}{2009/03/16}{Warning added to \cs{GlobalLayoutFrench}
%    when French is not the main language.}
%
%    \begin{macrocode}
\AtEndOfPackage{%
    \RequirePackage{keyval}%
    \define@key{FB}{StandardLayout}[true]%
                      {\csname FBStandardLayout#1\endcsname
                       \ifFBStandardLayout
                         \FBReduceListSpacingfalse
                         \FBCompactItemizefalse
                         \FBStandardItemLabelstrue
                         \FBIndentFirstfalse
                         \FBFrenchFootnotesfalse
                         \FBAutoSpaceFootnotesfalse
                         \FBGlobalLayoutFrenchfalse
                       \else
                         \FBReduceListSpacingtrue
                         \FBCompactItemizetrue
                         \FBStandardItemLabelsfalse
                         \FBIndentFirsttrue
                         \FBFrenchFootnotestrue
                         \FBAutoSpaceFootnotestrue
                       \fi}%
    \define@key{FB}{GlobalLayoutFrench}[true]%
                      {\csname FBGlobalLayoutFrench#1\endcsname
                       \ifFBGlobalLayoutFrench
                          \iflanguage{french}%
                            {\FBReduceListSpacingtrue
                             \FBCompactItemizetrue
                             \FBStandardItemLabelsfalse
                             \FBIndentFirsttrue
                             \FBFrenchFootnotestrue
                             \FBAutoSpaceFootnotestrue}%
                            {\PackageWarning{frenchb.ldf}%
                              {Option `GlobalLayoutFrench' skipped:
                               \MessageBreak French is *not*
                               babel's last option.\MessageBreak}}%
                       \fi}%
    \define@key{FB}{ReduceListSpacing}[true]%
                      {\csname FBReduceListSpacing#1\endcsname}%
    \define@key{FB}{CompactItemize}[true]%
                      {\csname FBCompactItemize#1\endcsname}%
    \define@key{FB}{StandardItemLabels}[true]%
                      {\csname FBStandardItemLabels#1\endcsname}%
    \define@key{FB}{ItemLabels}{%
        \renewcommand*{\FrenchLabelItem}{#1}}%
    \define@key{FB}{ItemLabeli}{%
        \renewcommand*{\Frlabelitemi}{#1}}%
    \define@key{FB}{ItemLabelii}{%
        \renewcommand*{\Frlabelitemii}{#1}}%
    \define@key{FB}{ItemLabeliii}{%
        \renewcommand*{\Frlabelitemiii}{#1}}%
    \define@key{FB}{ItemLabeliv}{%
        \renewcommand*{\Frlabelitemiv}{#1}}%
    \define@key{FB}{StandardLists}[true]%
                      {\csname FBStandardLists#1\endcsname
                       \ifFBStandardLists
                         \FBReduceListSpacingfalse
                         \FBCompactItemizefalse
                         \FBStandardItemLabelstrue
                       \else
                         \FBReduceListSpacingtrue
                         \FBCompactItemizetrue
                         \FBStandardItemLabelsfalse
                       \fi}%
    \define@key{FB}{IndentFirst}[true]%
                      {\csname FBIndentFirst#1\endcsname}%
    \define@key{FB}{FrenchFootnotes}[true]%
                      {\csname FBFrenchFootnotes#1\endcsname}%
    \define@key{FB}{AutoSpaceFootnotes}[true]%
                      {\csname FBAutoSpaceFootnotes#1\endcsname}%
    \define@key{FB}{AutoSpacePunctuation}[true]%
                      {\csname FBAutoSpacePunctuation#1\endcsname}%
    \define@key{FB}{OriginalTypewriter}[true]%
                      {\csname FBOriginalTypewriter#1\endcsname}%
    \define@key{FB}{ThinColonSpace}[true]%
                      {\csname FBThinColonSpace#1\endcsname}%
    \define@key{FB}{ThinSpaceInFrenchNumbers}[true]%
                      {\csname FBThinSpaceInFrenchNumbers#1\endcsname}%
    \define@key{FB}{FrenchSuperscripts}[true]%
                      {\csname FBFrenchSuperscripts#1\endcsname}
    \define@key{FB}{LowercaseSuperscripts}[true]%
                      {\csname FBLowercaseSuperscripts#1\endcsname}
    \define@key{FB}{PartNameFull}[true]%
                      {\csname FBPartNameFull#1\endcsname}%
    \define@key{FB}{ShowOptions}[true]%
                      {\csname FBShowOptions#1\endcsname}%
%    \end{macrocode}
%    Inputing French quotes as \emph{single characters} when they are
%    available on the keyboard (through a compose key for instance)
%    is more comfortable than typing |\og| and |\fg|.
%    The purpose of the following code is to map the French quote
%    characters to |\og\ignorespaces| and |{\fg}| respectively when
%    the current language is French, and to |\guillemotleft| and
%    |\guillemotright| otherwise (think of German quotes); thus correct
%    unbreakable spaces will be added automatically to French quotes.
%    The quote characters typed in depend on the input encoding,
%    it can be single-byte (latin1, latin9, applemac,\dots) or
%    multi-bytes (utf-8, utf8x).  We first check whether XeTeX is used
%    or not, if not the package |inputenc| has to be loaded before the
%    |\begin{document}| with the proper coding option, so we check if
%    |\DeclareInputText| is defined.
%    \begin{macrocode}
    \define@key{FB}{og}{%
       \newcommand*{\FB@@og}{\iflanguage{french}%
                               {\FB@og\ignorespaces}{\guillemotleft}}%
       \expandafter\ifx\csname XeTeXrevision\endcsname\relax
         \AtBeginDocument{%
           \@ifundefined{DeclareInputText}%
             {\PackageWarning{frenchb.ldf}%
               {Option `og' requires package inputenc.\MessageBreak}%
             }%
             {\@ifundefined{uc@dclc}%
%    \end{macrocode}
%    if |\uc@dclc| is undefined, utf8x is not loaded\dots
%    \begin{macrocode}
               {\@ifundefined{DeclareUnicodeCharacter}%
%    \end{macrocode}
%    if |\DeclareUnicodeCharacter| is undefined, utf8 is not loaded
%    either, we assume 8-bit character input encoding.
%    Package MULEenc.sty (from CJK) defines |\mule@def| to map
%    characters to control sequences.
%    \begin{macrocode}
                  {\@tempcnta`#1\relax
                     \@ifundefined{mule@def}%
                       {\DeclareInputText{\the\@tempcnta}{\FB@@og}}%
                       {\mule@def{11}{{\FB@@og}}}%
                  }%
%    \end{macrocode}
%    utf8 loaded, use |\DeclareUnicodeCharacter|,
%    \begin{macrocode}
                  {\DeclareUnicodeCharacter{00AB}{\FB@@og}}%
               }%
%    \end{macrocode}
%    utf8x loaded, use |\uc@dclc|,
%    \begin{macrocode}
               {\uc@dclc{171}{default}{\FB@@og}}%
             }%
         }%
%    \end{macrocode}
%    XeTeX in use, the following trick for defining the active quote
%    character is borrowed from \file{inputenc.dtx}.
%    \begin{macrocode}
       \else
         \catcode`#1=\active
         \bgroup
           \uccode`\~`#1%
           \uppercase{%
         \egroup
         \def~%
         }{\FB@@og}%
       \fi
    }%
%    \end{macrocode}
%    Same code for the closing quote.
%    \begin{macrocode}
    \define@key{FB}{fg}{%
       \newcommand*{\FB@@fg}{\iflanguage{french}%
                               {\FB@fg}{\guillemotright}}%
       \expandafter\ifx\csname XeTeXrevision\endcsname\relax
         \AtBeginDocument{%
           \@ifundefined{DeclareInputText}%
             {\PackageWarning{frenchb.ldf}%
               {Option `fg' requires package inputenc.\MessageBreak}%
             }%
             {\@ifundefined{uc@dclc}%
               {\@ifundefined{DeclareUnicodeCharacter}%
                  {\@tempcnta`#1\relax
                     \@ifundefined{mule@def}%
                       {\DeclareInputText{\the\@tempcnta}{{\FB@@fg}}}%
                       {\mule@def{27}{{\FB@@fg}}}%
                  }%
                  {\DeclareUnicodeCharacter{00BB}{{\FB@@fg}}}%
               }%
               {\uc@dclc{187}{default}{{\FB@@fg}}}%
             }%
         }%
       \else
         \catcode`#1=\active
         \bgroup
           \uccode`\~`#1%
           \uppercase{%
         \egroup
         \def~%
         }{{\FB@@fg}}%
       \fi
    }%
}
%    \end{macrocode}
%  \end{macro}
%
% \begin{macro}{\FBprocess@options}
%    |\FBprocess@options| processes the options, it is called \emph{once}
%    at |\begin{document}|.
%    \begin{macrocode}
\newcommand*{\FBprocess@options}{%
%    \end{macrocode}
%    Nothing has to be done here for |StandardLayout| and
%    |StandardLists| (the involved flags have already been set in
%    |\frenchbsetup{}| or before (at babel's EndOfPackage).
%
%    The next three options deal with the layout of lists in French.
%
%    |ReduceListSpacing| reduces the vertical spaces between list
%    items in French (done by changing |\list| to |\listFB|).
%    When |GlobalLayoutFrench| is true (the default), the same is
%    done outside French except for languages that force a different
%    setting.
%    \begin{macrocode}
  \ifFBReduceListSpacing
    \addto\extrasfrench{\let\list\listFB
                        \let\endlist\endlistFB}%
    \addto\noextrasfrench{\ifFBGlobalLayoutFrench
                            \let\list\listFB
                            \let\endlist\endlistFB
                          \else
                            \let\list\listORI
                            \let\endlist\endlistORI
                          \fi}%
  \else
    \addto\extrasfrench{\let\list\listORI
                        \let\endlist\endlistORI}%
    \addto\noextrasfrench{\let\list\listORI
                          \let\endlist\endlistORI}%
  \fi
%    \end{macrocode}
%
%    |CompactItemize| suppresses the vertical spacing between list
%    items in French (done by changing |\itemize| to |\itemizeFB|).
%    When |GlobalLayoutFrench| is true the same is done outside French.
%    \begin{macrocode}
  \ifFBCompactItemize
    \addto\extrasfrench{\let\itemize\itemizeFB
                        \let\enditemize\enditemizeFB}%
    \addto\noextrasfrench{\ifFBGlobalLayoutFrench
                             \let\itemize\itemizeFB
                             \let\enditemize\enditemizeFB
                          \else
                             \let\itemize\itemizeORI
                             \let\enditemize\enditemizeORI
                          \fi}%
  \else
    \addto\extrasfrench{\let\itemize\itemizeORI
                        \let\enditemize\enditemizeORI}%
    \addto\noextrasfrench{\let\itemize\itemizeORI
                          \let\enditemize\enditemizeORI}%
  \fi
%    \end{macrocode}
%
%    |StandardItemLabels| resets labelitems in French to their
%    standard values set by the \LaTeX{} class and packages loaded.
%    When |GlobalLayoutFrench| is true labelitems are identical inside
%    and outside French.
%    \begin{macrocode}
  \ifFBStandardItemLabels
    \addto\extrasfrench{\bbl@nonfrenchlabelitems}%
    \addto\noextrasfrench{\bbl@nonfrenchlabelitems}%
  \else
    \addto\extrasfrench{\bbl@frenchlabelitems}%
    \addto\noextrasfrench{\ifFBGlobalLayoutFrench
                            \bbl@frenchlabelitems
                          \else
                            \bbl@nonfrenchlabelitems
                          \fi}%
  \fi
%    \end{macrocode}
%
%    |IndentFirst| forces the first paragraphs of sections to be
%    indented just like the other ones in French.
%    When |GlobalLayoutFrench| is true (the default), the same is
%    done outside French except for languages that force a different
%    setting.
%    \begin{macrocode}
  \ifFBIndentFirst
    \addto\extrasfrench{\bbl@frenchindent}%
    \addto\noextrasfrench{\ifFBGlobalLayoutFrench
                             \bbl@frenchindent
                          \else
                             \bbl@nonfrenchindent
                          \fi}%
  \else
    \addto\extrasfrench{\bbl@nonfrenchindent}%
    \addto\noextrasfrench{\bbl@nonfrenchindent}%
  \fi
%    \end{macrocode}
%
%    The layout of footnotes is handled at the |\begin{document}|
%    depending on the values of flags |FrenchFootnotes|
%    and |AutoSpaceFootnotes| (see section~\ref{sec-footnotes}),
%    nothing has to be done here for footnotes.
%
%    |AutoSpacePunctuation| adds an unbreakable space (in French only)
%    before the four active characters (:;!?) even if none has been
%    typed before them.
%    \begin{macrocode}
  \ifFBAutoSpacePunctuation
     \autospace@beforeFDP
  \else
     \noautospace@beforeFDP
  \fi
%    \end{macrocode}
%
%    When |OriginalTypewriter| is set to |false| (the default),
%    |\ttfamily|, |\rmfamily| and |\sffamily| are redefined as
%    |\ttfamilyFB|, |\rmfamilyFB| and |\sffamilyFB| respectively
%    to prevent addition of automatic spaces before the four active
%    characters in computer code.
%    \begin{macrocode}
  \ifFBOriginalTypewriter
  \else
     \let\ttfamily\ttfamilyFB
     \let\rmfamily\rmfamilyFB
     \let\sffamily\sffamilyFB
  \fi
%    \end{macrocode}
%
%    |ThinColonSpace| changes the normal unbreakable space typeset in
%     French before `:' to a thin space.
%    \begin{macrocode}
  \ifFBThinColonSpace\renewcommand*{\Fcolonspace}{\thinspace}\fi
%    \end{macrocode}
%
%    When |true|, |ThinSpaceInFrenchNumbers| redefines |numprint.sty|'s
%    command |\npstylefrench| to set |\npthousandsep| to |\,|
%    (thinspace) instead of |~| (default) . This option has no effect
%    if package |numprint.sty| is not loaded with `|autolanguage|'.
%    As old versions of |numprint.sty| did not define |\npstylefrench|,
%    we have to provide this command.
%    \begin{macrocode}
  \@ifpackageloaded{numprint}%
  {\ifnprt@autolanguage
     \providecommand*{\npstylefrench}{}%
     \ifFBThinSpaceInFrenchNumbers
       \renewcommand*\npstylefrench{%
          \npthousandsep{\,}%
          \npdecimalsign{,}%
          \npproductsign{\cdot}%
          \npunitseparator{\,}%
          \npdegreeseparator{}%
          \nppercentseparator{\nprt@unitsep}%
          }%
     \else
       \renewcommand*\npstylefrench{%
          \npthousandsep{~}%
          \npdecimalsign{,}%
          \npproductsign{\cdot}%
          \npunitseparator{\,}%
          \npdegreeseparator{}%
          \nppercentseparator{\nprt@unitsep}%
          }%
     \fi
     \npaddtolanguage{french}{french}%
   \fi}{}%
%    \end{macrocode}
%
%    |FrenchSuperscripts|: if |true| |\up=\fup|, else
%    |\up=\textsuperscript|. Anyway |\up*=\FB@up@fake|. The star-form
%    |\up*{}| is provided for fonts that lack some superior letters:
%    Adobe Jenson Pro and Utopia Expert have no ``g superior'' for
%    instance.
%    \begin{macrocode}
  \ifFBFrenchSuperscripts
    \DeclareRobustCommand*{\up}{\@ifstar{\FB@up@fake}{\fup}}%
  \else
    \DeclareRobustCommand*{\up}{\@ifstar{\FB@up@fake}%
                                        {\textsuperscript}}%
  \fi
%    \end{macrocode}
%
%    |LowercaseSuperscripts|: if |true| let |\FB@lc| be |\lowercase|,
%     else |\FB@lc| is redefined to do nothing.
%    \begin{macrocode}
  \ifFBLowercaseSuperscripts
  \else
    \renewcommand*{\FB@lc}[1]{##1}%
  \fi
%    \end{macrocode}
%
%    |PartNameFull|: if |false|, redefine |\partname|.
%    \begin{macrocode}
  \ifFBPartNameFull
  \else\addto\captionsfrench{\def\partname{Partie}}\fi
%    \end{macrocode}
%
%    |ShowOptions|: if |true|, print the list of all options to the
%    \file{.log} file.
%    \begin{macrocode}
  \ifFBShowOptions
    \GenericWarning{* }{%
     * **** List of possible options for frenchb ****\MessageBreak
     [Default values between brackets when frenchb is loaded *LAST*]%
     \MessageBreak
     ShowOptions=true [false]\MessageBreak
     StandardLayout=true [false]\MessageBreak
     GlobalLayoutFrench=false [true]\MessageBreak
     StandardLists=true [false]\MessageBreak
     ReduceListSpacing=false [true]\MessageBreak
     CompactItemize=false [true]\MessageBreak
     StandardItemLabels=true [false]\MessageBreak
     ItemLabels=\textemdash, \textbullet,
        \protect\ding{43},... [\textendash]\MessageBreak
     ItemLabeli=\textemdash, \textbullet,
        \protect\ding{43},... [\textendash]\MessageBreak
     ItemLabelii=\textemdash, \textbullet,
        \protect\ding{43},... [\textendash]\MessageBreak
     ItemLabeliii=\textemdash, \textbullet,
        \protect\ding{43},... [\textendash]\MessageBreak
     ItemLabeliv=\textemdash, \textbullet,
        \protect\ding{43},... [\textendash]\MessageBreak
     IndentFirst=false [true]\MessageBreak
     FrenchFootnotes=false [true]\MessageBreak
     AutoSpaceFootnotes=false [true]\MessageBreak
     AutoSpacePunctuation=false [true]\MessageBreak
     OriginalTypewriter=true [false]\MessageBreak
     ThinColonSpace=true [false]\MessageBreak
     ThinSpaceInFrenchNumbers=true [false]\MessageBreak
     FrenchSuperscripts=false [true]\MessageBreak
     LowercaseSuperscripts=false [true]\MessageBreak
     PartNameFull=false [true]\MessageBreak
     og= <left quote character>, fg= <right quote character>
     \MessageBreak
     *********************************************
     \MessageBreak\protect\frenchbsetup{ShowOptions}}
  \fi
}
%    \end{macrocode}
%  \end{macro}
%
% \changes{v2.0}{2006/12/15}{AtBeginDocument, save again the
%    definitions of the `list' and `itemize' environments and the
%    values of labelitems.  As of frenchb v.1.6, `ORI' values were
%    set when reading frenchb.ldf, later changes were ignored.}
%
% \changes{v2.0}{2006/12/06}{Added warning for OT1 encoding.}
%
% \changes{v2.1b}{2008/04/07}{Disable some commands in bookmarks.}
%
%    At |\begin{document}| we save again the definitions of the `list'
%    and `itemize' environments and the values of labelitems so that
%    all changes made in the preamble are taken into account in
%    languages other than French and in French with the StandardLayout
%    option.  We also have to provide an |\xspace| command in case the
%    |xspace.sty| package is not loaded.
%
%    \begin{macrocode}
\AtBeginDocument{%
   \let\listORI\list
   \let\endlistORI\endlist
   \let\itemizeORI\itemize
   \let\enditemizeORI\enditemize
   \let\@ltiORI\labelitemi
   \let\@ltiiORI\labelitemii
   \let\@ltiiiORI\labelitemiii
   \let\@ltivORI\labelitemiv
   \providecommand*{\xspace}{\relax}%
%    \end{macrocode}
%    Let's redefine some commands in \file{hyperref}'s bookmarks.
%    \begin{macrocode}
   \@ifundefined{pdfstringdefDisableCommands}{}%
     {\pdfstringdefDisableCommands{%
        \let\up\relax
        \def\ieme{e\xspace}%
        \def\iemes{es\xspace}%
        \def\ier{er\xspace}%
        \def\iers{ers\xspace}%
        \def\iere{re\xspace}%
        \def\ieres{res\xspace}%
        \def\FrenchEnumerate#1{#1\degre\space}%
        \def\FrenchPopularEnumerate#1{#1\degre)\space}%
        \def\No{N\degre\space}%
        \def\no{n\degre\space}%
        \def\Nos{N\degre\space}%
        \def\nos{n\degre\space}%
        \def\og{\guillemotleft\space}%
        \def\fg{\space\guillemotright}%
        \let\bsc\textsc
        \let\degres\degre
     }}%
%    \end{macrocode}
%    It is time to process the options set with |\frenchboptions{}|.
%    Then execute either |\extrasfrench| and |\captionsfrench| or
%    |\noextrasfrench| according to the current language at the
%    |\begin{document}| (these three commands are updated by
%    |\FBprocess@options|).
%    \begin{macrocode}
   \FBprocess@options
   \iflanguage{french}{\extrasfrench\captionsfrench}{\noextrasfrench}%
%    \end{macrocode}
%    Some warnings are issued when output font encodings are not
%    properly set. With XeLaTeX, \file{fontspec.sty} and
%    \file{xunicode.sty} should be loaded; with (pdf)\LaTeX, a warning
%    is issued when OT1 encoding is in use at the |\begin{document}|.
%    Mind that |\encodingdefault| is defined as `long', defining
%    |\FBOTone| with |\newcommand*| would fail!
%    \begin{macrocode}
   \expandafter\ifx\csname XeTeXrevision\endcsname\relax
      \begingroup \newcommand{\FBOTone}{OT1}%
      \ifx\encodingdefault\FBOTone
        \PackageWarning{frenchb.ldf}%
           {OT1 encoding should not be used for French.
            \MessageBreak
            Add \protect\usepackage[T1]{fontenc} to the
            preamble\MessageBreak of your document,}
      \fi
     \endgroup
   \else
     \@ifundefined{DeclareUTFcharacter}%
       {\PackageWarning{frenchb.ldf}%
         {Add \protect\usepackage{fontspec} *and*\MessageBreak
          \protect\usepackage{xunicode} to the preamble\MessageBreak
          of your document,}}%
       {}%
    \fi
}
%    \end{macrocode}
%
%  \subsection{Clean up and exit}
%
%    Load |frenchb.cfg| (should do nothing, just for compatibility).
%    \begin{macrocode}
\loadlocalcfg{frenchb}
%    \end{macrocode}
%    Final cleaning.
%    The macro |\ldf@quit| takes care for setting the main language
%    to be switched on at |\begin{document}| and resetting the
%    category code of \texttt{@} to its original value.
%    The config file searched for has to be |frenchb.cfg|, and
%    |\CurrentOption| has been set to `french', so
%    |\ldf@finish\CurrentOption| cannot be used: we first load
%    |frenchb.cfg|, then call |\ldf@quit\CurrentOption|.
%    \begin{macrocode}
\FBclean@on@exit
\ldf@quit\CurrentOption
%    \end{macrocode}
% \iffalse
%</code>
%<*dtx>
% \fi
%%
%% \CharacterTable
%%  {Upper-case    \A\B\C\D\E\F\G\H\I\J\K\L\M\N\O\P\Q\R\S\T\U\V\W\X\Y\Z
%%   Lower-case    \a\b\c\d\e\f\g\h\i\j\k\l\m\n\o\p\q\r\s\t\u\v\w\x\y\z
%%   Digits        \0\1\2\3\4\5\6\7\8\9
%%   Exclamation   \!     Double quote  \"     Hash (number) \#
%%   Dollar        \$     Percent       \%     Ampersand     \&
%%   Acute accent  \'     Left paren    \(     Right paren   \)
%%   Asterisk      \*     Plus          \+     Comma         \,
%%   Minus         \-     Point         \.     Solidus       \/
%%   Colon         \:     Semicolon     \;     Less than     \<
%%   Equals        \=     Greater than  \>     Question mark \?
%%   Commercial at \@     Left bracket  \[     Backslash     \\
%%   Right bracket \]     Circumflex    \^     Underscore    \_
%%   Grave accent  \`     Left brace    \{     Vertical bar  \|
%%   Right brace   \}     Tilde         \~}
%%
% \iffalse
%</dtx>
% \fi
%
% \Finale
\endinput
}
\bbl@tempa{afrikaans}{%%
%% This file will generate fast loadable files and documentation
%% driver files from the doc files in this package when run through
%% LaTeX or TeX.
%%
%% Copyright 1989-2005 Johannes L. Braams and any individual authors
%% listed elsewhere in this file.  All rights reserved.
%% 
%% This file is part of the Babel system.
%% --------------------------------------
%% 
%% It may be distributed and/or modified under the
%% conditions of the LaTeX Project Public License, either version 1.3
%% of this license or (at your option) any later version.
%% The latest version of this license is in
%%   http://www.latex-project.org/lppl.txt
%% and version 1.3 or later is part of all distributions of LaTeX
%% version 2003/12/01 or later.
%% 
%% This work has the LPPL maintenance status "maintained".
%% 
%% The Current Maintainer of this work is Johannes Braams.
%% 
%% The list of all files belonging to the LaTeX base distribution is
%% given in the file `manifest.bbl. See also `legal.bbl' for additional
%% information.
%% 
%% The list of derived (unpacked) files belonging to the distribution
%% and covered by LPPL is defined by the unpacking scripts (with
%% extension .ins) which are part of the distribution.
%%
%% --------------- start of docstrip commands ------------------
%%
\def\filedate{1999/04/11}
\def\batchfile{dutch.ins}
\input docstrip.tex

{\ifx\generate\undefined
\Msg{**********************************************}
\Msg{*}
\Msg{* This installation requires docstrip}
\Msg{* version 2.3c or later.}
\Msg{*}
\Msg{* An older version of docstrip has been input}
\Msg{*}
\Msg{**********************************************}
\errhelp{Move or rename old docstrip.tex.}
\errmessage{Old docstrip in input path}
\batchmode
\csname @@end\endcsname
\fi}

\declarepreamble\mainpreamble
This is a generated file.

Copyright 1989-2005 Johannes L. Braams and any individual authors
listed elsewhere in this file.  All rights reserved.

This file was generated from file(s) of the Babel system.
---------------------------------------------------------

It may be distributed and/or modified under the
conditions of the LaTeX Project Public License, either version 1.3
of this license or (at your option) any later version.
The latest version of this license is in
  http://www.latex-project.org/lppl.txt
and version 1.3 or later is part of all distributions of LaTeX
version 2003/12/01 or later.

This work has the LPPL maintenance status "maintained".

The Current Maintainer of this work is Johannes Braams.

This file may only be distributed together with a copy of the Babel
system. You may however distribute the Babel system without
such generated files.

The list of all files belonging to the Babel distribution is
given in the file `manifest.bbl'. See also `legal.bbl for additional
information.

The list of derived (unpacked) files belonging to the distribution
and covered by LPPL is defined by the unpacking scripts (with
extension .ins) which are part of the distribution.
\endpreamble

\declarepreamble\fdpreamble
This is a generated file.

Copyright 1989-2005 Johannes L. Braams and any individual authors
listed elsewhere in this file.  All rights reserved.

This file was generated from file(s) of the Babel system.
---------------------------------------------------------

It may be distributed and/or modified under the
conditions of the LaTeX Project Public License, either version 1.3
of this license or (at your option) any later version.
The latest version of this license is in
  http://www.latex-project.org/lppl.txt
and version 1.3 or later is part of all distributions of LaTeX
version 2003/12/01 or later.

This work has the LPPL maintenance status "maintained".

The Current Maintainer of this work is Johannes Braams.

This file may only be distributed together with a copy of the Babel
system. You may however distribute the Babel system without
such generated files.

The list of all files belonging to the Babel distribution is
given in the file `manifest.bbl'. See also `legal.bbl for additional
information.

In particular, permission is granted to customize the declarations in
this file to serve the needs of your installation.

However, NO PERMISSION is granted to distribute a modified version
of this file under its original name.

\endpreamble

\keepsilent

\usedir{tex/generic/babel} 

\usepreamble\mainpreamble
\generate{\file{dutch.ldf}{\from{dutch.dtx}{code}}
          }
\usepreamble\fdpreamble

\ifToplevel{
\Msg{***********************************************************}
\Msg{*}
\Msg{* To finish the installation you have to move the following}
\Msg{* files into a directory searched by TeX:}
\Msg{*}
\Msg{* \space\space All *.def, *.fd, *.ldf, *.sty}
\Msg{*}
\Msg{* To produce the documentation run the files ending with}
\Msg{* '.dtx' and `.fdd' through LaTeX.}
\Msg{*}
\Msg{* Happy TeXing}
\Msg{***********************************************************}
}
 
\endinput
}
\bbl@tempa{american}{%%
%% This file will generate fast loadable files and documentation
%% driver files from the doc files in this package when run through
%% LaTeX or TeX.
%%
%% Copyright 1989-2005 Johannes L. Braams and any individual authors
%% listed elsewhere in this file.  All rights reserved.
%% 
%% This file is part of the Babel system.
%% --------------------------------------
%% 
%% It may be distributed and/or modified under the
%% conditions of the LaTeX Project Public License, either version 1.3
%% of this license or (at your option) any later version.
%% The latest version of this license is in
%%   http://www.latex-project.org/lppl.txt
%% and version 1.3 or later is part of all distributions of LaTeX
%% version 2003/12/01 or later.
%% 
%% This work has the LPPL maintenance status "maintained".
%% 
%% The Current Maintainer of this work is Johannes Braams.
%% 
%% The list of all files belonging to the LaTeX base distribution is
%% given in the file `manifest.bbl. See also `legal.bbl' for additional
%% information.
%% 
%% The list of derived (unpacked) files belonging to the distribution
%% and covered by LPPL is defined by the unpacking scripts (with
%% extension .ins) which are part of the distribution.
%%
%% --------------- start of docstrip commands ------------------
%%
\def\filedate{1999/04/11}
\def\batchfile{english.ins}
\input docstrip.tex

{\ifx\generate\undefined
\Msg{**********************************************}
\Msg{*}
\Msg{* This installation requires docstrip}
\Msg{* version 2.3c or later.}
\Msg{*}
\Msg{* An older version of docstrip has been input}
\Msg{*}
\Msg{**********************************************}
\errhelp{Move or rename old docstrip.tex.}
\errmessage{Old docstrip in input path}
\batchmode
\csname @@end\endcsname
\fi}

\declarepreamble\mainpreamble
This is a generated file.

Copyright 1989-2005 Johannes L. Braams and any individual authors
listed elsewhere in this file.  All rights reserved.

This file was generated from file(s) of the Babel system.
---------------------------------------------------------

It may be distributed and/or modified under the
conditions of the LaTeX Project Public License, either version 1.3
of this license or (at your option) any later version.
The latest version of this license is in
  http://www.latex-project.org/lppl.txt
and version 1.3 or later is part of all distributions of LaTeX
version 2003/12/01 or later.

This work has the LPPL maintenance status "maintained".

The Current Maintainer of this work is Johannes Braams.

This file may only be distributed together with a copy of the Babel
system. You may however distribute the Babel system without
such generated files.

The list of all files belonging to the Babel distribution is
given in the file `manifest.bbl'. See also `legal.bbl for additional
information.

The list of derived (unpacked) files belonging to the distribution
and covered by LPPL is defined by the unpacking scripts (with
extension .ins) which are part of the distribution.
\endpreamble

\declarepreamble\fdpreamble
This is a generated file.

Copyright 1989-2005 Johannes L. Braams and any individual authors
listed elsewhere in this file.  All rights reserved.

This file was generated from file(s) of the Babel system.
---------------------------------------------------------

It may be distributed and/or modified under the
conditions of the LaTeX Project Public License, either version 1.3
of this license or (at your option) any later version.
The latest version of this license is in
  http://www.latex-project.org/lppl.txt
and version 1.3 or later is part of all distributions of LaTeX
version 2003/12/01 or later.

This work has the LPPL maintenance status "maintained".

The Current Maintainer of this work is Johannes Braams.

This file may only be distributed together with a copy of the Babel
system. You may however distribute the Babel system without
such generated files.

The list of all files belonging to the Babel distribution is
given in the file `manifest.bbl'. See also `legal.bbl for additional
information.

In particular, permission is granted to customize the declarations in
this file to serve the needs of your installation.

However, NO PERMISSION is granted to distribute a modified version
of this file under its original name.

\endpreamble

\keepsilent

\usedir{tex/generic/babel} 

\usepreamble\mainpreamble
\generate{\file{english.ldf}{\from{english.dtx}{code}}
          }
\usepreamble\fdpreamble

\ifToplevel{
\Msg{***********************************************************}
\Msg{*}
\Msg{* To finish the installation you have to move the following}
\Msg{* files into a directory searched by TeX:}
\Msg{*}
\Msg{* \space\space All *.def, *.fd, *.ldf, *.sty}
\Msg{*}
\Msg{* To produce the documentation run the files ending with}
\Msg{* '.dtx' and `.fdd' through LaTeX.}
\Msg{*}
\Msg{* Happy TeXing}
\Msg{***********************************************************}
}
 
\endinput
}
\bbl@tempa{australian}{%%
%% This file will generate fast loadable files and documentation
%% driver files from the doc files in this package when run through
%% LaTeX or TeX.
%%
%% Copyright 1989-2005 Johannes L. Braams and any individual authors
%% listed elsewhere in this file.  All rights reserved.
%% 
%% This file is part of the Babel system.
%% --------------------------------------
%% 
%% It may be distributed and/or modified under the
%% conditions of the LaTeX Project Public License, either version 1.3
%% of this license or (at your option) any later version.
%% The latest version of this license is in
%%   http://www.latex-project.org/lppl.txt
%% and version 1.3 or later is part of all distributions of LaTeX
%% version 2003/12/01 or later.
%% 
%% This work has the LPPL maintenance status "maintained".
%% 
%% The Current Maintainer of this work is Johannes Braams.
%% 
%% The list of all files belonging to the LaTeX base distribution is
%% given in the file `manifest.bbl. See also `legal.bbl' for additional
%% information.
%% 
%% The list of derived (unpacked) files belonging to the distribution
%% and covered by LPPL is defined by the unpacking scripts (with
%% extension .ins) which are part of the distribution.
%%
%% --------------- start of docstrip commands ------------------
%%
\def\filedate{1999/04/11}
\def\batchfile{english.ins}
\input docstrip.tex

{\ifx\generate\undefined
\Msg{**********************************************}
\Msg{*}
\Msg{* This installation requires docstrip}
\Msg{* version 2.3c or later.}
\Msg{*}
\Msg{* An older version of docstrip has been input}
\Msg{*}
\Msg{**********************************************}
\errhelp{Move or rename old docstrip.tex.}
\errmessage{Old docstrip in input path}
\batchmode
\csname @@end\endcsname
\fi}

\declarepreamble\mainpreamble
This is a generated file.

Copyright 1989-2005 Johannes L. Braams and any individual authors
listed elsewhere in this file.  All rights reserved.

This file was generated from file(s) of the Babel system.
---------------------------------------------------------

It may be distributed and/or modified under the
conditions of the LaTeX Project Public License, either version 1.3
of this license or (at your option) any later version.
The latest version of this license is in
  http://www.latex-project.org/lppl.txt
and version 1.3 or later is part of all distributions of LaTeX
version 2003/12/01 or later.

This work has the LPPL maintenance status "maintained".

The Current Maintainer of this work is Johannes Braams.

This file may only be distributed together with a copy of the Babel
system. You may however distribute the Babel system without
such generated files.

The list of all files belonging to the Babel distribution is
given in the file `manifest.bbl'. See also `legal.bbl for additional
information.

The list of derived (unpacked) files belonging to the distribution
and covered by LPPL is defined by the unpacking scripts (with
extension .ins) which are part of the distribution.
\endpreamble

\declarepreamble\fdpreamble
This is a generated file.

Copyright 1989-2005 Johannes L. Braams and any individual authors
listed elsewhere in this file.  All rights reserved.

This file was generated from file(s) of the Babel system.
---------------------------------------------------------

It may be distributed and/or modified under the
conditions of the LaTeX Project Public License, either version 1.3
of this license or (at your option) any later version.
The latest version of this license is in
  http://www.latex-project.org/lppl.txt
and version 1.3 or later is part of all distributions of LaTeX
version 2003/12/01 or later.

This work has the LPPL maintenance status "maintained".

The Current Maintainer of this work is Johannes Braams.

This file may only be distributed together with a copy of the Babel
system. You may however distribute the Babel system without
such generated files.

The list of all files belonging to the Babel distribution is
given in the file `manifest.bbl'. See also `legal.bbl for additional
information.

In particular, permission is granted to customize the declarations in
this file to serve the needs of your installation.

However, NO PERMISSION is granted to distribute a modified version
of this file under its original name.

\endpreamble

\keepsilent

\usedir{tex/generic/babel} 

\usepreamble\mainpreamble
\generate{\file{english.ldf}{\from{english.dtx}{code}}
          }
\usepreamble\fdpreamble

\ifToplevel{
\Msg{***********************************************************}
\Msg{*}
\Msg{* To finish the installation you have to move the following}
\Msg{* files into a directory searched by TeX:}
\Msg{*}
\Msg{* \space\space All *.def, *.fd, *.ldf, *.sty}
\Msg{*}
\Msg{* To produce the documentation run the files ending with}
\Msg{* '.dtx' and `.fdd' through LaTeX.}
\Msg{*}
\Msg{* Happy TeXing}
\Msg{***********************************************************}
}
 
\endinput
}
\bbl@tempa{austrian}{% \iffalse meta-comm

% Copyright 1989-2008 Johannes L. Braams and any individual auth
% listed elsewhere in this file.  All rights reserv

% This file is part of the Babel syst
% -----------------------------------

% It may be distributed and/or modified under
% conditions of the LaTeX Project Public License, either version
% of this license or (at your option) any later versi
% The latest version of this license is
%   http://www.latex-project.org/lppl.
% and version 1.3 or later is part of all distributions of La
% version 2003/12/01 or lat

% This work has the LPPL maintenance status "maintaine

% The Current Maintainer of this work is Johannes Braa

% The list of all files belonging to the Babel system
% given in the file `manifest.bbl. See also `legal.bbl' for additio
% informati

% The list of derived (unpacked) files belonging to the distribut
% and covered by LPPL is defined by the unpacking scripts (w
% extension .ins) which are part of the distributi
%
% \CheckSum{3

% \iffa
%    Tell the \LaTeX\ system who we are and write an entry on
%    transcri
%<*d
\ProvidesFile{germanb.d
%</d
%<code>\ProvidesLanguage{germa
%
%\ProvidesFile{germanb.d
        [2008/06/01 v2.6m German support from the babel syst
%\iffa
%% File `germanb.d
%% Babel package for LaTeX version
%% Copyright (C) 1989 - 2
%%           by Johannes Braams, TeXn

%% Germanb Language Definition F
%% Copyright (C) 1989 - 2
%%           by Bernd Raichle raichle at azu.Informatik.Uni-Stuttgart
%%              Johannes Braams, TeXn
% This file is based on german.tex version 2.
%                       by Bernd Raichle, Hubert Partl et.

%% Please report errors to: J.L. Bra
%%                          babel at braams.xs4all

%<*filedriv
\documentclass{ltxd
\font\manual=logo10 % font used for the METAFONT logo, e
\newcommand*\MF{{\manual META}\-{\manual FON
\newcommand*\TeXhax{\TeX h
\newcommand*\babel{\textsf{babe
\newcommand*\langvar{$\langle \it lang \rangl
\newcommand*\note[1
\newcommand*\Lopt[1]{\textsf{#
\newcommand*\file[1]{\texttt{#
\begin{docume
 \DocInput{germanb.d
\end{docume
%</filedriv
%
% \GetFileInfo{germanb.d

% \changes{germanb-1.0a}{1990/05/14}{Incorporated Nico's commen
% \changes{germanb-1.0b}{1990/05/22}{fixed typo in definition
%    austrian language found by Werenfried S
%    \texttt{nspit@fys.ruu.n
% \changes{germanb-1.0c}{1990/07/16}{Fixed some typ
% \changes{germanb-1.1}{1990/07/30}{When using PostScript fonts w
%    the Adobe fontencoding, the dieresis-accent is located elsewhe
%    modified co
% \changes{germanb-1.1a}{1990/08/27}{Modified the documentat
%    somewh
% \changes{germanb-2.0}{1991/04/23}{Modified for babel 3
% \changes{germanb-2.0a}{1991/05/25}{Removed some problems in cha
%    l
% \changes{germanb-2.1}{1991/05/29}{Removed bug found by van der Me
% \changes{germanb-2.2}{1991/06/11}{Removed global assignmen
%    brought uptodate with \file{german.tex} v2.
% \changes{germanb-2.2a}{1991/07/15}{Renamed \file{babel.sty}
%    \file{babel.co
% \changes{germanb-2.3}{1991/11/05}{Rewritten parts of the code to
%    the new features of babel version 3
% \changes{germanb-2.3e}{1991/11/10}{Brought up-to-date w
%    \file{german.tex} v2.3e (plus some bug fixes) [b
% \changes{germanb-2.5}{1994/02/08}{Update or \LaTe
% \changes{germanb-2.5c}{1994/06/26}{Removed the use of \cs{fileda
%    and moved the identification after the loading
%    \file{babel.de
% \changes{germanb-2.6a}{1995/02/15}{Moved the identification to
%    top of the fi
% \changes{germanb-2.6a}{1995/02/15}{Rewrote the code that handles
%    active double quote charact
% \changes{germanb-2.6d}{1996/07/10}{Replaced \cs{undefined} w
%    \cs{@undefined} and \cs{empty} with \cs{@empty} for consiste
%    with \LaTe
% \changes{germanb-2.6d}{1996/10/10}{Moved the definition
%    \cs{atcatcode} right to the beginning

%  \section{The German langua

%    The file \file{\filename}\footnote{The file described in t
%    section has version number \fileversion\ and was last revised
%    \filedate.}  defines all the language definition macros for
%    German language as well as for the Austrian dialect of t
%    language\footnote{This file is a re-implementation of Hub
%    Partl's \file{german.sty} version 2.5b, see~\cite{HP}

%    For this language the character |"| is made active.
%    table~\ref{tab:german-quote} an overview is given of
%    purpose. One of the reasons for this is that in the Ger
%    language some character combinations change when a word is bro
%    between the combination. Also the vertical placement of
%    umlaut can be controlled this w
%    \begin{table}[h
%     \begin{cent
%     \begin{tabular}{lp{8c
%      |"a| & |\"a|, also implemented for the ot
%                  lowercase and uppercase vowels.
%      |"s| & to produce the German \ss{} (like |\ss{}|).
%      |"z| & to produce the German \ss{} (like |\ss{}|).
%      |"ck|& for |ck| to be hyphenated as |k-k|.
%      |"ff|& for |ff| to be hyphenated as |ff-
%                  this is also implemented for l, m, n, p, r and
%      |"S| & for |SS| to be |\uppercase{"s}|.
%      |"Z| & for |SZ| to be |\uppercase{"z}|.
%      \verb="|= & disable ligature at this position.
%      |"-| & an explicit hyphen sign, allowing hyphenat
%             in the rest of the word.
%      |""| & like |"-|, but producing no hyphen s
%             (for compund words with hyphen, e.g.\ |x-""y|).
%      |"~| & for a compound word mark without a breakpoint.
%      |"=| & for a compound word mark with a breakpoint, allow
%             hyphenation in the composing words.
%      |"`| & for German left double quotes (looks like ,,).
%      |"'| & for German right double quotes.
%      |"<| & for French left double quotes (similar to $<<$).
%      |">| & for French right double quotes (similar to $>>$).
%     \end{tabul
%     \caption{The extra definitions m
%              by \file{german.ldf}}\label{tab:german-quo
%     \end{cent
%    \end{tab
%    The quotes in table~\ref{tab:german-quote} can also be typeset
%    using the commands in table~\ref{tab:more-quot
%    \begin{table}[h
%     \begin{cent
%     \begin{tabular}{lp{8c
%      |\glqq| & for German left double quotes (looks like ,,).
%      |\grqq| & for German right double quotes (looks like ``).
%      |\glq|  & for German left single quotes (looks like ,).
%      |\grq|  & for German right single quotes (looks like `).
%      |\flqq| & for French left double quotes (similar to $<<$).
%      |\frqq| & for French right double quotes (similar to $>>$)
%      |\flq|  & for (French) left single quotes (similar to $<$).
%      |\frq|  & for (French) right single quotes (similar to $>$).
%      |\dq|   & the original quotes character (|"|).
%     \end{tabul
%     \caption{More commands which produce quotes, defi
%              by \file{german.ldf}}\label{tab:more-quo
%     \end{cent
%    \end{tab

% \StopEventuall

%    When this file was read through the option \Lopt{germanb} we m
%    it behave as if \Lopt{german} was specifi
% \changes{german-2.6l}{2008/03/17}{Making germanb behave like ger
%    needs some more work besides defining \cs{CurrentOptio
% \changes{germanb-2.6m}{2008/06/01}{Correted a ty
%    \begin{macroco
\def\bbl@tempa{germa
\ifx\CurrentOption\bbl@te
  \def\CurrentOption{germ
  \ifx\l@german\@undefi
    \@nopatterns{Germ
    \adddialect\l@germ

  \let\l@germanb\l@ger
  \AtBeginDocumen
    \let\captionsgermanb\captionsger
    \let\dategermanb\dateger
    \let\extrasgermanb\extrasger
    \let\noextrasgermanb\noextrasger


%    \end{macroco

%    The macro |\LdfInit| takes care of preventing that this file
%    loaded more than once, checking the category code of
%    \texttt{@} sign, e
% \changes{germanb-2.6d}{1996/11/02}{Now use \cs{LdfInit} to perf
%    initial check
%    \begin{macroco
%<*co
\LdfInit\CurrentOption{captions\CurrentOpti
%    \end{macroco

%    When this file is read as an option, i.e., by the |\usepacka
%    command, \texttt{german} will be an `unknown' language, so
%    have to make it known.  So we check for the existence
%    |\l@german| to see whether we have to do something he

% \changes{germanb-2.0}{1991/04/23}{Now use \cs{adddialect}
%    language undefin
% \changes{germanb-2.2d}{1991/10/27}{Removed use of \cs{@ifundefine
% \changes{germanb-2.3e}{1991/11/10}{Added warning, if no ger
%    patterns load
% \changes{germanb-2.5c}{1994/06/26}{Now use \cs{@nopatterns}
%    produce the warni
%    \begin{macroco
\ifx\l@german\@undefi
  \@nopatterns{Germ
  \adddialect\l@germ

%    \end{macroco

%    For the Austrian version of these definitions we just add anot
%    languag
% \changes{germanb-2.0}{1991/04/23}{Now use \cs{adddialect}
%    austri
%    \begin{macroco
\adddialect\l@austrian\l@ger
%    \end{macroco

%    The next step consists of defining commands to switch to (
%    from) the German langua

%  \begin{macro}{\captionsgerm
%  \begin{macro}{\captionsaustri
%    Either the macro |\captionsgerman| or the ma
%    |\captionsaustrian| will define all strings used in the f
%    standard document classes provided with \LaT

% \changes{germanb-2.2}{1991/06/06}{Removed \cs{global} definitio
% \changes{germanb-2.2}{1991/06/06}{\cs{pagename} should
%    \cs{headpagenam
% \changes{germanb-2.3e}{1991/11/10}{Added \cs{prefacenam
%    \cs{seename} and \cs{alsonam
% \changes{germanb-2.4}{1993/07/15}{\cs{headpagename} should
%    \cs{pagenam
% \changes{germanb-2.6b}{1995/07/04}{Added \cs{proofname}
%    AMS-\LaT
% \changes{germanb-2.6d}{1996/07/10}{Construct control sequence on
%    f
% \changes{germanb-2.6j}{2000/09/20}{Added \cs{glossarynam
%    \begin{macroco
\@namedef{captions\CurrentOption
  \def\prefacename{Vorwor
  \def\refname{Literatu
  \def\abstractname{Zusammenfassun
  \def\bibname{Literaturverzeichni
  \def\chaptername{Kapite
  \def\appendixname{Anhan
  \def\contentsname{Inhaltsverzeichnis}%    % oder nur: Inh
  \def\listfigurename{Abbildungsverzeichni
  \def\listtablename{Tabellenverzeichni
  \def\indexname{Inde
  \def\figurename{Abbildun
  \def\tablename{Tabelle}%                  % oder: Ta
  \def\partname{Tei
  \def\enclname{Anlage(n)}%                 % oder: Beilage
  \def\ccname{Verteiler}%                   % oder: Kopien
  \def\headtoname{A
  \def\pagename{Seit
  \def\seename{sieh
  \def\alsoname{siehe auc
  \def\proofname{Bewei
  \def\glossaryname{Glossa

%    \end{macroco
%  \end{mac
%  \end{mac

%  \begin{macro}{\dategerm
%    The macro |\dategerman| redefines the comm
%    |\today| to produce German dat
% \changes{germanb-2.3e}{1991/11/10}{Added \cs{month@germa
% \changes{germanb-2.6f}{1997/10/01}{Use \cs{edef} to def
%    \cs{today} to save memo
% \changes{germanb-2.6f}{1998/03/28}{use \cs{def} instead
%    \cs{ede
%    \begin{macroco
\def\month@german{\ifcase\month
  Januar\or Februar\or M\"arz\or April\or Mai\or Juni
  Juli\or August\or September\or Oktober\or November\or Dezember\
\def\dategerman{\def\today{\number\day.~\month@ger
    \space\number\yea
%    \end{macroco
%  \end{mac

%  \begin{macro}{\dateaustri
%    The macro |\dateaustrian| redefines the comm
%    |\today| to produce Austrian version of the German dat
% \changes{germanb-2.6f}{1997/10/01}{Use \cs{edef} to def
%    \cs{today} to save memo
% \changes{germanb-2.6f}{1998/03/28}{use \cs{def} instead
%    \cs{ede
%    \begin{macroco
\def\dateaustrian{\def\today{\number\day.~\ifnum1=\mo
  J\"anner\else \month@german\fi \space\number\yea
%    \end{macroco
%  \end{mac

%  \begin{macro}{\extrasgerm
%  \begin{macro}{\extrasaustri
% \changes{germanb-2.0b}{1991/05/29}{added some comment chars
%    prevent white spa
% \changes{germanb-2.2}{1991/06/11}{Save all redefined macr
%  \begin{macro}{\noextrasgerm
%  \begin{macro}{\noextrasaustri
% \changes{germanb-1.1}{1990/07/30}{Added \cs{dieresi
% \changes{germanb-2.0b}{1991/05/29}{added some comment chars
%    prevent white spa
% \changes{germanb-2.2}{1991/06/11}{Try to restore everything to
%    former sta
% \changes{germanb-2.6d}{1996/07/10}{Construct control seque
%    \cs{extrasgerman} or \cs{extrasaustrian} on the f

%    Either the macro |\extrasgerman| or the macros |\extrasaustri
%    will perform all the extra definitions needed for the Ger
%    language. The macro |\noextrasgerman| is used to cancel
%    actions of |\extrasgerman

%    For German (as well as for Dutch) the \texttt{"} character
%    made active. This is done once, later on its definition may va
%    \begin{macroco
\initiate@active@char
\@namedef{extras\CurrentOption
  \languageshorthands{germa
\expandafter\addto\csname extras\CurrentOption\endcsnam
  \bbl@activate{
%    \end{macroco
%    Don't forget to turn the shorthands off aga
% \changes{germanb-2.6i}{1999/12/16}{Deactivate shorthands ouside
%    Germ
%    \begin{macroco
\addto\noextrasgerman{\bbl@deactivate{
%    \end{macroco

% \changes{germanb-2.6a}{1995/02/15}{All the code to handle the act
%    double quote has been moved to \file{babel.de

%    In order for \TeX\ to be able to hyphenate German words wh
%    contain `\ss' (in the \texttt{OT1} position |^^Y|) we have
%    give the character a nonzero |\lccode| (see Appendix H, the \
%    boo
% \changes{germanb-2.6c}{1996/04/08}{Use decimal number instead
%    hat-notation as the hat may be activat
%    \begin{macroco
\expandafter\addto\csname extras\CurrentOption\endcsnam
  \babel@savevariable{\lccode2
  \lccode25=
%    \end{macroco
% \changes{germanb-2.6a}{1995/02/15}{Removeed \cs{3} as it is
%    longer in \file{german.ld

%    The umlaut accent macro |\"| is changed to lower the umlaut do
%    The redefinition is done with the help of |\umlautlo
%    \begin{macroco
\expandafter\addto\csname extras\CurrentOption\endcsnam
  \babel@save\"\umlautl
\@namedef{noextras\CurrentOption}{\umlauthi
%    \end{macroco
%    The german hyphenation patterns can be used with |\lefthyphenm
%    and |\righthyphenmin| set to
% \changes{germanb-2.6a}{1995/05/13}{use \cs{germanhyphenmins} to st
%    the correct valu
% \changes{germanb-2.6j}{2000/09/22}{Now use \cs{providehyphenmins}
%    provide a default val
%    \begin{macroco
\providehyphenmins{\CurrentOption}{\tw@\t
%    \end{macroco
%    For German texts we need to make sure that |\frenchspacing|
%    turned
% \changes{germanb-2.6k}{2001/01/26}{Turn frenchspacing on, as
%    \texttt{german.st
%    \begin{macroco
\expandafter\addto\csname extras\CurrentOption\endcsnam
  \bbl@frenchspaci
\expandafter\addto\csname noextras\CurrentOption\endcsnam
  \bbl@nonfrenchspaci
%    \end{macroco
%  \end{mac
%  \end{mac
%  \end{mac
%  \end{mac

% \changes{germanb-2.6a}{1995/02/15}{\cs{umlautlow}
%    \cs{umlauthigh} moved to \file{glyphs.dtx}, as well
%    \cs{newumlaut} (now \cs{lower@umlau

%    The code above is necessary because we need an extra act
%    character. This character is then used as indicated
%    table~\ref{tab:german-quot

%    To be able to define the function of |"|, we first defin
%    couple of `support' macr

% \changes{germanb-2.3e}{1991/11/10}{Added \cs{save@sf@q} macro
%    rewrote all quote macros to use
% \changes{germanb-2.3h}{1991/02/16}{moved definition
%    \cs{allowhyphens}, \cs{set@low@box} and \cs{save@sf@q}
%    \file{babel.co
% \changes{germanb-2.6a}{1995/02/15}{Moved all quotation characters
%    \file{glyphs.dt

%  \begin{macro}{\
%    We save the original double quote character in |\dq| to k
%    it available, the math accent |\"| can now be typed as |
%    \begin{macroco
\begingroup \catcode`\
\def\x{\endgr
  \def\@SS{\mathchar"701
  \def\dq{

%    \end{macroco
%  \end{mac
% \changes{germanb-2.6c}{1996/01/24}{Moved \cs{german@dq@disc}
%    babel.def, calling it \cs{bbl@dis

% \changes{germanb-2.6a}{1995/02/15}{Use \cs{ddot} instead
%    \cs{@MATHUMLAU

%    Now we can define the doublequote macros: the umlau
% \changes{germanb-2.6c}{1996/05/30}{added the \cs{allowhyphen
%    \begin{macroco
\declare@shorthand{german}{"a}{\textormath{\"{a}\allowhyphens}{\ddot
\declare@shorthand{german}{"o}{\textormath{\"{o}\allowhyphens}{\ddot
\declare@shorthand{german}{"u}{\textormath{\"{u}\allowhyphens}{\ddot
\declare@shorthand{german}{"A}{\textormath{\"{A}\allowhyphens}{\ddot
\declare@shorthand{german}{"O}{\textormath{\"{O}\allowhyphens}{\ddot
\declare@shorthand{german}{"U}{\textormath{\"{U}\allowhyphens}{\ddot
%    \end{macroco
%    trem
%    \begin{macroco
\declare@shorthand{german}{"e}{\textormath{\"{e}}{\ddot
\declare@shorthand{german}{"E}{\textormath{\"{E}}{\ddot
\declare@shorthand{german}{"i}{\textormath{\"{\i
                              {\ddot\imat
\declare@shorthand{german}{"I}{\textormath{\"{I}}{\ddot
%    \end{macroco
%    german es-zet (sharp
% \changes{germanb-2.6f}{1997/05/08}{use \cs{SS} instead
%    \texttt{SS}, removed braces after \cs{ss
%    \begin{macroco
\declare@shorthand{german}{"s}{\textormath{\ss}{\@SS{
\declare@shorthand{german}{"S}{\
\declare@shorthand{german}{"z}{\textormath{\ss}{\@SS{
\declare@shorthand{german}{"Z}{
%    \end{macroco
%    german and french quot
%    \begin{macroco
\declare@shorthand{german}{"`}{\gl
\declare@shorthand{german}{"'}{\gr
\declare@shorthand{german}{"<}{\fl
\declare@shorthand{german}{">}{\fr
%    \end{macroco
%    discretionary comma
%    \begin{macroco
\declare@shorthand{german}{"c}{\textormath{\bbl@disc ck}{
\declare@shorthand{german}{"C}{\textormath{\bbl@disc CK}{
\declare@shorthand{german}{"F}{\textormath{\bbl@disc F{FF}}{
\declare@shorthand{german}{"l}{\textormath{\bbl@disc l{ll}}{
\declare@shorthand{german}{"L}{\textormath{\bbl@disc L{LL}}{
\declare@shorthand{german}{"m}{\textormath{\bbl@disc m{mm}}{
\declare@shorthand{german}{"M}{\textormath{\bbl@disc M{MM}}{
\declare@shorthand{german}{"n}{\textormath{\bbl@disc n{nn}}{
\declare@shorthand{german}{"N}{\textormath{\bbl@disc N{NN}}{
\declare@shorthand{german}{"p}{\textormath{\bbl@disc p{pp}}{
\declare@shorthand{german}{"P}{\textormath{\bbl@disc P{PP}}{
\declare@shorthand{german}{"r}{\textormath{\bbl@disc r{rr}}{
\declare@shorthand{german}{"R}{\textormath{\bbl@disc R{RR}}{
\declare@shorthand{german}{"t}{\textormath{\bbl@disc t{tt}}{
\declare@shorthand{german}{"T}{\textormath{\bbl@disc T{TT}}{
%    \end{macroco
%    We need to treat |"f| a bit differently in order to preserve
%    ff-ligatur
% \changes{germanb-2.6f}{1998/06/15}{Copied the coding for \texttt{
%    from german.dtx version 2.5
%    \begin{macroco
\declare@shorthand{german}{"f}{\textormath{\bbl@discff}{
\def\bbl@discff{\penalty
  \afterassignment\bbl@insertff \let\bbl@nextff
\def\bbl@insertf
  \if f\bbl@nex
    \expandafter\@firstoftwo\else\expandafter\@secondoftwo
  {\relax\discretionary{ff-}{f}{ff}\allowhyphens}{f\bbl@nextf
\let\bbl@nextf
%    \end{macroco
%    and some additional comman
%    \begin{macroco
\declare@shorthand{german}{"-}{\nobreak\-\bbl@allowhyphe
\declare@shorthand{german}{"|
  \textormath{\penalty\@M\discretionary{-}{}{\kern.03e
              \allowhyphens}
\declare@shorthand{german}{""}{\hskip\z@sk
\declare@shorthand{german}{"~}{\textormath{\leavevmode\hbox{-}}{
\declare@shorthand{german}{"=}{\penalty\@M-\hskip\z@sk
%    \end{macroco

%  \begin{macro}{\mdq
%  \begin{macro}{\mdqo
%  \begin{macro}{\
%    All that's left to do now is to  define a couple of comma
%    for reasons of compatibility with \file{german.st
% \changes{germanb-2.6f}{1998/06/07}{Now use \cs{shorthandon}
%    \cs{shorthandoff
%    \begin{macroco
\def\mdqon{\shorthandon{
\def\mdqoff{\shorthandoff{
\def\ck{\allowhyphens\discretionary{k-}{k}{ck}\allowhyphe
%    \end{macroco
%  \end{mac
%  \end{mac
%  \end{mac

%    The macro |\ldf@finish| takes care of looking fo
%    configuration file, setting the main language to be switched
%    at |\begin{document}| and resetting the category code
%    \texttt{@} to its original val
% \changes{germanb-2.6d}{1996/11/02}{Now use \cs{ldf@finish} to w
%    u
%    \begin{macroco
\ldf@finish\CurrentOpt
%</co
%    \end{macroco

% \Fin

%% \CharacterTa
%%  {Upper-case    \A\B\C\D\E\F\G\H\I\J\K\L\M\N\O\P\Q\R\S\T\U\V\W\X\
%%   Lower-case    \a\b\c\d\e\f\g\h\i\j\k\l\m\n\o\p\q\r\s\t\u\v\w\x\
%%   Digits        \0\1\2\3\4\5\6\7\
%%   Exclamation   \!     Double quote  \"     Hash (number)
%%   Dollar        \$     Percent       \%     Ampersand
%%   Acute accent  \'     Left paren    \(     Right paren
%%   Asterisk      \*     Plus          \+     Comma
%%   Minus         \-     Point         \.     Solidus
%%   Colon         \:     Semicolon     \;     Less than
%%   Equals        \=     Greater than  \>     Question mark
%%   Commercial at \@     Left bracket  \[     Backslash
%%   Right bracket \]     Circumflex    \^     Underscore
%%   Grave accent  \`     Left brace    \{     Vertical bar
%%   Right brace   \}     Tilde

\endin
}
\bbl@tempa{bahasa}{\input{bahasai.ldf}}
\bbl@tempa{bahasai}{\input{bahasai.ldf}}
\bbl@tempa{bahasam}{% \iffalse meta-com

% Copyright 1989-2008 Johannes L. Braams and any individual aut
% listed elsewhere in this file.  All rights reser

% This file is part of the Babel sys
% ----------------------------------

% It may be distributed and/or modified under
% conditions of the LaTeX Project Public License, either version
% of this license or (at your option) any later vers
% The latest version of this license i
%   http://www.latex-project.org/lppl
% and version 1.3 or later is part of all distributions of L
% version 2003/12/01 or la

% This work has the LPPL maintenance status "maintain

% The Current Maintainer of this work is Johannes Bra

% The list of all files belonging to the Babel syste
% given in the file `manifest.bbl. See also `legal.bbl' for additi
% informat

% The list of derived (unpacked) files belonging to the distribu
% and covered by LPPL is defined by the unpacking scripts (
% extension .ins) which are part of the distribut
%
% \CheckSum{
%\iff
%    Tell the \LaTeX\ system who we are and write an entry on
%    transcr
%<*
\ProvidesFile{bahasam.
%</
%<code>\ProvidesLanguage{baha

%\ProvidesFile{bahasam.
       [2008/01/27 v1.0k Bahasa Malaysia support from the babel sys
%\iff
%% File `bahasam.
%% Babel package for LaTeX versio
%% Copyright (C) 1989 -
%%           by Johannes Braams, TeX

%% Bahasa Malaysia Language Definition
%% Copyright (C) 1994 -
%%           by J"org Knappen, (joerg.knappen at alpha.ntp.springer
%              Terry Mart (mart at vkpmzd.kph.uni-mainz
%              Institut f\"ur Kernph
%              Johannes Gutenberg-Universit\"at M
%              D-55099 M
%              Ger

%% Copyright (C) 2005,
%%           by Bob Margolis, (bob.margolis at ntlworld.
%              derived from J"ork Knappen's work - see ab
%%           [With help from Awangku Merali Pengiran Mohamed (Saraw
%               gratefully acknowled
%               Yate
%

%% Please report errors to: Bob Marg
%%                          bob.margolis at ntlworld
%%                          J.L. Br
%%                          babel at braams.xs4al

%    This file is part of the babel system, it provides the so
%    code for the  Bahasa Malaysia language defini
%    file.  The original version of this file was written by T
%    Mart (mart@vkpmzd.kph.uni-mainz.de) and J"org Kna
%    (knappen@vkpmzd.kph.uni-mainz.
%<*filedri
\documentclass{ltx
\newcommand*\TeXhax{\TeX
\newcommand*\babel{\textsf{bab
\newcommand*\langvar{$\langle \it lang \rang
\newcommand*\note[
\newcommand*\Lopt[1]{\textsf{
\newcommand*\file[1]{\texttt{
\begin{docum
 \DocInput{bahasam.
\end{docum
%</filedri

% \GetFileInfo{bahasam.

% \changes{bahasa-0.9c}{1994/06/26}{Removed the use of \cs{filed
%    and moved identification after the loading of \file{babel.d
% \changes{bahasa-1.0d}{1996/07/10}{Replaced \cs{undefined}
%    \cs{@undefined} and \cs{empty} with \cs{@empty} for consist
%    with \LaT
% \changes{bahasa-1.0e}{1996/10/10}{Moved the definitio
%    \cs{atcatcode} right to the beginni
% \changes{bahasam-0.9f}{2005/11/22}{A number of changes to make
%    specific to Bahasa Maya

%  \section{The Bahasa Malaysia langu

%    The file \file{\filename}\footnote{The file described in
%    section has version number \fileversion\ and was last revise
%    \filedate.}  defines all the language definition macros for
%    Bahasa Malaysia language. Bahasa just m
%    `language' in Bahasa Malaysia. A number of terms differ from those
%    in bahasa indone

%    For this language currently no special definitions are neede
%    availa

% \StopEventual

%    The macro |\LdfInit| takes care of preventing that this fil
%    loaded more than once, checking the category code of
%    \texttt{@} sign,
% \changes{bahasa-1.0e}{1996/11/02}{Now use \cs{LdfInit} to per
%    initial che
% \changes{bahasam-v1.0j}{2005/11/23}{Make it possible that this
%    is loaded by variuos opti
%    \begin{macroc
%<*c
\LdfInit\CurrentOption{date\CurrentOpt
%    \end{macroc

%    When this file is read as an option, i.e. by the |\usepack
%    command, \texttt{bahasa} could be an `unknown' language in w
%    case we have to make it known. So we check for the existenc
%    |\l@bahasa| to see whether we have to do something h

%    For both Bahasa Malaysia and Bahasa Indonesia the same se
%    hyphenation patterns can be used which are available in the
%    \file{inhyph.tex}. However it could be loaded using any of
%    possible Babel options fot the Malaysian and Indone
%    languase. So first we try to find out whether this is the c

% \changes{bahasa-0.9c}{1994/06/26}{Now use \cs{@patterns} to pro
%    the warn
%    \begin{macroc
\ifx\l@malay\@undef
  \ifx\l@meyalu\@undef
    \ifx\l@bahasam\@undef
      \ifx\l@bahasa\@undef
        \ifx\l@bahasai\@undef
          \ifx\l@indon\@undef
            \ifx\l@indonesian\@undef
              \@nopatterns{Bahasa Malay
              \adddialect\l@malay0\r
            \
              \let\l@malay\l@indone

          \
            \let\l@malay\l@i

        \
          \let\l@malay\l@bah

      \
        \let\l@malay\l@ba

    \
      \let\l@malay\l@bah

  \
    \let\l@malay\l@me


%    \end{macroc

%    Now that we are sure the |\l@malay| has some valid definitio
%    need to make sure that a name to access the hyphenation patte
%    corresponding to the option used, is availa
%    \begin{macroc
\expandafter\expandafter\expandafter
  \expandafter\cs
  \expandafter l\expandafter @\CurrentOption\endcs
  \l@m
%    \end{macroc

%    The next step consists of defining commands to switch to
%    from) the Bahasa langu

% \begin{macro}{\captionsbaha
%    The macro |\captionsbahasam| defines all strings used in the
%    standard documentclasses provided with \La
% \changes{bahasa-1.0b}{1995/07/04}{Added \cs{proofname}
%    AMS-\La
% \changes{bahasa-1.0d}{1996/07/09}{Replaced `Proof' by `Bu
%    (PR2214
% \changes{bahasa-1.0h}{2000/09/19}{Added \cs{glossaryna
% \changes{bahasa-1.0i}{2003/11/17}{Inserted translation for Gloss
% \changes{bahasam-1.0k}{2008/01/27}{Inserted changes from Awangku Mera
%    \begin{macroc
\@namedef{captions\CurrentOptio
  \def\prefacename{Praka
  \def\refname{Rujuk
  \def\abstractname{Abstrak}% (sometime it's called 'intis
                              %  or 'ikhtis
  \def\bibname{Bibliogra
  \def\chaptername{B
  \def\appendixname{Lampir
  \def\contentsname{Kandung
  \def\listfigurename{Senarai Gamb
  \def\listtablename{Senarai Jadu
  \def\indexname{Inde
  \def\figurename{Gamb
  \def\tablename{Jadu
  \def\partname{Bahagi
%  Subject:  Per
%  From:
  \def\enclname{Lampir
  \def\ccname{sk}% (short form for 'Salinan Kepa
  \def\headtoname{Kepa
  \def\pagename{Halam
%  Notes (Endnotes): Cat
  \def\seename{sila  ruj
  \def\alsoname{rujuk ju
  \def\proofname{Buk
  \def\glossaryname{Istil

%    \end{macroc
% \end{ma

% \begin{macro}{\datebaha
%    The macro |\datebahasam| redefines the command |\today| to pro
%    Bahasa Malaysian da
% \changes{bahasa-1.0f}{1997/10/01}{Use \cs{edef} to define \cs{tod
% \changes{bahasa~1.0f}{1998/03/28}{use \cs{def} instead of \cs{e
%    to save mem
% \changes{bahasa-1.0g}{1999/03/12}{Februari should be spelle
%    Pebru
% \changes{bahasam-1.0k}{2008/01/27}{Februari restored to BM spelli
%    see Collins Kamus Dwibahasa 2
%    \begin{macroc
\@namedef{date\CurrentOptio
  \def\today{\number\day~\ifcase\mont
    Januari\or Februari\or Mac\or April\or Mei\or Ju
    Julai\or Ogos\or September\or Oktober\or November\or Disembe
    \space \number\ye
%    \end{macroc
% \end{ma


% \begin{macro}{\extrasbaha
% \begin{macro}{\noextrasbaha
%    The macro |\extrasbahasa| will perform all the extra definit
%    needed for the Bahasa language. The macro |\extrasbahasa| is
%    to cancel the actions of |\extrasbahasa|.  For the moment t
%    macros are empty but they are defined for compatibility with
%    other language definition fi

%    \begin{macroc
\@namedef{extras\CurrentOptio
\@namedef{noextras\CurrentOptio
%    \end{macroc
% \end{ma
% \end{ma

%  \begin{macro}{\bahasamhyphenm
%    The bahasam hyphenation patterns should be used
%    |\lefthyphenmin| set to~2 and |\righthyphenmin| set t
% \changes{bahasa-1.0e}{1996/08/07}{use \cs{bahasamhyphenmins} to s
%    the correct val
% \changes{bahasa-1.0h}{2000/09/22}{Now use \cs{providehyphenmins
%    provide a default va
%    \begin{macroc
\providehyphenmins{\CurrentOption}{\tw@\
%    \end{macroc
%  \end{ma

%    The macro |\ldf@finish| takes care of looking f
%    configuration file, setting the main language to be switche
%    at |\begin{document}| and resetting the category cod
%    \texttt{@} to its original va
% \changes{bahasa-1.0e}{1996/11/02}{Now use \cs{ldf@finish} to wrap
%    \begin{macroc
\ldf@finish{\CurrentOpt
%</c
%    \end{macroc

% \Fi

%% \CharacterT
%%  {Upper-case    \A\B\C\D\E\F\G\H\I\J\K\L\M\N\O\P\Q\R\S\T\U\V\W\X
%%   Lower-case    \a\b\c\d\e\f\g\h\i\j\k\l\m\n\o\p\q\r\s\t\u\v\w\x
%%   Digits        \0\1\2\3\4\5\6\7
%%   Exclamation   \!     Double quote  \"     Hash (number
%%   Dollar        \$     Percent       \%     Ampersand
%%   Acute accent  \'     Left paren    \(     Right paren
%%   Asterisk      \*     Plus          \+     Comma
%%   Minus         \-     Point         \.     Solidus
%%   Colon         \:     Semicolon     \;     Less than
%%   Equals        \=     Greater than  \>     Question mar
%%   Commercial at \@     Left bracket  \[     Backslash
%%   Right bracket \]     Circumflex    \^     Underscore
%%   Grave accent  \`     Left brace    \{     Vertical bar
%%   Right brace   \}     Tilde

\endi
}
\bbl@tempa{brazil}{% \iffalse meta-comment
%
% Copyright 1989-2008 Johannes L. Braams and any individual authors
% listed elsewhere in this file.  All rights reserved.
% 
% This file is part of the Babel system.
% --------------------------------------
% 
% It may be distributed and/or modified under the
% conditions of the LaTeX Project Public License, either version 1.3
% of this license or (at your option) any later version.
% The latest version of this license is in
%   http://www.latex-project.org/lppl.txt
% and version 1.3 or later is part of all distributions of LaTeX
% version 2003/12/01 or later.
% 
% This work has the LPPL maintenance status "maintained".
% 
% The Current Maintainer of this work is Johannes Braams.
% 
% The list of all files belonging to the Babel system is
% given in the file `manifest.bbl. See also `legal.bbl' for additional
% information.
% 
% The list of derived (unpacked) files belonging to the distribution
% and covered by LPPL is defined by the unpacking scripts (with
% extension .ins) which are part of the distribution.
% \fi
% \CheckSum{320}
% \iffalse
%    Tell the \LaTeX\ system who we are and write an entry on the
%    transcript.
%<*dtx>
\ProvidesFile{portuges.dtx}
%</dtx>
%<code>\ProvidesLanguage{portuges}
%\fi
%\ProvidesFile{portuges.dtx}
        [2008/03/18 v1.2q Portuguese support from the babel system]
%\iffalse
%% File `portuges.dtx'
%% Babel package for LaTeX version 2e
%% Copyright (C) 1989 - 2008
%%           by Johannes Braams, TeXniek
%
%% Portuguese Language Definition File
%% Copyright (C) 1989 - 2008
%%           by Johannes Braams, TeXniek
%
%% Please report errors to: J.L. Braams
%%                          babel at braams.cistron.nl
%
%    This file is part of the babel system, it provides the source
%    code for the Portuguese language definition file.  The Portuguese
%    words were contributed by Jose Pedro Ramalhete, (JRAMALHE@CERNVM
%    or Jose-Pedro_Ramalhete@MACMAIL).
%
%    Arnaldo Viegas de Lima <arnaldo@VNET.IBM.COM> contributed
%    brasilian translations and suggestions for enhancements.
%<*filedriver>
\documentclass{ltxdoc}
\newcommand*\TeXhax{\TeX hax}
\newcommand*\babel{\textsf{babel}}
\newcommand*\langvar{$\langle \it lang \rangle$}
\newcommand*\note[1]{}
\newcommand*\Lopt[1]{\textsf{#1}}
\newcommand*\file[1]{\texttt{#1}}
\begin{document}
 \DocInput{portuges.dtx}
\end{document}
%</filedriver>
%\fi
%
% \GetFileInfo{portuges.dtx}
%
% \changes{portuges-1.0a}{1991/07/15}{Renamed \file{babel.sty} in
%    \file{babel.com}}
% \changes{portuges-1.1}{1992/02/16}{Brought up-to-date with babel 3.2a}
% \changes{portuges-1.2}{1994/02/26}{Update for \LaTeXe}
% \changes{portuges-1.2d}{1994/06/26}{Removed the use of \cs{filedate}
%    and moved identification after the loading of \file{babel.def}}
% \changes{portuges-1.2g}{1995/06/04}{Enhanced support for brasilian}
% \changes{portuges-1.2j}{1996/07/11}{Replaced \cs{undefined} with
%    \cs{@undefined} and \cs{empty} with \cs{@empty} for consistency
%    with \LaTeX} 
% \changes{portuges-1.2j}{1996/10/10}{Moved the definition of
%    \cs{atcatcode} right to the beginning.}
%
%  \section{The Portuguese language}
%
%    The file \file{\filename}\footnote{The file described in this
%    section has version number \fileversion\ and was last revised on
%    \filedate.  Contributions were made by Jose Pedro Ramalhete
%    (\texttt{JRAMALHE@CERNVM} or
%    \texttt{Jose-Pedro\_Ramalhete@MACMAIL}) and Arnaldo Viegas de
%    Lima \texttt{arnaldo@VNET.IBM.COM}.}  defines all the
%    language-specific macros for the Portuguese language as well as
%    for the Brasilian version of this language.
%
%    For this language the character |"| is made active. In
%    table~\ref{tab:port-quote} an overview is given of its purpose.
%
%    \begin{table}[htb]
%     \centering
%     \begin{tabular}{lp{8cm}}
%       \verb="|= & disable ligature at this position.\\
%        |"-| & an explicit hyphen sign, allowing hyphenation
%               in the rest of the word.\\
%        |""| & like \verb="-=, but producing no hyphen sign (for
%              words that should break at some sign such as
%              ``entrada/salida.''\\
%        |"<| & for French left double quotes (similar to $<<$).\\
%        |">| & for French right double quotes (similar to $>>$).\\
%        |\-| & like the old |\-|, but allowing hyphenation
%               in the rest of the word. \\
%     \end{tabular}
%     \caption{The extra definitions made by \file{portuges.ldf}}
%     \label{tab:port-quote}
%    \end{table}
%
% \StopEventually{}
%
%    The macro |\LdfInit| takes care of preventing that this file is
%    loaded more than once, checking the category code of the
%    \texttt{@} sign, etc.
% \changes{portuges-1.2j}{1996/11/03}{Now use \cs{LdfInit} to perform
%    initial checks} 
%    \begin{macrocode}
%<*code>
\LdfInit\CurrentOption{captions\CurrentOption}
%    \end{macrocode}
%
%    When this file is read as an option, i.e. by the |\usepackage|
%    command, \texttt{portuges} will be an `unknown' language in which
%    case we have to make it known. So we check for the existence of
%    |\l@portuges| to see whether we have to do something here. Since
%    it is possible to load this file with any of the following four
%    options to babel: \Lopt{portuges}, \Lopt{portuguese},
%    \Lopt{brazil} and \Lopt{brazilian} we also allow that the
%    hyphenation patterns are loaded under any of these four names. We
%    just have to find out which one was used.
%
% \changes{portuges-1.0b}{1991/10/29}{Removed use of cs{@ifundefined}}
% \changes{portuges-1.1}{1992/02/16}{Added a warning when no
%    hyphenation patterns were loaded.}
% \changes{portuges-1.2d}{1994/06/26}{Now use \cs{@nopatterns} to
%    produce the warning}
%    \begin{macrocode}
\ifx\l@portuges\@undefined
  \ifx\l@portuguese\@undefined
    \ifx\l@brazil\@undefined
      \ifx\l@brazilian\@undefined
        \@nopatterns{Portuguese}
        \adddialect\l@portuges0
      \else
        \let\l@portuges\l@brazilian
      \fi
    \else
      \let\l@portuges\l@brazil
    \fi
  \else
    \let\l@portuges\l@portuguese
  \fi
\fi
%    \end{macrocode}
%    By now |\l@portuges| is defined. When the language definition
%    file was loaded under a different name we make sure that the
%    hyphenation patterns can be found.
%    \begin{macrocode}
\expandafter\ifx\csname l@\CurrentOption\endcsname\relax
  \expandafter\let\csname l@\CurrentOption\endcsname\l@portuges
\fi
%    \end{macrocode}
%
%    Now we have to decide whether this language definition file was
%    loaded for Portuguese or Brasilian use. This can be done by
%    checking the contents of |\CurrentOption|. When it doesn't
%    contain either `portuges' or `portuguese' we make |\bbl@tempb|
%    empty. 
%    \begin{macrocode}
\def\bbl@tempa{portuguese}
\ifx\CurrentOption\bbl@tempa
  \let\bbl@tempb\@empty
\else
  \def\bbl@tempa{portuges}
  \ifx\CurrentOption\bbl@tempa
    \let\bbl@tempb\@empty
  \else
    \def\bbl@tempb{brazil}
  \fi
\fi
\ifx\bbl@tempb\@empty
%    \end{macrocode}
%
%    The next step consists of defining commands to switch to (and from)
%    the Portuguese language.
%
% \begin{macro}{\captionsportuges}
%    The macro |\captionsportuges| defines all strings used
%    in the four standard documentclasses provided with \LaTeX.
% \changes{portuges-1.1}{1992/02/16}{Added \cs{seename}, \cs{alsoname}
%    and \cs{prefacename}}
% \changes{portuges-1.1}{1993/07/15}{\cs{headpagename} should be
%    \cs{pagename}}
% \changes{portuges-1.2e}{1994/11/09}{Added a few missing
%    translations}
% \changes{portuges-1.2h}{1995/07/04}{Added \cs{proofname} for
%    AMS-\LaTeX}
% \changes{portuges-1.2i}{1995/11/25}{Substituted `Prova' for `Proof'}
%    \begin{macrocode}
  \@namedef{captions\CurrentOption}{%
    \def\prefacename{Pref\'acio}%
    \def\refname{Refer\^encias}%
    \def\abstractname{Resumo}%
    \def\bibname{Bibliografia}%
    \def\chaptername{Cap\'{\i}tulo}%
    \def\appendixname{Ap\^endice}%
%    \end{macrocode}
%    Some discussion took place around the correct translations for
%    `Table of Contents' and `Index'. the translations differ for
%    Portuguese and Brasilian based the following history:
%    \begin{quote}
%      The whole issue is that some books without a real index at the
%      end misused the term `\'Indice' as table of contents. Then,
%      what happens is that some books apeared with `\'Indice' at the
%      begining and a `\'Indice Remissivo' at the end. Remissivo is a
%      redundant word in this case, but was introduced to make up the
%      difference. So in Brasil people started using `Sum\'ario' and
%      `\'Indice Remissivo'. In Portugal this seems not to be very
%      common, therefore we chose `\'Indice' instead of `\'Indice
%      Remissivo'.
%    \end{quote}
%    \begin{macrocode}
    \def\contentsname{Conte\'udo}%
    \def\listfigurename{Lista de Figuras}%
    \def\listtablename{Lista de Tabelas}%
    \def\indexname{\'Indice}%
    \def\figurename{Figura}%
    \def\tablename{Tabela}%
    \def\partname{Parte}%
    \def\enclname{Anexo}%
    \def\ccname{Com c\'opia a}%
    \def\headtoname{Para}%
    \def\pagename{P\'agina}%
    \def\seename{ver}%
    \def\alsoname{ver tamb\'em}%
%    \end{macrocode}
%    An alternate term for `Proof' could be `Prova'.
% \changes{portuges-1.2m}{2000/09/20}{Added \cs{glossaryname}}
% \changes{portuges-1.2p}{2003/05/23}{Substituted `Gloss\'ario' for
%    `Glossary'}
%    \begin{macrocode}
    \def\proofname{Demonstra\c{c}\~ao}%
    \def\glossaryname{Gloss\'ario}%
    }
%    \end{macrocode}
% \end{macro}
%
% \begin{macro}{\dateportuges}
%    The macro |\dateportuges| redefines the command |\today| to
%    produce Portuguese dates.
% \changes{portuges-1.2k}{1997/10/01}{Use \cs{edef} to define
%    \cs{today} to save memory}
% \changes{portuges-1.2k}{1998/03/28}{use \cs{def} instead of
%    \cs{edef}} 
% \changes{portuges-1.2n}{2001/01/27}{Removed spurious space after
%    Dezembro}
%    \begin{macrocode}
  \@namedef{date\CurrentOption}{%
    \def\today{\number\day\space de\space\ifcase\month\or
      Janeiro\or Fevereiro\or Mar\c{c}o\or Abril\or Maio\or Junho\or
      Julho\or Agosto\or Setembro\or Outubro\or Novembro\or Dezembro%
      \fi
      \space de\space\number\year}}
\else
%    \end{macrocode}
% \end{macro}
%
%    For the Brasilian version of these definitions we just add a
%    ``dialect''. 
%    \begin{macrocode}
  \expandafter
    \adddialect\csname l@\CurrentOption\endcsname\l@portuges
%    \end{macrocode}
%
% \begin{macro}{\captionsbrazil}
% \changes{portuges-1.2g}{1995/06/04}{The captions for brasilian and
%    portuguese are different now}
%
%    The ``captions'' are different for both versions of the language,
%    so we define the macro |\captionsbrazil| here.
% \changes{portuges-1.2i}{1995/11/25}{Added \cs{proofname} for
%    AMS-\LaTeX}
% \changes{portuges-1.2m}{2000/09/20}{Added \cs{glossaryname}}
% \changes{portuges-1.2q}{2008/03/18}{Substituted `Gloss\'ario' for
%    `Glossary'}
%    \begin{macrocode}
  \@namedef{captions\CurrentOption}{%
    \def\prefacename{Pref\'acio}%
    \def\refname{Refer\^encias}%
    \def\abstractname{Resumo}%
    \def\bibname{Refer\^encias Bibliogr\'aficas}%
    \def\chaptername{Cap\'{\i}tulo}%
    \def\appendixname{Ap\^endice}%
    \def\contentsname{Sum\'ario}%
    \def\listfigurename{Lista de Figuras}%
    \def\listtablename{Lista de Tabelas}%
    \def\indexname{\'Indice Remissivo}%
    \def\figurename{Figura}%
    \def\tablename{Tabela}%
    \def\partname{Parte}%
    \def\enclname{Anexo}%
    \def\ccname{C\'opia para}%
    \def\headtoname{Para}%
    \def\pagename{P\'agina}%
    \def\seename{veja}%
    \def\alsoname{veja tamb\'em}%
    \def\proofname{Demonstra\c{c}\~ao}%
    \def\glossaryname{Gloss\'ario}%
    }
%    \end{macrocode}
% \end{macro}
%
% \begin{macro}{\datebrazil}
%    The macro |\datebrazil| redefines the command
%    |\today| to produce Brasilian dates, for which the names
%    of the months are not capitalized.
% \changes{portuges-1.2k}{1997/10/01}{Use \cs{edef} to define
%    \cs{today} to save memory}
% \changes{portuges-1.2k}{1998/03/28}{use \cs{def} instead of
%    \cs{edef}} 
% \changes{portuges-1.2n}{2001/01/27}{Removed spurious space after
%    dezembro}
%    \begin{macrocode}
  \@namedef{date\CurrentOption}{%
    \def\today{\number\day\space de\space\ifcase\month\or
      janeiro\or fevereiro\or mar\c{c}o\or abril\or maio\or junho\or
      julho\or agosto\or setembro\or outubro\or novembro\or dezembro%
      \fi
      \space de\space\number\year}}
\fi
%    \end{macrocode}
% \end{macro}
%
%  \begin{macro}{\portugeshyphenmins}
% \changes{portuges-1.2g}{1995/06/04}{Added setting of hyphenmin
%    values}
%    Set correct values for |\lefthyphenmin| and |\righthyphenmin|.
% \changes{portuges-1.2m}{2000/09/22}{Now use \cs{providehyphenmins} to
%    provide a default value}
% \changes{portuges-1.2o}{2001/02/16}{Set \cs{righthyphenmin} to 3 if
%    not provided by the pattern file.}
%    \begin{macrocode}
\providehyphenmins{\CurrentOption}{\tw@\thr@@}
%    \end{macrocode}
%  \end{macro}
%
% \begin{macro}{\extrasportuges}
% \changes{portuges-1.2g}{1995/06/04}{Added the definition of some
%    \texttt{"} shorthands}
% \begin{macro}{\noextrasportuges}
%    The macro |\extrasportuges| will perform all the extra
%    definitions needed for the Portuguese language. The macro
%    |\noextrasportuges| is used to cancel the actions of
%    |\extrasportuges|.
%
%    For Portuguese the \texttt{"} character is made active. This is
%    done once, later on its definition may vary. Other languages in
%    the same document may also use the \texttt{"} character for
%    shorthands; we specify that the portuguese group of shorthands
%    should be used.
%
%    \begin{macrocode}
\initiate@active@char{"}
\@namedef{extras\CurrentOption}{\languageshorthands{portuges}}
\expandafter\addto\csname extras\CurrentOption\endcsname{%
  \bbl@activate{"}}
%    \end{macrocode}
%    Don't forget to turn the shorthands off again.
% \changes{portuges-1.2m}{1999/12/17}{Deactivate shorthands ouside of
%    Basque}
%    \begin{macrocode}
\addto\noextrasportuges{\bbl@deactivate{"}}
%    \end{macrocode}
%    First we define access to the guillemets for quotations,
% \changes{portuges-1.2k}{1997/04/03}{Removed empty groups after
%    guillemot characters}
%    \begin{macrocode}
\declare@shorthand{portuges}{"<}{%
  \textormath{\guillemotleft}{\mbox{\guillemotleft}}}
\declare@shorthand{portuges}{">}{%
  \textormath{\guillemotright}{\mbox{\guillemotright}}}
%    \end{macrocode}
%    then we define two shorthands to be able to specify hyphenation
%    breakpoints that behave a little different from |\-|.
%    \begin{macrocode}
\declare@shorthand{portuges}{"-}{\nobreak-\bbl@allowhyphens}
\declare@shorthand{portuges}{""}{\hskip\z@skip}
%    \end{macrocode}
%    And we want to have a shorthand for disabling a ligature.
%    \begin{macrocode}
\declare@shorthand{portuges}{"|}{%
  \textormath{\discretionary{-}{}{\kern.03em}}{}}
%    \end{macrocode}
% \end{macro}
% \end{macro}
%
%  \begin{macro}{\-}
%
%    All that is left now is the redefinition of |\-|. The new version
%    of |\-| should indicate an extra hyphenation position, while
%    allowing other hyphenation positions to be generated
%    automatically. The standard behaviour of \TeX\ in this respect is
%    very unfortunate for languages such as Dutch and German, where
%    long compound words are quite normal and all one needs is a means
%    to indicate an extra hyphenation position on top of the ones that
%    \TeX\ can generate from the hyphenation patterns.
%    \begin{macrocode}
\expandafter\addto\csname extras\CurrentOption\endcsname{%
  \babel@save\-}
\expandafter\addto\csname extras\CurrentOption\endcsname{%
  \def\-{\allowhyphens\discretionary{-}{}{}\allowhyphens}}
%    \end{macrocode}
%  \end{macro}
%
%  \begin{macro}{\ord}
% \changes{portuges-1.2g}{1995/06/04}{Added macro}
%  \begin{macro}{\ro}
% \changes{portuges-1.2g}{1995/06/04}{Added macro}
%  \begin{macro}{\orda}
% \changes{portuges-1.2g}{1995/06/04}{Added macro}
%  \begin{macro}{\ra}
% \changes{portuges-1.2g}{1995/06/04}{Added macro}
%    We also provide an easy way to typeset ordinals, both in the male
%    (|\ord| or |\ro|) and the female (|orda| or |\ra|) form.
%    \begin{macrocode}
\def\ord{$^{\rm o}$}
\def\orda{$^{\rm a}$}
\let\ro\ord\let\ra\orda
%    \end{macrocode}
%  \end{macro}
%  \end{macro}
%  \end{macro}
%  \end{macro}
%
%    The macro |\ldf@finish| takes care of looking for a
%    configuration file, setting the main language to be switched on
%    at |\begin{document}| and resetting the category code of
%    \texttt{@} to its original value.
% \changes{portuges-1.2j}{1996/11/03}{ow use \cs{ldf@finish} to wrap
%    up} 
%    \begin{macrocode}
\ldf@finish\CurrentOption
%</code>
%    \end{macrocode}
%
% \Finale
%%
%% \CharacterTable
%%  {Upper-case    \A\B\C\D\E\F\G\H\I\J\K\L\M\N\O\P\Q\R\S\T\U\V\W\X\Y\Z
%%   Lower-case    \a\b\c\d\e\f\g\h\i\j\k\l\m\n\o\p\q\r\s\t\u\v\w\x\y\z
%%   Digits        \0\1\2\3\4\5\6\7\8\9
%%   Exclamation   \!     Double quote  \"     Hash (number) \#
%%   Dollar        \$     Percent       \%     Ampersand     \&
%%   Acute accent  \'     Left paren    \(     Right paren   \)
%%   Asterisk      \*     Plus          \+     Comma         \,
%%   Minus         \-     Point         \.     Solidus       \/
%%   Colon         \:     Semicolon     \;     Less than     \<
%%   Equals        \=     Greater than  \>     Question mark \?
%%   Commercial at \@     Left bracket  \[     Backslash     \\
%%   Right bracket \]     Circumflex    \^     Underscore    \_
%%   Grave accent  \`     Left brace    \{     Vertical bar  \|
%%   Right brace   \}     Tilde         \~}
%%
\endinput
}
\bbl@tempa{brazilian}{% \iffalse meta-comment
%
% Copyright 1989-2008 Johannes L. Braams and any individual authors
% listed elsewhere in this file.  All rights reserved.
% 
% This file is part of the Babel system.
% --------------------------------------
% 
% It may be distributed and/or modified under the
% conditions of the LaTeX Project Public License, either version 1.3
% of this license or (at your option) any later version.
% The latest version of this license is in
%   http://www.latex-project.org/lppl.txt
% and version 1.3 or later is part of all distributions of LaTeX
% version 2003/12/01 or later.
% 
% This work has the LPPL maintenance status "maintained".
% 
% The Current Maintainer of this work is Johannes Braams.
% 
% The list of all files belonging to the Babel system is
% given in the file `manifest.bbl. See also `legal.bbl' for additional
% information.
% 
% The list of derived (unpacked) files belonging to the distribution
% and covered by LPPL is defined by the unpacking scripts (with
% extension .ins) which are part of the distribution.
% \fi
% \CheckSum{320}
% \iffalse
%    Tell the \LaTeX\ system who we are and write an entry on the
%    transcript.
%<*dtx>
\ProvidesFile{portuges.dtx}
%</dtx>
%<code>\ProvidesLanguage{portuges}
%\fi
%\ProvidesFile{portuges.dtx}
        [2008/03/18 v1.2q Portuguese support from the babel system]
%\iffalse
%% File `portuges.dtx'
%% Babel package for LaTeX version 2e
%% Copyright (C) 1989 - 2008
%%           by Johannes Braams, TeXniek
%
%% Portuguese Language Definition File
%% Copyright (C) 1989 - 2008
%%           by Johannes Braams, TeXniek
%
%% Please report errors to: J.L. Braams
%%                          babel at braams.cistron.nl
%
%    This file is part of the babel system, it provides the source
%    code for the Portuguese language definition file.  The Portuguese
%    words were contributed by Jose Pedro Ramalhete, (JRAMALHE@CERNVM
%    or Jose-Pedro_Ramalhete@MACMAIL).
%
%    Arnaldo Viegas de Lima <arnaldo@VNET.IBM.COM> contributed
%    brasilian translations and suggestions for enhancements.
%<*filedriver>
\documentclass{ltxdoc}
\newcommand*\TeXhax{\TeX hax}
\newcommand*\babel{\textsf{babel}}
\newcommand*\langvar{$\langle \it lang \rangle$}
\newcommand*\note[1]{}
\newcommand*\Lopt[1]{\textsf{#1}}
\newcommand*\file[1]{\texttt{#1}}
\begin{document}
 \DocInput{portuges.dtx}
\end{document}
%</filedriver>
%\fi
%
% \GetFileInfo{portuges.dtx}
%
% \changes{portuges-1.0a}{1991/07/15}{Renamed \file{babel.sty} in
%    \file{babel.com}}
% \changes{portuges-1.1}{1992/02/16}{Brought up-to-date with babel 3.2a}
% \changes{portuges-1.2}{1994/02/26}{Update for \LaTeXe}
% \changes{portuges-1.2d}{1994/06/26}{Removed the use of \cs{filedate}
%    and moved identification after the loading of \file{babel.def}}
% \changes{portuges-1.2g}{1995/06/04}{Enhanced support for brasilian}
% \changes{portuges-1.2j}{1996/07/11}{Replaced \cs{undefined} with
%    \cs{@undefined} and \cs{empty} with \cs{@empty} for consistency
%    with \LaTeX} 
% \changes{portuges-1.2j}{1996/10/10}{Moved the definition of
%    \cs{atcatcode} right to the beginning.}
%
%  \section{The Portuguese language}
%
%    The file \file{\filename}\footnote{The file described in this
%    section has version number \fileversion\ and was last revised on
%    \filedate.  Contributions were made by Jose Pedro Ramalhete
%    (\texttt{JRAMALHE@CERNVM} or
%    \texttt{Jose-Pedro\_Ramalhete@MACMAIL}) and Arnaldo Viegas de
%    Lima \texttt{arnaldo@VNET.IBM.COM}.}  defines all the
%    language-specific macros for the Portuguese language as well as
%    for the Brasilian version of this language.
%
%    For this language the character |"| is made active. In
%    table~\ref{tab:port-quote} an overview is given of its purpose.
%
%    \begin{table}[htb]
%     \centering
%     \begin{tabular}{lp{8cm}}
%       \verb="|= & disable ligature at this position.\\
%        |"-| & an explicit hyphen sign, allowing hyphenation
%               in the rest of the word.\\
%        |""| & like \verb="-=, but producing no hyphen sign (for
%              words that should break at some sign such as
%              ``entrada/salida.''\\
%        |"<| & for French left double quotes (similar to $<<$).\\
%        |">| & for French right double quotes (similar to $>>$).\\
%        |\-| & like the old |\-|, but allowing hyphenation
%               in the rest of the word. \\
%     \end{tabular}
%     \caption{The extra definitions made by \file{portuges.ldf}}
%     \label{tab:port-quote}
%    \end{table}
%
% \StopEventually{}
%
%    The macro |\LdfInit| takes care of preventing that this file is
%    loaded more than once, checking the category code of the
%    \texttt{@} sign, etc.
% \changes{portuges-1.2j}{1996/11/03}{Now use \cs{LdfInit} to perform
%    initial checks} 
%    \begin{macrocode}
%<*code>
\LdfInit\CurrentOption{captions\CurrentOption}
%    \end{macrocode}
%
%    When this file is read as an option, i.e. by the |\usepackage|
%    command, \texttt{portuges} will be an `unknown' language in which
%    case we have to make it known. So we check for the existence of
%    |\l@portuges| to see whether we have to do something here. Since
%    it is possible to load this file with any of the following four
%    options to babel: \Lopt{portuges}, \Lopt{portuguese},
%    \Lopt{brazil} and \Lopt{brazilian} we also allow that the
%    hyphenation patterns are loaded under any of these four names. We
%    just have to find out which one was used.
%
% \changes{portuges-1.0b}{1991/10/29}{Removed use of cs{@ifundefined}}
% \changes{portuges-1.1}{1992/02/16}{Added a warning when no
%    hyphenation patterns were loaded.}
% \changes{portuges-1.2d}{1994/06/26}{Now use \cs{@nopatterns} to
%    produce the warning}
%    \begin{macrocode}
\ifx\l@portuges\@undefined
  \ifx\l@portuguese\@undefined
    \ifx\l@brazil\@undefined
      \ifx\l@brazilian\@undefined
        \@nopatterns{Portuguese}
        \adddialect\l@portuges0
      \else
        \let\l@portuges\l@brazilian
      \fi
    \else
      \let\l@portuges\l@brazil
    \fi
  \else
    \let\l@portuges\l@portuguese
  \fi
\fi
%    \end{macrocode}
%    By now |\l@portuges| is defined. When the language definition
%    file was loaded under a different name we make sure that the
%    hyphenation patterns can be found.
%    \begin{macrocode}
\expandafter\ifx\csname l@\CurrentOption\endcsname\relax
  \expandafter\let\csname l@\CurrentOption\endcsname\l@portuges
\fi
%    \end{macrocode}
%
%    Now we have to decide whether this language definition file was
%    loaded for Portuguese or Brasilian use. This can be done by
%    checking the contents of |\CurrentOption|. When it doesn't
%    contain either `portuges' or `portuguese' we make |\bbl@tempb|
%    empty. 
%    \begin{macrocode}
\def\bbl@tempa{portuguese}
\ifx\CurrentOption\bbl@tempa
  \let\bbl@tempb\@empty
\else
  \def\bbl@tempa{portuges}
  \ifx\CurrentOption\bbl@tempa
    \let\bbl@tempb\@empty
  \else
    \def\bbl@tempb{brazil}
  \fi
\fi
\ifx\bbl@tempb\@empty
%    \end{macrocode}
%
%    The next step consists of defining commands to switch to (and from)
%    the Portuguese language.
%
% \begin{macro}{\captionsportuges}
%    The macro |\captionsportuges| defines all strings used
%    in the four standard documentclasses provided with \LaTeX.
% \changes{portuges-1.1}{1992/02/16}{Added \cs{seename}, \cs{alsoname}
%    and \cs{prefacename}}
% \changes{portuges-1.1}{1993/07/15}{\cs{headpagename} should be
%    \cs{pagename}}
% \changes{portuges-1.2e}{1994/11/09}{Added a few missing
%    translations}
% \changes{portuges-1.2h}{1995/07/04}{Added \cs{proofname} for
%    AMS-\LaTeX}
% \changes{portuges-1.2i}{1995/11/25}{Substituted `Prova' for `Proof'}
%    \begin{macrocode}
  \@namedef{captions\CurrentOption}{%
    \def\prefacename{Pref\'acio}%
    \def\refname{Refer\^encias}%
    \def\abstractname{Resumo}%
    \def\bibname{Bibliografia}%
    \def\chaptername{Cap\'{\i}tulo}%
    \def\appendixname{Ap\^endice}%
%    \end{macrocode}
%    Some discussion took place around the correct translations for
%    `Table of Contents' and `Index'. the translations differ for
%    Portuguese and Brasilian based the following history:
%    \begin{quote}
%      The whole issue is that some books without a real index at the
%      end misused the term `\'Indice' as table of contents. Then,
%      what happens is that some books apeared with `\'Indice' at the
%      begining and a `\'Indice Remissivo' at the end. Remissivo is a
%      redundant word in this case, but was introduced to make up the
%      difference. So in Brasil people started using `Sum\'ario' and
%      `\'Indice Remissivo'. In Portugal this seems not to be very
%      common, therefore we chose `\'Indice' instead of `\'Indice
%      Remissivo'.
%    \end{quote}
%    \begin{macrocode}
    \def\contentsname{Conte\'udo}%
    \def\listfigurename{Lista de Figuras}%
    \def\listtablename{Lista de Tabelas}%
    \def\indexname{\'Indice}%
    \def\figurename{Figura}%
    \def\tablename{Tabela}%
    \def\partname{Parte}%
    \def\enclname{Anexo}%
    \def\ccname{Com c\'opia a}%
    \def\headtoname{Para}%
    \def\pagename{P\'agina}%
    \def\seename{ver}%
    \def\alsoname{ver tamb\'em}%
%    \end{macrocode}
%    An alternate term for `Proof' could be `Prova'.
% \changes{portuges-1.2m}{2000/09/20}{Added \cs{glossaryname}}
% \changes{portuges-1.2p}{2003/05/23}{Substituted `Gloss\'ario' for
%    `Glossary'}
%    \begin{macrocode}
    \def\proofname{Demonstra\c{c}\~ao}%
    \def\glossaryname{Gloss\'ario}%
    }
%    \end{macrocode}
% \end{macro}
%
% \begin{macro}{\dateportuges}
%    The macro |\dateportuges| redefines the command |\today| to
%    produce Portuguese dates.
% \changes{portuges-1.2k}{1997/10/01}{Use \cs{edef} to define
%    \cs{today} to save memory}
% \changes{portuges-1.2k}{1998/03/28}{use \cs{def} instead of
%    \cs{edef}} 
% \changes{portuges-1.2n}{2001/01/27}{Removed spurious space after
%    Dezembro}
%    \begin{macrocode}
  \@namedef{date\CurrentOption}{%
    \def\today{\number\day\space de\space\ifcase\month\or
      Janeiro\or Fevereiro\or Mar\c{c}o\or Abril\or Maio\or Junho\or
      Julho\or Agosto\or Setembro\or Outubro\or Novembro\or Dezembro%
      \fi
      \space de\space\number\year}}
\else
%    \end{macrocode}
% \end{macro}
%
%    For the Brasilian version of these definitions we just add a
%    ``dialect''. 
%    \begin{macrocode}
  \expandafter
    \adddialect\csname l@\CurrentOption\endcsname\l@portuges
%    \end{macrocode}
%
% \begin{macro}{\captionsbrazil}
% \changes{portuges-1.2g}{1995/06/04}{The captions for brasilian and
%    portuguese are different now}
%
%    The ``captions'' are different for both versions of the language,
%    so we define the macro |\captionsbrazil| here.
% \changes{portuges-1.2i}{1995/11/25}{Added \cs{proofname} for
%    AMS-\LaTeX}
% \changes{portuges-1.2m}{2000/09/20}{Added \cs{glossaryname}}
% \changes{portuges-1.2q}{2008/03/18}{Substituted `Gloss\'ario' for
%    `Glossary'}
%    \begin{macrocode}
  \@namedef{captions\CurrentOption}{%
    \def\prefacename{Pref\'acio}%
    \def\refname{Refer\^encias}%
    \def\abstractname{Resumo}%
    \def\bibname{Refer\^encias Bibliogr\'aficas}%
    \def\chaptername{Cap\'{\i}tulo}%
    \def\appendixname{Ap\^endice}%
    \def\contentsname{Sum\'ario}%
    \def\listfigurename{Lista de Figuras}%
    \def\listtablename{Lista de Tabelas}%
    \def\indexname{\'Indice Remissivo}%
    \def\figurename{Figura}%
    \def\tablename{Tabela}%
    \def\partname{Parte}%
    \def\enclname{Anexo}%
    \def\ccname{C\'opia para}%
    \def\headtoname{Para}%
    \def\pagename{P\'agina}%
    \def\seename{veja}%
    \def\alsoname{veja tamb\'em}%
    \def\proofname{Demonstra\c{c}\~ao}%
    \def\glossaryname{Gloss\'ario}%
    }
%    \end{macrocode}
% \end{macro}
%
% \begin{macro}{\datebrazil}
%    The macro |\datebrazil| redefines the command
%    |\today| to produce Brasilian dates, for which the names
%    of the months are not capitalized.
% \changes{portuges-1.2k}{1997/10/01}{Use \cs{edef} to define
%    \cs{today} to save memory}
% \changes{portuges-1.2k}{1998/03/28}{use \cs{def} instead of
%    \cs{edef}} 
% \changes{portuges-1.2n}{2001/01/27}{Removed spurious space after
%    dezembro}
%    \begin{macrocode}
  \@namedef{date\CurrentOption}{%
    \def\today{\number\day\space de\space\ifcase\month\or
      janeiro\or fevereiro\or mar\c{c}o\or abril\or maio\or junho\or
      julho\or agosto\or setembro\or outubro\or novembro\or dezembro%
      \fi
      \space de\space\number\year}}
\fi
%    \end{macrocode}
% \end{macro}
%
%  \begin{macro}{\portugeshyphenmins}
% \changes{portuges-1.2g}{1995/06/04}{Added setting of hyphenmin
%    values}
%    Set correct values for |\lefthyphenmin| and |\righthyphenmin|.
% \changes{portuges-1.2m}{2000/09/22}{Now use \cs{providehyphenmins} to
%    provide a default value}
% \changes{portuges-1.2o}{2001/02/16}{Set \cs{righthyphenmin} to 3 if
%    not provided by the pattern file.}
%    \begin{macrocode}
\providehyphenmins{\CurrentOption}{\tw@\thr@@}
%    \end{macrocode}
%  \end{macro}
%
% \begin{macro}{\extrasportuges}
% \changes{portuges-1.2g}{1995/06/04}{Added the definition of some
%    \texttt{"} shorthands}
% \begin{macro}{\noextrasportuges}
%    The macro |\extrasportuges| will perform all the extra
%    definitions needed for the Portuguese language. The macro
%    |\noextrasportuges| is used to cancel the actions of
%    |\extrasportuges|.
%
%    For Portuguese the \texttt{"} character is made active. This is
%    done once, later on its definition may vary. Other languages in
%    the same document may also use the \texttt{"} character for
%    shorthands; we specify that the portuguese group of shorthands
%    should be used.
%
%    \begin{macrocode}
\initiate@active@char{"}
\@namedef{extras\CurrentOption}{\languageshorthands{portuges}}
\expandafter\addto\csname extras\CurrentOption\endcsname{%
  \bbl@activate{"}}
%    \end{macrocode}
%    Don't forget to turn the shorthands off again.
% \changes{portuges-1.2m}{1999/12/17}{Deactivate shorthands ouside of
%    Basque}
%    \begin{macrocode}
\addto\noextrasportuges{\bbl@deactivate{"}}
%    \end{macrocode}
%    First we define access to the guillemets for quotations,
% \changes{portuges-1.2k}{1997/04/03}{Removed empty groups after
%    guillemot characters}
%    \begin{macrocode}
\declare@shorthand{portuges}{"<}{%
  \textormath{\guillemotleft}{\mbox{\guillemotleft}}}
\declare@shorthand{portuges}{">}{%
  \textormath{\guillemotright}{\mbox{\guillemotright}}}
%    \end{macrocode}
%    then we define two shorthands to be able to specify hyphenation
%    breakpoints that behave a little different from |\-|.
%    \begin{macrocode}
\declare@shorthand{portuges}{"-}{\nobreak-\bbl@allowhyphens}
\declare@shorthand{portuges}{""}{\hskip\z@skip}
%    \end{macrocode}
%    And we want to have a shorthand for disabling a ligature.
%    \begin{macrocode}
\declare@shorthand{portuges}{"|}{%
  \textormath{\discretionary{-}{}{\kern.03em}}{}}
%    \end{macrocode}
% \end{macro}
% \end{macro}
%
%  \begin{macro}{\-}
%
%    All that is left now is the redefinition of |\-|. The new version
%    of |\-| should indicate an extra hyphenation position, while
%    allowing other hyphenation positions to be generated
%    automatically. The standard behaviour of \TeX\ in this respect is
%    very unfortunate for languages such as Dutch and German, where
%    long compound words are quite normal and all one needs is a means
%    to indicate an extra hyphenation position on top of the ones that
%    \TeX\ can generate from the hyphenation patterns.
%    \begin{macrocode}
\expandafter\addto\csname extras\CurrentOption\endcsname{%
  \babel@save\-}
\expandafter\addto\csname extras\CurrentOption\endcsname{%
  \def\-{\allowhyphens\discretionary{-}{}{}\allowhyphens}}
%    \end{macrocode}
%  \end{macro}
%
%  \begin{macro}{\ord}
% \changes{portuges-1.2g}{1995/06/04}{Added macro}
%  \begin{macro}{\ro}
% \changes{portuges-1.2g}{1995/06/04}{Added macro}
%  \begin{macro}{\orda}
% \changes{portuges-1.2g}{1995/06/04}{Added macro}
%  \begin{macro}{\ra}
% \changes{portuges-1.2g}{1995/06/04}{Added macro}
%    We also provide an easy way to typeset ordinals, both in the male
%    (|\ord| or |\ro|) and the female (|orda| or |\ra|) form.
%    \begin{macrocode}
\def\ord{$^{\rm o}$}
\def\orda{$^{\rm a}$}
\let\ro\ord\let\ra\orda
%    \end{macrocode}
%  \end{macro}
%  \end{macro}
%  \end{macro}
%  \end{macro}
%
%    The macro |\ldf@finish| takes care of looking for a
%    configuration file, setting the main language to be switched on
%    at |\begin{document}| and resetting the category code of
%    \texttt{@} to its original value.
% \changes{portuges-1.2j}{1996/11/03}{ow use \cs{ldf@finish} to wrap
%    up} 
%    \begin{macrocode}
\ldf@finish\CurrentOption
%</code>
%    \end{macrocode}
%
% \Finale
%%
%% \CharacterTable
%%  {Upper-case    \A\B\C\D\E\F\G\H\I\J\K\L\M\N\O\P\Q\R\S\T\U\V\W\X\Y\Z
%%   Lower-case    \a\b\c\d\e\f\g\h\i\j\k\l\m\n\o\p\q\r\s\t\u\v\w\x\y\z
%%   Digits        \0\1\2\3\4\5\6\7\8\9
%%   Exclamation   \!     Double quote  \"     Hash (number) \#
%%   Dollar        \$     Percent       \%     Ampersand     \&
%%   Acute accent  \'     Left paren    \(     Right paren   \)
%%   Asterisk      \*     Plus          \+     Comma         \,
%%   Minus         \-     Point         \.     Solidus       \/
%%   Colon         \:     Semicolon     \;     Less than     \<
%%   Equals        \=     Greater than  \>     Question mark \?
%%   Commercial at \@     Left bracket  \[     Backslash     \\
%%   Right bracket \]     Circumflex    \^     Underscore    \_
%%   Grave accent  \`     Left brace    \{     Vertical bar  \|
%%   Right brace   \}     Tilde         \~}
%%
\endinput
}
\bbl@tempa{british}{%%
%% This file will generate fast loadable files and documentation
%% driver files from the doc files in this package when run through
%% LaTeX or TeX.
%%
%% Copyright 1989-2005 Johannes L. Braams and any individual authors
%% listed elsewhere in this file.  All rights reserved.
%% 
%% This file is part of the Babel system.
%% --------------------------------------
%% 
%% It may be distributed and/or modified under the
%% conditions of the LaTeX Project Public License, either version 1.3
%% of this license or (at your option) any later version.
%% The latest version of this license is in
%%   http://www.latex-project.org/lppl.txt
%% and version 1.3 or later is part of all distributions of LaTeX
%% version 2003/12/01 or later.
%% 
%% This work has the LPPL maintenance status "maintained".
%% 
%% The Current Maintainer of this work is Johannes Braams.
%% 
%% The list of all files belonging to the LaTeX base distribution is
%% given in the file `manifest.bbl. See also `legal.bbl' for additional
%% information.
%% 
%% The list of derived (unpacked) files belonging to the distribution
%% and covered by LPPL is defined by the unpacking scripts (with
%% extension .ins) which are part of the distribution.
%%
%% --------------- start of docstrip commands ------------------
%%
\def\filedate{1999/04/11}
\def\batchfile{english.ins}
\input docstrip.tex

{\ifx\generate\undefined
\Msg{**********************************************}
\Msg{*}
\Msg{* This installation requires docstrip}
\Msg{* version 2.3c or later.}
\Msg{*}
\Msg{* An older version of docstrip has been input}
\Msg{*}
\Msg{**********************************************}
\errhelp{Move or rename old docstrip.tex.}
\errmessage{Old docstrip in input path}
\batchmode
\csname @@end\endcsname
\fi}

\declarepreamble\mainpreamble
This is a generated file.

Copyright 1989-2005 Johannes L. Braams and any individual authors
listed elsewhere in this file.  All rights reserved.

This file was generated from file(s) of the Babel system.
---------------------------------------------------------

It may be distributed and/or modified under the
conditions of the LaTeX Project Public License, either version 1.3
of this license or (at your option) any later version.
The latest version of this license is in
  http://www.latex-project.org/lppl.txt
and version 1.3 or later is part of all distributions of LaTeX
version 2003/12/01 or later.

This work has the LPPL maintenance status "maintained".

The Current Maintainer of this work is Johannes Braams.

This file may only be distributed together with a copy of the Babel
system. You may however distribute the Babel system without
such generated files.

The list of all files belonging to the Babel distribution is
given in the file `manifest.bbl'. See also `legal.bbl for additional
information.

The list of derived (unpacked) files belonging to the distribution
and covered by LPPL is defined by the unpacking scripts (with
extension .ins) which are part of the distribution.
\endpreamble

\declarepreamble\fdpreamble
This is a generated file.

Copyright 1989-2005 Johannes L. Braams and any individual authors
listed elsewhere in this file.  All rights reserved.

This file was generated from file(s) of the Babel system.
---------------------------------------------------------

It may be distributed and/or modified under the
conditions of the LaTeX Project Public License, either version 1.3
of this license or (at your option) any later version.
The latest version of this license is in
  http://www.latex-project.org/lppl.txt
and version 1.3 or later is part of all distributions of LaTeX
version 2003/12/01 or later.

This work has the LPPL maintenance status "maintained".

The Current Maintainer of this work is Johannes Braams.

This file may only be distributed together with a copy of the Babel
system. You may however distribute the Babel system without
such generated files.

The list of all files belonging to the Babel distribution is
given in the file `manifest.bbl'. See also `legal.bbl for additional
information.

In particular, permission is granted to customize the declarations in
this file to serve the needs of your installation.

However, NO PERMISSION is granted to distribute a modified version
of this file under its original name.

\endpreamble

\keepsilent

\usedir{tex/generic/babel} 

\usepreamble\mainpreamble
\generate{\file{english.ldf}{\from{english.dtx}{code}}
          }
\usepreamble\fdpreamble

\ifToplevel{
\Msg{***********************************************************}
\Msg{*}
\Msg{* To finish the installation you have to move the following}
\Msg{* files into a directory searched by TeX:}
\Msg{*}
\Msg{* \space\space All *.def, *.fd, *.ldf, *.sty}
\Msg{*}
\Msg{* To produce the documentation run the files ending with}
\Msg{* '.dtx' and `.fdd' through LaTeX.}
\Msg{*}
\Msg{* Happy TeXing}
\Msg{***********************************************************}
}
 
\endinput
}
\bbl@tempa{canadian}{%%
%% This file will generate fast loadable files and documentation
%% driver files from the doc files in this package when run through
%% LaTeX or TeX.
%%
%% Copyright 1989-2005 Johannes L. Braams and any individual authors
%% listed elsewhere in this file.  All rights reserved.
%% 
%% This file is part of the Babel system.
%% --------------------------------------
%% 
%% It may be distributed and/or modified under the
%% conditions of the LaTeX Project Public License, either version 1.3
%% of this license or (at your option) any later version.
%% The latest version of this license is in
%%   http://www.latex-project.org/lppl.txt
%% and version 1.3 or later is part of all distributions of LaTeX
%% version 2003/12/01 or later.
%% 
%% This work has the LPPL maintenance status "maintained".
%% 
%% The Current Maintainer of this work is Johannes Braams.
%% 
%% The list of all files belonging to the LaTeX base distribution is
%% given in the file `manifest.bbl. See also `legal.bbl' for additional
%% information.
%% 
%% The list of derived (unpacked) files belonging to the distribution
%% and covered by LPPL is defined by the unpacking scripts (with
%% extension .ins) which are part of the distribution.
%%
%% --------------- start of docstrip commands ------------------
%%
\def\filedate{1999/04/11}
\def\batchfile{english.ins}
\input docstrip.tex

{\ifx\generate\undefined
\Msg{**********************************************}
\Msg{*}
\Msg{* This installation requires docstrip}
\Msg{* version 2.3c or later.}
\Msg{*}
\Msg{* An older version of docstrip has been input}
\Msg{*}
\Msg{**********************************************}
\errhelp{Move or rename old docstrip.tex.}
\errmessage{Old docstrip in input path}
\batchmode
\csname @@end\endcsname
\fi}

\declarepreamble\mainpreamble
This is a generated file.

Copyright 1989-2005 Johannes L. Braams and any individual authors
listed elsewhere in this file.  All rights reserved.

This file was generated from file(s) of the Babel system.
---------------------------------------------------------

It may be distributed and/or modified under the
conditions of the LaTeX Project Public License, either version 1.3
of this license or (at your option) any later version.
The latest version of this license is in
  http://www.latex-project.org/lppl.txt
and version 1.3 or later is part of all distributions of LaTeX
version 2003/12/01 or later.

This work has the LPPL maintenance status "maintained".

The Current Maintainer of this work is Johannes Braams.

This file may only be distributed together with a copy of the Babel
system. You may however distribute the Babel system without
such generated files.

The list of all files belonging to the Babel distribution is
given in the file `manifest.bbl'. See also `legal.bbl for additional
information.

The list of derived (unpacked) files belonging to the distribution
and covered by LPPL is defined by the unpacking scripts (with
extension .ins) which are part of the distribution.
\endpreamble

\declarepreamble\fdpreamble
This is a generated file.

Copyright 1989-2005 Johannes L. Braams and any individual authors
listed elsewhere in this file.  All rights reserved.

This file was generated from file(s) of the Babel system.
---------------------------------------------------------

It may be distributed and/or modified under the
conditions of the LaTeX Project Public License, either version 1.3
of this license or (at your option) any later version.
The latest version of this license is in
  http://www.latex-project.org/lppl.txt
and version 1.3 or later is part of all distributions of LaTeX
version 2003/12/01 or later.

This work has the LPPL maintenance status "maintained".

The Current Maintainer of this work is Johannes Braams.

This file may only be distributed together with a copy of the Babel
system. You may however distribute the Babel system without
such generated files.

The list of all files belonging to the Babel distribution is
given in the file `manifest.bbl'. See also `legal.bbl for additional
information.

In particular, permission is granted to customize the declarations in
this file to serve the needs of your installation.

However, NO PERMISSION is granted to distribute a modified version
of this file under its original name.

\endpreamble

\keepsilent

\usedir{tex/generic/babel} 

\usepreamble\mainpreamble
\generate{\file{english.ldf}{\from{english.dtx}{code}}
          }
\usepreamble\fdpreamble

\ifToplevel{
\Msg{***********************************************************}
\Msg{*}
\Msg{* To finish the installation you have to move the following}
\Msg{* files into a directory searched by TeX:}
\Msg{*}
\Msg{* \space\space All *.def, *.fd, *.ldf, *.sty}
\Msg{*}
\Msg{* To produce the documentation run the files ending with}
\Msg{* '.dtx' and `.fdd' through LaTeX.}
\Msg{*}
\Msg{* Happy TeXing}
\Msg{***********************************************************}
}
 
\endinput
}
\bbl@tempa{canadien}{% \iffalse meta-comment
%
% Copyright 1989-2009 Johannes L. Braams and any individual authors
% listed elsewhere in this file.  All rights reserved.
% 
% This file is part of the Babel system.
% --------------------------------------
% 
% It may be distributed and/or modified under the
% conditions of the LaTeX Project Public License, either version 1.3
% of this license or (at your option) any later version.
% The latest version of this license is in
%   http://www.latex-project.org/lppl.txt
% and version 1.3 or later is part of all distributions of LaTeX
% version 2003/12/01 or later.
% 
% This work has the LPPL maintenance status "maintained".
% 
% The Current Maintainer of this work is Johannes Braams.
% 
% The list of all files belonging to the Babel system is
% given in the file `manifest.bbl. See also `legal.bbl' for additional
% information.
% 
% The list of derived (unpacked) files belonging to the distribution
% and covered by LPPL is defined by the unpacking scripts (with
% extension .ins) which are part of the distribution.
% \fi
% \CheckSum{2135}
%
% \iffalse
%    Tell the \LaTeX\ system who we are and write an entry on the
%    transcript. Nothing to write to the .cfg file, if generated.
%<*dtx>
\ProvidesFile{frenchb.dtx}
%</dtx>
% \changes{v2.1d}{2008/05/04}{Argument of \cs{ProvidesLanguage} changed
%     from `french' to `frenchb', otherwise \cs{listfiles} prints
%     no date/version information.  The bug with \cs{listfiles}
%     (introduced in v.1.5!), was pointed out by Ulrike Fischer.}
%<code>\ProvidesLanguage{frenchb}
%\ProvidesFile{frenchb.dtx}
%<*!cfg>
        [2009/03/16 v2.3d French support from the babel system]
%</!cfg>
%<*cfg>
%% frenchb.cfg: configuration file for frenchb.ldf
%% Daniel Flipo Daniel.Flipo at univ-lille1.fr
%</cfg>
%%    File `frenchb.dtx'
%%    Babel package for LaTeX version 2e
%%    Copyright (C) 1989 - 2009
%%              by Johannes Braams, TeXniek
%
%<*!cfg>
%%    Frenchb language Definition File
%%    Copyright (C) 1989 - 2009
%%              by Johannes Braams, TeXniek
%%                 Daniel Flipo, GUTenberg
%
%%    Please report errors to: Daniel Flipo, GUTenberg
%%                             Daniel.Flipo at univ-lille1.fr
%</!cfg>
%
%    This file is part of the babel system, it provides the source
%    code for the French language definition file.
%
%<*filedriver>
\documentclass[a4paper]{ltxdoc}
\DeclareFontEncoding{T1}{}{}
\DeclareFontSubstitution{T1}{lmr}{m}{n}
\DeclareTextCommand{\guillemotleft}{OT1}{%
  {\fontencoding{T1}\fontfamily{lmr}\selectfont\char19}}
\DeclareTextCommand{\guillemotright}{OT1}{%
  {\fontencoding{T1}\fontfamily{lmr}\selectfont\char20}}
\newcommand*\TeXhax{\TeX hax}
\newcommand*\babel{\textsf{babel}}
\newcommand*\langvar{$\langle \mathit lang \rangle$}
\newcommand*\note[1]{}
\newcommand*\Lopt[1]{\textsf{#1}}
\newcommand*\file[1]{\texttt{#1}}
\begin{document}
\setlength{\parindent}{0pt}
\begin{center}
  \textbf{\Large A Babel language definition file for French}\\[3mm]^^A\]
  Daniel \textsc{Flipo}\\
  \texttt{Daniel.Flipo@univ-lille1.fr}
\end{center}
 \RecordChanges
 \DocInput{frenchb.dtx}
\end{document}
%</filedriver>
% \fi
% \GetFileInfo{frenchb.dtx}
%
%  \section{The French language}
%
%    The file \file{\filename}\footnote{The file described in this
%    section has version number \fileversion\ and was last revised on
%    \filedate.}, defines all the language definition macros for the
%    French language.
%
%    Customisation for the French language is achieved following the
%    book ``Lexique des r\`egles typographiques en usage \`a
%    l'Imprimerie nationale'' troisi\`eme \'edition (1994),
%    ISBN-2-11-081075-0.
%
%    First version released: 1.1 (1996/05/31) as part of
%    \babel-3.6beta.
%
%    |frenchb| has been improved using helpful suggestions from many
%    people, mainly from Jacques Andr\'e, Michel Bovani, Thierry Bouche,
%    and Vincent Jalby.  Thanks to all of them!
%
%    This new version (2.x) has been designed to be used with \LaTeXe{}
%    and Plain\TeX{} formats only. \LaTeX-2.09 is no longer supported.
%    Changes between version 1.6 and \fileversion{} are listed in
%    subsection~\ref{ssec-changes} p.~\pageref{ssec-changes}.
%
%    An extensive documentation is available in French here:\\
%    |http://daniel.flipo.free.fr/frenchb|
%
%  \subsection{Basic interface}
%
%    In a multilingual document, some typographic rules are language
%    dependent, i.e. spaces before `double punctuation' (|:| |;| |!|
%    |?|) in French, others concern the general layout (i.e. layout of
%    lists, footnotes, indentation of first paragraphs of sections) and
%    should apply to the whole document.
%
%    Starting with version~2.2, |frenchb| behaves differently according
%    to \babel's \emph{main language} defined as the \emph{last}
%    option\footnote{Its name is kept in \texttt{\textbackslash
%           bbl@main@language}.} at \babel's loading.  When French is
%    not \babel's main language, |frenchb| no longer alters the global
%    layout of the document (even in parts where French is the current
%    language): the layout of lists, footnotes, indentation of first
%    paragraphs of sections are not customised by |frenchb|.
%
%    When French is loaded as the last option of \babel, |frenchb|
%    makes the following changes to the global layout, \emph{both in
%    French and in all other languages}\footnote{%
%       For each item, hooks are provided to reset standard
%       \LaTeX{} settings or to emulate the behavior of former versions
%       of \texttt{frenchb} (see command
%       \texttt{\textbackslash frenchbsetup\{\}},
%       section~\ref{ssec-custom}).}:
%    \begin{enumerate}
%    \item the first paragraph of each section is indented
%          (\LaTeX{} only);
%    \item the default items in itemize environment are set to `--'
%          instead of `\textbullet', and all vertical spacing and glue
%          is deleted; it is possible to change `--' to something else
%          (`---' for instance) using |\frenchbsetup{}|;
%    \item vertical spacing in general \LaTeX{} lists is
%          shortened;
%    \item footnotes are displayed ``\`a la fran\c{c}aise''.
%    \end{enumerate}
%
%    Regarding local typography, the command |\selectlanguage{french}|
%    switches to the French language\footnote{%
%      \texttt{\textbackslash selectlanguage\{francais\}}
%      and \texttt{\textbackslash selectlanguage\{frenchb\}} are kept
%      for backward compatibility but should no longer be used.},
%    with the following effects:
%    \begin{enumerate}
%    \item French hyphenation patterns are made active;
%    \item `double punctuation' (|:| |;| |!| |?|) is made
%           active%\footnote{Actually, they are active in the whole
%           document, only their expansions differ in French and
%           outside French} for correct spacing in French;
%    \item |\today| prints the date in French;
%    \item the caption names are translated into French
%          (\LaTeX{} only);
%    \item the space after |\dots| is removed in French.
%    \end{enumerate}
%
%    Some commands are provided in |frenchb| to make typesetting
%    easier:
%    \begin{enumerate}
%    \item French quotation marks can be entered using the commands
%          |\og| and |\fg| which work in \LaTeXe and Plain\TeX,
%          their appearance depending on what is available to draw
%          them; even if you use \LaTeXe{} \emph{and} |T1|-encoding,
%          you should refrain from entering them as
%          |<<~French quotation marks~>>|: |\og| and |\fg| provide
%          better horizontal spacing.
%          |\og| and |\fg| can be used outside French, they typeset
%          then English quotes `` and ''.
%    \item A command |\up| is provided to typeset superscripts like
%          |M\up{me}| (abbreviation for ``Madame''), |1\up{er}| (for
%          ``premier'').  Other commands are also provided for
%          ordinals: |\ier|, |\iere|, |\iers|, |\ieres|, |\ieme|,
%          |\iemes| (|3\iemes| prints 3\textsuperscript{es}).
%    \item Family names should be typeset in small capitals and never
%          be hyphenated, the macro |\bsc| (boxed small caps) does
%          this, e.g., |Leslie~\bsc{Lamport}| will produce
%          Leslie~\mbox{\textsc{Lamport}}. Note that composed names
%          (such as Dupont-Durant) may now be hyphenated on explicit
%          hyphens, this differs from |frenchb|~v.1.x.
%    \item Commands |\primo|, |\secundo|, |\tertio| and |\quarto|
%          print 1\textsuperscript{o}, 2\textsuperscript{o},
%          3\textsuperscript{o}, 4\textsuperscript{o}.
%          |\FrenchEnumerate{6}| prints 6\textsuperscript{o}.
%    \item Abbreviations for ``Num\'ero(s)'' and ``num\'ero(s)''
%          (N\textsuperscript{o} N\textsuperscript{os}
%          n\textsuperscript{o} and n\textsuperscript{os}~)
%          are obtained via the commands |\No|, |\Nos|, |\no|, |\nos|.
%    \item Two commands are provided to typeset the symbol for
%          ``degr\'e'': |\degre| prints the raw character and
%          |\degres| should be used to typeset temperatures (e.g.,
%          ``|20~\degres C|'' with an unbreakable space), or for
%          alcohols' strengths (e.g., ``|45\degres|'' with \emph{no}
%          space in French).
%    \item In math mode the comma has to be surrounded with
%          braces to avoid a spurious space being inserted after it,
%          in decimal numbers for instance (see the \TeX{}book p.~134).
%          The command |\DecimalMathComma| makes the comma be an
%          ordinary character \emph{in French only} (no space added);
%          as a counterpart, if |\DecimalMathComma| is active, an
%          explicit space has to be added in lists and intervals:
%          |$[0,\ 1]$|, |$(x,\ y)$|. |\StandardMathComma| switches back
%          to the standard behaviour of the comma.
%    \item A command |\nombre| was provided in 1.x versions to easily
%          format numbers in slices of three digits separated either
%          by a comma in English or with a space in French; |\nombre|
%          is now mapped to |\numprint| from \file{numprint.sty}, see
%          \file{numprint.pdf} for more information.
%    \item |frenchb| has been designed to take advantage of the |xspace|
%          package if present: adding |\usepackage{xspace}| in the
%          preamble will force macros like |\fg|, |\ier|, |\ieme|,
%          |\dots|, \dots, to respect the spaces you type after them,
%          for instance typing `|1\ier juin|' will print
%          `1\textsuperscript{er} juin' (no need for a forced space
%          after |1\ier|).
%    \end{enumerate}
%
%  \subsection{Customisation}
%  \label{ssec-custom}
%
%     Up to version 1.6, customisation of |frenchb| was achieved
%     by entering commands in \file{frenchb.cfg}.  This possibility
%     remains for compatibility, but \emph{should not longer be used}.
%     Version 2.0 introduces a new command |\frenchbsetup{}| using
%     the \file{keyval} syntax which should make it easier to choose
%     among the many options available. The command |\frenchbsetup{}|
%     is to appear in the preamble only (after loading \babel).
%
%     \vspace{.5\baselineskip}
%     |\frenchbsetup{ShowOptions}| prints all available options to
%     the \file{.log} file, it is just meant as a remainder of the
%     list of offered options. As usual with \file{keyval} syntax,
%     boolean options (as |ShowOptions|) can be entered as
%     |ShowOptions=true| or just |ShowOptions|, the `|=true|' part
%     can be omitted.
%
%     \vspace{.5\baselineskip}
%     The other options are listed below. Their default value is shown
%     between brackets, sometimes followed be a `\texttt{*}'.
%     The `\texttt{*}' means that the default shown applies when
%     |frenchb| is loaded as the \emph{last} option of \babel{}
%     ---\babel's \emph{main language}---, and is toggled otherwise:
%     \begin{itemize}
%     \item |StandardLayout=true [false*]| forces |frenchb| not to
%       interfere with the layout: no action on any kind of lists,
%       first paragraphs of sections are not indented (as in English),
%       no action on footnotes. This option replaces the former
%       command |\StandardLayout|.  It can be used to avoid conflicts
%       with classes or packages which customise lists or footnotes.
%     \item |GlobalLayoutFrench=false [true*]| can be used, when French
%       is the main language, to emulate what prior versions of
%       |frenchb| (pre-2.2) did: lists, and first paragraphs
%       of sections will be displayed the standard way in other
%       languages than French, and ``\`a la fran\c{c}aise'' in French.
%       Note that the layout of footnotes is language independent
%       anyway (see below |FrenchFootnotes| and |AutoSpaceFootnotes|).
%       This option replaces the former command |\FrenchLayout|.
%     \item |ReduceListSpacing=false [true*]|; |frenchb| normally
%       reduces the values of the vertical spaces used in the
%       environment |list| in French; setting this option to |false|
%       reverts to the standard settings of |list|.  This option
%       replaces the former command |\FrenchListSpacingfalse|.
%     \item |CompactItemize=false [true*]|; |frenchb| normally
%       suppresses any vertical space between items of |itemize| lists
%       in French; setting this option to |false| reverts to the
%       standard settings of |itemize| lists.  This option replaces
%       the former command |\FrenchItemizeSpacingfalse|.
%     \item |StandardItemLabels=true [false*]| when set to |true| this
%       option stops |frenchb| from changing the labels in |itemize|
%       lists in French.
%     \item |ItemLabels=\textemdash|, |\textbullet|, |\ding{43}|,
%       \dots, |[\textendash*]|; when |StandardItemLabels=false| (the
%       default), this option enables to choose the label used in
%       |itemize| lists for all levels.  The next three options do
%       the same but each one for one level only. Note that the
%       example |\ding{43}| requires |\usepackage{pifont}|.
%     \item |ItemLabeli=\textemdash|, |\textbullet|, |\ding{43}|,
%       \dots,|[\textendash*]|
%     \item |ItemLabelii=\textemdash|, |\textbullet|, |\ding{43}|,
%       \dots, |[\textendash*]|
%     \item |ItemLabeliii=\textemdash|, |\textbullet|, |\ding{43}|,
%       \dots, |[\textendash*]|
%     \item |ItemLabeliv=\textemdash|, |\textbullet|, |\ding{43}|,
%       \dots, |[\textendash*]|
%     \item |StandardLists=true [false*]| forbids |frenchb| to
%       customise any kind of list. Do activate the option
%       |StandardLists| when using classes or packages that customise
%       lists too (|enumitem|, |paralist|, \dots{}) to avoid conflicts.
%       This option is just a shorthand for |ReduceListSpacing=false|
%       and |CompactItemize=false| and |StandardItemLabels=true|.
%     \item |IndentFirst=false [true*]|; |frenchb| normally forces
%       indentation of the first paragraph of sections.
%       When this option is set to |false|, the first paragraph of
%       will look the same in French and in English (not indented).
%     \item |FrenchFootnotes=false [true*]| reverts to the standard
%       layout of footnotes. By default |frenchb| typesets leading
%       numbers as `1.\hspace{.5em}' instead of `$\hbox{}^1$', but
%       has no effect on footnotes numbered with symbols (as in the
%       |\thanks| command).  The former commands |\StandardFootnotes|
%       and |\FrenchFootnotes| are still there, |\StandardFootnotes|
%       can be useful when some footnotes are numbered with letters
%       (inside minipages for instance).
%     \item |AutoSpaceFootnotes=false [true*]| ; by default |frenchb|
%       adds a thin space in the running text before the number or
%       symbol calling the footnote.  Making this option |false|
%       reverts to the standard setting (no space added).
%     \item |FrenchSuperscripts=false [true]| ; then
%       |\up=\textsuperscript| (option added in version 2.1).
%       Should only be made |false| to recompile older documents.
%       By default |\up| now relies on |\fup| designed to produce
%       better looking superscripts.
%     \item |AutoSpacePunctuation=false [true]|; in French, the user
%       \emph{should} input a space before the four characters `|:;!?|'
%       but as many people forget about it (even among native French
%       writers!), the default behaviour of |frenchb| is to
%       automatically add a |\thinspace| before `|;|' `|!|' `|?|' and a
%       normal (unbreakable) space before~`|:|' (this is recommended by
%       the French Imprimerie nationale).  This is convenient in most
%       cases but can lead to addition of spurious spaces in URLs or in
%       MS-DOS paths but only if they are no typed using |\texttt| or
%       verbatim mode. When the current font is a monospaced
%       (typewriter) font, |AutoSpacePunctuation| is locally switched
%       to |false|, no spurious space is added in that case, so the
%       default behaviour of of |frenchb| in that area should be fine
%       in most circumstances.
%
%       Choosing |AutoSpacePunctuation=false| will ensure that
%       a proper space will be added before `|:;!?|' \emph{if and only
%       if} a (normal) space has been typed in. Those who are unsure
%       about their typing in this area should stick to the default
%       option and type |\string;| |\string:| |\string!| |\string?|
%       instead of |;| |:| |!| |?| in case an unwanted space is
%       added by |frenchb|.
%     \item |ThinColonSpace=true [false]| changes the normal
%       (unbreakable) space added before the colon `:' to a thin space,
%       so that the same amount of space is added before any of the
%       four double punctuation characters. The default setting is
%       supported by the French Imprimerie nationale.
%     \item |LowercaseSuperscripts=false [true]| ; by default |frenchb|
%       inhibits the uppercasing of superscripts (for instance when they
%       are moved to page headers). Making this option |false|
%       will disable this behaviour (not recommended).
%     \item |PartNameFull=false [true]|; when true, |frenchb| numbers
%       the title of |\part{}| commands as ``Premi\`ere partie'',
%       ``Deuxi\`eme partie'' and so on. With some classes which change
%       the|\part{}| command (AMS and SMF classes do so), you will get
%       ``Premi\`ere partie~I'', ``Deuxi\`eme partie~II'' instead;
%       when this occurs, this option should be set to |false|,
%       part titles will then be printed as ``Partie I'', ``Partie II''.
%     \item |og=|\texttt{\guillemotleft}, |fg=|\texttt{\guillemotright};
%       when guillemets characters are available on the keyboard
%       (through a compose key for instance), it is nice to use them
%       instead of typing |\og| and |\fg|. This option tells |frenchb|
%       which characters are opening and closing French guillemets
%       (they depend on the input encoding), then you can type either
%       \texttt{\guillemotleft{} guillemets \guillemotright}, or
%       \texttt{\guillemotleft{}guillemets\guillemotright} (with or
%       without spaces), to get properly typeset French quotes.
%       This option requires \file{inputenc} to be loaded with the
%       proper encoding, it works with 8-bits encodings (latin1,
%       latin9, ansinew,  applemac,\dots) and multi-byte encodings
%       (utf8 and utf8x).
%     \end{itemize}
%
%  \subsection{Hyphenation checks}
%  \label{ssec-hyphen}
%
%    Once you have built your format, a good precaution would be to
%    perform some basic tests about hyphenation in French. For
%    \LaTeXe{} I suggest this:
%    \begin{itemize}
%    \item run the following file, with the encoding suitable for
%      your machine (\textit{my-encoding} will be |latin1| for
%      \textsc{unix} machines, |ansinew| for PCs running~Windows,
%      |applemac| or |latin1| for Macintoshs, or |utf8|\dots\\[3mm]^^A\]
%      |%%% Test file for French hyphenation.|\\
%      |\documentclass{article}|\\
%      |\usepackage[|\textit{my-encoding}|]{inputenc}|\\
%      |\usepackage[T1]{fontenc} % Use LM fonts|\\
%      |\usepackage{lmodern}     % for French|\\
%      |\usepackage[frenchb]{babel}|\\
%      |\begin{document}|\\
%      |\showhyphens{signal container \'ev\'enement alg\`ebre}|\\
%      |\showhyphens{|\texttt{signal container \'ev\'enement
%                     alg\`ebre}|}|\\
%      |\end{document}|
%    \item check the hyphenations proposed by \TeX{} in your log-file;
%      in French you should get with both 7-bit and 8-bit encodings\\
%      \texttt{si-gnal contai-ner \'ev\'e-ne-ment al-g\`ebre}.\\
%      Do not care about how accented characters are displayed in the
%      log-file, what matters is the position of the `|-|' hyphen
%      signs \emph{only}.
%    \end{itemize}
%    If they are all correct, your installation (probably) works fine,
%    if one (or more) is (are) wrong, ask a local wizard to see what's
%    going wrong and perform the test again (or e-mail me about what
%    happens).\\
%    Frequent mismatches:
%    \begin{itemize}
%    \item you get |sig-nal con-tainer|, this probably means that the
%    hyphenation patterns you are using are for US-English, not for
%    French;
%    \item you get no hyphen at all in \texttt{\'ev\'e-ne-ment}, this
%    probably means that you are using CM fonts and the macro
%    |\accent| to produce accented characters.
%    Using 8-bits fonts with built-in accented characters avoids
%    this kind of mismatch.
%    \end{itemize}
%
%    \textbf{Options' order} -- Please remember that options are read
%    in the order they appear inside the |\frenchbsetup| command.
%    Someone wishing that |frenchb| leaves the layout of lists
%    and footnotes untouched but caring for indentation of first
%    paragraph of sections could choose
%    |\frenchbsetup{StandardLayout,IndentFirst}| and get the expected
%    layout. Choosing |\frenchbsetup{IndentFirst,StandardLayout}|
%    would not lead to the expected result: option |IndentFirst| would
%    be overwritten by |StandardLayout|.
%
%  \subsection{Changes}
%  \label{ssec-changes}
%
%  \subsubsection*{What's new in version 2.0?}
%
%    Here is the list of all changes:
%    \begin{itemize}
%    \item Support for \LaTeX-2.09 and for \LaTeXe{} in compatibility
%      mode has been dropped. This version is meant for \LaTeXe{} and
%      Plain based formats (like \file{bplain}). \LaTeXe{} formats
%      based on ml\TeX{} are no longer supported either (plenty of
%      good 8-bits fonts are available now, so T1 encoding should
%      be preferred for typesetting in French). A warning is issued
%      when OT1 encoding is in use at the |\begin{document}|.
%    \item Customisation should now be handled by command
%      |\frenchbsetup{}|, \file{frenchb.cfg} (kept for compatibility)
%      should no longer be used. See section~\ref{ssec-custom} for
%      the list of available options.
%    \item Captions in figures and table have changed in French: former
%      abbreviations ``Fig.'' and ``Tab.'' have been replaced by full
%      names ``Figure'' and ``Table''.  If this leads to formatting
%      problems in captions, you can add the following two commands to
%      your preamble (after loading \babel) to get the former captions\\
%      |\addto\captionsfrench{\def\figurename{{\scshape Fig.}}}|\\
%      |\addto\captionsfrench{\def\tablename{{\scshape Tab.}}}|.
%    \item The |\nombre| command is now provided by the \file{numprint}
%      package which has to be loaded \emph{after} \babel{} with the
%      option |autolanguage| if number formatting should depend on the
%      current language.
%    \item The |\bsc| command no longer uses an |\hbox| to stop
%      hyphenation of names but a |\kern0pt| instead. This change
%      enables \file{microtype} to fine tune the length of the
%      argument of |\bsc|; as a side-effect, compound names like
%      Dupont-Durand can now be hyphenated on  explicit hyphens.
%      You can get back to the former behaviour of |\bsc| by adding\\
%      |\renewcommand*{\bsc}[1]{\leavevmode\hbox{\scshape #1}}|\\
%      to the preamble of your document.
%    \item Footnotes are now displayed ``\`a la fran\c caise'' for the
%      whole document, except with an explicit\\
%      |\frenchbsetup{AutoSpaceFootnotes=false,FrenchFootnotes=false}|.\\
%      Add this command if you want standard footnotes. It is still
%      possible to revert locally to the standard layout of footnotes
%      by adding |\StandardFootnotes| (inside a |minipage| environment
%      for instance).
%    \end{itemize}
%
%  \subsubsection*{What's new in version 2.1?}
%
%      New command |\fup| to typeset better looking superscripts.
%      Former command |\up| is now defined as |\fup|, but an option
%      |\frenchbsetup{FrenchSuperscripts=false}| is provided for
%      backward compatibility.  |\fup| was designed using ideas from
%      Jacques Andr\'e, Thierry Bouche and Ren\'e Fritz, thanks to them!
%
%  \subsubsection*{What's new in version 2.2?}
%
%      Starting with version~2.2a, |frenchb| alters the layout of
%      lists, footnotes, and the indentation of first paragraphs of
%      sections) \emph{only if} French is the ``main language''
%      (i.e. babel's last language option). The layout is global for
%      the whole document: lists, etc. look the same in French and in
%      other languages, everything is typeset ``\`a la fran\c caise''
%      if French is the ``main language'', otherwise |frenchb| doesn't
%      change anything regarding lists, footnotes, and indentation of
%      paragraphs.
%
%  \subsubsection*{What's new in version 2.3?}
%
%      Starting with version~2.3a, |frenchb| no longer inserts spaces
%      automatically before `|:;!?|' when a typewriter font is in use;
%      this was suggested by Yannis Haralambous to prevent
%      spurious spaces in computer source code or expressions like
%      \texttt{C\string:/foo}, \texttt{http\string://foo.bar},
%      etc.  An option (|OriginalTypewriter|) is provided to get back
%      to the former behaviour of |frenchb|.
%
%      Another probably invisible change: lowercase conversion in
%      |\up{}| is now achieved by the \LaTeX{} command |\MakeLowercase|
%      instead of \TeX's |\lowercase| command.  This prevents error
%      messages when diacritics are used inside |\up{}| (diacritics
%      should \emph{never} be used in superscripts though!).
%
% \StopEventually{}
%
%  \subsection{File frenchb.cfg}
%  \label{sec-cfg}
%
%    \file{frenchb.cfg} is now a dummy file just kept for compatibility
%    with previous versions.
%
% \iffalse
%<*cfg>
% \fi
%    \begin{macrocode}
%%%%%%%%%%%%%%%%%%%%%%%%%%%%%%%%%%%%%%%%%%%%%%%%%%%%%%%%%%%%%%%%%%%%%%
%%%%%%%%%  WARNING: THIS  FILE SHOULD  NO  LONGER  BE  USED  %%%%%%%%%
%% If you want to customise frenchb, please DO NOT hack into the code!
%% Do no put any code in this file either, please use the new command
%% \frenchbsetup{} with the proper options to customise frenchb.
%% 
%% Add \frenchbsetup{ShowOptions} to your preamble to see the list of
%% available options and/or read the documentation.
%%%%%%%%%%%%%%%%%%%%%%%%%%%%%%%%%%%%%%%%%%%%%%%%%%%%%%%%%%%%%%%%%%%%%%
%    \end{macrocode}
% \iffalse
%</cfg>
% \fi
%
%  \section{\TeX{}nical details}
%
%  \subsection{Initial setup}
%
% \changes{v2.1d}{2008/05/02}{Argument of \cs{ProvidesLanguage} changed
%     above from `french' to `frenchb' (otherwise \cs{listfiles} prints
%     no date/version information).  The real name of current language
%     (french) as to be corrected before calling \cs{LdfInit}.}
%
% \iffalse
%<*code>
% \fi
%
%    While this file was read through the option \Lopt{frenchb} we make
%    it behave as if \Lopt{french} was specified.
%    \begin{macrocode}
\def\CurrentOption{french}
%    \end{macrocode}
%
%    The macro |\LdfInit| takes care of preventing that this file is
%    loaded more than once, checking the category code of the
%    \texttt{@} sign, etc.
%
%    \begin{macrocode}
\LdfInit\CurrentOption\datefrench
%    \end{macrocode}
%
% \changes{v2.1d}{2008/05/04}{Avoid warning ``\cs{end} occurred
%   when \cs{ifx} ... incomplete'' with LaTeX-2.09.}
%
%  \begin{macro}{\ifLaTeXe}
%    No support is provided for late \LaTeX-2.09: issue a warning
%    and exit if \LaTeX-2.09 is in use. Plain is still supported.
%    \begin{macrocode}
\newif\ifLaTeXe
\let\bbl@tempa\relax
\ifx\magnification\@undefined
   \ifx\@compatibilitytrue\@undefined
     \PackageError{frenchb.ldf}
        {LaTeX-2.09 format is no longer supported.\MessageBreak
         Aborting here}
        {Please upgrade to LaTeX2e!}
     \let\bbl@tempa\endinput
   \else
     \LaTeXetrue
   \fi
\fi
\bbl@tempa
%    \end{macrocode}
%  \end{macro}
%
%    Check if hyphenation patterns for the French language have been
%    loaded in language.dat; we allow for the names `french',
%    `francais', `canadien' or `acadian'. The latter two are both
%    names used in Canada for variants of French that are in use in
%    that country.
%
%    \begin{macrocode}
\ifx\l@french\@undefined
  \ifx\l@francais\@undefined
    \ifx\l@canadien\@undefined
      \ifx\l@acadian\@undefined
        \@nopatterns{French}
        \adddialect\l@french0
      \else
        \let\l@french\l@acadian
      \fi
    \else
      \let\l@french\l@canadien
    \fi
  \else
    \let\l@french\l@francais
  \fi
\fi
%    \end{macrocode}
%    Now |\l@french| is always defined.
%
%    The internal name for the French language is |french|;
%    |francais| and |frenchb| are synonymous for |french|:
%    first let both names use the same hyphenation patterns.
%    Later we will have to set aliases for |\captionsfrench|,
%    |\datefrench|, |\extrasfrench| and |\noextrasfrench|.
%    As French uses the standard values of |\lefthyphenmin| (2)
%    and |\righthyphenmin| (3), no special setting is required here.
%
%    \begin{macrocode}
\ifx\l@francais\@undefined
  \let\l@francais\l@french
\fi
\ifx\l@frenchb\@undefined
  \let\l@frenchb\l@french
\fi
%    \end{macrocode}
%    When this language definition file was loaded for one of the
%    Canadian versions of French we need to make sure that a suitable
%    hyphenation pattern register will be found by \TeX.
%    \begin{macrocode}
\ifx\l@canadien\@undefined
  \let\l@canadien\l@french
\fi
\ifx\l@acadian\@undefined
  \let\l@acadian\l@french
\fi
%    \end{macrocode}
%
%    This language definition can be loaded for different variants of
%    the French language. The `key' \babel\ macros are only defined
%    once, using `french' as the language name, but |frenchb| and
%    |francais| are synonymous.
%    \begin{macrocode}
\def\datefrancais{\datefrench}
\def\datefrenchb{\datefrench}
\def\extrasfrancais{\extrasfrench}
\def\extrasfrenchb{\extrasfrench}
\def\noextrasfrancais{\noextrasfrench}
\def\noextrasfrenchb{\noextrasfrench}
%    \end{macrocode}
%
% \begin{macro}{\extrasfrench}
% \begin{macro}{\noextrasfrench}
%    The macro |\extrasfrench| will perform all the extra
%    definitions needed for the French language.
%    The macro |\noextrasfrench| is used to cancel the actions of
%    |\extrasfrench|.\\
%    In French, character ``apostrophe'' is a letter in expressions
%    like |l'ambulance| (French  hyphenation patterns provide entries
%    for this kind of words).  This means that the |\lccode| of
%    ``apostrophe'' has to be non null in French for proper hyphenation
%    of those expressions, and has to be reset to null when exiting
%    French.
%
%    \begin{macrocode}
\@namedef{extras\CurrentOption}{\lccode`\'=`\'}
\@namedef{noextras\CurrentOption}{\lccode`\'=0}
%    \end{macrocode}
% \end{macro}
% \end{macro}
%
%    One more thing |\extrasfrench| needs to do is to make sure that
%    |\frenchspacing| is in effect.  |\noextrasfrench| will switch
%    |\frenchspacing| off again.
%    \begin{macrocode}
  \expandafter\addto\csname extras\CurrentOption\endcsname{%
    \bbl@frenchspacing}
  \expandafter\addto\csname noextras\CurrentOption\endcsname{%
    \bbl@nonfrenchspacing}
%    \end{macrocode}
%
%  \subsection{Punctuation}
%  \label{sec-punct}
%
%    As long as no better solution is available%
%    \footnote{Lua\TeX, or pdf\TeX{} might provide alternatives in
%       the future\dots},
%    the `double punctuation' characters (|;| |!| |?| and |:|) have to
%    be made |\active| for an automatic control of the amount of space
%    to insert before them. Before doing so, we have to save the
%    standard definition of |\@makecaption| (which includes two ':')
%    to compare it later to its definition at the |\begin{document}|.
%    \begin{macrocode}
\long\def\STD@makecaption#1#2{%
  \vskip\abovecaptionskip
  \sbox\@tempboxa{#1: #2}%
  \ifdim \wd\@tempboxa >\hsize
    #1: #2\par
  \else
    \global \@minipagefalse
    \hb@xt@\hsize{\hfil\box\@tempboxa\hfil}%
  \fi
  \vskip\belowcaptionskip}%
%    \end{macrocode}
%
%    We define a new `if' |\FBpunct@active| which will be made false
%    whenever a better alternative will be available. Another `if'
%    |\FBAutoSpacePunctuation| needs to be defined now.
%    \begin{macrocode}
\newif\ifFBpunct@active          \FBpunct@activetrue
\newif\ifFBAutoSpacePunctuation  \FBAutoSpacePunctuationtrue
%    \end{macrocode}
%    The following code makes the four characters |;| |!| |?| and |:|
%    `active' and provides their definitions.
%    \begin{macrocode}
\ifFBpunct@active
  \initiate@active@char{:}
  \initiate@active@char{;}
  \initiate@active@char{!}
  \initiate@active@char{?}
%    \end{macrocode}
%    We first tune the amount of space before \texttt{;}
%    \texttt{!}  \texttt{?} and \texttt{:}.  This should only happen
%    in horizontal mode, hence the test |\ifhmode|.
%
%    In horizontal mode, if a space has been typed before `;' we
%    remove it and put an unbreakable |\thinspace| instead. If no
%    space has been typed, we add |\FDP@thinspace| which will be
%    defined, up to the user's wishes, as an automatic added
%    thin space, or as |\@empty|.
%    \begin{macrocode}
  \declare@shorthand{french}{;}{%
      \ifhmode
      \ifdim\lastskip>\z@
          \unskip\penalty\@M\thinspace
          \else
            \FDP@thinspace
        \fi
      \fi
%    \end{macrocode}
%    Now we can insert a |;| character.
%    \begin{macrocode}
      \string;}
%    \end{macrocode}
%    The next three definitions are very similar.
%    \begin{macrocode}
  \declare@shorthand{french}{!}{%
      \ifhmode
        \ifdim\lastskip>\z@
          \unskip\penalty\@M\thinspace
        \else
          \FDP@thinspace
        \fi
      \fi
      \string!}
  \declare@shorthand{french}{?}{%
      \ifhmode
        \ifdim\lastskip>\z@
          \unskip\penalty\@M\thinspace
        \else
          \FDP@thinspace
        \fi
      \fi
      \string?}
%    \end{macrocode}
%    According to the I.N. specifications, the `:' requires a normal
%    space before it, but some people prefer a |\thinspace| (just
%    like the other three). We define |\Fcolonspace| to hold the
%    required amount of space (user customisable).
%    \begin{macrocode}
  \newcommand*{\Fcolonspace}{\space}
  \declare@shorthand{french}{:}{%
      \ifhmode
        \ifdim\lastskip>\z@
          \unskip\penalty\@M\Fcolonspace
        \else
          \FDP@colonspace
        \fi
      \fi
      \string:}
%    \end{macrocode}
%
% \changes{v2.3a}{2008/10/10}{\cs{NoAutoSpaceBeforeFDP} and
%    \cs{AutoSpaceBeforeFDP} now set the flag
%    \cs{ifFBAutoSpacePunctuation} accordingly (LaTeX only).}
%
%  \begin{macro}{\AutoSpaceBeforeFDP}
%  \begin{macro}{\NoAutoSpaceBeforeFDP}
%    |\FDP@thinspace| and |\FDP@space| are defined as unbreakable
%    spaces by |\autospace@beforeFDP| or as |\@empty| by
%    |\noautospace@beforeFDP| (internal commands), user commands
%    |\AutoSpaceBeforeFDP| and |\NoAutoSpaceBeforeFDP| do the same and
%    take care of the flag |\ifFBAutoSpacePunctuation| in \LaTeX{}.
%    Set the default now for Plain (done later for \LaTeX).
%    \begin{macrocode}
  \def\autospace@beforeFDP{%
          \def\FDP@thinspace{\penalty\@M\thinspace}%
          \def\FDP@colonspace{\penalty\@M\Fcolonspace}}
  \def\noautospace@beforeFDP{\let\FDP@thinspace\@empty
                            \let\FDP@colonspace\@empty}
  \ifLaTeXe
    \def\AutoSpaceBeforeFDP{\autospace@beforeFDP
                            \FBAutoSpacePunctuationtrue}
    \def\NoAutoSpaceBeforeFDP{\noautospace@beforeFDP
                              \FBAutoSpacePunctuationfalse}
  \else
    \let\AutoSpaceBeforeFDP\autospace@beforeFDP
    \let\NoAutoSpaceBeforeFDP\noautospace@beforeFDP
    \AutoSpaceBeforeFDP
  \fi
%    \end{macrocode}
% \end{macro}
% \end{macro}
%
% \changes{v2.3a}{2008/10/10}{In LaTeX, frenchb no longer adds spaces
%     before `double punctuation' characters in computer code.
%     Suggested by Yannis Haralambous.}
%
% \changes{v2.3c}{2009/02/07}{Commands \cs{ttfamily}, \cs{rmfamily}
%    and \cs{sffamily} have to be robust.  Bug introduced in 2.3a,
%    pointed out by Manuel P\'egouri\'e-Gonnard.}
%
%    In \LaTeXe{} |\ttfamily| (and hence |\texttt|) will be redefined
%    `AtBeginDocument' as |\ttfamilyFB| so that no space
%    is added before the four |; : ! ?| characters, even if
%    |AutoSpacePunctuation| is true.  |\rmfamily| and |\sffamily| need
%    to be redefined also (|\ttfamily| is not always used inside a
%    group, its effect can be cancelled by |\rmfamily| or |\sffamily|).
%
%    These redefinitions can be canceled if necessary, for instance to
%    recompile older documents, see option |OriginalTypewriter| below.
%    \begin{macrocode}
  \ifLaTeXe
    \let\ttfamilyORI\ttfamily
    \let\rmfamilyORI\rmfamily
    \let\sffamilyORI\sffamily
    \DeclareRobustCommand\ttfamilyFB{%
         \noautospace@beforeFDP\ttfamilyORI}%
    \DeclareRobustCommand\rmfamilyFB{%
         \ifFBAutoSpacePunctuation
            \autospace@beforeFDP\rmfamilyORI
         \else
            \noautospace@beforeFDP\rmfamilyORI
         \fi}%
    \DeclareRobustCommand\sffamilyFB{%
         \ifFBAutoSpacePunctuation
            \autospace@beforeFDP\sffamilyORI
         \else
            \noautospace@beforeFDP\sffamilyORI
         \fi}%
  \fi
%    \end{macrocode}
%
%    When the active characters appear in an environment where their
%    French behaviour is not wanted they should give an `expected'
%    result. Therefore we define shorthands at system level as well.
%    \begin{macrocode}
  \declare@shorthand{system}{:}{\string:}
  \declare@shorthand{system}{!}{\string!}
  \declare@shorthand{system}{?}{\string?}
  \declare@shorthand{system}{;}{\string;}
%    \end{macrocode}
%    We specify that the French group of shorthands should be used.
%    \begin{macrocode}
  \addto\extrasfrench{%
    \languageshorthands{french}%
%    \end{macrocode}
%    These characters are `turned on' once, later their definition may
%    vary. Don't misunderstand the following code: they keep being
%    active all along the document, even when leaving French.
%    \begin{macrocode}
    \bbl@activate{:}\bbl@activate{;}%
    \bbl@activate{!}\bbl@activate{?}%
  }
  \addto\noextrasfrench{%
  \bbl@deactivate{:}\bbl@deactivate{;}%
  \bbl@deactivate{!}\bbl@deactivate{?}}
\fi
%    \end{macrocode}
%
%  \subsection{French quotation marks}
%
%  \begin{macro}{\og}
%  \begin{macro}{\fg}
%    The top macros for quotation marks will be called |\og|
%    (``\underline{o}uvrez \underline{g}uillemets'') and |\fg|
%    (``\underline{f}ermez \underline{g}uillemets'').
%    Another option for typesetting quotes in multilingual texts
%    is to use the package |csquotes.sty| and its command |\enquote|.
%
%    \begin{macrocode}
\newcommand*{\og}{\@empty}
\newcommand*{\fg}{\@empty}
%    \end{macrocode}
%  \end{macro}
%  \end{macro}
%
%  \begin{macro}{\guillemotleft}
%  \begin{macro}{\guillemotright}
%    \LaTeX{} users are supposed to use 8-bit output encodings (T1,
%    LY1,\dots) to typeset French, those who still stick to OT1 should
%    call |aeguill.sty| or a similar package. In both cases the
%    commands |\guillemotleft| and |\guillemotright| will print the
%    French opening and closing quote characters from the output font.
%    For XeLaTeX, |\guillemotleft| and |\guillemotright| are defined
%    by package \file{xunicode.sty}.
%    We will check `AtBeginDocument' that the proper output encodings
%    are in use (see end of section~\ref{sec-keyval}).
%
%    We give the following definitions for Plain users only as a (poor)
%    fall-back, they are welcome to change them for anything better.
%    \begin{macrocode}
\ifLaTeXe
\else
  \ifx\guillemotleft\@undefined
    \def\guillemotleft{\leavevmode\raise0.25ex
                       \hbox{$\scriptscriptstyle\ll$}}
  \fi
  \ifx\guillemotright\@undefined
    \def\guillemotright{\raise0.25ex
                        \hbox{$\scriptscriptstyle\gg$}}
  \fi
  \let\xspace\relax
\fi
%    \end{macrocode}
%  \end{macro}
%  \end{macro}
%
%    The next step is to provide correct spacing after |\guillemotleft|
%    and before |\guillemotright|: a space precedes and follows
%    quotation marks but no line break is allowed neither \emph{after}
%    the opening one, nor \emph{before} the closing one.
%    |\FBguill@spacing| which does the spacing, has been fine tuned by
%    Thierry Bouche.  French quotes (including spacing) are printed by
%    |\FB@og| and |\FB@fg|, the expansion of the top level commands
%    |\og| and |\og| is different in and outside French.
%    We'll try to be smart to users of David~Carlisle's |xspace|
%    package: if this package is loaded there will be no need for |{}|
%    or |\ | to get a space after |\fg|, otherwise |\xspace| will be
%    defined as |\relax| (done at the end of this file).
%
%    \begin{macrocode}
\newcommand*{\FBguill@spacing}{\penalty\@M\hskip.8\fontdimen2\font
                                            plus.3\fontdimen3\font
                                           minus.8\fontdimen4\font}
\DeclareRobustCommand*{\FB@og}{\leavevmode
                               \guillemotleft\FBguill@spacing}
\DeclareRobustCommand*{\FB@fg}{\ifdim\lastskip>\z@\unskip\fi
                               \FBguill@spacing\guillemotright\xspace}
%    \end{macrocode}
%
%    The top level definitions for French quotation marks are switched
%    on and off through the |\extrasfrench| |\noextrasfrench|
%    mechanism. Outside French, |\og| and |\fg| will typeset standard
%    English opening and closing double quotes.
%
%    \begin{macrocode}
\ifLaTeXe
  \def\bbl@frenchguillemets{\renewcommand*{\og}{\FB@og}%
                            \renewcommand*{\fg}{\FB@fg}}
  \def\bbl@nonfrenchguillemets{\renewcommand*{\og}{\textquotedblleft}%
            \renewcommand*{\fg}{\ifdim\lastskip>\z@\unskip\fi
                                   \textquotedblright}}
\else
   \def\bbl@frenchguillemets{\let\og\FB@og
                             \let\fg\FB@fg}
   \def\bbl@nonfrenchguillemets{\def\og{``}%
                     \def\fg{\ifdim\lastskip>\z@\unskip\fi ''}}
\fi
\expandafter\addto\csname extras\CurrentOption\endcsname{%
  \bbl@frenchguillemets}
\expandafter\addto\csname noextras\CurrentOption\endcsname{%
  \bbl@nonfrenchguillemets}
%    \end{macrocode}
%
%  \subsection{Date in French}
%
% \begin{macro}{\datefrench}
%    The macro |\datefrench| redefines the command |\today| to
%    produce French dates.
%
% \changes{v2.0}{2006/11/06}{2 '\cs{relax}' added in
%    \cs{today}'s definition.}
%
% \changes{v2.1a}{2008/03/25}{\cs{today} changed (correction in 2.0
%    was wrong: \cs{today} was printed without spaces in toc).}
%
%    \begin{macrocode}
\@namedef{date\CurrentOption}{%
  \def\today{{\number\day}\ifnum1=\day {\ier}\fi \space
    \ifcase\month
      \or janvier\or f\'evrier\or mars\or avril\or mai\or juin\or
      juillet\or ao\^ut\or septembre\or octobre\or novembre\or
      d\'ecembre\fi
    \space \number\year}}
%    \end{macrocode}
% \end{macro}
%
%  \subsection{Extra utilities}
%
%    Let's provide the French user with some extra utilities.
%
% \changes{v2.1a}{2008/03/24}{Command \cs{fup} added to produce
%    better superscripts than \cs{textsuperscript}.}
%
%  \begin{macro}{\up}
%
% \changes{v2.1c}{2008/04/29}{Provide a temporary definition
%    (hyperref safe) of \cs{up} in case it has to be expanded in
%    the preamble (by beamer's \cs{title} command for instance).}
%
%  \begin{macro}{\fup}
%
% \changes{v2.1b}{2008/04/02}{Command \cs{fup} changed to use
%    real superscripts from fourier v. 1.6.}
%
% \changes{v2.2a}{2008/05/08}{\cs{newif} and \cs{newdimen} moved
%    before \cs{ifLaTeXe} to avoid an error with plainTeX.}
%
% \changes{v2.3a}{2008/09/30}{\cs{lowercase} changed to
%    \cs{MakeLowercase} as the former doesn't work for non ASCII
%    characters in encodings like applemac, utf-8,\dots}
%
%    |\up| eases the typesetting of superscripts like
%    `1\textsuperscript{er}'.  Up to version 2.0 of |frenchb| |\up| was
%    just a shortcut for |\textsuperscript| in \LaTeXe, but several
%    users complained that |\textsuperscript| typesets superscripts
%    too high and too big, so we now define |\fup| as an attempt to
%    produce better looking superscripts.  |\up| is defined as |\fup|
%    but can be redefined by |\frenchbsetup{FrenchSuperscripts=false}|
%    as |\textsuperscript| for compatibility with previous versions.
%
%    When a font has built-in superscripts, the best thing to do is
%    to just use them, otherwise |\fup| has to simulate superscripts
%    by scaling and raising ordinary letters.  Scaling is done using
%    package \file{scalefnt} which will be loaded at the end of
%    \babel's loading (|frenchb| being an option of babel, it cannot
%    load a package while being read).
%
%    \begin{macrocode}
\newif\ifFB@poorman
\newdimen\FB@Mht
\ifLaTeXe
  \AtEndOfPackage{\RequirePackage{scalefnt}}
%    \end{macrocode}
%    |\FB@up@fake| holds the definition of fake superscripts.
%    The scaling ratio is 0.65, raising is computed to put the top of
%    lower case letters (like `m') just under the top  of upper case
%    letters (like `M'), precisely 12\% down.  The chosen settings
%    look correct for most fonts, but can be tuned by the end-user
%    if necessary by changing |\FBsupR| and |\FBsupS| commands.
%
%    |\FB@lc| is defined as |\MakeLowercase| to inhibit the uppercasing
%    of superscripts (this may happen in page headers with the standard
%    classes but is wrong); |\FB@lc| can be redefined to do nothing
%    by option |LowercaseSuperscripts=false| of |\frenchbsetup{}|.
%    \begin{macrocode}
  \newcommand*{\FBsupR}{-0.12}
  \newcommand*{\FBsupS}{0.65}
  \newcommand*{\FB@lc}[1]{\MakeLowercase{#1}}
  \DeclareRobustCommand*{\FB@up@fake}[1]{%
    \settoheight{\FB@Mht}{M}%
    \addtolength{\FB@Mht}{\FBsupR \FB@Mht}%
    \addtolength{\FB@Mht}{-\FBsupS ex}%
    \raisebox{\FB@Mht}{\scalefont{\FBsupS}{\FB@lc{#1}}}%
    }
%    \end{macrocode}
%    The only packages I currently know to take advantage of real
%    superscripts are a) \file{xltxtra} used in conjunction with
%    XeLaTeX and OpenType fonts having the font feature
%    'VerticalPosition=Superior' (\file{xltxtra} defines
%    |\realsuperscript| and |\fakesuperscript|) and b) \file{fourier}
%    (from version 1.6) when Expert Utopia fonts are available.
%
%    |\FB@up| checks whether the current font is a Type1 `Expert'
%    (or `Pro') font with real superscripts or not (the code works
%    currently only with \file{fourier-1.6} but could work with any
%    Expert Type1 font with built-in superscripts, see below), and
%    decides to use real or fake superscripts.
%    It works as follows: the content of |\f@family| (family name of
%    the current font) is split by |\FB@split| into two pieces, the
%    first three characters (`|fut|' for Fourier, `|ppl|' for Adobe's
%    Palatino, \dots) stored in |\FB@firstthree| and the rest stored
%    in |\FB@suffix| which is expected to be `|x|' or `|j|' for expert
%    fonts.
%    \begin{macrocode}
  \def\FB@split#1#2#3#4\@nil{\def\FB@firstthree{#1#2#3}%
                             \def\FB@suffix{#4}}
  \def\FB@x{x}
  \def\FB@j{j}
  \DeclareRobustCommand*{\FB@up}[1]{%
    \bgroup \FB@poormantrue
      \expandafter\FB@split\f@family\@nil
%    \end{macrocode}
%    Then |\FB@up| looks for a \file{.fd} file named \file{t1fut-sup.fd}
%    (Fourier) or \file{t1ppl-sup.fd} (Palatino), etc. supposed to
%    define the subfamily (|fut-sup| or |ppl-sup|, etc.) giving access
%    to the built-in superscripts.  If the \file{.fd} file is not found
%    by |\IfFileExists|, |\FB@up| falls back on fake superscripts,
%    otherwise |\FB@suffix| is checked to decide whether to use fake or
%    real superscripts.
%    \begin{macrocode}
      \edef\reserved@a{\lowercase{%
         \noexpand\IfFileExists{\f@encoding\FB@firstthree -sup.fd}}}%
      \reserved@a
        {\ifx\FB@suffix\FB@x \FB@poormanfalse\fi
         \ifx\FB@suffix\FB@j \FB@poormanfalse\fi
         \ifFB@poorman \FB@up@fake{#1}%
         \else         \FB@up@real{#1}%
         \fi}%
        {\FB@up@fake{#1}}%
    \egroup}
%    \end{macrocode}
%    |\FB@up@real| just picks up the superscripts from the subfamily
%    (and forces lowercase).
%    \begin{macrocode}
  \newcommand*{\FB@up@real}[1]{\bgroup
       \fontfamily{\FB@firstthree -sup}\selectfont \FB@lc{#1}\egroup}
%    \end{macrocode}
%    |\fup| is now defined as |\FB@up| unless |\realsuperscript| is
%    defined (occurs with XeLaTeX calling \file{xltxtra.sty}).
%    \begin{macrocode}
  \DeclareRobustCommand*{\fup}[1]{%
    \@ifundefined{realsuperscript}%
      {\FB@up{#1}}%
      {\bgroup\let\fakesuperscript\FB@up@fake
            \realsuperscript{\FB@lc{#1}}\egroup}}
%    \end{macrocode}
%    Temporary definition of |up| (redefined `AtBeginDocument').
%    \begin{macrocode}
  \newcommand*{\up}{\relax}
%    \end{macrocode}
%    Poor man's definition of |\up| for Plain. In \LaTeXe,
%    |\up| will be defined as |\fup| or |\textsuperscript| later on
%    while processing the options of |\frenchbsetup{}|.
%    \begin{macrocode}
\else
  \newcommand*{\up}[1]{\leavevmode\raise1ex\hbox{\sevenrm #1}}
\fi
%    \end{macrocode}
%  \end{macro}
%  \end{macro}
%
%  \begin{macro}{\ieme}
%  \begin{macro}{\ier}
%  \begin{macro}{\iere}
%  \begin{macro}{\iemes}
%  \begin{macro}{\iers}
%  \begin{macro}{\ieres}
%  Some handy macros for those who don't know how to abbreviate ordinals:
%    \begin{macrocode}
\def\ieme{\up{\lowercase{e}}\xspace}
\def\iemes{\up{\lowercase{es}}\xspace}
\def\ier{\up{\lowercase{er}}\xspace}
\def\iers{\up{\lowercase{ers}}\xspace}
\def\iere{\up{\lowercase{re}}\xspace}
\def\ieres{\up{\lowercase{res}}\xspace}
%    \end{macrocode}
%  \end{macro}
%  \end{macro}
%  \end{macro}
%  \end{macro}
%  \end{macro}
%  \end{macro}
%
% \changes{v2.1c}{2008/04/29}{Added commands \cs{Nos} and \cs{nos}.}
%
%  \begin{macro}{\No}
%  \begin{macro}{\no}
%  \begin{macro}{\Nos}
%  \begin{macro}{\nos}
%  \begin{macro}{\primo}
%  \begin{macro}{\fprimo)}
%    And some more macros relying on |\up| for numbering,
%    first two support macros.
%    \begin{macrocode}
\newcommand*{\FrenchEnumerate}[1]{%
                       #1\up{\lowercase{o}}\kern+.3em}
\newcommand*{\FrenchPopularEnumerate}[1]{%
                       #1\up{\lowercase{o}})\kern+.3em}
%    \end{macrocode}
%
%    Typing |\primo| should result in `$1^{\rm o}$\kern+.3em',
%    \begin{macrocode}
\def\primo{\FrenchEnumerate1}
\def\secundo{\FrenchEnumerate2}
\def\tertio{\FrenchEnumerate3}
\def\quarto{\FrenchEnumerate4}
%    \end{macrocode}
%    while typing |\fprimo)| gives `1$^{\rm o}$)\kern+.3em.
%    \begin{macrocode}
\def\fprimo){\FrenchPopularEnumerate1}
\def\fsecundo){\FrenchPopularEnumerate2}
\def\ftertio){\FrenchPopularEnumerate3}
\def\fquarto){\FrenchPopularEnumerate4}
%    \end{macrocode}
%
%    Let's provide four macros for the common abbreviations
%    of ``Num\'ero''.
%    \begin{macrocode}
\DeclareRobustCommand*{\No}{N\up{\lowercase{o}}\kern+.2em}
\DeclareRobustCommand*{\no}{n\up{\lowercase{o}}\kern+.2em}
\DeclareRobustCommand*{\Nos}{N\up{\lowercase{os}}\kern+.2em}
\DeclareRobustCommand*{\nos}{n\up{\lowercase{os}}\kern+.2em}
%    \end{macrocode}
%  \end{macro}
%  \end{macro}
%  \end{macro}
%  \end{macro}
%  \end{macro}
%  \end{macro}
%
%  \begin{macro}{\bsc}
%    As family names should be written in small capitals and never be
%    hyphenated, we provide a command (its name comes from Boxed Small
%    Caps) to input them easily.  Note that this command has changed
%    with version~2 of |frenchb|: a |\kern0pt| is used instead of |\hbox|
%    because |\hbox| would break microtype's font expansion; as a
%    (positive?) side effect, composed names (such as Dupont-Durand)
%    can now be hyphenated on explicit hyphens.
%    Usage: |Jean~\bsc{Duchemin}|.
%
% \changes{v2.0}{2006/11/06}{\cs{hbox} dropped, replaced by
%    \cs{kern0pt}.}
%
%    \begin{macrocode}
\DeclareRobustCommand*{\bsc}[1]{\leavevmode\begingroup\kern0pt
                                           \scshape #1\endgroup}
\ifLaTeXe\else\let\scshape\relax\fi
%    \end{macrocode}
%  \end{macro}
%
%    Some definitions for special characters.  We won't define |\tilde|
%    as a Text Symbol not to conflict with the macro |\tilde| for math
%    mode and use the name |\tild| instead. Note that |\boi| may
%    \emph{not} be used in math mode, its name in math mode is
%    |\backslash|.  |\degre|  can be accessed by the command |\r{}|
%    for ring accent.
%
%    \begin{macrocode}
\ifLaTeXe
  \DeclareTextSymbol{\at}{T1}{64}
  \DeclareTextSymbol{\circonflexe}{T1}{94}
  \DeclareTextSymbol{\tild}{T1}{126}
  \DeclareTextSymbolDefault{\at}{T1}
  \DeclareTextSymbolDefault{\circonflexe}{T1}
  \DeclareTextSymbolDefault{\tild}{T1}
  \DeclareRobustCommand*{\boi}{\textbackslash}
  \DeclareRobustCommand*{\degre}{\r{}}
\else
  \def\T@one{T1}
  \ifx\f@encoding\T@one
    \newcommand*{\degre}{\char6}
  \else
    \newcommand*{\degre}{\char23}
  \fi
  \newcommand*{\at}{\char64}
  \newcommand*{\circonflexe}{\char94}
  \newcommand*{\tild}{\char126}
  \newcommand*{\boi}{$\backslash$}
\fi
%    \end{macrocode}
%
%  \begin{macro}{\degres}
%    We now define a macro |\degres| for typesetting the abbreviation
%    for `degrees' (as in `degrees Celsius'). As the bounding box of
%    the character `degree' has \emph{very} different widths in CM/EC
%    and PostScript fonts, we fix the width of the bounding box of
%    |\degres| to 0.3\,em, this lets the symbol `degree' stick to the
%    preceding (e.g., |45\degres|) or following character
%    (e.g., |20~\degres C|).
%
%    If the \TeX{} Companion fonts are available (\file{textcomp.sty}),
%    we pick up |\textdegree| from them instead of using emulating
%    `degrees' from the |\r{}| accent. Otherwise we overwrite the
%    (poor) definition of |\textdegree| given in \file{latin1.def},
%    \file{applemac.def} etc. (called by  \file{inputenc.sty}) by
%    our definition of |\degres|. We also advice the user (once only)
%    to use TS1-encoding.
%
% \changes{v2.1c}{2008/04/29}{Provide a temporary definition (hyperref
%    safe) of \cs{degres} in case it has to be expanded in the preamble
%    (by beamer's \cs{title} command for instance).}
%
%    \begin{macrocode}
\ifLaTeXe
  \newcommand*{\degres}{\degre}
  \def\Warning@degree@TSone{%
        \PackageWarning{frenchb.ldf}{%
           Degrees would look better in TS1-encoding:
           \MessageBreak add \protect
           \usepackage{textcomp} to the preamble.
           \MessageBreak Degrees used}}
  \AtBeginDocument{\expandafter\ifx\csname M@TS1\endcsname\relax
                     \DeclareRobustCommand*{\degres}{%
                       \leavevmode\hbox to 0.3em{\hss\degre\hss}%
                       \Warning@degree@TSone
                       \global\let\Warning@degree@TSone\relax}%
                      \let\textdegree\degres
                   \else
                     \DeclareRobustCommand*{\degres}{%
                         \hbox{\UseTextSymbol{TS1}{\textdegree}}}%
                   \fi}
\else
  \newcommand*{\degres}{%
    \leavevmode\hbox to 0.3em{\hss\degre\hss}}
\fi
%    \end{macrocode}
%  \end{macro}
%
%  \subsection{Formatting numbers}
%  \label{sec-numbers}
%
%  \begin{macro}{\DecimalMathComma}
%  \begin{macro}{\StandardMathComma}
%    As mentioned in the \TeX{}book p.~134, the comma is of type
%    |\mathpunct| in math mode: it is automatically followed by a
%    space. This is convenient in lists and intervals but
%    unpleasant when the comma is used as a decimal separator
%    in French: it has to be entered as |{,}|.
%    |\DecimalMathComma| makes the comma be an ordinary character
%    (of type |\mathord|) in French \emph{only} (no space added);
%    |\StandardMathComma| switches back to the standard behaviour
%    of the comma.
%    \begin{macrocode}
\newcount\std@mcc
\newcount\dec@mcc
\std@mcc=\mathcode`\,
\dec@mcc=\std@mcc
\@tempcnta=\std@mcc
\divide\@tempcnta by "1000
\multiply\@tempcnta by "1000
\advance\dec@mcc by -\@tempcnta
\newcommand*{\DecimalMathComma}{\iflanguage{french}%
                                 {\mathcode`\,=\dec@mcc}{}%
              \addto\extrasfrench{\mathcode`\,=\dec@mcc}}
\newcommand*{\StandardMathComma}{\mathcode`\,=\std@mcc
             \addto\extrasfrench{\mathcode`\,=\std@mcc}}
\expandafter\addto\csname noextras\CurrentOption\endcsname{%
   \mathcode`\,=\std@mcc}
%    \end{macrocode}
%  \end{macro}
%  \end{macro}
%
%  \begin{macro}{\nombre}
%
% \changes{v2.0}{2006/11/06}{\cs{nombre} requires now numprint.sty.}
%
%    The command |\nombre| is now borrowed from |numprint.sty| for
%    \LaTeXe.  There is no point to maintain the former tricky code
%    when a package is dedicated to do the same job and more.
%    For Plain based formats, |\nombre| no longer formats numbers,
%    it prints them as is and issues a warning about the change.
%
%    Fake command |\nombre| for Plain based formats, warning users of
%    |frenchb| v.1.x. of the change.
%    \begin{macrocode}
\newcommand*{\nombre}[1]{{#1}\message{%
     *** \noexpand\nombre no longer formats numbers\string! ***}}%
%    \end{macrocode}
%  \end{macro}
%
%    The next definitions only make sense for \LaTeXe. Let's cleanup
%    and exit if the format in Plain based.
%
%    \begin{macrocode}
\let\FBstop@here\relax
\def\FBclean@on@exit{\let\ifLaTeXe\@undefined
                     \let\LaTeXetrue\@undefined
                     \let\LaTeXefalse\@undefined}
\ifx\magnification\@undefined
\else
   \def\FBstop@here{\let\STD@makecaption\relax
                    \FBclean@on@exit
                    \ldf@quit\CurrentOption\endinput}
\fi
\FBstop@here
%    \end{macrocode}
%
%    What follows now is for \LaTeXe{} \emph{only}.
%    We redefine |\nombre| for \LaTeXe. A warning is issued
%    at the first call of |\nombre| if |\numprint| is not
%    defined, suggesting what to do.  The package |numprint|
%    is \emph{not} loaded automatically by |frenchb| because of
%    possible options conflict.
%
%    \begin{macrocode}
\renewcommand*{\nombre}[1]{\Warning@nombre\numprint{#1}}
\newcommand*{\Warning@nombre}{%
   \@ifundefined{numprint}%
      {\PackageWarning{frenchb.ldf}{%
         \protect\nombre\space now relies on package numprint.sty,
         \MessageBreak add \protect
         \usepackage[autolanguage]{numprint}\MessageBreak
         to your preamble *after* loading babel, \MessageBreak
         see file numprint.pdf for other options.\MessageBreak
         \protect\nombre\space called}%
       \global\let\Warning@nombre\relax
       \global\let\numprint\relax
      }{}%
}
%    \end{macrocode}
%
% \changes{v2.0c}{2007/06/25}{There is no need to define here
%    numprint's command \cs{npstylefrench}, it will be redefined
%    `AtBeginDocument' by \cs{FBprocess@options}.}
%
% \changes{v2.0c}{2007/06/25}{\cs{ThinSpaceInFrenchNumbers} added
%     for compatibility with frenchb-1.x.}
%
%    \begin{macrocode}
\newcommand*{\ThinSpaceInFrenchNumbers}{%
   \PackageWarning{frenchb.ldf}{%
         Type \protect\frenchbsetup{ThinSpaceInFrenchNumbers}
         \MessageBreak Command \protect\ThinSpaceInFrenchNumbers\space
         is no longer\MessageBreak  defined in frenchb v.2,}}
%    \end{macrocode}
%
%  \subsection{Caption names}
%
%    The next step consists of defining the French equivalents for
%    the \LaTeX{} caption names.
%
% \begin{macro}{\captionsfrench}
%    Let's first define  |\captionsfrench| which sets all strings used
%    in the four standard document classes provided with \LaTeX.
%
% \changes{v2.0}{2006/11/06}{`Fig.' changed to `Figure' and
%     `Tab.' to `Table'.}
%
% \changes{v2.0}{2006/12/15}{Set \cs{CaptionSeparator} in
%     \cs{extrasfrench} now instead of \cs{captionsfrench}
%     because it has to be reset when leaving French.}
%
%    \begin{macrocode}
\@namedef{captions\CurrentOption}{%
   \def\refname{R\'ef\'erences}%
   \def\abstractname{R\'esum\'e}%
   \def\bibname{Bibliographie}%
   \def\prefacename{Pr\'eface}%
   \def\chaptername{Chapitre}%
   \def\appendixname{Annexe}%
   \def\contentsname{Table des mati\`eres}%
   \def\listfigurename{Table des figures}%
   \def\listtablename{Liste des tableaux}%
   \def\indexname{Index}%
   \def\figurename{{\scshape Figure}}%
   \def\tablename{{\scshape Table}}%
%    \end{macrocode}
%   ``Premi\`ere partie'' instead of ``Part I''.
%    \begin{macrocode}
   \def\partname{\protect\@Fpt partie}%
   \def\@Fpt{{\ifcase\value{part}\or Premi\`ere\or Deuxi\`eme\or
   Troisi\`eme\or Quatri\`eme\or Cinqui\`eme\or Sixi\`eme\or
   Septi\`eme\or Huiti\`eme\or Neuvi\`eme\or Dixi\`eme\or Onzi\`eme\or
   Douzi\`eme\or Treizi\`eme\or Quatorzi\`eme\or Quinzi\`eme\or
   Seizi\`eme\or Dix-septi\`eme\or Dix-huiti\`eme\or Dix-neuvi\`eme\or
   Vingti\`eme\fi}\space\def\thepart{}}%
   \def\pagename{page}%
   \def\seename{{\emph{voir}}}%
   \def\alsoname{{\emph{voir aussi}}}%
   \def\enclname{P.~J. }%
   \def\ccname{Copie \`a }%
   \def\headtoname{}%
   \def\proofname{D\'emonstration}%
   \def\glossaryname{Glossaire}%
   }
%    \end{macrocode}
% \end{macro}
%
%    As some users who choose |frenchb| or |francais| as option of
%    \babel, might customise |\captionsfrenchb| or |\captionsfrancais|
%    in the preamble, we merge their changes at the |\begin{document}|
%    when they do so.
%    The other variants of French (canadien, acadian) are defined by
%    checking if the relevant option was used and then adding one extra
%    level of expansion.
%
%    \begin{macrocode}
\AtBeginDocument{\let\captions@French\captionsfrench
                 \@ifundefined{captionsfrenchb}%
                    {\let\captions@Frenchb\relax}%
                    {\let\captions@Frenchb\captionsfrenchb}%
                 \@ifundefined{captionsfrancais}%
                    {\let\captions@Francais\relax}%
                    {\let\captions@Francais\captionsfrancais}%
                 \def\captionsfrench{\captions@French
                        \captions@Francais\captions@Frenchb}%
                 \def\captionsfrancais{\captionsfrench}%
                 \def\captionsfrenchb{\captionsfrench}%
                 \iflanguage{french}{\captionsfrench}{}%
                }
\@ifpackagewith{babel}{canadien}{%
  \def\captionscanadien{\captionsfrench}%
  \def\datecanadien{\datefrench}%
  \def\extrascanadien{\extrasfrench}%
  \def\noextrascanadien{\noextrasfrench}%
  }{}
\@ifpackagewith{babel}{acadian}{%
  \def\captionsacadian{\captionsfrench}%
  \def\dateacadian{\datefrench}%
  \def\extrasacadian{\extrasfrench}%
  \def\noextrasacadian{\noextrasfrench}%
  }{}
%    \end{macrocode}
%
% \begin{macro}{\CaptionSeparator}
%    Let's consider now captions in figures and tables.
%    In French, captions in figures and tables should be printed with
%    endash (`--') instead of the standard `:'.
%
%    The standard definition of |\@makecaption| (e.g., the one provided
%    in article.cls, report.cls, book.cls which is frozen for \LaTeXe{}
%    according to Frank Mittelbach), has been saved in
%    |\STD@makecaption| before making `:' active
%    (see section~\ref{sec-punct}). `AtBeginDocument' we compare it to
%    its current definition (some classes like koma-script classes,
%    AMS classes, ua-thesis.cls\dots change it).
%    If they are identical, |frenchb| just adds a hook called
%    |\CaptionSeparator| to |\@makecaption|, |\CaptionSeparator|
%    defaults to `: ' as in the standard |\@makecaption|, and will be
%    changed to ` -- ' in French.
%    If the definitions differ, |frenchb| doesn't overwrite the changes,
%    but prints a message in the .log file.
%
%    \begin{macrocode}
\def\CaptionSeparator{\string:\space}
\long\def\FB@makecaption#1#2{%
  \vskip\abovecaptionskip
  \sbox\@tempboxa{#1\CaptionSeparator #2}%
  \ifdim \wd\@tempboxa >\hsize
    #1\CaptionSeparator #2\par
  \else
    \global \@minipagefalse
    \hb@xt@\hsize{\hfil\box\@tempboxa\hfil}%
  \fi
  \vskip\belowcaptionskip}
\AtBeginDocument{%
  \ifx\@makecaption\STD@makecaption
      \global\let\@makecaption\FB@makecaption
  \else
    \@ifundefined{@makecaption}{}%
       {\PackageWarning{frenchb.ldf}%
        {The definition of \protect\@makecaption\space
         has been changed,\MessageBreak
         frenchb will NOT customise it;\MessageBreak reported}%
       }%
  \fi
  \let\FB@makecaption\relax
  \let\STD@makecaption\relax
}
\expandafter\addto\csname extras\CurrentOption\endcsname{%
   \def\CaptionSeparator{\space\textendash\space}}
\expandafter\addto\csname noextras\CurrentOption\endcsname{%
    \def\CaptionSeparator{\string:\space}}
%    \end{macrocode}
% \end{macro}
%
%  \subsection{French lists}
%  \label{sec-lists}
%
%  \begin{macro}{\listFB}
%  \begin{macro}{\listORI}
%    Vertical spacing in general lists should be shorter in French
%    texts than the defaults provided by \LaTeX.
%    Note that the easy way, just changing values of vertical spacing
%    parameters when entering French and restoring them to their
%    defaults on exit would not work; as most lists are based on
%    |\list| we will define a variant of |\list| (|\listFB|) to
%    be used in French.
%
%    The amount of vertical space before and after a list is given by
%    |\topsep| + |\parskip| (+ |\partopsep| if the list starts a new
%    paragraph). IMHO, |\parskip| should be added \emph{only} when
%    the list starts a new paragraph, so I subtract |\parskip| from
%    |\topsep| and add it back to |\partopsep|; this will normally
%    make no difference because |\parskip|'s default value is 0pt, but
%    will be noticeable when |\parskip| is \emph{not} null.
%
%    |\endlist| is not redefined, but |\endlistORI| is provided for
%    the users who prefer to define their own lists from the original
%    command, they can code: |\begin{listORI}{}{} \end{listORI}|.
%    \begin{macrocode}
\let\listORI\list
\let\endlistORI\endlist
\def\FB@listsettings{%
      \setlength{\itemsep}{0.4ex plus 0.2ex minus 0.2ex}%
      \setlength{\parsep}{0.4ex plus 0.2ex minus 0.2ex}%
      \setlength{\topsep}{0.8ex plus 0.4ex minus 0.4ex}%
      \setlength{\partopsep}{0.4ex plus 0.2ex minus 0.2ex}%
%    \end{macrocode}
%    |\parskip| is of type `skip', its mean value only (\emph{not
%    the glue}) should be subtracted from |\topsep| and added to
%    |\partopsep|, so convert |\parskip| to a `dimen' using
%    |\@tempdima|.
%    \begin{macrocode}
      \@tempdima=\parskip
      \addtolength{\topsep}{-\@tempdima}%
      \addtolength{\partopsep}{\@tempdima}}%
\def\listFB#1#2{\listORI{#1}{\FB@listsettings #2}}%
\let\endlistFB\endlist
%    \end{macrocode}
%  \end{macro}
%  \end{macro}
%
%  \begin{macro}{\itemizeFB}
%  \begin{macro}{\itemizeORI}
%  \begin{macro}{\bbl@frenchlabelitems}
%  \begin{macro}{\bbl@nonfrenchlabelitems}
%    Let's now consider French itemize lists.  They differ from those
%    provided by the standard \LaTeXe{} classes:
%    \begin{itemize}
%      \item vertical spacing between items, before and after
%         the list, should be \emph{null} with \emph{no glue} added;
%      \item the item labels of a first level list should be vertically
%          aligned on the paragraph's first character (i.e. at
%          |\parindent| from the left margin);
%      \item the `\textbullet' is never used in French itemize-lists,
%          a long dash `--' is preferred for all levels. The item label
%          used in French is stored in |\FrenchLabelItem}|, it defaults
%          to `--' and can be changed using |\frenchbsetup{}| (see
%          section~\ref{sec-keyval}).
%    \end{itemize}
%
%    \begin{macrocode}
\newcommand*{\FrenchLabelItem}{\textendash}
\newcommand*{\Frlabelitemi}{\FrenchLabelItem}
\newcommand*{\Frlabelitemii}{\FrenchLabelItem}
\newcommand*{\Frlabelitemiii}{\FrenchLabelItem}
\newcommand*{\Frlabelitemiv}{\FrenchLabelItem}
%    \end{macrocode}
%    |\bbl@frenchlabelitems| saves current itemize labels and changes
%    them to their value in French. This code should never be executed
%    twice in a row, so we need a new flag that will be set and reset
%    by |\bbl@nonfrenchlabelitems| and |\bbl@frenchlabelitems|.
%    \begin{macrocode}
\newif\ifFB@enterFrench  \FB@enterFrenchtrue
\def\bbl@frenchlabelitems{%
  \ifFB@enterFrench
    \let\@ltiORI\labelitemi
    \let\@ltiiORI\labelitemii
    \let\@ltiiiORI\labelitemiii
    \let\@ltivORI\labelitemiv
    \let\labelitemi\Frlabelitemi
    \let\labelitemii\Frlabelitemii
    \let\labelitemiii\Frlabelitemiii
    \let\labelitemiv\Frlabelitemiv
    \FB@enterFrenchfalse
  \fi
}
\let\itemizeORI\itemize
\let\enditemizeORI\enditemize
\let\enditemizeFB\enditemize
\def\itemizeFB{%
    \ifnum \@itemdepth >\thr@@\@toodeep\else
      \advance\@itemdepth\@ne
      \edef\@itemitem{labelitem\romannumeral\the\@itemdepth}%
      \expandafter
      \listORI
      \csname\@itemitem\endcsname
      {\settowidth{\labelwidth}{\csname\@itemitem\endcsname}%
       \setlength{\leftmargin}{\labelwidth}%
       \addtolength{\leftmargin}{\labelsep}%
       \ifnum\@listdepth=0
         \setlength{\itemindent}{\parindent}%
       \else
         \addtolength{\leftmargin}{\parindent}%
       \fi
       \setlength{\itemsep}{\z@}%
       \setlength{\parsep}{\z@}%
       \setlength{\topsep}{\z@}%
       \setlength{\partopsep}{\z@}%
%    \end{macrocode}
%    |\parskip| is of type `skip', its mean value only (\emph{not
%    the glue}) should be subtracted from |\topsep| and added to
%    |\partopsep|, so convert |\parskip| to a `dimen' using
%    |\@tempdima|.
%    \begin{macrocode}
       \@tempdima=\parskip
       \addtolength{\topsep}{-\@tempdima}%
       \addtolength{\partopsep}{\@tempdima}}%
    \fi}
%    \end{macrocode}
%    The user's changes in labelitems are saved when leaving French for
%    further use when switching back to French.  This code should never
%    be executed twice in a row (toggle with |\bbl@frenchlabelitems|).
%    \begin{macrocode}
\def\bbl@nonfrenchlabelitems{%
  \ifFB@enterFrench
  \else
      \let\Frlabelitemi\labelitemi
      \let\Frlabelitemii\labelitemii
      \let\Frlabelitemiii\labelitemiii
      \let\Frlabelitemiv\labelitemiv
      \let\labelitemi\@ltiORI
      \let\labelitemii\@ltiiORI
      \let\labelitemiii\@ltiiiORI
      \let\labelitemiv\@ltivORI
      \FB@enterFrenchtrue
  \fi
}
%    \end{macrocode}
%  \end{macro}
%  \end{macro}
%  \end{macro}
%  \end{macro}
%
%  \subsection{French indentation of sections}
%  \label{sec-indent}
%
%  \begin{macro}{\bbl@frenchindent}
%  \begin{macro}{\bbl@nonfrenchindent}
%    In French the first paragraph of each section should be indented,
%    this is another difference with US-English. This is controlled by
%    the flag |\if@afterindent|.
%
% \changes{v2.3d}{2009/03/16}{Bug correction: previous versions of
%    frenchb set the flag \cs{if@afterindent} to false outside
%    French which is correct for English but wrong for some languages
%    like Spanish.  Pointed out by Juan Jos\'e Torrens.}
%
%    We need to save the value of the flag |\if@afterindent|
%    `AtBeginDocument' before eventually changing its value.
%
%    \begin{macrocode}
\AtBeginDocument{\ifx\@afterindentfalse\@afterindenttrue
                       \let\@aifORI\@afterindenttrue
                 \else \let\@aifORI\@afterindentfalse
                 \fi
}
\def\bbl@frenchindent{\let\@afterindentfalse\@afterindenttrue
                      \@afterindenttrue}
\def\bbl@nonfrenchindent{\let\@afterindentfalse\@aifORI
                         \@afterindentfalse}
%    \end{macrocode}
%  \end{macro}
%  \end{macro}
%
%  \subsection{Formatting footnotes}
%  \label{sec-footnotes}
%
% \changes{v2.0}{2006/11/06}{Footnotes are now printed
%     by default `\`a la fran\c caise' for the whole document.}
%
% \changes{v2.0b}{2007/04/18}{Footnotes: Just do nothing
%    (except warning) when the bigfoot package is loaded.}
%
%    The |bigfoot| package deeply changes the way footnotes are
%    handled. When |bigfoot| is loaded, we just warn the user that
%    |frenchb| will drop the customisation of footnotes.
%
%    The layout of footnotes is controlled by two flags
%    |\ifFBAutoSpaceFootnotes| and |\ifFBFrenchFootnotes| which are
%    set by options of |\frenchbsetup{}| (see section~\ref{sec-keyval}).
%    Notice that the layout of footnotes \emph{does not depend} on the
%    current language (just think of two footnotes on the same page
%    looking different because one was called in a French part, the
%    other one in English!).
%
%    When |\ifFBAutoSpaceFootnotes| is true, |\@footnotemark| (whose
%    definition is saved at the |\begin{document}| in order to include
%    any customisation that packages might have done) is redefined to
%    add a thin space before the number or symbol calling a footnote
%    (any space typed in is removed first).  This has no effect on
%    the layout of the footnote itself.
%
%    \begin{macrocode}
\AtBeginDocument{\@ifpackageloaded{bigfoot}%
                   {\PackageWarning{frenchb.ldf}%
                     {bigfoot package in use.\MessageBreak
                      frenchb will NOT customise footnotes;\MessageBreak
                      reported}}%
                   {\let\@footnotemarkORI\@footnotemark
                    \def\@footnotemarkFB{\leavevmode\unskip\unkern
                                         \,\@footnotemarkORI}%
                    \ifFBAutoSpaceFootnotes
                      \let\@footnotemark\@footnotemarkFB
                    \fi}%
                }
%    \end{macrocode}
%
%    We then define |\@makefntextFB|, a variant of |\@makefntext|
%    which is responsible for the layout of footnotes, to match the
%    specifications of the French `Imprimerie Nationale':  footnotes
%    will be indented by |\parindentFFN|, numbers (if any) typeset on
%    the baseline (instead of superscripts) and followed by a dot
%    and an half quad space. Whenever symbols are used to number
%    footnotes (as in |\thanks| for instance), we switch back to the
%    standard layout (the French layout of footnotes is meant for
%    footnotes numbered by Arabic or Roman digits).
%
% \changes{v2.0}{2006/11/06}{\cs{parindentFFN} not changed if
%    already defined (required by JA for cah-gut.cls).}
%
% \changes{v2.3b}{2008/12/06}{New commands \cs{dotFFN} and
%    \cs{kernFFN} for more flexibility (suggested by JA).}
%
%    The value of |\parindentFFN| will be redefined at the
%    |\begin{document}|, as the maximum of |\parindent| and 1.5em
%    \emph{unless} it has been set in the preamble (the weird value
%    10in is just for testing whether |\parindentFFN| has been set
%    or not).
%
%    \begin{macrocode}
\newcommand*{\dotFFN}{.}
\newcommand*{\kernFFN}{\kern .5em}
\newdimen\parindentFFN
\parindentFFN=10in
\def\ftnISsymbol{\@fnsymbol\c@footnote}
\long\def\@makefntextFB#1{\ifx\thefootnote\ftnISsymbol
                            \@makefntextORI{#1}%
                          \else
                            \parindent=\parindentFFN
                            \rule\z@\footnotesep
                            \setbox\@tempboxa\hbox{\@thefnmark}%
                            \ifdim\wd\@tempboxa>\z@
                              \llap{\@thefnmark}\dotFFN\kernFFN
                            \fi #1
                          \fi}%
%    \end{macrocode}
%
%    We save the standard definition of |\@makefntext| at the
%    |\begin{document}|, and then redefine |\@makefntext| according to
%    the value of flag |\ifFBFrenchFootnotes| (true or false).
%
%    \begin{macrocode}
\AtBeginDocument{\@ifpackageloaded{bigfoot}{}%
                  {\ifdim\parindentFFN<10in
                   \else
                      \parindentFFN=\parindent
                      \ifdim\parindentFFN<1.5em\parindentFFN=1.5em\fi
                   \fi
                   \let\@makefntextORI\@makefntext
                   \long\def\@makefntext#1{%
                      \ifFBFrenchFootnotes
                         \@makefntextFB{#1}%
                      \else
                         \@makefntextORI{#1}%
                      \fi}%
                  }%
                }
%    \end{macrocode}
%
%    For compatibility reasons, we provide definitions for the commands
%    dealing with the layout of footnotes in |frenchb| version~1.6.
%    |\frenchbsetup{}| (see in section \ref{sec-keyval}) should be
%    preferred for setting these options.  |\StandardFootnotes| may
%    still be used locally (in minipages for instance), that's why the
%    test |\ifFBFrenchFootnotes| is done inside |\@makefntext|.
%    \begin{macrocode}
\newcommand*{\AddThinSpaceBeforeFootnotes}{\FBAutoSpaceFootnotestrue}
\newcommand*{\FrenchFootnotes}{\FBFrenchFootnotestrue}
\newcommand*{\StandardFootnotes}{\FBFrenchFootnotesfalse}
%    \end{macrocode}
%
%  \subsection{Global layout}
%  \label{sec-global}
%
%    In multilingual documents, some typographic rules must depend
%    on the current language (e.g., hyphenation, typesetting of
%    numbers, spacing before double punctuation\dots), others should,
%    IMHO, be kept global to the document: especially the layout of
%    lists (see~\ref{sec-lists}) and footnotes
%    (see~\ref{sec-footnotes}), and the indentation of the first
%    paragraph of sections (see~\ref{sec-indent}).
%
%    From version 2.2 on, if |frenchb| is \babel's ``main language''
%    (i.e. last language option at \babel's loading), |frenchb|
%    customises the layout (i.e. lists, indentation of the first
%    paragraphs of sections and footnotes) in the whole document
%    regardless the current language.   On the other hand, if |frenchb|
%    is \emph{not} \babel's ``main language'', it leaves the layout
%    unchanged both in French and in other languages.
%
%  \begin{macro}{\FrenchLayout}
%  \begin{macro}{\StandardLayout}
%    The former commands |\FrenchLayout| and |\StandardLayout| are kept
%    for compatibility reasons but should no longer be used.
%
% \changes{v2.0g}{2008/03/23}{Flag \cs{ifFBStandardLayout} not checked
%     by \cs{FBprocess@options}, low-level flags have to be set
%     one by one.}
%
%    \begin{macrocode}
\newcommand*{\FrenchLayout}{%
    \FBGlobalLayoutFrenchtrue
    \PackageWarning{frenchb.ldf}%
    {\protect\FrenchLayout\space is obsolete.  Please use\MessageBreak
     \protect\frenchbsetup{GlobalLayoutFrench} instead.}%
}
\newcommand*{\StandardLayout}{%
  \FBReduceListSpacingfalse
  \FBCompactItemizefalse
  \FBStandardItemLabelstrue
  \FBIndentFirstfalse
  \FBFrenchFootnotesfalse
  \FBAutoSpaceFootnotesfalse
  \PackageWarning{frenchb.ldf}%
    {\protect\StandardLayout\space is obsolete.  Please use\MessageBreak
    \protect\frenchbsetup{StandardLayout} instead.}%
}
\@onlypreamble\FrenchLayout
\@onlypreamble\StandardLayout
%    \end{macrocode}
%  \end{macro}
%  \end{macro}
%
%  \subsection{Dots\dots}
%  \label{sec-dots}
%
%  \begin{macro}{\FBtextellipsis}
%    \LaTeXe's standard definition of |\dots| in text-mode is
%    |\textellipsis| which includes a |\kern| at the end;
%    this space is not wanted in some cases (before a closing brace
%    for instance) and |\kern| breaks hyphenation of the next word.
%    We define |\FBtextellipsis| for French (in \LaTeXe{} only).
%
%    The |\if| construction in the \LaTeXe{} definition of |\dots|
%    doesn't allow the use of |xspace| (|xspace| is always followed
%    by a |\fi|), so we use the AMS-\LaTeX{} construction of |\dots|;
%    this has to be done `AtBeginDocument' not to be overwritten
%    when \file{amsmath.sty} is loaded after \babel.
%
% \changes{v2.0}{2006/11/06}{Added special case for LY1 encoding,
%    see  bug report from Bruno Voisin (2004/05/18).}
%
%    LY1 has a ready made character for |\textellipsis|, it should be
%    used in French too (pointed out by Bruno Voisin).
%
%    \begin{macrocode}
\DeclareTextSymbol{\FBtextellipsis}{LY1}{133}
\DeclareTextCommandDefault{\FBtextellipsis}{%
    .\kern\fontdimen3\font.\kern\fontdimen3\font.\xspace}
%    \end{macrocode}
%    |\Mdots@| and |\Tdots@ORI| hold the definitions of |\dots| in
%    Math and Text mode. They default to those of amsmath-2.0, and
%    will revert to standard \LaTeX{} definitions `AtBeginDocument',
%    if amsmath has not been loaded. |\Mdots@| doesn't change when
%    switching from/to French, while |\Tdots@| is |\FBtextellipsis|
%    in French and |\Tdots@ORI| otherwise.
%    \begin{macrocode}
\newcommand*{\Tdots@ORI}{\@xp\textellipsis}
\newcommand*{\Tdots@}{\Tdots@ORI}
\newcommand*{\Mdots@}{\@xp\mdots@}
\AtBeginDocument{\DeclareRobustCommand*{\dots}{\relax
                 \csname\ifmmode M\else T\fi dots@\endcsname}%
                 \@ifundefined{@xp}{\let\@xp\relax}{}%
                 \@ifundefined{mdots@}{\let\Tdots@ORI\textellipsis
                                       \let\Mdots@\mathellipsis}{}}
\def\bbl@frenchdots{\let\Tdots@\FBtextellipsis}
\def\bbl@nonfrenchdots{\let\Tdots@\Tdots@ORI}
\expandafter\addto\csname extras\CurrentOption\endcsname{%
    \bbl@frenchdots}
\expandafter\addto\csname noextras\CurrentOption\endcsname{%
    \bbl@nonfrenchdots}
%    \end{macrocode}
%  \end{macro}
%
%  \subsection{Setup options: keyval stuff}
%  \label{sec-keyval}
%
% \changes{v2.0}{2006/11/06}{New command \cs{frenchbsetup} added
%     for global customisation.}
%
% \changes{v2.0c}{2007/06/25}{Option ThinSpaceInFrenchNumbers added.}
%
% \changes{v2.0d}{2007/07/15}{Options og and fg changed: limit
%     the definition to French so that quote characters can be used
%     in German.}
%
% \changes{v2.0e}{2007/10/05}{New option: StandardLists.}
%
% \changes{v2.0f}{2008/03/23}{Two typos corrected in
%    option StandardLists: [false] $\to$ [true] and
%    StandardLayout $\to$ StandardLists.}
%
% \changes{v2.0f}{2008/03/23}{StandardLayout option had no
%     effect on lists.  Test moved to \cs{FBprocess@options}.}
%
% \changes{v2.0g}{2008/03/23}{Revert previous change to
%     StandardLayout. This option must set the three flags
%     \cs{FBReduceListSpacingfalse}, \cs{FBCompactItemizefalse},
%     and \cs{FBStandardItemLabeltrue} instead of
%     \cs{FBStandardListstrue}, so that later options can still
%     change their value before executing \cs{FBprocess@options}.
%     Same thing for option StandardLists.}
%
% \changes{v2.1a}{2008/03/24}{New option: FrenchSuperscripts
%     to define \cs{up} as \cs{fup} or as \cs{textsuperscript}.}
%
% \changes{v2.1a}{2008/03/30}{New option: LowercaseSuperscripts.}
%
% \changes{v2.2a}{2008/05/08}{The global layout of the document is
%     no longer changed when frenchb is not the last option of babel
%     (\cs{bbl@main@language}). Suggested by Ulrike Fischer.}
%
% \changes{v2.2a}{2008/05/08}{Values of flags
%     \cs{ifFBReduceListSpacing}, \cs{ifFBCompactItemize},
%     \cs{ifFBStandardItemLabels}, \cs{ifFBIndentFirst},
%     \cs{ifFBFrenchFootnotes}, \cs{ifFBAutoSpaceFootnotes} changed:
%     default now means `StandardLayout', it will be changed to
%     `FrenchLayout' AtEndOfPackage only if french is
%     \cs{bbl@main@language}.}
%
% \changes{v2.2a}{2008/05/08}{When frenchb is babel's last option,
%     French becomes the document's main language, so
%     GlobalLayoutFrench applies.}
%
% \changes{v2.3a}{2008/10/10}{New option: OriginalTypewriter. Now
%    frenchb switches to \cs{noautospace@beforeFDP} when a tt-font is
%    in use.  When OriginalTypewriter is set to true, frenchb behaves
%    as in pre-2.3 versions.}
%
%    We first define a collection of conditionals with their defaults
%    (true or false).
%
%    \begin{macrocode}
\newif\ifFBStandardLayout           \FBStandardLayouttrue
\newif\ifFBGlobalLayoutFrench       \FBGlobalLayoutFrenchfalse
\newif\ifFBReduceListSpacing        \FBReduceListSpacingfalse
\newif\ifFBCompactItemize           \FBCompactItemizefalse
\newif\ifFBStandardItemLabels       \FBStandardItemLabelstrue
\newif\ifFBStandardLists            \FBStandardListstrue
\newif\ifFBIndentFirst              \FBIndentFirstfalse
\newif\ifFBFrenchFootnotes          \FBFrenchFootnotesfalse
\newif\ifFBAutoSpaceFootnotes       \FBAutoSpaceFootnotesfalse
\newif\ifFBOriginalTypewriter       \FBOriginalTypewriterfalse
\newif\ifFBThinColonSpace           \FBThinColonSpacefalse
\newif\ifFBThinSpaceInFrenchNumbers \FBThinSpaceInFrenchNumbersfalse
\newif\ifFBFrenchSuperscripts       \FBFrenchSuperscriptstrue
\newif\ifFBLowercaseSuperscripts    \FBLowercaseSuperscriptstrue
\newif\ifFBPartNameFull             \FBPartNameFulltrue
\newif\ifFBShowOptions              \FBShowOptionsfalse
%    \end{macrocode}
%
%    The defaults values of these flags have been set so that |frenchb|
%    does not change anything regarding the global layout.
%    |\bbl@main@language| (set by the last option of babel) controls
%    the global layout of the document.  We check the current language
%    `AtEndOfPackage' (it is |\bbl@main@language|); if it is French,
%    the values of some flags have to be changed to ensure a French
%    looking layout for the whole document (even in parts written in
%    languages other than French); the end-user will then be able to
%    customise the values of all these flags with |\frenchbsetup{}|.
%    \begin{macrocode}
\AtEndOfPackage{%
   \iflanguage{french}{\FBReduceListSpacingtrue
                       \FBCompactItemizetrue
                       \FBStandardItemLabelsfalse
                       \FBIndentFirsttrue
                       \FBFrenchFootnotestrue
                       \FBAutoSpaceFootnotestrue
                       \FBGlobalLayoutFrenchtrue}%
                      {}%
}
%    \end{macrocode}
%
%  \begin{macro}{\frenchbsetup}
%    From version 2.0 on, all setup options are handled by \emph{one}
%    command |\frenchbsetup| using the keyval syntax.
%    Let's now define this command which reads and sets the options
%    to be processed later (at |\begin{document}|) by
%    |\FBprocess@options|. It  can only be called in the preamble.
%    \begin{macrocode}
\newcommand*{\frenchbsetup}[1]{%
  \setkeys{FB}{#1}%
}%
\@onlypreamble\frenchbsetup
%    \end{macrocode}
%    |frenchb| being an option of babel, it cannot load a package
%    (keyval) while |frenchb.ldf| is read, so we defer the loading of
%    \file{keyval} and the options setup at the end of \babel's loading.
%
%    |StandardLayout| resets the layout in French to the standard layout
%    defined par the \LaTeX{} class and packages loaded. It deals with
%    lists, indentation of first paragraphs of sections and footnotes.
%    Other keys, entered \emph{after} |StandardLayout| in
%    |\frenchbsetup|, can overrule some of the |StandardLayout|
%     settings.
%
%    |GlobalLayoutFrench| forces the layout in French and (as far as
%    possible) outside French to meet the French typographic standards.
%
% \changes{v2.3d}{2009/03/16}{Warning added to \cs{GlobalLayoutFrench}
%    when French is not the main language.}
%
%    \begin{macrocode}
\AtEndOfPackage{%
    \RequirePackage{keyval}%
    \define@key{FB}{StandardLayout}[true]%
                      {\csname FBStandardLayout#1\endcsname
                       \ifFBStandardLayout
                         \FBReduceListSpacingfalse
                         \FBCompactItemizefalse
                         \FBStandardItemLabelstrue
                         \FBIndentFirstfalse
                         \FBFrenchFootnotesfalse
                         \FBAutoSpaceFootnotesfalse
                         \FBGlobalLayoutFrenchfalse
                       \else
                         \FBReduceListSpacingtrue
                         \FBCompactItemizetrue
                         \FBStandardItemLabelsfalse
                         \FBIndentFirsttrue
                         \FBFrenchFootnotestrue
                         \FBAutoSpaceFootnotestrue
                       \fi}%
    \define@key{FB}{GlobalLayoutFrench}[true]%
                      {\csname FBGlobalLayoutFrench#1\endcsname
                       \ifFBGlobalLayoutFrench
                          \iflanguage{french}%
                            {\FBReduceListSpacingtrue
                             \FBCompactItemizetrue
                             \FBStandardItemLabelsfalse
                             \FBIndentFirsttrue
                             \FBFrenchFootnotestrue
                             \FBAutoSpaceFootnotestrue}%
                            {\PackageWarning{frenchb.ldf}%
                              {Option `GlobalLayoutFrench' skipped:
                               \MessageBreak French is *not*
                               babel's last option.\MessageBreak}}%
                       \fi}%
    \define@key{FB}{ReduceListSpacing}[true]%
                      {\csname FBReduceListSpacing#1\endcsname}%
    \define@key{FB}{CompactItemize}[true]%
                      {\csname FBCompactItemize#1\endcsname}%
    \define@key{FB}{StandardItemLabels}[true]%
                      {\csname FBStandardItemLabels#1\endcsname}%
    \define@key{FB}{ItemLabels}{%
        \renewcommand*{\FrenchLabelItem}{#1}}%
    \define@key{FB}{ItemLabeli}{%
        \renewcommand*{\Frlabelitemi}{#1}}%
    \define@key{FB}{ItemLabelii}{%
        \renewcommand*{\Frlabelitemii}{#1}}%
    \define@key{FB}{ItemLabeliii}{%
        \renewcommand*{\Frlabelitemiii}{#1}}%
    \define@key{FB}{ItemLabeliv}{%
        \renewcommand*{\Frlabelitemiv}{#1}}%
    \define@key{FB}{StandardLists}[true]%
                      {\csname FBStandardLists#1\endcsname
                       \ifFBStandardLists
                         \FBReduceListSpacingfalse
                         \FBCompactItemizefalse
                         \FBStandardItemLabelstrue
                       \else
                         \FBReduceListSpacingtrue
                         \FBCompactItemizetrue
                         \FBStandardItemLabelsfalse
                       \fi}%
    \define@key{FB}{IndentFirst}[true]%
                      {\csname FBIndentFirst#1\endcsname}%
    \define@key{FB}{FrenchFootnotes}[true]%
                      {\csname FBFrenchFootnotes#1\endcsname}%
    \define@key{FB}{AutoSpaceFootnotes}[true]%
                      {\csname FBAutoSpaceFootnotes#1\endcsname}%
    \define@key{FB}{AutoSpacePunctuation}[true]%
                      {\csname FBAutoSpacePunctuation#1\endcsname}%
    \define@key{FB}{OriginalTypewriter}[true]%
                      {\csname FBOriginalTypewriter#1\endcsname}%
    \define@key{FB}{ThinColonSpace}[true]%
                      {\csname FBThinColonSpace#1\endcsname}%
    \define@key{FB}{ThinSpaceInFrenchNumbers}[true]%
                      {\csname FBThinSpaceInFrenchNumbers#1\endcsname}%
    \define@key{FB}{FrenchSuperscripts}[true]%
                      {\csname FBFrenchSuperscripts#1\endcsname}
    \define@key{FB}{LowercaseSuperscripts}[true]%
                      {\csname FBLowercaseSuperscripts#1\endcsname}
    \define@key{FB}{PartNameFull}[true]%
                      {\csname FBPartNameFull#1\endcsname}%
    \define@key{FB}{ShowOptions}[true]%
                      {\csname FBShowOptions#1\endcsname}%
%    \end{macrocode}
%    Inputing French quotes as \emph{single characters} when they are
%    available on the keyboard (through a compose key for instance)
%    is more comfortable than typing |\og| and |\fg|.
%    The purpose of the following code is to map the French quote
%    characters to |\og\ignorespaces| and |{\fg}| respectively when
%    the current language is French, and to |\guillemotleft| and
%    |\guillemotright| otherwise (think of German quotes); thus correct
%    unbreakable spaces will be added automatically to French quotes.
%    The quote characters typed in depend on the input encoding,
%    it can be single-byte (latin1, latin9, applemac,\dots) or
%    multi-bytes (utf-8, utf8x).  We first check whether XeTeX is used
%    or not, if not the package |inputenc| has to be loaded before the
%    |\begin{document}| with the proper coding option, so we check if
%    |\DeclareInputText| is defined.
%    \begin{macrocode}
    \define@key{FB}{og}{%
       \newcommand*{\FB@@og}{\iflanguage{french}%
                               {\FB@og\ignorespaces}{\guillemotleft}}%
       \expandafter\ifx\csname XeTeXrevision\endcsname\relax
         \AtBeginDocument{%
           \@ifundefined{DeclareInputText}%
             {\PackageWarning{frenchb.ldf}%
               {Option `og' requires package inputenc.\MessageBreak}%
             }%
             {\@ifundefined{uc@dclc}%
%    \end{macrocode}
%    if |\uc@dclc| is undefined, utf8x is not loaded\dots
%    \begin{macrocode}
               {\@ifundefined{DeclareUnicodeCharacter}%
%    \end{macrocode}
%    if |\DeclareUnicodeCharacter| is undefined, utf8 is not loaded
%    either, we assume 8-bit character input encoding.
%    Package MULEenc.sty (from CJK) defines |\mule@def| to map
%    characters to control sequences.
%    \begin{macrocode}
                  {\@tempcnta`#1\relax
                     \@ifundefined{mule@def}%
                       {\DeclareInputText{\the\@tempcnta}{\FB@@og}}%
                       {\mule@def{11}{{\FB@@og}}}%
                  }%
%    \end{macrocode}
%    utf8 loaded, use |\DeclareUnicodeCharacter|,
%    \begin{macrocode}
                  {\DeclareUnicodeCharacter{00AB}{\FB@@og}}%
               }%
%    \end{macrocode}
%    utf8x loaded, use |\uc@dclc|,
%    \begin{macrocode}
               {\uc@dclc{171}{default}{\FB@@og}}%
             }%
         }%
%    \end{macrocode}
%    XeTeX in use, the following trick for defining the active quote
%    character is borrowed from \file{inputenc.dtx}.
%    \begin{macrocode}
       \else
         \catcode`#1=\active
         \bgroup
           \uccode`\~`#1%
           \uppercase{%
         \egroup
         \def~%
         }{\FB@@og}%
       \fi
    }%
%    \end{macrocode}
%    Same code for the closing quote.
%    \begin{macrocode}
    \define@key{FB}{fg}{%
       \newcommand*{\FB@@fg}{\iflanguage{french}%
                               {\FB@fg}{\guillemotright}}%
       \expandafter\ifx\csname XeTeXrevision\endcsname\relax
         \AtBeginDocument{%
           \@ifundefined{DeclareInputText}%
             {\PackageWarning{frenchb.ldf}%
               {Option `fg' requires package inputenc.\MessageBreak}%
             }%
             {\@ifundefined{uc@dclc}%
               {\@ifundefined{DeclareUnicodeCharacter}%
                  {\@tempcnta`#1\relax
                     \@ifundefined{mule@def}%
                       {\DeclareInputText{\the\@tempcnta}{{\FB@@fg}}}%
                       {\mule@def{27}{{\FB@@fg}}}%
                  }%
                  {\DeclareUnicodeCharacter{00BB}{{\FB@@fg}}}%
               }%
               {\uc@dclc{187}{default}{{\FB@@fg}}}%
             }%
         }%
       \else
         \catcode`#1=\active
         \bgroup
           \uccode`\~`#1%
           \uppercase{%
         \egroup
         \def~%
         }{{\FB@@fg}}%
       \fi
    }%
}
%    \end{macrocode}
%  \end{macro}
%
% \begin{macro}{\FBprocess@options}
%    |\FBprocess@options| processes the options, it is called \emph{once}
%    at |\begin{document}|.
%    \begin{macrocode}
\newcommand*{\FBprocess@options}{%
%    \end{macrocode}
%    Nothing has to be done here for |StandardLayout| and
%    |StandardLists| (the involved flags have already been set in
%    |\frenchbsetup{}| or before (at babel's EndOfPackage).
%
%    The next three options deal with the layout of lists in French.
%
%    |ReduceListSpacing| reduces the vertical spaces between list
%    items in French (done by changing |\list| to |\listFB|).
%    When |GlobalLayoutFrench| is true (the default), the same is
%    done outside French except for languages that force a different
%    setting.
%    \begin{macrocode}
  \ifFBReduceListSpacing
    \addto\extrasfrench{\let\list\listFB
                        \let\endlist\endlistFB}%
    \addto\noextrasfrench{\ifFBGlobalLayoutFrench
                            \let\list\listFB
                            \let\endlist\endlistFB
                          \else
                            \let\list\listORI
                            \let\endlist\endlistORI
                          \fi}%
  \else
    \addto\extrasfrench{\let\list\listORI
                        \let\endlist\endlistORI}%
    \addto\noextrasfrench{\let\list\listORI
                          \let\endlist\endlistORI}%
  \fi
%    \end{macrocode}
%
%    |CompactItemize| suppresses the vertical spacing between list
%    items in French (done by changing |\itemize| to |\itemizeFB|).
%    When |GlobalLayoutFrench| is true the same is done outside French.
%    \begin{macrocode}
  \ifFBCompactItemize
    \addto\extrasfrench{\let\itemize\itemizeFB
                        \let\enditemize\enditemizeFB}%
    \addto\noextrasfrench{\ifFBGlobalLayoutFrench
                             \let\itemize\itemizeFB
                             \let\enditemize\enditemizeFB
                          \else
                             \let\itemize\itemizeORI
                             \let\enditemize\enditemizeORI
                          \fi}%
  \else
    \addto\extrasfrench{\let\itemize\itemizeORI
                        \let\enditemize\enditemizeORI}%
    \addto\noextrasfrench{\let\itemize\itemizeORI
                          \let\enditemize\enditemizeORI}%
  \fi
%    \end{macrocode}
%
%    |StandardItemLabels| resets labelitems in French to their
%    standard values set by the \LaTeX{} class and packages loaded.
%    When |GlobalLayoutFrench| is true labelitems are identical inside
%    and outside French.
%    \begin{macrocode}
  \ifFBStandardItemLabels
    \addto\extrasfrench{\bbl@nonfrenchlabelitems}%
    \addto\noextrasfrench{\bbl@nonfrenchlabelitems}%
  \else
    \addto\extrasfrench{\bbl@frenchlabelitems}%
    \addto\noextrasfrench{\ifFBGlobalLayoutFrench
                            \bbl@frenchlabelitems
                          \else
                            \bbl@nonfrenchlabelitems
                          \fi}%
  \fi
%    \end{macrocode}
%
%    |IndentFirst| forces the first paragraphs of sections to be
%    indented just like the other ones in French.
%    When |GlobalLayoutFrench| is true (the default), the same is
%    done outside French except for languages that force a different
%    setting.
%    \begin{macrocode}
  \ifFBIndentFirst
    \addto\extrasfrench{\bbl@frenchindent}%
    \addto\noextrasfrench{\ifFBGlobalLayoutFrench
                             \bbl@frenchindent
                          \else
                             \bbl@nonfrenchindent
                          \fi}%
  \else
    \addto\extrasfrench{\bbl@nonfrenchindent}%
    \addto\noextrasfrench{\bbl@nonfrenchindent}%
  \fi
%    \end{macrocode}
%
%    The layout of footnotes is handled at the |\begin{document}|
%    depending on the values of flags |FrenchFootnotes|
%    and |AutoSpaceFootnotes| (see section~\ref{sec-footnotes}),
%    nothing has to be done here for footnotes.
%
%    |AutoSpacePunctuation| adds an unbreakable space (in French only)
%    before the four active characters (:;!?) even if none has been
%    typed before them.
%    \begin{macrocode}
  \ifFBAutoSpacePunctuation
     \autospace@beforeFDP
  \else
     \noautospace@beforeFDP
  \fi
%    \end{macrocode}
%
%    When |OriginalTypewriter| is set to |false| (the default),
%    |\ttfamily|, |\rmfamily| and |\sffamily| are redefined as
%    |\ttfamilyFB|, |\rmfamilyFB| and |\sffamilyFB| respectively
%    to prevent addition of automatic spaces before the four active
%    characters in computer code.
%    \begin{macrocode}
  \ifFBOriginalTypewriter
  \else
     \let\ttfamily\ttfamilyFB
     \let\rmfamily\rmfamilyFB
     \let\sffamily\sffamilyFB
  \fi
%    \end{macrocode}
%
%    |ThinColonSpace| changes the normal unbreakable space typeset in
%     French before `:' to a thin space.
%    \begin{macrocode}
  \ifFBThinColonSpace\renewcommand*{\Fcolonspace}{\thinspace}\fi
%    \end{macrocode}
%
%    When |true|, |ThinSpaceInFrenchNumbers| redefines |numprint.sty|'s
%    command |\npstylefrench| to set |\npthousandsep| to |\,|
%    (thinspace) instead of |~| (default) . This option has no effect
%    if package |numprint.sty| is not loaded with `|autolanguage|'.
%    As old versions of |numprint.sty| did not define |\npstylefrench|,
%    we have to provide this command.
%    \begin{macrocode}
  \@ifpackageloaded{numprint}%
  {\ifnprt@autolanguage
     \providecommand*{\npstylefrench}{}%
     \ifFBThinSpaceInFrenchNumbers
       \renewcommand*\npstylefrench{%
          \npthousandsep{\,}%
          \npdecimalsign{,}%
          \npproductsign{\cdot}%
          \npunitseparator{\,}%
          \npdegreeseparator{}%
          \nppercentseparator{\nprt@unitsep}%
          }%
     \else
       \renewcommand*\npstylefrench{%
          \npthousandsep{~}%
          \npdecimalsign{,}%
          \npproductsign{\cdot}%
          \npunitseparator{\,}%
          \npdegreeseparator{}%
          \nppercentseparator{\nprt@unitsep}%
          }%
     \fi
     \npaddtolanguage{french}{french}%
   \fi}{}%
%    \end{macrocode}
%
%    |FrenchSuperscripts|: if |true| |\up=\fup|, else
%    |\up=\textsuperscript|. Anyway |\up*=\FB@up@fake|. The star-form
%    |\up*{}| is provided for fonts that lack some superior letters:
%    Adobe Jenson Pro and Utopia Expert have no ``g superior'' for
%    instance.
%    \begin{macrocode}
  \ifFBFrenchSuperscripts
    \DeclareRobustCommand*{\up}{\@ifstar{\FB@up@fake}{\fup}}%
  \else
    \DeclareRobustCommand*{\up}{\@ifstar{\FB@up@fake}%
                                        {\textsuperscript}}%
  \fi
%    \end{macrocode}
%
%    |LowercaseSuperscripts|: if |true| let |\FB@lc| be |\lowercase|,
%     else |\FB@lc| is redefined to do nothing.
%    \begin{macrocode}
  \ifFBLowercaseSuperscripts
  \else
    \renewcommand*{\FB@lc}[1]{##1}%
  \fi
%    \end{macrocode}
%
%    |PartNameFull|: if |false|, redefine |\partname|.
%    \begin{macrocode}
  \ifFBPartNameFull
  \else\addto\captionsfrench{\def\partname{Partie}}\fi
%    \end{macrocode}
%
%    |ShowOptions|: if |true|, print the list of all options to the
%    \file{.log} file.
%    \begin{macrocode}
  \ifFBShowOptions
    \GenericWarning{* }{%
     * **** List of possible options for frenchb ****\MessageBreak
     [Default values between brackets when frenchb is loaded *LAST*]%
     \MessageBreak
     ShowOptions=true [false]\MessageBreak
     StandardLayout=true [false]\MessageBreak
     GlobalLayoutFrench=false [true]\MessageBreak
     StandardLists=true [false]\MessageBreak
     ReduceListSpacing=false [true]\MessageBreak
     CompactItemize=false [true]\MessageBreak
     StandardItemLabels=true [false]\MessageBreak
     ItemLabels=\textemdash, \textbullet,
        \protect\ding{43},... [\textendash]\MessageBreak
     ItemLabeli=\textemdash, \textbullet,
        \protect\ding{43},... [\textendash]\MessageBreak
     ItemLabelii=\textemdash, \textbullet,
        \protect\ding{43},... [\textendash]\MessageBreak
     ItemLabeliii=\textemdash, \textbullet,
        \protect\ding{43},... [\textendash]\MessageBreak
     ItemLabeliv=\textemdash, \textbullet,
        \protect\ding{43},... [\textendash]\MessageBreak
     IndentFirst=false [true]\MessageBreak
     FrenchFootnotes=false [true]\MessageBreak
     AutoSpaceFootnotes=false [true]\MessageBreak
     AutoSpacePunctuation=false [true]\MessageBreak
     OriginalTypewriter=true [false]\MessageBreak
     ThinColonSpace=true [false]\MessageBreak
     ThinSpaceInFrenchNumbers=true [false]\MessageBreak
     FrenchSuperscripts=false [true]\MessageBreak
     LowercaseSuperscripts=false [true]\MessageBreak
     PartNameFull=false [true]\MessageBreak
     og= <left quote character>, fg= <right quote character>
     \MessageBreak
     *********************************************
     \MessageBreak\protect\frenchbsetup{ShowOptions}}
  \fi
}
%    \end{macrocode}
%  \end{macro}
%
% \changes{v2.0}{2006/12/15}{AtBeginDocument, save again the
%    definitions of the `list' and `itemize' environments and the
%    values of labelitems.  As of frenchb v.1.6, `ORI' values were
%    set when reading frenchb.ldf, later changes were ignored.}
%
% \changes{v2.0}{2006/12/06}{Added warning for OT1 encoding.}
%
% \changes{v2.1b}{2008/04/07}{Disable some commands in bookmarks.}
%
%    At |\begin{document}| we save again the definitions of the `list'
%    and `itemize' environments and the values of labelitems so that
%    all changes made in the preamble are taken into account in
%    languages other than French and in French with the StandardLayout
%    option.  We also have to provide an |\xspace| command in case the
%    |xspace.sty| package is not loaded.
%
%    \begin{macrocode}
\AtBeginDocument{%
   \let\listORI\list
   \let\endlistORI\endlist
   \let\itemizeORI\itemize
   \let\enditemizeORI\enditemize
   \let\@ltiORI\labelitemi
   \let\@ltiiORI\labelitemii
   \let\@ltiiiORI\labelitemiii
   \let\@ltivORI\labelitemiv
   \providecommand*{\xspace}{\relax}%
%    \end{macrocode}
%    Let's redefine some commands in \file{hyperref}'s bookmarks.
%    \begin{macrocode}
   \@ifundefined{pdfstringdefDisableCommands}{}%
     {\pdfstringdefDisableCommands{%
        \let\up\relax
        \def\ieme{e\xspace}%
        \def\iemes{es\xspace}%
        \def\ier{er\xspace}%
        \def\iers{ers\xspace}%
        \def\iere{re\xspace}%
        \def\ieres{res\xspace}%
        \def\FrenchEnumerate#1{#1\degre\space}%
        \def\FrenchPopularEnumerate#1{#1\degre)\space}%
        \def\No{N\degre\space}%
        \def\no{n\degre\space}%
        \def\Nos{N\degre\space}%
        \def\nos{n\degre\space}%
        \def\og{\guillemotleft\space}%
        \def\fg{\space\guillemotright}%
        \let\bsc\textsc
        \let\degres\degre
     }}%
%    \end{macrocode}
%    It is time to process the options set with |\frenchboptions{}|.
%    Then execute either |\extrasfrench| and |\captionsfrench| or
%    |\noextrasfrench| according to the current language at the
%    |\begin{document}| (these three commands are updated by
%    |\FBprocess@options|).
%    \begin{macrocode}
   \FBprocess@options
   \iflanguage{french}{\extrasfrench\captionsfrench}{\noextrasfrench}%
%    \end{macrocode}
%    Some warnings are issued when output font encodings are not
%    properly set. With XeLaTeX, \file{fontspec.sty} and
%    \file{xunicode.sty} should be loaded; with (pdf)\LaTeX, a warning
%    is issued when OT1 encoding is in use at the |\begin{document}|.
%    Mind that |\encodingdefault| is defined as `long', defining
%    |\FBOTone| with |\newcommand*| would fail!
%    \begin{macrocode}
   \expandafter\ifx\csname XeTeXrevision\endcsname\relax
      \begingroup \newcommand{\FBOTone}{OT1}%
      \ifx\encodingdefault\FBOTone
        \PackageWarning{frenchb.ldf}%
           {OT1 encoding should not be used for French.
            \MessageBreak
            Add \protect\usepackage[T1]{fontenc} to the
            preamble\MessageBreak of your document,}
      \fi
     \endgroup
   \else
     \@ifundefined{DeclareUTFcharacter}%
       {\PackageWarning{frenchb.ldf}%
         {Add \protect\usepackage{fontspec} *and*\MessageBreak
          \protect\usepackage{xunicode} to the preamble\MessageBreak
          of your document,}}%
       {}%
    \fi
}
%    \end{macrocode}
%
%  \subsection{Clean up and exit}
%
%    Load |frenchb.cfg| (should do nothing, just for compatibility).
%    \begin{macrocode}
\loadlocalcfg{frenchb}
%    \end{macrocode}
%    Final cleaning.
%    The macro |\ldf@quit| takes care for setting the main language
%    to be switched on at |\begin{document}| and resetting the
%    category code of \texttt{@} to its original value.
%    The config file searched for has to be |frenchb.cfg|, and
%    |\CurrentOption| has been set to `french', so
%    |\ldf@finish\CurrentOption| cannot be used: we first load
%    |frenchb.cfg|, then call |\ldf@quit\CurrentOption|.
%    \begin{macrocode}
\FBclean@on@exit
\ldf@quit\CurrentOption
%    \end{macrocode}
% \iffalse
%</code>
%<*dtx>
% \fi
%%
%% \CharacterTable
%%  {Upper-case    \A\B\C\D\E\F\G\H\I\J\K\L\M\N\O\P\Q\R\S\T\U\V\W\X\Y\Z
%%   Lower-case    \a\b\c\d\e\f\g\h\i\j\k\l\m\n\o\p\q\r\s\t\u\v\w\x\y\z
%%   Digits        \0\1\2\3\4\5\6\7\8\9
%%   Exclamation   \!     Double quote  \"     Hash (number) \#
%%   Dollar        \$     Percent       \%     Ampersand     \&
%%   Acute accent  \'     Left paren    \(     Right paren   \)
%%   Asterisk      \*     Plus          \+     Comma         \,
%%   Minus         \-     Point         \.     Solidus       \/
%%   Colon         \:     Semicolon     \;     Less than     \<
%%   Equals        \=     Greater than  \>     Question mark \?
%%   Commercial at \@     Left bracket  \[     Backslash     \\
%%   Right bracket \]     Circumflex    \^     Underscore    \_
%%   Grave accent  \`     Left brace    \{     Vertical bar  \|
%%   Right brace   \}     Tilde         \~}
%%
% \iffalse
%</dtx>
% \fi
%
% \Finale
\endinput
}
\bbl@tempa{francais}{% \iffalse meta-comment
%
% Copyright 1989-2009 Johannes L. Braams and any individual authors
% listed elsewhere in this file.  All rights reserved.
% 
% This file is part of the Babel system.
% --------------------------------------
% 
% It may be distributed and/or modified under the
% conditions of the LaTeX Project Public License, either version 1.3
% of this license or (at your option) any later version.
% The latest version of this license is in
%   http://www.latex-project.org/lppl.txt
% and version 1.3 or later is part of all distributions of LaTeX
% version 2003/12/01 or later.
% 
% This work has the LPPL maintenance status "maintained".
% 
% The Current Maintainer of this work is Johannes Braams.
% 
% The list of all files belonging to the Babel system is
% given in the file `manifest.bbl. See also `legal.bbl' for additional
% information.
% 
% The list of derived (unpacked) files belonging to the distribution
% and covered by LPPL is defined by the unpacking scripts (with
% extension .ins) which are part of the distribution.
% \fi
% \CheckSum{2135}
%
% \iffalse
%    Tell the \LaTeX\ system who we are and write an entry on the
%    transcript. Nothing to write to the .cfg file, if generated.
%<*dtx>
\ProvidesFile{frenchb.dtx}
%</dtx>
% \changes{v2.1d}{2008/05/04}{Argument of \cs{ProvidesLanguage} changed
%     from `french' to `frenchb', otherwise \cs{listfiles} prints
%     no date/version information.  The bug with \cs{listfiles}
%     (introduced in v.1.5!), was pointed out by Ulrike Fischer.}
%<code>\ProvidesLanguage{frenchb}
%\ProvidesFile{frenchb.dtx}
%<*!cfg>
        [2009/03/16 v2.3d French support from the babel system]
%</!cfg>
%<*cfg>
%% frenchb.cfg: configuration file for frenchb.ldf
%% Daniel Flipo Daniel.Flipo at univ-lille1.fr
%</cfg>
%%    File `frenchb.dtx'
%%    Babel package for LaTeX version 2e
%%    Copyright (C) 1989 - 2009
%%              by Johannes Braams, TeXniek
%
%<*!cfg>
%%    Frenchb language Definition File
%%    Copyright (C) 1989 - 2009
%%              by Johannes Braams, TeXniek
%%                 Daniel Flipo, GUTenberg
%
%%    Please report errors to: Daniel Flipo, GUTenberg
%%                             Daniel.Flipo at univ-lille1.fr
%</!cfg>
%
%    This file is part of the babel system, it provides the source
%    code for the French language definition file.
%
%<*filedriver>
\documentclass[a4paper]{ltxdoc}
\DeclareFontEncoding{T1}{}{}
\DeclareFontSubstitution{T1}{lmr}{m}{n}
\DeclareTextCommand{\guillemotleft}{OT1}{%
  {\fontencoding{T1}\fontfamily{lmr}\selectfont\char19}}
\DeclareTextCommand{\guillemotright}{OT1}{%
  {\fontencoding{T1}\fontfamily{lmr}\selectfont\char20}}
\newcommand*\TeXhax{\TeX hax}
\newcommand*\babel{\textsf{babel}}
\newcommand*\langvar{$\langle \mathit lang \rangle$}
\newcommand*\note[1]{}
\newcommand*\Lopt[1]{\textsf{#1}}
\newcommand*\file[1]{\texttt{#1}}
\begin{document}
\setlength{\parindent}{0pt}
\begin{center}
  \textbf{\Large A Babel language definition file for French}\\[3mm]^^A\]
  Daniel \textsc{Flipo}\\
  \texttt{Daniel.Flipo@univ-lille1.fr}
\end{center}
 \RecordChanges
 \DocInput{frenchb.dtx}
\end{document}
%</filedriver>
% \fi
% \GetFileInfo{frenchb.dtx}
%
%  \section{The French language}
%
%    The file \file{\filename}\footnote{The file described in this
%    section has version number \fileversion\ and was last revised on
%    \filedate.}, defines all the language definition macros for the
%    French language.
%
%    Customisation for the French language is achieved following the
%    book ``Lexique des r\`egles typographiques en usage \`a
%    l'Imprimerie nationale'' troisi\`eme \'edition (1994),
%    ISBN-2-11-081075-0.
%
%    First version released: 1.1 (1996/05/31) as part of
%    \babel-3.6beta.
%
%    |frenchb| has been improved using helpful suggestions from many
%    people, mainly from Jacques Andr\'e, Michel Bovani, Thierry Bouche,
%    and Vincent Jalby.  Thanks to all of them!
%
%    This new version (2.x) has been designed to be used with \LaTeXe{}
%    and Plain\TeX{} formats only. \LaTeX-2.09 is no longer supported.
%    Changes between version 1.6 and \fileversion{} are listed in
%    subsection~\ref{ssec-changes} p.~\pageref{ssec-changes}.
%
%    An extensive documentation is available in French here:\\
%    |http://daniel.flipo.free.fr/frenchb|
%
%  \subsection{Basic interface}
%
%    In a multilingual document, some typographic rules are language
%    dependent, i.e. spaces before `double punctuation' (|:| |;| |!|
%    |?|) in French, others concern the general layout (i.e. layout of
%    lists, footnotes, indentation of first paragraphs of sections) and
%    should apply to the whole document.
%
%    Starting with version~2.2, |frenchb| behaves differently according
%    to \babel's \emph{main language} defined as the \emph{last}
%    option\footnote{Its name is kept in \texttt{\textbackslash
%           bbl@main@language}.} at \babel's loading.  When French is
%    not \babel's main language, |frenchb| no longer alters the global
%    layout of the document (even in parts where French is the current
%    language): the layout of lists, footnotes, indentation of first
%    paragraphs of sections are not customised by |frenchb|.
%
%    When French is loaded as the last option of \babel, |frenchb|
%    makes the following changes to the global layout, \emph{both in
%    French and in all other languages}\footnote{%
%       For each item, hooks are provided to reset standard
%       \LaTeX{} settings or to emulate the behavior of former versions
%       of \texttt{frenchb} (see command
%       \texttt{\textbackslash frenchbsetup\{\}},
%       section~\ref{ssec-custom}).}:
%    \begin{enumerate}
%    \item the first paragraph of each section is indented
%          (\LaTeX{} only);
%    \item the default items in itemize environment are set to `--'
%          instead of `\textbullet', and all vertical spacing and glue
%          is deleted; it is possible to change `--' to something else
%          (`---' for instance) using |\frenchbsetup{}|;
%    \item vertical spacing in general \LaTeX{} lists is
%          shortened;
%    \item footnotes are displayed ``\`a la fran\c{c}aise''.
%    \end{enumerate}
%
%    Regarding local typography, the command |\selectlanguage{french}|
%    switches to the French language\footnote{%
%      \texttt{\textbackslash selectlanguage\{francais\}}
%      and \texttt{\textbackslash selectlanguage\{frenchb\}} are kept
%      for backward compatibility but should no longer be used.},
%    with the following effects:
%    \begin{enumerate}
%    \item French hyphenation patterns are made active;
%    \item `double punctuation' (|:| |;| |!| |?|) is made
%           active%\footnote{Actually, they are active in the whole
%           document, only their expansions differ in French and
%           outside French} for correct spacing in French;
%    \item |\today| prints the date in French;
%    \item the caption names are translated into French
%          (\LaTeX{} only);
%    \item the space after |\dots| is removed in French.
%    \end{enumerate}
%
%    Some commands are provided in |frenchb| to make typesetting
%    easier:
%    \begin{enumerate}
%    \item French quotation marks can be entered using the commands
%          |\og| and |\fg| which work in \LaTeXe and Plain\TeX,
%          their appearance depending on what is available to draw
%          them; even if you use \LaTeXe{} \emph{and} |T1|-encoding,
%          you should refrain from entering them as
%          |<<~French quotation marks~>>|: |\og| and |\fg| provide
%          better horizontal spacing.
%          |\og| and |\fg| can be used outside French, they typeset
%          then English quotes `` and ''.
%    \item A command |\up| is provided to typeset superscripts like
%          |M\up{me}| (abbreviation for ``Madame''), |1\up{er}| (for
%          ``premier'').  Other commands are also provided for
%          ordinals: |\ier|, |\iere|, |\iers|, |\ieres|, |\ieme|,
%          |\iemes| (|3\iemes| prints 3\textsuperscript{es}).
%    \item Family names should be typeset in small capitals and never
%          be hyphenated, the macro |\bsc| (boxed small caps) does
%          this, e.g., |Leslie~\bsc{Lamport}| will produce
%          Leslie~\mbox{\textsc{Lamport}}. Note that composed names
%          (such as Dupont-Durant) may now be hyphenated on explicit
%          hyphens, this differs from |frenchb|~v.1.x.
%    \item Commands |\primo|, |\secundo|, |\tertio| and |\quarto|
%          print 1\textsuperscript{o}, 2\textsuperscript{o},
%          3\textsuperscript{o}, 4\textsuperscript{o}.
%          |\FrenchEnumerate{6}| prints 6\textsuperscript{o}.
%    \item Abbreviations for ``Num\'ero(s)'' and ``num\'ero(s)''
%          (N\textsuperscript{o} N\textsuperscript{os}
%          n\textsuperscript{o} and n\textsuperscript{os}~)
%          are obtained via the commands |\No|, |\Nos|, |\no|, |\nos|.
%    \item Two commands are provided to typeset the symbol for
%          ``degr\'e'': |\degre| prints the raw character and
%          |\degres| should be used to typeset temperatures (e.g.,
%          ``|20~\degres C|'' with an unbreakable space), or for
%          alcohols' strengths (e.g., ``|45\degres|'' with \emph{no}
%          space in French).
%    \item In math mode the comma has to be surrounded with
%          braces to avoid a spurious space being inserted after it,
%          in decimal numbers for instance (see the \TeX{}book p.~134).
%          The command |\DecimalMathComma| makes the comma be an
%          ordinary character \emph{in French only} (no space added);
%          as a counterpart, if |\DecimalMathComma| is active, an
%          explicit space has to be added in lists and intervals:
%          |$[0,\ 1]$|, |$(x,\ y)$|. |\StandardMathComma| switches back
%          to the standard behaviour of the comma.
%    \item A command |\nombre| was provided in 1.x versions to easily
%          format numbers in slices of three digits separated either
%          by a comma in English or with a space in French; |\nombre|
%          is now mapped to |\numprint| from \file{numprint.sty}, see
%          \file{numprint.pdf} for more information.
%    \item |frenchb| has been designed to take advantage of the |xspace|
%          package if present: adding |\usepackage{xspace}| in the
%          preamble will force macros like |\fg|, |\ier|, |\ieme|,
%          |\dots|, \dots, to respect the spaces you type after them,
%          for instance typing `|1\ier juin|' will print
%          `1\textsuperscript{er} juin' (no need for a forced space
%          after |1\ier|).
%    \end{enumerate}
%
%  \subsection{Customisation}
%  \label{ssec-custom}
%
%     Up to version 1.6, customisation of |frenchb| was achieved
%     by entering commands in \file{frenchb.cfg}.  This possibility
%     remains for compatibility, but \emph{should not longer be used}.
%     Version 2.0 introduces a new command |\frenchbsetup{}| using
%     the \file{keyval} syntax which should make it easier to choose
%     among the many options available. The command |\frenchbsetup{}|
%     is to appear in the preamble only (after loading \babel).
%
%     \vspace{.5\baselineskip}
%     |\frenchbsetup{ShowOptions}| prints all available options to
%     the \file{.log} file, it is just meant as a remainder of the
%     list of offered options. As usual with \file{keyval} syntax,
%     boolean options (as |ShowOptions|) can be entered as
%     |ShowOptions=true| or just |ShowOptions|, the `|=true|' part
%     can be omitted.
%
%     \vspace{.5\baselineskip}
%     The other options are listed below. Their default value is shown
%     between brackets, sometimes followed be a `\texttt{*}'.
%     The `\texttt{*}' means that the default shown applies when
%     |frenchb| is loaded as the \emph{last} option of \babel{}
%     ---\babel's \emph{main language}---, and is toggled otherwise:
%     \begin{itemize}
%     \item |StandardLayout=true [false*]| forces |frenchb| not to
%       interfere with the layout: no action on any kind of lists,
%       first paragraphs of sections are not indented (as in English),
%       no action on footnotes. This option replaces the former
%       command |\StandardLayout|.  It can be used to avoid conflicts
%       with classes or packages which customise lists or footnotes.
%     \item |GlobalLayoutFrench=false [true*]| can be used, when French
%       is the main language, to emulate what prior versions of
%       |frenchb| (pre-2.2) did: lists, and first paragraphs
%       of sections will be displayed the standard way in other
%       languages than French, and ``\`a la fran\c{c}aise'' in French.
%       Note that the layout of footnotes is language independent
%       anyway (see below |FrenchFootnotes| and |AutoSpaceFootnotes|).
%       This option replaces the former command |\FrenchLayout|.
%     \item |ReduceListSpacing=false [true*]|; |frenchb| normally
%       reduces the values of the vertical spaces used in the
%       environment |list| in French; setting this option to |false|
%       reverts to the standard settings of |list|.  This option
%       replaces the former command |\FrenchListSpacingfalse|.
%     \item |CompactItemize=false [true*]|; |frenchb| normally
%       suppresses any vertical space between items of |itemize| lists
%       in French; setting this option to |false| reverts to the
%       standard settings of |itemize| lists.  This option replaces
%       the former command |\FrenchItemizeSpacingfalse|.
%     \item |StandardItemLabels=true [false*]| when set to |true| this
%       option stops |frenchb| from changing the labels in |itemize|
%       lists in French.
%     \item |ItemLabels=\textemdash|, |\textbullet|, |\ding{43}|,
%       \dots, |[\textendash*]|; when |StandardItemLabels=false| (the
%       default), this option enables to choose the label used in
%       |itemize| lists for all levels.  The next three options do
%       the same but each one for one level only. Note that the
%       example |\ding{43}| requires |\usepackage{pifont}|.
%     \item |ItemLabeli=\textemdash|, |\textbullet|, |\ding{43}|,
%       \dots,|[\textendash*]|
%     \item |ItemLabelii=\textemdash|, |\textbullet|, |\ding{43}|,
%       \dots, |[\textendash*]|
%     \item |ItemLabeliii=\textemdash|, |\textbullet|, |\ding{43}|,
%       \dots, |[\textendash*]|
%     \item |ItemLabeliv=\textemdash|, |\textbullet|, |\ding{43}|,
%       \dots, |[\textendash*]|
%     \item |StandardLists=true [false*]| forbids |frenchb| to
%       customise any kind of list. Do activate the option
%       |StandardLists| when using classes or packages that customise
%       lists too (|enumitem|, |paralist|, \dots{}) to avoid conflicts.
%       This option is just a shorthand for |ReduceListSpacing=false|
%       and |CompactItemize=false| and |StandardItemLabels=true|.
%     \item |IndentFirst=false [true*]|; |frenchb| normally forces
%       indentation of the first paragraph of sections.
%       When this option is set to |false|, the first paragraph of
%       will look the same in French and in English (not indented).
%     \item |FrenchFootnotes=false [true*]| reverts to the standard
%       layout of footnotes. By default |frenchb| typesets leading
%       numbers as `1.\hspace{.5em}' instead of `$\hbox{}^1$', but
%       has no effect on footnotes numbered with symbols (as in the
%       |\thanks| command).  The former commands |\StandardFootnotes|
%       and |\FrenchFootnotes| are still there, |\StandardFootnotes|
%       can be useful when some footnotes are numbered with letters
%       (inside minipages for instance).
%     \item |AutoSpaceFootnotes=false [true*]| ; by default |frenchb|
%       adds a thin space in the running text before the number or
%       symbol calling the footnote.  Making this option |false|
%       reverts to the standard setting (no space added).
%     \item |FrenchSuperscripts=false [true]| ; then
%       |\up=\textsuperscript| (option added in version 2.1).
%       Should only be made |false| to recompile older documents.
%       By default |\up| now relies on |\fup| designed to produce
%       better looking superscripts.
%     \item |AutoSpacePunctuation=false [true]|; in French, the user
%       \emph{should} input a space before the four characters `|:;!?|'
%       but as many people forget about it (even among native French
%       writers!), the default behaviour of |frenchb| is to
%       automatically add a |\thinspace| before `|;|' `|!|' `|?|' and a
%       normal (unbreakable) space before~`|:|' (this is recommended by
%       the French Imprimerie nationale).  This is convenient in most
%       cases but can lead to addition of spurious spaces in URLs or in
%       MS-DOS paths but only if they are no typed using |\texttt| or
%       verbatim mode. When the current font is a monospaced
%       (typewriter) font, |AutoSpacePunctuation| is locally switched
%       to |false|, no spurious space is added in that case, so the
%       default behaviour of of |frenchb| in that area should be fine
%       in most circumstances.
%
%       Choosing |AutoSpacePunctuation=false| will ensure that
%       a proper space will be added before `|:;!?|' \emph{if and only
%       if} a (normal) space has been typed in. Those who are unsure
%       about their typing in this area should stick to the default
%       option and type |\string;| |\string:| |\string!| |\string?|
%       instead of |;| |:| |!| |?| in case an unwanted space is
%       added by |frenchb|.
%     \item |ThinColonSpace=true [false]| changes the normal
%       (unbreakable) space added before the colon `:' to a thin space,
%       so that the same amount of space is added before any of the
%       four double punctuation characters. The default setting is
%       supported by the French Imprimerie nationale.
%     \item |LowercaseSuperscripts=false [true]| ; by default |frenchb|
%       inhibits the uppercasing of superscripts (for instance when they
%       are moved to page headers). Making this option |false|
%       will disable this behaviour (not recommended).
%     \item |PartNameFull=false [true]|; when true, |frenchb| numbers
%       the title of |\part{}| commands as ``Premi\`ere partie'',
%       ``Deuxi\`eme partie'' and so on. With some classes which change
%       the|\part{}| command (AMS and SMF classes do so), you will get
%       ``Premi\`ere partie~I'', ``Deuxi\`eme partie~II'' instead;
%       when this occurs, this option should be set to |false|,
%       part titles will then be printed as ``Partie I'', ``Partie II''.
%     \item |og=|\texttt{\guillemotleft}, |fg=|\texttt{\guillemotright};
%       when guillemets characters are available on the keyboard
%       (through a compose key for instance), it is nice to use them
%       instead of typing |\og| and |\fg|. This option tells |frenchb|
%       which characters are opening and closing French guillemets
%       (they depend on the input encoding), then you can type either
%       \texttt{\guillemotleft{} guillemets \guillemotright}, or
%       \texttt{\guillemotleft{}guillemets\guillemotright} (with or
%       without spaces), to get properly typeset French quotes.
%       This option requires \file{inputenc} to be loaded with the
%       proper encoding, it works with 8-bits encodings (latin1,
%       latin9, ansinew,  applemac,\dots) and multi-byte encodings
%       (utf8 and utf8x).
%     \end{itemize}
%
%  \subsection{Hyphenation checks}
%  \label{ssec-hyphen}
%
%    Once you have built your format, a good precaution would be to
%    perform some basic tests about hyphenation in French. For
%    \LaTeXe{} I suggest this:
%    \begin{itemize}
%    \item run the following file, with the encoding suitable for
%      your machine (\textit{my-encoding} will be |latin1| for
%      \textsc{unix} machines, |ansinew| for PCs running~Windows,
%      |applemac| or |latin1| for Macintoshs, or |utf8|\dots\\[3mm]^^A\]
%      |%%% Test file for French hyphenation.|\\
%      |\documentclass{article}|\\
%      |\usepackage[|\textit{my-encoding}|]{inputenc}|\\
%      |\usepackage[T1]{fontenc} % Use LM fonts|\\
%      |\usepackage{lmodern}     % for French|\\
%      |\usepackage[frenchb]{babel}|\\
%      |\begin{document}|\\
%      |\showhyphens{signal container \'ev\'enement alg\`ebre}|\\
%      |\showhyphens{|\texttt{signal container \'ev\'enement
%                     alg\`ebre}|}|\\
%      |\end{document}|
%    \item check the hyphenations proposed by \TeX{} in your log-file;
%      in French you should get with both 7-bit and 8-bit encodings\\
%      \texttt{si-gnal contai-ner \'ev\'e-ne-ment al-g\`ebre}.\\
%      Do not care about how accented characters are displayed in the
%      log-file, what matters is the position of the `|-|' hyphen
%      signs \emph{only}.
%    \end{itemize}
%    If they are all correct, your installation (probably) works fine,
%    if one (or more) is (are) wrong, ask a local wizard to see what's
%    going wrong and perform the test again (or e-mail me about what
%    happens).\\
%    Frequent mismatches:
%    \begin{itemize}
%    \item you get |sig-nal con-tainer|, this probably means that the
%    hyphenation patterns you are using are for US-English, not for
%    French;
%    \item you get no hyphen at all in \texttt{\'ev\'e-ne-ment}, this
%    probably means that you are using CM fonts and the macro
%    |\accent| to produce accented characters.
%    Using 8-bits fonts with built-in accented characters avoids
%    this kind of mismatch.
%    \end{itemize}
%
%    \textbf{Options' order} -- Please remember that options are read
%    in the order they appear inside the |\frenchbsetup| command.
%    Someone wishing that |frenchb| leaves the layout of lists
%    and footnotes untouched but caring for indentation of first
%    paragraph of sections could choose
%    |\frenchbsetup{StandardLayout,IndentFirst}| and get the expected
%    layout. Choosing |\frenchbsetup{IndentFirst,StandardLayout}|
%    would not lead to the expected result: option |IndentFirst| would
%    be overwritten by |StandardLayout|.
%
%  \subsection{Changes}
%  \label{ssec-changes}
%
%  \subsubsection*{What's new in version 2.0?}
%
%    Here is the list of all changes:
%    \begin{itemize}
%    \item Support for \LaTeX-2.09 and for \LaTeXe{} in compatibility
%      mode has been dropped. This version is meant for \LaTeXe{} and
%      Plain based formats (like \file{bplain}). \LaTeXe{} formats
%      based on ml\TeX{} are no longer supported either (plenty of
%      good 8-bits fonts are available now, so T1 encoding should
%      be preferred for typesetting in French). A warning is issued
%      when OT1 encoding is in use at the |\begin{document}|.
%    \item Customisation should now be handled by command
%      |\frenchbsetup{}|, \file{frenchb.cfg} (kept for compatibility)
%      should no longer be used. See section~\ref{ssec-custom} for
%      the list of available options.
%    \item Captions in figures and table have changed in French: former
%      abbreviations ``Fig.'' and ``Tab.'' have been replaced by full
%      names ``Figure'' and ``Table''.  If this leads to formatting
%      problems in captions, you can add the following two commands to
%      your preamble (after loading \babel) to get the former captions\\
%      |\addto\captionsfrench{\def\figurename{{\scshape Fig.}}}|\\
%      |\addto\captionsfrench{\def\tablename{{\scshape Tab.}}}|.
%    \item The |\nombre| command is now provided by the \file{numprint}
%      package which has to be loaded \emph{after} \babel{} with the
%      option |autolanguage| if number formatting should depend on the
%      current language.
%    \item The |\bsc| command no longer uses an |\hbox| to stop
%      hyphenation of names but a |\kern0pt| instead. This change
%      enables \file{microtype} to fine tune the length of the
%      argument of |\bsc|; as a side-effect, compound names like
%      Dupont-Durand can now be hyphenated on  explicit hyphens.
%      You can get back to the former behaviour of |\bsc| by adding\\
%      |\renewcommand*{\bsc}[1]{\leavevmode\hbox{\scshape #1}}|\\
%      to the preamble of your document.
%    \item Footnotes are now displayed ``\`a la fran\c caise'' for the
%      whole document, except with an explicit\\
%      |\frenchbsetup{AutoSpaceFootnotes=false,FrenchFootnotes=false}|.\\
%      Add this command if you want standard footnotes. It is still
%      possible to revert locally to the standard layout of footnotes
%      by adding |\StandardFootnotes| (inside a |minipage| environment
%      for instance).
%    \end{itemize}
%
%  \subsubsection*{What's new in version 2.1?}
%
%      New command |\fup| to typeset better looking superscripts.
%      Former command |\up| is now defined as |\fup|, but an option
%      |\frenchbsetup{FrenchSuperscripts=false}| is provided for
%      backward compatibility.  |\fup| was designed using ideas from
%      Jacques Andr\'e, Thierry Bouche and Ren\'e Fritz, thanks to them!
%
%  \subsubsection*{What's new in version 2.2?}
%
%      Starting with version~2.2a, |frenchb| alters the layout of
%      lists, footnotes, and the indentation of first paragraphs of
%      sections) \emph{only if} French is the ``main language''
%      (i.e. babel's last language option). The layout is global for
%      the whole document: lists, etc. look the same in French and in
%      other languages, everything is typeset ``\`a la fran\c caise''
%      if French is the ``main language'', otherwise |frenchb| doesn't
%      change anything regarding lists, footnotes, and indentation of
%      paragraphs.
%
%  \subsubsection*{What's new in version 2.3?}
%
%      Starting with version~2.3a, |frenchb| no longer inserts spaces
%      automatically before `|:;!?|' when a typewriter font is in use;
%      this was suggested by Yannis Haralambous to prevent
%      spurious spaces in computer source code or expressions like
%      \texttt{C\string:/foo}, \texttt{http\string://foo.bar},
%      etc.  An option (|OriginalTypewriter|) is provided to get back
%      to the former behaviour of |frenchb|.
%
%      Another probably invisible change: lowercase conversion in
%      |\up{}| is now achieved by the \LaTeX{} command |\MakeLowercase|
%      instead of \TeX's |\lowercase| command.  This prevents error
%      messages when diacritics are used inside |\up{}| (diacritics
%      should \emph{never} be used in superscripts though!).
%
% \StopEventually{}
%
%  \subsection{File frenchb.cfg}
%  \label{sec-cfg}
%
%    \file{frenchb.cfg} is now a dummy file just kept for compatibility
%    with previous versions.
%
% \iffalse
%<*cfg>
% \fi
%    \begin{macrocode}
%%%%%%%%%%%%%%%%%%%%%%%%%%%%%%%%%%%%%%%%%%%%%%%%%%%%%%%%%%%%%%%%%%%%%%
%%%%%%%%%  WARNING: THIS  FILE SHOULD  NO  LONGER  BE  USED  %%%%%%%%%
%% If you want to customise frenchb, please DO NOT hack into the code!
%% Do no put any code in this file either, please use the new command
%% \frenchbsetup{} with the proper options to customise frenchb.
%% 
%% Add \frenchbsetup{ShowOptions} to your preamble to see the list of
%% available options and/or read the documentation.
%%%%%%%%%%%%%%%%%%%%%%%%%%%%%%%%%%%%%%%%%%%%%%%%%%%%%%%%%%%%%%%%%%%%%%
%    \end{macrocode}
% \iffalse
%</cfg>
% \fi
%
%  \section{\TeX{}nical details}
%
%  \subsection{Initial setup}
%
% \changes{v2.1d}{2008/05/02}{Argument of \cs{ProvidesLanguage} changed
%     above from `french' to `frenchb' (otherwise \cs{listfiles} prints
%     no date/version information).  The real name of current language
%     (french) as to be corrected before calling \cs{LdfInit}.}
%
% \iffalse
%<*code>
% \fi
%
%    While this file was read through the option \Lopt{frenchb} we make
%    it behave as if \Lopt{french} was specified.
%    \begin{macrocode}
\def\CurrentOption{french}
%    \end{macrocode}
%
%    The macro |\LdfInit| takes care of preventing that this file is
%    loaded more than once, checking the category code of the
%    \texttt{@} sign, etc.
%
%    \begin{macrocode}
\LdfInit\CurrentOption\datefrench
%    \end{macrocode}
%
% \changes{v2.1d}{2008/05/04}{Avoid warning ``\cs{end} occurred
%   when \cs{ifx} ... incomplete'' with LaTeX-2.09.}
%
%  \begin{macro}{\ifLaTeXe}
%    No support is provided for late \LaTeX-2.09: issue a warning
%    and exit if \LaTeX-2.09 is in use. Plain is still supported.
%    \begin{macrocode}
\newif\ifLaTeXe
\let\bbl@tempa\relax
\ifx\magnification\@undefined
   \ifx\@compatibilitytrue\@undefined
     \PackageError{frenchb.ldf}
        {LaTeX-2.09 format is no longer supported.\MessageBreak
         Aborting here}
        {Please upgrade to LaTeX2e!}
     \let\bbl@tempa\endinput
   \else
     \LaTeXetrue
   \fi
\fi
\bbl@tempa
%    \end{macrocode}
%  \end{macro}
%
%    Check if hyphenation patterns for the French language have been
%    loaded in language.dat; we allow for the names `french',
%    `francais', `canadien' or `acadian'. The latter two are both
%    names used in Canada for variants of French that are in use in
%    that country.
%
%    \begin{macrocode}
\ifx\l@french\@undefined
  \ifx\l@francais\@undefined
    \ifx\l@canadien\@undefined
      \ifx\l@acadian\@undefined
        \@nopatterns{French}
        \adddialect\l@french0
      \else
        \let\l@french\l@acadian
      \fi
    \else
      \let\l@french\l@canadien
    \fi
  \else
    \let\l@french\l@francais
  \fi
\fi
%    \end{macrocode}
%    Now |\l@french| is always defined.
%
%    The internal name for the French language is |french|;
%    |francais| and |frenchb| are synonymous for |french|:
%    first let both names use the same hyphenation patterns.
%    Later we will have to set aliases for |\captionsfrench|,
%    |\datefrench|, |\extrasfrench| and |\noextrasfrench|.
%    As French uses the standard values of |\lefthyphenmin| (2)
%    and |\righthyphenmin| (3), no special setting is required here.
%
%    \begin{macrocode}
\ifx\l@francais\@undefined
  \let\l@francais\l@french
\fi
\ifx\l@frenchb\@undefined
  \let\l@frenchb\l@french
\fi
%    \end{macrocode}
%    When this language definition file was loaded for one of the
%    Canadian versions of French we need to make sure that a suitable
%    hyphenation pattern register will be found by \TeX.
%    \begin{macrocode}
\ifx\l@canadien\@undefined
  \let\l@canadien\l@french
\fi
\ifx\l@acadian\@undefined
  \let\l@acadian\l@french
\fi
%    \end{macrocode}
%
%    This language definition can be loaded for different variants of
%    the French language. The `key' \babel\ macros are only defined
%    once, using `french' as the language name, but |frenchb| and
%    |francais| are synonymous.
%    \begin{macrocode}
\def\datefrancais{\datefrench}
\def\datefrenchb{\datefrench}
\def\extrasfrancais{\extrasfrench}
\def\extrasfrenchb{\extrasfrench}
\def\noextrasfrancais{\noextrasfrench}
\def\noextrasfrenchb{\noextrasfrench}
%    \end{macrocode}
%
% \begin{macro}{\extrasfrench}
% \begin{macro}{\noextrasfrench}
%    The macro |\extrasfrench| will perform all the extra
%    definitions needed for the French language.
%    The macro |\noextrasfrench| is used to cancel the actions of
%    |\extrasfrench|.\\
%    In French, character ``apostrophe'' is a letter in expressions
%    like |l'ambulance| (French  hyphenation patterns provide entries
%    for this kind of words).  This means that the |\lccode| of
%    ``apostrophe'' has to be non null in French for proper hyphenation
%    of those expressions, and has to be reset to null when exiting
%    French.
%
%    \begin{macrocode}
\@namedef{extras\CurrentOption}{\lccode`\'=`\'}
\@namedef{noextras\CurrentOption}{\lccode`\'=0}
%    \end{macrocode}
% \end{macro}
% \end{macro}
%
%    One more thing |\extrasfrench| needs to do is to make sure that
%    |\frenchspacing| is in effect.  |\noextrasfrench| will switch
%    |\frenchspacing| off again.
%    \begin{macrocode}
  \expandafter\addto\csname extras\CurrentOption\endcsname{%
    \bbl@frenchspacing}
  \expandafter\addto\csname noextras\CurrentOption\endcsname{%
    \bbl@nonfrenchspacing}
%    \end{macrocode}
%
%  \subsection{Punctuation}
%  \label{sec-punct}
%
%    As long as no better solution is available%
%    \footnote{Lua\TeX, or pdf\TeX{} might provide alternatives in
%       the future\dots},
%    the `double punctuation' characters (|;| |!| |?| and |:|) have to
%    be made |\active| for an automatic control of the amount of space
%    to insert before them. Before doing so, we have to save the
%    standard definition of |\@makecaption| (which includes two ':')
%    to compare it later to its definition at the |\begin{document}|.
%    \begin{macrocode}
\long\def\STD@makecaption#1#2{%
  \vskip\abovecaptionskip
  \sbox\@tempboxa{#1: #2}%
  \ifdim \wd\@tempboxa >\hsize
    #1: #2\par
  \else
    \global \@minipagefalse
    \hb@xt@\hsize{\hfil\box\@tempboxa\hfil}%
  \fi
  \vskip\belowcaptionskip}%
%    \end{macrocode}
%
%    We define a new `if' |\FBpunct@active| which will be made false
%    whenever a better alternative will be available. Another `if'
%    |\FBAutoSpacePunctuation| needs to be defined now.
%    \begin{macrocode}
\newif\ifFBpunct@active          \FBpunct@activetrue
\newif\ifFBAutoSpacePunctuation  \FBAutoSpacePunctuationtrue
%    \end{macrocode}
%    The following code makes the four characters |;| |!| |?| and |:|
%    `active' and provides their definitions.
%    \begin{macrocode}
\ifFBpunct@active
  \initiate@active@char{:}
  \initiate@active@char{;}
  \initiate@active@char{!}
  \initiate@active@char{?}
%    \end{macrocode}
%    We first tune the amount of space before \texttt{;}
%    \texttt{!}  \texttt{?} and \texttt{:}.  This should only happen
%    in horizontal mode, hence the test |\ifhmode|.
%
%    In horizontal mode, if a space has been typed before `;' we
%    remove it and put an unbreakable |\thinspace| instead. If no
%    space has been typed, we add |\FDP@thinspace| which will be
%    defined, up to the user's wishes, as an automatic added
%    thin space, or as |\@empty|.
%    \begin{macrocode}
  \declare@shorthand{french}{;}{%
      \ifhmode
      \ifdim\lastskip>\z@
          \unskip\penalty\@M\thinspace
          \else
            \FDP@thinspace
        \fi
      \fi
%    \end{macrocode}
%    Now we can insert a |;| character.
%    \begin{macrocode}
      \string;}
%    \end{macrocode}
%    The next three definitions are very similar.
%    \begin{macrocode}
  \declare@shorthand{french}{!}{%
      \ifhmode
        \ifdim\lastskip>\z@
          \unskip\penalty\@M\thinspace
        \else
          \FDP@thinspace
        \fi
      \fi
      \string!}
  \declare@shorthand{french}{?}{%
      \ifhmode
        \ifdim\lastskip>\z@
          \unskip\penalty\@M\thinspace
        \else
          \FDP@thinspace
        \fi
      \fi
      \string?}
%    \end{macrocode}
%    According to the I.N. specifications, the `:' requires a normal
%    space before it, but some people prefer a |\thinspace| (just
%    like the other three). We define |\Fcolonspace| to hold the
%    required amount of space (user customisable).
%    \begin{macrocode}
  \newcommand*{\Fcolonspace}{\space}
  \declare@shorthand{french}{:}{%
      \ifhmode
        \ifdim\lastskip>\z@
          \unskip\penalty\@M\Fcolonspace
        \else
          \FDP@colonspace
        \fi
      \fi
      \string:}
%    \end{macrocode}
%
% \changes{v2.3a}{2008/10/10}{\cs{NoAutoSpaceBeforeFDP} and
%    \cs{AutoSpaceBeforeFDP} now set the flag
%    \cs{ifFBAutoSpacePunctuation} accordingly (LaTeX only).}
%
%  \begin{macro}{\AutoSpaceBeforeFDP}
%  \begin{macro}{\NoAutoSpaceBeforeFDP}
%    |\FDP@thinspace| and |\FDP@space| are defined as unbreakable
%    spaces by |\autospace@beforeFDP| or as |\@empty| by
%    |\noautospace@beforeFDP| (internal commands), user commands
%    |\AutoSpaceBeforeFDP| and |\NoAutoSpaceBeforeFDP| do the same and
%    take care of the flag |\ifFBAutoSpacePunctuation| in \LaTeX{}.
%    Set the default now for Plain (done later for \LaTeX).
%    \begin{macrocode}
  \def\autospace@beforeFDP{%
          \def\FDP@thinspace{\penalty\@M\thinspace}%
          \def\FDP@colonspace{\penalty\@M\Fcolonspace}}
  \def\noautospace@beforeFDP{\let\FDP@thinspace\@empty
                            \let\FDP@colonspace\@empty}
  \ifLaTeXe
    \def\AutoSpaceBeforeFDP{\autospace@beforeFDP
                            \FBAutoSpacePunctuationtrue}
    \def\NoAutoSpaceBeforeFDP{\noautospace@beforeFDP
                              \FBAutoSpacePunctuationfalse}
  \else
    \let\AutoSpaceBeforeFDP\autospace@beforeFDP
    \let\NoAutoSpaceBeforeFDP\noautospace@beforeFDP
    \AutoSpaceBeforeFDP
  \fi
%    \end{macrocode}
% \end{macro}
% \end{macro}
%
% \changes{v2.3a}{2008/10/10}{In LaTeX, frenchb no longer adds spaces
%     before `double punctuation' characters in computer code.
%     Suggested by Yannis Haralambous.}
%
% \changes{v2.3c}{2009/02/07}{Commands \cs{ttfamily}, \cs{rmfamily}
%    and \cs{sffamily} have to be robust.  Bug introduced in 2.3a,
%    pointed out by Manuel P\'egouri\'e-Gonnard.}
%
%    In \LaTeXe{} |\ttfamily| (and hence |\texttt|) will be redefined
%    `AtBeginDocument' as |\ttfamilyFB| so that no space
%    is added before the four |; : ! ?| characters, even if
%    |AutoSpacePunctuation| is true.  |\rmfamily| and |\sffamily| need
%    to be redefined also (|\ttfamily| is not always used inside a
%    group, its effect can be cancelled by |\rmfamily| or |\sffamily|).
%
%    These redefinitions can be canceled if necessary, for instance to
%    recompile older documents, see option |OriginalTypewriter| below.
%    \begin{macrocode}
  \ifLaTeXe
    \let\ttfamilyORI\ttfamily
    \let\rmfamilyORI\rmfamily
    \let\sffamilyORI\sffamily
    \DeclareRobustCommand\ttfamilyFB{%
         \noautospace@beforeFDP\ttfamilyORI}%
    \DeclareRobustCommand\rmfamilyFB{%
         \ifFBAutoSpacePunctuation
            \autospace@beforeFDP\rmfamilyORI
         \else
            \noautospace@beforeFDP\rmfamilyORI
         \fi}%
    \DeclareRobustCommand\sffamilyFB{%
         \ifFBAutoSpacePunctuation
            \autospace@beforeFDP\sffamilyORI
         \else
            \noautospace@beforeFDP\sffamilyORI
         \fi}%
  \fi
%    \end{macrocode}
%
%    When the active characters appear in an environment where their
%    French behaviour is not wanted they should give an `expected'
%    result. Therefore we define shorthands at system level as well.
%    \begin{macrocode}
  \declare@shorthand{system}{:}{\string:}
  \declare@shorthand{system}{!}{\string!}
  \declare@shorthand{system}{?}{\string?}
  \declare@shorthand{system}{;}{\string;}
%    \end{macrocode}
%    We specify that the French group of shorthands should be used.
%    \begin{macrocode}
  \addto\extrasfrench{%
    \languageshorthands{french}%
%    \end{macrocode}
%    These characters are `turned on' once, later their definition may
%    vary. Don't misunderstand the following code: they keep being
%    active all along the document, even when leaving French.
%    \begin{macrocode}
    \bbl@activate{:}\bbl@activate{;}%
    \bbl@activate{!}\bbl@activate{?}%
  }
  \addto\noextrasfrench{%
  \bbl@deactivate{:}\bbl@deactivate{;}%
  \bbl@deactivate{!}\bbl@deactivate{?}}
\fi
%    \end{macrocode}
%
%  \subsection{French quotation marks}
%
%  \begin{macro}{\og}
%  \begin{macro}{\fg}
%    The top macros for quotation marks will be called |\og|
%    (``\underline{o}uvrez \underline{g}uillemets'') and |\fg|
%    (``\underline{f}ermez \underline{g}uillemets'').
%    Another option for typesetting quotes in multilingual texts
%    is to use the package |csquotes.sty| and its command |\enquote|.
%
%    \begin{macrocode}
\newcommand*{\og}{\@empty}
\newcommand*{\fg}{\@empty}
%    \end{macrocode}
%  \end{macro}
%  \end{macro}
%
%  \begin{macro}{\guillemotleft}
%  \begin{macro}{\guillemotright}
%    \LaTeX{} users are supposed to use 8-bit output encodings (T1,
%    LY1,\dots) to typeset French, those who still stick to OT1 should
%    call |aeguill.sty| or a similar package. In both cases the
%    commands |\guillemotleft| and |\guillemotright| will print the
%    French opening and closing quote characters from the output font.
%    For XeLaTeX, |\guillemotleft| and |\guillemotright| are defined
%    by package \file{xunicode.sty}.
%    We will check `AtBeginDocument' that the proper output encodings
%    are in use (see end of section~\ref{sec-keyval}).
%
%    We give the following definitions for Plain users only as a (poor)
%    fall-back, they are welcome to change them for anything better.
%    \begin{macrocode}
\ifLaTeXe
\else
  \ifx\guillemotleft\@undefined
    \def\guillemotleft{\leavevmode\raise0.25ex
                       \hbox{$\scriptscriptstyle\ll$}}
  \fi
  \ifx\guillemotright\@undefined
    \def\guillemotright{\raise0.25ex
                        \hbox{$\scriptscriptstyle\gg$}}
  \fi
  \let\xspace\relax
\fi
%    \end{macrocode}
%  \end{macro}
%  \end{macro}
%
%    The next step is to provide correct spacing after |\guillemotleft|
%    and before |\guillemotright|: a space precedes and follows
%    quotation marks but no line break is allowed neither \emph{after}
%    the opening one, nor \emph{before} the closing one.
%    |\FBguill@spacing| which does the spacing, has been fine tuned by
%    Thierry Bouche.  French quotes (including spacing) are printed by
%    |\FB@og| and |\FB@fg|, the expansion of the top level commands
%    |\og| and |\og| is different in and outside French.
%    We'll try to be smart to users of David~Carlisle's |xspace|
%    package: if this package is loaded there will be no need for |{}|
%    or |\ | to get a space after |\fg|, otherwise |\xspace| will be
%    defined as |\relax| (done at the end of this file).
%
%    \begin{macrocode}
\newcommand*{\FBguill@spacing}{\penalty\@M\hskip.8\fontdimen2\font
                                            plus.3\fontdimen3\font
                                           minus.8\fontdimen4\font}
\DeclareRobustCommand*{\FB@og}{\leavevmode
                               \guillemotleft\FBguill@spacing}
\DeclareRobustCommand*{\FB@fg}{\ifdim\lastskip>\z@\unskip\fi
                               \FBguill@spacing\guillemotright\xspace}
%    \end{macrocode}
%
%    The top level definitions for French quotation marks are switched
%    on and off through the |\extrasfrench| |\noextrasfrench|
%    mechanism. Outside French, |\og| and |\fg| will typeset standard
%    English opening and closing double quotes.
%
%    \begin{macrocode}
\ifLaTeXe
  \def\bbl@frenchguillemets{\renewcommand*{\og}{\FB@og}%
                            \renewcommand*{\fg}{\FB@fg}}
  \def\bbl@nonfrenchguillemets{\renewcommand*{\og}{\textquotedblleft}%
            \renewcommand*{\fg}{\ifdim\lastskip>\z@\unskip\fi
                                   \textquotedblright}}
\else
   \def\bbl@frenchguillemets{\let\og\FB@og
                             \let\fg\FB@fg}
   \def\bbl@nonfrenchguillemets{\def\og{``}%
                     \def\fg{\ifdim\lastskip>\z@\unskip\fi ''}}
\fi
\expandafter\addto\csname extras\CurrentOption\endcsname{%
  \bbl@frenchguillemets}
\expandafter\addto\csname noextras\CurrentOption\endcsname{%
  \bbl@nonfrenchguillemets}
%    \end{macrocode}
%
%  \subsection{Date in French}
%
% \begin{macro}{\datefrench}
%    The macro |\datefrench| redefines the command |\today| to
%    produce French dates.
%
% \changes{v2.0}{2006/11/06}{2 '\cs{relax}' added in
%    \cs{today}'s definition.}
%
% \changes{v2.1a}{2008/03/25}{\cs{today} changed (correction in 2.0
%    was wrong: \cs{today} was printed without spaces in toc).}
%
%    \begin{macrocode}
\@namedef{date\CurrentOption}{%
  \def\today{{\number\day}\ifnum1=\day {\ier}\fi \space
    \ifcase\month
      \or janvier\or f\'evrier\or mars\or avril\or mai\or juin\or
      juillet\or ao\^ut\or septembre\or octobre\or novembre\or
      d\'ecembre\fi
    \space \number\year}}
%    \end{macrocode}
% \end{macro}
%
%  \subsection{Extra utilities}
%
%    Let's provide the French user with some extra utilities.
%
% \changes{v2.1a}{2008/03/24}{Command \cs{fup} added to produce
%    better superscripts than \cs{textsuperscript}.}
%
%  \begin{macro}{\up}
%
% \changes{v2.1c}{2008/04/29}{Provide a temporary definition
%    (hyperref safe) of \cs{up} in case it has to be expanded in
%    the preamble (by beamer's \cs{title} command for instance).}
%
%  \begin{macro}{\fup}
%
% \changes{v2.1b}{2008/04/02}{Command \cs{fup} changed to use
%    real superscripts from fourier v. 1.6.}
%
% \changes{v2.2a}{2008/05/08}{\cs{newif} and \cs{newdimen} moved
%    before \cs{ifLaTeXe} to avoid an error with plainTeX.}
%
% \changes{v2.3a}{2008/09/30}{\cs{lowercase} changed to
%    \cs{MakeLowercase} as the former doesn't work for non ASCII
%    characters in encodings like applemac, utf-8,\dots}
%
%    |\up| eases the typesetting of superscripts like
%    `1\textsuperscript{er}'.  Up to version 2.0 of |frenchb| |\up| was
%    just a shortcut for |\textsuperscript| in \LaTeXe, but several
%    users complained that |\textsuperscript| typesets superscripts
%    too high and too big, so we now define |\fup| as an attempt to
%    produce better looking superscripts.  |\up| is defined as |\fup|
%    but can be redefined by |\frenchbsetup{FrenchSuperscripts=false}|
%    as |\textsuperscript| for compatibility with previous versions.
%
%    When a font has built-in superscripts, the best thing to do is
%    to just use them, otherwise |\fup| has to simulate superscripts
%    by scaling and raising ordinary letters.  Scaling is done using
%    package \file{scalefnt} which will be loaded at the end of
%    \babel's loading (|frenchb| being an option of babel, it cannot
%    load a package while being read).
%
%    \begin{macrocode}
\newif\ifFB@poorman
\newdimen\FB@Mht
\ifLaTeXe
  \AtEndOfPackage{\RequirePackage{scalefnt}}
%    \end{macrocode}
%    |\FB@up@fake| holds the definition of fake superscripts.
%    The scaling ratio is 0.65, raising is computed to put the top of
%    lower case letters (like `m') just under the top  of upper case
%    letters (like `M'), precisely 12\% down.  The chosen settings
%    look correct for most fonts, but can be tuned by the end-user
%    if necessary by changing |\FBsupR| and |\FBsupS| commands.
%
%    |\FB@lc| is defined as |\MakeLowercase| to inhibit the uppercasing
%    of superscripts (this may happen in page headers with the standard
%    classes but is wrong); |\FB@lc| can be redefined to do nothing
%    by option |LowercaseSuperscripts=false| of |\frenchbsetup{}|.
%    \begin{macrocode}
  \newcommand*{\FBsupR}{-0.12}
  \newcommand*{\FBsupS}{0.65}
  \newcommand*{\FB@lc}[1]{\MakeLowercase{#1}}
  \DeclareRobustCommand*{\FB@up@fake}[1]{%
    \settoheight{\FB@Mht}{M}%
    \addtolength{\FB@Mht}{\FBsupR \FB@Mht}%
    \addtolength{\FB@Mht}{-\FBsupS ex}%
    \raisebox{\FB@Mht}{\scalefont{\FBsupS}{\FB@lc{#1}}}%
    }
%    \end{macrocode}
%    The only packages I currently know to take advantage of real
%    superscripts are a) \file{xltxtra} used in conjunction with
%    XeLaTeX and OpenType fonts having the font feature
%    'VerticalPosition=Superior' (\file{xltxtra} defines
%    |\realsuperscript| and |\fakesuperscript|) and b) \file{fourier}
%    (from version 1.6) when Expert Utopia fonts are available.
%
%    |\FB@up| checks whether the current font is a Type1 `Expert'
%    (or `Pro') font with real superscripts or not (the code works
%    currently only with \file{fourier-1.6} but could work with any
%    Expert Type1 font with built-in superscripts, see below), and
%    decides to use real or fake superscripts.
%    It works as follows: the content of |\f@family| (family name of
%    the current font) is split by |\FB@split| into two pieces, the
%    first three characters (`|fut|' for Fourier, `|ppl|' for Adobe's
%    Palatino, \dots) stored in |\FB@firstthree| and the rest stored
%    in |\FB@suffix| which is expected to be `|x|' or `|j|' for expert
%    fonts.
%    \begin{macrocode}
  \def\FB@split#1#2#3#4\@nil{\def\FB@firstthree{#1#2#3}%
                             \def\FB@suffix{#4}}
  \def\FB@x{x}
  \def\FB@j{j}
  \DeclareRobustCommand*{\FB@up}[1]{%
    \bgroup \FB@poormantrue
      \expandafter\FB@split\f@family\@nil
%    \end{macrocode}
%    Then |\FB@up| looks for a \file{.fd} file named \file{t1fut-sup.fd}
%    (Fourier) or \file{t1ppl-sup.fd} (Palatino), etc. supposed to
%    define the subfamily (|fut-sup| or |ppl-sup|, etc.) giving access
%    to the built-in superscripts.  If the \file{.fd} file is not found
%    by |\IfFileExists|, |\FB@up| falls back on fake superscripts,
%    otherwise |\FB@suffix| is checked to decide whether to use fake or
%    real superscripts.
%    \begin{macrocode}
      \edef\reserved@a{\lowercase{%
         \noexpand\IfFileExists{\f@encoding\FB@firstthree -sup.fd}}}%
      \reserved@a
        {\ifx\FB@suffix\FB@x \FB@poormanfalse\fi
         \ifx\FB@suffix\FB@j \FB@poormanfalse\fi
         \ifFB@poorman \FB@up@fake{#1}%
         \else         \FB@up@real{#1}%
         \fi}%
        {\FB@up@fake{#1}}%
    \egroup}
%    \end{macrocode}
%    |\FB@up@real| just picks up the superscripts from the subfamily
%    (and forces lowercase).
%    \begin{macrocode}
  \newcommand*{\FB@up@real}[1]{\bgroup
       \fontfamily{\FB@firstthree -sup}\selectfont \FB@lc{#1}\egroup}
%    \end{macrocode}
%    |\fup| is now defined as |\FB@up| unless |\realsuperscript| is
%    defined (occurs with XeLaTeX calling \file{xltxtra.sty}).
%    \begin{macrocode}
  \DeclareRobustCommand*{\fup}[1]{%
    \@ifundefined{realsuperscript}%
      {\FB@up{#1}}%
      {\bgroup\let\fakesuperscript\FB@up@fake
            \realsuperscript{\FB@lc{#1}}\egroup}}
%    \end{macrocode}
%    Temporary definition of |up| (redefined `AtBeginDocument').
%    \begin{macrocode}
  \newcommand*{\up}{\relax}
%    \end{macrocode}
%    Poor man's definition of |\up| for Plain. In \LaTeXe,
%    |\up| will be defined as |\fup| or |\textsuperscript| later on
%    while processing the options of |\frenchbsetup{}|.
%    \begin{macrocode}
\else
  \newcommand*{\up}[1]{\leavevmode\raise1ex\hbox{\sevenrm #1}}
\fi
%    \end{macrocode}
%  \end{macro}
%  \end{macro}
%
%  \begin{macro}{\ieme}
%  \begin{macro}{\ier}
%  \begin{macro}{\iere}
%  \begin{macro}{\iemes}
%  \begin{macro}{\iers}
%  \begin{macro}{\ieres}
%  Some handy macros for those who don't know how to abbreviate ordinals:
%    \begin{macrocode}
\def\ieme{\up{\lowercase{e}}\xspace}
\def\iemes{\up{\lowercase{es}}\xspace}
\def\ier{\up{\lowercase{er}}\xspace}
\def\iers{\up{\lowercase{ers}}\xspace}
\def\iere{\up{\lowercase{re}}\xspace}
\def\ieres{\up{\lowercase{res}}\xspace}
%    \end{macrocode}
%  \end{macro}
%  \end{macro}
%  \end{macro}
%  \end{macro}
%  \end{macro}
%  \end{macro}
%
% \changes{v2.1c}{2008/04/29}{Added commands \cs{Nos} and \cs{nos}.}
%
%  \begin{macro}{\No}
%  \begin{macro}{\no}
%  \begin{macro}{\Nos}
%  \begin{macro}{\nos}
%  \begin{macro}{\primo}
%  \begin{macro}{\fprimo)}
%    And some more macros relying on |\up| for numbering,
%    first two support macros.
%    \begin{macrocode}
\newcommand*{\FrenchEnumerate}[1]{%
                       #1\up{\lowercase{o}}\kern+.3em}
\newcommand*{\FrenchPopularEnumerate}[1]{%
                       #1\up{\lowercase{o}})\kern+.3em}
%    \end{macrocode}
%
%    Typing |\primo| should result in `$1^{\rm o}$\kern+.3em',
%    \begin{macrocode}
\def\primo{\FrenchEnumerate1}
\def\secundo{\FrenchEnumerate2}
\def\tertio{\FrenchEnumerate3}
\def\quarto{\FrenchEnumerate4}
%    \end{macrocode}
%    while typing |\fprimo)| gives `1$^{\rm o}$)\kern+.3em.
%    \begin{macrocode}
\def\fprimo){\FrenchPopularEnumerate1}
\def\fsecundo){\FrenchPopularEnumerate2}
\def\ftertio){\FrenchPopularEnumerate3}
\def\fquarto){\FrenchPopularEnumerate4}
%    \end{macrocode}
%
%    Let's provide four macros for the common abbreviations
%    of ``Num\'ero''.
%    \begin{macrocode}
\DeclareRobustCommand*{\No}{N\up{\lowercase{o}}\kern+.2em}
\DeclareRobustCommand*{\no}{n\up{\lowercase{o}}\kern+.2em}
\DeclareRobustCommand*{\Nos}{N\up{\lowercase{os}}\kern+.2em}
\DeclareRobustCommand*{\nos}{n\up{\lowercase{os}}\kern+.2em}
%    \end{macrocode}
%  \end{macro}
%  \end{macro}
%  \end{macro}
%  \end{macro}
%  \end{macro}
%  \end{macro}
%
%  \begin{macro}{\bsc}
%    As family names should be written in small capitals and never be
%    hyphenated, we provide a command (its name comes from Boxed Small
%    Caps) to input them easily.  Note that this command has changed
%    with version~2 of |frenchb|: a |\kern0pt| is used instead of |\hbox|
%    because |\hbox| would break microtype's font expansion; as a
%    (positive?) side effect, composed names (such as Dupont-Durand)
%    can now be hyphenated on explicit hyphens.
%    Usage: |Jean~\bsc{Duchemin}|.
%
% \changes{v2.0}{2006/11/06}{\cs{hbox} dropped, replaced by
%    \cs{kern0pt}.}
%
%    \begin{macrocode}
\DeclareRobustCommand*{\bsc}[1]{\leavevmode\begingroup\kern0pt
                                           \scshape #1\endgroup}
\ifLaTeXe\else\let\scshape\relax\fi
%    \end{macrocode}
%  \end{macro}
%
%    Some definitions for special characters.  We won't define |\tilde|
%    as a Text Symbol not to conflict with the macro |\tilde| for math
%    mode and use the name |\tild| instead. Note that |\boi| may
%    \emph{not} be used in math mode, its name in math mode is
%    |\backslash|.  |\degre|  can be accessed by the command |\r{}|
%    for ring accent.
%
%    \begin{macrocode}
\ifLaTeXe
  \DeclareTextSymbol{\at}{T1}{64}
  \DeclareTextSymbol{\circonflexe}{T1}{94}
  \DeclareTextSymbol{\tild}{T1}{126}
  \DeclareTextSymbolDefault{\at}{T1}
  \DeclareTextSymbolDefault{\circonflexe}{T1}
  \DeclareTextSymbolDefault{\tild}{T1}
  \DeclareRobustCommand*{\boi}{\textbackslash}
  \DeclareRobustCommand*{\degre}{\r{}}
\else
  \def\T@one{T1}
  \ifx\f@encoding\T@one
    \newcommand*{\degre}{\char6}
  \else
    \newcommand*{\degre}{\char23}
  \fi
  \newcommand*{\at}{\char64}
  \newcommand*{\circonflexe}{\char94}
  \newcommand*{\tild}{\char126}
  \newcommand*{\boi}{$\backslash$}
\fi
%    \end{macrocode}
%
%  \begin{macro}{\degres}
%    We now define a macro |\degres| for typesetting the abbreviation
%    for `degrees' (as in `degrees Celsius'). As the bounding box of
%    the character `degree' has \emph{very} different widths in CM/EC
%    and PostScript fonts, we fix the width of the bounding box of
%    |\degres| to 0.3\,em, this lets the symbol `degree' stick to the
%    preceding (e.g., |45\degres|) or following character
%    (e.g., |20~\degres C|).
%
%    If the \TeX{} Companion fonts are available (\file{textcomp.sty}),
%    we pick up |\textdegree| from them instead of using emulating
%    `degrees' from the |\r{}| accent. Otherwise we overwrite the
%    (poor) definition of |\textdegree| given in \file{latin1.def},
%    \file{applemac.def} etc. (called by  \file{inputenc.sty}) by
%    our definition of |\degres|. We also advice the user (once only)
%    to use TS1-encoding.
%
% \changes{v2.1c}{2008/04/29}{Provide a temporary definition (hyperref
%    safe) of \cs{degres} in case it has to be expanded in the preamble
%    (by beamer's \cs{title} command for instance).}
%
%    \begin{macrocode}
\ifLaTeXe
  \newcommand*{\degres}{\degre}
  \def\Warning@degree@TSone{%
        \PackageWarning{frenchb.ldf}{%
           Degrees would look better in TS1-encoding:
           \MessageBreak add \protect
           \usepackage{textcomp} to the preamble.
           \MessageBreak Degrees used}}
  \AtBeginDocument{\expandafter\ifx\csname M@TS1\endcsname\relax
                     \DeclareRobustCommand*{\degres}{%
                       \leavevmode\hbox to 0.3em{\hss\degre\hss}%
                       \Warning@degree@TSone
                       \global\let\Warning@degree@TSone\relax}%
                      \let\textdegree\degres
                   \else
                     \DeclareRobustCommand*{\degres}{%
                         \hbox{\UseTextSymbol{TS1}{\textdegree}}}%
                   \fi}
\else
  \newcommand*{\degres}{%
    \leavevmode\hbox to 0.3em{\hss\degre\hss}}
\fi
%    \end{macrocode}
%  \end{macro}
%
%  \subsection{Formatting numbers}
%  \label{sec-numbers}
%
%  \begin{macro}{\DecimalMathComma}
%  \begin{macro}{\StandardMathComma}
%    As mentioned in the \TeX{}book p.~134, the comma is of type
%    |\mathpunct| in math mode: it is automatically followed by a
%    space. This is convenient in lists and intervals but
%    unpleasant when the comma is used as a decimal separator
%    in French: it has to be entered as |{,}|.
%    |\DecimalMathComma| makes the comma be an ordinary character
%    (of type |\mathord|) in French \emph{only} (no space added);
%    |\StandardMathComma| switches back to the standard behaviour
%    of the comma.
%    \begin{macrocode}
\newcount\std@mcc
\newcount\dec@mcc
\std@mcc=\mathcode`\,
\dec@mcc=\std@mcc
\@tempcnta=\std@mcc
\divide\@tempcnta by "1000
\multiply\@tempcnta by "1000
\advance\dec@mcc by -\@tempcnta
\newcommand*{\DecimalMathComma}{\iflanguage{french}%
                                 {\mathcode`\,=\dec@mcc}{}%
              \addto\extrasfrench{\mathcode`\,=\dec@mcc}}
\newcommand*{\StandardMathComma}{\mathcode`\,=\std@mcc
             \addto\extrasfrench{\mathcode`\,=\std@mcc}}
\expandafter\addto\csname noextras\CurrentOption\endcsname{%
   \mathcode`\,=\std@mcc}
%    \end{macrocode}
%  \end{macro}
%  \end{macro}
%
%  \begin{macro}{\nombre}
%
% \changes{v2.0}{2006/11/06}{\cs{nombre} requires now numprint.sty.}
%
%    The command |\nombre| is now borrowed from |numprint.sty| for
%    \LaTeXe.  There is no point to maintain the former tricky code
%    when a package is dedicated to do the same job and more.
%    For Plain based formats, |\nombre| no longer formats numbers,
%    it prints them as is and issues a warning about the change.
%
%    Fake command |\nombre| for Plain based formats, warning users of
%    |frenchb| v.1.x. of the change.
%    \begin{macrocode}
\newcommand*{\nombre}[1]{{#1}\message{%
     *** \noexpand\nombre no longer formats numbers\string! ***}}%
%    \end{macrocode}
%  \end{macro}
%
%    The next definitions only make sense for \LaTeXe. Let's cleanup
%    and exit if the format in Plain based.
%
%    \begin{macrocode}
\let\FBstop@here\relax
\def\FBclean@on@exit{\let\ifLaTeXe\@undefined
                     \let\LaTeXetrue\@undefined
                     \let\LaTeXefalse\@undefined}
\ifx\magnification\@undefined
\else
   \def\FBstop@here{\let\STD@makecaption\relax
                    \FBclean@on@exit
                    \ldf@quit\CurrentOption\endinput}
\fi
\FBstop@here
%    \end{macrocode}
%
%    What follows now is for \LaTeXe{} \emph{only}.
%    We redefine |\nombre| for \LaTeXe. A warning is issued
%    at the first call of |\nombre| if |\numprint| is not
%    defined, suggesting what to do.  The package |numprint|
%    is \emph{not} loaded automatically by |frenchb| because of
%    possible options conflict.
%
%    \begin{macrocode}
\renewcommand*{\nombre}[1]{\Warning@nombre\numprint{#1}}
\newcommand*{\Warning@nombre}{%
   \@ifundefined{numprint}%
      {\PackageWarning{frenchb.ldf}{%
         \protect\nombre\space now relies on package numprint.sty,
         \MessageBreak add \protect
         \usepackage[autolanguage]{numprint}\MessageBreak
         to your preamble *after* loading babel, \MessageBreak
         see file numprint.pdf for other options.\MessageBreak
         \protect\nombre\space called}%
       \global\let\Warning@nombre\relax
       \global\let\numprint\relax
      }{}%
}
%    \end{macrocode}
%
% \changes{v2.0c}{2007/06/25}{There is no need to define here
%    numprint's command \cs{npstylefrench}, it will be redefined
%    `AtBeginDocument' by \cs{FBprocess@options}.}
%
% \changes{v2.0c}{2007/06/25}{\cs{ThinSpaceInFrenchNumbers} added
%     for compatibility with frenchb-1.x.}
%
%    \begin{macrocode}
\newcommand*{\ThinSpaceInFrenchNumbers}{%
   \PackageWarning{frenchb.ldf}{%
         Type \protect\frenchbsetup{ThinSpaceInFrenchNumbers}
         \MessageBreak Command \protect\ThinSpaceInFrenchNumbers\space
         is no longer\MessageBreak  defined in frenchb v.2,}}
%    \end{macrocode}
%
%  \subsection{Caption names}
%
%    The next step consists of defining the French equivalents for
%    the \LaTeX{} caption names.
%
% \begin{macro}{\captionsfrench}
%    Let's first define  |\captionsfrench| which sets all strings used
%    in the four standard document classes provided with \LaTeX.
%
% \changes{v2.0}{2006/11/06}{`Fig.' changed to `Figure' and
%     `Tab.' to `Table'.}
%
% \changes{v2.0}{2006/12/15}{Set \cs{CaptionSeparator} in
%     \cs{extrasfrench} now instead of \cs{captionsfrench}
%     because it has to be reset when leaving French.}
%
%    \begin{macrocode}
\@namedef{captions\CurrentOption}{%
   \def\refname{R\'ef\'erences}%
   \def\abstractname{R\'esum\'e}%
   \def\bibname{Bibliographie}%
   \def\prefacename{Pr\'eface}%
   \def\chaptername{Chapitre}%
   \def\appendixname{Annexe}%
   \def\contentsname{Table des mati\`eres}%
   \def\listfigurename{Table des figures}%
   \def\listtablename{Liste des tableaux}%
   \def\indexname{Index}%
   \def\figurename{{\scshape Figure}}%
   \def\tablename{{\scshape Table}}%
%    \end{macrocode}
%   ``Premi\`ere partie'' instead of ``Part I''.
%    \begin{macrocode}
   \def\partname{\protect\@Fpt partie}%
   \def\@Fpt{{\ifcase\value{part}\or Premi\`ere\or Deuxi\`eme\or
   Troisi\`eme\or Quatri\`eme\or Cinqui\`eme\or Sixi\`eme\or
   Septi\`eme\or Huiti\`eme\or Neuvi\`eme\or Dixi\`eme\or Onzi\`eme\or
   Douzi\`eme\or Treizi\`eme\or Quatorzi\`eme\or Quinzi\`eme\or
   Seizi\`eme\or Dix-septi\`eme\or Dix-huiti\`eme\or Dix-neuvi\`eme\or
   Vingti\`eme\fi}\space\def\thepart{}}%
   \def\pagename{page}%
   \def\seename{{\emph{voir}}}%
   \def\alsoname{{\emph{voir aussi}}}%
   \def\enclname{P.~J. }%
   \def\ccname{Copie \`a }%
   \def\headtoname{}%
   \def\proofname{D\'emonstration}%
   \def\glossaryname{Glossaire}%
   }
%    \end{macrocode}
% \end{macro}
%
%    As some users who choose |frenchb| or |francais| as option of
%    \babel, might customise |\captionsfrenchb| or |\captionsfrancais|
%    in the preamble, we merge their changes at the |\begin{document}|
%    when they do so.
%    The other variants of French (canadien, acadian) are defined by
%    checking if the relevant option was used and then adding one extra
%    level of expansion.
%
%    \begin{macrocode}
\AtBeginDocument{\let\captions@French\captionsfrench
                 \@ifundefined{captionsfrenchb}%
                    {\let\captions@Frenchb\relax}%
                    {\let\captions@Frenchb\captionsfrenchb}%
                 \@ifundefined{captionsfrancais}%
                    {\let\captions@Francais\relax}%
                    {\let\captions@Francais\captionsfrancais}%
                 \def\captionsfrench{\captions@French
                        \captions@Francais\captions@Frenchb}%
                 \def\captionsfrancais{\captionsfrench}%
                 \def\captionsfrenchb{\captionsfrench}%
                 \iflanguage{french}{\captionsfrench}{}%
                }
\@ifpackagewith{babel}{canadien}{%
  \def\captionscanadien{\captionsfrench}%
  \def\datecanadien{\datefrench}%
  \def\extrascanadien{\extrasfrench}%
  \def\noextrascanadien{\noextrasfrench}%
  }{}
\@ifpackagewith{babel}{acadian}{%
  \def\captionsacadian{\captionsfrench}%
  \def\dateacadian{\datefrench}%
  \def\extrasacadian{\extrasfrench}%
  \def\noextrasacadian{\noextrasfrench}%
  }{}
%    \end{macrocode}
%
% \begin{macro}{\CaptionSeparator}
%    Let's consider now captions in figures and tables.
%    In French, captions in figures and tables should be printed with
%    endash (`--') instead of the standard `:'.
%
%    The standard definition of |\@makecaption| (e.g., the one provided
%    in article.cls, report.cls, book.cls which is frozen for \LaTeXe{}
%    according to Frank Mittelbach), has been saved in
%    |\STD@makecaption| before making `:' active
%    (see section~\ref{sec-punct}). `AtBeginDocument' we compare it to
%    its current definition (some classes like koma-script classes,
%    AMS classes, ua-thesis.cls\dots change it).
%    If they are identical, |frenchb| just adds a hook called
%    |\CaptionSeparator| to |\@makecaption|, |\CaptionSeparator|
%    defaults to `: ' as in the standard |\@makecaption|, and will be
%    changed to ` -- ' in French.
%    If the definitions differ, |frenchb| doesn't overwrite the changes,
%    but prints a message in the .log file.
%
%    \begin{macrocode}
\def\CaptionSeparator{\string:\space}
\long\def\FB@makecaption#1#2{%
  \vskip\abovecaptionskip
  \sbox\@tempboxa{#1\CaptionSeparator #2}%
  \ifdim \wd\@tempboxa >\hsize
    #1\CaptionSeparator #2\par
  \else
    \global \@minipagefalse
    \hb@xt@\hsize{\hfil\box\@tempboxa\hfil}%
  \fi
  \vskip\belowcaptionskip}
\AtBeginDocument{%
  \ifx\@makecaption\STD@makecaption
      \global\let\@makecaption\FB@makecaption
  \else
    \@ifundefined{@makecaption}{}%
       {\PackageWarning{frenchb.ldf}%
        {The definition of \protect\@makecaption\space
         has been changed,\MessageBreak
         frenchb will NOT customise it;\MessageBreak reported}%
       }%
  \fi
  \let\FB@makecaption\relax
  \let\STD@makecaption\relax
}
\expandafter\addto\csname extras\CurrentOption\endcsname{%
   \def\CaptionSeparator{\space\textendash\space}}
\expandafter\addto\csname noextras\CurrentOption\endcsname{%
    \def\CaptionSeparator{\string:\space}}
%    \end{macrocode}
% \end{macro}
%
%  \subsection{French lists}
%  \label{sec-lists}
%
%  \begin{macro}{\listFB}
%  \begin{macro}{\listORI}
%    Vertical spacing in general lists should be shorter in French
%    texts than the defaults provided by \LaTeX.
%    Note that the easy way, just changing values of vertical spacing
%    parameters when entering French and restoring them to their
%    defaults on exit would not work; as most lists are based on
%    |\list| we will define a variant of |\list| (|\listFB|) to
%    be used in French.
%
%    The amount of vertical space before and after a list is given by
%    |\topsep| + |\parskip| (+ |\partopsep| if the list starts a new
%    paragraph). IMHO, |\parskip| should be added \emph{only} when
%    the list starts a new paragraph, so I subtract |\parskip| from
%    |\topsep| and add it back to |\partopsep|; this will normally
%    make no difference because |\parskip|'s default value is 0pt, but
%    will be noticeable when |\parskip| is \emph{not} null.
%
%    |\endlist| is not redefined, but |\endlistORI| is provided for
%    the users who prefer to define their own lists from the original
%    command, they can code: |\begin{listORI}{}{} \end{listORI}|.
%    \begin{macrocode}
\let\listORI\list
\let\endlistORI\endlist
\def\FB@listsettings{%
      \setlength{\itemsep}{0.4ex plus 0.2ex minus 0.2ex}%
      \setlength{\parsep}{0.4ex plus 0.2ex minus 0.2ex}%
      \setlength{\topsep}{0.8ex plus 0.4ex minus 0.4ex}%
      \setlength{\partopsep}{0.4ex plus 0.2ex minus 0.2ex}%
%    \end{macrocode}
%    |\parskip| is of type `skip', its mean value only (\emph{not
%    the glue}) should be subtracted from |\topsep| and added to
%    |\partopsep|, so convert |\parskip| to a `dimen' using
%    |\@tempdima|.
%    \begin{macrocode}
      \@tempdima=\parskip
      \addtolength{\topsep}{-\@tempdima}%
      \addtolength{\partopsep}{\@tempdima}}%
\def\listFB#1#2{\listORI{#1}{\FB@listsettings #2}}%
\let\endlistFB\endlist
%    \end{macrocode}
%  \end{macro}
%  \end{macro}
%
%  \begin{macro}{\itemizeFB}
%  \begin{macro}{\itemizeORI}
%  \begin{macro}{\bbl@frenchlabelitems}
%  \begin{macro}{\bbl@nonfrenchlabelitems}
%    Let's now consider French itemize lists.  They differ from those
%    provided by the standard \LaTeXe{} classes:
%    \begin{itemize}
%      \item vertical spacing between items, before and after
%         the list, should be \emph{null} with \emph{no glue} added;
%      \item the item labels of a first level list should be vertically
%          aligned on the paragraph's first character (i.e. at
%          |\parindent| from the left margin);
%      \item the `\textbullet' is never used in French itemize-lists,
%          a long dash `--' is preferred for all levels. The item label
%          used in French is stored in |\FrenchLabelItem}|, it defaults
%          to `--' and can be changed using |\frenchbsetup{}| (see
%          section~\ref{sec-keyval}).
%    \end{itemize}
%
%    \begin{macrocode}
\newcommand*{\FrenchLabelItem}{\textendash}
\newcommand*{\Frlabelitemi}{\FrenchLabelItem}
\newcommand*{\Frlabelitemii}{\FrenchLabelItem}
\newcommand*{\Frlabelitemiii}{\FrenchLabelItem}
\newcommand*{\Frlabelitemiv}{\FrenchLabelItem}
%    \end{macrocode}
%    |\bbl@frenchlabelitems| saves current itemize labels and changes
%    them to their value in French. This code should never be executed
%    twice in a row, so we need a new flag that will be set and reset
%    by |\bbl@nonfrenchlabelitems| and |\bbl@frenchlabelitems|.
%    \begin{macrocode}
\newif\ifFB@enterFrench  \FB@enterFrenchtrue
\def\bbl@frenchlabelitems{%
  \ifFB@enterFrench
    \let\@ltiORI\labelitemi
    \let\@ltiiORI\labelitemii
    \let\@ltiiiORI\labelitemiii
    \let\@ltivORI\labelitemiv
    \let\labelitemi\Frlabelitemi
    \let\labelitemii\Frlabelitemii
    \let\labelitemiii\Frlabelitemiii
    \let\labelitemiv\Frlabelitemiv
    \FB@enterFrenchfalse
  \fi
}
\let\itemizeORI\itemize
\let\enditemizeORI\enditemize
\let\enditemizeFB\enditemize
\def\itemizeFB{%
    \ifnum \@itemdepth >\thr@@\@toodeep\else
      \advance\@itemdepth\@ne
      \edef\@itemitem{labelitem\romannumeral\the\@itemdepth}%
      \expandafter
      \listORI
      \csname\@itemitem\endcsname
      {\settowidth{\labelwidth}{\csname\@itemitem\endcsname}%
       \setlength{\leftmargin}{\labelwidth}%
       \addtolength{\leftmargin}{\labelsep}%
       \ifnum\@listdepth=0
         \setlength{\itemindent}{\parindent}%
       \else
         \addtolength{\leftmargin}{\parindent}%
       \fi
       \setlength{\itemsep}{\z@}%
       \setlength{\parsep}{\z@}%
       \setlength{\topsep}{\z@}%
       \setlength{\partopsep}{\z@}%
%    \end{macrocode}
%    |\parskip| is of type `skip', its mean value only (\emph{not
%    the glue}) should be subtracted from |\topsep| and added to
%    |\partopsep|, so convert |\parskip| to a `dimen' using
%    |\@tempdima|.
%    \begin{macrocode}
       \@tempdima=\parskip
       \addtolength{\topsep}{-\@tempdima}%
       \addtolength{\partopsep}{\@tempdima}}%
    \fi}
%    \end{macrocode}
%    The user's changes in labelitems are saved when leaving French for
%    further use when switching back to French.  This code should never
%    be executed twice in a row (toggle with |\bbl@frenchlabelitems|).
%    \begin{macrocode}
\def\bbl@nonfrenchlabelitems{%
  \ifFB@enterFrench
  \else
      \let\Frlabelitemi\labelitemi
      \let\Frlabelitemii\labelitemii
      \let\Frlabelitemiii\labelitemiii
      \let\Frlabelitemiv\labelitemiv
      \let\labelitemi\@ltiORI
      \let\labelitemii\@ltiiORI
      \let\labelitemiii\@ltiiiORI
      \let\labelitemiv\@ltivORI
      \FB@enterFrenchtrue
  \fi
}
%    \end{macrocode}
%  \end{macro}
%  \end{macro}
%  \end{macro}
%  \end{macro}
%
%  \subsection{French indentation of sections}
%  \label{sec-indent}
%
%  \begin{macro}{\bbl@frenchindent}
%  \begin{macro}{\bbl@nonfrenchindent}
%    In French the first paragraph of each section should be indented,
%    this is another difference with US-English. This is controlled by
%    the flag |\if@afterindent|.
%
% \changes{v2.3d}{2009/03/16}{Bug correction: previous versions of
%    frenchb set the flag \cs{if@afterindent} to false outside
%    French which is correct for English but wrong for some languages
%    like Spanish.  Pointed out by Juan Jos\'e Torrens.}
%
%    We need to save the value of the flag |\if@afterindent|
%    `AtBeginDocument' before eventually changing its value.
%
%    \begin{macrocode}
\AtBeginDocument{\ifx\@afterindentfalse\@afterindenttrue
                       \let\@aifORI\@afterindenttrue
                 \else \let\@aifORI\@afterindentfalse
                 \fi
}
\def\bbl@frenchindent{\let\@afterindentfalse\@afterindenttrue
                      \@afterindenttrue}
\def\bbl@nonfrenchindent{\let\@afterindentfalse\@aifORI
                         \@afterindentfalse}
%    \end{macrocode}
%  \end{macro}
%  \end{macro}
%
%  \subsection{Formatting footnotes}
%  \label{sec-footnotes}
%
% \changes{v2.0}{2006/11/06}{Footnotes are now printed
%     by default `\`a la fran\c caise' for the whole document.}
%
% \changes{v2.0b}{2007/04/18}{Footnotes: Just do nothing
%    (except warning) when the bigfoot package is loaded.}
%
%    The |bigfoot| package deeply changes the way footnotes are
%    handled. When |bigfoot| is loaded, we just warn the user that
%    |frenchb| will drop the customisation of footnotes.
%
%    The layout of footnotes is controlled by two flags
%    |\ifFBAutoSpaceFootnotes| and |\ifFBFrenchFootnotes| which are
%    set by options of |\frenchbsetup{}| (see section~\ref{sec-keyval}).
%    Notice that the layout of footnotes \emph{does not depend} on the
%    current language (just think of two footnotes on the same page
%    looking different because one was called in a French part, the
%    other one in English!).
%
%    When |\ifFBAutoSpaceFootnotes| is true, |\@footnotemark| (whose
%    definition is saved at the |\begin{document}| in order to include
%    any customisation that packages might have done) is redefined to
%    add a thin space before the number or symbol calling a footnote
%    (any space typed in is removed first).  This has no effect on
%    the layout of the footnote itself.
%
%    \begin{macrocode}
\AtBeginDocument{\@ifpackageloaded{bigfoot}%
                   {\PackageWarning{frenchb.ldf}%
                     {bigfoot package in use.\MessageBreak
                      frenchb will NOT customise footnotes;\MessageBreak
                      reported}}%
                   {\let\@footnotemarkORI\@footnotemark
                    \def\@footnotemarkFB{\leavevmode\unskip\unkern
                                         \,\@footnotemarkORI}%
                    \ifFBAutoSpaceFootnotes
                      \let\@footnotemark\@footnotemarkFB
                    \fi}%
                }
%    \end{macrocode}
%
%    We then define |\@makefntextFB|, a variant of |\@makefntext|
%    which is responsible for the layout of footnotes, to match the
%    specifications of the French `Imprimerie Nationale':  footnotes
%    will be indented by |\parindentFFN|, numbers (if any) typeset on
%    the baseline (instead of superscripts) and followed by a dot
%    and an half quad space. Whenever symbols are used to number
%    footnotes (as in |\thanks| for instance), we switch back to the
%    standard layout (the French layout of footnotes is meant for
%    footnotes numbered by Arabic or Roman digits).
%
% \changes{v2.0}{2006/11/06}{\cs{parindentFFN} not changed if
%    already defined (required by JA for cah-gut.cls).}
%
% \changes{v2.3b}{2008/12/06}{New commands \cs{dotFFN} and
%    \cs{kernFFN} for more flexibility (suggested by JA).}
%
%    The value of |\parindentFFN| will be redefined at the
%    |\begin{document}|, as the maximum of |\parindent| and 1.5em
%    \emph{unless} it has been set in the preamble (the weird value
%    10in is just for testing whether |\parindentFFN| has been set
%    or not).
%
%    \begin{macrocode}
\newcommand*{\dotFFN}{.}
\newcommand*{\kernFFN}{\kern .5em}
\newdimen\parindentFFN
\parindentFFN=10in
\def\ftnISsymbol{\@fnsymbol\c@footnote}
\long\def\@makefntextFB#1{\ifx\thefootnote\ftnISsymbol
                            \@makefntextORI{#1}%
                          \else
                            \parindent=\parindentFFN
                            \rule\z@\footnotesep
                            \setbox\@tempboxa\hbox{\@thefnmark}%
                            \ifdim\wd\@tempboxa>\z@
                              \llap{\@thefnmark}\dotFFN\kernFFN
                            \fi #1
                          \fi}%
%    \end{macrocode}
%
%    We save the standard definition of |\@makefntext| at the
%    |\begin{document}|, and then redefine |\@makefntext| according to
%    the value of flag |\ifFBFrenchFootnotes| (true or false).
%
%    \begin{macrocode}
\AtBeginDocument{\@ifpackageloaded{bigfoot}{}%
                  {\ifdim\parindentFFN<10in
                   \else
                      \parindentFFN=\parindent
                      \ifdim\parindentFFN<1.5em\parindentFFN=1.5em\fi
                   \fi
                   \let\@makefntextORI\@makefntext
                   \long\def\@makefntext#1{%
                      \ifFBFrenchFootnotes
                         \@makefntextFB{#1}%
                      \else
                         \@makefntextORI{#1}%
                      \fi}%
                  }%
                }
%    \end{macrocode}
%
%    For compatibility reasons, we provide definitions for the commands
%    dealing with the layout of footnotes in |frenchb| version~1.6.
%    |\frenchbsetup{}| (see in section \ref{sec-keyval}) should be
%    preferred for setting these options.  |\StandardFootnotes| may
%    still be used locally (in minipages for instance), that's why the
%    test |\ifFBFrenchFootnotes| is done inside |\@makefntext|.
%    \begin{macrocode}
\newcommand*{\AddThinSpaceBeforeFootnotes}{\FBAutoSpaceFootnotestrue}
\newcommand*{\FrenchFootnotes}{\FBFrenchFootnotestrue}
\newcommand*{\StandardFootnotes}{\FBFrenchFootnotesfalse}
%    \end{macrocode}
%
%  \subsection{Global layout}
%  \label{sec-global}
%
%    In multilingual documents, some typographic rules must depend
%    on the current language (e.g., hyphenation, typesetting of
%    numbers, spacing before double punctuation\dots), others should,
%    IMHO, be kept global to the document: especially the layout of
%    lists (see~\ref{sec-lists}) and footnotes
%    (see~\ref{sec-footnotes}), and the indentation of the first
%    paragraph of sections (see~\ref{sec-indent}).
%
%    From version 2.2 on, if |frenchb| is \babel's ``main language''
%    (i.e. last language option at \babel's loading), |frenchb|
%    customises the layout (i.e. lists, indentation of the first
%    paragraphs of sections and footnotes) in the whole document
%    regardless the current language.   On the other hand, if |frenchb|
%    is \emph{not} \babel's ``main language'', it leaves the layout
%    unchanged both in French and in other languages.
%
%  \begin{macro}{\FrenchLayout}
%  \begin{macro}{\StandardLayout}
%    The former commands |\FrenchLayout| and |\StandardLayout| are kept
%    for compatibility reasons but should no longer be used.
%
% \changes{v2.0g}{2008/03/23}{Flag \cs{ifFBStandardLayout} not checked
%     by \cs{FBprocess@options}, low-level flags have to be set
%     one by one.}
%
%    \begin{macrocode}
\newcommand*{\FrenchLayout}{%
    \FBGlobalLayoutFrenchtrue
    \PackageWarning{frenchb.ldf}%
    {\protect\FrenchLayout\space is obsolete.  Please use\MessageBreak
     \protect\frenchbsetup{GlobalLayoutFrench} instead.}%
}
\newcommand*{\StandardLayout}{%
  \FBReduceListSpacingfalse
  \FBCompactItemizefalse
  \FBStandardItemLabelstrue
  \FBIndentFirstfalse
  \FBFrenchFootnotesfalse
  \FBAutoSpaceFootnotesfalse
  \PackageWarning{frenchb.ldf}%
    {\protect\StandardLayout\space is obsolete.  Please use\MessageBreak
    \protect\frenchbsetup{StandardLayout} instead.}%
}
\@onlypreamble\FrenchLayout
\@onlypreamble\StandardLayout
%    \end{macrocode}
%  \end{macro}
%  \end{macro}
%
%  \subsection{Dots\dots}
%  \label{sec-dots}
%
%  \begin{macro}{\FBtextellipsis}
%    \LaTeXe's standard definition of |\dots| in text-mode is
%    |\textellipsis| which includes a |\kern| at the end;
%    this space is not wanted in some cases (before a closing brace
%    for instance) and |\kern| breaks hyphenation of the next word.
%    We define |\FBtextellipsis| for French (in \LaTeXe{} only).
%
%    The |\if| construction in the \LaTeXe{} definition of |\dots|
%    doesn't allow the use of |xspace| (|xspace| is always followed
%    by a |\fi|), so we use the AMS-\LaTeX{} construction of |\dots|;
%    this has to be done `AtBeginDocument' not to be overwritten
%    when \file{amsmath.sty} is loaded after \babel.
%
% \changes{v2.0}{2006/11/06}{Added special case for LY1 encoding,
%    see  bug report from Bruno Voisin (2004/05/18).}
%
%    LY1 has a ready made character for |\textellipsis|, it should be
%    used in French too (pointed out by Bruno Voisin).
%
%    \begin{macrocode}
\DeclareTextSymbol{\FBtextellipsis}{LY1}{133}
\DeclareTextCommandDefault{\FBtextellipsis}{%
    .\kern\fontdimen3\font.\kern\fontdimen3\font.\xspace}
%    \end{macrocode}
%    |\Mdots@| and |\Tdots@ORI| hold the definitions of |\dots| in
%    Math and Text mode. They default to those of amsmath-2.0, and
%    will revert to standard \LaTeX{} definitions `AtBeginDocument',
%    if amsmath has not been loaded. |\Mdots@| doesn't change when
%    switching from/to French, while |\Tdots@| is |\FBtextellipsis|
%    in French and |\Tdots@ORI| otherwise.
%    \begin{macrocode}
\newcommand*{\Tdots@ORI}{\@xp\textellipsis}
\newcommand*{\Tdots@}{\Tdots@ORI}
\newcommand*{\Mdots@}{\@xp\mdots@}
\AtBeginDocument{\DeclareRobustCommand*{\dots}{\relax
                 \csname\ifmmode M\else T\fi dots@\endcsname}%
                 \@ifundefined{@xp}{\let\@xp\relax}{}%
                 \@ifundefined{mdots@}{\let\Tdots@ORI\textellipsis
                                       \let\Mdots@\mathellipsis}{}}
\def\bbl@frenchdots{\let\Tdots@\FBtextellipsis}
\def\bbl@nonfrenchdots{\let\Tdots@\Tdots@ORI}
\expandafter\addto\csname extras\CurrentOption\endcsname{%
    \bbl@frenchdots}
\expandafter\addto\csname noextras\CurrentOption\endcsname{%
    \bbl@nonfrenchdots}
%    \end{macrocode}
%  \end{macro}
%
%  \subsection{Setup options: keyval stuff}
%  \label{sec-keyval}
%
% \changes{v2.0}{2006/11/06}{New command \cs{frenchbsetup} added
%     for global customisation.}
%
% \changes{v2.0c}{2007/06/25}{Option ThinSpaceInFrenchNumbers added.}
%
% \changes{v2.0d}{2007/07/15}{Options og and fg changed: limit
%     the definition to French so that quote characters can be used
%     in German.}
%
% \changes{v2.0e}{2007/10/05}{New option: StandardLists.}
%
% \changes{v2.0f}{2008/03/23}{Two typos corrected in
%    option StandardLists: [false] $\to$ [true] and
%    StandardLayout $\to$ StandardLists.}
%
% \changes{v2.0f}{2008/03/23}{StandardLayout option had no
%     effect on lists.  Test moved to \cs{FBprocess@options}.}
%
% \changes{v2.0g}{2008/03/23}{Revert previous change to
%     StandardLayout. This option must set the three flags
%     \cs{FBReduceListSpacingfalse}, \cs{FBCompactItemizefalse},
%     and \cs{FBStandardItemLabeltrue} instead of
%     \cs{FBStandardListstrue}, so that later options can still
%     change their value before executing \cs{FBprocess@options}.
%     Same thing for option StandardLists.}
%
% \changes{v2.1a}{2008/03/24}{New option: FrenchSuperscripts
%     to define \cs{up} as \cs{fup} or as \cs{textsuperscript}.}
%
% \changes{v2.1a}{2008/03/30}{New option: LowercaseSuperscripts.}
%
% \changes{v2.2a}{2008/05/08}{The global layout of the document is
%     no longer changed when frenchb is not the last option of babel
%     (\cs{bbl@main@language}). Suggested by Ulrike Fischer.}
%
% \changes{v2.2a}{2008/05/08}{Values of flags
%     \cs{ifFBReduceListSpacing}, \cs{ifFBCompactItemize},
%     \cs{ifFBStandardItemLabels}, \cs{ifFBIndentFirst},
%     \cs{ifFBFrenchFootnotes}, \cs{ifFBAutoSpaceFootnotes} changed:
%     default now means `StandardLayout', it will be changed to
%     `FrenchLayout' AtEndOfPackage only if french is
%     \cs{bbl@main@language}.}
%
% \changes{v2.2a}{2008/05/08}{When frenchb is babel's last option,
%     French becomes the document's main language, so
%     GlobalLayoutFrench applies.}
%
% \changes{v2.3a}{2008/10/10}{New option: OriginalTypewriter. Now
%    frenchb switches to \cs{noautospace@beforeFDP} when a tt-font is
%    in use.  When OriginalTypewriter is set to true, frenchb behaves
%    as in pre-2.3 versions.}
%
%    We first define a collection of conditionals with their defaults
%    (true or false).
%
%    \begin{macrocode}
\newif\ifFBStandardLayout           \FBStandardLayouttrue
\newif\ifFBGlobalLayoutFrench       \FBGlobalLayoutFrenchfalse
\newif\ifFBReduceListSpacing        \FBReduceListSpacingfalse
\newif\ifFBCompactItemize           \FBCompactItemizefalse
\newif\ifFBStandardItemLabels       \FBStandardItemLabelstrue
\newif\ifFBStandardLists            \FBStandardListstrue
\newif\ifFBIndentFirst              \FBIndentFirstfalse
\newif\ifFBFrenchFootnotes          \FBFrenchFootnotesfalse
\newif\ifFBAutoSpaceFootnotes       \FBAutoSpaceFootnotesfalse
\newif\ifFBOriginalTypewriter       \FBOriginalTypewriterfalse
\newif\ifFBThinColonSpace           \FBThinColonSpacefalse
\newif\ifFBThinSpaceInFrenchNumbers \FBThinSpaceInFrenchNumbersfalse
\newif\ifFBFrenchSuperscripts       \FBFrenchSuperscriptstrue
\newif\ifFBLowercaseSuperscripts    \FBLowercaseSuperscriptstrue
\newif\ifFBPartNameFull             \FBPartNameFulltrue
\newif\ifFBShowOptions              \FBShowOptionsfalse
%    \end{macrocode}
%
%    The defaults values of these flags have been set so that |frenchb|
%    does not change anything regarding the global layout.
%    |\bbl@main@language| (set by the last option of babel) controls
%    the global layout of the document.  We check the current language
%    `AtEndOfPackage' (it is |\bbl@main@language|); if it is French,
%    the values of some flags have to be changed to ensure a French
%    looking layout for the whole document (even in parts written in
%    languages other than French); the end-user will then be able to
%    customise the values of all these flags with |\frenchbsetup{}|.
%    \begin{macrocode}
\AtEndOfPackage{%
   \iflanguage{french}{\FBReduceListSpacingtrue
                       \FBCompactItemizetrue
                       \FBStandardItemLabelsfalse
                       \FBIndentFirsttrue
                       \FBFrenchFootnotestrue
                       \FBAutoSpaceFootnotestrue
                       \FBGlobalLayoutFrenchtrue}%
                      {}%
}
%    \end{macrocode}
%
%  \begin{macro}{\frenchbsetup}
%    From version 2.0 on, all setup options are handled by \emph{one}
%    command |\frenchbsetup| using the keyval syntax.
%    Let's now define this command which reads and sets the options
%    to be processed later (at |\begin{document}|) by
%    |\FBprocess@options|. It  can only be called in the preamble.
%    \begin{macrocode}
\newcommand*{\frenchbsetup}[1]{%
  \setkeys{FB}{#1}%
}%
\@onlypreamble\frenchbsetup
%    \end{macrocode}
%    |frenchb| being an option of babel, it cannot load a package
%    (keyval) while |frenchb.ldf| is read, so we defer the loading of
%    \file{keyval} and the options setup at the end of \babel's loading.
%
%    |StandardLayout| resets the layout in French to the standard layout
%    defined par the \LaTeX{} class and packages loaded. It deals with
%    lists, indentation of first paragraphs of sections and footnotes.
%    Other keys, entered \emph{after} |StandardLayout| in
%    |\frenchbsetup|, can overrule some of the |StandardLayout|
%     settings.
%
%    |GlobalLayoutFrench| forces the layout in French and (as far as
%    possible) outside French to meet the French typographic standards.
%
% \changes{v2.3d}{2009/03/16}{Warning added to \cs{GlobalLayoutFrench}
%    when French is not the main language.}
%
%    \begin{macrocode}
\AtEndOfPackage{%
    \RequirePackage{keyval}%
    \define@key{FB}{StandardLayout}[true]%
                      {\csname FBStandardLayout#1\endcsname
                       \ifFBStandardLayout
                         \FBReduceListSpacingfalse
                         \FBCompactItemizefalse
                         \FBStandardItemLabelstrue
                         \FBIndentFirstfalse
                         \FBFrenchFootnotesfalse
                         \FBAutoSpaceFootnotesfalse
                         \FBGlobalLayoutFrenchfalse
                       \else
                         \FBReduceListSpacingtrue
                         \FBCompactItemizetrue
                         \FBStandardItemLabelsfalse
                         \FBIndentFirsttrue
                         \FBFrenchFootnotestrue
                         \FBAutoSpaceFootnotestrue
                       \fi}%
    \define@key{FB}{GlobalLayoutFrench}[true]%
                      {\csname FBGlobalLayoutFrench#1\endcsname
                       \ifFBGlobalLayoutFrench
                          \iflanguage{french}%
                            {\FBReduceListSpacingtrue
                             \FBCompactItemizetrue
                             \FBStandardItemLabelsfalse
                             \FBIndentFirsttrue
                             \FBFrenchFootnotestrue
                             \FBAutoSpaceFootnotestrue}%
                            {\PackageWarning{frenchb.ldf}%
                              {Option `GlobalLayoutFrench' skipped:
                               \MessageBreak French is *not*
                               babel's last option.\MessageBreak}}%
                       \fi}%
    \define@key{FB}{ReduceListSpacing}[true]%
                      {\csname FBReduceListSpacing#1\endcsname}%
    \define@key{FB}{CompactItemize}[true]%
                      {\csname FBCompactItemize#1\endcsname}%
    \define@key{FB}{StandardItemLabels}[true]%
                      {\csname FBStandardItemLabels#1\endcsname}%
    \define@key{FB}{ItemLabels}{%
        \renewcommand*{\FrenchLabelItem}{#1}}%
    \define@key{FB}{ItemLabeli}{%
        \renewcommand*{\Frlabelitemi}{#1}}%
    \define@key{FB}{ItemLabelii}{%
        \renewcommand*{\Frlabelitemii}{#1}}%
    \define@key{FB}{ItemLabeliii}{%
        \renewcommand*{\Frlabelitemiii}{#1}}%
    \define@key{FB}{ItemLabeliv}{%
        \renewcommand*{\Frlabelitemiv}{#1}}%
    \define@key{FB}{StandardLists}[true]%
                      {\csname FBStandardLists#1\endcsname
                       \ifFBStandardLists
                         \FBReduceListSpacingfalse
                         \FBCompactItemizefalse
                         \FBStandardItemLabelstrue
                       \else
                         \FBReduceListSpacingtrue
                         \FBCompactItemizetrue
                         \FBStandardItemLabelsfalse
                       \fi}%
    \define@key{FB}{IndentFirst}[true]%
                      {\csname FBIndentFirst#1\endcsname}%
    \define@key{FB}{FrenchFootnotes}[true]%
                      {\csname FBFrenchFootnotes#1\endcsname}%
    \define@key{FB}{AutoSpaceFootnotes}[true]%
                      {\csname FBAutoSpaceFootnotes#1\endcsname}%
    \define@key{FB}{AutoSpacePunctuation}[true]%
                      {\csname FBAutoSpacePunctuation#1\endcsname}%
    \define@key{FB}{OriginalTypewriter}[true]%
                      {\csname FBOriginalTypewriter#1\endcsname}%
    \define@key{FB}{ThinColonSpace}[true]%
                      {\csname FBThinColonSpace#1\endcsname}%
    \define@key{FB}{ThinSpaceInFrenchNumbers}[true]%
                      {\csname FBThinSpaceInFrenchNumbers#1\endcsname}%
    \define@key{FB}{FrenchSuperscripts}[true]%
                      {\csname FBFrenchSuperscripts#1\endcsname}
    \define@key{FB}{LowercaseSuperscripts}[true]%
                      {\csname FBLowercaseSuperscripts#1\endcsname}
    \define@key{FB}{PartNameFull}[true]%
                      {\csname FBPartNameFull#1\endcsname}%
    \define@key{FB}{ShowOptions}[true]%
                      {\csname FBShowOptions#1\endcsname}%
%    \end{macrocode}
%    Inputing French quotes as \emph{single characters} when they are
%    available on the keyboard (through a compose key for instance)
%    is more comfortable than typing |\og| and |\fg|.
%    The purpose of the following code is to map the French quote
%    characters to |\og\ignorespaces| and |{\fg}| respectively when
%    the current language is French, and to |\guillemotleft| and
%    |\guillemotright| otherwise (think of German quotes); thus correct
%    unbreakable spaces will be added automatically to French quotes.
%    The quote characters typed in depend on the input encoding,
%    it can be single-byte (latin1, latin9, applemac,\dots) or
%    multi-bytes (utf-8, utf8x).  We first check whether XeTeX is used
%    or not, if not the package |inputenc| has to be loaded before the
%    |\begin{document}| with the proper coding option, so we check if
%    |\DeclareInputText| is defined.
%    \begin{macrocode}
    \define@key{FB}{og}{%
       \newcommand*{\FB@@og}{\iflanguage{french}%
                               {\FB@og\ignorespaces}{\guillemotleft}}%
       \expandafter\ifx\csname XeTeXrevision\endcsname\relax
         \AtBeginDocument{%
           \@ifundefined{DeclareInputText}%
             {\PackageWarning{frenchb.ldf}%
               {Option `og' requires package inputenc.\MessageBreak}%
             }%
             {\@ifundefined{uc@dclc}%
%    \end{macrocode}
%    if |\uc@dclc| is undefined, utf8x is not loaded\dots
%    \begin{macrocode}
               {\@ifundefined{DeclareUnicodeCharacter}%
%    \end{macrocode}
%    if |\DeclareUnicodeCharacter| is undefined, utf8 is not loaded
%    either, we assume 8-bit character input encoding.
%    Package MULEenc.sty (from CJK) defines |\mule@def| to map
%    characters to control sequences.
%    \begin{macrocode}
                  {\@tempcnta`#1\relax
                     \@ifundefined{mule@def}%
                       {\DeclareInputText{\the\@tempcnta}{\FB@@og}}%
                       {\mule@def{11}{{\FB@@og}}}%
                  }%
%    \end{macrocode}
%    utf8 loaded, use |\DeclareUnicodeCharacter|,
%    \begin{macrocode}
                  {\DeclareUnicodeCharacter{00AB}{\FB@@og}}%
               }%
%    \end{macrocode}
%    utf8x loaded, use |\uc@dclc|,
%    \begin{macrocode}
               {\uc@dclc{171}{default}{\FB@@og}}%
             }%
         }%
%    \end{macrocode}
%    XeTeX in use, the following trick for defining the active quote
%    character is borrowed from \file{inputenc.dtx}.
%    \begin{macrocode}
       \else
         \catcode`#1=\active
         \bgroup
           \uccode`\~`#1%
           \uppercase{%
         \egroup
         \def~%
         }{\FB@@og}%
       \fi
    }%
%    \end{macrocode}
%    Same code for the closing quote.
%    \begin{macrocode}
    \define@key{FB}{fg}{%
       \newcommand*{\FB@@fg}{\iflanguage{french}%
                               {\FB@fg}{\guillemotright}}%
       \expandafter\ifx\csname XeTeXrevision\endcsname\relax
         \AtBeginDocument{%
           \@ifundefined{DeclareInputText}%
             {\PackageWarning{frenchb.ldf}%
               {Option `fg' requires package inputenc.\MessageBreak}%
             }%
             {\@ifundefined{uc@dclc}%
               {\@ifundefined{DeclareUnicodeCharacter}%
                  {\@tempcnta`#1\relax
                     \@ifundefined{mule@def}%
                       {\DeclareInputText{\the\@tempcnta}{{\FB@@fg}}}%
                       {\mule@def{27}{{\FB@@fg}}}%
                  }%
                  {\DeclareUnicodeCharacter{00BB}{{\FB@@fg}}}%
               }%
               {\uc@dclc{187}{default}{{\FB@@fg}}}%
             }%
         }%
       \else
         \catcode`#1=\active
         \bgroup
           \uccode`\~`#1%
           \uppercase{%
         \egroup
         \def~%
         }{{\FB@@fg}}%
       \fi
    }%
}
%    \end{macrocode}
%  \end{macro}
%
% \begin{macro}{\FBprocess@options}
%    |\FBprocess@options| processes the options, it is called \emph{once}
%    at |\begin{document}|.
%    \begin{macrocode}
\newcommand*{\FBprocess@options}{%
%    \end{macrocode}
%    Nothing has to be done here for |StandardLayout| and
%    |StandardLists| (the involved flags have already been set in
%    |\frenchbsetup{}| or before (at babel's EndOfPackage).
%
%    The next three options deal with the layout of lists in French.
%
%    |ReduceListSpacing| reduces the vertical spaces between list
%    items in French (done by changing |\list| to |\listFB|).
%    When |GlobalLayoutFrench| is true (the default), the same is
%    done outside French except for languages that force a different
%    setting.
%    \begin{macrocode}
  \ifFBReduceListSpacing
    \addto\extrasfrench{\let\list\listFB
                        \let\endlist\endlistFB}%
    \addto\noextrasfrench{\ifFBGlobalLayoutFrench
                            \let\list\listFB
                            \let\endlist\endlistFB
                          \else
                            \let\list\listORI
                            \let\endlist\endlistORI
                          \fi}%
  \else
    \addto\extrasfrench{\let\list\listORI
                        \let\endlist\endlistORI}%
    \addto\noextrasfrench{\let\list\listORI
                          \let\endlist\endlistORI}%
  \fi
%    \end{macrocode}
%
%    |CompactItemize| suppresses the vertical spacing between list
%    items in French (done by changing |\itemize| to |\itemizeFB|).
%    When |GlobalLayoutFrench| is true the same is done outside French.
%    \begin{macrocode}
  \ifFBCompactItemize
    \addto\extrasfrench{\let\itemize\itemizeFB
                        \let\enditemize\enditemizeFB}%
    \addto\noextrasfrench{\ifFBGlobalLayoutFrench
                             \let\itemize\itemizeFB
                             \let\enditemize\enditemizeFB
                          \else
                             \let\itemize\itemizeORI
                             \let\enditemize\enditemizeORI
                          \fi}%
  \else
    \addto\extrasfrench{\let\itemize\itemizeORI
                        \let\enditemize\enditemizeORI}%
    \addto\noextrasfrench{\let\itemize\itemizeORI
                          \let\enditemize\enditemizeORI}%
  \fi
%    \end{macrocode}
%
%    |StandardItemLabels| resets labelitems in French to their
%    standard values set by the \LaTeX{} class and packages loaded.
%    When |GlobalLayoutFrench| is true labelitems are identical inside
%    and outside French.
%    \begin{macrocode}
  \ifFBStandardItemLabels
    \addto\extrasfrench{\bbl@nonfrenchlabelitems}%
    \addto\noextrasfrench{\bbl@nonfrenchlabelitems}%
  \else
    \addto\extrasfrench{\bbl@frenchlabelitems}%
    \addto\noextrasfrench{\ifFBGlobalLayoutFrench
                            \bbl@frenchlabelitems
                          \else
                            \bbl@nonfrenchlabelitems
                          \fi}%
  \fi
%    \end{macrocode}
%
%    |IndentFirst| forces the first paragraphs of sections to be
%    indented just like the other ones in French.
%    When |GlobalLayoutFrench| is true (the default), the same is
%    done outside French except for languages that force a different
%    setting.
%    \begin{macrocode}
  \ifFBIndentFirst
    \addto\extrasfrench{\bbl@frenchindent}%
    \addto\noextrasfrench{\ifFBGlobalLayoutFrench
                             \bbl@frenchindent
                          \else
                             \bbl@nonfrenchindent
                          \fi}%
  \else
    \addto\extrasfrench{\bbl@nonfrenchindent}%
    \addto\noextrasfrench{\bbl@nonfrenchindent}%
  \fi
%    \end{macrocode}
%
%    The layout of footnotes is handled at the |\begin{document}|
%    depending on the values of flags |FrenchFootnotes|
%    and |AutoSpaceFootnotes| (see section~\ref{sec-footnotes}),
%    nothing has to be done here for footnotes.
%
%    |AutoSpacePunctuation| adds an unbreakable space (in French only)
%    before the four active characters (:;!?) even if none has been
%    typed before them.
%    \begin{macrocode}
  \ifFBAutoSpacePunctuation
     \autospace@beforeFDP
  \else
     \noautospace@beforeFDP
  \fi
%    \end{macrocode}
%
%    When |OriginalTypewriter| is set to |false| (the default),
%    |\ttfamily|, |\rmfamily| and |\sffamily| are redefined as
%    |\ttfamilyFB|, |\rmfamilyFB| and |\sffamilyFB| respectively
%    to prevent addition of automatic spaces before the four active
%    characters in computer code.
%    \begin{macrocode}
  \ifFBOriginalTypewriter
  \else
     \let\ttfamily\ttfamilyFB
     \let\rmfamily\rmfamilyFB
     \let\sffamily\sffamilyFB
  \fi
%    \end{macrocode}
%
%    |ThinColonSpace| changes the normal unbreakable space typeset in
%     French before `:' to a thin space.
%    \begin{macrocode}
  \ifFBThinColonSpace\renewcommand*{\Fcolonspace}{\thinspace}\fi
%    \end{macrocode}
%
%    When |true|, |ThinSpaceInFrenchNumbers| redefines |numprint.sty|'s
%    command |\npstylefrench| to set |\npthousandsep| to |\,|
%    (thinspace) instead of |~| (default) . This option has no effect
%    if package |numprint.sty| is not loaded with `|autolanguage|'.
%    As old versions of |numprint.sty| did not define |\npstylefrench|,
%    we have to provide this command.
%    \begin{macrocode}
  \@ifpackageloaded{numprint}%
  {\ifnprt@autolanguage
     \providecommand*{\npstylefrench}{}%
     \ifFBThinSpaceInFrenchNumbers
       \renewcommand*\npstylefrench{%
          \npthousandsep{\,}%
          \npdecimalsign{,}%
          \npproductsign{\cdot}%
          \npunitseparator{\,}%
          \npdegreeseparator{}%
          \nppercentseparator{\nprt@unitsep}%
          }%
     \else
       \renewcommand*\npstylefrench{%
          \npthousandsep{~}%
          \npdecimalsign{,}%
          \npproductsign{\cdot}%
          \npunitseparator{\,}%
          \npdegreeseparator{}%
          \nppercentseparator{\nprt@unitsep}%
          }%
     \fi
     \npaddtolanguage{french}{french}%
   \fi}{}%
%    \end{macrocode}
%
%    |FrenchSuperscripts|: if |true| |\up=\fup|, else
%    |\up=\textsuperscript|. Anyway |\up*=\FB@up@fake|. The star-form
%    |\up*{}| is provided for fonts that lack some superior letters:
%    Adobe Jenson Pro and Utopia Expert have no ``g superior'' for
%    instance.
%    \begin{macrocode}
  \ifFBFrenchSuperscripts
    \DeclareRobustCommand*{\up}{\@ifstar{\FB@up@fake}{\fup}}%
  \else
    \DeclareRobustCommand*{\up}{\@ifstar{\FB@up@fake}%
                                        {\textsuperscript}}%
  \fi
%    \end{macrocode}
%
%    |LowercaseSuperscripts|: if |true| let |\FB@lc| be |\lowercase|,
%     else |\FB@lc| is redefined to do nothing.
%    \begin{macrocode}
  \ifFBLowercaseSuperscripts
  \else
    \renewcommand*{\FB@lc}[1]{##1}%
  \fi
%    \end{macrocode}
%
%    |PartNameFull|: if |false|, redefine |\partname|.
%    \begin{macrocode}
  \ifFBPartNameFull
  \else\addto\captionsfrench{\def\partname{Partie}}\fi
%    \end{macrocode}
%
%    |ShowOptions|: if |true|, print the list of all options to the
%    \file{.log} file.
%    \begin{macrocode}
  \ifFBShowOptions
    \GenericWarning{* }{%
     * **** List of possible options for frenchb ****\MessageBreak
     [Default values between brackets when frenchb is loaded *LAST*]%
     \MessageBreak
     ShowOptions=true [false]\MessageBreak
     StandardLayout=true [false]\MessageBreak
     GlobalLayoutFrench=false [true]\MessageBreak
     StandardLists=true [false]\MessageBreak
     ReduceListSpacing=false [true]\MessageBreak
     CompactItemize=false [true]\MessageBreak
     StandardItemLabels=true [false]\MessageBreak
     ItemLabels=\textemdash, \textbullet,
        \protect\ding{43},... [\textendash]\MessageBreak
     ItemLabeli=\textemdash, \textbullet,
        \protect\ding{43},... [\textendash]\MessageBreak
     ItemLabelii=\textemdash, \textbullet,
        \protect\ding{43},... [\textendash]\MessageBreak
     ItemLabeliii=\textemdash, \textbullet,
        \protect\ding{43},... [\textendash]\MessageBreak
     ItemLabeliv=\textemdash, \textbullet,
        \protect\ding{43},... [\textendash]\MessageBreak
     IndentFirst=false [true]\MessageBreak
     FrenchFootnotes=false [true]\MessageBreak
     AutoSpaceFootnotes=false [true]\MessageBreak
     AutoSpacePunctuation=false [true]\MessageBreak
     OriginalTypewriter=true [false]\MessageBreak
     ThinColonSpace=true [false]\MessageBreak
     ThinSpaceInFrenchNumbers=true [false]\MessageBreak
     FrenchSuperscripts=false [true]\MessageBreak
     LowercaseSuperscripts=false [true]\MessageBreak
     PartNameFull=false [true]\MessageBreak
     og= <left quote character>, fg= <right quote character>
     \MessageBreak
     *********************************************
     \MessageBreak\protect\frenchbsetup{ShowOptions}}
  \fi
}
%    \end{macrocode}
%  \end{macro}
%
% \changes{v2.0}{2006/12/15}{AtBeginDocument, save again the
%    definitions of the `list' and `itemize' environments and the
%    values of labelitems.  As of frenchb v.1.6, `ORI' values were
%    set when reading frenchb.ldf, later changes were ignored.}
%
% \changes{v2.0}{2006/12/06}{Added warning for OT1 encoding.}
%
% \changes{v2.1b}{2008/04/07}{Disable some commands in bookmarks.}
%
%    At |\begin{document}| we save again the definitions of the `list'
%    and `itemize' environments and the values of labelitems so that
%    all changes made in the preamble are taken into account in
%    languages other than French and in French with the StandardLayout
%    option.  We also have to provide an |\xspace| command in case the
%    |xspace.sty| package is not loaded.
%
%    \begin{macrocode}
\AtBeginDocument{%
   \let\listORI\list
   \let\endlistORI\endlist
   \let\itemizeORI\itemize
   \let\enditemizeORI\enditemize
   \let\@ltiORI\labelitemi
   \let\@ltiiORI\labelitemii
   \let\@ltiiiORI\labelitemiii
   \let\@ltivORI\labelitemiv
   \providecommand*{\xspace}{\relax}%
%    \end{macrocode}
%    Let's redefine some commands in \file{hyperref}'s bookmarks.
%    \begin{macrocode}
   \@ifundefined{pdfstringdefDisableCommands}{}%
     {\pdfstringdefDisableCommands{%
        \let\up\relax
        \def\ieme{e\xspace}%
        \def\iemes{es\xspace}%
        \def\ier{er\xspace}%
        \def\iers{ers\xspace}%
        \def\iere{re\xspace}%
        \def\ieres{res\xspace}%
        \def\FrenchEnumerate#1{#1\degre\space}%
        \def\FrenchPopularEnumerate#1{#1\degre)\space}%
        \def\No{N\degre\space}%
        \def\no{n\degre\space}%
        \def\Nos{N\degre\space}%
        \def\nos{n\degre\space}%
        \def\og{\guillemotleft\space}%
        \def\fg{\space\guillemotright}%
        \let\bsc\textsc
        \let\degres\degre
     }}%
%    \end{macrocode}
%    It is time to process the options set with |\frenchboptions{}|.
%    Then execute either |\extrasfrench| and |\captionsfrench| or
%    |\noextrasfrench| according to the current language at the
%    |\begin{document}| (these three commands are updated by
%    |\FBprocess@options|).
%    \begin{macrocode}
   \FBprocess@options
   \iflanguage{french}{\extrasfrench\captionsfrench}{\noextrasfrench}%
%    \end{macrocode}
%    Some warnings are issued when output font encodings are not
%    properly set. With XeLaTeX, \file{fontspec.sty} and
%    \file{xunicode.sty} should be loaded; with (pdf)\LaTeX, a warning
%    is issued when OT1 encoding is in use at the |\begin{document}|.
%    Mind that |\encodingdefault| is defined as `long', defining
%    |\FBOTone| with |\newcommand*| would fail!
%    \begin{macrocode}
   \expandafter\ifx\csname XeTeXrevision\endcsname\relax
      \begingroup \newcommand{\FBOTone}{OT1}%
      \ifx\encodingdefault\FBOTone
        \PackageWarning{frenchb.ldf}%
           {OT1 encoding should not be used for French.
            \MessageBreak
            Add \protect\usepackage[T1]{fontenc} to the
            preamble\MessageBreak of your document,}
      \fi
     \endgroup
   \else
     \@ifundefined{DeclareUTFcharacter}%
       {\PackageWarning{frenchb.ldf}%
         {Add \protect\usepackage{fontspec} *and*\MessageBreak
          \protect\usepackage{xunicode} to the preamble\MessageBreak
          of your document,}}%
       {}%
    \fi
}
%    \end{macrocode}
%
%  \subsection{Clean up and exit}
%
%    Load |frenchb.cfg| (should do nothing, just for compatibility).
%    \begin{macrocode}
\loadlocalcfg{frenchb}
%    \end{macrocode}
%    Final cleaning.
%    The macro |\ldf@quit| takes care for setting the main language
%    to be switched on at |\begin{document}| and resetting the
%    category code of \texttt{@} to its original value.
%    The config file searched for has to be |frenchb.cfg|, and
%    |\CurrentOption| has been set to `french', so
%    |\ldf@finish\CurrentOption| cannot be used: we first load
%    |frenchb.cfg|, then call |\ldf@quit\CurrentOption|.
%    \begin{macrocode}
\FBclean@on@exit
\ldf@quit\CurrentOption
%    \end{macrocode}
% \iffalse
%</code>
%<*dtx>
% \fi
%%
%% \CharacterTable
%%  {Upper-case    \A\B\C\D\E\F\G\H\I\J\K\L\M\N\O\P\Q\R\S\T\U\V\W\X\Y\Z
%%   Lower-case    \a\b\c\d\e\f\g\h\i\j\k\l\m\n\o\p\q\r\s\t\u\v\w\x\y\z
%%   Digits        \0\1\2\3\4\5\6\7\8\9
%%   Exclamation   \!     Double quote  \"     Hash (number) \#
%%   Dollar        \$     Percent       \%     Ampersand     \&
%%   Acute accent  \'     Left paren    \(     Right paren   \)
%%   Asterisk      \*     Plus          \+     Comma         \,
%%   Minus         \-     Point         \.     Solidus       \/
%%   Colon         \:     Semicolon     \;     Less than     \<
%%   Equals        \=     Greater than  \>     Question mark \?
%%   Commercial at \@     Left bracket  \[     Backslash     \\
%%   Right bracket \]     Circumflex    \^     Underscore    \_
%%   Grave accent  \`     Left brace    \{     Vertical bar  \|
%%   Right brace   \}     Tilde         \~}
%%
% \iffalse
%</dtx>
% \fi
%
% \Finale
\endinput
}
\bbl@tempa{french}{% \iffalse meta-comment
%
% Copyright 1989-2009 Johannes L. Braams and any individual authors
% listed elsewhere in this file.  All rights reserved.
% 
% This file is part of the Babel system.
% --------------------------------------
% 
% It may be distributed and/or modified under the
% conditions of the LaTeX Project Public License, either version 1.3
% of this license or (at your option) any later version.
% The latest version of this license is in
%   http://www.latex-project.org/lppl.txt
% and version 1.3 or later is part of all distributions of LaTeX
% version 2003/12/01 or later.
% 
% This work has the LPPL maintenance status "maintained".
% 
% The Current Maintainer of this work is Johannes Braams.
% 
% The list of all files belonging to the Babel system is
% given in the file `manifest.bbl. See also `legal.bbl' for additional
% information.
% 
% The list of derived (unpacked) files belonging to the distribution
% and covered by LPPL is defined by the unpacking scripts (with
% extension .ins) which are part of the distribution.
% \fi
% \CheckSum{2135}
%
% \iffalse
%    Tell the \LaTeX\ system who we are and write an entry on the
%    transcript. Nothing to write to the .cfg file, if generated.
%<*dtx>
\ProvidesFile{frenchb.dtx}
%</dtx>
% \changes{v2.1d}{2008/05/04}{Argument of \cs{ProvidesLanguage} changed
%     from `french' to `frenchb', otherwise \cs{listfiles} prints
%     no date/version information.  The bug with \cs{listfiles}
%     (introduced in v.1.5!), was pointed out by Ulrike Fischer.}
%<code>\ProvidesLanguage{frenchb}
%\ProvidesFile{frenchb.dtx}
%<*!cfg>
        [2009/03/16 v2.3d French support from the babel system]
%</!cfg>
%<*cfg>
%% frenchb.cfg: configuration file for frenchb.ldf
%% Daniel Flipo Daniel.Flipo at univ-lille1.fr
%</cfg>
%%    File `frenchb.dtx'
%%    Babel package for LaTeX version 2e
%%    Copyright (C) 1989 - 2009
%%              by Johannes Braams, TeXniek
%
%<*!cfg>
%%    Frenchb language Definition File
%%    Copyright (C) 1989 - 2009
%%              by Johannes Braams, TeXniek
%%                 Daniel Flipo, GUTenberg
%
%%    Please report errors to: Daniel Flipo, GUTenberg
%%                             Daniel.Flipo at univ-lille1.fr
%</!cfg>
%
%    This file is part of the babel system, it provides the source
%    code for the French language definition file.
%
%<*filedriver>
\documentclass[a4paper]{ltxdoc}
\DeclareFontEncoding{T1}{}{}
\DeclareFontSubstitution{T1}{lmr}{m}{n}
\DeclareTextCommand{\guillemotleft}{OT1}{%
  {\fontencoding{T1}\fontfamily{lmr}\selectfont\char19}}
\DeclareTextCommand{\guillemotright}{OT1}{%
  {\fontencoding{T1}\fontfamily{lmr}\selectfont\char20}}
\newcommand*\TeXhax{\TeX hax}
\newcommand*\babel{\textsf{babel}}
\newcommand*\langvar{$\langle \mathit lang \rangle$}
\newcommand*\note[1]{}
\newcommand*\Lopt[1]{\textsf{#1}}
\newcommand*\file[1]{\texttt{#1}}
\begin{document}
\setlength{\parindent}{0pt}
\begin{center}
  \textbf{\Large A Babel language definition file for French}\\[3mm]^^A\]
  Daniel \textsc{Flipo}\\
  \texttt{Daniel.Flipo@univ-lille1.fr}
\end{center}
 \RecordChanges
 \DocInput{frenchb.dtx}
\end{document}
%</filedriver>
% \fi
% \GetFileInfo{frenchb.dtx}
%
%  \section{The French language}
%
%    The file \file{\filename}\footnote{The file described in this
%    section has version number \fileversion\ and was last revised on
%    \filedate.}, defines all the language definition macros for the
%    French language.
%
%    Customisation for the French language is achieved following the
%    book ``Lexique des r\`egles typographiques en usage \`a
%    l'Imprimerie nationale'' troisi\`eme \'edition (1994),
%    ISBN-2-11-081075-0.
%
%    First version released: 1.1 (1996/05/31) as part of
%    \babel-3.6beta.
%
%    |frenchb| has been improved using helpful suggestions from many
%    people, mainly from Jacques Andr\'e, Michel Bovani, Thierry Bouche,
%    and Vincent Jalby.  Thanks to all of them!
%
%    This new version (2.x) has been designed to be used with \LaTeXe{}
%    and Plain\TeX{} formats only. \LaTeX-2.09 is no longer supported.
%    Changes between version 1.6 and \fileversion{} are listed in
%    subsection~\ref{ssec-changes} p.~\pageref{ssec-changes}.
%
%    An extensive documentation is available in French here:\\
%    |http://daniel.flipo.free.fr/frenchb|
%
%  \subsection{Basic interface}
%
%    In a multilingual document, some typographic rules are language
%    dependent, i.e. spaces before `double punctuation' (|:| |;| |!|
%    |?|) in French, others concern the general layout (i.e. layout of
%    lists, footnotes, indentation of first paragraphs of sections) and
%    should apply to the whole document.
%
%    Starting with version~2.2, |frenchb| behaves differently according
%    to \babel's \emph{main language} defined as the \emph{last}
%    option\footnote{Its name is kept in \texttt{\textbackslash
%           bbl@main@language}.} at \babel's loading.  When French is
%    not \babel's main language, |frenchb| no longer alters the global
%    layout of the document (even in parts where French is the current
%    language): the layout of lists, footnotes, indentation of first
%    paragraphs of sections are not customised by |frenchb|.
%
%    When French is loaded as the last option of \babel, |frenchb|
%    makes the following changes to the global layout, \emph{both in
%    French and in all other languages}\footnote{%
%       For each item, hooks are provided to reset standard
%       \LaTeX{} settings or to emulate the behavior of former versions
%       of \texttt{frenchb} (see command
%       \texttt{\textbackslash frenchbsetup\{\}},
%       section~\ref{ssec-custom}).}:
%    \begin{enumerate}
%    \item the first paragraph of each section is indented
%          (\LaTeX{} only);
%    \item the default items in itemize environment are set to `--'
%          instead of `\textbullet', and all vertical spacing and glue
%          is deleted; it is possible to change `--' to something else
%          (`---' for instance) using |\frenchbsetup{}|;
%    \item vertical spacing in general \LaTeX{} lists is
%          shortened;
%    \item footnotes are displayed ``\`a la fran\c{c}aise''.
%    \end{enumerate}
%
%    Regarding local typography, the command |\selectlanguage{french}|
%    switches to the French language\footnote{%
%      \texttt{\textbackslash selectlanguage\{francais\}}
%      and \texttt{\textbackslash selectlanguage\{frenchb\}} are kept
%      for backward compatibility but should no longer be used.},
%    with the following effects:
%    \begin{enumerate}
%    \item French hyphenation patterns are made active;
%    \item `double punctuation' (|:| |;| |!| |?|) is made
%           active%\footnote{Actually, they are active in the whole
%           document, only their expansions differ in French and
%           outside French} for correct spacing in French;
%    \item |\today| prints the date in French;
%    \item the caption names are translated into French
%          (\LaTeX{} only);
%    \item the space after |\dots| is removed in French.
%    \end{enumerate}
%
%    Some commands are provided in |frenchb| to make typesetting
%    easier:
%    \begin{enumerate}
%    \item French quotation marks can be entered using the commands
%          |\og| and |\fg| which work in \LaTeXe and Plain\TeX,
%          their appearance depending on what is available to draw
%          them; even if you use \LaTeXe{} \emph{and} |T1|-encoding,
%          you should refrain from entering them as
%          |<<~French quotation marks~>>|: |\og| and |\fg| provide
%          better horizontal spacing.
%          |\og| and |\fg| can be used outside French, they typeset
%          then English quotes `` and ''.
%    \item A command |\up| is provided to typeset superscripts like
%          |M\up{me}| (abbreviation for ``Madame''), |1\up{er}| (for
%          ``premier'').  Other commands are also provided for
%          ordinals: |\ier|, |\iere|, |\iers|, |\ieres|, |\ieme|,
%          |\iemes| (|3\iemes| prints 3\textsuperscript{es}).
%    \item Family names should be typeset in small capitals and never
%          be hyphenated, the macro |\bsc| (boxed small caps) does
%          this, e.g., |Leslie~\bsc{Lamport}| will produce
%          Leslie~\mbox{\textsc{Lamport}}. Note that composed names
%          (such as Dupont-Durant) may now be hyphenated on explicit
%          hyphens, this differs from |frenchb|~v.1.x.
%    \item Commands |\primo|, |\secundo|, |\tertio| and |\quarto|
%          print 1\textsuperscript{o}, 2\textsuperscript{o},
%          3\textsuperscript{o}, 4\textsuperscript{o}.
%          |\FrenchEnumerate{6}| prints 6\textsuperscript{o}.
%    \item Abbreviations for ``Num\'ero(s)'' and ``num\'ero(s)''
%          (N\textsuperscript{o} N\textsuperscript{os}
%          n\textsuperscript{o} and n\textsuperscript{os}~)
%          are obtained via the commands |\No|, |\Nos|, |\no|, |\nos|.
%    \item Two commands are provided to typeset the symbol for
%          ``degr\'e'': |\degre| prints the raw character and
%          |\degres| should be used to typeset temperatures (e.g.,
%          ``|20~\degres C|'' with an unbreakable space), or for
%          alcohols' strengths (e.g., ``|45\degres|'' with \emph{no}
%          space in French).
%    \item In math mode the comma has to be surrounded with
%          braces to avoid a spurious space being inserted after it,
%          in decimal numbers for instance (see the \TeX{}book p.~134).
%          The command |\DecimalMathComma| makes the comma be an
%          ordinary character \emph{in French only} (no space added);
%          as a counterpart, if |\DecimalMathComma| is active, an
%          explicit space has to be added in lists and intervals:
%          |$[0,\ 1]$|, |$(x,\ y)$|. |\StandardMathComma| switches back
%          to the standard behaviour of the comma.
%    \item A command |\nombre| was provided in 1.x versions to easily
%          format numbers in slices of three digits separated either
%          by a comma in English or with a space in French; |\nombre|
%          is now mapped to |\numprint| from \file{numprint.sty}, see
%          \file{numprint.pdf} for more information.
%    \item |frenchb| has been designed to take advantage of the |xspace|
%          package if present: adding |\usepackage{xspace}| in the
%          preamble will force macros like |\fg|, |\ier|, |\ieme|,
%          |\dots|, \dots, to respect the spaces you type after them,
%          for instance typing `|1\ier juin|' will print
%          `1\textsuperscript{er} juin' (no need for a forced space
%          after |1\ier|).
%    \end{enumerate}
%
%  \subsection{Customisation}
%  \label{ssec-custom}
%
%     Up to version 1.6, customisation of |frenchb| was achieved
%     by entering commands in \file{frenchb.cfg}.  This possibility
%     remains for compatibility, but \emph{should not longer be used}.
%     Version 2.0 introduces a new command |\frenchbsetup{}| using
%     the \file{keyval} syntax which should make it easier to choose
%     among the many options available. The command |\frenchbsetup{}|
%     is to appear in the preamble only (after loading \babel).
%
%     \vspace{.5\baselineskip}
%     |\frenchbsetup{ShowOptions}| prints all available options to
%     the \file{.log} file, it is just meant as a remainder of the
%     list of offered options. As usual with \file{keyval} syntax,
%     boolean options (as |ShowOptions|) can be entered as
%     |ShowOptions=true| or just |ShowOptions|, the `|=true|' part
%     can be omitted.
%
%     \vspace{.5\baselineskip}
%     The other options are listed below. Their default value is shown
%     between brackets, sometimes followed be a `\texttt{*}'.
%     The `\texttt{*}' means that the default shown applies when
%     |frenchb| is loaded as the \emph{last} option of \babel{}
%     ---\babel's \emph{main language}---, and is toggled otherwise:
%     \begin{itemize}
%     \item |StandardLayout=true [false*]| forces |frenchb| not to
%       interfere with the layout: no action on any kind of lists,
%       first paragraphs of sections are not indented (as in English),
%       no action on footnotes. This option replaces the former
%       command |\StandardLayout|.  It can be used to avoid conflicts
%       with classes or packages which customise lists or footnotes.
%     \item |GlobalLayoutFrench=false [true*]| can be used, when French
%       is the main language, to emulate what prior versions of
%       |frenchb| (pre-2.2) did: lists, and first paragraphs
%       of sections will be displayed the standard way in other
%       languages than French, and ``\`a la fran\c{c}aise'' in French.
%       Note that the layout of footnotes is language independent
%       anyway (see below |FrenchFootnotes| and |AutoSpaceFootnotes|).
%       This option replaces the former command |\FrenchLayout|.
%     \item |ReduceListSpacing=false [true*]|; |frenchb| normally
%       reduces the values of the vertical spaces used in the
%       environment |list| in French; setting this option to |false|
%       reverts to the standard settings of |list|.  This option
%       replaces the former command |\FrenchListSpacingfalse|.
%     \item |CompactItemize=false [true*]|; |frenchb| normally
%       suppresses any vertical space between items of |itemize| lists
%       in French; setting this option to |false| reverts to the
%       standard settings of |itemize| lists.  This option replaces
%       the former command |\FrenchItemizeSpacingfalse|.
%     \item |StandardItemLabels=true [false*]| when set to |true| this
%       option stops |frenchb| from changing the labels in |itemize|
%       lists in French.
%     \item |ItemLabels=\textemdash|, |\textbullet|, |\ding{43}|,
%       \dots, |[\textendash*]|; when |StandardItemLabels=false| (the
%       default), this option enables to choose the label used in
%       |itemize| lists for all levels.  The next three options do
%       the same but each one for one level only. Note that the
%       example |\ding{43}| requires |\usepackage{pifont}|.
%     \item |ItemLabeli=\textemdash|, |\textbullet|, |\ding{43}|,
%       \dots,|[\textendash*]|
%     \item |ItemLabelii=\textemdash|, |\textbullet|, |\ding{43}|,
%       \dots, |[\textendash*]|
%     \item |ItemLabeliii=\textemdash|, |\textbullet|, |\ding{43}|,
%       \dots, |[\textendash*]|
%     \item |ItemLabeliv=\textemdash|, |\textbullet|, |\ding{43}|,
%       \dots, |[\textendash*]|
%     \item |StandardLists=true [false*]| forbids |frenchb| to
%       customise any kind of list. Do activate the option
%       |StandardLists| when using classes or packages that customise
%       lists too (|enumitem|, |paralist|, \dots{}) to avoid conflicts.
%       This option is just a shorthand for |ReduceListSpacing=false|
%       and |CompactItemize=false| and |StandardItemLabels=true|.
%     \item |IndentFirst=false [true*]|; |frenchb| normally forces
%       indentation of the first paragraph of sections.
%       When this option is set to |false|, the first paragraph of
%       will look the same in French and in English (not indented).
%     \item |FrenchFootnotes=false [true*]| reverts to the standard
%       layout of footnotes. By default |frenchb| typesets leading
%       numbers as `1.\hspace{.5em}' instead of `$\hbox{}^1$', but
%       has no effect on footnotes numbered with symbols (as in the
%       |\thanks| command).  The former commands |\StandardFootnotes|
%       and |\FrenchFootnotes| are still there, |\StandardFootnotes|
%       can be useful when some footnotes are numbered with letters
%       (inside minipages for instance).
%     \item |AutoSpaceFootnotes=false [true*]| ; by default |frenchb|
%       adds a thin space in the running text before the number or
%       symbol calling the footnote.  Making this option |false|
%       reverts to the standard setting (no space added).
%     \item |FrenchSuperscripts=false [true]| ; then
%       |\up=\textsuperscript| (option added in version 2.1).
%       Should only be made |false| to recompile older documents.
%       By default |\up| now relies on |\fup| designed to produce
%       better looking superscripts.
%     \item |AutoSpacePunctuation=false [true]|; in French, the user
%       \emph{should} input a space before the four characters `|:;!?|'
%       but as many people forget about it (even among native French
%       writers!), the default behaviour of |frenchb| is to
%       automatically add a |\thinspace| before `|;|' `|!|' `|?|' and a
%       normal (unbreakable) space before~`|:|' (this is recommended by
%       the French Imprimerie nationale).  This is convenient in most
%       cases but can lead to addition of spurious spaces in URLs or in
%       MS-DOS paths but only if they are no typed using |\texttt| or
%       verbatim mode. When the current font is a monospaced
%       (typewriter) font, |AutoSpacePunctuation| is locally switched
%       to |false|, no spurious space is added in that case, so the
%       default behaviour of of |frenchb| in that area should be fine
%       in most circumstances.
%
%       Choosing |AutoSpacePunctuation=false| will ensure that
%       a proper space will be added before `|:;!?|' \emph{if and only
%       if} a (normal) space has been typed in. Those who are unsure
%       about their typing in this area should stick to the default
%       option and type |\string;| |\string:| |\string!| |\string?|
%       instead of |;| |:| |!| |?| in case an unwanted space is
%       added by |frenchb|.
%     \item |ThinColonSpace=true [false]| changes the normal
%       (unbreakable) space added before the colon `:' to a thin space,
%       so that the same amount of space is added before any of the
%       four double punctuation characters. The default setting is
%       supported by the French Imprimerie nationale.
%     \item |LowercaseSuperscripts=false [true]| ; by default |frenchb|
%       inhibits the uppercasing of superscripts (for instance when they
%       are moved to page headers). Making this option |false|
%       will disable this behaviour (not recommended).
%     \item |PartNameFull=false [true]|; when true, |frenchb| numbers
%       the title of |\part{}| commands as ``Premi\`ere partie'',
%       ``Deuxi\`eme partie'' and so on. With some classes which change
%       the|\part{}| command (AMS and SMF classes do so), you will get
%       ``Premi\`ere partie~I'', ``Deuxi\`eme partie~II'' instead;
%       when this occurs, this option should be set to |false|,
%       part titles will then be printed as ``Partie I'', ``Partie II''.
%     \item |og=|\texttt{\guillemotleft}, |fg=|\texttt{\guillemotright};
%       when guillemets characters are available on the keyboard
%       (through a compose key for instance), it is nice to use them
%       instead of typing |\og| and |\fg|. This option tells |frenchb|
%       which characters are opening and closing French guillemets
%       (they depend on the input encoding), then you can type either
%       \texttt{\guillemotleft{} guillemets \guillemotright}, or
%       \texttt{\guillemotleft{}guillemets\guillemotright} (with or
%       without spaces), to get properly typeset French quotes.
%       This option requires \file{inputenc} to be loaded with the
%       proper encoding, it works with 8-bits encodings (latin1,
%       latin9, ansinew,  applemac,\dots) and multi-byte encodings
%       (utf8 and utf8x).
%     \end{itemize}
%
%  \subsection{Hyphenation checks}
%  \label{ssec-hyphen}
%
%    Once you have built your format, a good precaution would be to
%    perform some basic tests about hyphenation in French. For
%    \LaTeXe{} I suggest this:
%    \begin{itemize}
%    \item run the following file, with the encoding suitable for
%      your machine (\textit{my-encoding} will be |latin1| for
%      \textsc{unix} machines, |ansinew| for PCs running~Windows,
%      |applemac| or |latin1| for Macintoshs, or |utf8|\dots\\[3mm]^^A\]
%      |%%% Test file for French hyphenation.|\\
%      |\documentclass{article}|\\
%      |\usepackage[|\textit{my-encoding}|]{inputenc}|\\
%      |\usepackage[T1]{fontenc} % Use LM fonts|\\
%      |\usepackage{lmodern}     % for French|\\
%      |\usepackage[frenchb]{babel}|\\
%      |\begin{document}|\\
%      |\showhyphens{signal container \'ev\'enement alg\`ebre}|\\
%      |\showhyphens{|\texttt{signal container \'ev\'enement
%                     alg\`ebre}|}|\\
%      |\end{document}|
%    \item check the hyphenations proposed by \TeX{} in your log-file;
%      in French you should get with both 7-bit and 8-bit encodings\\
%      \texttt{si-gnal contai-ner \'ev\'e-ne-ment al-g\`ebre}.\\
%      Do not care about how accented characters are displayed in the
%      log-file, what matters is the position of the `|-|' hyphen
%      signs \emph{only}.
%    \end{itemize}
%    If they are all correct, your installation (probably) works fine,
%    if one (or more) is (are) wrong, ask a local wizard to see what's
%    going wrong and perform the test again (or e-mail me about what
%    happens).\\
%    Frequent mismatches:
%    \begin{itemize}
%    \item you get |sig-nal con-tainer|, this probably means that the
%    hyphenation patterns you are using are for US-English, not for
%    French;
%    \item you get no hyphen at all in \texttt{\'ev\'e-ne-ment}, this
%    probably means that you are using CM fonts and the macro
%    |\accent| to produce accented characters.
%    Using 8-bits fonts with built-in accented characters avoids
%    this kind of mismatch.
%    \end{itemize}
%
%    \textbf{Options' order} -- Please remember that options are read
%    in the order they appear inside the |\frenchbsetup| command.
%    Someone wishing that |frenchb| leaves the layout of lists
%    and footnotes untouched but caring for indentation of first
%    paragraph of sections could choose
%    |\frenchbsetup{StandardLayout,IndentFirst}| and get the expected
%    layout. Choosing |\frenchbsetup{IndentFirst,StandardLayout}|
%    would not lead to the expected result: option |IndentFirst| would
%    be overwritten by |StandardLayout|.
%
%  \subsection{Changes}
%  \label{ssec-changes}
%
%  \subsubsection*{What's new in version 2.0?}
%
%    Here is the list of all changes:
%    \begin{itemize}
%    \item Support for \LaTeX-2.09 and for \LaTeXe{} in compatibility
%      mode has been dropped. This version is meant for \LaTeXe{} and
%      Plain based formats (like \file{bplain}). \LaTeXe{} formats
%      based on ml\TeX{} are no longer supported either (plenty of
%      good 8-bits fonts are available now, so T1 encoding should
%      be preferred for typesetting in French). A warning is issued
%      when OT1 encoding is in use at the |\begin{document}|.
%    \item Customisation should now be handled by command
%      |\frenchbsetup{}|, \file{frenchb.cfg} (kept for compatibility)
%      should no longer be used. See section~\ref{ssec-custom} for
%      the list of available options.
%    \item Captions in figures and table have changed in French: former
%      abbreviations ``Fig.'' and ``Tab.'' have been replaced by full
%      names ``Figure'' and ``Table''.  If this leads to formatting
%      problems in captions, you can add the following two commands to
%      your preamble (after loading \babel) to get the former captions\\
%      |\addto\captionsfrench{\def\figurename{{\scshape Fig.}}}|\\
%      |\addto\captionsfrench{\def\tablename{{\scshape Tab.}}}|.
%    \item The |\nombre| command is now provided by the \file{numprint}
%      package which has to be loaded \emph{after} \babel{} with the
%      option |autolanguage| if number formatting should depend on the
%      current language.
%    \item The |\bsc| command no longer uses an |\hbox| to stop
%      hyphenation of names but a |\kern0pt| instead. This change
%      enables \file{microtype} to fine tune the length of the
%      argument of |\bsc|; as a side-effect, compound names like
%      Dupont-Durand can now be hyphenated on  explicit hyphens.
%      You can get back to the former behaviour of |\bsc| by adding\\
%      |\renewcommand*{\bsc}[1]{\leavevmode\hbox{\scshape #1}}|\\
%      to the preamble of your document.
%    \item Footnotes are now displayed ``\`a la fran\c caise'' for the
%      whole document, except with an explicit\\
%      |\frenchbsetup{AutoSpaceFootnotes=false,FrenchFootnotes=false}|.\\
%      Add this command if you want standard footnotes. It is still
%      possible to revert locally to the standard layout of footnotes
%      by adding |\StandardFootnotes| (inside a |minipage| environment
%      for instance).
%    \end{itemize}
%
%  \subsubsection*{What's new in version 2.1?}
%
%      New command |\fup| to typeset better looking superscripts.
%      Former command |\up| is now defined as |\fup|, but an option
%      |\frenchbsetup{FrenchSuperscripts=false}| is provided for
%      backward compatibility.  |\fup| was designed using ideas from
%      Jacques Andr\'e, Thierry Bouche and Ren\'e Fritz, thanks to them!
%
%  \subsubsection*{What's new in version 2.2?}
%
%      Starting with version~2.2a, |frenchb| alters the layout of
%      lists, footnotes, and the indentation of first paragraphs of
%      sections) \emph{only if} French is the ``main language''
%      (i.e. babel's last language option). The layout is global for
%      the whole document: lists, etc. look the same in French and in
%      other languages, everything is typeset ``\`a la fran\c caise''
%      if French is the ``main language'', otherwise |frenchb| doesn't
%      change anything regarding lists, footnotes, and indentation of
%      paragraphs.
%
%  \subsubsection*{What's new in version 2.3?}
%
%      Starting with version~2.3a, |frenchb| no longer inserts spaces
%      automatically before `|:;!?|' when a typewriter font is in use;
%      this was suggested by Yannis Haralambous to prevent
%      spurious spaces in computer source code or expressions like
%      \texttt{C\string:/foo}, \texttt{http\string://foo.bar},
%      etc.  An option (|OriginalTypewriter|) is provided to get back
%      to the former behaviour of |frenchb|.
%
%      Another probably invisible change: lowercase conversion in
%      |\up{}| is now achieved by the \LaTeX{} command |\MakeLowercase|
%      instead of \TeX's |\lowercase| command.  This prevents error
%      messages when diacritics are used inside |\up{}| (diacritics
%      should \emph{never} be used in superscripts though!).
%
% \StopEventually{}
%
%  \subsection{File frenchb.cfg}
%  \label{sec-cfg}
%
%    \file{frenchb.cfg} is now a dummy file just kept for compatibility
%    with previous versions.
%
% \iffalse
%<*cfg>
% \fi
%    \begin{macrocode}
%%%%%%%%%%%%%%%%%%%%%%%%%%%%%%%%%%%%%%%%%%%%%%%%%%%%%%%%%%%%%%%%%%%%%%
%%%%%%%%%  WARNING: THIS  FILE SHOULD  NO  LONGER  BE  USED  %%%%%%%%%
%% If you want to customise frenchb, please DO NOT hack into the code!
%% Do no put any code in this file either, please use the new command
%% \frenchbsetup{} with the proper options to customise frenchb.
%% 
%% Add \frenchbsetup{ShowOptions} to your preamble to see the list of
%% available options and/or read the documentation.
%%%%%%%%%%%%%%%%%%%%%%%%%%%%%%%%%%%%%%%%%%%%%%%%%%%%%%%%%%%%%%%%%%%%%%
%    \end{macrocode}
% \iffalse
%</cfg>
% \fi
%
%  \section{\TeX{}nical details}
%
%  \subsection{Initial setup}
%
% \changes{v2.1d}{2008/05/02}{Argument of \cs{ProvidesLanguage} changed
%     above from `french' to `frenchb' (otherwise \cs{listfiles} prints
%     no date/version information).  The real name of current language
%     (french) as to be corrected before calling \cs{LdfInit}.}
%
% \iffalse
%<*code>
% \fi
%
%    While this file was read through the option \Lopt{frenchb} we make
%    it behave as if \Lopt{french} was specified.
%    \begin{macrocode}
\def\CurrentOption{french}
%    \end{macrocode}
%
%    The macro |\LdfInit| takes care of preventing that this file is
%    loaded more than once, checking the category code of the
%    \texttt{@} sign, etc.
%
%    \begin{macrocode}
\LdfInit\CurrentOption\datefrench
%    \end{macrocode}
%
% \changes{v2.1d}{2008/05/04}{Avoid warning ``\cs{end} occurred
%   when \cs{ifx} ... incomplete'' with LaTeX-2.09.}
%
%  \begin{macro}{\ifLaTeXe}
%    No support is provided for late \LaTeX-2.09: issue a warning
%    and exit if \LaTeX-2.09 is in use. Plain is still supported.
%    \begin{macrocode}
\newif\ifLaTeXe
\let\bbl@tempa\relax
\ifx\magnification\@undefined
   \ifx\@compatibilitytrue\@undefined
     \PackageError{frenchb.ldf}
        {LaTeX-2.09 format is no longer supported.\MessageBreak
         Aborting here}
        {Please upgrade to LaTeX2e!}
     \let\bbl@tempa\endinput
   \else
     \LaTeXetrue
   \fi
\fi
\bbl@tempa
%    \end{macrocode}
%  \end{macro}
%
%    Check if hyphenation patterns for the French language have been
%    loaded in language.dat; we allow for the names `french',
%    `francais', `canadien' or `acadian'. The latter two are both
%    names used in Canada for variants of French that are in use in
%    that country.
%
%    \begin{macrocode}
\ifx\l@french\@undefined
  \ifx\l@francais\@undefined
    \ifx\l@canadien\@undefined
      \ifx\l@acadian\@undefined
        \@nopatterns{French}
        \adddialect\l@french0
      \else
        \let\l@french\l@acadian
      \fi
    \else
      \let\l@french\l@canadien
    \fi
  \else
    \let\l@french\l@francais
  \fi
\fi
%    \end{macrocode}
%    Now |\l@french| is always defined.
%
%    The internal name for the French language is |french|;
%    |francais| and |frenchb| are synonymous for |french|:
%    first let both names use the same hyphenation patterns.
%    Later we will have to set aliases for |\captionsfrench|,
%    |\datefrench|, |\extrasfrench| and |\noextrasfrench|.
%    As French uses the standard values of |\lefthyphenmin| (2)
%    and |\righthyphenmin| (3), no special setting is required here.
%
%    \begin{macrocode}
\ifx\l@francais\@undefined
  \let\l@francais\l@french
\fi
\ifx\l@frenchb\@undefined
  \let\l@frenchb\l@french
\fi
%    \end{macrocode}
%    When this language definition file was loaded for one of the
%    Canadian versions of French we need to make sure that a suitable
%    hyphenation pattern register will be found by \TeX.
%    \begin{macrocode}
\ifx\l@canadien\@undefined
  \let\l@canadien\l@french
\fi
\ifx\l@acadian\@undefined
  \let\l@acadian\l@french
\fi
%    \end{macrocode}
%
%    This language definition can be loaded for different variants of
%    the French language. The `key' \babel\ macros are only defined
%    once, using `french' as the language name, but |frenchb| and
%    |francais| are synonymous.
%    \begin{macrocode}
\def\datefrancais{\datefrench}
\def\datefrenchb{\datefrench}
\def\extrasfrancais{\extrasfrench}
\def\extrasfrenchb{\extrasfrench}
\def\noextrasfrancais{\noextrasfrench}
\def\noextrasfrenchb{\noextrasfrench}
%    \end{macrocode}
%
% \begin{macro}{\extrasfrench}
% \begin{macro}{\noextrasfrench}
%    The macro |\extrasfrench| will perform all the extra
%    definitions needed for the French language.
%    The macro |\noextrasfrench| is used to cancel the actions of
%    |\extrasfrench|.\\
%    In French, character ``apostrophe'' is a letter in expressions
%    like |l'ambulance| (French  hyphenation patterns provide entries
%    for this kind of words).  This means that the |\lccode| of
%    ``apostrophe'' has to be non null in French for proper hyphenation
%    of those expressions, and has to be reset to null when exiting
%    French.
%
%    \begin{macrocode}
\@namedef{extras\CurrentOption}{\lccode`\'=`\'}
\@namedef{noextras\CurrentOption}{\lccode`\'=0}
%    \end{macrocode}
% \end{macro}
% \end{macro}
%
%    One more thing |\extrasfrench| needs to do is to make sure that
%    |\frenchspacing| is in effect.  |\noextrasfrench| will switch
%    |\frenchspacing| off again.
%    \begin{macrocode}
  \expandafter\addto\csname extras\CurrentOption\endcsname{%
    \bbl@frenchspacing}
  \expandafter\addto\csname noextras\CurrentOption\endcsname{%
    \bbl@nonfrenchspacing}
%    \end{macrocode}
%
%  \subsection{Punctuation}
%  \label{sec-punct}
%
%    As long as no better solution is available%
%    \footnote{Lua\TeX, or pdf\TeX{} might provide alternatives in
%       the future\dots},
%    the `double punctuation' characters (|;| |!| |?| and |:|) have to
%    be made |\active| for an automatic control of the amount of space
%    to insert before them. Before doing so, we have to save the
%    standard definition of |\@makecaption| (which includes two ':')
%    to compare it later to its definition at the |\begin{document}|.
%    \begin{macrocode}
\long\def\STD@makecaption#1#2{%
  \vskip\abovecaptionskip
  \sbox\@tempboxa{#1: #2}%
  \ifdim \wd\@tempboxa >\hsize
    #1: #2\par
  \else
    \global \@minipagefalse
    \hb@xt@\hsize{\hfil\box\@tempboxa\hfil}%
  \fi
  \vskip\belowcaptionskip}%
%    \end{macrocode}
%
%    We define a new `if' |\FBpunct@active| which will be made false
%    whenever a better alternative will be available. Another `if'
%    |\FBAutoSpacePunctuation| needs to be defined now.
%    \begin{macrocode}
\newif\ifFBpunct@active          \FBpunct@activetrue
\newif\ifFBAutoSpacePunctuation  \FBAutoSpacePunctuationtrue
%    \end{macrocode}
%    The following code makes the four characters |;| |!| |?| and |:|
%    `active' and provides their definitions.
%    \begin{macrocode}
\ifFBpunct@active
  \initiate@active@char{:}
  \initiate@active@char{;}
  \initiate@active@char{!}
  \initiate@active@char{?}
%    \end{macrocode}
%    We first tune the amount of space before \texttt{;}
%    \texttt{!}  \texttt{?} and \texttt{:}.  This should only happen
%    in horizontal mode, hence the test |\ifhmode|.
%
%    In horizontal mode, if a space has been typed before `;' we
%    remove it and put an unbreakable |\thinspace| instead. If no
%    space has been typed, we add |\FDP@thinspace| which will be
%    defined, up to the user's wishes, as an automatic added
%    thin space, or as |\@empty|.
%    \begin{macrocode}
  \declare@shorthand{french}{;}{%
      \ifhmode
      \ifdim\lastskip>\z@
          \unskip\penalty\@M\thinspace
          \else
            \FDP@thinspace
        \fi
      \fi
%    \end{macrocode}
%    Now we can insert a |;| character.
%    \begin{macrocode}
      \string;}
%    \end{macrocode}
%    The next three definitions are very similar.
%    \begin{macrocode}
  \declare@shorthand{french}{!}{%
      \ifhmode
        \ifdim\lastskip>\z@
          \unskip\penalty\@M\thinspace
        \else
          \FDP@thinspace
        \fi
      \fi
      \string!}
  \declare@shorthand{french}{?}{%
      \ifhmode
        \ifdim\lastskip>\z@
          \unskip\penalty\@M\thinspace
        \else
          \FDP@thinspace
        \fi
      \fi
      \string?}
%    \end{macrocode}
%    According to the I.N. specifications, the `:' requires a normal
%    space before it, but some people prefer a |\thinspace| (just
%    like the other three). We define |\Fcolonspace| to hold the
%    required amount of space (user customisable).
%    \begin{macrocode}
  \newcommand*{\Fcolonspace}{\space}
  \declare@shorthand{french}{:}{%
      \ifhmode
        \ifdim\lastskip>\z@
          \unskip\penalty\@M\Fcolonspace
        \else
          \FDP@colonspace
        \fi
      \fi
      \string:}
%    \end{macrocode}
%
% \changes{v2.3a}{2008/10/10}{\cs{NoAutoSpaceBeforeFDP} and
%    \cs{AutoSpaceBeforeFDP} now set the flag
%    \cs{ifFBAutoSpacePunctuation} accordingly (LaTeX only).}
%
%  \begin{macro}{\AutoSpaceBeforeFDP}
%  \begin{macro}{\NoAutoSpaceBeforeFDP}
%    |\FDP@thinspace| and |\FDP@space| are defined as unbreakable
%    spaces by |\autospace@beforeFDP| or as |\@empty| by
%    |\noautospace@beforeFDP| (internal commands), user commands
%    |\AutoSpaceBeforeFDP| and |\NoAutoSpaceBeforeFDP| do the same and
%    take care of the flag |\ifFBAutoSpacePunctuation| in \LaTeX{}.
%    Set the default now for Plain (done later for \LaTeX).
%    \begin{macrocode}
  \def\autospace@beforeFDP{%
          \def\FDP@thinspace{\penalty\@M\thinspace}%
          \def\FDP@colonspace{\penalty\@M\Fcolonspace}}
  \def\noautospace@beforeFDP{\let\FDP@thinspace\@empty
                            \let\FDP@colonspace\@empty}
  \ifLaTeXe
    \def\AutoSpaceBeforeFDP{\autospace@beforeFDP
                            \FBAutoSpacePunctuationtrue}
    \def\NoAutoSpaceBeforeFDP{\noautospace@beforeFDP
                              \FBAutoSpacePunctuationfalse}
  \else
    \let\AutoSpaceBeforeFDP\autospace@beforeFDP
    \let\NoAutoSpaceBeforeFDP\noautospace@beforeFDP
    \AutoSpaceBeforeFDP
  \fi
%    \end{macrocode}
% \end{macro}
% \end{macro}
%
% \changes{v2.3a}{2008/10/10}{In LaTeX, frenchb no longer adds spaces
%     before `double punctuation' characters in computer code.
%     Suggested by Yannis Haralambous.}
%
% \changes{v2.3c}{2009/02/07}{Commands \cs{ttfamily}, \cs{rmfamily}
%    and \cs{sffamily} have to be robust.  Bug introduced in 2.3a,
%    pointed out by Manuel P\'egouri\'e-Gonnard.}
%
%    In \LaTeXe{} |\ttfamily| (and hence |\texttt|) will be redefined
%    `AtBeginDocument' as |\ttfamilyFB| so that no space
%    is added before the four |; : ! ?| characters, even if
%    |AutoSpacePunctuation| is true.  |\rmfamily| and |\sffamily| need
%    to be redefined also (|\ttfamily| is not always used inside a
%    group, its effect can be cancelled by |\rmfamily| or |\sffamily|).
%
%    These redefinitions can be canceled if necessary, for instance to
%    recompile older documents, see option |OriginalTypewriter| below.
%    \begin{macrocode}
  \ifLaTeXe
    \let\ttfamilyORI\ttfamily
    \let\rmfamilyORI\rmfamily
    \let\sffamilyORI\sffamily
    \DeclareRobustCommand\ttfamilyFB{%
         \noautospace@beforeFDP\ttfamilyORI}%
    \DeclareRobustCommand\rmfamilyFB{%
         \ifFBAutoSpacePunctuation
            \autospace@beforeFDP\rmfamilyORI
         \else
            \noautospace@beforeFDP\rmfamilyORI
         \fi}%
    \DeclareRobustCommand\sffamilyFB{%
         \ifFBAutoSpacePunctuation
            \autospace@beforeFDP\sffamilyORI
         \else
            \noautospace@beforeFDP\sffamilyORI
         \fi}%
  \fi
%    \end{macrocode}
%
%    When the active characters appear in an environment where their
%    French behaviour is not wanted they should give an `expected'
%    result. Therefore we define shorthands at system level as well.
%    \begin{macrocode}
  \declare@shorthand{system}{:}{\string:}
  \declare@shorthand{system}{!}{\string!}
  \declare@shorthand{system}{?}{\string?}
  \declare@shorthand{system}{;}{\string;}
%    \end{macrocode}
%    We specify that the French group of shorthands should be used.
%    \begin{macrocode}
  \addto\extrasfrench{%
    \languageshorthands{french}%
%    \end{macrocode}
%    These characters are `turned on' once, later their definition may
%    vary. Don't misunderstand the following code: they keep being
%    active all along the document, even when leaving French.
%    \begin{macrocode}
    \bbl@activate{:}\bbl@activate{;}%
    \bbl@activate{!}\bbl@activate{?}%
  }
  \addto\noextrasfrench{%
  \bbl@deactivate{:}\bbl@deactivate{;}%
  \bbl@deactivate{!}\bbl@deactivate{?}}
\fi
%    \end{macrocode}
%
%  \subsection{French quotation marks}
%
%  \begin{macro}{\og}
%  \begin{macro}{\fg}
%    The top macros for quotation marks will be called |\og|
%    (``\underline{o}uvrez \underline{g}uillemets'') and |\fg|
%    (``\underline{f}ermez \underline{g}uillemets'').
%    Another option for typesetting quotes in multilingual texts
%    is to use the package |csquotes.sty| and its command |\enquote|.
%
%    \begin{macrocode}
\newcommand*{\og}{\@empty}
\newcommand*{\fg}{\@empty}
%    \end{macrocode}
%  \end{macro}
%  \end{macro}
%
%  \begin{macro}{\guillemotleft}
%  \begin{macro}{\guillemotright}
%    \LaTeX{} users are supposed to use 8-bit output encodings (T1,
%    LY1,\dots) to typeset French, those who still stick to OT1 should
%    call |aeguill.sty| or a similar package. In both cases the
%    commands |\guillemotleft| and |\guillemotright| will print the
%    French opening and closing quote characters from the output font.
%    For XeLaTeX, |\guillemotleft| and |\guillemotright| are defined
%    by package \file{xunicode.sty}.
%    We will check `AtBeginDocument' that the proper output encodings
%    are in use (see end of section~\ref{sec-keyval}).
%
%    We give the following definitions for Plain users only as a (poor)
%    fall-back, they are welcome to change them for anything better.
%    \begin{macrocode}
\ifLaTeXe
\else
  \ifx\guillemotleft\@undefined
    \def\guillemotleft{\leavevmode\raise0.25ex
                       \hbox{$\scriptscriptstyle\ll$}}
  \fi
  \ifx\guillemotright\@undefined
    \def\guillemotright{\raise0.25ex
                        \hbox{$\scriptscriptstyle\gg$}}
  \fi
  \let\xspace\relax
\fi
%    \end{macrocode}
%  \end{macro}
%  \end{macro}
%
%    The next step is to provide correct spacing after |\guillemotleft|
%    and before |\guillemotright|: a space precedes and follows
%    quotation marks but no line break is allowed neither \emph{after}
%    the opening one, nor \emph{before} the closing one.
%    |\FBguill@spacing| which does the spacing, has been fine tuned by
%    Thierry Bouche.  French quotes (including spacing) are printed by
%    |\FB@og| and |\FB@fg|, the expansion of the top level commands
%    |\og| and |\og| is different in and outside French.
%    We'll try to be smart to users of David~Carlisle's |xspace|
%    package: if this package is loaded there will be no need for |{}|
%    or |\ | to get a space after |\fg|, otherwise |\xspace| will be
%    defined as |\relax| (done at the end of this file).
%
%    \begin{macrocode}
\newcommand*{\FBguill@spacing}{\penalty\@M\hskip.8\fontdimen2\font
                                            plus.3\fontdimen3\font
                                           minus.8\fontdimen4\font}
\DeclareRobustCommand*{\FB@og}{\leavevmode
                               \guillemotleft\FBguill@spacing}
\DeclareRobustCommand*{\FB@fg}{\ifdim\lastskip>\z@\unskip\fi
                               \FBguill@spacing\guillemotright\xspace}
%    \end{macrocode}
%
%    The top level definitions for French quotation marks are switched
%    on and off through the |\extrasfrench| |\noextrasfrench|
%    mechanism. Outside French, |\og| and |\fg| will typeset standard
%    English opening and closing double quotes.
%
%    \begin{macrocode}
\ifLaTeXe
  \def\bbl@frenchguillemets{\renewcommand*{\og}{\FB@og}%
                            \renewcommand*{\fg}{\FB@fg}}
  \def\bbl@nonfrenchguillemets{\renewcommand*{\og}{\textquotedblleft}%
            \renewcommand*{\fg}{\ifdim\lastskip>\z@\unskip\fi
                                   \textquotedblright}}
\else
   \def\bbl@frenchguillemets{\let\og\FB@og
                             \let\fg\FB@fg}
   \def\bbl@nonfrenchguillemets{\def\og{``}%
                     \def\fg{\ifdim\lastskip>\z@\unskip\fi ''}}
\fi
\expandafter\addto\csname extras\CurrentOption\endcsname{%
  \bbl@frenchguillemets}
\expandafter\addto\csname noextras\CurrentOption\endcsname{%
  \bbl@nonfrenchguillemets}
%    \end{macrocode}
%
%  \subsection{Date in French}
%
% \begin{macro}{\datefrench}
%    The macro |\datefrench| redefines the command |\today| to
%    produce French dates.
%
% \changes{v2.0}{2006/11/06}{2 '\cs{relax}' added in
%    \cs{today}'s definition.}
%
% \changes{v2.1a}{2008/03/25}{\cs{today} changed (correction in 2.0
%    was wrong: \cs{today} was printed without spaces in toc).}
%
%    \begin{macrocode}
\@namedef{date\CurrentOption}{%
  \def\today{{\number\day}\ifnum1=\day {\ier}\fi \space
    \ifcase\month
      \or janvier\or f\'evrier\or mars\or avril\or mai\or juin\or
      juillet\or ao\^ut\or septembre\or octobre\or novembre\or
      d\'ecembre\fi
    \space \number\year}}
%    \end{macrocode}
% \end{macro}
%
%  \subsection{Extra utilities}
%
%    Let's provide the French user with some extra utilities.
%
% \changes{v2.1a}{2008/03/24}{Command \cs{fup} added to produce
%    better superscripts than \cs{textsuperscript}.}
%
%  \begin{macro}{\up}
%
% \changes{v2.1c}{2008/04/29}{Provide a temporary definition
%    (hyperref safe) of \cs{up} in case it has to be expanded in
%    the preamble (by beamer's \cs{title} command for instance).}
%
%  \begin{macro}{\fup}
%
% \changes{v2.1b}{2008/04/02}{Command \cs{fup} changed to use
%    real superscripts from fourier v. 1.6.}
%
% \changes{v2.2a}{2008/05/08}{\cs{newif} and \cs{newdimen} moved
%    before \cs{ifLaTeXe} to avoid an error with plainTeX.}
%
% \changes{v2.3a}{2008/09/30}{\cs{lowercase} changed to
%    \cs{MakeLowercase} as the former doesn't work for non ASCII
%    characters in encodings like applemac, utf-8,\dots}
%
%    |\up| eases the typesetting of superscripts like
%    `1\textsuperscript{er}'.  Up to version 2.0 of |frenchb| |\up| was
%    just a shortcut for |\textsuperscript| in \LaTeXe, but several
%    users complained that |\textsuperscript| typesets superscripts
%    too high and too big, so we now define |\fup| as an attempt to
%    produce better looking superscripts.  |\up| is defined as |\fup|
%    but can be redefined by |\frenchbsetup{FrenchSuperscripts=false}|
%    as |\textsuperscript| for compatibility with previous versions.
%
%    When a font has built-in superscripts, the best thing to do is
%    to just use them, otherwise |\fup| has to simulate superscripts
%    by scaling and raising ordinary letters.  Scaling is done using
%    package \file{scalefnt} which will be loaded at the end of
%    \babel's loading (|frenchb| being an option of babel, it cannot
%    load a package while being read).
%
%    \begin{macrocode}
\newif\ifFB@poorman
\newdimen\FB@Mht
\ifLaTeXe
  \AtEndOfPackage{\RequirePackage{scalefnt}}
%    \end{macrocode}
%    |\FB@up@fake| holds the definition of fake superscripts.
%    The scaling ratio is 0.65, raising is computed to put the top of
%    lower case letters (like `m') just under the top  of upper case
%    letters (like `M'), precisely 12\% down.  The chosen settings
%    look correct for most fonts, but can be tuned by the end-user
%    if necessary by changing |\FBsupR| and |\FBsupS| commands.
%
%    |\FB@lc| is defined as |\MakeLowercase| to inhibit the uppercasing
%    of superscripts (this may happen in page headers with the standard
%    classes but is wrong); |\FB@lc| can be redefined to do nothing
%    by option |LowercaseSuperscripts=false| of |\frenchbsetup{}|.
%    \begin{macrocode}
  \newcommand*{\FBsupR}{-0.12}
  \newcommand*{\FBsupS}{0.65}
  \newcommand*{\FB@lc}[1]{\MakeLowercase{#1}}
  \DeclareRobustCommand*{\FB@up@fake}[1]{%
    \settoheight{\FB@Mht}{M}%
    \addtolength{\FB@Mht}{\FBsupR \FB@Mht}%
    \addtolength{\FB@Mht}{-\FBsupS ex}%
    \raisebox{\FB@Mht}{\scalefont{\FBsupS}{\FB@lc{#1}}}%
    }
%    \end{macrocode}
%    The only packages I currently know to take advantage of real
%    superscripts are a) \file{xltxtra} used in conjunction with
%    XeLaTeX and OpenType fonts having the font feature
%    'VerticalPosition=Superior' (\file{xltxtra} defines
%    |\realsuperscript| and |\fakesuperscript|) and b) \file{fourier}
%    (from version 1.6) when Expert Utopia fonts are available.
%
%    |\FB@up| checks whether the current font is a Type1 `Expert'
%    (or `Pro') font with real superscripts or not (the code works
%    currently only with \file{fourier-1.6} but could work with any
%    Expert Type1 font with built-in superscripts, see below), and
%    decides to use real or fake superscripts.
%    It works as follows: the content of |\f@family| (family name of
%    the current font) is split by |\FB@split| into two pieces, the
%    first three characters (`|fut|' for Fourier, `|ppl|' for Adobe's
%    Palatino, \dots) stored in |\FB@firstthree| and the rest stored
%    in |\FB@suffix| which is expected to be `|x|' or `|j|' for expert
%    fonts.
%    \begin{macrocode}
  \def\FB@split#1#2#3#4\@nil{\def\FB@firstthree{#1#2#3}%
                             \def\FB@suffix{#4}}
  \def\FB@x{x}
  \def\FB@j{j}
  \DeclareRobustCommand*{\FB@up}[1]{%
    \bgroup \FB@poormantrue
      \expandafter\FB@split\f@family\@nil
%    \end{macrocode}
%    Then |\FB@up| looks for a \file{.fd} file named \file{t1fut-sup.fd}
%    (Fourier) or \file{t1ppl-sup.fd} (Palatino), etc. supposed to
%    define the subfamily (|fut-sup| or |ppl-sup|, etc.) giving access
%    to the built-in superscripts.  If the \file{.fd} file is not found
%    by |\IfFileExists|, |\FB@up| falls back on fake superscripts,
%    otherwise |\FB@suffix| is checked to decide whether to use fake or
%    real superscripts.
%    \begin{macrocode}
      \edef\reserved@a{\lowercase{%
         \noexpand\IfFileExists{\f@encoding\FB@firstthree -sup.fd}}}%
      \reserved@a
        {\ifx\FB@suffix\FB@x \FB@poormanfalse\fi
         \ifx\FB@suffix\FB@j \FB@poormanfalse\fi
         \ifFB@poorman \FB@up@fake{#1}%
         \else         \FB@up@real{#1}%
         \fi}%
        {\FB@up@fake{#1}}%
    \egroup}
%    \end{macrocode}
%    |\FB@up@real| just picks up the superscripts from the subfamily
%    (and forces lowercase).
%    \begin{macrocode}
  \newcommand*{\FB@up@real}[1]{\bgroup
       \fontfamily{\FB@firstthree -sup}\selectfont \FB@lc{#1}\egroup}
%    \end{macrocode}
%    |\fup| is now defined as |\FB@up| unless |\realsuperscript| is
%    defined (occurs with XeLaTeX calling \file{xltxtra.sty}).
%    \begin{macrocode}
  \DeclareRobustCommand*{\fup}[1]{%
    \@ifundefined{realsuperscript}%
      {\FB@up{#1}}%
      {\bgroup\let\fakesuperscript\FB@up@fake
            \realsuperscript{\FB@lc{#1}}\egroup}}
%    \end{macrocode}
%    Temporary definition of |up| (redefined `AtBeginDocument').
%    \begin{macrocode}
  \newcommand*{\up}{\relax}
%    \end{macrocode}
%    Poor man's definition of |\up| for Plain. In \LaTeXe,
%    |\up| will be defined as |\fup| or |\textsuperscript| later on
%    while processing the options of |\frenchbsetup{}|.
%    \begin{macrocode}
\else
  \newcommand*{\up}[1]{\leavevmode\raise1ex\hbox{\sevenrm #1}}
\fi
%    \end{macrocode}
%  \end{macro}
%  \end{macro}
%
%  \begin{macro}{\ieme}
%  \begin{macro}{\ier}
%  \begin{macro}{\iere}
%  \begin{macro}{\iemes}
%  \begin{macro}{\iers}
%  \begin{macro}{\ieres}
%  Some handy macros for those who don't know how to abbreviate ordinals:
%    \begin{macrocode}
\def\ieme{\up{\lowercase{e}}\xspace}
\def\iemes{\up{\lowercase{es}}\xspace}
\def\ier{\up{\lowercase{er}}\xspace}
\def\iers{\up{\lowercase{ers}}\xspace}
\def\iere{\up{\lowercase{re}}\xspace}
\def\ieres{\up{\lowercase{res}}\xspace}
%    \end{macrocode}
%  \end{macro}
%  \end{macro}
%  \end{macro}
%  \end{macro}
%  \end{macro}
%  \end{macro}
%
% \changes{v2.1c}{2008/04/29}{Added commands \cs{Nos} and \cs{nos}.}
%
%  \begin{macro}{\No}
%  \begin{macro}{\no}
%  \begin{macro}{\Nos}
%  \begin{macro}{\nos}
%  \begin{macro}{\primo}
%  \begin{macro}{\fprimo)}
%    And some more macros relying on |\up| for numbering,
%    first two support macros.
%    \begin{macrocode}
\newcommand*{\FrenchEnumerate}[1]{%
                       #1\up{\lowercase{o}}\kern+.3em}
\newcommand*{\FrenchPopularEnumerate}[1]{%
                       #1\up{\lowercase{o}})\kern+.3em}
%    \end{macrocode}
%
%    Typing |\primo| should result in `$1^{\rm o}$\kern+.3em',
%    \begin{macrocode}
\def\primo{\FrenchEnumerate1}
\def\secundo{\FrenchEnumerate2}
\def\tertio{\FrenchEnumerate3}
\def\quarto{\FrenchEnumerate4}
%    \end{macrocode}
%    while typing |\fprimo)| gives `1$^{\rm o}$)\kern+.3em.
%    \begin{macrocode}
\def\fprimo){\FrenchPopularEnumerate1}
\def\fsecundo){\FrenchPopularEnumerate2}
\def\ftertio){\FrenchPopularEnumerate3}
\def\fquarto){\FrenchPopularEnumerate4}
%    \end{macrocode}
%
%    Let's provide four macros for the common abbreviations
%    of ``Num\'ero''.
%    \begin{macrocode}
\DeclareRobustCommand*{\No}{N\up{\lowercase{o}}\kern+.2em}
\DeclareRobustCommand*{\no}{n\up{\lowercase{o}}\kern+.2em}
\DeclareRobustCommand*{\Nos}{N\up{\lowercase{os}}\kern+.2em}
\DeclareRobustCommand*{\nos}{n\up{\lowercase{os}}\kern+.2em}
%    \end{macrocode}
%  \end{macro}
%  \end{macro}
%  \end{macro}
%  \end{macro}
%  \end{macro}
%  \end{macro}
%
%  \begin{macro}{\bsc}
%    As family names should be written in small capitals and never be
%    hyphenated, we provide a command (its name comes from Boxed Small
%    Caps) to input them easily.  Note that this command has changed
%    with version~2 of |frenchb|: a |\kern0pt| is used instead of |\hbox|
%    because |\hbox| would break microtype's font expansion; as a
%    (positive?) side effect, composed names (such as Dupont-Durand)
%    can now be hyphenated on explicit hyphens.
%    Usage: |Jean~\bsc{Duchemin}|.
%
% \changes{v2.0}{2006/11/06}{\cs{hbox} dropped, replaced by
%    \cs{kern0pt}.}
%
%    \begin{macrocode}
\DeclareRobustCommand*{\bsc}[1]{\leavevmode\begingroup\kern0pt
                                           \scshape #1\endgroup}
\ifLaTeXe\else\let\scshape\relax\fi
%    \end{macrocode}
%  \end{macro}
%
%    Some definitions for special characters.  We won't define |\tilde|
%    as a Text Symbol not to conflict with the macro |\tilde| for math
%    mode and use the name |\tild| instead. Note that |\boi| may
%    \emph{not} be used in math mode, its name in math mode is
%    |\backslash|.  |\degre|  can be accessed by the command |\r{}|
%    for ring accent.
%
%    \begin{macrocode}
\ifLaTeXe
  \DeclareTextSymbol{\at}{T1}{64}
  \DeclareTextSymbol{\circonflexe}{T1}{94}
  \DeclareTextSymbol{\tild}{T1}{126}
  \DeclareTextSymbolDefault{\at}{T1}
  \DeclareTextSymbolDefault{\circonflexe}{T1}
  \DeclareTextSymbolDefault{\tild}{T1}
  \DeclareRobustCommand*{\boi}{\textbackslash}
  \DeclareRobustCommand*{\degre}{\r{}}
\else
  \def\T@one{T1}
  \ifx\f@encoding\T@one
    \newcommand*{\degre}{\char6}
  \else
    \newcommand*{\degre}{\char23}
  \fi
  \newcommand*{\at}{\char64}
  \newcommand*{\circonflexe}{\char94}
  \newcommand*{\tild}{\char126}
  \newcommand*{\boi}{$\backslash$}
\fi
%    \end{macrocode}
%
%  \begin{macro}{\degres}
%    We now define a macro |\degres| for typesetting the abbreviation
%    for `degrees' (as in `degrees Celsius'). As the bounding box of
%    the character `degree' has \emph{very} different widths in CM/EC
%    and PostScript fonts, we fix the width of the bounding box of
%    |\degres| to 0.3\,em, this lets the symbol `degree' stick to the
%    preceding (e.g., |45\degres|) or following character
%    (e.g., |20~\degres C|).
%
%    If the \TeX{} Companion fonts are available (\file{textcomp.sty}),
%    we pick up |\textdegree| from them instead of using emulating
%    `degrees' from the |\r{}| accent. Otherwise we overwrite the
%    (poor) definition of |\textdegree| given in \file{latin1.def},
%    \file{applemac.def} etc. (called by  \file{inputenc.sty}) by
%    our definition of |\degres|. We also advice the user (once only)
%    to use TS1-encoding.
%
% \changes{v2.1c}{2008/04/29}{Provide a temporary definition (hyperref
%    safe) of \cs{degres} in case it has to be expanded in the preamble
%    (by beamer's \cs{title} command for instance).}
%
%    \begin{macrocode}
\ifLaTeXe
  \newcommand*{\degres}{\degre}
  \def\Warning@degree@TSone{%
        \PackageWarning{frenchb.ldf}{%
           Degrees would look better in TS1-encoding:
           \MessageBreak add \protect
           \usepackage{textcomp} to the preamble.
           \MessageBreak Degrees used}}
  \AtBeginDocument{\expandafter\ifx\csname M@TS1\endcsname\relax
                     \DeclareRobustCommand*{\degres}{%
                       \leavevmode\hbox to 0.3em{\hss\degre\hss}%
                       \Warning@degree@TSone
                       \global\let\Warning@degree@TSone\relax}%
                      \let\textdegree\degres
                   \else
                     \DeclareRobustCommand*{\degres}{%
                         \hbox{\UseTextSymbol{TS1}{\textdegree}}}%
                   \fi}
\else
  \newcommand*{\degres}{%
    \leavevmode\hbox to 0.3em{\hss\degre\hss}}
\fi
%    \end{macrocode}
%  \end{macro}
%
%  \subsection{Formatting numbers}
%  \label{sec-numbers}
%
%  \begin{macro}{\DecimalMathComma}
%  \begin{macro}{\StandardMathComma}
%    As mentioned in the \TeX{}book p.~134, the comma is of type
%    |\mathpunct| in math mode: it is automatically followed by a
%    space. This is convenient in lists and intervals but
%    unpleasant when the comma is used as a decimal separator
%    in French: it has to be entered as |{,}|.
%    |\DecimalMathComma| makes the comma be an ordinary character
%    (of type |\mathord|) in French \emph{only} (no space added);
%    |\StandardMathComma| switches back to the standard behaviour
%    of the comma.
%    \begin{macrocode}
\newcount\std@mcc
\newcount\dec@mcc
\std@mcc=\mathcode`\,
\dec@mcc=\std@mcc
\@tempcnta=\std@mcc
\divide\@tempcnta by "1000
\multiply\@tempcnta by "1000
\advance\dec@mcc by -\@tempcnta
\newcommand*{\DecimalMathComma}{\iflanguage{french}%
                                 {\mathcode`\,=\dec@mcc}{}%
              \addto\extrasfrench{\mathcode`\,=\dec@mcc}}
\newcommand*{\StandardMathComma}{\mathcode`\,=\std@mcc
             \addto\extrasfrench{\mathcode`\,=\std@mcc}}
\expandafter\addto\csname noextras\CurrentOption\endcsname{%
   \mathcode`\,=\std@mcc}
%    \end{macrocode}
%  \end{macro}
%  \end{macro}
%
%  \begin{macro}{\nombre}
%
% \changes{v2.0}{2006/11/06}{\cs{nombre} requires now numprint.sty.}
%
%    The command |\nombre| is now borrowed from |numprint.sty| for
%    \LaTeXe.  There is no point to maintain the former tricky code
%    when a package is dedicated to do the same job and more.
%    For Plain based formats, |\nombre| no longer formats numbers,
%    it prints them as is and issues a warning about the change.
%
%    Fake command |\nombre| for Plain based formats, warning users of
%    |frenchb| v.1.x. of the change.
%    \begin{macrocode}
\newcommand*{\nombre}[1]{{#1}\message{%
     *** \noexpand\nombre no longer formats numbers\string! ***}}%
%    \end{macrocode}
%  \end{macro}
%
%    The next definitions only make sense for \LaTeXe. Let's cleanup
%    and exit if the format in Plain based.
%
%    \begin{macrocode}
\let\FBstop@here\relax
\def\FBclean@on@exit{\let\ifLaTeXe\@undefined
                     \let\LaTeXetrue\@undefined
                     \let\LaTeXefalse\@undefined}
\ifx\magnification\@undefined
\else
   \def\FBstop@here{\let\STD@makecaption\relax
                    \FBclean@on@exit
                    \ldf@quit\CurrentOption\endinput}
\fi
\FBstop@here
%    \end{macrocode}
%
%    What follows now is for \LaTeXe{} \emph{only}.
%    We redefine |\nombre| for \LaTeXe. A warning is issued
%    at the first call of |\nombre| if |\numprint| is not
%    defined, suggesting what to do.  The package |numprint|
%    is \emph{not} loaded automatically by |frenchb| because of
%    possible options conflict.
%
%    \begin{macrocode}
\renewcommand*{\nombre}[1]{\Warning@nombre\numprint{#1}}
\newcommand*{\Warning@nombre}{%
   \@ifundefined{numprint}%
      {\PackageWarning{frenchb.ldf}{%
         \protect\nombre\space now relies on package numprint.sty,
         \MessageBreak add \protect
         \usepackage[autolanguage]{numprint}\MessageBreak
         to your preamble *after* loading babel, \MessageBreak
         see file numprint.pdf for other options.\MessageBreak
         \protect\nombre\space called}%
       \global\let\Warning@nombre\relax
       \global\let\numprint\relax
      }{}%
}
%    \end{macrocode}
%
% \changes{v2.0c}{2007/06/25}{There is no need to define here
%    numprint's command \cs{npstylefrench}, it will be redefined
%    `AtBeginDocument' by \cs{FBprocess@options}.}
%
% \changes{v2.0c}{2007/06/25}{\cs{ThinSpaceInFrenchNumbers} added
%     for compatibility with frenchb-1.x.}
%
%    \begin{macrocode}
\newcommand*{\ThinSpaceInFrenchNumbers}{%
   \PackageWarning{frenchb.ldf}{%
         Type \protect\frenchbsetup{ThinSpaceInFrenchNumbers}
         \MessageBreak Command \protect\ThinSpaceInFrenchNumbers\space
         is no longer\MessageBreak  defined in frenchb v.2,}}
%    \end{macrocode}
%
%  \subsection{Caption names}
%
%    The next step consists of defining the French equivalents for
%    the \LaTeX{} caption names.
%
% \begin{macro}{\captionsfrench}
%    Let's first define  |\captionsfrench| which sets all strings used
%    in the four standard document classes provided with \LaTeX.
%
% \changes{v2.0}{2006/11/06}{`Fig.' changed to `Figure' and
%     `Tab.' to `Table'.}
%
% \changes{v2.0}{2006/12/15}{Set \cs{CaptionSeparator} in
%     \cs{extrasfrench} now instead of \cs{captionsfrench}
%     because it has to be reset when leaving French.}
%
%    \begin{macrocode}
\@namedef{captions\CurrentOption}{%
   \def\refname{R\'ef\'erences}%
   \def\abstractname{R\'esum\'e}%
   \def\bibname{Bibliographie}%
   \def\prefacename{Pr\'eface}%
   \def\chaptername{Chapitre}%
   \def\appendixname{Annexe}%
   \def\contentsname{Table des mati\`eres}%
   \def\listfigurename{Table des figures}%
   \def\listtablename{Liste des tableaux}%
   \def\indexname{Index}%
   \def\figurename{{\scshape Figure}}%
   \def\tablename{{\scshape Table}}%
%    \end{macrocode}
%   ``Premi\`ere partie'' instead of ``Part I''.
%    \begin{macrocode}
   \def\partname{\protect\@Fpt partie}%
   \def\@Fpt{{\ifcase\value{part}\or Premi\`ere\or Deuxi\`eme\or
   Troisi\`eme\or Quatri\`eme\or Cinqui\`eme\or Sixi\`eme\or
   Septi\`eme\or Huiti\`eme\or Neuvi\`eme\or Dixi\`eme\or Onzi\`eme\or
   Douzi\`eme\or Treizi\`eme\or Quatorzi\`eme\or Quinzi\`eme\or
   Seizi\`eme\or Dix-septi\`eme\or Dix-huiti\`eme\or Dix-neuvi\`eme\or
   Vingti\`eme\fi}\space\def\thepart{}}%
   \def\pagename{page}%
   \def\seename{{\emph{voir}}}%
   \def\alsoname{{\emph{voir aussi}}}%
   \def\enclname{P.~J. }%
   \def\ccname{Copie \`a }%
   \def\headtoname{}%
   \def\proofname{D\'emonstration}%
   \def\glossaryname{Glossaire}%
   }
%    \end{macrocode}
% \end{macro}
%
%    As some users who choose |frenchb| or |francais| as option of
%    \babel, might customise |\captionsfrenchb| or |\captionsfrancais|
%    in the preamble, we merge their changes at the |\begin{document}|
%    when they do so.
%    The other variants of French (canadien, acadian) are defined by
%    checking if the relevant option was used and then adding one extra
%    level of expansion.
%
%    \begin{macrocode}
\AtBeginDocument{\let\captions@French\captionsfrench
                 \@ifundefined{captionsfrenchb}%
                    {\let\captions@Frenchb\relax}%
                    {\let\captions@Frenchb\captionsfrenchb}%
                 \@ifundefined{captionsfrancais}%
                    {\let\captions@Francais\relax}%
                    {\let\captions@Francais\captionsfrancais}%
                 \def\captionsfrench{\captions@French
                        \captions@Francais\captions@Frenchb}%
                 \def\captionsfrancais{\captionsfrench}%
                 \def\captionsfrenchb{\captionsfrench}%
                 \iflanguage{french}{\captionsfrench}{}%
                }
\@ifpackagewith{babel}{canadien}{%
  \def\captionscanadien{\captionsfrench}%
  \def\datecanadien{\datefrench}%
  \def\extrascanadien{\extrasfrench}%
  \def\noextrascanadien{\noextrasfrench}%
  }{}
\@ifpackagewith{babel}{acadian}{%
  \def\captionsacadian{\captionsfrench}%
  \def\dateacadian{\datefrench}%
  \def\extrasacadian{\extrasfrench}%
  \def\noextrasacadian{\noextrasfrench}%
  }{}
%    \end{macrocode}
%
% \begin{macro}{\CaptionSeparator}
%    Let's consider now captions in figures and tables.
%    In French, captions in figures and tables should be printed with
%    endash (`--') instead of the standard `:'.
%
%    The standard definition of |\@makecaption| (e.g., the one provided
%    in article.cls, report.cls, book.cls which is frozen for \LaTeXe{}
%    according to Frank Mittelbach), has been saved in
%    |\STD@makecaption| before making `:' active
%    (see section~\ref{sec-punct}). `AtBeginDocument' we compare it to
%    its current definition (some classes like koma-script classes,
%    AMS classes, ua-thesis.cls\dots change it).
%    If they are identical, |frenchb| just adds a hook called
%    |\CaptionSeparator| to |\@makecaption|, |\CaptionSeparator|
%    defaults to `: ' as in the standard |\@makecaption|, and will be
%    changed to ` -- ' in French.
%    If the definitions differ, |frenchb| doesn't overwrite the changes,
%    but prints a message in the .log file.
%
%    \begin{macrocode}
\def\CaptionSeparator{\string:\space}
\long\def\FB@makecaption#1#2{%
  \vskip\abovecaptionskip
  \sbox\@tempboxa{#1\CaptionSeparator #2}%
  \ifdim \wd\@tempboxa >\hsize
    #1\CaptionSeparator #2\par
  \else
    \global \@minipagefalse
    \hb@xt@\hsize{\hfil\box\@tempboxa\hfil}%
  \fi
  \vskip\belowcaptionskip}
\AtBeginDocument{%
  \ifx\@makecaption\STD@makecaption
      \global\let\@makecaption\FB@makecaption
  \else
    \@ifundefined{@makecaption}{}%
       {\PackageWarning{frenchb.ldf}%
        {The definition of \protect\@makecaption\space
         has been changed,\MessageBreak
         frenchb will NOT customise it;\MessageBreak reported}%
       }%
  \fi
  \let\FB@makecaption\relax
  \let\STD@makecaption\relax
}
\expandafter\addto\csname extras\CurrentOption\endcsname{%
   \def\CaptionSeparator{\space\textendash\space}}
\expandafter\addto\csname noextras\CurrentOption\endcsname{%
    \def\CaptionSeparator{\string:\space}}
%    \end{macrocode}
% \end{macro}
%
%  \subsection{French lists}
%  \label{sec-lists}
%
%  \begin{macro}{\listFB}
%  \begin{macro}{\listORI}
%    Vertical spacing in general lists should be shorter in French
%    texts than the defaults provided by \LaTeX.
%    Note that the easy way, just changing values of vertical spacing
%    parameters when entering French and restoring them to their
%    defaults on exit would not work; as most lists are based on
%    |\list| we will define a variant of |\list| (|\listFB|) to
%    be used in French.
%
%    The amount of vertical space before and after a list is given by
%    |\topsep| + |\parskip| (+ |\partopsep| if the list starts a new
%    paragraph). IMHO, |\parskip| should be added \emph{only} when
%    the list starts a new paragraph, so I subtract |\parskip| from
%    |\topsep| and add it back to |\partopsep|; this will normally
%    make no difference because |\parskip|'s default value is 0pt, but
%    will be noticeable when |\parskip| is \emph{not} null.
%
%    |\endlist| is not redefined, but |\endlistORI| is provided for
%    the users who prefer to define their own lists from the original
%    command, they can code: |\begin{listORI}{}{} \end{listORI}|.
%    \begin{macrocode}
\let\listORI\list
\let\endlistORI\endlist
\def\FB@listsettings{%
      \setlength{\itemsep}{0.4ex plus 0.2ex minus 0.2ex}%
      \setlength{\parsep}{0.4ex plus 0.2ex minus 0.2ex}%
      \setlength{\topsep}{0.8ex plus 0.4ex minus 0.4ex}%
      \setlength{\partopsep}{0.4ex plus 0.2ex minus 0.2ex}%
%    \end{macrocode}
%    |\parskip| is of type `skip', its mean value only (\emph{not
%    the glue}) should be subtracted from |\topsep| and added to
%    |\partopsep|, so convert |\parskip| to a `dimen' using
%    |\@tempdima|.
%    \begin{macrocode}
      \@tempdima=\parskip
      \addtolength{\topsep}{-\@tempdima}%
      \addtolength{\partopsep}{\@tempdima}}%
\def\listFB#1#2{\listORI{#1}{\FB@listsettings #2}}%
\let\endlistFB\endlist
%    \end{macrocode}
%  \end{macro}
%  \end{macro}
%
%  \begin{macro}{\itemizeFB}
%  \begin{macro}{\itemizeORI}
%  \begin{macro}{\bbl@frenchlabelitems}
%  \begin{macro}{\bbl@nonfrenchlabelitems}
%    Let's now consider French itemize lists.  They differ from those
%    provided by the standard \LaTeXe{} classes:
%    \begin{itemize}
%      \item vertical spacing between items, before and after
%         the list, should be \emph{null} with \emph{no glue} added;
%      \item the item labels of a first level list should be vertically
%          aligned on the paragraph's first character (i.e. at
%          |\parindent| from the left margin);
%      \item the `\textbullet' is never used in French itemize-lists,
%          a long dash `--' is preferred for all levels. The item label
%          used in French is stored in |\FrenchLabelItem}|, it defaults
%          to `--' and can be changed using |\frenchbsetup{}| (see
%          section~\ref{sec-keyval}).
%    \end{itemize}
%
%    \begin{macrocode}
\newcommand*{\FrenchLabelItem}{\textendash}
\newcommand*{\Frlabelitemi}{\FrenchLabelItem}
\newcommand*{\Frlabelitemii}{\FrenchLabelItem}
\newcommand*{\Frlabelitemiii}{\FrenchLabelItem}
\newcommand*{\Frlabelitemiv}{\FrenchLabelItem}
%    \end{macrocode}
%    |\bbl@frenchlabelitems| saves current itemize labels and changes
%    them to their value in French. This code should never be executed
%    twice in a row, so we need a new flag that will be set and reset
%    by |\bbl@nonfrenchlabelitems| and |\bbl@frenchlabelitems|.
%    \begin{macrocode}
\newif\ifFB@enterFrench  \FB@enterFrenchtrue
\def\bbl@frenchlabelitems{%
  \ifFB@enterFrench
    \let\@ltiORI\labelitemi
    \let\@ltiiORI\labelitemii
    \let\@ltiiiORI\labelitemiii
    \let\@ltivORI\labelitemiv
    \let\labelitemi\Frlabelitemi
    \let\labelitemii\Frlabelitemii
    \let\labelitemiii\Frlabelitemiii
    \let\labelitemiv\Frlabelitemiv
    \FB@enterFrenchfalse
  \fi
}
\let\itemizeORI\itemize
\let\enditemizeORI\enditemize
\let\enditemizeFB\enditemize
\def\itemizeFB{%
    \ifnum \@itemdepth >\thr@@\@toodeep\else
      \advance\@itemdepth\@ne
      \edef\@itemitem{labelitem\romannumeral\the\@itemdepth}%
      \expandafter
      \listORI
      \csname\@itemitem\endcsname
      {\settowidth{\labelwidth}{\csname\@itemitem\endcsname}%
       \setlength{\leftmargin}{\labelwidth}%
       \addtolength{\leftmargin}{\labelsep}%
       \ifnum\@listdepth=0
         \setlength{\itemindent}{\parindent}%
       \else
         \addtolength{\leftmargin}{\parindent}%
       \fi
       \setlength{\itemsep}{\z@}%
       \setlength{\parsep}{\z@}%
       \setlength{\topsep}{\z@}%
       \setlength{\partopsep}{\z@}%
%    \end{macrocode}
%    |\parskip| is of type `skip', its mean value only (\emph{not
%    the glue}) should be subtracted from |\topsep| and added to
%    |\partopsep|, so convert |\parskip| to a `dimen' using
%    |\@tempdima|.
%    \begin{macrocode}
       \@tempdima=\parskip
       \addtolength{\topsep}{-\@tempdima}%
       \addtolength{\partopsep}{\@tempdima}}%
    \fi}
%    \end{macrocode}
%    The user's changes in labelitems are saved when leaving French for
%    further use when switching back to French.  This code should never
%    be executed twice in a row (toggle with |\bbl@frenchlabelitems|).
%    \begin{macrocode}
\def\bbl@nonfrenchlabelitems{%
  \ifFB@enterFrench
  \else
      \let\Frlabelitemi\labelitemi
      \let\Frlabelitemii\labelitemii
      \let\Frlabelitemiii\labelitemiii
      \let\Frlabelitemiv\labelitemiv
      \let\labelitemi\@ltiORI
      \let\labelitemii\@ltiiORI
      \let\labelitemiii\@ltiiiORI
      \let\labelitemiv\@ltivORI
      \FB@enterFrenchtrue
  \fi
}
%    \end{macrocode}
%  \end{macro}
%  \end{macro}
%  \end{macro}
%  \end{macro}
%
%  \subsection{French indentation of sections}
%  \label{sec-indent}
%
%  \begin{macro}{\bbl@frenchindent}
%  \begin{macro}{\bbl@nonfrenchindent}
%    In French the first paragraph of each section should be indented,
%    this is another difference with US-English. This is controlled by
%    the flag |\if@afterindent|.
%
% \changes{v2.3d}{2009/03/16}{Bug correction: previous versions of
%    frenchb set the flag \cs{if@afterindent} to false outside
%    French which is correct for English but wrong for some languages
%    like Spanish.  Pointed out by Juan Jos\'e Torrens.}
%
%    We need to save the value of the flag |\if@afterindent|
%    `AtBeginDocument' before eventually changing its value.
%
%    \begin{macrocode}
\AtBeginDocument{\ifx\@afterindentfalse\@afterindenttrue
                       \let\@aifORI\@afterindenttrue
                 \else \let\@aifORI\@afterindentfalse
                 \fi
}
\def\bbl@frenchindent{\let\@afterindentfalse\@afterindenttrue
                      \@afterindenttrue}
\def\bbl@nonfrenchindent{\let\@afterindentfalse\@aifORI
                         \@afterindentfalse}
%    \end{macrocode}
%  \end{macro}
%  \end{macro}
%
%  \subsection{Formatting footnotes}
%  \label{sec-footnotes}
%
% \changes{v2.0}{2006/11/06}{Footnotes are now printed
%     by default `\`a la fran\c caise' for the whole document.}
%
% \changes{v2.0b}{2007/04/18}{Footnotes: Just do nothing
%    (except warning) when the bigfoot package is loaded.}
%
%    The |bigfoot| package deeply changes the way footnotes are
%    handled. When |bigfoot| is loaded, we just warn the user that
%    |frenchb| will drop the customisation of footnotes.
%
%    The layout of footnotes is controlled by two flags
%    |\ifFBAutoSpaceFootnotes| and |\ifFBFrenchFootnotes| which are
%    set by options of |\frenchbsetup{}| (see section~\ref{sec-keyval}).
%    Notice that the layout of footnotes \emph{does not depend} on the
%    current language (just think of two footnotes on the same page
%    looking different because one was called in a French part, the
%    other one in English!).
%
%    When |\ifFBAutoSpaceFootnotes| is true, |\@footnotemark| (whose
%    definition is saved at the |\begin{document}| in order to include
%    any customisation that packages might have done) is redefined to
%    add a thin space before the number or symbol calling a footnote
%    (any space typed in is removed first).  This has no effect on
%    the layout of the footnote itself.
%
%    \begin{macrocode}
\AtBeginDocument{\@ifpackageloaded{bigfoot}%
                   {\PackageWarning{frenchb.ldf}%
                     {bigfoot package in use.\MessageBreak
                      frenchb will NOT customise footnotes;\MessageBreak
                      reported}}%
                   {\let\@footnotemarkORI\@footnotemark
                    \def\@footnotemarkFB{\leavevmode\unskip\unkern
                                         \,\@footnotemarkORI}%
                    \ifFBAutoSpaceFootnotes
                      \let\@footnotemark\@footnotemarkFB
                    \fi}%
                }
%    \end{macrocode}
%
%    We then define |\@makefntextFB|, a variant of |\@makefntext|
%    which is responsible for the layout of footnotes, to match the
%    specifications of the French `Imprimerie Nationale':  footnotes
%    will be indented by |\parindentFFN|, numbers (if any) typeset on
%    the baseline (instead of superscripts) and followed by a dot
%    and an half quad space. Whenever symbols are used to number
%    footnotes (as in |\thanks| for instance), we switch back to the
%    standard layout (the French layout of footnotes is meant for
%    footnotes numbered by Arabic or Roman digits).
%
% \changes{v2.0}{2006/11/06}{\cs{parindentFFN} not changed if
%    already defined (required by JA for cah-gut.cls).}
%
% \changes{v2.3b}{2008/12/06}{New commands \cs{dotFFN} and
%    \cs{kernFFN} for more flexibility (suggested by JA).}
%
%    The value of |\parindentFFN| will be redefined at the
%    |\begin{document}|, as the maximum of |\parindent| and 1.5em
%    \emph{unless} it has been set in the preamble (the weird value
%    10in is just for testing whether |\parindentFFN| has been set
%    or not).
%
%    \begin{macrocode}
\newcommand*{\dotFFN}{.}
\newcommand*{\kernFFN}{\kern .5em}
\newdimen\parindentFFN
\parindentFFN=10in
\def\ftnISsymbol{\@fnsymbol\c@footnote}
\long\def\@makefntextFB#1{\ifx\thefootnote\ftnISsymbol
                            \@makefntextORI{#1}%
                          \else
                            \parindent=\parindentFFN
                            \rule\z@\footnotesep
                            \setbox\@tempboxa\hbox{\@thefnmark}%
                            \ifdim\wd\@tempboxa>\z@
                              \llap{\@thefnmark}\dotFFN\kernFFN
                            \fi #1
                          \fi}%
%    \end{macrocode}
%
%    We save the standard definition of |\@makefntext| at the
%    |\begin{document}|, and then redefine |\@makefntext| according to
%    the value of flag |\ifFBFrenchFootnotes| (true or false).
%
%    \begin{macrocode}
\AtBeginDocument{\@ifpackageloaded{bigfoot}{}%
                  {\ifdim\parindentFFN<10in
                   \else
                      \parindentFFN=\parindent
                      \ifdim\parindentFFN<1.5em\parindentFFN=1.5em\fi
                   \fi
                   \let\@makefntextORI\@makefntext
                   \long\def\@makefntext#1{%
                      \ifFBFrenchFootnotes
                         \@makefntextFB{#1}%
                      \else
                         \@makefntextORI{#1}%
                      \fi}%
                  }%
                }
%    \end{macrocode}
%
%    For compatibility reasons, we provide definitions for the commands
%    dealing with the layout of footnotes in |frenchb| version~1.6.
%    |\frenchbsetup{}| (see in section \ref{sec-keyval}) should be
%    preferred for setting these options.  |\StandardFootnotes| may
%    still be used locally (in minipages for instance), that's why the
%    test |\ifFBFrenchFootnotes| is done inside |\@makefntext|.
%    \begin{macrocode}
\newcommand*{\AddThinSpaceBeforeFootnotes}{\FBAutoSpaceFootnotestrue}
\newcommand*{\FrenchFootnotes}{\FBFrenchFootnotestrue}
\newcommand*{\StandardFootnotes}{\FBFrenchFootnotesfalse}
%    \end{macrocode}
%
%  \subsection{Global layout}
%  \label{sec-global}
%
%    In multilingual documents, some typographic rules must depend
%    on the current language (e.g., hyphenation, typesetting of
%    numbers, spacing before double punctuation\dots), others should,
%    IMHO, be kept global to the document: especially the layout of
%    lists (see~\ref{sec-lists}) and footnotes
%    (see~\ref{sec-footnotes}), and the indentation of the first
%    paragraph of sections (see~\ref{sec-indent}).
%
%    From version 2.2 on, if |frenchb| is \babel's ``main language''
%    (i.e. last language option at \babel's loading), |frenchb|
%    customises the layout (i.e. lists, indentation of the first
%    paragraphs of sections and footnotes) in the whole document
%    regardless the current language.   On the other hand, if |frenchb|
%    is \emph{not} \babel's ``main language'', it leaves the layout
%    unchanged both in French and in other languages.
%
%  \begin{macro}{\FrenchLayout}
%  \begin{macro}{\StandardLayout}
%    The former commands |\FrenchLayout| and |\StandardLayout| are kept
%    for compatibility reasons but should no longer be used.
%
% \changes{v2.0g}{2008/03/23}{Flag \cs{ifFBStandardLayout} not checked
%     by \cs{FBprocess@options}, low-level flags have to be set
%     one by one.}
%
%    \begin{macrocode}
\newcommand*{\FrenchLayout}{%
    \FBGlobalLayoutFrenchtrue
    \PackageWarning{frenchb.ldf}%
    {\protect\FrenchLayout\space is obsolete.  Please use\MessageBreak
     \protect\frenchbsetup{GlobalLayoutFrench} instead.}%
}
\newcommand*{\StandardLayout}{%
  \FBReduceListSpacingfalse
  \FBCompactItemizefalse
  \FBStandardItemLabelstrue
  \FBIndentFirstfalse
  \FBFrenchFootnotesfalse
  \FBAutoSpaceFootnotesfalse
  \PackageWarning{frenchb.ldf}%
    {\protect\StandardLayout\space is obsolete.  Please use\MessageBreak
    \protect\frenchbsetup{StandardLayout} instead.}%
}
\@onlypreamble\FrenchLayout
\@onlypreamble\StandardLayout
%    \end{macrocode}
%  \end{macro}
%  \end{macro}
%
%  \subsection{Dots\dots}
%  \label{sec-dots}
%
%  \begin{macro}{\FBtextellipsis}
%    \LaTeXe's standard definition of |\dots| in text-mode is
%    |\textellipsis| which includes a |\kern| at the end;
%    this space is not wanted in some cases (before a closing brace
%    for instance) and |\kern| breaks hyphenation of the next word.
%    We define |\FBtextellipsis| for French (in \LaTeXe{} only).
%
%    The |\if| construction in the \LaTeXe{} definition of |\dots|
%    doesn't allow the use of |xspace| (|xspace| is always followed
%    by a |\fi|), so we use the AMS-\LaTeX{} construction of |\dots|;
%    this has to be done `AtBeginDocument' not to be overwritten
%    when \file{amsmath.sty} is loaded after \babel.
%
% \changes{v2.0}{2006/11/06}{Added special case for LY1 encoding,
%    see  bug report from Bruno Voisin (2004/05/18).}
%
%    LY1 has a ready made character for |\textellipsis|, it should be
%    used in French too (pointed out by Bruno Voisin).
%
%    \begin{macrocode}
\DeclareTextSymbol{\FBtextellipsis}{LY1}{133}
\DeclareTextCommandDefault{\FBtextellipsis}{%
    .\kern\fontdimen3\font.\kern\fontdimen3\font.\xspace}
%    \end{macrocode}
%    |\Mdots@| and |\Tdots@ORI| hold the definitions of |\dots| in
%    Math and Text mode. They default to those of amsmath-2.0, and
%    will revert to standard \LaTeX{} definitions `AtBeginDocument',
%    if amsmath has not been loaded. |\Mdots@| doesn't change when
%    switching from/to French, while |\Tdots@| is |\FBtextellipsis|
%    in French and |\Tdots@ORI| otherwise.
%    \begin{macrocode}
\newcommand*{\Tdots@ORI}{\@xp\textellipsis}
\newcommand*{\Tdots@}{\Tdots@ORI}
\newcommand*{\Mdots@}{\@xp\mdots@}
\AtBeginDocument{\DeclareRobustCommand*{\dots}{\relax
                 \csname\ifmmode M\else T\fi dots@\endcsname}%
                 \@ifundefined{@xp}{\let\@xp\relax}{}%
                 \@ifundefined{mdots@}{\let\Tdots@ORI\textellipsis
                                       \let\Mdots@\mathellipsis}{}}
\def\bbl@frenchdots{\let\Tdots@\FBtextellipsis}
\def\bbl@nonfrenchdots{\let\Tdots@\Tdots@ORI}
\expandafter\addto\csname extras\CurrentOption\endcsname{%
    \bbl@frenchdots}
\expandafter\addto\csname noextras\CurrentOption\endcsname{%
    \bbl@nonfrenchdots}
%    \end{macrocode}
%  \end{macro}
%
%  \subsection{Setup options: keyval stuff}
%  \label{sec-keyval}
%
% \changes{v2.0}{2006/11/06}{New command \cs{frenchbsetup} added
%     for global customisation.}
%
% \changes{v2.0c}{2007/06/25}{Option ThinSpaceInFrenchNumbers added.}
%
% \changes{v2.0d}{2007/07/15}{Options og and fg changed: limit
%     the definition to French so that quote characters can be used
%     in German.}
%
% \changes{v2.0e}{2007/10/05}{New option: StandardLists.}
%
% \changes{v2.0f}{2008/03/23}{Two typos corrected in
%    option StandardLists: [false] $\to$ [true] and
%    StandardLayout $\to$ StandardLists.}
%
% \changes{v2.0f}{2008/03/23}{StandardLayout option had no
%     effect on lists.  Test moved to \cs{FBprocess@options}.}
%
% \changes{v2.0g}{2008/03/23}{Revert previous change to
%     StandardLayout. This option must set the three flags
%     \cs{FBReduceListSpacingfalse}, \cs{FBCompactItemizefalse},
%     and \cs{FBStandardItemLabeltrue} instead of
%     \cs{FBStandardListstrue}, so that later options can still
%     change their value before executing \cs{FBprocess@options}.
%     Same thing for option StandardLists.}
%
% \changes{v2.1a}{2008/03/24}{New option: FrenchSuperscripts
%     to define \cs{up} as \cs{fup} or as \cs{textsuperscript}.}
%
% \changes{v2.1a}{2008/03/30}{New option: LowercaseSuperscripts.}
%
% \changes{v2.2a}{2008/05/08}{The global layout of the document is
%     no longer changed when frenchb is not the last option of babel
%     (\cs{bbl@main@language}). Suggested by Ulrike Fischer.}
%
% \changes{v2.2a}{2008/05/08}{Values of flags
%     \cs{ifFBReduceListSpacing}, \cs{ifFBCompactItemize},
%     \cs{ifFBStandardItemLabels}, \cs{ifFBIndentFirst},
%     \cs{ifFBFrenchFootnotes}, \cs{ifFBAutoSpaceFootnotes} changed:
%     default now means `StandardLayout', it will be changed to
%     `FrenchLayout' AtEndOfPackage only if french is
%     \cs{bbl@main@language}.}
%
% \changes{v2.2a}{2008/05/08}{When frenchb is babel's last option,
%     French becomes the document's main language, so
%     GlobalLayoutFrench applies.}
%
% \changes{v2.3a}{2008/10/10}{New option: OriginalTypewriter. Now
%    frenchb switches to \cs{noautospace@beforeFDP} when a tt-font is
%    in use.  When OriginalTypewriter is set to true, frenchb behaves
%    as in pre-2.3 versions.}
%
%    We first define a collection of conditionals with their defaults
%    (true or false).
%
%    \begin{macrocode}
\newif\ifFBStandardLayout           \FBStandardLayouttrue
\newif\ifFBGlobalLayoutFrench       \FBGlobalLayoutFrenchfalse
\newif\ifFBReduceListSpacing        \FBReduceListSpacingfalse
\newif\ifFBCompactItemize           \FBCompactItemizefalse
\newif\ifFBStandardItemLabels       \FBStandardItemLabelstrue
\newif\ifFBStandardLists            \FBStandardListstrue
\newif\ifFBIndentFirst              \FBIndentFirstfalse
\newif\ifFBFrenchFootnotes          \FBFrenchFootnotesfalse
\newif\ifFBAutoSpaceFootnotes       \FBAutoSpaceFootnotesfalse
\newif\ifFBOriginalTypewriter       \FBOriginalTypewriterfalse
\newif\ifFBThinColonSpace           \FBThinColonSpacefalse
\newif\ifFBThinSpaceInFrenchNumbers \FBThinSpaceInFrenchNumbersfalse
\newif\ifFBFrenchSuperscripts       \FBFrenchSuperscriptstrue
\newif\ifFBLowercaseSuperscripts    \FBLowercaseSuperscriptstrue
\newif\ifFBPartNameFull             \FBPartNameFulltrue
\newif\ifFBShowOptions              \FBShowOptionsfalse
%    \end{macrocode}
%
%    The defaults values of these flags have been set so that |frenchb|
%    does not change anything regarding the global layout.
%    |\bbl@main@language| (set by the last option of babel) controls
%    the global layout of the document.  We check the current language
%    `AtEndOfPackage' (it is |\bbl@main@language|); if it is French,
%    the values of some flags have to be changed to ensure a French
%    looking layout for the whole document (even in parts written in
%    languages other than French); the end-user will then be able to
%    customise the values of all these flags with |\frenchbsetup{}|.
%    \begin{macrocode}
\AtEndOfPackage{%
   \iflanguage{french}{\FBReduceListSpacingtrue
                       \FBCompactItemizetrue
                       \FBStandardItemLabelsfalse
                       \FBIndentFirsttrue
                       \FBFrenchFootnotestrue
                       \FBAutoSpaceFootnotestrue
                       \FBGlobalLayoutFrenchtrue}%
                      {}%
}
%    \end{macrocode}
%
%  \begin{macro}{\frenchbsetup}
%    From version 2.0 on, all setup options are handled by \emph{one}
%    command |\frenchbsetup| using the keyval syntax.
%    Let's now define this command which reads and sets the options
%    to be processed later (at |\begin{document}|) by
%    |\FBprocess@options|. It  can only be called in the preamble.
%    \begin{macrocode}
\newcommand*{\frenchbsetup}[1]{%
  \setkeys{FB}{#1}%
}%
\@onlypreamble\frenchbsetup
%    \end{macrocode}
%    |frenchb| being an option of babel, it cannot load a package
%    (keyval) while |frenchb.ldf| is read, so we defer the loading of
%    \file{keyval} and the options setup at the end of \babel's loading.
%
%    |StandardLayout| resets the layout in French to the standard layout
%    defined par the \LaTeX{} class and packages loaded. It deals with
%    lists, indentation of first paragraphs of sections and footnotes.
%    Other keys, entered \emph{after} |StandardLayout| in
%    |\frenchbsetup|, can overrule some of the |StandardLayout|
%     settings.
%
%    |GlobalLayoutFrench| forces the layout in French and (as far as
%    possible) outside French to meet the French typographic standards.
%
% \changes{v2.3d}{2009/03/16}{Warning added to \cs{GlobalLayoutFrench}
%    when French is not the main language.}
%
%    \begin{macrocode}
\AtEndOfPackage{%
    \RequirePackage{keyval}%
    \define@key{FB}{StandardLayout}[true]%
                      {\csname FBStandardLayout#1\endcsname
                       \ifFBStandardLayout
                         \FBReduceListSpacingfalse
                         \FBCompactItemizefalse
                         \FBStandardItemLabelstrue
                         \FBIndentFirstfalse
                         \FBFrenchFootnotesfalse
                         \FBAutoSpaceFootnotesfalse
                         \FBGlobalLayoutFrenchfalse
                       \else
                         \FBReduceListSpacingtrue
                         \FBCompactItemizetrue
                         \FBStandardItemLabelsfalse
                         \FBIndentFirsttrue
                         \FBFrenchFootnotestrue
                         \FBAutoSpaceFootnotestrue
                       \fi}%
    \define@key{FB}{GlobalLayoutFrench}[true]%
                      {\csname FBGlobalLayoutFrench#1\endcsname
                       \ifFBGlobalLayoutFrench
                          \iflanguage{french}%
                            {\FBReduceListSpacingtrue
                             \FBCompactItemizetrue
                             \FBStandardItemLabelsfalse
                             \FBIndentFirsttrue
                             \FBFrenchFootnotestrue
                             \FBAutoSpaceFootnotestrue}%
                            {\PackageWarning{frenchb.ldf}%
                              {Option `GlobalLayoutFrench' skipped:
                               \MessageBreak French is *not*
                               babel's last option.\MessageBreak}}%
                       \fi}%
    \define@key{FB}{ReduceListSpacing}[true]%
                      {\csname FBReduceListSpacing#1\endcsname}%
    \define@key{FB}{CompactItemize}[true]%
                      {\csname FBCompactItemize#1\endcsname}%
    \define@key{FB}{StandardItemLabels}[true]%
                      {\csname FBStandardItemLabels#1\endcsname}%
    \define@key{FB}{ItemLabels}{%
        \renewcommand*{\FrenchLabelItem}{#1}}%
    \define@key{FB}{ItemLabeli}{%
        \renewcommand*{\Frlabelitemi}{#1}}%
    \define@key{FB}{ItemLabelii}{%
        \renewcommand*{\Frlabelitemii}{#1}}%
    \define@key{FB}{ItemLabeliii}{%
        \renewcommand*{\Frlabelitemiii}{#1}}%
    \define@key{FB}{ItemLabeliv}{%
        \renewcommand*{\Frlabelitemiv}{#1}}%
    \define@key{FB}{StandardLists}[true]%
                      {\csname FBStandardLists#1\endcsname
                       \ifFBStandardLists
                         \FBReduceListSpacingfalse
                         \FBCompactItemizefalse
                         \FBStandardItemLabelstrue
                       \else
                         \FBReduceListSpacingtrue
                         \FBCompactItemizetrue
                         \FBStandardItemLabelsfalse
                       \fi}%
    \define@key{FB}{IndentFirst}[true]%
                      {\csname FBIndentFirst#1\endcsname}%
    \define@key{FB}{FrenchFootnotes}[true]%
                      {\csname FBFrenchFootnotes#1\endcsname}%
    \define@key{FB}{AutoSpaceFootnotes}[true]%
                      {\csname FBAutoSpaceFootnotes#1\endcsname}%
    \define@key{FB}{AutoSpacePunctuation}[true]%
                      {\csname FBAutoSpacePunctuation#1\endcsname}%
    \define@key{FB}{OriginalTypewriter}[true]%
                      {\csname FBOriginalTypewriter#1\endcsname}%
    \define@key{FB}{ThinColonSpace}[true]%
                      {\csname FBThinColonSpace#1\endcsname}%
    \define@key{FB}{ThinSpaceInFrenchNumbers}[true]%
                      {\csname FBThinSpaceInFrenchNumbers#1\endcsname}%
    \define@key{FB}{FrenchSuperscripts}[true]%
                      {\csname FBFrenchSuperscripts#1\endcsname}
    \define@key{FB}{LowercaseSuperscripts}[true]%
                      {\csname FBLowercaseSuperscripts#1\endcsname}
    \define@key{FB}{PartNameFull}[true]%
                      {\csname FBPartNameFull#1\endcsname}%
    \define@key{FB}{ShowOptions}[true]%
                      {\csname FBShowOptions#1\endcsname}%
%    \end{macrocode}
%    Inputing French quotes as \emph{single characters} when they are
%    available on the keyboard (through a compose key for instance)
%    is more comfortable than typing |\og| and |\fg|.
%    The purpose of the following code is to map the French quote
%    characters to |\og\ignorespaces| and |{\fg}| respectively when
%    the current language is French, and to |\guillemotleft| and
%    |\guillemotright| otherwise (think of German quotes); thus correct
%    unbreakable spaces will be added automatically to French quotes.
%    The quote characters typed in depend on the input encoding,
%    it can be single-byte (latin1, latin9, applemac,\dots) or
%    multi-bytes (utf-8, utf8x).  We first check whether XeTeX is used
%    or not, if not the package |inputenc| has to be loaded before the
%    |\begin{document}| with the proper coding option, so we check if
%    |\DeclareInputText| is defined.
%    \begin{macrocode}
    \define@key{FB}{og}{%
       \newcommand*{\FB@@og}{\iflanguage{french}%
                               {\FB@og\ignorespaces}{\guillemotleft}}%
       \expandafter\ifx\csname XeTeXrevision\endcsname\relax
         \AtBeginDocument{%
           \@ifundefined{DeclareInputText}%
             {\PackageWarning{frenchb.ldf}%
               {Option `og' requires package inputenc.\MessageBreak}%
             }%
             {\@ifundefined{uc@dclc}%
%    \end{macrocode}
%    if |\uc@dclc| is undefined, utf8x is not loaded\dots
%    \begin{macrocode}
               {\@ifundefined{DeclareUnicodeCharacter}%
%    \end{macrocode}
%    if |\DeclareUnicodeCharacter| is undefined, utf8 is not loaded
%    either, we assume 8-bit character input encoding.
%    Package MULEenc.sty (from CJK) defines |\mule@def| to map
%    characters to control sequences.
%    \begin{macrocode}
                  {\@tempcnta`#1\relax
                     \@ifundefined{mule@def}%
                       {\DeclareInputText{\the\@tempcnta}{\FB@@og}}%
                       {\mule@def{11}{{\FB@@og}}}%
                  }%
%    \end{macrocode}
%    utf8 loaded, use |\DeclareUnicodeCharacter|,
%    \begin{macrocode}
                  {\DeclareUnicodeCharacter{00AB}{\FB@@og}}%
               }%
%    \end{macrocode}
%    utf8x loaded, use |\uc@dclc|,
%    \begin{macrocode}
               {\uc@dclc{171}{default}{\FB@@og}}%
             }%
         }%
%    \end{macrocode}
%    XeTeX in use, the following trick for defining the active quote
%    character is borrowed from \file{inputenc.dtx}.
%    \begin{macrocode}
       \else
         \catcode`#1=\active
         \bgroup
           \uccode`\~`#1%
           \uppercase{%
         \egroup
         \def~%
         }{\FB@@og}%
       \fi
    }%
%    \end{macrocode}
%    Same code for the closing quote.
%    \begin{macrocode}
    \define@key{FB}{fg}{%
       \newcommand*{\FB@@fg}{\iflanguage{french}%
                               {\FB@fg}{\guillemotright}}%
       \expandafter\ifx\csname XeTeXrevision\endcsname\relax
         \AtBeginDocument{%
           \@ifundefined{DeclareInputText}%
             {\PackageWarning{frenchb.ldf}%
               {Option `fg' requires package inputenc.\MessageBreak}%
             }%
             {\@ifundefined{uc@dclc}%
               {\@ifundefined{DeclareUnicodeCharacter}%
                  {\@tempcnta`#1\relax
                     \@ifundefined{mule@def}%
                       {\DeclareInputText{\the\@tempcnta}{{\FB@@fg}}}%
                       {\mule@def{27}{{\FB@@fg}}}%
                  }%
                  {\DeclareUnicodeCharacter{00BB}{{\FB@@fg}}}%
               }%
               {\uc@dclc{187}{default}{{\FB@@fg}}}%
             }%
         }%
       \else
         \catcode`#1=\active
         \bgroup
           \uccode`\~`#1%
           \uppercase{%
         \egroup
         \def~%
         }{{\FB@@fg}}%
       \fi
    }%
}
%    \end{macrocode}
%  \end{macro}
%
% \begin{macro}{\FBprocess@options}
%    |\FBprocess@options| processes the options, it is called \emph{once}
%    at |\begin{document}|.
%    \begin{macrocode}
\newcommand*{\FBprocess@options}{%
%    \end{macrocode}
%    Nothing has to be done here for |StandardLayout| and
%    |StandardLists| (the involved flags have already been set in
%    |\frenchbsetup{}| or before (at babel's EndOfPackage).
%
%    The next three options deal with the layout of lists in French.
%
%    |ReduceListSpacing| reduces the vertical spaces between list
%    items in French (done by changing |\list| to |\listFB|).
%    When |GlobalLayoutFrench| is true (the default), the same is
%    done outside French except for languages that force a different
%    setting.
%    \begin{macrocode}
  \ifFBReduceListSpacing
    \addto\extrasfrench{\let\list\listFB
                        \let\endlist\endlistFB}%
    \addto\noextrasfrench{\ifFBGlobalLayoutFrench
                            \let\list\listFB
                            \let\endlist\endlistFB
                          \else
                            \let\list\listORI
                            \let\endlist\endlistORI
                          \fi}%
  \else
    \addto\extrasfrench{\let\list\listORI
                        \let\endlist\endlistORI}%
    \addto\noextrasfrench{\let\list\listORI
                          \let\endlist\endlistORI}%
  \fi
%    \end{macrocode}
%
%    |CompactItemize| suppresses the vertical spacing between list
%    items in French (done by changing |\itemize| to |\itemizeFB|).
%    When |GlobalLayoutFrench| is true the same is done outside French.
%    \begin{macrocode}
  \ifFBCompactItemize
    \addto\extrasfrench{\let\itemize\itemizeFB
                        \let\enditemize\enditemizeFB}%
    \addto\noextrasfrench{\ifFBGlobalLayoutFrench
                             \let\itemize\itemizeFB
                             \let\enditemize\enditemizeFB
                          \else
                             \let\itemize\itemizeORI
                             \let\enditemize\enditemizeORI
                          \fi}%
  \else
    \addto\extrasfrench{\let\itemize\itemizeORI
                        \let\enditemize\enditemizeORI}%
    \addto\noextrasfrench{\let\itemize\itemizeORI
                          \let\enditemize\enditemizeORI}%
  \fi
%    \end{macrocode}
%
%    |StandardItemLabels| resets labelitems in French to their
%    standard values set by the \LaTeX{} class and packages loaded.
%    When |GlobalLayoutFrench| is true labelitems are identical inside
%    and outside French.
%    \begin{macrocode}
  \ifFBStandardItemLabels
    \addto\extrasfrench{\bbl@nonfrenchlabelitems}%
    \addto\noextrasfrench{\bbl@nonfrenchlabelitems}%
  \else
    \addto\extrasfrench{\bbl@frenchlabelitems}%
    \addto\noextrasfrench{\ifFBGlobalLayoutFrench
                            \bbl@frenchlabelitems
                          \else
                            \bbl@nonfrenchlabelitems
                          \fi}%
  \fi
%    \end{macrocode}
%
%    |IndentFirst| forces the first paragraphs of sections to be
%    indented just like the other ones in French.
%    When |GlobalLayoutFrench| is true (the default), the same is
%    done outside French except for languages that force a different
%    setting.
%    \begin{macrocode}
  \ifFBIndentFirst
    \addto\extrasfrench{\bbl@frenchindent}%
    \addto\noextrasfrench{\ifFBGlobalLayoutFrench
                             \bbl@frenchindent
                          \else
                             \bbl@nonfrenchindent
                          \fi}%
  \else
    \addto\extrasfrench{\bbl@nonfrenchindent}%
    \addto\noextrasfrench{\bbl@nonfrenchindent}%
  \fi
%    \end{macrocode}
%
%    The layout of footnotes is handled at the |\begin{document}|
%    depending on the values of flags |FrenchFootnotes|
%    and |AutoSpaceFootnotes| (see section~\ref{sec-footnotes}),
%    nothing has to be done here for footnotes.
%
%    |AutoSpacePunctuation| adds an unbreakable space (in French only)
%    before the four active characters (:;!?) even if none has been
%    typed before them.
%    \begin{macrocode}
  \ifFBAutoSpacePunctuation
     \autospace@beforeFDP
  \else
     \noautospace@beforeFDP
  \fi
%    \end{macrocode}
%
%    When |OriginalTypewriter| is set to |false| (the default),
%    |\ttfamily|, |\rmfamily| and |\sffamily| are redefined as
%    |\ttfamilyFB|, |\rmfamilyFB| and |\sffamilyFB| respectively
%    to prevent addition of automatic spaces before the four active
%    characters in computer code.
%    \begin{macrocode}
  \ifFBOriginalTypewriter
  \else
     \let\ttfamily\ttfamilyFB
     \let\rmfamily\rmfamilyFB
     \let\sffamily\sffamilyFB
  \fi
%    \end{macrocode}
%
%    |ThinColonSpace| changes the normal unbreakable space typeset in
%     French before `:' to a thin space.
%    \begin{macrocode}
  \ifFBThinColonSpace\renewcommand*{\Fcolonspace}{\thinspace}\fi
%    \end{macrocode}
%
%    When |true|, |ThinSpaceInFrenchNumbers| redefines |numprint.sty|'s
%    command |\npstylefrench| to set |\npthousandsep| to |\,|
%    (thinspace) instead of |~| (default) . This option has no effect
%    if package |numprint.sty| is not loaded with `|autolanguage|'.
%    As old versions of |numprint.sty| did not define |\npstylefrench|,
%    we have to provide this command.
%    \begin{macrocode}
  \@ifpackageloaded{numprint}%
  {\ifnprt@autolanguage
     \providecommand*{\npstylefrench}{}%
     \ifFBThinSpaceInFrenchNumbers
       \renewcommand*\npstylefrench{%
          \npthousandsep{\,}%
          \npdecimalsign{,}%
          \npproductsign{\cdot}%
          \npunitseparator{\,}%
          \npdegreeseparator{}%
          \nppercentseparator{\nprt@unitsep}%
          }%
     \else
       \renewcommand*\npstylefrench{%
          \npthousandsep{~}%
          \npdecimalsign{,}%
          \npproductsign{\cdot}%
          \npunitseparator{\,}%
          \npdegreeseparator{}%
          \nppercentseparator{\nprt@unitsep}%
          }%
     \fi
     \npaddtolanguage{french}{french}%
   \fi}{}%
%    \end{macrocode}
%
%    |FrenchSuperscripts|: if |true| |\up=\fup|, else
%    |\up=\textsuperscript|. Anyway |\up*=\FB@up@fake|. The star-form
%    |\up*{}| is provided for fonts that lack some superior letters:
%    Adobe Jenson Pro and Utopia Expert have no ``g superior'' for
%    instance.
%    \begin{macrocode}
  \ifFBFrenchSuperscripts
    \DeclareRobustCommand*{\up}{\@ifstar{\FB@up@fake}{\fup}}%
  \else
    \DeclareRobustCommand*{\up}{\@ifstar{\FB@up@fake}%
                                        {\textsuperscript}}%
  \fi
%    \end{macrocode}
%
%    |LowercaseSuperscripts|: if |true| let |\FB@lc| be |\lowercase|,
%     else |\FB@lc| is redefined to do nothing.
%    \begin{macrocode}
  \ifFBLowercaseSuperscripts
  \else
    \renewcommand*{\FB@lc}[1]{##1}%
  \fi
%    \end{macrocode}
%
%    |PartNameFull|: if |false|, redefine |\partname|.
%    \begin{macrocode}
  \ifFBPartNameFull
  \else\addto\captionsfrench{\def\partname{Partie}}\fi
%    \end{macrocode}
%
%    |ShowOptions|: if |true|, print the list of all options to the
%    \file{.log} file.
%    \begin{macrocode}
  \ifFBShowOptions
    \GenericWarning{* }{%
     * **** List of possible options for frenchb ****\MessageBreak
     [Default values between brackets when frenchb is loaded *LAST*]%
     \MessageBreak
     ShowOptions=true [false]\MessageBreak
     StandardLayout=true [false]\MessageBreak
     GlobalLayoutFrench=false [true]\MessageBreak
     StandardLists=true [false]\MessageBreak
     ReduceListSpacing=false [true]\MessageBreak
     CompactItemize=false [true]\MessageBreak
     StandardItemLabels=true [false]\MessageBreak
     ItemLabels=\textemdash, \textbullet,
        \protect\ding{43},... [\textendash]\MessageBreak
     ItemLabeli=\textemdash, \textbullet,
        \protect\ding{43},... [\textendash]\MessageBreak
     ItemLabelii=\textemdash, \textbullet,
        \protect\ding{43},... [\textendash]\MessageBreak
     ItemLabeliii=\textemdash, \textbullet,
        \protect\ding{43},... [\textendash]\MessageBreak
     ItemLabeliv=\textemdash, \textbullet,
        \protect\ding{43},... [\textendash]\MessageBreak
     IndentFirst=false [true]\MessageBreak
     FrenchFootnotes=false [true]\MessageBreak
     AutoSpaceFootnotes=false [true]\MessageBreak
     AutoSpacePunctuation=false [true]\MessageBreak
     OriginalTypewriter=true [false]\MessageBreak
     ThinColonSpace=true [false]\MessageBreak
     ThinSpaceInFrenchNumbers=true [false]\MessageBreak
     FrenchSuperscripts=false [true]\MessageBreak
     LowercaseSuperscripts=false [true]\MessageBreak
     PartNameFull=false [true]\MessageBreak
     og= <left quote character>, fg= <right quote character>
     \MessageBreak
     *********************************************
     \MessageBreak\protect\frenchbsetup{ShowOptions}}
  \fi
}
%    \end{macrocode}
%  \end{macro}
%
% \changes{v2.0}{2006/12/15}{AtBeginDocument, save again the
%    definitions of the `list' and `itemize' environments and the
%    values of labelitems.  As of frenchb v.1.6, `ORI' values were
%    set when reading frenchb.ldf, later changes were ignored.}
%
% \changes{v2.0}{2006/12/06}{Added warning for OT1 encoding.}
%
% \changes{v2.1b}{2008/04/07}{Disable some commands in bookmarks.}
%
%    At |\begin{document}| we save again the definitions of the `list'
%    and `itemize' environments and the values of labelitems so that
%    all changes made in the preamble are taken into account in
%    languages other than French and in French with the StandardLayout
%    option.  We also have to provide an |\xspace| command in case the
%    |xspace.sty| package is not loaded.
%
%    \begin{macrocode}
\AtBeginDocument{%
   \let\listORI\list
   \let\endlistORI\endlist
   \let\itemizeORI\itemize
   \let\enditemizeORI\enditemize
   \let\@ltiORI\labelitemi
   \let\@ltiiORI\labelitemii
   \let\@ltiiiORI\labelitemiii
   \let\@ltivORI\labelitemiv
   \providecommand*{\xspace}{\relax}%
%    \end{macrocode}
%    Let's redefine some commands in \file{hyperref}'s bookmarks.
%    \begin{macrocode}
   \@ifundefined{pdfstringdefDisableCommands}{}%
     {\pdfstringdefDisableCommands{%
        \let\up\relax
        \def\ieme{e\xspace}%
        \def\iemes{es\xspace}%
        \def\ier{er\xspace}%
        \def\iers{ers\xspace}%
        \def\iere{re\xspace}%
        \def\ieres{res\xspace}%
        \def\FrenchEnumerate#1{#1\degre\space}%
        \def\FrenchPopularEnumerate#1{#1\degre)\space}%
        \def\No{N\degre\space}%
        \def\no{n\degre\space}%
        \def\Nos{N\degre\space}%
        \def\nos{n\degre\space}%
        \def\og{\guillemotleft\space}%
        \def\fg{\space\guillemotright}%
        \let\bsc\textsc
        \let\degres\degre
     }}%
%    \end{macrocode}
%    It is time to process the options set with |\frenchboptions{}|.
%    Then execute either |\extrasfrench| and |\captionsfrench| or
%    |\noextrasfrench| according to the current language at the
%    |\begin{document}| (these three commands are updated by
%    |\FBprocess@options|).
%    \begin{macrocode}
   \FBprocess@options
   \iflanguage{french}{\extrasfrench\captionsfrench}{\noextrasfrench}%
%    \end{macrocode}
%    Some warnings are issued when output font encodings are not
%    properly set. With XeLaTeX, \file{fontspec.sty} and
%    \file{xunicode.sty} should be loaded; with (pdf)\LaTeX, a warning
%    is issued when OT1 encoding is in use at the |\begin{document}|.
%    Mind that |\encodingdefault| is defined as `long', defining
%    |\FBOTone| with |\newcommand*| would fail!
%    \begin{macrocode}
   \expandafter\ifx\csname XeTeXrevision\endcsname\relax
      \begingroup \newcommand{\FBOTone}{OT1}%
      \ifx\encodingdefault\FBOTone
        \PackageWarning{frenchb.ldf}%
           {OT1 encoding should not be used for French.
            \MessageBreak
            Add \protect\usepackage[T1]{fontenc} to the
            preamble\MessageBreak of your document,}
      \fi
     \endgroup
   \else
     \@ifundefined{DeclareUTFcharacter}%
       {\PackageWarning{frenchb.ldf}%
         {Add \protect\usepackage{fontspec} *and*\MessageBreak
          \protect\usepackage{xunicode} to the preamble\MessageBreak
          of your document,}}%
       {}%
    \fi
}
%    \end{macrocode}
%
%  \subsection{Clean up and exit}
%
%    Load |frenchb.cfg| (should do nothing, just for compatibility).
%    \begin{macrocode}
\loadlocalcfg{frenchb}
%    \end{macrocode}
%    Final cleaning.
%    The macro |\ldf@quit| takes care for setting the main language
%    to be switched on at |\begin{document}| and resetting the
%    category code of \texttt{@} to its original value.
%    The config file searched for has to be |frenchb.cfg|, and
%    |\CurrentOption| has been set to `french', so
%    |\ldf@finish\CurrentOption| cannot be used: we first load
%    |frenchb.cfg|, then call |\ldf@quit\CurrentOption|.
%    \begin{macrocode}
\FBclean@on@exit
\ldf@quit\CurrentOption
%    \end{macrocode}
% \iffalse
%</code>
%<*dtx>
% \fi
%%
%% \CharacterTable
%%  {Upper-case    \A\B\C\D\E\F\G\H\I\J\K\L\M\N\O\P\Q\R\S\T\U\V\W\X\Y\Z
%%   Lower-case    \a\b\c\d\e\f\g\h\i\j\k\l\m\n\o\p\q\r\s\t\u\v\w\x\y\z
%%   Digits        \0\1\2\3\4\5\6\7\8\9
%%   Exclamation   \!     Double quote  \"     Hash (number) \#
%%   Dollar        \$     Percent       \%     Ampersand     \&
%%   Acute accent  \'     Left paren    \(     Right paren   \)
%%   Asterisk      \*     Plus          \+     Comma         \,
%%   Minus         \-     Point         \.     Solidus       \/
%%   Colon         \:     Semicolon     \;     Less than     \<
%%   Equals        \=     Greater than  \>     Question mark \?
%%   Commercial at \@     Left bracket  \[     Backslash     \\
%%   Right bracket \]     Circumflex    \^     Underscore    \_
%%   Grave accent  \`     Left brace    \{     Vertical bar  \|
%%   Right brace   \}     Tilde         \~}
%%
% \iffalse
%</dtx>
% \fi
%
% \Finale
\endinput
}%
\bbl@tempa{german}{% \iffalse meta-comm

% Copyright 1989-2008 Johannes L. Braams and any individual auth
% listed elsewhere in this file.  All rights reserv

% This file is part of the Babel syst
% -----------------------------------

% It may be distributed and/or modified under
% conditions of the LaTeX Project Public License, either version
% of this license or (at your option) any later versi
% The latest version of this license is
%   http://www.latex-project.org/lppl.
% and version 1.3 or later is part of all distributions of La
% version 2003/12/01 or lat

% This work has the LPPL maintenance status "maintaine

% The Current Maintainer of this work is Johannes Braa

% The list of all files belonging to the Babel system
% given in the file `manifest.bbl. See also `legal.bbl' for additio
% informati

% The list of derived (unpacked) files belonging to the distribut
% and covered by LPPL is defined by the unpacking scripts (w
% extension .ins) which are part of the distributi
%
% \CheckSum{3

% \iffa
%    Tell the \LaTeX\ system who we are and write an entry on
%    transcri
%<*d
\ProvidesFile{germanb.d
%</d
%<code>\ProvidesLanguage{germa
%
%\ProvidesFile{germanb.d
        [2008/06/01 v2.6m German support from the babel syst
%\iffa
%% File `germanb.d
%% Babel package for LaTeX version
%% Copyright (C) 1989 - 2
%%           by Johannes Braams, TeXn

%% Germanb Language Definition F
%% Copyright (C) 1989 - 2
%%           by Bernd Raichle raichle at azu.Informatik.Uni-Stuttgart
%%              Johannes Braams, TeXn
% This file is based on german.tex version 2.
%                       by Bernd Raichle, Hubert Partl et.

%% Please report errors to: J.L. Bra
%%                          babel at braams.xs4all

%<*filedriv
\documentclass{ltxd
\font\manual=logo10 % font used for the METAFONT logo, e
\newcommand*\MF{{\manual META}\-{\manual FON
\newcommand*\TeXhax{\TeX h
\newcommand*\babel{\textsf{babe
\newcommand*\langvar{$\langle \it lang \rangl
\newcommand*\note[1
\newcommand*\Lopt[1]{\textsf{#
\newcommand*\file[1]{\texttt{#
\begin{docume
 \DocInput{germanb.d
\end{docume
%</filedriv
%
% \GetFileInfo{germanb.d

% \changes{germanb-1.0a}{1990/05/14}{Incorporated Nico's commen
% \changes{germanb-1.0b}{1990/05/22}{fixed typo in definition
%    austrian language found by Werenfried S
%    \texttt{nspit@fys.ruu.n
% \changes{germanb-1.0c}{1990/07/16}{Fixed some typ
% \changes{germanb-1.1}{1990/07/30}{When using PostScript fonts w
%    the Adobe fontencoding, the dieresis-accent is located elsewhe
%    modified co
% \changes{germanb-1.1a}{1990/08/27}{Modified the documentat
%    somewh
% \changes{germanb-2.0}{1991/04/23}{Modified for babel 3
% \changes{germanb-2.0a}{1991/05/25}{Removed some problems in cha
%    l
% \changes{germanb-2.1}{1991/05/29}{Removed bug found by van der Me
% \changes{germanb-2.2}{1991/06/11}{Removed global assignmen
%    brought uptodate with \file{german.tex} v2.
% \changes{germanb-2.2a}{1991/07/15}{Renamed \file{babel.sty}
%    \file{babel.co
% \changes{germanb-2.3}{1991/11/05}{Rewritten parts of the code to
%    the new features of babel version 3
% \changes{germanb-2.3e}{1991/11/10}{Brought up-to-date w
%    \file{german.tex} v2.3e (plus some bug fixes) [b
% \changes{germanb-2.5}{1994/02/08}{Update or \LaTe
% \changes{germanb-2.5c}{1994/06/26}{Removed the use of \cs{fileda
%    and moved the identification after the loading
%    \file{babel.de
% \changes{germanb-2.6a}{1995/02/15}{Moved the identification to
%    top of the fi
% \changes{germanb-2.6a}{1995/02/15}{Rewrote the code that handles
%    active double quote charact
% \changes{germanb-2.6d}{1996/07/10}{Replaced \cs{undefined} w
%    \cs{@undefined} and \cs{empty} with \cs{@empty} for consiste
%    with \LaTe
% \changes{germanb-2.6d}{1996/10/10}{Moved the definition
%    \cs{atcatcode} right to the beginning

%  \section{The German langua

%    The file \file{\filename}\footnote{The file described in t
%    section has version number \fileversion\ and was last revised
%    \filedate.}  defines all the language definition macros for
%    German language as well as for the Austrian dialect of t
%    language\footnote{This file is a re-implementation of Hub
%    Partl's \file{german.sty} version 2.5b, see~\cite{HP}

%    For this language the character |"| is made active.
%    table~\ref{tab:german-quote} an overview is given of
%    purpose. One of the reasons for this is that in the Ger
%    language some character combinations change when a word is bro
%    between the combination. Also the vertical placement of
%    umlaut can be controlled this w
%    \begin{table}[h
%     \begin{cent
%     \begin{tabular}{lp{8c
%      |"a| & |\"a|, also implemented for the ot
%                  lowercase and uppercase vowels.
%      |"s| & to produce the German \ss{} (like |\ss{}|).
%      |"z| & to produce the German \ss{} (like |\ss{}|).
%      |"ck|& for |ck| to be hyphenated as |k-k|.
%      |"ff|& for |ff| to be hyphenated as |ff-
%                  this is also implemented for l, m, n, p, r and
%      |"S| & for |SS| to be |\uppercase{"s}|.
%      |"Z| & for |SZ| to be |\uppercase{"z}|.
%      \verb="|= & disable ligature at this position.
%      |"-| & an explicit hyphen sign, allowing hyphenat
%             in the rest of the word.
%      |""| & like |"-|, but producing no hyphen s
%             (for compund words with hyphen, e.g.\ |x-""y|).
%      |"~| & for a compound word mark without a breakpoint.
%      |"=| & for a compound word mark with a breakpoint, allow
%             hyphenation in the composing words.
%      |"`| & for German left double quotes (looks like ,,).
%      |"'| & for German right double quotes.
%      |"<| & for French left double quotes (similar to $<<$).
%      |">| & for French right double quotes (similar to $>>$).
%     \end{tabul
%     \caption{The extra definitions m
%              by \file{german.ldf}}\label{tab:german-quo
%     \end{cent
%    \end{tab
%    The quotes in table~\ref{tab:german-quote} can also be typeset
%    using the commands in table~\ref{tab:more-quot
%    \begin{table}[h
%     \begin{cent
%     \begin{tabular}{lp{8c
%      |\glqq| & for German left double quotes (looks like ,,).
%      |\grqq| & for German right double quotes (looks like ``).
%      |\glq|  & for German left single quotes (looks like ,).
%      |\grq|  & for German right single quotes (looks like `).
%      |\flqq| & for French left double quotes (similar to $<<$).
%      |\frqq| & for French right double quotes (similar to $>>$)
%      |\flq|  & for (French) left single quotes (similar to $<$).
%      |\frq|  & for (French) right single quotes (similar to $>$).
%      |\dq|   & the original quotes character (|"|).
%     \end{tabul
%     \caption{More commands which produce quotes, defi
%              by \file{german.ldf}}\label{tab:more-quo
%     \end{cent
%    \end{tab

% \StopEventuall

%    When this file was read through the option \Lopt{germanb} we m
%    it behave as if \Lopt{german} was specifi
% \changes{german-2.6l}{2008/03/17}{Making germanb behave like ger
%    needs some more work besides defining \cs{CurrentOptio
% \changes{germanb-2.6m}{2008/06/01}{Correted a ty
%    \begin{macroco
\def\bbl@tempa{germa
\ifx\CurrentOption\bbl@te
  \def\CurrentOption{germ
  \ifx\l@german\@undefi
    \@nopatterns{Germ
    \adddialect\l@germ

  \let\l@germanb\l@ger
  \AtBeginDocumen
    \let\captionsgermanb\captionsger
    \let\dategermanb\dateger
    \let\extrasgermanb\extrasger
    \let\noextrasgermanb\noextrasger


%    \end{macroco

%    The macro |\LdfInit| takes care of preventing that this file
%    loaded more than once, checking the category code of
%    \texttt{@} sign, e
% \changes{germanb-2.6d}{1996/11/02}{Now use \cs{LdfInit} to perf
%    initial check
%    \begin{macroco
%<*co
\LdfInit\CurrentOption{captions\CurrentOpti
%    \end{macroco

%    When this file is read as an option, i.e., by the |\usepacka
%    command, \texttt{german} will be an `unknown' language, so
%    have to make it known.  So we check for the existence
%    |\l@german| to see whether we have to do something he

% \changes{germanb-2.0}{1991/04/23}{Now use \cs{adddialect}
%    language undefin
% \changes{germanb-2.2d}{1991/10/27}{Removed use of \cs{@ifundefine
% \changes{germanb-2.3e}{1991/11/10}{Added warning, if no ger
%    patterns load
% \changes{germanb-2.5c}{1994/06/26}{Now use \cs{@nopatterns}
%    produce the warni
%    \begin{macroco
\ifx\l@german\@undefi
  \@nopatterns{Germ
  \adddialect\l@germ

%    \end{macroco

%    For the Austrian version of these definitions we just add anot
%    languag
% \changes{germanb-2.0}{1991/04/23}{Now use \cs{adddialect}
%    austri
%    \begin{macroco
\adddialect\l@austrian\l@ger
%    \end{macroco

%    The next step consists of defining commands to switch to (
%    from) the German langua

%  \begin{macro}{\captionsgerm
%  \begin{macro}{\captionsaustri
%    Either the macro |\captionsgerman| or the ma
%    |\captionsaustrian| will define all strings used in the f
%    standard document classes provided with \LaT

% \changes{germanb-2.2}{1991/06/06}{Removed \cs{global} definitio
% \changes{germanb-2.2}{1991/06/06}{\cs{pagename} should
%    \cs{headpagenam
% \changes{germanb-2.3e}{1991/11/10}{Added \cs{prefacenam
%    \cs{seename} and \cs{alsonam
% \changes{germanb-2.4}{1993/07/15}{\cs{headpagename} should
%    \cs{pagenam
% \changes{germanb-2.6b}{1995/07/04}{Added \cs{proofname}
%    AMS-\LaT
% \changes{germanb-2.6d}{1996/07/10}{Construct control sequence on
%    f
% \changes{germanb-2.6j}{2000/09/20}{Added \cs{glossarynam
%    \begin{macroco
\@namedef{captions\CurrentOption
  \def\prefacename{Vorwor
  \def\refname{Literatu
  \def\abstractname{Zusammenfassun
  \def\bibname{Literaturverzeichni
  \def\chaptername{Kapite
  \def\appendixname{Anhan
  \def\contentsname{Inhaltsverzeichnis}%    % oder nur: Inh
  \def\listfigurename{Abbildungsverzeichni
  \def\listtablename{Tabellenverzeichni
  \def\indexname{Inde
  \def\figurename{Abbildun
  \def\tablename{Tabelle}%                  % oder: Ta
  \def\partname{Tei
  \def\enclname{Anlage(n)}%                 % oder: Beilage
  \def\ccname{Verteiler}%                   % oder: Kopien
  \def\headtoname{A
  \def\pagename{Seit
  \def\seename{sieh
  \def\alsoname{siehe auc
  \def\proofname{Bewei
  \def\glossaryname{Glossa

%    \end{macroco
%  \end{mac
%  \end{mac

%  \begin{macro}{\dategerm
%    The macro |\dategerman| redefines the comm
%    |\today| to produce German dat
% \changes{germanb-2.3e}{1991/11/10}{Added \cs{month@germa
% \changes{germanb-2.6f}{1997/10/01}{Use \cs{edef} to def
%    \cs{today} to save memo
% \changes{germanb-2.6f}{1998/03/28}{use \cs{def} instead
%    \cs{ede
%    \begin{macroco
\def\month@german{\ifcase\month
  Januar\or Februar\or M\"arz\or April\or Mai\or Juni
  Juli\or August\or September\or Oktober\or November\or Dezember\
\def\dategerman{\def\today{\number\day.~\month@ger
    \space\number\yea
%    \end{macroco
%  \end{mac

%  \begin{macro}{\dateaustri
%    The macro |\dateaustrian| redefines the comm
%    |\today| to produce Austrian version of the German dat
% \changes{germanb-2.6f}{1997/10/01}{Use \cs{edef} to def
%    \cs{today} to save memo
% \changes{germanb-2.6f}{1998/03/28}{use \cs{def} instead
%    \cs{ede
%    \begin{macroco
\def\dateaustrian{\def\today{\number\day.~\ifnum1=\mo
  J\"anner\else \month@german\fi \space\number\yea
%    \end{macroco
%  \end{mac

%  \begin{macro}{\extrasgerm
%  \begin{macro}{\extrasaustri
% \changes{germanb-2.0b}{1991/05/29}{added some comment chars
%    prevent white spa
% \changes{germanb-2.2}{1991/06/11}{Save all redefined macr
%  \begin{macro}{\noextrasgerm
%  \begin{macro}{\noextrasaustri
% \changes{germanb-1.1}{1990/07/30}{Added \cs{dieresi
% \changes{germanb-2.0b}{1991/05/29}{added some comment chars
%    prevent white spa
% \changes{germanb-2.2}{1991/06/11}{Try to restore everything to
%    former sta
% \changes{germanb-2.6d}{1996/07/10}{Construct control seque
%    \cs{extrasgerman} or \cs{extrasaustrian} on the f

%    Either the macro |\extrasgerman| or the macros |\extrasaustri
%    will perform all the extra definitions needed for the Ger
%    language. The macro |\noextrasgerman| is used to cancel
%    actions of |\extrasgerman

%    For German (as well as for Dutch) the \texttt{"} character
%    made active. This is done once, later on its definition may va
%    \begin{macroco
\initiate@active@char
\@namedef{extras\CurrentOption
  \languageshorthands{germa
\expandafter\addto\csname extras\CurrentOption\endcsnam
  \bbl@activate{
%    \end{macroco
%    Don't forget to turn the shorthands off aga
% \changes{germanb-2.6i}{1999/12/16}{Deactivate shorthands ouside
%    Germ
%    \begin{macroco
\addto\noextrasgerman{\bbl@deactivate{
%    \end{macroco

% \changes{germanb-2.6a}{1995/02/15}{All the code to handle the act
%    double quote has been moved to \file{babel.de

%    In order for \TeX\ to be able to hyphenate German words wh
%    contain `\ss' (in the \texttt{OT1} position |^^Y|) we have
%    give the character a nonzero |\lccode| (see Appendix H, the \
%    boo
% \changes{germanb-2.6c}{1996/04/08}{Use decimal number instead
%    hat-notation as the hat may be activat
%    \begin{macroco
\expandafter\addto\csname extras\CurrentOption\endcsnam
  \babel@savevariable{\lccode2
  \lccode25=
%    \end{macroco
% \changes{germanb-2.6a}{1995/02/15}{Removeed \cs{3} as it is
%    longer in \file{german.ld

%    The umlaut accent macro |\"| is changed to lower the umlaut do
%    The redefinition is done with the help of |\umlautlo
%    \begin{macroco
\expandafter\addto\csname extras\CurrentOption\endcsnam
  \babel@save\"\umlautl
\@namedef{noextras\CurrentOption}{\umlauthi
%    \end{macroco
%    The german hyphenation patterns can be used with |\lefthyphenm
%    and |\righthyphenmin| set to
% \changes{germanb-2.6a}{1995/05/13}{use \cs{germanhyphenmins} to st
%    the correct valu
% \changes{germanb-2.6j}{2000/09/22}{Now use \cs{providehyphenmins}
%    provide a default val
%    \begin{macroco
\providehyphenmins{\CurrentOption}{\tw@\t
%    \end{macroco
%    For German texts we need to make sure that |\frenchspacing|
%    turned
% \changes{germanb-2.6k}{2001/01/26}{Turn frenchspacing on, as
%    \texttt{german.st
%    \begin{macroco
\expandafter\addto\csname extras\CurrentOption\endcsnam
  \bbl@frenchspaci
\expandafter\addto\csname noextras\CurrentOption\endcsnam
  \bbl@nonfrenchspaci
%    \end{macroco
%  \end{mac
%  \end{mac
%  \end{mac
%  \end{mac

% \changes{germanb-2.6a}{1995/02/15}{\cs{umlautlow}
%    \cs{umlauthigh} moved to \file{glyphs.dtx}, as well
%    \cs{newumlaut} (now \cs{lower@umlau

%    The code above is necessary because we need an extra act
%    character. This character is then used as indicated
%    table~\ref{tab:german-quot

%    To be able to define the function of |"|, we first defin
%    couple of `support' macr

% \changes{germanb-2.3e}{1991/11/10}{Added \cs{save@sf@q} macro
%    rewrote all quote macros to use
% \changes{germanb-2.3h}{1991/02/16}{moved definition
%    \cs{allowhyphens}, \cs{set@low@box} and \cs{save@sf@q}
%    \file{babel.co
% \changes{germanb-2.6a}{1995/02/15}{Moved all quotation characters
%    \file{glyphs.dt

%  \begin{macro}{\
%    We save the original double quote character in |\dq| to k
%    it available, the math accent |\"| can now be typed as |
%    \begin{macroco
\begingroup \catcode`\
\def\x{\endgr
  \def\@SS{\mathchar"701
  \def\dq{

%    \end{macroco
%  \end{mac
% \changes{germanb-2.6c}{1996/01/24}{Moved \cs{german@dq@disc}
%    babel.def, calling it \cs{bbl@dis

% \changes{germanb-2.6a}{1995/02/15}{Use \cs{ddot} instead
%    \cs{@MATHUMLAU

%    Now we can define the doublequote macros: the umlau
% \changes{germanb-2.6c}{1996/05/30}{added the \cs{allowhyphen
%    \begin{macroco
\declare@shorthand{german}{"a}{\textormath{\"{a}\allowhyphens}{\ddot
\declare@shorthand{german}{"o}{\textormath{\"{o}\allowhyphens}{\ddot
\declare@shorthand{german}{"u}{\textormath{\"{u}\allowhyphens}{\ddot
\declare@shorthand{german}{"A}{\textormath{\"{A}\allowhyphens}{\ddot
\declare@shorthand{german}{"O}{\textormath{\"{O}\allowhyphens}{\ddot
\declare@shorthand{german}{"U}{\textormath{\"{U}\allowhyphens}{\ddot
%    \end{macroco
%    trem
%    \begin{macroco
\declare@shorthand{german}{"e}{\textormath{\"{e}}{\ddot
\declare@shorthand{german}{"E}{\textormath{\"{E}}{\ddot
\declare@shorthand{german}{"i}{\textormath{\"{\i
                              {\ddot\imat
\declare@shorthand{german}{"I}{\textormath{\"{I}}{\ddot
%    \end{macroco
%    german es-zet (sharp
% \changes{germanb-2.6f}{1997/05/08}{use \cs{SS} instead
%    \texttt{SS}, removed braces after \cs{ss
%    \begin{macroco
\declare@shorthand{german}{"s}{\textormath{\ss}{\@SS{
\declare@shorthand{german}{"S}{\
\declare@shorthand{german}{"z}{\textormath{\ss}{\@SS{
\declare@shorthand{german}{"Z}{
%    \end{macroco
%    german and french quot
%    \begin{macroco
\declare@shorthand{german}{"`}{\gl
\declare@shorthand{german}{"'}{\gr
\declare@shorthand{german}{"<}{\fl
\declare@shorthand{german}{">}{\fr
%    \end{macroco
%    discretionary comma
%    \begin{macroco
\declare@shorthand{german}{"c}{\textormath{\bbl@disc ck}{
\declare@shorthand{german}{"C}{\textormath{\bbl@disc CK}{
\declare@shorthand{german}{"F}{\textormath{\bbl@disc F{FF}}{
\declare@shorthand{german}{"l}{\textormath{\bbl@disc l{ll}}{
\declare@shorthand{german}{"L}{\textormath{\bbl@disc L{LL}}{
\declare@shorthand{german}{"m}{\textormath{\bbl@disc m{mm}}{
\declare@shorthand{german}{"M}{\textormath{\bbl@disc M{MM}}{
\declare@shorthand{german}{"n}{\textormath{\bbl@disc n{nn}}{
\declare@shorthand{german}{"N}{\textormath{\bbl@disc N{NN}}{
\declare@shorthand{german}{"p}{\textormath{\bbl@disc p{pp}}{
\declare@shorthand{german}{"P}{\textormath{\bbl@disc P{PP}}{
\declare@shorthand{german}{"r}{\textormath{\bbl@disc r{rr}}{
\declare@shorthand{german}{"R}{\textormath{\bbl@disc R{RR}}{
\declare@shorthand{german}{"t}{\textormath{\bbl@disc t{tt}}{
\declare@shorthand{german}{"T}{\textormath{\bbl@disc T{TT}}{
%    \end{macroco
%    We need to treat |"f| a bit differently in order to preserve
%    ff-ligatur
% \changes{germanb-2.6f}{1998/06/15}{Copied the coding for \texttt{
%    from german.dtx version 2.5
%    \begin{macroco
\declare@shorthand{german}{"f}{\textormath{\bbl@discff}{
\def\bbl@discff{\penalty
  \afterassignment\bbl@insertff \let\bbl@nextff
\def\bbl@insertf
  \if f\bbl@nex
    \expandafter\@firstoftwo\else\expandafter\@secondoftwo
  {\relax\discretionary{ff-}{f}{ff}\allowhyphens}{f\bbl@nextf
\let\bbl@nextf
%    \end{macroco
%    and some additional comman
%    \begin{macroco
\declare@shorthand{german}{"-}{\nobreak\-\bbl@allowhyphe
\declare@shorthand{german}{"|
  \textormath{\penalty\@M\discretionary{-}{}{\kern.03e
              \allowhyphens}
\declare@shorthand{german}{""}{\hskip\z@sk
\declare@shorthand{german}{"~}{\textormath{\leavevmode\hbox{-}}{
\declare@shorthand{german}{"=}{\penalty\@M-\hskip\z@sk
%    \end{macroco

%  \begin{macro}{\mdq
%  \begin{macro}{\mdqo
%  \begin{macro}{\
%    All that's left to do now is to  define a couple of comma
%    for reasons of compatibility with \file{german.st
% \changes{germanb-2.6f}{1998/06/07}{Now use \cs{shorthandon}
%    \cs{shorthandoff
%    \begin{macroco
\def\mdqon{\shorthandon{
\def\mdqoff{\shorthandoff{
\def\ck{\allowhyphens\discretionary{k-}{k}{ck}\allowhyphe
%    \end{macroco
%  \end{mac
%  \end{mac
%  \end{mac

%    The macro |\ldf@finish| takes care of looking fo
%    configuration file, setting the main language to be switched
%    at |\begin{document}| and resetting the category code
%    \texttt{@} to its original val
% \changes{germanb-2.6d}{1996/11/02}{Now use \cs{ldf@finish} to w
%    u
%    \begin{macroco
\ldf@finish\CurrentOpt
%</co
%    \end{macroco

% \Fin

%% \CharacterTa
%%  {Upper-case    \A\B\C\D\E\F\G\H\I\J\K\L\M\N\O\P\Q\R\S\T\U\V\W\X\
%%   Lower-case    \a\b\c\d\e\f\g\h\i\j\k\l\m\n\o\p\q\r\s\t\u\v\w\x\
%%   Digits        \0\1\2\3\4\5\6\7\
%%   Exclamation   \!     Double quote  \"     Hash (number)
%%   Dollar        \$     Percent       \%     Ampersand
%%   Acute accent  \'     Left paren    \(     Right paren
%%   Asterisk      \*     Plus          \+     Comma
%%   Minus         \-     Point         \.     Solidus
%%   Colon         \:     Semicolon     \;     Less than
%%   Equals        \=     Greater than  \>     Question mark
%%   Commercial at \@     Left bracket  \[     Backslash
%%   Right bracket \]     Circumflex    \^     Underscore
%%   Grave accent  \`     Left brace    \{     Vertical bar
%%   Right brace   \}     Tilde

\endin
}
\bbl@tempa{hebrew}{%
  \input{rlbabel.def}%
  % \iffalse meta-comment
%
% Copyright 1989-2005 Johannes L. Braams and any individual authors
% listed elsewhere in this file.  All rights reserved.
% 
% This file is part of the Babel system.
% --------------------------------------
% 
% It may be distributed and/or modified under the
% conditions of the LaTeX Project Public License, either version 1.3
% of this license or (at your option) any later version.
% The latest version of this license is in
%   http://www.latex-project.org/lppl.txt
% and version 1.3 or later is part of all distributions of LaTeX
% version 2003/12/01 or later.
% 
% This work has the LPPL maintenance status "maintained".
% 
% The Current Maintainer of this work is Johannes Braams.
% 
% The list of all files belonging to the Babel system is
% given in the file `manifest.bbl. See also `legal.bbl' for additional
% information.
% 
% The list of derived (unpacked) files belonging to the distribution
% and covered by LPPL is defined by the unpacking scripts (with
% extension .ins) which are part of the distribution.
% \fi
% \CheckSum{3345}
%
% \iffalse meta-comment
%% Hebrew language definition and additional packages.
%% Copyright (C) 1997 -- 2005 Boris Lavva.
%
%% Babel package for LaTeX version 2e
%% Copyright (C) 1989 -- 2005 by Johannes Braams,
%%                            TeXniek
%%                            All rights reserved.
%<*calendar>
%% TeX & LaTeX macros for computing Hebrew date from Gregorian one
%% Copyright (C) 1991 by Michail Rozman, misha@iop.tartu.ew.su
%%
%</calendar>
% \fi
%
%
% \iffalse
%<hebrew>\ProvidesFile{hebrew.ldf}
%<rightleft>\ProvidesFile{rlbabel.def}
%<calendar>\ProvidesPackage{hebcal}
%<*driver>
\ProvidesFile{hebrew.drv}
%</driver>
% \fi
% \ProvidesFile{hebrew.dtx}
        [2005/03/30 v2.3h %
% \iffalse
%<hebrew>         Hebrew language definition from the babel system
%<rightleft>         Right-to-Left support from the babel system
%<calendar>         Hebrew calendar
%<driver>         Driver file for hebrew support
% \fi
    Hebrew language support from the babel system]
%
% \iffalse
% \subsection{A driver for this document}
%
% The next bit of code contains the documentation driver file for
% \TeX{}, i.e., the file that will produce the documentation you are
% currently reading. It will be extracted from this file by the \dst{}
%  program.
%
%    \begin{macrocode}
%<*driver>
\documentclass{ltxdoc}
\providecommand\babel{\textsf{babel}}
\providecommand\file[1]{\texttt{#1}}
\makeatletter
%    \end{macrocode}
%
%    The code lines are numbered within sections,
%    \begin{macrocode}
\@addtoreset{CodelineNo}{section}
\renewcommand\theCodelineNo{%
  \reset@font\scriptsize\thesection.\arabic{CodelineNo}}
%    \end{macrocode}
%    which should also be visible in the index; hence this
%    redefinition of a macro from \file{doc.sty}.
%    \begin{macrocode}
\renewcommand\codeline@wrindex[1]{\if@filesw
        \immediate\write\@indexfile
            {\string\indexentry{#1}%
            {\number\c@section.\number\c@CodelineNo}}\fi}
%    \end{macrocode}
%
%    The glossary environment is used or the change log, but its
%    definition needs changing for this document.
%    \begin{macrocode}
\renewenvironment{theglossary}{%
    \glossary@prologue%
    \GlossaryParms \let\item\@idxitem \ignorespaces}%
   {}
\makeatother
\DisableCrossrefs
\CodelineIndex
\RecordChanges
\title{Hebrew language support from the \babel\ system}
\author{Boris Lavva}
\date{Printed \today}
\begin{document}
   \maketitle
   \tableofcontents
   \DocInput{hebrew.dtx}
   \DocInput{hebinp.dtx}
   \DocInput{hebrew.fdd}
   \DocInput{heb209.dtx}
   \clearpage
   \def\filename{index}
   \PrintIndex
   \clearpage
   \def\filename{changes}
   \PrintChanges
\end{document}
%</driver>
%    \end{macrocode}
% \fi
%
% \providecommand\babel{\textsf{babel}}
% \providecommand\dst{\textsc{docstrip}}
% \providecommand\file[1]{\texttt{#1}}
% \providecommand\pkg[1]{\texttt{#1}}
% \providecommand\XeT{X\kern-.125em\lower.5ex\hbox{E}\kern-.1667emT\@}
% \providecommand\scrunch{\setlength{\itemsep}{-.05in}}
% \GetFileInfo{hebrew.dtx}
%
% \changes{hebrew~0.1}{??/??/??}{%
%    Preliminary \LaTeX\ Hebrew option (by Sergio Fogel)}
% \changes{hebrew~0.2}{??/??/??}{%
%    Corrections and additions (by Rama Porrat)}
% \changes{hebrew~0.6}{??/??/??}{Additions (by Yael Dubinsky)}
% \changes{hebrew~1.2}{??/??/??}{%
%    Bilingual tables, penalties, documentation and more changes
%    (by Yaniv Bargury)}
% \changes{hebrew~1.30}{1992/05/15}{%
%    Font selection, various (by Alon Ziv)}
% \changes{hebrew~1.31}{1993/02/22}{Bug fixes (by Alon Ziv)}
% \changes{hebrew~1.32}{1993/03/10}{Made font-change command 
%    for numbers `\cs{protect}'ed (by Alon Ziv)}
% \changes{hebrew~1.33}{1993/03/11}{%
%    Made \cs{refstepcounter} work using \cs{@ltor} (by Alon Ziv)}
% \changes{hebrew~1.34}{1993/03/22}{%
%    Moved font loading to another file. Added \cs{mainsec}. 
%    Made all text strings be produced by control codes (similar to
%    \LaTeX 2.09 Mar '92). Fixed \cs{noindent} (by Alon Ziv)}
% \changes{hebrew~1.35}{1993/03/22}{%
%    Moved the texts to a file selected by the current encoding 
%    (by Alon Ziv)}
% \changes{hebrew~1.36}{1993/03/24}{Use \TeX\ tricks to redefine 
%    \cs{theXXXX} without keeping old definitions.
%    Use only \cs{@eng} for direction/font change (removed \cs{@ltor}).
%    Switched from use of \cs{mainsec} to code taken from \babel\
%    system (by Alon Ziv)}
% \changes{hebrew~1.37}{1993/03/28}{%
%    Use \cs{add@around} in defining font size commands. Small bug
%    fixes (by Alon Ziv)}
% \changes{hebrew~1.38}{1993/04/20}{%
%    \cs{everypar} changed so that \cs{noindent} works unmodified 
%    (by Alon Ziv, thanks to Chris Rowley)}
% \changes{hebrew~1.39}{1993/08/10}{%
%    Redefined primitive sectioning commands. Changed \cs{include} so
%    it finds \texttt{.h}, \texttt{.xet}, and \texttt{.ltx} files (no
%    extension needed). Reinstated use of \cs{@ltor} (by Alon Ziv)}
% \changes{hebrew~1.40}{1993/09/01}{Added the \cs{@brackets} hack
%    (by Alon Ziv)}
% \changes{hebrew~1.41}{1993/09/09}{%
%    Reworked towards using NFSS2. Changed some macro names to be more
%    logical: renamed \cs{@ltor} to \cs{@number}, \cs{@eng} to
%    \cs{@latin}, and (in \texttt{hebrew.ldf}) \cs{@heb} to
%    \cs{@hebrew} (by Alon Ziv)}
% \changes{hebrew~1.42}{1993/09/22}{%
%    Made list environments work better. Fixed thebibliography
%    environment (by Alon Ziv)}
% \changes{hebrew~2.0a}{1998/01/01}{%
%    Completely rewritten for \LaTeXe\ and \babel\ support. Various
%    input and font encodings (with NFSS2) are supported too. The
%    original \pkg{hebrew.sty} is divided to a number of packages and
%    definition files for better readability and extensibility. Added
%    some user- and code-level documentation inside the \texttt{.dtx}
%    and \texttt{.fdd} files, and \LaTeX -driven installation with
%    \pkg{hebrew.ins} (by Boris Lavva)}
% \changes{hebrew~2.1}{2000/11/23}{%
%    corrections from Sivan Toledo: sender name in letter, and section name in
%    headings. (by Tzafrir Cohen)}
% \changes{hebrew~2.2}{2000/12/11}{%
%    renamed hebrew letters to heb* (e.g.: alef renamed to hebalef)
%    (by Tzafrir Cohen)}
% \changes{hebrew~2.3}{2001/02/27}{
%    added several \cs{@ifclassloaded}\{slides\} to allow the use of the
%    slides class. (by Tzafrir Cohen)}
% \changes{hebrew~2.3a}{2001/07/09}{
%    The documentation should now be built fine (broken since at least 
%    2.1, and probably 2.0) (by Tzafrir Cohen)}
% \changes{hebrew~2.3b}{2001/08/16}{
%    minor clean-ups. The documentation builds now with no warnings.
%    Also removed \cs{R} from the caption macro (added in 2.1)
%    Added internal \cs{@ensure@L} and \cs{@ensure@R} 
%    (Is there a real need for them? Maybe should they be exposed?)
%    (by Tzafrir Cohen)}
% \changes{hebrew~2.3c}{2001/10/05}{
%    a temporary fix to the \cs{gim} macro. Should be replaced by stuff 
%    from hebcal.
%    (by Tzafrir Cohen)}
% \changes{hebrew~2.3d}{2002/01/04}{
%    Initial support for the prosper class. Added \cs{arabicnorl} .
%    (by Tzafrir Cohen)}
% \changes{hebrew~2.3e}{2002/08/09}{
%    Removing hebtech from this distriution (not relevant to babel),
%    added \cs{HeblatexEncoding}. some docs cleanup
%    (by Tzafrir Cohen)}
% \changes{hebrew~2.3f}{2002/12/26}{
%    redefined \cs{list} instead of redefining every environment 
%    that uses it. some pscolor handling, removed HeblatexEncoding 
%    (don't use 2.3e) (by Tzafrir Cohen)}
% \changes{hebrew~2.3g}{2003/06/05}{
%    Reimplemented the printing of Hebrew numerals and Hebrew
%    counters; modified \pkg{hebcal.sty} to use this implementation
%    when typesetting Hebrew dates; added option |full| to package
%    \pkg{hebcal}; also removed some gratuitous
%    spaces inserted by \pkg{hebcal.sty} by adding comment marks.
%    CAUTION: the changes to \pkg{hebcal.sty} make it dependent on
%    \pkg{babel} and not useable as a stand-alone package. Is this a
%    problem? (by Ron Artstein)}
%
% \section{The Hebrew language}\label{sec:hebrew}
%
%    The file \file{\filename}\footnote{The Hebrew language support
%    files described in this section have version number \fileversion\
%    and were last revised on \filedate.} provides the following
%    packages and files for Hebrew language support:
%    \begin{description}
%    \item[\file{hebrew.ldf}] file defines all the language-specific
%    macros for the Hebrew language. 
%    \item[\file{rlbabel.def}] file is used by |hebrew.ldf| for
%    bidirectional versions of the major \LaTeX{} commands and
%    environments. It is designed to be used with other right-to-left
%    languages, not only with Hebrew.
%    \item[\pkg{hebcal.sty}] package defines a set of macros for
%    computing Hebrew date from Gregorian one.
%    \end{description}
%
%    Additional Hebrew input and font encoding definition files that
%    should be included and used with \file{hebrew.ldf} are:
%    \begin{description}
%    \item[\file{hebinp.dtx}] provides Hebrew input encodings, such as
%          ISO 8859-8, MS Windows codepage 1255 or IBM PC codepage 862
%          (see Section~\ref{sec:hebinp} on page~\pageref{sec:hebinp}).
%    \item[\file{hebrew.fdd}] contains Hebrew font encodings, related
%          font definition files and \pkg{hebfont} package that
%          provides Hebrew font switching commands (see
%          Section~\ref{sec:hebfdd} on page~\pageref{sec:hebfdd} for
%          further details).
%    \end{description}
%
%    \LaTeX~2.09 compatibility files are included with
%    \file{heb209.dtx} and gives possibility to compile existing
%    \LaTeX~2.09 Hebrew documents with small (if any) changes (see
%    Section~\ref{sec:heb209} on page~\pageref{sec:heb209} for
%    details).
%
%    Finally, optional document class \pkg{hebtech} may be useful for
%    writing theses and dissertations in both Hebrew and English (and
%    any other languages included with \babel). It designed to meet
%    requirements of the Graduate School of the Technion --- Israel
%    Institute of Technology. 
%
%    \emph{As of version 2.3e hebtech is no longer distributed together
%    with heblatex. It should be part of a new "hebclasses" package}
%
% \subsection{Acknowledgement}
%
%    The following people have contributed to Hebrew package in one
%    way or another, knowingly or unknowingly. In alphabetical order:
%    Irina Abramovici, Yaniv Bargury, Yael Dubinsky, Sergio Fogel,
%    Dan Haran, Rama Porrat, Michail Rozman, Alon Ziv.
%
%    Tatiana Samoilov and Vitaly Surazhsky found a number of serious
%    bugs in preliminary version of Hebrew package.
%
%    A number of other people have contributed comments and
%    information. Specific contributions are acknowledged within the
%    document.
%
%    I want to thank my wife, Vita, and son, Mishka, for their
%    infinite love and patience.
%
%    If you made a contribution and I haven't mentioned it, don't
%    worry, it was an accident. I'm sorry. Just tell me and I will add
%    you to the next version.
%
% \StopEventually{}
%
% \subsection{The {\normalfont\dst{}} modules}
%
%    The following modules are used in the implementation to direct
%    \dst{} in generating external files:
% \begin{center}
% \begin{tabular}{@{}ll}
%   driver    & produce a documentation driver file \\[4pt]
%   hebrew    & produce Hebrew language support file\\
%   rightleft & create right-to-left support file\\
%   calendar  & create Hebrew calendar package
% \end{tabular}
% \end{center}
%    A typical \dst{} command file would then have entries like:
%    \begin{quote}
%       |\generateFile{hebrew.ldf}{t}{\from{hebrew.dtx}{hebrew}}|
%    \end{quote}
%
% \subsection{Hebrew language definitions}
%
%    The macro |\LdfInit| takes care of preventing that this file is
%    loaded more than once, checking the category code of the |@|
%    sign, etc.
%    \begin{macrocode}
%<*hebrew>
\LdfInit{hebrew}{captionshebrew}
%    \end{macrocode}
%
%    When this file is read as an option, i.e., by the |\usepackage|
%    command, |hebrew| will be an `unknown' language, in which case we
%    have to make it known. So we check for the existence of
%    |\l@hebrew| to see whether we have to do something here. 
%
%    \begin{macrocode}
\ifx\l@hebrew\@undefined
  \@nopatterns{Hebrew}%
  \adddialect\l@hebrew0
\fi
%    \end{macrocode}
%
%  \begin{macro}{\hebrewencoding}
%    \emph{FIX DOCS REGARDING 8BIT}
%
%    Typesetting Hebrew texts implies that a special input and output
%    encoding needs to be used. Generally, the user may choose
%    between different available Hebrew encodings provided. The
%    current support for Hebrew uses all available fonts from the
%    Hebrew University of Jerusalem encoded in `old-code' 7-bit
%    encoding also known as Israeli Standard SI-960. We define for
%    these fonts the Local Hebrew Encoding |LHE| (see the file
%    |hebrew.fdd| for more details), and the |LHE| encoding definition
%    file should be loaded by default.
%
%    Other fonts are available in windows-cp1255 (a superset of ISO-8859-8
%    with nikud). For those, the encoding |HE8| should be used. Such fonts
%    are, e.g., windows' TrueType fonts (once cnverted to Type1 or MetaFont)
%    and IBM's Type1 fonts.
%
%    However, if an user wants to use another font encoding, for
%    example, cyrillic encoding T2 and extended latin encoding T1, ---
%    he/she has to load the corresponding file \emph{before} the
%    \pkg{hebrew} package. This may be done in the following way:
%    \begin{quote}
%      |\usepackage[LHE,T2,T1]{fontenc}|\\
%      |\usepackage[hebrew,russian,english]{babel}|
%    \end{quote}
%    We make sure that the |LHE| encoding is known to \LaTeX{} at end
%    of this package.
%
%    Also note that if you want to use the encoding |HE8| , you should define 
%    the following in your document, \emph{before loading babel}:
%    \begin{quote}
%      |\def\HeblatexEncoding{HE8}|\\
%      |\def\HeblatexEncodingFile{he8enc}|
%    \end{quote}
% \changes{hebrew-2.3h}{2004/02/20}{Make LHE the default encoding for
%    compatibility reasons}
%    \begin{macrocode}
\providecommand{\HeblatexEncoding}{LHE}%
\providecommand{\HeblatexEncodingFile}{lheenc}%
\newcommand{\heblatex@set@encoding}[2]{
}
\AtEndOfPackage{%
  \@ifpackageloaded{fontenc}{%
    \@ifl@aded{def}{%
      \HeblatexEncodingFile}{\def\hebrewencoding{\HeblatexEncoding}}{}%
  }{%
    \input{\HeblatexEncodingFile.def}%
    \def\hebrewencoding{\HeblatexEncoding}%
  }}
%    \end{macrocode}
%  \end{macro}
%
%    We also need to load inputenc package with one of the Hebrew
%    input encodings. By default, we set up the |8859-8| codepage.
%    If an user wants to use many input encodings in the same
%    document, for example, the MS Windows Hebrew codepage |cp1255|
%    and the standard IBM PC Russian codepage |cp866|, he/she has to
%    load the corresponding file \emph{before} the hebrew package
%    too. This may be done in the following way:
%    \begin{quote}
%      |\usepackage[cp1255,cp866]{inputenc}|\\
%      |\usepackage[hebrew,russian,english]{babel}|
%    \end{quote}
%
%    An user can switch input encodings in the document using the
%    command |\inputencoding|, for example, to use the |cp1255|:
%    \begin{quote}
%       |\inputencoding{cp1255}|
%    \end{quote}
%    \begin{macrocode}
\AtEndOfPackage{%
  \@ifpackageloaded{inputenc}{}{\RequirePackage[8859-8]{inputenc}}}
%    \end{macrocode}
%
%    The next step consists of defining commands to switch to (and
%    from) the Hebrew language.
%
%  \begin{macro}{\hebrewhyphenmins}
%    This macro is used to store the correct values of the hyphenation
%    parameters |\lefthyphenmin| and |\righthyphenmin|. They are set
%    to~2.
% \changes{hebrew~2.0b}{2000/09/22}{Now use \cs{providehyphenmins} to
%    provide a default value}
%    \begin{macrocode}
\providehyphenmins{\CurrentOption}{\tw@\tw@}
%    \end{macrocode}
%  \end{macro}
%
% \begin{macro}{\captionshebrew}
%    The macro |\captionshebrew| replaces all captions used in the four
%    standard document classes provided with \LaTeXe with their Hebrew
%    equivalents.
% \changes{hebrew-2.0b}{2000/09/20}{Added \cs{glossaryname}}
%    \begin{macrocode}
\addto\captionshebrew{%
  \def\prefacename{\@ensure@R{\hebmem\hebbet\hebvav\hebalef}}%
  \def\refname{\@ensure@R{\hebresh\hebshin\hebyod\hebmem\hebtav\ %
    \hebmem\hebqof\hebvav\hebresh\hebvav\hebtav}}%
  \def\abstractname{\@ensure@R{\hebtav\hebqof\hebtsadi\hebyod\hebresh}}%
  \def\bibname{\@ensure@R{\hebbet\hebyod\hebbet\heblamed\hebyod\hebvav%
    \hebgimel\hebresh\hebpe\hebyod\hebhe}}%
  \def\chaptername{\@ensure@R{\hebpe\hebresh\hebqof}}%
  \def\appendixname{\@ensure@R{\hebnun\hebsamekh\hebpe\hebhet}}%
  \def\contentsname{\@ensure@R{%
    \hebtav\hebvav\hebkaf\hebfinalnun\ %
    \hebayin\hebnun\hebyod\hebyod\hebnun\hebyod\hebfinalmem}}%
  \def\listfigurename{\@ensure@R{%
    \hebresh\hebshin\hebyod\hebmem\hebtav\ %
    \hebalef\hebyod\hebvav\hebresh\hebyod\hebfinalmem}}%
  \def\listtablename{\@ensure@R{%
    \hebresh\hebshin\hebyod\hebmem\hebtav\
    \hebtet\hebbet\heblamed\hebalef\hebvav\hebtav}}%
  \def\indexname{\@ensure@R{\hebmem\hebpe\hebtav\hebhet}}%
  \def\figurename{\@ensure@R{\hebalef\hebyod\hebvav\hebresh}}%
  \def\tablename{\@ensure@R{\hebtet\hebbet\heblamed\hebhe}}%
  \def\partname{\@ensure@R{\hebhet\heblamed\hebqof}}%
  \def\enclname{\@ensure@R{\hebresh\hebtsadi"\hebbet}}%
  \def\ccname{\@ensure@R{\hebhe\hebayin\hebtav\hebqof\hebyod%
    \hebfinalmem}}%
  \def\headtoname{\@ensure@R{\hebalef\heblamed}}%
  \def\pagename{\@ensure@R{\hebayin\hebmem\hebvav\hebdalet}}%
  \def\psname{\@ensure@R{\hebnun.\hebbet.}}%
  \def\seename{\@ensure@R{\hebresh\hebalef\hebhe}}%
  \def\alsoname{\@ensure@R{\hebresh\hebalef\hebhe \hebgimel%
    \hebmemesof}}%
  \def\proofname{\@ensure@R{\hebhe\hebvav\hebkaf\hebhet\hebhe}}
  \def\glossaryname{\@ensure@L{Glossary}}% <-- Needs translation
}
%    \end{macrocode}
% \end{macro}
%  \begin{macro}{\slidelabel}
%    Here we fix the macro |slidelabel| of the seminar package. Note 
%    that this still won't work well enough when overlays will be 
%    involved
%    \begin{macrocode}
\@ifclassloaded{seminar}{%
  \def\slidelabel{\bf \if@rl\R{\hebshin\hebqof\hebfinalpe{} \theslide}%
                      \else\L{Slide \theslide}%
                      \fi}%
}{}
%    \end{macrocode}
% \end{macro}
%
%    Here we provide an user with translation of Gregorian dates
%    to Hebrew. In addition, the \pkg{hebcal} package can be used
%    to create Hebrew calendar dates.
%
%  \begin{macro}{\hebmonth}
%    The macro |\hebmonth{|\emph{month}|}| produces month names in
%    Hebrew.
%    \begin{macrocode}
\def\hebmonth#1{%
  \ifcase#1\or \hebyod\hebnun\hebvav\hebalef\hebresh\or %
     \hebpe\hebbet\hebresh\hebvav\hebalef\hebresh\or %
     \hebmem\hebresh\hebfinaltsadi\or %
     \hebalef\hebpe\hebresh\hebyod\heblamed\or %
     \hebmem\hebalef\hebyod\or \hebyod\hebvav\hebnun\hebyod\or %
     \hebyod\hebvav\heblamed\hebyod\or %
     \hebalef\hebvav\hebgimel\hebvav\hebsamekh\hebtet\or %
     \hebsamekh\hebpe\hebtet\hebmem\hebbet\hebresh\or %
     \hebalef\hebvav\hebqof\hebtet\hebvav\hebbet\hebresh\or %
     \hebnun\hebvav\hebbet\hebmem\hebbet\hebresh\or %
     \hebdalet\hebtsadi\hebmem\hebbet\hebresh\fi}
%    \end{macrocode}
%  \end{macro}
%
%  \begin{macro}{\hebdate}
%    The macro |\hebdate{|\emph{day}|}{|\emph{month}|}{|\emph{year}|}|
%    translates a given Gregorian date to Hebrew.
%    \begin{macrocode}
\def\hebdate#1#2#3{%
  \beginR\beginL\number#1\endL\ \hebbet\hebmonth{#2}
         \beginL\number#3\endL\endR}
%    \end{macrocode}
%  \end{macro}
%
%  \begin{macro}{\hebday}
%    The macro |\hebday| will replace |\today| command when in Hebrew
%    mode.
%    \begin{macrocode}
\def\hebday{\hebdate{\day}{\month}{\year}}
%    \end{macrocode}
%  \end{macro}
%
% \begin{macro}{\datehebrew}
%    The macro |\datehebrew| redefines the command |\today| to produce
%    Gregorian dates in Hebrew. It uses the macro |\hebday|.
%    \begin{macrocode}
\def\datehebrew{\let\today=\hebday}
%    \end{macrocode}
% \end{macro}
%
%    The macro |\extrashebrew| will perform all the extra definitions
%    needed for the Hebrew language. The macro |\noextrashebrew|
%    is used to cancel the actions of |\extrashebrew|.
%
% \begin{macro}{\extrashebrew}
%    We switch font encoding to Hebrew and direction to
%    right-to-left. We cannot use the regular language switching
%    commands (for example, |\sethebrew| and |\unsethebrew| or
%    |\selectlanguage{hebrew}|), when in restricted horizontal mode,
%    because it will result in \emph{unbalanced} |\beginR| or
%    |\beginL| primitives.
%    Instead, in \TeX 's restricted horizontal mode, the
%    |\L{|\emph{latin text}|}| and |\R{|\emph{hebrew text}|}|, or
%    |\embox{|\emph{latin text}|}| and |\hmbox{|\emph{hebrew text}|}|
%    should be used.
%
%    Hence, we use |\beginR| and |\beginL| switching commands only
%    when not in restricted horizontal mode.
%    \begin{macrocode}
\addto\extrashebrew{%
  \tohebrew%
  \ifhmode\ifinner\else\beginR\fi\fi}
%    \end{macrocode}
% \end{macro}
%
% \begin{macro}{\noextrashebrew}
%    The macro |\noextrashebrew| is used to cancel the actions of
%    |\extrashebrew|. We switch back to the previous font encoding and
%    restore left-to-right direction.
%    \begin{macrocode}
\addto\noextrashebrew{%
  \fromhebrew%
  \ifhmode\ifinner\else\beginL\fi\fi}
%    \end{macrocode}
% \end{macro}
%
%    Generally, we can switch to- and from- Hebrew by means of
%    standard \babel -defined commands, for example,
%    \begin{quote}
%       |\selectlanguage{hebrew}|
%    \end{quote}
%    or
%    \begin{quote}
%       |\begin{otherlanguage}{hebrew}|\\
%       \hspace*{1.5em} some Hebrew text\\
%       |\end{otherlanguage}|
%    \end{quote}
%    Now we define two additional commands that offer the possibility
%    to switch to and from Hebrew language. These commands are
%    backward compatible with the previous versions of
%    \pkg{hebrew.sty}.
%
%  \begin{macro}{\sethebrew}
%  \begin{macro}{\unsethebrew}
%    The command |\sethebrew| will switch from the current font encoding
%    to the hebrew font encoding, and from the current direction of
%    text to the right-to-left mode. The command |\unsethebrew| switches
%    back.
%
%    Both commands use standard right-to-left switching macros
%    |\setrllanguage{|\emph{ r2l language name}|}| and
%    |\unsetrllanguage{|\emph{r2l language name}|}|, that
%    defined in the \file{rlbabel.def} file.
%    \begin{macrocode}
\def\sethebrew{\setrllanguage{hebrew}}
\def\unsethebrew{\unsetrllanguage{hebrew}}
%    \end{macrocode}
%  \end{macro}
%  \end{macro}
%
%  \begin{macro}{\hebrewtext}
%  \begin{macro}{\nohebrewtext}
%    The following two commands are \emph{obsolete} and work only
%    in \LaTeX 2.09 compatibility mode. They are synonyms of
%    |\sethebrew| and |\unsethebrew| defined above.
%    \begin{macrocode}
\if@compatibility
  \let\hebrewtext=\sethebrew
  \let\nohebrewtext=\unsethebrew
\fi
%    \end{macrocode}
%  \end{macro}
%  \end{macro}
%
%  \begin{macro}{\tohebrew}
%  \begin{macro}{\fromhebrew}
%    These two commands change only the current font encoding to- and
%    from- Hebrew encoding. Their implementation uses
%    |\@torl{|\emph{language name}|}| and |\@fromrl| macros defined in
%    \file{rlbabel.def} file. Both commands may be useful \emph{only}
%    for package and class writers, not for regular users.
%    \begin{macrocode}
\def\tohebrew{\@torl{hebrew}}%
\def\fromhebrew{\@fromrl}
%    \end{macrocode}
%  \end{macro}
%  \end{macro}
%
%  \begin{macro}{\@hebrew}
%    Sometimes we need to preserve Hebrew mode without knowing in
%    which environment we are located now. For these cases, the
%    |\@hebrew{|\emph{hebrew text}|}| macro will be useful. Not that
%    this macro is similar to the |\@number| and |\@latin| macros
%    defined in \file{rlbabel.def} file.
%    \begin{macrocode}
\def\@@hebrew#1{\beginR{{\tohebrew#1}}\endR}
\def\@hebrew{\protect\@@hebrew}
%    \end{macrocode}
%  \end{macro}
%
%  \subsubsection{Hebrew numerals}
%
%    We provide commands to print numbers in the traditional
%    notation using Hebrew letters. We need commands that print 
%    a Hebrew number from a decimal input, as well as commands 
%    to print the value of a counter as a Hebrew number.  
%  \begin{macro}{\if@gim@apost}
%  \begin{macro}{\if@gim@final}
%    Hebrew numbers can be written in various styles: with or without
%    apostrophes, and with the letters kaf, mem, nun, pe, tsadi as either
%    final or initial forms when they are the last letters in the
%    sequence. We provide two flags to set the style options.
%    \begin{macrocode}
\newif\if@gim@apost  % whether we print apostrophes
\newif\if@gim@final  % whether we use final or initial letters
%    \end{macrocode}
%  \end{macro}
%  \end{macro}
%  \begin{macro}{\hebrewnumeral}
%  \begin{macro}{\Hebrewnumeral}
%  \begin{macro}{\Hebrewnumeralfinal}
%    The commands that print a Hebrew number 
%    must specify the style locally: relying on a global style
%    option could cause a counter to
%    print in an inconsistent manner---for instance, page numbers
%    might appear in different styles if the global style option
%    changed mid-way through a document.
%    The commands only allow three of the four possible flag
%    combinations (I do not know of a use that requires the
%    combination of final letters and no apostrophes --RA).
%
%    Each command sets the style flags and calls |\@hebrew@numeral|.
%    Double braces are used in order to protect the values of
%    |\@tempcnta| and |\@tempcntb|, which are changed by this call;
%    they also keep the flag assignments local (this is not important
%    because the global values are never used).
%    \begin{macrocode}
\newcommand*{\hebrewnumeral}[1]      % no apostrophe, no final letters
 {{\@gim@finalfalse\@gim@apostfalse\@hebrew@numeral{#1}}}
\newcommand*{\Hebrewnumeral}[1]      % apostrophe, no final letters
 {{\@gim@finalfalse\@gim@aposttrue\@hebrew@numeral{#1}}}
\newcommand*{\Hebrewnumeralfinal}[1] % apostrophe, final letters
 {{\@gim@finaltrue\@gim@aposttrue\@hebrew@numeral{#1}}}
%    \end{macrocode}
%  \end{macro}
%  \end{macro}
%  \end{macro}
%  \begin{macro}{\alph}
%  \begin{macro}{\@alph}
%  \begin{macro}{\Alph}
%  \begin{macro}{\@Alph}
%  \begin{macro}{\Alphfinal}
%  \begin{macro}{\@Alphfinal}
%    Counter-printing commands are based on the above commands. The
%    natural name for the counter-printing commands is |\alph|, because
%    Hebrew numerals are the only way to represent numbers with
%    Hebrew letters (kaf always means~20, never~11). Hebrew has no
%    uppercase letters, hence no need for the familiar meaning of |\Alph|;
%    we therefore define |\alph| to print counters as Hebrew numerals
%    without apostrophes, and |\Alph| to print with apostrophes. A third
%    form, |\Alphfinal|, is provided to print with apostrophes and final
%    letters, as is required for Hebrew year designators. The commands
%    |\alph| and |\Alph| are defined in \pkg{latex.ltx}, and we only
%    need to redefine the internal commands |\@alph| and 
%    |\@Alph|; for |\Alphfinal| we need to provide both a wrapper and
%    an internal command. 
%    The counter printing commands are made semi-robust: without the
%    |\protect|, commands like |\theenumii| break (I'm not quite clear
%    on why this happens, --RA); at the same time, we cannot make the 
%    commands too robust (e.g.~with |\DeclareRobustCommand|) because
%    this would enter the command name rather than its value into
%    files like |.aux|, |.toc| etc\@.
%    The old meanings of meaning of |\@alph| and |\@Alph| are saved
%    upon entering Hebrew mode and restored upon exiting it.
%    \begin{macrocode}
\addto\extrashebrew{%
  \let\saved@alph=\@alph%
  \let\saved@Alph=\@Alph%
  \renewcommand*{\@alph}[1]{\protect\hebrewnumeral{\number#1}}%
  \renewcommand*{\@Alph}[1]{\protect\Hebrewnumeral{\number#1}}%
  \def\Alphfinal#1{\expandafter\@Alphfinal\csname c@#1\endcsname}%
  \providecommand*{\@Alphfinal}[1]{\protect\Hebrewnumeralfinal{\number#1}}}
\addto\noextrashebrew{%
  \let\@alph=\saved@alph%
  \let\@Alph=\saved@Alph}
%    \end{macrocode}
%    Note that |\alph| (without apostrophes) is already the
%    appropriate choice for the second-level enumerate label, and
%    |\Alph| (with apostrophes) is an appropriate choice for appendix;
%    however, the default \LaTeX\ labels need to be redefined for
%    appropriate cross-referencing, see below.
%    \LaTeX\ default class files specify |\Alph| for
%    the fourth-level enumerate level, this should probably be changed.
%    Also, the way labels get flushed left by default looks inappropriate
%    for Hebrew numerals, so we should redefine |\labelenumii| as well
%    as |\labelenumiv| (presently not implemented).
%  \end{macro}
%  \end{macro}
%  \end{macro}
%  \end{macro}
%  \end{macro}
%  \end{macro}
%  \begin{macro}{\theenumii}
%  \begin{macro}{\theenumiv}
%  \begin{macro}{\label}
%    Cross-references to counter labels need to be printed according
%    to the language environment in which a label was issued, not
%    the environment in which it is called: for example, a label~(1b) 
%    issued in a Latin environment should be referred to as~(1b) in a
%    Hebrew text, and label~(2dalet) issued in a Hebrew environment
%    should be referred to as~(2dalet) in a Latin text. This was the
%    unanimous opinion in a poll sent to the Ivri\TeX\ list. 
%    We therefore redefine |\theenumii| and |\theenumiv|, so that an
%    explicit language instruction gets written to the |.aux| file.
%    \begin{macrocode}
\renewcommand{\theenumii}
   {\if@rl\protect\hebrewnumeral{\number\c@enumii}%
    \else\protect\L{\protect\@@alph{\number\c@enumii}}\fi}
\renewcommand{\theenumiv}
   {\if@rl\protect\Hebrewnumeral{\number\c@enumiv}%
    \else\protect\L{\protect\@@Alph{\number\c@enumiv}}\fi}
%    \end{macrocode}
%    We also need to control for the font and direction in which a
%    counter label is printed. Direction is straightforward: a Latin
%    label like~(1b) should be written left-to-right when called in a
%    Hebrew text, and a Hebrew label like~(2dalet) should be written
%    right-to-left when called in a Latin text. The font question is
%    more delicate, because we should decide whether the numerals
%    should be typeset in the font of the language enviroment in which
%    the label was issued, or that of the environment in which it is
%    called. 
%    \begin{itemize}
%     \item
%      A purely numeric label like~(23) looks best if it is set in the
%      font of the surrounding language.
%     \item
%      But a mixed alphanumeric label like~(1b) lookes weird if
%      the~`1' is taken from the Hebrew font; likewise, (2dalet) looks
%      weird if the~`2' is taken from a Latin font.
%     \item
%      Finally, mixing the two possibilities is worst, because a
%      single Hebrew sentence referring to examples~(1b) and~(2) would
%      take the~`1' from the Latin font and the~`2' from the Hebrew
%      font, and this looks really awful. (It is also very hard to
%      implement). 
%    \end{itemize}
%    In light of the conflicting considerations it seems like there's
%    no perfect solution. I have chosen to implement the top option,
%    where numerals are taken from the font of the surrounding
%    language, because it seems to me that reference to purely numeric
%    labels is the most common, so this gives a good solution to the
%    majority of cases and a mediocre solution to the minority.
%
%    We redefine the |\label| command which writes to the
%    |.aux| file. Depending on the language environment we issue
%    appropriate |\beginR/L|$\cdots$|\endR/L| commands to control the
%    direction without affecting the font. Since these commands do not
%    affect the value of |\if@rl|, we cannot use the macro
%    |\@brackets| to determine the correct brackets to be used with
%    |\p@enumiii|; instead, we let the language environment determine an
%    explicit definition.
%    \begin{macrocode}
\def\label#1{\@bsphack
  \if@rl
    \def\p@enumiii{\p@enumii)\theenumii(}%
    \protected@write\@auxout{}%
         {\string\newlabel{#1}{{\beginR\@currentlabel\endR}{\thepage}}}%
  \else
    \def\p@enumiii{\p@enumii(\theenumii)}%
    \protected@write\@auxout{}%
         {\string\newlabel{#1}{{\beginL\@currentlabel\endL}{\thepage}}}%
  \fi
  \@esphack}
%    \end{macrocode}
%    NOTE: it appears that the definition of |\label| is
%    language-independent and thus belongs in \pkg{rlbabel.def}, but
%    this is not the case. The decision to typeset label numerals
%    in the font of the surrounding language is reasonable for Hebrew,
%    because mixed-font (1b) and (2dalet) are somewhat acceptable. The
%    same may not be acceptable for Arabic, whose numeral glyphs are
%    radically different from those in the Latin fonts. The decision
%    about the direction may also be different for Arabic, which is
%    more right-to-left oriented than Hebrew (two examples: dates like
%    15/6/2003 are written left-to-right in Hebrew but right-to-left
%    in Arabic; equations like $1+2=3$ are written left-to-right in
%    Hebrew but right-to-left in Arabic elementary school textbooks
%    using Arabic numeral glyphs). My personal hunch is that a label
%    like~(1b) in an Arabic text would be typeset left-to-right if
%    the~`1' is a Western glyph, but right-to-left if the~`1' is an
%    Arabic glyph. But this is just a guess, I'd have to ask Arab
%    typesetters to find the correct answer. --RA.
%  \end{macro}
%  \end{macro}
%  \end{macro}
%  \begin{macro}{\appendix}
%    The following code provides for the proper printing of appendix
%    numbers in tables of contents. Section and chapter headings are
%    normally bilingual: regardless of the text language, the author
%    supplies each section/chapter with two headings---one for the
%    Hebrew table of contents and one for the Latin table of contents.
%    It makes sense that the label should be a Latin letter in the
%    Latin table of contents and a Hebrew letter in the Hebrew table
%    of contents. The definition is similar to that of |\theenumii|
%    and |\theenumiv| above, but additional |\protect| commands ensure
%    that the entire condition is written the |.aux| file. The
%    appendix number will therefore be typeset according to the
%    environment in which it is used rather than issued: a Hebrew
%    number (with apostrophes) in a Hebrew environment and a Latin
%    capital letter in a Latin environment (the command 
%    |\@@Alph| is set in \pkg{rlbabel.def} to hold the default meaning
%    of \LaTeX\ [latin] |\@Alph|, regardless of the mode in which it is
%    issued). The net result is that
%    the second appendix will be marked with~`B' in the Latin table of
%    contents and with `bet' in the Hebrew table of contents; the mark
%    in the main text will depend on the language of the appendix itself.
%    \begin{macrocode}
\@ifclassloaded{letter}{}{%
\@ifclassloaded{slides}{}{%
  \let\@@appendix=\appendix%
  \@ifclassloaded{article}{%
    \renewcommand\appendix{\@@appendix%
      \renewcommand\thesection
        {\protect\if@rl\protect\Hebrewnumeral{\number\c@section}%
         \protect\else\@@Alph\c@section\protect\fi}}}
   {\renewcommand\appendix{\@@appendix%
      \renewcommand\thechapter
        {\protect\if@rl\protect\Hebrewnumeral{\number\c@chapter}%
         \protect\else\@@Alph\c@chapter\protect\fi}}}}}
%    \end{macrocode}
%    QUESTION: is this also the appropriate way to refer to an
%    appendix in the text, or should we retain the original label the
%    same way we did with |enumerate| labels? 
%    ANOTHER QUESTION: are similar redefinitions needed for other
%    counters that generate texts in bilingual lists like |.lof/.fol|
%    and |.lot/.tol|? --RA.
%  \end{macro}
%  \begin{macro}{\@hebrew@numeral}
%    The command |\@hebrew@numeral| prints a Hebrew number. The groups
%    of thousands, millions, billions are separated by apostrophes and
%    typeset without apostrophes or final letters; the remainder
%    (under 1000) is typeset conventionally, with the selected styles
%    for apostrophes and final letters. 
%    The function calls on |\gim@no@mil| to typeset each
%    three-digit block. The algorithm 
%    is recursive, but the maximum recursion depth is~4 because \TeX\
%    only allows numbers up to $2^{31}-1 = 2{,}147{,}483{,}647$.
%    The typesetting routine is wrapped in |\@hebrew| in order to
%    ensure that numbers are always typeset in Hebrew mode.
%
%    Initialize: |\@tempcnta| holds the value, |\@tempcntb| is used for
%    calculations.
%    \begin{macrocode}
\newcommand*{\@hebrew@numeral}[1]
{\@hebrew{\@tempcnta=#1\@tempcntb=#1\relax
 \divide\@tempcntb by 1000
%    \end{macrocode}
%    If we're under 1000, call |\gim@nomil|
%    \begin{macrocode}
 \ifnum\@tempcntb=0\gim@nomil\@tempcnta\relax
%    \end{macrocode}
%    If we're above 1000 then force no apostrophe and no final letter
%    styles for the value above~1000, recur for the value above~1000,
%    add an apostrophe, and call |\gim@nomil| for the remainder.
%    \begin{macrocode}
 \else{\@gim@apostfalse\@gim@finalfalse\@hebrew@numeral\@tempcntb}'%
      \multiply\@tempcntb by 1000\relax
      \advance\@tempcnta by -\@tempcntb\relax
      \gim@nomil\@tempcnta\relax
 \fi
}}
%    \end{macrocode}
%    NOTE: is it the case that 15,000 and 16,000 are written as
%    yod-he and yod-vav, rather than tet-vav and tet-zayin? This
%    vaguely rings a bell, but I'm not certain. If this is the case,
%    then the current behavior is incorrect and should be changed. --RA.
%  \end{macro}
%  \begin{macro}{\gim@nomil}
%    The command |\gim@nomil| typesets an integer between 0~and~999
%    (for~0 it typesets nothing). The code has been modified from the
%    old |hebcal.sty|
%    (appropriate credits---Boris Lavva and Michail Rozman ?).
%    |\@tempcnta| holds the total value that remains to be typeset.
%    At each stage we find the highest valued letter that is 
%    less than or equal to |\@tempcnta|, and call on |\gim@print| to
%    subtract this value and print the letter.
%
%    Initialize: |\@tempcnta| holds the value, there is no previous
%    letter. 
%    \begin{macrocode}
\newcommand*{\gim@nomil}[1]{\@tempcnta=#1\@gim@prevfalse
%    \end{macrocode}
% Find the hundreds digit.
%    \begin{macrocode}
  \@tempcntb=\@tempcnta\divide\@tempcntb by 100\relax % hundreds digit
  \ifcase\@tempcntb                     % print nothing if no hundreds
     \or\gim@print{100}{\hebqof}%
     \or\gim@print{200}{\hebresh}%
     \or\gim@print{300}{\hebshin}%
     \or\gim@print{400}{\hebtav}%
     \or\hebtav\@gim@prevtrue\gim@print{500}{\hebqof}%
     \or\hebtav\@gim@prevtrue\gim@print{600}{\hebresh}%
     \or\hebtav\@gim@prevtrue\gim@print{700}{\hebshin}%
     \or\hebtav\@gim@prevtrue\gim@print{800}{\hebtav}%
     \or\hebtav\@gim@prevtrue\hebtav\gim@print{900}{\hebqof}%
  \fi
%    \end{macrocode}
%    Find the tens digit. The numbers 15 and 16 are traditionally
%    printed as tet-vav ($9+6$) and tet-zayin ($9+7$) to avoid
%    spelling the Lord's name.
%    \begin{macrocode}
  \@tempcntb=\@tempcnta\divide\@tempcntb by 10\relax      % tens digit
  \ifcase\@tempcntb                         % print nothing if no tens
      \or                                   % number between 10 and 19
              \ifnum\@tempcnta = 16 \gim@print {9}{\hebtet}% tet-zayin
         \else\ifnum\@tempcnta = 15 \gim@print {9}{\hebtet}% tet-vav
         \else                      \gim@print{10}{\hebyod}%
              \fi % \@tempcnta = 15
              \fi % \@tempcnta = 16
%    \end{macrocode}
%    Initial or final forms are selected according to the current
%    style option; |\gim@print| will force a non-final letter in
%    non-final position by means of a local style change.
%    \begin{macrocode}
      \or\gim@print{20}{\if@gim@final\hebfinalkaf\else\hebkaf\fi}%
      \or\gim@print{30}{\heblamed}%
      \or\gim@print{40}{\if@gim@final\hebfinalmem\else\hebmem\fi}%
      \or\gim@print{50}{\if@gim@final\hebfinalnun\else\hebnun\fi}%
      \or\gim@print{60}{\hebsamekh}%
      \or\gim@print{70}{\hebayin}%
      \or\gim@print{80}{\if@gim@final\hebfinalpe\else\hebpe\fi}%
      \or\gim@print{90}{\if@gim@final\hebfinaltsadi\else\hebtsadi\fi}%
  \fi
%    \end{macrocode}
%    Print the ones digit.
%    \begin{macrocode}
  \ifcase\@tempcnta                         % print nothing if no ones
      \or\gim@print{1}{\hebalef}%
      \or\gim@print{2}{\hebbet}%
      \or\gim@print{3}{\hebgimel}%
      \or\gim@print{4}{\hebdalet}%
      \or\gim@print{5}{\hebhe}%
      \or\gim@print{6}{\hebvav}%
      \or\gim@print{7}{\hebzayin}%
      \or\gim@print{8}{\hebhet}%
      \or\gim@print{9}{\hebtet}%
  \fi
}
%    \end{macrocode}
%  \end{macro}
%  \begin{macro}{\gim@print}
%  \begin{macro}{\if@gim@prev}
%    The actual printing routine typesets a digit with the appropriate
%    apostrophes: if a number sequence consists of a
%    single letter then it is followed by a single apostrophe, and if
%    it consists of more than one letter then a double
%    apostrophe is inserted before the last letter.
%    We typeset the letters one at a time, keeping a flag that tells
%    us if any previous letters had been typeset.
%    \begin{macrocode}
\newif\if@gim@prev % flag if a previous letter has been typeset
%    \end{macrocode}
%    For each letter, we
%    first subtract its value from the total. Then, 
%    \begin{itemize}
%     \item
%      if the result is zero then this is the last letter; we check
%      the flag to see if this is the only letter and print it with
%      the appropriate apostrophe;
%     \item
%      if the result is not zero then there remain additional letters
%      to be typeset; we print without an apostrophe and set the
%      `previous letter' flag. 
%    \end{itemize}
%    |\@tempcnta| holds the total value that remains to be typeset.
%    We first deduct the letter's value from |\@tempcnta|,
%    so |\@tempcnta| is zero if and only if this is the last letter.
%    \begin{macrocode}
\newcommand*{\gim@print}[2]{%   #2 is a letter, #1 is its value.
  \advance\@tempcnta by -#1\relax% deduct the value from the remainder
%    \end{macrocode}
%    If this is the last letter, we print with the appropriate
%    apostrophe (depending on the style option):
%    if there is a preceding letter, print |"x| if the style calls for
%    apostrophes, |x| if it doesn't;
%    otherwise, this is the only letter: print |x'| if the style calls
%    for apostrophes, |x| if it doesn't.
%    \begin{macrocode}
  \ifnum\@tempcnta=0% if this is the last letter
     \if@gim@prev\if@gim@apost"\fi#2%
     \else#2\if@gim@apost'\fi\fi%
%    \end{macrocode}
%    If this is not the last letter: print a non-final form (by
%    forcing a local style option) and set the `previous letter' flag.
%    \begin{macrocode}
  \else{\@gim@finalfalse#2}\@gim@prevtrue\fi}
%    \end{macrocode}
%  \end{macro}
%  \end{macro}
%
%  \begin{macro}{\hebr}
%  \begin{macro}{\gim}
%    The older Hebrew counter commands |\hebr| and |\gim| are retained
%    in order to keep older documents from breaking. They are set to
%    be equivalent to |\alph|, and their use is deprecated. Note that
%    |\hebr| gives different results than it had in the past---it
%    now typesets 11 as yod-alef rather than kaf.
%    \begin{macrocode}
\let\hebr=\alph
\let\gim=\alph
%    \end{macrocode}
%  \end{macro}
%  \end{macro}
%
%    For backward compatibility with `older' \pkg{hebrew.sty}
%    packages, we define Hebrew equivalents of some useful \LaTeX\
%    commands. Note, however, that 8-bit macros defined in Hebrew
%    are no longer supported.
%    \begin{macrocode}
\def\hebcopy{\protect\R{\hebhe\hebayin\hebtav\hebqof}}
\def\hebincl{\protect\R{\hebresh\hebtsadi"\hebbet}}
\def\hebpage{\protect\R{\hebayin\hebmem\hebvav\hebdalet}}
\def\hebto{\protect\R{\hebayin\hebdalet}}
%    \end{macrocode}
%    |\hadgesh| produce ``poor man's bold'' (heavy printout), when
%    used with normal font glyphs. It is advisable to use bold font
%    (for example, \emph{Dead Sea}) instead of this macro.
%    \begin{macrocode}
\def\hadgesh#1{\leavevmode\setbox0=\hbox{#1}%
  \kern-.025em\copy0\kern-\wd0
  \kern.05em\copy0\kern-\wd0
  \kern-.025em\raise.0433em\box0 }
%    \end{macrocode}
%    |\piska| and |\piskapiska| sometimes used in `older' hebrew
%    sources, and should not be used in \LaTeXe.
%    \begin{macrocode}
\if@compatibility
  \def\piska#1{\item{#1}\hangindent=-\hangindent}
  \def\piskapiska#1{\itemitem{#1}\hangindent=-\hangindent}
\fi
%    \end{macrocode}
%    The following commands are simply synonyms for the standard ones,
%    provided with \LaTeXe.
%    \begin{macrocode}
\let\makafgadol=\textendash
\let\makafanak=\textemdash
\let\geresh=\textquoteright
\let\opengeresh=\textquoteright
\let\closegeresh=\textquoteleft
\let\openquote=\textquotedblright
\let\closequote=\textquotedblleft
\let\leftquotation=\textquotedblright
\let\rightquotation=\textquotedblleft
%    \end{macrocode}
%
%    We need to ensure that Hebrew is used as the default
%    right-to-left language at |\begin{document}|. The mechanism of
%    defining the |\@rllanguagename| is the same as in \babel 's
%    |\languagename|: the last right-to-left language in the
%    |\usepackage{babel}| line is set as the default right-to-left
%    language at document beginning.
%
%    For example, the following code:
%    \begin{quote}
%       |\usepackage[russian,hebrew,arabic,greek,english]{babel}|
%    \end{quote}
%    will set the Arabic language as the default right-to-left
%    language and the English language as the default language.
%    As a result, the commands |\L{}| and |\embox{}| will use English
%    and |\R{}| and |\hmbox{}| will use Arabic by default. These
%    defaults can be changed with the next |\sethebrew| or
%    |\selectlanguage{|\emph{language name}|}| command.
%    \begin{macrocode}
\AtBeginDocument{\def\@rllanguagename{hebrew}}
%    \end{macrocode}
%    
%    The macro |\ldf@finish| takes care of looking for a configuration
%    file, setting the main language to be switched on at
%    |\begin{document}| and resetting the category code of |@| to its
%    original value.
%    \begin{macrocode}
\ldf@finish{hebrew}
%</hebrew>
%    \end{macrocode}
%
% \subsection{Right to left support}
%
%    This file \pkg{rlbabel.def} defines necessary bidirectional macro
%    support for \LaTeXe. It is designed for use not only with Hebrew,
%    but with any Right-to-Left languages, supported by \babel. The
%    macros provided in this file are language and encoding
%    independent.
%
%    Right-to-left languages will use \TeX\ extensions, namely \TeX\
%    primitives |\beginL|, |\endL| and |\beginR|, |\endR|, currently
%    implemented only in $\varepsilon$-\TeX\ and in \TeX{-}{-}\XeT.
%
%    If $\varepsilon$-\TeX\ is used, we should switch it to the
%    \emph{enhanced} mode:
%    \begin{macrocode}
%<*rightleft>
\ifx\TeXXeTstate\undefined\else%
   \TeXXeTstate=1
\fi
%    \end{macrocode}
%
%    Note, that $\varepsilon$-\TeX 's format file should be created
%    for \emph{extended} mode. Mode can be checked by running
%    $\varepsilon$-\TeX\ on some \TeX{} file, for example:
%    \begin{quote}
%    |This is e-TeX, Version 3.14159-1.1 (Web2c 7.0)|\\
%    |entering extended mode|
%    \end{quote}
%    The second line should be \texttt{entering extended mode}.
%
%    We check if user uses Right-to-Left enabled engine instead of
%    regular Knuth's \TeX:
%    \begin{macrocode}
\ifx\beginL\@undefined%
   \newlinechar`\^^J
   \typeout{^^JTo avoid this error message,^^J%
     run TeX--XeT or e-TeX engine instead of regular TeX.^^J}
   \errmessage{Right-to-Left Support Error: use TeX--XeT or e-TeX
     engine}%
\fi
%    \end{macrocode}
%
% \subsubsection{Switching from LR to RL mode and back}
% 
%    \cs{@torl} and \cs{@fromrl} are called each time the horizontal
%    direction changes. They do all that is necessary besides changing
%    the direction. Currently their task is to change the encoding
%    information and mode (condition \cs{if@rl}). They should not
%    normally be called by users: user-level macros, such as
%    \cs{sethebrew} and \cs{unsethebrew}, as well as \babel 's
%    \cs{selectlanguage} are defined in language-definition files and
%    should be used to change default language (and direction). 
%    
%    Local direction changing commands (for small pieces of text):
%    |\L{}|, |\R{}|, |\embox{}| and |\hmbox{}| are defined below in
%    this file in language-independent manner.
%
% \begin{macro}{\if@rl}
%    \begin{description}\scrunch
%       \item[|\@rltrue|] means that the main mode is currently
%                          Right-to-Left.
%       \item[|\@rlfalse|] means that the main mode is currently
%                          Left-to-Right.
%    \end{description}
%    \begin{macrocode}
\newif\if@rl
%    \end{macrocode}
% \end{macro}
%
% \begin{macro}{\if@rlmain}
%    This is the main direction of the document. Unlike |\if@rl|
%    it is set once and never changes.
%    \begin{description}\scrunch
%       \item[|\@rltrue|]  means that the document is Right-to-Left.
%       \item[|\@rlfalse|] means that the document is Left-to-Right.
%    \end{description}
%    Practically |\if@rlmain| is set according to the value of |\if@rl|
%    in the beginning of the run.
%    \begin{macrocode}
\AtBeginDocument{% Here we set the main document direction
  \newif\if@rlmain% 
  \if@rl% e.g: if the options to babel were [english,hebrew]
    \@rlmaintrue%
  \else%  e.g: if the options to babel were [hebrew,english]
    \@rlmainfalse%
  \fi%
}
%    \end{macrocode}
% \end{macro}
%
% \begin{macro}{\@torl}
%    Switches current direction to Right-to-Left: saves current
%    Left-to-Right encoding in |\lr@encodingdefault|, sets required
%    Right-to-Left language name in |\@rllanguagename| (similar to
%    \babel 's |\languagename|) and changes derection.
%
%    The Right-to-Left language encoding should be defined in |.ldf|
%    file as special macro created by concatenation of the language
%    name and string \texttt{encoding}, for example, for Hebrew it
%    will be |\hebrewencoding|.
%    \begin{macrocode}
\DeclareRobustCommand{\@torl}[1]{%
  \if@rl\else%
     \let\lr@encodingdefault=\encodingdefault%
  \fi%
  \def\@rllanguagename{#1}%
  \def\encodingdefault{\csname#1encoding\endcsname}%
  \fontencoding{\encodingdefault}%
  \selectfont%
  \@rltrue}
%    \end{macrocode}
% \end{macro}
%
% \begin{macro}{\@fromrl}
%    Opposite to |\@torl|, switches current direction to
%    Left-to-Right: restores saved Left-to-Right encoding
%    (|\lr@encodingdefault|) and changes direction.
%    \begin{macrocode}
\DeclareRobustCommand{\@fromrl}{%
  \if@rl%
     \let\encodingdefault=\lr@encodingdefault%
  \fi%
  \fontencoding{\encodingdefault}%
  \selectfont%
  \@rlfalse}
%    \end{macrocode}
% \end{macro}
%
% \begin{macro}{\selectlanguage}
%    This standard \babel 's macro should be redefined to support
%    bidirectional tables. We divide |\selectlanguage| implementation
%    to two parts, and the first part calls the second
%    |\@@selectlanguage|.
%    \begin{macrocode}
\expandafter\def\csname selectlanguage \endcsname#1{%
  \edef\languagename{%
    \ifnum\escapechar=\expandafter`\string#1\@empty
    \else \string#1\@empty\fi}%
  \@@selectlanguage{\languagename}}
%    \end{macrocode}
% \end{macro}
%
% \begin{macro}{\@@selectlanguage}
%    This new internal macro redefines a final part of the standard
%    \babel 's |\select|\-|language| implementation.
%
%    Standard \LaTeX\ provides us with 3 tables: Table of Contents
%    (|.toc|), List of Figures (|.lof|), and List of Tables
%    (|.lot|). In multi-lingual texts mixing Left-to-Right languages
%    with Right-to-Left ones, the use of various directions in one
%    table results in very ugly output. Therefore, these 3 standard
%    tables will be used now only for Left-to-Right languages, and we
%    will add 3 Right-to-Left tables (their extensions are simply
%    reversed ones): RL Table of Contents (|.cot|), RL List of Figures
%    (|.fol|), and RL List of Tables (|.lof|).
%    \begin{macrocode}
\def\@@selectlanguage#1{%
  \select@language{#1}%
  \if@filesw
     \protected@write\@auxout{}{\string\select@language{#1}}%
     \if@rl%
       \addtocontents{cot}{\xstring\select@language{#1}}%
       \addtocontents{fol}{\xstring\select@language{#1}}%
       \addtocontents{tol}{\xstring\select@language{#1}}%     
     \else%
       \addtocontents{toc}{\xstring\select@language{#1}}%
       \addtocontents{lof}{\xstring\select@language{#1}}%
       \addtocontents{lot}{\xstring\select@language{#1}}%
     \fi%
  \fi}
%    \end{macrocode}
% \end{macro}
%
% \begin{macro}{\setrllanguage}
% \begin{macro}{\unsetrllanguage}
%    The |\setrllanguage| and |\unsetrllanguage| pair of macros is
%    proved to very useful in bilingual texts, for example, in
%    Hebrew-English texts. The language-specific commands, for example,
%    |\sethebrew| and |\unsethebrew| use these macros as basis.
%    
%    Implementation saves and restores other language in
%    |\other@languagename| variable, and uses internal macro
%    |\@@selectlanguage|, defined above, to switch between languages.
%    \begin{macrocode}
\let\other@languagename=\languagename
\DeclareRobustCommand{\setrllanguage}[1]{%
   \if@rl\else%
     \let\other@languagename=\languagename%
   \fi%
     \def\languagename{#1}%
     \@@selectlanguage{\languagename}}
%    \end{macrocode}
%
%    \begin{macrocode}
\DeclareRobustCommand{\unsetrllanguage}[1]{%
   \if@rl%
     \let\languagename=\other@languagename%
   \fi
   \@@selectlanguage{\languagename}}
%    \end{macrocode}
% \end{macro}
% \end{macro}
%
% \begin{macro}{\L}
% \begin{macro}{\R}
% \begin{macro}{\HeblatexRedefineL}
%    Macros for changing direction, originally taken from TUGboat.
%    Usage: |\L{|\emph{Left to Right text}|}| and |\R{|\emph{Right to
%    Left text}|}|. Numbers should also be enclosed in |\L{}|, as in
%    |\L{123}|.
%
%    Note, that these macros do not receive language name as
%    parameter. Instead, the saved |\@rllanguagename| will be
%    used. We assume that each Right-to-Left language defines
%    |\to|\emph{languagename} and |\from|\emph{languagename} macros in
%    language definition file, for example, for Hebrew: |\tohebrew|
%    and |\fromhebrew| macros in \pkg{hebrew.ldf} file.
%
%    The macros \cs{L} and \cs{R} include `protect' to to make them robust and
%    allow use, for example, in tables.
%
%    Due to the fact that some packages have different definitions for \cs{L}
%    the macro |\HeblatexRedefineL| is provided to overide them. This may
%    be required with hyperref, for instance.
%    \begin{macrocode}
\let\next=\
\def\HeblatexRedefineL{%
  \def\L{\protect\pL}%
}
\HeblatexRedefineL
\def\pL{\protect\afterassignment\moreL \let\next= }
\def\moreL{\bracetext \aftergroup\endL \beginL\csname
  from\@rllanguagename\endcsname}
%    \end{macrocode}
%
%    \begin{macrocode}
\def\R{\protect\pR}
\def\pR{\protect\afterassignment\moreR \let\next= }
\def\moreR{\bracetext \aftergroup\endR \beginR\csname
  to\@rllanguagename\endcsname}
\def\bracetext{\ifcat\next{\else\ifcat\next}\fi
  \errmessage{Missing left brace has been substituted}\fi \bgroup}
\everydisplay{\if@rl\aftergroup\beginR\fi }
%    \end{macrocode}
% \end{macro}
% \end{macro}
% \end{macro}
%
% \begin{macro}{\@ensure@R}
% \begin{macro}{\@ensure@L}
%    Two small internal macros, a-la |\ensuremath|
%    \begin{macrocode}
\def\@ensure@R#1{\if@rl#1\else\R{#1}\fi}
\def\@ensure@L#1{\if@rl\L{#1}\else#1\fi}
%    \end{macrocode}
% \end{macro}
% \end{macro}
%
%    Take care of Right-to-Left indentation in every paragraph.
%    Originally, \cs{noindent} was redefined for right-to-left by
%    Yaniv Bargury, then the implementation was rewritten by Alon Ziv
%    using an idea by Chris Rowley: \cs{noindent} now works
%    unmodified.
%    \begin{macrocode}
\def\rl@everypar{\if@rl{\setbox\z@\lastbox\beginR\usebox\z@}\fi}
\let\o@everypar=\everypar
\def\everypar#1{\o@everypar{\rl@everypar#1}}
%    \end{macrocode}
%
% \begin{macro}{\hmbox}
% \begin{macro}{\embox}
%    Useful vbox commands. All text in math formulas is best enclosed
%    in these: LR text in |\embox| and RL text in |\hmbox|. |\mbox{}|
%    is useless for both cases, since it typesets in Left-to-Right
%    even for Right-to-Left languages (additions by Yaniv Bargury). 
%    \begin{macrocode}
\newcommand{\hmbox}[1]{\mbox{\R{#1}}}
\newcommand{\embox}[1]{\mbox{\L{#1}}}
%    \end{macrocode}
% \end{macro}
% \end{macro}
%
% \begin{macro}{\@brackets}
%    When in Right-to-Left mode, brackets should be swapped. This
%    macro receives 3 parameters: left bracket, content, right
%    bracket. Brackets can be square brackets, braces, or
%    parentheses.
%    \begin{macrocode}
\def\@brackets#1#2#3{\protect\if@rl #3#2#1\protect\else
  #1#2#3\protect\fi}
%    \end{macrocode}
% \end{macro}
%
% \begin{macro}{\@number}
% \begin{macro}{\@latin}
%    \cs{@number} preserves numbers direction from Left to Right.
%    \cs{@latin} in addition switches current encoding to the latin.
%    \begin{macrocode}
\def\@@number#1{\ifmmode\else\beginL\fi#1\ifmmode\else\endL\fi}
\def\@@latin#1{\@@number{{\@fromrl#1}}}
\def\@number{\protect\@@number}
\def\@latin{\protect\@@latin}
%    \end{macrocode}
% \end{macro}
% \end{macro}
%
% \subsubsection{Counters}
% 
%     To make counter references work in Right to Left text, we need
%     to surround their original definitions with an
%     |\@number{|\ldots|}| or |\@latin{|\ldots|}|. Note, that
%     language-specific counters, such as \cs{hebr} or \cs{gim} are
%     provided with language definition file.
%
%    We start with saving the original definitions:
%    \begin{macrocode}
\let\@@arabic=\@arabic
\let\@@roman=\@roman
\let\@@Roman=\@Roman
\let\@@alph=\@alph
\let\@@Alph=\@Alph
%    \end{macrocode}
%
% \begin{macro}{\@arabic}
% \begin{macro}{\@roman}
% \begin{macro}{\@Roman}
%    Arabic and roman numbers should be from Left to Right. In
%    addition, roman numerals, both lower- and upper-case should be in
%    latin encoding.
%    \begin{macrocode}
\def\@arabic#1{\@number{\@@arabic#1}}
\def\@roman#1{\@latin{\@@roman#1}}
\def\@Roman#1{\@latin{\@@Roman#1}}
%    \end{macrocode}
% \end{macro}
% \end{macro}
% \end{macro}
%
% \begin{macro}{\arabicnorl}
% This macro preserves the original definition of |\arabic|
% (overrides the overriding of |\@arabic|)
%    \begin{macrocode}
\def\arabicnorl#1{\expandafter\@@arabic\csname c@#1\endcsname}
%    \end{macrocode}
% \end{macro}
%
% \begin{macro}{\make@lr}
%    In Right to Left documents all counters defined in the standard
%    document classes \emph{article}, \emph{report} and \emph{book}
%    provided with \LaTeXe, such as |\thesection|, |\thefigure|,
%    |\theequation| should be typed as numbers from left to right. To
%    ensure direction, we use the following
%    |\make@lr{|\emph{counter}|}| macro:
%    \begin{macrocode}
\def\make@lr#1{\begingroup
    \toks@=\expandafter{#1}%
    \edef\x{\endgroup
  \def\noexpand#1{\noexpand\@number{\the\toks@}}}%
  \x}
%    \end{macrocode}
%
%    \begin{macrocode}
\@ifclassloaded{letter}{}{%
  \@ifclassloaded{slides}{}{%
    \make@lr\thesection
    \make@lr\thesubsection
    \make@lr\thesubsubsection
    \make@lr\theparagraph
    \make@lr\thesubparagraph
    \make@lr\thefigure
    \make@lr\thetable
  }
  \make@lr\theequation
}
%    \end{macrocode}
% \end{macro}
%
% \subsubsection{Preserving logos}
% 
%    Preserve \TeX, \LaTeX\ and \LaTeXe\ logos.
% \begin{macro}{\TeX}
%    \begin{macrocode}
\let\@@TeX\TeX
\def\TeX{\@latin{\@@TeX}}
%    \end{macrocode}
% \end{macro}
%
% \begin{macro}{\LaTeX}
%    \begin{macrocode}
\let\@@LaTeX\LaTeX
\def\LaTeX{\@latin{\@@LaTeX}}
%    \end{macrocode}
% \end{macro}
%
% \begin{macro}{\LaTeXe}
%    \begin{macrocode}
\let\@@LaTeXe\LaTeXe
\def\LaTeXe{\@latin{\@@LaTeXe}}
%    \end{macrocode}
% \end{macro}
%
% \subsubsection{List environments}
%
%    List environments in Right-to-Left languages, are ticked and
%    indented from the right instead of from the left. All the
%    definitions that caused indentation are revised for Right-to-Left
%    languages. \LaTeX\ keeps track on the indentation with the
%    \cs{leftmargin} and \cs{rightmargin} values.
%
% \begin{macro}{list}
%    Thus we need to override the definition of the |\list| macro: when
%    in RTL mode, the right margins are the begining of the line.
%    \begin{macrocode}
\def\list#1#2{%
  \ifnum \@listdepth >5\relax
    \@toodeep
  \else
    \global\advance\@listdepth\@ne
  \fi
  \rightmargin\z@
  \listparindent\z@
  \itemindent\z@
  \csname @list\romannumeral\the\@listdepth\endcsname
  \def\@itemlabel{#1}%
  \let\makelabel\@mklab
  \@nmbrlistfalse
  #2\relax
  \@trivlist
  \parskip\parsep
  \parindent\listparindent
  \advance\linewidth -\rightmargin
  \advance\linewidth -\leftmargin
%    \end{macrocode}
%    The only change in the macro is the |\if@rl| case:
%    \begin{macrocode}
  \if@rl
    \advance\@totalleftmargin \rightmargin
  \else
    \advance\@totalleftmargin \leftmargin
  \fi
  \parshape \@ne \@totalleftmargin \linewidth
  \ignorespaces}
%    \end{macrocode}
% \end{macro}
%
% \begin{macro}{\labelenumii}
% \begin{macro}{\p@enumiii}
%    The \cs{labelenumii} and \cs{p@enumiii} commands use
%    \emph{parentheses}. They are revised to work Right-to-Left with
%    the help of \cs{@brackets} macro defined above.
%    \begin{macrocode}
\def\labelenumii{\@brackets(\theenumii)}
\def\p@enumiii{\p@enumii\@brackets(\theenumii)}
%    \end{macrocode}
% \end{macro}
% \end{macro}
%
% \subsubsection{Tables of moving stuff}
%
%    Tables of moving arguments: table of contents (|toc|), list of
%    figures (|lof|) and list of tables (|lot|) are handles here. These
%    three default \LaTeX\ tables will be used now exclusively for
%    Left to Right stuff.
%
%    Three additional Right-to-Left tables: RL table of contents
%    (|cot|), RL list of figures (|fol|), and RL list of tables
%    (|tol|) are added.
%    These three tables will be used exclusively for Right to
%    Left stuff.
%
% \begin{macro}{\@tableofcontents}
% \begin{macro}{\@listoffigures}
% \begin{macro}{\@listoftables}
%    We define 3 new macros similar to the standard \LaTeX\ tables,
%    but with one parameter --- table file extension. These macros
%    will help us to define our additional tables below.
%    \begin{macrocode}
\@ifclassloaded{letter}{}{% other
\@ifclassloaded{slides}{}{% other
  \@ifclassloaded{article}{% article
    \newcommand\@tableofcontents[1]{%
      \section*{\contentsname\@mkboth%
        {\MakeUppercase\contentsname}%
        {\MakeUppercase\contentsname}}%
      \@starttoc{#1}}
    \newcommand\@listoffigures[1]{%
      \section*{\listfigurename\@mkboth%
        {\MakeUppercase\listfigurename}%
        {\MakeUppercase\listfigurename}}%
      \@starttoc{#1}}
    \newcommand\@listoftables[1]{%
      \section*{\listtablename\@mkboth%
        {\MakeUppercase\listtablename}%
        {\MakeUppercase\listtablename}}%
      \@starttoc{#1}}}%
  {% else report or book
    \newcommand\@tableofcontents[1]{%
      \@restonecolfalse\if@twocolumn\@restonecoltrue\onecolumn%
      \fi\chapter*{\contentsname\@mkboth%
        {\MakeUppercase\contentsname}%
        {\MakeUppercase\contentsname}}%
      \@starttoc{#1}\if@restonecol\twocolumn\fi}
    \newcommand\@listoffigures[1]{%
      \@restonecolfalse\if@twocolumn\@restonecoltrue\onecolumn%
      \fi\chapter*{\listfigurename\@mkboth%
        {\MakeUppercase\listfigurename}%
        {\MakeUppercase\listfigurename}}%
      \@starttoc{#1}\if@restonecol\twocolumn\fi}
    \newcommand\@listoftables[1]{%
      \if@twocolumn\@restonecoltrue\onecolumn\else\@restonecolfalse\fi%
      \chapter*{\listtablename\@mkboth%      
        {\MakeUppercase\listtablename}%
        {\MakeUppercase\listtablename}}%
      \@starttoc{#1}\if@restonecol\twocolumn\fi}}%
%    \end{macrocode}
% \end{macro}
% \end{macro}
% \end{macro}
%
% \begin{macro}{\lrtableofcontents}
% \begin{macro}{\lrlistoffigures}
% \begin{macro}{\lrlistoftables}
%    Left-to-Right tables are called now |\lr|\emph{xxx} and defined
%    with the aid of three macros defined above (extensions |toc|,
%    |lof|, and |lot|).
%    \begin{macrocode}
  \newcommand\lrtableofcontents{\@tableofcontents{toc}}%
  \newcommand\lrlistoffigures{\@listoffigures{lof}}%
  \newcommand\lrlistoftables{\@listoftables{lot}}%
%    \end{macrocode}
% \end{macro}
% \end{macro}
% \end{macro}
%
% \begin{macro}{\rltableofcontents}
% \begin{macro}{\rllistoffigures}
% \begin{macro}{\rllistoftables}
%    Right-to-Left tables will be called |\rl|\emph{xxx} and defined
%    with the aid of three macros defined above (extensions |cot|,
%    |fol|, and |tol|).
%    \begin{macrocode}
  \newcommand\rltableofcontents{\@tableofcontents{cot}}%
  \newcommand\rllistoffigures{\@listoffigures{fol}}%
  \newcommand\rllistoftables{\@listoftables{tol}}%
%    \end{macrocode}
% \end{macro}
% \end{macro}
% \end{macro}
%
% \begin{macro}{\tableofcontents}
% \begin{macro}{\listoffigures}
% \begin{macro}{\listoftables}
%    Let |\|\emph{xxx} be |\rl|\emph{xxx} if the current direction is
%    Right-to-Left and |\lr|\emph{xxx} if it is Left-to-Right.
%    \begin{macrocode}
  \renewcommand\tableofcontents{\if@rl\rltableofcontents%
                                \else\lrtableofcontents\fi}
  \renewcommand\listoffigures{\if@rl\rllistoffigures%
                              \else\lrlistoffigures\fi}
  \renewcommand\listoftables{\if@rl\rllistoftables%
                             \else\lrlistoftables\fi}}}
%    \end{macrocode}
% \end{macro}
% \end{macro}
% \end{macro}
%
% \begin{macro}{\@dottedtocline}
%    The following makes problems when making a Right-to-Left tables,
%    since it uses \cs{leftskip} and \cs{rightskip} which are both
%    mode dependent.
%    \begin{macrocode}
\def\@dottedtocline#1#2#3#4#5{%
  \ifnum #1>\c@tocdepth \else
    \vskip \z@ \@plus.2\p@
    {\if@rl\rightskip\else\leftskip\fi #2\relax 
      \if@rl\leftskip\else\rightskip\fi \@tocrmarg \parfillskip
      -\if@rl\leftskip\else\rightskip\fi
     \parindent #2\relax\@afterindenttrue
     \interlinepenalty\@M
     \leavevmode
     \@tempdima #3\relax
     \advance\if@rl\rightskip\else\leftskip\fi \@tempdima 
     \null\nobreak\hskip -\if@rl\rightskip\else\leftskip\fi
     {#4}\nobreak
     \leaders\hbox{$\m@th
        \mkern \@dotsep mu\hbox{.}\mkern \@dotsep
        mu$}\hfill
     \nobreak
     \hb@xt@\@pnumwidth{\hfil\normalfont \normalcolor \beginL#5\endL}%
     \par}%
  \fi}
%    \end{macrocode}
% \end{macro}
%
% \begin{macro}{\l@part}
%    This standard macro was redefined for table of contents since it
%    uses \cs{rightskip} which is mode dependent.
%    \begin{macrocode}
\@ifclassloaded{letter}{}{% other
\@ifclassloaded{slides}{}{% other
\renewcommand*\l@part[2]{%
  \ifnum \c@tocdepth >-2\relax
    \addpenalty{-\@highpenalty}%
    \addvspace{2.25em \@plus\p@}%
    \begingroup
      \setlength\@tempdima{3em}%
      \parindent \z@ \if@rl\leftskip\else\rightskip\fi \@pnumwidth
      \parfillskip -\@pnumwidth
      {\leavevmode
       \large \bfseries #1\hfil \hb@xt@\@pnumwidth{\hss#2}}\par
       \nobreak
         \global\@nobreaktrue
         \everypar{\global\@nobreakfalse\everypar{}}%
    \endgroup
  \fi}}}
%    \end{macrocode}
% \end{macro}
%
% \begin{macro}{\@part}
%    Part is redefined to support new Right-to-Left table of contents
%    (|cot|) as well as the Left-to-Right one (|toc|).
%    \begin{macrocode}
\@ifclassloaded{article}{% article class
  \def\@part[#1]#2{%
    \ifnum \c@secnumdepth >\m@ne
      \refstepcounter{part}%
      \addcontentsline{toc}{part}{\thepart\hspace{1em}#1}%
      \addcontentsline{cot}{part}{\thepart\hspace{1em}#1}%
    \else
      \addcontentsline{toc}{part}{#1}%
      \addcontentsline{cot}{part}{#1}%
    \fi
    {\parindent \z@ \raggedright
     \interlinepenalty \@M
     \normalfont
     \ifnum \c@secnumdepth >\m@ne
       \Large\bfseries \partname~\thepart
       \par\nobreak
     \fi
     \huge \bfseries #2%
     \markboth{}{}\par}%
    \nobreak
    \vskip 3ex
    \@afterheading}%
}{% report and book classes
  \def\@part[#1]#2{%
    \ifnum \c@secnumdepth >-2\relax
      \refstepcounter{part}%
      \addcontentsline{toc}{part}{\thepart\hspace{1em}#1}%
      \addcontentsline{cot}{part}{\thepart\hspace{1em}#1}%
    \else
      \addcontentsline{toc}{part}{#1}%
      \addcontentsline{cot}{part}{#1}%
    \fi
    \markboth{}{}%
    {\centering
     \interlinepenalty \@M
     \normalfont
     \ifnum \c@secnumdepth >-2\relax
       \huge\bfseries \partname~\thepart
       \par
       \vskip 20\p@
     \fi
     \Huge \bfseries #2\par}%
     \@endpart}}
%    \end{macrocode} 
% \end{macro}
%
% \begin{macro}{\@sect}
%    Section was redefined from the \pkg{latex.ltx} file. It is
%    changed to support both Left-to-Right (|toc|) and Right-to-Left
%    (|cot|) table of contents simultaneously.
%    \begin{macrocode}
\def\@sect#1#2#3#4#5#6[#7]#8{%
  \ifnum #2>\c@secnumdepth
    \let\@svsec\@empty
  \else
    \refstepcounter{#1}%
    \protected@edef\@svsec{\@seccntformat{#1}\relax}%
  \fi
  \@tempskipa #5\relax
  \ifdim \@tempskipa>\z@
    \begingroup
      #6{%
        \@hangfrom{\hskip #3\relax\@svsec}%
          \interlinepenalty \@M #8\@@par}%
    \endgroup
    \csname #1mark\endcsname{#7}%
    \addcontentsline{toc}{#1}{%
      \ifnum #2>\c@secnumdepth \else
        \protect\numberline{\csname the#1\endcsname}%
      \fi
      #7}%
    \addcontentsline{cot}{#1}{%
      \ifnum #2>\c@secnumdepth \else
        \protect\numberline{\csname the#1\endcsname}%
      \fi
      #7}%
  \else
    \def\@svsechd{%
      #6{\hskip #3\relax
      \@svsec #8}%
      \csname #1mark\endcsname{#7}%
      \addcontentsline{toc}{#1}{%
        \ifnum #2>\c@secnumdepth \else
          \protect\numberline{\csname the#1\endcsname}%
        \fi
        #7}%
      \addcontentsline{cot}{#1}{%
        \ifnum #2>\c@secnumdepth \else
          \protect\numberline{\csname the#1\endcsname}%
        \fi
        #7}}%
  \fi
  \@xsect{#5}}
%    \end{macrocode} 
% \end{macro}
%
% \begin{macro}{\@caption}
%    Caption was redefined from the \pkg{latex.ltx} file. It is
%    changed to support Left-to-Right list of figures and list of
%    tables (|lof| and |lot|) as well as new Right-to-Left lists 
%    (|fol| and |tol|) simultaneously.
%    \begin{macrocode}
\long\def\@caption#1[#2]#3{%
  \par
  \addcontentsline{\csname ext@#1\endcsname}{#1}%
    {\protect\numberline{\csname the#1\endcsname}%
    {\ignorespaces #2}}%
  \def\@fignm{figure}
  \ifx#1\@fignm\addcontentsline{fol}{#1}%
     {\protect\numberline{\csname the#1\endcsname}%
     {\ignorespaces #2}}\fi%
  \def\@tblnm{table}
  \ifx#1\@tblnm\addcontentsline{tol}{#1}%
     {\protect\numberline{\csname the#1\endcsname}%
     {\ignorespaces #2}}\fi%
  \begingroup
    \@parboxrestore
    \if@minipage
      \@setminipage
    \fi
    \normalsize
    \@makecaption{\csname fnum@#1\endcsname}{\ignorespaces #3}\par
  \endgroup}
%    \end{macrocode} 
% \end{macro}
%
% \begin{macro}{\l@chapter}
%    This standard macro was redefined for table of contents since it
%    uses \cs{rightskip} which is mode dependent.
%    \begin{macrocode}
\@ifclassloaded{letter}{}{%
\@ifclassloaded{slides}{}{%
  \@ifclassloaded{article}{}{%
    \renewcommand*\l@chapter[2]{%
      \ifnum \c@tocdepth >\m@ne
      \addpenalty{-\@highpenalty}%
      \vskip 1.0em \@plus\p@
      \setlength\@tempdima{1.5em}%
      \begingroup
         \parindent \z@ \if@rl\leftskip\else\rightskip\fi \@pnumwidth
         \parfillskip -\@pnumwidth
         \leavevmode \bfseries
         \advance\if@rl\rightskip\else\leftskip\fi\@tempdima
         \hskip -\if@rl\rightskip\else\leftskip\fi
         #1\nobreak\hfil \nobreak\hb@xt@\@pnumwidth{\hss#2}\par
         \penalty\@highpenalty
      \endgroup
      \fi}}}}
%    \end{macrocode} 
% \end{macro}
%
% \begin{macro}{\l@section}
% \begin{macro}{\l@subsection}
% \begin{macro}{\l@subsubsection}
% \begin{macro}{\l@paragraph}
% \begin{macro}{\l@subparagraph}
%    The toc entry for section did not work in article style.
%    Also it does not print dots, which is funny when most of your
%    work is divided into sections.
%
%    It was revised to use |\@dottedtocline| as in \pkg{report.sty}
%    (by Yaniv Bargury) and was updated later for all kinds of
%    sections (by Boris Lavva).
%    \begin{macrocode}
\@ifclassloaded{article}{%
\renewcommand*\l@section{\@dottedtocline{1}{1.5em}{2.3em}}
\renewcommand*\l@subsection{\@dottedtocline{2}{3.8em}{3.2em}}
\renewcommand*\l@subsubsection{\@dottedtocline{3}{7.0em}{4.1em}}
\renewcommand*\l@paragraph{\@dottedtocline{4}{10em}{5em}}
\renewcommand*\l@subparagraph{\@dottedtocline{5}{12em}{6em}}}{}
%    \end{macrocode} 
% \end{macro}
% \end{macro}
% \end{macro}
% \end{macro}
% \end{macro}
%
% \subsubsection{Two-column mode}
%
%    This is the support of \texttt{twocolumn} option for the standard
%    \LaTeXe\ classes.
%    The following code was originally borrowed from the Arab\TeX\
%    package, file \pkg{latexext.sty}, copyright by Klaus Lagally,
%    Institut fuer Informatik, Universitaet Stuttgart. It was updated
%    for this package by Boris Lavva.
%
% \begin{macro}{\@outputdblcol}
% \begin{macro}{\set@outputdblcol}
% \begin{macro}{rl@outputdblcol}
%    First column is \cs{@leftcolumn} will be shown at the right side,
%    Second column is \cs{@outputbox} will be shown at the left side. 
%    
%    |\set@outputdblcol| IS CURRENTLY DISABLED. TODO: REMOVE IT [tzafrir]
%    \begin{macrocode}
\let\@@outputdblcol\@outputdblcol
%\def\set@outputdblcol{%
%  \if@rl\renewcommand{\@outputdblcol}{\rl@outputdblcol}%
%  \else\renewcommand{\@outputdblcol}{\@@outputdblcol}\fi}
\renewcommand{\@outputdblcol}{%
  \if@rlmain%
    \rl@outputdblcol%
  \else%
    \@@outputdblcol%
  \fi%
}
\newcommand{\rl@outputdblcol}{%
  \if@firstcolumn
    \global \@firstcolumnfalse
    \global \setbox\@leftcolumn \box\@outputbox
  \else
    \global \@firstcolumntrue
    \setbox\@outputbox \vbox {\hb@xt@\textwidth {%
                              \hskip\columnwidth%
                              \hfil\vrule\@width\columnseprule\hfil
                              \hb@xt@\columnwidth {%
                                \box\@leftcolumn \hss}%
                              \hb@xt@\columnwidth {%
                                \hskip-\textwidth%
                                \box\@outputbox \hss}%
                              \hskip\columnsep%
                              \hskip\columnwidth}}%
    \@combinedblfloats
    \@outputpage
    \begingroup
      \@dblfloatplacement
      \@startdblcolumn
      \@whilesw\if@fcolmade \fi
        {\@outputpage
         \@startdblcolumn}%
    \endgroup
 \fi}
%    \end{macrocode} 
% \end{macro}
% \end{macro}
% \end{macro}
%
% \subsubsection{Footnotes}
%
% \begin{macro}{\footnoterule}
%    The Right-to-Left footnote rule is simply reversed default
%    Left-to-Right one. Footnotes can be used in RL or LR main 
%    modes, but changing mode while a footnote is pending is still 
%    unsolved.
%    \begin{macrocode}
\let\@@footnoterule=\footnoterule
\def\footnoterule{\if@rl\hb@xt@\hsize{\hss\vbox{\@@footnoterule}}%
                  \else\@@footnoterule\fi} 
%    \end{macrocode} 
% \end{macro}
%
% \subsubsection{Headings and two-side support}
%
%    When using \texttt{headings} or \texttt{myheadings} modes, we
%    have to ensure that the language and direction of heading is the
%    same as the whole chapter/part of the document. This is
%    implementing by setting special variable \cs{headlanguage} when
%    starting new chapter/part.
%
%    In addition, when selecting the \texttt{twoside} option (default in
%    \texttt{book} document class), the LR and RL modes need to be set
%    properly for things on the heading and footing. This is done
%    here too.
%
% \begin{macro}{ps@headings}
% \begin{macro}{ps@myheadings}
% \begin{macro}{headeven}
% \begin{macro}{headodd}
%    First, we will support the standard \pkg{letter} class:
%    \begin{macrocode}
\@ifclassloaded{letter}{%
  \def\headodd{\protect\if@rl\beginR\fi\headtoname{}
               \ignorespaces\toname
               \hfil \@date
               \hfil \pagename{} \thepage\protect\if@rl\endR\fi}
  \if@twoside
     \def\ps@headings{%
         \let\@oddfoot\@empty\let\@evenfoot\@empty
         \def\@oddhead{\select@language{\headlanguage}\headodd}
         \let\@evenhead\@oddhead}
  \else
     \def\ps@headings{%
         \let\@oddfoot\@empty
         \def\@oddhead{\select@language{\headlanguage}\headodd}}
  \fi  
  \def\headfirst{\protect\if@rl\beginR\fi\fromlocation \hfill %
                 \telephonenum\protect\if@rl\endR\fi}
  \def\ps@firstpage{%
     \let\@oddhead\@empty
     \def\@oddfoot{\raisebox{-45\p@}[\z@]{%
        \hb@xt@\textwidth{\hspace*{100\p@}%
          \ifcase \@ptsize\relax
             \normalsize
          \or
             \small
          \or
             \footnotesize
          \fi
        \select@language{\headlanguage}\headfirst}}\hss}}
%
  \renewcommand{\opening}[1]{%
     \let\headlanguage=\languagename%
     \ifx\@empty\fromaddress%
        \thispagestyle{firstpage}%
        {\raggedleft\@date\par}%
     \else  % home address
        \thispagestyle{empty}%
        {\raggedleft
        \if@rl\begin{tabular}{@{\beginR\csname%
          to\@rllanguagename\endcsname}r@{\endR}}\ignorespaces
           \fromaddress \\*[2\parskip]%
           \@date \end{tabular}\par%
        \else\begin{tabular}{l}\ignorespaces
           \fromaddress \\*[2\parskip]%
           \@date \end{tabular}\par%
        \fi}%
     \fi
     \vspace{2\parskip}%
     {\raggedright \toname \\ \toaddress \par}%
     \vspace{2\parskip}%
     #1\par\nobreak}
}
%    \end{macrocode}
%    Then, the \pkg{article}, \pkg{report} and \pkg{book} document
%    classes are supported. Note, that in one-sided mode
%    \cs{markright} was changed to \cs{markboth}.
%    \begin{macrocode}
{% article, report, book
  \def\headeven{\protect\if@rl\beginR\thepage\hfil\rightmark\endR
                \protect\else\thepage\hfil{\slshape\leftmark}
                \protect\fi}
  \def\headodd{\protect\if@rl\beginR\leftmark\hfil\thepage\endR
               \protect\else{\slshape\rightmark}\hfil\thepage
               \protect\fi}
  \@ifclassloaded{article}{% article
    \if@twoside   % two-sided
       \def\ps@headings{%
         \let\@oddfoot\@empty\let\@evenfoot\@empty
         \def\@evenhead{\select@language{\headlanguage}\headeven}%
         \def\@oddhead{\select@language{\headlanguage}\headodd}%
         \let\@mkboth\markboth
         \def\sectionmark##1{%
           \markboth {\MakeUppercase{%
               \ifnum \c@secnumdepth >\z@
                  \thesection\quad
               \fi
               ##1}}{}}%
         \def\subsectionmark##1{%
           \markright{%
             \ifnum \c@secnumdepth >\@ne
                \thesubsection\quad
             \fi
        ##1}}}
    \else          % one-sided
       \def\ps@headings{%
         \let\@oddfoot\@empty
         \def\@oddhead{\headodd}%
         \let\@mkboth\markboth
         \def\sectionmark##1{%
           \markboth{\MakeUppercase{%
               \ifnum \c@secnumdepth >\m@ne
                  \thesection\quad
               \fi
               ##1}}{\MakeUppercase{%
               \ifnum \c@secnumdepth >\m@ne
                  \thesection\quad
               \fi
               ##1}}}}
    \fi
%
    \def\ps@myheadings{%
      \let\@oddfoot\@empty\let\@evenfoot\@empty
      \def\@evenhead{\select@language{\headlanguage}\headeven}%
      \def\@oddhead{\select@language{\headlanguage}\headodd}%
      \let\@mkboth\@gobbletwo
      \let\sectionmark\@gobble
      \let\subsectionmark\@gobble
  }}{% report and book
    \if@twoside  % two-sided
       \def\ps@headings{%
         \let\@oddfoot\@empty\let\@evenfoot\@empty
         \def\@evenhead{\select@language{\headlanguage}\headeven}
         \def\@oddhead{\select@language{\headlanguage}\headodd}
         \let\@mkboth\markboth
         \def\chaptermark##1{%
           \markboth{\MakeUppercase{%
               \ifnum \c@secnumdepth >\m@ne
                  \@chapapp\ \thechapter. \ %
               \fi
               ##1}}{}}%
         \def\sectionmark##1{%
           \markright {\MakeUppercase{%
               \ifnum \c@secnumdepth >\z@
                  \thesection. \ %
               \fi
               ##1}}}}
    \else  % one-sided
       \def\ps@headings{%
         \let\@oddfoot\@empty
         \def\@oddhead{\select@language{\headlanguage}\headodd}
         \let\@mkboth\markboth
         \def\chaptermark##1{%
           \markboth{\MakeUppercase{%
               \ifnum \c@secnumdepth >\m@ne
                  \@chapapp\ \thechapter. \ %
               \fi
               ##1}}{\MakeUppercase{%
               \ifnum \c@secnumdepth >\m@ne
                  \@chapapp\ \thechapter. \ %
               \fi
               ##1}}}}
    \fi
    \def\ps@myheadings{%
      \let\@oddfoot\@empty\let\@evenfoot\@empty
      \def\@evenhead{\select@language{\headlanguage}\headeven}%
      \def\@oddhead{\select@language{\headlanguage}\headodd}%
      \let\@mkboth\@gobbletwo
      \let\chaptermark\@gobble
      \let\sectionmark\@gobble
  }}}
%    \end{macrocode} 
% \end{macro}
% \end{macro}
% \end{macro}
% \end{macro}
%
%    
%    \subsubsection{Postscript Porblems}
%    Any command that is implemented by PostScript directives, e.g
%    commands from the ps-tricks package, needs to be fixed, because the
%    PostScript directives are being interpeted after the document has been
%    converted by \TeX to visual Hebrew (DVI, PostScript and PDF have visual
%    Hebrew). 
%
%    For instance: Suppose you wrote in your document: 
%
%    |\textcolor{cyan}{some ltr text}| 
%
%    This would be interpeted by \TeX to something like:
%
%    |[postscript:make color cyan]some LTR text[postscript:make color black]|
%
%
%    However, with the bidirectionality support we get:
%
%    |\textcolor{cyan}{\hebalef\hebbet}|
%
%    Translated to: 
%
%    |[postscript:make color black]{bet}{alef}[postscript:make color cyan]|
%
%    While we want:
%
%    |[postscript:make color cyan]{bet}{alef}[postscript:make color black]|
%
%    The following code will probably work at least with code that stays in the
%    same line:
%    \begin{macro}{@textcolor}
% \begin{macrocode}
\AtBeginDocument{%
  %I assume that \@textcolor is only defined by the package color
  \ifx\@textcolor\@undefined\else% 
    % If that macro was defined before the beginning of the document,
    % that is: the package was loaded: redefine it with bidi support
    \def\@textcolor#1#2#3{% 
      \if@rl%
        \beginL\protect\leavevmode{\color#1{#2}\beginR#3\endR}\endL%
      \else%
        \protect\leavevmode{\color#1{#2}#3}%
      \fi%
    }%
  \fi%
}
% \end{macrocode}
% \end{macro}
% \begin{macro}{\thetrueSlideCounter}
%    This macro probably needs to be overriden for when using |prosper|,
%    (waiting for feedback. Tzafrir)
%    \begin{macrocode}
\@ifclassloaded{prosper}{%
  \def\thetrueSlideCounter{\arabicnorl{trueSlideCounter}}
}{}
%    \end{macrocode}
% \end{macro}
%
% \subsubsection{Miscellaneous internal \LaTeX\ macros}
%
% \begin{macro}{\raggedright}
% \begin{macro}{\raggedleft}
%    \cs{raggedright} was changed from \pkg{latex.ltx} file to support
%    Right-to-Left mode, because of the bug in its implementation.
%    \begin{macrocode}
\def\raggedright{%
  \let\\\@centercr
  \leftskip\z@skip\rightskip\@flushglue
  \parindent\z@\parfillskip\z@skip}
%    \end{macrocode} 
%    Swap meanings of \cs{raggedright} and \cs{raggedleft} in
%    Right-to-Left mode.
%    \begin{macrocode}
\let\@@raggedleft=\raggedleft
\let\@@raggedright=\raggedright
\renewcommand\raggedleft{\if@rl\@@raggedright%
                         \else\@@raggedleft\fi}
\renewcommand\raggedright{\if@rl\@@raggedleft%
                          \else\@@raggedright\fi}
%    \end{macrocode} 
% \end{macro}
% \end{macro}
%
% \begin{macro}{\author}
%    \cs{author} is inserted with \texttt{tabular} environment, and
%    will be used in restricted horizontal mode. Therefore we have to
%    add explicit direction change command when in Right-to-Left
%    mode.
%    \begin{macrocode}
\let\@@author=\author
\renewcommand{\author}[1]{\@@author{\if@rl\beginR #1\endR\else #1\fi}}
%    \end{macrocode}
% \end{macro}
%
% \begin{macro}{\MakeUppercase}
% \begin{macro}{\MakeLowercase}
%    There are no uppercase and lowercase letters in most
%    Right-to-Left languages, therefore we should redefine
%    \cs{MakeUppercase} and \cs{MakeLowercase} \LaTeXe\ commands.
%    \begin{macrocode}
\let\@@MakeUppercase=\MakeUppercase
\def\MakeUppercase#1{\if@rl#1\else\@@MakeUppercase{#1}\fi}
\let\@@MakeLowercase=\MakeLowercase
\def\MakeLowercase#1{\if@rl#1\else\@@MakeLowercase{#1}\fi}
%    \end{macrocode}
% \end{macro}
% \end{macro}
%
% \begin{macro}{\underline}
%    We should explicitly use \cs{L} and \cs{R} commands in
%    \cs{underline}d text.
%    \begin{macrocode}
\let\@@@underline=\underline
\def\underline#1{\@@@underline{\if@rl\R{#1}\else #1\fi}}
%    \end{macrocode}
% \end{macro}
%
%    \cs{undertext} was added for \LaTeX 2.09 compatibility mode.
%    \begin{macrocode}
\if@compatibility
   \let\undertext=\underline
\fi
%    \end{macrocode} 
%
% \begin{macro}{\@xnthm}
% \begin{macro}{\@opargbegintheorem}
%    The following has been inserted to correct the appearance of the
%    number in \cs{newtheorem} to reorder theorem number components. A
%    similar correction  in the definition of \cs{@opargbegintheorem}
%    was added too.
%    \begin{macrocode}
\def\@xnthm#1#2[#3]{%
  \expandafter\@ifdefinable\csname #1\endcsname
  {\@definecounter{#1}\@addtoreset{#1}{#3}%
    \expandafter\xdef\csname the#1\endcsname{\noexpand\@number
      {\expandafter\noexpand\csname the#3\endcsname \@thmcountersep
        \@thmcounter{#1}}}%
    \global\@namedef{#1}{\@thm{#1}{#2}}%
    \global\@namedef{end#1}{\@endtheorem}}}
%
\def\@opargbegintheorem#1#2#3{%
  \trivlist
      \item[\hskip \labelsep{\bfseries #1\ #2\ 
          \@brackets({#3})}]\itshape}
%    \end{macrocode} 
% \end{macro}
% \end{macro}
%
% \begin{macro}{\@chapter}
% \begin{macro}{\@schapter}
%    The following was added for pretty printing of the chapter
%    numbers, for supporting Right-to-Left tables (\texttt{cot},
%    \texttt{fol}, and \texttt{tol}), to save \cs{headlanguage}
%    for use in running headers, and to start two-column mode
%    depending on chapter's main language.
%    \begin{macrocode}
\@ifclassloaded{article}{}{%
  % For pretty priniting
  \def\@@chapapp{Chapter}
  \def\@@thechapter{\@@arabic\c@chapter}
  \def\@chapter[#1]#2{%
    \let\headlanguage=\languagename%
    %\set@outputdblcol%
    \ifnum \c@secnumdepth >\m@ne
       \refstepcounter{chapter}%
       \typeout{\@@chapapp\space\@@thechapter.}%
       \addcontentsline{toc}{chapter}%
       {\protect\numberline{\thechapter}#1}
       \addcontentsline{cot}{chapter}%
       {\protect\numberline{\thechapter}#1}
    \else
       \addcontentsline{toc}{chapter}{#1}%
       \addcontentsline{cot}{chapter}{#1}%
    \fi
    \chaptermark{#1}
    \addtocontents{lof}{\protect\addvspace{10\p@}}%
    \addtocontents{fol}{\protect\addvspace{10\p@}}%
    \addtocontents{lot}{\protect\addvspace{10\p@}}%
    \addtocontents{tol}{\protect\addvspace{10\p@}}%
    \if@twocolumn
       \@topnewpage[\@makechapterhead{#2}]%
    \else
       \@makechapterhead{#2}%
       \@afterheading
    \fi}
  %
  \def\@schapter#1{%
    \let\headlanguage=\languagename%
    %\set@outputdblcol%    
    \if@twocolumn
       \@topnewpage[\@makeschapterhead{#1}]%
    \else
       \@makeschapterhead{#1}%
       \@afterheading
    \fi}}
%    \end{macrocode} 
% \end{macro}
% \end{macro}
%
% \begin{macro}{\appendix}
%    Changed mainly for pretty printing of appendix numbers, and to
%    start two-column mode with the right language (if needed).
%    \begin{macrocode}
\@ifclassloaded{letter}{}{% other
\@ifclassloaded{slides}{}{% other
  \@ifclassloaded{article}{% article
    \renewcommand\appendix{\par
      \setcounter{section}{0}%
      \setcounter{subsection}{0}%
      \renewcommand\thesection{\@Alph\c@section}}
  }{% report and book
    \renewcommand\appendix{\par
      %\set@outputdblcol%
      \setcounter{chapter}{0}%
      \setcounter{section}{0}%
      \renewcommand\@chapapp{\appendixname}%
      % For pretty priniting
      \def\@@chapapp{Appendix}%
      \def\@@thechapter{\@@Alph\c@chapter}
      \renewcommand\thechapter{\@Alph\c@chapter}}}}}
%    \end{macrocode} 
% \end{macro}
%
% \subsubsection{Bibliography and citations}
%
% \begin{macro}{\@cite}
% \begin{macro}{\@biblabel}
% \begin{macro}{\@lbibitem}
%    Citations are produced by the macro
%    |\@cite{|\emph{LABEL}|}{|\emph{NOTE}|}|. Both the citation label
%    and the note is typeset in the current direction. We have to use
%    \cs{@brackets} macro in \cs{@cite} and \cs{@biblabel} macros. In
%    addition, when using \emph{alpha} or similar bibliography style,
%    the \cs{@lbibitem} is used and have to be update to support bot
%    Right-to-Left and Left-to-Right citations.
%
%    \begin{macrocode}
\def\@cite#1#2{\@brackets[{#1\if@tempswa , #2\fi}]}
\def\@biblabel#1{\@brackets[{#1}]}
\def\@lbibitem[#1]#2{\item[\@biblabel{#1}\hfill]\if@filesw
      {\let\protect\noexpand
       \immediate
       \if@rl\write\@auxout{\string\bibcite{#2}{\R{#1}}}%
       \else\write\@auxout{\string\bibcite{#2}{\L{#1}}}\fi%
      }\fi\ignorespaces}
%    \end{macrocode}
% \end{macro}
% \end{macro}
% \end{macro}
%
% \begin{environment}{thebibliography}
%    Use \cs{rightmargin} instead of \cs{leftmargin} when in RL mode.
%    \begin{macrocode}
\@ifclassloaded{letter}{}{% other
\@ifclassloaded{slides}{}{% other
\@ifclassloaded{article}{%
  \renewenvironment{thebibliography}[1]
  {\section*{\refname\@mkboth%
      {\MakeUppercase\refname}%
      {\MakeUppercase\refname}}%
    \list{\@biblabel{\@arabic\c@enumiv}}%
    {\settowidth\labelwidth{\@biblabel{#1}}%
      \if@rl\leftmargin\else\rightmargin\fi\labelwidth
      \advance\if@rl\leftmargin\else\rightmargin\fi\labelsep
      \@openbib@code
      \usecounter{enumiv}%
      \let\p@enumiv\@empty
      \renewcommand\theenumiv{\@arabic\c@enumiv}}%
    \sloppy
    \clubpenalty4000
    \@clubpenalty \clubpenalty
    \widowpenalty4000%
    \sfcode`\.\@m}
  {\def\@noitemerr
    {\@latex@warning{Empty `thebibliography' environment}}%
     \endlist}}%
{\renewenvironment{thebibliography}[1]{%
    \chapter*{\bibname\@mkboth%
      {\MakeUppercase\bibname}%
      {\MakeUppercase\bibname}}%
    \list{\@biblabel{\@arabic\c@enumiv}}%
    {\settowidth\labelwidth{\@biblabel{#1}}%
      \if@rl\leftmargin\else\rightmargin\fi\labelwidth
      \advance\if@rl\leftmargin\else\rightmargin\fi\labelsep
      \@openbib@code
      \usecounter{enumiv}%
      \let\p@enumiv\@empty
      \renewcommand\theenumiv{\@arabic\c@enumiv}}%
    \sloppy
    \clubpenalty4000
    \@clubpenalty \clubpenalty
    \widowpenalty4000%
    \sfcode`\.\@m}
  {\def\@noitemerr
    {\@latex@warning{Empty `thebibliography' environment}}%
     \endlist}}}}
%    \end{macrocode}
% \end{environment}
%
% \begin{macro}{\@verbatim}
%    All kinds of verbs (\cs{verb},\cs{verb*},\texttt{verbatim} and
%    \texttt{verbatim*}) now can be used in Right-to-Left mode. Errors
%    in latin mode solved too.
%    \begin{macrocode}
\def\@verbatim{%
  \let\do\@makeother \dospecials%
  \obeylines \verbatim@font \@noligs}
%    \end{macrocode}
% \end{macro}
%
% \begin{macro}{\@makecaption}
%    Captions are set always centered. This allows us to use bilingual
%    captions, for example: |\caption{\R{RLtext} \\ \L{LRtext}}|,
%    which will be formatted as:
%    \begin{center}
%    Right to left caption here (RLtext) \\
%    Left to right caption here (LRtext)
%    \end{center}
%    See also \cs{bcaption} command below.
%    \begin{macrocode}
\long\def\@makecaption#1#2{%
  \vskip\abovecaptionskip%
  \begin{center}%
    #1: #2%
  \end{center} \par%
  \vskip\belowcaptionskip}
%    \end{macrocode} 
% \end{macro}
%
% \subsubsection{Additional bidirectional commands}
%
%    \begin{itemize}
%    \item Section headings are typeset with the default global
%    direction.
%    \item Text in section headings in the reverse language \emph{do
%    not} have to be protected for the reflection command, as in:
%    |\protect\L{|\emph{Latin Text}|}|, because \cs{L} and \cs{R} are
%    robust now.
%    \item Table of contents, list of figures and list of tables
%    should be typeset with the \cs{tableofcontents},
%    \cs{listoffigures} and \cs{listoftables} commands respectively.
%    \item The above tables will be typeset in the main direction (and
%    language) in effect where the above commands are placed.
%    \item Only 2 tables of each kind are supported: one for
%    Right-to-Left and another for Left-to-Right directions.
%    \end{itemize}
%
%    How to include line to both tables? One has to use bidirectional
%    sectioning commands as following:
%    \begin{enumerate}
%    \item Use the |\b|\emph{xxx} version of the sectioning commands
%    in the text instead of the |\|\emph{xxx} version (\emph{xxx} is
%    one of: \texttt{part}, \texttt{chapter}, \texttt{section}, 
%    \texttt{subsection}, \texttt{subsubsection}, \texttt{caption}).
%    \item Syntax of the |\b|\emph{xxx} command is
%        |\b|\emph{xxx}|{|\emph{RL text}|}{|\emph{LR text}|}|.
%    Both arguments are typeset in proper direction by default (no
%    need to change direction for the text inside).
%    \item The section header inside the document will be typeset in
%    the global direction in effect at the time. i.e. The |{|\emph{RL
%    text}|}| will be typeset if Right-to-Left mode is in effect and
%    |{|\emph{LR text}|}| otherwise.
%    \end{enumerate}
%
% \begin{macro}{\bpart}
%    \begin{macrocode}
\newcommand{\bpart}[2]{\part{\protect\if@rl%
    #1 \protect\else #2 \protect\fi}}
%    \end{macrocode}
% \end{macro}
%
% \begin{macro}{\bchapter}
%    \begin{macrocode}
\newcommand{\bchapter}[2]{\chapter{\protect\if@rl%
    #1 \protect\else #2 \protect\fi}}
%    \end{macrocode}
% \end{macro}
%
% \begin{macro}{\bsection}
%    \begin{macrocode}
\newcommand{\bsection}[2]{\section{\protect\if@rl%
    #1 \protect\else #2 \protect\fi}}
%    \end{macrocode}
% \end{macro}
%
% \begin{macro}{\bsubsection}
%    \begin{macrocode}
\newcommand{\bsubsection}[2]{\subsection{\protect\if@rl%
    #1 \protect\else #2 \protect\fi}}
%    \end{macrocode}
% \end{macro}
%
% \begin{macro}{\bsubsubsection}
%    \begin{macrocode}
\newcommand{\bsubsubsection}[2]{\subsubsection{\protect\if@rl%
    #1 \protect\else #2 \protect\fi}}
%    \end{macrocode}
% \end{macro}
%
% \begin{macro}{\bcaption}
%    \begin{macrocode}
\newcommand{\bcaption}[2]{%
  \caption[\protect\if@rl \R{#1}\protect\else \L{#2}\protect\fi]{%
    \if@rl\R{#1}\protect\\ \L{#2}
    \else\L{#2}\protect\\ \R{#1}\fi}}
%    \end{macrocode}
% \end{macro}
%
%    The following definition is a modified version of \cs{bchapter}, meant
%    as a bilingual twin for \cs{chapter*} and \cs{section*}
%    (added by Irina Abramovici).
%
% \begin{macro}{\bchapternn}
%    \begin{macrocode}
\newcommand{\bchapternn}[2]{\chapter*{\protect\if@rl% 
    #1 \protect\else #2 \protect\fi}}
%    \end{macrocode}
% \end{macro}
%
% \begin{macro}{\bsectionnn}
%    \begin{macrocode}
\newcommand{\bsectionnn}[2]{\section*{\protect\if@rl%
    #1 \protect\else #2 \protect\fi}}
%    \end{macrocode}
% \end{macro}
%
%    Finally, at end of \babel\ package, the \cs{headlanguage} and
%    two-column mode will be initialized according to the current
%    language.
%    \begin{macrocode}
\AtEndOfPackage{\rlAtEndOfPackage}
%
\def\rlAtEndOfPackage{%
  \global\let\headlanguage=\languagename%\set@outputdblcol%
}
%</rightleft>
%    \end{macrocode}
%
% \subsection{Hebrew calendar}
%
%    The original version of the package \pkg{hebcal.sty}\footnote{The
%    following description of \pkg{hebcal} package is based on the
%    comments included with original source by the author, Michail
%    Rozman.} for \TeX\ and \LaTeX2.09, entitled ``\TeX{} \& \LaTeX{}
%    macros for computing Hebrew date from Gregorian one'' was created
%    by Michail Rozman, |misha@iop.tartu.ew.su|\footnote{Please direct
%    any comments, bug reports, questions, etc. about the package to
%    this address.}
%
%    \begin{tabular}{@{}lr@{}c@{}ll}
%    Released: &Tammuz 12, 5751&--&June 24, 1991   &\\
%    Corrected:&Shebat 10, 5752&--&January 15, 1992&by Rama Porrat\\
%    Corrected:&Adar II 5, 5752&--&March 10, 1992  &by Misha\\
%    Corrected:&Tebeth, 5756   &--&January 1996    &Dan Haran\\
%              &&&&(haran@math.tau.ac.il)
%    \end{tabular}
%
%    The package was adjusted for \babel{} and \LaTeXe{} by Boris
%    Lavva.
%
%    Changes to the printing routine (only) by Ron Artstein, June 1,
%    2003.
%
%    This package should be included \emph{after} the \pkg{babel} with
%    \pkg{hebrew} option, as following:
%    \begin{quote}
%       |\documentclass[|\ldots|]{|\ldots|}|\\
%       |\usepackage[hebrew,|\ldots|,|\emph{other languages}|,|
%                            \ldots|]{babel}|\\
%       |\usepackage{hebcal}|
%    \end{quote}
%
%    Two main user-level commands are provided by this package:
%
%    \DescribeMacro{\Hebrewtoday}
%    Computes today's Hebrew date and prints it. If we are presently
%    in Hebrew mode, the date will be printed in Hebrew, otherwise ---
%    in English (like Shebat 10, 5752).
%
%    \DescribeMacro{\Hebrewdate}
%    Computes the Hebrew date from the given Gregorian date and
%    prints it. If we are presently in Hebrew mode, the date will be
%    printed in Hebrew, otherwise --- in English (like Shebat 10,
%    5752). An example of usage is shown below:
%    \begin{quote}
%       |\newcount\hd \newcount\hm \newcount\hy|\\
%       |\hd=10 \hm=3 \hy=1992|\\
%       |\Hebrewdate{\hd}{\hm}{\hy}|
%    \end{quote}
%
%    \DescribeMacro{full}
%    The package option |full| sets the flag |\@full@hebrew@year|,
%    which causes years from the current millenium to be printed with
%    the thousands digit (he-tav-shin-samekh-gimel). Without this 
%    option, thousands are not printed for the current millenium.
%    NOTE: should this be a command option rather than a package
%    option? --RA.
%
% \subsubsection{Introduction}
%
%    The Hebrew calendar is inherently complicated: it is lunisolar --
%    each year starts close to the autumn equinox, but each month must
%    strictly start at a new moon.  Thus Hebrew calendar must be
%    harmonized simultaneously with both lunar and solar events. In
%    addition, for reasons of the religious practice, the year cannot
%    start on Sunday, Wednesday or Friday.
%
%    For the full description of Hebrew calendar and for the list of
%    references see:
%    \begin{quote}
%      Nachum Dershowitz and Edward M. Reingold,
%      \emph{``Calendarical Calculations''}, Software--Pract.Exper.,
%      vol. 20 (9), pp.899--928 (September 1990).
%    \end{quote}
%    |C| translation of |LISP| programs from the above article
%    available from Mr. Wayne Geiser, |geiser%pictel@uunet.uu.net|.
%
%    The 4\textsuperscript{th} distribution (July 1989) of hdate/hcal
%    (Hebrew calendar programs similar to UNIX date/cal) by Mr. Amos
%    Shapir, |amos@shum.huji.ac.il|, contains short and very clear
%    description of algorithms.
%
% \subsubsection{Registers, Commands, Formatting Macros}
%
%    The command |\Hebrewtoday| produces today's date for Hebrew
%    calendar. It is similar to the standard \LaTeXe{} command
%    |\today|. In addition three numerical registers |\Hebrewday|,
%    |\Hebrewmonth| and |\Hebrewyear| are set.
%    For setting this registers without producing of date string
%    command |\Hebrewsetreg| can be used.
%
%    The command 
%    |\Hebrewdate{|\emph{Gday}|}{|\emph{Gmonth}|}{|\emph{Gyear}|}|
%    produces Hebrew calendar date corresponding to Gregorian date 
%    |Gday.Gmonth.Gyear|. Three numerical registers |\Hebrewday|,
%    |\Hebrewmonth| and |\Hebrewyear| are set.
%
%    For converting arbitrary Gregorian date |Gday.Gmonth.Gyear|
%    to Hebrew date |Hday.Hmonth.Hyear| without producing date string
%    the command:
%    \begin{center}
%      |\HebrewFromGregorian{|\emph{Gday}|}{|\emph{Gmonth}|}{|%
%      \emph{Gyear}|}{|\emph{Hday}|}{|\emph{Hmonth}|}{|\emph{Hyear}|}|
%    \end{center}
%    can be used.
%
%    \begin{macrocode}
%<*calendar>
\newif\if@full@hebrew@year
\@full@hebrew@yearfalse
\DeclareOption{full}{\@full@hebrew@yeartrue}
\ProcessOptions
\newcount\Hebrewday  \newcount\Hebrewmonth   \newcount\Hebrewyear
%    \end{macrocode}
%
% \begin{macro}{\Hebrewdate}
%    Hebrew calendar date corresponding to Gregorian date
%    |Gday.Gmonth.Gyear|. If Hebrew (right-to-left) fonts \& macros
%    are not loaded, we have to use English format.
%    \begin{macrocode}
\def\Hebrewdate#1#2#3{%
    \HebrewFromGregorian{#1}{#2}{#3}
                        {\Hebrewday}{\Hebrewmonth}{\Hebrewyear}%
    \ifundefined{if@rl}% 
       \FormatForEnglish{\Hebrewday}{\Hebrewmonth}{\Hebrewyear}%
    \else%
       \FormatDate{\Hebrewday}{\Hebrewmonth}{\Hebrewyear}%
    \fi}
%    \end{macrocode}
% \end{macro}
%
% \begin{macro}{\Hebrewtoday}
%    Today's date in Hebrew calendar.
%    \begin{macrocode}
\def\Hebrewtoday{\Hebrewdate{\day}{\month}{\year}}
\let\hebrewtoday=\Hebrewtoday
%    \end{macrocode}
% \end{macro}
%
% \begin{macro}{\Hebrewsetreg}
%    Set registers: today's date in hebrew calendar.
%    \begin{macrocode}
\def\Hebrewsetreg{%
    \HebrewFromGregorian{\day}{\month}{\year}
                        {\Hebrewday}{\Hebrewmonth}{\Hebrewyear}}
%    \end{macrocode}
% \end{macro}
%
% \begin{macro}{\FormatDate}
%    Prints a Hebrew calendar date |Hebrewday.Hebrewmonth.Hebrewyear|.
%    \begin{macrocode}
\def\FormatDate#1#2#3{%
        \if@rl%
            \FormatForHebrew{#1}{#2}{#3}%
        \else%
            \FormatForEnglish{#1}{#2}{#3}
        \fi}
%    \end{macrocode}
% \end{macro}
%
%    To prepare another language version of Hebrew calendar commands,
%    one should change or add commands here.
%
%    We start with Hebrew language macros.
% \begin{macro}{\HebrewYearName}
%    Prints Hebrew year as a Hebrew number. Disambiguates strings by
%    adding lamed-pe-gimel to years of the first Jewish millenium and
%    to years divisible by 1000. Suppresses the thousands digit in the
%    current millenium unless the package option |full| is selected.
%    NOTE: should this be provided as a command option rather than a
%    package option? --RA.
%    \begin{macrocode}
\def\HebrewYearName#1{{%
   \@tempcnta=#1\divide\@tempcnta by 1000\multiply\@tempcnta by 1000
   \ifnum#1=\@tempcnta\relax % divisible by 1000: disambiguate
     \Hebrewnumeralfinal{#1}\ )\heblamed\hebpe"\hebgimel(%
   \else % not divisible by 1000
     \ifnum#1<1000\relax     % first millennium: disambiguate
       \Hebrewnumeralfinal{#1}\ )\heblamed\hebpe"\hebgimel(%
     \else 
       \ifnum#1<5000
         \Hebrewnumeralfinal{#1}%
       \else
         \ifnum#1<6000 % current millenium, print without thousands
           \@tempcnta=#1\relax
           \if@full@hebrew@year\else\advance\@tempcnta by -5000\fi
           \Hebrewnumeralfinal{\@tempcnta}%
         \else % #1>6000
           \Hebrewnumeralfinal{#1}%
         \fi
       \fi
     \fi
   \fi}}
%    \end{macrocode}
% \end{macro}
%
% \begin{macro}{\HebrewMonthName}
%    The macro |\HebrewMonthName{|\emph{month}|}{|\emph{year}|}|
%    returns the name of month in the `year'.
%    \begin{macrocode}
\def\HebrewMonthName#1#2{%
    \ifnum #1 = 7 %
    \CheckLeapHebrewYear{#2}%
        \if@HebrewLeap \hebalef\hebdalet\hebresh\ \hebbet'%
           \else \hebalef\hebdalet\hebresh%
        \fi%
    \else%
        \ifcase#1%
           % nothing for 0           
           \or\hebtav\hebshin\hebresh\hebyod%
           \or\hebhet\hebshin\hebvav\hebfinalnun%
           \or\hebkaf\hebsamekh\heblamed\hebvav%
           \or\hebtet\hebbet\hebtav%
           \or\hebshin\hebbet\hebtet%
           \or\hebalef\hebdalet\hebresh\ \hebalef'%
           \or\hebalef\hebdalet\hebresh\ \hebbet'%
           \or\hebnun\hebyod\hebsamekh\hebfinalnun%
           \or\hebalef\hebyod\hebyod\hebresh%
           \or\hebsamekh\hebyod\hebvav\hebfinalnun%
           \or\hebtav\hebmem\hebvav\hebzayin%
           \or\hebalef\hebbet%
           \or\hebalef\heblamed\hebvav\heblamed%
        \fi%
    \fi}
%    \end{macrocode}
% \end{macro}
%
% \begin{macro}{\HebrewDayName}
%    Name of day in Hebrew letters (gimatria).
%    \begin{macrocode}
\def\HebrewDayName#1{\Hebrewnumeral{#1}}
%    \end{macrocode}
% \end{macro}
%
%
% \begin{macro}{\FormatForHebrew}
%    The macro |\FormatForHebrew{|\emph{hday}|}{|\emph{hmonth}
%    |}{|\emph{hyear}|}| returns the formatted Hebrew date in Hebrew
%    language.
%    \begin{macrocode}
\def\FormatForHebrew#1#2#3{%
  \HebrewDayName{#1}~\hebbet\HebrewMonthName{#2}{#3},~%
  \HebrewYearName{#3}}
%    \end{macrocode}
% \end{macro}
%
%    We continue with two English language macros for Hebrew calendar.
% \begin{macro}{\HebrewMonthNameInEnglish}
%    The macro |\HebrewMonthNameInEnglish{|\emph{month}|}{|%
%    \emph{year}|}| is similar to |\Hebrew|\-|Month|\-|Name| described
%    above. It returns the name of month in the Hebrew `year' in
%    English. 
%    \begin{macrocode}
\def\HebrewMonthNameInEnglish#1#2{%
    \ifnum #1 = 7%
    \CheckLeapHebrewYear{#2}%
        \if@HebrewLeap Adar II\else Adar\fi%
    \else%
        \ifcase #1%
            % nothing for 0
            \or Tishrei%
            \or Heshvan%
            \or Kislev%
            \or Tebeth%
            \or Shebat%
            \or Adar I%
            \or Adar II%
            \or Nisan%
            \or Iyar%
            \or Sivan%
            \or Tammuz%
            \or Av%
            \or Elul%
        \fi
    \fi}
%    \end{macrocode}
% \end{macro}
%
% \begin{macro}{\FormatForEnglish}
%    The macro |\FormatForEnglish{|\emph{hday}|}{|\emph{hmonth}
%    |}{|\emph{hyear}|}| is similar to |\Format|\-|For|\-|Hebrew|
%    macro described above and returns the formatted Hebrew date in
%    English.
%    \begin{macrocode}
\def\FormatForEnglish#1#2#3{%
    \HebrewMonthNameInEnglish{#2}{#3} \number#1,\ \number#3}
%    \end{macrocode}
% \end{macro}
%
% \subsubsection{Auxiliary Macros}
%
%    \begin{macrocode}
\newcount\@common
%    \end{macrocode}
% \begin{macro}{\Remainder}
%    |\Remainder{|\emph{a}|}{|\emph{b}|}{|\emph{c}|}| calculates 
%    $c = a\%b == a-b\times\frac{a}{b}$
%    \begin{macrocode}
\def\Remainder#1#2#3{%
    #3 = #1%                   %  c = a
    \divide #3 by #2%          %  c = a/b
    \multiply #3 by -#2%       %  c = -b(a/b)
    \advance #3 by #1}%        %  c = a - b(a/b)
%    \end{macrocode}
% \end{macro}
%    \begin{macrocode}
\newif\if@Divisible
%    \end{macrocode}
% \begin{macro}{\CheckIfDivisible}
%    |\CheckIfDivisible{|\emph{a}|}{|\emph{b}|}| sets
%    |\@Divisibletrue| if $a\%b == 0$
%    \begin{macrocode}
\def\CheckIfDivisible#1#2{%
    {%
      \countdef\tmp = 0% \tmp == \count0 - temporary variable
      \Remainder{#1}{#2}{\tmp}%
      \ifnum \tmp = 0%
          \global\@Divisibletrue%
      \else%
          \global\@Divisiblefalse%
      \fi}}
%    \end{macrocode}
% \end{macro}
%
% \begin{macro}{\ifundefined}
%    From the \TeX book, ex. 7.7: 
%    \begin{quote}
%       |\ifundefined{|\emph{command}|}<true text>\else<false text>\fi|
%    \end{quote}
%    \begin{macrocode}
\def\ifundefined#1{\expandafter\ifx\csname#1\endcsname\relax}
%    \end{macrocode}
% \end{macro}
%
% \subsubsection{Gregorian Part}
%
%    \begin{macrocode}
\newif\if@GregorianLeap
%    \end{macrocode}
% \begin{macro}{\IfGregorianLeap}
%    Conditional which is true if Gregorian `year' is a leap year:
%    $((year\%4==0)\wedge(year\%100\neq 0))\vee(year\%400==0)$
%    \begin{macrocode}
\def\IfGregorianLeap#1{%
    \CheckIfDivisible{#1}{4}%
    \if@Divisible%
        \CheckIfDivisible{#1}{100}%
        \if@Divisible%
            \CheckIfDivisible{#1}{400}%
            \if@Divisible%
                \@GregorianLeaptrue%
            \else%
                \@GregorianLeapfalse%
            \fi%
        \else%
            \@GregorianLeaptrue%
        \fi%
    \else%
        \@GregorianLeapfalse%
    \fi%
    \if@GregorianLeap}
%    \end{macrocode}
% \end{macro}
%
% \begin{macro}{\GregorianDaysInPriorMonths}
%    The macro |\GregorianDaysInPriorMonths{|\emph{month}|}{|^^A
%    \emph{year}|}{|\emph{days}|}| calculates the number of days in
%    months prior to `month' in the `year'.
%    \begin{macrocode}
\def\GregorianDaysInPriorMonths#1#2#3{%
    {%
        #3 = \ifcase #1%
               0 \or%             % no month number 0
               0 \or%
              31 \or%
              59 \or%
              90 \or%
             120 \or%
             151 \or%
             181 \or%
             212 \or%
             243 \or%
             273 \or%
             304 \or%
             334%
        \fi%
        \IfGregorianLeap{#2}%
            \ifnum #1 > 2%        % if month after February
                \advance #3 by 1% % add leap day
            \fi%
        \fi%
        \global\@common = #3}%
    #3 = \@common}
%    \end{macrocode}
% \end{macro}
%
% \begin{macro}{\GregorianDaysInPriorYears}
%    The macro |\GregorianDaysInPriorYears{|\emph{year}|}{|^^A
%    \emph{days}|}| calculates the number of days in years prior to
%    the `year'.
%    \begin{macrocode}
\def\GregorianDaysInPriorYears#1#2{%
     {%
         \countdef\tmpc = 4%      % \tmpc==\count4
         \countdef\tmpb = 2%      % \tmpb==\count2
         \tmpb = #1%              %
         \advance \tmpb by -1%    %
         \tmpc = \tmpb%           % \tmpc = \tmpb = year-1
         \multiply \tmpc by 365%  % Days in prior years =
         #2 = \tmpc%              % = 365*(year-1) ...
         \tmpc = \tmpb%           %
         \divide \tmpc by 4%      % \tmpc = (year-1)/4
         \advance #2 by \tmpc%    % ... plus Julian leap days ...
         \tmpc = \tmpb%           %
         \divide \tmpc by 100%    % \tmpc = (year-1)/100
         \advance #2 by -\tmpc%   % ... minus century years ...
         \tmpc = \tmpb%           %
         \divide \tmpc by 400%    % \tmpc = (year-1)/400
         \advance #2 by \tmpc%    % ... plus 4-century years.
         \global\@common = #2}%
    #2 = \@common}
%    \end{macrocode}
% \end{macro}
%
% \begin{macro}{\AbsoluteFromGregorian}
%    The macro |\AbsoluteFromGregorian{|\emph{day}|}{|\emph{month}^^A
%    |}{|\emph{year}|}{|\emph{absdate}|}| calculates the absolute date
%    (days since $01.01.0001$) from Gregorian date |day.month.year|.
%    \begin{macrocode}
\def\AbsoluteFromGregorian#1#2#3#4{%
    {%
        \countdef\tmpd = 0%       % \tmpd==\count0
        #4 = #1%                  % days so far this month
        \GregorianDaysInPriorMonths{#2}{#3}{\tmpd}%
        \advance #4 by \tmpd%     % add days in prior months
        \GregorianDaysInPriorYears{#3}{\tmpd}%
        \advance #4 by \tmpd%     % add days in prior years
        \global\@common = #4}%
    #4 = \@common}
%    \end{macrocode}
% \end{macro}
%
% \subsubsection{Hebrew Part}
%
%    \begin{macrocode}
\newif\if@HebrewLeap
%    \end{macrocode}
% \begin{macro}{\CheckLeapHebrewYear}
%    Set |\@HebrewLeaptrue| if Hebrew `year' is a leap year, i.e.\ if
%    $(1+7\times year)\%19 < 7$ then \emph{true} else \emph{false}
%    \begin{macrocode}
\def\CheckLeapHebrewYear#1{%
    {%
        \countdef\tmpa = 0%       % \tmpa==\count0
        \countdef\tmpb = 1%       % \tmpb==\count1
%
        \tmpa = #1%
        \multiply \tmpa by 7%
        \advance \tmpa by 1%
        \Remainder{\tmpa}{19}{\tmpb}%
        \ifnum \tmpb < 7%         % \tmpb = (7*year+1)%19
            \global\@HebrewLeaptrue%
        \else%
            \global\@HebrewLeapfalse%
        \fi}}
%    \end{macrocode}
% \end{macro}
%
% \begin{macro}{\HebrewElapsedMonths}
%    The macro |\HebrewElapsedMonths{|\emph{year}|}{|\emph{months}|}|
%    determines the number of months elapsed from the Sunday prior to
%    the start of the Hebrew calendar to the mean conjunction of
%    Tishri of Hebrew `year'.
%    \begin{macrocode}
\def\HebrewElapsedMonths#1#2{%
    {%
        \countdef\tmpa = 0%       % \tmpa==\count0
        \countdef\tmpb = 1%       % \tmpb==\count1
        \countdef\tmpc = 2%       % \tmpc==\count2
%
        \tmpa = #1%               %
        \advance \tmpa by -1%     %
        #2 = \tmpa%               % #2 = \tmpa = year-1
        \divide #2 by 19%         % Number of complete Meton cycles
        \multiply #2 by 235%      % #2 = 235*((year-1)/19)
%
        \Remainder{\tmpa}{19}{\tmpb}% \tmpa = years%19-years this cycle
        \tmpc = \tmpb%            %
        \multiply \tmpb by 12%    %
        \advance #2 by \tmpb%     % add regular months this cycle
%
        \multiply \tmpc by 7%     %
        \advance \tmpc by 1%      %
        \divide \tmpc by 19%      % \tmpc = (1+7*((year-1)%19))/19 -
%                                 %  number of leap months this cycle
        \advance #2 by \tmpc%     %  add leap months
%
        \global\@common = #2}%
    #2 = \@common}
%    \end{macrocode}
% \end{macro}
%
% \begin{macro}{\HebrewElapsedDays}
%    The macro |\HebrewElapsedDays{|\emph{year}|}{|\emph{days}|}|
%    determines the number of days elapsed from the Sunday prior to
%    the start of the Hebrew calendar to the mean conjunction of
%    Tishri of Hebrew `year'.
%    \begin{macrocode}
\def\HebrewElapsedDays#1#2{%
    {%
        \countdef\tmpa = 0%       % \tmpa==\count0
        \countdef\tmpb = 1%       % \tmpb==\count1
        \countdef\tmpc = 2%       % \tmpc==\count2
%
        \HebrewElapsedMonths{#1}{#2}%
        \tmpa = #2%               %
        \multiply \tmpa by 13753% %
        \advance \tmpa by 5604%   % \tmpa=MonthsElapsed*13758 + 5604
        \Remainder{\tmpa}{25920}{\tmpc}% \tmpc == ConjunctionParts
        \divide \tmpa by 25920%
%
        \multiply #2 by 29%
        \advance #2 by 1%
        \advance #2 by \tmpa%     %  #2 = 1 + MonthsElapsed*29 +
%                                 %          PartsElapsed/25920
        \Remainder{#2}{7}{\tmpa}% %  \tmpa == DayOfWeek
        \ifnum \tmpc < 19440%
            \ifnum \tmpc < 9924%
            \else%                % New moon at 9 h. 204 p. or later
                \ifnum \tmpa = 2% % on Tuesday ...
                    \CheckLeapHebrewYear{#1}% of a common year
                    \if@HebrewLeap%
                    \else%
                        \advance #2 by 1%
                    \fi%
                \fi%
            \fi%
            \ifnum \tmpc < 16789%
            \else%                 % New moon at 15 h. 589 p. or later
                \ifnum \tmpa = 1%  % on Monday ...
                    \advance #1 by -1%
                    \CheckLeapHebrewYear{#1}% at the end of leap year
                    \if@HebrewLeap%
                        \advance #2 by 1%
                    \fi%
                \fi%
            \fi%
        \else%
            \advance #2 by 1%      %  new moon at or after midday
        \fi%
%
        \Remainder{#2}{7}{\tmpa}%  %  \tmpa == DayOfWeek
        \ifnum \tmpa = 0%          %  if Sunday ...
            \advance #2 by 1%
        \else%                     %
            \ifnum \tmpa = 3%      %  Wednesday ...
                \advance #2 by 1%
            \else%
                \ifnum \tmpa = 5%  %  or Friday
                     \advance #2 by 1%
                \fi%
            \fi%
        \fi%
        \global\@common = #2}%
    #2 = \@common}
%    \end{macrocode}
% \end{macro}
%
% \begin{macro}{\DaysInHebrewYear}
%    The macro |\DaysInHebrewYear{|\emph{year}|}{|\emph{days}|}|
%    calculates the number of days in Hebrew `year'.
%    \begin{macrocode}
\def\DaysInHebrewYear#1#2{%
    {%
        \countdef\tmpe = 12%   % \tmpe==\count12
%
        \HebrewElapsedDays{#1}{\tmpe}%
        \advance #1 by 1%
        \HebrewElapsedDays{#1}{#2}%
        \advance #2 by -\tmpe%
        \global\@common = #2}%
    #2 = \@common}
%    \end{macrocode}
% \end{macro}
%
% \begin{macro}{\HebrewDaysInPriorMonths}
%    The macro |\HebrewDaysInPriorMonths{|\emph{month}|}{|^^A
%    \emph{year}|}{|\emph{days}|}| calculates the nu\-mber of days in
%    months prior to `month' in the `year'.
%    \begin{macrocode}
\def\HebrewDaysInPriorMonths#1#2#3{%
    {%
        \countdef\tmpf= 14%    % \tmpf==\count14
%
        #3 = \ifcase #1%       % Days in prior month of regular year
               0 \or%          % no month number 0
               0 \or%          % Tishri
              30 \or%          % Heshvan
              59 \or%          % Kislev
              89 \or%          % Tebeth
             118 \or%          % Shebat
             148 \or%          % Adar I
             148 \or%          % Adar II
             177 \or%          % Nisan
             207 \or%          % Iyar
             236 \or%          % Sivan
             266 \or%          % Tammuz
             295 \or%          % Av
             325 \or%          % Elul
             400%              % Dummy
        \fi%
        \CheckLeapHebrewYear{#2}%
        \if@HebrewLeap%            % in leap year
            \ifnum #1 > 6%         % if month after Adar I
                \advance #3 by 30% % add  30 days
            \fi%
        \fi%
        \DaysInHebrewYear{#2}{\tmpf}%
        \ifnum #1 > 3%
            \ifnum \tmpf = 353%    %
                \advance #3 by -1% %
            \fi%                   %  Short Kislev
            \ifnum \tmpf = 383%    %
                \advance #3 by -1% %
            \fi%                   %
        \fi%
%
        \ifnum #1 > 2%
            \ifnum \tmpf = 355%    %
                \advance #3 by 1%  %
            \fi%                   %  Long Heshvan
            \ifnum \tmpf = 385%    %
                \advance #3 by 1%  %
            \fi%                   %
        \fi%
        \global\@common = #3}%
    #3 = \@common}
%    \end{macrocode}
% \end{macro}
%
% \begin{macro}{\AbsoluteFromHebrew}
%    The macro |\AbsoluteFromHebrew{|\emph{day}|}{|\emph{month}^^A
%    |}{|\emph{year}|}{|\emph{absdate}|}| calculates the absolute date
%    of Hebrew date |day.month.year|.
%    \begin{macrocode}
\def\AbsoluteFromHebrew#1#2#3#4{%
    {%
        #4 = #1%
        \HebrewDaysInPriorMonths{#2}{#3}{#1}%
        \advance #4 by #1%         % Add days in prior months this year
        \HebrewElapsedDays{#3}{#1}%
        \advance #4 by #1%         % Add days in prior years
        \advance #4 by -1373429%   % Subtract days before Gregorian
        \global\@common = #4}%     %   01.01.0001
    #4 = \@common}
%    \end{macrocode}
% \end{macro}
%
% \begin{macro}{\HebrewFromGregorian}
%    The macro |\HebrewFromGregorian{|\emph{Gday}|}{|\emph{Gmonth}^^A
%    |}{|\emph{Gyear}|}{|\emph{Hday}|}{|\emph{Hmonth}|}|\-|{|^^A
%    \emph{Hyear}|}| evaluates Hebrew date |Hday|, |Hmonth|, |Hyear|
%    from Gregorian date |Gday|, |Gmonth|, |Gyear|.
%    \begin{macrocode}
\def\HebrewFromGregorian#1#2#3#4#5#6{%
    {%
        \countdef\tmpx= 17%        % \tmpx==\count17
        \countdef\tmpy= 18%        % \tmpy==\count18
        \countdef\tmpz= 19%        % \tmpz==\count19
%
        #6 = #3%                   %
        \global\advance #6 by 3761%  approximation from above
        \AbsoluteFromGregorian{#1}{#2}{#3}{#4}%
        \tmpz = 1  \tmpy = 1%
        \AbsoluteFromHebrew{\tmpz}{\tmpy}{#6}{\tmpx}%
        \ifnum \tmpx > #4%              %
            \global\advance #6 by -1% Hyear = Gyear + 3760
            \AbsoluteFromHebrew{\tmpz}{\tmpy}{#6}{\tmpx}%
        \fi%                            %
        \advance #4 by -\tmpx%     % Days in this year
        \advance #4 by 1%          %
        #5 = #4%                   %
        \divide #5 by 30%          % Approximation for month from below
        \loop%                     % Search for month
            \HebrewDaysInPriorMonths{#5}{#6}{\tmpx}%
            \ifnum \tmpx < #4%
                \advance #5 by 1%
                \tmpy = \tmpx%
        \repeat%
        \global\advance #5 by -1%
        \global\advance #4 by -\tmpy}}
%</calendar>
%    \end{macrocode}
% \end{macro}
%
% \Finale
%%
%% \CharacterTable
%%  {Upper-case    \A\B\C\D\E\F\G\H\I\J\K\L\M\N\O\P\Q\R\S\T\U\V\W\X\Y\Z
%%   Lower-case    \a\b\c\d\e\f\g\h\i\j\k\l\m\n\o\p\q\r\s\t\u\v\w\x\y\z
%%   Digits        \0\1\2\3\4\5\6\7\8\9
%%   Exclamation   \!     Double quote  \"     Hash (number) \#
%%   Dollar        \$     Percent       \%     Ampersand     \&
%%   Acute accent  \'     Left paren    \(     Right paren   \)
%%   Asterisk      \*     Plus          \+     Comma         \,
%%   Minus         \-     Point         \.     Solidus       \/
%%   Colon         \:     Semicolon     \;     Less than     \<
%%   Equals        \=     Greater than  \>     Question mark \?
%%   Commercial at \@     Left bracket  \[     Backslash     \\
%%   Right bracket \]     Circumflex    \^     Underscore    \_
%%   Grave accent  \`     Left brace    \{     Vertical bar  \|
%%   Right brace   \}     Tilde         \~}
%%
\endinput
}
\bbl@tempa{hungarian}{%%
%% This file will generate fast loadable files and documentation
%% driver files from the doc files in this package when run through
%% LaTeX or TeX.
%%
%% Copyright 1989-2005 Johannes L. Braams and any individual authors
%% listed elsewhere in this file.  All rights reserved.
%% 
%% This file is part of the Babel system.
%% --------------------------------------
%% 
%% It may be distributed and/or modified under the
%% conditions of the LaTeX Project Public License, either version 1.3
%% of this license or (at your option) any later version.
%% The latest version of this license is in
%%   http://www.latex-project.org/lppl.txt
%% and version 1.3 or later is part of all distributions of LaTeX
%% version 2003/12/01 or later.
%% 
%% This work has the LPPL maintenance status "maintained".
%% 
%% The Current Maintainer of this work is Johannes Braams.
%% 
%% The list of all files belonging to the LaTeX base distribution is
%% given in the file `manifest.bbl. See also `legal.bbl' for additional
%% information.
%% 
%% The list of derived (unpacked) files belonging to the distribution
%% and covered by LPPL is defined by the unpacking scripts (with
%% extension .ins) which are part of the distribution.
%%
%% --------------- start of docstrip commands ------------------
%%
\def\filedate{1999/04/11}
\def\batchfile{magyar.ins}
\input docstrip.tex

{\ifx\generate\undefined
\Msg{**********************************************}
\Msg{*}
\Msg{* This installation requires docstrip}
\Msg{* version 2.3c or later.}
\Msg{*}
\Msg{* An older version of docstrip has been input}
\Msg{*}
\Msg{**********************************************}
\errhelp{Move or rename old docstrip.tex.}
\errmessage{Old docstrip in input path}
\batchmode
\csname @@end\endcsname
\fi}

\declarepreamble\mainpreamble
This is a generated file.

Copyright 1989-2005 Johannes L. Braams and any individual authors
listed elsewhere in this file.  All rights reserved.

This file was generated from file(s) of the Babel system.
---------------------------------------------------------

It may be distributed and/or modified under the
conditions of the LaTeX Project Public License, either version 1.3
of this license or (at your option) any later version.
The latest version of this license is in
  http://www.latex-project.org/lppl.txt
and version 1.3 or later is part of all distributions of LaTeX
version 2003/12/01 or later.

This work has the LPPL maintenance status "maintained".

The Current Maintainer of this work is Johannes Braams.

This file may only be distributed together with a copy of the Babel
system. You may however distribute the Babel system without
such generated files.

The list of all files belonging to the Babel distribution is
given in the file `manifest.bbl'. See also `legal.bbl for additional
information.

The list of derived (unpacked) files belonging to the distribution
and covered by LPPL is defined by the unpacking scripts (with
extension .ins) which are part of the distribution.
\endpreamble

\declarepreamble\fdpreamble
This is a generated file.

Copyright 1989-2005 Johannes L. Braams and any individual authors
listed elsewhere in this file.  All rights reserved.

This file was generated from file(s) of the Babel system.
---------------------------------------------------------

It may be distributed and/or modified under the
conditions of the LaTeX Project Public License, either version 1.3
of this license or (at your option) any later version.
The latest version of this license is in
  http://www.latex-project.org/lppl.txt
and version 1.3 or later is part of all distributions of LaTeX
version 2003/12/01 or later.

This work has the LPPL maintenance status "maintained".

The Current Maintainer of this work is Johannes Braams.

This file may only be distributed together with a copy of the Babel
system. You may however distribute the Babel system without
such generated files.

The list of all files belonging to the Babel distribution is
given in the file `manifest.bbl'. See also `legal.bbl for additional
information.

In particular, permission is granted to customize the declarations in
this file to serve the needs of your installation.

However, NO PERMISSION is granted to distribute a modified version
of this file under its original name.

\endpreamble

\keepsilent

\usedir{tex/generic/babel} 

\usepreamble\mainpreamble
\generate{\file{magyar.ldf}{\from{magyar.dtx}{code}}
          }
\usepreamble\fdpreamble

\ifToplevel{
\Msg{***********************************************************}
\Msg{*}
\Msg{* To finish the installation you have to move the following}
\Msg{* files into a directory searched by TeX:}
\Msg{*}
\Msg{* \space\space All *.def, *.fd, *.ldf, *.sty}
\Msg{*}
\Msg{* To produce the documentation run the files ending with}
\Msg{* '.dtx' and `.fdd' through LaTeX.}
\Msg{*}
\Msg{* Happy TeXing}
\Msg{***********************************************************}
}
 
\endinput
}
\bbl@tempa{indon}{\input{bahasai.ldf}}
\bbl@tempa{indonesian}{\input{bahasai.ldf}}
\bbl@tempa{lowersorbian}{% \iffalse meta-comme

% Copyright 1989-2008 Johannes L. Braams and any individual autho
% listed elsewhere in this file.  All rights reserve

% This file is part of the Babel syste
% ------------------------------------

% It may be distributed and/or modified under t
% conditions of the LaTeX Project Public License, either version 1
% of this license or (at your option) any later versio
% The latest version of this license is
%   http://www.latex-project.org/lppl.t
% and version 1.3 or later is part of all distributions of LaT
% version 2003/12/01 or late

% This work has the LPPL maintenance status "maintained

% The Current Maintainer of this work is Johannes Braam

% The list of all files belonging to the Babel system
% given in the file `manifest.bbl. See also `legal.bbl' for addition
% informatio

% The list of derived (unpacked) files belonging to the distributi
% and covered by LPPL is defined by the unpacking scripts (wi
% extension .ins) which are part of the distributio
% \
% \CheckSum{15
% \iffal

%    Tell the \LaTeX\ system who we are and write an entry on t
%    transcrip
%<*dt
\ProvidesFile{lsorbian.dt
%</dt
%<code>\ProvidesLanguage{lsorbia
%\
%\ProvidesFile{lsorbian.dt
        [2008/03/17 v1.0g Lower Sorbian support from the babel syste
%\iffal
%% File `lsorbian.dt
%% Babel package for LaTeX version
%% Copyright (C) 1989 - 20
%%           by Johannes Braams, TeXni

%% Lower Sorbian Language Definition Fi
%% Copyright (C) 1994 - 20
%%           by Eduard Wern
%           Werner, Eduard
%           Serbski institut z. t
%           Dw\'orni\v{s}\'cowa
%           02625 Budy\v{s}in/Bautz
%           Germany
%           (??)3591 497223
%           edi at kaihh.hanse.de

%% Please report errors to: Eduard Werner edi at kaihh.hanse.

%    This file is part of the babel system, it provides the sour
%    code for the Lower Sorbian definition fil
%<*filedrive
\documentclass{ltxdo
\newcommand*\TeXhax{\TeX ha
\newcommand*\babel{\textsf{babel
\newcommand*\langvar{$\langle \it lang \rangle
\newcommand*\note[1]
\newcommand*\Lopt[1]{\textsf{#1
\newcommand*\file[1]{\texttt{#1
\begin{documen
 \DocInput{lsorbian.dt
\end{documen
%</filedrive
%\

% \GetFileInfo{lsorbian.dt

% \changes{lsorbian-0.1}{1994/10/10}{First versio
% \changes{lsorbian-1.0d}{1996/10/10}{Replaced \cs{undefined} wi
%    \cs{@undefined} and \cs{empty} with \cs{@empty} for consisten
%    with \LaTeX, moved the definition of \cs{atcatcode} right to t
%    beginning

%  \section{The Lower Sorbian languag

%    The file \file{\filename}\footnote{The file described in th
%    section has version number \fileversion\ and was last revised
%    \filedate.  It was written by Eduard Wern
%    (\texttt{edi@kaihh.hanse.de}).}  It defines all t
%    language-specific macros for Lower Sorbia

% \StopEventually

%    The macro |\LdfInit| takes care of preventing that this file
%    loaded more than once, checking the category code of t
%    \texttt{@} sign, et
% \changes{lsorbian-1.0d}{1996/11/03}{Now use \cs{LdfInit} to perfo
%    initial check
% \changes{lsorbian-1.0g}{2007/10/19}{This file can be loaded und
%    more than one name
%    \begin{macrocod
%<*cod
\LdfInit\CurrentOption{date\CurrentOptio
%    \end{macrocod

%    When this file is read as an option, i.e. by the |\usepackag
%    command, \texttt{lsorbian} will be an `unknown' language, in whi
%    case we have to make it known. So we check for the existence
%    |\l@lsorbian| to see whether we have to do something her
% \changes{lsorbian-1.0g}{2007/10/19}{This file can be loaded und
%    more than one name
%
%    \babel\ also knwos the option \Lopt{lowersorbian} we have
%    check that as wel

%    \begin{macrocod
\ifx\l@lowersorbian\@undefin
  \ifx\l@lsorbian\@undefin
    \@nopatterns{Lsorbia
    \adddialect\l@lsorbian\
    \let\l@lowersorbian\l@lsorbi
  \el
    \let\l@lowersorbian\l@lsorbi
  \
\el
  \let\l@lsorbian\l@lowersorbi
\
%    \end{macrocod

%    The next step consists of defining commands to switch to (a
%    from) the Lower Sorbian languag

%  \begin{macro}{\captionslsorbia
%    The macro |\captionslsorbian| defines all strings used in the fo
%    standard documentclasses provided with \LaTe
% \changes{lsorbian-1.0b}{1995/07/04}{Added \cs{proofname} f
%    AMS-\LaTe
% \changes{lsorbian-1.0f}{2000/09/22}{Added \cs{glossaryname
% \changes{lsorbian-1.0g}{2007/10/19}{Make this work for more than o
%    option name
%    \begin{macrocod
\@namedef{captions\CurrentOption}
  \def\prefacename{Zawod
  \def\refname{Referency
  \def\abstractname{Abstrakt
  \def\bibname{Literatura
  \def\chaptername{Kapitl
  \def\appendixname{Dodawki
  \def\contentsname{Wop\'simje\'se
  \def\listfigurename{Zapis wobrazow
  \def\listtablename{Zapis tabulkow
  \def\indexname{Indeks
  \def\figurename{Wobraz
  \def\tablename{Tabulka
  \def\partname{\'Z\v el
  \def\enclname{P\'si\l oga
  \def\ccname{CC
  \def\headtoname{Komu
  \def\pagename{Strona
  \def\seename{gl.
  \def\alsoname{gl.~teke
  \def\proofname{Proof}%  <-- needs translati
  \def\glossaryname{Glossary}% <-- Needs translati

%    \end{macrocod
%  \end{macr

%  \begin{macro}{\newdatelsorbia
%    The macro |\newdatelsorbian| redefines the command |\today|
%    produce Lower Sorbian date
% \changes{lsorbian-1.0e}{1997/10/01}{Use \cs{edef} to defi
%    \cs{today} to save memor
% \changes{lsorbian-1.0e}{1998/03/28}{use \cs{def} instead
%    \cs{edef}
% \changes{lsorbian-1.0g}{2007/10/19}{Make this work for more than o
%    option name
%    \begin{macrocod
\@namedef{newdate\CurrentOption}
  \def\today{\number\day.~\ifcase\month\
    januara\or februara\or m\v erca\or apryla\or maja\
    junija\or julija\or awgusta\or septembra\or oktobra\
    nowembra\or decembra\
    \space \number\year
%    \end{macrocod
%  \end{macr

%  \begin{macro}{\olddatelsorbia
%    The macro |\olddatelsorbian| redefines the command |\today|
%    produce old-style Lower Sorbian date
% \changes{lsorbian-1.0g}{2007/10/19}{Make this work for more than o
%    option name
%    \begin{macrocod
\@namedef{olddate\CurrentOption}
  \def\today{\number\day.~\ifcase\month\
    wjelikego ro\v zka\
    ma\l ego ro\v zka\
    nal\v etnika\
    jat\v sownika\
    ro\v zownika\
    sma\v znika\
    pra\v znika\
    \v znje\'nca\
    po\v znje\'nca\
    winowca\
    nazymnika\o
    godownika\fi \space \number\year
%    \end{macrocod
%  \end{macr

%    The default will be the new-style date
% \changes{lsorbian-1.0g}{2007/10/19}{Make this work for more than o
%    option name
%    \begin{macrocod
\expandafter\let\csname date\CurrentOption\expandafter\endcsna
                \csname newdate\CurrentOption\endcsna
%    \end{macrocod

% \begin{macro}{\extraslsorbia
% \begin{macro}{\noextraslsorbia
%    The macro |\extraslsorbian| will perform all the ext
%    definitions needed for the lsorbian language. The mac
%    |\noextraslsorbian| is used to cancel the actions
%    |\extraslsorbian|.  For the moment these macros are empty b
%    they are defined for compatibility with the other langua
%    definition file

%    \begin{macrocod
\@namedef{extras\CurrentOption}
\@namedef{noextras\CurrentOption}
%    \end{macrocod
% \end{macr
% \end{macr

%    The macro |\ldf@finish| takes care of looking for
%    configuration file, setting the main language to be switched
%    at |\begin{document}| and resetting the category code
%    \texttt{@} to its original valu
% \changes{lsorbian-1.0d}{1996/11/03}{Now use \cs{ldf@finish} to wr
%    up
% \changes{lsorbian-1.0g}{2007/10/19}{Make this work for more than o
%    option nam
%    \begin{macrocod
\ldf@finish\CurrentOpti
%</cod
%    \end{macrocod

% \Fina

%% \CharacterTab
%%  {Upper-case    \A\B\C\D\E\F\G\H\I\J\K\L\M\N\O\P\Q\R\S\T\U\V\W\X\Y
%%   Lower-case    \a\b\c\d\e\f\g\h\i\j\k\l\m\n\o\p\q\r\s\t\u\v\w\x\y
%%   Digits        \0\1\2\3\4\5\6\7\8
%%   Exclamation   \!     Double quote  \"     Hash (number)
%%   Dollar        \$     Percent       \%     Ampersand
%%   Acute accent  \'     Left paren    \(     Right paren
%%   Asterisk      \*     Plus          \+     Comma
%%   Minus         \-     Point         \.     Solidus
%%   Colon         \:     Semicolon     \;     Less than
%%   Equals        \=     Greater than  \>     Question mark
%%   Commercial at \@     Left bracket  \[     Backslash
%%   Right bracket \]     Circumflex    \^     Underscore
%%   Grave accent  \`     Left brace    \{     Vertical bar
%%   Right brace   \}     Tilde         \

\endinp
}
\bbl@tempa{malay}{% \iffalse meta-com

% Copyright 1989-2008 Johannes L. Braams and any individual aut
% listed elsewhere in this file.  All rights reser

% This file is part of the Babel sys
% ----------------------------------

% It may be distributed and/or modified under
% conditions of the LaTeX Project Public License, either version
% of this license or (at your option) any later vers
% The latest version of this license i
%   http://www.latex-project.org/lppl
% and version 1.3 or later is part of all distributions of L
% version 2003/12/01 or la

% This work has the LPPL maintenance status "maintain

% The Current Maintainer of this work is Johannes Bra

% The list of all files belonging to the Babel syste
% given in the file `manifest.bbl. See also `legal.bbl' for additi
% informat

% The list of derived (unpacked) files belonging to the distribu
% and covered by LPPL is defined by the unpacking scripts (
% extension .ins) which are part of the distribut
%
% \CheckSum{
%\iff
%    Tell the \LaTeX\ system who we are and write an entry on
%    transcr
%<*
\ProvidesFile{bahasam.
%</
%<code>\ProvidesLanguage{baha

%\ProvidesFile{bahasam.
       [2008/01/27 v1.0k Bahasa Malaysia support from the babel sys
%\iff
%% File `bahasam.
%% Babel package for LaTeX versio
%% Copyright (C) 1989 -
%%           by Johannes Braams, TeX

%% Bahasa Malaysia Language Definition
%% Copyright (C) 1994 -
%%           by J"org Knappen, (joerg.knappen at alpha.ntp.springer
%              Terry Mart (mart at vkpmzd.kph.uni-mainz
%              Institut f\"ur Kernph
%              Johannes Gutenberg-Universit\"at M
%              D-55099 M
%              Ger

%% Copyright (C) 2005,
%%           by Bob Margolis, (bob.margolis at ntlworld.
%              derived from J"ork Knappen's work - see ab
%%           [With help from Awangku Merali Pengiran Mohamed (Saraw
%               gratefully acknowled
%               Yate
%

%% Please report errors to: Bob Marg
%%                          bob.margolis at ntlworld
%%                          J.L. Br
%%                          babel at braams.xs4al

%    This file is part of the babel system, it provides the so
%    code for the  Bahasa Malaysia language defini
%    file.  The original version of this file was written by T
%    Mart (mart@vkpmzd.kph.uni-mainz.de) and J"org Kna
%    (knappen@vkpmzd.kph.uni-mainz.
%<*filedri
\documentclass{ltx
\newcommand*\TeXhax{\TeX
\newcommand*\babel{\textsf{bab
\newcommand*\langvar{$\langle \it lang \rang
\newcommand*\note[
\newcommand*\Lopt[1]{\textsf{
\newcommand*\file[1]{\texttt{
\begin{docum
 \DocInput{bahasam.
\end{docum
%</filedri

% \GetFileInfo{bahasam.

% \changes{bahasa-0.9c}{1994/06/26}{Removed the use of \cs{filed
%    and moved identification after the loading of \file{babel.d
% \changes{bahasa-1.0d}{1996/07/10}{Replaced \cs{undefined}
%    \cs{@undefined} and \cs{empty} with \cs{@empty} for consist
%    with \LaT
% \changes{bahasa-1.0e}{1996/10/10}{Moved the definitio
%    \cs{atcatcode} right to the beginni
% \changes{bahasam-0.9f}{2005/11/22}{A number of changes to make
%    specific to Bahasa Maya

%  \section{The Bahasa Malaysia langu

%    The file \file{\filename}\footnote{The file described in
%    section has version number \fileversion\ and was last revise
%    \filedate.}  defines all the language definition macros for
%    Bahasa Malaysia language. Bahasa just m
%    `language' in Bahasa Malaysia. A number of terms differ from those
%    in bahasa indone

%    For this language currently no special definitions are neede
%    availa

% \StopEventual

%    The macro |\LdfInit| takes care of preventing that this fil
%    loaded more than once, checking the category code of
%    \texttt{@} sign,
% \changes{bahasa-1.0e}{1996/11/02}{Now use \cs{LdfInit} to per
%    initial che
% \changes{bahasam-v1.0j}{2005/11/23}{Make it possible that this
%    is loaded by variuos opti
%    \begin{macroc
%<*c
\LdfInit\CurrentOption{date\CurrentOpt
%    \end{macroc

%    When this file is read as an option, i.e. by the |\usepack
%    command, \texttt{bahasa} could be an `unknown' language in w
%    case we have to make it known. So we check for the existenc
%    |\l@bahasa| to see whether we have to do something h

%    For both Bahasa Malaysia and Bahasa Indonesia the same se
%    hyphenation patterns can be used which are available in the
%    \file{inhyph.tex}. However it could be loaded using any of
%    possible Babel options fot the Malaysian and Indone
%    languase. So first we try to find out whether this is the c

% \changes{bahasa-0.9c}{1994/06/26}{Now use \cs{@patterns} to pro
%    the warn
%    \begin{macroc
\ifx\l@malay\@undef
  \ifx\l@meyalu\@undef
    \ifx\l@bahasam\@undef
      \ifx\l@bahasa\@undef
        \ifx\l@bahasai\@undef
          \ifx\l@indon\@undef
            \ifx\l@indonesian\@undef
              \@nopatterns{Bahasa Malay
              \adddialect\l@malay0\r
            \
              \let\l@malay\l@indone

          \
            \let\l@malay\l@i

        \
          \let\l@malay\l@bah

      \
        \let\l@malay\l@ba

    \
      \let\l@malay\l@bah

  \
    \let\l@malay\l@me


%    \end{macroc

%    Now that we are sure the |\l@malay| has some valid definitio
%    need to make sure that a name to access the hyphenation patte
%    corresponding to the option used, is availa
%    \begin{macroc
\expandafter\expandafter\expandafter
  \expandafter\cs
  \expandafter l\expandafter @\CurrentOption\endcs
  \l@m
%    \end{macroc

%    The next step consists of defining commands to switch to
%    from) the Bahasa langu

% \begin{macro}{\captionsbaha
%    The macro |\captionsbahasam| defines all strings used in the
%    standard documentclasses provided with \La
% \changes{bahasa-1.0b}{1995/07/04}{Added \cs{proofname}
%    AMS-\La
% \changes{bahasa-1.0d}{1996/07/09}{Replaced `Proof' by `Bu
%    (PR2214
% \changes{bahasa-1.0h}{2000/09/19}{Added \cs{glossaryna
% \changes{bahasa-1.0i}{2003/11/17}{Inserted translation for Gloss
% \changes{bahasam-1.0k}{2008/01/27}{Inserted changes from Awangku Mera
%    \begin{macroc
\@namedef{captions\CurrentOptio
  \def\prefacename{Praka
  \def\refname{Rujuk
  \def\abstractname{Abstrak}% (sometime it's called 'intis
                              %  or 'ikhtis
  \def\bibname{Bibliogra
  \def\chaptername{B
  \def\appendixname{Lampir
  \def\contentsname{Kandung
  \def\listfigurename{Senarai Gamb
  \def\listtablename{Senarai Jadu
  \def\indexname{Inde
  \def\figurename{Gamb
  \def\tablename{Jadu
  \def\partname{Bahagi
%  Subject:  Per
%  From:
  \def\enclname{Lampir
  \def\ccname{sk}% (short form for 'Salinan Kepa
  \def\headtoname{Kepa
  \def\pagename{Halam
%  Notes (Endnotes): Cat
  \def\seename{sila  ruj
  \def\alsoname{rujuk ju
  \def\proofname{Buk
  \def\glossaryname{Istil

%    \end{macroc
% \end{ma

% \begin{macro}{\datebaha
%    The macro |\datebahasam| redefines the command |\today| to pro
%    Bahasa Malaysian da
% \changes{bahasa-1.0f}{1997/10/01}{Use \cs{edef} to define \cs{tod
% \changes{bahasa~1.0f}{1998/03/28}{use \cs{def} instead of \cs{e
%    to save mem
% \changes{bahasa-1.0g}{1999/03/12}{Februari should be spelle
%    Pebru
% \changes{bahasam-1.0k}{2008/01/27}{Februari restored to BM spelli
%    see Collins Kamus Dwibahasa 2
%    \begin{macroc
\@namedef{date\CurrentOptio
  \def\today{\number\day~\ifcase\mont
    Januari\or Februari\or Mac\or April\or Mei\or Ju
    Julai\or Ogos\or September\or Oktober\or November\or Disembe
    \space \number\ye
%    \end{macroc
% \end{ma


% \begin{macro}{\extrasbaha
% \begin{macro}{\noextrasbaha
%    The macro |\extrasbahasa| will perform all the extra definit
%    needed for the Bahasa language. The macro |\extrasbahasa| is
%    to cancel the actions of |\extrasbahasa|.  For the moment t
%    macros are empty but they are defined for compatibility with
%    other language definition fi

%    \begin{macroc
\@namedef{extras\CurrentOptio
\@namedef{noextras\CurrentOptio
%    \end{macroc
% \end{ma
% \end{ma

%  \begin{macro}{\bahasamhyphenm
%    The bahasam hyphenation patterns should be used
%    |\lefthyphenmin| set to~2 and |\righthyphenmin| set t
% \changes{bahasa-1.0e}{1996/08/07}{use \cs{bahasamhyphenmins} to s
%    the correct val
% \changes{bahasa-1.0h}{2000/09/22}{Now use \cs{providehyphenmins
%    provide a default va
%    \begin{macroc
\providehyphenmins{\CurrentOption}{\tw@\
%    \end{macroc
%  \end{ma

%    The macro |\ldf@finish| takes care of looking f
%    configuration file, setting the main language to be switche
%    at |\begin{document}| and resetting the category cod
%    \texttt{@} to its original va
% \changes{bahasa-1.0e}{1996/11/02}{Now use \cs{ldf@finish} to wrap
%    \begin{macroc
\ldf@finish{\CurrentOpt
%</c
%    \end{macroc

% \Fi

%% \CharacterT
%%  {Upper-case    \A\B\C\D\E\F\G\H\I\J\K\L\M\N\O\P\Q\R\S\T\U\V\W\X
%%   Lower-case    \a\b\c\d\e\f\g\h\i\j\k\l\m\n\o\p\q\r\s\t\u\v\w\x
%%   Digits        \0\1\2\3\4\5\6\7
%%   Exclamation   \!     Double quote  \"     Hash (number
%%   Dollar        \$     Percent       \%     Ampersand
%%   Acute accent  \'     Left paren    \(     Right paren
%%   Asterisk      \*     Plus          \+     Comma
%%   Minus         \-     Point         \.     Solidus
%%   Colon         \:     Semicolon     \;     Less than
%%   Equals        \=     Greater than  \>     Question mar
%%   Commercial at \@     Left bracket  \[     Backslash
%%   Right bracket \]     Circumflex    \^     Underscore
%%   Grave accent  \`     Left brace    \{     Vertical bar
%%   Right brace   \}     Tilde

\endi
}
\bbl@tempa{meyalu}{% \iffalse meta-com

% Copyright 1989-2008 Johannes L. Braams and any individual aut
% listed elsewhere in this file.  All rights reser

% This file is part of the Babel sys
% ----------------------------------

% It may be distributed and/or modified under
% conditions of the LaTeX Project Public License, either version
% of this license or (at your option) any later vers
% The latest version of this license i
%   http://www.latex-project.org/lppl
% and version 1.3 or later is part of all distributions of L
% version 2003/12/01 or la

% This work has the LPPL maintenance status "maintain

% The Current Maintainer of this work is Johannes Bra

% The list of all files belonging to the Babel syste
% given in the file `manifest.bbl. See also `legal.bbl' for additi
% informat

% The list of derived (unpacked) files belonging to the distribu
% and covered by LPPL is defined by the unpacking scripts (
% extension .ins) which are part of the distribut
%
% \CheckSum{
%\iff
%    Tell the \LaTeX\ system who we are and write an entry on
%    transcr
%<*
\ProvidesFile{bahasam.
%</
%<code>\ProvidesLanguage{baha

%\ProvidesFile{bahasam.
       [2008/01/27 v1.0k Bahasa Malaysia support from the babel sys
%\iff
%% File `bahasam.
%% Babel package for LaTeX versio
%% Copyright (C) 1989 -
%%           by Johannes Braams, TeX

%% Bahasa Malaysia Language Definition
%% Copyright (C) 1994 -
%%           by J"org Knappen, (joerg.knappen at alpha.ntp.springer
%              Terry Mart (mart at vkpmzd.kph.uni-mainz
%              Institut f\"ur Kernph
%              Johannes Gutenberg-Universit\"at M
%              D-55099 M
%              Ger

%% Copyright (C) 2005,
%%           by Bob Margolis, (bob.margolis at ntlworld.
%              derived from J"ork Knappen's work - see ab
%%           [With help from Awangku Merali Pengiran Mohamed (Saraw
%               gratefully acknowled
%               Yate
%

%% Please report errors to: Bob Marg
%%                          bob.margolis at ntlworld
%%                          J.L. Br
%%                          babel at braams.xs4al

%    This file is part of the babel system, it provides the so
%    code for the  Bahasa Malaysia language defini
%    file.  The original version of this file was written by T
%    Mart (mart@vkpmzd.kph.uni-mainz.de) and J"org Kna
%    (knappen@vkpmzd.kph.uni-mainz.
%<*filedri
\documentclass{ltx
\newcommand*\TeXhax{\TeX
\newcommand*\babel{\textsf{bab
\newcommand*\langvar{$\langle \it lang \rang
\newcommand*\note[
\newcommand*\Lopt[1]{\textsf{
\newcommand*\file[1]{\texttt{
\begin{docum
 \DocInput{bahasam.
\end{docum
%</filedri

% \GetFileInfo{bahasam.

% \changes{bahasa-0.9c}{1994/06/26}{Removed the use of \cs{filed
%    and moved identification after the loading of \file{babel.d
% \changes{bahasa-1.0d}{1996/07/10}{Replaced \cs{undefined}
%    \cs{@undefined} and \cs{empty} with \cs{@empty} for consist
%    with \LaT
% \changes{bahasa-1.0e}{1996/10/10}{Moved the definitio
%    \cs{atcatcode} right to the beginni
% \changes{bahasam-0.9f}{2005/11/22}{A number of changes to make
%    specific to Bahasa Maya

%  \section{The Bahasa Malaysia langu

%    The file \file{\filename}\footnote{The file described in
%    section has version number \fileversion\ and was last revise
%    \filedate.}  defines all the language definition macros for
%    Bahasa Malaysia language. Bahasa just m
%    `language' in Bahasa Malaysia. A number of terms differ from those
%    in bahasa indone

%    For this language currently no special definitions are neede
%    availa

% \StopEventual

%    The macro |\LdfInit| takes care of preventing that this fil
%    loaded more than once, checking the category code of
%    \texttt{@} sign,
% \changes{bahasa-1.0e}{1996/11/02}{Now use \cs{LdfInit} to per
%    initial che
% \changes{bahasam-v1.0j}{2005/11/23}{Make it possible that this
%    is loaded by variuos opti
%    \begin{macroc
%<*c
\LdfInit\CurrentOption{date\CurrentOpt
%    \end{macroc

%    When this file is read as an option, i.e. by the |\usepack
%    command, \texttt{bahasa} could be an `unknown' language in w
%    case we have to make it known. So we check for the existenc
%    |\l@bahasa| to see whether we have to do something h

%    For both Bahasa Malaysia and Bahasa Indonesia the same se
%    hyphenation patterns can be used which are available in the
%    \file{inhyph.tex}. However it could be loaded using any of
%    possible Babel options fot the Malaysian and Indone
%    languase. So first we try to find out whether this is the c

% \changes{bahasa-0.9c}{1994/06/26}{Now use \cs{@patterns} to pro
%    the warn
%    \begin{macroc
\ifx\l@malay\@undef
  \ifx\l@meyalu\@undef
    \ifx\l@bahasam\@undef
      \ifx\l@bahasa\@undef
        \ifx\l@bahasai\@undef
          \ifx\l@indon\@undef
            \ifx\l@indonesian\@undef
              \@nopatterns{Bahasa Malay
              \adddialect\l@malay0\r
            \
              \let\l@malay\l@indone

          \
            \let\l@malay\l@i

        \
          \let\l@malay\l@bah

      \
        \let\l@malay\l@ba

    \
      \let\l@malay\l@bah

  \
    \let\l@malay\l@me


%    \end{macroc

%    Now that we are sure the |\l@malay| has some valid definitio
%    need to make sure that a name to access the hyphenation patte
%    corresponding to the option used, is availa
%    \begin{macroc
\expandafter\expandafter\expandafter
  \expandafter\cs
  \expandafter l\expandafter @\CurrentOption\endcs
  \l@m
%    \end{macroc

%    The next step consists of defining commands to switch to
%    from) the Bahasa langu

% \begin{macro}{\captionsbaha
%    The macro |\captionsbahasam| defines all strings used in the
%    standard documentclasses provided with \La
% \changes{bahasa-1.0b}{1995/07/04}{Added \cs{proofname}
%    AMS-\La
% \changes{bahasa-1.0d}{1996/07/09}{Replaced `Proof' by `Bu
%    (PR2214
% \changes{bahasa-1.0h}{2000/09/19}{Added \cs{glossaryna
% \changes{bahasa-1.0i}{2003/11/17}{Inserted translation for Gloss
% \changes{bahasam-1.0k}{2008/01/27}{Inserted changes from Awangku Mera
%    \begin{macroc
\@namedef{captions\CurrentOptio
  \def\prefacename{Praka
  \def\refname{Rujuk
  \def\abstractname{Abstrak}% (sometime it's called 'intis
                              %  or 'ikhtis
  \def\bibname{Bibliogra
  \def\chaptername{B
  \def\appendixname{Lampir
  \def\contentsname{Kandung
  \def\listfigurename{Senarai Gamb
  \def\listtablename{Senarai Jadu
  \def\indexname{Inde
  \def\figurename{Gamb
  \def\tablename{Jadu
  \def\partname{Bahagi
%  Subject:  Per
%  From:
  \def\enclname{Lampir
  \def\ccname{sk}% (short form for 'Salinan Kepa
  \def\headtoname{Kepa
  \def\pagename{Halam
%  Notes (Endnotes): Cat
  \def\seename{sila  ruj
  \def\alsoname{rujuk ju
  \def\proofname{Buk
  \def\glossaryname{Istil

%    \end{macroc
% \end{ma

% \begin{macro}{\datebaha
%    The macro |\datebahasam| redefines the command |\today| to pro
%    Bahasa Malaysian da
% \changes{bahasa-1.0f}{1997/10/01}{Use \cs{edef} to define \cs{tod
% \changes{bahasa~1.0f}{1998/03/28}{use \cs{def} instead of \cs{e
%    to save mem
% \changes{bahasa-1.0g}{1999/03/12}{Februari should be spelle
%    Pebru
% \changes{bahasam-1.0k}{2008/01/27}{Februari restored to BM spelli
%    see Collins Kamus Dwibahasa 2
%    \begin{macroc
\@namedef{date\CurrentOptio
  \def\today{\number\day~\ifcase\mont
    Januari\or Februari\or Mac\or April\or Mei\or Ju
    Julai\or Ogos\or September\or Oktober\or November\or Disembe
    \space \number\ye
%    \end{macroc
% \end{ma


% \begin{macro}{\extrasbaha
% \begin{macro}{\noextrasbaha
%    The macro |\extrasbahasa| will perform all the extra definit
%    needed for the Bahasa language. The macro |\extrasbahasa| is
%    to cancel the actions of |\extrasbahasa|.  For the moment t
%    macros are empty but they are defined for compatibility with
%    other language definition fi

%    \begin{macroc
\@namedef{extras\CurrentOptio
\@namedef{noextras\CurrentOptio
%    \end{macroc
% \end{ma
% \end{ma

%  \begin{macro}{\bahasamhyphenm
%    The bahasam hyphenation patterns should be used
%    |\lefthyphenmin| set to~2 and |\righthyphenmin| set t
% \changes{bahasa-1.0e}{1996/08/07}{use \cs{bahasamhyphenmins} to s
%    the correct val
% \changes{bahasa-1.0h}{2000/09/22}{Now use \cs{providehyphenmins
%    provide a default va
%    \begin{macroc
\providehyphenmins{\CurrentOption}{\tw@\
%    \end{macroc
%  \end{ma

%    The macro |\ldf@finish| takes care of looking f
%    configuration file, setting the main language to be switche
%    at |\begin{document}| and resetting the category cod
%    \texttt{@} to its original va
% \changes{bahasa-1.0e}{1996/11/02}{Now use \cs{ldf@finish} to wrap
%    \begin{macroc
\ldf@finish{\CurrentOpt
%</c
%    \end{macroc

% \Fi

%% \CharacterT
%%  {Upper-case    \A\B\C\D\E\F\G\H\I\J\K\L\M\N\O\P\Q\R\S\T\U\V\W\X
%%   Lower-case    \a\b\c\d\e\f\g\h\i\j\k\l\m\n\o\p\q\r\s\t\u\v\w\x
%%   Digits        \0\1\2\3\4\5\6\7
%%   Exclamation   \!     Double quote  \"     Hash (number
%%   Dollar        \$     Percent       \%     Ampersand
%%   Acute accent  \'     Left paren    \(     Right paren
%%   Asterisk      \*     Plus          \+     Comma
%%   Minus         \-     Point         \.     Solidus
%%   Colon         \:     Semicolon     \;     Less than
%%   Equals        \=     Greater than  \>     Question mar
%%   Commercial at \@     Left bracket  \[     Backslash
%%   Right bracket \]     Circumflex    \^     Underscore
%%   Grave accent  \`     Left brace    \{     Vertical bar
%%   Right brace   \}     Tilde

\endi
}
\bbl@tempa{naustrian}{% \iffalse meta-comment
%
% Copyright 1989-2008 Johannes L. Braams and any individual authors
% listed elsewhere in this file.  All rights reserved.
% 
% This file is part of the Babel system.
% --------------------------------------
% 
% It may be distributed and/or modified under the
% conditions of the LaTeX Project Public License, either version 1.3
% of this license or (at your option) any later version.
% The latest version of this license is in
%   http://www.latex-project.org/lppl.txt
% and version 1.3 or later is part of all distributions of LaTeX
% version 2003/12/01 or later.
% 
% This work has the LPPL maintenance status "maintained".
% 
% The Current Maintainer of this work is Johannes Braams.
% 
% The list of all files belonging to the Babel system is
% given in the file `manifest.bbl. See also `legal.bbl' for additional
% information.
% 
% The list of derived (unpacked) files belonging to the distribution
% and covered by LPPL is defined by the unpacking scripts (with
% extension .ins) which are part of the distribution.
% \fi
% \CheckSum{266}
%
% \iffalse
%    Tell the \LaTeX\ system who we are and write an entry on the
%    transcript.
%<*dtx>
\ProvidesFile{ngermanb.dtx}
%</dtx>
%<code>\ProvidesLanguage{ngermanb}
%\fi
%\ProvidesFile{ngermanb.dtx}
        [2008/03/17 v2.6m new German support from the babel system]
%\iffalse
%% File `ngermanb.dtx'
%% Babel package for LaTeX version 2e
%% Copyright (C) 1989 - 2008
%%           by Johannes Braams, TeXniek
%
%% new Germanb Language Definition File
%% Copyright (C) 1989 - 2008
%%           by Bernd Raichle raichle at azu.Informatik.Uni-Stuttgart.de
%%              Johannes Braams, TeXniek,
%%              Walter Schmidt.
% This file is based on german.tex version 2.5e
%                       by Bernd Raichle, Hubert Partl et.al.
%
%% Please report errors to: J.L. Braams
%%                          babel at braams.xs4all.nl
%
%<*filedriver>
\documentclass{ltxdoc}
\font\manual=logo10 % font used for the METAFONT logo, etc.
\newcommand*\MF{{\manual META}\-{\manual FONT}}
\newcommand*\TeXhax{\TeX hax}
\newcommand*\babel{\textsf{babel}}
\newcommand*\langvar{$\langle \it lang \rangle$}
\newcommand*\note[1]{}
\newcommand*\Lopt[1]{\textsf{#1}}
\newcommand*\file[1]{\texttt{#1}}
\begin{document}
 \DocInput{ngermanb.dtx}
\end{document}
%</filedriver>
%\fi
% \GetFileInfo{ngermanb.dtx}
%
% \changes{ngermanb-2.6f}{1999/03/24}{Renamed from \file{germanb.ldf};
%          language names changed from \texttt{german} and \texttt{austrian}
%          to \texttt{ngerman} and \texttt{naustrian}.}
%
%  \section{The German language -- new orthography}
%
%    The file \file{\filename}\footnote{The file described in this
%    section has version number \fileversion\ and was last revised on
%    \filedate.}  defines all the language definition macros for the
%    German language with the `new orthography' introduced in
%    August 1998.  This includes also the Austrian dialect of this
%    language.
%  
%    As with the `traditional'  German orthography, 
%    the character |"| is made active, and 
%    the commands in  table~\ref{tab:german-quote} can be used, except
%    for |"ck| and |"ff| etc., which are no longer required.
%
%    The internal language names are |ngerman| and |naustrian|.
%
% \StopEventually{}
%
%    When this file was read through the option \Lopt{ngermanb} we make
%    it behave as if \Lopt{ngerman} was specified.
%    \begin{macrocode}
\def\bbl@tempa{ngermanb}
\ifx\CurrentOption\bbl@tempa
  \def\CurrentOption{ngerman}
\fi
%    \end{macrocode}
%
%    The macro |\LdfInit| takes care of preventing that this file is
%    loaded more than once, checking the category code of the
%    \texttt{@} sign, etc.
%    \begin{macrocode}
%<*code>
\LdfInit\CurrentOption{captions\CurrentOption}
%    \end{macrocode}
%
%    When this file is read as an option, i.e., by the |\usepackage|
%    command, \texttt{ngerman} will be an `unknown' language, so we
%    have to make it known.  So we check for the existence of
%    |\l@ngerman| to see whether we have to do something here.
%
%    \begin{macrocode}
\ifx\l@ngerman\@undefined
  \@nopatterns{ngerman}
  \adddialect\l@ngerman0
\fi
%    \end{macrocode}
%
%    For the Austrian version of these definitions we just add another
%    language. 
%    \begin{macrocode}
\adddialect\l@naustrian\l@ngerman
%    \end{macrocode}
%
%    The next step consists of defining commands to switch to (and
%    from) the German language.
%
%  \begin{macro}{\captionsngerman}
%  \begin{macro}{\captionsnaustrian}
%    Either the macro |\captionnsgerman| or the macro
%    |\captionsnaustrian| will define all strings used in the four
%    standard document classes provided with \LaTeX.
%
% \changes{ngermanb-2.6k}{2000/09/20}{Added \cs{glossaryname}}
%    \begin{macrocode}
\@namedef{captions\CurrentOption}{%
  \def\prefacename{Vorwort}%
  \def\refname{Literatur}%
  \def\abstractname{Zusammenfassung}%
  \def\bibname{Literaturverzeichnis}%
  \def\chaptername{Kapitel}%
  \def\appendixname{Anhang}%
  \def\contentsname{Inhaltsverzeichnis}%    % oder nur: Inhalt
  \def\listfigurename{Abbildungsverzeichnis}%
  \def\listtablename{Tabellenverzeichnis}%
  \def\indexname{Index}%
  \def\figurename{Abbildung}%
  \def\tablename{Tabelle}%                  % oder: Tafel
  \def\partname{Teil}%
  \def\enclname{Anlage(n)}%                 % oder: Beilage(n)
  \def\ccname{Verteiler}%                   % oder: Kopien an
  \def\headtoname{An}%
  \def\pagename{Seite}%
  \def\seename{siehe}%
  \def\alsoname{siehe auch}%
  \def\proofname{Beweis}%
  \def\glossaryname{Glossar}%
  }
%    \end{macrocode}
%  \end{macro}
%  \end{macro}
%
%  \begin{macro}{\datengerman}
%    The macro |\datengerman| redefines the command
%    |\today| to produce German dates.
%    \begin{macrocode}
\def\month@ngerman{\ifcase\month\or
  Januar\or Februar\or M\"arz\or April\or Mai\or Juni\or
  Juli\or August\or September\or Oktober\or November\or Dezember\fi}
\def\datengerman{\def\today{\number\day.~\month@ngerman
    \space\number\year}}
%    \end{macrocode}
%  \end{macro}
%
%  \begin{macro}{\dateanustrian}
%    The macro |\datenaustrian| redefines the command
%    |\today| to produce Austrian version of the German dates.
%    \begin{macrocode}
\def\datenaustrian{\def\today{\number\day.~\ifnum1=\month
  J\"anner\else \month@ngerman\fi \space\number\year}}
%    \end{macrocode}
%  \end{macro}
%
%  \begin{macro}{\extrasngerman}
%  \begin{macro}{\extrasnaustrian}
%  \begin{macro}{\noextrasngerman}
%  \begin{macro}{\noextrasnaustrian}
%    Either the macro |\extrasngerman| or the macros |\extrasnaustrian|
%    will perform all the extra definitions needed for the German
%    language. The macro |\noextrasngerman| is used to cancel the
%    actions of |\extrasngerman|. 
%
%    For German (as well as for Dutch) the \texttt{"} character is
%    made active. This is done once, later on its definition may vary.
%    \begin{macrocode}
\initiate@active@char{"}
\@namedef{extras\CurrentOption}{%
  \languageshorthands{ngerman}}
\expandafter\addto\csname extras\CurrentOption\endcsname{%
  \bbl@activate{"}}
%    \end{macrocode}
%    Don't forget to turn the shorthands off again.
% \changes{ngermanb-2.6j}{1999/12/16}{Deactivate shorthands ouside of
%    German}
%    \begin{macrocode}
\addto\noextrasngerman{\bbl@deactivate{"}}
%    \end{macrocode}
%
%
%    In order for \TeX\ to be able to hyphenate German words which
%    contain `\ss' (in the \texttt{OT1} position |^^Y|) we have to
%    give the character a nonzero |\lccode| (see Appendix H, the \TeX
%    book).
%    \begin{macrocode}
\expandafter\addto\csname extras\CurrentOption\endcsname{%
  \babel@savevariable{\lccode25}%
  \lccode25=25}
%    \end{macrocode}
%
%    The umlaut accent macro |\"| is changed to lower the umlaut dots.
%    The redefinition is done with the help of |\umlautlow|.
%    \begin{macrocode}
\expandafter\addto\csname extras\CurrentOption\endcsname{%
  \babel@save\"\umlautlow}
\@namedef{noextras\CurrentOption}{\umlauthigh}
%    \end{macrocode}
%    The current 
%    version of the `new' German hyphenation patterns (\file{dehyphn.tex}
%    is to be used with |\lefthyphenmin| and |\righthyphenmin| set to~2. 
% \changes{ngermanb-2.6k}{2000/09/22}{Now use \cs{providehyphenmins} to
%    provide a default value}
%    \begin{macrocode}
\providehyphenmins{\CurrentOption}{\tw@\tw@}
%    \end{macrocode}
%    For German texts we need to make sure that |\frenchspacing| is
%    turned on.
% \changes{ngermanb-2.6m}{2001/01/26}{Turn frenchspacing on, as in
%    \texttt{german.sty}}
%    \begin{macrocode}
\expandafter\addto\csname extras\CurrentOption\endcsname{%
  \bbl@frenchspacing}
\expandafter\addto\csname noextras\CurrentOption\endcsname{%
  \bbl@nonfrenchspacing}
%    \end{macrocode}
%  \end{macro}
%  \end{macro}
%  \end{macro}
%  \end{macro}
%
%    The code above is necessary because we need an extra active
%    character. This character is then used as indicated in
%    table~\ref{tab:german-quote}.
%
%    To be able to define the function of |"|, we first define a
%    couple of `support' macros.
%
%
%  \begin{macro}{\dq}
%    We save the original double quote character in |\dq| to keep
%    it available, the math accent |\"| can now be typed as |"|.
%    \begin{macrocode}
\begingroup \catcode`\"12
\def\x{\endgroup
  \def\@SS{\mathchar"7019 }
  \def\dq{"}}
\x
%    \end{macrocode}
%  \end{macro}
%
%    Now we can define the doublequote macros: the umlauts,
%    \begin{macrocode}
\declare@shorthand{ngerman}{"a}{\textormath{\"{a}\allowhyphens}{\ddot a}}
\declare@shorthand{ngerman}{"o}{\textormath{\"{o}\allowhyphens}{\ddot o}}
\declare@shorthand{ngerman}{"u}{\textormath{\"{u}\allowhyphens}{\ddot u}}
\declare@shorthand{ngerman}{"A}{\textormath{\"{A}\allowhyphens}{\ddot A}}
\declare@shorthand{ngerman}{"O}{\textormath{\"{O}\allowhyphens}{\ddot O}}
\declare@shorthand{ngerman}{"U}{\textormath{\"{U}\allowhyphens}{\ddot U}}
%    \end{macrocode}
%    tremas,
%    \begin{macrocode}
\declare@shorthand{ngerman}{"e}{\textormath{\"{e}}{\ddot e}}
\declare@shorthand{ngerman}{"E}{\textormath{\"{E}}{\ddot E}}
\declare@shorthand{ngerman}{"i}{\textormath{\"{\i}}%
                              {\ddot\imath}}
\declare@shorthand{ngerman}{"I}{\textormath{\"{I}}{\ddot I}}
%    \end{macrocode}
%    german es-zet (sharp s),
%    \begin{macrocode}
\declare@shorthand{ngerman}{"s}{\textormath{\ss}{\@SS{}}}
\declare@shorthand{ngerman}{"S}{\SS}
\declare@shorthand{ngerman}{"z}{\textormath{\ss}{\@SS{}}}
\declare@shorthand{ngerman}{"Z}{SZ}
%    \end{macrocode}
%    german and french quotes,
%    \begin{macrocode}
\declare@shorthand{ngerman}{"`}{\glqq}
\declare@shorthand{ngerman}{"'}{\grqq}
\declare@shorthand{ngerman}{"<}{\flqq}
\declare@shorthand{ngerman}{">}{\frqq}
%    \end{macrocode}
%    and some additional commands:
%    \begin{macrocode}
\declare@shorthand{ngerman}{"-}{\nobreak\-\bbl@allowhyphens}
\declare@shorthand{ngerman}{"|}{%
  \textormath{\penalty\@M\discretionary{-}{}{\kern.03em}%
              \allowhyphens}{}}
\declare@shorthand{ngerman}{""}{\hskip\z@skip}
\declare@shorthand{ngerman}{"~}{\textormath{\leavevmode\hbox{-}}{-}}
\declare@shorthand{ngerman}{"=}{\penalty\@M-\hskip\z@skip}
%    \end{macrocode}
%
%  \begin{macro}{\mdqon}
%  \begin{macro}{\mdqoff}
%    All that's left to do now is to  define a couple of commands
%    for reasons of compatibility with \file{german.sty}.
%    \begin{macrocode}
\def\mdqon{\shorthandon{"}}
\def\mdqoff{\shorthandoff{"}}
%    \end{macrocode}
%  \end{macro}
%  \end{macro}
%
%    The macro |\ldf@finish| takes care of looking for a
%    configuration file, setting the main language to be switched on
%    at |\begin{document}| and resetting the category code of
%    \texttt{@} to its original value.
%    \begin{macrocode}
\ldf@finish\CurrentOption
%</code>
%    \end{macrocode}
%
% \Finale
%%
%% \CharacterTable
%%  {Upper-case    \A\B\C\D\E\F\G\H\I\J\K\L\M\N\O\P\Q\R\S\T\U\V\W\X\Y\Z
%%   Lower-case    \a\b\c\d\e\f\g\h\i\j\k\l\m\n\o\p\q\r\s\t\u\v\w\x\y\z
%%   Digits        \0\1\2\3\4\5\6\7\8\9
%%   Exclamation   \!     Double quote  \"     Hash (number) \#
%%   Dollar        \$     Percent       \%     Ampersand     \&
%%   Acute accent  \'     Left paren    \(     Right paren   \)
%%   Asterisk      \*     Plus          \+     Comma         \,
%%   Minus         \-     Point         \.     Solidus       \/
%%   Colon         \:     Semicolon     \;     Less than     \<
%%   Equals        \=     Greater than  \>     Question mark \?
%%   Commercial at \@     Left bracket  \[     Backslash     \\
%%   Right bracket \]     Circumflex    \^     Underscore    \_
%%   Grave accent  \`     Left brace    \{     Vertical bar  \|
%%   Right brace   \}     Tilde         \~}
%%
\endinput
}
\bbl@tempa{newzealand}{%%
%% This file will generate fast loadable files and documentation
%% driver files from the doc files in this package when run through
%% LaTeX or TeX.
%%
%% Copyright 1989-2005 Johannes L. Braams and any individual authors
%% listed elsewhere in this file.  All rights reserved.
%% 
%% This file is part of the Babel system.
%% --------------------------------------
%% 
%% It may be distributed and/or modified under the
%% conditions of the LaTeX Project Public License, either version 1.3
%% of this license or (at your option) any later version.
%% The latest version of this license is in
%%   http://www.latex-project.org/lppl.txt
%% and version 1.3 or later is part of all distributions of LaTeX
%% version 2003/12/01 or later.
%% 
%% This work has the LPPL maintenance status "maintained".
%% 
%% The Current Maintainer of this work is Johannes Braams.
%% 
%% The list of all files belonging to the LaTeX base distribution is
%% given in the file `manifest.bbl. See also `legal.bbl' for additional
%% information.
%% 
%% The list of derived (unpacked) files belonging to the distribution
%% and covered by LPPL is defined by the unpacking scripts (with
%% extension .ins) which are part of the distribution.
%%
%% --------------- start of docstrip commands ------------------
%%
\def\filedate{1999/04/11}
\def\batchfile{english.ins}
\input docstrip.tex

{\ifx\generate\undefined
\Msg{**********************************************}
\Msg{*}
\Msg{* This installation requires docstrip}
\Msg{* version 2.3c or later.}
\Msg{*}
\Msg{* An older version of docstrip has been input}
\Msg{*}
\Msg{**********************************************}
\errhelp{Move or rename old docstrip.tex.}
\errmessage{Old docstrip in input path}
\batchmode
\csname @@end\endcsname
\fi}

\declarepreamble\mainpreamble
This is a generated file.

Copyright 1989-2005 Johannes L. Braams and any individual authors
listed elsewhere in this file.  All rights reserved.

This file was generated from file(s) of the Babel system.
---------------------------------------------------------

It may be distributed and/or modified under the
conditions of the LaTeX Project Public License, either version 1.3
of this license or (at your option) any later version.
The latest version of this license is in
  http://www.latex-project.org/lppl.txt
and version 1.3 or later is part of all distributions of LaTeX
version 2003/12/01 or later.

This work has the LPPL maintenance status "maintained".

The Current Maintainer of this work is Johannes Braams.

This file may only be distributed together with a copy of the Babel
system. You may however distribute the Babel system without
such generated files.

The list of all files belonging to the Babel distribution is
given in the file `manifest.bbl'. See also `legal.bbl for additional
information.

The list of derived (unpacked) files belonging to the distribution
and covered by LPPL is defined by the unpacking scripts (with
extension .ins) which are part of the distribution.
\endpreamble

\declarepreamble\fdpreamble
This is a generated file.

Copyright 1989-2005 Johannes L. Braams and any individual authors
listed elsewhere in this file.  All rights reserved.

This file was generated from file(s) of the Babel system.
---------------------------------------------------------

It may be distributed and/or modified under the
conditions of the LaTeX Project Public License, either version 1.3
of this license or (at your option) any later version.
The latest version of this license is in
  http://www.latex-project.org/lppl.txt
and version 1.3 or later is part of all distributions of LaTeX
version 2003/12/01 or later.

This work has the LPPL maintenance status "maintained".

The Current Maintainer of this work is Johannes Braams.

This file may only be distributed together with a copy of the Babel
system. You may however distribute the Babel system without
such generated files.

The list of all files belonging to the Babel distribution is
given in the file `manifest.bbl'. See also `legal.bbl for additional
information.

In particular, permission is granted to customize the declarations in
this file to serve the needs of your installation.

However, NO PERMISSION is granted to distribute a modified version
of this file under its original name.

\endpreamble

\keepsilent

\usedir{tex/generic/babel} 

\usepreamble\mainpreamble
\generate{\file{english.ldf}{\from{english.dtx}{code}}
          }
\usepreamble\fdpreamble

\ifToplevel{
\Msg{***********************************************************}
\Msg{*}
\Msg{* To finish the installation you have to move the following}
\Msg{* files into a directory searched by TeX:}
\Msg{*}
\Msg{* \space\space All *.def, *.fd, *.ldf, *.sty}
\Msg{*}
\Msg{* To produce the documentation run the files ending with}
\Msg{* '.dtx' and `.fdd' through LaTeX.}
\Msg{*}
\Msg{* Happy TeXing}
\Msg{***********************************************************}
}
 
\endinput
}
\bbl@tempa{ngerman}{% \iffalse meta-comment
%
% Copyright 1989-2008 Johannes L. Braams and any individual authors
% listed elsewhere in this file.  All rights reserved.
% 
% This file is part of the Babel system.
% --------------------------------------
% 
% It may be distributed and/or modified under the
% conditions of the LaTeX Project Public License, either version 1.3
% of this license or (at your option) any later version.
% The latest version of this license is in
%   http://www.latex-project.org/lppl.txt
% and version 1.3 or later is part of all distributions of LaTeX
% version 2003/12/01 or later.
% 
% This work has the LPPL maintenance status "maintained".
% 
% The Current Maintainer of this work is Johannes Braams.
% 
% The list of all files belonging to the Babel system is
% given in the file `manifest.bbl. See also `legal.bbl' for additional
% information.
% 
% The list of derived (unpacked) files belonging to the distribution
% and covered by LPPL is defined by the unpacking scripts (with
% extension .ins) which are part of the distribution.
% \fi
% \CheckSum{266}
%
% \iffalse
%    Tell the \LaTeX\ system who we are and write an entry on the
%    transcript.
%<*dtx>
\ProvidesFile{ngermanb.dtx}
%</dtx>
%<code>\ProvidesLanguage{ngermanb}
%\fi
%\ProvidesFile{ngermanb.dtx}
        [2008/03/17 v2.6m new German support from the babel system]
%\iffalse
%% File `ngermanb.dtx'
%% Babel package for LaTeX version 2e
%% Copyright (C) 1989 - 2008
%%           by Johannes Braams, TeXniek
%
%% new Germanb Language Definition File
%% Copyright (C) 1989 - 2008
%%           by Bernd Raichle raichle at azu.Informatik.Uni-Stuttgart.de
%%              Johannes Braams, TeXniek,
%%              Walter Schmidt.
% This file is based on german.tex version 2.5e
%                       by Bernd Raichle, Hubert Partl et.al.
%
%% Please report errors to: J.L. Braams
%%                          babel at braams.xs4all.nl
%
%<*filedriver>
\documentclass{ltxdoc}
\font\manual=logo10 % font used for the METAFONT logo, etc.
\newcommand*\MF{{\manual META}\-{\manual FONT}}
\newcommand*\TeXhax{\TeX hax}
\newcommand*\babel{\textsf{babel}}
\newcommand*\langvar{$\langle \it lang \rangle$}
\newcommand*\note[1]{}
\newcommand*\Lopt[1]{\textsf{#1}}
\newcommand*\file[1]{\texttt{#1}}
\begin{document}
 \DocInput{ngermanb.dtx}
\end{document}
%</filedriver>
%\fi
% \GetFileInfo{ngermanb.dtx}
%
% \changes{ngermanb-2.6f}{1999/03/24}{Renamed from \file{germanb.ldf};
%          language names changed from \texttt{german} and \texttt{austrian}
%          to \texttt{ngerman} and \texttt{naustrian}.}
%
%  \section{The German language -- new orthography}
%
%    The file \file{\filename}\footnote{The file described in this
%    section has version number \fileversion\ and was last revised on
%    \filedate.}  defines all the language definition macros for the
%    German language with the `new orthography' introduced in
%    August 1998.  This includes also the Austrian dialect of this
%    language.
%  
%    As with the `traditional'  German orthography, 
%    the character |"| is made active, and 
%    the commands in  table~\ref{tab:german-quote} can be used, except
%    for |"ck| and |"ff| etc., which are no longer required.
%
%    The internal language names are |ngerman| and |naustrian|.
%
% \StopEventually{}
%
%    When this file was read through the option \Lopt{ngermanb} we make
%    it behave as if \Lopt{ngerman} was specified.
%    \begin{macrocode}
\def\bbl@tempa{ngermanb}
\ifx\CurrentOption\bbl@tempa
  \def\CurrentOption{ngerman}
\fi
%    \end{macrocode}
%
%    The macro |\LdfInit| takes care of preventing that this file is
%    loaded more than once, checking the category code of the
%    \texttt{@} sign, etc.
%    \begin{macrocode}
%<*code>
\LdfInit\CurrentOption{captions\CurrentOption}
%    \end{macrocode}
%
%    When this file is read as an option, i.e., by the |\usepackage|
%    command, \texttt{ngerman} will be an `unknown' language, so we
%    have to make it known.  So we check for the existence of
%    |\l@ngerman| to see whether we have to do something here.
%
%    \begin{macrocode}
\ifx\l@ngerman\@undefined
  \@nopatterns{ngerman}
  \adddialect\l@ngerman0
\fi
%    \end{macrocode}
%
%    For the Austrian version of these definitions we just add another
%    language. 
%    \begin{macrocode}
\adddialect\l@naustrian\l@ngerman
%    \end{macrocode}
%
%    The next step consists of defining commands to switch to (and
%    from) the German language.
%
%  \begin{macro}{\captionsngerman}
%  \begin{macro}{\captionsnaustrian}
%    Either the macro |\captionnsgerman| or the macro
%    |\captionsnaustrian| will define all strings used in the four
%    standard document classes provided with \LaTeX.
%
% \changes{ngermanb-2.6k}{2000/09/20}{Added \cs{glossaryname}}
%    \begin{macrocode}
\@namedef{captions\CurrentOption}{%
  \def\prefacename{Vorwort}%
  \def\refname{Literatur}%
  \def\abstractname{Zusammenfassung}%
  \def\bibname{Literaturverzeichnis}%
  \def\chaptername{Kapitel}%
  \def\appendixname{Anhang}%
  \def\contentsname{Inhaltsverzeichnis}%    % oder nur: Inhalt
  \def\listfigurename{Abbildungsverzeichnis}%
  \def\listtablename{Tabellenverzeichnis}%
  \def\indexname{Index}%
  \def\figurename{Abbildung}%
  \def\tablename{Tabelle}%                  % oder: Tafel
  \def\partname{Teil}%
  \def\enclname{Anlage(n)}%                 % oder: Beilage(n)
  \def\ccname{Verteiler}%                   % oder: Kopien an
  \def\headtoname{An}%
  \def\pagename{Seite}%
  \def\seename{siehe}%
  \def\alsoname{siehe auch}%
  \def\proofname{Beweis}%
  \def\glossaryname{Glossar}%
  }
%    \end{macrocode}
%  \end{macro}
%  \end{macro}
%
%  \begin{macro}{\datengerman}
%    The macro |\datengerman| redefines the command
%    |\today| to produce German dates.
%    \begin{macrocode}
\def\month@ngerman{\ifcase\month\or
  Januar\or Februar\or M\"arz\or April\or Mai\or Juni\or
  Juli\or August\or September\or Oktober\or November\or Dezember\fi}
\def\datengerman{\def\today{\number\day.~\month@ngerman
    \space\number\year}}
%    \end{macrocode}
%  \end{macro}
%
%  \begin{macro}{\dateanustrian}
%    The macro |\datenaustrian| redefines the command
%    |\today| to produce Austrian version of the German dates.
%    \begin{macrocode}
\def\datenaustrian{\def\today{\number\day.~\ifnum1=\month
  J\"anner\else \month@ngerman\fi \space\number\year}}
%    \end{macrocode}
%  \end{macro}
%
%  \begin{macro}{\extrasngerman}
%  \begin{macro}{\extrasnaustrian}
%  \begin{macro}{\noextrasngerman}
%  \begin{macro}{\noextrasnaustrian}
%    Either the macro |\extrasngerman| or the macros |\extrasnaustrian|
%    will perform all the extra definitions needed for the German
%    language. The macro |\noextrasngerman| is used to cancel the
%    actions of |\extrasngerman|. 
%
%    For German (as well as for Dutch) the \texttt{"} character is
%    made active. This is done once, later on its definition may vary.
%    \begin{macrocode}
\initiate@active@char{"}
\@namedef{extras\CurrentOption}{%
  \languageshorthands{ngerman}}
\expandafter\addto\csname extras\CurrentOption\endcsname{%
  \bbl@activate{"}}
%    \end{macrocode}
%    Don't forget to turn the shorthands off again.
% \changes{ngermanb-2.6j}{1999/12/16}{Deactivate shorthands ouside of
%    German}
%    \begin{macrocode}
\addto\noextrasngerman{\bbl@deactivate{"}}
%    \end{macrocode}
%
%
%    In order for \TeX\ to be able to hyphenate German words which
%    contain `\ss' (in the \texttt{OT1} position |^^Y|) we have to
%    give the character a nonzero |\lccode| (see Appendix H, the \TeX
%    book).
%    \begin{macrocode}
\expandafter\addto\csname extras\CurrentOption\endcsname{%
  \babel@savevariable{\lccode25}%
  \lccode25=25}
%    \end{macrocode}
%
%    The umlaut accent macro |\"| is changed to lower the umlaut dots.
%    The redefinition is done with the help of |\umlautlow|.
%    \begin{macrocode}
\expandafter\addto\csname extras\CurrentOption\endcsname{%
  \babel@save\"\umlautlow}
\@namedef{noextras\CurrentOption}{\umlauthigh}
%    \end{macrocode}
%    The current 
%    version of the `new' German hyphenation patterns (\file{dehyphn.tex}
%    is to be used with |\lefthyphenmin| and |\righthyphenmin| set to~2. 
% \changes{ngermanb-2.6k}{2000/09/22}{Now use \cs{providehyphenmins} to
%    provide a default value}
%    \begin{macrocode}
\providehyphenmins{\CurrentOption}{\tw@\tw@}
%    \end{macrocode}
%    For German texts we need to make sure that |\frenchspacing| is
%    turned on.
% \changes{ngermanb-2.6m}{2001/01/26}{Turn frenchspacing on, as in
%    \texttt{german.sty}}
%    \begin{macrocode}
\expandafter\addto\csname extras\CurrentOption\endcsname{%
  \bbl@frenchspacing}
\expandafter\addto\csname noextras\CurrentOption\endcsname{%
  \bbl@nonfrenchspacing}
%    \end{macrocode}
%  \end{macro}
%  \end{macro}
%  \end{macro}
%  \end{macro}
%
%    The code above is necessary because we need an extra active
%    character. This character is then used as indicated in
%    table~\ref{tab:german-quote}.
%
%    To be able to define the function of |"|, we first define a
%    couple of `support' macros.
%
%
%  \begin{macro}{\dq}
%    We save the original double quote character in |\dq| to keep
%    it available, the math accent |\"| can now be typed as |"|.
%    \begin{macrocode}
\begingroup \catcode`\"12
\def\x{\endgroup
  \def\@SS{\mathchar"7019 }
  \def\dq{"}}
\x
%    \end{macrocode}
%  \end{macro}
%
%    Now we can define the doublequote macros: the umlauts,
%    \begin{macrocode}
\declare@shorthand{ngerman}{"a}{\textormath{\"{a}\allowhyphens}{\ddot a}}
\declare@shorthand{ngerman}{"o}{\textormath{\"{o}\allowhyphens}{\ddot o}}
\declare@shorthand{ngerman}{"u}{\textormath{\"{u}\allowhyphens}{\ddot u}}
\declare@shorthand{ngerman}{"A}{\textormath{\"{A}\allowhyphens}{\ddot A}}
\declare@shorthand{ngerman}{"O}{\textormath{\"{O}\allowhyphens}{\ddot O}}
\declare@shorthand{ngerman}{"U}{\textormath{\"{U}\allowhyphens}{\ddot U}}
%    \end{macrocode}
%    tremas,
%    \begin{macrocode}
\declare@shorthand{ngerman}{"e}{\textormath{\"{e}}{\ddot e}}
\declare@shorthand{ngerman}{"E}{\textormath{\"{E}}{\ddot E}}
\declare@shorthand{ngerman}{"i}{\textormath{\"{\i}}%
                              {\ddot\imath}}
\declare@shorthand{ngerman}{"I}{\textormath{\"{I}}{\ddot I}}
%    \end{macrocode}
%    german es-zet (sharp s),
%    \begin{macrocode}
\declare@shorthand{ngerman}{"s}{\textormath{\ss}{\@SS{}}}
\declare@shorthand{ngerman}{"S}{\SS}
\declare@shorthand{ngerman}{"z}{\textormath{\ss}{\@SS{}}}
\declare@shorthand{ngerman}{"Z}{SZ}
%    \end{macrocode}
%    german and french quotes,
%    \begin{macrocode}
\declare@shorthand{ngerman}{"`}{\glqq}
\declare@shorthand{ngerman}{"'}{\grqq}
\declare@shorthand{ngerman}{"<}{\flqq}
\declare@shorthand{ngerman}{">}{\frqq}
%    \end{macrocode}
%    and some additional commands:
%    \begin{macrocode}
\declare@shorthand{ngerman}{"-}{\nobreak\-\bbl@allowhyphens}
\declare@shorthand{ngerman}{"|}{%
  \textormath{\penalty\@M\discretionary{-}{}{\kern.03em}%
              \allowhyphens}{}}
\declare@shorthand{ngerman}{""}{\hskip\z@skip}
\declare@shorthand{ngerman}{"~}{\textormath{\leavevmode\hbox{-}}{-}}
\declare@shorthand{ngerman}{"=}{\penalty\@M-\hskip\z@skip}
%    \end{macrocode}
%
%  \begin{macro}{\mdqon}
%  \begin{macro}{\mdqoff}
%    All that's left to do now is to  define a couple of commands
%    for reasons of compatibility with \file{german.sty}.
%    \begin{macrocode}
\def\mdqon{\shorthandon{"}}
\def\mdqoff{\shorthandoff{"}}
%    \end{macrocode}
%  \end{macro}
%  \end{macro}
%
%    The macro |\ldf@finish| takes care of looking for a
%    configuration file, setting the main language to be switched on
%    at |\begin{document}| and resetting the category code of
%    \texttt{@} to its original value.
%    \begin{macrocode}
\ldf@finish\CurrentOption
%</code>
%    \end{macrocode}
%
% \Finale
%%
%% \CharacterTable
%%  {Upper-case    \A\B\C\D\E\F\G\H\I\J\K\L\M\N\O\P\Q\R\S\T\U\V\W\X\Y\Z
%%   Lower-case    \a\b\c\d\e\f\g\h\i\j\k\l\m\n\o\p\q\r\s\t\u\v\w\x\y\z
%%   Digits        \0\1\2\3\4\5\6\7\8\9
%%   Exclamation   \!     Double quote  \"     Hash (number) \#
%%   Dollar        \$     Percent       \%     Ampersand     \&
%%   Acute accent  \'     Left paren    \(     Right paren   \)
%%   Asterisk      \*     Plus          \+     Comma         \,
%%   Minus         \-     Point         \.     Solidus       \/
%%   Colon         \:     Semicolon     \;     Less than     \<
%%   Equals        \=     Greater than  \>     Question mark \?
%%   Commercial at \@     Left bracket  \[     Backslash     \\
%%   Right bracket \]     Circumflex    \^     Underscore    \_
%%   Grave accent  \`     Left brace    \{     Vertical bar  \|
%%   Right brace   \}     Tilde         \~}
%%
\endinput
}
\bbl@tempa{nynorsk}{% \iffalse meta-comment
%
% Copyright 1989-2005 Johannes L. Braams and any individual authors
% listed elsewhere in this file.  All rights reserved.
% 
% This file is part of the Babel system.
% --------------------------------------
% 
% It may be distributed and/or modified under the
% conditions of the LaTeX Project Public License, either version 1.3
% of this license or (at your option) any later version.
% The latest version of this license is in
%   http://www.latex-project.org/lppl.txt
% and version 1.3 or later is part of all distributions of LaTeX
% version 2003/12/01 or later.
% 
% This work has the LPPL maintenance status "maintained".
% 
% The Current Maintainer of this work is Johannes Braams.
% 
% The list of all files belonging to the Babel system is
% given in the file `manifest.bbl. See also `legal.bbl' for additional
% information.
% 
% The list of derived (unpacked) files belonging to the distribution
% and covered by LPPL is defined by the unpacking scripts (with
% extension .ins) which are part of the distribution.
% \fi
%\CheckSum{305}
% \iffalse
%    Tell the \LaTeX\ system who we are and write an entry on the
%    transcript.
%<*dtx>
\ProvidesFile{norsk.dtx}
%</dtx>
%<code>\ProvidesLanguage{norsk}
%\fi
%\ProvidesFile{norsk.dtx}
        [2012/08/06 v2.0i Norsk support from the babel system]
%\iffalse
%%File `norsk.dtx'
%% Babel package for LaTeX version 2e
%% Copyright (C) 1989 - 2005
%%           by Johannes Braams, TeXniek
%
%% Please report errors to: J.L. Braams
%%                          babel at braams.cistron.nl
%
%    This file is part of the babel system, it provides the source
%    code for the Norwegian language definition file.  Contributions
%    were made by Haavard Helstrup (HAAVARD@CERNVM) and Alv Kjetil
%    Holme (HOLMEA@CERNVM); the `nynorsk' variant has been supplied by
%    Per Steinar Iversen (iversen@vxcern.cern.ch) and Terje Engeset
%    Petterst (TERJEEP@VSFYS1.FI.UIB.NO)
%
%    Rune Kleveland (runekl at math.uio.no) added the shorthand
%    definitions 
%<*filedriver>
\documentclass{ltxdoc}
\newcommand*\TeXhax{\TeX hax}
\newcommand*\babel{\textsf{babel}}
\newcommand*\langvar{$\langle \it lang \rangle$}
\newcommand*\note[1]{}
\newcommand*\Lopt[1]{\textsf{#1}}
\newcommand*\file[1]{\texttt{#1}}
\begin{document}
 \DocInput{norsk.dtx}
\end{document}
%</filedriver>
%\fi
% \GetFileInfo{norsk.dtx}
%
% \changes{norsk-1.0a}{1991/07/15}{Renamed \file{babel.sty} in
%    \file{babel.com}}
% \changes{norsk-1.1a}{1992/02/16}{Brought up-to-date with babel 3.2a}
% \changes{norsk-1.1c}{1993/11/11}{Added a couple of translations
%    (from Per Norman Oma, TeX@itk.unit.no)}
% \changes{norsk-1.2a}{1994/02/27}{Update for \LaTeXe}
% \changes{norsk-1.2d}{1994/06/26}{Removed the use of \cs{filedate}
%    and moved identification after the loading of \file{babel.def}}
% \changes{norsk-1.2h}{1996/07/12}{Replaced \cs{undefined} with
%    \cs{@undefined} and \cs{empty} with \cs{@empty} for consistency
%    with \LaTeX} 
% \changes{norsk-1.2h}{1996/10/10}{Moved the definition of
%    \cs{atcatcode} right to the beginning.}
%
%
%  \section{The Norwegian language}
%
%    The file \file{\filename}\footnote{The file described in this
%    section has version number \fileversion\ and was last revised on
%    \filedate.  Contributions were made by Haavard Helstrup
%    (\texttt{HAAVARD@CERNVM)} and Alv Kjetil Holme
%    (\texttt{HOLMEA@CERNVM}); the `nynorsk' variant has been supplied
%    by Per Steinar Iversen \texttt{iversen@vxcern.cern.ch}) and Terje
%    Engeset Petterst (\texttt{TERJEEP@VSFYS1.FI.UIB.NO)}; the
%    shorthand definitions were provided by Rune Kleveland
%    (\texttt{runekl@math.uio.no}).} defines all the language definition
%    macros for the Norwegian language as well as for an alternative
%    variant `nynorsk' of this language. 
%
%    For this language the character |"| is made active. In
%    table~\ref{tab:norsk-quote} an overview is given of its purpose.
%    \begin{table}[htb]
%     \begin{center}
%     \begin{tabular}{lp{.7\textwidth}}
%      |"ff|& for |ff| to be hyphenated as |ff-f|,
%             this is also implemented for b, d, f, g, l, m, n,
%             p, r, s, and t. (|o"ppussing|)                        \\
%      |"ee|& Hyphenate |"ee| as |\'e-e|. (|komit"een|)             \\
%      |"-| & an explicit hyphen sign, allowing hyphenation in the
%             composing words. Use this for compound words when the
%             hyphenation patterns fail to hyphenate
%             properly. (|alpin"-anlegg|)                           \\
%      \verb="|= & Like |"-|, but inserts 0.03em space.  Use it if
%             the compound point is spanned by a ligature.
%             (\verb=hoff"|intriger=)                               \\
%      |""| & Like |"-|, but producing no hyphen sign.
%             (|i""g\aa{}r|)                                        \\
%      |"~| & Like |-|, but allows no hyphenation at all. (|E"~cup|)\\
%      |"=| & Like |-|, but allowing hyphenation in the composing
%             words. (|marksistisk"=leninistisk|)                   \\
%      |"<| & for French left double quotes (similar to $<<$).      \\
%      |">| & for French right double quotes (similar to $>>$).     \\
%     \end{tabular}
%     \caption{The extra definitions made
%              by \file{norsk.sty}}\label{tab:norsk-quote}
%     \end{center}
%    \end{table}
% \changes{norsk-2.0a}{1998/06/24}{Describe the use of double quote as
%    active character}
%
%    Rune Kleveland distributes a Norwegian dictionary for ispell
%    (570000 words). It can be found at
%    |http://www.uio.no/~runekl/dictionary.html|. 
%
%    This dictionary supports the spellings |spi"sslede| for
%    `spisslede' (hyphenated spiss-slede) and other such words, and
%    also suggest the spelling |spi"sslede| for `spisslede' and
%    `spissslede'.
%
% \StopEventually{}
%
%    The macro |\LdfInit| takes care of preventing that this file is
%    loaded more than once, checking the category code of the
%    \texttt{@} sign, etc.
% \changes{norsk-1.2h}{1996/11/03}{Now use \cs{LdfInit} to perform
%    initial checks} 
%    \begin{macrocode}
%<*code>
\LdfInit\CurrentOption{captions\CurrentOption}
%    \end{macrocode}
%
%    When this file is read as an option, i.e. by the |\usepackage|
%    command, \texttt{norsk} will be an `unknown' language in which
%    case we have to make it known.  So we check for the existence of
%    |\l@norsk| to see whether we have to do something here.
%
% \changes{norsk-1.0c}{1991/10/29}{Removed use of \cs{@ifundefined}}
% \changes{norsk-1.1a}{1992/02/16}{Added a warning when no hyphenation
%    patterns were loaded.}
% \changes{norsk-1.2d}{1994/06/26}{Now use \cs{@nopatterns} to produce
%    the warning}
%    \begin{macrocode}
\ifx\l@norsk\@undefined
    \@nopatterns{Norsk}
    \adddialect\l@norsk0\fi
%    \end{macrocode}
%
%  \begin{macro}{\norskhyphenmins}
%     Some sets of Norwegian hyphenation patterns can be used with
%     |\lefthyphenmin| set to~1 and |\righthyphenmin| set to~2, but
%     the most common set |nohyph.tex| can't.  So we use
%     |\lefthyphenmin=2| by default.
% \changes{norsk-1.2f}{1995/07/02}{Added setting of hyphenmin
%    parameters}
% \changes{norsk-2.0a}{1998/06/24}{Changed setting of hyphenmin
%    parameters to 2~2} 
% \changes{norsk-2.0e}{2000/09/22}{Now use \cs{providehyphenmins} to
%    provide a default value}
%    \begin{macrocode}
\providehyphenmins{\CurrentOption}{\tw@\tw@}
%    \end{macrocode}
%  \end{macro}
%
%    Now we have to decide which version of the captions should be
%    made available. This can be done by checking the contents of
%    |\CurrentOption|. 
%    \begin{macrocode}
\def\bbl@tempa{norsk}
\ifx\CurrentOption\bbl@tempa
%    \end{macrocode}
%
%    The next step consists of defining commands to switch to (and
%    from) the Norwegian language.
%
% \begin{macro}{\captionsnorsk}
%    The macro |\captionsnorsk| defines all strings used
%    in the four standard documentclasses provided with \LaTeX.
% \changes{norsk-1.1a}{1992/02/16}{Added \cs{seename}, \cs{alsoname} and
%    \cs{prefacename}}
% \changes{norsk-1.1b}{1993/07/15}{\cs{headpagename} should be
%    \cs{pagename}}
% \changes{norsk-1.2f}{1995/07/02}{Added \cs{proofname} for
%    AMS-\LaTeX}
% \changes{norsk-1.2g}{1996/04/01}{Replaced `Proof' by its
%    translation} 
% \changes{norsk-2.0e}{2000/09/20}{Added \cs{glossaryname}}
% \changes{norsk-2.0g}{1996/04/01}{Replaced `Glossary' by its
%    translation} 
%    \begin{macrocode}
  \def\captionsnorsk{%
    \def\prefacename{Forord}%
    \def\refname{Referanser}%
    \def\abstractname{Sammendrag}%
    \def\bibname{Bibliografi}%     or Litteraturoversikt
    %                              or Litteratur or Referanser
    \def\chaptername{Kapittel}%
    \def\appendixname{Tillegg}%    or Appendiks
    \def\contentsname{Innhold}%
    \def\listfigurename{Figurer}%  or Figurliste
    \def\listtablename{Tabeller}%  or Tabelliste
    \def\indexname{Register}%
    \def\figurename{Figur}%
    \def\tablename{Tabell}%
    \def\partname{Del}%
    \def\enclname{Vedlegg}%
    \def\ccname{Kopi sendt}%
    \def\headtoname{Til}% in letter
    \def\pagename{Side}%
    \def\seename{Se}%
    \def\alsoname{Se ogs\aa{}}%
    \def\proofname{Bevis}%
    \def\glossaryname{Ordliste}%
    }
\else
%    \end{macrocode}
% \end{macro}
%
%    For the `nynorsk' version of these definitions we just add a
%    ``dialect''.
%    \begin{macrocode}
  \adddialect\l@nynorsk\l@norsk
%    \end{macrocode}
%
% \begin{macro}{\captionsnynorsk}
%    The macro |\captionsnynorsk| defines all strings used in the four
%    standard documentclasses provided with \LaTeX, but using a
%    different spelling than in the command |\captionsnorsk|.
% \changes{norsk-1.1a}{1992/02/16}{Added \cs{seename}, \cs{alsoname} and
%    \cs{prefacename}}
% \changes{norsk-1.1b}{1993/07/15}{\cs{headpagename} should be
%    \cs{pagename}}
% \changes{norsk-1.2g}{1996/04/01}{Replaced `Proof' by its
%    translation} 
% \changes{norsk-2.0e}{2000/09/20}{Added \cs{glossaryname}}
% \changes{norsk-2.0g}{1996/04/01}{Replaced `Glossary' by its
%    translation} 
% \changes{norks-2.0h}{2001/01/12}{Changed \cs{ccname} and \cs{alsoname}}
%    \begin{macrocode}
  \def\captionsnynorsk{%
    \def\prefacename{Forord}%
    \def\refname{Referansar}%
    \def\abstractname{Samandrag}%
    \def\bibname{Litteratur}%     or Litteraturoversyn
     %                             or Referansar
    \def\chaptername{Kapittel}%
    \def\appendixname{Tillegg}%   or Appendiks
    \def\contentsname{Innhald}%
    \def\listfigurename{Figurar}% or Figurliste
    \def\listtablename{Tabellar}% or Tabelliste
    \def\indexname{Register}%
    \def\figurename{Figur}%
    \def\tablename{Tabell}%
    \def\partname{Del}%
    \def\enclname{Vedlegg}%
    \def\ccname{Kopi til}%
    \def\headtoname{Til}% in letter
    \def\pagename{Side}%
    \def\seename{Sj\aa{}}%
    \def\alsoname{Sj\aa{} \`{o}g}%
    \def\proofname{Bevis}%
    \def\glossaryname{Ordliste}%
    }
\fi
%    \end{macrocode}
% \end{macro}
%
% \begin{macro}{\datenorsk}
%    The macro |\datenorsk| redefines the command |\today| to produce
%    Norwegian dates.
% \changes{norsk-1.2i}{1997/10/01}{Use \cs{edef} to define
%    \cs{today} to save memory}
% \changes{norsk-1.2i}{1998/03/28}{use \cs{def} instead of \cs{edef}}
% \changes{norsk-2.0i}{2012/08/06}{Removed extra space after `desember'}
%    \begin{macrocode}
\@namedef{date\CurrentOption}{%
  \def\today{\number\day.~\ifcase\month\or
    januar\or februar\or mars\or april\or mai\or juni\or
    juli\or august\or september\or oktober\or november\or
    desember\fi
    \space\number\year}}
%    \end{macrocode}
% \end{macro}
%
% \begin{macro}{\extrasnorsk}
% \begin{macro}{\extrasnynorsk}
%    The macro |\extrasnorsk| will perform all the extra definitions
%    needed for the Norwegian language. The macro |\noextrasnorsk| is
%    used to cancel the actions of |\extrasnorsk|.  
%
%    Norwegian typesetting requires |\frencspacing| to be in effect.
%    \begin{macrocode}
\@namedef{extras\CurrentOption}{\bbl@frenchspacing}
\@namedef{noextras\CurrentOption}{\bbl@nonfrenchspacing}
%    \end{macrocode}
% \end{macro}
% \end{macro}
%
%    For Norsk the \texttt{"} character is made active. This is done
%    once, later on its definition may vary.
% \changes{norsk-2.0a}{1998/06/24}{Made double quote character active}
%    \begin{macrocode}
\initiate@active@char{"}
\expandafter\addto\csname extras\CurrentOption\endcsname{%
  \languageshorthands{norsk}}
\expandafter\addto\csname extras\CurrentOption\endcsname{%
  \bbl@activate{"}}
%    \end{macrocode}
%    Don't forget to turn the shorthands off again.
% \changes{norsk-2.0c}{1999/12/17}{Deactivate shorthands ouside of
%    Norsk}
%    \begin{macrocode}
\expandafter\addto\csname noextras\CurrentOption\endcsname{%
  \bbl@deactivate{"}}
%    \end{macrocode}
%
%    The code above is necessary because we need to define a number of
%    shorthand commands. These sharthand commands are then used as
%    indicated in table~\ref{tab:norsk-quote}.
%
%    To be able to define the function of |"|, we first define a
%    couple of `support' macros.
%
%  \begin{macro}{\dq}
%    We save the original double quote character in |\dq| to keep
%    it available, the math accent |\"| can now be typed as |"|.
%    \begin{macrocode}
\begingroup \catcode`\"12
\def\x{\endgroup
  \def\@SS{\mathchar"7019 }
  \def\dq{"}}
\x
%    \end{macrocode}
%  \end{macro}
%
%    Now we can define the discretionary shorthand commands.
%    The number of words where such hyphenation is required is for
%    each character
%    \begin{center}
%      \begin{tabular}{*{11}c}
%        b&d&f &g&k &l &n&p &r&s &t \\
%        4&4&15&3&43&30&8&12&1&33&35
%       \end{tabular}
%    \end{center}
%    taken from a list of 83000 ispell-roots.
%
% \changes{norsk-2.0d}{2000/02/29}{Shorthands are the same for both
%    spelling variants, no need to use \cs{CurrentOption}}
%    \begin{macrocode}
\declare@shorthand{norsk}{"b}{\textormath{\bbl@disc b{bb}}{b}}
\declare@shorthand{norsk}{"B}{\textormath{\bbl@disc B{BB}}{B}}
\declare@shorthand{norsk}{"d}{\textormath{\bbl@disc d{dd}}{d}}
\declare@shorthand{norsk}{"D}{\textormath{\bbl@disc D{DD}}{D}}
\declare@shorthand{norsk}{"e}{\textormath{\bbl@disc e{\'e}}{}}
\declare@shorthand{norsk}{"E}{\textormath{\bbl@disc E{\'E}}{}}
\declare@shorthand{norsk}{"F}{\textormath{\bbl@disc F{FF}}{F}}
\declare@shorthand{norsk}{"g}{\textormath{\bbl@disc g{gg}}{g}}
\declare@shorthand{norsk}{"G}{\textormath{\bbl@disc G{GG}}{G}}
\declare@shorthand{norsk}{"k}{\textormath{\bbl@disc k{kk}}{k}}
\declare@shorthand{norsk}{"K}{\textormath{\bbl@disc K{KK}}{K}}
\declare@shorthand{norsk}{"l}{\textormath{\bbl@disc l{ll}}{l}}
\declare@shorthand{norsk}{"L}{\textormath{\bbl@disc L{LL}}{L}}
\declare@shorthand{norsk}{"n}{\textormath{\bbl@disc n{nn}}{n}}
\declare@shorthand{norsk}{"N}{\textormath{\bbl@disc N{NN}}{N}}
\declare@shorthand{norsk}{"p}{\textormath{\bbl@disc p{pp}}{p}}
\declare@shorthand{norsk}{"P}{\textormath{\bbl@disc P{PP}}{P}}
\declare@shorthand{norsk}{"r}{\textormath{\bbl@disc r{rr}}{r}}
\declare@shorthand{norsk}{"R}{\textormath{\bbl@disc R{RR}}{R}}
\declare@shorthand{norsk}{"s}{\textormath{\bbl@disc s{ss}}{s}}
\declare@shorthand{norsk}{"S}{\textormath{\bbl@disc S{SS}}{S}}
\declare@shorthand{norsk}{"t}{\textormath{\bbl@disc t{tt}}{t}}
\declare@shorthand{norsk}{"T}{\textormath{\bbl@disc T{TT}}{T}}
%    \end{macrocode}
%    We need to treat |"f| a bit differently in order to preserve the
%    ff-ligature. 
% \changes{norsk-2.0b}{1999/11/19}{Copied the coding for \texttt{"f}
%    from germanb.dtx version 2.6g} 
%    \begin{macrocode}
\declare@shorthand{norsk}{"f}{\textormath{\bbl@discff}{f}}
\def\bbl@discff{\penalty\@M
  \afterassignment\bbl@insertff \let\bbl@nextff= }
\def\bbl@insertff{%
  \if f\bbl@nextff
    \expandafter\@firstoftwo\else\expandafter\@secondoftwo\fi
  {\relax\discretionary{ff-}{f}{ff}\allowhyphens}{f\bbl@nextff}}
\let\bbl@nextff=f
%    \end{macrocode}
%    We now  define the French double quotes and some commands 
%    concerning hyphenation:
% \changes{norsk-2.0b}{1999/11/22}{added the french double quotes}
% \changes{norsk-2.0d}{2000/01/28}{Use \cs{bbl@allowhyphens} in
%    \texttt{"-}}
%    \begin{macrocode}
\declare@shorthand{norsk}{"<}{\flqq}
\declare@shorthand{norsk}{">}{\frqq}
\declare@shorthand{norsk}{"-}{\penalty\@M\-\bbl@allowhyphens}
\declare@shorthand{norsk}{"|}{%
  \textormath{\penalty\@M\discretionary{-}{}{\kern.03em}%
              \allowhyphens}{}}
\declare@shorthand{norsk}{""}{\hskip\z@skip}
\declare@shorthand{norsk}{"~}{\textormath{\leavevmode\hbox{-}}{-}}
\declare@shorthand{norsk}{"=}{\penalty\@M-\hskip\z@skip}
%    \end{macrocode}
%
%    The macro |\ldf@finish| takes care of looking for a
%    configuration file, setting the main language to be switched on
%    at |\begin{document}| and resetting the category code of
%    \texttt{@} to its original value.
% \changes{norsk-1.2h}{1996/11/03}{Now use \cs{ldf@finish} to wrap up}
%    \begin{macrocode}
\ldf@finish\CurrentOption
%</code>
%    \end{macrocode}
%
% \Finale
%%
%% \CharacterTable
%%  {Upper-case    \A\B\C\D\E\F\G\H\I\J\K\L\M\N\O\P\Q\R\S\T\U\V\W\X\Y\Z
%%   Lower-case    \a\b\c\d\e\f\g\h\i\j\k\l\m\n\o\p\q\r\s\t\u\v\w\x\y\z
%%   Digits        \0\1\2\3\4\5\6\7\8\9
%%   Exclamation   \!     Double quote  \"     Hash (number) \#
%%   Dollar        \$     Percent       \%     Ampersand     \&
%%   Acute accent  \'     Left paren    \(     Right paren   \)
%%   Asterisk      \*     Plus          \+     Comma         \,
%%   Minus         \-     Point         \.     Solidus       \/
%%   Colon         \:     Semicolon     \;     Less than     \<
%%   Equals        \=     Greater than  \>     Question mark \?
%%   Commercial at \@     Left bracket  \[     Backslash     \\
%%   Right bracket \]     Circumflex    \^     Underscore    \_
%%   Grave accent  \`     Left brace    \{     Vertical bar  \|
%%   Right brace   \}     Tilde         \~}
%%
\endinput
}
\bbl@tempa{polutonikogreek}{%
  %%
%% This file will generate fast loadable files and documentation
%% driver files from the doc files in this package when run through
%% LaTeX or TeX.
%%
%% Copyright 1989-2007 Johannes L. Braams and any individual authors
%% listed elsewhere in this file.  All rights reserved.
%% 
%% This file is part of the Babel system.
%% --------------------------------------
%% 
%% It may be distributed and/or modified under the
%% conditions of the LaTeX Project Public License, either version 1.3
%% of this license or (at your option) any later version.
%% The latest version of this license is in
%%   http://www.latex-project.org/lppl.txt
%% and version 1.3 or later is part of all distributions of LaTeX
%% version 2003/12/01 or later.
%% 
%% This work has the LPPL maintenance status "maintained".
%% 
%% The Current Maintainer of this work is Johannes Braams.
%% 
%% The list of all files belonging to the LaTeX base distribution is
%% given in the file `manifest.bbl. See also `legal.bbl' for additional
%% information.
%% 
%% The list of derived (unpacked) files belonging to the distribution
%% and covered by LPPL is defined by the unpacking scripts (with
%% extension .ins) which are part of the distribution.
%%
%% --------------- start of docstrip commands ------------------
%%
\def\filedate{2007/10/20}
\def\batchfile{greek.ins}
\input docstrip.tex

{\ifx\generate\undefined
\Msg{**********************************************}
\Msg{*}
\Msg{* This installation requires docstrip}
\Msg{* version 2.3c or later.}
\Msg{*}
\Msg{* An older version of docstrip has been input}
\Msg{*}
\Msg{**********************************************}
\errhelp{Move or rename old docstrip.tex.}
\errmessage{Old docstrip in input path}
\batchmode
\csname @@end\endcsname
\fi}

\declarepreamble\mainpreamble
This is a generated file.

Copyright 1989-2007 Johannes L. Braams and any individual authors
listed elsewhere in this file.  All rights reserved.

This file was generated from file(s) of the Babel system.
---------------------------------------------------------

It may be distributed and/or modified under the
conditions of the LaTeX Project Public License, either version 1.3
of this license or (at your option) any later version.
The latest version of this license is in
  http://www.latex-project.org/lppl.txt
and version 1.3 or later is part of all distributions of LaTeX
version 2003/12/01 or later.

This work has the LPPL maintenance status "maintained".

The Current Maintainer of this work is Johannes Braams.

This file may only be distributed together with a copy of the Babel
system. You may however distribute the Babel system without
such generated files.

The list of all files belonging to the Babel distribution is
given in the file `manifest.bbl'. See also `legal.bbl for additional
information.

The list of derived (unpacked) files belonging to the distribution
and covered by LPPL is defined by the unpacking scripts (with
extension .ins) which are part of the distribution.
\endpreamble

\declarepreamble\fdpreamble
This is a generated file.

Copyright 1989-2005 Johannes L. Braams and any individual authors
listed elsewhere in this file.  All rights reserved.

This file was generated from file(s) of the Babel system.
---------------------------------------------------------

It may be distributed and/or modified under the
conditions of the LaTeX Project Public License, either version 1.3
of this license or (at your option) any later version.
The latest version of this license is in
  http://www.latex-project.org/lppl.txt
and version 1.3 or later is part of all distributions of LaTeX
version 2003/12/01 or later.

This work has the LPPL maintenance status "maintained".

The Current Maintainer of this work is Johannes Braams.

This file may only be distributed together with a copy of the Babel
system. You may however distribute the Babel system without
such generated files.

The list of all files belonging to the Babel distribution is
given in the file `manifest.bbl'. See also `legal.bbl for additional
information.

In particular, permission is granted to customize the declarations in
this file to serve the needs of your installation.

However, NO PERMISSION is granted to distribute a modified version
of this file under its original name.

\endpreamble

\keepsilent

\usedir{tex/generic/babel} 

\usepreamble\mainpreamble
\generate{\file{greek.ldf}{\from{greek.dtx}{code}}
          \file{athnum.sty}{\from{athnum.dtx}{package}}
          \file{grmath.sty}{\from{grmath.dtx}{package}}
          \file{grsymb.sty}{\from{grsymb.dtx}{package}}
          }
\usepreamble\fdpreamble
\generate{\file{lgrenc.def}{\from{greek.fdd}{LGRenc}}
          \file{lgrcmr.fd}{\from{greek.fdd}{fd,LGRcmr}}
          \file{lgrcmro.fd}{\from{greek.fdd}{fd,LGRcmro}}
          \file{lgrcmtt.fd}{\from{greek.fdd}{fd,LGRcmtt}}
          \file{lgrcmss.fd}{\from{greek.fdd}{fd,LGRcmss}}
          \file{lgrlcmtt.fd}{\from{greek.fdd}{fd,LGRlcmtt}}
          \file{lgrlcmss.fd}{\from{greek.fdd}{fd,LGRlcmss}}
          }

\ifToplevel{
\Msg{***********************************************************}
\Msg{*}
\Msg{* To finish the installation you have to move the following}
\Msg{* files into a directory searched by TeX:}
\Msg{*}
\Msg{* \space\space All *.def, *.fd, *.ldf, *.sty}
\Msg{*}
\Msg{* To produce the documentation run the files ending with}
\Msg{* '.dtx' and `.fdd' through LaTeX.}
\Msg{*}
\Msg{* Happy TeXing}
\Msg{***********************************************************}
}
 
\endinput
%
  \languageattribute{greek}{polutoniko}}
\bbl@tempa{portuguese}{% \iffalse meta-comment
%
% Copyright 1989-2008 Johannes L. Braams and any individual authors
% listed elsewhere in this file.  All rights reserved.
% 
% This file is part of the Babel system.
% --------------------------------------
% 
% It may be distributed and/or modified under the
% conditions of the LaTeX Project Public License, either version 1.3
% of this license or (at your option) any later version.
% The latest version of this license is in
%   http://www.latex-project.org/lppl.txt
% and version 1.3 or later is part of all distributions of LaTeX
% version 2003/12/01 or later.
% 
% This work has the LPPL maintenance status "maintained".
% 
% The Current Maintainer of this work is Johannes Braams.
% 
% The list of all files belonging to the Babel system is
% given in the file `manifest.bbl. See also `legal.bbl' for additional
% information.
% 
% The list of derived (unpacked) files belonging to the distribution
% and covered by LPPL is defined by the unpacking scripts (with
% extension .ins) which are part of the distribution.
% \fi
% \CheckSum{320}
% \iffalse
%    Tell the \LaTeX\ system who we are and write an entry on the
%    transcript.
%<*dtx>
\ProvidesFile{portuges.dtx}
%</dtx>
%<code>\ProvidesLanguage{portuges}
%\fi
%\ProvidesFile{portuges.dtx}
        [2008/03/18 v1.2q Portuguese support from the babel system]
%\iffalse
%% File `portuges.dtx'
%% Babel package for LaTeX version 2e
%% Copyright (C) 1989 - 2008
%%           by Johannes Braams, TeXniek
%
%% Portuguese Language Definition File
%% Copyright (C) 1989 - 2008
%%           by Johannes Braams, TeXniek
%
%% Please report errors to: J.L. Braams
%%                          babel at braams.cistron.nl
%
%    This file is part of the babel system, it provides the source
%    code for the Portuguese language definition file.  The Portuguese
%    words were contributed by Jose Pedro Ramalhete, (JRAMALHE@CERNVM
%    or Jose-Pedro_Ramalhete@MACMAIL).
%
%    Arnaldo Viegas de Lima <arnaldo@VNET.IBM.COM> contributed
%    brasilian translations and suggestions for enhancements.
%<*filedriver>
\documentclass{ltxdoc}
\newcommand*\TeXhax{\TeX hax}
\newcommand*\babel{\textsf{babel}}
\newcommand*\langvar{$\langle \it lang \rangle$}
\newcommand*\note[1]{}
\newcommand*\Lopt[1]{\textsf{#1}}
\newcommand*\file[1]{\texttt{#1}}
\begin{document}
 \DocInput{portuges.dtx}
\end{document}
%</filedriver>
%\fi
%
% \GetFileInfo{portuges.dtx}
%
% \changes{portuges-1.0a}{1991/07/15}{Renamed \file{babel.sty} in
%    \file{babel.com}}
% \changes{portuges-1.1}{1992/02/16}{Brought up-to-date with babel 3.2a}
% \changes{portuges-1.2}{1994/02/26}{Update for \LaTeXe}
% \changes{portuges-1.2d}{1994/06/26}{Removed the use of \cs{filedate}
%    and moved identification after the loading of \file{babel.def}}
% \changes{portuges-1.2g}{1995/06/04}{Enhanced support for brasilian}
% \changes{portuges-1.2j}{1996/07/11}{Replaced \cs{undefined} with
%    \cs{@undefined} and \cs{empty} with \cs{@empty} for consistency
%    with \LaTeX} 
% \changes{portuges-1.2j}{1996/10/10}{Moved the definition of
%    \cs{atcatcode} right to the beginning.}
%
%  \section{The Portuguese language}
%
%    The file \file{\filename}\footnote{The file described in this
%    section has version number \fileversion\ and was last revised on
%    \filedate.  Contributions were made by Jose Pedro Ramalhete
%    (\texttt{JRAMALHE@CERNVM} or
%    \texttt{Jose-Pedro\_Ramalhete@MACMAIL}) and Arnaldo Viegas de
%    Lima \texttt{arnaldo@VNET.IBM.COM}.}  defines all the
%    language-specific macros for the Portuguese language as well as
%    for the Brasilian version of this language.
%
%    For this language the character |"| is made active. In
%    table~\ref{tab:port-quote} an overview is given of its purpose.
%
%    \begin{table}[htb]
%     \centering
%     \begin{tabular}{lp{8cm}}
%       \verb="|= & disable ligature at this position.\\
%        |"-| & an explicit hyphen sign, allowing hyphenation
%               in the rest of the word.\\
%        |""| & like \verb="-=, but producing no hyphen sign (for
%              words that should break at some sign such as
%              ``entrada/salida.''\\
%        |"<| & for French left double quotes (similar to $<<$).\\
%        |">| & for French right double quotes (similar to $>>$).\\
%        |\-| & like the old |\-|, but allowing hyphenation
%               in the rest of the word. \\
%     \end{tabular}
%     \caption{The extra definitions made by \file{portuges.ldf}}
%     \label{tab:port-quote}
%    \end{table}
%
% \StopEventually{}
%
%    The macro |\LdfInit| takes care of preventing that this file is
%    loaded more than once, checking the category code of the
%    \texttt{@} sign, etc.
% \changes{portuges-1.2j}{1996/11/03}{Now use \cs{LdfInit} to perform
%    initial checks} 
%    \begin{macrocode}
%<*code>
\LdfInit\CurrentOption{captions\CurrentOption}
%    \end{macrocode}
%
%    When this file is read as an option, i.e. by the |\usepackage|
%    command, \texttt{portuges} will be an `unknown' language in which
%    case we have to make it known. So we check for the existence of
%    |\l@portuges| to see whether we have to do something here. Since
%    it is possible to load this file with any of the following four
%    options to babel: \Lopt{portuges}, \Lopt{portuguese},
%    \Lopt{brazil} and \Lopt{brazilian} we also allow that the
%    hyphenation patterns are loaded under any of these four names. We
%    just have to find out which one was used.
%
% \changes{portuges-1.0b}{1991/10/29}{Removed use of cs{@ifundefined}}
% \changes{portuges-1.1}{1992/02/16}{Added a warning when no
%    hyphenation patterns were loaded.}
% \changes{portuges-1.2d}{1994/06/26}{Now use \cs{@nopatterns} to
%    produce the warning}
%    \begin{macrocode}
\ifx\l@portuges\@undefined
  \ifx\l@portuguese\@undefined
    \ifx\l@brazil\@undefined
      \ifx\l@brazilian\@undefined
        \@nopatterns{Portuguese}
        \adddialect\l@portuges0
      \else
        \let\l@portuges\l@brazilian
      \fi
    \else
      \let\l@portuges\l@brazil
    \fi
  \else
    \let\l@portuges\l@portuguese
  \fi
\fi
%    \end{macrocode}
%    By now |\l@portuges| is defined. When the language definition
%    file was loaded under a different name we make sure that the
%    hyphenation patterns can be found.
%    \begin{macrocode}
\expandafter\ifx\csname l@\CurrentOption\endcsname\relax
  \expandafter\let\csname l@\CurrentOption\endcsname\l@portuges
\fi
%    \end{macrocode}
%
%    Now we have to decide whether this language definition file was
%    loaded for Portuguese or Brasilian use. This can be done by
%    checking the contents of |\CurrentOption|. When it doesn't
%    contain either `portuges' or `portuguese' we make |\bbl@tempb|
%    empty. 
%    \begin{macrocode}
\def\bbl@tempa{portuguese}
\ifx\CurrentOption\bbl@tempa
  \let\bbl@tempb\@empty
\else
  \def\bbl@tempa{portuges}
  \ifx\CurrentOption\bbl@tempa
    \let\bbl@tempb\@empty
  \else
    \def\bbl@tempb{brazil}
  \fi
\fi
\ifx\bbl@tempb\@empty
%    \end{macrocode}
%
%    The next step consists of defining commands to switch to (and from)
%    the Portuguese language.
%
% \begin{macro}{\captionsportuges}
%    The macro |\captionsportuges| defines all strings used
%    in the four standard documentclasses provided with \LaTeX.
% \changes{portuges-1.1}{1992/02/16}{Added \cs{seename}, \cs{alsoname}
%    and \cs{prefacename}}
% \changes{portuges-1.1}{1993/07/15}{\cs{headpagename} should be
%    \cs{pagename}}
% \changes{portuges-1.2e}{1994/11/09}{Added a few missing
%    translations}
% \changes{portuges-1.2h}{1995/07/04}{Added \cs{proofname} for
%    AMS-\LaTeX}
% \changes{portuges-1.2i}{1995/11/25}{Substituted `Prova' for `Proof'}
%    \begin{macrocode}
  \@namedef{captions\CurrentOption}{%
    \def\prefacename{Pref\'acio}%
    \def\refname{Refer\^encias}%
    \def\abstractname{Resumo}%
    \def\bibname{Bibliografia}%
    \def\chaptername{Cap\'{\i}tulo}%
    \def\appendixname{Ap\^endice}%
%    \end{macrocode}
%    Some discussion took place around the correct translations for
%    `Table of Contents' and `Index'. the translations differ for
%    Portuguese and Brasilian based the following history:
%    \begin{quote}
%      The whole issue is that some books without a real index at the
%      end misused the term `\'Indice' as table of contents. Then,
%      what happens is that some books apeared with `\'Indice' at the
%      begining and a `\'Indice Remissivo' at the end. Remissivo is a
%      redundant word in this case, but was introduced to make up the
%      difference. So in Brasil people started using `Sum\'ario' and
%      `\'Indice Remissivo'. In Portugal this seems not to be very
%      common, therefore we chose `\'Indice' instead of `\'Indice
%      Remissivo'.
%    \end{quote}
%    \begin{macrocode}
    \def\contentsname{Conte\'udo}%
    \def\listfigurename{Lista de Figuras}%
    \def\listtablename{Lista de Tabelas}%
    \def\indexname{\'Indice}%
    \def\figurename{Figura}%
    \def\tablename{Tabela}%
    \def\partname{Parte}%
    \def\enclname{Anexo}%
    \def\ccname{Com c\'opia a}%
    \def\headtoname{Para}%
    \def\pagename{P\'agina}%
    \def\seename{ver}%
    \def\alsoname{ver tamb\'em}%
%    \end{macrocode}
%    An alternate term for `Proof' could be `Prova'.
% \changes{portuges-1.2m}{2000/09/20}{Added \cs{glossaryname}}
% \changes{portuges-1.2p}{2003/05/23}{Substituted `Gloss\'ario' for
%    `Glossary'}
%    \begin{macrocode}
    \def\proofname{Demonstra\c{c}\~ao}%
    \def\glossaryname{Gloss\'ario}%
    }
%    \end{macrocode}
% \end{macro}
%
% \begin{macro}{\dateportuges}
%    The macro |\dateportuges| redefines the command |\today| to
%    produce Portuguese dates.
% \changes{portuges-1.2k}{1997/10/01}{Use \cs{edef} to define
%    \cs{today} to save memory}
% \changes{portuges-1.2k}{1998/03/28}{use \cs{def} instead of
%    \cs{edef}} 
% \changes{portuges-1.2n}{2001/01/27}{Removed spurious space after
%    Dezembro}
%    \begin{macrocode}
  \@namedef{date\CurrentOption}{%
    \def\today{\number\day\space de\space\ifcase\month\or
      Janeiro\or Fevereiro\or Mar\c{c}o\or Abril\or Maio\or Junho\or
      Julho\or Agosto\or Setembro\or Outubro\or Novembro\or Dezembro%
      \fi
      \space de\space\number\year}}
\else
%    \end{macrocode}
% \end{macro}
%
%    For the Brasilian version of these definitions we just add a
%    ``dialect''. 
%    \begin{macrocode}
  \expandafter
    \adddialect\csname l@\CurrentOption\endcsname\l@portuges
%    \end{macrocode}
%
% \begin{macro}{\captionsbrazil}
% \changes{portuges-1.2g}{1995/06/04}{The captions for brasilian and
%    portuguese are different now}
%
%    The ``captions'' are different for both versions of the language,
%    so we define the macro |\captionsbrazil| here.
% \changes{portuges-1.2i}{1995/11/25}{Added \cs{proofname} for
%    AMS-\LaTeX}
% \changes{portuges-1.2m}{2000/09/20}{Added \cs{glossaryname}}
% \changes{portuges-1.2q}{2008/03/18}{Substituted `Gloss\'ario' for
%    `Glossary'}
%    \begin{macrocode}
  \@namedef{captions\CurrentOption}{%
    \def\prefacename{Pref\'acio}%
    \def\refname{Refer\^encias}%
    \def\abstractname{Resumo}%
    \def\bibname{Refer\^encias Bibliogr\'aficas}%
    \def\chaptername{Cap\'{\i}tulo}%
    \def\appendixname{Ap\^endice}%
    \def\contentsname{Sum\'ario}%
    \def\listfigurename{Lista de Figuras}%
    \def\listtablename{Lista de Tabelas}%
    \def\indexname{\'Indice Remissivo}%
    \def\figurename{Figura}%
    \def\tablename{Tabela}%
    \def\partname{Parte}%
    \def\enclname{Anexo}%
    \def\ccname{C\'opia para}%
    \def\headtoname{Para}%
    \def\pagename{P\'agina}%
    \def\seename{veja}%
    \def\alsoname{veja tamb\'em}%
    \def\proofname{Demonstra\c{c}\~ao}%
    \def\glossaryname{Gloss\'ario}%
    }
%    \end{macrocode}
% \end{macro}
%
% \begin{macro}{\datebrazil}
%    The macro |\datebrazil| redefines the command
%    |\today| to produce Brasilian dates, for which the names
%    of the months are not capitalized.
% \changes{portuges-1.2k}{1997/10/01}{Use \cs{edef} to define
%    \cs{today} to save memory}
% \changes{portuges-1.2k}{1998/03/28}{use \cs{def} instead of
%    \cs{edef}} 
% \changes{portuges-1.2n}{2001/01/27}{Removed spurious space after
%    dezembro}
%    \begin{macrocode}
  \@namedef{date\CurrentOption}{%
    \def\today{\number\day\space de\space\ifcase\month\or
      janeiro\or fevereiro\or mar\c{c}o\or abril\or maio\or junho\or
      julho\or agosto\or setembro\or outubro\or novembro\or dezembro%
      \fi
      \space de\space\number\year}}
\fi
%    \end{macrocode}
% \end{macro}
%
%  \begin{macro}{\portugeshyphenmins}
% \changes{portuges-1.2g}{1995/06/04}{Added setting of hyphenmin
%    values}
%    Set correct values for |\lefthyphenmin| and |\righthyphenmin|.
% \changes{portuges-1.2m}{2000/09/22}{Now use \cs{providehyphenmins} to
%    provide a default value}
% \changes{portuges-1.2o}{2001/02/16}{Set \cs{righthyphenmin} to 3 if
%    not provided by the pattern file.}
%    \begin{macrocode}
\providehyphenmins{\CurrentOption}{\tw@\thr@@}
%    \end{macrocode}
%  \end{macro}
%
% \begin{macro}{\extrasportuges}
% \changes{portuges-1.2g}{1995/06/04}{Added the definition of some
%    \texttt{"} shorthands}
% \begin{macro}{\noextrasportuges}
%    The macro |\extrasportuges| will perform all the extra
%    definitions needed for the Portuguese language. The macro
%    |\noextrasportuges| is used to cancel the actions of
%    |\extrasportuges|.
%
%    For Portuguese the \texttt{"} character is made active. This is
%    done once, later on its definition may vary. Other languages in
%    the same document may also use the \texttt{"} character for
%    shorthands; we specify that the portuguese group of shorthands
%    should be used.
%
%    \begin{macrocode}
\initiate@active@char{"}
\@namedef{extras\CurrentOption}{\languageshorthands{portuges}}
\expandafter\addto\csname extras\CurrentOption\endcsname{%
  \bbl@activate{"}}
%    \end{macrocode}
%    Don't forget to turn the shorthands off again.
% \changes{portuges-1.2m}{1999/12/17}{Deactivate shorthands ouside of
%    Basque}
%    \begin{macrocode}
\addto\noextrasportuges{\bbl@deactivate{"}}
%    \end{macrocode}
%    First we define access to the guillemets for quotations,
% \changes{portuges-1.2k}{1997/04/03}{Removed empty groups after
%    guillemot characters}
%    \begin{macrocode}
\declare@shorthand{portuges}{"<}{%
  \textormath{\guillemotleft}{\mbox{\guillemotleft}}}
\declare@shorthand{portuges}{">}{%
  \textormath{\guillemotright}{\mbox{\guillemotright}}}
%    \end{macrocode}
%    then we define two shorthands to be able to specify hyphenation
%    breakpoints that behave a little different from |\-|.
%    \begin{macrocode}
\declare@shorthand{portuges}{"-}{\nobreak-\bbl@allowhyphens}
\declare@shorthand{portuges}{""}{\hskip\z@skip}
%    \end{macrocode}
%    And we want to have a shorthand for disabling a ligature.
%    \begin{macrocode}
\declare@shorthand{portuges}{"|}{%
  \textormath{\discretionary{-}{}{\kern.03em}}{}}
%    \end{macrocode}
% \end{macro}
% \end{macro}
%
%  \begin{macro}{\-}
%
%    All that is left now is the redefinition of |\-|. The new version
%    of |\-| should indicate an extra hyphenation position, while
%    allowing other hyphenation positions to be generated
%    automatically. The standard behaviour of \TeX\ in this respect is
%    very unfortunate for languages such as Dutch and German, where
%    long compound words are quite normal and all one needs is a means
%    to indicate an extra hyphenation position on top of the ones that
%    \TeX\ can generate from the hyphenation patterns.
%    \begin{macrocode}
\expandafter\addto\csname extras\CurrentOption\endcsname{%
  \babel@save\-}
\expandafter\addto\csname extras\CurrentOption\endcsname{%
  \def\-{\allowhyphens\discretionary{-}{}{}\allowhyphens}}
%    \end{macrocode}
%  \end{macro}
%
%  \begin{macro}{\ord}
% \changes{portuges-1.2g}{1995/06/04}{Added macro}
%  \begin{macro}{\ro}
% \changes{portuges-1.2g}{1995/06/04}{Added macro}
%  \begin{macro}{\orda}
% \changes{portuges-1.2g}{1995/06/04}{Added macro}
%  \begin{macro}{\ra}
% \changes{portuges-1.2g}{1995/06/04}{Added macro}
%    We also provide an easy way to typeset ordinals, both in the male
%    (|\ord| or |\ro|) and the female (|orda| or |\ra|) form.
%    \begin{macrocode}
\def\ord{$^{\rm o}$}
\def\orda{$^{\rm a}$}
\let\ro\ord\let\ra\orda
%    \end{macrocode}
%  \end{macro}
%  \end{macro}
%  \end{macro}
%  \end{macro}
%
%    The macro |\ldf@finish| takes care of looking for a
%    configuration file, setting the main language to be switched on
%    at |\begin{document}| and resetting the category code of
%    \texttt{@} to its original value.
% \changes{portuges-1.2j}{1996/11/03}{ow use \cs{ldf@finish} to wrap
%    up} 
%    \begin{macrocode}
\ldf@finish\CurrentOption
%</code>
%    \end{macrocode}
%
% \Finale
%%
%% \CharacterTable
%%  {Upper-case    \A\B\C\D\E\F\G\H\I\J\K\L\M\N\O\P\Q\R\S\T\U\V\W\X\Y\Z
%%   Lower-case    \a\b\c\d\e\f\g\h\i\j\k\l\m\n\o\p\q\r\s\t\u\v\w\x\y\z
%%   Digits        \0\1\2\3\4\5\6\7\8\9
%%   Exclamation   \!     Double quote  \"     Hash (number) \#
%%   Dollar        \$     Percent       \%     Ampersand     \&
%%   Acute accent  \'     Left paren    \(     Right paren   \)
%%   Asterisk      \*     Plus          \+     Comma         \,
%%   Minus         \-     Point         \.     Solidus       \/
%%   Colon         \:     Semicolon     \;     Less than     \<
%%   Equals        \=     Greater than  \>     Question mark \?
%%   Commercial at \@     Left bracket  \[     Backslash     \\
%%   Right bracket \]     Circumflex    \^     Underscore    \_
%%   Grave accent  \`     Left brace    \{     Vertical bar  \|
%%   Right brace   \}     Tilde         \~}
%%
\endinput
}
\bbl@tempa{russian}{% \iffalse meta-comment
%
% Copyright 1989-2008 Johannes L. Braams and any individual authors
% listed elsewhere in this file.  All rights reserved.
% 
% This file is part of the Babel system.
% --------------------------------------
% 
% It may be distributed and/or modified under the
% conditions of the LaTeX Project Public License, either version 1.3
% of this license or (at your option) any later version.
% The latest version of this license is in
%   http://www.latex-project.org/lppl.txt
% and version 1.3 or later is part of all distributions of LaTeX
% version 2003/12/01 or later.
% 
% This work has the LPPL maintenance status "maintained".
% 
% The Current Maintainer of this work is Johannes Braams.
% 
% The list of all files belonging to the Babel system is
% given in the file `manifest.bbl. See also `legal.bbl' for additional
% information.
% 
% The list of derived (unpacked) files belonging to the distribution
% and covered by LPPL is defined by the unpacking scripts (with
% extension .ins) which are part of the distribution.
% \fi
% \CheckSum{1461}
%
% \iffalse
%    Tell the \LaTeX\ system who we are and write an entry on the
%    transcript.
%<*dtx>
\ProvidesFile{russianb.dtx}
%</dtx>
%<code>\ProvidesLanguage{russianb}
        [2008/03/21 v1.1r Russian support from the babel system]
%
%% File `russianb.dtx'
%% Babel package for LaTeX version 2e
%% Copyright (C) 1989 - 2008
%%           by Johannes Braams, TeXniek
%
%% Russianb Language Definition File
%% Copyright (C) 1995 - 2008
%%           by Olga Lapko cyrtug at mir.msk.su
%%              Johannes Braams, TeXniek
%
%% Adapted to the new T2 and X2 Cyrillic encodings
%%           by Vladimir Volovich TeX at vvv.vsu.ru
%%              Werner Lemberg wl at gnu.org
%
%% Please report errors to: J.L. Braams
%%                          babel at braams dot xs4all dot nl
%
%<*filedriver>
\documentclass{ltxdoc}
\newcommand\TeXhax{\TeX hax}
\newcommand\babel{\textsf{babel}}
\newcommand\langvar{$\langle \it lang \rangle$}
\newcommand\note[1]{}
\newcommand\Lopt[1]{\textsf{#1}}
\newcommand\file[1]{\texttt{#1}}
\newcommand\pkg[1]{\texttt{#1}}
\begin{document}
 \DocInput{russianb.dtx}
\end{document}
%</filedriver>
%\fi
% \GetFileInfo{russianb.dtx}
%
% \changes{russianb-1.1c}{1996/07/11}{Replaced \cs{undefined} with
%    \cs{@undefined} and \cs{empty} with \cs{@empty} for consistency
%    with \LaTeX}
% \changes{russianb-1.1d}{1996/10/10}{Moved the definition of
%    \cs{atcatcode} right to the beginning.}
% \changes{russianb-1.1k}{1999/08/19}{replaced all \cs{penalty}\cs{@M}
%    with \cs{nobreak}}
%
%  \section{The Russian language}
%
%    The file \file{\filename}\footnote{The file described in this section
%    has version number \fileversion\ and was last revised on \filedate.
%    This file was initially derived from the original version of
%    \file{german.sty}, which has some definitions for Russian. Later the
%    definitions from \file{russian.sty} version 1.0b (for \LaTeX\ 2.09),
%    \file{russian.sty} version v2.5c (for \LaTeXe) and \file{francais.sty}
%    version 4.5c and \file{germanb.sty} version 2.5c were added.} defines
%    all the language-specific macros for the Russian language. It needs the
%    file \file{cyrcod} for success documentation with Russian encodings
%    (see below).
%
%    For this language the character |"| is made active. In
%    table~\ref{tab:russian-quote} an overview is given of its purpose.
%
% \changes{russianb-1.1f}{1998/06/26}{%
%    Added definitions of Cyrillic emdash stuff and thinspace}
%
%    \begin{table}[htb]
%      \begin{center}
%      \begin{tabular}{lp{8cm}}
%       \verb="|= & disable ligature at this position.               \\
%       |"-| & an explicit hyphen sign, allowing hyphenation
%                   in the rest of the word.                         \\
%       |"---| & Cyrillic emdash in plain text.                      \\
%       |"--~| & Cyrillic emdash in compound names (surnames).       \\
%       |"--*| & Cyrillic emdash for denoting direct speech.         \\
%       |""| & like |"-|, but producing no hyphen sign
%                   (for compund words with hyphen, e.g.\ |x-""y|
%                   or some other signs  as ``disable/enable'').     \\
%       |"~| & for a compound word mark without a breakpoint.        \\
%       |"=| & for a compound word mark with a breakpoint, allowing
%              hyphenation in the composing words.                   \\
%       |",| & thinspace for initials with a breakpoint
%               in following surname.                                \\
%       |"`| & for German left double quotes
%                   (looks like ,\kern-0.08em,).                     \\
%       |"'| & for German right double quotes (looks like ``).       \\%^^A''
%       |"<| & for French left double quotes (looks like $<\!\!<$).  \\
%       |">| & for French right double quotes (looks like $>\!\!>$). \\
%      \end{tabular}
%      \caption{The extra definitions made
%               by \file{russianb}}\label{tab:russian-quote}
%      \end{center}
%    \end{table}
%
%    The quotes in table~\ref{tab:russian-quote} can also be typeset by
%    using the commands in table~\ref{tab:rmore-quote}.
%
%    \begin{table}[htb]
%      \begin{center}
%      \begin{tabular}{lp{8cm}}
%       |\cdash---| & Cyrillic emdash in plain text.                    \\
%       |\cdash--~| & Cyrillic emdash in compound names (surnames).     \\
%       |\cdash--*| & Cyrillic emdash for denoting direct speech.       \\
%       |\glqq| & for German left double quotes
%                    (looks like ,\kern-0.08em,).                       \\
%       |\grqq| & for German right double quotes (looks like ``).       \\%^^A''
%       |\flqq| & for French left double quotes (looks like $<\!\!<$).  \\
%       |\frqq| & for French right double quotes (looks like $>\!\!>$). \\
%       |\dq|   & the original quotes character (|"|).                  \\
%      \end{tabular}
%      \caption{More commands which produce quotes, defined
%               by \babel}\label{tab:rmore-quote}
%      \end{center}
%    \end{table}
%
%    The French quotes are also available as ligatures `|<<|' and `|>>|' in
%    8-bit Cyrillic font encodings (\texttt{LCY}, \texttt{X2}, \texttt{T2*})
%    and as `|<|' and `|>|' characters in 7-bit Cyrillic font encodings
%    (\texttt{OT2} and \texttt{LWN}).
%
%    The quotation marks traditionally used in Russian were borrowed from
%    other languages (e.g., French and German) so they keep their original
%    names.
%
% \StopEventually{}
%
%    The macro |\LdfInit| takes care of preventing that this file is loaded
%    more than once, checking the category code of the \texttt{@} sign, etc.
%
% \changes{russianb-1.1d}{1996/11/03}{Now use \cs{LdfInit} to perform
%    initial checks}
% \changes{russianb-1.1e}{1996/12/29}{Added closing brace to second
%    argument of \cs{LdfInit}}
%    \begin{macrocode}
%<*code>
\LdfInit{russian}{captionsrussian}
%    \end{macrocode}
%
%    When this file is read as an option, i.e., by the |\usepackage|
%    command, \texttt{russianb} will be an `unknown' language, in which case
%    we have to make it known. So we check for the existence of |\l@russian|
%    to see whether we have to do something here.
%
%    \begin{macrocode}
\ifx\l@russian\@undefined
  \@nopatterns{Russian}
  \adddialect\l@russian0
\fi
%    \end{macrocode}
%
%  \begin{macro}{\latinencoding}
%
%    We need to know the encoding for text that is supposed to be which is
%    active at the end of the \babel\ package. If the \pkg{fontenc} package
%    is loaded later, then\ldots too bad!
%
%    \begin{macrocode}
\let\latinencoding\cf@encoding
%    \end{macrocode}
%
%  \end{macro}
%
%    The user may choose between different available Cyrillic
%    encodings---e.g., \texttt{X2}, \texttt{LCY}, or \texttt{LWN}.\@
%    Hopefully, \texttt{X2} will eventually replace the two latter encodings
%    (\texttt{LCY} and \texttt{LWN}).\@ If the user wants to use another
%    font encoding than the default (\texttt{T2A}), he has to load the
%    corresponding file \emph{before} \file{russianb.sty}. This may be done
%    in the following way:
%
%    \begin{verbatim}
%      % override the default X2 encoding used in Babel
%      \usepackage[LCY,OT1]{fontenc}
%      \usepackage[english,russian]{babel}
%    \end{verbatim}
%    \unskip
%
%    Note: for the Russian language, the \texttt{T2A} encoding is better than
%    \texttt{X2}, because \texttt{X2} does not contain Latin letters, and
%    users should be very careful to switch the language every time they
%    want to typeset a Latin word inside a Russian phrase or vice versa.
%
%    We parse the |\cdp@list| containing the encodings known to \LaTeX\ in
%    the order they were loaded. We set the |\cyrillicencoding| to the
%    \emph{last} loaded encoding in the list of supported Cyrillic
%    encodings: \texttt{OT2}, \texttt{LWN}, \texttt{LCY}, \texttt{X2},
%    \texttt{T2C}, \texttt{T2B}, \texttt{T2A}, if any.
%
%    \begin{macrocode}
\def\reserved@a#1#2{%
   \edef\reserved@b{#1}%
   \edef\reserved@c{#2}%
   \ifx\reserved@b\reserved@c
     \let\cyrillicencoding\reserved@c
   \fi}
\def\cdp@elt#1#2#3#4{%
   \reserved@a{#1}{OT2}%
   \reserved@a{#1}{LWN}%
   \reserved@a{#1}{LCY}%
   \reserved@a{#1}{X2}%
   \reserved@a{#1}{T2C}%
   \reserved@a{#1}{T2B}%
   \reserved@a{#1}{T2A}}
\cdp@list
%    \end{macrocode}
%
%    Now, if |\cyrillicencoding| is undefined, then the user did not load
%    any of supported encodings. So, we have to set |\cyrillicencoding| to
%    some default value. We test the presence of the encoding definition
%    files in the order from less preferable to more preferable encodings.
%    We use the lowercase names (i.e., \file{lcyenc.def} instead of
%    \file{LCYenc.def}).
%
%    \begin{macrocode}
\ifx\cyrillicencoding\undefined
  \IfFileExists{ot2enc.def}{\def\cyrillicencoding{OT2}}\relax
  \IfFileExists{lwnenc.def}{\def\cyrillicencoding{LWN}}\relax
  \IfFileExists{lcyenc.def}{\def\cyrillicencoding{LCY}}\relax
  \IfFileExists{x2enc.def}{\def\cyrillicencoding{X2}}\relax
  \IfFileExists{t2cenc.def}{\def\cyrillicencoding{T2C}}\relax
  \IfFileExists{t2benc.def}{\def\cyrillicencoding{T2B}}\relax
  \IfFileExists{t2aenc.def}{\def\cyrillicencoding{T2A}}\relax
%    \end{macrocode}
%
%    If |\cyrillicencoding| is still undefined, then the user seems not to
%    have a properly installed distribution. A fatal error.
%
%    \begin{macrocode}
  \ifx\cyrillicencoding\undefined
    \PackageError{babel}%
      {No Cyrillic encoding definition files were found}%
      {Your installation is incomplete.\MessageBreak
       You need at least one of the following files:\MessageBreak
       \space\space
       x2enc.def, t2aenc.def, t2benc.def, t2cenc.def,\MessageBreak
       \space\space
       lcyenc.def, lwnenc.def, ot2enc.def.}%
  \else
%    \end{macrocode}
%
%    We avoid |\usepackage[\cyrillicencoding]{fontenc}| because we don't
%    want to force the switch of |\encodingdefault|.
%
%    \begin{macrocode}
    \lowercase
      \expandafter{\expandafter\input\cyrillicencoding enc.def\relax}%
  \fi
\fi
%    \end{macrocode}
%
%    \begin{verbatim}
%      \PackageInfo{babel}
%        {Using `\cyrillicencoding' as a default Cyrillic encoding}%
%    \end{verbatim}
%    \unskip
%
%    \begin{macrocode}
\DeclareRobustCommand{\Russian}{%
  \fontencoding\cyrillicencoding\selectfont
  \let\encodingdefault\cyrillicencoding
  \expandafter\set@hyphenmins\russianhyphenmins
  \language\l@russian}%
\DeclareRobustCommand{\English}{%
  \fontencoding\latinencoding\selectfont
  \let\encodingdefault\latinencoding
  \expandafter\set@hyphenmins\englishhyphenmins
  \language\l@english}%
\let\Rus\Russian
\let\Eng\English
\let\cyrillictext\Russian
\let\cyr\Russian
%    \end{macrocode}
%
%    Since the \texttt{X2} encoding does not contain Latin letters, we
%    should make some redefinitions of \LaTeX\ macros which implicitly
%    produce Latin letters.
%
%    \begin{macrocode}
\expandafter\ifx\csname T@X2\endcsname\relax\else
%    \end{macrocode}
%
%    We put |\latinencoding| in braces to avoid problems with
%    |\@alph| inside minipages (e.g., footnotes inside minipages) where
%    |\@alph| is expanded and we get for example `|\fontencoding OT1|'
%    (|\fontencoding| is robust).
%
%    \begin{macrocode}
  \def\@alph#1{{\fontencoding{\latinencoding}\selectfont
    \ifcase#1\or
      a\or b\or c\or d\or e\or f\or g\or h\or
      i\or j\or k\or l\or m\or n\or o\or p\or
      q\or r\or s\or t\or u\or v\or w\or x\or
      y\or z\else\@ctrerr\fi}}%
  \def\@Alph#1{{\fontencoding{\latinencoding}\selectfont
    \ifcase#1\or
      A\or B\or C\or D\or E\or F\or G\or H\or
      I\or J\or K\or L\or M\or N\or O\or P\or
      Q\or R\or S\or T\or U\or V\or W\or X\or
      Y\or Z\else\@ctrerr\fi}}%
%    \end{macrocode}
%
%    Unfortunately, the commands |\AA| and |\aa| are not encoding dependent
%    in \LaTeX\ (unlike e.g., |\oe| or |\DH|). They are defined as |\r{A}| and
%    |\r{a}|. This leads to unpredictable results when the font encoding
%    does not contain the Latin letters `A' and `a' (like \texttt{X2}).
%
%    \begin{macrocode}
  \DeclareTextSymbolDefault{\AA}{OT1}
  \DeclareTextSymbolDefault{\aa}{OT1}
  \DeclareTextCommand{\aa}{OT1}{\r a}
  \DeclareTextCommand{\AA}{OT1}{\r A}
\fi
%    \end{macrocode}
%
%    The following block redefines the character class of uppercase Greek
%    letters and some accents, if it is equal to 7 (variable family), to
%    avoid incorrect results if the font encoding in some math family does
%    not contain these characters in places of OT1 encoding. The code was
%    taken from |amsmath.dtx|. See comments and further explanation there.
%
% \changes{russianb-1.1n}{2001/02/21}{As this code generates a
%    textfont 7 error it is commented out for now.}
%    \begin{macrocode}
% \begingroup\catcode`\"=12
% % uppercase greek letters:
% \def\@tempa#1{\expandafter\@tempb\meaning#1\relax\relax\relax\relax
%   "0000\@nil#1}
% \def\@tempb#1"#2#3#4#5#6\@nil#7{%
%   \ifnum"#2=7 \count@"1#3#4#5\relax
%     \ifnum\count@<"1000 \else \global\mathchardef#7="0#3#4#5\relax \fi
%   \fi}
% \@tempa\Gamma\@tempa\Delta\@tempa\Theta\@tempa\Lambda\@tempa\Xi
% \@tempa\Pi\@tempa\Sigma\@tempa\Upsilon\@tempa\Phi\@tempa\Psi
% \@tempa\Omega
% % some accents:
% \def\@tempa#1#2\@nil{\def\@tempc{#1}}\def\@tempb{\mathaccent}
% \expandafter\@tempa\hat\relax\relax\@nil
% \ifx\@tempb\@tempc
%   \def\@tempa#1\@nil{#1}%
%   \def\@tempb#1{\afterassignment\@tempa\mathchardef\@tempc=}%
%   \def\do#1"#2{}
%   \def\@tempd#1{\expandafter\@tempb#1\@nil
%     \ifnum\@tempc>"FFF
%       \xdef#1{\mathaccent"\expandafter\do\meaning\@tempc\space}%
%     \fi}
%   \@tempd\hat\@tempd\check\@tempd\tilde\@tempd\acute\@tempd\grave
%   \@tempd\dot\@tempd\ddot\@tempd\breve\@tempd\bar
% \fi
% \endgroup
%    \end{macrocode}
%
%    The user should use the \pkg{inputenc} package when any 8-bit Cyrillic
%    font encoding is used, selecting one of the Cyrillic input encodings.
%    We do not assume any default input encoding, so the user should
%    explicitly call the \pkg{inputenc} package by |\usepackage{inputenc}|.
%    We also removed |\AtBeginDocument|, so \pkg{inputenc} should be used
%    before \babel.
%
% \changes{russianb-1.1l}{1999/08/27}{Made not using inputenc a
%    warning instead of an error} 
%    \begin{macrocode}
\@ifpackageloaded{inputenc}{}{%
  \def\reserved@a{LWN}%
  \ifx\reserved@a\cyrillicencoding\else
    \def\reserved@a{OT2}%
    \ifx\reserved@a\cyrillicencoding\else
      \PackageWarning{babel}%
        {No input encoding specified for Russian language}
  \fi\fi}
%    \end{macrocode}
%
%    Now we define two commands that offer the possibility to switch between
%    Cyrillic and Roman encodings.
%
%  \begin{macro}{\cyrillictext}
%  \begin{macro}{\latintext}
%
%    The command |\cyrillictext| will switch from Latin font encoding to the
%    Cyrillic font encoding, the command |\latintext| switches back. This
%    assumes that the `normal' font encoding is a Latin one. These commands
%    are \emph{declarations}, for shorter peaces of text the commands
%    |\textlatin| and |\textcyrillic| can be used.
%
% \changes{russianb-1.1o}{2003/10/12}{\cs{latintext} is already
%    defined by the core of \babel}
%    \begin{macrocode}
%\DeclareRobustCommand{\latintext}{%
%  \fontencoding{\latinencoding}\selectfont
%  \def\encodingdefault{\latinencoding}}
\let\lat\latintext
%    \end{macrocode}
%
%  \end{macro}
%  \end{macro}
%
%  \begin{macro}{\textcyrillic}
%  \begin{macro}{\textlatin}
%
%    These commands take an argument which is then typeset using the
%    requested font encoding.
% \changes{russianb-1.1o}{2003/10/12}{\cs{textlatin} already defined
%    by the core of \babel}
%    \begin{macrocode}
\DeclareTextFontCommand{\textcyrillic}{\cyrillictext}
%\DeclareTextFontCommand{\textlatin}{\latintext}
%    \end{macrocode}
%
%  \end{macro}
%  \end{macro}
%
%    We make the \TeX
%    \begin{macrocode}
%\ifx\ltxTeX\undefined\let\ltxTeX\TeX\fi
%\ProvideTextCommandDefault{\TeX}{\textlatin{\ltxTeX}}
%    \end{macrocode}
%    and \LaTeX\ logos encoding independent.
%    \begin{macrocode}
%\ifx\ltxLaTeX\undefined\let\ltxLaTeX\LaTeX\fi
%\ProvideTextCommandDefault{\LaTeX}{\textlatin{\ltxLaTeX}}
%    \end{macrocode}
%
%    The next step consists of defining commands to switch to (and
%    from) the Russian language.
%
% \begin{macro}{\captionsrussian}
%
%    The macro |\captionsrussian| defines all strings used in the four
%    standard document classes provided with \LaTeX. The two commands |\cyr|
%    and |\lat| activate Cyrillic resp.\ Latin encoding.
%
%    \begin{macrocode}
\addto\captionsrussian{%
%   FIXME: Where is the \prefacename used?
  \def\prefacename{%
    {\cyr\CYRP\cyrr\cyre\cyrd\cyri\cyrs\cyrl\cyro\cyrv\cyri\cyre}}%
%   {\cyr\CYRV\cyrv\cyre\cyrd\cyre\cyrn\cyri\cyre}}%
  \def\refname{%
    {\cyr\CYRS\cyrp\cyri\cyrs\cyro\cyrk
      \ \cyrl\cyri\cyrt\cyre\cyrr\cyra\cyrt\cyru\cyrr\cyrery}}%
% \def\refname{%
%   {\cyr\CYRL\cyri\cyrt\cyre\cyrr\cyra\cyrt\cyru\cyrr\cyra}}%
  \def\abstractname{%
    {\cyr\CYRA\cyrn\cyrn\cyro\cyrt\cyra\cyrc\cyri\cyrya}}%
  \def\bibname{%
    {\cyr\CYRL\cyri\cyrt\cyre\cyrr\cyra\cyrt\cyru\cyrr\cyra}}%
% \def\bibname{%
%   {\cyr\CYRB\cyri\cyrb\cyrl\cyri\cyro
%    \cyrg\cyrr\cyra\cyrf\cyri\cyrya}}%
% for reports according to GOST:
% \def\bibname{%
%   {\cyr\CYRS\cyrp\cyri\cyrs\cyro\cyrk
%     \ \cyri\cyrs\cyrp\cyro\cyrl\cyrsftsn\cyrz\cyro\cyrv\cyra\cyrn
%     \cyrn\cyrery\cyrh\ \cyri\cyrs\cyrt\cyro\cyrch\cyrn\cyri
%     \cyrk\cyro\cyrv}}%
  \def\chaptername{{\cyr\CYRG\cyrl\cyra\cyrv\cyra}}%
% \@ifundefined{chapter}{}{%
%   \def\chaptername{{\cyr\CYRG\cyrl\cyra\cyrv\cyra}}}%
  \def\appendixname{%
    {\cyr\CYRP\cyrr\cyri\cyrl\cyro\cyrzh\cyre\cyrn\cyri\cyre}}%
%    \end{macrocode}
%
%    There are two names for the Table of Contents that are used in Russian
%    publications. For books (and reports) the second variant is
%    appropriate, but for proceedings the first variant is preferred:
%
%    \begin{macrocode}
  \@ifundefined{thechapter}%
    {\def\contentsname{%
      {\cyr\CYRS\cyro\cyrd\cyre\cyrr\cyrzh\cyra\cyrn\cyri\cyre}}}%
    {\def\contentsname{%
      {\cyr\CYRO\cyrg\cyrl\cyra\cyrv\cyrl\cyre\cyrn\cyri\cyre}}}%
  \def\listfigurename{%
    {\cyr\CYRS\cyrp\cyri\cyrs\cyro\cyrk
      \ \cyri\cyrl\cyrl\cyryu\cyrs\cyrt\cyrr\cyra\cyrc\cyri\cyrishrt}}%
% \def\listfigurename{%
%   {\cyr\CYRS\cyrp\cyri\cyrs\cyro\cyrk
%     \ \cyrr\cyri\cyrs\cyru\cyrn\cyrk\cyro\cyrv}}%
  \def\listtablename{%
    {\cyr\CYRS\cyrp\cyri\cyrs\cyro\cyrk
      \ \cyrt\cyra\cyrb\cyrl\cyri\cyrc}}%
  \def\indexname{%
    {\cyr\CYRP\cyrr\cyre\cyrd\cyrm\cyre\cyrt\cyrn\cyrery\cyrishrt
      \ \cyru\cyrk\cyra\cyrz\cyra\cyrt\cyre\cyrl\cyrsftsn}}%
  \def\authorname{%
    {\cyr\CYRI\cyrm\cyre\cyrn\cyrn\cyro\cyrishrt
      \ \cyru\cyrk\cyra\cyrz\cyra\cyrt\cyre\cyrl\cyrsftsn}}%
  \def\figurename{{\cyr\CYRR\cyri\cyrs.}}%
  \def\tablename{{\cyr\CYRT\cyra\cyrb\cyrl\cyri\cyrc\cyra}}%
  \def\partname{{\cyr\CYRCH\cyra\cyrs\cyrt\cyrsftsn}}%
  \def\enclname{{\cyr\cyrv\cyrk\cyrl.}}%
  \def\ccname{{\cyr\cyri\cyrs\cyrh.}}%
% \def\ccname{{\cyr\cyri\cyrz}}%
  \def\headtoname{{\cyr\cyrv\cyrh.}}%
% \def\headtoname{{\cyr\cyrv}}%
  \def\pagename{{\cyr\cyrs.}}%
% \def\pagename{{\cyr\cyrs\cyrt\cyrr.}}%
  \def\seename{{\cyr\cyrs\cyrm.}}%
  \def\alsoname{{\cyr\cyrs\cyrm.\ \cyrt\cyra\cyrk\cyrzh\cyre}}%
%    \end{macrocode}
% \changes{russianb-1.1m}{2000/09/20}{Added \cs{glossaryname}}
%    \begin{macrocode}
  \def\proofname{{\cyr\CYRD\cyro\cyrk\cyra\cyrz\cyra\cyrt
      \cyre\cyrl\cyrsftsn\cyrs\cyrt\cyrv\cyro}}%
  \def\glossaryname{Glossary}% <-- Needs translation
  }
%    \end{macrocode}
%
% \end{macro}
%
% \begin{macro}{\daterussian}
%
%    The macro |\daterussian| redefines the command |\today| to produce
%    Russian dates.
%
%    \begin{macrocode}
\def\daterussian{%
  \def\today{\number\day~\ifcase\month\or
    \cyrya\cyrn\cyrv\cyra\cyrr\cyrya\or
    \cyrf\cyre\cyrv\cyrr\cyra\cyrl\cyrya\or
    \cyrm\cyra\cyrr\cyrt\cyra\or
    \cyra\cyrp\cyrr\cyre\cyrl\cyrya\or
    \cyrm\cyra\cyrya\or
    \cyri\cyryu\cyrn\cyrya\or
    \cyri\cyryu\cyrl\cyrya\or
    \cyra\cyrv\cyrg\cyru\cyrs\cyrt\cyra\or
    \cyrs\cyre\cyrn\cyrt\cyrya\cyrb\cyrr\cyrya\or
    \cyro\cyrk\cyrt\cyrya\cyrb\cyrr\cyrya\or
    \cyrn\cyro\cyrya\cyrb\cyrr\cyrya\or
    \cyrd\cyre\cyrk\cyra\cyrb\cyrr\cyrya\fi
    \ \number\year~\cyrg.}}
%    \end{macrocode}
%
% \end{macro}
%
% \begin{macro}{\extrasrussian}
%
%    The macro |\extrasrussian| will perform all the extra definitions
%    needed for the Russian language. The macro |\noextrasrussian| is used
%    to cancel the actions of |\extrasrussian|.
%
% \changes{russianb-1.1b}{1996/02/20}{Added switch to \texttt{LWN}
%    encoding}
%
%    The first action we define is to switch on the selected Cyrillic
%    encoding whenever we enter `russian'.
%
%    \begin{macrocode}
\addto\extrasrussian{\cyrillictext}
%    \end{macrocode}
%
%    When the encoding definition file was processed by \LaTeX\ the current
%    font encoding is stored in |\latinencoding|, assuming that \LaTeX\ uses
%    \texttt{T1} or \texttt{OT1} as default. Therefore we switch back to
%    |\latinencoding| whenever the Russian language is no longer `active'.
%
%    \begin{macrocode}
\addto\noextrasrussian{\latintext}
%    \end{macrocode}
%
%  \begin{macro}{\verbatim@font}
%
% \changes{russianb-1.1b}{1996/02/20}{Added changing of
%    \cs{verbatim@font}}
%
%    In order to get both Latin and Cyrillic letters in verbatim text we
%    need to change the definition of an internal \LaTeX\ command somewhat:
%
%    \begin{macrocode}
%\def\verbatim@font{%
%  \let\encodingdefault\latinencoding
%  \normalfont\ttfamily
%  \expandafter\def\csname\cyrillicencoding-cmd\endcsname##1##2{%
%    \ifx\protect\@typeset@protect
%      \begingroup\UseTextSymbol\cyrillicencoding##1\endgroup
%    \else\noexpand##1\fi}}
%    \end{macrocode}
%
%  \end{macro}
%
%    The category code of the characters `\texttt{:}', `\texttt{;}',
%    `\texttt{!}', and `\texttt{?}' is made |\active| to insert a little
%    white space.
%
%    For Russian (as well as for German) the \texttt{"} character also is
%    made active.
%
%    Note: It is \emph{very} questionable whether the Russian typesetting
%    tradition requires additional spacing before those punctuation signs.
%    Therefore, we make the corresponding code optional. If you need it,
%    then define the \texttt{frenchpunct} docstrip option in
%    \file{babel.ins}.
%
% \changes{russianb-1.1f}{1998/06/26}{%
%    Added a hook to insert space
%    or not before `double punctuation' (from frenchb).}
%
%    Borrowed from french.
%    Some users dislike automatic insertion of a space before
%    `double punctuation', and prefer to decide themselves whether a
%    space should be added or not; so a hook |\NoAutoSpaceBeforeFDP|
%    is provided: if this command is added (in file |russianb.cfg|, or
%    anywhere in a document) |russianb| will respect your typing, and
%    introduce a suitable space before `double punctuation' \emph{if
%    and only if} a space is typed in the source file before those
%    signs.
%
%    The command |\AutoSpaceBeforeFDP| switches back to the
%    default behavior of |russianb|.
%
% \changes{russianb-1.1a}{1995/03/07}{Use the new mechanism for dealing
%    with active characters}
%
%    \begin{macrocode}
%<*frenchpunct>
\initiate@active@char{:}
\initiate@active@char{;}
%</frenchpunct>
%<*frenchpunct|spanishligs>
\initiate@active@char{!}
\initiate@active@char{?}
%</frenchpunct|spanishligs>
\initiate@active@char{"}
%    \end{macrocode}
%
%    The code above is necessary because we need extra active characters.
%    The character |"| is used as indicated in
%    table~\ref{tab:russian-quote}.
%
%    We specify that the Russian group of shorthands should be used.
%
%    \begin{macrocode}
\addto\extrasrussian{\languageshorthands{russian}}
%    \end{macrocode}
%
%    These characters are `turned on' once, later their definition may
%    vary.
%
%    \begin{macrocode}
\addto\extrasrussian{%
%<frenchpunct>  \bbl@activate{:}\bbl@activate{;}%
%<frenchpunct|spanishligs>  \bbl@activate{!}\bbl@activate{?}%
  \bbl@activate{"}}
\addto\noextrasrussian{%
%<frenchpunct>  \bbl@deactivate{:}\bbl@deactivate{;}%
%<frenchpunct|spanishligs>  \bbl@deactivate{!}\bbl@deactivate{?}%
  \bbl@deactivate{"}}
%    \end{macrocode}
%
%   The \texttt{X2} and \texttt{T2*} encodings do not contain
%   |spanish_shriek| and |spanish_query| symbols; as a consequence, the
%   ligatures `|?`|' and `|!`|' do not work with them (these characters are
%   useless for Cyrillic texts anyway). But we define the shorthands to
%   emulate these ligatures (optionally).
%
%   We do not use |\latinencoding| here (but instead explicitly use
%   \texttt{OT1}) because the user may choose \texttt{T2A} to be the primary
%   encoding, but it does not contain these characters.
%
%    \begin{macrocode}
%<*spanishligs>
\declare@shorthand{russian}{?`}{\UseTextSymbol{OT1}\textquestiondown}
\declare@shorthand{russian}{!`}{\UseTextSymbol{OT1}\textexclamdown}
%</spanishligs>
%    \end{macrocode}
%
% \begin{macro}{\russian@sh@;@}
% \begin{macro}{\russian@sh@:@}
% \begin{macro}{\russian@sh@!@}
% \begin{macro}{\russian@sh@?@}
%
%    We have to reduce the amount of white space before \texttt{;},
%    \texttt{:} and \texttt{!}. This should only happen in horizontal mode,
%    hence the test with |\ifhmode|.
%
% \changes{russianb-1.1a}{1995/07/04}{Use new \cs{DefineActiveNoArg}}
% \changes{russianb-1.1a}{1995/07/04}{Use the more general mechanism of
%    \cs{declare@shorthand}}
% \changes{russianb-1.1b}{1996/02/08}{Updated to reflect the latest
%    french definitions}
%
%    \begin{macrocode}
%<*frenchpunct>
\declare@shorthand{russian}{;}{%
  \ifhmode
%    \end{macrocode}
%
% \changes{russianb-1.1f}{1998/06/26}{%
%    \thinspace changed to kern.1em (space bit thinner)}
% \changes{russianb-1.1f}{1998/06/26}{%
%    Added a hook to insert space
%    or not before `double punctuation' (from frenchb).}
%
%    In horizontal mode we check for the presence of a `space', `unskip' if
%    it exists and place a |0.1em| kerning.
%
%    \begin{macrocode}
    \ifdim\lastskip>\z@
      \unskip\nobreak\kern.1em
    \else
%    \end{macrocode}
%    If no space has been typed, we add |\FDP@thinspace|
%    which will be
%    defined, up to the user's wishes, as an automatic added
%    thinspace, or as |\@empty|.
%
%    \begin{macrocode}
        \FDP@thinspace
    \fi
  \fi
%    \end{macrocode}
%
%    Now we can insert a `|;|' character.
%
%    \begin{macrocode}
  \string;}
%    \end{macrocode}
%
%    The other definitions are very similar.
%
%    \begin{macrocode}
\declare@shorthand{russian}{:}{%
  \ifhmode
    \ifdim\lastskip>\z@
      \unskip\nobreak\kern.1em
    \else
        \FDP@thinspace
    \fi
  \fi
  \string:}
%    \end{macrocode}
%
%    \begin{macrocode}
\declare@shorthand{russian}{!}{%
  \ifhmode
    \ifdim\lastskip>\z@
      \unskip\nobreak\kern.1em
    \else
        \FDP@thinspace
    \fi
  \fi
  \string!}
%    \end{macrocode}
%
%    \begin{macrocode}
\declare@shorthand{russian}{?}{%
  \ifhmode
    \ifdim\lastskip>\z@
      \unskip\nobreak\kern.1em
    \else
        \FDP@thinspace
    \fi
  \fi
  \string?}
%    \end{macrocode}
%
% \end{macro}
% \end{macro}
% \end{macro}
% \end{macro}
%
%
% \changes{russianb-1.1f}{1998/06/26}{%
%    Added a hook to insert space
%    or not before `double punctuation' (from frenchb).}
%  \begin{macro}{\AutoSpaceBeforeFDP}
%  \begin{macro}{\NoAutoSpaceBeforeFDP}
%  \begin{macro}{\FDP@thinspace}
%    |\FDP@thinspace| is defined as unbreakable
%    spaces if |\AutoSpaceBeforeFDP| is activated or as |\@empty| if
%    |\NoAutoSpaceBeforeFDP| is in use.
%    The default is |\AutoSpaceBeforeFDP|.
%    \begin{macrocode}
\def\AutoSpaceBeforeFDP{%
      \def\FDP@thinspace{\nobreak\kern.1em}}
\def\NoAutoSpaceBeforeFDP{\let\FDP@thinspace\@empty}
\AutoSpaceBeforeFDP
%    \end{macrocode}
%  \end{macro}
%  \end{macro}
%  \end{macro}
%
%  \begin{macro}{\FDPon}
%  \begin{macro}{\FDPoff}
% \changes{russianb-1.1f}{1998/06/26}{One more chance to avoid
%       spaces before double punctuation}
%
%     The next macros allow to switch on/off activeness of double
%     punctuation signs.
%
%    \begin{macrocode}
\def\FDPon{\bbl@activate{:}%
        \bbl@activate{;}%
        \bbl@activate{?}%
        \bbl@activate{!}}
\def\FDPoff{\bbl@deactivate{:}%
        \bbl@deactivate{;}%
        \bbl@deactivate{?}%
        \bbl@deactivate{!}}
%    \end{macrocode}
%  \end{macro}
%  \end{macro}
%
%  \begin{macro}{\system@sh@:@}
%  \begin{macro}{\system@sh@!@}
%  \begin{macro}{\system@sh@?@}
%  \begin{macro}{\system@sh@;@}
%
% \changes{russianb-1.1a}{1995/07/04}{Added system level shorthands}
%
%    When the active characters appear in an environment where their
%    Russian behaviour is not wanted they should give an `expected'
%    result. Therefore we define shorthands at system level as well.
%
%    \begin{macrocode}
\declare@shorthand{system}{:}{\string:}
\declare@shorthand{system}{;}{\string;}
%</frenchpunct>
%<*frenchpunct&!spanishligs>
\declare@shorthand{system}{!}{\string!}
\declare@shorthand{system}{?}{\string?}
%</frenchpunct&!spanishligs>
%    \end{macrocode}
%
%  \end{macro}
%  \end{macro}
%  \end{macro}
%  \end{macro}
%
%    To be able to define the function of `|"|', we first define a couple of
%    `support' macros.
%
%  \begin{macro}{\dq}
%
%    We save the original double quote character in |\dq| to keep it
%    available, the math accent |\"| can now be typed as `|"|'.
%
%    \begin{macrocode}
\begingroup \catcode`\"12
\def\reserved@a{\endgroup
  \def\@SS{\mathchar"7019 }
  \def\dq{"}}
\reserved@a
%    \end{macrocode}
%
%  \end{macro}
%
% \changes{russianb-1.1a}{1995/07/04}{Use \cs{ddot} instead of
%    \cs{@MATHUMLAUT}}
%
%    Now we can define the doublequote macros: german and french quotes.
%    We use definitions of these quotes made in babel.sty.
%    The french quotes are contained in the \texttt{T2*} encodings.
%
%    \begin{macrocode}
\declare@shorthand{russian}{"`}{\glqq}
\declare@shorthand{russian}{"'}{\grqq}
\declare@shorthand{russian}{"<}{\flqq}
\declare@shorthand{russian}{">}{\frqq}
%    \end{macrocode}
%
%    Some additional commands:
%
%    \begin{macrocode}
\declare@shorthand{russian}{""}{\hskip\z@skip}
\declare@shorthand{russian}{"~}{\textormath{\leavevmode\hbox{-}}{-}}
\declare@shorthand{russian}{"=}{\nobreak-\hskip\z@skip}
\declare@shorthand{russian}{"|}{%
  \textormath{\nobreak\discretionary{-}{}{\kern.03em}%
              \allowhyphens}{}}
%    \end{macrocode}
%
%    The next two macros for |"-| and |"---| are somewhat different.
%    We must check whether the second token is a hyphen character:
%
%    \begin{macrocode}
\declare@shorthand{russian}{"-}{%
%    \end{macrocode}
%
%    If the next token is `|-|', we typeset an emdash, otherwise a hyphen
%    sign:
%
%    \begin{macrocode}
  \def\russian@sh@tmp{%
    \if\russian@sh@next-\expandafter\russian@sh@emdash
    \else\expandafter\russian@sh@hyphen\fi
  }%
%    \end{macrocode}
%
%    \TeX\ looks for the next token after the first `|-|': the meaning of
%    this token is written to |\russian@sh@next| and |\russian@sh@tmp| is
%    called.
%
%    \begin{macrocode}
  \futurelet\russian@sh@next\russian@sh@tmp}
%    \end{macrocode}
%
%    Here are the definitions of hyphen and emdash. First the hyphen:
%
%    \begin{macrocode}
\def\russian@sh@hyphen{%
  \nobreak\-\bbl@allowhyphens}
%    \end{macrocode}
%
% \changes{russianb-1.1f}{1998/06/26}{%
%    Rearranging of cyrillic emdash stuff}
%
%    For the emdash definition, there are the two parameters: we must `eat'
%    two last hyphen signs of our emdash\dots :
%    \begin{macrocode}
\def\russian@sh@emdash#1#2{\cdash-#1#2}
%    \end{macrocode}
%  \begin{macro}{\cdash}
%    \dots\ these two parameters are useful for another macro:
%    |\cdash|:
%    \begin{macrocode}
%\ifx\cdash\undefined % should be defined earlier
\def\cdash#1#2#3{\def\tempx@{#3}%
\def\tempa@{-}\def\tempb@{~}\def\tempc@{*}%
 \ifx\tempx@\tempa@\@Acdash\else
  \ifx\tempx@\tempb@\@Bcdash\else
   \ifx\tempx@\tempc@\@Ccdash\else
    \errmessage{Wrong usage of cdash}\fi\fi\fi}
%    \end{macrocode}
%   second parameter (or third for |\cdash|) shows what kind of emdash
%   to create in next step
%      \begin{center}
%      \begin{tabular}{@{}p{.1\hsize}@{}p{.9\hsize}@{}}
%       |"---| & ordinary (plain) Cyrillic emdash inside text:
%       an unbreakable thinspace will be inserted before only in case of
%       a \textit{space} before the dash (it is necessary for dashes after
%       display maths formulae: there could be lists, enumerations etc.\
%       started with ``--- where $a$ is ...'' i.e., the dash starts a line).
%       (Firstly there were planned rather soft rules for user: he may put
%       a space before the dash or not.  But it is difficult to place this
%       thinspace automatically, i.e., by checking modes because after
%       display formulae \TeX{} uses horizontal mode.  Maybe there is a
%       misunderstanding?  Maybe there is another way?)  After a dash
%       a breakable thinspace is always placed; \\
%   \end{tabular}
%   \end{center}
%    \begin{macrocode}
% What is more grammatically: .2em or .2\fontdimen6\font ?
\def\@Acdash{\ifdim\lastskip>\z@\unskip\nobreak\hskip.2em\fi
  \cyrdash\hskip.2em\ignorespaces}%
%    \end{macrocode}
%      \begin{center}
%      \begin{tabular}{@{}p{.1\hsize}@{}p{.9\hsize}@{}}
%       |"--~| & emdash in compound names or surnames
%       (like Mendeleev--Klapeiron); this dash has no space characters
%       around; after the dash some space is added
%       |\exhyphenalty| \\
%   \end{tabular}
%   \end{center}
%    \begin{macrocode}
\def\@Bcdash{\leavevmode\ifdim\lastskip>\z@\unskip\fi
 \nobreak\cyrdash\penalty\exhyphenpenalty\hskip\z@skip\ignorespaces}%
%    \end{macrocode}
%      \begin{center}
%      \begin{tabular}{@{}p{.1\hsize}@{}p{.9\hsize}@{}}
%       |"--*| & for denoting direct speech (a space like |\enskip|
%       must follow the emdash); \\
%   \end{tabular}
%   \end{center}
%    \begin{macrocode}
\def\@Ccdash{\leavevmode
 \nobreak\cyrdash\nobreak\hskip.35em\ignorespaces}%
%\fi
%    \end{macrocode}
%  \end{macro}
%
%  \begin{macro}{\cyrdash}
%   Finally the macro for ``body'' of the Cyrillic emdash.
%   The |\cyrdash| macro will be defined in case this macro hasn't been
%   defined in a fontenc file.  For T2* fonts, cyrdash will be placed in
%   the code of the English emdash thus it uses ligature |---|.
%    \begin{macrocode}
% Is there an IF necessary?
\ifx\cyrdash\undefined
  \def\cyrdash{\hbox to.8em{--\hss--}}
\fi
%    \end{macrocode}
%  \end{macro}
%
% \changes{russianb-1.1f}{1998/06/26}{%
%    Add macro for thinspace between initials}
%
%    Here a really new macro---to place thinspace between initials.
%    This macro used instead of |\,| allows hyphenation in the following
%    surname.
%
% \changes{russianb-1.1r}{2004/11/21}{Removed the commanet character
%    before the next code line, see R3669}
%    \begin{macrocode}
\declare@shorthand{russian}{",}{\nobreak\hskip.2em\ignorespaces}
%    \end{macrocode}
%
% \changes{russianb-1.1f}{1998/06/26}{%
%    Add commands for switch on/off
%    doublequote activeness.  Borrowed from german.}
%
%  \begin{macro}{\mdqon}
%  \begin{macro}{\mdqoff}
%    All that's left to do now is to  define a couple of commands
%    for |"|.
%    \begin{macrocode}
\def\mdqon{\bbl@activate{"}}
\def\mdqoff{\bbl@deactivate{"}}
%    \end{macrocode}
%  \end{macro}
%  \end{macro}
%
%    The Russian hyphenation patterns can be used with |\lefthyphenmin|
%    and |\righthyphenmin| set to~2.
%
% \changes{russianb-1.1a}{1995/07/04}{use \cs{russianhyphenmins} to
%    store the correct values}
% \changes{russianb-1.1m}{2000/09/22}{Now use \cs{providehyphenmins} to
%    provide a default value}
%    \begin{macrocode}
\providehyphenmins{\CurrentOption}{\tw@\tw@}
% temporary hack:
\ifx\englishhyphenmins\undefined
  \def\englishhyphenmins{\tw@\thr@@}
\fi
%    \end{macrocode}
%
%    Now the action |\extrasrussian| has to execute is to make sure that the
%    command |\frenchspacing| is in effect. If this is not the case the
%    execution of |\noextrasrussian| will switch it off again.
%
%    \begin{macrocode}
\addto\extrasrussian{\bbl@frenchspacing}
\addto\noextrasrussian{\bbl@nonfrenchspacing}
%    \end{macrocode}
%
% \end{macro}
%
%    Next we add a new enumeration style for Russian manuscripts with
%    Cyrillic letters, and later on we define some math operator names in
%    accordance with Russian typesetting traditions.
%
%  \begin{macro}{\Asbuk}
%
%    We begin by defining |\Asbuk| which works like |\Alph|, but produces
%    (uppercase) Cyrillic letters intead of Latin ones. The letters YO,
%    ISHRT, HRDSN, ERY, and SFTSN are skipped, as usual for such
%    enumeration.
%
%    \begin{macrocode}
\def\Asbuk#1{\expandafter\@Asbuk\csname c@#1\endcsname}
\def\@Asbuk#1{\ifcase#1\or
  \CYRA\or\CYRB\or\CYRV\or\CYRG\or\CYRD\or\CYRE\or\CYRZH\or
  \CYRZ\or\CYRI\or\CYRK\or\CYRL\or\CYRM\or\CYRN\or\CYRO\or
  \CYRP\or\CYRR\or\CYRS\or\CYRT\or\CYRU\or\CYRF\or\CYRH\or
  \CYRC\or\CYRCH\or\CYRSH\or\CYRSHCH\or\CYREREV\or\CYRYU\or
  \CYRYA\else\@ctrerr\fi}
%    \end{macrocode}
%
%  \end{macro}
%
%  \begin{macro}{\asbuk}
%
%    The macro |\asbuk| is similar to |\alph|; it produces lowercase
%    Russian letters.
%
%    \begin{macrocode}
\def\asbuk#1{\expandafter\@asbuk\csname c@#1\endcsname}
\def\@asbuk#1{\ifcase#1\or
  \cyra\or\cyrb\or\cyrv\or\cyrg\or\cyrd\or\cyre\or\cyrzh\or
  \cyrz\or\cyri\or\cyrk\or\cyrl\or\cyrm\or\cyrn\or\cyro\or
  \cyrp\or\cyrr\or\cyrs\or\cyrt\or\cyru\or\cyrf\or\cyrh\or
  \cyrc\or\cyrch\or\cyrsh\or\cyrshch\or\cyrerev\or\cyryu\or
  \cyrya\else\@ctrerr\fi}
%    \end{macrocode}
%
%  \end{macro}
%
% Set up default Cyrillic math alphabets. To use Cyrillic letters in
% math mode user should load the |textmath| package \emph{before}
% loading fontenc package (or \babel).  Note, that by default Cyrillic
% letters are taken from upright font in math mode (unlike Latin
% letters).
%    \begin{macrocode}
%\RequirePackage{textmath}
\@ifundefined{sym\cyrillicencoding letters}{}{%
\SetSymbolFont{\cyrillicencoding letters}{bold}\cyrillicencoding
  \rmdefault\bfdefault\updefault
\DeclareSymbolFontAlphabet\cyrmathrm{\cyrillicencoding letters}
%    \end{macrocode}
%    And we need a few commands to be able to switch to different variants.
%    \begin{macrocode}
\DeclareMathAlphabet\cyrmathbf\cyrillicencoding
  \rmdefault\bfdefault\updefault
\DeclareMathAlphabet\cyrmathsf\cyrillicencoding
  \sfdefault\mddefault\updefault
\DeclareMathAlphabet\cyrmathit\cyrillicencoding
  \rmdefault\mddefault\itdefault
\DeclareMathAlphabet\cyrmathtt\cyrillicencoding
  \ttdefault\mddefault\updefault
%
\SetMathAlphabet\cyrmathsf{bold}\cyrillicencoding
  \sfdefault\bfdefault\updefault
\SetMathAlphabet\cyrmathit{bold}\cyrillicencoding
  \rmdefault\bfdefault\itdefault
}
%    \end{macrocode}
%
%    Some math functions in Russian math books have other names: e.g.,
%    \texttt{sinh} in Russian is written as \texttt{sh} etc. So we define a
%    number of new math operators.
%
%    |\sinh|:
%    \begin{macrocode}
\def\sh{\mathop{\operator@font sh}\nolimits}
%    \end{macrocode}
%    |\cosh|:
%    \begin{macrocode}
\def\ch{\mathop{\operator@font ch}\nolimits}
%    \end{macrocode}
%    |\tan|:
%    \begin{macrocode}
\def\tg{\mathop{\operator@font tg}\nolimits}
%    \end{macrocode}
%    |\arctan|:
%    \begin{macrocode}
\def\arctg{\mathop{\operator@font arctg}\nolimits}
%    \end{macrocode}
%    arcctg:
%    \begin{macrocode}
\def\arcctg{\mathop{\operator@font arcctg}\nolimits}
%    \end{macrocode}
%    The following macro conflicts with |\th| defined in Latin~1 encoding:
%
%    |\tanh|:
% \changes{russianb-1.1q}{2004/05/21}{Change definition of \cs{th}
%    only for this language}
%    \begin{macrocode}
\addto\extrasrussian{%
  \babel@save{\th}%
  \let\ltx@th\th
  \def\th{\textormath{\ltx@th}%
                     {\mathop{\operator@font th}\nolimits}}%
  }
%    \end{macrocode}
%    |\cot|:
%    \begin{macrocode}
\def\ctg{\mathop{\operator@font ctg}\nolimits}
%    \end{macrocode}
%    |\coth|:
%    \begin{macrocode}
\def\cth{\mathop{\operator@font cth}\nolimits}
%    \end{macrocode}
%    |\csc|:
%    \begin{macrocode}
\def\cosec{\mathop{\operator@font cosec}\nolimits}
%    \end{macrocode}
%
%    And finally some other Russian mathematical symbols:
%    \begin{macrocode}
\def\Prob{\mathop{\kern\z@\mathsf{P}}\nolimits}
\def\Variance{\mathop{\kern\z@\mathsf{D}}\nolimits}
\def\nod{\mathop{\cyrmathrm{\cyrn.\cyro.\cyrd.}}\nolimits}
\def\nok{\mathop{\cyrmathrm{\cyrn.\cyro.\cyrk.}}\nolimits}
\def\NOD{\mathop{\cyrmathrm{\CYRN\CYRO\CYRD}}\nolimits}
\def\NOK{\mathop{\cyrmathrm{\CYRN\CYRO\CYRK}}\nolimits}
\def\Proj{\mathop{\cyrmathrm{\CYRP\cyrr}}\nolimits}
%    \end{macrocode}
%
% This is for compatibility with older Russian packages.
%    \begin{macrocode}
\DeclareRobustCommand{\No}{%
   \ifmmode{\nfss@text{\textnumero}}\else\textnumero\fi}
%    \end{macrocode}
%
%    The macro |\ldf@finish| takes care of looking for a configuration file,
%    setting the main language to be switched on at |\begin{document}| and
%    resetting the category code of \texttt{@} to its original value.
%
% \changes{russianb-1.1d}{1996/11/03}{Now use \cs{ldf@finish} to wrap
%    up}
%
%    \begin{macrocode}
\ldf@finish{russian}
%</code>
%    \end{macrocode}
%
% \Finale
%%
%% \CharacterTable
%%  {Upper-case    \A\B\C\D\E\F\G\H\I\J\K\L\M\N\O\P\Q\R\S\T\U\V\W\X\Y\Z
%%   Lower-case    \a\b\c\d\e\f\g\h\i\j\k\l\m\n\o\p\q\r\s\t\u\v\w\x\y\z
%%   Digits        \0\1\2\3\4\5\6\7\8\9
%%   Exclamation   \!     Double quote  \"     Hash (number) \#
%%   Dollar        \$     Percent       \%     Ampersand     \&
%%   Acute accent  \'     Left paren    \(     Right paren   \)
%%   Asterisk      \*     Plus          \+     Comma         \,
%%   Minus         \-     Point         \.     Solidus       \/
%%   Colon         \:     Semicolon     \;     Less than     \<
%%   Equals        \=     Greater than  \>     Question mark \?
%%   Commercial at \@     Left bracket  \[     Backslash     \\
%%   Right bracket \]     Circumflex    \^     Underscore    \_
%%   Grave accent  \`     Left brace    \{     Vertical bar  \|
%%   Right brace   \}     Tilde         \~}
%%
\endinput
}
\bbl@tempa{UKenglish}{%%
%% This file will generate fast loadable files and documentation
%% driver files from the doc files in this package when run through
%% LaTeX or TeX.
%%
%% Copyright 1989-2005 Johannes L. Braams and any individual authors
%% listed elsewhere in this file.  All rights reserved.
%% 
%% This file is part of the Babel system.
%% --------------------------------------
%% 
%% It may be distributed and/or modified under the
%% conditions of the LaTeX Project Public License, either version 1.3
%% of this license or (at your option) any later version.
%% The latest version of this license is in
%%   http://www.latex-project.org/lppl.txt
%% and version 1.3 or later is part of all distributions of LaTeX
%% version 2003/12/01 or later.
%% 
%% This work has the LPPL maintenance status "maintained".
%% 
%% The Current Maintainer of this work is Johannes Braams.
%% 
%% The list of all files belonging to the LaTeX base distribution is
%% given in the file `manifest.bbl. See also `legal.bbl' for additional
%% information.
%% 
%% The list of derived (unpacked) files belonging to the distribution
%% and covered by LPPL is defined by the unpacking scripts (with
%% extension .ins) which are part of the distribution.
%%
%% --------------- start of docstrip commands ------------------
%%
\def\filedate{1999/04/11}
\def\batchfile{english.ins}
\input docstrip.tex

{\ifx\generate\undefined
\Msg{**********************************************}
\Msg{*}
\Msg{* This installation requires docstrip}
\Msg{* version 2.3c or later.}
\Msg{*}
\Msg{* An older version of docstrip has been input}
\Msg{*}
\Msg{**********************************************}
\errhelp{Move or rename old docstrip.tex.}
\errmessage{Old docstrip in input path}
\batchmode
\csname @@end\endcsname
\fi}

\declarepreamble\mainpreamble
This is a generated file.

Copyright 1989-2005 Johannes L. Braams and any individual authors
listed elsewhere in this file.  All rights reserved.

This file was generated from file(s) of the Babel system.
---------------------------------------------------------

It may be distributed and/or modified under the
conditions of the LaTeX Project Public License, either version 1.3
of this license or (at your option) any later version.
The latest version of this license is in
  http://www.latex-project.org/lppl.txt
and version 1.3 or later is part of all distributions of LaTeX
version 2003/12/01 or later.

This work has the LPPL maintenance status "maintained".

The Current Maintainer of this work is Johannes Braams.

This file may only be distributed together with a copy of the Babel
system. You may however distribute the Babel system without
such generated files.

The list of all files belonging to the Babel distribution is
given in the file `manifest.bbl'. See also `legal.bbl for additional
information.

The list of derived (unpacked) files belonging to the distribution
and covered by LPPL is defined by the unpacking scripts (with
extension .ins) which are part of the distribution.
\endpreamble

\declarepreamble\fdpreamble
This is a generated file.

Copyright 1989-2005 Johannes L. Braams and any individual authors
listed elsewhere in this file.  All rights reserved.

This file was generated from file(s) of the Babel system.
---------------------------------------------------------

It may be distributed and/or modified under the
conditions of the LaTeX Project Public License, either version 1.3
of this license or (at your option) any later version.
The latest version of this license is in
  http://www.latex-project.org/lppl.txt
and version 1.3 or later is part of all distributions of LaTeX
version 2003/12/01 or later.

This work has the LPPL maintenance status "maintained".

The Current Maintainer of this work is Johannes Braams.

This file may only be distributed together with a copy of the Babel
system. You may however distribute the Babel system without
such generated files.

The list of all files belonging to the Babel distribution is
given in the file `manifest.bbl'. See also `legal.bbl for additional
information.

In particular, permission is granted to customize the declarations in
this file to serve the needs of your installation.

However, NO PERMISSION is granted to distribute a modified version
of this file under its original name.

\endpreamble

\keepsilent

\usedir{tex/generic/babel} 

\usepreamble\mainpreamble
\generate{\file{english.ldf}{\from{english.dtx}{code}}
          }
\usepreamble\fdpreamble

\ifToplevel{
\Msg{***********************************************************}
\Msg{*}
\Msg{* To finish the installation you have to move the following}
\Msg{* files into a directory searched by TeX:}
\Msg{*}
\Msg{* \space\space All *.def, *.fd, *.ldf, *.sty}
\Msg{*}
\Msg{* To produce the documentation run the files ending with}
\Msg{* '.dtx' and `.fdd' through LaTeX.}
\Msg{*}
\Msg{* Happy TeXing}
\Msg{***********************************************************}
}
 
\endinput
}
\bbl@tempa{ukrainian}{% \iffalse meta-commen

% Copyright 1989-2008 Johannes L. Braams and any individual author
% listed elsewhere in this file.  All rights reserved
%
% This file is part of the Babel system
% -------------------------------------
%
% It may be distributed and/or modified under th
% conditions of the LaTeX Project Public License, either version 1.
% of this license or (at your option) any later version
% The latest version of this license is i
%   http://www.latex-project.org/lppl.tx
% and version 1.3 or later is part of all distributions of LaTe
% version 2003/12/01 or later
%
% This work has the LPPL maintenance status "maintained"
%
% The Current Maintainer of this work is Johannes Braams
%
% The list of all files belonging to the Babel system i
% given in the file `manifest.bbl. See also `legal.bbl' for additiona
% information
%
% The list of derived (unpacked) files belonging to the distributio
% and covered by LPPL is defined by the unpacking scripts (wit
% extension .ins) which are part of the distribution
% \f
% \CheckSum{1472

% \iffals
%    Tell the \LaTeX\ system who we are and write an entry on th
%    transcript
%<*dtx
\ProvidesFile{ukraineb.dtx
%</dtx
%<code>\ProvidesLanguage{ukraineb
        [2008/03/21 v1.1l Ukrainian support from the babel system

%% File `ukraineb.dtx
%% Babel package for LaTeX version 2
%% Copyright (C) 1989 - 200
%%           by Johannes Braams, TeXnie

%% ukraineb Language Definition Fil
%% Copyright (C) 1997 - 200
%%           by Andrij Shvaika ashv at icmp.lviv.u

%% derived from the Russianb Language Definition Fil
%% Copyright (C) 1995 - 200
%%           by Olga Lapko cyrtug at mir.msk.s
%%              Johannes Braams, TeXnie
% adapted to the new T2 and X2 Cyrillic encoding
%           by Vladimir Volovich TeX at vvv.vsu.r
%              Werner Lemberg wl at gnu.or

%% Please report errors to: J.L. Braam
%%                          babel at braams.xs4all.n

%<*filedriver
\documentclass{ltxdoc
\newcommand\TeXhax{\TeX hax
\newcommand\babel{\textsf{babel}
\newcommand\langvar{$\langle \it lang \rangle$
\newcommand\note[1]{
\newcommand\Lopt[1]{\textsf{#1}
\newcommand\file[1]{\texttt{#1}
\newcommand\pkg[1]{\texttt{#1}
\begin{document
 \DocInput{ukraineb.dtx
\end{document
%</filedriver
%\f
% \GetFileInfo{ukraineb.dtx

% \changes{ukraineb-1.1e}{1999/08/19}{replaced all \cs{penalty}\cs{@M
%    with \cs{nobreak}

%  \section{The Ukrainian language

%    The file \file{\filename}\footnote{The file described in thi
%    section has version number \fileversion
%    This file was derived from the \file{russianb.dtx} version 1.1g.
%    defines all the language-specific macros for the Ukrainia
%    language. It needs the file \file{cyrcod} for success documentatio
%    with Ukrainian encodings (see below)

%    For this language the character |"| is made active. I
%    table~\ref{tab:ukrainian-quote} an overview is given of it
%    purpose.

%    \begin{table}[htb
%      \begin{center
%      \begin{tabular}{lp{8cm}
%       \verb="|= & disable ligature at this position.               \
%       |"-| & an explicit hyphen sign, allowing hyphenatio
%                   in the rest of the word.                         \
%       |"---| & Cyrillic emdash in plain text.                      \
%       |"--~| & Cyrillic emdash in compound names (surnames).       \
%       |"--*| & Cyrillic emdash for denoting direct speech.         \
%       |""| & like |"-|, but producing no hyphen sig
%                   (for compund words with hyphen, e.g.\ |x-""y
%                   or some other signs  as ``disable/enable'').     \
%       |"~| & for a compound word mark without a breakpoint.        \
%       |"=| & for a compound word mark with a breakpoint, allowin
%              hyphenation in the composing words.                   \
%       |",| & thinspace for initials with a breakpoin
%               in following surname.                                \
%       |"`| & for German left double quote
%                   (looks like ,\kern-0.08em,).                     \
%       |"'| & for German right double quotes (looks like ``).       \\%^^A'
%       |"<| & for French left double quotes (looks like $<\!\!<$).  \
%       |">| & for French right double quotes (looks like $>\!\!>$). \
%      \end{tabular
%      \caption{The extra definitions mad
%               by \file{ukraineb}}\label{tab:ukrainian-quote
%      \end{center
%    \end{table

%    The quotes in table~\ref{tab:ukrainian-quote} (see, als
%    table~\ref{tab:russian-quote}) can also be typeset by using the command
%    in table~\ref{tab:umore-quote} (see, also table~\ref{tab:rmore-quote})

%    \begin{table}[htb
%      \begin{center
%      \begin{tabular}{lp{8cm}
%       |\cdash---| & Cyrillic emdash in plain text.                    \
%       |\cdash--~| & Cyrillic emdash in compound names (surnames).     \
%       |\cdash--*| & Cyrillic emdash for denoting direct speech.       \
%       |\glqq| & for German left double quote
%                    (looks like ,\kern-0.08em,).                       \
%       |\grqq| & for German right double quotes (looks like ``).       \\%^^A'
%       |\flqq| & for French left double quotes (looks like $<\!\!<$).  \
%       |\frqq| & for French right double quotes (looks like $>\!\!>$). \
%       |\dq|   & the original quotes character (|"|).                  \
%      \end{tabular
%      \caption{More commands which produce quotes, define
%               by \babel}\label{tab:umore-quote
%      \end{center
%    \end{table

%    The French quotes are also available as ligatures `|<<|' and `|>>|' i
%    8-bit Cyrillic font encodings (\texttt{LCY}, \texttt{X2}, \texttt{T2*}
%    and as `|<|' and `|>|' characters in 7-bit Cyrillic font encoding
%    (\texttt{OT2} and \texttt{LWN})

%    The quotation marks traditionally used in Ukrainian and Russia
%    languages were borrowed from other languages (e.g. French and German
%    so they keep their original names

% \StopEventually{

%    The macro |\LdfInit| takes care of preventing that this file is loade
%    more than once, checking the category code of the \texttt{@} sign, etc

%    \begin{macrocode
%<*code
\LdfInit{ukrainian}{captionsukrainian
%    \end{macrocode

%    When this file is read as an option, i.e., by the |\usepackage
%    command, \texttt{ukraineb} will be an `unknown' language, in which cas
%    we have to make it known. So we check for the existence of |\l@ukrainian
%    to see whether we have to do something here

%    \begin{macrocode
\ifx\l@ukrainian\@undefine
  \@nopatterns{Ukrainian
  \adddialect\l@ukrainian
\f
%    \end{macrocode

%  \begin{macro}{\latinencoding

%    We need to know the encoding for text that is supposed to be which i
%    active at the end of the \babel\ package. If the \pkg{fontenc} packag
%    is loaded later, then\ldots too bad

%    \begin{macrocode
\let\latinencoding\cf@encodin
%    \end{macrocode

%  \end{macro

%    The user may choose between different available Cyrilli
%    encodings---e.g., \texttt{X2}, \texttt{LCY}, or \texttt{LWN}.\
%    Hopefully, \texttt{X2} will eventually replace the two latter encoding
%    (\texttt{LCY} and \texttt{LWN}).\@ If the user wants to use anothe
%    font encoding than the default (\texttt{T2A}), he has to load th
%    corresponding file \emph{before} \file{ukraineb.sty}. This may be don
%    in the following way

%    \begin{verbatim
%      % override the default X2 encoding used in Babe
%      \usepackage[LCY,OT1]{fontenc
%      \usepackage[english,ukrainian]{babel
%    \end{verbatim
%    \unski

%    Note: for the Ukrainian language, the \texttt{T2A} encoding is better tha
%    \texttt{X2}, because \texttt{X2} does not contain Latin letters, an
%    users should be very careful to switch the language every time the
%    want to typeset a Latin word inside a Ukrainian phrase or vice versa

%    We parse the |\cdp@list| containing the encodings known to \LaTeX\ i
%    the order they were loaded. We set the |\cyrillicencoding| to th
%    \emph{last} loaded encoding in the list of supported Cyrilli
%    encodings: \texttt{OT2}, \texttt{LWN}, \texttt{LCY}, \texttt{X2}
%    \texttt{T2C}, \texttt{T2B}, \texttt{T2A}, if any

%    \begin{macrocode
\def\reserved@a#1#2{
   \edef\reserved@b{#1}
   \edef\reserved@c{#2}
   \ifx\reserved@b\reserved@
     \let\cyrillicencoding\reserved@
   \fi
\def\cdp@elt#1#2#3#4{
   \reserved@a{#1}{OT2}
   \reserved@a{#1}{LWN}
   \reserved@a{#1}{LCY}
   \reserved@a{#1}{X2}
   \reserved@a{#1}{T2C}
   \reserved@a{#1}{T2B}
   \reserved@a{#1}{T2A}
\cdp@lis
%    \end{macrocode

%    Now, if |\cyrillicencoding| is undefined, then the user did not loa
%    any of supported encodings. So, we have to set |\cyrillicencoding| t
%    some default value. We test the presence of the encoding definitio
%    files in the order from less preferable to more preferable encodings
%    We use the lowercase names (i.e., \file{lcyenc.def} instead o
%    \file{LCYenc.def})

%    \begin{macrocode
\ifx\cyrillicencoding\undefine
  \IfFileExists{ot2enc.def}{\def\cyrillicencoding{OT2}}\rela
  \IfFileExists{lwnenc.def}{\def\cyrillicencoding{LWN}}\rela
  \IfFileExists{lcyenc.def}{\def\cyrillicencoding{LCY}}\rela
  \IfFileExists{x2enc.def}{\def\cyrillicencoding{X2}}\rela
  \IfFileExists{t2cenc.def}{\def\cyrillicencoding{T2C}}\rela
  \IfFileExists{t2benc.def}{\def\cyrillicencoding{T2B}}\rela
  \IfFileExists{t2aenc.def}{\def\cyrillicencoding{T2A}}\rela
%    \end{macrocode

%    If |\cyrillicencoding| is still undefined, then the user seems not t
%    have a properly installed distribution. A fatal error

%    \begin{macrocode
  \ifx\cyrillicencoding\undefine
    \PackageError{babel}
      {No Cyrillic encoding definition files were found}
      {Your installation is incomplete.\MessageBrea
       You need at least one of the following files:\MessageBrea
       \space\spac
       x2enc.def, t2aenc.def, t2benc.def, t2cenc.def,\MessageBrea
       \space\spac
       lcyenc.def, lwnenc.def, ot2enc.def.}
  \els
%    \end{macrocode

%    We avoid |\usepackage[\cyrillicencoding]{fontenc}| because we don'
%    want to force the switch of |\encodingdefault|

%    \begin{macrocode
    \lowercas
      \expandafter{\expandafter\input\cyrillicencoding enc.def\relax}
  \f
\f
%    \end{macrocode

%    \begin{verbatim
%      \PackageInfo{babel
%        {Using `\cyrillicencoding' as a default Cyrillic encoding}
%    \end{verbatim
%    \unski

%    \begin{macrocode
\DeclareRobustCommand{\Ukrainian}{
  \fontencoding\cyrillicencoding\selectfon
  \let\encodingdefault\cyrillicencodin
  \expandafter\set@hyphenmins\ukrainianhyphenmin
  \language\l@ukrainian}
\DeclareRobustCommand{\English}{
  \fontencoding\latinencoding\selectfon
  \let\encodingdefault\latinencodin
  \expandafter\set@hyphenmins\englishhyphenmin
  \language\l@english}
\let\Ukr\Ukrainia
\let\Eng\Englis
\let\cyrillictext\Ukrainia
\let\cyr\Ukrainia
%    \end{macrocode

%    Since the \texttt{X2} encoding does not contain Latin letters, w
%    should make some redefinitions of \LaTeX\ macros which implicitl
%    produce Latin letters

%    \begin{macrocode
\expandafter\ifx\csname T@X2\endcsname\relax\els
%    \end{macrocode

%    We put |\latinencoding| in braces to avoid problems wit
%    |\@alph| inside minipages (e.g., footnotes inside minipages) wher
%    |\@alph| is expanded and we get for example `|\fontencoding OT1|
%    (|\fontencoding| is robust)

%    \begin{macrocode
  \def\@alph#1{{\fontencoding{\latinencoding}\selectfon
    \ifcase#1\o
      a\or b\or c\or d\or e\or f\or g\or h\o
      i\or j\or k\or l\or m\or n\or o\or p\o
      q\or r\or s\or t\or u\or v\or w\or x\o
      y\or z\else\@ctrerr\fi}}
  \def\@Alph#1{{\fontencoding{\latinencoding}\selectfon
    \ifcase#1\o
      A\or B\or C\or D\or E\or F\or G\or H\o
      I\or J\or K\or L\or M\or N\or O\or P\o
      Q\or R\or S\or T\or U\or V\or W\or X\o
      Y\or Z\else\@ctrerr\fi}}
%    \end{macrocode

%    Unfortunately, the commands |\AA| and |\aa| are not encoding dependen
%    in \LaTeX\ (unlike e.g., |\oe| or |\DH|). They are defined as |\r{A}| an
%    |\r{a}|. This leads to unpredictable results when the font encodin
%    does not contain the Latin letters `A' and `a' (like \texttt{X2})

%    \begin{macrocode
  \DeclareTextSymbolDefault{\AA}{OT1
  \DeclareTextSymbolDefault{\aa}{OT1
  \DeclareTextCommand{\aa}{OT1}{\r a
  \DeclareTextCommand{\AA}{OT1}{\r A
\f
%    \end{macrocode

%    The following block redefines the character class of uppercase Gree
%    letters and some accents, if it is equal to 7 (variable family), t
%    avoid incorrect results if the font encoding in some math family doe
%    not contain these characters in places of OT1 encoding. The code wa
%    taken from |amsmath.dtx|. See comments and further explanation there

% \changes{ukraineb-1.1i}{2001/02/21}{As this code generates
%    textfont 7 error it is commented out for now.
%    \begin{macrocode
% \begingroup\catcode`\"=1
% % uppercase greek letters
% \def\@tempa#1{\expandafter\@tempb\meaning#1\relax\relax\relax\rela
%   "0000\@nil#1
% \def\@tempb#1"#2#3#4#5#6\@nil#7{
%   \ifnum"#2=7 \count@"1#3#4#5\rela
%     \ifnum\count@<"1000 \else \global\mathchardef#7="0#3#4#5\relax \f
%   \fi
% \@tempa\Gamma\@tempa\Delta\@tempa\Theta\@tempa\Lambda\@tempa\X
% \@tempa\Pi\@tempa\Sigma\@tempa\Upsilon\@tempa\Phi\@tempa\Ps
% \@tempa\Omeg
% % some accents
% \def\@tempa#1#2\@nil{\def\@tempc{#1}}\def\@tempb{\mathaccent
% \expandafter\@tempa\hat\relax\relax\@ni
% \ifx\@tempb\@temp
%   \def\@tempa#1\@nil{#1}
%   \def\@tempb#1{\afterassignment\@tempa\mathchardef\@tempc=}
%   \def\do#1"#2{
%   \def\@tempd#1{\expandafter\@tempb#1\@ni
%     \ifnum\@tempc>"FF
%       \xdef#1{\mathaccent"\expandafter\do\meaning\@tempc\space}
%     \fi
%   \@tempd\hat\@tempd\check\@tempd\tilde\@tempd\acute\@tempd\grav
%   \@tempd\dot\@tempd\ddot\@tempd\breve\@tempd\ba
% \f
% \endgrou
%    \end{macrocode

%    The user must use the \pkg{inputenc} package when any 8-bit Cyrilli
%    font encoding is used, selecting one of the Cyrillic input encodings
%    We do not assume any default input encoding, so the user shoul
%    explicitly call the \pkg{inputenc} package by |\usepackage{inputenc}|
%    We also removed |\AtBeginDocument|, so \pkg{inputenc} should be use
%    before \babel

% \changes{ukraineb-1.1f}{1999/08/27}{Made not using inputenc
%    warning instead of an error
%    \begin{macrocode
\@ifpackageloaded{inputenc}{}{
  \def\reserved@a{LWN}
  \ifx\reserved@a\cyrillicencoding\els
    \def\reserved@a{OT2}
    \ifx\reserved@a\cyrillicencoding\els
      \PackageWarning{babel}
        {No input encoding specified for Ukrainian language
  \fi\fi
%    \end{macrocode

%    Now we define two commands that offer the possibility to switch betwee
%    Cyrillic and Roman encodings

%  \begin{macro}{\cyrillictext
%  \begin{macro}{\latintext

%    The command |\cyrillictext| will switch from Latin font encoding to th
%    Cyrillic font encoding, the command |\latintext| switches back. Thi
%    assumes that the `normal' font encoding is a Latin one. These command
%    are \emph{declarations}, for shorter peaces of text the command
%    |\textlatin| and |\textcyrillic| can be used

% \changes{ukraineb-1.1j}{2003/10/12}{\cs{latintext} is alread
%    defined by the core of \babel
%    \begin{macrocode
%\DeclareRobustCommand{\latintext}{
%  \fontencoding{\latinencoding}\selectfon
%  \def\encodingdefault{\latinencoding}
\let\lat\latintex
%    \end{macrocode

%  \end{macro
%  \end{macro

%  \begin{macro}{\textcyrillic
%  \begin{macro}{\textlatin

%    These commands take an argument which is then typeset using th
%    requested font encoding
% \changes{ukraineb-1.1j}{2003/10/12}{\cs{latintext} is alread
%    defined by the core of \babel
%    \begin{macrocode
\DeclareTextFontCommand{\textcyrillic}{\cyrillictext
%\DeclareTextFontCommand{\textlatin}{\latintext
%    \end{macrocode

%  \end{macro
%  \end{macro

%    We make the \Te
%    \begin{macrocode
%\ifx\ltxTeX\undefined\let\ltxTeX\TeX\f
%\ProvideTextCommandDefault{\TeX}{\textlatin{\ltxTeX}
%    \end{macrocode
%    and \LaTeX\ logos encoding independent
%    \begin{macrocode
%\ifx\ltxLaTeX\undefined\let\ltxLaTeX\LaTeX\f
%\ProvideTextCommandDefault{\LaTeX}{\textlatin{\ltxLaTeX}
%    \end{macrocode

%    The next step consists of defining commands to switch to (an
%    from) the Ukrainian language

% \begin{macro}{\captionsukrainian

%    The macro |\captionsukrainian| defines all strings used in the fou
%    standard document classes provided with \LaTeX. The two commands |\cyr
%    and |\lat| activate Cyrillic resp.\ Latin encoding
% \changes{ukraineb-1.1d}{1999/04/03}{replace \cs{CYRUKRI} wit
%    \cs{CYRII} in \cs{authorname}
% \changes{ukraineb-1.1g}{2000/09/20}{Added \cs{glossaryname}
% \changes{ukraineb-1.1h}{2001/02/13}{Added translation fo
%    `Glossary'
%    \begin{macrocode
\addto\captionsukrainian{
  \def\prefacename{{\cyr\CYRV\cyrs\cyrt\cyru\cyrp}}
% \def\prefacename{{\cyr\CYRP\cyre\cyrr\cyre\cyrd\cyrm\cyro\cyrv\cyra}}
  \def\refname{
    {\cyr\CYRL\cyrii\cyrt\cyre\cyrr\cyra\cyrt\cyru\cyrr\cyra}}
%  \def\refname{
%    {\cyr\CYRP\cyre\cyrr\cyre\cyrl\cyrii\cyr
%         \ \cyrp\cyro\cyrs\cyri\cyrl\cyra\cyrn\cyrsftsn}}
  \def\abstractname{
    {\cyr\CYRA\cyrn\cyro\cyrt\cyra\cyrc\cyrii\cyrya}}
%  \def\abstractname{{\cyr\CYRR\cyre\cyrf\cyre\cyrr\cyra\cyrt}}
  \def\bibname{
    {\cyr\CYRB\cyrii\cyrb\cyrl\cyrii\cyro\cyrgup\cyrr\cyra\cyrf\cyrii\cyrya}}
% \def\bibname{{\cyr\CYRL\cyrii\cyrt\cyre\cyrr\cyra\cyrt\cyru\cyrr\cyra}}
  \def\chaptername{{\cyr\CYRR\cyro\cyrz\cyrd\cyrii\cyrl}}
%  \def\chaptername{{\cyr\CYRG\cyrl\cyra\cyrv\cyra}}
  \def\appendixname{{\cyr\CYRD\cyro\cyrd\cyra\cyrt\cyro\cyrk}}
  \def\contentsname{{\cyr\CYRZ\cyrm\cyrii\cyrs\cyrt}}
  \def\listfigurename{{\cyr\CYRP\cyre\cyrr\cyre\cyrl\cyrii\cyr
         \ \cyrii\cyrl\cyryu\cyrs\cyrt\cyrr\cyra\cyrc\cyrii\cyrishrt}}
  \def\listtablename{{\cyr\CYRP\cyre\cyrr\cyre\cyrl\cyrii\cyr
         \ \cyrt\cyra\cyrb\cyrl\cyri\cyrc\cyrsftsn}}
  \def\indexname{{\cyr\CYRP\cyro\cyrk\cyra\cyrzh\cyrch\cyri\cyrk}}
  \def\authorname{{\cyr\CYRII\cyrm\cyre\cyrn\cyrn\cyri\cyrishr
         \ \cyrp\cyro\cyrk\cyra\cyrzh\cyrch\cyri\cyrk}}
  \def\figurename{{\cyr\CYRR\cyri\cyrs.}}
%  \def\figurename{\cyr\CYRR\cyri\cyrs\cyru\cyrn\cyro\cyrk}}
  \def\tablename{{\cyr\CYRT\cyra\cyrb\cyrl.}}
%  \def\tablename{\cyr\CYRT\cyra\cyrb\cyrl\cyri\cyrc\cyrya}}
  \def\partname{{\cyr\CYRCH\cyra\cyrs\cyrt\cyri\cyrn\cyra}}
  \def\enclname{{\cyr\cyrv\cyrk\cyrl\cyra\cyrd\cyrk\cyra}}
  \def\ccname{{\cyr\cyrk\cyro\cyrp\cyrii\cyrya}}
  \def\headtoname{{\cyr\CYRD\cyro}}
  \def\pagename{{\cyr\cyrs.}}
%  \def\pagename{{\cyr\cyrs\cyrt\cyro\cyrr\cyrii\cyrn\cyrk\cyra}}
  \def\seename{{\cyr\cyrd\cyri\cyrv.}}
  \def\alsoname{{\cyr\cyrd\cyri\cyrv.\ \cyrt\cyra\cyrk\cyro\cyrzh}
  \def\proofname{{\cyr\CYRD\cyro\cyrv\cyre\cyrd\cyre\cyrn\cyrn\cyrya}}
  \def\glossaryname{{\cyr\CYRS\cyrl\cyro\cyrv\cyrn\cyri\cyrk\
                   \cyrt\cyre\cyrr\cyrm\cyrii\cyrn\cyrii\cyrv}}

%    \end{macrocode

% \end{macro

% \begin{macro}{\dateukrainian

%    The macro |\dateukrainian| redefines the command |\today| to produc
%    Ukrainian dates

%    \begin{macrocode
\def\dateukrainian{
  \def\today{\number\day~\ifcase\month\o
    \cyrs\cyrii\cyrch\cyrn\cyrya\o
    \cyrl\cyryu\cyrt\cyro\cyrg\cyro\o
    \cyrb\cyre\cyrr\cyre\cyrz\cyrn\cyrya\o
    \cyrk\cyrv\cyrii\cyrt\cyrn\cyrya\o
    \cyrt\cyrr\cyra\cyrv\cyrn\cyrya\o
    \cyrch\cyre\cyrr\cyrv\cyrn\cyrya\o
    \cyrl\cyri\cyrp\cyrn\cyrya\o
    \cyrs\cyre\cyrr\cyrp\cyrn\cyrya\o
    \cyrv\cyre\cyrr\cyre\cyrs\cyrn\cyrya\o
    \cyrzh\cyro\cyrv\cyrt\cyrn\cyrya\o
    \cyrl\cyri\cyrs\cyrt\cyro\cyrp\cyra\cyrd\cyra\o
    \cyrg\cyrr\cyru\cyrd\cyrn\cyrya\f
    \space\number\year~\cyrr.}
%    \end{macrocode

% \end{macro

% \begin{macro}{\extrasukrainian

%    The macro |\extrasukrainian| will perform all the extra definition
%    needed for the Ukrainian language. The macro |\noextrasukrainian
%    is used to cancel the actions of |\extrasukrainian|

%    The first action we define is to switch on the selected Cyrilli
%    encoding whenever we enter `ukrainian'

%    \begin{macrocode
\addto\extrasukrainian{\cyrillictext
%    \end{macrocode

%    When the encoding definition file was processed by \LaTeX\ the curren
%    font encoding is stored in |\latinencoding|, assuming that \LaTeX\ use
%    \texttt{T1} or \texttt{OT1} as default. Therefore we switch back t
%    |\latinencoding| whenever the Ukrainian language is no longer `active'

%    \begin{macrocode
\addto\noextrasukrainian{\latintext
%    \end{macrocode

%    Next we must allow hyphenation in the Ukrainian words with apostroph
%    whenever we enter `ukrainian'. This solution was proposed b
%    Vladimir Volovich <vvv@vvv.vsu.ru

%    \begin{macrocode
\addto\extrasukrainian{\lccode`\'=`\'
\addto\noextrasukrainian{\lccode`\'=0
%    \end{macrocode

%  \begin{macro}{\verbatim@font

%    In order to get both Latin and Cyrillic letters in verbatim text w
%    need to change the definition of an internal \LaTeX\ command somewhat

%    \begin{macrocode
%\def\verbatim@font{
%  \let\encodingdefault\latinencodin
%  \normalfont\ttfamil
%  \expandafter\def\csname\cyrillicencoding-cmd\endcsname##1##2{
%    \ifx\protect\@typeset@protec
%      \begingroup\UseTextSymbol\cyrillicencoding##1\endgrou
%    \else\noexpand##1\fi}
%    \end{macrocode

%  \end{macro

%    The category code of the characters `\texttt{:}', `\texttt{;}'
%    `\texttt{!}', and `\texttt{?}' is made |\active| to insert a littl
%    white space

%    For Ukrainian (as well as for Russian and German) the \texttt{"
%    character also is made active

%    Note: It is \emph{very} questionable whether the Russian typesettin
%    tradition requires additional spacing before those punctuation signs
%    Therefore, we make the corresponding code optional. If you need it
%    then define the \texttt{frenchpunct} docstrip option i
%    \file{babel.ins}

%    Borrowed from french
%    Some users dislike automatic insertion of a space befor
%    `double punctuation', and prefer to decide themselves whether
%    space should be added or not; so a hook |\NoAutoSpaceBeforeFDP
%    is provided: if this command is added (in file |ukraineb.cfg|, o
%    anywhere in a document) |ukraineb| will respect your typing, an
%    introduce a suitable space before `double punctuation' \emph{i
%    and only if} a space is typed in the source file before thos
%    signs

%    The command |\AutoSpaceBeforeFDP| switches back to th
%    default behavior of |ukraineb|

%    \begin{macrocode
%<*frenchpunct
\initiate@active@char{:
\initiate@active@char{;
%</frenchpunct
%<*frenchpunct|spanishligs
\initiate@active@char{!
\initiate@active@char{?
%</frenchpunct|spanishligs
\initiate@active@char{"
%    \end{macrocode

%    The code above is necessary because we need extra active characters
%    The character |"| is used as indicated i
%    table~\ref{tab:ukrainian-quote}

%    We specify that the Ukrainian group of shorthands should be used

%    \begin{macrocode
\addto\extrasukrainian{\languageshorthands{ukrainian}
%    \end{macrocode

%    These characters are `turned on' once, later their definition ma
%    vary

%    \begin{macrocode
\addto\extrasukrainian{
%<frenchpunct>  \bbl@activate{:}\bbl@activate{;}
%<frenchpunct|spanishligs>  \bbl@activate{!}\bbl@activate{?}
  \bbl@activate{"}
\addto\noextrasukrainian{
%<frenchpunct>  \bbl@deactivate{:}\bbl@deactivate{;}
%<frenchpunct|spanishligs>  \bbl@deactivate{!}\bbl@deactivate{?}
  \bbl@deactivate{"}
%    \end{macrocode

%   The \texttt{X2} and \texttt{T2*} encodings do not contai
%   |spanish_shriek| and |spanish_query| symbols; as a consequence, th
%   ligatures `|?`|' and `|!`|' do not work with them (these characters ar
%   useless for Cyrillic texts anyway). But we define the shorthands t
%   emulate these ligatures (optionally)

%   We do not use |\latinencoding| here (but instead explicitly us
%   \texttt{OT1}) because the user may choose \texttt{T2A} to be the primar
%   encoding, but it does not contain these characters

%    \begin{macrocode
%<*spanishligs
\declare@shorthand{ukrainian}{?`}{\UseTextSymbol{OT1}\textquestiondown
\declare@shorthand{ukrainian}{!`}{\UseTextSymbol{OT1}\textexclamdown
%</spanishligs
%    \end{macrocode

% \begin{macro}{\ukrainian@sh@;@
% \begin{macro}{\ukrainian@sh@:@
% \begin{macro}{\ukrainian@sh@!@
% \begin{macro}{\ukrainian@sh@?@

%    We have to reduce the amount of white space before \texttt{;}
%    \texttt{:} and \texttt{!}. This should only happen in horizontal mode
%    hence the test with |\ifhmode|

%    \begin{macrocode
%<*frenchpunct
\declare@shorthand{ukrainian}{;}{
  \ifhmod
%    \end{macrocode

%    In horizontal mode we check for the presence of a `space', `unskip' i
%    it exists and place a |0.1em| kerning

%    \begin{macrocode
    \ifdim\lastskip>\z
      \unskip\nobreak\kern.1e
    \els
%    \end{macrocode
%    If no space has been typed, we add |\FDP@thinspace
%    which will b
%    defined, up to the user's wishes, as an automatic adde
%    thinspace, or as |\@empty|

%    \begin{macrocode
        \FDP@thinspac
    \f
  \f
%    \end{macrocode

%    Now we can insert a `|;|' character

%    \begin{macrocode
  \string;
%    \end{macrocode

%    The other definitions are very similar

%    \begin{macrocode
\declare@shorthand{ukrainian}{:}{
  \ifhmod
    \ifdim\lastskip>\z
      \unskip\nobreak\kern.1e
    \els
        \FDP@thinspac
    \f
  \f
  \string:
%    \end{macrocode

%    \begin{macrocode
\declare@shorthand{ukrainian}{!}{
  \ifhmod
    \ifdim\lastskip>\z
      \unskip\nobreak\kern.1e
    \els
        \FDP@thinspac
    \f
  \f
  \string!
%    \end{macrocode

%    \begin{macrocode
\declare@shorthand{ukrainian}{?}{
  \ifhmod
    \ifdim\lastskip>\z
      \unskip\nobreak\kern.1e
    \els
        \FDP@thinspac
    \f
  \f
  \string?
%    \end{macrocode

% \end{macro
% \end{macro
% \end{macro
% \end{macro


%  \begin{macro}{\AutoSpaceBeforeFDP
%  \begin{macro}{\NoAutoSpaceBeforeFDP
%  \begin{macro}{\FDP@thinspace
%    |\FDP@thinspace| is defined as unbreakabl
%    spaces if |\AutoSpaceBeforeFDP| is activated or as |\@empty| i
%    |\NoAutoSpaceBeforeFDP| is in use
%    The default is |\AutoSpaceBeforeFDP|
%    \begin{macrocode
\def\AutoSpaceBeforeFDP{
      \def\FDP@thinspace{\nobreak\kern.1em}
\def\NoAutoSpaceBeforeFDP{\let\FDP@thinspace\@empty
\AutoSpaceBeforeFD
%    \end{macrocode
%  \end{macro
%  \end{macro
%  \end{macro

%  \begin{macro}{\FDPon
%  \begin{macro}{\FDPoff

%     The next macros allow to switch on/off activeness of doubl
%     punctuation signs

%    \begin{macrocode
\def\FDPon{\bbl@activate{:}
        \bbl@activate{;}
        \bbl@activate{?}
        \bbl@activate{!}
\def\FDPoff{\bbl@deactivate{:}
        \bbl@deactivate{;}
        \bbl@deactivate{?}
        \bbl@deactivate{!}
%    \end{macrocode
%  \end{macro
%  \end{macro

%  \begin{macro}{\system@sh@:@
%  \begin{macro}{\system@sh@!@
%  \begin{macro}{\system@sh@?@
%  \begin{macro}{\system@sh@;@

%    When the active characters appear in an environment where thei
%    Ukrainian behaviour is not wanted they should give an `expected
%    result. Therefore we define shorthands at system level as well

%    \begin{macrocode
\declare@shorthand{system}{:}{\string:
\declare@shorthand{system}{;}{\string;
%</frenchpunct
%<*frenchpunct&!spanishligs
\declare@shorthand{system}{!}{\string!
\declare@shorthand{system}{?}{\string?
%</frenchpunct&!spanishligs
%    \end{macrocode

%  \end{macro
%  \end{macro
%  \end{macro
%  \end{macro

%    To be able to define the function of `|"|', we first define a couple o
%    `support' macros

%  \begin{macro}{\dq

%    We save the original double quote character in |\dq| to keep i
%    available, the math accent |\"| can now be typed as `|"|'

%    \begin{macrocode
\begingroup \catcode`\"1
\def\reserved@a{\endgrou
  \def\@SS{\mathchar"7019
  \def\dq{"}
\reserved@
%    \end{macrocode

%  \end{macro

%    Now we can define the doublequote macros: german and french quotes
%    We use definitions of these quotes made in babel.sty
%    The french quotes are contained in the \texttt{T2*} encodings

%    \begin{macrocode
\declare@shorthand{ukrainian}{"`}{\glqq
\declare@shorthand{ukrainian}{"'}{\grqq
\declare@shorthand{ukrainian}{"<}{\flqq
\declare@shorthand{ukrainian}{">}{\frqq
%    \end{macrocode

%    Some additional commands

%    \begin{macrocode
\declare@shorthand{ukrainian}{""}{\hskip\z@skip
\declare@shorthand{ukrainian}{"~}{\textormath{\leavevmode\hbox{-}}{-}
\declare@shorthand{ukrainian}{"=}{\nobreak-\hskip\z@skip
\declare@shorthand{ukrainian}{"|}{
  \textormath{\nobreak\discretionary{-}{}{\kern.03em}
              \allowhyphens}{}
%    \end{macrocode

%    The next two macros for |"-| and |"---| are somewhat different
%    We must check whether the second token is a hyphen character

%    \begin{macrocode
\declare@shorthand{ukrainian}{"-}{
%    \end{macrocode

%    If the next token is `|-|', we typeset an emdash, otherwise a hyphe
%    sign

%    \begin{macrocode
  \def\ukrainian@sh@tmp{
    \if\ukrainian@sh@next-\expandafter\ukrainian@sh@emdas
    \else\expandafter\ukrainian@sh@hyphen\f
  }
%    \end{macrocode

%    \TeX\ looks for the next token after the first `|-|': the meaning o
%    this token is written to |\ukrainian@sh@next| and |\ukrainian@sh@tmp| i
%    called

%    \begin{macrocode
  \futurelet\ukrainian@sh@next\ukrainian@sh@tmp
%    \end{macrocode

%    Here are the definitions of hyphen and emdash. First the hyphen

%    \begin{macrocode
\def\ukrainian@sh@hyphen{
  \nobreak\-\bbl@allowhyphens
%    \end{macrocode

%    For the emdash definition, there are the two parameters: we must `eat
%    two last hyphen signs of our emdash\dots
%    \begin{macrocode
\def\ukrainian@sh@emdash#1#2{\cdash-#1#2
%    \end{macrocode
%  \begin{macro}{\cdash
%    \dots\ these two parameters are useful for another macro
%    |\cdash|
%    \begin{macrocode
%\ifx\cdash\undefined % should be defined earlie
\def\cdash#1#2#3{\def\tempx@{#3}
\def\tempa@{-}\def\tempb@{~}\def\tempc@{*}
 \ifx\tempx@\tempa@\@Acdash\els
  \ifx\tempx@\tempb@\@Bcdash\els
   \ifx\tempx@\tempc@\@Ccdash\els
    \errmessage{Wrong usage of cdash}\fi\fi\fi
%    \end{macrocode
%   second parameter (or third for |\cdash|) shows what kind of emdas
%   to create in next ste
%      \begin{center
%      \begin{tabular}{@{}p{.1\hsize}@{}p{.9\hsize}@{}
%       |"---| & ordinary (plain) Cyrillic emdash inside text
%       an unbreakable thinspace will be inserted before only in case o
%       a \textit{space} before the dash (it is necessary for dashes afte
%       display maths formulae: there could be lists, enumerations etc.
%       started with ``--- where $a$ is ...'' i.e., the dash starts a line)
%       (Firstly there were planned rather soft rules for user: he may pu
%       a space before the dash or not.  But it is difficult to place thi
%       thinspace automatically, i.e., by checking modes because afte
%       display formulae \TeX{} uses horizontal mode.  Maybe there is
%       misunderstanding?  Maybe there is another way?)  After a das
%       a breakable thinspace is always placed; \
%   \end{tabular
%   \end{center
%    \begin{macrocode
% What is more grammatically: .2em or .2\fontdimen6\font
\def\@Acdash{\ifdim\lastskip>\z@\unskip\nobreak\hskip.2em\f
  \cyrdash\hskip.2em\ignorespaces}
%    \end{macrocode
%      \begin{center
%      \begin{tabular}{@{}p{.1\hsize}@{}p{.9\hsize}@{}
%       |"--~| & emdash in compound names or surname
%       (like Mendeleev--Klapeiron); this dash has no space character
%       around; after the dash some space is adde
%       |\exhyphenalty| \
%   \end{tabular
%   \end{center
%    \begin{macrocode
\def\@Bcdash{\leavevmode\ifdim\lastskip>\z@\unskip\f
 \nobreak\cyrdash\penalty\exhyphenpenalty\hskip\z@skip\ignorespaces}
%    \end{macrocode
%      \begin{center
%      \begin{tabular}{@{}p{.1\hsize}@{}p{.9\hsize}@{}
%       |"--*| & for denoting direct speech (a space like |\enskip
%       must follow the emdash); \
%   \end{tabular
%   \end{center
%    \begin{macrocode
\def\@Ccdash{\leavevmod
 \nobreak\cyrdash\nobreak\hskip.35em\ignorespaces}
%\f
%    \end{macrocode
%  \end{macro

%  \begin{macro}{\cyrdash
%   Finally the macro for ``body'' of the Cyrillic emdash
%   The |\cyrdash| macro will be defined in case this macro hasn't bee
%   defined in a fontenc file.  For T2* fonts, cyrdash will be placed i
%   the code of the English emdash thus it uses ligature |---|
%    \begin{macrocode
% Is there an IF necessary
\ifx\cyrdash\undefine
  \def\cyrdash{\hbox to.8em{--\hss--}
\f
%    \end{macrocode
%  \end{macro

%    Here a really new macro---to place thinspace between initials
%    This macro used instead of |\,| allows hyphenation in the followin
%    surname

%    \begin{macrocode
\declare@shorthand{ukrainian}{",}{\nobreak\hskip.2em\ignorespaces
%    \end{macrocode

%  \begin{macro}{\mdqon
%  \begin{macro}{\mdqoff
%    All that's left to do now is to  define a couple of command
%    for |"|
%    \begin{macrocode
\def\mdqon{\bbl@activate{"}
\def\mdqoff{\bbl@deactivate{"}
%    \end{macrocode
%  \end{macro
%  \end{macro

%    The Ukrainian hyphenation patterns can be used with |\lefthyphenmin
%    and |\righthyphenmin| set to~2
% \changes{ukraineb-1.1g}{2000/09/22}{Now use \cs{providehyphenmins} t
%    provide a default value
%    \begin{macrocode
\providehyphenmins{\CurrentOption}{\tw@\tw@
% temporary hack
\ifx\englishhyphenmins\undefine
  \def\englishhyphenmins{\tw@\thr@@
\f
%    \end{macrocode

%    Now the action |\extrasukrainian| has to execute is to make sure that th
%    command |\frenchspacing| is in effect. If this is not the case th
%    execution of |\noextrasukrainian| will switch it off again

%    \begin{macrocode
\addto\extrasukrainian{\bbl@frenchspacing
\addto\noextrasukrainian{\bbl@nonfrenchspacing
%    \end{macrocode

% \end{macro

%    Next we add a new enumeration style for Ukrainian manuscripts wit
%    Cyrillic letters, and later on we define some math operator names i
%    accordance with Ukrainian and Russian typesetting traditions

%  \begin{macro}{\Asbuk

%    We begin by defining |\Asbuk| which works like |\Alph|, but produce
%    (uppercase) Cyrillic letters intead of Latin ones. The letters CYRGUP
%    and SFTSN are skipped, as usual for such enumeration

%    \begin{macrocode
\def\Asbuk#1{\expandafter\@Asbuk\csname c@#1\endcsname
\def\@Asbuk#1{\ifcase#1\o
  \CYRA\or\CYRB\or\CYRV\or\CYRG\or\CYRD\or\CYRE\or\CYRIE\o
  \CYRZH\or\CYRZ\or\CYRI\or\CYRII\or\CYRYI\or\CYRISHRT\o
  \CYRK\or\CYRL\or\CYRM\or\CYRN\or\CYRO\or\CYRP\or\CYRR\o
  \CYRS\or\CYRT\or\CYRU\or\CYRF\or\CYRH\or\CYRC\or\CYRCH\o
  \CYRSH\or\CYRSHCH\or\CYRYU\or\CYRYA\else\@ctrerr\fi
%    \end{macrocode

%  \end{macro

%  \begin{macro}{\asbuk

%    The macro |\asbuk| is similar to |\alph|; it produces lowercas
%    Ukrainian letters

%    \begin{macrocode
\def\asbuk#1{\expandafter\@asbuk\csname c@#1\endcsname
\def\@asbuk#1{\ifcase#1\o
  \cyra\or\cyrb\or\cyrv\or\cyrg\or\cyrd\or\cyre\or\cyrie\o
  \cyrzh\or\cyrz\or\cyri\or\cyrii\or\cyryi\or\cyrishrt\o
  \cyrk\or\cyrl\or\cyrm\or\cyrn\or\cyro\or\cyrp\or\cyrr\o
  \cyrs\or\cyrt\or\cyru\or\cyrf\or\cyrh\or\cyrc\or\cyrch\o
  \cyrsh\or\cyrshch\or\cyryu\or\cyrya\else\@ctrerr\fi
%    \end{macrocode

%  \end{macro

%    Set up default Cyrillic math alphabets. The math groups for cyrilli
%    letters are defined in the encoding definition files. First, declar
%    a new alphabet for symbols, |\cyrmathrm|, based on the symbol fon
%    for Cyrillic letters defined in the encoding definition file. Note
%    that by default Cyrillic letters are taken from upright font in mat
%    mode (unlike Latin letters)
%    \begin{macrocode
%\RequirePackage{textmath
\@ifundefined{sym\cyrillicencoding letters}{}{
\SetSymbolFont{\cyrillicencoding letters}{bold}\cyrillicencodin
  \rmdefault\bfdefault\updefaul
\DeclareSymbolFontAlphabet\cyrmathrm{\cyrillicencoding letters
%    \end{macrocode
%    And we need a few commands to be able to switch to different variants
%    \begin{macrocode
\DeclareMathAlphabet\cyrmathbf\cyrillicencodin
  \rmdefault\bfdefault\updefaul
\DeclareMathAlphabet\cyrmathsf\cyrillicencodin
  \sfdefault\mddefault\updefaul
\DeclareMathAlphabet\cyrmathit\cyrillicencodin
  \rmdefault\mddefault\itdefaul
\DeclareMathAlphabet\cyrmathtt\cyrillicencodin
  \ttdefault\mddefault\updefaul

\SetMathAlphabet\cyrmathsf{bold}\cyrillicencodin
  \sfdefault\bfdefault\updefaul
\SetMathAlphabet\cyrmathit{bold}\cyrillicencodin
  \rmdefault\bfdefault\itdefaul

%    \end{macrocode

%    Some math functions in Ukrainian and Russian math books have othe
%    names: e.g., \texttt{sinh} in Russian is written as \texttt{sh} etc
%    So we define a number of new math operators

%    |\sinh|
%    \begin{macrocode
\def\sh{\mathop{\operator@font sh}\nolimits
%    \end{macrocode
%    |\cosh|
%    \begin{macrocode
\def\ch{\mathop{\operator@font ch}\nolimits
%    \end{macrocode
%    |\tan|
%    \begin{macrocode
\def\tg{\mathop{\operator@font tg}\nolimits
%    \end{macrocode
%    |\arctan|
%    \begin{macrocode
\def\arctg{\mathop{\operator@font arctg}\nolimits
%    \end{macrocode
%    arcctg
%    \begin{macrocode
\def\arcctg{\mathop{\operator@font arcctg}\nolimits
%    \end{macrocode
%    The following macro conflicts with |\th| defined in Latin~1 encoding

%    |\tanh|
% \changes{ukraineb-1.1k}{2004/05/21}{Change definition of \cs{th
%    only for this language
%    \begin{macrocode
\addto\extrasrussian{
  \babel@save{\th}
  \let\ltx@th\t
  \def\th{\textormath{\ltx@th}
                     {\mathop{\operator@font th}\nolimits}}

%    \end{macrocode
%    |\cot|
%    \begin{macrocode
\def\ctg{\mathop{\operator@font ctg}\nolimits
%    \end{macrocode
%    |\coth|
%    \begin{macrocode
\def\cth{\mathop{\operator@font cth}\nolimits
%    \end{macrocode
%    |\csc|
%    \begin{macrocode
\def\cosec{\mathop{\operator@font cosec}\nolimits
%    \end{macrocode

%    And finally some other Ukrainian and Russian mathematical symbols
%    \begin{macrocode
\def\Prob{\mathop{\kern\z@\mathsf{P}}\nolimits
\def\Variance{\mathop{\kern\z@\mathsf{D}}\nolimits
\def\nsd{\mathop{\cyrmathrm{\cyrn.\cyrs.\cyrd.}}\nolimits
\def\nsk{\mathop{\cyrmathrm{\cyrn.\cyrs.\cyrk.}}\nolimits
\def\NSD{\mathop{\cyrmathrm{\CYRN\CYRS\CYRD}}\nolimits
\def\NSK{\mathop{\cyrmathrm{\CYRN\CYRS\CYRK}}\nolimits
  \def\nod{\mathop{\cyrmathrm{\cyrn.\cyro.\cyrd.}}\nolimits}    % ?????
  \def\nok{\mathop{\cyrmathrm{\cyrn.\cyro.\cyrk.}}\nolimits}    % ?????
  \def\NOD{\mathop{\cyrmathrm{\CYRN\CYRO\CYRD}}\nolimits}       % ?????
  \def\NOK{\mathop{\cyrmathrm{\CYRN\CYRO\CYRK}}\nolimits}       % ?????
\def\Proj{\mathop{\cyrmathrm{\CYRP\cyrr}}\nolimits
%    \end{macrocode

% This is for compatibility with older Ukrainian packages
%    \begin{macrocode
\DeclareRobustCommand{\No}{
   \ifmmode{\nfss@text{\textnumero}}\else\textnumero\fi
%    \end{macrocode

%    The macro |\ldf@finish| takes care of looking for a configuration file
%    setting the main language to be switched on at |\begin{document}| an
%    resetting the category code of \texttt{@} to its original value

%    \begin{macrocode
\ldf@finish{ukrainian
%</code
%    \end{macrocode

% \Final
%
%% \CharacterTabl
%%  {Upper-case    \A\B\C\D\E\F\G\H\I\J\K\L\M\N\O\P\Q\R\S\T\U\V\W\X\Y\
%%   Lower-case    \a\b\c\d\e\f\g\h\i\j\k\l\m\n\o\p\q\r\s\t\u\v\w\x\y\
%%   Digits        \0\1\2\3\4\5\6\7\8\
%%   Exclamation   \!     Double quote  \"     Hash (number) \
%%   Dollar        \$     Percent       \%     Ampersand     \
%%   Acute accent  \'     Left paren    \(     Right paren   \
%%   Asterisk      \*     Plus          \+     Comma         \
%%   Minus         \-     Point         \.     Solidus       \
%%   Colon         \:     Semicolon     \;     Less than     \
%%   Equals        \=     Greater than  \>     Question mark \
%%   Commercial at \@     Left bracket  \[     Backslash     \
%%   Right bracket \]     Circumflex    \^     Underscore    \
%%   Grave accent  \`     Left brace    \{     Vertical bar  \
%%   Right brace   \}     Tilde         \~
%
\endinpu
}
\bbl@tempa{uppersorbian}{% \iffalse meta-commen

% Copyright 1989-2008 Johannes L. Braams and any individual author
% listed elsewhere in this file.  All rights reserved
%
% This file is part of the Babel system
% -------------------------------------
%
% It may be distributed and/or modified under th
% conditions of the LaTeX Project Public License, either version 1.
% of this license or (at your option) any later version
% The latest version of this license is i
%   http://www.latex-project.org/lppl.tx
% and version 1.3 or later is part of all distributions of LaTe
% version 2003/12/01 or later
%
% This work has the LPPL maintenance status "maintained"
%
% The Current Maintainer of this work is Johannes Braams
%
% The list of all files belonging to the Babel system i
% given in the file `manifest.bbl. See also `legal.bbl' for additiona
% information
%
% The list of derived (unpacked) files belonging to the distributio
% and covered by LPPL is defined by the unpacking scripts (wit
% extension .ins) which are part of the distribution
% \f
% \CheckSum{344
% \iffals
%    Tell the \LaTeX\ system who we are and write an entry on th
%    transcript
%<*dtx
\ProvidesFile{usorbian.dtx
%</dtx
%<code>\ProvidesLanguage{usorbian
%\f
%\ProvidesFile{usorbian.dtx
        [2008/03/17 v1.0k Upper Sorbian support from the babel system
%\iffals
%% File `usorbian.dtx
%% Babel package for LaTeX version 2
%% Copyright (C) 1989 - 200
%%           by Johannes Braams, TeXnie

%% Upper Sorbian Language Definition Fil
%% Copyright (C) 1994 - 200
%%           by Eduard Werne
%           Werner, Eduard"
%           Serbski institut z. t.
%           Dw\'orni\v{s}\'cowa
%           02625 Budy\v{s}in/Bautze
%           Germany"
%           (??)3591 497223"
%           edi at kaihh.hanse.de"

%% Please report errors to: Eduard Werner edi at kaihh.hanse.d
%
%    This file is part of the babel system, it provides the sourc
%    code for the Upper Sorbian definition file
%<*filedriver
\documentclass{ltxdoc
\newcommand*\TeXhax{\TeX hax
\newcommand*\babel{\textsf{babel}
\newcommand*\langvar{$\langle \it lang \rangle$
\newcommand*\note[1]{
\newcommand*\Lopt[1]{\textsf{#1}
\newcommand*\file[1]{\texttt{#1}
\newfont{\logo}{logo10
\newcommand*\MF{{\logo METAFONT}
\begin{document
 \DocInput{usorbian.dtx
\end{document
%</filedriver
%\f
% \GetFileInfo{usorbian.dtx

% \changes{usorbian-0.1}{1994/10/10}{First version
% \changes{usorbian-0.1b}{1994/10/18}{Made it possible to run throug
%    \LaTeX; added \cs{MF} and removed extra \cs{end{macro}}
% \changes{usorbian-1.0d}{1996/07/13}{Replaced \cs{undefined} wit
%    \cs{@undefined} and \cs{empty} with \cs{@empty} for consistenc
%    with \LaTeX
% \changes{usorbian-1.0e}{1996/10/10}{Moved the definition o
%    \cs{atcatcode} right to the beginning.

%  \section{The Upper Sorbian language

%    The file \file{\filename}\footnote{The file described in thi
%    section has version number \fileversion\ and was last revised o
%    \filedate.  It was written by Eduard Werne
%    (\texttt{edi@kaihh.hanse.de}).}  It defines all th
%    language-specific macros for Upper Sorbian

% \StopEventually{

%    The macro |\LdfInit| takes care of preventing that this file i
%    loaded more than once, checking the category code of th
%    \texttt{@} sign, etc
% \changes{usorbian-1.0e}{1996/11/03}{Now use \cs{LdfInit} to perfor
%    initial checks}
% \changes{usorbian-1.0j}{2007/10/19}{This file can be loaded unde
%    more than one name.
%    \begin{macrocode
%<*code
\LdfInit\CurrentOption{date\CurrentOption
%    \end{macrocode

%    When this file is read as an option, i.e. by the |\usepackage
%    command, \texttt{usorbian} will be an `unknown' language, in whic
%    case we have to make it known. So we check for the existence o
%    |\l@usorbian| to see whether we have to do something here.
% \changes{usorbian-1.0j}{2007/10/19}{Check for the optio
%    lowersorbian
%    A
%    \babel\ also knwos the option \Lopt{uppersorbian} we have t
%    check that as well

%    \begin{macrocode
\ifx\l@uppersorbian\@undefine
  \ifx\l@usorbian\@undefine
    \@nopatterns{Usorbian
    \adddialect\l@usorbian\z
    \let\l@uppersorbian\l@usorbia
  \els
    \let\l@uppersorbian\l@usorbia
  \f
\els
  \let\l@usorbian\l@uppersorbia
\f
%    \end{macrocode

%    The next step consists of defining commands to switch to (an
%    from) the Upper Sorbian language

% \begin{macro}{\captionsusorbian
%    The macro |\captionsusorbian| defines all strings used in the fou
%    standard documentclasses provided with \LaTeX
% \changes{usorbian-0.1c}{1994/11/27}{Removed two typos (Kapitel an
%    Dodatki)
% \changes{usorbian-1.0b}{1995/07/04}{Added \cs{proofname} fo
%    AMS-\LaTeX
% \changes{usorbian-1.0i}{2000/09/22}{Added \cs{glossaryname}
% \changes{usorbian-1.0j}{2007/10/19}{Make this work for more than on
%    option name
%    \begin{macrocode
\@namedef{captions\CurrentOption}{
  \def\prefacename{Zawod}
  \def\refname{Referency}
  \def\abstractname{Abstrakt}
  \def\bibname{Literatura}
  \def\chaptername{Kapitl}
  \def\appendixname{Dodawki}
  \def\contentsname{Wobsah}
  \def\listfigurename{Zapis wobrazow}
  \def\listtablename{Zapis tabulkow}
  \def\indexname{Indeks}
  \def\figurename{Wobraz}
  \def\tablename{Tabulka}
  \def\partname{D\'z\v el}
  \def\enclname{P\v r\l oha}
  \def\ccname{CC}
  \def\headtoname{Komu}
  \def\pagename{Strona}
  \def\seename{hl.}
  \def\alsoname{hl.~te\v z
  \def\proofname{Proof}%  <-- needs translatio
  \def\glossaryname{Glossary}% <-- Needs translatio
  }
%    \end{macrocode
% \end{macro

% \begin{macro}{\newdateusorbian
%    The macro |\newdateusorbian| redefines the command |\today| t
%    produce Upper Sorbian dates
% \changes{usorbian-1.0g}{1997/10/01}{Use \cs{edef} to defin
%    \cs{today} to save memory
% \changes{usorbian-1.0g}{1998/03/28}{use \cs{def} instead o
%    \cs{edef}
% \changes{usorbian-1.0j}{2007/10/19}{Make this work for more than on
%    option name
%    \begin{macrocode
\@namedef{newdate\CurrentOption}{
  \def\today{\number\day.~\ifcase\month\o
    januara\or februara\or m\v erca\or apryla\or meje\or junija\o
    julija\or awgusta\or septembra\or oktobra\o
    nowembra\or decembra\f
    \space \number\year}
%    \end{macrocode
% \end{macro

% \begin{macro}{\olddateusorbian
%    The macro |\olddateusorbian| redefines the command |\today| t
%    produce old-style Upper Sorbian dates
% \changes{usorbian-1.0g}{1997/10/01}{Use \cs{edef} to defin
%    \cs{today} to save memory
% \changes{usorbian-1.0g}{1998/03/28}{use \cs{def} instead o
%    \cs{edef}
% \changes{usorbian-1.0j}{2007/10/19}{Make this work for more than on
%    option name
%    \begin{macrocode
\@namedef{olddate\CurrentOption}{
  \def\today{\number\day.~\ifcase\month\o
    wulkeho r\'o\v zka\or ma\l eho r\'o\v zka\or nal\v etnika\o
    jutrownika\or r\'o\v zownika\or  sma\v znika\or pra\v znika\o
    \v znjenca\or po\v znjenca\or winowca\or nazymnika\o
    hodownika\fi \space \number\year}
%    \end{macrocode
% \end{macro

%    The default will be the new-style dates
% \changes{usorbian-1.0j}{2007/10/19}{Make this work for more than on
%    option name
%    \begin{macrocode
\expandafter\let\csname date\CurrentOption\expandafter\endcsnam
                \csname newdate\CurrentOption\endcsnam
%    \end{macrocode

% \begin{macro}{\extrasusorbian
%    The macro |\extrasusorbian| will perform all the extr
%    definitions needed for the Upper Sorbian language. It's pirate
%    from |germanb.sty|.  The macro |\noextrasusorbian| is used t
%    cancel the actions of |\extrasusorbian|

%    Because for Upper Sorbian (as well as for Dutch) the \texttt{"
%    character is made active. This is done once, later on it
%    definition may vary
% \changes{usorbian-1.0j}{2007/10/19}{Make this work for more than on
%    option name
%    \begin{macrocode
\initiate@active@char{"
\@namedef{extras\CurrentOption}{\languageshorthands{usorbian}
\expandafter\addto\csname extras\CurrentOption\endcsname{
  \bbl@activate{"}
%    \end{macrocode
%    Don't forget to turn the shorthands off again
% \changes{usorbian-1.0h}{1999/12/17}{Deactivate shorthands ouside o
%    Upper Sorbian
%    \begin{macrocode
\expandafter\addto\csname extras\CurrentOption\endcsname{
  \bbl@deactivate{"}
%    \end{macrocode

%    In order for \TeX\ to be able to hyphenate German Upper Sorbia
%    words which contain `\ss' we have to give the character a nonzer
%    |\lccode| (see Appendix H, the \TeX book). As some of the othe
%    language definitions turn the character |^| into a shorthand w
%    need to make sure that it has it's orginial definition here
% \changes{usorbian-1.0k}{2008/03/17}{Make sure the caret has th
%    right \cs{catcdoe}}
%    \begin{macrocode
\begingroup \catcode`\^
\def\x{\endgrou
  \expandafter\addto\csname extras\CurrentOption\endcsname{
    \babel@savevariable{\lccode`\^^Y}
    \lccode`\^^Y`\^^Y}
\
%    \end{macrocode
%    The umlaut accent macro |\"| is changed to lower the umlaut dots
%    The redefinition is done with the help of |\umlautlow|
%    \begin{macrocode
\expandafter\addto\csname extras\CurrentOption\endcsname{
  \babel@save\"\umlautlow
\expandafter\addto\csname noextras\CurrentOption\endcsname{
  \umlauthigh
%    \end{macrocode
%    The Upper Sorbian hyphenation patterns can be used wit
%    |\lefthyphenmin| and |\righthyphenmin| set to~2
% \changes{usorbian-1.0i}{2000/09/22}{Now use \cs{providehyphenmins} t
%    provide a default value
%    \begin{macrocode
\providehyphenmins{\CurrentOption}{\tw@\tw@
%    \end{macrocode
% \end{macro

% \changes{usorbian-1.0a}{1995/05/27}{Removed stuff that has bee
%    moved to \file{babel.def}

%  \begin{macro}{\dq
%    We save the original double quote character in |\dq| to keep i
%    available, the math accent |\"| can now be typed as |"|.  Also w
%    store the original meaning of the command |\"| for future use
%    \begin{macrocode
\begingroup \catcode`\"1
\def\x{\endgrou
  \def\@SS{\mathchar"7019
  \def\dq{"}
\
%    \end{macrocode
% \end{macro

%    Now we can define the doublequote macros: the umlauts
%    \begin{macrocode
\declare@shorthand{usorbian}{"a}{\textormath{\"{a}}{\ddot a}
\declare@shorthand{usorbian}{"o}{\textormath{\"{o}}{\ddot o}
\declare@shorthand{usorbian}{"u}{\textormath{\"{u}}{\ddot u}
\declare@shorthand{usorbian}{"A}{\textormath{\"{A}}{\ddot A}
\declare@shorthand{usorbian}{"O}{\textormath{\"{O}}{\ddot O}
\declare@shorthand{usorbian}{"U}{\textormath{\"{U}}{\ddot U}
%    \end{macrocode
%    tremas
%    \begin{macrocode
\declare@shorthand{usorbian}{"e}{\textormath{\"{e}}{\ddot e}
\declare@shorthand{usorbian}{"E}{\textormath{\"{E}}{\ddot E}
\declare@shorthand{usorbian}{"i}{\textormath{\"{\i}}{\ddot\imath}
\declare@shorthand{usorbian}{"I}{\textormath{\"{I}}{\ddot I}
%    \end{macrocode
%    usorbian es-zet (sharp s)
%    \begin{macrocode
\declare@shorthand{usorbian}{"s}{\textormath{\ss{}}{\@SS{}}
\declare@shorthand{usorbian}{"S}{SS
%    \end{macrocode
%    german and french quotes
% \changes{usorbian-1.0f}{1997/04/03}{Removed empty groups afte
%    double quote and guillemot characters
%    \begin{macrocode
\declare@shorthand{usorbian}{"`}{
  \textormath{\quotedblbase}{\mbox{\quotedblbase}}
\declare@shorthand{usorbian}{"'}{
  \textormath{\textquotedblleft}{\mbox{\textquotedblleft}}
\declare@shorthand{usorbian}{"<}{
  \textormath{\guillemotleft}{\mbox{\guillemotleft}}
\declare@shorthand{usorbian}{">}{
  \textormath{\guillemotright}{\mbox{\guillemotright}}
%    \end{macrocode
%    discretionary command
% \changes{usorbian-1.0c}{1996/01/24}{Now use \cs{bbl@disc}
%    \begin{macrocode
\declare@shorthand{usorbian}{"c}{\textormath{\bbl@disc ck}{c}
\declare@shorthand{usorbian}{"C}{\textormath{\bbl@disc CK}{C}
\declare@shorthand{usorbian}{"f}{\textormath{\bbl@disc f{ff}}{f}
\declare@shorthand{usorbian}{"F}{\textormath{\bbl@disc F{FF}}{F}
\declare@shorthand{usorbian}{"l}{\textormath{\bbl@disc l{ll}}{l}
\declare@shorthand{usorbian}{"L}{\textormath{\bbl@disc L{LL}}{L}
\declare@shorthand{usorbian}{"m}{\textormath{\bbl@disc m{mm}}{m}
\declare@shorthand{usorbian}{"M}{\textormath{\bbl@disc M{MM}}{M}
\declare@shorthand{usorbian}{"n}{\textormath{\bbl@disc n{nn}}{n}
\declare@shorthand{usorbian}{"N}{\textormath{\bbl@disc N{NN}}{N}
\declare@shorthand{usorbian}{"p}{\textormath{\bbl@disc p{pp}}{p}
\declare@shorthand{usorbian}{"P}{\textormath{\bbl@disc P{PP}}{P}
\declare@shorthand{usorbian}{"t}{\textormath{\bbl@disc t{tt}}{t}
\declare@shorthand{usorbian}{"T}{\textormath{\bbl@disc T{TT}}{T}
%    \end{macrocode
%    and some additional commands
%    \begin{macrocode
\declare@shorthand{usorbian}{"-}{\nobreak\-\bbl@allowhyphens
\declare@shorthand{usorbian}{"|}{
  \textormath{\nobreak\discretionary{-}{}{\kern.03em}
              \allowhyphens}{}
\declare@shorthand{usorbian}{""}{\hskip\z@skip
%    \end{macrocode

%  \begin{macro}{\mdqon
%  \begin{macro}{\mdqoff
%  \begin{macro}{\ck
%    All that's left to do now is to  define a couple of command
%    for reasons of compatibility with \file{german.sty}
% \changes{usorbian-1.0g}{1998/06/07}{Now use \cs{shorthandon} an
%    \cs{shorthandoff}}
%    \begin{macrocode
\def\mdqon{\shorthandon{"}
\def\mdqoff{\shorthandoff{"}
\def\ck{\allowhyphens\discretionary{k-}{k}{ck}\allowhyphens
%    \end{macrocode
%  \end{macro
%  \end{macro
%  \end{macro

%    The macro |\ldf@finish| takes care of looking for
%    configuration file, setting the main language to be switched o
%    at |\begin{document}| and resetting the category code o
%    \texttt{@} to its original value
% \changes{usorbian-1.0e}{1996/11/03}{Now use \cs{ldf@finish} to wra
%    up}
% \changes{usorbian-1.0j}{2007/10/19}{Make this work for more than on
%    option name
%    \begin{macrocode
\ldf@finish\CurrentOptio
%</code
%    \end{macrocode

% \Final
%
%% \CharacterTabl
%%  {Upper-case    \A\B\C\D\E\F\G\H\I\J\K\L\M\N\O\P\Q\R\S\T\U\V\W\X\Y\
%%   Lower-case    \a\b\c\d\e\f\g\h\i\j\k\l\m\n\o\p\q\r\s\t\u\v\w\x\y\
%%   Digits        \0\1\2\3\4\5\6\7\8\
%%   Exclamation   \!     Double quote  \"     Hash (number) \
%%   Dollar        \$     Percent       \%     Ampersand     \
%%   Acute accent  \'     Left paren    \(     Right paren   \
%%   Asterisk      \*     Plus          \+     Comma         \
%%   Minus         \-     Point         \.     Solidus       \
%%   Colon         \:     Semicolon     \;     Less than     \
%%   Equals        \=     Greater than  \>     Question mark \
%%   Commercial at \@     Left bracket  \[     Backslash     \
%%   Right bracket \]     Circumflex    \^     Underscore    \
%%   Grave accent  \`     Left brace    \{     Vertical bar  \
%%   Right brace   \}     Tilde         \~
%
\endinpu
}
\bbl@tempa{USenglish}{%%
%% This file will generate fast loadable files and documentation
%% driver files from the doc files in this package when run through
%% LaTeX or TeX.
%%
%% Copyright 1989-2005 Johannes L. Braams and any individual authors
%% listed elsewhere in this file.  All rights reserved.
%% 
%% This file is part of the Babel system.
%% --------------------------------------
%% 
%% It may be distributed and/or modified under the
%% conditions of the LaTeX Project Public License, either version 1.3
%% of this license or (at your option) any later version.
%% The latest version of this license is in
%%   http://www.latex-project.org/lppl.txt
%% and version 1.3 or later is part of all distributions of LaTeX
%% version 2003/12/01 or later.
%% 
%% This work has the LPPL maintenance status "maintained".
%% 
%% The Current Maintainer of this work is Johannes Braams.
%% 
%% The list of all files belonging to the LaTeX base distribution is
%% given in the file `manifest.bbl. See also `legal.bbl' for additional
%% information.
%% 
%% The list of derived (unpacked) files belonging to the distribution
%% and covered by LPPL is defined by the unpacking scripts (with
%% extension .ins) which are part of the distribution.
%%
%% --------------- start of docstrip commands ------------------
%%
\def\filedate{1999/04/11}
\def\batchfile{english.ins}
\input docstrip.tex

{\ifx\generate\undefined
\Msg{**********************************************}
\Msg{*}
\Msg{* This installation requires docstrip}
\Msg{* version 2.3c or later.}
\Msg{*}
\Msg{* An older version of docstrip has been input}
\Msg{*}
\Msg{**********************************************}
\errhelp{Move or rename old docstrip.tex.}
\errmessage{Old docstrip in input path}
\batchmode
\csname @@end\endcsname
\fi}

\declarepreamble\mainpreamble
This is a generated file.

Copyright 1989-2005 Johannes L. Braams and any individual authors
listed elsewhere in this file.  All rights reserved.

This file was generated from file(s) of the Babel system.
---------------------------------------------------------

It may be distributed and/or modified under the
conditions of the LaTeX Project Public License, either version 1.3
of this license or (at your option) any later version.
The latest version of this license is in
  http://www.latex-project.org/lppl.txt
and version 1.3 or later is part of all distributions of LaTeX
version 2003/12/01 or later.

This work has the LPPL maintenance status "maintained".

The Current Maintainer of this work is Johannes Braams.

This file may only be distributed together with a copy of the Babel
system. You may however distribute the Babel system without
such generated files.

The list of all files belonging to the Babel distribution is
given in the file `manifest.bbl'. See also `legal.bbl for additional
information.

The list of derived (unpacked) files belonging to the distribution
and covered by LPPL is defined by the unpacking scripts (with
extension .ins) which are part of the distribution.
\endpreamble

\declarepreamble\fdpreamble
This is a generated file.

Copyright 1989-2005 Johannes L. Braams and any individual authors
listed elsewhere in this file.  All rights reserved.

This file was generated from file(s) of the Babel system.
---------------------------------------------------------

It may be distributed and/or modified under the
conditions of the LaTeX Project Public License, either version 1.3
of this license or (at your option) any later version.
The latest version of this license is in
  http://www.latex-project.org/lppl.txt
and version 1.3 or later is part of all distributions of LaTeX
version 2003/12/01 or later.

This work has the LPPL maintenance status "maintained".

The Current Maintainer of this work is Johannes Braams.

This file may only be distributed together with a copy of the Babel
system. You may however distribute the Babel system without
such generated files.

The list of all files belonging to the Babel distribution is
given in the file `manifest.bbl'. See also `legal.bbl for additional
information.

In particular, permission is granted to customize the declarations in
this file to serve the needs of your installation.

However, NO PERMISSION is granted to distribute a modified version
of this file under its original name.

\endpreamble

\keepsilent

\usedir{tex/generic/babel} 

\usepreamble\mainpreamble
\generate{\file{english.ldf}{\from{english.dtx}{code}}
          }
\usepreamble\fdpreamble

\ifToplevel{
\Msg{***********************************************************}
\Msg{*}
\Msg{* To finish the installation you have to move the following}
\Msg{* files into a directory searched by TeX:}
\Msg{*}
\Msg{* \space\space All *.def, *.fd, *.ldf, *.sty}
\Msg{*}
\Msg{* To produce the documentation run the files ending with}
\Msg{* '.dtx' and `.fdd' through LaTeX.}
\Msg{*}
\Msg{* Happy TeXing}
\Msg{***********************************************************}
}
 
\endinput
}
%    \end{macrocode}
%    Now, we make sure an option is explicitly declared for any
%    language set as global options.
%    \begin{macrocode}
\@for\bbl@a:=\@classoptionslist\do{%
  \ifx\bbl@a\@empty\else
    \@ifundefined{ds@\bbl@a}%
      {\IfFileExists{\bbl@a.ldf}%
        {\edef\bbl@b{\noexpand\bbl@tempa{\bbl@a}%
           {\noexpand\input{\bbl@a.ldf}}}%
         \bbl@b}%
        {}}%
      {}%
  \fi}
%    \end{macrocode}
%    The goal of the following piece of code is to catch a language
%    given both as global option and the last package option, which
%    formerly had not a well-defined behaviour.  Since there is a
%    previous |\ProcessOptions|, |\@curroptions| is already
%    defined. Note only language options are non empty.
%    \begin{macrocode}
\@for\bbl@a:=\@curroptions\do{%
  \ifx\bbl@a\@empty\else
    \expandafter\ifx\csname ds@\bbl@a\endcsname\@empty\else
      \edef\bbl@lastoption{\bbl@a}%
    \fi
  \fi}
%    \end{macrocode}
%    For all those languages for which the option name is the same as
%    the name of the language specific file we specify a default
%    option, which tries to load the file specified. If this doesn't
%    succeed an error is signalled.
% \changes{babel~3.6i}{1997/03/12}{Added default option}
% \changes{babel~3.9a}{1997/03/12}{Rewritten the error message}
%    \begin{macrocode}
\DeclareOption*{%
  \@expandtwoargs\in@{\string=}{\CurrentOption}%
  \ifin@\else
    \InputIfFileExists{\CurrentOption.ldf}%
     {}%
     {\PackageError{babel}{%
        Unknow option `\CurrentOption'. Either you misspelled it\MessageBreak
        or the language definition file \CurrentOption.ldf was not found}{%
        Valid options are: shorthands=..., KeepShorthandsActive,\MessageBreak
        activeacute, activegrave, noconfig, no..., no...,\MessageBreak
        or a valid language name.}}%
      \csname\CurrentOption.ldf-h@@k\endcsname
  \fi}
%    \end{macrocode}
%    Another way to extend the list of `known' options for \babel\ is
%    to create the file \file{bblopts.cfg} in which one can add option
%    declarations. However, this mechanism is deprecated -- if you
%    want an alternative name for a language, just create a new |.ldf|
%    file loading the actual one. You can also set the name
%    of the file with the package option |config=<name>|, which will
%    load |<name>.cfg| instead. 
% \changes{babel~3.6i}{1997/03/15}{Added the possibility to have a
%    \file{bblopts.cfg} file with option declarations.}
% \changes{babel~3.9a}{2012/06/28}{Added the \cs{AtEndOfLanguage}
%    mechanism, to be used mainly with the local cfg file.}
% \changes{babel~3.9a}{2012/06/31}{Now you can set the name of the
%    local cfg file.}
%    \begin{macrocode}
\def\AtEndOfLanguage#1{%
  \@ifundefined{#1.ldf-h@@k}%
    {\expandafter\let\csname#1.ldf-h@@k\endcsname\@empty}%
    {}%
    \expandafter\g@addto@macro\csname#1.ldf-h@@k\endcsname}
\ifx\bbl@opt@config\@nnil
  \InputIfFileExists{bblopts.cfg}%
    {\typeout{*************************************^^J%
             * Local config file bblopts.cfg used^^J%
             *}}%
    {}%
\else
  \InputIfFileExists{\bbl@opt@config.cfg}%
    {\typeout{*************************************^^J%
             * Local config file \bbl@opt@config.cfg used^^J%
             *}}%
    {\PackageError{babel}{%
       Local config file `\bbl@opt@config.cfg' not found}{%
       Perhaps you misspelled it.}}%
\fi
%    \end{macrocode}
%    The options have to be processed in the order in which the user
%    specified them:
%    \begin{macrocode}
\ProcessOptions*
%    \end{macrocode}
%    This finishes the second pass. Now the third one begins, which
%    loads the main language set with the key |main|. A warning [??
%    error] is raised if the main language is not the same as the last
%    named one, or if the value of the key |main| is not a language.
%    \begin{macrocode}
\ifx\bbl@opt@main\@nnil
  \ifx\bbl@lastoption\@undefined\else
    \ifx\bbl@lastoption\bbl@main@language\else
       \PackageWarning{babel}{%
         Last named language is `\bbl@lastoption', but the main\MessageBreak
         language has been set to `\bbl@main@language'. The main\MessageBreak
         language cannot be both a global and a package option.\MessageBreak
         Use `main=\bbl@lastoption' as package option.\MessageBreak
         Reported }%
    \fi
  \fi
\else
  \ifx\bbl@loadmain\@undefined
    \PackageError{babel}{%
      Unknown language `\bbl@opt@main' in key `main'}{!!!!!}%
  \else
    \bbl@loadmain
    \DeclareOption*{}
    \ProcessOptions*
  \fi
\fi
%    \end{macrocode}
% \changes{babel~3.7c}{1999/03/13}{Added an error message for when no
%    language option was specified}
%    In order to catch the case where the user forgot to specify a
%    language we check whether |\bbl@main@language|, has become
%    defined. If not, no language has been loaded and an error
%    message is displayed.
% \changes{babel~3.7c}{1999/04/09}{No longer us a redefinition of an
%    internal macro, just check \cs{bbl@main@language} and load
%    \file{babel.def}}
% \changes{babel~3.9a}{2012/06/24}{Now babel is not loaded to prevent
%    the document from raising errors after fixing it}
%    \begin{macrocode}
\ifx\bbl@main@language\@undefined
  \PackageError{babel}{%
    You haven't specified a language option}{%
    You need to specify a language, either as a global
    option\MessageBreak
    or as an optional argument to the \string\usepackage\space
    command; \MessageBreak
    You shouldn't try to proceed from here, type x to quit.}
\fi
%    \end{macrocode}
%
%  \begin{macro}{\substitutefontfamily}
%    The command |\substitutefontfamily| creates an \file{.fd} file on
%    the fly. The first argument is an encoding mnemonic, the second
%    and third arguments are font family names.
% \changes{babel~3.7j}{2003/06/15}{create file with lowercase name}
%    \begin{macrocode}
\def\substitutefontfamily#1#2#3{%
  \lowercase{\immediate\openout15=#1#2.fd\relax}%
  \immediate\write15{%
    \string\ProvidesFile{#1#2.fd}%
    [\the\year/\two@digits{\the\month}/\two@digits{\the\day}
     \space generated font description file]^^J
    \string\DeclareFontFamily{#1}{#2}{}^^J
    \string\DeclareFontShape{#1}{#2}{m}{n}{<->ssub * #3/m/n}{}^^J
    \string\DeclareFontShape{#1}{#2}{m}{it}{<->ssub * #3/m/it}{}^^J
    \string\DeclareFontShape{#1}{#2}{m}{sl}{<->ssub * #3/m/sl}{}^^J
    \string\DeclareFontShape{#1}{#2}{m}{sc}{<->ssub * #3/m/sc}{}^^J
    \string\DeclareFontShape{#1}{#2}{b}{n}{<->ssub * #3/bx/n}{}^^J
    \string\DeclareFontShape{#1}{#2}{b}{it}{<->ssub * #3/bx/it}{}^^J
    \string\DeclareFontShape{#1}{#2}{b}{sl}{<->ssub * #3/bx/sl}{}^^J
    \string\DeclareFontShape{#1}{#2}{b}{sc}{<->ssub * #3/bx/sc}{}^^J
    }%
  \closeout15
  }
%    \end{macrocode}
%    This command should only be used in the preamble of a document.
%    \begin{macrocode}
\@onlypreamble\substitutefontfamily
%    \end{macrocode}
%  \end{macro}
%
%    \begin{macrocode}
%</package>
%    \end{macrocode}
%
% \section{The Kernel of Babel}
%
%    The kernel of the \babel\ system is stored in either
%    \file{hyphen.cfg} or \file{switch.def} and \file{babel.def}. The
%    file \file{hyphen.cfg} is a file that can be loaded into the
%    format, which is necessary when you want to be able to switch
%    hyphenation patterns. The file \file{babel.def} contains some
%    \TeX\ code that can be read in at run time. When \file{babel.def}
%    is loaded it checks if \file{hyphen.cfg} is in the format; if
%    not the file \file{switch.def} is loaded.
%
%    Because plain \TeX\ users might want to use some of the features
%    of the \babel{} system too, care has to be taken that plain \TeX\
%    can process the files. For this reason the current format will
%    have to be checked in a number of places. Some of the code below
%    is common to plain \TeX\ and \LaTeX, some of it is for the
%    \LaTeX\ case only.
%
%    When the command |\AtBeginDocument| doesn't exist we assume that
%    we are dealing with a plain-based format. In that case the file
%    \file{plain.def} is needed.
%
%    \begin{macrocode}
%<*kernel|core>
\ifx\AtBeginDocument\@undefined
%    \end{macrocode}
%    But we need to use the second part of \file{plain.def} (when we
%    load it from \file{switch.def}) which we can do by defining
%    |\adddialect|.
% \changes{babel~3.7c}{1999/04/20}{define \cs{adddialect} before
%    loading \file{plain.def} here}
%    \begin{macrocode}
%<kernel&!patterns>  \def\adddialect{}
  \input plain.def\relax
\fi
%</kernel|core>
%    \end{macrocode}
%
%    Check the presence of the command |\iflanguage|, if it is
%    undefined read the file \file{switch.def}.
% \changes{babel~3.0d}{1991/10/29}{Removed use of \cs{@ifundefined}}
%    \begin{macrocode}
%<*core>
\ifx\iflanguage\@undefined
  \input switch.def\relax
\fi
%</core>
%    \end{macrocode}
% \changes{babel~3.6a}{1996/11/02}{Removed \cs{babel@core@loaded}, no
%    longer needed with the advent of \cs{LdfInit}}
%
%  \subsection{Encoding issues (part 1)}
%
%    The first thing we need to do is to determine, at
%    |\begin{document}|, which latin fontencoding to use.
%
%  \begin{macro}{\latinencoding}
% \changes{babel~3.6i}{1997/03/15}{Macro added, moved from
%    \file{.ldf} files}
%    When text is being typeset in an encoding other than `latin'
%    (\texttt{OT1} or \texttt{T1}), it would be nice to still have
%    Roman numerals come out in the Latin encoding.
%    So we first assume that the current encoding at the end
%    of processing the package is the Latin encoding.
%    \begin{macrocode}
%<*core>
\AtEndOfPackage{\edef\latinencoding{\cf@encoding}}
%    \end{macrocode}
%    But this might be overruled with a later loading of the package
%    \pkg{fontenc}. Therefor we check at the execution of
%    |\begin{document}| whether it was loaded with the \Lopt{T1}
%    option. The normal way to do this (using |\@ifpackageloaded|) is
%    disabled for this package. Now we have to revert to parsing the
%    internal macro |\@filelist| which contains all the filenames
%    loaded.
% \changes{babel~3.6k}{1999/03/15}{Use T1 encoding when it is a known
%    encoding}
% \changes{babel~3.6m}{1999/04/06}{Can't use \cs{@ifpackageloaded}
%    need to parse \cs{@filelist}}
% \changes{babel~3.6n}{1999/04/07}{moved checking for fontenc right to
%    the top of \file{babel.sty}}
% \changes{babel~3.6n}{1999/04/07}{Added a check for `manual' selection
%    of \texttt{T1} encoding, without loading \pkg{fontenc}}
% \changes{babel~3.6q}{1999/04/12}{Better solution then parsing
%    \cs{@filelist}, use \cs{@ifl@aded}}
% \changes{babel~3.6u}{1999/04/20}{Moved this code to
%    \file{babel.def}}
%    \begin{macrocode}
\AtBeginDocument{%
  \gdef\latinencoding{OT1}%
  \ifx\cf@encoding\bbl@t@one
    \xdef\latinencoding{\bbl@t@one}%
  \else
    \@ifl@aded{def}{t1enc}{\xdef\latinencoding{\bbl@t@one}}{}%
  \fi
  }
%    \end{macrocode}
%  \end{macro}
%
%  \begin{macro}{\latintext}
% \changes{babel~3.6i}{1997/03/15}{Macro added, moved from
%    \file{.ldf} files}
%    Then we can define the command |\latintext| which is a
%    declarative switch to a latin font-encoding.
%    \begin{macrocode}
\DeclareRobustCommand{\latintext}{%
  \fontencoding{\latinencoding}\selectfont
  \def\encodingdefault{\latinencoding}}
%    \end{macrocode}
%  \end{macro}
%
%  \begin{macro}{\textlatin}
% \changes{babel~3.6i}{1997/03/15}{Macro added, moved from
%    \file{.ldf} files}
% \changes{babel~3.7j}{2003/03/19}{added \cs{leavevmode} to prevent a
%    paragraph starting \emph{inside} the group}
% \changes{babel~3.7k}{2003/10/12}{Use \cs{DeclareTextFontComand}}
%    This command takes an argument which is then typeset using the
%    requested font encoding. In order to avoid many encoding switches
%    it operates in a local scope.
%    \begin{macrocode}
\ifx\@undefined\DeclareTextFontCommand
  \DeclareRobustCommand{\textlatin}[1]{\leavevmode{\latintext #1}}
\else
  \DeclareTextFontCommand{\textlatin}{\latintext}
\fi
%</core>
%    \end{macrocode}
%  \end{macro}
%
%    We also need to redefine a number of commands to ensure that the
%    right font encoding is used, but this can't be done before
%    \file{babel.def} is loaded.
% \changes{babel~3.6o}{1999/04/07}{Moved the rest of the font encoding
%    related definitions to their original place}
%
% \subsection{Multiple languages}
%
%    With \TeX\ version~3.0 it has become possible to load hyphenation
%    patterns for more than one language. This means that some extra
%    administration has to be taken care of.  The user has to know for
%    which languages patterns have been loaded, and what values of
%    |\language| have been used.
%
%    Some discussion has been going on in the \TeX\ world about how to
%    use |\language|. Some have suggested to set a fixed standard,
%    i.\,e., patterns for each language should \emph{always} be loaded
%    in the same location. It has also been suggested to use the
%    \textsc{iso} list for this purpose. Others have pointed out that
%    the \textsc{iso} list contains more than 256~languages, which
%    have \emph{not} been numbered consecutively.
%
%    I think the best way to use |\language|, is to use it
%    dynamically.  This code implements an algorithm to do so. It uses
%    an external file in which the person who maintains a \TeX\
%    environment has to record for which languages he has hyphenation
%    patterns \emph{and} in which files these are stored\footnote{This
%    is because different operating systems sometimes use \emph{very}
%    different file-naming conventions.}. When hyphenation exceptions
%    are stored in a separate file this can be indicated by naming
%    that file \emph{after} the file with the hyphenation patterns.
%
%    This ``configuration file'' can contain empty lines and comments,
%    as well as lines which start with an equals (\texttt{=})
%    sign. Such a line will instruct \LaTeX\ that the hyphenation
%    patterns just processed have to be known under an alternative
%    name. Here is an example:
%  \begin{verbatim}
%    % File    : language.dat
%    % Purpose : tell iniTeX what files with patterns to load.
%    english    english.hyphenations
%    =british
%
%    dutch      hyphen.dutch exceptions.dutch % Nederlands
%    german hyphen.ger
%  \end{verbatim}
%
%    As the file \file{switch.def} needs to be read only once, we
%    check whether it was read before.  If it was, the command
%    |\iflanguage| is already defined, so we can stop processing.
%    \begin{macrocode}
%<*kernel>
%<*!patterns>
\expandafter\ifx\csname iflanguage\endcsname\relax \else
\expandafter\endinput
\fi
%</!patterns>
%    \end{macrocode}
%
%  \begin{macro}{\language}
%    Plain \TeX\ version~3.0 provides the primitive |\language| that
%    is used to store the current language. When used with a pre-3.0
%    version this function has to be implemented by allocating a
%    counter.
%    \begin{macrocode}
\ifx\language\@undefined
  \csname newcount\endcsname\language
\fi
%    \end{macrocode}
%  \end{macro}
%
%  \begin{macro}{\last@language}
%    Another counter is used to store the last language defined.  For
%    pre-3.0 formats an extra counter has to be allocated,
%    \begin{macrocode}
\ifx\newlanguage\@undefined
  \csname newcount\endcsname\last@language
%    \end{macrocode}
%    plain \TeX\ version 3.0 uses |\count 19| for this purpose.
%    \begin{macrocode}
\else
  \countdef\last@language=19
\fi
%    \end{macrocode}
%  \end{macro}
%
%  \begin{macro}{\addlanguage}
%
%    To add languages to \TeX's memory plain \TeX\ version~3.0
%    supplies |\newlanguage|, in a pre-3.0 environment a similar macro
%    has to be provided. For both cases a new macro is defined here,
%    because the original |\newlanguage| was defined to be |\outer|.
%
%    For a format based on plain version~2.x, the definition of
%    |\newlanguage| can not be copied because |\count 19| is used for
%    other purposes in these formats. Therefor |\addlanguage| is
%    defined using a definition based on the macros used to define
%    |\newlanguage| in plain \TeX\ version~3.0.
% \changes{babel~3.2}{1991/11/11}{Added a \texttt{\%}, removed
%    \texttt{by}}
%    \begin{macrocode}
\ifx\newlanguage\@undefined
  \def\addlanguage#1{%
    \global\advance\last@language \@ne
    \ifnum\last@language<\@cclvi
    \else
        \errmessage{No room for a new \string\language!}%
    \fi
    \global\chardef#1\last@language
    \wlog{\string#1 = \string\language\the\last@language}}
%    \end{macrocode}
%
%    For formats based on plain version~3.0 the definition of
%    |\newlanguage| can be simply copied, removing |\outer|.
%
%    \begin{macrocode}
\else
  \def\addlanguage{\alloc@9\language\chardef\@cclvi}
\fi
%    \end{macrocode}
%  \end{macro}
%
%  \begin{macro}{\adddialect}
%    The macro |\adddialect| can be used to add the name of a dialect
%    or variant language, for which an already defined hyphenation
%    table can be used.
% \changes{babel~3.2}{1991/11/11}{Added \cs{relax}}
%    \begin{macrocode}
\def\adddialect#1#2{%
    \global\chardef#1#2\relax
    \wlog{\string#1 = a dialect from \string\language#2}}
%    \end{macrocode}
%  \end{macro}
%
%  \begin{macro}{\iflanguage}
%    Users might want to test (in a private package for instance)
%    which language is currently active. For this we provide a test
%    macro, |\iflanguage|, that has three arguments.  It checks
%    whether the first argument is a known language. If so, it
%    compares the first argument with the value of |\language|. Then,
%    depending on the result of the comparison, it executes either the
%    second or the third argument.
% \changes{babel~3.0a}{1991/05/29}{Added \cs{@bsphack} and
%    \cs{@esphack}}
% \changes{babel~3.0c}{1991/07/21}{Added comment character after
%    \texttt{\#2}}
% \changes{babel~3.0d}{1991/08/08}{Removed superfluous
%    \cs{expandafter}}
% \changes{babel~3.0d}{1991/10/07}{Removed space hacks and use of
%    \cs{@ifundefined}}
% \changes{babel~3.2}{1991/11/11}{Rephrased \cs{ifnum} test}
% \changes{babel~3.7a}{1998/06/10}{Now evaluate the \cs{ifnum} test
%    \emph{after} the \cs{fi} from the \cs{ifx} test and use
%    \cs{@firstoftwo} and \cs{@secondoftwo}}
% \changes{babel~3.7b}{1998/06/29}{Slight enhancement: added braces
%    around first argument of \cs{bbl@afterfi}}
%    \begin{macrocode}
\def\iflanguage#1{%
  \expandafter\ifx\csname l@#1\endcsname\relax
    \@nolanerr{#1}%
  \else
    \bbl@afterfi{\ifnum\csname l@#1\endcsname=\language
      \expandafter\@firstoftwo
    \else
      \expandafter\@secondoftwo
    \fi}%
  \fi}
%    \end{macrocode}
%  \end{macro}
%
%  \begin{macro}{\selectlanguage}
%    The macro |\selectlanguage| checks whether the language is
%    already defined before it performs its actual task, which is to
%    update |\language| and activate language-specific definitions.
%
%    To allow the call of |\selectlanguage| either with a control
%    sequence name or with a simple string as argument, we have to use
%    a trick to delete the optional escape character.
%
%    To convert a control sequence to a string, we use the |\string|
%    primitive.  Next we have to look at the first character of this
%    string and compare it with the escape character.  Because this
%    escape character can be changed by setting the internal integer
%    |\escapechar| to a character number, we have to compare this
%    number with the character of the string.  To do this we have to
%    use \TeX's backquote notation to specify the character as a
%    number.
%
%    If the first character of the |\string|'ed argument is the
%    current escape character, the comparison has stripped this
%    character and the rest in the `then' part consists of the rest of
%    the control sequence name.  Otherwise we know that either the
%    argument is not a control sequence or |\escapechar| is set to a
%    value outside of the character range~$0$--$255$.
%
%    If the user gives an empty argument, we provide a default
%    argument for |\string|.  This argument should expand to nothing.
%
% \changes{babel~3.0c}{1991/06/06}{Made \cs{selectlanguage}
%    robust}
% \changes{babel~3.2}{1991/11/11}{Modified to allow arguments that
%    start with an escape character}
% \changes{babel~3.2a}{1991/11/17}{Simplified the modification to
%    allow the use in a \cs{write} command}
% \changes{babel~3.5b}{1995/05/13}{Store the name of the current
%    language in a control sequence instead of passing the whole macro
%    construct to strip the escape character in the argument of
%    \cs{selectlanguage }.}
% \changes{babel~3.5f}{1995/08/30}{Added a missing percent character}
% \changes{babel~3.5f}{1995/11/16}{Moved check for escape character
%    one level down in the expansion}
%    \begin{macrocode}
\edef\selectlanguage{%
  \noexpand\protect
  \expandafter\noexpand\csname selectlanguage \endcsname
  }
%    \end{macrocode}
%    Because the command |\selectlanguage| could be used in a moving
%    argument it expands to \verb*=\protect\selectlanguage =.
%    Therefor, we have to make sure that a macro |\protect| exists.
%    If it doesn't it is |\let| to |\relax|.
%    \begin{macrocode}
\ifx\@undefined\protect\let\protect\relax\fi
%    \end{macrocode}
%    As \LaTeX$\:$2.09 writes to files \textit{expanded} whereas
%    \LaTeXe\ takes care \textit{not} to expand the arguments of
%    |\write| statements we need to be a bit clever about the way we
%    add information to \file{.aux} files. Therefor we introduce the
%    macro |\xstring| which should expand to the right amount of
%    |\string|'s.
%    \begin{macrocode}
\ifx\documentclass\@undefined
  \def\xstring{\string\string\string}
\else
  \let\xstring\string
\fi
%    \end{macrocode}
%
% \changes{babel~3.5b}{1995/03/04}{Changed the name of the internal
%    macro to \cs{selectlanguage }.}
% \changes{babel~3.5b}{1995/03/05}{Added an extra level of expansion to
%    separate the switching mechanism from writing to aux files}
% \changes{babel~3.7f}{2000/09/25}{Use \cs{aftergroup} to keep the
%    language grouping correct in auxiliary files {PR3091}}
%    Since version 3.5 \babel\ writes entries to the auxiliary files in
%    order to typeset table of contents etc. in the correct language
%    environment.
%  \begin{macro}{\bbl@pop@language}
%    \emph{But} when the language change happens \emph{inside} a group
%    the end of the group doesn't write anything to the auxiliary
%    files. Therefor we need \TeX's |aftergroup| mechanism to help
%    us. The command |\aftergroup| stores the token immediately
%    following it to be executed when the current group is closed. So
%    we define a temporary control sequence |\bbl@pop@language| to be
%    executed at the end of the group. It calls |\bbl@set@language|
%    with the name of the current language as its argument.
%
% \changes{babel~3.7j}{2003/03/18}{Introduce the language stack
%    mechanism}
%  \begin{macro}{\bbl@language@stack}
%    The previous solution works for one level of nesting groups, but
%    as soon as more levels are used it is no longer adequate. For
%    that case we need to keep track of the nested languages using a
%    stack mechanism. This stack is called |\bbl@language@stack| and
%    initially empty.
%    \begin{macrocode}
\xdef\bbl@language@stack{}
%    \end{macrocode}
%    When using a stack we need a mechanism to push an element on the
%    stack and to retrieve the information afterwards.
%  \begin{macro}{\bbl@push@language}
%  \begin{macro}{\bbl@pop@language}
%    The stack is simply a list of languagenames, separated with a `+'
%    sign; the push function can be simple:
%    \begin{macrocode}
\def\bbl@push@language{%
  \xdef\bbl@language@stack{\languagename+\bbl@language@stack}%
  }
%    \end{macrocode}
%    Retrieving information from the stack is a little bit less simple,
%    as we need to remove the element from the stack while storing it
%    in the macro |\languagename|. For this we first define a helper function.
%  \begin{macro}{\bbl@pop@lang}
%    This macro stores its first element (which is delimited by the
%    `+'-sign) in |\languagename| and stores the rest of the string
%    (delimited by `-') in its third argument.
%    \begin{macrocode}
\def\bbl@pop@lang#1+#2-#3{%
  \def\languagename{#1}\xdef#3{#2}%
  }
%    \end{macrocode}
%  \end{macro}
%    The reason for the somewhat weird arrangement of arguments to the
%    helper function is the fact it is called in the following way:
%    \begin{macrocode}
\def\bbl@pop@language{%
  \expandafter\bbl@pop@lang\bbl@language@stack-\bbl@language@stack
%    \end{macrocode}
%    This means that before |\bbl@pop@lang| is executed \TeX\ first
%    \emph{expands} the stack, stored in |\bbl@language@stack|. The
%    result of that is that the argument string of |\bbl@pop@lang|
%    contains one or more language names, each followed by a `+'-sign
%    (zero language names won't occur as this macro will only be
%    called after something has been pushed on the stack) followed by
%    the `-'-sign and finally the reference to the stack.
%    \begin{macrocode}$$
  \expandafter\bbl@set@language\expandafter{\languagename}%
  }
%    \end{macrocode}
%    Once the name of the previous language is retrieved from the stack,
%    it is fed to |\bbl@set@language| to do the actual work of
%    switching everything that needs switching.
%  \end{macro}
%  \end{macro}
%  \end{macro}
%
% \changes{babel~3.7j}{2003/03/18}{Now use the language stack mechanism}
%    \begin{macrocode}
\expandafter\def\csname selectlanguage \endcsname#1{%
  \bbl@push@language
  \aftergroup\bbl@pop@language
  \bbl@set@language{#1}}
%    \end{macrocode}
% \changes{babel~3.7m}{2003/11/12}{Removed the superfluous empty
%    definition of \cs{bbl@pop@language}}
%  \end{macro}
%
%  \begin{macro}{\bbl@set@language}
% \changes{babel~3.5f}{1995/11/16}{Now also define \cs{languagename}
%    at this level}
% \changes{babel~3.7f}{2000/09/25}{Macro \cs{bbl@set@language}
%    introduced}
%    The macro |\bbl@set@language| takes care of switching the language
%    environment \emph{and} of writing entries on the auxiliary files.
%    \begin{macrocode}
\def\bbl@set@language#1{%
  \edef\languagename{%
    \ifnum\escapechar=\expandafter`\string#1\@empty
    \else \string#1\@empty\fi}%
  \select@language{\languagename}%
%    \end{macrocode}
%    We also write a command to change the current language in the
%    auxiliary files.
% \changes{babel~3.5a}{1995/02/17}{write the language change to the
%    auxiliary files}
%    \begin{macrocode}
  \if@filesw
    \protected@write\@auxout{}{\string\select@language{\languagename}}%
    \addtocontents{toc}{\xstring\select@language{\languagename}}%
    \addtocontents{lof}{\xstring\select@language{\languagename}}%
    \addtocontents{lot}{\xstring\select@language{\languagename}}%
  \fi}
%    \end{macrocode}
%  \end{macro}
%
%    First, check if the user asks for a known language. If so,
%    update the value of |\language| and call |\originalTeX|
%    to bring \TeX\ in a certain pre-defined state.
% \changes{babel~3.0a}{1991/05/29}{Added \cs{@bsphack} and
%    \cs{@esphack}}
% \changes{babel~3.0d}{1991/08/08}{Removed superfluous
%    \cs{expandafter}}
% \changes{babel~3.0d}{1991/10/07}{Removed space hacks and use of
%    \cs{@ifundefined}}
% \changes{babel~3.2a}{1991/11/17}{Added \cs{relax} as first command
%    to stop an expansion if \cs{protect} is empty}
% \changes{babel~3.6a}{1996/11/07}{Check for the existence of
%    \cs{date...} instead of \cs{l@...}}
% \changes{babel~3.7m}{2003/11/16}{Check for the existence of both
%    \cs{l@...} and \cs{date...}}
% \changes{babel~3.8l}{2008/07/06}{Use \cs{bbl@patterns}}
%    \begin{macrocode}
\def\select@language#1{%
  \expandafter\ifx\csname l@#1\endcsname\relax
    \@nolanerr{#1}%
  \else
    \expandafter\ifx\csname date#1\endcsname\relax
      \@noopterr{#1}%
    \else
      \bbl@patterns{\languagename}%
      \originalTeX
%    \end{macrocode}
%    The name of the language is stored in the control sequence
%    |\languagename|. The contents of this control sequence could be
%    tested in the following way:
%  \begin{verbatim}
%    \edef\tmp{\string english}
%    \ifx\languagename\tmp
%        ...
%    \else
%        ...
%    \fi
%  \end{verbatim}
%    The construction with |\string| is necessary because
%    |\languagename| returns the name with characters of category code
%    \texttt{12} (other).  Then we have to \emph{re}define
%    |\originalTeX| to compensate for the things that have been
%    activated.  To save memory space for the macro definition of
%    |\originalTeX|, we construct the control sequence name for the
%    |\noextras|\langvar\ command at definition time by expanding the
%    |\csname| primitive.
% \changes{babel~3.0a}{1991/06/06}{Replaced \cs{gdef} with \cs{def}}
% \changes{babel~3.1}{1991/10/31}{\cs{originalTeX} should only be
%    executed once}
% \changes{babel~3.2a}{1991/11/17}{Added three \cs{expandafter}s
%    to save macro space for \cs{originalTeX}}
% \changes{babel~3.2a}{1991/11/20}{Moved definition of
%    \cs{originalTeX} before \cs{extras\langvar}}
% \changes{babel~3.2a}{1991/11/24}{Set \cs{originalTeX} to
%    \cs{empty}, because it should be expandable.}
%    \begin{macrocode}
      \expandafter\def\expandafter\originalTeX
          \expandafter{\csname noextras#1\endcsname
                       \let\originalTeX\@empty}%
%    \end{macrocode}
% \changes{babel~3.6d}{1997/01/07}{set the language shorthands to
%    `none' before switching on the extras}
%    \begin{macrocode}
      \languageshorthands{none}%
      \babel@beginsave
%    \end{macrocode}
%    Now activate the language-specific definitions. This is done by
%    constructing the names of three macros by concatenating three
%    words with the argument of |\selectlanguage|, and calling these
%    macros.
% \changes{babel~3.5b}{1995/05/13}{Separated the setting of the
%    hyphenmin values}
%    \begin{macrocode}
      \csname captions#1\endcsname
      \csname date#1\endcsname
      \csname extras#1\endcsname\relax
%    \end{macrocode}
%    The switching of the values of |\lefthyphenmin| and
%    |\righthyphenmin| is somewhat different. First we save their
%    current values, then we check if |\|\langvar|hyphenmins| is
%    defined. If it is not, we set default values (2 and 3), otherwise
%    the values in |\|\langvar|hyphenmins| will be used.
% \changes{babel~3.5b}{1995/06/05}{Addedd default setting of hyphenmin
%    parameters}
%    \begin{macrocode}
      \babel@savevariable\lefthyphenmin
      \babel@savevariable\righthyphenmin
      \expandafter\ifx\csname #1hyphenmins\endcsname\relax
        \set@hyphenmins\tw@\thr@@\relax
      \else
        \expandafter\expandafter\expandafter\set@hyphenmins
          \csname #1hyphenmins\endcsname\relax
      \fi
    \fi
  \fi}
%    \end{macrocode}
%  \end{macro}
%
%  \begin{environment}{otherlanguage}
%    The \Lenv{otherlanguage} environment can be used as an
%    alternative to using the |\selectlanguage| declarative
%    command. When you are typesetting a document which mixes
%    left-to-right and right-to-left typesetting you have to use this
%    environment in order to let things work as you expect them to.
%
%    The first thing this environment does is store the name of the
%    language in |\languagename|; it then calls
%    \verb*=\selectlanguage = to switch on everything that is needed for
%    this language The |\ignorespaces| command is necessary to hide
%    the environment when it is entered in horizontal mode.
% \changes{babel~3.5d}{1995/06/22}{environment added}
% \changes{babel~3.5e}{1995/07/07}{changed name}
% \changes{babel~3.7j}{2003/03/18}{rely on \cs{selectlanguage } to
%    keep track of the nesting}
%    \begin{macrocode}
\long\def\otherlanguage#1{%
  \csname selectlanguage \endcsname{#1}%
  \ignorespaces
  }
%    \end{macrocode}
%    The |\endotherlanguage| part of the environment calls
%    |\originalTeX| to restore (most of) the settings and tries to
%    hide itself when it is called in horizontal mode.
%    \begin{macrocode}
\long\def\endotherlanguage{%
  \originalTeX
  \global\@ignoretrue\ignorespaces
  }
%    \end{macrocode}
%  \end{environment}
%
%
%  \begin{environment}{otherlanguage*}
%    The \Lenv{otherlanguage} environment is meant to be used when a
%    large part of text from a different language needs to be typeset,
%    but without changing the translation of words such as `figure'.
%
%    This environment makes use of |\foreign@language|.
% \changes{babel~3.5f}{1996/05/29}{environment added}
% \changes{babel~3.6d}{1997/01/07}{Introduced \cs{foreign@language}}
%    \begin{macrocode}
\expandafter\def\csname otherlanguage*\endcsname#1{%
  \foreign@language{#1}%
  }
%    \end{macrocode}
%    At the end of the environment we need to switch off the extra
%    definitions. The grouping mechanism of the environment will take
%    care of resetting the correct hyphenation rules.
%    \begin{macrocode}
\expandafter\def\csname endotherlanguage*\endcsname{%
  \csname noextras\languagename\endcsname
  }
%    \end{macrocode}
%  \end{environment}
%
%  \begin{macro}{\foreignlanguage}
%    The |\foreignlanguage| command is another substitute for the
%    |\selectlanguage| command. This command takes two arguments, the
%    first argument is the name of the language to use for typesetting
%    the text specified in the second argument.
%
%    Unlike |\selectlanguage| this command doesn't switch
%    \emph{everything}, it only switches the hyphenation rules and the
%    extra definitions for the language specified. It does this within
%    a group and assumes the |\extras|\langvar\ command doesn't make
%    any |\global| changes. The coding is very similar to part of
%    |\selectlanguage|.
% \changes{babel~3.5d}{1995/06/22}{Macro added}
% \changes{babel~3.6d}{1997/01/07}{Introduced \cs{foreign@language}}
% \changes{babel~3.7a}{1998/03/12}{Added executing \cs{originalTeX}}
%    \begin{macrocode}
\def\foreignlanguage{\protect\csname foreignlanguage \endcsname}
\expandafter\def\csname foreignlanguage \endcsname#1#2{%
  \begingroup
    \originalTeX
    \foreign@language{#1}%
    #2%
    \csname noextras#1\endcsname
  \endgroup
  }
%    \end{macrocode}
%  \end{macro}
%
%  \begin{macro}{\foreign@language}
% \changes{babel~3.6d}{1997/01/07}{New macro}
%    This macro does the work for |\foreignlanguage| and the
%    \Lenv{otherlanguage*} environment.
%    \begin{macrocode}
\def\foreign@language#1{%
%    \end{macrocode}
%    First we need to store the name of the language and check that it
%    is a known language.
%    \begin{macrocode}
  \def\languagename{#1}%
  \expandafter\ifx\csname l@#1\endcsname\relax
    \@nolanerr{#1}%
  \else
%    \end{macrocode}
%    If it is we can select the proper hyphenation table and switch on
%    the extra definitions for this language.
% \changes{babel~3.6d}{1997/01/07}{set the language shorthands to
%    `none' before switching on the extras}
% \changes{babel~3.8l}{2008/07/06}{use \cs{bbl@patterns}}
%    \begin{macrocode}
    \bbl@patterns{\languagename}%
    \languageshorthands{none}%
%    \end{macrocode}
%    Then we set the left- and right hyphenmin variables.
% \changes{babel~3.6d}{1997/01/07}{Added \cs{relax} to prevent
%    disappearance of the first token after this command.}
%    \begin{macrocode}
    \csname extras#1\endcsname
    \expandafter\ifx\csname #1hyphenmins\endcsname\relax
      \set@hyphenmins\tw@\thr@@\relax
    \else
      \expandafter\expandafter\expandafter\set@hyphenmins
        \csname #1hyphenmins\endcsname\relax
    \fi
  \fi
  }
%    \end{macrocode}
%  \end{macro}
%
%  \begin{macro}{\bbl@patterns}
% \changes{babel~3.8l}{2008/07/06}{Macro added}
%    This macro selects the hyphenation patterns by changing the
%    \cs{language} register.  If special hyphenation patterns
%    are available specifically for the current font encoding,
%    use them instead of the default.
%    \begin{macrocode}
\def\bbl@patterns#1{%
  \language=\expandafter\ifx\csname l@#1:\f@encoding\endcsname\relax
    \csname l@#1\endcsname
  \else
    \csname l@#1:\f@encoding\endcsname
  \fi\relax
}
%    \end{macrocode}
%  \end{macro}
%
%  \begin{environment}{hyphenrules}
% \changes{babel~3.7e}{2000/01/28}{Added environment hyphenrules}
%    The environment \Lenv{hyphenrules} can be used to select
%    \emph{just} the hyphenation rules. This environment does
%    \emph{not} change |\languagename| and when the hyphenation rules
%    specified were not loaded it has no effect.
% \changes{babel~3.8j}{2008/03/16}{Also set the hyphenmin paramters to
%    the correct value (PR3997)} 
% \changes{babel~3.8l}{2008/07/06}{Use \cs{bbl@patterns}}
%    \begin{macrocode}
\def\hyphenrules#1{%
  \expandafter\ifx\csname l@#1\endcsname\@undefined
    \@nolanerr{#1}%
  \else
    \bbl@patterns{#1}%
    \languageshorthands{none}%
       \expandafter\ifx\csname #1hyphenmins\endcsname\relax
         \set@hyphenmins\tw@\thr@@\relax
       \else
         \expandafter\expandafter\expandafter\set@hyphenmins
         \csname #1hyphenmins\endcsname\relax
       \fi
  \fi
  }
\def\endhyphenrules{}
%    \end{macrocode}
%  \end{environment}
%
%  \begin{macro}{\providehyphenmins}
% \changes{babel~3.7f}{2000/02/18}{added macro}
%    The macro |\providehyphenmins| should be used in the language
%    definition files to provide a \emph{default} setting for the
%    hyphenation parameters |\lefthyphenmin| and |\righthyphenmin|. If
%    the macro |\|\langvar|hyphenmins| is already defined this command
%    has no effect.
%    \begin{macrocode}
\def\providehyphenmins#1#2{%
  \expandafter\ifx\csname #1hyphenmins\endcsname\relax
    \@namedef{#1hyphenmins}{#2}%
  \fi}
%    \end{macrocode}
%  \end{macro}
%
%  \begin{macro}{\set@hyphenmins}
%    This macro sets the values of |\lefthyphenmin| and
%    |\righthyphenmin|. It expects two values as its argument.
%    \begin{macrocode}
\def\set@hyphenmins#1#2{\lefthyphenmin#1\righthyphenmin#2}
%    \end{macrocode}
%  \end{macro}
%
%  \begin{macro}{\LdfInit}
% \changes{babel~3.6a}{1996/10/16}{Macro added}
%    This macro is defined in two versions. The first version is to be
%    part of the `kernel' of \babel, ie. the part that is loaded in
%    the format; the second version is defined in \file{babel.def}.
%    The version in the format just checks the category code of the
%    ampersand and then loads \file{babel.def}.
%    \begin{macrocode}
\def\LdfInit{%
  \chardef\atcatcode=\catcode`\@
  \catcode`\@=11\relax
  \input babel.def\relax
%    \end{macrocode}
%    The category code of the ampersand is restored and the macro
%    calls itself again with the new definition from
%    \file{babel.def}
%    \begin{macrocode}
  \catcode`\@=\atcatcode \let\atcatcode\relax
  \LdfInit}
%</kernel>
%    \end{macrocode}
%    The second version of this macro takes two arguments. The first
%    argument is the name of the language that will be defined in the
%    language definition file; the second argument is either a control
%    sequence or a string from which a control sequence should be
%    constructed. The existence of the control sequence indicates that
%    the file has been processed before.
%
%    At the start of processing a language definition file we always
%    check the category code of the ampersand. We make sure that it is
%    a `letter' during the processing of the file.
%    \begin{macrocode}
%<*core>
\def\LdfInit#1#2{%
  \chardef\atcatcode=\catcode`\@
  \catcode`\@=11\relax
%    \end{macrocode}
%    Another character that needs to have the correct category code
%    during processing of language definition files is the equals sign,
%    `=', because it is sometimes used in constructions with the
%    |\let| primitive. Therefor we store its current catcode and
%    restore it later on.
% \changes{babel~3.7o}{2003/11/26}{make sure the equals sign has its
%    default category code}
%    \begin{macrocode}
  \chardef\eqcatcode=\catcode`\=
  \catcode`\==12\relax
%    \end{macrocode}
%    Now we check whether we should perhaps stop the processing of
%    this file. To do this we first need to check whether the second
%    argument that is passed to |\LdfInit| is a control sequence. We
%    do that by looking at the first token after passing |#2| through
%    |string|. When it is equal to |\@backslashchar| we are dealing
%    with a control sequence which we can compare with |\@undefined|.
%    \begin{macrocode}
  \let\bbl@tempa\relax
  \expandafter\if\expandafter\@backslashchar
                 \expandafter\@car\string#2\@nil
    \ifx#2\@undefined
    \else
%    \end{macrocode}
%    If so, we call |\ldf@quit| (but after the end of this |\if|
%    construction) to set the main language, restore the category code
%    of the @-sign and call |\endinput|.
%    \begin{macrocode}
      \def\bbl@tempa{\ldf@quit{#1}}
    \fi
  \else
%    \end{macrocode}
%    When |#2| was \emph{not} a control sequence we construct one and
%    compare it with |\relax|.
%    \begin{macrocode}
    \expandafter\ifx\csname#2\endcsname\relax
    \else
      \def\bbl@tempa{\ldf@quit{#1}}
    \fi
  \fi
  \bbl@tempa
%    \end{macrocode}
%    Finally we check |\originalTeX|.
%    \begin{macrocode}
  \ifx\originalTeX\@undefined
    \let\originalTeX\@empty
  \else
    \originalTeX
  \fi}
%    \end{macrocode}
%  \end{macro}
%
%  \begin{macro}{\ldf@quit}
% \changes{babel~3.6a}{1996/10/29}{Macro added}
%    This macro interrupts the processing of a language definition file.
% \changes{babel~3.7o}{2003/11/26}{Also restore the category code of
%    the equals sign}
%    \begin{macrocode}
\def\ldf@quit#1{%
  \expandafter\main@language\expandafter{#1}%
  \catcode`\@=\atcatcode \let\atcatcode\relax
  \catcode`\==\eqcatcode \let\eqcatcode\relax
  \endinput
}
%    \end{macrocode}
%  \end{macro}
%
%  \begin{macro}{\ldf@finish}
% \changes{babel~3.6a}{1996/10/16}{Macro added}
%    This macro takes one argument. It is the name of the language
%    that was defined in the language definition file.
%
%    We load the local configuration file if one is present, we set
%    the main language (taking into account that the argument might be
%    a control sequence that needs to be expanded) and reset the
%    category code of the @-sign.
% \changes{babel~3.7o}{2003/11/26}{Also restore the category code of
%    the equals sign}
%    \begin{macrocode}
\def\ldf@finish#1{%
  \loadlocalcfg{#1}%
  \expandafter\main@language\expandafter{#1}%
  \catcode`\@=\atcatcode \let\atcatcode\relax
  \catcode`\==\eqcatcode \let\eqcatcode\relax
  }
%    \end{macrocode}
%  \end{macro}
%
%    After the preamble of the document the commands |\LdfInit|,
%    |\ldf@quit| and |\ldf@finish| are no longer needed. Therefor
%    they are turned into warning messages in \LaTeX.
%    \begin{macrocode}
\@onlypreamble\LdfInit
\@onlypreamble\ldf@quit
\@onlypreamble\ldf@finish
%    \end{macrocode}
%
%  \begin{macro}{\main@language}
% \changes{babel~3.5a}{1995/02/17}{Macro added}
% \changes{babel~3.6a}{1996/10/16}{\cs{main@language} now also sets
%    \cs{languagename} and \cs{l@languagename} for use by other
%    packages in the preamble of a document}
%  \begin{macro}{\bbl@main@language}
% \changes{babel~3.5a}{1995/02/17}{Macro added}
%    This command should be used in the various language definition
%    files. It stores its argument in |\bbl@main@language|; to be used
%    to switch to the correct language at the beginning of the
%    document.
% \changes{babel~3.8l}{2008/07/06}{Use \cs{bbl@patterns}}
%    \begin{macrocode}
\def\main@language#1{%
  \def\bbl@main@language{#1}%
  \let\languagename\bbl@main@language
  \bbl@patterns{\languagename}%
  }
%    \end{macrocode}
%    The default is to use English as the main language.
% \changes{babel~3.6c}{1997/01/05}{When \file{hyphen.cfg} is not
%    loaded in the format \cs{l@english} might not be defined; assume
%    english is language 0}
% \changes{babel~3.9a}{2012-05-17}{Languages are best assigned with
%    \cs{chardef}, not \cs{let}}
%    \begin{macrocode}
\ifx\l@english\@undefined
  \chardef\l@english\z@
\fi
\main@language{english}
%    \end{macrocode}
%    We also have to make sure that some code gets executed at the
%    beginning of the document.
%    \begin{macrocode}
\AtBeginDocument{%
  \expandafter\selectlanguage\expandafter{\bbl@main@language}}
%</core>
%    \end{macrocode}
%  \end{macro}
%  \end{macro}
%
%  \begin{macro}{\originalTeX}
%    The macro|\originalTeX| should be known to \TeX\ at this moment.
%    As it has to be expandable we |\let| it to |\@empty| instead of
%    |\relax|.
% \changes{babel~3.2a}{1991/11/24}{Set \cs{originalTeX} to
%    \cs{empty}, because it should be expandable.}
%    \begin{macrocode}
%<*kernel>
\ifx\originalTeX\@undefined\let\originalTeX\@empty\fi
%    \end{macrocode}
%    Because this part of the code can be included in a format, we
%    make sure that the macro which initialises the save mechanism,
%    |\babel@beginsave|, is not considered to be undefined.
%    \begin{macrocode}
\ifx\babel@beginsave\@undefined\let\babel@beginsave\relax\fi
%    \end{macrocode}
%  \end{macro}
%
%  \begin{macro}{\@nolanerr}
% \changes{babel~3.4e}{1994/06/25}{Use \cs{PackageError} in \LaTeXe\
%    mode}
%  \begin{macro}{\@nopatterns}
% \changes{babel~3.4e}{1994/06/25}{Macro added}
%    The \babel\ package will signal an error when a documents tries
%    to select a language that hasn't been defined earlier. When a
%    user selects a language for which no hyphenation patterns were
%    loaded into the format he will be given a warning about that
%    fact. We revert to the patterns for |\language|=0 in that case.
%    In most formats that will be (US)english, but it might also be
%    empty.
%  \begin{macro}{\@noopterr}
% \changes{babel~3.7m}{2003/11/16}{Macro added}
%    When the package was loaded without options not everything will
%    work as expected. An error message is issued in that case.
%
%    When the format knows about |\PackageError| it must be \LaTeXe,
%    so we can safely use its error handling interface. Otherwise
%    we'll have to `keep it simple'.
% \changes{babel~3.0d}{1991/10/07}{Added a percent sign to remove
%    unwanted white space}
% \changes{babel~3.5a}{1995/02/15}{Added \cs{@activated} to log active
%    characters}
% \changes{babel~3.5c}{1995/06/19}{Added missing closing brace}
%    \begin{macrocode}
\ifx\PackageError\@undefined
  \def\@nolanerr#1{%
    \errhelp{Your command will be ignored, type <return> to proceed}%
    \errmessage{You haven't defined the language #1\space yet}}
  \def\@nopatterns#1{%
    \message{No hyphenation patterns were loaded for}%
    \message{the language `#1'}%
    \message{I will use the patterns loaded for \bbl@nulllanguage\space
          instead}}
  \def\@noopterr#1{%
    \errmessage{The option #1 was not specified in \string\usepackage}
    \errhelp{You may continue, but expect unexpected results}}
  \def\@activated#1{%
    \wlog{Package babel Info: Making #1 an active character}}
\else
  \newcommand*{\@nolanerr}[1]{%
    \PackageError{babel}%
                 {You haven't defined the language #1\space yet}%
        {Your command will be ignored, type <return> to proceed}}
  \newcommand*{\@nopatterns}[1]{%
    \PackageWarningNoLine{babel}%
        {No hyphenation patterns were loaded for\MessageBreak
          the language `#1'\MessageBreak
          I will use the patterns loaded for \bbl@nulllanguage\space
          instead}}
  \newcommand*{\@noopterr}[1]{%
    \PackageError{babel}%
                 {You haven't loaded the option #1\space yet}%
             {You may proceed, but expect unexpected results}}
  \newcommand*{\@activated}[1]{%
    \PackageInfo{babel}{%
      Making #1 an active character}}
\fi
%    \end{macrocode}
%  \end{macro}
%  \end{macro}
%  \end{macro}
%
%    The following code is meant to be read by ini\TeX\ because it
%    should instruct \TeX\ to read hyphenation patterns. To this end
%    the \texttt{docstrip} option \texttt{patterns} can be used to
%    include this code in the file \file{hyphen.cfg}.
%    \begin{macrocode}
%<*patterns>
%    \end{macrocode}
%
% \changes{babel~3.5g}{1996/07/09}{Removed the use of
%    \cs{patterns@loaded} altogether}
%
%  \begin{macro}{\process@line}
% \changes{babel~3.5b}{1995/04/28}{added macro}
%    Each line in the file \file{language.dat} is processed by
%    |\process@line| after it is read. The first thing this macro does
%    is to check whether the line starts with \texttt{=}.
%    When the first token of a line is an \texttt{=}, the macro
%    |\process@synonym| is called; otherwise the macro
%    |\process@language| will continue.
% \changes{babel~3.5g}{1996/07/09}{Simplified code, removing
%    \cs{bbl@eq@}}
% \changes{babel~3.7c}{1999/04/09}{added an extra argument in order to
%    prevent a trailing space from becoming part of the control
%    sequence when defining a synonym (PR 2851)}
%    \begin{macrocode}
\def\process@line#1#2 #3/{%
  \ifx=#1
    \process@synonym#2 /
  \else
    \process@language#1#2 #3/%
  \fi
  }
%    \end{macrocode}
%  \end{macro}
%
%  \begin{macro}{\process@synonym}
% \changes{babel~3.5b}{1995/04/28}{added macro}
%    This macro takes care of the lines which start with an
%    \texttt{=}. It needs an empty token register to begin with.
%    \begin{macrocode}
\toks@{}
\def\process@synonym#1 /{%
  \ifnum\last@language=\m@ne
%    \end{macrocode}
%    When no languages have been loaded yet, the name following the
%    \texttt{=} will be a synonym for hyphenation register 0.
%    \begin{macrocode}
    \expandafter\chardef\csname l@#1\endcsname0\relax
    \wlog{\string\l@#1=\string\language0}
%    \end{macrocode}
%    As no hyphenation patterns are read in yet, we can not yet set
%    the hyphenmin parameters. Therefor a command to do so is stored
%    in a token register and executed when the first pattern file has
%    been processed.
% \changes{babel~3.7c}{1999/04/27}{Use a token register to temporarily
%    store a command to set hyphenmin parameters for the synonym which
%    is defined \emph{before} the first pattern file is processed}
%    \begin{macrocode}
    \toks@\expandafter{\the\toks@
      \expandafter\let\csname #1hyphenmins\expandafter\endcsname
      \csname\languagename hyphenmins\endcsname}%
  \else
%    \end{macrocode}
%    Otherwise the name will be a synonym for the language loaded last.
%    \begin{macrocode}
    \expandafter\chardef\csname l@#1\endcsname\last@language
    \wlog{\string\l@#1=\string\language\the\last@language}
%    \end{macrocode}
%    We also need to copy the hyphenmin parameters for the synonym.
% \changes{babel~3.7c}{1999/04/22}{Now also store hyphenmin parameters
%    for language synonyms}
% \changes{babel~3.9a}{2012/06/25}{Added \cs{bbl@languages}}
%    \begin{macrocode}
    \expandafter\let\csname #1hyphenmins\expandafter\endcsname
    \csname\languagename hyphenmins\endcsname
  \fi
  \xdef\bbl@languages{%
    \@ifundefined{bbl@languages}\@empty{\bbl@languages,}%
    #1/\the\last@language//}%
  }
%    \end{macrocode}
%  \end{macro}
%
%  \begin{macro}{\process@language}
%    The macro |\process@language| is used to process a non-empty line
%    from the `configuration file'. It has three arguments, each
%    delimited by white space. The third argument is optional,
%    so a |/| character is expected to delimit the last
%    argument.  The first argument is the `name' of a language; the
%    second is the name of the file that contains the patterns. The
%    optional third argument is the name of a file containing
%    hyphenation exceptions.
%
%    The first thing to do is call |\addlanguage| to allocate a
%    pattern register and to make that register `active'.
% \changes{babel~3.0d}{1991/08/08}{Removed superfluous
%    \cs{expandafter}}
% \changes{babel~3.0d}{1991/08/21}{Reinserted \cs{expandafter}}
% \changes{babel~3.0d}{1991/10/27}{Added the collection of pattern
%    names.}
% \changes{babel~3.7c}{1999/04/22}{Also store \cs{languagename} for
%    possible later use in \cs{process@synonym}}
%    \begin{macrocode}
\def\process@language#1 #2 #3/{%
  \expandafter\addlanguage\csname l@#1\endcsname
  \expandafter\language\csname l@#1\endcsname
  \def\languagename{#1}%
%    \end{macrocode}
%    Then the `name' of the language that will be loaded now is
%    added to the token register |\toks8|. and finally
%    the pattern file is read.
%    \begin{macrocode}
  \global\toks8\expandafter{\the\toks8#1, }%
%    \end{macrocode}
% \changes{babel~3.7f}{2000/02/18}{Allow for the encoding to be used
%    as part of the language name} 
%    For some hyphenation patterns it is needed to load them with a
%    specific font encoding selected. This can be specified in the
%    file \file{language.dat} by adding for instance `\texttt{:T1}' to
%    the name of the language. The macro |\bbl@get@enc| extracts the
%    font encoding from the language name and stores it in
%    |\bbl@hyph@enc|.
%    \begin{macrocode}
  \begingroup
    \bbl@get@enc#1:\@@@
    \ifx\bbl@hyph@enc\@empty
    \else
      \fontencoding{\bbl@hyph@enc}\selectfont
    \fi
%    \end{macrocode}
%
% \changes{babel~3.4e}{1994/06/24}{Added code to detect assignments to
%    left- and righthyphenmin in the patternfile.}
%    Some pattern files contain assignments to |\lefthyphenmin| and
%    |\righthyphenmin|. \TeX\ does not keep track of these
%    assignments. Therefor we try to detect such assignments and
%    store them in the |\|\langvar|hyphenmins| macro. When no
%    assignments were made we provide a default setting.
%    \begin{macrocode}
    \lefthyphenmin\m@ne
%    \end{macrocode}
%    Some pattern files contain changes to the |\lccode| en |\uccode|
%    arrays. Such changes should remain local to the language;
%    therefor we process the pattern file in a group; the |\patterns|
%    command acts globally so its effect will be remembered.
% \changes{babel~3.7a}{1998/03/27}{Read pattern files in a group}
% \changes{babel~3.7c}{1999/04/05}{need to set hyphenmin values
%    globally}
% \changes{babel~3.7c}{1999/04/22}{Set \cs{lefthyphenmin} to \cs{m@ne}
%    \emph{inside} the group; explicitly set the hyphenmin parameters
%    for language 0}
%    \begin{macrocode}
    \input #2\relax
%    \end{macrocode}
%    Now we globally store the settings of |\lefthyphenmin| and
%    |\righthyphenmin| and close the group.
% \changes{babel~3.7c}{1999/04/25}{Only set hyphenmin values when the
%    pattern file changed them}
%    \begin{macrocode}
    \ifnum\lefthyphenmin=\m@ne
    \else
      \expandafter\xdef\csname #1hyphenmins\endcsname{%
        \the\lefthyphenmin\the\righthyphenmin}%
    \fi
  \endgroup
%    \end{macrocode}
%    If the counter |\language| is still equal to zero we set the
%    hyphenmin parameters to the values for the language loaded on
%    pattern register 0.
%    \begin{macrocode}
  \ifnum\the\language=\z@
    \expandafter\ifx\csname #1hyphenmins\endcsname\relax
      \set@hyphenmins\tw@\thr@@\relax
    \else
      \expandafter\expandafter\expandafter\set@hyphenmins
        \csname #1hyphenmins\endcsname
    \fi
%    \end{macrocode}
%    Now execute the contents of token register zero as it may
%    contain commands which set the hyphenmin parameters for synonyms
%    that were defined before the first pattern file is read in.
% \changes{babel~3.7c}{1999/04/27}{Added the execution of the contents
%    of \cs{toks@}}
%    \begin{macrocode}
    \the\toks@
  \fi
%    \end{macrocode}
%    Empty the token register after use.
%    \begin{macrocode}
  \toks@{}%
%    \end{macrocode}
%    When the hyphenation patterns have been processed we need to see
%    if a file with hyphenation exceptions needs to be read. This is
%    the case when the third argument is not empty and when it does
%    not contain a space token.
% \changes{babel~3.5b}{1995/04/28}{Added optional reading of file with
%    hyphenation exceptions}
% \changes{babel~3.5f}{1995/07/25}{Use \cs{empty} instead of
%    \cs{@empty} as the latter is unknown in plain}
%    \begin{macrocode}
  \def\bbl@tempa{#3}%
  \let\bbl@tempb\@empty
  \ifx\bbl@tempa\@empty
  \else
    \ifx\bbl@tempa\space
    \else
      \input #3\relax
      \def\bbl@tempb{#3}%
    \fi
  \fi
%    \end{macrocode}
% \changes{babel~3.9a}{2012/06/25}{Added \cs{bbl@languages}}
%    \cs{bbl@languages} saves a snapshot of the loaded languagues in the
%    form  \meta{language}/\meta{number}/\meta{patterns-file}/\meta{exceptions-file}
%    \begin{macrocode}
  \xdef\bbl@languages{%
    \@ifundefined{bbl@languages}\@empty{\bbl@languages,}%
    #1/\the\language/#2/\bbl@tempb}%
  }
%    \end{macrocode}
%
%  \begin{macro}{\bbl@get@enc}
% \changes{babel~3.7f}{2000/02/18}{Added macro}
%  \begin{macro}{\bbl@hyph@enc}
%    The macro |\bbl@get@enc| extracts the font encoding from the
%    language name and stores it in |\bbl@hyph@enc|. It uses delimited
%    arguments to achieve this.
%    \begin{macrocode}
\def\bbl@get@enc#1:#2\@@@{%
%    \end{macrocode}
%    First store both arguments in temporary macros,
%    \begin{macrocode}
  \def\bbl@tempa{#1}%
  \def\bbl@tempb{#2}%
%    \end{macrocode}
%    then, if the second argument was empty, no font encoding was
%    specified and we're done.
% \changes{babel~3.9a}{2012/06/25}{\cs{bbl@hyph@enc} is set globally}
%    \begin{macrocode}
  \ifx\bbl@tempb\@empty
    \global\let\bbl@hyph@enc\@empty
  \else
%    \end{macrocode}
%    But if the second argument was \emph{not} empty it will now have
%    a superfluous colon attached to it which we need to remove. This
%    done by feeding it to |\bbl@get@enc|. The string that we are
%    after will then be in the first argument and be stored in
%    |\bbl@tempa|.
%    \begin{macrocode}
    \bbl@get@enc#2\@@@
    \xdef\bbl@hyph@enc{\bbl@tempa}%
  \fi}
%    \end{macrocode}
%  \end{macro}
%  \end{macro}
%  \end{macro}
%
%  \begin{macro}{\readconfigfile}
%    The configuration file can now be opened for reading.
%    \begin{macrocode}
\openin1 = language.dat
%    \end{macrocode}
%
%    See if the file exists, if not, use the default hyphenation file
%    \file{hyphen.tex}. The user will be informed about this.
%
%    \begin{macrocode}
\ifeof1
  \message{I couldn't find the file language.dat,\space
           I will try the file hyphen.tex}
  \input hyphen.tex\relax
  \def\l@english{0}%
  \def\languagename{english}%
\else
%    \end{macrocode}
%
%    Pattern registers are allocated using count register
%    |\last@language|. Its initial value is~0. The definition of the
%    macro |\newlanguage| is such that it first increments the count
%    register and then defines the language. In order to have the
%    first patterns loaded in pattern register number~0 we initialize
%    |\last@language| with the value~$-1$.
%
% \changes{babel~3.1}{1991/05/21}{Removed use of \cs{toks0}}
%    \begin{macrocode}
  \last@language\m@ne
%    \end{macrocode}
%
%    We now read lines from the file until the end is found
%
%    \begin{macrocode}
  \loop
%    \end{macrocode}
%
%    While reading from the input, it is useful to switch off
%    recognition of the end-of-line character. This saves us stripping
%    off spaces from the contents of the control sequence.
%
%    \begin{macrocode}
    \endlinechar\m@ne
    \read1 to \bbl@line
    \endlinechar`\^^M
%    \end{macrocode}
%
%    Empty lines are skipped.
%    \begin{macrocode}
    \ifx\bbl@line\@empty
    \else
%    \end{macrocode}
%
%    Now we add a space and a |/| character to the end of
%    |\bbl@line|. This is needed to be able to recognize the third,
%    optional, argument of |\process@language| later on.
% \changes{babel~3.5b}{1995/04/28}{Now add a \cs{space} and a /
%    character}
% \changes{babel~3.8m}{2008/07/08}{Store the name of the language
%    loaded in register 0 (PR 4039)} 
%    \begin{macrocode}
      \edef\bbl@line{\bbl@line\space/}%
      \expandafter\process@line\bbl@line
      \ifx\bbl@defaultlanguage\@undefined
        \let\bbl@defaultlanguage\languagename
      \fi
    \fi
%    \end{macrocode}
%
%    Check for the end of the file.  To avoid a new \texttt{if}
%    control sequence we create the necessary |\iftrue| or |\iffalse|
%    with the help of |\csname|.  But there is one complication with
%    this approach: when skipping the \texttt{loop...repeat} \TeX\ has
%    to read |\if|/|\fi| pairs.  So we have to insert a `dummy'
%    |\iftrue|.
% \changes{babel~3.1}{1991/10/31}{Removed the extra \texttt{if}
%    control sequence}
%    \begin{macrocode}
    \iftrue \csname fi\endcsname
    \csname if\ifeof1 false\else true\fi\endcsname
  \repeat
%    \end{macrocode}
%
%    Reactivate the default patterns,
% \changes{babel~3.8m}{2008/07/08}{Also restore the name of the
%    language in \cs{languagename} (PR 4039)} 
%    \begin{macrocode}
  \language=0
  \let\languagename\bbl@defaultlanguage
  \let\bbl@defaultlanguage\@undefined
\fi
%    \end{macrocode}
%    and close the configuration file.
% \changes{babel~3.2a}{1991/11/20}{Free macro space for
%    \cs{process@language}}
%    \begin{macrocode}
\closein1
%    \end{macrocode}
%    Also remove some macros from memory
%    \begin{macrocode}
\let\process@language\@undefined
\let\process@synonym\@undefined
\let\process@line\@undefined
\let\bbl@tempa\@undefined
\let\bbl@tempb\@undefined
\let\bbl@eq@\@undefined
\let\bbl@line\@undefined
\let\bbl@get@enc\@undefined
%    \end{macrocode}
%
% \changes{babel~3.5f}{1995/11/08}{Moved the fiddling with \cs{dump}
%     to \file{bbplain.dtx} as it is no longer needed for \LaTeX}
%    We add a message about the fact that babel is loaded in the
%    format and with which language patterns to the \cs{everyjob}
%    register.
% \changes{babel~3.6h}{1997/01/23}{Added a couple of \cs{expandafter}s
%    to copy the contents of \cs{toks8} into \cs{everyjob} instead of
%    the reference}
%    \begin{macrocode}
\ifx\addto@hook\@undefined
\else
  \expandafter\addto@hook\expandafter\everyjob\expandafter{%
    \expandafter\typeout\expandafter{\the\toks8 loaded.}}
\fi
%    \end{macrocode}
%    Here the code for ini\TeX\ ends.
%    \begin{macrocode}
%</patterns>
%</kernel>
%    \end{macrocode}
%  \end{macro}
%
% \subsection{Support for active characters}
%
%  \begin{macro}{\bbl@add@special}
% \changes{babel~3.2}{1991/11/10}{Added macro}
%    The macro |\bbl@add@special| is used to add a new character (or
%    single character control sequence) to the macro |\dospecials|
%    (and |\@sanitize| if \LaTeX\ is used).
%
%    To keep all changes local, we begin a new group.  Then we
%    redefine the macros |\do| and |\@makeother| to add themselves and
%    the given character without expansion.
%    \begin{macrocode}
%<*core|shorthands>
\def\bbl@add@special#1{\begingroup
    \def\do{\noexpand\do\noexpand}%
    \def\@makeother{\noexpand\@makeother\noexpand}%
%    \end{macrocode}
%    To add the character to the macros, we expand the original macros
%    with the additional character inside the redefinition of the
%    macros.  Because |\@sanitize| can be undefined, we put the
%    definition inside a conditional.
%    \begin{macrocode}
    \edef\x{\endgroup
      \def\noexpand\dospecials{\dospecials\do#1}%
      \expandafter\ifx\csname @sanitize\endcsname\relax \else
        \def\noexpand\@sanitize{\@sanitize\@makeother#1}%
      \fi}%
%    \end{macrocode}
%    The macro |\x| contains at this moment the following:\\
%    |\endgroup\def\dospecials{|\textit{old contents}%
%    |\do|\meta{char}|}|.\\
%    If |\@sanitize| is defined, it contains an additional definition
%    of this macro.  The last thing we have to do, is the expansion of
%    |\x|.  Then |\endgroup| is executed, which restores the old
%    meaning of |\x|, |\do| and |\@makeother|.  After the group is
%    closed, the new definition of |\dospecials| (and |\@sanitize|) is
%    assigned.
%    \begin{macrocode}
  \x}
%    \end{macrocode}
%  \end{macro}
%
%  \begin{macro}{\bbl@remove@special}
% \changes{babel~3.2}{1991/11/10}{Added macro}
%    The companion of the former macro is |\bbl@remove@special|.  It
%    is used to remove a character from the set macros |\dospecials|
%    and |\@sanitize|.
%
%    To keep all changes local, we begin a new group.  Then we define
%    a help macro |\x|, which expands to empty if the characters
%    match, otherwise it expands to its nonexpandable input.  Because
%    \TeX\ inserts a |\relax|, if the corresponding |\else| or |\fi|
%    is scanned before the comparison is evaluated, we provide a `stop
%    sign' which should expand to nothing.
%    \begin{macrocode}
\def\bbl@remove@special#1{\begingroup
    \def\x##1##2{\ifnum`#1=`##2\noexpand\@empty
                 \else\noexpand##1\noexpand##2\fi}%
%    \end{macrocode}
%    With the help of this macro we define |\do| and |\make@other|.
%    \begin{macrocode}
    \def\do{\x\do}%
    \def\@makeother{\x\@makeother}%
%    \end{macrocode}
%    The rest of the work is similar to |\bbl@add@special|.
%    \begin{macrocode}
    \edef\x{\endgroup
      \def\noexpand\dospecials{\dospecials}%
      \expandafter\ifx\csname @sanitize\endcsname\relax \else
        \def\noexpand\@sanitize{\@sanitize}%
      \fi}%
  \x}
%    \end{macrocode}
%  \end{macro}
%
%  \subsection{Shorthands}
%
%  \begin{macro}{\initiate@active@char}
% \changes{babel~3.5a}{1995/02/11}{Added macro}
% \changes{babel~3.5b}{1995/03/03}{Renamed macro}
%    A language definition file can call this macro to make a
%    character active. This macro takes one argument, the character
%    that is to be made active. When the character was already active
%    this macro does nothing. Otherwise, this macro defines the
%    control sequence |\normal@char|\m{char} to expand to the
%    character in its `normal state' and it defines the active
%    character to expand to |\normal@char|\m{char} by default
%    (\m{char} being the character to be made active). Later its
%    definition can be changed to expand to |\active@char|\m{char}
%    by calling |\bbl@activate{|\m{char}|}|.
%
%    For example, to make the double quote character active one could
%    have the following line in a language definition file:
%  \begin{verbatim}
%    \initiate@active@char{"}
%  \end{verbatim}
%
%  \begin{macro}{\bbl@afterelse}
%  \begin{macro}{\bbl@afterfi}
%    Because the code that is used in the handling of active
%    characters may need to look ahead, we take extra care to `throw'
%    it over the |\else| and |\fi| parts of an
%    |\if|-statement\footnote{This code is based on code presented in
%    TUGboat vol. 12, no2, June 1991 in ``An expansion Power Lemma''
%    by Sonja Maus.}. These macros will break if another |\if...\fi|
%    statement appears in one of the arguments.
% \changes{babel~3.6i}{1997/02/20}{Made \cs{bbl@afterelse} and
%    \cs{bbl@afterfi} \cs{long}}
%    \begin{macrocode}
\long\def\bbl@afterelse#1\else#2\fi{\fi#1}
\long\def\bbl@afterfi#1\fi{\fi#1}
%    \end{macrocode}
%  \end{macro}
%  \end{macro}
%
% \changes{babel~3.7a}{1997/02/23}{Commented out \c{peek@token} and
%    \cs{test@token} as shorthands are made expandable again}
%
%  \begin{macro}{\peek@token}
% \changes{babel~3.5f}{1995/12/06}{macro added}
% \changes{babel~3.6i}{1998/03/10}{Renamed \cs{test@token} to
%    \cs{bbl@test@token} to prevent a clash with Arab\TeX}
%    To prevent error messages when a shorthand, which
%    normally takes an argument, sees a |\par|, or |}|, or similar
%    tokens, we need to be able to `peek' at what is coming up next in
%    the input stream. Depending on the category code of the token
%    that is seen, we need to either continue the code for the active
%    character, or insert the non-active version of that character in
%    the output. The macro |\peek@token| therefore takes two
%    arguments, with which it constructs the control sequence to
%    expand next. It |\let|'s |\bbl@nexta| and |\bbl@nextb| to the two
%    possible macros. This is necessary for |\bbl@test@token| to take
%    the right decision.
%    \begin{macrocode}
%\def\peek@token#1#2{%
%  \expandafter\let\expandafter\bbl@nexta\csname #1\string#2\endcsname
%  \expandafter\let\expandafter\bbl@nextb
%    \csname system@active\string#2\endcsname
%  \futurelet\bbl@token\bbl@test@token}
%    \end{macrocode}
%
%  \begin{macro}{\bbl@test@token}
% \changes{babel~3.5f}{1995/12/06}{macro added}
% \changes{babel~3.6i}{1998/03/10}{renamed \cs{bbl@token} to
%    \cs{bbl@test@token} to prevent a clash with Arab\TeX}
%    When the result of peeking at the next token has yielded a token
%    with category `letter', `other' or `active' it is safe to proceed
%    with evaluating the code for the shorthand. When a token is found
%    with any other category code proceeding is unsafe and therefor
%    the original shorthand character is inserted in the output. The
%    macro that calls |\bbl@test@token| needs to setup |\bbl@nexta|
%    and |\bbl@nextb| in order to achieve this.
%    \begin{macrocode}
%\def\bbl@test@token{%
%  \let\bbl@next\bbl@nexta
%  \ifcat\noexpand\bbl@token a%
%  \else
%    \ifcat\noexpand\bbl@token=%
%    \else
%      \ifcat\noexpand\bbl@token\noexpand\bbl@next
%      \else
%        \let\bbl@next\bbl@nextb
%      \fi
%    \fi
%  \fi
%  \bbl@next}
%    \end{macrocode}
%  \end{macro}
%  \end{macro}
%
%^^A
%^^A Tekens met mathcode >"8000 zorgen voor problemen.
%^^A hier kan op getest worden door ze catcode13 te geven 
%^^A en te vragen of er een undefined macro ontstaat:
%^^A \ifx#1\undefined{matchcode<"8000}\else get active definition 
%^^A using \let \fi
%^^A
%    The macro |\initiate@active@char| takes all the necessary actions
%    to make its argument a shorthand character. The real work is
%    performed once for each character.
% \changes{babel~3.7c}{1999/04/30}{Only execute
%    \cs{initiate@active@char} once for each character}
%    \begin{macrocode}
\def\initiate@active@char#1{%
  \expandafter\ifx\csname active@char\string##1\endcsname\relax
    \bbl@afterfi{\@initiate@active@char{#1}}%
  \fi}
%    \end{macrocode}
%    Note that the definition of |\@initiate@active@char| needs an
%    active character, for this the |~| is used. Some of the changes
%    we need, do not have to become available later on, so we do it
%    inside a group.
%    \begin{macrocode}
\begingroup
  \catcode`\~\active
  \def\x{\endgroup
    \def\@initiate@active@char##1{%
%    \end{macrocode}
%    If the character is already active we provide the default
%    expansion under this shorthand mechanism.
% \changes{babel~3.5f}{1996/01/09}{Deal correctly with already active
%    characters, provide top level expansion and define all lower
%    level expansion macros outside of the \cs{else} branch.}
% \changes{babel~3.5g}{1996/08/13}{Top level expansion of
%    \cs{normal@char char} where char is already active, should be the
%    expansion of the active character, not the active character
%    itself as this causes an endless loop}
% \changes{babel~3.7d}{1999/08/19}{Make sure the active character
%    doesn't get expanded more then once by the \cs{edef} by adding
%    \cs{expandafter}\cs{strip@prefix}\cs{meaning}}
% \changes{babel~3.7e}{1999/09/06}{previous change was rubbish; use
%    \cs{let} instead of \cs{edef}}
%    \begin{macrocode}
      \ifcat\noexpand##1\noexpand~\relax
        \@ifundefined{normal@char\string##1}{%
          \expandafter\let\csname normal@char\string##1\endcsname##1%
          \expandafter\gdef
            \expandafter##1%
            \expandafter{%
              \expandafter\active@prefix\expandafter##1%
              \csname normal@char\string##1\endcsname}}{}%
      \else
%    \end{macrocode}
%    Otherwise we write a message in the transcript file,
%    \begin{macrocode}
        \@activated{##1}%
%    \end{macrocode}
%    and define |\normal@char|\m{char} to expand to the character in
%    its default state.
%    \begin{macrocode}
        \@namedef{normal@char\string##1}{##1}%
%    \end{macrocode}
%    If we are making the right quote active we need to change
%    |\pr@m@s| as well.
% \changes{babel~3.5a}{1995/03/10}{Added a check for right quote and
%    adapt \cs{pr@m@s} if necessary}
% \changes{babel~3.7f}{1999/12/18}{The redefinition needs to take
%    place one level higher, \cs{prim@s} needs to be redefined.}
%    \begin{macrocode}
        \ifx##1'%
          \let\prim@s\bbl@prim@s
%    \end{macrocode}
%    Also, make sure that a single |'| in math mode `does the right
%    thing'.
% \changes{babel~3.7f}{1999/12/18}{Insert a check for math mode in the
%    definition of \cs{normal@char'}}
% \changes{babel~3.7g}{2000/10/02}{use \cs{textormath} to get rid of
%    the \cs{fi} (PR 3266)}
%    \begin{macrocode}
          \@namedef{normal@char\string##1}{%
            \textormath{##1}{^\bgroup\prim@s}}%
        \fi
%    \end{macrocode}
%    If we are using the caret as a shorthand character special care
%    should be taken to make sure math still works. Therefor an extra
%    level of expansion is introduced with a check for math mode on
%    the upper level.
% \changes{babel~3.7f}{1999/12/18}{Introduced an extra level of
%    expansion in the definition of an active caret}
% \changes{babel~3.9a}{2012/06/20}{Added a couple of missing comment
%    characters (PR 4146)}
%    \begin{macrocode}
        \ifx##1^%
          \gdef\bbl@act@caret{%
            \ifmmode
              \csname normal@char\string^\endcsname
            \else
              \bbl@afterfi
              {\if@safe@actives
                \bbl@afterelse\csname normal@char\string##1\endcsname
               \else
                \bbl@afterfi\csname user@active\string##1\endcsname
               \fi}%
            \fi}%
        \fi
%    \end{macrocode}
%    To prevent problems with the loading of other packages after
%    \babel\ we reset the catcode of the character at the end of the
%    package.
% \changes{babel~3.5f}{1995/12/01}{Restore the category code of a
%    shorthand char at end of package}
% \changes{babel~3.6f}{1997/01/14}{Made restoring of the category code
%    of shorthand characters optional}
% \changes{babel~3.7a}{1997/03/21}{Use \cs{@ifpackagewith} to
%    determine whether shorthand characters need to remain active}
% \changes{babel~3.9a}{2012/07/04}{Catcodes are also restored after
%    each language, to prevent incompatibilities. Use \cs{string} instead
%    of \cs{noexpand} and add \cs{relax}}
%    \begin{macrocode}
        \@ifpackagewith{babel}{KeepShorthandsActive}{}{%
          \edef\bbl@tempa{\catcode`\string##1\the\catcode`##1\relax}%
          \expandafter\AtEndOfLanguage\expandafter\CurrentOption
            \expandafter{\bbl@tempa}%
          \expandafter\AtEndOfPackage\expandafter{\bbl@tempa}}%
%    \end{macrocode}
%    Now we set the lowercase code of the |~| equal to that of the
%    character to be made active and execute the rest of the code
%    inside a |\lowercase| `environment'.
% \changes{babel~3.5f}{1996/01/25}{store the \cs{lccode} of the tie
%    before changing it}
%    \begin{macrocode}
        \@tempcnta=\lccode`\~
        \lccode`~=`##1%
        \lowercase{%
%    \end{macrocode}
%    Make the character active and add it to |\dospecials| and
%    |\@sanitize|.
%    \begin{macrocode}
          \catcode`~\active
          \expandafter\bbl@add@special
            \csname \string##1\endcsname
%    \end{macrocode}
%    Also re-activate it again at |\begin{document}|.
%    \begin{macrocode}
            \AtBeginDocument{%
              \catcode`##1\active
%    \end{macrocode}
%    We also need to make sure that the shorthands are active during
%    the processing of the \file{.aux} file. Otherwise some citations
%    may give unexpected results in the printout when a shorthand was
%    used in the optional argument of |\bibitem| for example.
% \changes{babel~3.6i}{1997/03/01}{Make shorthands active during
%    \file{.aux} file processing}
%    \begin{macrocode}
              \if@filesw
                \immediate\write\@mainaux{%
                  \string\catcode`##1\string\active}%
              \fi}%
%    \end{macrocode}
%    Define the character to expand to
%    \begin{center}
%    |\active@prefix| \m{char} |\normal@char|\m{char}
%    \end{center}
%    (where |\active@char|\m{char} is \emph{one} control sequence!).
% \changes{babel~3.5f}{1996/01/25}{restore the \cs{lccode} of the tie}
%    \begin{macrocode}
          \expandafter\gdef
            \expandafter~%
            \expandafter{%
            \expandafter\active@prefix\expandafter##1%
            \csname normal@char\string##1\endcsname}}%
        \lccode`\~\@tempcnta
      \fi
%    \end{macrocode}
%    For the active caret we first expand to |\bbl@act@caret| in order
%    to be able to handle math mode correctly.
% \changes{babel~3.7f}{2000/09/25}{Make an exception for the active
%    caret which needs an extra level of expansion}
%    \begin{macrocode}
      \ifx##1^%
        \@namedef{active@char\string##1}{\bbl@act@caret}%
      \else
%    \end{macrocode}
%    We define the first level expansion of |\active@char|\m{char} to
%    check the status of the |@safe@actives| flag. If it is set to
%    true we expand to the `normal' version of this character,
%    otherwise we call |\@active@char|\m{char}.
%    \begin{macrocode}
        \@namedef{active@char\string##1}{%
          \if@safe@actives
            \bbl@afterelse\csname normal@char\string##1\endcsname
          \else
            \bbl@afterfi\csname user@active\string##1\endcsname
          \fi}%
      \fi
%    \end{macrocode}
%    The next level of the code checks whether a user has defined a
%    shorthand for himself with this character. First we check for a
%    single character shorthand. If that doesn't exist we check for a
%    shorthand with an argument.
% \changes{babel~3.5d}{1995/07/02}{Skip the user-level active char
%    with argument if no shorthands with arguments were defined}
% \changes{babel~3.8b}{2004/04/19}{Now use \cs{bbl@sh@select}}
%    \begin{macrocode}
      \@namedef{user@active\string##1}{%
        \expandafter\ifx
        \csname \user@group @sh@\string##1@\endcsname
        \relax
          \bbl@afterelse\bbl@sh@select\user@group##1%
        {user@active@arg\string##1}{language@active\string##1}%
        \else
          \bbl@afterfi\csname \user@group @sh@\string##1@\endcsname
        \fi}%
%    \end{macrocode}
%    When there is also no user-level shorthand with an argument we
%    will check whether there is a language defined shorthand for
%    this active character. Before the next token is absorbed as
%    argument we need to make sure that this is safe. Therefor
%    |\peek@token| is called to decide that.
% \changes{babel~3.5f}{1995/12/07}{use \cs{peek@token} to check whether
%    it is safe to proceed}
% \changes{babel~3.6i}{1997/02/20}{Remove the use of \cs{peek@token}
%    again and make the \cs{...active@arg...} commands \cs{long}}
% \changes{babel~3.7e}{1999/09/24}{pass the argument on with braces in
%    order to prevent it from breaking up}
% \changes{babel~3.7f}{2000/02/18}{remove the braces again}
%    \begin{macrocode}
      \long\@namedef{user@active@arg\string##1}####1{%
        \expandafter\ifx
        \csname \user@group @sh@\string##1@\string####1@\endcsname
        \relax
          \bbl@afterelse
          \csname language@active\string##1\endcsname####1%
        \else
          \bbl@afterfi
          \csname \user@group @sh@\string##1@\string####1@%
          \endcsname
        \fi}%
%    \end{macrocode}
%    In order to do the right thing when a shorthand with an argument
%    is used by itself at the end of the line we provide a definition
%    for the case of an empty argument. For that case we let the
%    shorthand character expand to its non-active self.
%    \begin{macrocode}
      \@namedef{\user@group @sh@\string##1@@}{%
        \csname normal@char\string##1\endcsname}%
%    \end{macrocode}
%
%    Like the shorthands that can be defined by the user, a language
%    definition file can also define shorthands with and without an
%    argument, so we need two more macros to check if they exist.
% \changes{babel~3.5d}{1995/07/02}{Skip the language-level active char
%    with argument if no shorthands with arguments were defined}
% \changes{babel~3.8b}{2004/04/19}{Now use \cs{bbl@sh@select}}
%    \begin{macrocode}
      \@namedef{language@active\string##1}{%
        \expandafter\ifx
        \csname \language@group @sh@\string##1@\endcsname
        \relax
          \bbl@afterelse\bbl@sh@select\language@group##1%
          {language@active@arg\string##1}{system@active\string##1}%
        \else
          \bbl@afterfi
          \csname \language@group @sh@\string##1@\endcsname
        \fi}%
%    \end{macrocode}
% \changes{babel~3.5f}{1995/12/07}{use \cs{peek@token} to check whether
%    it is safe to proceed}
% \changes{babel~3.6i}{1997/02/20}{Remove the use of \cs{peek@token}
%    again}
% \changes{babel~3.7e}{1999/09/24}{pass the argument on with braces in
%    order to prevent it from breaking up}
% \changes{babel~3.7f}{2000/02/18}{remove the braces again}
%    \begin{macrocode}
      \long\@namedef{language@active@arg\string##1}####1{%
        \expandafter\ifx
        \csname \language@group @sh@\string##1@\string####1@\endcsname
        \relax
          \bbl@afterelse
          \csname system@active\string##1\endcsname####1%
        \else
          \bbl@afterfi
          \csname \language@group @sh@\string##1@\string####1@%
          \endcsname
        \fi}%
%    \end{macrocode}
%    And the same goes for the system level.
% \changes{babel~3.8b}{2004/04/19}{Now use \cs{bbl@sh@select}}
%    \begin{macrocode}
      \@namedef{system@active\string##1}{%
        \expandafter\ifx
        \csname \system@group @sh@\string##1@\endcsname
        \relax
          \bbl@afterelse\bbl@sh@select\system@group##1%
          {system@active@arg\string##1}{normal@char\string##1}%
        \else
          \bbl@afterfi\csname \system@group @sh@\string##1@\endcsname
        \fi}%
%    \end{macrocode}
%    When no shorthands were found the `normal' version of the active
%    character is inserted.
% \changes{babel~3.5f}{1995/12/07}{use \cs{peek@token} to check whether
%    it is safe to proceed}
% \changes{babel~3.6i}{1997/02/20}{Remove the use of \cs{peek@token}
%    again}
%    \begin{macrocode}
      \long\@namedef{system@active@arg\string##1}####1{%
        \expandafter\ifx
        \csname \system@group @sh@\string##1@\string####1@\endcsname
        \relax
          \bbl@afterelse\csname normal@char\string##1\endcsname####1%
        \else
          \bbl@afterfi
          \csname \system@group @sh@\string##1@\string####1@\endcsname
        \fi}%
%    \end{macrocode}
%    When a shorthand combination such as |''| ends up in a heading
%    \TeX\ would see |\protect'\protect'|. To prevent this from
%    happening a shorthand needs to be defined at user level.
% \changes{babel~3.7f}{1999/12/09}{Added an extra shorthand
%    combination on user level to catch an interfering \cs{protect}}
%    \begin{macrocode}
      \@namedef{user@sh@\string##1@\string\protect@}{%
        \csname user@active\string##1\endcsname}%
      }%
    }\x
%    \end{macrocode}
%  \end{macro}
%
%  \begin{macro}{\bbl@sh@select}
%    This command helps the shorthand supporting macros to select how
%    to proceed. Note that this macro needs to be expandable as do all
%    the shorthand macros in order for them to work in expansion-only
%    environments such as the argument of |\hyphenation|.
%
%    This macro expects the name of a group of shorthands in its first
%    argument and a shorthand character in its second argument. It
%    will expand to either |\bbl@firstcs| or |\bbl@scndcs|. Hence two
%    more arguments need to follow it.
% \changes{babel~3.8b}{2004/04/19}{Added command}
%    \begin{macrocode}
\def\bbl@sh@select#1#2{%
  \expandafter\ifx\csname#1@sh@\string#2@sel\endcsname\relax
    \bbl@afterelse\bbl@scndcs
  \else
    \bbl@afterfi\csname#1@sh@\string#2@sel\endcsname
  \fi
}
%    \end{macrocode}
%  \end{macro}
%
%  \begin{macro}{\active@prefix}
%    The command |\active@prefix| which is used in the expansion of
%    active characters has a function similar to |\OT1-cmd| in that it
%    |\protect|s the active character whenever |\protect| is
%    \emph{not} |\@typeset@protect|.
% \changes{babel~3.5d}{1995/07/02}{\cs{@protected@cmd} has vanished
%    from \file{ltoutenc.dtx}}
% \changes{babel~3.7o}{2003/11/17}{Added handling of the situation
%    where \cs{protect} is set to \cs{@unexpandable@protect}}
%    \begin{macrocode}
\def\active@prefix#1{%
  \ifx\protect\@typeset@protect
  \else
%    \end{macrocode}
%    When |\protect| is set to |\@unexpandable@protect| we make sure
%    that the active character is als \emph{not} expanded by inserting
%    |\noexpand| in front of it. The |\@gobble| is needed to remove
%    a token such as |\activechar:| (when the double colon was the
%    active character to be dealt with).
%    \begin{macrocode}
    \ifx\protect\@unexpandable@protect
      \bbl@afterelse\bbl@afterfi\noexpand#1\@gobble
    \else
      \bbl@afterfi\bbl@afterfi\protect#1\@gobble
    \fi
  \fi}
%    \end{macrocode}
%  \end{macro}
%
%  \begin{macro}{\if@safe@actives}
%    In some circumstances it is necessary to be able to change the
%    expansion of an active character on the fly. For this purpose the
%    switch |@safe@actives| is available. The setting of this switch
%    should be checked in the first level expansion of
%    |\active@char|\m{char}.
%    \begin{macrocode}
\newif\if@safe@actives
\@safe@activesfalse
%    \end{macrocode}
%  \end{macro}
%
%  \begin{macro}{\bbl@restore@actives}
% \changes{babel~3.7m}{2003/11/15}{New macro added}
%    When the output routine kicks in while the
%    active characters were made ``safe'' this must be undone in
%    the headers to prevent unexpected typeset results. For this
%    situation we define a command to make them ``unsafe'' again.
%    \begin{macrocode}
\def\bbl@restore@actives{\if@safe@actives\@safe@activesfalse\fi}
%    \end{macrocode}
%  \end{macro}
%
%  \begin{macro}{\bbl@activate}
% \changes{babel~3.5a}{1995/02/11}{Added macro}
%
%    This macro takes one argument, like |\initiate@active@char|. The
%    macro is used to change the definition of an active character to
%    expand to |\active@char|\m{char} instead of
%    |\normal@char|\m{char}.
%    \begin{macrocode}
\def\bbl@activate#1{%
  \expandafter\def
  \expandafter#1\expandafter{%
    \expandafter\active@prefix
    \expandafter#1\csname active@char\string#1\endcsname}%
}
%    \end{macrocode}
%  \end{macro}
%
%  \begin{macro}{\bbl@deactivate}
% \changes{babel~3.5a}{1995/02/11}{Added macro}
%    This macro takes one argument, like |\bbl@activate|. The macro
%    doesn't really make a character non-active; it changes its
%    definition to expand to |\normal@char|\m{char}.
%    \begin{macrocode}
\def\bbl@deactivate#1{%
  \expandafter\def
  \expandafter#1\expandafter{%
    \expandafter\active@prefix
    \expandafter#1\csname normal@char\string#1\endcsname}%
}
%    \end{macrocode}
%  \end{macro}
%
%  \begin{macro}{\bbl@firstcs}
%  \begin{macro}{\bbl@scndcs}
%    These macros have two arguments. They use one of their arguments
%    to build a control sequence from.
%    \begin{macrocode}
\def\bbl@firstcs#1#2{\csname#1\endcsname}
\def\bbl@scndcs#1#2{\csname#2\endcsname}
%    \end{macrocode}
%  \end{macro}
%  \end{macro}
%
%  \begin{macro}{\declare@shorthand}
%    The command |\declare@shorthand| is used to declare a shorthand
%    on a certain level. It takes three arguments:
%    \begin{enumerate}
%    \item a name for the collection of shorthands, i.e. `system', or
%      `dutch';
%    \item the character (sequence) that makes up the shorthand,
%      i.e. |~| or |"a|;
%    \item the code to be executed when the shorthand is encountered.
%    \end{enumerate}
% \changes{babel~3.5d}{1995/07/02}{Make a `note' when a shorthand with
%    an argument is defined.}
% \changes{babel~3.6i}{1997/02/23}{Make it possible to distinguish the
%    constructed control sequences for the case with argument}
% \changes{babel~3.8b}{2004/04/19}{We need to support shorthands with
%    and without argument in different groups; added the name of the
%    group to the storage macro}
% \changes{babel~3.9a}{2012/07/03}{Check if shorthands are redefined}
%    \begin{macrocode}
\def\declare@shorthand#1#2{\@decl@short{#1}#2\@nil}
\def\@decl@short#1#2#3\@nil#4{%
  \def\bbl@tempa{#3}%
  \ifx\bbl@tempa\@empty
    \expandafter\let\csname #1@sh@\string#2@sel\endcsname\bbl@scndcs
    \@ifundefined{#1@sh@\string#2@}{}%
      {\def\bbl@tempa{#4}%
       \expandafter\ifx\csname#1@sh@\string#2@\endcsname\bbl@tempa
       \else
         \PackageWarning{Babel}%
           {Redefining #1 shorthand \string#2\MessageBreak
            in language \CurrentOption}%
       \fi}%
    \@namedef{#1@sh@\string#2@}{#4}%
  \else
    \expandafter\let\csname #1@sh@\string#2@sel\endcsname\bbl@firstcs
    \@ifundefined{#1@sh@\string#2@\string#3@}{}%
      {\def\bbl@tempa{#4}%
       \expandafter\ifx\csname#1@sh@\string#2@\string#3@\endcsname\bbl@tempa
       \else
         \PackageWarning{Babel}%
           {Redefining #1 shorthand \string#2\string#3\MessageBreak
            in language \CurrentOption}%
       \fi}%
    \@namedef{#1@sh@\string#2@\string#3@}{#4}%
  \fi}
%    \end{macrocode}
%  \end{macro}
%
%  \begin{macro}{\textormath}
%    Some of the shorthands that will be declared by the language
%    definition files have to be usable in both text and mathmode. To
%    achieve this the helper macro |\textormath| is provided.
%    \begin{macrocode}
\def\textormath#1#2{%
  \ifmmode
    \bbl@afterelse#2%
  \else
    \bbl@afterfi#1%
  \fi}
%    \end{macrocode}
%  \end{macro}
%
%  \begin{macro}{\user@group}
%  \begin{macro}{\language@group}
%  \begin{macro}{\system@group}
%    The current concept of `shorthands' supports three levels or
%    groups of shorthands. For each level the name of the level or
%    group is stored in a macro. The default is to have a user group;
%    use language group `english' and have a system group called
%    `system'.
% \changes{babel~3.6i}{1997/02/24}{Have a user group called `user' by
%    default}
%    \begin{macrocode}
\def\user@group{user}
\def\language@group{english}
\def\system@group{system}
%    \end{macrocode}
%  \end{macro}
%  \end{macro}
%  \end{macro}
%
%  \begin{macro}{\useshorthands}
%    This is the user level command to tell \LaTeX\ that user level
%    shorthands will be used in the document. It takes one argument,
%    the character that starts a shorthand.
% \changes{babel~3.7j}{2001/11/11}{When \TeX\ has seen a character
%    its category code is fixed; need to use a `stand-in' for the
%    call of \cs{bbl@activate}} 
%    \begin{macrocode}
\def\useshorthands#1{%
%    \end{macrocode}
%    First note that this is user level.
%    \begin{macrocode}
  \def\user@group{user}%
%    \end{macrocode}
%    Then initialize the character for use as a shorthand character.
%    \begin{macrocode}
  \initiate@active@char{#1}%
%    \end{macrocode}
%    Now that \TeX\ has seen the character its category code is
%    fixed, but for the actions of |\bbl@activate| to succeed we need
%    it to be active. Hence the trick with the |\lccode| to circumvent
%    this.
% \changes{babel~3.7j}{2003/09/11}{The change from 11/112001 was
%    incomplete} 
%    \begin{macrocode}
  \@tempcnta\lccode`\~
  \lccode`~=`#1%
  \lowercase{\catcode`~\active\bbl@activate{~}}%
  \lccode`\~\@tempcnta}
%    \end{macrocode}
%  \end{macro}
%
%  \begin{macro}{\defineshorthand}
%    Currently we only support one group of user level shorthands,
%    called `user'.
%    \begin{macrocode}
\def\defineshorthand{\declare@shorthand{user}}
%    \end{macrocode}
%  \end{macro}
%
%  \begin{macro}{\languageshorthands}
%    A user level command to change the language from which shorthands
%    are used.
%    \begin{macrocode}
\def\languageshorthands#1{\def\language@group{#1}}
%    \end{macrocode}
%  \end{macro}
%
%  \begin{macro}{\aliasshorthand}
% \changes{babel~3.5f}{1996/01/25}{New command}
%    \begin{macrocode}
\def\aliasshorthand#1#2{%
%    \end{macrocode}
%    First the new shorthand needs to be initialized,
%    \begin{macrocode}
  \expandafter\ifx\csname active@char\string#2\endcsname\relax
     \ifx\document\@notprerr
       \@notshorthand{#2}
     \else
       \initiate@active@char{#2}%
%    \end{macrocode}
%    Then we need to use the |\lccode| trick to make the new shorthand
%    behave like the old one. Therefore we save the current |\lccode|
%    of the |~|-character and restore it later. Then we |\let| the new
%    shorthand character be equal to the original. 
%    \begin{macrocode}
       \@tempcnta\lccode`\~
       \lccode`~=`#2%
       \lowercase{\let~#1}%
       \lccode`\~\@tempcnta
     \fi
   \fi
}
%    \end{macrocode}
%  \end{macro}
%
%  \begin{macro}{\@notshorthand}
% \changes{v3.8d}{2004/11/20}{Error message added}
%    
%    \begin{macrocode}
\def\@notshorthand#1{%
       \PackageError{babel}{%
         The character `\string #1' should be made
         a shorthand character;\MessageBreak
         add the command \string\useshorthands\string{#1\string} to
         the preamble.\MessageBreak
         I will ignore your instruction}{}%
   }
%    \end{macrocode}
%  \end{macro}
%
%  \begin{macro}{\shorthandon}
% \changes{babel~3.7a}{1998/06/07}{Added command}
%  \begin{macro}{\shorthandoff}
% \changes{babel~3.7a}{1998/06/07}{Added command}
%    The first level definition of these macros just passes the
%    argument on to |\bbl@switch@sh|, adding |\@nil| at the end to
%    denote the end of the list of characters.
%    \begin{macrocode}
\newcommand*\shorthandon[1]{\bbl@switch@sh{on}#1\@nil}
\newcommand*\shorthandoff[1]{\bbl@switch@sh{off}#1\@nil}
%    \end{macrocode}
%
%  \begin{macro}{\bbl@switch@sh}
% \changes{babel~3.7a}{1998/06/07}{Added command}
%    The macro |\bbl@switch@sh| takes the list of characters apart one
%    by  one and subsequently switches the category code of the
%    shorthand character according to the first argument of
%    |\bbl@switch@sh|.
%    \begin{macrocode}
\def\bbl@switch@sh#1#2#3\@nil{%
%    \end{macrocode}
%    But before any of this switching takes place we make sure that
%    the character we are dealing with is known as a shorthand
%    character. If it is, a macro such as |\active@char"| should
%    exist.
%    \begin{macrocode}
  \@ifundefined{active@char\string#2}{%
    \PackageError{babel}{%
      The character '\string #2' is not a shorthand character
      in \languagename}{%
      Maybe you made a typing mistake?\MessageBreak
      I will ignore your instruction}}{%
    \csname bbl@switch@sh@#1\endcsname#2}%
%    \end{macrocode}
%    Now that, as the first character in the list has been taken care
%    of, we pass the rest of the list back to |\bbl@switch@sh|.
%    \begin{macrocode}
  \ifx#3\@empty\else
    \bbl@afterfi\bbl@switch@sh{#1}#3\@nil
  \fi}
%    \end{macrocode}
%  \end{macro}
%
%  \begin{macro}{\bbl@switch@sh@off}
%    All that is left to do is define the actual switching
%    macros. Switching off is easy, we just set the category code to
%    `other' (12).
%    \begin{macrocode}
\def\bbl@switch@sh@off#1{\catcode`#112\relax}
%    \end{macrocode}
%  \end{macro}
%
%  \begin{macro}{\bbl@switch@sh@on}
%    But switching the shorthand character back on is a bit more
%    tricky. It involves making sure that we have an active character
%    to begin with when the macro is being defined. It also needs the
%    use of |\lowercase| and |\lccode| trickery to get everything to
%    work out as expected. And to keep things local that need to
%    remain local a group is opened, which is closed as soon as |\x|
%    gets executed.
% \changes{babel~3.8j}{2008/03/21}{Added a group in order to protect
%    the current lowercase code of the tilde (PR 3851)} 
%    \begin{macrocode}
\begingroup
  \catcode`\~\active
  \def\x{\endgroup
    \def\bbl@switch@sh@on##1{%
      \begingroup
      \lccode`~=`##1%
      \lowercase{\endgroup
        \catcode`~\active
        }%
      }%
    }
%    \end{macrocode}
%  \end{macro}
%    The next operation makes the above definition effective.
%    \begin{macrocode}
\x
%
%    \end{macrocode}
%  \end{macro}
%  \end{macro}
%
%    To prevent problems with constructs such as |\char"01A| when the
%    double quote is made active, we define a shorthand on
%    system level.
% \changes{babel~3.5a}{1995/03/10}{Replaced 16 system shorthands to
%    deal with hex numbers by one}
%    \begin{macrocode}
\declare@shorthand{system}{"}{\csname normal@char\string"\endcsname}
%    \end{macrocode}
%
%    When the right quote is made active we need to take care of
%    handling it correctly in mathmode. Therefore we define a
%    shorthand at system level to make it expand to a non-active right
%    quote in textmode, but expand to its original definition in
%    mathmode. (Note that the right quote is `active' in mathmode
%    because of its mathcode.)
% \changes{babel~3.5a}{1995/03/10}{Added a system shorthand for the
%    right quote}
%    \begin{macrocode}
\declare@shorthand{system}{'}{%
  \textormath{\csname normal@char\string'\endcsname}%
             {\sp\bgroup\prim@s}}
%    \end{macrocode}
%
%    When the left quote is  made active we need to take care of
%    handling it correctly when it is followed by for instance an open
%    brace token. Therefore we define a shorthand at system level to
%    make it expand to a non-active left quote.
% \changes{babel~3.5f}{1996/03/06}{Added a system shorthand for the
%    left quote}
%    \begin{macrocode}
\declare@shorthand{system}{`}{\csname normal@char\string`\endcsname}
%    \end{macrocode}
%
%  \begin{macro}{\bbl@prim@s}
% \changes{babel~3.7f}{1999/12/01}{Need to redefine \cs{prim@s} as
%    well as plain \TeX's definition uses \cs{next}}
%  \begin{macro}{\bbl@pr@m@s}
% \changes{babel~3.5a}{1995/03/10}{Added macro}
%    One of the internal macros that are involved in substituting
%    |\prime| for each right quote in mathmode is |\prim@s|. This
%    checks if the next character is a right quote. When the right
%    quote is active, the definition of this macro needs to be adapted
%    to look for an active right quote.
%    \begin{macrocode}
\def\bbl@prim@s{%
  \prime\futurelet\@let@token\bbl@pr@m@s}
\begingroup
  \catcode`\'\active\let'\relax
  \def\x{\endgroup
    \def\bbl@pr@m@s{%
      \ifx'\@let@token
        \expandafter\pr@@@s
      \else
        \ifx^\@let@token
          \expandafter\expandafter\expandafter\pr@@@t
        \else
          \egroup
        \fi
      \fi}%
    }
\x
%    \end{macrocode}
%  \end{macro}
%  \end{macro}
%
%    \begin{macrocode}
%</core|shorthands>
%    \end{macrocode}
%
%    Normally the |~| is active and expands to \verb*=\penalty\@M\ =.
%    When it is written to the \file{.aux} file it is written
%    expanded. To prevent that and to be able to use the character |~|
%    as a start character for a shorthand, it is redefined here as a
%    one character shorthand on system level.
% \changes{babel~3.5f}{1996/04/02}{No need to reset the category code
%    of the tilde as \cs{initiate@active@char} now correctly deals
%    with active characters}
%    \begin{macrocode}
%<*core>
\initiate@active@char{~}
\declare@shorthand{system}{~}{\leavevmode\nobreak\ }
\bbl@activate{~}
%    \end{macrocode}
%
%  \begin{macro}{\OT1dqpos}
%  \begin{macro}{\T1dqpos}
%    The position of the double quote character is different for the
%    OT1 and T1 encodings. It will later be selected using the
%    |\f@encoding| macro. Therefor we define two macros here to store
%    the position of the character in these encodings.
%    \begin{macrocode}
\expandafter\def\csname OT1dqpos\endcsname{127}
\expandafter\def\csname T1dqpos\endcsname{4}
%    \end{macrocode}
%    When the macro |\f@encoding| is undefined (as it is in plain
%    \TeX) we define it here to expand to \texttt{OT1}
%    \begin{macrocode}
\ifx\f@encoding\@undefined
  \def\f@encoding{OT1}
\fi
%    \end{macrocode}
%  \end{macro}
%  \end{macro}
%
% \subsection{Conditional loading of shorthands}
%
% !!! To be documented
% \changes{babel~3.9a}{2012/06/16}{Added code}
%    \begin{macrocode}
\ifx\bbl@opt@shorthands\@nnil\else
%    \begin{macrocode}
% TO DO: package options are expanded by LaTeX, and ~ raises
% an error, but not \string~. Is there a way to fix it?

% Note the value is that at the expansion time, eg, in the preample
% shorhands are usually deactivated
%    \begin{macrocode}
\def\babelshorthand#1{%
  \@ifundefined{bbl@@\languagename @@\bbl@sh@string#1\@empty}%
    {#1}%
    {\@nameuse{bbl@@\languagename @@\bbl@sh@string#1\@empty}}}

 \let\bbl@s@initiate@active@char\initiate@active@char
 \def\initiate@active@char#1{%
     \bbl@ifshorthand{#1}%
       {\bbl@s@initiate@active@char{#1}}%
       {\@namedef{active@char\string#1}{}}}%
 \let\bbl@s@declare@shorthand\declare@shorthand
 \def\declare@shorthand#1#2{%
     \expandafter\bbl@ifshorthand\expandafter{\@car#2\@nil}%
       {\bbl@s@declare@shorthand{#1}{#2}}%
       {\def\bbl@tempa{#2}%
        \@namedef{bbl@@#1@@\bbl@sh@string#2\@empty}}}%
 \let\bbl@s@switch@sh@on\bbl@switch@sh@on
 \def\bbl@switch@sh@on#1{%
     \bbl@ifshorthand{#1}%
       {\bbl@s@switch@sh@on{#1}}%
       {}}%
 \let\bbl@s@switch@sh@off\bbl@switch@sh@off
 \def\bbl@switch@sh@off#1{%
     \bbl@ifshorthand{#1}%
       {\bbl@s@switch@sh@off{#1}}%
       {}}%
 \let\bbl@s@activate\bbl@activate
 \def\bbl@activate#1{%
     \bbl@ifshorthand{#1}%
       {\bbl@s@activate{#1}}%
       {}}%
 \let\bbl@s@deactivate\bbl@deactivate
 \def\bbl@deactivate#1{%
     \bbl@ifshorthand{#1}%
       {\bbl@s@deactivate{#1}}%
       {}}

\fi
%    \end{macrocode}
%  \subsection{Language attributes}
%
%    Language attributes provide a means to give the user control over
%    which features of the language definition files he wants to
%    enable.
% \changes{babel~3.7c}{1998/07/02}{Added support for language
%    attributes}
%  \begin{macro}{\languageattribute}
%    The macro |\languageattribute| checks whether its arguments are
%    valid and then activates the selected language attribute.
%    \begin{macrocode}
\newcommand\languageattribute[2]{%
%    \end{macrocode}
%    First check whether the language is known.
%    \begin{macrocode}
  \expandafter\ifx\csname l@#1\endcsname\relax
    \@nolanerr{#1}%
  \else
%    \end{macrocode}
%    Than process each attribute in the list.
%    \begin{macrocode}
    \@for\bbl@attr:=#2\do{%
%    \end{macrocode}
%    We want to make sure that each attribute is selected only once;
%    therefor we store the already selected attributes in
%    |\bbl@known@attribs|. When that control sequence is not yet
%    defined this attribute is certainly not selected before.
%    \begin{macrocode}
      \ifx\bbl@known@attribs\@undefined
        \in@false
      \else
%    \end{macrocode}
%    Now we need to see if the attribute occurs in the list of
%    already selected attributes.
%    \begin{macrocode}
        \edef\bbl@tempa{\noexpand\in@{,#1-\bbl@attr,}%
          {,\bbl@known@attribs,}}%
        \bbl@tempa
      \fi
%    \end{macrocode}
%    When the attribute was in the list we issue a warning; this might
%    not be the users intention.
%    \begin{macrocode}
      \ifin@
        \PackageWarning{Babel}{%
          You have more than once selected the attribute
          '\bbl@attr'\MessageBreak for language #1}%
      \else
%    \end{macrocode}
%    When we end up here the attribute is not selected before. So, we
%    add it to the list of selected attributes and execute the
%    associated \TeX-code.
%    \begin{macrocode}
        \edef\bbl@tempa{%
          \noexpand\bbl@add@list\noexpand\bbl@known@attribs{#1-\bbl@attr}}%
        \bbl@tempa
        \edef\bbl@tempa{#1-\bbl@attr}%
        \expandafter\bbl@ifknown@ttrib\expandafter{\bbl@tempa}\bbl@attributes%
        {\csname#1@attr@\bbl@attr\endcsname}%
        {\@attrerr{#1}{\bbl@attr}}%
     \fi
      }
  \fi}
%    \end{macrocode}
%    This command should only be used in the preamble of a document.
%    \begin{macrocode}
\@onlypreamble\languageattribute
%    \end{macrocode}
%    The error text to be issued when an unknown attribute is
%    selected.
%    \begin{macrocode}
  \newcommand*{\@attrerr}[2]{%
    \PackageError{babel}%
                 {The attribute #2 is unknown for language #1.}%
        {Your command will be ignored, type <return> to proceed}}
%    \end{macrocode}
%  \end{macro}
%
%  \begin{macro}{\bbl@declare@ttribute}
%    This command adds the new language/attribute combination to the
%    list of known attributes.
%    \begin{macrocode}
\def\bbl@declare@ttribute#1#2#3{%
  \bbl@add@list\bbl@attributes{#1-#2}%
%    \end{macrocode}
%    Then it defines a control sequence to be executed when the
%    attribute is used in a document. The result of this should be
%    that the macro |\extras...| for the current language is extended,
%    otherwise the attribute will not work as its code is removed from
%    memory at |\begin{document}|.
%    \begin{macrocode}
  \expandafter\def\csname#1@attr@#2\endcsname{#3}%
  }
%    \end{macrocode}
%  \end{macro}
%
%  \begin{macro}{\bbl@ifattributeset}
% \changes{babel~3.7f}{2000/02/12}{macro added}
%    This internal macro has 4 arguments. It can be used to interpret
%    \TeX\ code based on whether a certain attribute was set. This
%    command should appear inside the argument to |\AtBeginDocument|
%    because the attributes are set in the document preamble,
%    \emph{after} \babel\ is loaded.
%
%    The first argument is the language, the second argument the
%    attribute being checked, and the third and fourth arguments are
%    the true and false clauses.
%    \begin{macrocode}
\def\bbl@ifattributeset#1#2#3#4{%
%    \end{macrocode}
%    First we need to find out if any attributes were set; if not
%    we're done.
%    \begin{macrocode}
  \ifx\bbl@known@attribs\@undefined
    \in@false
  \else
%    \end{macrocode}
%    The we need to check the list of known attributes.
%    \begin{macrocode}
    \edef\bbl@tempa{\noexpand\in@{,#1-#2,}%
      {,\bbl@known@attribs,}}%
    \bbl@tempa
  \fi
%    \end{macrocode}
%    When we're this far |\ifin@| has a value indicating if the
%    attribute in question was set or not. Just to be safe the code to
%    be executed is `thrown over the |\fi|'.
%    \begin{macrocode}
  \ifin@
    \bbl@afterelse#3%
  \else
    \bbl@afterfi#4%
  \fi
  }
%    \end{macrocode}
%  \end{macro}
%
%  \begin{macro}{\bbl@add@list}
%    This internal macro adds its second argument to a comma
%    separated list in its first argument. When the list is not
%    defined yet (or empty), it will be initiated
%    \begin{macrocode}
\def\bbl@add@list#1#2{%
  \ifx#1\@undefined
    \def#1{#2}%
  \else
    \ifx#1\@empty
      \def#1{#2}%
    \else
      \edef#1{#1,#2}%
    \fi
  \fi
  }
%    \end{macrocode}
%  \end{macro}
%
%  \begin{macro}{\bbl@ifknown@ttrib}
%    An internal macro to check whether a given language/attribute is
%    known. The macro takes 4 arguments, the language/attribute, the
%    attribute list, the \TeX-code to be executed when the attribute
%    is known and the \TeX-code to be executed otherwise.
%    \begin{macrocode}
\def\bbl@ifknown@ttrib#1#2{%
%    \end{macrocode}
%    We first assume the attribute is unknown.
%    \begin{macrocode}
  \let\bbl@tempa\@secondoftwo
%    \end{macrocode}
%    Then we loop over the list of known attributes, trying to find a
%    match.
%    \begin{macrocode}
  \@for\bbl@tempb:=#2\do{%
    \expandafter\in@\expandafter{\expandafter,\bbl@tempb,}{,#1,}%
    \ifin@
%    \end{macrocode}
%    When a match is found the definition of |\bbl@tempa| is changed.
%    \begin{macrocode}
      \let\bbl@tempa\@firstoftwo
    \else
    \fi}%
%    \end{macrocode}
%    Finally we execute |\bbl@tempa|.
%    \begin{macrocode}
  \bbl@tempa
}
%    \end{macrocode}
%  \end{macro}
%
%  \begin{macro}{\bbl@clear@ttribs}
%    This macro removes all the attribute code from \LaTeX's memory at
%    |\begin{document}| time (if any is present).
% \changes{babel~3.7e}{1999/09/24}{When \cs{bbl@attributes} is
%    undefined this should be a no-op} 
%    \begin{macrocode}
\def\bbl@clear@ttribs{%
  \ifx\bbl@attributes\@undefined\else
    \@for\bbl@tempa:=\bbl@attributes\do{%
      \expandafter\bbl@clear@ttrib\bbl@tempa.
      }%
    \let\bbl@attributes\@undefined
  \fi
  }
\def\bbl@clear@ttrib#1-#2.{%
  \expandafter\let\csname#1@attr@#2\endcsname\@undefined}
\AtBeginDocument{\bbl@clear@ttribs}
%    \end{macrocode}
%  \end{macro}
%
%  \subsection{Support for saving macro definitions}
%
%    To save the meaning of control sequences using |\babel@save|, we
%    use temporary control sequences.  To save hash table entries for
%    these control sequences, we don't use the name of the control
%    sequence to be saved to construct the temporary name.  Instead we
%    simply use the value of a counter, which is reset to zero each
%    time we begin to save new values.  This works well because we
%    release the saved meanings before we begin to save a new set of
%    control sequence meanings (see |\selectlanguage| and
%    |\originalTeX|).
%
%  \begin{macro}{\babel@savecnt}
% \changes{babel~3.2}{1991/11/10}{Added macro}
%  \begin{macro}{\babel@beginsave}
% \changes{babel~3.2}{1991/11/10}{Added macro}
%    The initialization of a new save cycle: reset the counter to
%    zero.
%    \begin{macrocode}
\def\babel@beginsave{\babel@savecnt\z@}
%    \end{macrocode}
%    Before it's forgotten, allocate the counter and initialize all.
%    \begin{macrocode}
\newcount\babel@savecnt
\babel@beginsave
%    \end{macrocode}
%  \end{macro}
%  \end{macro}
%
%  \begin{macro}{\babel@save}
% \changes{babel~3.2}{1991/11/10}{Added macro}
%    The macro |\babel@save|\meta{csname} saves the current meaning of
%    the control sequence \meta{csname} to
%    |\originalTeX|\footnote{\cs{originalTeX} has to be
%    expandable, i.\,e.\ you shouldn't let it to \cs{relax}.}.
%    To do this, we let the current meaning to a temporary control
%    sequence, the restore commands are appended to |\originalTeX| and
%    the counter is incremented.
% \changes{babel~3.2c}{1992/03/17}{missing backslash led to errors
%    when executing \cs{originalTeX}}
% \changes{babel~3.2d}{1992/07/02}{saving in \cs{babel@i} and
%    restoring from \cs{@babel@i} doesn't work very well...}
%    \begin{macrocode}
\def\babel@save#1{%
  \expandafter\let\csname babel@\number\babel@savecnt\endcsname #1\relax
  \begingroup
    \toks@\expandafter{\originalTeX \let#1=}%
    \edef\x{\endgroup
      \def\noexpand\originalTeX{\the\toks@ \expandafter\noexpand
         \csname babel@\number\babel@savecnt\endcsname\relax}}%
  \x
  \advance\babel@savecnt\@ne}
%    \end{macrocode}
%  \end{macro}
%
%  \begin{macro}{\babel@savevariable}
% \changes{babel~3.2}{1991/11/10}{Added macro}
%    The macro |\babel@savevariable|\meta{variable} saves the value of
%    the variable.  \meta{variable} can be anything allowed after the
%    |\the| primitive.
%    \begin{macrocode}
\def\babel@savevariable#1{\begingroup
    \toks@\expandafter{\originalTeX #1=}%
    \edef\x{\endgroup
      \def\noexpand\originalTeX{\the\toks@ \the#1\relax}}%
  \x}
%    \end{macrocode}
%  \end{macro}
%
%  \begin{macro}{\bbl@frenchspacing}
%  \begin{macro}{\bbl@nonfrenchspacing}
%    Some languages need to have |\frenchspacing| in effect. Others
%    don't want that. The command |\bbl@frenchspacing| switches it on
%    when it isn't already in effect and |\bbl@nonfrenchspacing|
%    switches it off if necessary.
%    \begin{macrocode}
\def\bbl@frenchspacing{%
  \ifnum\the\sfcode`\.=\@m
    \let\bbl@nonfrenchspacing\relax
  \else
    \frenchspacing
    \let\bbl@nonfrenchspacing\nonfrenchspacing
  \fi}
\let\bbl@nonfrenchspacing\nonfrenchspacing
%    \end{macrocode}
%  \end{macro}
%  \end{macro}
%
% \subsection{Support for extending macros}
%
%  \begin{macro}{\addto}
%    For each language four control sequences have to be defined that
%    control the language-specific definitions. To be able to add
%    something to these macro once they have been defined the macro
%    |\addto| is introduced. It takes two arguments, a \meta{control
%    sequence} and \TeX-code to be added to the \meta{control
%    sequence}.
%
%    If the \meta{control sequence} has not been defined before it is
%    defined now.
% \changes{babel~3.1}{1991/11/05}{Added macro}
% \changes{babel~3.4}{1994/02/04}{Changed to use toks register}
% \changes{babel~3.6b}{1996/12/30}{Also check if control sequence
%    expands to \cs{relax}}
%    \begin{macrocode}
\def\addto#1#2{%
  \ifx#1\@undefined
    \def#1{#2}%
  \else
%    \end{macrocode}
%    The control sequence could also expand to |\relax|, in which case
%    a circular definition results. The net result is a stack overflow.
%    \begin{macrocode}
    \ifx#1\relax
      \def#1{#2}%
    \else
%    \end{macrocode}
%    Otherwise the replacement text for the \meta{control sequence} is
%    expanded and stored in a token register, together with the
%    \TeX-code to be added.  Finally the \meta{control sequence} is
%    \emph{re}defined, using the contents of the token register.
%    \begin{macrocode}
      {\toks@\expandafter{#1#2}%
        \xdef#1{\the\toks@}}%
    \fi
  \fi
}
%    \end{macrocode}
%  \end{macro}
%
% \subsection{Macros common to a number of languages}
%
%  \begin{macro}{\allowhyphens}
% \changes{babel~3.2b}{1992/02/16}{Moved macro from language
%    definition files}
% \changes{babel~3.7a}{1998/03/12}{Make \cs{allowhyphens} a no-op for
%    T1 fontencoding}
%    This macro makes hyphenation possible. Basically its definition
%    is nothing more than |\nobreak| |\hskip| \texttt{0pt plus
%    0pt}\footnote{\TeX\ begins and ends a word for hyphenation at a
%    glue node. The penalty prevents a linebreak at this glue node.}.
%    \begin{macrocode}
\def\bbl@t@one{T1}
\def\allowhyphens{%
  \ifx\cf@encoding\bbl@t@one\else\bbl@allowhyphens\fi}
\def\bbl@allowhyphens{\nobreak\hskip\z@skip}
%    \end{macrocode}
%  \end{macro}
%
%  \begin{macro}{\set@low@box}
% \changes{babel~3.2b}{1992/02/16}{Moved macro from language
%    definition files}
%    The following macro is used to lower quotes to the same level as
%    the comma.  It prepares its argument in box register~0.
%    \begin{macrocode}
\def\set@low@box#1{\setbox\tw@\hbox{,}\setbox\z@\hbox{#1}%
    \dimen\z@\ht\z@ \advance\dimen\z@ -\ht\tw@%
    \setbox\z@\hbox{\lower\dimen\z@ \box\z@}\ht\z@\ht\tw@ \dp\z@\dp\tw@}
%    \end{macrocode}
%  \end{macro}
%
%  \begin{macro}{\save@sf@q}
% \changes{babel~3.2b}{1992/02/16}{Moved macro from language
%    definition files}
%    The macro |\save@sf@q| is used to save and reset the current
%    space factor.
% \changes{babel~3.7f}{2000/09/19}{PR3119, don't start a paragraph in
%    a local group}
%    \begin{macrocode}
\def\save@sf@q #1{\leavevmode
 \begingroup 
  \edef\@SF{\spacefactor \the\spacefactor}#1\@SF
 \endgroup
}
%    \end{macrocode}
%  \end{macro}
%
%  \begin{macro}{\bbl@disc}
% \changes{babel~3.5f}{1996/01/24}{Macro moved from language
%    definition files}
%    For some languages the macro |\bbl@disc| is used to ease the
%    insertion of discretionaries for letters that behave `abnormally'
%    at a breakpoint.
%    \begin{macrocode}
\def\bbl@disc#1#2{%
  \nobreak\discretionary{#2-}{}{#1}\allowhyphens}
%    \end{macrocode}
%  \end{macro}
%
% \changes{babel~3.5c}{1995/06/14}{Repaired a typo (itlaic, PR1652)}
%
%  \subsection{Making glyphs available}
%
%    The file \file{\filename}\footnote{The file described in this
%    section has version number \fileversion, and was last revised on
%    \filedate.} makes a number of glyphs available that either do not
%    exist in the \texttt{OT1} encoding and have to be `faked', or
%    that are not accessible through \file{T1enc.def}.
%
%  \subsection{Quotation marks}
%
%  \begin{macro}{\quotedblbase}
%    In the \texttt{T1} encoding the opening double quote at the
%    baseline is available as a separate character, accessible via
%    |\quotedblbase|. In the \texttt{OT1} encoding it is not
%    available, therefor we make it available by lowering the normal
%    open quote character to the baseline.
%    \begin{macrocode}
\ProvideTextCommand{\quotedblbase}{OT1}{%
  \save@sf@q{\set@low@box{\textquotedblright\/}%
    \box\z@\kern-.04em\allowhyphens}}
%    \end{macrocode}
%    Make sure that when an encoding other than \texttt{OT1} or
%    \texttt{T1} is used this glyph can still be typeset.
%    \begin{macrocode}
\ProvideTextCommandDefault{\quotedblbase}{%
  \UseTextSymbol{OT1}{\quotedblbase}}
%    \end{macrocode}
%  \end{macro}
%
%  \begin{macro}{\quotesinglbase}
%    We also need the single quote character at the baseline.
%    \begin{macrocode}
\ProvideTextCommand{\quotesinglbase}{OT1}{%
  \save@sf@q{\set@low@box{\textquoteright\/}%
    \box\z@\kern-.04em\allowhyphens}}
%    \end{macrocode}
%    Make sure that when an encoding other than \texttt{OT1} or
%    \texttt{T1} is used this glyph can still be typeset.
%    \begin{macrocode}
\ProvideTextCommandDefault{\quotesinglbase}{%
  \UseTextSymbol{OT1}{\quotesinglbase}}
%    \end{macrocode}
%  \end{macro}
%
%  \begin{macro}{\guillemotleft}
%  \begin{macro}{\guillemotright}
%    The guillemet characters are not available in \texttt{OT1}
%    encoding. They are faked.
%    \begin{macrocode}
\ProvideTextCommand{\guillemotleft}{OT1}{%
  \ifmmode
    \ll
  \else
    \save@sf@q{\nobreak
      \raise.2ex\hbox{$\scriptscriptstyle\ll$}\allowhyphens}%
  \fi}
\ProvideTextCommand{\guillemotright}{OT1}{%
  \ifmmode
    \gg
  \else
    \save@sf@q{\nobreak
      \raise.2ex\hbox{$\scriptscriptstyle\gg$}\allowhyphens}%
  \fi}
%    \end{macrocode}
%    Make sure that when an encoding other than \texttt{OT1} or
%    \texttt{T1} is used these glyphs can still be typeset.
%    \begin{macrocode}
\ProvideTextCommandDefault{\guillemotleft}{%
  \UseTextSymbol{OT1}{\guillemotleft}}
\ProvideTextCommandDefault{\guillemotright}{%
  \UseTextSymbol{OT1}{\guillemotright}}
%    \end{macrocode}
%  \end{macro}
%  \end{macro}
%
%  \begin{macro}{\guilsinglleft}
%  \begin{macro}{\guilsinglright}
%    The single guillemets are not available in \texttt{OT1}
%    encoding. They are faked.
%    \begin{macrocode}
\ProvideTextCommand{\guilsinglleft}{OT1}{%
  \ifmmode
    <%
  \else
    \save@sf@q{\nobreak
      \raise.2ex\hbox{$\scriptscriptstyle<$}\allowhyphens}%
  \fi}
\ProvideTextCommand{\guilsinglright}{OT1}{%
  \ifmmode
    >%
  \else
    \save@sf@q{\nobreak
      \raise.2ex\hbox{$\scriptscriptstyle>$}\allowhyphens}%
  \fi}
%    \end{macrocode}
%    Make sure that when an encoding other than \texttt{OT1} or
%    \texttt{T1} is used these glyphs can still be typeset.
%    \begin{macrocode}
\ProvideTextCommandDefault{\guilsinglleft}{%
  \UseTextSymbol{OT1}{\guilsinglleft}}
\ProvideTextCommandDefault{\guilsinglright}{%
  \UseTextSymbol{OT1}{\guilsinglright}}
%    \end{macrocode}
%  \end{macro}
%  \end{macro}
%
%
%  \subsection{Letters}
%
%  \begin{macro}{\ij}
%  \begin{macro}{\IJ}
%    The dutch language uses the letter `ij'. It is available in
%    \texttt{T1} encoded fonts, but not in the \texttt{OT1} encoded
%    fonts. Therefor we fake it for the \texttt{OT1} encoding.
% \changes{dutch-3.7a}{1995/02/04}{Changed the kerning in the faked ij
%    to match the dc-version of it}
%    \begin{macrocode}
\DeclareTextCommand{\ij}{OT1}{%
  \allowhyphens i\kern-0.02em j\allowhyphens}
\DeclareTextCommand{\IJ}{OT1}{%
  \allowhyphens I\kern-0.02em J\allowhyphens}
\DeclareTextCommand{\ij}{T1}{\char188}
\DeclareTextCommand{\IJ}{T1}{\char156}
%    \end{macrocode}
%    Make sure that when an encoding other than \texttt{OT1} or
%    \texttt{T1} is used these glyphs can still be typeset.
%    \begin{macrocode}
\ProvideTextCommandDefault{\ij}{%
  \UseTextSymbol{OT1}{\ij}}
\ProvideTextCommandDefault{\IJ}{%
  \UseTextSymbol{OT1}{\IJ}}
%    \end{macrocode}
%  \end{macro}
%  \end{macro}
%
%  \begin{macro}{\dj}
%  \begin{macro}{\DJ}
%    The croatian language needs the letters |\dj| and |\DJ|; they are
%    available in the \texttt{T1} encoding, but not in the
%    \texttt{OT1} encoding by default.
%
%    Some code to construct these glyphs for the \texttt{OT1} encoding
%    was made available to me by Stipcevic Mario,
%    (\texttt{stipcevic@olimp.irb.hr}).
% \changes{babel~3.5f}{1996/03/28}{New definition of \cs{dj}, see PR
%    2058}
%    \begin{macrocode}
\def\crrtic@{\hrule height0.1ex width0.3em}
\def\crttic@{\hrule height0.1ex width0.33em}
%
\def\ddj@{%
  \setbox0\hbox{d}\dimen@=\ht0
  \advance\dimen@1ex
  \dimen@.45\dimen@
  \dimen@ii\expandafter\rem@pt\the\fontdimen\@ne\font\dimen@
  \advance\dimen@ii.5ex
  \leavevmode\rlap{\raise\dimen@\hbox{\kern\dimen@ii\vbox{\crrtic@}}}}
\def\DDJ@{%
  \setbox0\hbox{D}\dimen@=.55\ht0
  \dimen@ii\expandafter\rem@pt\the\fontdimen\@ne\font\dimen@
  \advance\dimen@ii.15ex %            correction for the dash position
  \advance\dimen@ii-.15\fontdimen7\font %     correction for cmtt font
  \dimen\thr@@\expandafter\rem@pt\the\fontdimen7\font\dimen@
  \leavevmode\rlap{\raise\dimen@\hbox{\kern\dimen@ii\vbox{\crttic@}}}}
%
\DeclareTextCommand{\dj}{OT1}{\ddj@ d}
\DeclareTextCommand{\DJ}{OT1}{\DDJ@ D}
%    \end{macrocode}
%    Make sure that when an encoding other than \texttt{OT1} or
%    \texttt{T1} is used these glyphs can still be typeset.
%    \begin{macrocode}
\ProvideTextCommandDefault{\dj}{%
  \UseTextSymbol{OT1}{\dj}}
\ProvideTextCommandDefault{\DJ}{%
  \UseTextSymbol{OT1}{\DJ}}
%    \end{macrocode}
%  \end{macro}
%  \end{macro}
%
%  \begin{macro}{\SS}
%    For the \texttt{T1} encoding |\SS| is defined and selects a
%    specific glyph from the font, but for other encodings it is not
%    available. Therefor we make it available here.
%    \begin{macrocode}
\DeclareTextCommand{\SS}{OT1}{SS}
\ProvideTextCommandDefault{\SS}{\UseTextSymbol{OT1}{\SS}}
%    \end{macrocode}
%  \end{macro}
%
% \subsection{Shorthands for quotation marks}
%
%    Shorthands are provided for a number of different quotation
%    marks, which make them usable both outside and inside mathmode.
%
%  \begin{macro}{\glq}
%  \begin{macro}{\grq}
% \changes{babel~3.7a}{1997/04/25}{Make the definition of \cs{grq}
%    dependent on the font encoding}
% \changes{babel~3.8b}{2004/05/02}{Made \cs{glq} fontencoding
%    dependent as well} 
%    The `german' single quotes.
%    \begin{macrocode}
\ProvideTextCommand{\glq}{OT1}{%
  \textormath{\quotesinglbase}{\mbox{\quotesinglbase}}}
\ProvideTextCommand{\glq}{T1}{%
  \textormath{\quotesinglbase}{\mbox{\quotesinglbase}}}
\ProvideTextCommandDefault{\glq}{\UseTextSymbol{OT1}\glq}
%    \end{macrocode}
%    The definition of |\grq| depends on the fontencoding. With
%    \texttt{T1} encoding no extra kerning is needed.
%    \begin{macrocode}
\ProvideTextCommand{\grq}{T1}{%
  \textormath{\textquoteleft}{\mbox{\textquoteleft}}}
\ProvideTextCommand{\grq}{OT1}{%
  \save@sf@q{\kern-.0125em%
  \textormath{\textquoteleft}{\mbox{\textquoteleft}}%
  \kern.07em\relax}}
\ProvideTextCommandDefault{\grq}{\UseTextSymbol{OT1}\grq}
%    \end{macrocode}
%  \end{macro}
%  \end{macro}
%
%  \begin{macro}{\glqq}
%  \begin{macro}{\grqq}
% \changes{babel~3.7a}{1997/04/25}{Make the definition of \cs{grqq}
%    dependent on the font encoding}
% \changes{babel~3.8b}{2004/05/02}{Made \cs{grqq} fontencoding
%    dependent as well} 
%    The `german' double quotes.
%    \begin{macrocode}
\ProvideTextCommand{\glqq}{OT1}{%
  \textormath{\quotedblbase}{\mbox{\quotedblbase}}}
\ProvideTextCommand{\glqq}{T1}{%
  \textormath{\quotedblbase}{\mbox{\quotedblbase}}}
\ProvideTextCommandDefault{\glqq}{\UseTextSymbol{OT1}\glqq}
%    \end{macrocode}
%    The definition of |\grqq| depends on the fontencoding. With
%    \texttt{T1} encoding no extra kerning is needed.
%    \begin{macrocode}
\ProvideTextCommand{\grqq}{T1}{%
  \textormath{\textquotedblleft}{\mbox{\textquotedblleft}}}
\ProvideTextCommand{\grqq}{OT1}{%
  \save@sf@q{\kern-.07em%
  \textormath{\textquotedblleft}{\mbox{\textquotedblleft}}%
  \kern.07em\relax}}
\ProvideTextCommandDefault{\grqq}{\UseTextSymbol{OT1}\grqq}
%    \end{macrocode}
%  \end{macro}
%  \end{macro}
%
%  \begin{macro}{\flq}
%  \begin{macro}{\frq}
% \changes{babel~3.5f}{1995/08/07}{corrected spelling of
%    \cs{quilsingl...}}
% \changes{babel~3.5f}{1995/09/05}{now use \cs{textormath} in these
%    definitions}
% \changes{babel~3.8b}{2004/05/02}{Made \cs{flq} and \cs{frq}
%    fontencoding dependent} 
%    The `french' single guillemets.
%    \begin{macrocode}
\ProvideTextCommand{\flq}{OT1}{%
  \textormath{\guilsinglleft}{\mbox{\guilsinglleft}}}
\ProvideTextCommand{\flq}{T1}{%
  \textormath{\guilsinglleft}{\mbox{\guilsinglleft}}}
\ProvideTextCommandDefault{\flq}{\UseTextSymbol{OT1}\flq}
%    \end{macrocode}
%    
%    \begin{macrocode}
\ProvideTextCommand{\frq}{OT1}{%
  \textormath{\guilsinglright}{\mbox{\guilsinglright}}}
\ProvideTextCommand{\frq}{T1}{%
  \textormath{\guilsinglright}{\mbox{\guilsinglright}}}
\ProvideTextCommandDefault{\frq}{\UseTextSymbol{OT1}\frq}
%    \end{macrocode}
%  \end{macro}
%  \end{macro}
%
%  \begin{macro}{\flqq}
%  \begin{macro}{\frqq}
% \changes{babel~3.5f}{1995/08/07}{corrected spelling of
%    \cs{quillemot...}}
% \changes{babel~3.5f}{1995/09/05}{now use \cs{textormath} in these
%    definitions}
% \changes{babel~3.8b}{2004/05/02}{Made \cs{flqq} and \cs{frqq}
%    fontencoding dependent} 
%    The `french' double guillemets.
%    \begin{macrocode}
\ProvideTextCommand{\flqq}{OT1}{%
  \textormath{\guillemotleft}{\mbox{\guillemotleft}}}
\ProvideTextCommand{\flqq}{T1}{%
  \textormath{\guillemotleft}{\mbox{\guillemotleft}}}
\ProvideTextCommandDefault{\flqq}{\UseTextSymbol{OT1}\flqq}
%    \end{macrocode}
%    
%    \begin{macrocode}
\ProvideTextCommand{\frqq}{OT1}{%
  \textormath{\guillemotright}{\mbox{\guillemotright}}}
\ProvideTextCommand{\frqq}{T1}{%
  \textormath{\guillemotright}{\mbox{\guillemotright}}}
\ProvideTextCommandDefault{\frqq}{\UseTextSymbol{OT1}\frqq}
%    \end{macrocode}
%  \end{macro}
%  \end{macro}
%
%  \subsection{Umlauts and trema's}
%
%    The command |\"| needs to have a different effect for different
%    languages. For German for instance, the `umlaut' should be
%    positioned lower than the default position for placing it over
%    the letters a, o, u, A, O and U. When placed over an e, i, E or I
%    it can retain its normal position. For Dutch the same glyph is
%    always placed in the lower position.
%
%  \begin{macro}{\umlauthigh}
% \changes{v3.8a}{2004/02/19}{Use \cs{leavevmode}\cs{bgroup} to
%    prevent problems when this command occurs in vertical mode.}
%  \begin{macro}{\umlautlow}
%    To be able to provide both positions of |\"| we provide two
%    commands to switch the positioning, the default will be
%    |\umlauthigh| (the normal positioning).
%    \begin{macrocode}
\def\umlauthigh{%
  \def\bbl@umlauta##1{\leavevmode\bgroup%
      \expandafter\accent\csname\f@encoding dqpos\endcsname
      ##1\allowhyphens\egroup}%
  \let\bbl@umlaute\bbl@umlauta}
\def\umlautlow{%
  \def\bbl@umlauta{\protect\lower@umlaut}}
\def\umlautelow{%
  \def\bbl@umlaute{\protect\lower@umlaut}}
\umlauthigh
%    \end{macrocode}
%  \end{macro}
%  \end{macro}
%
%  \begin{macro}{\lower@umlaut}
%    The command |\lower@umlaut| is used to position the |\"| closer
%    the the letter.
%
%    We want the umlaut character lowered, nearer to the letter. To do
%    this we need an extra \meta{dimen} register.
%    \begin{macrocode}
\expandafter\ifx\csname U@D\endcsname\relax
  \csname newdimen\endcsname\U@D
\fi
%    \end{macrocode}
%    The following code fools \TeX's \texttt{make\_accent} procedure
%    about the current x-height of the font to force another placement
%    of the umlaut character.
%    \begin{macrocode}
\def\lower@umlaut#1{%
%    \end{macrocode}
%    First we have to save the current x-height of the font, because
%    we'll change this font dimension and this is always done
%    globally.
% \changes{v3.8a}{2004/02/19}{Use \cs{leavevmode}\cs{bgroup} to
%    prevent problems when this command occurs in vertical mode.}
%    \begin{macrocode}
  \leavevmode\bgroup
    \U@D 1ex%
%    \end{macrocode}
%    Then we compute the new x-height in such a way that the umlaut
%    character is lowered to the base character.  The value of
%    \texttt{.45ex} depends on the \MF\ parameters with which the
%    fonts were built.  (Just try out, which value will look best.)
%    \begin{macrocode}
    {\setbox\z@\hbox{%
      \expandafter\char\csname\f@encoding dqpos\endcsname}%
      \dimen@ -.45ex\advance\dimen@\ht\z@
%    \end{macrocode}
%    If the new x-height is too low, it is not changed.
%    \begin{macrocode}
      \ifdim 1ex<\dimen@ \fontdimen5\font\dimen@ \fi}%
%    \end{macrocode}
%    Finally we call the |\accent| primitive, reset the old x-height
%    and insert the base character in the argument.
% \changes{babel~3.5f}{1996/04/02}{Added a \cs{allowhyphens}}
% \changes{babel~3.5f}{1996/06/25}{removed \cs{allowhyphens}}
%    \begin{macrocode}
    \expandafter\accent\csname\f@encoding dqpos\endcsname
    \fontdimen5\font\U@D #1%
  \egroup}
%    \end{macrocode}
%  \end{macro}
%
%    For all vowels we declare |\"| to be a composite command which
%    uses |\bbl@umlauta| or |\bbl@umlaute| to position the umlaut
%    character. We need to be sure that these definitions override the
%    ones that are provided when the package \pkg{fontenc} with
%    option \Lopt{OT1} is used. Therefor these declarations are
%    postponed until the beginning of the document.
%    \begin{macrocode}
\AtBeginDocument{%
  \DeclareTextCompositeCommand{\"}{OT1}{a}{\bbl@umlauta{a}}%
  \DeclareTextCompositeCommand{\"}{OT1}{e}{\bbl@umlaute{e}}%
  \DeclareTextCompositeCommand{\"}{OT1}{i}{\bbl@umlaute{\i}}%
  \DeclareTextCompositeCommand{\"}{OT1}{\i}{\bbl@umlaute{\i}}%
  \DeclareTextCompositeCommand{\"}{OT1}{o}{\bbl@umlauta{o}}%
  \DeclareTextCompositeCommand{\"}{OT1}{u}{\bbl@umlauta{u}}%
  \DeclareTextCompositeCommand{\"}{OT1}{A}{\bbl@umlauta{A}}%
  \DeclareTextCompositeCommand{\"}{OT1}{E}{\bbl@umlaute{E}}%
  \DeclareTextCompositeCommand{\"}{OT1}{I}{\bbl@umlaute{I}}%
  \DeclareTextCompositeCommand{\"}{OT1}{O}{\bbl@umlauta{O}}%
  \DeclareTextCompositeCommand{\"}{OT1}{U}{\bbl@umlauta{U}}%
}
%    \end{macrocode}
%
% \subsection{The redefinition of the style commands}
%
%    The rest of the code in this file can only be processed by
%    \LaTeX, so we check the current format. If it is plain \TeX,
%    processing should stop here. But, because of the need to limit
%    the scope of the definition of |\format|, a macro that is used
%    locally in the following |\if|~statement, this comparison is done
%    inside a group. To prevent \TeX\ from complaining about an
%    unclosed group, the processing of the command |\endinput| is
%    deferred until after the group is closed. This is accomplished by
%    the command |\aftergroup|.
%    \begin{macrocode}
{\def\format{lplain}
\ifx\fmtname\format
\else
  \def\format{LaTeX2e}
  \ifx\fmtname\format
  \else
    \aftergroup\endinput
  \fi
\fi}
%    \end{macrocode}
%
%    Now that we're sure that the code is seen by \LaTeX\ only, we
%    have to find out what the main (primary) document style is
%    because we want to redefine some macros.  This is only necessary
%    for releases of \LaTeX\ dated before December~1991. Therefor
%    this part of the code can optionally be included in
%    \file{babel.def} by specifying the \texttt{docstrip} option
%    \texttt{names}.
%    \begin{macrocode}
%<*names>
%    \end{macrocode}
%
%    The standard styles can be distinguished by checking whether some
%    macros are defined. In table~\ref{styles} an overview is given of
%    the macros that can be used for this purpose.
%  \begin{table}[htb]
%  \begin{center}
% \DeleteShortVerb{\|}
%  \begin{tabular}{|lcp{8cm}|}
%   \hline
%   article         & : & both the \verb+\chapter+ and \verb+\opening+
%                         macros are undefined\\
%   report and book & : & the \verb+\chapter+ macro is defined and
%                         the \verb+\opening+ is undefined\\
%   letter          & : & the \verb+\chapter+ macro is undefined and
%                         the \verb+\opening+ is defined\\
%   \hline
%  \end{tabular}
% \caption{How to determine the main document style}\label{styles}
% \MakeShortVerb{\|}
%  \end{center}
%  \end{table}
%
%    \noindent The macros that have to be redefined for the
%    \texttt{report} and \texttt{book} document styles happen to be
%    the same, so there is no need to distinguish between those two
%    styles.
%
%  \begin{macro}{\doc@style}
%    First a parameter |\doc@style| is defined to identify the current
%    document style. This parameter might have been defined by a
%    document style that already uses macros instead of hard-wired
%    texts, such as \file{artikel1.sty}~\cite{BEP}, so the existence of
%    |\doc@style| is checked. If this macro is undefined, i.\,e., if
%    the document style is unknown and could therefore contain
%    hard-wired texts, |\doc@style| is defined to the default
%    value~`0'.
% \changes{babel~3.0d}{1991/10/29}{Removed use of \cs{@ifundefined}}
%    \begin{macrocode}
\ifx\@undefined\doc@style
  \def\doc@style{0}%
%    \end{macrocode}
%    This parameter is defined in the following \texttt{if}
%    construction (see table~\ref{styles}):
%
%    \begin{macrocode}
  \ifx\@undefined\opening
    \ifx\@undefined\chapter
      \def\doc@style{1}%
    \else
      \def\doc@style{2}%
    \fi
  \else
    \def\doc@style{3}%
  \fi%
\fi%
%    \end{macrocode}
%  \end{macro}
%
% \changes{babel~3.1}{1991/11/05}{Removed definition of
%    \cs{if@restonecol}}
%
%    \subsubsection{Redefinition of macros}
%
%    Now here comes the real work: we start to redefine things and
%    replace hard-wired texts by macros. These redefinitions should be
%    carried out conditionally, in case it has already been done.
%
%    For the \texttt{figure} and \texttt{table} environments we have
%    in all styles:
%    \begin{macrocode}
\@ifundefined{figurename}{\def\fnum@figure{\figurename{} \thefigure}}{}
\@ifundefined{tablename}{\def\fnum@table{\tablename{} \thetable}}{}
%    \end{macrocode}
%
%    The rest of the macros have to be treated differently for each
%    style.  When |\doc@style| still has its default value nothing
%    needs to be done.
%    \begin{macrocode}
\ifcase \doc@style\relax
\or
%    \end{macrocode}
%
%    This means that \file{babel.def} is read after the
%    \texttt{article} style, where no |\chapter| and |\opening|
%    commands are defined\footnote{A fact that was pointed out to me
%    by Nico Poppelier and was already used in Piet van Oostrum's
%    document style option~\texttt{nl}.}.
%
%    First we have the |\tableofcontents|,
%    |\listoffigures| and |\listoftables|:
%    \begin{macrocode}
\@ifundefined{contentsname}%
    {\def\tableofcontents{\section*{\contentsname\@mkboth
          {\uppercase{\contentsname}}{\uppercase{\contentsname}}}%
      \@starttoc{toc}}}{}

\@ifundefined{listfigurename}%
    {\def\listoffigures{\section*{\listfigurename\@mkboth
          {\uppercase{\listfigurename}}{\uppercase{\listfigurename}}}%
     \@starttoc{lof}}}{}

\@ifundefined{listtablename}%
    {\def\listoftables{\section*{\listtablename\@mkboth
          {\uppercase{\listtablename}}{\uppercase{\listtablename}}}%
      \@starttoc{lot}}}{}
%    \end{macrocode}
%
% Then the |\thebibliography| and |\theindex| environments.
%
%    \begin{macrocode}
\@ifundefined{refname}%
    {\def\thebibliography#1{\section*{\refname
      \@mkboth{\uppercase{\refname}}{\uppercase{\refname}}}%
      \list{[\arabic{enumi}]}{\settowidth\labelwidth{[#1]}%
        \leftmargin\labelwidth
        \advance\leftmargin\labelsep
        \usecounter{enumi}}%
        \def\newblock{\hskip.11em plus.33em minus.07em}%
        \sloppy\clubpenalty4000\widowpenalty\clubpenalty
        \sfcode`\.=1000\relax}}{}

\@ifundefined{indexname}%
    {\def\theindex{\@restonecoltrue\if@twocolumn\@restonecolfalse\fi
     \columnseprule \z@
     \columnsep 35pt\twocolumn[\section*{\indexname}]%
       \@mkboth{\uppercase{\indexname}}{\uppercase{\indexname}}%
       \thispagestyle{plain}%
       \parskip\z@ plus.3pt\parindent\z@\let\item\@idxitem}}{}
%    \end{macrocode}
%
% The |abstract| environment:
%
%    \begin{macrocode}
\@ifundefined{abstractname}%
    {\def\abstract{\if@twocolumn
    \section*{\abstractname}%
    \else \small
    \begin{center}%
    {\bf \abstractname\vspace{-.5em}\vspace{\z@}}%
    \end{center}%
    \quotation
    \fi}}{}
%    \end{macrocode}
%
% And last but not least, the macro |\part|:
%
%    \begin{macrocode}
\@ifundefined{partname}%
{\def\@part[#1]#2{\ifnum \c@secnumdepth >\m@ne
        \refstepcounter{part}%
        \addcontentsline{toc}{part}{\thepart
        \hspace{1em}#1}\else
      \addcontentsline{toc}{part}{#1}\fi
   {\parindent\z@ \raggedright
    \ifnum \c@secnumdepth >\m@ne
      \Large \bf \partname{} \thepart
      \par \nobreak
    \fi
    \huge \bf
    #2\markboth{}{}\par}%
    \nobreak
    \vskip 3ex\@afterheading}%
}{}
%    \end{macrocode}
%
%    This is all that needs to be done for the \texttt{article} style.
%
%    \begin{macrocode}
\or
%    \end{macrocode}
%
%    The next case is formed by the two styles \texttt{book} and
%    \texttt{report}.  Basically we have to do the same as for the
%    \texttt{article} style, except now we must also change the
%    |\chapter| command.
%
%    The tables of contents, figures and tables:
%    \begin{macrocode}
\@ifundefined{contentsname}%
    {\def\tableofcontents{\@restonecolfalse
      \if@twocolumn\@restonecoltrue\onecolumn
      \fi\chapter*{\contentsname\@mkboth
          {\uppercase{\contentsname}}{\uppercase{\contentsname}}}%
      \@starttoc{toc}%
      \csname if@restonecol\endcsname\twocolumn
      \csname fi\endcsname}}{}

\@ifundefined{listfigurename}%
    {\def\listoffigures{\@restonecolfalse
      \if@twocolumn\@restonecoltrue\onecolumn
      \fi\chapter*{\listfigurename\@mkboth
          {\uppercase{\listfigurename}}{\uppercase{\listfigurename}}}%
      \@starttoc{lof}%
      \csname if@restonecol\endcsname\twocolumn
      \csname fi\endcsname}}{}

\@ifundefined{listtablename}%
    {\def\listoftables{\@restonecolfalse
      \if@twocolumn\@restonecoltrue\onecolumn
      \fi\chapter*{\listtablename\@mkboth
          {\uppercase{\listtablename}}{\uppercase{\listtablename}}}%
      \@starttoc{lot}%
      \csname if@restonecol\endcsname\twocolumn
      \csname fi\endcsname}}{}
%    \end{macrocode}
%
%    Again, the |bibliography| and |index| environments; notice that
%    in this case we use |\bibname| instead of |\refname| as in the
%    definitions for the \texttt{article} style.  The reason for this
%    is that in the \texttt{article} document style the term
%    `References' is used in the definition of |\thebibliography|. In
%    the \texttt{report} and \texttt{book} document styles the term
%    `Bibliography' is used.
%    \begin{macrocode}
\@ifundefined{bibname}%
    {\def\thebibliography#1{\chapter*{\bibname
     \@mkboth{\uppercase{\bibname}}{\uppercase{\bibname}}}%
     \list{[\arabic{enumi}]}{\settowidth\labelwidth{[#1]}%
     \leftmargin\labelwidth \advance\leftmargin\labelsep
     \usecounter{enumi}}%
     \def\newblock{\hskip.11em plus.33em minus.07em}%
     \sloppy\clubpenalty4000\widowpenalty\clubpenalty
     \sfcode`\.=1000\relax}}{}

\@ifundefined{indexname}%
    {\def\theindex{\@restonecoltrue\if@twocolumn\@restonecolfalse\fi
    \columnseprule \z@
    \columnsep 35pt\twocolumn[\@makeschapterhead{\indexname}]%
      \@mkboth{\uppercase{\indexname}}{\uppercase{\indexname}}%
    \thispagestyle{plain}%
    \parskip\z@ plus.3pt\parindent\z@ \let\item\@idxitem}}{}
%    \end{macrocode}
%
% Here is the |abstract| environment:
%    \begin{macrocode}
\@ifundefined{abstractname}%
    {\def\abstract{\titlepage
    \null\vfil
    \begin{center}%
    {\bf \abstractname}%
    \end{center}}}{}
%    \end{macrocode}
%
%     And last but not least the |\chapter|, |\appendix| and
%    |\part| macros.
%    \begin{macrocode}
\@ifundefined{chaptername}{\def\@chapapp{\chaptername}}{}
%
\@ifundefined{appendixname}%
    {\def\appendix{\par
      \setcounter{chapter}{0}%
      \setcounter{section}{0}%
      \def\@chapapp{\appendixname}%
      \def\thechapter{\Alph{chapter}}}}{}
%
\@ifundefined{partname}%
    {\def\@part[#1]#2{\ifnum \c@secnumdepth >-2\relax
            \refstepcounter{part}%
            \addcontentsline{toc}{part}{\thepart
            \hspace{1em}#1}\else
            \addcontentsline{toc}{part}{#1}\fi
       \markboth{}{}%
       {\centering
        \ifnum \c@secnumdepth >-2\relax
          \huge\bf \partname{} \thepart
        \par
        \vskip 20pt \fi
        \Huge \bf
        #1\par}\@endpart}}{}%
%    \end{macrocode}
%
%    \begin{macrocode}
\or
%    \end{macrocode}
%
%    Now we address the case where \file{babel.def} is read after the
%    \texttt{letter} style. The \texttt{letter} document style
%    defines the macro |\opening| and some other macros that are
%    specific to \texttt{letter}. This means that we have to redefine
%    other macros, compared to the previous two cases.
%
%    First two macros for the material at the end of a letter, the
%    |\cc| and |\encl| macros.
%    \begin{macrocode}
\@ifundefined{ccname}%
    {\def\cc#1{\par\noindent
     \parbox[t]{\textwidth}%
     {\@hangfrom{\rm \ccname : }\ignorespaces #1\strut}\par}}{}

\@ifundefined{enclname}%
    {\def\encl#1{\par\noindent
     \parbox[t]{\textwidth}%
     {\@hangfrom{\rm \enclname : }\ignorespaces #1\strut}\par}}{}
%    \end{macrocode}
%
%    The last thing we have to do here is to redefine the
%    \texttt{headings} pagestyle:
% \changes{babel~3.3}{1993/07/11}{\cs{headpagename} should be
%    \cs{pagename}}
%    \begin{macrocode}
\@ifundefined{headtoname}%
    {\def\ps@headings{%
        \def\@oddhead{\sl \headtoname{} \ignorespaces\toname \hfil
                      \@date \hfil \pagename{} \thepage}%
        \def\@oddfoot{}}}{}
%    \end{macrocode}
%
%    This was the last of the four standard document styles, so if
%    |\doc@style| has another value we do nothing and just close the
%    \texttt{if} construction.
%    \begin{macrocode}
\fi
%    \end{macrocode}
%    Here ends the code that can be optionally included when a version
%    of \LaTeX\ is in use that is dated \emph{before} December~1991.
%    \begin{macrocode}
%</names>
%</core>
%    \end{macrocode}
%
% \subsection{Cross referencing macros}
%
%    The \LaTeX\ book states:
%  \begin{quote}
%    The \emph{key} argument is any sequence of letters, digits, and
%    punctuation symbols; upper- and lowercase letters are regarded as
%    different.
%  \end{quote}
%    When the above quote should still be true when a document is
%    typeset in a language that has active characters, special care
%    has to be taken of the category codes of these characters when
%    they appear in an argument of the cross referencing macros.
%
%    When a cross referencing command processes its argument, all
%    tokens in this argument should be character tokens with category
%    `letter' or `other'.
%
%    The only way to accomplish this in most cases is to use the trick
%    described in the \TeX book~\cite{DEK} (Appendix~D, page~382).
%    The primitive |\meaning| applied to a token expands to the
%    current meaning of this token.  For example, `|\meaning\A|' with
%    |\A| defined as `|\def\A#1{\B}|' expands to the characters
%    `|macro:#1->\B|' with all category codes set to `other' or
%    `space'.
%
%  \begin{macro}{\bbl@redefine}
% \changes{babel~3.5f}{1995/11/15}{Macro added}
%    To redefine a command, we save the old meaning of the macro.
%    Then we redefine it to call the original macro with the
%    `sanitized' argument.  The reason why we do it this way is that
%    we don't want to redefine the \LaTeX\ macros completely in case
%    their definitions change (they have changed in the past).
%
%    Because we need to redefine a number of commands we define the
%    command |\bbl@redefine| which takes care of this. It creates a
%    new control sequence, |\org@...|
%    \begin{macrocode}
%<*core|shorthands>
\def\bbl@redefine#1{%
  \edef\bbl@tempa{\expandafter\@gobble\string#1}%
  \expandafter\let\csname org@\bbl@tempa\endcsname#1
  \expandafter\def\csname\bbl@tempa\endcsname}
%    \end{macrocode}
%
%    This command should only be used in the preamble of the document.
%    \begin{macrocode}
\@onlypreamble\bbl@redefine
%    \end{macrocode}
%  \end{macro}
%
%  \begin{macro}{\bbl@redefine@long}
% \changes{babel~3.6f}{1997/01/14}{Macro added}
%    This version of |\babel@redefine| can be used to redefine |\long|
%    commands such as |\ifthenelse|.
%    \begin{macrocode}
\def\bbl@redefine@long#1{%
  \edef\bbl@tempa{\expandafter\@gobble\string#1}%
  \expandafter\let\csname org@\bbl@tempa\endcsname#1
  \expandafter\long\expandafter\def\csname\bbl@tempa\endcsname}
\@onlypreamble\bbl@redefine@long
%    \end{macrocode}
%  \end{macro}
%
%  \begin{macro}{\bbl@redefinerobust}
% \changes{babel~3.5f}{1995/11/15}{Macro added}
%    For commands that are redefined, but which \textit{might} be
%    robust we need a slightly more intelligent macro. A robust
%    command |foo| is defined to expand to |\protect|\verb*|\foo |. So
%    it is necessary to check whether \verb*|\foo | exists.
%    \begin{macrocode}
\def\bbl@redefinerobust#1{%
  \edef\bbl@tempa{\expandafter\@gobble\string#1}%
  \expandafter\ifx\csname \bbl@tempa\space\endcsname\relax
    \expandafter\let\csname org@\bbl@tempa\endcsname#1
    \expandafter\edef\csname\bbl@tempa\endcsname{\noexpand\protect
      \expandafter\noexpand\csname\bbl@tempa\space\endcsname}%
  \else
    \expandafter\let\csname org@\bbl@tempa\expandafter\endcsname
                    \csname\bbl@tempa\space\endcsname
  \fi
%    \end{macrocode}
%    The result of the code above is that the command that is being
%    redefined is always robust afterwards. Therefor all we need to do
%    now is define \verb*|\foo |.
% \changes{babel~3.5f}{1996/04/09}{Define \cs*{foo } instead of
%    \cs{foo}}
%    \begin{macrocode}
  \expandafter\def\csname\bbl@tempa\space\endcsname}
%    \end{macrocode}
%
%    This command should only be used in the preamble of the document.
%    \begin{macrocode}
\@onlypreamble\bbl@redefinerobust
%    \end{macrocode}
%  \end{macro}
%
%  \begin{macro}{\newlabel}
% \changes{babel~3.5f}{1995/11/15}{Now use \cs{bbl@redefine}}
%    The macro |\label| writes a line with a |\newlabel| command
%    into the |.aux| file to define labels.
%    \begin{macrocode}
%\bbl@redefine\newlabel#1#2{%
%  \@safe@activestrue\org@newlabel{#1}{#2}\@safe@activesfalse}
%    \end{macrocode}
%  \end{macro}
%
%  \begin{macro}{\@newl@bel}
% \changes{babel~3.6i}{1997/03/01}{Now redefine \cs{@newl@bel} instead
%    of \cs{@lbibitem} and \cs{newlabel}}
%    We need to change the definition of the \LaTeX-internal macro
%    |\@newl@bel|. This is needed because we need to make sure that
%    shorthand characters expand to their non-active version.
%
%^^A The following lines commented out in preparation for a change
%^^A in the LaTeX definition of \@enwl@bel... JLB 2000/10/01
%^^A
%^^A    To play it safe when redefining a \LaTeX-internal command we
%^^A    first check whether its definition didn't change.
%^^A    \begin{macrocode}
%^^A\CheckCommand*\@newl@bel[3]{%
%^^A  \@ifundefined{#1@#2}%
%^^A    \relax
%^^A    {\gdef \@multiplelabels {%
%^^A       \@latex@warning@no@line{There were multiply-defined labels}}%
%^^A     \@latex@warning@no@line{Label `#2' multiply defined}}%
%^^A  \global\@namedef{#1@#2}{#3}}
%^^A    \end{macrocode}
%^^A    Then we give the new definition.
%    \begin{macrocode}
\def\@newl@bel#1#2#3{%
%    \end{macrocode}
%    First we open a new group to keep the changed setting of
%    |\protect| local and then we set the |@safe@actives| switch to
%    true to make sure that any shorthand that appears in any of the
%    arguments immediately expands to its non-active self.
% \changes{babel~3.7a}{1997/12/19}{Call \cs{@safe@activestrue}
%    directly}
%    \begin{macrocode}
  {%
    \@safe@activestrue
    \@ifundefined{#1@#2}%
      \relax
      {%
        \gdef \@multiplelabels {%
          \@latex@warning@no@line{There were multiply-defined labels}}%
        \@latex@warning@no@line{Label `#2' multiply defined}%
      }%
    \global\@namedef{#1@#2}{#3}%
    }%
  }
%    \end{macrocode}
%  \end{macro}
%
%  \begin{macro}{\@testdef}
%    An internal \LaTeX\ macro used to test if the labels that have
%    been written on the |.aux| file have changed.  It is called by
%    the |\enddocument| macro. This macro needs to be completely
%    rewritten, using |\meaning|. The reason for this is that in some
%    cases the expansion of |\#1@#2| contains the same characters as
%    the |#3|; but the character codes differ. Therefor \LaTeX\ keeps
%    reporting that the labels may have changed.
% \changes{babel~3.4g}{1994/08/30}{Moved the \cs{def} inside the
%    macrocode environment}
% \changes{babel~3.5f}{1995/11/15}{Now use \cs{bbl@redefine}}
% \changes{babel~3.5f}{1996/01/09}{Complete rewrite of this macro as
%    the same character ended up with different category codes in the
%    labels that are being compared. Now use \cs{meaning}}
% \changes{babel~3.5f}{1996/01/16}{Use \cs{strip@prefix} only on
%    \cs{bbl@tempa} when it is not \cs{relax}}
% \changes{babel~3.6i}{1997/02/28}{Make sure that shorthands don't get
%    expanded at the wrong moment.}
% \changes{babel~3.6i}{1997/03/01}{\cs{@safe@activesfalse} is now
%    part of the label definition}
% \changes{babel~3.7a}{1998/03/13}{Removed \cs{@safe@activesfalse}
%    from the label definition}
%    \begin{macrocode}
\CheckCommand*\@testdef[3]{%
  \def\reserved@a{#3}%
  \expandafter \ifx \csname #1@#2\endcsname \reserved@a
  \else
    \@tempswatrue
  \fi}
%    \end{macrocode}
%    Now that we made sure that |\@testdef| still has the same
%    definition we can rewrite it. First we make the shorthands
%    `safe'.
%    \begin{macrocode}
\def\@testdef #1#2#3{%
  \@safe@activestrue
%    \end{macrocode}
%    Then we use |\bbl@tempa| as an `alias' for the macro that
%    contains the label which is being checked.
%    \begin{macrocode}
  \expandafter\let\expandafter\bbl@tempa\csname #1@#2\endcsname
%    \end{macrocode}
%    Then we define |\bbl@tempb| just as |\@newl@bel| does it.
%    \begin{macrocode}
  \def\bbl@tempb{#3}%
  \@safe@activesfalse
%    \end{macrocode}
%    When the label is defined we replace the definition of
%    |\bbl@tempa| by its meaning.
%    \begin{macrocode}
  \ifx\bbl@tempa\relax
  \else
    \edef\bbl@tempa{\expandafter\strip@prefix\meaning\bbl@tempa}%
  \fi
%    \end{macrocode}
%    We do the same for |\bbl@tempb|.
%    \begin{macrocode}
  \edef\bbl@tempb{\expandafter\strip@prefix\meaning\bbl@tempb}%
%    \end{macrocode}
%    If the label didn't change, |\bbl@tempa| and |\bbl@tempb| should
%    be identical macros.
%    \begin{macrocode}
  \ifx \bbl@tempa \bbl@tempb
  \else
    \@tempswatrue
  \fi}
%    \end{macrocode}
%  \end{macro}
%
%  \begin{macro}{\ref}
%  \begin{macro}{\pageref}
%    The same holds for the macro |\ref| that references a label
%    and |\pageref| to reference a page. So we redefine |\ref| and
%    |\pageref|. While we change these macros, we make them robust as
%    well (if they weren't already) to prevent problems if they should
%    become expanded at the wrong moment.
% \changes{babel~3.5b}{1995/03/07}{Made \cs{ref} and \cs{pageref}
%    robust (PR1353)}
% \changes{babel~3.5d}{1995/07/04}{use a different control sequence
%    while making \cs{ref} and \cs{pageref} robust}
% \changes{babel~3.5f}{1995/11/06}{redefine \cs*{ref } if it exists
%    instead of \cs{ref}}
% \changes{babel~3.5f}{1995/11/15}{Now use \cs{bbl@redefinerobust}}
% \changes{babel~3.5f}{1996/01/19}{redefine \cs{\@setref} instead of
%    \cs{ref} and \cs{pageref} in \LaTeXe.}
% \changes{babel~3.5f}{1996/01/21}{Reverse the previous change as it
%    inhibits the use of active characters in labels}
%    \begin{macrocode}
\bbl@redefinerobust\ref#1{%
  \@safe@activestrue\org@ref{#1}\@safe@activesfalse}
\bbl@redefinerobust\pageref#1{%
  \@safe@activestrue\org@pageref{#1}\@safe@activesfalse}
%    \end{macrocode}
%  \end{macro}
%  \end{macro}
%
%  \begin{macro}{\@citex}
% \changes{babel~3.5f}{1995/11/15}{Now use \cs{bbl@redefine}}
%    The macro used to cite from a bibliography, |\cite|, uses an
%    internal macro, |\@citex|.
%    It is this internal macro that picks up the argument(s),
%    so we redefine this internal macro and leave |\cite| alone. The
%    first argument is used for typesetting, so the shorthands need
%    only be deactivated in the second argument.
% \changes{babel~3.7g}{2000/10/01}{The shorthands need to be
%    deactivated for the second argument of \cs{@citex} only.}
%    \begin{macrocode}
\bbl@redefine\@citex[#1]#2{%
  \@safe@activestrue\edef\@tempa{#2}\@safe@activesfalse
  \org@@citex[#1]{\@tempa}}
%    \end{macrocode}
%    Unfortunately, the packages \pkg{natbib} and \pkg{cite} need a
%    different definition of |\@citex|...
%    To begin with, \pkg{natbib} has a definition for |\@citex| with
%    \emph{three} arguments... We only know that a package is loaded
%    when |\begin{document}| is executed, so we need to postpone the
%    different redefinition.
%    \begin{macrocode}
\AtBeginDocument{%
  \@ifpackageloaded{natbib}{%
%    \end{macrocode}
%    Notice that we use |\def| here instead of |\bbl@redefine| because
%    |\org@@citex| is already defined and we don't want to overwrite
%    that definition (it would result in parameter stack overflow
%    because of a circular definition).
%    \begin{macrocode}
    \def\@citex[#1][#2]#3{%
      \@safe@activestrue\edef\@tempa{#3}\@safe@activesfalse
      \org@@citex[#1][#2]{\@tempa}}%
  }{}}
%    \end{macrocode}
%    The package \pkg{cite} has a definition of |\@citex| where the
%    shorthands need to be turned off in both arguments.
%    \begin{macrocode}
\AtBeginDocument{%
  \@ifpackageloaded{cite}{%
    \def\@citex[#1]#2{%
      \@safe@activestrue\org@@citex[#1]{#2}\@safe@activesfalse}%
    }{}}
%    \end{macrocode}
%  \end{macro}
%
%  \begin{macro}{\nocite}
% \changes{babel~3.5f}{1995/11/15}{Now use \cs{bbl@redefine}}
%    The macro |\nocite| which is used to instruct BiB\TeX\ to
%    extract uncited references from the database.
%    \begin{macrocode}
\bbl@redefine\nocite#1{%
  \@safe@activestrue\org@nocite{#1}\@safe@activesfalse}
%    \end{macrocode}
%  \end{macro}
%
%  \begin{macro}{\bibcite}
% \changes{babel~3.5f}{1995/11/15}{Now use \cs{bbl@redefine}}
%    The macro that is used in the |.aux| file to define citation
%    labels. When packages such as \pkg{natbib} or \pkg{cite} are not
%    loaded its second argument is used to typeset the citation
%    label. In that case, this second argument can contain active
%    characters but is used in an environment where
%    |\@safe@activestrue| is in effect. This switch needs to be reset
%    inside the |\hbox| which contains the citation label. In order to
%    determine during \file{.aux} file processing which definition of
%    |\bibcite| is needed we define |\bibcite| in such a way that it
%    redefines itself with the proper definition.
% \changes{babel~3.6s}{1999/04/13}{Need to determine `online' which
%    definition of \cs{bibcite} is needed}
% \changes{babel~3.6v}{1999/04/21}{Also check for \pkg{cite} it can't
%    handle \cs{@safe@activesfalse} in its second argument}
%    \begin{macrocode}
\bbl@redefine\bibcite{%
%    \end{macrocode}
%    We call |\bbl@cite@choice| to select the proper definition for
%    |\bibcite|. This new definition is then activated.
%    \begin{macrocode}
  \bbl@cite@choice
  \bibcite}
%    \end{macrocode}
%  \end{macro}
%
%  \begin{macro}{\bbl@bibcite}
% \changes{babel~3.6v}{1999/04/21}{Macro \cs{bbl@bibcite} added}
%    The macro |\bbl@bibcite| holds the definition of |\bibcite|
%    needed when neither \pkg{natbib} nor \pkg{cite} is loaded.
%    \begin{macrocode}
\def\bbl@bibcite#1#2{%
  \org@bibcite{#1}{\@safe@activesfalse#2}}
%    \end{macrocode}
%  \end{macro}
%
%  \begin{macro}{\bbl@cite@choice}
% \changes{babel~3.6v}{1999/04/21}{Macro \cs{bbl@cite@choice} added}
%    The macro |\bbl@cite@choice| determines which definition of
%    |\bibcite| is needed.
%    \begin{macrocode}
\def\bbl@cite@choice{%
%    \end{macrocode}
%    First we give |\bibcite| its default definition.
%    \begin{macrocode}
  \global\let\bibcite\bbl@bibcite
%    \end{macrocode}
%    Then, when \pkg{natbib} is loaded we restore the original
%    definition of |\bibcite| .
%    \begin{macrocode}
  \@ifpackageloaded{natbib}{\global\let\bibcite\org@bibcite}{}%
%    \end{macrocode}
%    For \pkg{cite} we do the same.
%    \begin{macrocode}
  \@ifpackageloaded{cite}{\global\let\bibcite\org@bibcite}{}%
%    \end{macrocode}
%    Make sure this only happens once.
%    \begin{macrocode}
  \global\let\bbl@cite@choice\relax
  }
%    \end{macrocode}
%
%    When a document is run for the first time, no \file{.aux} file is
%    available, and |\bibcite| will not yet be properly defined. In
%    this case, this has to happen before the document starts.
%    \begin{macrocode}
\AtBeginDocument{\bbl@cite@choice}
%    \end{macrocode}
%  \end{macro}
%
%  \begin{macro}{\@bibitem}
% \changes{babel~3.5f}{1995/11/15}{Now use \cs{bbl@redefine}}
%    One of the two internal \LaTeX\ macros called by |\bibitem|
%    that write the citation label on the |.aux| file.
%    \begin{macrocode}
\bbl@redefine\@bibitem#1{%
  \@safe@activestrue\org@@bibitem{#1}\@safe@activesfalse}
%    \end{macrocode}
%  \end{macro}
%
%  \subsection{marks}
%
%  \begin{macro}{\markright}
% \changes{babel~3.6i}{1997/03/15}{Added redefinition of \cs{mark...}
%    commands}
%    Because the output routine is asynchronous, we must
%    pass the current language attribute to the head lines, together
%    with the text that is put into them. To achieve this we need to
%    adapt the definition of |\markright| and |\markboth| somewhat.
% \changes{babel~3.7c}{1999/04/08}{Removed the use of \cs{head@lang}
%    (PR 2990)}
% \changes{babel~3.7c}{1999/04/09}{Avoid expanding the arguments by
%    storing them in token registers}
% \changes{babel~3.7m}{2003/11/15}{added \cs{bbl@restore@actives} to
%    the mark}
% \changes{babel~3.8c}{2004/05/26}{No need to add \emph{anything} to
%    an empty mark; prevented this by checking the contents of the
%    argument}
% \changes{babel~3.8f}{2005/05/15}{Make the definition independent of
%    the original definition; expand \cs{languagename} before passing
%    it into the token registers} 
%    \begin{macrocode}
\bbl@redefine\markright#1{%
%    \end{macrocode}
%    First of all we temporarily store the language switching command,
%    using an expanded definition in order to get the current value of
%    |\languagename|. 
%    \begin{macrocode}
  \edef\bbl@tempb{\noexpand\protect
    \noexpand\foreignlanguage{\languagename}}%
%    \end{macrocode}
%    Then, we check whether the argument is empty; if it is, we
%    just make sure the scratch token register is empty.
%    \begin{macrocode}
  \def\bbl@arg{#1}%
  \ifx\bbl@arg\@empty
    \toks@{}%
  \else
%    \end{macrocode}
%    Next, we store the argument to |\markright| in the scratch token
%    register, together with the expansion of |\bbl@tempb| (containing
%    the language switching command) as defined before. This way
%    these commands will not be expanded by using |\edef| later
%    on, and we make sure that the text is typeset using the
%    correct language settings. While doing so, we make sure that
%    active characters that may end up in the mark are not disabled by
%    the output routine kicking in while \cs{@safe@activestrue} is in
%    effect.
%    \begin{macrocode}
    \expandafter\toks@\expandafter{%
             \bbl@tempb{\protect\bbl@restore@actives#1}}%
  \fi
%    \end{macrocode}
%    Then we define a temporary control sequence using |\edef|.
%    \begin{macrocode}
  \edef\bbl@tempa{%
%    \end{macrocode}
%     When |\bbl@tempa| is executed, only |\languagename| will be
%    expanded, because of the way the token register was filled.
%    \begin{macrocode}
    \noexpand\org@markright{\the\toks@}}%
  \bbl@tempa
}
%    \end{macrocode}
%  \end{macro}
%
%  \begin{macro}{\markboth}
%  \begin{macro}{\@mkboth}
%    The definition of |\markboth| is equivalent to that of
%    |\markright|, except that we need two token registers. The
%    documentclasses \cls{report} and \cls{book} define and set the
%    headings for the page. While doing so they also store a copy of
%    |\markboth| in |\@mkboth|. Therefor we need to check whether
%    |\@mkboth| has already been set. If so we neeed to do that again
%    with the new definition of |\makrboth|.
% \changes{babel~3.7m}{2003/11/15}{added \cs{bbl@restore@actives} to
%    the mark}
% \changes{babel~3.8c}{2004/05/26}{No need to add \emph{anything} to
%    an empty mark, prevented this by checking the contents of the
%    arguments} 
% \changes{babel~3.8f}{2005/05/15}{Make the definition independent of
%    the original definition; expand \cs{languagename} before passing
%    it into the token registers} 
% \changes{babel~3.8j}{2008/03/21}{Added setting of \cs{@mkboth} (PR
%    3826)} 
%    \begin{macrocode}
\ifx\@mkboth\markboth
  \def\bbl@tempc{\let\@mkboth\markboth}
\else
  \def\bbl@tempc{}
\fi
%    \end{macrocode}
%    Now we can start the new definition of |\markboth|
%    \begin{macrocode}
\bbl@redefine\markboth#1#2{%
  \edef\bbl@tempb{\noexpand\protect
    \noexpand\foreignlanguage{\languagename}}%
  \def\bbl@arg{#1}%
  \ifx\bbl@arg\@empty
    \toks@{}%
  \else
   \expandafter\toks@\expandafter{%
             \bbl@tempb{\protect\bbl@restore@actives#1}}%
  \fi
  \def\bbl@arg{#2}%
  \ifx\bbl@arg\@empty
    \toks8{}%
  \else
    \expandafter\toks8\expandafter{%
             \bbl@tempb{\protect\bbl@restore@actives#2}}%
  \fi
  \edef\bbl@tempa{%
    \noexpand\org@markboth{\the\toks@}{\the\toks8}}%
  \bbl@tempa
}
%    \end{macrocode}
%    and copy it to |\@mkboth| if necesary.
%    \begin{macrocode}
\bbl@tempc
%</core|shorthands>
%    \end{macrocode}
%  \end{macro}
%  \end{macro}
%
%  \subsection{Encoding issues (part 2)}
%
% \changes{babel~3.7c}{1999/04/16}{Removed redefinition of \cs{@roman}
%    and \cs{@Roman}}
%
%  \begin{macro}{\TeX}
%  \begin{macro}{\LaTeX}
% \changes{babel~3.7a}{1998/03/12}{Make \TeX\ and \LaTeX\ logos
%    encoding-independent}
%    Because documents may use font encodings other than one of the
%    latin encodings, we make sure that the logos of \TeX\ and
%    \LaTeX\ always come out in the right encoding.
%    \begin{macrocode}
%<*core>
\bbl@redefine\TeX{\textlatin{\org@TeX}}
\bbl@redefine\LaTeX{\textlatin{\org@LaTeX}}
%</core>
%    \end{macrocode}
%  \end{macro}
%  \end{macro}
%
%  \subsection{Preventing clashes with other packages}
%
%  \subsubsection{\pkg{ifthen}}
%
%  \begin{macro}{\ifthenelse}
% \changes{babel~3.5g}{1996/08/11}{Redefinition of \cs{ifthenelse}
%    added to circumvent problems with \cs{pageref} in the argument of
%    \cs{isodd}}
%    Sometimes a document writer wants to create a special effect
%    depending on the page a certain fragment of text appears on. This
%    can be achieved by the following piece of code:
% \begin{verbatim}
%    \ifthenelse{\isodd{\pageref{some:label}}}
%               {code for odd pages}
%               {code for even pages}
% \end{verbatim}
%    In order for this to work the argument of |\isodd| needs to be
%    fully expandable. With the above redefinition of |\pageref| it is
%    not in the case of this example. To overcome that, we add some
%    code to the definition of |\ifthenelse| to make things work.
%
%    The first thing we need to do is check if the package
%    \pkg{ifthen} is loaded. This should be done at |\begin{document}|
%    time.
%    \begin{macrocode}
%<*package>
\AtBeginDocument{%
  \@ifpackageloaded{ifthen}{%
%    \end{macrocode}
%    Then we can redefine |\ifthenelse|:
% \changes{babel~3.6f}{1997/01/14}{\cs{ifthenelse} needs to be long}
% \changes{babel~3.9a}{2012/06/22}{\cs{ref} is also taken into account}
%    \begin{macrocode}
    \bbl@redefine@long\ifthenelse#1#2#3{%
%    \end{macrocode}
%    We want to revert the definition of |\pageref| and |\ref| to
%    their original definition for the duration of |\ifthenelse|,
%    so we first need to store their current meanings.
%    \begin{macrocode}
      \let\bbl@tempa\pageref
      \let\pageref\org@pageref
      \let\bbl@tempb\ref
      \let\ref\org@ref
%    \end{macrocode}
%    Then we can set the |\@safe@actives| switch and call the original
%    |\ifthenelse|. In order to be able to use shorthands in the
%    second and third arguments of |\ifthenelse| the resetting of the
%    switch \emph{and} the definition of |\pageref| happens inside
%    those arguments. 
% \changes{babel~3.6i}{1997/02/25}{Now reset the @safe@actives switch
%    inside the 2nd and 3rd arguments of \cs{ifthenelse}}
% \changes{babel~3.7f}{2000/06/29}{\cs{pageref} needs to have its
%    babel definition reinstated in the second and third arguments}
%    \begin{macrocode}
      \@safe@activestrue
      \org@ifthenelse{#1}{%
        \let\pageref\bbl@tempa
        \let\ref\bbl@tempb
        \@safe@activesfalse
        #2}{%
        \let\pageref\bbl@tempa
        \let\ref\bbl@tempb
        \@safe@activesfalse
        #3}%
      }%
%    \end{macrocode}
%    When the package wasn't loaded we do nothing.
%    \begin{macrocode}
    }{}%
  }
%    \end{macrocode}
%  \end{macro}
%
%  \subsubsection{\pkg{varioref}}
%
%  \begin{macro}{\@@vpageref}
% \changes{babel~3.6a}{1996/10/29}{Redefinition of \cs{@@vpageref}
%    added to circumvent problems with active \texttt{:} in the
%    argument of \cs{vref} when \pkg{varioref} is used}
%  \begin{macro}{\vrefpagenum}
% \changes{babel~3.7o}{2003/11/18}{Added redefinition of
%    \cs{vrefpagenum} which deals with ranges of pages}
%  \begin{macro}{\Ref}
% \changes{babel~3.8g}{2005/05/21}{We also need to adapt \cs{Ref}
%    which needs to be able to uppercase the first letter of the
%    expansion of \cs{ref}} 
%    When the package varioref is in use we need to modify its
%    internal command |\@@vpageref| in order to prevent problems when
%    an active character ends up in the argument of |\vref|.
%    \begin{macrocode}
\AtBeginDocument{%
  \@ifpackageloaded{varioref}{%
    \bbl@redefine\@@vpageref#1[#2]#3{%
      \@safe@activestrue
      \org@@@vpageref{#1}[#2]{#3}%
      \@safe@activesfalse}%
%    \end{macrocode}
%    The same needs to happen for |\vrefpagenum|.
%    \begin{macrocode}
    \bbl@redefine\vrefpagenum#1#2{%
      \@safe@activestrue
      \org@vrefpagenum{#1}{#2}%
      \@safe@activesfalse}%
%    \end{macrocode}
%    The package \pkg{varioref} defines |\Ref| to be a robust command
%    wich uppercases the first character of the reference text. In
%    order to be able to do that it needs to access the exandable form
%    of |\ref|. So we employ a little trick here. We redefine the
%    (internal) command \verb*|\Ref | to call |\org@ref| instead of
%    |\ref|. The disadvantgage of this solution is that whenever the
%    derfinition of |\Ref| changes, this definition needs to be updated
%    as well.
%    \begin{macrocode}
    \expandafter\def\csname Ref \endcsname#1{%
      \protected@edef\@tempa{\org@ref{#1}}\expandafter\MakeUppercase\@tempa}
    }{}%
  }
%    \end{macrocode}
%  \end{macro}
%  \end{macro}
%  \end{macro}
%
%  \subsubsection{\pkg{hhline}}
%
%  \begin{macro}{\hhline}
%    Delaying the activation of the shorthand characters has introduced
%    a problem with the \pkg{hhline} package. The reason is that it
%    uses the `:' character which is made active by the french support
%    in \babel. Therefor we need to \emph{reload} the package when
%    the `:' is an active character.
%
%    So at |\begin{document}| we check whether \pkg{hhline} is loaded.
%    \begin{macrocode}
\AtBeginDocument{%
  \@ifpackageloaded{hhline}%
%    \end{macrocode}
%    Then we check whether the expansion of |\normal@char:| is not
%    equal to |\relax|.
% \changes{babel~3.8b}{2004/04/19}{added \cs{string} to prevent
%    unwanted expansion of the colon}
%    \begin{macrocode}
    {\expandafter\ifx\csname normal@char\string:\endcsname\relax
     \else
%    \end{macrocode}
%    In that case we simply reload the package. Note that this happens
%    \emph{after} the category code of the @-sign has been changed to
%    other, so we need to temporarily change it to letter again.
%    \begin{macrocode}
       \makeatletter
       \def\@currname{hhline}% \iffalse meta-comment
%
% Copyright 1993-2014
%
% The LaTeX3 Project and any individual authors listed elsewhere
% in this file.
%
% This file is part of the Standard LaTeX `Tools Bundle'.
% -------------------------------------------------------
%
% It may be distributed and/or modified under the
% conditions of the LaTeX Project Public License, either version 1.3c
% of this license or (at your option) any later version.
% The latest version of this license is in
%    http://www.latex-project.org/lppl.txt
% and version 1.3c or later is part of all distributions of LaTeX
% version 2005/12/01 or later.
%
% The list of all files belonging to the LaTeX `Tools Bundle' is
% given in the file `manifest.txt'.
%
% \fi
% \iffalse
%% File: hhline.dtx Copyright (C) 1991-1994 David Carlisle
%
%<package>\NeedsTeXFormat{LaTeX2e}
%<package>\ProvidesPackage{hhline}
%<package>         [2014/10/28 v2.03 Table rule package (DPC)]
%
%<*driver>
\documentclass{ltxdoc}
\usepackage{hhline}
\GetFileInfo{hhline.sty}
\begin{document}
\title{The \textsf{hhline} package\thanks{This file
        has version number \fileversion, last
        revised \filedate.}}
\author{David Carlisle\\carlisle@cs.man.ac.uk}
\date{\filedate}
 \maketitle
 \DeleteShortVerb{\|}
 \DocInput{hhline.dtx}
\end{document}
%</driver>
% \fi
%
%
% \changes{v1.00}{1991/06/04}{Initial Version}
% \changes{v2.00}{1991/11/06}
%     {Add tilde which allows \cmd\cline-like constructions.}
% \changes{v2.01}{1992/06/26}
%    {Re-issue for the new  doc and docstrip.}
% \changes{v2.02}{1994/03/14}
%    {Update for LaTeX2e.}
% \changes{v2.03}{1994/05/23}
%    {New style warning.}
%
%
% \CheckSum{244}
%
% \MakeShortVerb{\"}
%
% \begin{abstract}
% "\hhline" produces a line like "\hline", or a double line like
% "\hline\hline", except for its interaction with vertical lines.
% \end{abstract}
%
% \arrayrulewidth=1pt
% \doublerulesep=3pt
%
% \section{Introduction}
% The argument to "\hhline" is similar to the preamble of an {\tt
% array} or {\tt tabular}. It consists of a list of tokens with the
% following meanings:
% \[
% \begin{tabular}{cl}
%   "="   & A double hline the width of a column.\\
%   "-"   & A single hline the width of a column.\\[10pt]
%   "~"   & A column with no hline.\\[10pt]
%
%   "|"   & A vline which `cuts' through a double (or single) hline.\\
%   ":"   & A vline which is broken by a double hline.\\[10pt]
%
%   "#"   & A double hline segment between two vlines.\\
%   "t"   & The top half of a double hline segment.\\
%   "b"   & The bottom half of a double hline segment.\\
%
%   "*"   & "*{3}{==#}" expands to "==#==#==#",
%                   as in the {\tt*}-form for the preamble.
% \end{tabular}
% \]
% If a double vline is specified ("||" or "::") then the hlines
% produced by "\hhline" are broken. To obtain the effect of an hline
% `cutting through' the double vline, use a "#" or omit the vline
% specifiers, depending on whether or not you wish the double vline to
% break.
%
% The tokens {\tt t} and {\tt b} must be used between two vertical
% rules. "|tb|" produces the same lines  as "#", but is much less
% efficient. The main use for these are to make constructions like
% "|t:" (top left corner) and ":b|" (bottom right corner).
%
% If "\hhline" is used to make a single hline, then the argument
% should only contain the tokens "-", "~"  and "|" (and
% {\tt*}-expressions).
%
% An example using most of these features is:
% \[
% \vcenter{\hsize=2in\begin{verbatim}
% \begin{tabular}{||cc||c|c||}
% \hhline{|t:==:t:==:t|}
% a&b&c&d\\
% \hhline{|:==:|~|~||}
% 1&2&3&4\\
% \hhline{#==#~|=#}
% i&j&k&l\\
% \hhline{||--||--||}
% w&x&y&z\\
% \hhline{|b:==:b:==:b|}
% \end{tabular}
% \end{verbatim}
% }
% \qquad
% \begin{tabular}{||cc||c|c||}
% \hhline{|t:==:t:==:t|}
% a&b&c&d\\
% \hhline{|:==:|~|~||}
% 1&2&3&4\\
% \hhline{#==#~|=#}
% i&j&k&l\\
% \hhline{||--||--||}
% w&x&y&z\\
% \hhline{|b:==:b:==:b|}
% \end{tabular}
% \]
%
% The lines produced by \LaTeX's "\hline" consist of a single (\TeX\
% primitive) "\hrule". The lines produced by "\hhline" are made
% up of lots of small line segments. \TeX\ will place these very
% accurately in the {\tt .dvi} file, but the program that you use to
% print the {\tt .dvi} file may not line up these segments exactly. (A
% similar problem can occur with diagonal lines in the {\tt picture}
% environment.)
%
% If this effect causes a problem, you could try a different driver
% program, or if this is not possible, increasing "\arrayrulewidth"
% may help to reduce the effect.
%
% \StopEventually{}
%
% \section{The Macros}
%
%    \begin{macrocode}
%<*package>
%    \end{macrocode}
%
% \begin{macro}{\HH@box}
% Makes a box containing a double hline segment. The most common case,
% both rules of length "\doublerulesep" will be stored in "\box1", this
% is not initialised until "\hhline" is called as the user may change
% the parameters "\doublerulesep" and "\arrayrulewidth". The two
% arguments to "\HH@box" are the widths (ie lengths) of the top and
% bottom rules.
%    \begin{macrocode}
\def\HH@box#1#2{\vbox{%
  \hrule \@height \arrayrulewidth \@width #1
  \vskip \doublerulesep
  \hrule \@height \arrayrulewidth \@width #2}}
%    \end{macrocode}
% \end{macro}
%
% \begin{macro}{\HH@add}
% Build up the preamble in the register "\toks@".
%    \begin{macrocode}
\def\HH@add#1{\toks@\expandafter{\the\toks@#1}}
%    \end{macrocode}
% \end{macro}

% \begin{macro}{\HH@xexpast}
% \begin{macro}{\HH@xexnoop}
% We `borrow' the version of "\@xexpast" from Mittelbach's array.sty,
% as this allows "#" to appear in the argument list.
%    \begin{macrocode}
\def\HH@xexpast#1*#2#3#4\@@{%
   \@tempcnta #2
   \toks@={#1}\@temptokena={#3}%
   \let\the@toksz\relax \let\the@toks\relax
   \def\@tempa{\the@toksz}%
   \ifnum\@tempcnta >0 \@whilenum\@tempcnta >0\do
     {\edef\@tempa{\@tempa\the@toks}\advance \@tempcnta \m@ne}%
       \let \@tempb \HH@xexpast \else
       \let \@tempb \HH@xexnoop \fi
   \def\the@toksz{\the\toks@}\def\the@toks{\the\@temptokena}%
   \edef\@tempa{\@tempa}%
   \expandafter \@tempb \@tempa #4\@@}

\def\HH@xexnoop#1\@@{}
%    \end{macrocode}
% \end{macro}
% \end{macro}
%
% \begin{macro}{\hhline}
% Use a simplified version of "\@mkpream" to break apart the argument
% to "\hhline". Actually it is oversimplified, It assumes that the
% vertical rules are at the end of the column. If you were to specify
% "c|@{xx}|" in the array argument, then "\hhline" would not be
% able to access the first vertical rule. (It ought to have an "@"
% option, and add "\leaders" up to the width of a box containing the
% "@"-expression. We use a loop made with "\futurelet" rather
% than "\@tfor" so that we can use "#" to denote the crossing of
% a double hline with a double vline.\\
% "\if@firstamp" is true in the first column and false otherwise.\\
% "\if@tempswa"  is true if the previous entry was a vline
%                     (":", "|" or "#").
%    \begin{macrocode}
\def\hhline#1{\omit\@firstamptrue\@tempswafalse
%    \end{macrocode}
% Put two rules of width "\doublerulesep" in "\box1"
%    \begin{macrocode}
\global\setbox\@ne\HH@box\doublerulesep\doublerulesep
%    \end{macrocode}
% If Mittelbach's {\tt array.sty} is loaded, we do not need the negative
% "\hskip"'s around vertical rules.
%    \begin{macrocode}
  \xdef\@tempc{\ifx\extrarowheight\HH@undef\hskip-.5\arrayrulewidth\fi}%
%    \end{macrocode}
% Now expand the {\tt*}-forms and add dummy tokens ( "\relax" and
% "`" ) to either end of the token list. Call "\HH@let" to start
% processing the token list.
%    \begin{macrocode}
    \HH@xexpast\relax#1*0x\@@\toks@{}\expandafter\HH@let\@tempa`}
%    \end{macrocode}
% \end{macro}

% \begin{macro}{\HH@let}
% Discard the last token, look at the next one.
%    \begin{macrocode}
\def\HH@let#1{\futurelet\@tempb\HH@loop}
%    \end{macrocode}
% \end{macro}

% \begin{macro}{\HH@loop}
% The main loop. Note we use "\ifx" rather than "\if" in
% version~2 as the new token "~" is active.
%    \begin{macrocode}
\def\HH@loop{%
%    \end{macrocode}
% If next token is "`", stop the loop and put the lines into this row
% of the alignment.
%    \begin{macrocode}
  \ifx\@tempb`\def\next##1{\the\toks@\cr}\else\let\next\HH@let
%    \end{macrocode}
% "|", add a vertical rule (across either a double or
% single hline).
%    \begin{macrocode}
  \ifx\@tempb|\if@tempswa\HH@add{\hskip\doublerulesep}\fi\@tempswatrue
          \HH@add{\@tempc\vline\@tempc}\else
%    \end{macrocode}
% ":", add a broken vertical rule (across a double hline).
%    \begin{macrocode}
  \ifx\@tempb:\if@tempswa\HH@add{\hskip\doublerulesep}\fi\@tempswatrue
      \HH@add{\@tempc\HH@box\arrayrulewidth\arrayrulewidth\@tempc}\else
%    \end{macrocode}
% "#", add a double hline segment between two vlines.
%    \begin{macrocode}
  \ifx\@tempb##\if@tempswa\HH@add{\hskip\doublerulesep}\fi\@tempswatrue
         \HH@add{\@tempc\vline\@tempc\copy\@ne\@tempc\vline\@tempc}\else
%    \end{macrocode}
% "~", A column with no hline (this gives an effect similar to
% \verb+\cline+).
%    \begin{macrocode}
  \ifx\@tempb~\@tempswafalse
           \if@firstamp\@firstampfalse\else\HH@add{&\omit}\fi
              \HH@add{\hfil}\else
%    \end{macrocode}
% "-", add a single hline across the column.
%    \begin{macrocode}
  \ifx\@tempb-\@tempswafalse
           \if@firstamp\@firstampfalse\else\HH@add{&\omit}\fi
              \HH@add{\leaders\hrule\@height\arrayrulewidth\hfil}\else
%    \end{macrocode}
% "=", add a double hline across the column.
%    \begin{macrocode}
  \ifx\@tempb=\@tempswafalse
       \if@firstamp\@firstampfalse\else\HH@add{&\omit}\fi
%    \end{macrocode}
%     Put in as many copies of "\box1" as possible with
%     "\leaders", this may leave gaps at the ends, so put an extra box
%     at each end, overlapping the "\leaders".
%    \begin{macrocode}
       \HH@add
          {\rlap{\copy\@ne}\leaders\copy\@ne\hfil\llap{\copy\@ne}}\else
%    \end{macrocode}
% "t", add the top half of a double hline segment, in a "\rlap"
% so that it may be used with {\tt b}.
%    \begin{macrocode}
  \ifx\@tempb t\HH@add{\rlap{\HH@box\doublerulesep\z@}}\else
%    \end{macrocode}
% "b", add the bottom half of a double hline segment in a "\rlap"
% so that it may be used with {\tt t}.
%    \begin{macrocode}
  \ifx\@tempb b\HH@add{\rlap{\HH@box\z@\doublerulesep}}\else
%    \end{macrocode}
% Otherwise ignore the token, with a warning.
%    \begin{macrocode}
  \PackageWarning{hhline}%
      {\meaning\@tempb\space ignored in \noexpand\hhline argument%
       \MessageBreak}%
  \fi\fi\fi\fi\fi\fi\fi\fi\fi
%    \end{macrocode}
% Go around the loop again.
%    \begin{macrocode}
  \next}
%    \end{macrocode}
% \end{macro}
%
%    \begin{macrocode}
%</package>
%    \end{macrocode}
%
% \Finale
\endinput
\makeatother
     \fi}%
    {}}
%    \end{macrocode}
%  \end{macro}
%
%  \subsubsection{\pkg{hyperref}}
%
%  \begin{macro}{\pdfstringdefDisableCommands}
% \changes{babel~3.8j}{2008/03/16}{Inform \pkg{hyperref} to use
%    shorthands at system level (PR4006)}
%    Although a number of interworking problems between \pkg{babel}
%    and \pkg{hyperref} are tackled by \pkg{hyperref} itself we need
%    to take care of correctly handling the shorthand characters.
%    When they get expanded inside a bookmark a warning will appear in
%    the log file which can be prevented. This is done by informing
%    \pkg{hyperref} that it should the shorthands as defined on the
%    system level rather than at the user level.
%    
%    \begin{macrocode}
\AtBeginDocument{%
  \@ifundefined{pdfstringdefDisableCommands}%
    {}%
    {\pdfstringdefDisableCommands{%
       \languageshorthands{system}}%
    }%
}
%    \end{macrocode}
%  \end{macro}
%
%
%  \subsubsection{General}
%
%  \begin{macro}{\FOREIGNLANGUAGE}
%    The package \pkg{fancyhdr} treats the running head and fout lines
%    somewhat differently as the standard classes. A symptom of this is
%    that the command |\foreignlanguage| which \babel\ adds to the
%    marks can end up inside the argument of |\MakeUppercase|. To
%    prevent unexpected results we need to define |\FOREIGNLANGUAGE|
%    here.
% \changes{babel~3.7j}{2003/05/23}{Define \cs{FOREIGNLANGUAGE}
%    unconditionally}
%    \begin{macrocode}
\DeclareRobustCommand{\FOREIGNLANGUAGE}[1]{%
  \lowercase{\foreignlanguage{#1}}}
%</package>
%    \end{macrocode}
%  \end{macro}
%
%  \begin{macro}{\nfss@catcodes}
% \changes{babel~3.5g}{1996/08/18}{Need to add the double quote and
%    acute characters to \cs{nfss@catcodes} to prevent problems when
%    reading in .fd files}
%    \LaTeX's font selection scheme sometimes wants to read font
%    definition files in the middle of processing the document. In
%    order to guard against any characters having the wrong
%    |\catcode|s it always calls |\nfss@catcodes| before loading a
%    file. Unfortunately, the characters |"| and |'| are not dealt
%    with. Therefor we have to add them until \LaTeX\ does that
%    herself.
%    \begin{macrocode}
%<*core|shorthands>
\ifx\nfss@catcodes\@undefined
\else
  \addto\nfss@catcodes{%
    \@makeother\'%
    \@makeother\"%
    }
\fi
%    \end{macrocode}
%  \end{macro}
%
%    \begin{macrocode}
%</core|shorthands>
%    \end{macrocode}
%
% \section{Local Language Configuration}
%
%  \begin{macro}{\loadlocalcfg}
%    At some sites it may be necessary to add site-specific actions to
%    a language definition file. This can be done by creating a file
%    with the same name as the language definition file, but with the
%    extension \file{.cfg}. For instance the file \file{norsk.cfg}
%    will be loaded when the language definition file \file{norsk.ldf}
%    is loaded.
%
% \changes{babel~3.5d}{1995/06/22}{Added macro}
%    \begin{macrocode}
%<*core>
%    \end{macrocode}
%    For plain-based formats we don't want to override the definition
%    of |\loadlocalcfg| from \file{plain.def}.
%    \begin{macrocode}
\ifx\loadlocalcfg\@undefined
  \def\loadlocalcfg#1{%
    \InputIfFileExists{#1.cfg}
           {\typeout{*************************************^^J%
                     * Local config file #1.cfg used^^J%
                     *}%
            }
           {}}
\fi
%    \end{macrocode}
%    Just to be compatible with \LaTeX$\:$2.09 we add a few more lines
%    of code:
%    \begin{macrocode}
\ifx\@unexpandable@protect\@undefined
  \def\@unexpandable@protect{\noexpand\protect\noexpand}
  \long\def \protected@write#1#2#3{%
        \begingroup
         \let\thepage\relax
         #2%
         \let\protect\@unexpandable@protect
         \edef\reserved@a{\write#1{#3}}%
         \reserved@a
        \endgroup
        \if@nobreak\ifvmode\nobreak\fi\fi
  }
\fi
%</core>
%    \end{macrocode}
%  \end{macro}
%
%
% \clearpage
% \section{Driver files for the documented source code}
%
%    Since \babel\ version 3.4 all source files that are part of the
%    \babel\ system can be typeset separately. But to typeset
%    them all in one document, the file \file{babel.drv} can be used.
%    If you only want the information on how to use the \babel\ system
%    and what goodies are provided by the language-specific files, you
%    can run the file \file{user.drv} through \LaTeX\ to get a user
%    guide.
%
% \changes{babel~3.4b}{1994/05/18}{Use the ltxdoc class instead of
%    article}
% \changes{babel~3.7a}{1997/05/21}{Now need packages t1enc and
%    supertabular to be loaded; the documentation for icelandic needs
%    its \file{.ldf} file to be present}
% \changes{babel~3.8a}{2004/02/20}{Also load package url}
%    \begin{macrocode}
%<*driver>
\documentclass{ltxdoc}
\usepackage{url,t1enc,supertabular}
\usepackage[icelandic,english]{babel}
\DoNotIndex{\!,\',\,,\.,\-,\:,\;,\?,\/,\^,\`,\@M}
\DoNotIndex{\@,\@ne,\@m,\@afterheading,\@date,\@endpart}
\DoNotIndex{\@hangfrom,\@idxitem,\@makeschapterhead,\@mkboth}
\DoNotIndex{\@oddfoot,\@oddhead,\@restonecolfalse,\@restonecoltrue}
\DoNotIndex{\@starttoc,\@unused}
\DoNotIndex{\accent,\active}
\DoNotIndex{\addcontentsline,\advance,\Alph,\arabic}
\DoNotIndex{\baselineskip,\begin,\begingroup,\bf,\box,\c@secnumdepth}
\DoNotIndex{\catcode,\centering,\char,\chardef,\clubpenalty}
\DoNotIndex{\columnsep,\columnseprule,\crcr,\csname}
\DoNotIndex{\day,\def,\dimen,\discretionary,\divide,\dp,\do}
\DoNotIndex{\edef,\else,\@empty,\end,\endgroup,\endcsname,\endinput}
\DoNotIndex{\errhelp,\errmessage,\expandafter,\fi,\filedate}
\DoNotIndex{\fileversion,\fmtname,\fnum@figure,\fnum@table,\fontdimen}
\DoNotIndex{\gdef,\global}
\DoNotIndex{\hbox,\hidewidth,\hfil,\hskip,\hspace,\ht,\Huge,\huge}
\DoNotIndex{\ialign,\if@twocolumn,\ifcase,\ifcat,\ifhmode,\ifmmode}
\DoNotIndex{\ifnum,\ifx,\immediate,\ignorespaces,\input,\item}
\DoNotIndex{\kern}
\DoNotIndex{\labelsep,\Large,\large,\labelwidth,\lccode,\leftmargin}
\DoNotIndex{\lineskip,\leavevmode,\let,\list,\ll,\long,\lower}
\DoNotIndex{\m@ne,\mathchar,\mathaccent,\markboth,\month,\multiply}
\DoNotIndex{\newblock,\newbox,\newcount,\newdimen,\newif,\newwrite}
\DoNotIndex{\nobreak,\noexpand,\noindent,\null,\number}
\DoNotIndex{\onecolumn,\or}
\DoNotIndex{\p@,par, \parbox,\parindent,\parskip,\penalty}
\DoNotIndex{\protect,\ps@headings}
\DoNotIndex{\quotation}
\DoNotIndex{\raggedright,\raise,\refstepcounter,\relax,\rm,\setbox}
\DoNotIndex{\section,\setcounter,\settowidth,\scriptscriptstyle}
\DoNotIndex{\sfcode,\sl,\sloppy,\small,\space,\spacefactor,\strut}
\DoNotIndex{\string}
\DoNotIndex{\textwidth,\the,\thechapter,\thefigure,\thepage,\thepart}
\DoNotIndex{\thetable,\thispagestyle,\titlepage,\tracingmacros}
\DoNotIndex{\tw@,\twocolumn,\typeout,\uppercase,\usecounter}
\DoNotIndex{\vbox,\vfil,\vskip,\vspace,\vss}
\DoNotIndex{\widowpenalty,\write,\xdef,\year,\z@,\z@skip}
%    \end{macrocode}
%
%     Here |\dlqq| is defined so that  an example of |"'| can be
%     given.
%    \begin{macrocode}
\makeatletter
\gdef\dlqq{{\setbox\tw@=\hbox{,}\setbox\z@=\hbox{''}%
  \dimen\z@=\ht\z@ \advance\dimen\z@-\ht\tw@
  \setbox\z@=\hbox{\lower\dimen\z@\box\z@}\ht\z@=\ht\tw@
  \dp\z@=\dp\tw@ \box\z@\kern-.04em}}
%    \end{macrocode}
%
%    The code lines are numbered within sections,
%    \begin{macrocode}
%<*!user>
\@addtoreset{CodelineNo}{section}
\renewcommand\theCodelineNo{%
  \reset@font\scriptsize\thesection.\arabic{CodelineNo}}
%    \end{macrocode}
%    which should also be visible in the index; hence this
%    redefinition of a macro from \file{doc.sty}.
%    \begin{macrocode}
\renewcommand\codeline@wrindex[1]{\if@filesw
        \immediate\write\@indexfile
            {\string\indexentry{#1}%
            {\number\c@section.\number\c@CodelineNo}}\fi}
%    \end{macrocode}
%
%    The glossary environment is used or the change log, but its
%    definition needs changing for this document.
%    \begin{macrocode}
\renewenvironment{theglossary}{%
    \glossary@prologue%
    \GlossaryParms \let\item\@idxitem \ignorespaces}%
   {}
%</!user>
\makeatother
%    \end{macrocode}
%
%    A few shorthands used in the documentation
% \changes{babel~3.5g}{1996/07/06}{Added definition of \cs{Babel}}
%    \begin{macrocode}
\font\manual=logo10 % font used for the METAFONT logo, etc.
\newcommand*\MF{{\manual META}\-{\manual FONT}}
\newcommand*\TeXhax{\TeX hax}
\newcommand*\babel{\textsf{babel}}
\newcommand*\Babel{\textsf{Babel}}
\newcommand*\m[1]{\mbox{$\langle$\it#1\/$\rangle$}}
\newcommand*\langvar{\m{lang}}
%    \end{macrocode}
%
%     Some more definitions needed in the documentation.
%    \begin{macrocode}
%\newcommand*\note[1]{\textbf{#1}}
\newcommand*\note[1]{}
\newcommand*\bsl{\protect\bslash}
\newcommand*\Lopt[1]{\textsf{#1}}
\newcommand*\Lenv[1]{\textsf{#1}}
\newcommand*\file[1]{\texttt{#1}}
\newcommand*\cls[1]{\texttt{#1}}
\newcommand*\pkg[1]{\texttt{#1}}
\newcommand*\langdeffile[1]{%
%<-user>  \clearpage
  \DocInput{#1}}
%    \end{macrocode}
%
%    When a full index should be generated uncomment the line with
%    |\EnableCrossrefs|. Beware, processing may take some time.
%    Use |\DisableCrossrefs| when the index is ready.
%    \begin{macrocode}
%  \EnableCrossrefs
\DisableCrossrefs
%    \end{macrocode}
%
%    Inlude the change log.
%    \begin{macrocode}
%<-user>\RecordChanges
%    \end{macrocode}
%    The index should use the linenumbers of the code.
%    \begin{macrocode}
%<-user>\CodelineIndex
%    \end{macrocode}
%
% Set everything in |\MacroFont| instead of |\AltMacroFont|
%    \begin{macrocode}
\setcounter{StandardModuleDepth}{1}
%    \end{macrocode}
%
%    For the user guide we only want the description parts of all the
%    files.
%    \begin{macrocode}
%<user>\OnlyDescription
%    \end{macrocode}
%    Here starts the document
%    \begin{macrocode}
\begin{document}
\DocInput{babel.dtx}
%    \end{macrocode}
%
%    All the language definition files.
% \changes{babel~3.2e}{1992/07/07}{Added slovak}
% \changes{babel~3.3}{1993/07/11}{Added catalan and galician}
% \changes{babel~3.3}{1993/07/11}{Added turkish}
% \changes{babel~3.4}{1994/02/28}{Added bahasa}
% \changes{babel~3.5a}{1995/02/16}{Added breton, irish, scottish}
% \changes{babel~3.5b}{1995/05/19}{Added lsorbian, usorbian}
% \changes{babel~3.5c}{1995/06/14}{Changed the order of including the
%    language files somewhat (PR1652)}
% \changes{babel~3.5g}{1996/07/06}{Added greek}
% \changes{babel~3.6a}{1996/12/14}{Added welsh}
%^^A \changes{babel~3.6i}{1997/02/07}{Added sanskrit}
% \changes{babel~3.6i}{1997/02/22}{Added basque}
^^A% \changes{babel~3.6i}{1997/02/22}{Added kannada}
% \changes{babel~3.7a}{1997/05/21}{Added icelandic}
% \changes{babel~3.7b}{1998/06/25}{Added Latin}
% \changes{babel~3.7c}{1999/03/09}{Added ukrainian}
% \changes{babel~3.7c}{1999/05/09}{Added hebrew and serbian}
% \changes{babel~3.7e}{1999/11/22}{Added missing hebrew files}
% \changes{babel~3.7f}{2000/09/21}{Added bulgarian}
% \changes{babel~3.7f}{2000/09/26}{Added samin}
% \changes{babel~3.8a}{2004/02/20}{Added interlingua}
% \changes{babel~3.8h}{2005/11/23}{Added albanian and bahasam}
%    \begin{macrocode}
%<user>\clearpage
\langdeffile{esperanto.dtx}
\langdeffile{interlingua.dtx}
%
\langdeffile{dutch.dtx}
\langdeffile{english.dtx}
\langdeffile{germanb.dtx}
\langdeffile{ngermanb.dtx}
%
\langdeffile{breton.dtx}
\langdeffile{welsh.dtx}
\langdeffile{irish.dtx}
\langdeffile{scottish.dtx}
%
\langdeffile{greek.dtx}
%
\langdeffile{frenchb.dtx}
\langdeffile{italian.dtx}
\langdeffile{latin.dtx}
\langdeffile{portuges.dtx}
\langdeffile{spanish.dtx}
\langdeffile{catalan.dtx}
\langdeffile{galician.dtx}
\langdeffile{basque.dtx}
\langdeffile{romanian.dtx}
%
\langdeffile{danish.dtx}
\langdeffile{icelandic.dtx}
\langdeffile{norsk.dtx}
\langdeffile{swedish.dtx}
\langdeffile{samin.dtx}
%
\langdeffile{finnish.dtx}
\langdeffile{magyar.dtx}
\langdeffile{estonian.dtx}
%
\langdeffile{albanian.dtx}
\langdeffile{croatian.dtx}
\langdeffile{czech.dtx}
\langdeffile{polish.dtx}
\langdeffile{serbian.dtx}
\langdeffile{slovak.dtx}
\langdeffile{slovene.dtx}
\langdeffile{russianb.dtx}
\langdeffile{bulgarian.dtx}
\langdeffile{ukraineb.dtx}
%
\langdeffile{lsorbian.dtx}
\langdeffile{usorbian.dtx}
\langdeffile{turkish.dtx}
%
\langdeffile{hebrew.dtx}
\DocInput{hebinp.dtx}
\DocInput{hebrew.fdd}
\DocInput{heb209.dtx}
\langdeffile{bahasa.dtx}
\langdeffile{bahasam.dtx}
%\langdeffile{sanskrit.dtx}
%\langdeffile{kannada.dtx}
%\langdeffile{nagari.dtx}
%\langdeffile{tamil.dtx}
\clearpage
\DocInput{bbplain.dtx}
%    \end{macrocode}
%    Finally print the index and change log (not for the user guide).
%    \begin{macrocode}
%<*!user>
\clearpage
\def\filename{index}
\PrintIndex
\clearpage
\def\filename{changes}
\PrintChanges
%</!user>
\end{document}
%</driver>
%    \end{macrocode}
%
% \Finale
%
%%
%% \CharacterTable
%%  {Upper-case    \A\B\C\D\E\F\G\H\I\J\K\L\M\N\O\P\Q\R\S\T\U\V\W\X\Y\Z
%%   Lower-case    \a\b\c\d\e\f\g\h\i\j\k\l\m\n\o\p\q\r\s\t\u\v\w\x\y\z
%%   Digits        \0\1\2\3\4\5\6\7\8\9
%%   Exclamation   \!     Double quote  \"     Hash (number) \#
%%   Dollar        \$     Percent       \%     Ampersand     \&
%%   Acute accent  \'     Left paren    \(     Right paren   \)
%%   Asterisk      \*     Plus          \+     Comma         \,
%%   Minus         \-     Point         \.     Solidus       \/
%%   Colon         \:     Semicolon     \;     Less than     \<
%%   Equals        \=     Greater than  \>     Question mark \?
%%   Commercial at \@     Left bracket  \[     Backslash     \\
%%   Right bracket \]     Circumflex    \^     Underscore    \_
%%   Grave accent  \`     Left brace    \{     Vertical bar  \|
%%   Right brace   \}     Tilde         \~}
\endinput


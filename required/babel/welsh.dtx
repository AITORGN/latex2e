% \iffalse meta-comment
%
% Copyright 1989-2005 Johannes L. Braams and any individual authors
% listed elsewhere in this file.  All rights reserved.
% 
% This file is part of the Babel system.
% --------------------------------------
% 
% It may be distributed and/or modified under the
% conditions of the LaTeX Project Public License, either version 1.3
% of this license or (at your option) any later version.
% The latest version of this license is in
%   http://www.latex-project.org/lppl.txt
% and version 1.3 or later is part of all distributions of LaTeX
% version 2003/12/01 or later.
% 
% This work has the LPPL maintenance status "maintained".
% 
% The Current Maintainer of this work is Johannes Braams.
% 
% The list of all files belonging to the Babel system is
% given in the file `manifest.bbl. See also `legal.bbl' for additional
% information.
% 
% The list of derived (unpacked) files belonging to the distribution
% and covered by LPPL is defined by the unpacking scripts (with
% extension .ins) which are part of the distribution.
% \fi
% \CheckSum{97}
%
% \iffalse
%    Tell the \LaTeX\ system who we are and write an entry on the
%    transcript.
%<*dtx>
\ProvidesFile{welsh.dtx}
%</dtx>
%<code>\ProvidesLanguage{welsh}
%\fi
%\ProvidesFile{welsh.dtx}
        [2005/03/31 v1.0d Welsh support from the babel system]
%\iffalse
%% File `welsh.dtx'
%% Babel package for LaTeX version 2e
%% Copyright (C) 1989 -- 2005
%%           by Johannes Braams, TeXniek
%
%% Please report errors to: J.L. Braams
%%                          babel at braams.cistron.nl
%
%    This file is part of the babel system, it provides the source
%    code for the Welsh language definition file.
%<*filedriver>
\documentclass{ltxdoc}
\newcommand*{\TeXhax}{\TeX hax}
\newcommand*{\babel}{\textsf{babel}}
\newcommand*{\langvar}{$\langle \mathit lang \rangle$}
\newcommand*{\note}[1]{}
\newcommand*{\Lopt}[1]{\textsf{#1}}
\newcommand*{\file}[1]{\texttt{#1}}
\begin{document}
 \DocInput{welsh.dtx}
\end{document}
%</filedriver>
%\fi
% \GetFileInfo{welsh.dtx}
%
%  \section{The Welsh language}
%
%    The file \file{\filename}\footnote{The file described in this
%    section has version number \fileversion\ and was last revised on
%    \filedate.}  defines all the language definition macros for the
%    Welsh language.
%
%    For this language currently no special definitions are needed or
%    available.
%
% \StopEventually{}
%
%    The macro |\ldf@init| takes care of preventing that this file is
%    loaded more than once, checking the category code of the
%    \texttt{@} sign, etc.
%    \begin{macrocode}
%<*code>
\LdfInit{welsh}{captionswelsh}
%    \end{macrocode}
%
%    When this file is read as an option, i.e. by the |\usepackage|
%    command, \texttt{welsh} could be an `unknown' language in
%    which case we have to make it known.  So we check for the
%    existence of |\l@welsh| to see whether we have to do
%    something here.
%
%    \begin{macrocode}
\ifx\undefined\l@welsh
  \@nopatterns{welsh}
  \adddialect\l@welsh0\fi
%    \end{macrocode}
%    The next step consists of defining commands to switch to (and
%    from) the Welsh language.
%
%  \begin{macro}{\welshhyphenmins}
%    This macro is used to store the correct values of the hyphenation
%    parameters |\lefthyphenmin| and |\righthyphenmin|.
% \changes{welsh-1.0c}{2000/09/22}{Now use \cs{providehyphenmins} to
%    provide a default value}
%    \begin{macrocode}
\providehyphenmins{\CurrentOption}{\tw@\thr@@}
%    \end{macrocode}
%  \end{macro}
%
% \begin{macro}{\captionswelsh}
%    The macro |\captionswelsh| defines all strings used in the
%    four standard documentclasses provided with \LaTeX.
% \changes{welsh-1.0c}{2000/09/20}{Added \cs{glossaryname}}
% \changes{welsh-1.0d}{2005/03/31}{Provided the translation for
%    Glossary} 
%    \begin{macrocode}
\def\captionswelsh{%
  \def\prefacename{Rhagair}%
  \def\refname{Cyfeiriadau}%   
  \def\abstractname{Crynodeb}% 
  \def\bibname{Llyfryddiaeth}%
  \def\chaptername{Pennod}%
  \def\appendixname{Atodiad}%
  \def\contentsname{Cynnwys}%
  \def\listfigurename{Rhestr Ddarluniau}%
  \def\listtablename{Rhestr Dablau}%
  \def\indexname{Mynegai}%
  \def\figurename{Darlun}%
  \def\tablename{Taflen}%
  \def\partname{Rhan}%
  \def\enclname{amgae\"edig}%
  \def\ccname{cop\"\i au}%
  \def\headtoname{At}%  % `at' on letters meaning `to ( a person)'
                        % `to (a place)' is `i' in Welsh
  \def\pagename{tudalen}%
  \def\seename{gweler}%
  \def\alsoname{gweler hefyd}%
  \def\proofname{Prawf}%
  \def\glossaryname{Rhestr termau}%
  }
%    \end{macrocode}
% \end{macro}
%
% \begin{macro}{\datewelsh}
%    The macro |\datewelsh| redefines the command |\today| to
%    produce welsh dates.
% \changes{welsh-1.0b}{1997/10/01}{Use \cs{edef} to define
%    \cs{today} to save memory}
% \changes{welsh-1.0b}{1998/03/28}{use \cs{def} instead of
%    \cs{edef}}
% \changes{welsh-1.0d}{2005/03/31}{removed `a viz' from the definition
%    of \cs{today}} 
%    \begin{macrocode}
\def\datewelsh{%
  \def\today{\ifnum\day=1\relax 1\/$^{\mathrm{a\tilde{n}}}$\else
    \number\day\fi\space\ifcase\month\or
    Ionawr\or Chwefror\or Mawrth\or Ebrill\or
    Mai\or Mehefin\or Gorffennaf\or Awst\or
    Medi\or Hydref\or Tachwedd\or Rhagfyr\fi
  \space\number\year}}
%    \end{macrocode}
% \end{macro}
%
% \begin{macro}{\extraswelsh}
% \begin{macro}{\noextraswelsh}
%    The macro |\extraswelsh| will perform all the extra
%    definitions needed for the welsh language. The macro
%    |\noextraswelsh| is used to cancel the actions of
%    |\extraswelsh|.  For the moment these macros are empty but
%    they are defined for compatibility with the other
%    language definition files.
%
%    \begin{macrocode}
\addto\extraswelsh{}
\addto\noextraswelsh{}
%    \end{macrocode}
% \end{macro}
% \end{macro}
%
%    The macro |\ldf@finish| takes care of looking for a
%    configuration file, setting the main language to be switched on
%    at |\begin{document}| and resetting the category code of
%    \texttt{@} to its original value.
%    \begin{macrocode}
\ldf@finish{welsh}
%</code>
%    \end{macrocode}
%
% \Finale
%\endinput
%% \CharacterTable
%%  {Upper-case    \A\B\C\D\E\F\G\H\I\J\K\L\M\N\O\P\Q\R\S\T\U\V\W\X\Y\Z
%%   Lower-case    \a\b\c\d\e\f\g\h\i\j\k\l\m\n\o\p\q\r\s\t\u\v\w\x\y\z
%%   Digits        \0\1\2\3\4\5\6\7\8\9
%%   Exclamation   \!     Double quote  \"     Hash (number) \#
%%   Dollar        \$     Percent       \%     Ampersand     \&
%%   Acute accent  \'     Left paren    \(     Right paren   \)
%%   Asterisk      \*     Plus          \+     Comma         \,
%%   Minus         \-     Point         \.     Solidus       \/
%%   Colon         \:     Semicolon     \;     Less than     \<
%%   Equals        \=     Greater than  \>     Question mark \?
%%   Commercial at \@     Left bracket  \[     Backslash     \\
%%   Right bracket \]     Circumflex    \^     Underscore    \_
%%   Grave accent  \`     Left brace    \{     Vertical bar  \|
%%   Right brace   \}     Tilde         \~}
%%

% \iffalse meta-comment
%
% Copyright 1993 1994 1995 1996 1997 1998 1999 2000 2001 2002 2003 2004 2005 2006 2007 2008 2009
% The LaTeX3 Project and any individual authors listed elsewhere
% in this file. 
% 
% This file is part of the LaTeX base system.
% -------------------------------------------
% 
% It may be distributed and/or modified under the
% conditions of the LaTeX Project Public License, either version 1.3c
% of this license or (at your option) any later version.
% The latest version of this license is in
%    http://www.latex-project.org/lppl.txt
% and version 1.3c or later is part of all distributions of LaTeX 
% version 2005/12/01 or later.
% 
% This file has the LPPL maintenance status "maintained".
% 
% The list of all files belonging to the LaTeX base distribution is
% given in the file `manifest.txt'. See also `legal.txt' for additional
% information.
% 
% The list of derived (unpacked) files belonging to the distribution 
% and covered by LPPL is defined by the unpacking scripts (with 
% extension .ins) which are part of the distribution.
% 
% \fi
% Filename: ltnews13.tex 
%
% This is issue 13 of LaTeX News.

\documentclass
%    [lw35fonts]     % uncomment this line to get Palatino
   {ltnews}[2000/07/21]

% \usepackage[T1]{fontenc}

\publicationmonth{June}  
\publicationyear{2000}
\publicationissue{13}


\begin{document}

\maketitle

\raisefirstsection 
\section{Yearly release cycle}

We announced in \textit{\LaTeX{} News~11} that we intended to switch
to a 12-monthly release schedule.  With the present (June~2000)
release, this switch is being made: thus the next release of \LaTeX{}
will be dated June~2001.  We shall of course continue, as in the past,
to release patches as needed to fix significant bugs.


\section{PSNFSS: \small Quote of the Month}

\begin{quote}
  You should say in the \LaTeX{} News that Walter Schmidt has taken over
  \PSNFSS{} from me. It gives me a certain pleasure to be able to draw a
  line under that part of my life\ldots

  \begin{latexonly}
    \vspace{-\baselineskip}
  \end{latexonly}

  \begin{flushright}
    Sebastian Rahtz
  \end{flushright}
\end{quote}

\begin{latexonly}
  \vspace{-\baselineskip} 
\end{latexonly}

\noindent
The \PSNFSS{} material, which supports the use\latex{\\}
of common PostScript fonts with \LaTeX{}, has been thoroughly updated.
Most noticeably, the \package{mathpple} package, which used to be
distributed separately, is now part of the basic \PSNFSS{} bundle;
this package provides mathematical typesetting with the Palatino
typeface family.  In addition, numerous bugs and flaws have been fixed
and the distribution has been `cleaned up'.  The file
\file{changes.txt} contains a detailed list of these changes.
\latex{\looseness=-1}   %%!!!!! It worked!

The documentation (in \file{psnfss2e.pdf}) has been completely
rewritten to provide a comprehensive introduction to the use
of PostScript fonts. 

Notice that the new \PSNFSS{} needs updated files for font metrics,
virtual fonts and font definitions.  If you received the new
version~(8.1) as part of a complete \TeX{} system then these new font
files should also have been installed.  However, if you intend to
install or update \PSNFSS{} yourself, please read the instructions in
the file \file{00readme.txt} of the new \PSNFSS{} distribution.

Support for commercial PostScript fonts, such as
Lucida~Bright, has been removed from the basic distribution;
it is now available from \ctan{}:
\begin{latexonly}
  \file{\ctanhttp macros/latex/\\
    contrib/supported/psnfssx}.
\end{latexonly}
\begin{htmlonly}
  \url{http://www.tex.ac.uk/tex-archive/macros/latex/contrib/supported/psnfssx}.
\end{htmlonly}

\section{New AMS-\LaTeX{}}

Version 2.0 of AMS-\LaTeX{} was released on December 1, 1999. It can
be obtained via \url{ftp://ftp.ams.org/pub/tex/} or
\url{http://www.ams.org/tex/amslatex.html}, as well from \ctan{}:
\begin{latexonly}
  \file{\ctanhttp macros/latex/\\
    required/amslatex}.
\end{latexonly}
\begin{htmlonly}
  \url{http://www.tex.ac.uk/tex-archive/macros/latex/required/amslatex}.
\end{htmlonly}

This release consists chiefly of bug fixes and consolidation of the
existing features. The division of AMS-\LaTeX{} into two main parts
(the math packages;\latex{\\}
the AMS document classes) has been made more pronounced.
The files \file{diffs-m.txt}, \file{diffs-c.txt},
\file{amsmath.faq}, and \file{amsclass.faq} describe the\latex{\\}
changes and address some common questions.

The primary documentation files remain \file{amsldoc.tex}, for the
\package{amsmath} package, and \file{instr-l.tex}, for the AMS
document classes.\latex{\\}
The documentation for the \package{amsthm} package,\latex{\\}
however, has been moved from \file{amsldoc.tex}\latex{\\}
to a separate document \file{amsthdoc.tex}.

\begin{latexonly}
  \vfill
\end{latexonly}

\section{New input encoding \package{latin4}}

The package \package{inputenc} has, thanks to Hana Skoumalov\'a,
been extended to cover the \package{latin4} input encoding; this
covers Baltic and Scandinavian languages as well as Greenland
Inuit and Lappish.

\begin{latexonly}
  \vfill
\end{latexonly}

\section{New experimental code}

In \textit{\LaTeX{} News~12} we announced some ongoing work towards a
`Designer Interface for \LaTeX' and we presented some early results
thereof.  Since then, at Gutenberg\,2000 in Toulouse and TUG\,2000 in
Oxford, we described a new output routine and an improved method of
handling vertical mode material between paragraphs.  In combination
these support higher quality \emph{automated}\footnote
                             {The stress here is on automated!}
page-breaking and page make-up\latex{\\}
for complex pages---the best yet achieved with \TeX{}!

A paper describing the new output routine is 
\begin{latexonly}
  at\\
  \file{http://www.latex-project.org/papers/xo-pfloat.pdf%
   \hspace*{-16pt}}\\
\end{latexonly}
\begin{htmlonly}
  at \url{http://www.latex-project.org/papers/xo-pfloat.pdf}. 
\end{htmlonly}
All code examples and documentation are available 
\begin{latexonly}
  at\\
   \file{http://www.latex-project.org/code/experimental/%
   \latex{\hspace*{-5pt}}}.\\[3pt]
\end{latexonly}
\begin{htmlonly}
  at \url{http://www.latex-project.org/code/experimental/}.

\end{htmlonly}
This directory has been extended to contain
\begin{description}
 \item[galley] Prototype implementation of the interface\latex{\\}
   for manipulating vertical material in galleys.
 \item[xinitials] Prototype implementation of the interface\latex{\\}
   for paragraph initials (needs the \texttt{galley} package.
 \item[xtheorem] Contributed example using the \texttt{template}
   package to provide a designer interface for theorem environments.
 \item[xoutput] A prototype implementation of the new output routine
   as described in the \texttt{xo-pfloat.pdf} paper. Expected
   availability: at or shortly after\latex{\\}
   the TUG\,2000 conference.
\end{description}

\end{document}



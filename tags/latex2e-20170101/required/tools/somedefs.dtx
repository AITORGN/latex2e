% \iffalse meta-comment
%
% Copyright 1993 1994 1995 1996 1997 1998 1999 2000 2001 2002 2003 2004 2005
% 2006 2008 2009
% The LaTeX3 Project and any individual authors listed elsewhere
% in this file.
%
% This file is part of the Standard LaTeX `Tools Bundle'.
% -------------------------------------------------------
%
% It may be distributed and/or modified under the
% conditions of the LaTeX Project Public License, either version 1.3c
% of this license or (at your option) any later version.
% The latest version of this license is in
%    http://www.latex-project.org/lppl.txt
% and version 1.3c or later is part of all distributions of LaTeX
% version 2005/12/01 or later.
%
% The list of all files belonging to the LaTeX `Tools Bundle' is
% given in the file `manifest.txt'.
%
% \fi
% \title{The \textsf{somedefs} toolkit package}
% \date{long time ago in a different century\ldots}
% \author{Alan Jeffrey}
% \MaintainedByLaTeXTeam{tools}
% \maketitle
%
%
% \changes{v0.03}{1994/06/01}{Use new style error commands. DPC}
% \changes{v0.02}{1994/05/10}{Added a \cmd\relax, to stop arguments to
%    \cmd\newcommand\ being unbraced by \cmd\ProvidesCommand.  Added
%    an error message for commands which are requested but never
%    defined.  Spotted by DPC.}
%
% \section*{Overview}
%
% This is an example `programmers toolkit' package, for use by package
% writers.  It allows package writers to provide options which switch
% definitions on and off.  For example, a package |fred| might define a
% large number of commands, including |\foo| and |\baz|, so:
% \begin{verbatim}
%    \usepackage{fred}
% \end{verbatim}
% would use a lot of memory, even if |\foo| and |\baz| were the only
% commands needed.  However, if the author of |fred| used the |somedefs|
% package, then the user would be able to say:
% \begin{verbatim}
%    \usepackage[only,foo,baz]{fred}
% \end{verbatim}
% and only the commands |\foo| and |\baz| would be defined.
%
% To use the |somedefs| package in your own packages or classes, you
%  say:
% \begin{verbatim}
%    \RequirePackage{somedefs}
% \end{verbatim}
% You can then use four new commands:
% \begin{flushleft}\begin{itemize}
% \item |\UseAllDefinitions| which says that all the commands in the
%    file should be defined.
% \item |\UseSomeDefinitions| which says that only the commands
%    specified by |\UseDefinition| should be defined.
% \item |\UseDefinition{|\meta{name}|}| which says that the command
%    |\name| should be defined.
% \item |\ProvidesDefinition{|\meta{definition}|}| which provides one
%    definition, of the form |\definingcommand{\command}...|
% \end{itemize}\end{flushleft}
% For example, the package |fred| could say:
% \begin{verbatim}
%    \RequirePackage{somedefs}
%    \UseAllDefinitions
%    \DeclareOption{only}{\UseSomeDefinitions}
%    \DeclareOption*{\UseDefinition{\CurrentOption}}
%    \ProcessOptions
%    \ProvidesDefinition{\newcommand{\foo}{...}}
%    \ProvidesDefinition{\newcommand{\baz}{...}}
% \end{verbatim}
% One of the commands |\UseAllDefinitions| or |\UseSomeDefinitions|
% should always be used.  You may have some commands which need other
% commands, in which case you have to declare the options by hand.  For
% example, if the command |\bar| needs the command |\foo|, you could
% say:
% \begin{verbatim}
%    \DeclareOption{bar}{\UseDefinition{bar}\UseDefinition{foo}}
% \end{verbatim}
% For a longer example of the use of the |somedefs| package, look at the
% |rawfonts| package.
%
% \StopEventually{}
%
% \section*{Implementation}
%
% The driver for the documentation you're now reading.
%    \begin{macrocode}
%<*driver>
\documentclass{ltxdoc}
\begin{document}
\DocInput{somedefs.dtx}
\end{document}
%</driver>
%    \end{macrocode}
% This is a \LaTeXe{} package.
%    \begin{macrocode}
%<*package>
\NeedsTeXFormat{LaTeX2e}
\ProvidesPackage{somedefs}[1994/06/01 v0.03 Toolkit for optional definitions]
%    \end{macrocode}
% \begin{macro}{\UseSomeDefinitions}
% \begin{macro}{\UseAllDefinitions}
% \begin{macro}{\UseDefinition}
% \begin{macro}{\ProvidesDefinition}
% \begin{macro}{\@providesdefinition}
% \begin{macro}{\@provides@definition}
% \begin{macro}{\@unprovided@definition}
%    The package works by having |\UseDefinition{|\meta{name}|}| define
%    |\name| to be |\@unprovided@definition|.
%    If |\UseSomeDefinitions| has been called,
%    then |\ProvidesDefinition| looks to see if |\name|
%    is |\@unprovided@definition|.  If
%    |\UseAllDefinitions| has been called, then |\ProvidesDefinition|
%    does nothing.  If neither has been called, then
%    |\ProvidesDefinition| produces an error message.
%    \begin{macrocode}
\def\UseSomeDefinitions{%
   \let\ProvidesDefinition\@providesdefinition
}
\def\UseAllDefinitions{%
   \let\ProvidesDefinition\@firstofone
}
\def\UseDefinition#1{%
   \expandafter\let\csname#1\endcsname\@unprovided@definition
}
\def\ProvidesDefinition#1{%
   \PackageError{somedefs}%
     {No \noexpand\UseSomeDefinitions or \string\UseAllDefinitions}%
     {The package which used the `somedefs' package has an error.}%
}
\def\@providesdefinition#1{\@provides@definition#1\relax
   \@provides@definition}
\def\@provides@definition#1#2#3\@provides@definition{%
   \ifx#2\@unprovided@definition
      #1#2#3%
   \fi
}
\def\@unprovided@definition{%
   \PackageError{somedefs}%
     {Package `somedefs' error: this command was never defined}%
     {You have requested a command which does not exist.}%
}
\@onlypreamble\UseSomeDefinitions
\@onlypreamble\UseAllDefinitions
\@onlypreamble\UseDefinition
\@onlypreamble\ProvidesDefinition
\@onlypreamble\@providesdefinition
\@onlypreamble\@provides@definition
%    \end{macrocode}
% \end{macro}
% \end{macro}
% \end{macro}
% \end{macro}
% \end{macro}
% \end{macro}
% \end{macro}
% That's it!
%    \begin{macrocode}
%</package>
%    \end{macrocode}
%
% \Finale
%
% \endinput

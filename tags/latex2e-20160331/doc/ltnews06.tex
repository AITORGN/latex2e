% \iffalse meta-comment
%
% Copyright 1993 1994 1995 1996 1997 1998 1999 2000 2001 2002 2003 2004 2005 2006 2007 2008 2009
% The LaTeX3 Project and any individual authors listed elsewhere
% in this file. 
% 
% This file is part of the LaTeX base system.
% -------------------------------------------
% 
% It may be distributed and/or modified under the
% conditions of the LaTeX Project Public License, either version 1.3c
% of this license or (at your option) any later version.
% The latest version of this license is in
%    http://www.latex-project.org/lppl.txt
% and version 1.3c or later is part of all distributions of LaTeX 
% version 2005/12/01 or later.
% 
% This file has the LPPL maintenance status "maintained".
% 
% The list of all files belonging to the LaTeX base distribution is
% given in the file `manifest.txt'. See also `legal.txt' for additional
% information.
% 
% The list of derived (unpacked) files belonging to the distribution 
% and covered by LPPL is defined by the unpacking scripts (with 
% extension .ins) which are part of the distribution.
% 
% \fi
% Filename: ltnews06.tex

% This is issue 6 of LaTeX News.

\documentclass
%    [lw35fonts]
   {ltnews}

% \usepackage[T1]{fontenc}

\publicationmonth{December}
\publicationyear{1996}
\publicationissue{6}

\begin{document}

\maketitle

\section{Welcome to \LaTeXNews~6}
This issue of \emph{\LaTeXNews} accompanies the sixth release of the
new standard \LaTeX{}, \LaTeXe.

\section{Mono-case file names}
Previously \LaTeX\ has used some files with `mixed-case' file names
such as \texttt{T1cmr.fd} and \texttt{T1enc.def}.

These file names cause problems on some systems (in particular they
are illegal on the ISO 9660 CDROM format) and so in this release
all file names have been made lowercase (for example 
\texttt{t1cmr.fd} and \texttt{t1enc.def}).

This change should \emph{not} affect any document. Within \LaTeX,
encodings still have the usual uppercase names in uses such as
\verb|\usepackage[T1]{fontenc}| and \verb|\fontencoding{T1}|.
\LaTeX\ will automatically convert to the lowercase form while
constructing the file name.
\LaTeX\ will input the `fd' file under the old name if it fails to
find the file with the new name, so existing collections of fd files
should still work with this new release.

The change \emph{does} affect the configuration files that may be used
to make the \LaTeX\ format with initex. For example, the file
\texttt{fonttext.ltx} previously specified \verb|\input{T1cmr.fd}|.
It now has \verb|\input{t1cmr.fd}|. 
If you use a local file \texttt{fonttext.cfg}
you will need to make similar changes, as \verb|\input{T1cmr.fd}|
will not work as \texttt{T1cmr.fd} is no longer in the distribution.

The files affected by this change all have names of the form
\verb|*.fd| or \verb|*enc.def|.

\section{Another input encoding}
Thanks to work by S\o ren Sandmann, the \textsf{inputenc} package now
supports the IBM codepage~865 used in Scandinavia.

\section{Better user-defined math display environments}
Suppose that you want to define an environment for displaying text
that is numbered as an equation.  A straightforward way to do this is
as follows:
\begin{verbatim}
  \newenvironment{texteqn}
    {\begin{equation}
       \begin{minipage}{0.9\linewidth}}
      {\end{minipage}
     \end{equation}}
\end{verbatim}
However, if you have tried this then you will probably have noticed
that it does not work perfectly when used in the middle of a paragraph
because an inter-word space appears at the beginning of the first
line after the environment.

There is now an extra command (with a very long name) available that
you can use to avoid this problem; it should be inserted as shown here:
\begin{verbatim}
  \newenvironment{texteqn}
    {\begin{equation}
       \begin{minipage}{0.9\linewidth}}
      {\end{minipage}
     \end{equation}
     \ignorespacesafterend}
\end{verbatim}

\section{Docstrip improvements}
The \textsf{docstrip} program that is used to unpack the \LaTeX\ 
sources has undergone further development. The new version should be
able to process all old `batchfiles' but it allows a simpler syntax in
new `batchfiles' (no need to define \verb|\def\batchfile{|\ldots).

It also allows `target' directories to be specified when writing
files.  This directory support is disabled by default unless activated
in a local \texttt{docstrip.cfg} configuration file.
See \texttt{docstrip.dtx} for details.

\section{AMS \LaTeX\ update}
Since the last \LaTeX\ release in June, the American Mathematical
Society have re-issued the `AMS\LaTeX' classes and packages, fixing
several reported problems.

\section{Graphics package update}
The \LaTeX\ color and graphics packages have been updated slightly,
principally to support more dvi drivers, see the readme file in the
\textsf{graphics} distribution.

\section{EC Fonts released}
The first release of the Extended Computer Modern fonts has just been
made. (In January 1997.)

This release of \LaTeX\ does \emph{not} default to these `ec' fonts
as its T1 encoded fonts. By default it will
use the `dc' fonts if the T1 encoding is requested.

As noted in \texttt{install.txt} you may run \TeX\ on the install file
\texttt{ec.ins} \emph{after} unpacking the base distribution but
\emph{before} making the \LaTeX\ format. This will produce
suitable `fd' files making \LaTeX\ (including, for the first time, the
\textsf{slides} class) use the `ec' fonts as the default T1 encoded
font set.


\end{document}


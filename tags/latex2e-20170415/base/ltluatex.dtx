% \iffalse meta-comment
%
% Copyright 2015 2017
% The LaTeX3 Project and any individual authors listed elsewhere
% in this file.
%
% It may be distributed and/or modified under the conditions of
% the LaTeX Project Public License (LPPL), either version 1.3c of
% this license or (at your option) any later version.  The latest
% version of this license is in the file:
%
%   http://www.latex-project.org/lppl.txt
%
%
%
%<2ekernel>%%% From File: ltluatex.dtx
%<plain>\ifx\newluafunction\undefined\else\expandafter\endinput\fi
%<tex>\ifx
%<tex>  \ProvidesFile\undefined\begingroup\def\ProvidesFile
%<tex>  #1#2[#3]{\endgroup\immediate\write-1{File: #1 #3}}
%<tex>\fi
%<plain>\ProvidesFile{ltluatex.tex}
%<*driver>
\ProvidesFile{ltluatex.dtx}
%</driver>
%<*tex>
[2017/04/28 v1.1f
%</tex>
%<plain>  LuaTeX support for plain TeX (core)
%<*tex>
]
\edef\etatcatcode{\the\catcode`\@}
\catcode`\@=11
%</tex>
%<*driver>
\documentclass{ltxdoc}
\GetFileInfo{ltluatex.dtx}
\begin{document}
\title{\filename\\(Lua\TeX{}-specific support)}
\author{David Carlisle and Joseph Wright\footnote{Significant portions
  of the code here are adapted/simplified from the packages \textsf{luatex} and
  \textsf{luatexbase} written by Heiko Oberdiek, \'{E}lie Roux,
  Manuel P\'{e}gouri\'{e}-Gonnar and Philipp Gesang.}}
\date{\filedate}
\maketitle
\setcounter{tocdepth}{2}
\tableofcontents
\DocInput{\filename}
\end{document}
%</driver>
% \fi
%
%
% \section{Overview}
%
% Lua\TeX{} adds a number of engine-specific functions to \TeX{}. Several of
% these require set up that is best done in the kernel or need related support
% functions. This file provides \emph{basic} support for Lua\TeX{} at the
% \LaTeXe{} kernel level plus as a loadable file which can be used with
% plain \TeX{} and \LaTeX{}.
%
% This file contains code for both \TeX{} (to be stored as part of the format)
% and Lua (to be loaded at the start of each job). In the Lua code, the kernel
% uses the namespace |luatexbase|.
%
% The following |\count| registers are used here for register allocation:
% \begin{itemize}
%  \item[\texttt{\string\e@alloc@attribute@count}] Attributes (default~258)
%  \item[\texttt{\string\e@alloc@ccodetable@count}] Category code tables
%    (default~259)
%  \item[\texttt{\string\e@alloc@luafunction@count}] Lua functions
%    (default~260)
%  \item[\texttt{\string\e@alloc@whatsit@count}] User whatsits (default~261)
%  \item[\texttt{\string\e@alloc@bytecode@count}] Lua bytecodes (default~262)
%  \item[\texttt{\string\e@alloc@luachunk@count}] Lua chunks (default~263)
% \end{itemize}
% (|\count 256| is used for |\newmarks| allocation and |\count 257|
% is used for\linebreak
% |\newXeTeXintercharclass| with Xe\TeX{}, with code defined in
% \texttt{ltfinal.dtx}).
% With any \LaTeXe{} kernel from 2015 onward these registers are part of
% the block in the extended area reserved by the kernel (prior to 2015 the
% \LaTeXe{} kernel did not provide any functionality for the extended
% allocation area).
%
% \section{Core \TeX{} functionality}
%
% The commands defined here are defined for
% possible inclusion in a future \LaTeX{} format, however also extracted
% to the file |ltluatex.tex| which may be used with older \LaTeX\
% formats, and with plain \TeX.
%
% \noindent
% \DescribeMacro{\newattribute}
% |\newattribute{|\meta{attribute}|}|\\
% Defines a named \cs{attribute}, indexed from~$1$
% (\emph{i.e.}~|\attribute0| is never defined). Attributes initially
% have the marker value |-"7FFFFFFF| (`unset') set by the engine.
%
% \noindent
% \DescribeMacro{\newcatcodetable}
% |\newcatcodetable{|\meta{catcodetable}|}|\\
% Defines a named \cs{catcodetable}, indexed from~$1$
% (|\catcodetable0| is never assigned). A new catcode table will be
% populated with exactly those values assigned by Ini\TeX{} (as described
% in the Lua\TeX{} manual).
%
% \noindent
% \DescribeMacro{\newluafunction}
% |\newluafunction{|\meta{function}|}|\\
% Defines a named \cs{luafunction}, indexed from~$1$. (Lua indexes
% tables from $1$ so |\luafunction0| is not available).
%
% \noindent
% \DescribeMacro{\newwhatsit}
% |\newwhatsit{|\meta{whatsit}|}|\\
% Defines a custom \cs{whatsit}, indexed from~$1$.
%
% \noindent
% \DescribeMacro{\newluabytecode}
% |\newluabytecode{|\meta{bytecode}|}|\\
% Allocates a number for Lua bytecode register, indexed from~$1$.
%
% \noindent
% \DescribeMacro{\newluachunkname}
% |newluachunkname{|\meta{chunkname}|}|\\
% Allocates a number for Lua chunk register, indexed from~$1$.
% Also enters the name of the regiser (without backslash) into the
% \verb|lua.name| table to be used in stack traces.
%
% \noindent
% \DescribeMacro{\catcodetable@initex}
% \DescribeMacro{\catcodetable@string}
% \DescribeMacro{\catcodetable@latex}
% \DescribeMacro{\catcodetable@atletter}
% Predefined category code tables with the obvious assignments. Note
% that the |latex| and |atletter| tables set the full Unicode range
% to the codes predefined by the kernel.
%
% \noindent
% \DescribeMacro{\setattribute}
% \DescribeMacro{\unsetattribute}
% |\setattribute{|\meta{attribute}|}{|\meta{value}|}|\\
% |\unsetattribute{|\meta{attribute}|}|\\
% Set and unset attributes in a manner analogous to |\setlength|. Note that
% attributes take a marker value when unset so this operation is distinct
% from setting the value to zero.
%
% \section{Plain \TeX\ interface}
%
% The \textsf{ltluatex} interface may be used with plain \TeX\ using 
% |% \iffalse meta-comment
%
% Copyright 2015
% The LaTeX3 Project and any individual authors listed elsewhere
% in this file.
%
% It may be distributed and/or modified under the conditions of
% the LaTeX Project Public License (LPPL), either version 1.3c of
% this license or (at your option) any later version.  The latest
% version of this license is in the file:
%
%   http://www.latex-project.org/lppl.txt
%
%
%
%<2ekernel>%%% From File: ltluatex.dtx
%<plain>\ifx\newluafunction\undefined\else\expandafter\endinput\fi
%<tex>\ifx
%<tex>  \ProvidesFile\undefined\begingroup\def\ProvidesFile
%<tex>  #1#2[#3]{\endgroup\immediate\write-1{File: #1 #3}}
%<tex>\fi
%<plain>\ProvidesFile{ltluatex.tex}
%<*driver>
\ProvidesFile{ltluatex.dtx}
%</driver>
%<*tex>
[2015/08/01 v1.0a
%</tex>
%<plain>  LuaTeX support for plain TeX (core)
%<*tex>
]
\edef\etatcatcode{\the\catcode`\@}
\catcode`\@=11
%</tex>
%<*driver>
\documentclass{ltxdoc}
\GetFileInfo{ltluatex.dtx}
\begin{document}
\title{\filename\\(Lua\TeX{}-specific support)}
\author{David Carlisle and Joseph Wright\footnote{Significant portions
  of the code here are adapted/simplified from the packages \textsf{luatex} and
  \textsf{luatexbase} written by Heiko Oberdiek, \'{E}lie Roux,
  Manuel P\'{e}gouri\'{e}-Gonnar and Philipp Gesang.}}
\date{\filedate}
\maketitle
\setcounter{tocdepth}{2}
\tableofcontents
\DocInput{\filename}
\end{document}
%</driver>
% \fi
%
%
% \section{Overview}
%
% Lua\TeX{} adds a number of engine-specific functions to \TeX{}. Several of
% these require set up that is best done in the kernel or need related support
% functions. This file provides \emph{basic} support for Lua\TeX{} at the
% \LaTeXe{} kernel level plus as a loadable file which can be used with
% plain \TeX{} and \LaTeX{}.
%
% This file contains code for both \TeX{} (to be stored as part of the format)
% and Lua (to be loaded at the start of each job). In the Lua code, the kernel
% uses the namespace |luatexbase|.
%
% The following |\count| registers are used here for register allocation:
% \begin{itemize}
%  \item[\texttt{\string\e@alloc@attribute@count}] Attributes (default~258)
%  \item[\texttt{\string\e@alloc@ccodetable@count}] Category code tables
%    (default~259)
%  \item[\texttt{\string\e@alloc@luafunction@count}] Lua functions
%    (default~260)
%  \item[\texttt{\string\e@alloc@whatsit@count}] User whatsits (default~261)
% \end{itemize}
% (|\count256| is used for |\newmarks| allocation and |\count257|
% is used for\break
% |\newXeTeXintercharclass| with Xe\TeX{}, with code defined in
% \texttt{ltfinal.dtx}).
% With any \LaTeXe{} kernel from 2015 onward these registers are part of
% the block in the extended area reserved by the kernel (prior to 2015 the
% \LaTeXe{} kernel did not provide any functionality for the extended
% allocation area).
%
% \section{Core \TeX{} functionality}
%
% The commands defined here are defined for
% possible inclusion in a future \LaTeX{} format, however also extracted
% to the file |ltluatex.tex| which may be used with older \LaTeX\
% formats, and with plain \TeX.
%
% \DescribeMacro{\newattribute}
% |\newattribute{|\meta{attribute}|}|\\
% Defines a named \cs{attribute}, indexed from~$1$
% (\emph{i.e.}~|\attribute0| is never defined). Attributes initially
% have the marker value |-"7FFFFFFF| (`unset') set by the engine.
%
% \noindent
% \DescribeMacro{\newcatcodetable}
% |\newcatcodetable{|\meta{catcodetable}|}|\\
% Defines a named \cs{catcodetable}, indexed from~$1$
% (|\catcodetable0| is never assigned). A new catcode table will be
% populated with exactly those values assigned by Ini\TeX{} (as described
% in the Lua\TeX{} manual).
%
% \noindent
% \DescribeMacro{\newluafunction}
% |\newluafunction{|\meta{function}|}|\\
% Defines a named \cs{luafunction}, indexed from~$1$ (Lua indexes from
% $1$ so |\luafunction0| is not available).
%
% \noindent
% \DescribeMacro{\newwhatsit}
% |\newwhatsit{|\meta{whatsit}|}|\\
% Defines a custom \cs{whatsit}, indexed from~$1$.
%
% \noindent
% \DescribeMacro{\catcodetable@initex}
% \DescribeMacro{\catcodetable@string}
% \DescribeMacro{\catcodetable@latex}
% \DescribeMacro{\catcodetable@atletter}
% Predefined category code tables with the obvious assignments. Note
% that the |latex| and |atletter| tables set the full Unicode range
% to the codes predefined by the kernel.
%
% \noindent
% \DescribeMacro{\setattribute}
% \DescribeMacro{\unsetattribute}
% |\setattribute{|\meta{attribute}|}{|\meta{value}|}|\\
% |\unsetattribute{|\meta{attribute}|}|\\
% Set and unset attributes in a manner analogous to |\setlength|. Note that
% attributes take a marker value when unset so this operation is distinct
% from setting the value to zero.
%
% \noindent
% \DescribeMacro{\setcatcodetable}
% |\setcatcodetable{|\meta{table}|}{|\meta{catcodes}|}|\\
% Sets the \meta{table} (which must have been previously defined) to
% apply the \meta{catcodes} specified.
%
% \section{Plain \TeX\ interface}
%
% The \textsf{ltluatex} interface may be used with plain \TeX\ using 
% |% \iffalse meta-comment
%
% Copyright 2015
% The LaTeX3 Project and any individual authors listed elsewhere
% in this file.
%
% It may be distributed and/or modified under the conditions of
% the LaTeX Project Public License (LPPL), either version 1.3c of
% this license or (at your option) any later version.  The latest
% version of this license is in the file:
%
%   http://www.latex-project.org/lppl.txt
%
%
%
%<2ekernel>%%% From File: ltluatex.dtx
%<plain>\ifx\newluafunction\undefined\else\expandafter\endinput\fi
%<tex>\ifx
%<tex>  \ProvidesFile\undefined\begingroup\def\ProvidesFile
%<tex>  #1#2[#3]{\endgroup\immediate\write-1{File: #1 #3}}
%<tex>\fi
%<plain>\ProvidesFile{ltluatex.tex}
%<*driver>
\ProvidesFile{ltluatex.dtx}
%</driver>
%<*tex>
[2015/08/01 v1.0a
%</tex>
%<plain>  LuaTeX support for plain TeX (core)
%<*tex>
]
\edef\etatcatcode{\the\catcode`\@}
\catcode`\@=11
%</tex>
%<*driver>
\documentclass{ltxdoc}
\GetFileInfo{ltluatex.dtx}
\begin{document}
\title{\filename\\(Lua\TeX{}-specific support)}
\author{David Carlisle and Joseph Wright\footnote{Significant portions
  of the code here are adapted/simplified from the packages \textsf{luatex} and
  \textsf{luatexbase} written by Heiko Oberdiek, \'{E}lie Roux,
  Manuel P\'{e}gouri\'{e}-Gonnar and Philipp Gesang.}}
\date{\filedate}
\maketitle
\setcounter{tocdepth}{2}
\tableofcontents
\DocInput{\filename}
\end{document}
%</driver>
% \fi
%
%
% \section{Overview}
%
% Lua\TeX{} adds a number of engine-specific functions to \TeX{}. Several of
% these require set up that is best done in the kernel or need related support
% functions. This file provides \emph{basic} support for Lua\TeX{} at the
% \LaTeXe{} kernel level plus as a loadable file which can be used with
% plain \TeX{} and \LaTeX{}.
%
% This file contains code for both \TeX{} (to be stored as part of the format)
% and Lua (to be loaded at the start of each job). In the Lua code, the kernel
% uses the namespace |luatexbase|.
%
% The following |\count| registers are used here for register allocation:
% \begin{itemize}
%  \item[\texttt{\string\e@alloc@attribute@count}] Attributes (default~258)
%  \item[\texttt{\string\e@alloc@ccodetable@count}] Category code tables
%    (default~259)
%  \item[\texttt{\string\e@alloc@luafunction@count}] Lua functions
%    (default~260)
%  \item[\texttt{\string\e@alloc@whatsit@count}] User whatsits (default~261)
% \end{itemize}
% (|\count256| is used for |\newmarks| allocation and |\count257|
% is used for\break
% |\newXeTeXintercharclass| with Xe\TeX{}, with code defined in
% \texttt{ltfinal.dtx}).
% With any \LaTeXe{} kernel from 2015 onward these registers are part of
% the block in the extended area reserved by the kernel (prior to 2015 the
% \LaTeXe{} kernel did not provide any functionality for the extended
% allocation area).
%
% \section{Core \TeX{} functionality}
%
% The commands defined here are defined for
% possible inclusion in a future \LaTeX{} format, however also extracted
% to the file |ltluatex.tex| which may be used with older \LaTeX\
% formats, and with plain \TeX.
%
% \DescribeMacro{\newattribute}
% |\newattribute{|\meta{attribute}|}|\\
% Defines a named \cs{attribute}, indexed from~$1$
% (\emph{i.e.}~|\attribute0| is never defined). Attributes initially
% have the marker value |-"7FFFFFFF| (`unset') set by the engine.
%
% \noindent
% \DescribeMacro{\newcatcodetable}
% |\newcatcodetable{|\meta{catcodetable}|}|\\
% Defines a named \cs{catcodetable}, indexed from~$1$
% (|\catcodetable0| is never assigned). A new catcode table will be
% populated with exactly those values assigned by Ini\TeX{} (as described
% in the Lua\TeX{} manual).
%
% \noindent
% \DescribeMacro{\newluafunction}
% |\newluafunction{|\meta{function}|}|\\
% Defines a named \cs{luafunction}, indexed from~$1$ (Lua indexes from
% $1$ so |\luafunction0| is not available).
%
% \noindent
% \DescribeMacro{\newwhatsit}
% |\newwhatsit{|\meta{whatsit}|}|\\
% Defines a custom \cs{whatsit}, indexed from~$1$.
%
% \noindent
% \DescribeMacro{\catcodetable@initex}
% \DescribeMacro{\catcodetable@string}
% \DescribeMacro{\catcodetable@latex}
% \DescribeMacro{\catcodetable@atletter}
% Predefined category code tables with the obvious assignments. Note
% that the |latex| and |atletter| tables set the full Unicode range
% to the codes predefined by the kernel.
%
% \noindent
% \DescribeMacro{\setattribute}
% \DescribeMacro{\unsetattribute}
% |\setattribute{|\meta{attribute}|}{|\meta{value}|}|\\
% |\unsetattribute{|\meta{attribute}|}|\\
% Set and unset attributes in a manner analogous to |\setlength|. Note that
% attributes take a marker value when unset so this operation is distinct
% from setting the value to zero.
%
% \noindent
% \DescribeMacro{\setcatcodetable}
% |\setcatcodetable{|\meta{table}|}{|\meta{catcodes}|}|\\
% Sets the \meta{table} (which must have been previously defined) to
% apply the \meta{catcodes} specified.
%
% \section{Plain \TeX\ interface}
%
% The \textsf{ltluatex} interface may be used with plain \TeX\ using 
% |% \iffalse meta-comment
%
% Copyright 2015
% The LaTeX3 Project and any individual authors listed elsewhere
% in this file.
%
% It may be distributed and/or modified under the conditions of
% the LaTeX Project Public License (LPPL), either version 1.3c of
% this license or (at your option) any later version.  The latest
% version of this license is in the file:
%
%   http://www.latex-project.org/lppl.txt
%
%
%
%<2ekernel>%%% From File: ltluatex.dtx
%<plain>\ifx\newluafunction\undefined\else\expandafter\endinput\fi
%<tex>\ifx
%<tex>  \ProvidesFile\undefined\begingroup\def\ProvidesFile
%<tex>  #1#2[#3]{\endgroup\immediate\write-1{File: #1 #3}}
%<tex>\fi
%<plain>\ProvidesFile{ltluatex.tex}
%<*driver>
\ProvidesFile{ltluatex.dtx}
%</driver>
%<*tex>
[2015/08/01 v1.0a
%</tex>
%<plain>  LuaTeX support for plain TeX (core)
%<*tex>
]
\edef\etatcatcode{\the\catcode`\@}
\catcode`\@=11
%</tex>
%<*driver>
\documentclass{ltxdoc}
\GetFileInfo{ltluatex.dtx}
\begin{document}
\title{\filename\\(Lua\TeX{}-specific support)}
\author{David Carlisle and Joseph Wright\footnote{Significant portions
  of the code here are adapted/simplified from the packages \textsf{luatex} and
  \textsf{luatexbase} written by Heiko Oberdiek, \'{E}lie Roux,
  Manuel P\'{e}gouri\'{e}-Gonnar and Philipp Gesang.}}
\date{\filedate}
\maketitle
\setcounter{tocdepth}{2}
\tableofcontents
\DocInput{\filename}
\end{document}
%</driver>
% \fi
%
%
% \section{Overview}
%
% Lua\TeX{} adds a number of engine-specific functions to \TeX{}. Several of
% these require set up that is best done in the kernel or need related support
% functions. This file provides \emph{basic} support for Lua\TeX{} at the
% \LaTeXe{} kernel level plus as a loadable file which can be used with
% plain \TeX{} and \LaTeX{}.
%
% This file contains code for both \TeX{} (to be stored as part of the format)
% and Lua (to be loaded at the start of each job). In the Lua code, the kernel
% uses the namespace |luatexbase|.
%
% The following |\count| registers are used here for register allocation:
% \begin{itemize}
%  \item[\texttt{\string\e@alloc@attribute@count}] Attributes (default~258)
%  \item[\texttt{\string\e@alloc@ccodetable@count}] Category code tables
%    (default~259)
%  \item[\texttt{\string\e@alloc@luafunction@count}] Lua functions
%    (default~260)
%  \item[\texttt{\string\e@alloc@whatsit@count}] User whatsits (default~261)
% \end{itemize}
% (|\count256| is used for |\newmarks| allocation and |\count257|
% is used for\break
% |\newXeTeXintercharclass| with Xe\TeX{}, with code defined in
% \texttt{ltfinal.dtx}).
% With any \LaTeXe{} kernel from 2015 onward these registers are part of
% the block in the extended area reserved by the kernel (prior to 2015 the
% \LaTeXe{} kernel did not provide any functionality for the extended
% allocation area).
%
% \section{Core \TeX{} functionality}
%
% The commands defined here are defined for
% possible inclusion in a future \LaTeX{} format, however also extracted
% to the file |ltluatex.tex| which may be used with older \LaTeX\
% formats, and with plain \TeX.
%
% \DescribeMacro{\newattribute}
% |\newattribute{|\meta{attribute}|}|\\
% Defines a named \cs{attribute}, indexed from~$1$
% (\emph{i.e.}~|\attribute0| is never defined). Attributes initially
% have the marker value |-"7FFFFFFF| (`unset') set by the engine.
%
% \noindent
% \DescribeMacro{\newcatcodetable}
% |\newcatcodetable{|\meta{catcodetable}|}|\\
% Defines a named \cs{catcodetable}, indexed from~$1$
% (|\catcodetable0| is never assigned). A new catcode table will be
% populated with exactly those values assigned by Ini\TeX{} (as described
% in the Lua\TeX{} manual).
%
% \noindent
% \DescribeMacro{\newluafunction}
% |\newluafunction{|\meta{function}|}|\\
% Defines a named \cs{luafunction}, indexed from~$1$ (Lua indexes from
% $1$ so |\luafunction0| is not available).
%
% \noindent
% \DescribeMacro{\newwhatsit}
% |\newwhatsit{|\meta{whatsit}|}|\\
% Defines a custom \cs{whatsit}, indexed from~$1$.
%
% \noindent
% \DescribeMacro{\catcodetable@initex}
% \DescribeMacro{\catcodetable@string}
% \DescribeMacro{\catcodetable@latex}
% \DescribeMacro{\catcodetable@atletter}
% Predefined category code tables with the obvious assignments. Note
% that the |latex| and |atletter| tables set the full Unicode range
% to the codes predefined by the kernel.
%
% \noindent
% \DescribeMacro{\setattribute}
% \DescribeMacro{\unsetattribute}
% |\setattribute{|\meta{attribute}|}{|\meta{value}|}|\\
% |\unsetattribute{|\meta{attribute}|}|\\
% Set and unset attributes in a manner analogous to |\setlength|. Note that
% attributes take a marker value when unset so this operation is distinct
% from setting the value to zero.
%
% \noindent
% \DescribeMacro{\setcatcodetable}
% |\setcatcodetable{|\meta{table}|}{|\meta{catcodes}|}|\\
% Sets the \meta{table} (which must have been previously defined) to
% apply the \meta{catcodes} specified.
%
% \section{Plain \TeX\ interface}
%
% The \textsf{ltluatex} interface may be used with plain \TeX\ using 
% |\input{ltluatex}| this inputs |ltluatex.tex| which inputs
% |etex.src| (or |etex.sty| if used with \LaTeX)
% if it is not already input, and then defines some internal commands to
% allow the \textsf{ltluatex} interface to be defined.
%
% The \textsf{luatexbase} package interface may also be used in plain \TeX,
% as before, by inputting the package |\input luatexbase.sty|. The new
% version of \textsf{luatexbase} is based on this \textsf{ltluatex}
% code but implements a compatibility layer providing the interface
% of the original package.
%
% \section{Lua functionality}
%
% \begingroup
%
% \begingroup\lccode`~=`_
% \lowercase{\endgroup\let~}_
% \catcode`_=12
%
% \subsection{Allocators in Lua}
%
% \DescribeMacro{luatexbase.new_attribute}
% |luatexbase.new_attribute(|\meta{attribute}|)|\\
% Returns an allocation number for the \meta{attribute}, indexed from~$1$.
% The attribute will be initialised with the marker value |-"7FFFFFFF|
% (`unset'). The attribute allocation sequence is shared with the \TeX{}
% code but this function does \emph{not} define a token using
% |\attributedef|.
% The attribute name is recorded in the |attributes| table. A
% metatable is provided so that the table syntax can be used
% consistently for attributes declared in \TeX\ or lua.
%
% \noindent
% \DescribeMacro{luatexbase.new_whatsit}
% |luatexbase.new_whatsit(|\meta{whatsit}|)|\\
% Returns an allocation number for the custom \meta{whatsit}, indexed from~$1$.
%
% \noindent
% \DescribeMacro{luatexbase.new_bytecode}
% |luatexbase.new_bytecode(|\meta{bytecode}|)|\\
% Returns an allocation number for a bytecode register, indexed from~$1$.
% The optional \meta{name} argument is just used for logging.
%
% \noindent
% \DescribeMacro{luatexbase.new_chunkname}
% |luatexbase.new_chunkname(|\meta{chunkname}|)|\\
% Returns an allocation number for a lua chunk name for use with 
% |\directlua| and |\latelua|, indexed from~$1$.
% The number is returned and also \meta{name} argument is added to the
% |lua.name| array at that index.
%
%
% \subsection{Lua access to \TeX{} register numbers}
%
% \DescribeMacro{luatexbase.registernumber}
% |luatexbase.registernumer(|\meta{name}|)|\\
% Sometimes (notably in the case of Lua attributes) it is necessary to
% access a register \emph{by number} that has been allocated by \TeX{}.
% This package provides a function to look up the relevant number
% using Lua\TeX{}'s internal tables. After for example
% |\newattribute\myattrib|, |\myattrib| would be defined by (say)
% |\myattrib=\attribute15|.  |luatexbase.registernumer("myattrib")|
% would then return the register number, $15$ in this case. If the string passed
% as argument does not correspond to a token defined by |\attributedef|,
% |\countdef| or similar commands, the Lua value |false| is returned.
%
% As an example, consider the input:
%\begin{verbatim}
% \newcommand\test[1]{%
% \typeout{#1: \expandafter\meaning\csname#1\endcsname^^J
% \space\space\space\space
% \directlua{tex.write(luatexbase.registernumber("#1") or "bad input")}%
% }}
%
% \test{undefinedrubbish}
%
% \test{space}
%
% \test{hbox}
%
% \test{@MM}
%
% \test{@tempdima}
% \test{@tempdimb}
%
% \test{strutbox}
%
% \test{sixt@@n}
%
% \attrbutedef\myattr=12
% \myattr=200
% \test{myattr}
%
%\end{verbatim}
%
% If the demonstration code is processed with Lua\LaTeX{} then the following
% would be produced in the log and terminal output.
%\begin{verbatim}
% undefinedrubbish: \relax
%      bad input
% space: macro:->
%      bad input
% hbox: \hbox
%      bad input
% @MM: \mathchar"4E20
%      20000
% @tempdima: \dimen14
%      14
% @tempdimb: \dimen15
%      15
% strutbox: \char"B
%      11
% sixt@@n: \char"10
%      16
% myattr: \attribute12
%      12
%\end{verbatim}
%
% Notice how undefined commands, or commands unrelated to registers
% do not produce an error, just return |false| and so print
% |bad input| here. Note also that commands defined by |\newbox| work and
% return the number of the box register even though the actual command
% holding this number is a |\chardef| defined token (there is no
% |\boxdef|).
%
% \subsection{Module utilities}
%
% \DescribeMacro{luatexbase.provides_module}
% |luatexbase.provides_module(|\meta{info}|)|\\
% This function is used by modules to identify themselves; the |info| should be
% a table containing information about the module. The required field
% |name| must contain the name of the module. It is recommended to provide a
% field |date| in the usual \LaTeX{} format |yyyy/mm/dd|. Optional fields
% |version| (a string) and |description| may be used if present. This
% information will be recorded in the log. Other fields are ignored.
%
% \noindent
% \DescribeMacro{luatexbase.module_info}
% \DescribeMacro{luatexbase.module_warning}
% \DescribeMacro{luatexbase.module_error}
% |luatexbase.module_info(|\meta{module}, \meta{text}|)|\\
% |luatexbase.module_warning(|\meta{module}, \meta{text}|)|\\
% |luatexbase.module_error(|\meta{module}, \meta{text}|)|\\
% These functions are similar to \LaTeX{}'s |\PackageError|, |\PackageWarning|
% and |\PackageInfo| in the way they format the output.  No automatic line
% breaking is done, you may still use |\n| as usual for that, and the name of
% the package will be prepended to each output line.
%
% Note that |luatexbase.module_error| raises an actual Lua error with |error()|,
% which currently means a call stack will be dumped. While this may not
% look pretty, at least it provides useful information for tracking the
% error down.
%
% \subsection{Callback management}
%
% \noindent
% \DescribeMacro{luatexbase.add_to_callback}
% |luatexbase.add_to_callback(|^^A
% \meta{callback}, \meta{function}, \meta{description}|)|
% Registers the \meta{function} into the \meta{callback} with a textual
% \meta{description} of the function. Functions are inserted into the callback
% in the order loaded.
%
% \noindent
% \DescribeMacro{luatexbase.remove_from_callback}
% |luatexbase.remove_from_callback(|\meta{callback}, \meta{description}|)|
% Removes the function with \meta{description} from the \meta{callback}.
% The removed function and its description 
% are returned as the results of this function.
%
% \noindent
% \DescribeMacro{luatexbase.in_callback}
% |luatexbase.in_callback(|\meta{callback}, \meta{description}|)|
% Checks if the \meta{description} matches one of the functions added
% to the list for the \meta{callback}, returning a boolean value.
%
% \noindent
% \DescribeMacro{luatexbase.disable_callback}
% |luatexbase.disable_callback(|\meta{callback}|)|
% Sets the \meta{callback} to \texttt{false} as described in the Lua\TeX{}
% manual for the underlying \texttt{callback.register} built-in. Callbacks
% will only be set to false (and thus be skipped entirely) if there are
% no functions registered using the callback.
%
% \noindent
% \DescribeMacro{luatexbase.callback_descriptions}
% A list of the descriptions of functions registered to the specified
% callback is returned. |{}| is returned if there are no functions registered.
%
% \noindent
% \DescribeMacro{luatexbase.create_callback}
% |luatexbase.create_callback(|\meta{name},meta{type},\meta{default}|)|
% Defines a user defined callback. The last argument is a default
% functtion of |false|.
%
% \noindent
% \DescribeMacro{luatexbase.call_callback}
% |luatexbase.create_callback(|\meta{name},\ldots|)|
% Calls a user defined callback with the supplied arguments.
%
% \endgroup
%
% \CheckSum{425}
% \StopEventually{}
%
% \section{Implementation}
%
%    \begin{macrocode}
%<*2ekernel|tex|latexrelease>
%<2ekernel|latexrelease>\ifx\directlua\@undefined\else
%    \end{macrocode}
%
%
% \subsection{Minimum Lua\TeX{} version}
%
% Lua\TeX{} has changed a lot over time. In the kernel support for ancient
% versions is not provided: trying to build a format with a very old binary
% therefore gives some information in the log and loading stops. The cut-off
% selected here relates to the tree-searching behaviour of |require()|:
% from version~0.60, Lua\TeX{} will correctly find Lua files in the |texmf|
% tree without `help'.
%    \begin{macrocode}
%<latexrelease>\IncludeInRelease{2015/11/01}
%<latexrelease>                 {\newluafunction}{LuaTeX}%
\ifnum\luatexversion<60 %
  \wlog{***************************************************}
  \wlog{* LuaTeX version too old for ltluatex support *}
  \wlog{***************************************************}
  \expandafter\endinput
\fi
%    \end{macrocode}
%
% \subsection{Older \LaTeX{}/Plain \TeX\ setup}
% 
%    \begin{macrocode}
%<*tex>
%    \end{macrocode}
%
% Older \LaTeX{] formats don't have the primitives with `native' names:
% sort that out. I fthey already exist this will still be safe.
%    \begin{macrocode}
\directlua{tex.enableprimitives("",tex.extraprimitives("luatex"))}
%    \end{macrocode}
%
%    \begin{macrocode}
\ifx\e@alloc\@undefined
%    \end{macrocode}
%
% In pre-2014 \LaTeX{}, or plain \TeX{}, load |etex.{sty,src}|.
%    \begin{macrocode}
  \ifx\documentclass\@undefined
    \ifx\loccount\@undefined
      \input{etex.src}%
    \fi
    \catcode`\@=11 %
    \outer\expandafter\def\csname newfam\endcsname
                          {\alloc@8\fam\chardef\et@xmaxfam}
  \else
    \RequirePackage{etex}
    \expandafter\def\csname newfam\endcsname
                    {\alloc@8\fam\chardef\et@xmaxfam}
    \expandafter\let\expandafter\new@mathgroup\csname newfam\endcsname
  \fi
%    \end{macrocode}
%
% \subsubsection{Fixes to \texttt{etex.src}/\texttt{etex.sty}}
%
% These could and probably should be made directly in an
% update to etex.src which already has some luatex-specific
% code, but does not define the correct range for luatex.
%
%    \begin{macrocode}
% 2015-07-13 higher range in luatex
\edef \et@xmaxregs {\ifx\directlua\@undefined 32768\else 65536\fi}
% luatex/xetex also allow more math fam
\edef \et@xmaxfam {\ifx\Umathchar\@undefined\sixt@@n\else\@cclvi\fi}
%    \end{macrocode}
%
%    \begin{macrocode}
\count 270=\et@xmaxregs % locally allocates \count registers
\count 271=\et@xmaxregs % ditto for \dimen registers
\count 272=\et@xmaxregs % ditto for \skip registers
\count 273=\et@xmaxregs % ditto for \muskip registers
\count 274=\et@xmaxregs % ditto for \box registers
\count 275=\et@xmaxregs % ditto for \toks registers
\count 276=\et@xmaxregs % ditto for \marks classes
%    \end{macrocode}
%
% and 256 or 16 fam. (Done above due to plain/\LaTeX\ differences in
% \textsf{ltluatex}.)
%    \begin{macrocode}
% \outer\def\newfam{\alloc@8\fam\chardef\et@xmaxfam}
%    \end{macrocode}
%
% End of proposed changes to \texttt{etex.src}
%
% \subsubsection{luatex specific settings}
% 
% Switch to global cf |luatex.sty| to leave room for inserts
% not really needed for luatex but possibly most compatible
% with existing use.
%    \begin{macrocode}
\expandafter\let\csname newcount\expandafter\expandafter\endcsname
                \csname globcount\endcsname
\expandafter\let\csname newdimen\expandafter\expandafter\endcsname
                \csname globdimen\endcsname
\expandafter\let\csname newskip\expandafter\expandafter\endcsname
                \csname globskip\endcsname
\expandafter\let\csname newbox\expandafter\expandafter\endcsname
                \csname globbox\endcsname
%    \end{macrocode}
%
% Define|\e@alloc| as in latex (the existing macros in |etex.src|
% hard to extend to further register types as they assume specific
% 26x and 27x count range. For compatibility the existing register
% allocation is not changed.
%
%    \begin{macrocode}
\chardef\e@alloc@top=65535
\let\e@alloc@chardef\chardef
%    \end{macrocode}
%
%    \begin{macrocode}
\def\e@alloc#1#2#3#4#5#6{%
  \global\advance#3\@ne
  \e@ch@ck{#3}{#4}{#5}#1%
  \allocationnumber#3\relax
  \global#2#6\allocationnumber
  \wlog{\string#6=\string#1\the\allocationnumber}}%
%    \end{macrocode}
%
%    \begin{macrocode}
\gdef\e@ch@ck#1#2#3#4{%
  \ifnum#1<#2\else
    \ifnum#1=#2\relax
      #1\@cclvi
      \ifx\count#4\advance#1 10 \fi
    \fi
    \ifnum#1<#3\relax
    \else
      \errmessage{No room for a new \string#4}%
    \fi
  \fi}%
%    \end{macrocode}
%
% Two simple \LaTeX\ macros used in |ltlatex.sty|.
%    \begin{macrocode}
\long\def\@gobble#1{}
\long\def\@firstofone#1{#1}
%    \end{macrocode}
%
%    \begin{macrocode}
% Fix up allocations not to clash with |etex.src|.
%    \end{macrocode}
%
%    \begin{macrocode}
\expandafter\csname newcount\endcsname\e@alloc@attribute@count
\expandafter\csname newcount\endcsname\e@alloc@ccodetable@count
\expandafter\csname newcount\endcsname\e@alloc@luafunction@count
\expandafter\csname newcount\endcsname\e@alloc@whatsit@count
%    \end{macrocode}
%
% End of conditional setup for plain \TeX\ / old \LaTeX.
%    \begin{macrocode}
\fi
%</tex>
%    \end{macrocode}
%
%
% \subsection{Attributes}
%
% \begin{macro}{\newattribute}
% \changes{v1.0a}{0000/00/00}{Macro added}
%   As is generally the case for the Lua\TeX{} registers we start here
%   from~$1$. Notably, some code assumes that |\attribute0| is never used so
%   this is important in this case.
%    \begin{macrocode}
\ifx\e@alloc@attribute@count\@undefined
  \countdef\e@alloc@attribute@count=258
\fi
\def\newattribute#1{%
  \e@alloc\attribute\attributedef
    \e@alloc@attribute@count\m@ne\e@alloc@top#1%
}
\e@alloc@attribute@count=\z@
%    \end{macrocode}
% \end{macro}
%
% \begin{macro}{\setattribute}
% \begin{macro}{\unsetattribute}
%   Handy utilities.
%    \begin{macrocode}
\def\setattribute#1#2{#1=\numexpr#2\relax}
\def\unsetattribute#1{#1=-"7FFFFFFF\relax}
%    \end{macrocode}
% \end{macro}
% \end{macro}
%
% \subsection{Category code tables}
%
% \begin{macro}{\newcatcodetable}
% \changes{v1.0a}{0000/00/00}{Macro added}
%   Category code tables are allocated with a limit half of that used by Lua\TeX{}
%   for everything else. At the end of allocation there needs to be an
%   initialisation step. Table~$0$ is already taken (it's the global one for
%   current use) so the allocation starts at~$1$.
%    \begin{macrocode}
\ifx\e@alloc@ccodetable@count\@undefined
  \countdef\e@alloc@ccodetable@count=259
\fi
\def\newcatcodetable#1{%
  \e@alloc\catcodetable\chardef
    \e@alloc@ccodetable@count\m@ne{"8000}#1%
  \initcatcodetable\allocationnumber
}
\e@alloc@ccodetable@count=\z@
%    \end{macrocode}
% \end{macro}
%
% \begin{macro}{\catcodetable@initex}
% \changes{v1.0a}{0000/00/00}{Macro added}
% \begin{macro}{\catcodetable@string}
% \changes{v1.0a}{0000/00/00}{Macro added}
% \begin{macro}{\catcodetable@latex}
% \changes{v1.0a}{0000/00/00}{Macro added}
% \begin{macro}{\catcodetable@atletter}
% \changes{v1.0a}{0000/00/00}{Macro added}
%   Save a small set of standard tables. The Unicode data is read
%   here in a group avoiding any global definitions: that needs a bit
%   of effort so that in package/plain mode there is no effect on any
%   settings already in force.
%    \begin{macrocode}
\newcatcodetable\catcodetable@initex
\newcatcodetable\catcodetable@string
\begingroup
  \def\setrangecatcode#1#2#3{%
    \ifnum#1>#2 %
      \expandafter\@gobble
    \else
      \expandafter\@firstofone
    \fi
      {%
        \catcode#1=#3 %
        \expandafter\setrangecatcode\expandafter
          {\number\numexpr#1 + 1\relax}{#2}{#3}
      }%
  }
  \@firstofone{%
    \catcodetable\catcodetable@initex
      \catcode0=12 %
      \catcode12=12 %
      \catcode37=12 %
      \setrangecatcode{65}{90}{12}%
      \setrangecatcode{97}{122}{12}%
      \catcode92=12 %
      \catcode127=12 %
      \savecatcodetable\catcodetable@string
    \endgroup
  }%
\newcatcodetable\catcodetable@latex
\newcatcodetable\catcodetable@atletter
\begingroup
  \let\ENDGROUP\endgroup
  \let\begingroup\relax
  \let\endgroup\relax
  \let\global\relax
  \let\gdef\def
  \input{unicode-letters.def}%
  \let\endgroup\ENDGROUP
  \@firstofone{%
    \catcode64=12 %
    \savecatcodetable\catcodetable@latex
    \catcode64=11 %
    \savecatcodetable\catcodetable@atletter
   }
\endgroup
%    \end{macrocode}
% \end{macro}
% \end{macro}
% \end{macro}
% \end{macro}
%
% \subsection{Named Lua functions}
%
% \begin{macro}{\newluafunction}
% \changes{v1.0a}{0000/00/00}{Macro added}
%   Much the same story for allocating Lua\TeX{} functions except here they are
%   just numbers so are allocated in the same way as boxes. Lua index from~$1$
%   so once again slot~$0$ is skipped.
%    \begin{macrocode}
\ifx\e@alloc@luafunction@count\@undefined
  \countdef\e@alloc@luafunction@count=260
\fi
\def\newluafunction{%
  \e@alloc\luafunction\e@alloc@chardef
    \e@alloc@luafunction@count\m@ne\e@alloc@top
}
\e@alloc@luafunction@count=\z@
%    \end{macrocode}
% \end{macro}
%
% \subsection{Custom whatsits}
%
% \begin{macro}{\newwhatsit}
% \changes{v1.0a}{0000/00/00}{Macro added}
%   These are only settable from Lua but for consistency are definable
%   here.
%    \begin{macrocode}
\ifx\e@alloc@whatsit@count\@undefined
  \countdef\e@alloc@whatsit@count=261
\fi
\def\newwhatsit#1{%
  \e@alloc\whatsit\e@alloc@chardef
    \e@alloc@whatsit@count\m@ne\e@alloc@top#1%
}
\e@alloc@whatsit@count=\z@
%    \end{macrocode}
% \end{macro}
%
% \subsection{Lua loader}
%
% Load the Lua code at the start of every job.
% For the conversion of \TeX{} into numbers at the Lua side we need some
% known registers: for convenience we use a set of systematic names, which
% means using a group around the Lua loader.
%    \begin{macrocode}
%<2ekernel>\everyjob\expandafter{%
%<2ekernel>  \the\everyjob
  \begingroup
    \attributedef\attributezero=0 %
    \chardef     \charzero     =0 %
%    \end{macrocode}
% Note name change required on older luatex, for hash table access.
%    \begin{macrocode}
    \countdef    \CountZero    =0 % 
    \dimendef    \dimenzero    =0 %
    \mathchardef \mathcharzero =0 %
    \muskipdef   \muskipzero   =0 %
    \skipdef     \skipzero     =0 %
    \toksdef     \tokszero     =0 %
    \directlua{require("ltluatex")}
  \endgroup
%<2ekernel>}
%<latexrelease>\EndIncludeInRelease
%    \end{macrocode}
%
%    \begin{macrocode}
%<latexrelease>\IncludeInRelease{0000/00/00}
%<latexrelease>                 {\newluafunction}{LuaTeX}%
%<latexrelease>\let\e@alloc@attribute@count\@undefined
%<latexrelease>\let\newattribute\@undefined
%<latexrelease>\let\setattribute\@undefined
%<latexrelease>\let\unsetattribute\@undefined
%<latexrelease>\let\e@alloc@ccodetable@count\@undefined
%<latexrelease>\let\newcatcodetable\@undefined
%<latexrelease>\let\catcodetable@initex\@undefined
%<latexrelease>\let\catcodetable@string\@undefined
%<latexrelease>\let\catcodetable@latex\@undefined
%<latexrelease>\let\catcodetable@atletter\@undefined
%<latexrelease>\let\e@alloc@luafunction@count\@undefined
%<latexrelease>\let\newluafunction\@undefined
%<latexrelease>\let\e@alloc@luafunction@count\@undefined
%<latexrelease>\let\newwhatsit\@undefined
%<latexrelease>\let\e@alloc@whatsit@count\@undefined
%<latexrelease>\EndIncludeInRelease
%    \end{macrocode}
%
%    \begin{macrocode}
%<2ekernel|latexrelease>\fi
%</2ekernel|tex|latexrelease>
%    \end{macrocode}
%
% \subsection{Lua module preliminaries}
%
% \begingroup
%
%  \begingroup\lccode`~=`_
%  \lowercase{\endgroup\let~}_
%  \catcode`_=12
%
%    \begin{macrocode}
%<*lua>
%    \end{macrocode}
%
% Some set up for the Lua module which is needed for all of the Lua
% functionality added here.
%
% \begin{macro}{luatexbase}
% \changes{v1.0a}{0000/00/00}{Table added}
%   Set up the table for the returned functions. This is used to expose
%   all of the public functions.
%    \begin{macrocode}
luatexbase       = luatexbase or { }
local luatexbase = luatexbase
%    \end{macrocode}
% \end{macro}
%
% Some Lua best practice: use local versions of functions where possible.
%    \begin{macrocode}
local string_gsub      = string.gsub
local tex_count        = tex.count
local tex_setattribute = tex.setattribute
local tex_setcount     = tex.setcount
local texio_write_nl   = texio.write_nl
%    \end{macrocode}
%
% \subsection{Lua module utilities}
%
% \subsubsection{Module tracking}
%
% \begin{macro}{luatexbase.modules}
% \changes{v1.0a}{0000/00/00}{Function modified}
%   To allow tracking of module usage, a structure is provided to store
%   information and to return it.
%    \begin{macrocode}
local modules = modules or { }
%    \end{macrocode}
% \end{macro}
%
% \begin{macro}{luatexbase.provides_module}
% \changes{v1.0a}{0000/00/00}{Function added}
%   Modelled on |\ProvidesPackage|, we store much the same information but
%   with a little more structure.
%    \begin{macrocode}
local function provides_module(info)
  if not (info and info.name) then
    luatexbase_error("Missing module name for provides_modules")
    return
  end
  local function spaced(text)
    return text and (" " .. text) or ""
  end
  texio_write_nl(
    "log",
    "Lua module: " .. info.name
      .. spaced(info.date)
      .. spaced(info.version)
      .. spaced(info.description)
  )
  modules[info.name] = info
end
luatexbase.provides_module = provides_module
%    \end{macrocode}
% \end{macro}
%
% \subsubsection{Module messages}
%
% There are various warnings and errors that need to be given. For warnings
% we can get exactly the same formatting as from \TeX{}. For errors we have to
% make some changes. Here we give the text of the error in the \LaTeX{} format
% then force an error from Lua to halt the run. Splitting the message text is
% done using |\n| which takes the place of |\MessageBreak|.
%
% First an auxiliary for the formatting: this measures up the message
% leader so we always get the correct indent.
%    \begin{macrocode}
local function msg_format(mod, msg_type, text)
  local leader = ""
  if type == "Error" then
    leader = "! "
  end
  local cont
  if mod == "LaTeX" then
    cont = string_gsub(leader, ".", " ")
    leader = leader .. "LaTeX: "
  else
    first_head = leader .. "Module "  .. msg_type 
    cont = "(" .. mod .. ")"
      .. string_gsub(first_head, ".", " ")
    first_head =  leader .. "Module "  .. mod .. " " .. msg_type  .. ":"
  end
  return first_head .. " "
    .. string_gsub(
         text .. " on input line "
         .. tex.inputlineno, "\n", "\n" .. cont .. " "
      )
   .. "\n"
end
%    \end{macrocode}
%
% \begin{macro}{luatexbase.module_info}
% \changes{v1.0a}{0000/00/00}{Function added}
% \begin{macro}{luatexbase.module_warning}
% \changes{v1.0a}{0000/00/00}{Function added}
% \begin{macro}{luatexbase.module_error}
% \changes{v1.0a}{0000/00/00}{Function added}
%   Write messages.
%    \begin{macrocode}
local function module_info(mod, text)
  local i
  for _,i in ipairs(msg_format(mod, "Info", text):explode("\n")) do
    texio_write_nl("log", i)
  end
end
luatexbase.module_info = module_info
local function module_warning(mod, text)
  local i
  for _,i in ipairs(msg_format(mod, "Warning", text):explode("\n")) do
    texio_write_nl("term and log", i)
  end
end
luatexbase.module_warning = module_warning
local function module_error(mod, text)
  local i
  for _,i in ipairs(msg_format(mod, "Error", text):explode("\n")) do
    texio_write_nl("term and log", i)
  end
  texio_write_nl("term and log", "\n")
  error("See " .. mod .. " Error")
end
luatexbase.module_error = module_error
%    \end{macrocode}
% \end{macro}
% \end{macro}
% \end{macro}
%
% Dedicated versions for the rest of the code here.
%    \begin{macrocode}
local function luatexbase_warning(text)
  module_warning("luatexbase", text)
end
local function luatexbase_error(text)
  module_error("luatexbase", text)
end
%    \end{macrocode}
%
%
% \subsection{Accessing register numbers from Lua}
%
% Collect up the data from the \TeX{} level into a Lua table: from
% version~0.80, Lua\TeX{} makes that easy.
%    \begin{macrocode}
local luaregisterbasetable = { }
local registermap = {
  attributezero = "assign_attr"    ,
  charzero      = "char_given"     ,
  CountZero     = "assign_int"     ,
  dimenzero     = "assign_dimen"   ,
  mathcharzero  = "math_given"     ,
  muskipzero    = "assign_mu_skip" ,
  skipzero      = "assign_skip"    ,
  tokszero      = "assign_toks"    ,
}
local i, j
local createtoken
if tex.luatexversion >79 then
 createtoken   = newtoken.create
end
local hashtokens    = tex.hashtokens
local luatexversion = tex.luatexversion
for i,j in pairs (registermap) do
  if luatexversion < 80 then
    luaregisterbasetable[hashtokens()[i][1]] =
      hashtokens()[i][2]
  else
    luaregisterbasetable[j] = createtoken(i).mode
  end
end
%    \end{macrocode}
%
% \begin{macro}{luatexbase.registernumber}
%   Working out the correct return value can be done in two ways. For older
%   Lua\TeX{} releases it has to be extracted from the |hashtokens|. On the
%   other hand, newer Lua\TeX{}'s have |newtoken|, and whilst |.mode| isn't
%   currently documented, Hans Hagen pointed to this approach so we should be
%   OK.
%    \begin{macrocode}
local registernumber
if luatexversion < 80 then
  function registernumber(name)
    local nt = hashtokens()[name]
    if(nt and luaregisterbasetable[nt[1]]) then
      return nt[2] - luaregisterbasetable[nt[1]]
    else
      return false
    end
  end
else
  function registernumber(name)
    local nt = createtoken(name)
    if(luaregisterbasetable[nt.cmdname]) then
      return nt.mode - luaregisterbasetable[nt.cmdname]
    else
      return false
    end
  end
end
luatexbase.registernumber = registernumber
%    \end{macrocode}
% \end{macro}
%
% \subsection{Attribute allocation}
%
% \begin{macro}{luatexbase.new_attribute}
% \changes{v1.0a}{0000/00/00}{Function added}
%   As attributes are used for Lua manipulations its useful to be able
%   to assign from this end.
%    \begin{macrocode}
local attributes=setmetatable(
{},
{
__index = function(t,key)
return registernumber(key) or nil
end}
)
luatexbase.attributes=attributes
%    \end{macrocode}
%
%    \begin{macrocode}
local function new_attribute(name)
  tex_setcount("global", "e@alloc@attribute@count",
                          tex_count["e@alloc@attribute@count"] + 1)
  if tex_count["e@alloc@attribute@count"] > 65534 then
    luatexbase_error("No room for a new \\attribute")
    return -1
  end
  attributes[name]= tex_count["e@alloc@attribute@count"]
  texio_write_nl("Lua-only attribute " .. name .. " = " ..
                 tex_count["e@alloc@attribute@count"])
  return tex_count["e@alloc@attribute@count"]
end
luatexbase.new_attribute = new_attribute
%    \end{macrocode}
% \end{macro}
%
% \subsection{Custom whatsit allocation}
%
% \begin{macro}{luatexbase.new_whatsit}
% Much the same as for attribute allocation in Lua
%    \begin{macrocode}
local function new_whatsit(name)
  tex_setcount("global", "e@alloc@whatsit@count", 
                         tex_count["e@alloc@whatsit@count"] + 1)
  if tex_count["e@alloc@whatsit@count"] > 65534 then
    luatexbase_error("No room for a new custom whatsit")
    return -1
  end
  texio_write_nl("Custom whatsit " .. name .. " = " ..
                 tex_count["e@alloc@whatsit@count"])
  return tex_count["e@alloc@whatsit@count"]
end
luatexbase.new_whatsit = new_whatsit
%    \end{macrocode}
% \end{macro}
%
% \subsection{Bytecode register allocation}
%
% \begin{macro}{luatexbase.new_bytecode}
% Much the same as for attribute allocation in Lua.
% Currently we maintain the allocation count purely in lua, not using
% a \TeX\ counter.
% The optional \meta{name} argument is used in the log if given.
%    \begin{macrocode}
local bytecode_count = 0
local function new_bytecode(name)
  bytecode_count = bytecode_count + 1
  if bytecode_count > 65534 then
    luatexbase_error("No room for a new bytecode")
    return -1
  end
  texio_write_nl("Lua bytecode " .. (name or "") .. " = " .. 
                 bytecode_count .. "\n")
  return bytecode_count
end
luatexbase.new_bytecode = new_bytecode
%    \end{macrocode}
% \end{macro}
%
% \subsection{Lua chunk name allocation}
%
% \begin{macro}{luatexbase.new_chunkname}
% As for bytecode registers but here we also store the name in the
% |lua.name| table.
%    \begin{macrocode}
local chunkname_count = 0
local function new_chunkname(name)
  chunkname_count = chunkname_count + 1
  if chunkname_count > 65534 then
    luatexbase_error("No room for a new chunkname")
    return -1
  end
  lua.name[chunkname_count]=name
  texio_write_nl("Lua chunkname " .. name .. " = " .. 
                 chunkname_count .. "\n")
  return chunkname_count
end
luatexbase.new_chunkname = new_chunkname
%    \end{macrocode}
% \end{macro}
%
% \subsection{Lua callback management}
%
% The native mechanism for callbacks in Lua allows only one per function.
% That is extremely restrictive and so a mechanism is needed to add and
% remove callbacks from the appropriate hooks.
%
% \subsubsection{Housekeeping}
%
% The main table: keys are callback names, and values are the associated lists
% of functions. More precisely, the entries in the list are tables holding the
% actual function as |func| and the identifying description as |description|.
% Only callbacks with a non-empty list of functions have an entry in this
% list.
%    \begin{macrocode}
local callbacklist = callbacklist or { }
%    \end{macrocode}
%
% Numerical codes for callback types, and name-to-value association (the
% table keys are strings, the values are numbers).
%    \begin{macrocode}
local list, data, exclusive, simple = 1, 2, 3, 4
local types = {
  list      = list,
  data      = data,
  exclusive = exclusive,
  simple    = simple,
}
%    \end{macrocode}
%
% Now, list all predefined callbacks with their current type, based on the
% Lua\TeX{} manual version~0.80. A full list of the currently-available
% callbacks can be obtained using
%  \begin{verbatim}
%    \directlua{
%      for i,_ in pairs(callback.list()) do
%        texio.write_nl("- " .. i)
%      end
%    }
%    \bye
%  \end{verbatim}
% in plain Lua\TeX{}. (Some undocumented callbacks are omitted as they are
% to be removed.)
%    \begin{macrocode}
local callbacktypes = callbacktypes or {
%    \end{macrocode}
%   Section 4.1.1: file discovery callbacks.
%    \begin{macrocode}
  find_read_file     = exclusive,
  find_write_file    = exclusive,
  find_font_file     = data,
  find_output_file   = data,
  find_format_file   = data,
  find_vf_file       = data,
  find_map_file      = data,
  find_enc_file      = data,
  find_sfd_file      = data,
  find_pk_file       = data,
  find_data_file     = data,
  find_opentype_file = data,
  find_truetype_file = data,
  find_type1_file    = data,
  find_image_file    = data,
%    \end{macrocode}
% Section 4.1.2: file reading callbacks.
%    \begin{macrocode}
  open_read_file     = exclusive,
  read_font_file     = exclusive,
  read_vf_file       = exclusive,
  read_map_file      = exclusive,
  read_enc_file      = exclusive,
  read_sfd_file      = exclusive,
  read_pk_file       = exclusive,
  read_data_file     = exclusive,
  read_truetype_file = exclusive,
  read_type1_file    = exclusive,
  read_opentype_file = exclusive,
%    \end{macrocode}
% Section 4.1.3: data processing callbacks.
%    \begin{macrocode}
  process_input_buffer  = data,
  process_output_buffer = data,
  process_jobname       = data,
  token_filter          = exclusive,
%    \end{macrocode}
% Section 4.1.4: node list processing callbacks.
%    \begin{macrocode}
  buildpage_filter      = simple,
  pre_linebreak_filter  = list,
  linebreak_filter      = list,
  post_linebreak_filter = list,
  hpack_filter          = list,
  vpack_filter          = list,
  pre_output_filter     = list,
  hyphenate             = simple,
  ligaturing            = simple,
  kerning               = simple,
  mlist_to_hlist        = list,
%    \end{macrocode}
% Section 4.1.5: information reporting callbacks.
%    \begin{macrocode}
  pre_dump            = simple,
  start_run           = simple,
  stop_run            = simple,
  start_page_number   = simple,
  stop_page_number    = simple,
  show_error_hook     = simple,
  show_error_message  = simple,
  show_lua_error_hook = simple,
  start_file          = simple,
  stop_file           = simple,
%    \end{macrocode}
% Section 4.1.6: PDF-related callbacks.
%    \begin{macrocode}
  finish_pdffile = data,
  finish_pdfpage = data,
%    \end{macrocode}
% Section 4.1.7: font-related callbacks.
%    \begin{macrocode}
  define_font = exclusive,
%    \end{macrocode}
% Undocumented callbacks which are likely to get documented.
%    \begin{macrocode}
  find_cidmap_file           = data,
  pdf_stream_filter_callback = data,
}
luatexbase.callbacktypes=callbacktypes
%    \end{macrocode}
%
% \begin{macro}{callback.register}
% \changes{v1.0a}{0000/00/00}{Function modified}
%   Save the original function for registering callbacks and prevent the
%   original being used. The original is saved in a place that remains
%   available so other more sophisticated code can override the approach
%   taken by the kernel if desired.
%    \begin{macrocode}
local callback_register = callback_register or callback.register
function callback.register()
  luatexbase_error("Attempt to use callback.register() directly.")
end
%    \end{macrocode}
% \end{macro}
%
% \subsubsection{Handlers}
%
% The handler function is registered into the callback when the
% first function is added to this callback's list. Then, when the callback
% is called, then handler takes care of running all functions in the list.
% When the last function is removed from the callback's list, the handler
% is unregistered.
%
% More precisely, the functions below are used to generate a specialized
% function (closure) for a given callback, which is the actual handler.
%
% Handler for |data| callbacks.
%    \begin{macrocode}
local function data_handler(name)
  return function(data, ...)
    local i
    for _,i in ipairs(callbacklist[name]) do
      data = i.func(data,...)
    end
    return data
  end
end
%    \end{macrocode}
% Handler for |exclusive| callbacks. We can assume |callbacklist[name]| is not
% empty: otherwise, the function wouldn't be registered in the callback any
% more.
%    \begin{macrocode}
local function exclusive_handler(name)
  return function(...)
    return callbacklist[name][1].func(...)
  end
end
%    \end{macrocode}
% Handler for |list| callbacks.
%    \begin{macrocode}
local function list_handler(name)
  return function(head, ...)
    local ret
    local alltrue = true
    local i
    for _,i in ipairs(callbacklist[name]) do
      ret = i.func(head, ...)
      if ret == false then
        luatexbase_warning(
          "Function `i.description' returned false\n"
            .. "in callback `name'"
         )
         break
      end
      if ret ~= true then
        alltrue = false
        head = ret
      end
    end
    return alltrue and true or head
  end
end
%    \end{macrocode}
% Handler for |simple| callbacks.
%    \begin{macrocode}
local function simple_handler(name)
  return function(...)
    local i
    for _,i in ipairs(callbacklist[name]) do
      i.func(...)
    end
  end
end
%    \end{macrocode}
%
% Keep a handlers table for indexed access.
%    \begin{macrocode}
local handlers = {
  [data]      = data_handler,
  [exclusive] = exclusive_handler,
  [list]      = list_handler,
  [simple]    = simple_handler,
}
%    \end{macrocode}
%
% \subsubsection{Public functions for callback management}
%
% Defining user callbacks perhaps should be in package code,
% but impacts on |add_to_callback|.
% If a default function is not required, may may be declared as |false|.
% First we need a list of user callbacks.
%    \begin{macrocode}
local user_callbacks_defaults = { }
%    \end{macrocode}
%
% \begin{macro}{luatexbase.create_callback}
% \changes{v1.0a}{0000/00/00}{Function added}
%   The allocator itself.
%    \begin{macrocode}
local function create_callback(name, ctype, default)
  if not name or
    name == "" or
    callbacktypes[name] or
    not(default == false or  type(default) == "function")
    then
      luatexbase_error("Unable to create callback " .. name)
  end
  user_callbacks_defaults[name] = default
  callbacktypes[name] = types[ctype]
end
luatexbase.create_callback = create_callback
%    \end{macrocode}
% \end{macro}
%
% \begin{macro}{luatexbase.call_callback}
% \changes{v1.0a}{0000/00/00}{Function added}
%  Call a user defined callback. First check arguments.
%    \begin{macrocode}
local function call_callback(name,...)
  if not name or
    name == "" or
    user_callbacks_defaults[name] == nil
    then
        luatexbase_error("Unable to call callback " .. name)
  end
  local l = callbacklist[name]
  local f
  if not l then
    f = user_callbacks_defaults[name]
    if l == false then
	   return nil
	 end
  else
    f = handlers[callbacktypes[name]](name)
  end
  return f(...)
end
luatexbase.call_callback=call_callback
%    \end{macrocode}
% \end{macro}
%
% \begin{macro}{luatexbase.add_to_callback}
% \changes{v1.0a}{0000/00/00}{Function added}
%   Add a function to a callback. First check arguments.
%    \begin{macrocode}
local function add_to_callback(name, func, description)
  if
    not name or
    name == "" or
    not callbacktypes[name] or
    type(func) ~= "function" or
    not description or
    description == "" then
    luatexbase_error(
      "Unable to register callback.\n\n"
        .. "Correct usage:\n"
        .. "add_to_callback(<callback>, <function>, <description>)"
    )
    return
  end
%    \end{macrocode}
%   Then test if this callback is already in use. If not, initialise its list
%   and register the proper handler.
%    \begin{macrocode}
  local l = callbacklist[name]
  if l == nil then
    l = { }
    callbacklist[name] = l
%    \end{macrocode}
% If it is not a user defined callback use the primitive callback register.
%    \begin{macrocode}
    if user_callbacks_defaults[name] == nil then
      callback_register(name, handlers[callbacktypes[name]](name))
    end
  end
%    \end{macrocode}
%  Actually register the function and give an error if more than one
%  |exclusive| one is registered.
%    \begin{macrocode}
  local f = {
    func        = func,
    description = description,
  }
  local priority = #l + 1
  if callbacktypes[name] == exclusive then
    if #l == 1 then
      luatexbase_error(
        "Cannot add second callback to exclusive function `" ..
        name .. "'.")
    end
  end
  table.insert(l, priority, f)
%    \end{macrocode}
%  Keep user informed.
%    \begin{macrocode}
  texio_write_nl(
    "Inserting `" .. description .. "' at position "
      .. priority .. " in `" .. name .. "'."
  )
end
luatexbase.add_to_callback = add_to_callback
%    \end{macrocode}
% \end{macro}
%
% \begin{macro}{luatexbase.remove_from_callback}
% \changes{v1.0a}{0000/00/00}{Function added}
%   Remove a function from a callback. First check arguments.
%    \begin{macrocode}
local function remove_from_callback(name, description)
  if
    not name or
    name == "" or
    not callbacktypes[name] or
    not description or
    description == "" then
    luatexbase_error(
      "Unable to remove function from callback.\n\n"
        .. "Correct usage:\n"
        .. "remove_to_callback(<callback>, <description>)"
    )
    return
  end
  local l = callbacklist[name]
  if not l then
    luatexbase_error(
      "No callback list for `" .. name .. "'.")
  end
%    \end{macrocode}
%  Loop over the callback's function list until we find a matching entry.
%  Remove it and check if the list is empty: if so, unregister the
%   callback handler.
%    \begin{macrocode}
  local index = false
  local i,j
  local cb = {}
  for i,j in ipairs(l) do
    if j.description == description then
      index = i
      break
    end
  end
  if not index then
    luatexbase_error(
      "No callback `" .. description .. "' registered for `" ..
      name .. "'.")
    return
  end
  cb = l[index]
  table.remove(l, index)
  texio_write_nl(
    "Removing  `" .. description .. "' from `" .. name .. "'."
  )
  if #l == 0 then
    callbacklist[name] = nil
    callback_register(name, nil)
  end
  return cb.func,cb.description
end
luatexbase.remove_from_callback = remove_from_callback
%    \end{macrocode}
% \end{macro}
%
% \begin{macro}{luatexbase.in_callback}
% \changes{v1.0a}{0000/00/00}{Function added}
%   Look for a function description in a callback.
%    \begin{macrocode}
local function in_callback(name, description)
  if not name
    or name == ""
    or not callbacktypes[name]
    or not description then
      return false
  end
  local i
  for _, i in pairs(callbacklist[name]) do
    if i.description == description then
      return true
    end
  end
  return false
end
luatexbase.in_callback = in_callback
%    \end{macrocode}
% \end{macro}
%
% \begin{macro}{luatexbase.disable_callback}
% \changes{v1.0a}{0000/00/00}{Function added}
%   As we subvert the engine interface we need to provide a way to access
%   this functionality.
%    \begin{macrocode}
local function disable_callback(name)
  if(callbacklist[name] == nil) then
    callback_register(name, false)
  else
    luatexbase_error("Callback list for " .. name .. " not empty")
  end
end
luatexbase.disable_callback = disable_callback
%    \end{macrocode}
% \end{macro}
%
% \begin{macro}{luatexbase.in_callback}
% \changes{v1.0a}{0000/00/00}{Function added}
%   List the descriptions of functions registed for the given callback.
%    \begin{macrocode}
local function callback_descriptions (name)
  local d = {}
  if not name
    or name == ""
    or not callbacktypes[name]
    then
    return d
  else
  local i
  for k, i in pairs(callbacklist[name] or {}) do
    d[k]= i.description
    end
  end
  return d
end
luatexbase.callback_descriptions =callback_descriptions 
%    \end{macrocode}
% \end{macro}
% \endgroup
%
%    \begin{macrocode}
%</lua>
%    \end{macrocode}
%
% Reset the catcode of |@|.
%    \begin{macrocode}
%<tex>\catcode`\@=\etatcatcode\relax
%    \end{macrocode}
%
%
% \Finale
| this inputs |ltluatex.tex| which inputs
% |etex.src| (or |etex.sty| if used with \LaTeX)
% if it is not already input, and then defines some internal commands to
% allow the \textsf{ltluatex} interface to be defined.
%
% The \textsf{luatexbase} package interface may also be used in plain \TeX,
% as before, by inputting the package |\input luatexbase.sty|. The new
% version of \textsf{luatexbase} is based on this \textsf{ltluatex}
% code but implements a compatibility layer providing the interface
% of the original package.
%
% \section{Lua functionality}
%
% \begingroup
%
% \begingroup\lccode`~=`_
% \lowercase{\endgroup\let~}_
% \catcode`_=12
%
% \subsection{Allocators in Lua}
%
% \DescribeMacro{luatexbase.new_attribute}
% |luatexbase.new_attribute(|\meta{attribute}|)|\\
% Returns an allocation number for the \meta{attribute}, indexed from~$1$.
% The attribute will be initialised with the marker value |-"7FFFFFFF|
% (`unset'). The attribute allocation sequence is shared with the \TeX{}
% code but this function does \emph{not} define a token using
% |\attributedef|.
% The attribute name is recorded in the |attributes| table. A
% metatable is provided so that the table syntax can be used
% consistently for attributes declared in \TeX\ or lua.
%
% \noindent
% \DescribeMacro{luatexbase.new_whatsit}
% |luatexbase.new_whatsit(|\meta{whatsit}|)|\\
% Returns an allocation number for the custom \meta{whatsit}, indexed from~$1$.
%
% \noindent
% \DescribeMacro{luatexbase.new_bytecode}
% |luatexbase.new_bytecode(|\meta{bytecode}|)|\\
% Returns an allocation number for a bytecode register, indexed from~$1$.
% The optional \meta{name} argument is just used for logging.
%
% \noindent
% \DescribeMacro{luatexbase.new_chunkname}
% |luatexbase.new_chunkname(|\meta{chunkname}|)|\\
% Returns an allocation number for a lua chunk name for use with 
% |\directlua| and |\latelua|, indexed from~$1$.
% The number is returned and also \meta{name} argument is added to the
% |lua.name| array at that index.
%
%
% \subsection{Lua access to \TeX{} register numbers}
%
% \DescribeMacro{luatexbase.registernumber}
% |luatexbase.registernumer(|\meta{name}|)|\\
% Sometimes (notably in the case of Lua attributes) it is necessary to
% access a register \emph{by number} that has been allocated by \TeX{}.
% This package provides a function to look up the relevant number
% using Lua\TeX{}'s internal tables. After for example
% |\newattribute\myattrib|, |\myattrib| would be defined by (say)
% |\myattrib=\attribute15|.  |luatexbase.registernumer("myattrib")|
% would then return the register number, $15$ in this case. If the string passed
% as argument does not correspond to a token defined by |\attributedef|,
% |\countdef| or similar commands, the Lua value |false| is returned.
%
% As an example, consider the input:
%\begin{verbatim}
% \newcommand\test[1]{%
% \typeout{#1: \expandafter\meaning\csname#1\endcsname^^J
% \space\space\space\space
% \directlua{tex.write(luatexbase.registernumber("#1") or "bad input")}%
% }}
%
% \test{undefinedrubbish}
%
% \test{space}
%
% \test{hbox}
%
% \test{@MM}
%
% \test{@tempdima}
% \test{@tempdimb}
%
% \test{strutbox}
%
% \test{sixt@@n}
%
% \attrbutedef\myattr=12
% \myattr=200
% \test{myattr}
%
%\end{verbatim}
%
% If the demonstration code is processed with Lua\LaTeX{} then the following
% would be produced in the log and terminal output.
%\begin{verbatim}
% undefinedrubbish: \relax
%      bad input
% space: macro:->
%      bad input
% hbox: \hbox
%      bad input
% @MM: \mathchar"4E20
%      20000
% @tempdima: \dimen14
%      14
% @tempdimb: \dimen15
%      15
% strutbox: \char"B
%      11
% sixt@@n: \char"10
%      16
% myattr: \attribute12
%      12
%\end{verbatim}
%
% Notice how undefined commands, or commands unrelated to registers
% do not produce an error, just return |false| and so print
% |bad input| here. Note also that commands defined by |\newbox| work and
% return the number of the box register even though the actual command
% holding this number is a |\chardef| defined token (there is no
% |\boxdef|).
%
% \subsection{Module utilities}
%
% \DescribeMacro{luatexbase.provides_module}
% |luatexbase.provides_module(|\meta{info}|)|\\
% This function is used by modules to identify themselves; the |info| should be
% a table containing information about the module. The required field
% |name| must contain the name of the module. It is recommended to provide a
% field |date| in the usual \LaTeX{} format |yyyy/mm/dd|. Optional fields
% |version| (a string) and |description| may be used if present. This
% information will be recorded in the log. Other fields are ignored.
%
% \noindent
% \DescribeMacro{luatexbase.module_info}
% \DescribeMacro{luatexbase.module_warning}
% \DescribeMacro{luatexbase.module_error}
% |luatexbase.module_info(|\meta{module}, \meta{text}|)|\\
% |luatexbase.module_warning(|\meta{module}, \meta{text}|)|\\
% |luatexbase.module_error(|\meta{module}, \meta{text}|)|\\
% These functions are similar to \LaTeX{}'s |\PackageError|, |\PackageWarning|
% and |\PackageInfo| in the way they format the output.  No automatic line
% breaking is done, you may still use |\n| as usual for that, and the name of
% the package will be prepended to each output line.
%
% Note that |luatexbase.module_error| raises an actual Lua error with |error()|,
% which currently means a call stack will be dumped. While this may not
% look pretty, at least it provides useful information for tracking the
% error down.
%
% \subsection{Callback management}
%
% \noindent
% \DescribeMacro{luatexbase.add_to_callback}
% |luatexbase.add_to_callback(|^^A
% \meta{callback}, \meta{function}, \meta{description}|)|
% Registers the \meta{function} into the \meta{callback} with a textual
% \meta{description} of the function. Functions are inserted into the callback
% in the order loaded.
%
% \noindent
% \DescribeMacro{luatexbase.remove_from_callback}
% |luatexbase.remove_from_callback(|\meta{callback}, \meta{description}|)|
% Removes the function with \meta{description} from the \meta{callback}.
% The removed function and its description 
% are returned as the results of this function.
%
% \noindent
% \DescribeMacro{luatexbase.in_callback}
% |luatexbase.in_callback(|\meta{callback}, \meta{description}|)|
% Checks if the \meta{description} matches one of the functions added
% to the list for the \meta{callback}, returning a boolean value.
%
% \noindent
% \DescribeMacro{luatexbase.disable_callback}
% |luatexbase.disable_callback(|\meta{callback}|)|
% Sets the \meta{callback} to \texttt{false} as described in the Lua\TeX{}
% manual for the underlying \texttt{callback.register} built-in. Callbacks
% will only be set to false (and thus be skipped entirely) if there are
% no functions registered using the callback.
%
% \noindent
% \DescribeMacro{luatexbase.callback_descriptions}
% A list of the descriptions of functions registered to the specified
% callback is returned. |{}| is returned if there are no functions registered.
%
% \noindent
% \DescribeMacro{luatexbase.create_callback}
% |luatexbase.create_callback(|\meta{name},meta{type},\meta{default}|)|
% Defines a user defined callback. The last argument is a default
% functtion of |false|.
%
% \noindent
% \DescribeMacro{luatexbase.call_callback}
% |luatexbase.create_callback(|\meta{name},\ldots|)|
% Calls a user defined callback with the supplied arguments.
%
% \endgroup
%
% \CheckSum{425}
% \StopEventually{}
%
% \section{Implementation}
%
%    \begin{macrocode}
%<*2ekernel|tex|latexrelease>
%<2ekernel|latexrelease>\ifx\directlua\@undefined\else
%    \end{macrocode}
%
%
% \subsection{Minimum Lua\TeX{} version}
%
% Lua\TeX{} has changed a lot over time. In the kernel support for ancient
% versions is not provided: trying to build a format with a very old binary
% therefore gives some information in the log and loading stops. The cut-off
% selected here relates to the tree-searching behaviour of |require()|:
% from version~0.60, Lua\TeX{} will correctly find Lua files in the |texmf|
% tree without `help'.
%    \begin{macrocode}
%<latexrelease>\IncludeInRelease{2015/11/01}
%<latexrelease>                 {\newluafunction}{LuaTeX}%
\ifnum\luatexversion<60 %
  \wlog{***************************************************}
  \wlog{* LuaTeX version too old for ltluatex support *}
  \wlog{***************************************************}
  \expandafter\endinput
\fi
%    \end{macrocode}
%
% \subsection{Older \LaTeX{}/Plain \TeX\ setup}
% 
%    \begin{macrocode}
%<*tex>
%    \end{macrocode}
%
% Older \LaTeX{] formats don't have the primitives with `native' names:
% sort that out. I fthey already exist this will still be safe.
%    \begin{macrocode}
\directlua{tex.enableprimitives("",tex.extraprimitives("luatex"))}
%    \end{macrocode}
%
%    \begin{macrocode}
\ifx\e@alloc\@undefined
%    \end{macrocode}
%
% In pre-2014 \LaTeX{}, or plain \TeX{}, load |etex.{sty,src}|.
%    \begin{macrocode}
  \ifx\documentclass\@undefined
    \ifx\loccount\@undefined
      \input{etex.src}%
    \fi
    \catcode`\@=11 %
    \outer\expandafter\def\csname newfam\endcsname
                          {\alloc@8\fam\chardef\et@xmaxfam}
  \else
    \RequirePackage{etex}
    \expandafter\def\csname newfam\endcsname
                    {\alloc@8\fam\chardef\et@xmaxfam}
    \expandafter\let\expandafter\new@mathgroup\csname newfam\endcsname
  \fi
%    \end{macrocode}
%
% \subsubsection{Fixes to \texttt{etex.src}/\texttt{etex.sty}}
%
% These could and probably should be made directly in an
% update to etex.src which already has some luatex-specific
% code, but does not define the correct range for luatex.
%
%    \begin{macrocode}
% 2015-07-13 higher range in luatex
\edef \et@xmaxregs {\ifx\directlua\@undefined 32768\else 65536\fi}
% luatex/xetex also allow more math fam
\edef \et@xmaxfam {\ifx\Umathchar\@undefined\sixt@@n\else\@cclvi\fi}
%    \end{macrocode}
%
%    \begin{macrocode}
\count 270=\et@xmaxregs % locally allocates \count registers
\count 271=\et@xmaxregs % ditto for \dimen registers
\count 272=\et@xmaxregs % ditto for \skip registers
\count 273=\et@xmaxregs % ditto for \muskip registers
\count 274=\et@xmaxregs % ditto for \box registers
\count 275=\et@xmaxregs % ditto for \toks registers
\count 276=\et@xmaxregs % ditto for \marks classes
%    \end{macrocode}
%
% and 256 or 16 fam. (Done above due to plain/\LaTeX\ differences in
% \textsf{ltluatex}.)
%    \begin{macrocode}
% \outer\def\newfam{\alloc@8\fam\chardef\et@xmaxfam}
%    \end{macrocode}
%
% End of proposed changes to \texttt{etex.src}
%
% \subsubsection{luatex specific settings}
% 
% Switch to global cf |luatex.sty| to leave room for inserts
% not really needed for luatex but possibly most compatible
% with existing use.
%    \begin{macrocode}
\expandafter\let\csname newcount\expandafter\expandafter\endcsname
                \csname globcount\endcsname
\expandafter\let\csname newdimen\expandafter\expandafter\endcsname
                \csname globdimen\endcsname
\expandafter\let\csname newskip\expandafter\expandafter\endcsname
                \csname globskip\endcsname
\expandafter\let\csname newbox\expandafter\expandafter\endcsname
                \csname globbox\endcsname
%    \end{macrocode}
%
% Define|\e@alloc| as in latex (the existing macros in |etex.src|
% hard to extend to further register types as they assume specific
% 26x and 27x count range. For compatibility the existing register
% allocation is not changed.
%
%    \begin{macrocode}
\chardef\e@alloc@top=65535
\let\e@alloc@chardef\chardef
%    \end{macrocode}
%
%    \begin{macrocode}
\def\e@alloc#1#2#3#4#5#6{%
  \global\advance#3\@ne
  \e@ch@ck{#3}{#4}{#5}#1%
  \allocationnumber#3\relax
  \global#2#6\allocationnumber
  \wlog{\string#6=\string#1\the\allocationnumber}}%
%    \end{macrocode}
%
%    \begin{macrocode}
\gdef\e@ch@ck#1#2#3#4{%
  \ifnum#1<#2\else
    \ifnum#1=#2\relax
      #1\@cclvi
      \ifx\count#4\advance#1 10 \fi
    \fi
    \ifnum#1<#3\relax
    \else
      \errmessage{No room for a new \string#4}%
    \fi
  \fi}%
%    \end{macrocode}
%
% Two simple \LaTeX\ macros used in |ltlatex.sty|.
%    \begin{macrocode}
\long\def\@gobble#1{}
\long\def\@firstofone#1{#1}
%    \end{macrocode}
%
%    \begin{macrocode}
% Fix up allocations not to clash with |etex.src|.
%    \end{macrocode}
%
%    \begin{macrocode}
\expandafter\csname newcount\endcsname\e@alloc@attribute@count
\expandafter\csname newcount\endcsname\e@alloc@ccodetable@count
\expandafter\csname newcount\endcsname\e@alloc@luafunction@count
\expandafter\csname newcount\endcsname\e@alloc@whatsit@count
%    \end{macrocode}
%
% End of conditional setup for plain \TeX\ / old \LaTeX.
%    \begin{macrocode}
\fi
%</tex>
%    \end{macrocode}
%
%
% \subsection{Attributes}
%
% \begin{macro}{\newattribute}
% \changes{v1.0a}{0000/00/00}{Macro added}
%   As is generally the case for the Lua\TeX{} registers we start here
%   from~$1$. Notably, some code assumes that |\attribute0| is never used so
%   this is important in this case.
%    \begin{macrocode}
\ifx\e@alloc@attribute@count\@undefined
  \countdef\e@alloc@attribute@count=258
\fi
\def\newattribute#1{%
  \e@alloc\attribute\attributedef
    \e@alloc@attribute@count\m@ne\e@alloc@top#1%
}
\e@alloc@attribute@count=\z@
%    \end{macrocode}
% \end{macro}
%
% \begin{macro}{\setattribute}
% \begin{macro}{\unsetattribute}
%   Handy utilities.
%    \begin{macrocode}
\def\setattribute#1#2{#1=\numexpr#2\relax}
\def\unsetattribute#1{#1=-"7FFFFFFF\relax}
%    \end{macrocode}
% \end{macro}
% \end{macro}
%
% \subsection{Category code tables}
%
% \begin{macro}{\newcatcodetable}
% \changes{v1.0a}{0000/00/00}{Macro added}
%   Category code tables are allocated with a limit half of that used by Lua\TeX{}
%   for everything else. At the end of allocation there needs to be an
%   initialisation step. Table~$0$ is already taken (it's the global one for
%   current use) so the allocation starts at~$1$.
%    \begin{macrocode}
\ifx\e@alloc@ccodetable@count\@undefined
  \countdef\e@alloc@ccodetable@count=259
\fi
\def\newcatcodetable#1{%
  \e@alloc\catcodetable\chardef
    \e@alloc@ccodetable@count\m@ne{"8000}#1%
  \initcatcodetable\allocationnumber
}
\e@alloc@ccodetable@count=\z@
%    \end{macrocode}
% \end{macro}
%
% \begin{macro}{\catcodetable@initex}
% \changes{v1.0a}{0000/00/00}{Macro added}
% \begin{macro}{\catcodetable@string}
% \changes{v1.0a}{0000/00/00}{Macro added}
% \begin{macro}{\catcodetable@latex}
% \changes{v1.0a}{0000/00/00}{Macro added}
% \begin{macro}{\catcodetable@atletter}
% \changes{v1.0a}{0000/00/00}{Macro added}
%   Save a small set of standard tables. The Unicode data is read
%   here in a group avoiding any global definitions: that needs a bit
%   of effort so that in package/plain mode there is no effect on any
%   settings already in force.
%    \begin{macrocode}
\newcatcodetable\catcodetable@initex
\newcatcodetable\catcodetable@string
\begingroup
  \def\setrangecatcode#1#2#3{%
    \ifnum#1>#2 %
      \expandafter\@gobble
    \else
      \expandafter\@firstofone
    \fi
      {%
        \catcode#1=#3 %
        \expandafter\setrangecatcode\expandafter
          {\number\numexpr#1 + 1\relax}{#2}{#3}
      }%
  }
  \@firstofone{%
    \catcodetable\catcodetable@initex
      \catcode0=12 %
      \catcode12=12 %
      \catcode37=12 %
      \setrangecatcode{65}{90}{12}%
      \setrangecatcode{97}{122}{12}%
      \catcode92=12 %
      \catcode127=12 %
      \savecatcodetable\catcodetable@string
    \endgroup
  }%
\newcatcodetable\catcodetable@latex
\newcatcodetable\catcodetable@atletter
\begingroup
  \let\ENDGROUP\endgroup
  \let\begingroup\relax
  \let\endgroup\relax
  \let\global\relax
  \let\gdef\def
  %% This is the file `unicode-letters.def',
%% generated using the script ltunicode.dtx.
%%
%% The data here are derived from the files
%% - UnicodeData.txt 
%%   MD5 sum 01C77522E09E6A60D15C0E9983C06975
%% - EastAsianWidth.txt 
%%   Version 7.0.0 dated 2014-02-28, 23:15:00
%%   MD5 sum 1148EADA16B2123FA7D3DA310D52C4C6
%% - LineBreak.txt 
%%   Version 7.0.0 dated 2014-02-28, 23:15:00
%%   MD5 sum C95A5E60B2B527F4E77286F9C9CF09D1
%% which are maintained by the Unicode Consortium.
%%
%% Generated on 2015-08-06. 
%%
%% Copyright 2014-2015
%% The LaTeX3 Project and any individual authors listed elsewhere
%% in this file.
%%
%% This file is part of the LaTeX base system.
%% -------------------------------------------
%%
%% It may be distributed and/or modified under the
%% conditions of the LaTeX Project Public License, either version 1.3c
%% of this license or (at your option) any later version.
%% The latest version of this license is in
%%    http://www.latex-project.org/lppl.txt
%% and version 1.3c or later is part of all distributions of LaTeX
%% version 2005/12/01 or later.
%%
%% This file has the LPPL maintenance status "maintained".
%%
%% The list of all files belonging to the LaTeX base distribution is
%% given in the file `manifest.txt'. See also `legal.txt' for additional
%% information.
\begingroup
  \def\C#1 #2 #3 {%
    \XeTeXcheck{#1}%
    \global\uccode"#1="#2 %
    \global\lccode"#1="#3 %
  }
  \def\L#1 #2 #3 {%
    \C #1 #2 #3 %
    \global\catcode"#1=11 %
    \ifnum"#1="#3 %
    \else
      \global\sfcode"#1=999 %
    \fi
    \ifnum"#1<"10000 %
      \global\Umathcode"#1="7"01"#1 %
    \else
      \global\Umathcode"#1="0"01"#1 %
    \fi
  }
  \def\l#1 {\L#1 #1 #1 }
  \ifx\Umathcode\undefined
    \let\Umathcode\XeTeXmathcode
  \fi
  \def\XeTeXcheck#1{}
  \ifx\XeTeXversion\undefined
  \else
    \def\XeTeXcheck.#1.#2-#3\relax{#1}
     \ifnum\expandafter\XeTeXcheck\XeTeXrevision.-\relax>996 %
       \def\XeTeXcheck#1{}
     \else
       \def\XeTeXcheck#1{%
          \ifnum"#1>"FFFF %
            \long\def\XeTeXcheck##1\endgroup{\endgroup}
            \expandafter\XeTeXcheck
          \fi
       }
     \fi
  \fi
  \L 0041 0041 0061
  \L 0042 0042 0062
  \L 0043 0043 0063
  \L 0044 0044 0064
  \L 0045 0045 0065
  \L 0046 0046 0066
  \L 0047 0047 0067
  \L 0048 0048 0068
  \L 0049 0049 0069
  \L 004A 004A 006A
  \L 004B 004B 006B
  \L 004C 004C 006C
  \L 004D 004D 006D
  \L 004E 004E 006E
  \L 004F 004F 006F
  \L 0050 0050 0070
  \L 0051 0051 0071
  \L 0052 0052 0072
  \L 0053 0053 0073
  \L 0054 0054 0074
  \L 0055 0055 0075
  \L 0056 0056 0076
  \L 0057 0057 0077
  \L 0058 0058 0078
  \L 0059 0059 0079
  \L 005A 005A 007A
  \L 0061 0041 0061
  \L 0062 0042 0062
  \L 0063 0043 0063
  \L 0064 0044 0064
  \L 0065 0045 0065
  \L 0066 0046 0066
  \L 0067 0047 0067
  \L 0068 0048 0068
  \L 0069 0049 0069
  \L 006A 004A 006A
  \L 006B 004B 006B
  \L 006C 004C 006C
  \L 006D 004D 006D
  \L 006E 004E 006E
  \L 006F 004F 006F
  \L 0070 0050 0070
  \L 0071 0051 0071
  \L 0072 0052 0072
  \L 0073 0053 0073
  \L 0074 0054 0074
  \L 0075 0055 0075
  \L 0076 0056 0076
  \L 0077 0057 0077
  \L 0078 0058 0078
  \L 0079 0059 0079
  \L 007A 005A 007A
  \l 00AA
  \L 00B5 039C 00B5
  \l 00BA
  \L 00C0 00C0 00E0
  \L 00C1 00C1 00E1
  \L 00C2 00C2 00E2
  \L 00C3 00C3 00E3
  \L 00C4 00C4 00E4
  \L 00C5 00C5 00E5
  \L 00C6 00C6 00E6
  \L 00C7 00C7 00E7
  \L 00C8 00C8 00E8
  \L 00C9 00C9 00E9
  \L 00CA 00CA 00EA
  \L 00CB 00CB 00EB
  \L 00CC 00CC 00EC
  \L 00CD 00CD 00ED
  \L 00CE 00CE 00EE
  \L 00CF 00CF 00EF
  \L 00D0 00D0 00F0
  \L 00D1 00D1 00F1
  \L 00D2 00D2 00F2
  \L 00D3 00D3 00F3
  \L 00D4 00D4 00F4
  \L 00D5 00D5 00F5
  \L 00D6 00D6 00F6
  \L 00D8 00D8 00F8
  \L 00D9 00D9 00F9
  \L 00DA 00DA 00FA
  \L 00DB 00DB 00FB
  \L 00DC 00DC 00FC
  \L 00DD 00DD 00FD
  \L 00DE 00DE 00FE
  \l 00DF
  \L 00E0 00C0 00E0
  \L 00E1 00C1 00E1
  \L 00E2 00C2 00E2
  \L 00E3 00C3 00E3
  \L 00E4 00C4 00E4
  \L 00E5 00C5 00E5
  \L 00E6 00C6 00E6
  \L 00E7 00C7 00E7
  \L 00E8 00C8 00E8
  \L 00E9 00C9 00E9
  \L 00EA 00CA 00EA
  \L 00EB 00CB 00EB
  \L 00EC 00CC 00EC
  \L 00ED 00CD 00ED
  \L 00EE 00CE 00EE
  \L 00EF 00CF 00EF
  \L 00F0 00D0 00F0
  \L 00F1 00D1 00F1
  \L 00F2 00D2 00F2
  \L 00F3 00D3 00F3
  \L 00F4 00D4 00F4
  \L 00F5 00D5 00F5
  \L 00F6 00D6 00F6
  \L 00F8 00D8 00F8
  \L 00F9 00D9 00F9
  \L 00FA 00DA 00FA
  \L 00FB 00DB 00FB
  \L 00FC 00DC 00FC
  \L 00FD 00DD 00FD
  \L 00FE 00DE 00FE
  \L 00FF 0178 00FF
  \L 0100 0100 0101
  \L 0101 0100 0101
  \L 0102 0102 0103
  \L 0103 0102 0103
  \L 0104 0104 0105
  \L 0105 0104 0105
  \L 0106 0106 0107
  \L 0107 0106 0107
  \L 0108 0108 0109
  \L 0109 0108 0109
  \L 010A 010A 010B
  \L 010B 010A 010B
  \L 010C 010C 010D
  \L 010D 010C 010D
  \L 010E 010E 010F
  \L 010F 010E 010F
  \L 0110 0110 0111
  \L 0111 0110 0111
  \L 0112 0112 0113
  \L 0113 0112 0113
  \L 0114 0114 0115
  \L 0115 0114 0115
  \L 0116 0116 0117
  \L 0117 0116 0117
  \L 0118 0118 0119
  \L 0119 0118 0119
  \L 011A 011A 011B
  \L 011B 011A 011B
  \L 011C 011C 011D
  \L 011D 011C 011D
  \L 011E 011E 011F
  \L 011F 011E 011F
  \L 0120 0120 0121
  \L 0121 0120 0121
  \L 0122 0122 0123
  \L 0123 0122 0123
  \L 0124 0124 0125
  \L 0125 0124 0125
  \L 0126 0126 0127
  \L 0127 0126 0127
  \L 0128 0128 0129
  \L 0129 0128 0129
  \L 012A 012A 012B
  \L 012B 012A 012B
  \L 012C 012C 012D
  \L 012D 012C 012D
  \L 012E 012E 012F
  \L 012F 012E 012F
  \L 0130 0130 0069
  \L 0131 0049 0131
  \L 0132 0132 0133
  \L 0133 0132 0133
  \L 0134 0134 0135
  \L 0135 0134 0135
  \L 0136 0136 0137
  \L 0137 0136 0137
  \l 0138
  \L 0139 0139 013A
  \L 013A 0139 013A
  \L 013B 013B 013C
  \L 013C 013B 013C
  \L 013D 013D 013E
  \L 013E 013D 013E
  \L 013F 013F 0140
  \L 0140 013F 0140
  \L 0141 0141 0142
  \L 0142 0141 0142
  \L 0143 0143 0144
  \L 0144 0143 0144
  \L 0145 0145 0146
  \L 0146 0145 0146
  \L 0147 0147 0148
  \L 0148 0147 0148
  \l 0149
  \L 014A 014A 014B
  \L 014B 014A 014B
  \L 014C 014C 014D
  \L 014D 014C 014D
  \L 014E 014E 014F
  \L 014F 014E 014F
  \L 0150 0150 0151
  \L 0151 0150 0151
  \L 0152 0152 0153
  \L 0153 0152 0153
  \L 0154 0154 0155
  \L 0155 0154 0155
  \L 0156 0156 0157
  \L 0157 0156 0157
  \L 0158 0158 0159
  \L 0159 0158 0159
  \L 015A 015A 015B
  \L 015B 015A 015B
  \L 015C 015C 015D
  \L 015D 015C 015D
  \L 015E 015E 015F
  \L 015F 015E 015F
  \L 0160 0160 0161
  \L 0161 0160 0161
  \L 0162 0162 0163
  \L 0163 0162 0163
  \L 0164 0164 0165
  \L 0165 0164 0165
  \L 0166 0166 0167
  \L 0167 0166 0167
  \L 0168 0168 0169
  \L 0169 0168 0169
  \L 016A 016A 016B
  \L 016B 016A 016B
  \L 016C 016C 016D
  \L 016D 016C 016D
  \L 016E 016E 016F
  \L 016F 016E 016F
  \L 0170 0170 0171
  \L 0171 0170 0171
  \L 0172 0172 0173
  \L 0173 0172 0173
  \L 0174 0174 0175
  \L 0175 0174 0175
  \L 0176 0176 0177
  \L 0177 0176 0177
  \L 0178 0178 00FF
  \L 0179 0179 017A
  \L 017A 0179 017A
  \L 017B 017B 017C
  \L 017C 017B 017C
  \L 017D 017D 017E
  \L 017E 017D 017E
  \L 017F 0053 017F
  \L 0180 0243 0180
  \L 0181 0181 0253
  \L 0182 0182 0183
  \L 0183 0182 0183
  \L 0184 0184 0185
  \L 0185 0184 0185
  \L 0186 0186 0254
  \L 0187 0187 0188
  \L 0188 0187 0188
  \L 0189 0189 0256
  \L 018A 018A 0257
  \L 018B 018B 018C
  \L 018C 018B 018C
  \l 018D
  \L 018E 018E 01DD
  \L 018F 018F 0259
  \L 0190 0190 025B
  \L 0191 0191 0192
  \L 0192 0191 0192
  \L 0193 0193 0260
  \L 0194 0194 0263
  \L 0195 01F6 0195
  \L 0196 0196 0269
  \L 0197 0197 0268
  \L 0198 0198 0199
  \L 0199 0198 0199
  \L 019A 023D 019A
  \l 019B
  \L 019C 019C 026F
  \L 019D 019D 0272
  \L 019E 0220 019E
  \L 019F 019F 0275
  \L 01A0 01A0 01A1
  \L 01A1 01A0 01A1
  \L 01A2 01A2 01A3
  \L 01A3 01A2 01A3
  \L 01A4 01A4 01A5
  \L 01A5 01A4 01A5
  \L 01A6 01A6 0280
  \L 01A7 01A7 01A8
  \L 01A8 01A7 01A8
  \L 01A9 01A9 0283
  \l 01AA
  \l 01AB
  \L 01AC 01AC 01AD
  \L 01AD 01AC 01AD
  \L 01AE 01AE 0288
  \L 01AF 01AF 01B0
  \L 01B0 01AF 01B0
  \L 01B1 01B1 028A
  \L 01B2 01B2 028B
  \L 01B3 01B3 01B4
  \L 01B4 01B3 01B4
  \L 01B5 01B5 01B6
  \L 01B6 01B5 01B6
  \L 01B7 01B7 0292
  \L 01B8 01B8 01B9
  \L 01B9 01B8 01B9
  \l 01BA
  \l 01BB
  \L 01BC 01BC 01BD
  \L 01BD 01BC 01BD
  \l 01BE
  \L 01BF 01F7 01BF
  \l 01C0
  \l 01C1
  \l 01C2
  \l 01C3
  \L 01C4 01C4 01C6
  \L 01C5 01C4 01C6
  \L 01C6 01C4 01C6
  \L 01C7 01C7 01C9
  \L 01C8 01C7 01C9
  \L 01C9 01C7 01C9
  \L 01CA 01CA 01CC
  \L 01CB 01CA 01CC
  \L 01CC 01CA 01CC
  \L 01CD 01CD 01CE
  \L 01CE 01CD 01CE
  \L 01CF 01CF 01D0
  \L 01D0 01CF 01D0
  \L 01D1 01D1 01D2
  \L 01D2 01D1 01D2
  \L 01D3 01D3 01D4
  \L 01D4 01D3 01D4
  \L 01D5 01D5 01D6
  \L 01D6 01D5 01D6
  \L 01D7 01D7 01D8
  \L 01D8 01D7 01D8
  \L 01D9 01D9 01DA
  \L 01DA 01D9 01DA
  \L 01DB 01DB 01DC
  \L 01DC 01DB 01DC
  \L 01DD 018E 01DD
  \L 01DE 01DE 01DF
  \L 01DF 01DE 01DF
  \L 01E0 01E0 01E1
  \L 01E1 01E0 01E1
  \L 01E2 01E2 01E3
  \L 01E3 01E2 01E3
  \L 01E4 01E4 01E5
  \L 01E5 01E4 01E5
  \L 01E6 01E6 01E7
  \L 01E7 01E6 01E7
  \L 01E8 01E8 01E9
  \L 01E9 01E8 01E9
  \L 01EA 01EA 01EB
  \L 01EB 01EA 01EB
  \L 01EC 01EC 01ED
  \L 01ED 01EC 01ED
  \L 01EE 01EE 01EF
  \L 01EF 01EE 01EF
  \l 01F0
  \L 01F1 01F1 01F3
  \L 01F2 01F1 01F3
  \L 01F3 01F1 01F3
  \L 01F4 01F4 01F5
  \L 01F5 01F4 01F5
  \L 01F6 01F6 0195
  \L 01F7 01F7 01BF
  \L 01F8 01F8 01F9
  \L 01F9 01F8 01F9
  \L 01FA 01FA 01FB
  \L 01FB 01FA 01FB
  \L 01FC 01FC 01FD
  \L 01FD 01FC 01FD
  \L 01FE 01FE 01FF
  \L 01FF 01FE 01FF
  \L 0200 0200 0201
  \L 0201 0200 0201
  \L 0202 0202 0203
  \L 0203 0202 0203
  \L 0204 0204 0205
  \L 0205 0204 0205
  \L 0206 0206 0207
  \L 0207 0206 0207
  \L 0208 0208 0209
  \L 0209 0208 0209
  \L 020A 020A 020B
  \L 020B 020A 020B
  \L 020C 020C 020D
  \L 020D 020C 020D
  \L 020E 020E 020F
  \L 020F 020E 020F
  \L 0210 0210 0211
  \L 0211 0210 0211
  \L 0212 0212 0213
  \L 0213 0212 0213
  \L 0214 0214 0215
  \L 0215 0214 0215
  \L 0216 0216 0217
  \L 0217 0216 0217
  \L 0218 0218 0219
  \L 0219 0218 0219
  \L 021A 021A 021B
  \L 021B 021A 021B
  \L 021C 021C 021D
  \L 021D 021C 021D
  \L 021E 021E 021F
  \L 021F 021E 021F
  \L 0220 0220 019E
  \l 0221
  \L 0222 0222 0223
  \L 0223 0222 0223
  \L 0224 0224 0225
  \L 0225 0224 0225
  \L 0226 0226 0227
  \L 0227 0226 0227
  \L 0228 0228 0229
  \L 0229 0228 0229
  \L 022A 022A 022B
  \L 022B 022A 022B
  \L 022C 022C 022D
  \L 022D 022C 022D
  \L 022E 022E 022F
  \L 022F 022E 022F
  \L 0230 0230 0231
  \L 0231 0230 0231
  \L 0232 0232 0233
  \L 0233 0232 0233
  \l 0234
  \l 0235
  \l 0236
  \l 0237
  \l 0238
  \l 0239
  \L 023A 023A 2C65
  \L 023B 023B 023C
  \L 023C 023B 023C
  \L 023D 023D 019A
  \L 023E 023E 2C66
  \L 023F 2C7E 023F
  \L 0240 2C7F 0240
  \L 0241 0241 0242
  \L 0242 0241 0242
  \L 0243 0243 0180
  \L 0244 0244 0289
  \L 0245 0245 028C
  \L 0246 0246 0247
  \L 0247 0246 0247
  \L 0248 0248 0249
  \L 0249 0248 0249
  \L 024A 024A 024B
  \L 024B 024A 024B
  \L 024C 024C 024D
  \L 024D 024C 024D
  \L 024E 024E 024F
  \L 024F 024E 024F
  \L 0250 2C6F 0250
  \L 0251 2C6D 0251
  \L 0252 2C70 0252
  \L 0253 0181 0253
  \L 0254 0186 0254
  \l 0255
  \L 0256 0189 0256
  \L 0257 018A 0257
  \l 0258
  \L 0259 018F 0259
  \l 025A
  \L 025B 0190 025B
  \L 025C A7AB 025C
  \l 025D
  \l 025E
  \l 025F
  \L 0260 0193 0260
  \L 0261 A7AC 0261
  \l 0262
  \L 0263 0194 0263
  \l 0264
  \L 0265 A78D 0265
  \L 0266 A7AA 0266
  \l 0267
  \L 0268 0197 0268
  \L 0269 0196 0269
  \l 026A
  \L 026B 2C62 026B
  \L 026C A7AD 026C
  \l 026D
  \l 026E
  \L 026F 019C 026F
  \l 0270
  \L 0271 2C6E 0271
  \L 0272 019D 0272
  \l 0273
  \l 0274
  \L 0275 019F 0275
  \l 0276
  \l 0277
  \l 0278
  \l 0279
  \l 027A
  \l 027B
  \l 027C
  \L 027D 2C64 027D
  \l 027E
  \l 027F
  \L 0280 01A6 0280
  \l 0281
  \l 0282
  \L 0283 01A9 0283
  \l 0284
  \l 0285
  \l 0286
  \L 0287 A7B1 0287
  \L 0288 01AE 0288
  \L 0289 0244 0289
  \L 028A 01B1 028A
  \L 028B 01B2 028B
  \L 028C 0245 028C
  \l 028D
  \l 028E
  \l 028F
  \l 0290
  \l 0291
  \L 0292 01B7 0292
  \l 0293
  \l 0294
  \l 0295
  \l 0296
  \l 0297
  \l 0298
  \l 0299
  \l 029A
  \l 029B
  \l 029C
  \l 029D
  \L 029E A7B0 029E
  \l 029F
  \l 02A0
  \l 02A1
  \l 02A2
  \l 02A3
  \l 02A4
  \l 02A5
  \l 02A6
  \l 02A7
  \l 02A8
  \l 02A9
  \l 02AA
  \l 02AB
  \l 02AC
  \l 02AD
  \l 02AE
  \l 02AF
  \l 02B0
  \l 02B1
  \l 02B2
  \l 02B3
  \l 02B4
  \l 02B5
  \l 02B6
  \l 02B7
  \l 02B8
  \l 02B9
  \l 02BA
  \l 02BB
  \l 02BC
  \l 02BD
  \l 02BE
  \l 02BF
  \l 02C0
  \l 02C1
  \l 02C6
  \l 02C7
  \l 02C8
  \l 02C9
  \l 02CA
  \l 02CB
  \l 02CC
  \l 02CD
  \l 02CE
  \l 02CF
  \l 02D0
  \l 02D1
  \l 02E0
  \l 02E1
  \l 02E2
  \l 02E3
  \l 02E4
  \l 02EC
  \l 02EE
  \l 0300
  \l 0301
  \l 0302
  \l 0303
  \l 0304
  \l 0305
  \l 0306
  \l 0307
  \l 0308
  \l 0309
  \l 030A
  \l 030B
  \l 030C
  \l 030D
  \l 030E
  \l 030F
  \l 0310
  \l 0311
  \l 0312
  \l 0313
  \l 0314
  \l 0315
  \l 0316
  \l 0317
  \l 0318
  \l 0319
  \l 031A
  \l 031B
  \l 031C
  \l 031D
  \l 031E
  \l 031F
  \l 0320
  \l 0321
  \l 0322
  \l 0323
  \l 0324
  \l 0325
  \l 0326
  \l 0327
  \l 0328
  \l 0329
  \l 032A
  \l 032B
  \l 032C
  \l 032D
  \l 032E
  \l 032F
  \l 0330
  \l 0331
  \l 0332
  \l 0333
  \l 0334
  \l 0335
  \l 0336
  \l 0337
  \l 0338
  \l 0339
  \l 033A
  \l 033B
  \l 033C
  \l 033D
  \l 033E
  \l 033F
  \l 0340
  \l 0341
  \l 0342
  \l 0343
  \l 0344
  \L 0345 0399 0345
  \l 0346
  \l 0347
  \l 0348
  \l 0349
  \l 034A
  \l 034B
  \l 034C
  \l 034D
  \l 034E
  \l 034F
  \l 0350
  \l 0351
  \l 0352
  \l 0353
  \l 0354
  \l 0355
  \l 0356
  \l 0357
  \l 0358
  \l 0359
  \l 035A
  \l 035B
  \l 035C
  \l 035D
  \l 035E
  \l 035F
  \l 0360
  \l 0361
  \l 0362
  \l 0363
  \l 0364
  \l 0365
  \l 0366
  \l 0367
  \l 0368
  \l 0369
  \l 036A
  \l 036B
  \l 036C
  \l 036D
  \l 036E
  \l 036F
  \L 0370 0370 0371
  \L 0371 0370 0371
  \L 0372 0372 0373
  \L 0373 0372 0373
  \l 0374
  \L 0376 0376 0377
  \L 0377 0376 0377
  \l 037A
  \L 037B 03FD 037B
  \L 037C 03FE 037C
  \L 037D 03FF 037D
  \L 037F 037F 03F3
  \L 0386 0386 03AC
  \L 0388 0388 03AD
  \L 0389 0389 03AE
  \L 038A 038A 03AF
  \L 038C 038C 03CC
  \L 038E 038E 03CD
  \L 038F 038F 03CE
  \l 0390
  \L 0391 0391 03B1
  \L 0392 0392 03B2
  \L 0393 0393 03B3
  \L 0394 0394 03B4
  \L 0395 0395 03B5
  \L 0396 0396 03B6
  \L 0397 0397 03B7
  \L 0398 0398 03B8
  \L 0399 0399 03B9
  \L 039A 039A 03BA
  \L 039B 039B 03BB
  \L 039C 039C 03BC
  \L 039D 039D 03BD
  \L 039E 039E 03BE
  \L 039F 039F 03BF
  \L 03A0 03A0 03C0
  \L 03A1 03A1 03C1
  \L 03A3 03A3 03C3
  \L 03A4 03A4 03C4
  \L 03A5 03A5 03C5
  \L 03A6 03A6 03C6
  \L 03A7 03A7 03C7
  \L 03A8 03A8 03C8
  \L 03A9 03A9 03C9
  \L 03AA 03AA 03CA
  \L 03AB 03AB 03CB
  \L 03AC 0386 03AC
  \L 03AD 0388 03AD
  \L 03AE 0389 03AE
  \L 03AF 038A 03AF
  \l 03B0
  \L 03B1 0391 03B1
  \L 03B2 0392 03B2
  \L 03B3 0393 03B3
  \L 03B4 0394 03B4
  \L 03B5 0395 03B5
  \L 03B6 0396 03B6
  \L 03B7 0397 03B7
  \L 03B8 0398 03B8
  \L 03B9 0399 03B9
  \L 03BA 039A 03BA
  \L 03BB 039B 03BB
  \L 03BC 039C 03BC
  \L 03BD 039D 03BD
  \L 03BE 039E 03BE
  \L 03BF 039F 03BF
  \L 03C0 03A0 03C0
  \L 03C1 03A1 03C1
  \L 03C2 03A3 03C2
  \L 03C3 03A3 03C3
  \L 03C4 03A4 03C4
  \L 03C5 03A5 03C5
  \L 03C6 03A6 03C6
  \L 03C7 03A7 03C7
  \L 03C8 03A8 03C8
  \L 03C9 03A9 03C9
  \L 03CA 03AA 03CA
  \L 03CB 03AB 03CB
  \L 03CC 038C 03CC
  \L 03CD 038E 03CD
  \L 03CE 038F 03CE
  \L 03CF 03CF 03D7
  \L 03D0 0392 03D0
  \L 03D1 0398 03D1
  \l 03D2
  \l 03D3
  \l 03D4
  \L 03D5 03A6 03D5
  \L 03D6 03A0 03D6
  \L 03D7 03CF 03D7
  \L 03D8 03D8 03D9
  \L 03D9 03D8 03D9
  \L 03DA 03DA 03DB
  \L 03DB 03DA 03DB
  \L 03DC 03DC 03DD
  \L 03DD 03DC 03DD
  \L 03DE 03DE 03DF
  \L 03DF 03DE 03DF
  \L 03E0 03E0 03E1
  \L 03E1 03E0 03E1
  \L 03E2 03E2 03E3
  \L 03E3 03E2 03E3
  \L 03E4 03E4 03E5
  \L 03E5 03E4 03E5
  \L 03E6 03E6 03E7
  \L 03E7 03E6 03E7
  \L 03E8 03E8 03E9
  \L 03E9 03E8 03E9
  \L 03EA 03EA 03EB
  \L 03EB 03EA 03EB
  \L 03EC 03EC 03ED
  \L 03ED 03EC 03ED
  \L 03EE 03EE 03EF
  \L 03EF 03EE 03EF
  \L 03F0 039A 03F0
  \L 03F1 03A1 03F1
  \L 03F2 03F9 03F2
  \L 03F3 037F 03F3
  \L 03F4 03F4 03B8
  \L 03F5 0395 03F5
  \L 03F7 03F7 03F8
  \L 03F8 03F7 03F8
  \L 03F9 03F9 03F2
  \L 03FA 03FA 03FB
  \L 03FB 03FA 03FB
  \l 03FC
  \L 03FD 03FD 037B
  \L 03FE 03FE 037C
  \L 03FF 03FF 037D
  \L 0400 0400 0450
  \L 0401 0401 0451
  \L 0402 0402 0452
  \L 0403 0403 0453
  \L 0404 0404 0454
  \L 0405 0405 0455
  \L 0406 0406 0456
  \L 0407 0407 0457
  \L 0408 0408 0458
  \L 0409 0409 0459
  \L 040A 040A 045A
  \L 040B 040B 045B
  \L 040C 040C 045C
  \L 040D 040D 045D
  \L 040E 040E 045E
  \L 040F 040F 045F
  \L 0410 0410 0430
  \L 0411 0411 0431
  \L 0412 0412 0432
  \L 0413 0413 0433
  \L 0414 0414 0434
  \L 0415 0415 0435
  \L 0416 0416 0436
  \L 0417 0417 0437
  \L 0418 0418 0438
  \L 0419 0419 0439
  \L 041A 041A 043A
  \L 041B 041B 043B
  \L 041C 041C 043C
  \L 041D 041D 043D
  \L 041E 041E 043E
  \L 041F 041F 043F
  \L 0420 0420 0440
  \L 0421 0421 0441
  \L 0422 0422 0442
  \L 0423 0423 0443
  \L 0424 0424 0444
  \L 0425 0425 0445
  \L 0426 0426 0446
  \L 0427 0427 0447
  \L 0428 0428 0448
  \L 0429 0429 0449
  \L 042A 042A 044A
  \L 042B 042B 044B
  \L 042C 042C 044C
  \L 042D 042D 044D
  \L 042E 042E 044E
  \L 042F 042F 044F
  \L 0430 0410 0430
  \L 0431 0411 0431
  \L 0432 0412 0432
  \L 0433 0413 0433
  \L 0434 0414 0434
  \L 0435 0415 0435
  \L 0436 0416 0436
  \L 0437 0417 0437
  \L 0438 0418 0438
  \L 0439 0419 0439
  \L 043A 041A 043A
  \L 043B 041B 043B
  \L 043C 041C 043C
  \L 043D 041D 043D
  \L 043E 041E 043E
  \L 043F 041F 043F
  \L 0440 0420 0440
  \L 0441 0421 0441
  \L 0442 0422 0442
  \L 0443 0423 0443
  \L 0444 0424 0444
  \L 0445 0425 0445
  \L 0446 0426 0446
  \L 0447 0427 0447
  \L 0448 0428 0448
  \L 0449 0429 0449
  \L 044A 042A 044A
  \L 044B 042B 044B
  \L 044C 042C 044C
  \L 044D 042D 044D
  \L 044E 042E 044E
  \L 044F 042F 044F
  \L 0450 0400 0450
  \L 0451 0401 0451
  \L 0452 0402 0452
  \L 0453 0403 0453
  \L 0454 0404 0454
  \L 0455 0405 0455
  \L 0456 0406 0456
  \L 0457 0407 0457
  \L 0458 0408 0458
  \L 0459 0409 0459
  \L 045A 040A 045A
  \L 045B 040B 045B
  \L 045C 040C 045C
  \L 045D 040D 045D
  \L 045E 040E 045E
  \L 045F 040F 045F
  \L 0460 0460 0461
  \L 0461 0460 0461
  \L 0462 0462 0463
  \L 0463 0462 0463
  \L 0464 0464 0465
  \L 0465 0464 0465
  \L 0466 0466 0467
  \L 0467 0466 0467
  \L 0468 0468 0469
  \L 0469 0468 0469
  \L 046A 046A 046B
  \L 046B 046A 046B
  \L 046C 046C 046D
  \L 046D 046C 046D
  \L 046E 046E 046F
  \L 046F 046E 046F
  \L 0470 0470 0471
  \L 0471 0470 0471
  \L 0472 0472 0473
  \L 0473 0472 0473
  \L 0474 0474 0475
  \L 0475 0474 0475
  \L 0476 0476 0477
  \L 0477 0476 0477
  \L 0478 0478 0479
  \L 0479 0478 0479
  \L 047A 047A 047B
  \L 047B 047A 047B
  \L 047C 047C 047D
  \L 047D 047C 047D
  \L 047E 047E 047F
  \L 047F 047E 047F
  \L 0480 0480 0481
  \L 0481 0480 0481
  \l 0483
  \l 0484
  \l 0485
  \l 0486
  \l 0487
  \l 0488
  \l 0489
  \L 048A 048A 048B
  \L 048B 048A 048B
  \L 048C 048C 048D
  \L 048D 048C 048D
  \L 048E 048E 048F
  \L 048F 048E 048F
  \L 0490 0490 0491
  \L 0491 0490 0491
  \L 0492 0492 0493
  \L 0493 0492 0493
  \L 0494 0494 0495
  \L 0495 0494 0495
  \L 0496 0496 0497
  \L 0497 0496 0497
  \L 0498 0498 0499
  \L 0499 0498 0499
  \L 049A 049A 049B
  \L 049B 049A 049B
  \L 049C 049C 049D
  \L 049D 049C 049D
  \L 049E 049E 049F
  \L 049F 049E 049F
  \L 04A0 04A0 04A1
  \L 04A1 04A0 04A1
  \L 04A2 04A2 04A3
  \L 04A3 04A2 04A3
  \L 04A4 04A4 04A5
  \L 04A5 04A4 04A5
  \L 04A6 04A6 04A7
  \L 04A7 04A6 04A7
  \L 04A8 04A8 04A9
  \L 04A9 04A8 04A9
  \L 04AA 04AA 04AB
  \L 04AB 04AA 04AB
  \L 04AC 04AC 04AD
  \L 04AD 04AC 04AD
  \L 04AE 04AE 04AF
  \L 04AF 04AE 04AF
  \L 04B0 04B0 04B1
  \L 04B1 04B0 04B1
  \L 04B2 04B2 04B3
  \L 04B3 04B2 04B3
  \L 04B4 04B4 04B5
  \L 04B5 04B4 04B5
  \L 04B6 04B6 04B7
  \L 04B7 04B6 04B7
  \L 04B8 04B8 04B9
  \L 04B9 04B8 04B9
  \L 04BA 04BA 04BB
  \L 04BB 04BA 04BB
  \L 04BC 04BC 04BD
  \L 04BD 04BC 04BD
  \L 04BE 04BE 04BF
  \L 04BF 04BE 04BF
  \L 04C0 04C0 04CF
  \L 04C1 04C1 04C2
  \L 04C2 04C1 04C2
  \L 04C3 04C3 04C4
  \L 04C4 04C3 04C4
  \L 04C5 04C5 04C6
  \L 04C6 04C5 04C6
  \L 04C7 04C7 04C8
  \L 04C8 04C7 04C8
  \L 04C9 04C9 04CA
  \L 04CA 04C9 04CA
  \L 04CB 04CB 04CC
  \L 04CC 04CB 04CC
  \L 04CD 04CD 04CE
  \L 04CE 04CD 04CE
  \L 04CF 04C0 04CF
  \L 04D0 04D0 04D1
  \L 04D1 04D0 04D1
  \L 04D2 04D2 04D3
  \L 04D3 04D2 04D3
  \L 04D4 04D4 04D5
  \L 04D5 04D4 04D5
  \L 04D6 04D6 04D7
  \L 04D7 04D6 04D7
  \L 04D8 04D8 04D9
  \L 04D9 04D8 04D9
  \L 04DA 04DA 04DB
  \L 04DB 04DA 04DB
  \L 04DC 04DC 04DD
  \L 04DD 04DC 04DD
  \L 04DE 04DE 04DF
  \L 04DF 04DE 04DF
  \L 04E0 04E0 04E1
  \L 04E1 04E0 04E1
  \L 04E2 04E2 04E3
  \L 04E3 04E2 04E3
  \L 04E4 04E4 04E5
  \L 04E5 04E4 04E5
  \L 04E6 04E6 04E7
  \L 04E7 04E6 04E7
  \L 04E8 04E8 04E9
  \L 04E9 04E8 04E9
  \L 04EA 04EA 04EB
  \L 04EB 04EA 04EB
  \L 04EC 04EC 04ED
  \L 04ED 04EC 04ED
  \L 04EE 04EE 04EF
  \L 04EF 04EE 04EF
  \L 04F0 04F0 04F1
  \L 04F1 04F0 04F1
  \L 04F2 04F2 04F3
  \L 04F3 04F2 04F3
  \L 04F4 04F4 04F5
  \L 04F5 04F4 04F5
  \L 04F6 04F6 04F7
  \L 04F7 04F6 04F7
  \L 04F8 04F8 04F9
  \L 04F9 04F8 04F9
  \L 04FA 04FA 04FB
  \L 04FB 04FA 04FB
  \L 04FC 04FC 04FD
  \L 04FD 04FC 04FD
  \L 04FE 04FE 04FF
  \L 04FF 04FE 04FF
  \L 0500 0500 0501
  \L 0501 0500 0501
  \L 0502 0502 0503
  \L 0503 0502 0503
  \L 0504 0504 0505
  \L 0505 0504 0505
  \L 0506 0506 0507
  \L 0507 0506 0507
  \L 0508 0508 0509
  \L 0509 0508 0509
  \L 050A 050A 050B
  \L 050B 050A 050B
  \L 050C 050C 050D
  \L 050D 050C 050D
  \L 050E 050E 050F
  \L 050F 050E 050F
  \L 0510 0510 0511
  \L 0511 0510 0511
  \L 0512 0512 0513
  \L 0513 0512 0513
  \L 0514 0514 0515
  \L 0515 0514 0515
  \L 0516 0516 0517
  \L 0517 0516 0517
  \L 0518 0518 0519
  \L 0519 0518 0519
  \L 051A 051A 051B
  \L 051B 051A 051B
  \L 051C 051C 051D
  \L 051D 051C 051D
  \L 051E 051E 051F
  \L 051F 051E 051F
  \L 0520 0520 0521
  \L 0521 0520 0521
  \L 0522 0522 0523
  \L 0523 0522 0523
  \L 0524 0524 0525
  \L 0525 0524 0525
  \L 0526 0526 0527
  \L 0527 0526 0527
  \L 0528 0528 0529
  \L 0529 0528 0529
  \L 052A 052A 052B
  \L 052B 052A 052B
  \L 052C 052C 052D
  \L 052D 052C 052D
  \L 052E 052E 052F
  \L 052F 052E 052F
  \L 0531 0531 0561
  \L 0532 0532 0562
  \L 0533 0533 0563
  \L 0534 0534 0564
  \L 0535 0535 0565
  \L 0536 0536 0566
  \L 0537 0537 0567
  \L 0538 0538 0568
  \L 0539 0539 0569
  \L 053A 053A 056A
  \L 053B 053B 056B
  \L 053C 053C 056C
  \L 053D 053D 056D
  \L 053E 053E 056E
  \L 053F 053F 056F
  \L 0540 0540 0570
  \L 0541 0541 0571
  \L 0542 0542 0572
  \L 0543 0543 0573
  \L 0544 0544 0574
  \L 0545 0545 0575
  \L 0546 0546 0576
  \L 0547 0547 0577
  \L 0548 0548 0578
  \L 0549 0549 0579
  \L 054A 054A 057A
  \L 054B 054B 057B
  \L 054C 054C 057C
  \L 054D 054D 057D
  \L 054E 054E 057E
  \L 054F 054F 057F
  \L 0550 0550 0580
  \L 0551 0551 0581
  \L 0552 0552 0582
  \L 0553 0553 0583
  \L 0554 0554 0584
  \L 0555 0555 0585
  \L 0556 0556 0586
  \l 0559
  \L 0561 0531 0561
  \L 0562 0532 0562
  \L 0563 0533 0563
  \L 0564 0534 0564
  \L 0565 0535 0565
  \L 0566 0536 0566
  \L 0567 0537 0567
  \L 0568 0538 0568
  \L 0569 0539 0569
  \L 056A 053A 056A
  \L 056B 053B 056B
  \L 056C 053C 056C
  \L 056D 053D 056D
  \L 056E 053E 056E
  \L 056F 053F 056F
  \L 0570 0540 0570
  \L 0571 0541 0571
  \L 0572 0542 0572
  \L 0573 0543 0573
  \L 0574 0544 0574
  \L 0575 0545 0575
  \L 0576 0546 0576
  \L 0577 0547 0577
  \L 0578 0548 0578
  \L 0579 0549 0579
  \L 057A 054A 057A
  \L 057B 054B 057B
  \L 057C 054C 057C
  \L 057D 054D 057D
  \L 057E 054E 057E
  \L 057F 054F 057F
  \L 0580 0550 0580
  \L 0581 0551 0581
  \L 0582 0552 0582
  \L 0583 0553 0583
  \L 0584 0554 0584
  \L 0585 0555 0585
  \L 0586 0556 0586
  \l 0587
  \l 0591
  \l 0592
  \l 0593
  \l 0594
  \l 0595
  \l 0596
  \l 0597
  \l 0598
  \l 0599
  \l 059A
  \l 059B
  \l 059C
  \l 059D
  \l 059E
  \l 059F
  \l 05A0
  \l 05A1
  \l 05A2
  \l 05A3
  \l 05A4
  \l 05A5
  \l 05A6
  \l 05A7
  \l 05A8
  \l 05A9
  \l 05AA
  \l 05AB
  \l 05AC
  \l 05AD
  \l 05AE
  \l 05AF
  \l 05B0
  \l 05B1
  \l 05B2
  \l 05B3
  \l 05B4
  \l 05B5
  \l 05B6
  \l 05B7
  \l 05B8
  \l 05B9
  \l 05BA
  \l 05BB
  \l 05BC
  \l 05BD
  \l 05BF
  \l 05C1
  \l 05C2
  \l 05C4
  \l 05C5
  \l 05C7
  \l 05D0
  \l 05D1
  \l 05D2
  \l 05D3
  \l 05D4
  \l 05D5
  \l 05D6
  \l 05D7
  \l 05D8
  \l 05D9
  \l 05DA
  \l 05DB
  \l 05DC
  \l 05DD
  \l 05DE
  \l 05DF
  \l 05E0
  \l 05E1
  \l 05E2
  \l 05E3
  \l 05E4
  \l 05E5
  \l 05E6
  \l 05E7
  \l 05E8
  \l 05E9
  \l 05EA
  \l 05F0
  \l 05F1
  \l 05F2
  \l 0610
  \l 0611
  \l 0612
  \l 0613
  \l 0614
  \l 0615
  \l 0616
  \l 0617
  \l 0618
  \l 0619
  \l 061A
  \l 0620
  \l 0621
  \l 0622
  \l 0623
  \l 0624
  \l 0625
  \l 0626
  \l 0627
  \l 0628
  \l 0629
  \l 062A
  \l 062B
  \l 062C
  \l 062D
  \l 062E
  \l 062F
  \l 0630
  \l 0631
  \l 0632
  \l 0633
  \l 0634
  \l 0635
  \l 0636
  \l 0637
  \l 0638
  \l 0639
  \l 063A
  \l 063B
  \l 063C
  \l 063D
  \l 063E
  \l 063F
  \l 0640
  \l 0641
  \l 0642
  \l 0643
  \l 0644
  \l 0645
  \l 0646
  \l 0647
  \l 0648
  \l 0649
  \l 064A
  \l 064B
  \l 064C
  \l 064D
  \l 064E
  \l 064F
  \l 0650
  \l 0651
  \l 0652
  \l 0653
  \l 0654
  \l 0655
  \l 0656
  \l 0657
  \l 0658
  \l 0659
  \l 065A
  \l 065B
  \l 065C
  \l 065D
  \l 065E
  \l 065F
  \l 066E
  \l 066F
  \l 0670
  \l 0671
  \l 0672
  \l 0673
  \l 0674
  \l 0675
  \l 0676
  \l 0677
  \l 0678
  \l 0679
  \l 067A
  \l 067B
  \l 067C
  \l 067D
  \l 067E
  \l 067F
  \l 0680
  \l 0681
  \l 0682
  \l 0683
  \l 0684
  \l 0685
  \l 0686
  \l 0687
  \l 0688
  \l 0689
  \l 068A
  \l 068B
  \l 068C
  \l 068D
  \l 068E
  \l 068F
  \l 0690
  \l 0691
  \l 0692
  \l 0693
  \l 0694
  \l 0695
  \l 0696
  \l 0697
  \l 0698
  \l 0699
  \l 069A
  \l 069B
  \l 069C
  \l 069D
  \l 069E
  \l 069F
  \l 06A0
  \l 06A1
  \l 06A2
  \l 06A3
  \l 06A4
  \l 06A5
  \l 06A6
  \l 06A7
  \l 06A8
  \l 06A9
  \l 06AA
  \l 06AB
  \l 06AC
  \l 06AD
  \l 06AE
  \l 06AF
  \l 06B0
  \l 06B1
  \l 06B2
  \l 06B3
  \l 06B4
  \l 06B5
  \l 06B6
  \l 06B7
  \l 06B8
  \l 06B9
  \l 06BA
  \l 06BB
  \l 06BC
  \l 06BD
  \l 06BE
  \l 06BF
  \l 06C0
  \l 06C1
  \l 06C2
  \l 06C3
  \l 06C4
  \l 06C5
  \l 06C6
  \l 06C7
  \l 06C8
  \l 06C9
  \l 06CA
  \l 06CB
  \l 06CC
  \l 06CD
  \l 06CE
  \l 06CF
  \l 06D0
  \l 06D1
  \l 06D2
  \l 06D3
  \l 06D5
  \l 06D6
  \l 06D7
  \l 06D8
  \l 06D9
  \l 06DA
  \l 06DB
  \l 06DC
  \l 06DF
  \l 06E0
  \l 06E1
  \l 06E2
  \l 06E3
  \l 06E4
  \l 06E5
  \l 06E6
  \l 06E7
  \l 06E8
  \l 06EA
  \l 06EB
  \l 06EC
  \l 06ED
  \l 06EE
  \l 06EF
  \l 06FA
  \l 06FB
  \l 06FC
  \l 06FF
  \l 0710
  \l 0711
  \l 0712
  \l 0713
  \l 0714
  \l 0715
  \l 0716
  \l 0717
  \l 0718
  \l 0719
  \l 071A
  \l 071B
  \l 071C
  \l 071D
  \l 071E
  \l 071F
  \l 0720
  \l 0721
  \l 0722
  \l 0723
  \l 0724
  \l 0725
  \l 0726
  \l 0727
  \l 0728
  \l 0729
  \l 072A
  \l 072B
  \l 072C
  \l 072D
  \l 072E
  \l 072F
  \l 0730
  \l 0731
  \l 0732
  \l 0733
  \l 0734
  \l 0735
  \l 0736
  \l 0737
  \l 0738
  \l 0739
  \l 073A
  \l 073B
  \l 073C
  \l 073D
  \l 073E
  \l 073F
  \l 0740
  \l 0741
  \l 0742
  \l 0743
  \l 0744
  \l 0745
  \l 0746
  \l 0747
  \l 0748
  \l 0749
  \l 074A
  \l 074D
  \l 074E
  \l 074F
  \l 0750
  \l 0751
  \l 0752
  \l 0753
  \l 0754
  \l 0755
  \l 0756
  \l 0757
  \l 0758
  \l 0759
  \l 075A
  \l 075B
  \l 075C
  \l 075D
  \l 075E
  \l 075F
  \l 0760
  \l 0761
  \l 0762
  \l 0763
  \l 0764
  \l 0765
  \l 0766
  \l 0767
  \l 0768
  \l 0769
  \l 076A
  \l 076B
  \l 076C
  \l 076D
  \l 076E
  \l 076F
  \l 0770
  \l 0771
  \l 0772
  \l 0773
  \l 0774
  \l 0775
  \l 0776
  \l 0777
  \l 0778
  \l 0779
  \l 077A
  \l 077B
  \l 077C
  \l 077D
  \l 077E
  \l 077F
  \l 0780
  \l 0781
  \l 0782
  \l 0783
  \l 0784
  \l 0785
  \l 0786
  \l 0787
  \l 0788
  \l 0789
  \l 078A
  \l 078B
  \l 078C
  \l 078D
  \l 078E
  \l 078F
  \l 0790
  \l 0791
  \l 0792
  \l 0793
  \l 0794
  \l 0795
  \l 0796
  \l 0797
  \l 0798
  \l 0799
  \l 079A
  \l 079B
  \l 079C
  \l 079D
  \l 079E
  \l 079F
  \l 07A0
  \l 07A1
  \l 07A2
  \l 07A3
  \l 07A4
  \l 07A5
  \l 07A6
  \l 07A7
  \l 07A8
  \l 07A9
  \l 07AA
  \l 07AB
  \l 07AC
  \l 07AD
  \l 07AE
  \l 07AF
  \l 07B0
  \l 07B1
  \l 07CA
  \l 07CB
  \l 07CC
  \l 07CD
  \l 07CE
  \l 07CF
  \l 07D0
  \l 07D1
  \l 07D2
  \l 07D3
  \l 07D4
  \l 07D5
  \l 07D6
  \l 07D7
  \l 07D8
  \l 07D9
  \l 07DA
  \l 07DB
  \l 07DC
  \l 07DD
  \l 07DE
  \l 07DF
  \l 07E0
  \l 07E1
  \l 07E2
  \l 07E3
  \l 07E4
  \l 07E5
  \l 07E6
  \l 07E7
  \l 07E8
  \l 07E9
  \l 07EA
  \l 07EB
  \l 07EC
  \l 07ED
  \l 07EE
  \l 07EF
  \l 07F0
  \l 07F1
  \l 07F2
  \l 07F3
  \l 07F4
  \l 07F5
  \l 07FA
  \l 0800
  \l 0801
  \l 0802
  \l 0803
  \l 0804
  \l 0805
  \l 0806
  \l 0807
  \l 0808
  \l 0809
  \l 080A
  \l 080B
  \l 080C
  \l 080D
  \l 080E
  \l 080F
  \l 0810
  \l 0811
  \l 0812
  \l 0813
  \l 0814
  \l 0815
  \l 0816
  \l 0817
  \l 0818
  \l 0819
  \l 081A
  \l 081B
  \l 081C
  \l 081D
  \l 081E
  \l 081F
  \l 0820
  \l 0821
  \l 0822
  \l 0823
  \l 0824
  \l 0825
  \l 0826
  \l 0827
  \l 0828
  \l 0829
  \l 082A
  \l 082B
  \l 082C
  \l 082D
  \l 0840
  \l 0841
  \l 0842
  \l 0843
  \l 0844
  \l 0845
  \l 0846
  \l 0847
  \l 0848
  \l 0849
  \l 084A
  \l 084B
  \l 084C
  \l 084D
  \l 084E
  \l 084F
  \l 0850
  \l 0851
  \l 0852
  \l 0853
  \l 0854
  \l 0855
  \l 0856
  \l 0857
  \l 0858
  \l 0859
  \l 085A
  \l 085B
  \l 08A0
  \l 08A1
  \l 08A2
  \l 08A3
  \l 08A4
  \l 08A5
  \l 08A6
  \l 08A7
  \l 08A8
  \l 08A9
  \l 08AA
  \l 08AB
  \l 08AC
  \l 08AD
  \l 08AE
  \l 08AF
  \l 08B0
  \l 08B1
  \l 08B2
  \l 08E4
  \l 08E5
  \l 08E6
  \l 08E7
  \l 08E8
  \l 08E9
  \l 08EA
  \l 08EB
  \l 08EC
  \l 08ED
  \l 08EE
  \l 08EF
  \l 08F0
  \l 08F1
  \l 08F2
  \l 08F3
  \l 08F4
  \l 08F5
  \l 08F6
  \l 08F7
  \l 08F8
  \l 08F9
  \l 08FA
  \l 08FB
  \l 08FC
  \l 08FD
  \l 08FE
  \l 08FF
  \l 0900
  \l 0901
  \l 0902
  \l 0903
  \l 0904
  \l 0905
  \l 0906
  \l 0907
  \l 0908
  \l 0909
  \l 090A
  \l 090B
  \l 090C
  \l 090D
  \l 090E
  \l 090F
  \l 0910
  \l 0911
  \l 0912
  \l 0913
  \l 0914
  \l 0915
  \l 0916
  \l 0917
  \l 0918
  \l 0919
  \l 091A
  \l 091B
  \l 091C
  \l 091D
  \l 091E
  \l 091F
  \l 0920
  \l 0921
  \l 0922
  \l 0923
  \l 0924
  \l 0925
  \l 0926
  \l 0927
  \l 0928
  \l 0929
  \l 092A
  \l 092B
  \l 092C
  \l 092D
  \l 092E
  \l 092F
  \l 0930
  \l 0931
  \l 0932
  \l 0933
  \l 0934
  \l 0935
  \l 0936
  \l 0937
  \l 0938
  \l 0939
  \l 093A
  \l 093B
  \l 093C
  \l 093D
  \l 093E
  \l 093F
  \l 0940
  \l 0941
  \l 0942
  \l 0943
  \l 0944
  \l 0945
  \l 0946
  \l 0947
  \l 0948
  \l 0949
  \l 094A
  \l 094B
  \l 094C
  \l 094D
  \l 094E
  \l 094F
  \l 0950
  \l 0951
  \l 0952
  \l 0953
  \l 0954
  \l 0955
  \l 0956
  \l 0957
  \l 0958
  \l 0959
  \l 095A
  \l 095B
  \l 095C
  \l 095D
  \l 095E
  \l 095F
  \l 0960
  \l 0961
  \l 0962
  \l 0963
  \l 0971
  \l 0972
  \l 0973
  \l 0974
  \l 0975
  \l 0976
  \l 0977
  \l 0978
  \l 0979
  \l 097A
  \l 097B
  \l 097C
  \l 097D
  \l 097E
  \l 097F
  \l 0980
  \l 0981
  \l 0982
  \l 0983
  \l 0985
  \l 0986
  \l 0987
  \l 0988
  \l 0989
  \l 098A
  \l 098B
  \l 098C
  \l 098F
  \l 0990
  \l 0993
  \l 0994
  \l 0995
  \l 0996
  \l 0997
  \l 0998
  \l 0999
  \l 099A
  \l 099B
  \l 099C
  \l 099D
  \l 099E
  \l 099F
  \l 09A0
  \l 09A1
  \l 09A2
  \l 09A3
  \l 09A4
  \l 09A5
  \l 09A6
  \l 09A7
  \l 09A8
  \l 09AA
  \l 09AB
  \l 09AC
  \l 09AD
  \l 09AE
  \l 09AF
  \l 09B0
  \l 09B2
  \l 09B6
  \l 09B7
  \l 09B8
  \l 09B9
  \l 09BC
  \l 09BD
  \l 09BE
  \l 09BF
  \l 09C0
  \l 09C1
  \l 09C2
  \l 09C3
  \l 09C4
  \l 09C7
  \l 09C8
  \l 09CB
  \l 09CC
  \l 09CD
  \l 09CE
  \l 09D7
  \l 09DC
  \l 09DD
  \l 09DF
  \l 09E0
  \l 09E1
  \l 09E2
  \l 09E3
  \l 09F0
  \l 09F1
  \l 0A01
  \l 0A02
  \l 0A03
  \l 0A05
  \l 0A06
  \l 0A07
  \l 0A08
  \l 0A09
  \l 0A0A
  \l 0A0F
  \l 0A10
  \l 0A13
  \l 0A14
  \l 0A15
  \l 0A16
  \l 0A17
  \l 0A18
  \l 0A19
  \l 0A1A
  \l 0A1B
  \l 0A1C
  \l 0A1D
  \l 0A1E
  \l 0A1F
  \l 0A20
  \l 0A21
  \l 0A22
  \l 0A23
  \l 0A24
  \l 0A25
  \l 0A26
  \l 0A27
  \l 0A28
  \l 0A2A
  \l 0A2B
  \l 0A2C
  \l 0A2D
  \l 0A2E
  \l 0A2F
  \l 0A30
  \l 0A32
  \l 0A33
  \l 0A35
  \l 0A36
  \l 0A38
  \l 0A39
  \l 0A3C
  \l 0A3E
  \l 0A3F
  \l 0A40
  \l 0A41
  \l 0A42
  \l 0A47
  \l 0A48
  \l 0A4B
  \l 0A4C
  \l 0A4D
  \l 0A51
  \l 0A59
  \l 0A5A
  \l 0A5B
  \l 0A5C
  \l 0A5E
  \l 0A70
  \l 0A71
  \l 0A72
  \l 0A73
  \l 0A74
  \l 0A75
  \l 0A81
  \l 0A82
  \l 0A83
  \l 0A85
  \l 0A86
  \l 0A87
  \l 0A88
  \l 0A89
  \l 0A8A
  \l 0A8B
  \l 0A8C
  \l 0A8D
  \l 0A8F
  \l 0A90
  \l 0A91
  \l 0A93
  \l 0A94
  \l 0A95
  \l 0A96
  \l 0A97
  \l 0A98
  \l 0A99
  \l 0A9A
  \l 0A9B
  \l 0A9C
  \l 0A9D
  \l 0A9E
  \l 0A9F
  \l 0AA0
  \l 0AA1
  \l 0AA2
  \l 0AA3
  \l 0AA4
  \l 0AA5
  \l 0AA6
  \l 0AA7
  \l 0AA8
  \l 0AAA
  \l 0AAB
  \l 0AAC
  \l 0AAD
  \l 0AAE
  \l 0AAF
  \l 0AB0
  \l 0AB2
  \l 0AB3
  \l 0AB5
  \l 0AB6
  \l 0AB7
  \l 0AB8
  \l 0AB9
  \l 0ABC
  \l 0ABD
  \l 0ABE
  \l 0ABF
  \l 0AC0
  \l 0AC1
  \l 0AC2
  \l 0AC3
  \l 0AC4
  \l 0AC5
  \l 0AC7
  \l 0AC8
  \l 0AC9
  \l 0ACB
  \l 0ACC
  \l 0ACD
  \l 0AD0
  \l 0AE0
  \l 0AE1
  \l 0AE2
  \l 0AE3
  \l 0B01
  \l 0B02
  \l 0B03
  \l 0B05
  \l 0B06
  \l 0B07
  \l 0B08
  \l 0B09
  \l 0B0A
  \l 0B0B
  \l 0B0C
  \l 0B0F
  \l 0B10
  \l 0B13
  \l 0B14
  \l 0B15
  \l 0B16
  \l 0B17
  \l 0B18
  \l 0B19
  \l 0B1A
  \l 0B1B
  \l 0B1C
  \l 0B1D
  \l 0B1E
  \l 0B1F
  \l 0B20
  \l 0B21
  \l 0B22
  \l 0B23
  \l 0B24
  \l 0B25
  \l 0B26
  \l 0B27
  \l 0B28
  \l 0B2A
  \l 0B2B
  \l 0B2C
  \l 0B2D
  \l 0B2E
  \l 0B2F
  \l 0B30
  \l 0B32
  \l 0B33
  \l 0B35
  \l 0B36
  \l 0B37
  \l 0B38
  \l 0B39
  \l 0B3C
  \l 0B3D
  \l 0B3E
  \l 0B3F
  \l 0B40
  \l 0B41
  \l 0B42
  \l 0B43
  \l 0B44
  \l 0B47
  \l 0B48
  \l 0B4B
  \l 0B4C
  \l 0B4D
  \l 0B56
  \l 0B57
  \l 0B5C
  \l 0B5D
  \l 0B5F
  \l 0B60
  \l 0B61
  \l 0B62
  \l 0B63
  \l 0B71
  \l 0B82
  \l 0B83
  \l 0B85
  \l 0B86
  \l 0B87
  \l 0B88
  \l 0B89
  \l 0B8A
  \l 0B8E
  \l 0B8F
  \l 0B90
  \l 0B92
  \l 0B93
  \l 0B94
  \l 0B95
  \l 0B99
  \l 0B9A
  \l 0B9C
  \l 0B9E
  \l 0B9F
  \l 0BA3
  \l 0BA4
  \l 0BA8
  \l 0BA9
  \l 0BAA
  \l 0BAE
  \l 0BAF
  \l 0BB0
  \l 0BB1
  \l 0BB2
  \l 0BB3
  \l 0BB4
  \l 0BB5
  \l 0BB6
  \l 0BB7
  \l 0BB8
  \l 0BB9
  \l 0BBE
  \l 0BBF
  \l 0BC0
  \l 0BC1
  \l 0BC2
  \l 0BC6
  \l 0BC7
  \l 0BC8
  \l 0BCA
  \l 0BCB
  \l 0BCC
  \l 0BCD
  \l 0BD0
  \l 0BD7
  \l 0C00
  \l 0C01
  \l 0C02
  \l 0C03
  \l 0C05
  \l 0C06
  \l 0C07
  \l 0C08
  \l 0C09
  \l 0C0A
  \l 0C0B
  \l 0C0C
  \l 0C0E
  \l 0C0F
  \l 0C10
  \l 0C12
  \l 0C13
  \l 0C14
  \l 0C15
  \l 0C16
  \l 0C17
  \l 0C18
  \l 0C19
  \l 0C1A
  \l 0C1B
  \l 0C1C
  \l 0C1D
  \l 0C1E
  \l 0C1F
  \l 0C20
  \l 0C21
  \l 0C22
  \l 0C23
  \l 0C24
  \l 0C25
  \l 0C26
  \l 0C27
  \l 0C28
  \l 0C2A
  \l 0C2B
  \l 0C2C
  \l 0C2D
  \l 0C2E
  \l 0C2F
  \l 0C30
  \l 0C31
  \l 0C32
  \l 0C33
  \l 0C34
  \l 0C35
  \l 0C36
  \l 0C37
  \l 0C38
  \l 0C39
  \l 0C3D
  \l 0C3E
  \l 0C3F
  \l 0C40
  \l 0C41
  \l 0C42
  \l 0C43
  \l 0C44
  \l 0C46
  \l 0C47
  \l 0C48
  \l 0C4A
  \l 0C4B
  \l 0C4C
  \l 0C4D
  \l 0C55
  \l 0C56
  \l 0C58
  \l 0C59
  \l 0C60
  \l 0C61
  \l 0C62
  \l 0C63
  \l 0C81
  \l 0C82
  \l 0C83
  \l 0C85
  \l 0C86
  \l 0C87
  \l 0C88
  \l 0C89
  \l 0C8A
  \l 0C8B
  \l 0C8C
  \l 0C8E
  \l 0C8F
  \l 0C90
  \l 0C92
  \l 0C93
  \l 0C94
  \l 0C95
  \l 0C96
  \l 0C97
  \l 0C98
  \l 0C99
  \l 0C9A
  \l 0C9B
  \l 0C9C
  \l 0C9D
  \l 0C9E
  \l 0C9F
  \l 0CA0
  \l 0CA1
  \l 0CA2
  \l 0CA3
  \l 0CA4
  \l 0CA5
  \l 0CA6
  \l 0CA7
  \l 0CA8
  \l 0CAA
  \l 0CAB
  \l 0CAC
  \l 0CAD
  \l 0CAE
  \l 0CAF
  \l 0CB0
  \l 0CB1
  \l 0CB2
  \l 0CB3
  \l 0CB5
  \l 0CB6
  \l 0CB7
  \l 0CB8
  \l 0CB9
  \l 0CBC
  \l 0CBD
  \l 0CBE
  \l 0CBF
  \l 0CC0
  \l 0CC1
  \l 0CC2
  \l 0CC3
  \l 0CC4
  \l 0CC6
  \l 0CC7
  \l 0CC8
  \l 0CCA
  \l 0CCB
  \l 0CCC
  \l 0CCD
  \l 0CD5
  \l 0CD6
  \l 0CDE
  \l 0CE0
  \l 0CE1
  \l 0CE2
  \l 0CE3
  \l 0CF1
  \l 0CF2
  \l 0D01
  \l 0D02
  \l 0D03
  \l 0D05
  \l 0D06
  \l 0D07
  \l 0D08
  \l 0D09
  \l 0D0A
  \l 0D0B
  \l 0D0C
  \l 0D0E
  \l 0D0F
  \l 0D10
  \l 0D12
  \l 0D13
  \l 0D14
  \l 0D15
  \l 0D16
  \l 0D17
  \l 0D18
  \l 0D19
  \l 0D1A
  \l 0D1B
  \l 0D1C
  \l 0D1D
  \l 0D1E
  \l 0D1F
  \l 0D20
  \l 0D21
  \l 0D22
  \l 0D23
  \l 0D24
  \l 0D25
  \l 0D26
  \l 0D27
  \l 0D28
  \l 0D29
  \l 0D2A
  \l 0D2B
  \l 0D2C
  \l 0D2D
  \l 0D2E
  \l 0D2F
  \l 0D30
  \l 0D31
  \l 0D32
  \l 0D33
  \l 0D34
  \l 0D35
  \l 0D36
  \l 0D37
  \l 0D38
  \l 0D39
  \l 0D3A
  \l 0D3D
  \l 0D3E
  \l 0D3F
  \l 0D40
  \l 0D41
  \l 0D42
  \l 0D43
  \l 0D44
  \l 0D46
  \l 0D47
  \l 0D48
  \l 0D4A
  \l 0D4B
  \l 0D4C
  \l 0D4D
  \l 0D4E
  \l 0D57
  \l 0D60
  \l 0D61
  \l 0D62
  \l 0D63
  \l 0D7A
  \l 0D7B
  \l 0D7C
  \l 0D7D
  \l 0D7E
  \l 0D7F
  \l 0D82
  \l 0D83
  \l 0D85
  \l 0D86
  \l 0D87
  \l 0D88
  \l 0D89
  \l 0D8A
  \l 0D8B
  \l 0D8C
  \l 0D8D
  \l 0D8E
  \l 0D8F
  \l 0D90
  \l 0D91
  \l 0D92
  \l 0D93
  \l 0D94
  \l 0D95
  \l 0D96
  \l 0D9A
  \l 0D9B
  \l 0D9C
  \l 0D9D
  \l 0D9E
  \l 0D9F
  \l 0DA0
  \l 0DA1
  \l 0DA2
  \l 0DA3
  \l 0DA4
  \l 0DA5
  \l 0DA6
  \l 0DA7
  \l 0DA8
  \l 0DA9
  \l 0DAA
  \l 0DAB
  \l 0DAC
  \l 0DAD
  \l 0DAE
  \l 0DAF
  \l 0DB0
  \l 0DB1
  \l 0DB3
  \l 0DB4
  \l 0DB5
  \l 0DB6
  \l 0DB7
  \l 0DB8
  \l 0DB9
  \l 0DBA
  \l 0DBB
  \l 0DBD
  \l 0DC0
  \l 0DC1
  \l 0DC2
  \l 0DC3
  \l 0DC4
  \l 0DC5
  \l 0DC6
  \l 0DCA
  \l 0DCF
  \l 0DD0
  \l 0DD1
  \l 0DD2
  \l 0DD3
  \l 0DD4
  \l 0DD6
  \l 0DD8
  \l 0DD9
  \l 0DDA
  \l 0DDB
  \l 0DDC
  \l 0DDD
  \l 0DDE
  \l 0DDF
  \l 0DF2
  \l 0DF3
  \l 0E01
  \l 0E02
  \l 0E03
  \l 0E04
  \l 0E05
  \l 0E06
  \l 0E07
  \l 0E08
  \l 0E09
  \l 0E0A
  \l 0E0B
  \l 0E0C
  \l 0E0D
  \l 0E0E
  \l 0E0F
  \l 0E10
  \l 0E11
  \l 0E12
  \l 0E13
  \l 0E14
  \l 0E15
  \l 0E16
  \l 0E17
  \l 0E18
  \l 0E19
  \l 0E1A
  \l 0E1B
  \l 0E1C
  \l 0E1D
  \l 0E1E
  \l 0E1F
  \l 0E20
  \l 0E21
  \l 0E22
  \l 0E23
  \l 0E24
  \l 0E25
  \l 0E26
  \l 0E27
  \l 0E28
  \l 0E29
  \l 0E2A
  \l 0E2B
  \l 0E2C
  \l 0E2D
  \l 0E2E
  \l 0E2F
  \l 0E30
  \l 0E31
  \l 0E32
  \l 0E33
  \l 0E34
  \l 0E35
  \l 0E36
  \l 0E37
  \l 0E38
  \l 0E39
  \l 0E3A
  \l 0E40
  \l 0E41
  \l 0E42
  \l 0E43
  \l 0E44
  \l 0E45
  \l 0E46
  \l 0E47
  \l 0E48
  \l 0E49
  \l 0E4A
  \l 0E4B
  \l 0E4C
  \l 0E4D
  \l 0E4E
  \l 0E81
  \l 0E82
  \l 0E84
  \l 0E87
  \l 0E88
  \l 0E8A
  \l 0E8D
  \l 0E94
  \l 0E95
  \l 0E96
  \l 0E97
  \l 0E99
  \l 0E9A
  \l 0E9B
  \l 0E9C
  \l 0E9D
  \l 0E9E
  \l 0E9F
  \l 0EA1
  \l 0EA2
  \l 0EA3
  \l 0EA5
  \l 0EA7
  \l 0EAA
  \l 0EAB
  \l 0EAD
  \l 0EAE
  \l 0EAF
  \l 0EB0
  \l 0EB1
  \l 0EB2
  \l 0EB3
  \l 0EB4
  \l 0EB5
  \l 0EB6
  \l 0EB7
  \l 0EB8
  \l 0EB9
  \l 0EBB
  \l 0EBC
  \l 0EBD
  \l 0EC0
  \l 0EC1
  \l 0EC2
  \l 0EC3
  \l 0EC4
  \l 0EC6
  \l 0EC8
  \l 0EC9
  \l 0ECA
  \l 0ECB
  \l 0ECC
  \l 0ECD
  \l 0EDC
  \l 0EDD
  \l 0EDE
  \l 0EDF
  \l 0F00
  \l 0F18
  \l 0F19
  \l 0F35
  \l 0F37
  \l 0F39
  \l 0F3E
  \l 0F3F
  \l 0F40
  \l 0F41
  \l 0F42
  \l 0F43
  \l 0F44
  \l 0F45
  \l 0F46
  \l 0F47
  \l 0F49
  \l 0F4A
  \l 0F4B
  \l 0F4C
  \l 0F4D
  \l 0F4E
  \l 0F4F
  \l 0F50
  \l 0F51
  \l 0F52
  \l 0F53
  \l 0F54
  \l 0F55
  \l 0F56
  \l 0F57
  \l 0F58
  \l 0F59
  \l 0F5A
  \l 0F5B
  \l 0F5C
  \l 0F5D
  \l 0F5E
  \l 0F5F
  \l 0F60
  \l 0F61
  \l 0F62
  \l 0F63
  \l 0F64
  \l 0F65
  \l 0F66
  \l 0F67
  \l 0F68
  \l 0F69
  \l 0F6A
  \l 0F6B
  \l 0F6C
  \l 0F71
  \l 0F72
  \l 0F73
  \l 0F74
  \l 0F75
  \l 0F76
  \l 0F77
  \l 0F78
  \l 0F79
  \l 0F7A
  \l 0F7B
  \l 0F7C
  \l 0F7D
  \l 0F7E
  \l 0F7F
  \l 0F80
  \l 0F81
  \l 0F82
  \l 0F83
  \l 0F84
  \l 0F86
  \l 0F87
  \l 0F88
  \l 0F89
  \l 0F8A
  \l 0F8B
  \l 0F8C
  \l 0F8D
  \l 0F8E
  \l 0F8F
  \l 0F90
  \l 0F91
  \l 0F92
  \l 0F93
  \l 0F94
  \l 0F95
  \l 0F96
  \l 0F97
  \l 0F99
  \l 0F9A
  \l 0F9B
  \l 0F9C
  \l 0F9D
  \l 0F9E
  \l 0F9F
  \l 0FA0
  \l 0FA1
  \l 0FA2
  \l 0FA3
  \l 0FA4
  \l 0FA5
  \l 0FA6
  \l 0FA7
  \l 0FA8
  \l 0FA9
  \l 0FAA
  \l 0FAB
  \l 0FAC
  \l 0FAD
  \l 0FAE
  \l 0FAF
  \l 0FB0
  \l 0FB1
  \l 0FB2
  \l 0FB3
  \l 0FB4
  \l 0FB5
  \l 0FB6
  \l 0FB7
  \l 0FB8
  \l 0FB9
  \l 0FBA
  \l 0FBB
  \l 0FBC
  \l 0FC6
  \l 1000
  \l 1001
  \l 1002
  \l 1003
  \l 1004
  \l 1005
  \l 1006
  \l 1007
  \l 1008
  \l 1009
  \l 100A
  \l 100B
  \l 100C
  \l 100D
  \l 100E
  \l 100F
  \l 1010
  \l 1011
  \l 1012
  \l 1013
  \l 1014
  \l 1015
  \l 1016
  \l 1017
  \l 1018
  \l 1019
  \l 101A
  \l 101B
  \l 101C
  \l 101D
  \l 101E
  \l 101F
  \l 1020
  \l 1021
  \l 1022
  \l 1023
  \l 1024
  \l 1025
  \l 1026
  \l 1027
  \l 1028
  \l 1029
  \l 102A
  \l 102B
  \l 102C
  \l 102D
  \l 102E
  \l 102F
  \l 1030
  \l 1031
  \l 1032
  \l 1033
  \l 1034
  \l 1035
  \l 1036
  \l 1037
  \l 1038
  \l 1039
  \l 103A
  \l 103B
  \l 103C
  \l 103D
  \l 103E
  \l 103F
  \l 1050
  \l 1051
  \l 1052
  \l 1053
  \l 1054
  \l 1055
  \l 1056
  \l 1057
  \l 1058
  \l 1059
  \l 105A
  \l 105B
  \l 105C
  \l 105D
  \l 105E
  \l 105F
  \l 1060
  \l 1061
  \l 1062
  \l 1063
  \l 1064
  \l 1065
  \l 1066
  \l 1067
  \l 1068
  \l 1069
  \l 106A
  \l 106B
  \l 106C
  \l 106D
  \l 106E
  \l 106F
  \l 1070
  \l 1071
  \l 1072
  \l 1073
  \l 1074
  \l 1075
  \l 1076
  \l 1077
  \l 1078
  \l 1079
  \l 107A
  \l 107B
  \l 107C
  \l 107D
  \l 107E
  \l 107F
  \l 1080
  \l 1081
  \l 1082
  \l 1083
  \l 1084
  \l 1085
  \l 1086
  \l 1087
  \l 1088
  \l 1089
  \l 108A
  \l 108B
  \l 108C
  \l 108D
  \l 108E
  \l 108F
  \l 109A
  \l 109B
  \l 109C
  \l 109D
  \L 10A0 10A0 2D00
  \L 10A1 10A1 2D01
  \L 10A2 10A2 2D02
  \L 10A3 10A3 2D03
  \L 10A4 10A4 2D04
  \L 10A5 10A5 2D05
  \L 10A6 10A6 2D06
  \L 10A7 10A7 2D07
  \L 10A8 10A8 2D08
  \L 10A9 10A9 2D09
  \L 10AA 10AA 2D0A
  \L 10AB 10AB 2D0B
  \L 10AC 10AC 2D0C
  \L 10AD 10AD 2D0D
  \L 10AE 10AE 2D0E
  \L 10AF 10AF 2D0F
  \L 10B0 10B0 2D10
  \L 10B1 10B1 2D11
  \L 10B2 10B2 2D12
  \L 10B3 10B3 2D13
  \L 10B4 10B4 2D14
  \L 10B5 10B5 2D15
  \L 10B6 10B6 2D16
  \L 10B7 10B7 2D17
  \L 10B8 10B8 2D18
  \L 10B9 10B9 2D19
  \L 10BA 10BA 2D1A
  \L 10BB 10BB 2D1B
  \L 10BC 10BC 2D1C
  \L 10BD 10BD 2D1D
  \L 10BE 10BE 2D1E
  \L 10BF 10BF 2D1F
  \L 10C0 10C0 2D20
  \L 10C1 10C1 2D21
  \L 10C2 10C2 2D22
  \L 10C3 10C3 2D23
  \L 10C4 10C4 2D24
  \L 10C5 10C5 2D25
  \L 10C7 10C7 2D27
  \L 10CD 10CD 2D2D
  \l 10D0
  \l 10D1
  \l 10D2
  \l 10D3
  \l 10D4
  \l 10D5
  \l 10D6
  \l 10D7
  \l 10D8
  \l 10D9
  \l 10DA
  \l 10DB
  \l 10DC
  \l 10DD
  \l 10DE
  \l 10DF
  \l 10E0
  \l 10E1
  \l 10E2
  \l 10E3
  \l 10E4
  \l 10E5
  \l 10E6
  \l 10E7
  \l 10E8
  \l 10E9
  \l 10EA
  \l 10EB
  \l 10EC
  \l 10ED
  \l 10EE
  \l 10EF
  \l 10F0
  \l 10F1
  \l 10F2
  \l 10F3
  \l 10F4
  \l 10F5
  \l 10F6
  \l 10F7
  \l 10F8
  \l 10F9
  \l 10FA
  \l 10FC
  \l 10FD
  \l 10FE
  \l 10FF
  \l 1100
  \l 1101
  \l 1102
  \l 1103
  \l 1104
  \l 1105
  \l 1106
  \l 1107
  \l 1108
  \l 1109
  \l 110A
  \l 110B
  \l 110C
  \l 110D
  \l 110E
  \l 110F
  \l 1110
  \l 1111
  \l 1112
  \l 1113
  \l 1114
  \l 1115
  \l 1116
  \l 1117
  \l 1118
  \l 1119
  \l 111A
  \l 111B
  \l 111C
  \l 111D
  \l 111E
  \l 111F
  \l 1120
  \l 1121
  \l 1122
  \l 1123
  \l 1124
  \l 1125
  \l 1126
  \l 1127
  \l 1128
  \l 1129
  \l 112A
  \l 112B
  \l 112C
  \l 112D
  \l 112E
  \l 112F
  \l 1130
  \l 1131
  \l 1132
  \l 1133
  \l 1134
  \l 1135
  \l 1136
  \l 1137
  \l 1138
  \l 1139
  \l 113A
  \l 113B
  \l 113C
  \l 113D
  \l 113E
  \l 113F
  \l 1140
  \l 1141
  \l 1142
  \l 1143
  \l 1144
  \l 1145
  \l 1146
  \l 1147
  \l 1148
  \l 1149
  \l 114A
  \l 114B
  \l 114C
  \l 114D
  \l 114E
  \l 114F
  \l 1150
  \l 1151
  \l 1152
  \l 1153
  \l 1154
  \l 1155
  \l 1156
  \l 1157
  \l 1158
  \l 1159
  \l 115A
  \l 115B
  \l 115C
  \l 115D
  \l 115E
  \l 115F
  \l 1160
  \l 1161
  \l 1162
  \l 1163
  \l 1164
  \l 1165
  \l 1166
  \l 1167
  \l 1168
  \l 1169
  \l 116A
  \l 116B
  \l 116C
  \l 116D
  \l 116E
  \l 116F
  \l 1170
  \l 1171
  \l 1172
  \l 1173
  \l 1174
  \l 1175
  \l 1176
  \l 1177
  \l 1178
  \l 1179
  \l 117A
  \l 117B
  \l 117C
  \l 117D
  \l 117E
  \l 117F
  \l 1180
  \l 1181
  \l 1182
  \l 1183
  \l 1184
  \l 1185
  \l 1186
  \l 1187
  \l 1188
  \l 1189
  \l 118A
  \l 118B
  \l 118C
  \l 118D
  \l 118E
  \l 118F
  \l 1190
  \l 1191
  \l 1192
  \l 1193
  \l 1194
  \l 1195
  \l 1196
  \l 1197
  \l 1198
  \l 1199
  \l 119A
  \l 119B
  \l 119C
  \l 119D
  \l 119E
  \l 119F
  \l 11A0
  \l 11A1
  \l 11A2
  \l 11A3
  \l 11A4
  \l 11A5
  \l 11A6
  \l 11A7
  \l 11A8
  \l 11A9
  \l 11AA
  \l 11AB
  \l 11AC
  \l 11AD
  \l 11AE
  \l 11AF
  \l 11B0
  \l 11B1
  \l 11B2
  \l 11B3
  \l 11B4
  \l 11B5
  \l 11B6
  \l 11B7
  \l 11B8
  \l 11B9
  \l 11BA
  \l 11BB
  \l 11BC
  \l 11BD
  \l 11BE
  \l 11BF
  \l 11C0
  \l 11C1
  \l 11C2
  \l 11C3
  \l 11C4
  \l 11C5
  \l 11C6
  \l 11C7
  \l 11C8
  \l 11C9
  \l 11CA
  \l 11CB
  \l 11CC
  \l 11CD
  \l 11CE
  \l 11CF
  \l 11D0
  \l 11D1
  \l 11D2
  \l 11D3
  \l 11D4
  \l 11D5
  \l 11D6
  \l 11D7
  \l 11D8
  \l 11D9
  \l 11DA
  \l 11DB
  \l 11DC
  \l 11DD
  \l 11DE
  \l 11DF
  \l 11E0
  \l 11E1
  \l 11E2
  \l 11E3
  \l 11E4
  \l 11E5
  \l 11E6
  \l 11E7
  \l 11E8
  \l 11E9
  \l 11EA
  \l 11EB
  \l 11EC
  \l 11ED
  \l 11EE
  \l 11EF
  \l 11F0
  \l 11F1
  \l 11F2
  \l 11F3
  \l 11F4
  \l 11F5
  \l 11F6
  \l 11F7
  \l 11F8
  \l 11F9
  \l 11FA
  \l 11FB
  \l 11FC
  \l 11FD
  \l 11FE
  \l 11FF
  \l 1200
  \l 1201
  \l 1202
  \l 1203
  \l 1204
  \l 1205
  \l 1206
  \l 1207
  \l 1208
  \l 1209
  \l 120A
  \l 120B
  \l 120C
  \l 120D
  \l 120E
  \l 120F
  \l 1210
  \l 1211
  \l 1212
  \l 1213
  \l 1214
  \l 1215
  \l 1216
  \l 1217
  \l 1218
  \l 1219
  \l 121A
  \l 121B
  \l 121C
  \l 121D
  \l 121E
  \l 121F
  \l 1220
  \l 1221
  \l 1222
  \l 1223
  \l 1224
  \l 1225
  \l 1226
  \l 1227
  \l 1228
  \l 1229
  \l 122A
  \l 122B
  \l 122C
  \l 122D
  \l 122E
  \l 122F
  \l 1230
  \l 1231
  \l 1232
  \l 1233
  \l 1234
  \l 1235
  \l 1236
  \l 1237
  \l 1238
  \l 1239
  \l 123A
  \l 123B
  \l 123C
  \l 123D
  \l 123E
  \l 123F
  \l 1240
  \l 1241
  \l 1242
  \l 1243
  \l 1244
  \l 1245
  \l 1246
  \l 1247
  \l 1248
  \l 124A
  \l 124B
  \l 124C
  \l 124D
  \l 1250
  \l 1251
  \l 1252
  \l 1253
  \l 1254
  \l 1255
  \l 1256
  \l 1258
  \l 125A
  \l 125B
  \l 125C
  \l 125D
  \l 1260
  \l 1261
  \l 1262
  \l 1263
  \l 1264
  \l 1265
  \l 1266
  \l 1267
  \l 1268
  \l 1269
  \l 126A
  \l 126B
  \l 126C
  \l 126D
  \l 126E
  \l 126F
  \l 1270
  \l 1271
  \l 1272
  \l 1273
  \l 1274
  \l 1275
  \l 1276
  \l 1277
  \l 1278
  \l 1279
  \l 127A
  \l 127B
  \l 127C
  \l 127D
  \l 127E
  \l 127F
  \l 1280
  \l 1281
  \l 1282
  \l 1283
  \l 1284
  \l 1285
  \l 1286
  \l 1287
  \l 1288
  \l 128A
  \l 128B
  \l 128C
  \l 128D
  \l 1290
  \l 1291
  \l 1292
  \l 1293
  \l 1294
  \l 1295
  \l 1296
  \l 1297
  \l 1298
  \l 1299
  \l 129A
  \l 129B
  \l 129C
  \l 129D
  \l 129E
  \l 129F
  \l 12A0
  \l 12A1
  \l 12A2
  \l 12A3
  \l 12A4
  \l 12A5
  \l 12A6
  \l 12A7
  \l 12A8
  \l 12A9
  \l 12AA
  \l 12AB
  \l 12AC
  \l 12AD
  \l 12AE
  \l 12AF
  \l 12B0
  \l 12B2
  \l 12B3
  \l 12B4
  \l 12B5
  \l 12B8
  \l 12B9
  \l 12BA
  \l 12BB
  \l 12BC
  \l 12BD
  \l 12BE
  \l 12C0
  \l 12C2
  \l 12C3
  \l 12C4
  \l 12C5
  \l 12C8
  \l 12C9
  \l 12CA
  \l 12CB
  \l 12CC
  \l 12CD
  \l 12CE
  \l 12CF
  \l 12D0
  \l 12D1
  \l 12D2
  \l 12D3
  \l 12D4
  \l 12D5
  \l 12D6
  \l 12D8
  \l 12D9
  \l 12DA
  \l 12DB
  \l 12DC
  \l 12DD
  \l 12DE
  \l 12DF
  \l 12E0
  \l 12E1
  \l 12E2
  \l 12E3
  \l 12E4
  \l 12E5
  \l 12E6
  \l 12E7
  \l 12E8
  \l 12E9
  \l 12EA
  \l 12EB
  \l 12EC
  \l 12ED
  \l 12EE
  \l 12EF
  \l 12F0
  \l 12F1
  \l 12F2
  \l 12F3
  \l 12F4
  \l 12F5
  \l 12F6
  \l 12F7
  \l 12F8
  \l 12F9
  \l 12FA
  \l 12FB
  \l 12FC
  \l 12FD
  \l 12FE
  \l 12FF
  \l 1300
  \l 1301
  \l 1302
  \l 1303
  \l 1304
  \l 1305
  \l 1306
  \l 1307
  \l 1308
  \l 1309
  \l 130A
  \l 130B
  \l 130C
  \l 130D
  \l 130E
  \l 130F
  \l 1310
  \l 1312
  \l 1313
  \l 1314
  \l 1315
  \l 1318
  \l 1319
  \l 131A
  \l 131B
  \l 131C
  \l 131D
  \l 131E
  \l 131F
  \l 1320
  \l 1321
  \l 1322
  \l 1323
  \l 1324
  \l 1325
  \l 1326
  \l 1327
  \l 1328
  \l 1329
  \l 132A
  \l 132B
  \l 132C
  \l 132D
  \l 132E
  \l 132F
  \l 1330
  \l 1331
  \l 1332
  \l 1333
  \l 1334
  \l 1335
  \l 1336
  \l 1337
  \l 1338
  \l 1339
  \l 133A
  \l 133B
  \l 133C
  \l 133D
  \l 133E
  \l 133F
  \l 1340
  \l 1341
  \l 1342
  \l 1343
  \l 1344
  \l 1345
  \l 1346
  \l 1347
  \l 1348
  \l 1349
  \l 134A
  \l 134B
  \l 134C
  \l 134D
  \l 134E
  \l 134F
  \l 1350
  \l 1351
  \l 1352
  \l 1353
  \l 1354
  \l 1355
  \l 1356
  \l 1357
  \l 1358
  \l 1359
  \l 135A
  \l 135D
  \l 135E
  \l 135F
  \l 1380
  \l 1381
  \l 1382
  \l 1383
  \l 1384
  \l 1385
  \l 1386
  \l 1387
  \l 1388
  \l 1389
  \l 138A
  \l 138B
  \l 138C
  \l 138D
  \l 138E
  \l 138F
  \l 13A0
  \l 13A1
  \l 13A2
  \l 13A3
  \l 13A4
  \l 13A5
  \l 13A6
  \l 13A7
  \l 13A8
  \l 13A9
  \l 13AA
  \l 13AB
  \l 13AC
  \l 13AD
  \l 13AE
  \l 13AF
  \l 13B0
  \l 13B1
  \l 13B2
  \l 13B3
  \l 13B4
  \l 13B5
  \l 13B6
  \l 13B7
  \l 13B8
  \l 13B9
  \l 13BA
  \l 13BB
  \l 13BC
  \l 13BD
  \l 13BE
  \l 13BF
  \l 13C0
  \l 13C1
  \l 13C2
  \l 13C3
  \l 13C4
  \l 13C5
  \l 13C6
  \l 13C7
  \l 13C8
  \l 13C9
  \l 13CA
  \l 13CB
  \l 13CC
  \l 13CD
  \l 13CE
  \l 13CF
  \l 13D0
  \l 13D1
  \l 13D2
  \l 13D3
  \l 13D4
  \l 13D5
  \l 13D6
  \l 13D7
  \l 13D8
  \l 13D9
  \l 13DA
  \l 13DB
  \l 13DC
  \l 13DD
  \l 13DE
  \l 13DF
  \l 13E0
  \l 13E1
  \l 13E2
  \l 13E3
  \l 13E4
  \l 13E5
  \l 13E6
  \l 13E7
  \l 13E8
  \l 13E9
  \l 13EA
  \l 13EB
  \l 13EC
  \l 13ED
  \l 13EE
  \l 13EF
  \l 13F0
  \l 13F1
  \l 13F2
  \l 13F3
  \l 13F4
  \l 1401
  \l 1402
  \l 1403
  \l 1404
  \l 1405
  \l 1406
  \l 1407
  \l 1408
  \l 1409
  \l 140A
  \l 140B
  \l 140C
  \l 140D
  \l 140E
  \l 140F
  \l 1410
  \l 1411
  \l 1412
  \l 1413
  \l 1414
  \l 1415
  \l 1416
  \l 1417
  \l 1418
  \l 1419
  \l 141A
  \l 141B
  \l 141C
  \l 141D
  \l 141E
  \l 141F
  \l 1420
  \l 1421
  \l 1422
  \l 1423
  \l 1424
  \l 1425
  \l 1426
  \l 1427
  \l 1428
  \l 1429
  \l 142A
  \l 142B
  \l 142C
  \l 142D
  \l 142E
  \l 142F
  \l 1430
  \l 1431
  \l 1432
  \l 1433
  \l 1434
  \l 1435
  \l 1436
  \l 1437
  \l 1438
  \l 1439
  \l 143A
  \l 143B
  \l 143C
  \l 143D
  \l 143E
  \l 143F
  \l 1440
  \l 1441
  \l 1442
  \l 1443
  \l 1444
  \l 1445
  \l 1446
  \l 1447
  \l 1448
  \l 1449
  \l 144A
  \l 144B
  \l 144C
  \l 144D
  \l 144E
  \l 144F
  \l 1450
  \l 1451
  \l 1452
  \l 1453
  \l 1454
  \l 1455
  \l 1456
  \l 1457
  \l 1458
  \l 1459
  \l 145A
  \l 145B
  \l 145C
  \l 145D
  \l 145E
  \l 145F
  \l 1460
  \l 1461
  \l 1462
  \l 1463
  \l 1464
  \l 1465
  \l 1466
  \l 1467
  \l 1468
  \l 1469
  \l 146A
  \l 146B
  \l 146C
  \l 146D
  \l 146E
  \l 146F
  \l 1470
  \l 1471
  \l 1472
  \l 1473
  \l 1474
  \l 1475
  \l 1476
  \l 1477
  \l 1478
  \l 1479
  \l 147A
  \l 147B
  \l 147C
  \l 147D
  \l 147E
  \l 147F
  \l 1480
  \l 1481
  \l 1482
  \l 1483
  \l 1484
  \l 1485
  \l 1486
  \l 1487
  \l 1488
  \l 1489
  \l 148A
  \l 148B
  \l 148C
  \l 148D
  \l 148E
  \l 148F
  \l 1490
  \l 1491
  \l 1492
  \l 1493
  \l 1494
  \l 1495
  \l 1496
  \l 1497
  \l 1498
  \l 1499
  \l 149A
  \l 149B
  \l 149C
  \l 149D
  \l 149E
  \l 149F
  \l 14A0
  \l 14A1
  \l 14A2
  \l 14A3
  \l 14A4
  \l 14A5
  \l 14A6
  \l 14A7
  \l 14A8
  \l 14A9
  \l 14AA
  \l 14AB
  \l 14AC
  \l 14AD
  \l 14AE
  \l 14AF
  \l 14B0
  \l 14B1
  \l 14B2
  \l 14B3
  \l 14B4
  \l 14B5
  \l 14B6
  \l 14B7
  \l 14B8
  \l 14B9
  \l 14BA
  \l 14BB
  \l 14BC
  \l 14BD
  \l 14BE
  \l 14BF
  \l 14C0
  \l 14C1
  \l 14C2
  \l 14C3
  \l 14C4
  \l 14C5
  \l 14C6
  \l 14C7
  \l 14C8
  \l 14C9
  \l 14CA
  \l 14CB
  \l 14CC
  \l 14CD
  \l 14CE
  \l 14CF
  \l 14D0
  \l 14D1
  \l 14D2
  \l 14D3
  \l 14D4
  \l 14D5
  \l 14D6
  \l 14D7
  \l 14D8
  \l 14D9
  \l 14DA
  \l 14DB
  \l 14DC
  \l 14DD
  \l 14DE
  \l 14DF
  \l 14E0
  \l 14E1
  \l 14E2
  \l 14E3
  \l 14E4
  \l 14E5
  \l 14E6
  \l 14E7
  \l 14E8
  \l 14E9
  \l 14EA
  \l 14EB
  \l 14EC
  \l 14ED
  \l 14EE
  \l 14EF
  \l 14F0
  \l 14F1
  \l 14F2
  \l 14F3
  \l 14F4
  \l 14F5
  \l 14F6
  \l 14F7
  \l 14F8
  \l 14F9
  \l 14FA
  \l 14FB
  \l 14FC
  \l 14FD
  \l 14FE
  \l 14FF
  \l 1500
  \l 1501
  \l 1502
  \l 1503
  \l 1504
  \l 1505
  \l 1506
  \l 1507
  \l 1508
  \l 1509
  \l 150A
  \l 150B
  \l 150C
  \l 150D
  \l 150E
  \l 150F
  \l 1510
  \l 1511
  \l 1512
  \l 1513
  \l 1514
  \l 1515
  \l 1516
  \l 1517
  \l 1518
  \l 1519
  \l 151A
  \l 151B
  \l 151C
  \l 151D
  \l 151E
  \l 151F
  \l 1520
  \l 1521
  \l 1522
  \l 1523
  \l 1524
  \l 1525
  \l 1526
  \l 1527
  \l 1528
  \l 1529
  \l 152A
  \l 152B
  \l 152C
  \l 152D
  \l 152E
  \l 152F
  \l 1530
  \l 1531
  \l 1532
  \l 1533
  \l 1534
  \l 1535
  \l 1536
  \l 1537
  \l 1538
  \l 1539
  \l 153A
  \l 153B
  \l 153C
  \l 153D
  \l 153E
  \l 153F
  \l 1540
  \l 1541
  \l 1542
  \l 1543
  \l 1544
  \l 1545
  \l 1546
  \l 1547
  \l 1548
  \l 1549
  \l 154A
  \l 154B
  \l 154C
  \l 154D
  \l 154E
  \l 154F
  \l 1550
  \l 1551
  \l 1552
  \l 1553
  \l 1554
  \l 1555
  \l 1556
  \l 1557
  \l 1558
  \l 1559
  \l 155A
  \l 155B
  \l 155C
  \l 155D
  \l 155E
  \l 155F
  \l 1560
  \l 1561
  \l 1562
  \l 1563
  \l 1564
  \l 1565
  \l 1566
  \l 1567
  \l 1568
  \l 1569
  \l 156A
  \l 156B
  \l 156C
  \l 156D
  \l 156E
  \l 156F
  \l 1570
  \l 1571
  \l 1572
  \l 1573
  \l 1574
  \l 1575
  \l 1576
  \l 1577
  \l 1578
  \l 1579
  \l 157A
  \l 157B
  \l 157C
  \l 157D
  \l 157E
  \l 157F
  \l 1580
  \l 1581
  \l 1582
  \l 1583
  \l 1584
  \l 1585
  \l 1586
  \l 1587
  \l 1588
  \l 1589
  \l 158A
  \l 158B
  \l 158C
  \l 158D
  \l 158E
  \l 158F
  \l 1590
  \l 1591
  \l 1592
  \l 1593
  \l 1594
  \l 1595
  \l 1596
  \l 1597
  \l 1598
  \l 1599
  \l 159A
  \l 159B
  \l 159C
  \l 159D
  \l 159E
  \l 159F
  \l 15A0
  \l 15A1
  \l 15A2
  \l 15A3
  \l 15A4
  \l 15A5
  \l 15A6
  \l 15A7
  \l 15A8
  \l 15A9
  \l 15AA
  \l 15AB
  \l 15AC
  \l 15AD
  \l 15AE
  \l 15AF
  \l 15B0
  \l 15B1
  \l 15B2
  \l 15B3
  \l 15B4
  \l 15B5
  \l 15B6
  \l 15B7
  \l 15B8
  \l 15B9
  \l 15BA
  \l 15BB
  \l 15BC
  \l 15BD
  \l 15BE
  \l 15BF
  \l 15C0
  \l 15C1
  \l 15C2
  \l 15C3
  \l 15C4
  \l 15C5
  \l 15C6
  \l 15C7
  \l 15C8
  \l 15C9
  \l 15CA
  \l 15CB
  \l 15CC
  \l 15CD
  \l 15CE
  \l 15CF
  \l 15D0
  \l 15D1
  \l 15D2
  \l 15D3
  \l 15D4
  \l 15D5
  \l 15D6
  \l 15D7
  \l 15D8
  \l 15D9
  \l 15DA
  \l 15DB
  \l 15DC
  \l 15DD
  \l 15DE
  \l 15DF
  \l 15E0
  \l 15E1
  \l 15E2
  \l 15E3
  \l 15E4
  \l 15E5
  \l 15E6
  \l 15E7
  \l 15E8
  \l 15E9
  \l 15EA
  \l 15EB
  \l 15EC
  \l 15ED
  \l 15EE
  \l 15EF
  \l 15F0
  \l 15F1
  \l 15F2
  \l 15F3
  \l 15F4
  \l 15F5
  \l 15F6
  \l 15F7
  \l 15F8
  \l 15F9
  \l 15FA
  \l 15FB
  \l 15FC
  \l 15FD
  \l 15FE
  \l 15FF
  \l 1600
  \l 1601
  \l 1602
  \l 1603
  \l 1604
  \l 1605
  \l 1606
  \l 1607
  \l 1608
  \l 1609
  \l 160A
  \l 160B
  \l 160C
  \l 160D
  \l 160E
  \l 160F
  \l 1610
  \l 1611
  \l 1612
  \l 1613
  \l 1614
  \l 1615
  \l 1616
  \l 1617
  \l 1618
  \l 1619
  \l 161A
  \l 161B
  \l 161C
  \l 161D
  \l 161E
  \l 161F
  \l 1620
  \l 1621
  \l 1622
  \l 1623
  \l 1624
  \l 1625
  \l 1626
  \l 1627
  \l 1628
  \l 1629
  \l 162A
  \l 162B
  \l 162C
  \l 162D
  \l 162E
  \l 162F
  \l 1630
  \l 1631
  \l 1632
  \l 1633
  \l 1634
  \l 1635
  \l 1636
  \l 1637
  \l 1638
  \l 1639
  \l 163A
  \l 163B
  \l 163C
  \l 163D
  \l 163E
  \l 163F
  \l 1640
  \l 1641
  \l 1642
  \l 1643
  \l 1644
  \l 1645
  \l 1646
  \l 1647
  \l 1648
  \l 1649
  \l 164A
  \l 164B
  \l 164C
  \l 164D
  \l 164E
  \l 164F
  \l 1650
  \l 1651
  \l 1652
  \l 1653
  \l 1654
  \l 1655
  \l 1656
  \l 1657
  \l 1658
  \l 1659
  \l 165A
  \l 165B
  \l 165C
  \l 165D
  \l 165E
  \l 165F
  \l 1660
  \l 1661
  \l 1662
  \l 1663
  \l 1664
  \l 1665
  \l 1666
  \l 1667
  \l 1668
  \l 1669
  \l 166A
  \l 166B
  \l 166C
  \l 166F
  \l 1670
  \l 1671
  \l 1672
  \l 1673
  \l 1674
  \l 1675
  \l 1676
  \l 1677
  \l 1678
  \l 1679
  \l 167A
  \l 167B
  \l 167C
  \l 167D
  \l 167E
  \l 167F
  \l 1681
  \l 1682
  \l 1683
  \l 1684
  \l 1685
  \l 1686
  \l 1687
  \l 1688
  \l 1689
  \l 168A
  \l 168B
  \l 168C
  \l 168D
  \l 168E
  \l 168F
  \l 1690
  \l 1691
  \l 1692
  \l 1693
  \l 1694
  \l 1695
  \l 1696
  \l 1697
  \l 1698
  \l 1699
  \l 169A
  \l 16A0
  \l 16A1
  \l 16A2
  \l 16A3
  \l 16A4
  \l 16A5
  \l 16A6
  \l 16A7
  \l 16A8
  \l 16A9
  \l 16AA
  \l 16AB
  \l 16AC
  \l 16AD
  \l 16AE
  \l 16AF
  \l 16B0
  \l 16B1
  \l 16B2
  \l 16B3
  \l 16B4
  \l 16B5
  \l 16B6
  \l 16B7
  \l 16B8
  \l 16B9
  \l 16BA
  \l 16BB
  \l 16BC
  \l 16BD
  \l 16BE
  \l 16BF
  \l 16C0
  \l 16C1
  \l 16C2
  \l 16C3
  \l 16C4
  \l 16C5
  \l 16C6
  \l 16C7
  \l 16C8
  \l 16C9
  \l 16CA
  \l 16CB
  \l 16CC
  \l 16CD
  \l 16CE
  \l 16CF
  \l 16D0
  \l 16D1
  \l 16D2
  \l 16D3
  \l 16D4
  \l 16D5
  \l 16D6
  \l 16D7
  \l 16D8
  \l 16D9
  \l 16DA
  \l 16DB
  \l 16DC
  \l 16DD
  \l 16DE
  \l 16DF
  \l 16E0
  \l 16E1
  \l 16E2
  \l 16E3
  \l 16E4
  \l 16E5
  \l 16E6
  \l 16E7
  \l 16E8
  \l 16E9
  \l 16EA
  \l 16F1
  \l 16F2
  \l 16F3
  \l 16F4
  \l 16F5
  \l 16F6
  \l 16F7
  \l 16F8
  \l 1700
  \l 1701
  \l 1702
  \l 1703
  \l 1704
  \l 1705
  \l 1706
  \l 1707
  \l 1708
  \l 1709
  \l 170A
  \l 170B
  \l 170C
  \l 170E
  \l 170F
  \l 1710
  \l 1711
  \l 1712
  \l 1713
  \l 1714
  \l 1720
  \l 1721
  \l 1722
  \l 1723
  \l 1724
  \l 1725
  \l 1726
  \l 1727
  \l 1728
  \l 1729
  \l 172A
  \l 172B
  \l 172C
  \l 172D
  \l 172E
  \l 172F
  \l 1730
  \l 1731
  \l 1732
  \l 1733
  \l 1734
  \l 1740
  \l 1741
  \l 1742
  \l 1743
  \l 1744
  \l 1745
  \l 1746
  \l 1747
  \l 1748
  \l 1749
  \l 174A
  \l 174B
  \l 174C
  \l 174D
  \l 174E
  \l 174F
  \l 1750
  \l 1751
  \l 1752
  \l 1753
  \l 1760
  \l 1761
  \l 1762
  \l 1763
  \l 1764
  \l 1765
  \l 1766
  \l 1767
  \l 1768
  \l 1769
  \l 176A
  \l 176B
  \l 176C
  \l 176E
  \l 176F
  \l 1770
  \l 1772
  \l 1773
  \l 1780
  \l 1781
  \l 1782
  \l 1783
  \l 1784
  \l 1785
  \l 1786
  \l 1787
  \l 1788
  \l 1789
  \l 178A
  \l 178B
  \l 178C
  \l 178D
  \l 178E
  \l 178F
  \l 1790
  \l 1791
  \l 1792
  \l 1793
  \l 1794
  \l 1795
  \l 1796
  \l 1797
  \l 1798
  \l 1799
  \l 179A
  \l 179B
  \l 179C
  \l 179D
  \l 179E
  \l 179F
  \l 17A0
  \l 17A1
  \l 17A2
  \l 17A3
  \l 17A4
  \l 17A5
  \l 17A6
  \l 17A7
  \l 17A8
  \l 17A9
  \l 17AA
  \l 17AB
  \l 17AC
  \l 17AD
  \l 17AE
  \l 17AF
  \l 17B0
  \l 17B1
  \l 17B2
  \l 17B3
  \l 17B4
  \l 17B5
  \l 17B6
  \l 17B7
  \l 17B8
  \l 17B9
  \l 17BA
  \l 17BB
  \l 17BC
  \l 17BD
  \l 17BE
  \l 17BF
  \l 17C0
  \l 17C1
  \l 17C2
  \l 17C3
  \l 17C4
  \l 17C5
  \l 17C6
  \l 17C7
  \l 17C8
  \l 17C9
  \l 17CA
  \l 17CB
  \l 17CC
  \l 17CD
  \l 17CE
  \l 17CF
  \l 17D0
  \l 17D1
  \l 17D2
  \l 17D3
  \l 17D7
  \l 17DC
  \l 17DD
  \l 180B
  \l 180C
  \l 180D
  \l 1820
  \l 1821
  \l 1822
  \l 1823
  \l 1824
  \l 1825
  \l 1826
  \l 1827
  \l 1828
  \l 1829
  \l 182A
  \l 182B
  \l 182C
  \l 182D
  \l 182E
  \l 182F
  \l 1830
  \l 1831
  \l 1832
  \l 1833
  \l 1834
  \l 1835
  \l 1836
  \l 1837
  \l 1838
  \l 1839
  \l 183A
  \l 183B
  \l 183C
  \l 183D
  \l 183E
  \l 183F
  \l 1840
  \l 1841
  \l 1842
  \l 1843
  \l 1844
  \l 1845
  \l 1846
  \l 1847
  \l 1848
  \l 1849
  \l 184A
  \l 184B
  \l 184C
  \l 184D
  \l 184E
  \l 184F
  \l 1850
  \l 1851
  \l 1852
  \l 1853
  \l 1854
  \l 1855
  \l 1856
  \l 1857
  \l 1858
  \l 1859
  \l 185A
  \l 185B
  \l 185C
  \l 185D
  \l 185E
  \l 185F
  \l 1860
  \l 1861
  \l 1862
  \l 1863
  \l 1864
  \l 1865
  \l 1866
  \l 1867
  \l 1868
  \l 1869
  \l 186A
  \l 186B
  \l 186C
  \l 186D
  \l 186E
  \l 186F
  \l 1870
  \l 1871
  \l 1872
  \l 1873
  \l 1874
  \l 1875
  \l 1876
  \l 1877
  \l 1880
  \l 1881
  \l 1882
  \l 1883
  \l 1884
  \l 1885
  \l 1886
  \l 1887
  \l 1888
  \l 1889
  \l 188A
  \l 188B
  \l 188C
  \l 188D
  \l 188E
  \l 188F
  \l 1890
  \l 1891
  \l 1892
  \l 1893
  \l 1894
  \l 1895
  \l 1896
  \l 1897
  \l 1898
  \l 1899
  \l 189A
  \l 189B
  \l 189C
  \l 189D
  \l 189E
  \l 189F
  \l 18A0
  \l 18A1
  \l 18A2
  \l 18A3
  \l 18A4
  \l 18A5
  \l 18A6
  \l 18A7
  \l 18A8
  \l 18A9
  \l 18AA
  \l 18B0
  \l 18B1
  \l 18B2
  \l 18B3
  \l 18B4
  \l 18B5
  \l 18B6
  \l 18B7
  \l 18B8
  \l 18B9
  \l 18BA
  \l 18BB
  \l 18BC
  \l 18BD
  \l 18BE
  \l 18BF
  \l 18C0
  \l 18C1
  \l 18C2
  \l 18C3
  \l 18C4
  \l 18C5
  \l 18C6
  \l 18C7
  \l 18C8
  \l 18C9
  \l 18CA
  \l 18CB
  \l 18CC
  \l 18CD
  \l 18CE
  \l 18CF
  \l 18D0
  \l 18D1
  \l 18D2
  \l 18D3
  \l 18D4
  \l 18D5
  \l 18D6
  \l 18D7
  \l 18D8
  \l 18D9
  \l 18DA
  \l 18DB
  \l 18DC
  \l 18DD
  \l 18DE
  \l 18DF
  \l 18E0
  \l 18E1
  \l 18E2
  \l 18E3
  \l 18E4
  \l 18E5
  \l 18E6
  \l 18E7
  \l 18E8
  \l 18E9
  \l 18EA
  \l 18EB
  \l 18EC
  \l 18ED
  \l 18EE
  \l 18EF
  \l 18F0
  \l 18F1
  \l 18F2
  \l 18F3
  \l 18F4
  \l 18F5
  \l 1900
  \l 1901
  \l 1902
  \l 1903
  \l 1904
  \l 1905
  \l 1906
  \l 1907
  \l 1908
  \l 1909
  \l 190A
  \l 190B
  \l 190C
  \l 190D
  \l 190E
  \l 190F
  \l 1910
  \l 1911
  \l 1912
  \l 1913
  \l 1914
  \l 1915
  \l 1916
  \l 1917
  \l 1918
  \l 1919
  \l 191A
  \l 191B
  \l 191C
  \l 191D
  \l 191E
  \l 1920
  \l 1921
  \l 1922
  \l 1923
  \l 1924
  \l 1925
  \l 1926
  \l 1927
  \l 1928
  \l 1929
  \l 192A
  \l 192B
  \l 1930
  \l 1931
  \l 1932
  \l 1933
  \l 1934
  \l 1935
  \l 1936
  \l 1937
  \l 1938
  \l 1939
  \l 193A
  \l 193B
  \l 1950
  \l 1951
  \l 1952
  \l 1953
  \l 1954
  \l 1955
  \l 1956
  \l 1957
  \l 1958
  \l 1959
  \l 195A
  \l 195B
  \l 195C
  \l 195D
  \l 195E
  \l 195F
  \l 1960
  \l 1961
  \l 1962
  \l 1963
  \l 1964
  \l 1965
  \l 1966
  \l 1967
  \l 1968
  \l 1969
  \l 196A
  \l 196B
  \l 196C
  \l 196D
  \l 1970
  \l 1971
  \l 1972
  \l 1973
  \l 1974
  \l 1980
  \l 1981
  \l 1982
  \l 1983
  \l 1984
  \l 1985
  \l 1986
  \l 1987
  \l 1988
  \l 1989
  \l 198A
  \l 198B
  \l 198C
  \l 198D
  \l 198E
  \l 198F
  \l 1990
  \l 1991
  \l 1992
  \l 1993
  \l 1994
  \l 1995
  \l 1996
  \l 1997
  \l 1998
  \l 1999
  \l 199A
  \l 199B
  \l 199C
  \l 199D
  \l 199E
  \l 199F
  \l 19A0
  \l 19A1
  \l 19A2
  \l 19A3
  \l 19A4
  \l 19A5
  \l 19A6
  \l 19A7
  \l 19A8
  \l 19A9
  \l 19AA
  \l 19AB
  \l 19B0
  \l 19B1
  \l 19B2
  \l 19B3
  \l 19B4
  \l 19B5
  \l 19B6
  \l 19B7
  \l 19B8
  \l 19B9
  \l 19BA
  \l 19BB
  \l 19BC
  \l 19BD
  \l 19BE
  \l 19BF
  \l 19C0
  \l 19C1
  \l 19C2
  \l 19C3
  \l 19C4
  \l 19C5
  \l 19C6
  \l 19C7
  \l 19C8
  \l 19C9
  \l 1A00
  \l 1A01
  \l 1A02
  \l 1A03
  \l 1A04
  \l 1A05
  \l 1A06
  \l 1A07
  \l 1A08
  \l 1A09
  \l 1A0A
  \l 1A0B
  \l 1A0C
  \l 1A0D
  \l 1A0E
  \l 1A0F
  \l 1A10
  \l 1A11
  \l 1A12
  \l 1A13
  \l 1A14
  \l 1A15
  \l 1A16
  \l 1A17
  \l 1A18
  \l 1A19
  \l 1A1A
  \l 1A1B
  \l 1A20
  \l 1A21
  \l 1A22
  \l 1A23
  \l 1A24
  \l 1A25
  \l 1A26
  \l 1A27
  \l 1A28
  \l 1A29
  \l 1A2A
  \l 1A2B
  \l 1A2C
  \l 1A2D
  \l 1A2E
  \l 1A2F
  \l 1A30
  \l 1A31
  \l 1A32
  \l 1A33
  \l 1A34
  \l 1A35
  \l 1A36
  \l 1A37
  \l 1A38
  \l 1A39
  \l 1A3A
  \l 1A3B
  \l 1A3C
  \l 1A3D
  \l 1A3E
  \l 1A3F
  \l 1A40
  \l 1A41
  \l 1A42
  \l 1A43
  \l 1A44
  \l 1A45
  \l 1A46
  \l 1A47
  \l 1A48
  \l 1A49
  \l 1A4A
  \l 1A4B
  \l 1A4C
  \l 1A4D
  \l 1A4E
  \l 1A4F
  \l 1A50
  \l 1A51
  \l 1A52
  \l 1A53
  \l 1A54
  \l 1A55
  \l 1A56
  \l 1A57
  \l 1A58
  \l 1A59
  \l 1A5A
  \l 1A5B
  \l 1A5C
  \l 1A5D
  \l 1A5E
  \l 1A60
  \l 1A61
  \l 1A62
  \l 1A63
  \l 1A64
  \l 1A65
  \l 1A66
  \l 1A67
  \l 1A68
  \l 1A69
  \l 1A6A
  \l 1A6B
  \l 1A6C
  \l 1A6D
  \l 1A6E
  \l 1A6F
  \l 1A70
  \l 1A71
  \l 1A72
  \l 1A73
  \l 1A74
  \l 1A75
  \l 1A76
  \l 1A77
  \l 1A78
  \l 1A79
  \l 1A7A
  \l 1A7B
  \l 1A7C
  \l 1A7F
  \l 1AA7
  \l 1AB0
  \l 1AB1
  \l 1AB2
  \l 1AB3
  \l 1AB4
  \l 1AB5
  \l 1AB6
  \l 1AB7
  \l 1AB8
  \l 1AB9
  \l 1ABA
  \l 1ABB
  \l 1ABC
  \l 1ABD
  \l 1ABE
  \l 1B00
  \l 1B01
  \l 1B02
  \l 1B03
  \l 1B04
  \l 1B05
  \l 1B06
  \l 1B07
  \l 1B08
  \l 1B09
  \l 1B0A
  \l 1B0B
  \l 1B0C
  \l 1B0D
  \l 1B0E
  \l 1B0F
  \l 1B10
  \l 1B11
  \l 1B12
  \l 1B13
  \l 1B14
  \l 1B15
  \l 1B16
  \l 1B17
  \l 1B18
  \l 1B19
  \l 1B1A
  \l 1B1B
  \l 1B1C
  \l 1B1D
  \l 1B1E
  \l 1B1F
  \l 1B20
  \l 1B21
  \l 1B22
  \l 1B23
  \l 1B24
  \l 1B25
  \l 1B26
  \l 1B27
  \l 1B28
  \l 1B29
  \l 1B2A
  \l 1B2B
  \l 1B2C
  \l 1B2D
  \l 1B2E
  \l 1B2F
  \l 1B30
  \l 1B31
  \l 1B32
  \l 1B33
  \l 1B34
  \l 1B35
  \l 1B36
  \l 1B37
  \l 1B38
  \l 1B39
  \l 1B3A
  \l 1B3B
  \l 1B3C
  \l 1B3D
  \l 1B3E
  \l 1B3F
  \l 1B40
  \l 1B41
  \l 1B42
  \l 1B43
  \l 1B44
  \l 1B45
  \l 1B46
  \l 1B47
  \l 1B48
  \l 1B49
  \l 1B4A
  \l 1B4B
  \l 1B6B
  \l 1B6C
  \l 1B6D
  \l 1B6E
  \l 1B6F
  \l 1B70
  \l 1B71
  \l 1B72
  \l 1B73
  \l 1B80
  \l 1B81
  \l 1B82
  \l 1B83
  \l 1B84
  \l 1B85
  \l 1B86
  \l 1B87
  \l 1B88
  \l 1B89
  \l 1B8A
  \l 1B8B
  \l 1B8C
  \l 1B8D
  \l 1B8E
  \l 1B8F
  \l 1B90
  \l 1B91
  \l 1B92
  \l 1B93
  \l 1B94
  \l 1B95
  \l 1B96
  \l 1B97
  \l 1B98
  \l 1B99
  \l 1B9A
  \l 1B9B
  \l 1B9C
  \l 1B9D
  \l 1B9E
  \l 1B9F
  \l 1BA0
  \l 1BA1
  \l 1BA2
  \l 1BA3
  \l 1BA4
  \l 1BA5
  \l 1BA6
  \l 1BA7
  \l 1BA8
  \l 1BA9
  \l 1BAA
  \l 1BAB
  \l 1BAC
  \l 1BAD
  \l 1BAE
  \l 1BAF
  \l 1BBA
  \l 1BBB
  \l 1BBC
  \l 1BBD
  \l 1BBE
  \l 1BBF
  \l 1BC0
  \l 1BC1
  \l 1BC2
  \l 1BC3
  \l 1BC4
  \l 1BC5
  \l 1BC6
  \l 1BC7
  \l 1BC8
  \l 1BC9
  \l 1BCA
  \l 1BCB
  \l 1BCC
  \l 1BCD
  \l 1BCE
  \l 1BCF
  \l 1BD0
  \l 1BD1
  \l 1BD2
  \l 1BD3
  \l 1BD4
  \l 1BD5
  \l 1BD6
  \l 1BD7
  \l 1BD8
  \l 1BD9
  \l 1BDA
  \l 1BDB
  \l 1BDC
  \l 1BDD
  \l 1BDE
  \l 1BDF
  \l 1BE0
  \l 1BE1
  \l 1BE2
  \l 1BE3
  \l 1BE4
  \l 1BE5
  \l 1BE6
  \l 1BE7
  \l 1BE8
  \l 1BE9
  \l 1BEA
  \l 1BEB
  \l 1BEC
  \l 1BED
  \l 1BEE
  \l 1BEF
  \l 1BF0
  \l 1BF1
  \l 1BF2
  \l 1BF3
  \l 1C00
  \l 1C01
  \l 1C02
  \l 1C03
  \l 1C04
  \l 1C05
  \l 1C06
  \l 1C07
  \l 1C08
  \l 1C09
  \l 1C0A
  \l 1C0B
  \l 1C0C
  \l 1C0D
  \l 1C0E
  \l 1C0F
  \l 1C10
  \l 1C11
  \l 1C12
  \l 1C13
  \l 1C14
  \l 1C15
  \l 1C16
  \l 1C17
  \l 1C18
  \l 1C19
  \l 1C1A
  \l 1C1B
  \l 1C1C
  \l 1C1D
  \l 1C1E
  \l 1C1F
  \l 1C20
  \l 1C21
  \l 1C22
  \l 1C23
  \l 1C24
  \l 1C25
  \l 1C26
  \l 1C27
  \l 1C28
  \l 1C29
  \l 1C2A
  \l 1C2B
  \l 1C2C
  \l 1C2D
  \l 1C2E
  \l 1C2F
  \l 1C30
  \l 1C31
  \l 1C32
  \l 1C33
  \l 1C34
  \l 1C35
  \l 1C36
  \l 1C37
  \l 1C4D
  \l 1C4E
  \l 1C4F
  \l 1C5A
  \l 1C5B
  \l 1C5C
  \l 1C5D
  \l 1C5E
  \l 1C5F
  \l 1C60
  \l 1C61
  \l 1C62
  \l 1C63
  \l 1C64
  \l 1C65
  \l 1C66
  \l 1C67
  \l 1C68
  \l 1C69
  \l 1C6A
  \l 1C6B
  \l 1C6C
  \l 1C6D
  \l 1C6E
  \l 1C6F
  \l 1C70
  \l 1C71
  \l 1C72
  \l 1C73
  \l 1C74
  \l 1C75
  \l 1C76
  \l 1C77
  \l 1C78
  \l 1C79
  \l 1C7A
  \l 1C7B
  \l 1C7C
  \l 1C7D
  \l 1CD0
  \l 1CD1
  \l 1CD2
  \l 1CD4
  \l 1CD5
  \l 1CD6
  \l 1CD7
  \l 1CD8
  \l 1CD9
  \l 1CDA
  \l 1CDB
  \l 1CDC
  \l 1CDD
  \l 1CDE
  \l 1CDF
  \l 1CE0
  \l 1CE1
  \l 1CE2
  \l 1CE3
  \l 1CE4
  \l 1CE5
  \l 1CE6
  \l 1CE7
  \l 1CE8
  \l 1CE9
  \l 1CEA
  \l 1CEB
  \l 1CEC
  \l 1CED
  \l 1CEE
  \l 1CEF
  \l 1CF0
  \l 1CF1
  \l 1CF2
  \l 1CF3
  \l 1CF4
  \l 1CF5
  \l 1CF6
  \l 1CF8
  \l 1CF9
  \l 1D00
  \l 1D01
  \l 1D02
  \l 1D03
  \l 1D04
  \l 1D05
  \l 1D06
  \l 1D07
  \l 1D08
  \l 1D09
  \l 1D0A
  \l 1D0B
  \l 1D0C
  \l 1D0D
  \l 1D0E
  \l 1D0F
  \l 1D10
  \l 1D11
  \l 1D12
  \l 1D13
  \l 1D14
  \l 1D15
  \l 1D16
  \l 1D17
  \l 1D18
  \l 1D19
  \l 1D1A
  \l 1D1B
  \l 1D1C
  \l 1D1D
  \l 1D1E
  \l 1D1F
  \l 1D20
  \l 1D21
  \l 1D22
  \l 1D23
  \l 1D24
  \l 1D25
  \l 1D26
  \l 1D27
  \l 1D28
  \l 1D29
  \l 1D2A
  \l 1D2B
  \l 1D2C
  \l 1D2D
  \l 1D2E
  \l 1D2F
  \l 1D30
  \l 1D31
  \l 1D32
  \l 1D33
  \l 1D34
  \l 1D35
  \l 1D36
  \l 1D37
  \l 1D38
  \l 1D39
  \l 1D3A
  \l 1D3B
  \l 1D3C
  \l 1D3D
  \l 1D3E
  \l 1D3F
  \l 1D40
  \l 1D41
  \l 1D42
  \l 1D43
  \l 1D44
  \l 1D45
  \l 1D46
  \l 1D47
  \l 1D48
  \l 1D49
  \l 1D4A
  \l 1D4B
  \l 1D4C
  \l 1D4D
  \l 1D4E
  \l 1D4F
  \l 1D50
  \l 1D51
  \l 1D52
  \l 1D53
  \l 1D54
  \l 1D55
  \l 1D56
  \l 1D57
  \l 1D58
  \l 1D59
  \l 1D5A
  \l 1D5B
  \l 1D5C
  \l 1D5D
  \l 1D5E
  \l 1D5F
  \l 1D60
  \l 1D61
  \l 1D62
  \l 1D63
  \l 1D64
  \l 1D65
  \l 1D66
  \l 1D67
  \l 1D68
  \l 1D69
  \l 1D6A
  \l 1D6B
  \l 1D6C
  \l 1D6D
  \l 1D6E
  \l 1D6F
  \l 1D70
  \l 1D71
  \l 1D72
  \l 1D73
  \l 1D74
  \l 1D75
  \l 1D76
  \l 1D77
  \l 1D78
  \L 1D79 A77D 1D79
  \l 1D7A
  \l 1D7B
  \l 1D7C
  \L 1D7D 2C63 1D7D
  \l 1D7E
  \l 1D7F
  \l 1D80
  \l 1D81
  \l 1D82
  \l 1D83
  \l 1D84
  \l 1D85
  \l 1D86
  \l 1D87
  \l 1D88
  \l 1D89
  \l 1D8A
  \l 1D8B
  \l 1D8C
  \l 1D8D
  \l 1D8E
  \l 1D8F
  \l 1D90
  \l 1D91
  \l 1D92
  \l 1D93
  \l 1D94
  \l 1D95
  \l 1D96
  \l 1D97
  \l 1D98
  \l 1D99
  \l 1D9A
  \l 1D9B
  \l 1D9C
  \l 1D9D
  \l 1D9E
  \l 1D9F
  \l 1DA0
  \l 1DA1
  \l 1DA2
  \l 1DA3
  \l 1DA4
  \l 1DA5
  \l 1DA6
  \l 1DA7
  \l 1DA8
  \l 1DA9
  \l 1DAA
  \l 1DAB
  \l 1DAC
  \l 1DAD
  \l 1DAE
  \l 1DAF
  \l 1DB0
  \l 1DB1
  \l 1DB2
  \l 1DB3
  \l 1DB4
  \l 1DB5
  \l 1DB6
  \l 1DB7
  \l 1DB8
  \l 1DB9
  \l 1DBA
  \l 1DBB
  \l 1DBC
  \l 1DBD
  \l 1DBE
  \l 1DBF
  \l 1DC0
  \l 1DC1
  \l 1DC2
  \l 1DC3
  \l 1DC4
  \l 1DC5
  \l 1DC6
  \l 1DC7
  \l 1DC8
  \l 1DC9
  \l 1DCA
  \l 1DCB
  \l 1DCC
  \l 1DCD
  \l 1DCE
  \l 1DCF
  \l 1DD0
  \l 1DD1
  \l 1DD2
  \l 1DD3
  \l 1DD4
  \l 1DD5
  \l 1DD6
  \l 1DD7
  \l 1DD8
  \l 1DD9
  \l 1DDA
  \l 1DDB
  \l 1DDC
  \l 1DDD
  \l 1DDE
  \l 1DDF
  \l 1DE0
  \l 1DE1
  \l 1DE2
  \l 1DE3
  \l 1DE4
  \l 1DE5
  \l 1DE6
  \l 1DE7
  \l 1DE8
  \l 1DE9
  \l 1DEA
  \l 1DEB
  \l 1DEC
  \l 1DED
  \l 1DEE
  \l 1DEF
  \l 1DF0
  \l 1DF1
  \l 1DF2
  \l 1DF3
  \l 1DF4
  \l 1DF5
  \l 1DFC
  \l 1DFD
  \l 1DFE
  \l 1DFF
  \L 1E00 1E00 1E01
  \L 1E01 1E00 1E01
  \L 1E02 1E02 1E03
  \L 1E03 1E02 1E03
  \L 1E04 1E04 1E05
  \L 1E05 1E04 1E05
  \L 1E06 1E06 1E07
  \L 1E07 1E06 1E07
  \L 1E08 1E08 1E09
  \L 1E09 1E08 1E09
  \L 1E0A 1E0A 1E0B
  \L 1E0B 1E0A 1E0B
  \L 1E0C 1E0C 1E0D
  \L 1E0D 1E0C 1E0D
  \L 1E0E 1E0E 1E0F
  \L 1E0F 1E0E 1E0F
  \L 1E10 1E10 1E11
  \L 1E11 1E10 1E11
  \L 1E12 1E12 1E13
  \L 1E13 1E12 1E13
  \L 1E14 1E14 1E15
  \L 1E15 1E14 1E15
  \L 1E16 1E16 1E17
  \L 1E17 1E16 1E17
  \L 1E18 1E18 1E19
  \L 1E19 1E18 1E19
  \L 1E1A 1E1A 1E1B
  \L 1E1B 1E1A 1E1B
  \L 1E1C 1E1C 1E1D
  \L 1E1D 1E1C 1E1D
  \L 1E1E 1E1E 1E1F
  \L 1E1F 1E1E 1E1F
  \L 1E20 1E20 1E21
  \L 1E21 1E20 1E21
  \L 1E22 1E22 1E23
  \L 1E23 1E22 1E23
  \L 1E24 1E24 1E25
  \L 1E25 1E24 1E25
  \L 1E26 1E26 1E27
  \L 1E27 1E26 1E27
  \L 1E28 1E28 1E29
  \L 1E29 1E28 1E29
  \L 1E2A 1E2A 1E2B
  \L 1E2B 1E2A 1E2B
  \L 1E2C 1E2C 1E2D
  \L 1E2D 1E2C 1E2D
  \L 1E2E 1E2E 1E2F
  \L 1E2F 1E2E 1E2F
  \L 1E30 1E30 1E31
  \L 1E31 1E30 1E31
  \L 1E32 1E32 1E33
  \L 1E33 1E32 1E33
  \L 1E34 1E34 1E35
  \L 1E35 1E34 1E35
  \L 1E36 1E36 1E37
  \L 1E37 1E36 1E37
  \L 1E38 1E38 1E39
  \L 1E39 1E38 1E39
  \L 1E3A 1E3A 1E3B
  \L 1E3B 1E3A 1E3B
  \L 1E3C 1E3C 1E3D
  \L 1E3D 1E3C 1E3D
  \L 1E3E 1E3E 1E3F
  \L 1E3F 1E3E 1E3F
  \L 1E40 1E40 1E41
  \L 1E41 1E40 1E41
  \L 1E42 1E42 1E43
  \L 1E43 1E42 1E43
  \L 1E44 1E44 1E45
  \L 1E45 1E44 1E45
  \L 1E46 1E46 1E47
  \L 1E47 1E46 1E47
  \L 1E48 1E48 1E49
  \L 1E49 1E48 1E49
  \L 1E4A 1E4A 1E4B
  \L 1E4B 1E4A 1E4B
  \L 1E4C 1E4C 1E4D
  \L 1E4D 1E4C 1E4D
  \L 1E4E 1E4E 1E4F
  \L 1E4F 1E4E 1E4F
  \L 1E50 1E50 1E51
  \L 1E51 1E50 1E51
  \L 1E52 1E52 1E53
  \L 1E53 1E52 1E53
  \L 1E54 1E54 1E55
  \L 1E55 1E54 1E55
  \L 1E56 1E56 1E57
  \L 1E57 1E56 1E57
  \L 1E58 1E58 1E59
  \L 1E59 1E58 1E59
  \L 1E5A 1E5A 1E5B
  \L 1E5B 1E5A 1E5B
  \L 1E5C 1E5C 1E5D
  \L 1E5D 1E5C 1E5D
  \L 1E5E 1E5E 1E5F
  \L 1E5F 1E5E 1E5F
  \L 1E60 1E60 1E61
  \L 1E61 1E60 1E61
  \L 1E62 1E62 1E63
  \L 1E63 1E62 1E63
  \L 1E64 1E64 1E65
  \L 1E65 1E64 1E65
  \L 1E66 1E66 1E67
  \L 1E67 1E66 1E67
  \L 1E68 1E68 1E69
  \L 1E69 1E68 1E69
  \L 1E6A 1E6A 1E6B
  \L 1E6B 1E6A 1E6B
  \L 1E6C 1E6C 1E6D
  \L 1E6D 1E6C 1E6D
  \L 1E6E 1E6E 1E6F
  \L 1E6F 1E6E 1E6F
  \L 1E70 1E70 1E71
  \L 1E71 1E70 1E71
  \L 1E72 1E72 1E73
  \L 1E73 1E72 1E73
  \L 1E74 1E74 1E75
  \L 1E75 1E74 1E75
  \L 1E76 1E76 1E77
  \L 1E77 1E76 1E77
  \L 1E78 1E78 1E79
  \L 1E79 1E78 1E79
  \L 1E7A 1E7A 1E7B
  \L 1E7B 1E7A 1E7B
  \L 1E7C 1E7C 1E7D
  \L 1E7D 1E7C 1E7D
  \L 1E7E 1E7E 1E7F
  \L 1E7F 1E7E 1E7F
  \L 1E80 1E80 1E81
  \L 1E81 1E80 1E81
  \L 1E82 1E82 1E83
  \L 1E83 1E82 1E83
  \L 1E84 1E84 1E85
  \L 1E85 1E84 1E85
  \L 1E86 1E86 1E87
  \L 1E87 1E86 1E87
  \L 1E88 1E88 1E89
  \L 1E89 1E88 1E89
  \L 1E8A 1E8A 1E8B
  \L 1E8B 1E8A 1E8B
  \L 1E8C 1E8C 1E8D
  \L 1E8D 1E8C 1E8D
  \L 1E8E 1E8E 1E8F
  \L 1E8F 1E8E 1E8F
  \L 1E90 1E90 1E91
  \L 1E91 1E90 1E91
  \L 1E92 1E92 1E93
  \L 1E93 1E92 1E93
  \L 1E94 1E94 1E95
  \L 1E95 1E94 1E95
  \l 1E96
  \l 1E97
  \l 1E98
  \l 1E99
  \l 1E9A
  \L 1E9B 1E60 1E9B
  \l 1E9C
  \l 1E9D
  \L 1E9E 1E9E 00DF
  \l 1E9F
  \L 1EA0 1EA0 1EA1
  \L 1EA1 1EA0 1EA1
  \L 1EA2 1EA2 1EA3
  \L 1EA3 1EA2 1EA3
  \L 1EA4 1EA4 1EA5
  \L 1EA5 1EA4 1EA5
  \L 1EA6 1EA6 1EA7
  \L 1EA7 1EA6 1EA7
  \L 1EA8 1EA8 1EA9
  \L 1EA9 1EA8 1EA9
  \L 1EAA 1EAA 1EAB
  \L 1EAB 1EAA 1EAB
  \L 1EAC 1EAC 1EAD
  \L 1EAD 1EAC 1EAD
  \L 1EAE 1EAE 1EAF
  \L 1EAF 1EAE 1EAF
  \L 1EB0 1EB0 1EB1
  \L 1EB1 1EB0 1EB1
  \L 1EB2 1EB2 1EB3
  \L 1EB3 1EB2 1EB3
  \L 1EB4 1EB4 1EB5
  \L 1EB5 1EB4 1EB5
  \L 1EB6 1EB6 1EB7
  \L 1EB7 1EB6 1EB7
  \L 1EB8 1EB8 1EB9
  \L 1EB9 1EB8 1EB9
  \L 1EBA 1EBA 1EBB
  \L 1EBB 1EBA 1EBB
  \L 1EBC 1EBC 1EBD
  \L 1EBD 1EBC 1EBD
  \L 1EBE 1EBE 1EBF
  \L 1EBF 1EBE 1EBF
  \L 1EC0 1EC0 1EC1
  \L 1EC1 1EC0 1EC1
  \L 1EC2 1EC2 1EC3
  \L 1EC3 1EC2 1EC3
  \L 1EC4 1EC4 1EC5
  \L 1EC5 1EC4 1EC5
  \L 1EC6 1EC6 1EC7
  \L 1EC7 1EC6 1EC7
  \L 1EC8 1EC8 1EC9
  \L 1EC9 1EC8 1EC9
  \L 1ECA 1ECA 1ECB
  \L 1ECB 1ECA 1ECB
  \L 1ECC 1ECC 1ECD
  \L 1ECD 1ECC 1ECD
  \L 1ECE 1ECE 1ECF
  \L 1ECF 1ECE 1ECF
  \L 1ED0 1ED0 1ED1
  \L 1ED1 1ED0 1ED1
  \L 1ED2 1ED2 1ED3
  \L 1ED3 1ED2 1ED3
  \L 1ED4 1ED4 1ED5
  \L 1ED5 1ED4 1ED5
  \L 1ED6 1ED6 1ED7
  \L 1ED7 1ED6 1ED7
  \L 1ED8 1ED8 1ED9
  \L 1ED9 1ED8 1ED9
  \L 1EDA 1EDA 1EDB
  \L 1EDB 1EDA 1EDB
  \L 1EDC 1EDC 1EDD
  \L 1EDD 1EDC 1EDD
  \L 1EDE 1EDE 1EDF
  \L 1EDF 1EDE 1EDF
  \L 1EE0 1EE0 1EE1
  \L 1EE1 1EE0 1EE1
  \L 1EE2 1EE2 1EE3
  \L 1EE3 1EE2 1EE3
  \L 1EE4 1EE4 1EE5
  \L 1EE5 1EE4 1EE5
  \L 1EE6 1EE6 1EE7
  \L 1EE7 1EE6 1EE7
  \L 1EE8 1EE8 1EE9
  \L 1EE9 1EE8 1EE9
  \L 1EEA 1EEA 1EEB
  \L 1EEB 1EEA 1EEB
  \L 1EEC 1EEC 1EED
  \L 1EED 1EEC 1EED
  \L 1EEE 1EEE 1EEF
  \L 1EEF 1EEE 1EEF
  \L 1EF0 1EF0 1EF1
  \L 1EF1 1EF0 1EF1
  \L 1EF2 1EF2 1EF3
  \L 1EF3 1EF2 1EF3
  \L 1EF4 1EF4 1EF5
  \L 1EF5 1EF4 1EF5
  \L 1EF6 1EF6 1EF7
  \L 1EF7 1EF6 1EF7
  \L 1EF8 1EF8 1EF9
  \L 1EF9 1EF8 1EF9
  \L 1EFA 1EFA 1EFB
  \L 1EFB 1EFA 1EFB
  \L 1EFC 1EFC 1EFD
  \L 1EFD 1EFC 1EFD
  \L 1EFE 1EFE 1EFF
  \L 1EFF 1EFE 1EFF
  \L 1F00 1F08 1F00
  \L 1F01 1F09 1F01
  \L 1F02 1F0A 1F02
  \L 1F03 1F0B 1F03
  \L 1F04 1F0C 1F04
  \L 1F05 1F0D 1F05
  \L 1F06 1F0E 1F06
  \L 1F07 1F0F 1F07
  \L 1F08 1F08 1F00
  \L 1F09 1F09 1F01
  \L 1F0A 1F0A 1F02
  \L 1F0B 1F0B 1F03
  \L 1F0C 1F0C 1F04
  \L 1F0D 1F0D 1F05
  \L 1F0E 1F0E 1F06
  \L 1F0F 1F0F 1F07
  \L 1F10 1F18 1F10
  \L 1F11 1F19 1F11
  \L 1F12 1F1A 1F12
  \L 1F13 1F1B 1F13
  \L 1F14 1F1C 1F14
  \L 1F15 1F1D 1F15
  \L 1F18 1F18 1F10
  \L 1F19 1F19 1F11
  \L 1F1A 1F1A 1F12
  \L 1F1B 1F1B 1F13
  \L 1F1C 1F1C 1F14
  \L 1F1D 1F1D 1F15
  \L 1F20 1F28 1F20
  \L 1F21 1F29 1F21
  \L 1F22 1F2A 1F22
  \L 1F23 1F2B 1F23
  \L 1F24 1F2C 1F24
  \L 1F25 1F2D 1F25
  \L 1F26 1F2E 1F26
  \L 1F27 1F2F 1F27
  \L 1F28 1F28 1F20
  \L 1F29 1F29 1F21
  \L 1F2A 1F2A 1F22
  \L 1F2B 1F2B 1F23
  \L 1F2C 1F2C 1F24
  \L 1F2D 1F2D 1F25
  \L 1F2E 1F2E 1F26
  \L 1F2F 1F2F 1F27
  \L 1F30 1F38 1F30
  \L 1F31 1F39 1F31
  \L 1F32 1F3A 1F32
  \L 1F33 1F3B 1F33
  \L 1F34 1F3C 1F34
  \L 1F35 1F3D 1F35
  \L 1F36 1F3E 1F36
  \L 1F37 1F3F 1F37
  \L 1F38 1F38 1F30
  \L 1F39 1F39 1F31
  \L 1F3A 1F3A 1F32
  \L 1F3B 1F3B 1F33
  \L 1F3C 1F3C 1F34
  \L 1F3D 1F3D 1F35
  \L 1F3E 1F3E 1F36
  \L 1F3F 1F3F 1F37
  \L 1F40 1F48 1F40
  \L 1F41 1F49 1F41
  \L 1F42 1F4A 1F42
  \L 1F43 1F4B 1F43
  \L 1F44 1F4C 1F44
  \L 1F45 1F4D 1F45
  \L 1F48 1F48 1F40
  \L 1F49 1F49 1F41
  \L 1F4A 1F4A 1F42
  \L 1F4B 1F4B 1F43
  \L 1F4C 1F4C 1F44
  \L 1F4D 1F4D 1F45
  \l 1F50
  \L 1F51 1F59 1F51
  \l 1F52
  \L 1F53 1F5B 1F53
  \l 1F54
  \L 1F55 1F5D 1F55
  \l 1F56
  \L 1F57 1F5F 1F57
  \L 1F59 1F59 1F51
  \L 1F5B 1F5B 1F53
  \L 1F5D 1F5D 1F55
  \L 1F5F 1F5F 1F57
  \L 1F60 1F68 1F60
  \L 1F61 1F69 1F61
  \L 1F62 1F6A 1F62
  \L 1F63 1F6B 1F63
  \L 1F64 1F6C 1F64
  \L 1F65 1F6D 1F65
  \L 1F66 1F6E 1F66
  \L 1F67 1F6F 1F67
  \L 1F68 1F68 1F60
  \L 1F69 1F69 1F61
  \L 1F6A 1F6A 1F62
  \L 1F6B 1F6B 1F63
  \L 1F6C 1F6C 1F64
  \L 1F6D 1F6D 1F65
  \L 1F6E 1F6E 1F66
  \L 1F6F 1F6F 1F67
  \L 1F70 1FBA 1F70
  \L 1F71 1FBB 1F71
  \L 1F72 1FC8 1F72
  \L 1F73 1FC9 1F73
  \L 1F74 1FCA 1F74
  \L 1F75 1FCB 1F75
  \L 1F76 1FDA 1F76
  \L 1F77 1FDB 1F77
  \L 1F78 1FF8 1F78
  \L 1F79 1FF9 1F79
  \L 1F7A 1FEA 1F7A
  \L 1F7B 1FEB 1F7B
  \L 1F7C 1FFA 1F7C
  \L 1F7D 1FFB 1F7D
  \L 1F80 1F88 1F80
  \L 1F81 1F89 1F81
  \L 1F82 1F8A 1F82
  \L 1F83 1F8B 1F83
  \L 1F84 1F8C 1F84
  \L 1F85 1F8D 1F85
  \L 1F86 1F8E 1F86
  \L 1F87 1F8F 1F87
  \L 1F88 1F88 1F80
  \L 1F89 1F89 1F81
  \L 1F8A 1F8A 1F82
  \L 1F8B 1F8B 1F83
  \L 1F8C 1F8C 1F84
  \L 1F8D 1F8D 1F85
  \L 1F8E 1F8E 1F86
  \L 1F8F 1F8F 1F87
  \L 1F90 1F98 1F90
  \L 1F91 1F99 1F91
  \L 1F92 1F9A 1F92
  \L 1F93 1F9B 1F93
  \L 1F94 1F9C 1F94
  \L 1F95 1F9D 1F95
  \L 1F96 1F9E 1F96
  \L 1F97 1F9F 1F97
  \L 1F98 1F98 1F90
  \L 1F99 1F99 1F91
  \L 1F9A 1F9A 1F92
  \L 1F9B 1F9B 1F93
  \L 1F9C 1F9C 1F94
  \L 1F9D 1F9D 1F95
  \L 1F9E 1F9E 1F96
  \L 1F9F 1F9F 1F97
  \L 1FA0 1FA8 1FA0
  \L 1FA1 1FA9 1FA1
  \L 1FA2 1FAA 1FA2
  \L 1FA3 1FAB 1FA3
  \L 1FA4 1FAC 1FA4
  \L 1FA5 1FAD 1FA5
  \L 1FA6 1FAE 1FA6
  \L 1FA7 1FAF 1FA7
  \L 1FA8 1FA8 1FA0
  \L 1FA9 1FA9 1FA1
  \L 1FAA 1FAA 1FA2
  \L 1FAB 1FAB 1FA3
  \L 1FAC 1FAC 1FA4
  \L 1FAD 1FAD 1FA5
  \L 1FAE 1FAE 1FA6
  \L 1FAF 1FAF 1FA7
  \L 1FB0 1FB8 1FB0
  \L 1FB1 1FB9 1FB1
  \l 1FB2
  \L 1FB3 1FBC 1FB3
  \l 1FB4
  \l 1FB6
  \l 1FB7
  \L 1FB8 1FB8 1FB0
  \L 1FB9 1FB9 1FB1
  \L 1FBA 1FBA 1F70
  \L 1FBB 1FBB 1F71
  \L 1FBC 1FBC 1FB3
  \L 1FBE 0399 1FBE
  \l 1FC2
  \L 1FC3 1FCC 1FC3
  \l 1FC4
  \l 1FC6
  \l 1FC7
  \L 1FC8 1FC8 1F72
  \L 1FC9 1FC9 1F73
  \L 1FCA 1FCA 1F74
  \L 1FCB 1FCB 1F75
  \L 1FCC 1FCC 1FC3
  \L 1FD0 1FD8 1FD0
  \L 1FD1 1FD9 1FD1
  \l 1FD2
  \l 1FD3
  \l 1FD6
  \l 1FD7
  \L 1FD8 1FD8 1FD0
  \L 1FD9 1FD9 1FD1
  \L 1FDA 1FDA 1F76
  \L 1FDB 1FDB 1F77
  \L 1FE0 1FE8 1FE0
  \L 1FE1 1FE9 1FE1
  \l 1FE2
  \l 1FE3
  \l 1FE4
  \L 1FE5 1FEC 1FE5
  \l 1FE6
  \l 1FE7
  \L 1FE8 1FE8 1FE0
  \L 1FE9 1FE9 1FE1
  \L 1FEA 1FEA 1F7A
  \L 1FEB 1FEB 1F7B
  \L 1FEC 1FEC 1FE5
  \l 1FF2
  \L 1FF3 1FFC 1FF3
  \l 1FF4
  \l 1FF6
  \l 1FF7
  \L 1FF8 1FF8 1F78
  \L 1FF9 1FF9 1F79
  \L 1FFA 1FFA 1F7C
  \L 1FFB 1FFB 1F7D
  \L 1FFC 1FFC 1FF3
  \l 2071
  \l 207F
  \l 2090
  \l 2091
  \l 2092
  \l 2093
  \l 2094
  \l 2095
  \l 2096
  \l 2097
  \l 2098
  \l 2099
  \l 209A
  \l 209B
  \l 209C
  \l 20D0
  \l 20D1
  \l 20D2
  \l 20D3
  \l 20D4
  \l 20D5
  \l 20D6
  \l 20D7
  \l 20D8
  \l 20D9
  \l 20DA
  \l 20DB
  \l 20DC
  \l 20DD
  \l 20DE
  \l 20DF
  \l 20E0
  \l 20E1
  \l 20E2
  \l 20E3
  \l 20E4
  \l 20E5
  \l 20E6
  \l 20E7
  \l 20E8
  \l 20E9
  \l 20EA
  \l 20EB
  \l 20EC
  \l 20ED
  \l 20EE
  \l 20EF
  \l 20F0
  \l 2102
  \l 2107
  \l 210A
  \l 210B
  \l 210C
  \l 210D
  \l 210E
  \l 210F
  \l 2110
  \l 2111
  \l 2112
  \l 2113
  \l 2115
  \l 2119
  \l 211A
  \l 211B
  \l 211C
  \l 211D
  \l 2124
  \L 2126 2126 03C9
  \l 2128
  \L 212A 212A 006B
  \L 212B 212B 00E5
  \l 212C
  \l 212D
  \l 212F
  \l 2130
  \l 2131
  \L 2132 2132 214E
  \l 2133
  \l 2134
  \l 2135
  \l 2136
  \l 2137
  \l 2138
  \l 2139
  \l 213C
  \l 213D
  \l 213E
  \l 213F
  \l 2145
  \l 2146
  \l 2147
  \l 2148
  \l 2149
  \L 214E 2132 214E
  \C 2160 2160 2170
  \C 2161 2161 2171
  \C 2162 2162 2172
  \C 2163 2163 2173
  \C 2164 2164 2174
  \C 2165 2165 2175
  \C 2166 2166 2176
  \C 2167 2167 2177
  \C 2168 2168 2178
  \C 2169 2169 2179
  \C 216A 216A 217A
  \C 216B 216B 217B
  \C 216C 216C 217C
  \C 216D 216D 217D
  \C 216E 216E 217E
  \C 216F 216F 217F
  \C 2170 2160 2170
  \C 2171 2161 2171
  \C 2172 2162 2172
  \C 2173 2163 2173
  \C 2174 2164 2174
  \C 2175 2165 2175
  \C 2176 2166 2176
  \C 2177 2167 2177
  \C 2178 2168 2178
  \C 2179 2169 2179
  \C 217A 216A 217A
  \C 217B 216B 217B
  \C 217C 216C 217C
  \C 217D 216D 217D
  \C 217E 216E 217E
  \C 217F 216F 217F
  \L 2183 2183 2184
  \L 2184 2183 2184
  \C 24B6 24B6 24D0
  \C 24B7 24B7 24D1
  \C 24B8 24B8 24D2
  \C 24B9 24B9 24D3
  \C 24BA 24BA 24D4
  \C 24BB 24BB 24D5
  \C 24BC 24BC 24D6
  \C 24BD 24BD 24D7
  \C 24BE 24BE 24D8
  \C 24BF 24BF 24D9
  \C 24C0 24C0 24DA
  \C 24C1 24C1 24DB
  \C 24C2 24C2 24DC
  \C 24C3 24C3 24DD
  \C 24C4 24C4 24DE
  \C 24C5 24C5 24DF
  \C 24C6 24C6 24E0
  \C 24C7 24C7 24E1
  \C 24C8 24C8 24E2
  \C 24C9 24C9 24E3
  \C 24CA 24CA 24E4
  \C 24CB 24CB 24E5
  \C 24CC 24CC 24E6
  \C 24CD 24CD 24E7
  \C 24CE 24CE 24E8
  \C 24CF 24CF 24E9
  \C 24D0 24B6 24D0
  \C 24D1 24B7 24D1
  \C 24D2 24B8 24D2
  \C 24D3 24B9 24D3
  \C 24D4 24BA 24D4
  \C 24D5 24BB 24D5
  \C 24D6 24BC 24D6
  \C 24D7 24BD 24D7
  \C 24D8 24BE 24D8
  \C 24D9 24BF 24D9
  \C 24DA 24C0 24DA
  \C 24DB 24C1 24DB
  \C 24DC 24C2 24DC
  \C 24DD 24C3 24DD
  \C 24DE 24C4 24DE
  \C 24DF 24C5 24DF
  \C 24E0 24C6 24E0
  \C 24E1 24C7 24E1
  \C 24E2 24C8 24E2
  \C 24E3 24C9 24E3
  \C 24E4 24CA 24E4
  \C 24E5 24CB 24E5
  \C 24E6 24CC 24E6
  \C 24E7 24CD 24E7
  \C 24E8 24CE 24E8
  \C 24E9 24CF 24E9
  \L 2C00 2C00 2C30
  \L 2C01 2C01 2C31
  \L 2C02 2C02 2C32
  \L 2C03 2C03 2C33
  \L 2C04 2C04 2C34
  \L 2C05 2C05 2C35
  \L 2C06 2C06 2C36
  \L 2C07 2C07 2C37
  \L 2C08 2C08 2C38
  \L 2C09 2C09 2C39
  \L 2C0A 2C0A 2C3A
  \L 2C0B 2C0B 2C3B
  \L 2C0C 2C0C 2C3C
  \L 2C0D 2C0D 2C3D
  \L 2C0E 2C0E 2C3E
  \L 2C0F 2C0F 2C3F
  \L 2C10 2C10 2C40
  \L 2C11 2C11 2C41
  \L 2C12 2C12 2C42
  \L 2C13 2C13 2C43
  \L 2C14 2C14 2C44
  \L 2C15 2C15 2C45
  \L 2C16 2C16 2C46
  \L 2C17 2C17 2C47
  \L 2C18 2C18 2C48
  \L 2C19 2C19 2C49
  \L 2C1A 2C1A 2C4A
  \L 2C1B 2C1B 2C4B
  \L 2C1C 2C1C 2C4C
  \L 2C1D 2C1D 2C4D
  \L 2C1E 2C1E 2C4E
  \L 2C1F 2C1F 2C4F
  \L 2C20 2C20 2C50
  \L 2C21 2C21 2C51
  \L 2C22 2C22 2C52
  \L 2C23 2C23 2C53
  \L 2C24 2C24 2C54
  \L 2C25 2C25 2C55
  \L 2C26 2C26 2C56
  \L 2C27 2C27 2C57
  \L 2C28 2C28 2C58
  \L 2C29 2C29 2C59
  \L 2C2A 2C2A 2C5A
  \L 2C2B 2C2B 2C5B
  \L 2C2C 2C2C 2C5C
  \L 2C2D 2C2D 2C5D
  \L 2C2E 2C2E 2C5E
  \L 2C30 2C00 2C30
  \L 2C31 2C01 2C31
  \L 2C32 2C02 2C32
  \L 2C33 2C03 2C33
  \L 2C34 2C04 2C34
  \L 2C35 2C05 2C35
  \L 2C36 2C06 2C36
  \L 2C37 2C07 2C37
  \L 2C38 2C08 2C38
  \L 2C39 2C09 2C39
  \L 2C3A 2C0A 2C3A
  \L 2C3B 2C0B 2C3B
  \L 2C3C 2C0C 2C3C
  \L 2C3D 2C0D 2C3D
  \L 2C3E 2C0E 2C3E
  \L 2C3F 2C0F 2C3F
  \L 2C40 2C10 2C40
  \L 2C41 2C11 2C41
  \L 2C42 2C12 2C42
  \L 2C43 2C13 2C43
  \L 2C44 2C14 2C44
  \L 2C45 2C15 2C45
  \L 2C46 2C16 2C46
  \L 2C47 2C17 2C47
  \L 2C48 2C18 2C48
  \L 2C49 2C19 2C49
  \L 2C4A 2C1A 2C4A
  \L 2C4B 2C1B 2C4B
  \L 2C4C 2C1C 2C4C
  \L 2C4D 2C1D 2C4D
  \L 2C4E 2C1E 2C4E
  \L 2C4F 2C1F 2C4F
  \L 2C50 2C20 2C50
  \L 2C51 2C21 2C51
  \L 2C52 2C22 2C52
  \L 2C53 2C23 2C53
  \L 2C54 2C24 2C54
  \L 2C55 2C25 2C55
  \L 2C56 2C26 2C56
  \L 2C57 2C27 2C57
  \L 2C58 2C28 2C58
  \L 2C59 2C29 2C59
  \L 2C5A 2C2A 2C5A
  \L 2C5B 2C2B 2C5B
  \L 2C5C 2C2C 2C5C
  \L 2C5D 2C2D 2C5D
  \L 2C5E 2C2E 2C5E
  \L 2C60 2C60 2C61
  \L 2C61 2C60 2C61
  \L 2C62 2C62 026B
  \L 2C63 2C63 1D7D
  \L 2C64 2C64 027D
  \L 2C65 023A 2C65
  \L 2C66 023E 2C66
  \L 2C67 2C67 2C68
  \L 2C68 2C67 2C68
  \L 2C69 2C69 2C6A
  \L 2C6A 2C69 2C6A
  \L 2C6B 2C6B 2C6C
  \L 2C6C 2C6B 2C6C
  \L 2C6D 2C6D 0251
  \L 2C6E 2C6E 0271
  \L 2C6F 2C6F 0250
  \L 2C70 2C70 0252
  \l 2C71
  \L 2C72 2C72 2C73
  \L 2C73 2C72 2C73
  \l 2C74
  \L 2C75 2C75 2C76
  \L 2C76 2C75 2C76
  \l 2C77
  \l 2C78
  \l 2C79
  \l 2C7A
  \l 2C7B
  \l 2C7C
  \l 2C7D
  \L 2C7E 2C7E 023F
  \L 2C7F 2C7F 0240
  \L 2C80 2C80 2C81
  \L 2C81 2C80 2C81
  \L 2C82 2C82 2C83
  \L 2C83 2C82 2C83
  \L 2C84 2C84 2C85
  \L 2C85 2C84 2C85
  \L 2C86 2C86 2C87
  \L 2C87 2C86 2C87
  \L 2C88 2C88 2C89
  \L 2C89 2C88 2C89
  \L 2C8A 2C8A 2C8B
  \L 2C8B 2C8A 2C8B
  \L 2C8C 2C8C 2C8D
  \L 2C8D 2C8C 2C8D
  \L 2C8E 2C8E 2C8F
  \L 2C8F 2C8E 2C8F
  \L 2C90 2C90 2C91
  \L 2C91 2C90 2C91
  \L 2C92 2C92 2C93
  \L 2C93 2C92 2C93
  \L 2C94 2C94 2C95
  \L 2C95 2C94 2C95
  \L 2C96 2C96 2C97
  \L 2C97 2C96 2C97
  \L 2C98 2C98 2C99
  \L 2C99 2C98 2C99
  \L 2C9A 2C9A 2C9B
  \L 2C9B 2C9A 2C9B
  \L 2C9C 2C9C 2C9D
  \L 2C9D 2C9C 2C9D
  \L 2C9E 2C9E 2C9F
  \L 2C9F 2C9E 2C9F
  \L 2CA0 2CA0 2CA1
  \L 2CA1 2CA0 2CA1
  \L 2CA2 2CA2 2CA3
  \L 2CA3 2CA2 2CA3
  \L 2CA4 2CA4 2CA5
  \L 2CA5 2CA4 2CA5
  \L 2CA6 2CA6 2CA7
  \L 2CA7 2CA6 2CA7
  \L 2CA8 2CA8 2CA9
  \L 2CA9 2CA8 2CA9
  \L 2CAA 2CAA 2CAB
  \L 2CAB 2CAA 2CAB
  \L 2CAC 2CAC 2CAD
  \L 2CAD 2CAC 2CAD
  \L 2CAE 2CAE 2CAF
  \L 2CAF 2CAE 2CAF
  \L 2CB0 2CB0 2CB1
  \L 2CB1 2CB0 2CB1
  \L 2CB2 2CB2 2CB3
  \L 2CB3 2CB2 2CB3
  \L 2CB4 2CB4 2CB5
  \L 2CB5 2CB4 2CB5
  \L 2CB6 2CB6 2CB7
  \L 2CB7 2CB6 2CB7
  \L 2CB8 2CB8 2CB9
  \L 2CB9 2CB8 2CB9
  \L 2CBA 2CBA 2CBB
  \L 2CBB 2CBA 2CBB
  \L 2CBC 2CBC 2CBD
  \L 2CBD 2CBC 2CBD
  \L 2CBE 2CBE 2CBF
  \L 2CBF 2CBE 2CBF
  \L 2CC0 2CC0 2CC1
  \L 2CC1 2CC0 2CC1
  \L 2CC2 2CC2 2CC3
  \L 2CC3 2CC2 2CC3
  \L 2CC4 2CC4 2CC5
  \L 2CC5 2CC4 2CC5
  \L 2CC6 2CC6 2CC7
  \L 2CC7 2CC6 2CC7
  \L 2CC8 2CC8 2CC9
  \L 2CC9 2CC8 2CC9
  \L 2CCA 2CCA 2CCB
  \L 2CCB 2CCA 2CCB
  \L 2CCC 2CCC 2CCD
  \L 2CCD 2CCC 2CCD
  \L 2CCE 2CCE 2CCF
  \L 2CCF 2CCE 2CCF
  \L 2CD0 2CD0 2CD1
  \L 2CD1 2CD0 2CD1
  \L 2CD2 2CD2 2CD3
  \L 2CD3 2CD2 2CD3
  \L 2CD4 2CD4 2CD5
  \L 2CD5 2CD4 2CD5
  \L 2CD6 2CD6 2CD7
  \L 2CD7 2CD6 2CD7
  \L 2CD8 2CD8 2CD9
  \L 2CD9 2CD8 2CD9
  \L 2CDA 2CDA 2CDB
  \L 2CDB 2CDA 2CDB
  \L 2CDC 2CDC 2CDD
  \L 2CDD 2CDC 2CDD
  \L 2CDE 2CDE 2CDF
  \L 2CDF 2CDE 2CDF
  \L 2CE0 2CE0 2CE1
  \L 2CE1 2CE0 2CE1
  \L 2CE2 2CE2 2CE3
  \L 2CE3 2CE2 2CE3
  \l 2CE4
  \L 2CEB 2CEB 2CEC
  \L 2CEC 2CEB 2CEC
  \L 2CED 2CED 2CEE
  \L 2CEE 2CED 2CEE
  \l 2CEF
  \l 2CF0
  \l 2CF1
  \L 2CF2 2CF2 2CF3
  \L 2CF3 2CF2 2CF3
  \L 2D00 10A0 2D00
  \L 2D01 10A1 2D01
  \L 2D02 10A2 2D02
  \L 2D03 10A3 2D03
  \L 2D04 10A4 2D04
  \L 2D05 10A5 2D05
  \L 2D06 10A6 2D06
  \L 2D07 10A7 2D07
  \L 2D08 10A8 2D08
  \L 2D09 10A9 2D09
  \L 2D0A 10AA 2D0A
  \L 2D0B 10AB 2D0B
  \L 2D0C 10AC 2D0C
  \L 2D0D 10AD 2D0D
  \L 2D0E 10AE 2D0E
  \L 2D0F 10AF 2D0F
  \L 2D10 10B0 2D10
  \L 2D11 10B1 2D11
  \L 2D12 10B2 2D12
  \L 2D13 10B3 2D13
  \L 2D14 10B4 2D14
  \L 2D15 10B5 2D15
  \L 2D16 10B6 2D16
  \L 2D17 10B7 2D17
  \L 2D18 10B8 2D18
  \L 2D19 10B9 2D19
  \L 2D1A 10BA 2D1A
  \L 2D1B 10BB 2D1B
  \L 2D1C 10BC 2D1C
  \L 2D1D 10BD 2D1D
  \L 2D1E 10BE 2D1E
  \L 2D1F 10BF 2D1F
  \L 2D20 10C0 2D20
  \L 2D21 10C1 2D21
  \L 2D22 10C2 2D22
  \L 2D23 10C3 2D23
  \L 2D24 10C4 2D24
  \L 2D25 10C5 2D25
  \L 2D27 10C7 2D27
  \L 2D2D 10CD 2D2D
  \l 2D30
  \l 2D31
  \l 2D32
  \l 2D33
  \l 2D34
  \l 2D35
  \l 2D36
  \l 2D37
  \l 2D38
  \l 2D39
  \l 2D3A
  \l 2D3B
  \l 2D3C
  \l 2D3D
  \l 2D3E
  \l 2D3F
  \l 2D40
  \l 2D41
  \l 2D42
  \l 2D43
  \l 2D44
  \l 2D45
  \l 2D46
  \l 2D47
  \l 2D48
  \l 2D49
  \l 2D4A
  \l 2D4B
  \l 2D4C
  \l 2D4D
  \l 2D4E
  \l 2D4F
  \l 2D50
  \l 2D51
  \l 2D52
  \l 2D53
  \l 2D54
  \l 2D55
  \l 2D56
  \l 2D57
  \l 2D58
  \l 2D59
  \l 2D5A
  \l 2D5B
  \l 2D5C
  \l 2D5D
  \l 2D5E
  \l 2D5F
  \l 2D60
  \l 2D61
  \l 2D62
  \l 2D63
  \l 2D64
  \l 2D65
  \l 2D66
  \l 2D67
  \l 2D6F
  \l 2D7F
  \l 2D80
  \l 2D81
  \l 2D82
  \l 2D83
  \l 2D84
  \l 2D85
  \l 2D86
  \l 2D87
  \l 2D88
  \l 2D89
  \l 2D8A
  \l 2D8B
  \l 2D8C
  \l 2D8D
  \l 2D8E
  \l 2D8F
  \l 2D90
  \l 2D91
  \l 2D92
  \l 2D93
  \l 2D94
  \l 2D95
  \l 2D96
  \l 2DA0
  \l 2DA1
  \l 2DA2
  \l 2DA3
  \l 2DA4
  \l 2DA5
  \l 2DA6
  \l 2DA8
  \l 2DA9
  \l 2DAA
  \l 2DAB
  \l 2DAC
  \l 2DAD
  \l 2DAE
  \l 2DB0
  \l 2DB1
  \l 2DB2
  \l 2DB3
  \l 2DB4
  \l 2DB5
  \l 2DB6
  \l 2DB8
  \l 2DB9
  \l 2DBA
  \l 2DBB
  \l 2DBC
  \l 2DBD
  \l 2DBE
  \l 2DC0
  \l 2DC1
  \l 2DC2
  \l 2DC3
  \l 2DC4
  \l 2DC5
  \l 2DC6
  \l 2DC8
  \l 2DC9
  \l 2DCA
  \l 2DCB
  \l 2DCC
  \l 2DCD
  \l 2DCE
  \l 2DD0
  \l 2DD1
  \l 2DD2
  \l 2DD3
  \l 2DD4
  \l 2DD5
  \l 2DD6
  \l 2DD8
  \l 2DD9
  \l 2DDA
  \l 2DDB
  \l 2DDC
  \l 2DDD
  \l 2DDE
  \l 2DE0
  \l 2DE1
  \l 2DE2
  \l 2DE3
  \l 2DE4
  \l 2DE5
  \l 2DE6
  \l 2DE7
  \l 2DE8
  \l 2DE9
  \l 2DEA
  \l 2DEB
  \l 2DEC
  \l 2DED
  \l 2DEE
  \l 2DEF
  \l 2DF0
  \l 2DF1
  \l 2DF2
  \l 2DF3
  \l 2DF4
  \l 2DF5
  \l 2DF6
  \l 2DF7
  \l 2DF8
  \l 2DF9
  \l 2DFA
  \l 2DFB
  \l 2DFC
  \l 2DFD
  \l 2DFE
  \l 2DFF
  \l 2E2F
  \l 3005
  \l 3006
  \l 302A
  \l 302B
  \l 302C
  \l 302D
  \l 302E
  \l 302F
  \l 3031
  \l 3032
  \l 3033
  \l 3034
  \l 3035
  \l 303B
  \l 303C
  \l 3041
  \l 3042
  \l 3043
  \l 3044
  \l 3045
  \l 3046
  \l 3047
  \l 3048
  \l 3049
  \l 304A
  \l 304B
  \l 304C
  \l 304D
  \l 304E
  \l 304F
  \l 3050
  \l 3051
  \l 3052
  \l 3053
  \l 3054
  \l 3055
  \l 3056
  \l 3057
  \l 3058
  \l 3059
  \l 305A
  \l 305B
  \l 305C
  \l 305D
  \l 305E
  \l 305F
  \l 3060
  \l 3061
  \l 3062
  \l 3063
  \l 3064
  \l 3065
  \l 3066
  \l 3067
  \l 3068
  \l 3069
  \l 306A
  \l 306B
  \l 306C
  \l 306D
  \l 306E
  \l 306F
  \l 3070
  \l 3071
  \l 3072
  \l 3073
  \l 3074
  \l 3075
  \l 3076
  \l 3077
  \l 3078
  \l 3079
  \l 307A
  \l 307B
  \l 307C
  \l 307D
  \l 307E
  \l 307F
  \l 3080
  \l 3081
  \l 3082
  \l 3083
  \l 3084
  \l 3085
  \l 3086
  \l 3087
  \l 3088
  \l 3089
  \l 308A
  \l 308B
  \l 308C
  \l 308D
  \l 308E
  \l 308F
  \l 3090
  \l 3091
  \l 3092
  \l 3093
  \l 3094
  \l 3095
  \l 3096
  \l 3099
  \l 309A
  \l 309D
  \l 309E
  \l 309F
  \l 30A1
  \l 30A2
  \l 30A3
  \l 30A4
  \l 30A5
  \l 30A6
  \l 30A7
  \l 30A8
  \l 30A9
  \l 30AA
  \l 30AB
  \l 30AC
  \l 30AD
  \l 30AE
  \l 30AF
  \l 30B0
  \l 30B1
  \l 30B2
  \l 30B3
  \l 30B4
  \l 30B5
  \l 30B6
  \l 30B7
  \l 30B8
  \l 30B9
  \l 30BA
  \l 30BB
  \l 30BC
  \l 30BD
  \l 30BE
  \l 30BF
  \l 30C0
  \l 30C1
  \l 30C2
  \l 30C3
  \l 30C4
  \l 30C5
  \l 30C6
  \l 30C7
  \l 30C8
  \l 30C9
  \l 30CA
  \l 30CB
  \l 30CC
  \l 30CD
  \l 30CE
  \l 30CF
  \l 30D0
  \l 30D1
  \l 30D2
  \l 30D3
  \l 30D4
  \l 30D5
  \l 30D6
  \l 30D7
  \l 30D8
  \l 30D9
  \l 30DA
  \l 30DB
  \l 30DC
  \l 30DD
  \l 30DE
  \l 30DF
  \l 30E0
  \l 30E1
  \l 30E2
  \l 30E3
  \l 30E4
  \l 30E5
  \l 30E6
  \l 30E7
  \l 30E8
  \l 30E9
  \l 30EA
  \l 30EB
  \l 30EC
  \l 30ED
  \l 30EE
  \l 30EF
  \l 30F0
  \l 30F1
  \l 30F2
  \l 30F3
  \l 30F4
  \l 30F5
  \l 30F6
  \l 30F7
  \l 30F8
  \l 30F9
  \l 30FA
  \l 30FC
  \l 30FD
  \l 30FE
  \l 30FF
  \l 3105
  \l 3106
  \l 3107
  \l 3108
  \l 3109
  \l 310A
  \l 310B
  \l 310C
  \l 310D
  \l 310E
  \l 310F
  \l 3110
  \l 3111
  \l 3112
  \l 3113
  \l 3114
  \l 3115
  \l 3116
  \l 3117
  \l 3118
  \l 3119
  \l 311A
  \l 311B
  \l 311C
  \l 311D
  \l 311E
  \l 311F
  \l 3120
  \l 3121
  \l 3122
  \l 3123
  \l 3124
  \l 3125
  \l 3126
  \l 3127
  \l 3128
  \l 3129
  \l 312A
  \l 312B
  \l 312C
  \l 312D
  \l 3131
  \l 3132
  \l 3133
  \l 3134
  \l 3135
  \l 3136
  \l 3137
  \l 3138
  \l 3139
  \l 313A
  \l 313B
  \l 313C
  \l 313D
  \l 313E
  \l 313F
  \l 3140
  \l 3141
  \l 3142
  \l 3143
  \l 3144
  \l 3145
  \l 3146
  \l 3147
  \l 3148
  \l 3149
  \l 314A
  \l 314B
  \l 314C
  \l 314D
  \l 314E
  \l 314F
  \l 3150
  \l 3151
  \l 3152
  \l 3153
  \l 3154
  \l 3155
  \l 3156
  \l 3157
  \l 3158
  \l 3159
  \l 315A
  \l 315B
  \l 315C
  \l 315D
  \l 315E
  \l 315F
  \l 3160
  \l 3161
  \l 3162
  \l 3163
  \l 3164
  \l 3165
  \l 3166
  \l 3167
  \l 3168
  \l 3169
  \l 316A
  \l 316B
  \l 316C
  \l 316D
  \l 316E
  \l 316F
  \l 3170
  \l 3171
  \l 3172
  \l 3173
  \l 3174
  \l 3175
  \l 3176
  \l 3177
  \l 3178
  \l 3179
  \l 317A
  \l 317B
  \l 317C
  \l 317D
  \l 317E
  \l 317F
  \l 3180
  \l 3181
  \l 3182
  \l 3183
  \l 3184
  \l 3185
  \l 3186
  \l 3187
  \l 3188
  \l 3189
  \l 318A
  \l 318B
  \l 318C
  \l 318D
  \l 318E
  \l 31A0
  \l 31A1
  \l 31A2
  \l 31A3
  \l 31A4
  \l 31A5
  \l 31A6
  \l 31A7
  \l 31A8
  \l 31A9
  \l 31AA
  \l 31AB
  \l 31AC
  \l 31AD
  \l 31AE
  \l 31AF
  \l 31B0
  \l 31B1
  \l 31B2
  \l 31B3
  \l 31B4
  \l 31B5
  \l 31B6
  \l 31B7
  \l 31B8
  \l 31B9
  \l 31BA
  \l 31F0
  \l 31F1
  \l 31F2
  \l 31F3
  \l 31F4
  \l 31F5
  \l 31F6
  \l 31F7
  \l 31F8
  \l 31F9
  \l 31FA
  \l 31FB
  \l 31FC
  \l 31FD
  \l 31FE
  \l 31FF
  \l 3400
  \l 3401
  \l 3402
  \l 3403
  \l 3404
  \l 3405
  \l 3406
  \l 3407
  \l 3408
  \l 3409
  \l 340A
  \l 340B
  \l 340C
  \l 340D
  \l 340E
  \l 340F
  \l 3410
  \l 3411
  \l 3412
  \l 3413
  \l 3414
  \l 3415
  \l 3416
  \l 3417
  \l 3418
  \l 3419
  \l 341A
  \l 341B
  \l 341C
  \l 341D
  \l 341E
  \l 341F
  \l 3420
  \l 3421
  \l 3422
  \l 3423
  \l 3424
  \l 3425
  \l 3426
  \l 3427
  \l 3428
  \l 3429
  \l 342A
  \l 342B
  \l 342C
  \l 342D
  \l 342E
  \l 342F
  \l 3430
  \l 3431
  \l 3432
  \l 3433
  \l 3434
  \l 3435
  \l 3436
  \l 3437
  \l 3438
  \l 3439
  \l 343A
  \l 343B
  \l 343C
  \l 343D
  \l 343E
  \l 343F
  \l 3440
  \l 3441
  \l 3442
  \l 3443
  \l 3444
  \l 3445
  \l 3446
  \l 3447
  \l 3448
  \l 3449
  \l 344A
  \l 344B
  \l 344C
  \l 344D
  \l 344E
  \l 344F
  \l 3450
  \l 3451
  \l 3452
  \l 3453
  \l 3454
  \l 3455
  \l 3456
  \l 3457
  \l 3458
  \l 3459
  \l 345A
  \l 345B
  \l 345C
  \l 345D
  \l 345E
  \l 345F
  \l 3460
  \l 3461
  \l 3462
  \l 3463
  \l 3464
  \l 3465
  \l 3466
  \l 3467
  \l 3468
  \l 3469
  \l 346A
  \l 346B
  \l 346C
  \l 346D
  \l 346E
  \l 346F
  \l 3470
  \l 3471
  \l 3472
  \l 3473
  \l 3474
  \l 3475
  \l 3476
  \l 3477
  \l 3478
  \l 3479
  \l 347A
  \l 347B
  \l 347C
  \l 347D
  \l 347E
  \l 347F
  \l 3480
  \l 3481
  \l 3482
  \l 3483
  \l 3484
  \l 3485
  \l 3486
  \l 3487
  \l 3488
  \l 3489
  \l 348A
  \l 348B
  \l 348C
  \l 348D
  \l 348E
  \l 348F
  \l 3490
  \l 3491
  \l 3492
  \l 3493
  \l 3494
  \l 3495
  \l 3496
  \l 3497
  \l 3498
  \l 3499
  \l 349A
  \l 349B
  \l 349C
  \l 349D
  \l 349E
  \l 349F
  \l 34A0
  \l 34A1
  \l 34A2
  \l 34A3
  \l 34A4
  \l 34A5
  \l 34A6
  \l 34A7
  \l 34A8
  \l 34A9
  \l 34AA
  \l 34AB
  \l 34AC
  \l 34AD
  \l 34AE
  \l 34AF
  \l 34B0
  \l 34B1
  \l 34B2
  \l 34B3
  \l 34B4
  \l 34B5
  \l 34B6
  \l 34B7
  \l 34B8
  \l 34B9
  \l 34BA
  \l 34BB
  \l 34BC
  \l 34BD
  \l 34BE
  \l 34BF
  \l 34C0
  \l 34C1
  \l 34C2
  \l 34C3
  \l 34C4
  \l 34C5
  \l 34C6
  \l 34C7
  \l 34C8
  \l 34C9
  \l 34CA
  \l 34CB
  \l 34CC
  \l 34CD
  \l 34CE
  \l 34CF
  \l 34D0
  \l 34D1
  \l 34D2
  \l 34D3
  \l 34D4
  \l 34D5
  \l 34D6
  \l 34D7
  \l 34D8
  \l 34D9
  \l 34DA
  \l 34DB
  \l 34DC
  \l 34DD
  \l 34DE
  \l 34DF
  \l 34E0
  \l 34E1
  \l 34E2
  \l 34E3
  \l 34E4
  \l 34E5
  \l 34E6
  \l 34E7
  \l 34E8
  \l 34E9
  \l 34EA
  \l 34EB
  \l 34EC
  \l 34ED
  \l 34EE
  \l 34EF
  \l 34F0
  \l 34F1
  \l 34F2
  \l 34F3
  \l 34F4
  \l 34F5
  \l 34F6
  \l 34F7
  \l 34F8
  \l 34F9
  \l 34FA
  \l 34FB
  \l 34FC
  \l 34FD
  \l 34FE
  \l 34FF
  \l 3500
  \l 3501
  \l 3502
  \l 3503
  \l 3504
  \l 3505
  \l 3506
  \l 3507
  \l 3508
  \l 3509
  \l 350A
  \l 350B
  \l 350C
  \l 350D
  \l 350E
  \l 350F
  \l 3510
  \l 3511
  \l 3512
  \l 3513
  \l 3514
  \l 3515
  \l 3516
  \l 3517
  \l 3518
  \l 3519
  \l 351A
  \l 351B
  \l 351C
  \l 351D
  \l 351E
  \l 351F
  \l 3520
  \l 3521
  \l 3522
  \l 3523
  \l 3524
  \l 3525
  \l 3526
  \l 3527
  \l 3528
  \l 3529
  \l 352A
  \l 352B
  \l 352C
  \l 352D
  \l 352E
  \l 352F
  \l 3530
  \l 3531
  \l 3532
  \l 3533
  \l 3534
  \l 3535
  \l 3536
  \l 3537
  \l 3538
  \l 3539
  \l 353A
  \l 353B
  \l 353C
  \l 353D
  \l 353E
  \l 353F
  \l 3540
  \l 3541
  \l 3542
  \l 3543
  \l 3544
  \l 3545
  \l 3546
  \l 3547
  \l 3548
  \l 3549
  \l 354A
  \l 354B
  \l 354C
  \l 354D
  \l 354E
  \l 354F
  \l 3550
  \l 3551
  \l 3552
  \l 3553
  \l 3554
  \l 3555
  \l 3556
  \l 3557
  \l 3558
  \l 3559
  \l 355A
  \l 355B
  \l 355C
  \l 355D
  \l 355E
  \l 355F
  \l 3560
  \l 3561
  \l 3562
  \l 3563
  \l 3564
  \l 3565
  \l 3566
  \l 3567
  \l 3568
  \l 3569
  \l 356A
  \l 356B
  \l 356C
  \l 356D
  \l 356E
  \l 356F
  \l 3570
  \l 3571
  \l 3572
  \l 3573
  \l 3574
  \l 3575
  \l 3576
  \l 3577
  \l 3578
  \l 3579
  \l 357A
  \l 357B
  \l 357C
  \l 357D
  \l 357E
  \l 357F
  \l 3580
  \l 3581
  \l 3582
  \l 3583
  \l 3584
  \l 3585
  \l 3586
  \l 3587
  \l 3588
  \l 3589
  \l 358A
  \l 358B
  \l 358C
  \l 358D
  \l 358E
  \l 358F
  \l 3590
  \l 3591
  \l 3592
  \l 3593
  \l 3594
  \l 3595
  \l 3596
  \l 3597
  \l 3598
  \l 3599
  \l 359A
  \l 359B
  \l 359C
  \l 359D
  \l 359E
  \l 359F
  \l 35A0
  \l 35A1
  \l 35A2
  \l 35A3
  \l 35A4
  \l 35A5
  \l 35A6
  \l 35A7
  \l 35A8
  \l 35A9
  \l 35AA
  \l 35AB
  \l 35AC
  \l 35AD
  \l 35AE
  \l 35AF
  \l 35B0
  \l 35B1
  \l 35B2
  \l 35B3
  \l 35B4
  \l 35B5
  \l 35B6
  \l 35B7
  \l 35B8
  \l 35B9
  \l 35BA
  \l 35BB
  \l 35BC
  \l 35BD
  \l 35BE
  \l 35BF
  \l 35C0
  \l 35C1
  \l 35C2
  \l 35C3
  \l 35C4
  \l 35C5
  \l 35C6
  \l 35C7
  \l 35C8
  \l 35C9
  \l 35CA
  \l 35CB
  \l 35CC
  \l 35CD
  \l 35CE
  \l 35CF
  \l 35D0
  \l 35D1
  \l 35D2
  \l 35D3
  \l 35D4
  \l 35D5
  \l 35D6
  \l 35D7
  \l 35D8
  \l 35D9
  \l 35DA
  \l 35DB
  \l 35DC
  \l 35DD
  \l 35DE
  \l 35DF
  \l 35E0
  \l 35E1
  \l 35E2
  \l 35E3
  \l 35E4
  \l 35E5
  \l 35E6
  \l 35E7
  \l 35E8
  \l 35E9
  \l 35EA
  \l 35EB
  \l 35EC
  \l 35ED
  \l 35EE
  \l 35EF
  \l 35F0
  \l 35F1
  \l 35F2
  \l 35F3
  \l 35F4
  \l 35F5
  \l 35F6
  \l 35F7
  \l 35F8
  \l 35F9
  \l 35FA
  \l 35FB
  \l 35FC
  \l 35FD
  \l 35FE
  \l 35FF
  \l 3600
  \l 3601
  \l 3602
  \l 3603
  \l 3604
  \l 3605
  \l 3606
  \l 3607
  \l 3608
  \l 3609
  \l 360A
  \l 360B
  \l 360C
  \l 360D
  \l 360E
  \l 360F
  \l 3610
  \l 3611
  \l 3612
  \l 3613
  \l 3614
  \l 3615
  \l 3616
  \l 3617
  \l 3618
  \l 3619
  \l 361A
  \l 361B
  \l 361C
  \l 361D
  \l 361E
  \l 361F
  \l 3620
  \l 3621
  \l 3622
  \l 3623
  \l 3624
  \l 3625
  \l 3626
  \l 3627
  \l 3628
  \l 3629
  \l 362A
  \l 362B
  \l 362C
  \l 362D
  \l 362E
  \l 362F
  \l 3630
  \l 3631
  \l 3632
  \l 3633
  \l 3634
  \l 3635
  \l 3636
  \l 3637
  \l 3638
  \l 3639
  \l 363A
  \l 363B
  \l 363C
  \l 363D
  \l 363E
  \l 363F
  \l 3640
  \l 3641
  \l 3642
  \l 3643
  \l 3644
  \l 3645
  \l 3646
  \l 3647
  \l 3648
  \l 3649
  \l 364A
  \l 364B
  \l 364C
  \l 364D
  \l 364E
  \l 364F
  \l 3650
  \l 3651
  \l 3652
  \l 3653
  \l 3654
  \l 3655
  \l 3656
  \l 3657
  \l 3658
  \l 3659
  \l 365A
  \l 365B
  \l 365C
  \l 365D
  \l 365E
  \l 365F
  \l 3660
  \l 3661
  \l 3662
  \l 3663
  \l 3664
  \l 3665
  \l 3666
  \l 3667
  \l 3668
  \l 3669
  \l 366A
  \l 366B
  \l 366C
  \l 366D
  \l 366E
  \l 366F
  \l 3670
  \l 3671
  \l 3672
  \l 3673
  \l 3674
  \l 3675
  \l 3676
  \l 3677
  \l 3678
  \l 3679
  \l 367A
  \l 367B
  \l 367C
  \l 367D
  \l 367E
  \l 367F
  \l 3680
  \l 3681
  \l 3682
  \l 3683
  \l 3684
  \l 3685
  \l 3686
  \l 3687
  \l 3688
  \l 3689
  \l 368A
  \l 368B
  \l 368C
  \l 368D
  \l 368E
  \l 368F
  \l 3690
  \l 3691
  \l 3692
  \l 3693
  \l 3694
  \l 3695
  \l 3696
  \l 3697
  \l 3698
  \l 3699
  \l 369A
  \l 369B
  \l 369C
  \l 369D
  \l 369E
  \l 369F
  \l 36A0
  \l 36A1
  \l 36A2
  \l 36A3
  \l 36A4
  \l 36A5
  \l 36A6
  \l 36A7
  \l 36A8
  \l 36A9
  \l 36AA
  \l 36AB
  \l 36AC
  \l 36AD
  \l 36AE
  \l 36AF
  \l 36B0
  \l 36B1
  \l 36B2
  \l 36B3
  \l 36B4
  \l 36B5
  \l 36B6
  \l 36B7
  \l 36B8
  \l 36B9
  \l 36BA
  \l 36BB
  \l 36BC
  \l 36BD
  \l 36BE
  \l 36BF
  \l 36C0
  \l 36C1
  \l 36C2
  \l 36C3
  \l 36C4
  \l 36C5
  \l 36C6
  \l 36C7
  \l 36C8
  \l 36C9
  \l 36CA
  \l 36CB
  \l 36CC
  \l 36CD
  \l 36CE
  \l 36CF
  \l 36D0
  \l 36D1
  \l 36D2
  \l 36D3
  \l 36D4
  \l 36D5
  \l 36D6
  \l 36D7
  \l 36D8
  \l 36D9
  \l 36DA
  \l 36DB
  \l 36DC
  \l 36DD
  \l 36DE
  \l 36DF
  \l 36E0
  \l 36E1
  \l 36E2
  \l 36E3
  \l 36E4
  \l 36E5
  \l 36E6
  \l 36E7
  \l 36E8
  \l 36E9
  \l 36EA
  \l 36EB
  \l 36EC
  \l 36ED
  \l 36EE
  \l 36EF
  \l 36F0
  \l 36F1
  \l 36F2
  \l 36F3
  \l 36F4
  \l 36F5
  \l 36F6
  \l 36F7
  \l 36F8
  \l 36F9
  \l 36FA
  \l 36FB
  \l 36FC
  \l 36FD
  \l 36FE
  \l 36FF
  \l 3700
  \l 3701
  \l 3702
  \l 3703
  \l 3704
  \l 3705
  \l 3706
  \l 3707
  \l 3708
  \l 3709
  \l 370A
  \l 370B
  \l 370C
  \l 370D
  \l 370E
  \l 370F
  \l 3710
  \l 3711
  \l 3712
  \l 3713
  \l 3714
  \l 3715
  \l 3716
  \l 3717
  \l 3718
  \l 3719
  \l 371A
  \l 371B
  \l 371C
  \l 371D
  \l 371E
  \l 371F
  \l 3720
  \l 3721
  \l 3722
  \l 3723
  \l 3724
  \l 3725
  \l 3726
  \l 3727
  \l 3728
  \l 3729
  \l 372A
  \l 372B
  \l 372C
  \l 372D
  \l 372E
  \l 372F
  \l 3730
  \l 3731
  \l 3732
  \l 3733
  \l 3734
  \l 3735
  \l 3736
  \l 3737
  \l 3738
  \l 3739
  \l 373A
  \l 373B
  \l 373C
  \l 373D
  \l 373E
  \l 373F
  \l 3740
  \l 3741
  \l 3742
  \l 3743
  \l 3744
  \l 3745
  \l 3746
  \l 3747
  \l 3748
  \l 3749
  \l 374A
  \l 374B
  \l 374C
  \l 374D
  \l 374E
  \l 374F
  \l 3750
  \l 3751
  \l 3752
  \l 3753
  \l 3754
  \l 3755
  \l 3756
  \l 3757
  \l 3758
  \l 3759
  \l 375A
  \l 375B
  \l 375C
  \l 375D
  \l 375E
  \l 375F
  \l 3760
  \l 3761
  \l 3762
  \l 3763
  \l 3764
  \l 3765
  \l 3766
  \l 3767
  \l 3768
  \l 3769
  \l 376A
  \l 376B
  \l 376C
  \l 376D
  \l 376E
  \l 376F
  \l 3770
  \l 3771
  \l 3772
  \l 3773
  \l 3774
  \l 3775
  \l 3776
  \l 3777
  \l 3778
  \l 3779
  \l 377A
  \l 377B
  \l 377C
  \l 377D
  \l 377E
  \l 377F
  \l 3780
  \l 3781
  \l 3782
  \l 3783
  \l 3784
  \l 3785
  \l 3786
  \l 3787
  \l 3788
  \l 3789
  \l 378A
  \l 378B
  \l 378C
  \l 378D
  \l 378E
  \l 378F
  \l 3790
  \l 3791
  \l 3792
  \l 3793
  \l 3794
  \l 3795
  \l 3796
  \l 3797
  \l 3798
  \l 3799
  \l 379A
  \l 379B
  \l 379C
  \l 379D
  \l 379E
  \l 379F
  \l 37A0
  \l 37A1
  \l 37A2
  \l 37A3
  \l 37A4
  \l 37A5
  \l 37A6
  \l 37A7
  \l 37A8
  \l 37A9
  \l 37AA
  \l 37AB
  \l 37AC
  \l 37AD
  \l 37AE
  \l 37AF
  \l 37B0
  \l 37B1
  \l 37B2
  \l 37B3
  \l 37B4
  \l 37B5
  \l 37B6
  \l 37B7
  \l 37B8
  \l 37B9
  \l 37BA
  \l 37BB
  \l 37BC
  \l 37BD
  \l 37BE
  \l 37BF
  \l 37C0
  \l 37C1
  \l 37C2
  \l 37C3
  \l 37C4
  \l 37C5
  \l 37C6
  \l 37C7
  \l 37C8
  \l 37C9
  \l 37CA
  \l 37CB
  \l 37CC
  \l 37CD
  \l 37CE
  \l 37CF
  \l 37D0
  \l 37D1
  \l 37D2
  \l 37D3
  \l 37D4
  \l 37D5
  \l 37D6
  \l 37D7
  \l 37D8
  \l 37D9
  \l 37DA
  \l 37DB
  \l 37DC
  \l 37DD
  \l 37DE
  \l 37DF
  \l 37E0
  \l 37E1
  \l 37E2
  \l 37E3
  \l 37E4
  \l 37E5
  \l 37E6
  \l 37E7
  \l 37E8
  \l 37E9
  \l 37EA
  \l 37EB
  \l 37EC
  \l 37ED
  \l 37EE
  \l 37EF
  \l 37F0
  \l 37F1
  \l 37F2
  \l 37F3
  \l 37F4
  \l 37F5
  \l 37F6
  \l 37F7
  \l 37F8
  \l 37F9
  \l 37FA
  \l 37FB
  \l 37FC
  \l 37FD
  \l 37FE
  \l 37FF
  \l 3800
  \l 3801
  \l 3802
  \l 3803
  \l 3804
  \l 3805
  \l 3806
  \l 3807
  \l 3808
  \l 3809
  \l 380A
  \l 380B
  \l 380C
  \l 380D
  \l 380E
  \l 380F
  \l 3810
  \l 3811
  \l 3812
  \l 3813
  \l 3814
  \l 3815
  \l 3816
  \l 3817
  \l 3818
  \l 3819
  \l 381A
  \l 381B
  \l 381C
  \l 381D
  \l 381E
  \l 381F
  \l 3820
  \l 3821
  \l 3822
  \l 3823
  \l 3824
  \l 3825
  \l 3826
  \l 3827
  \l 3828
  \l 3829
  \l 382A
  \l 382B
  \l 382C
  \l 382D
  \l 382E
  \l 382F
  \l 3830
  \l 3831
  \l 3832
  \l 3833
  \l 3834
  \l 3835
  \l 3836
  \l 3837
  \l 3838
  \l 3839
  \l 383A
  \l 383B
  \l 383C
  \l 383D
  \l 383E
  \l 383F
  \l 3840
  \l 3841
  \l 3842
  \l 3843
  \l 3844
  \l 3845
  \l 3846
  \l 3847
  \l 3848
  \l 3849
  \l 384A
  \l 384B
  \l 384C
  \l 384D
  \l 384E
  \l 384F
  \l 3850
  \l 3851
  \l 3852
  \l 3853
  \l 3854
  \l 3855
  \l 3856
  \l 3857
  \l 3858
  \l 3859
  \l 385A
  \l 385B
  \l 385C
  \l 385D
  \l 385E
  \l 385F
  \l 3860
  \l 3861
  \l 3862
  \l 3863
  \l 3864
  \l 3865
  \l 3866
  \l 3867
  \l 3868
  \l 3869
  \l 386A
  \l 386B
  \l 386C
  \l 386D
  \l 386E
  \l 386F
  \l 3870
  \l 3871
  \l 3872
  \l 3873
  \l 3874
  \l 3875
  \l 3876
  \l 3877
  \l 3878
  \l 3879
  \l 387A
  \l 387B
  \l 387C
  \l 387D
  \l 387E
  \l 387F
  \l 3880
  \l 3881
  \l 3882
  \l 3883
  \l 3884
  \l 3885
  \l 3886
  \l 3887
  \l 3888
  \l 3889
  \l 388A
  \l 388B
  \l 388C
  \l 388D
  \l 388E
  \l 388F
  \l 3890
  \l 3891
  \l 3892
  \l 3893
  \l 3894
  \l 3895
  \l 3896
  \l 3897
  \l 3898
  \l 3899
  \l 389A
  \l 389B
  \l 389C
  \l 389D
  \l 389E
  \l 389F
  \l 38A0
  \l 38A1
  \l 38A2
  \l 38A3
  \l 38A4
  \l 38A5
  \l 38A6
  \l 38A7
  \l 38A8
  \l 38A9
  \l 38AA
  \l 38AB
  \l 38AC
  \l 38AD
  \l 38AE
  \l 38AF
  \l 38B0
  \l 38B1
  \l 38B2
  \l 38B3
  \l 38B4
  \l 38B5
  \l 38B6
  \l 38B7
  \l 38B8
  \l 38B9
  \l 38BA
  \l 38BB
  \l 38BC
  \l 38BD
  \l 38BE
  \l 38BF
  \l 38C0
  \l 38C1
  \l 38C2
  \l 38C3
  \l 38C4
  \l 38C5
  \l 38C6
  \l 38C7
  \l 38C8
  \l 38C9
  \l 38CA
  \l 38CB
  \l 38CC
  \l 38CD
  \l 38CE
  \l 38CF
  \l 38D0
  \l 38D1
  \l 38D2
  \l 38D3
  \l 38D4
  \l 38D5
  \l 38D6
  \l 38D7
  \l 38D8
  \l 38D9
  \l 38DA
  \l 38DB
  \l 38DC
  \l 38DD
  \l 38DE
  \l 38DF
  \l 38E0
  \l 38E1
  \l 38E2
  \l 38E3
  \l 38E4
  \l 38E5
  \l 38E6
  \l 38E7
  \l 38E8
  \l 38E9
  \l 38EA
  \l 38EB
  \l 38EC
  \l 38ED
  \l 38EE
  \l 38EF
  \l 38F0
  \l 38F1
  \l 38F2
  \l 38F3
  \l 38F4
  \l 38F5
  \l 38F6
  \l 38F7
  \l 38F8
  \l 38F9
  \l 38FA
  \l 38FB
  \l 38FC
  \l 38FD
  \l 38FE
  \l 38FF
  \l 3900
  \l 3901
  \l 3902
  \l 3903
  \l 3904
  \l 3905
  \l 3906
  \l 3907
  \l 3908
  \l 3909
  \l 390A
  \l 390B
  \l 390C
  \l 390D
  \l 390E
  \l 390F
  \l 3910
  \l 3911
  \l 3912
  \l 3913
  \l 3914
  \l 3915
  \l 3916
  \l 3917
  \l 3918
  \l 3919
  \l 391A
  \l 391B
  \l 391C
  \l 391D
  \l 391E
  \l 391F
  \l 3920
  \l 3921
  \l 3922
  \l 3923
  \l 3924
  \l 3925
  \l 3926
  \l 3927
  \l 3928
  \l 3929
  \l 392A
  \l 392B
  \l 392C
  \l 392D
  \l 392E
  \l 392F
  \l 3930
  \l 3931
  \l 3932
  \l 3933
  \l 3934
  \l 3935
  \l 3936
  \l 3937
  \l 3938
  \l 3939
  \l 393A
  \l 393B
  \l 393C
  \l 393D
  \l 393E
  \l 393F
  \l 3940
  \l 3941
  \l 3942
  \l 3943
  \l 3944
  \l 3945
  \l 3946
  \l 3947
  \l 3948
  \l 3949
  \l 394A
  \l 394B
  \l 394C
  \l 394D
  \l 394E
  \l 394F
  \l 3950
  \l 3951
  \l 3952
  \l 3953
  \l 3954
  \l 3955
  \l 3956
  \l 3957
  \l 3958
  \l 3959
  \l 395A
  \l 395B
  \l 395C
  \l 395D
  \l 395E
  \l 395F
  \l 3960
  \l 3961
  \l 3962
  \l 3963
  \l 3964
  \l 3965
  \l 3966
  \l 3967
  \l 3968
  \l 3969
  \l 396A
  \l 396B
  \l 396C
  \l 396D
  \l 396E
  \l 396F
  \l 3970
  \l 3971
  \l 3972
  \l 3973
  \l 3974
  \l 3975
  \l 3976
  \l 3977
  \l 3978
  \l 3979
  \l 397A
  \l 397B
  \l 397C
  \l 397D
  \l 397E
  \l 397F
  \l 3980
  \l 3981
  \l 3982
  \l 3983
  \l 3984
  \l 3985
  \l 3986
  \l 3987
  \l 3988
  \l 3989
  \l 398A
  \l 398B
  \l 398C
  \l 398D
  \l 398E
  \l 398F
  \l 3990
  \l 3991
  \l 3992
  \l 3993
  \l 3994
  \l 3995
  \l 3996
  \l 3997
  \l 3998
  \l 3999
  \l 399A
  \l 399B
  \l 399C
  \l 399D
  \l 399E
  \l 399F
  \l 39A0
  \l 39A1
  \l 39A2
  \l 39A3
  \l 39A4
  \l 39A5
  \l 39A6
  \l 39A7
  \l 39A8
  \l 39A9
  \l 39AA
  \l 39AB
  \l 39AC
  \l 39AD
  \l 39AE
  \l 39AF
  \l 39B0
  \l 39B1
  \l 39B2
  \l 39B3
  \l 39B4
  \l 39B5
  \l 39B6
  \l 39B7
  \l 39B8
  \l 39B9
  \l 39BA
  \l 39BB
  \l 39BC
  \l 39BD
  \l 39BE
  \l 39BF
  \l 39C0
  \l 39C1
  \l 39C2
  \l 39C3
  \l 39C4
  \l 39C5
  \l 39C6
  \l 39C7
  \l 39C8
  \l 39C9
  \l 39CA
  \l 39CB
  \l 39CC
  \l 39CD
  \l 39CE
  \l 39CF
  \l 39D0
  \l 39D1
  \l 39D2
  \l 39D3
  \l 39D4
  \l 39D5
  \l 39D6
  \l 39D7
  \l 39D8
  \l 39D9
  \l 39DA
  \l 39DB
  \l 39DC
  \l 39DD
  \l 39DE
  \l 39DF
  \l 39E0
  \l 39E1
  \l 39E2
  \l 39E3
  \l 39E4
  \l 39E5
  \l 39E6
  \l 39E7
  \l 39E8
  \l 39E9
  \l 39EA
  \l 39EB
  \l 39EC
  \l 39ED
  \l 39EE
  \l 39EF
  \l 39F0
  \l 39F1
  \l 39F2
  \l 39F3
  \l 39F4
  \l 39F5
  \l 39F6
  \l 39F7
  \l 39F8
  \l 39F9
  \l 39FA
  \l 39FB
  \l 39FC
  \l 39FD
  \l 39FE
  \l 39FF
  \l 3A00
  \l 3A01
  \l 3A02
  \l 3A03
  \l 3A04
  \l 3A05
  \l 3A06
  \l 3A07
  \l 3A08
  \l 3A09
  \l 3A0A
  \l 3A0B
  \l 3A0C
  \l 3A0D
  \l 3A0E
  \l 3A0F
  \l 3A10
  \l 3A11
  \l 3A12
  \l 3A13
  \l 3A14
  \l 3A15
  \l 3A16
  \l 3A17
  \l 3A18
  \l 3A19
  \l 3A1A
  \l 3A1B
  \l 3A1C
  \l 3A1D
  \l 3A1E
  \l 3A1F
  \l 3A20
  \l 3A21
  \l 3A22
  \l 3A23
  \l 3A24
  \l 3A25
  \l 3A26
  \l 3A27
  \l 3A28
  \l 3A29
  \l 3A2A
  \l 3A2B
  \l 3A2C
  \l 3A2D
  \l 3A2E
  \l 3A2F
  \l 3A30
  \l 3A31
  \l 3A32
  \l 3A33
  \l 3A34
  \l 3A35
  \l 3A36
  \l 3A37
  \l 3A38
  \l 3A39
  \l 3A3A
  \l 3A3B
  \l 3A3C
  \l 3A3D
  \l 3A3E
  \l 3A3F
  \l 3A40
  \l 3A41
  \l 3A42
  \l 3A43
  \l 3A44
  \l 3A45
  \l 3A46
  \l 3A47
  \l 3A48
  \l 3A49
  \l 3A4A
  \l 3A4B
  \l 3A4C
  \l 3A4D
  \l 3A4E
  \l 3A4F
  \l 3A50
  \l 3A51
  \l 3A52
  \l 3A53
  \l 3A54
  \l 3A55
  \l 3A56
  \l 3A57
  \l 3A58
  \l 3A59
  \l 3A5A
  \l 3A5B
  \l 3A5C
  \l 3A5D
  \l 3A5E
  \l 3A5F
  \l 3A60
  \l 3A61
  \l 3A62
  \l 3A63
  \l 3A64
  \l 3A65
  \l 3A66
  \l 3A67
  \l 3A68
  \l 3A69
  \l 3A6A
  \l 3A6B
  \l 3A6C
  \l 3A6D
  \l 3A6E
  \l 3A6F
  \l 3A70
  \l 3A71
  \l 3A72
  \l 3A73
  \l 3A74
  \l 3A75
  \l 3A76
  \l 3A77
  \l 3A78
  \l 3A79
  \l 3A7A
  \l 3A7B
  \l 3A7C
  \l 3A7D
  \l 3A7E
  \l 3A7F
  \l 3A80
  \l 3A81
  \l 3A82
  \l 3A83
  \l 3A84
  \l 3A85
  \l 3A86
  \l 3A87
  \l 3A88
  \l 3A89
  \l 3A8A
  \l 3A8B
  \l 3A8C
  \l 3A8D
  \l 3A8E
  \l 3A8F
  \l 3A90
  \l 3A91
  \l 3A92
  \l 3A93
  \l 3A94
  \l 3A95
  \l 3A96
  \l 3A97
  \l 3A98
  \l 3A99
  \l 3A9A
  \l 3A9B
  \l 3A9C
  \l 3A9D
  \l 3A9E
  \l 3A9F
  \l 3AA0
  \l 3AA1
  \l 3AA2
  \l 3AA3
  \l 3AA4
  \l 3AA5
  \l 3AA6
  \l 3AA7
  \l 3AA8
  \l 3AA9
  \l 3AAA
  \l 3AAB
  \l 3AAC
  \l 3AAD
  \l 3AAE
  \l 3AAF
  \l 3AB0
  \l 3AB1
  \l 3AB2
  \l 3AB3
  \l 3AB4
  \l 3AB5
  \l 3AB6
  \l 3AB7
  \l 3AB8
  \l 3AB9
  \l 3ABA
  \l 3ABB
  \l 3ABC
  \l 3ABD
  \l 3ABE
  \l 3ABF
  \l 3AC0
  \l 3AC1
  \l 3AC2
  \l 3AC3
  \l 3AC4
  \l 3AC5
  \l 3AC6
  \l 3AC7
  \l 3AC8
  \l 3AC9
  \l 3ACA
  \l 3ACB
  \l 3ACC
  \l 3ACD
  \l 3ACE
  \l 3ACF
  \l 3AD0
  \l 3AD1
  \l 3AD2
  \l 3AD3
  \l 3AD4
  \l 3AD5
  \l 3AD6
  \l 3AD7
  \l 3AD8
  \l 3AD9
  \l 3ADA
  \l 3ADB
  \l 3ADC
  \l 3ADD
  \l 3ADE
  \l 3ADF
  \l 3AE0
  \l 3AE1
  \l 3AE2
  \l 3AE3
  \l 3AE4
  \l 3AE5
  \l 3AE6
  \l 3AE7
  \l 3AE8
  \l 3AE9
  \l 3AEA
  \l 3AEB
  \l 3AEC
  \l 3AED
  \l 3AEE
  \l 3AEF
  \l 3AF0
  \l 3AF1
  \l 3AF2
  \l 3AF3
  \l 3AF4
  \l 3AF5
  \l 3AF6
  \l 3AF7
  \l 3AF8
  \l 3AF9
  \l 3AFA
  \l 3AFB
  \l 3AFC
  \l 3AFD
  \l 3AFE
  \l 3AFF
  \l 3B00
  \l 3B01
  \l 3B02
  \l 3B03
  \l 3B04
  \l 3B05
  \l 3B06
  \l 3B07
  \l 3B08
  \l 3B09
  \l 3B0A
  \l 3B0B
  \l 3B0C
  \l 3B0D
  \l 3B0E
  \l 3B0F
  \l 3B10
  \l 3B11
  \l 3B12
  \l 3B13
  \l 3B14
  \l 3B15
  \l 3B16
  \l 3B17
  \l 3B18
  \l 3B19
  \l 3B1A
  \l 3B1B
  \l 3B1C
  \l 3B1D
  \l 3B1E
  \l 3B1F
  \l 3B20
  \l 3B21
  \l 3B22
  \l 3B23
  \l 3B24
  \l 3B25
  \l 3B26
  \l 3B27
  \l 3B28
  \l 3B29
  \l 3B2A
  \l 3B2B
  \l 3B2C
  \l 3B2D
  \l 3B2E
  \l 3B2F
  \l 3B30
  \l 3B31
  \l 3B32
  \l 3B33
  \l 3B34
  \l 3B35
  \l 3B36
  \l 3B37
  \l 3B38
  \l 3B39
  \l 3B3A
  \l 3B3B
  \l 3B3C
  \l 3B3D
  \l 3B3E
  \l 3B3F
  \l 3B40
  \l 3B41
  \l 3B42
  \l 3B43
  \l 3B44
  \l 3B45
  \l 3B46
  \l 3B47
  \l 3B48
  \l 3B49
  \l 3B4A
  \l 3B4B
  \l 3B4C
  \l 3B4D
  \l 3B4E
  \l 3B4F
  \l 3B50
  \l 3B51
  \l 3B52
  \l 3B53
  \l 3B54
  \l 3B55
  \l 3B56
  \l 3B57
  \l 3B58
  \l 3B59
  \l 3B5A
  \l 3B5B
  \l 3B5C
  \l 3B5D
  \l 3B5E
  \l 3B5F
  \l 3B60
  \l 3B61
  \l 3B62
  \l 3B63
  \l 3B64
  \l 3B65
  \l 3B66
  \l 3B67
  \l 3B68
  \l 3B69
  \l 3B6A
  \l 3B6B
  \l 3B6C
  \l 3B6D
  \l 3B6E
  \l 3B6F
  \l 3B70
  \l 3B71
  \l 3B72
  \l 3B73
  \l 3B74
  \l 3B75
  \l 3B76
  \l 3B77
  \l 3B78
  \l 3B79
  \l 3B7A
  \l 3B7B
  \l 3B7C
  \l 3B7D
  \l 3B7E
  \l 3B7F
  \l 3B80
  \l 3B81
  \l 3B82
  \l 3B83
  \l 3B84
  \l 3B85
  \l 3B86
  \l 3B87
  \l 3B88
  \l 3B89
  \l 3B8A
  \l 3B8B
  \l 3B8C
  \l 3B8D
  \l 3B8E
  \l 3B8F
  \l 3B90
  \l 3B91
  \l 3B92
  \l 3B93
  \l 3B94
  \l 3B95
  \l 3B96
  \l 3B97
  \l 3B98
  \l 3B99
  \l 3B9A
  \l 3B9B
  \l 3B9C
  \l 3B9D
  \l 3B9E
  \l 3B9F
  \l 3BA0
  \l 3BA1
  \l 3BA2
  \l 3BA3
  \l 3BA4
  \l 3BA5
  \l 3BA6
  \l 3BA7
  \l 3BA8
  \l 3BA9
  \l 3BAA
  \l 3BAB
  \l 3BAC
  \l 3BAD
  \l 3BAE
  \l 3BAF
  \l 3BB0
  \l 3BB1
  \l 3BB2
  \l 3BB3
  \l 3BB4
  \l 3BB5
  \l 3BB6
  \l 3BB7
  \l 3BB8
  \l 3BB9
  \l 3BBA
  \l 3BBB
  \l 3BBC
  \l 3BBD
  \l 3BBE
  \l 3BBF
  \l 3BC0
  \l 3BC1
  \l 3BC2
  \l 3BC3
  \l 3BC4
  \l 3BC5
  \l 3BC6
  \l 3BC7
  \l 3BC8
  \l 3BC9
  \l 3BCA
  \l 3BCB
  \l 3BCC
  \l 3BCD
  \l 3BCE
  \l 3BCF
  \l 3BD0
  \l 3BD1
  \l 3BD2
  \l 3BD3
  \l 3BD4
  \l 3BD5
  \l 3BD6
  \l 3BD7
  \l 3BD8
  \l 3BD9
  \l 3BDA
  \l 3BDB
  \l 3BDC
  \l 3BDD
  \l 3BDE
  \l 3BDF
  \l 3BE0
  \l 3BE1
  \l 3BE2
  \l 3BE3
  \l 3BE4
  \l 3BE5
  \l 3BE6
  \l 3BE7
  \l 3BE8
  \l 3BE9
  \l 3BEA
  \l 3BEB
  \l 3BEC
  \l 3BED
  \l 3BEE
  \l 3BEF
  \l 3BF0
  \l 3BF1
  \l 3BF2
  \l 3BF3
  \l 3BF4
  \l 3BF5
  \l 3BF6
  \l 3BF7
  \l 3BF8
  \l 3BF9
  \l 3BFA
  \l 3BFB
  \l 3BFC
  \l 3BFD
  \l 3BFE
  \l 3BFF
  \l 3C00
  \l 3C01
  \l 3C02
  \l 3C03
  \l 3C04
  \l 3C05
  \l 3C06
  \l 3C07
  \l 3C08
  \l 3C09
  \l 3C0A
  \l 3C0B
  \l 3C0C
  \l 3C0D
  \l 3C0E
  \l 3C0F
  \l 3C10
  \l 3C11
  \l 3C12
  \l 3C13
  \l 3C14
  \l 3C15
  \l 3C16
  \l 3C17
  \l 3C18
  \l 3C19
  \l 3C1A
  \l 3C1B
  \l 3C1C
  \l 3C1D
  \l 3C1E
  \l 3C1F
  \l 3C20
  \l 3C21
  \l 3C22
  \l 3C23
  \l 3C24
  \l 3C25
  \l 3C26
  \l 3C27
  \l 3C28
  \l 3C29
  \l 3C2A
  \l 3C2B
  \l 3C2C
  \l 3C2D
  \l 3C2E
  \l 3C2F
  \l 3C30
  \l 3C31
  \l 3C32
  \l 3C33
  \l 3C34
  \l 3C35
  \l 3C36
  \l 3C37
  \l 3C38
  \l 3C39
  \l 3C3A
  \l 3C3B
  \l 3C3C
  \l 3C3D
  \l 3C3E
  \l 3C3F
  \l 3C40
  \l 3C41
  \l 3C42
  \l 3C43
  \l 3C44
  \l 3C45
  \l 3C46
  \l 3C47
  \l 3C48
  \l 3C49
  \l 3C4A
  \l 3C4B
  \l 3C4C
  \l 3C4D
  \l 3C4E
  \l 3C4F
  \l 3C50
  \l 3C51
  \l 3C52
  \l 3C53
  \l 3C54
  \l 3C55
  \l 3C56
  \l 3C57
  \l 3C58
  \l 3C59
  \l 3C5A
  \l 3C5B
  \l 3C5C
  \l 3C5D
  \l 3C5E
  \l 3C5F
  \l 3C60
  \l 3C61
  \l 3C62
  \l 3C63
  \l 3C64
  \l 3C65
  \l 3C66
  \l 3C67
  \l 3C68
  \l 3C69
  \l 3C6A
  \l 3C6B
  \l 3C6C
  \l 3C6D
  \l 3C6E
  \l 3C6F
  \l 3C70
  \l 3C71
  \l 3C72
  \l 3C73
  \l 3C74
  \l 3C75
  \l 3C76
  \l 3C77
  \l 3C78
  \l 3C79
  \l 3C7A
  \l 3C7B
  \l 3C7C
  \l 3C7D
  \l 3C7E
  \l 3C7F
  \l 3C80
  \l 3C81
  \l 3C82
  \l 3C83
  \l 3C84
  \l 3C85
  \l 3C86
  \l 3C87
  \l 3C88
  \l 3C89
  \l 3C8A
  \l 3C8B
  \l 3C8C
  \l 3C8D
  \l 3C8E
  \l 3C8F
  \l 3C90
  \l 3C91
  \l 3C92
  \l 3C93
  \l 3C94
  \l 3C95
  \l 3C96
  \l 3C97
  \l 3C98
  \l 3C99
  \l 3C9A
  \l 3C9B
  \l 3C9C
  \l 3C9D
  \l 3C9E
  \l 3C9F
  \l 3CA0
  \l 3CA1
  \l 3CA2
  \l 3CA3
  \l 3CA4
  \l 3CA5
  \l 3CA6
  \l 3CA7
  \l 3CA8
  \l 3CA9
  \l 3CAA
  \l 3CAB
  \l 3CAC
  \l 3CAD
  \l 3CAE
  \l 3CAF
  \l 3CB0
  \l 3CB1
  \l 3CB2
  \l 3CB3
  \l 3CB4
  \l 3CB5
  \l 3CB6
  \l 3CB7
  \l 3CB8
  \l 3CB9
  \l 3CBA
  \l 3CBB
  \l 3CBC
  \l 3CBD
  \l 3CBE
  \l 3CBF
  \l 3CC0
  \l 3CC1
  \l 3CC2
  \l 3CC3
  \l 3CC4
  \l 3CC5
  \l 3CC6
  \l 3CC7
  \l 3CC8
  \l 3CC9
  \l 3CCA
  \l 3CCB
  \l 3CCC
  \l 3CCD
  \l 3CCE
  \l 3CCF
  \l 3CD0
  \l 3CD1
  \l 3CD2
  \l 3CD3
  \l 3CD4
  \l 3CD5
  \l 3CD6
  \l 3CD7
  \l 3CD8
  \l 3CD9
  \l 3CDA
  \l 3CDB
  \l 3CDC
  \l 3CDD
  \l 3CDE
  \l 3CDF
  \l 3CE0
  \l 3CE1
  \l 3CE2
  \l 3CE3
  \l 3CE4
  \l 3CE5
  \l 3CE6
  \l 3CE7
  \l 3CE8
  \l 3CE9
  \l 3CEA
  \l 3CEB
  \l 3CEC
  \l 3CED
  \l 3CEE
  \l 3CEF
  \l 3CF0
  \l 3CF1
  \l 3CF2
  \l 3CF3
  \l 3CF4
  \l 3CF5
  \l 3CF6
  \l 3CF7
  \l 3CF8
  \l 3CF9
  \l 3CFA
  \l 3CFB
  \l 3CFC
  \l 3CFD
  \l 3CFE
  \l 3CFF
  \l 3D00
  \l 3D01
  \l 3D02
  \l 3D03
  \l 3D04
  \l 3D05
  \l 3D06
  \l 3D07
  \l 3D08
  \l 3D09
  \l 3D0A
  \l 3D0B
  \l 3D0C
  \l 3D0D
  \l 3D0E
  \l 3D0F
  \l 3D10
  \l 3D11
  \l 3D12
  \l 3D13
  \l 3D14
  \l 3D15
  \l 3D16
  \l 3D17
  \l 3D18
  \l 3D19
  \l 3D1A
  \l 3D1B
  \l 3D1C
  \l 3D1D
  \l 3D1E
  \l 3D1F
  \l 3D20
  \l 3D21
  \l 3D22
  \l 3D23
  \l 3D24
  \l 3D25
  \l 3D26
  \l 3D27
  \l 3D28
  \l 3D29
  \l 3D2A
  \l 3D2B
  \l 3D2C
  \l 3D2D
  \l 3D2E
  \l 3D2F
  \l 3D30
  \l 3D31
  \l 3D32
  \l 3D33
  \l 3D34
  \l 3D35
  \l 3D36
  \l 3D37
  \l 3D38
  \l 3D39
  \l 3D3A
  \l 3D3B
  \l 3D3C
  \l 3D3D
  \l 3D3E
  \l 3D3F
  \l 3D40
  \l 3D41
  \l 3D42
  \l 3D43
  \l 3D44
  \l 3D45
  \l 3D46
  \l 3D47
  \l 3D48
  \l 3D49
  \l 3D4A
  \l 3D4B
  \l 3D4C
  \l 3D4D
  \l 3D4E
  \l 3D4F
  \l 3D50
  \l 3D51
  \l 3D52
  \l 3D53
  \l 3D54
  \l 3D55
  \l 3D56
  \l 3D57
  \l 3D58
  \l 3D59
  \l 3D5A
  \l 3D5B
  \l 3D5C
  \l 3D5D
  \l 3D5E
  \l 3D5F
  \l 3D60
  \l 3D61
  \l 3D62
  \l 3D63
  \l 3D64
  \l 3D65
  \l 3D66
  \l 3D67
  \l 3D68
  \l 3D69
  \l 3D6A
  \l 3D6B
  \l 3D6C
  \l 3D6D
  \l 3D6E
  \l 3D6F
  \l 3D70
  \l 3D71
  \l 3D72
  \l 3D73
  \l 3D74
  \l 3D75
  \l 3D76
  \l 3D77
  \l 3D78
  \l 3D79
  \l 3D7A
  \l 3D7B
  \l 3D7C
  \l 3D7D
  \l 3D7E
  \l 3D7F
  \l 3D80
  \l 3D81
  \l 3D82
  \l 3D83
  \l 3D84
  \l 3D85
  \l 3D86
  \l 3D87
  \l 3D88
  \l 3D89
  \l 3D8A
  \l 3D8B
  \l 3D8C
  \l 3D8D
  \l 3D8E
  \l 3D8F
  \l 3D90
  \l 3D91
  \l 3D92
  \l 3D93
  \l 3D94
  \l 3D95
  \l 3D96
  \l 3D97
  \l 3D98
  \l 3D99
  \l 3D9A
  \l 3D9B
  \l 3D9C
  \l 3D9D
  \l 3D9E
  \l 3D9F
  \l 3DA0
  \l 3DA1
  \l 3DA2
  \l 3DA3
  \l 3DA4
  \l 3DA5
  \l 3DA6
  \l 3DA7
  \l 3DA8
  \l 3DA9
  \l 3DAA
  \l 3DAB
  \l 3DAC
  \l 3DAD
  \l 3DAE
  \l 3DAF
  \l 3DB0
  \l 3DB1
  \l 3DB2
  \l 3DB3
  \l 3DB4
  \l 3DB5
  \l 3DB6
  \l 3DB7
  \l 3DB8
  \l 3DB9
  \l 3DBA
  \l 3DBB
  \l 3DBC
  \l 3DBD
  \l 3DBE
  \l 3DBF
  \l 3DC0
  \l 3DC1
  \l 3DC2
  \l 3DC3
  \l 3DC4
  \l 3DC5
  \l 3DC6
  \l 3DC7
  \l 3DC8
  \l 3DC9
  \l 3DCA
  \l 3DCB
  \l 3DCC
  \l 3DCD
  \l 3DCE
  \l 3DCF
  \l 3DD0
  \l 3DD1
  \l 3DD2
  \l 3DD3
  \l 3DD4
  \l 3DD5
  \l 3DD6
  \l 3DD7
  \l 3DD8
  \l 3DD9
  \l 3DDA
  \l 3DDB
  \l 3DDC
  \l 3DDD
  \l 3DDE
  \l 3DDF
  \l 3DE0
  \l 3DE1
  \l 3DE2
  \l 3DE3
  \l 3DE4
  \l 3DE5
  \l 3DE6
  \l 3DE7
  \l 3DE8
  \l 3DE9
  \l 3DEA
  \l 3DEB
  \l 3DEC
  \l 3DED
  \l 3DEE
  \l 3DEF
  \l 3DF0
  \l 3DF1
  \l 3DF2
  \l 3DF3
  \l 3DF4
  \l 3DF5
  \l 3DF6
  \l 3DF7
  \l 3DF8
  \l 3DF9
  \l 3DFA
  \l 3DFB
  \l 3DFC
  \l 3DFD
  \l 3DFE
  \l 3DFF
  \l 3E00
  \l 3E01
  \l 3E02
  \l 3E03
  \l 3E04
  \l 3E05
  \l 3E06
  \l 3E07
  \l 3E08
  \l 3E09
  \l 3E0A
  \l 3E0B
  \l 3E0C
  \l 3E0D
  \l 3E0E
  \l 3E0F
  \l 3E10
  \l 3E11
  \l 3E12
  \l 3E13
  \l 3E14
  \l 3E15
  \l 3E16
  \l 3E17
  \l 3E18
  \l 3E19
  \l 3E1A
  \l 3E1B
  \l 3E1C
  \l 3E1D
  \l 3E1E
  \l 3E1F
  \l 3E20
  \l 3E21
  \l 3E22
  \l 3E23
  \l 3E24
  \l 3E25
  \l 3E26
  \l 3E27
  \l 3E28
  \l 3E29
  \l 3E2A
  \l 3E2B
  \l 3E2C
  \l 3E2D
  \l 3E2E
  \l 3E2F
  \l 3E30
  \l 3E31
  \l 3E32
  \l 3E33
  \l 3E34
  \l 3E35
  \l 3E36
  \l 3E37
  \l 3E38
  \l 3E39
  \l 3E3A
  \l 3E3B
  \l 3E3C
  \l 3E3D
  \l 3E3E
  \l 3E3F
  \l 3E40
  \l 3E41
  \l 3E42
  \l 3E43
  \l 3E44
  \l 3E45
  \l 3E46
  \l 3E47
  \l 3E48
  \l 3E49
  \l 3E4A
  \l 3E4B
  \l 3E4C
  \l 3E4D
  \l 3E4E
  \l 3E4F
  \l 3E50
  \l 3E51
  \l 3E52
  \l 3E53
  \l 3E54
  \l 3E55
  \l 3E56
  \l 3E57
  \l 3E58
  \l 3E59
  \l 3E5A
  \l 3E5B
  \l 3E5C
  \l 3E5D
  \l 3E5E
  \l 3E5F
  \l 3E60
  \l 3E61
  \l 3E62
  \l 3E63
  \l 3E64
  \l 3E65
  \l 3E66
  \l 3E67
  \l 3E68
  \l 3E69
  \l 3E6A
  \l 3E6B
  \l 3E6C
  \l 3E6D
  \l 3E6E
  \l 3E6F
  \l 3E70
  \l 3E71
  \l 3E72
  \l 3E73
  \l 3E74
  \l 3E75
  \l 3E76
  \l 3E77
  \l 3E78
  \l 3E79
  \l 3E7A
  \l 3E7B
  \l 3E7C
  \l 3E7D
  \l 3E7E
  \l 3E7F
  \l 3E80
  \l 3E81
  \l 3E82
  \l 3E83
  \l 3E84
  \l 3E85
  \l 3E86
  \l 3E87
  \l 3E88
  \l 3E89
  \l 3E8A
  \l 3E8B
  \l 3E8C
  \l 3E8D
  \l 3E8E
  \l 3E8F
  \l 3E90
  \l 3E91
  \l 3E92
  \l 3E93
  \l 3E94
  \l 3E95
  \l 3E96
  \l 3E97
  \l 3E98
  \l 3E99
  \l 3E9A
  \l 3E9B
  \l 3E9C
  \l 3E9D
  \l 3E9E
  \l 3E9F
  \l 3EA0
  \l 3EA1
  \l 3EA2
  \l 3EA3
  \l 3EA4
  \l 3EA5
  \l 3EA6
  \l 3EA7
  \l 3EA8
  \l 3EA9
  \l 3EAA
  \l 3EAB
  \l 3EAC
  \l 3EAD
  \l 3EAE
  \l 3EAF
  \l 3EB0
  \l 3EB1
  \l 3EB2
  \l 3EB3
  \l 3EB4
  \l 3EB5
  \l 3EB6
  \l 3EB7
  \l 3EB8
  \l 3EB9
  \l 3EBA
  \l 3EBB
  \l 3EBC
  \l 3EBD
  \l 3EBE
  \l 3EBF
  \l 3EC0
  \l 3EC1
  \l 3EC2
  \l 3EC3
  \l 3EC4
  \l 3EC5
  \l 3EC6
  \l 3EC7
  \l 3EC8
  \l 3EC9
  \l 3ECA
  \l 3ECB
  \l 3ECC
  \l 3ECD
  \l 3ECE
  \l 3ECF
  \l 3ED0
  \l 3ED1
  \l 3ED2
  \l 3ED3
  \l 3ED4
  \l 3ED5
  \l 3ED6
  \l 3ED7
  \l 3ED8
  \l 3ED9
  \l 3EDA
  \l 3EDB
  \l 3EDC
  \l 3EDD
  \l 3EDE
  \l 3EDF
  \l 3EE0
  \l 3EE1
  \l 3EE2
  \l 3EE3
  \l 3EE4
  \l 3EE5
  \l 3EE6
  \l 3EE7
  \l 3EE8
  \l 3EE9
  \l 3EEA
  \l 3EEB
  \l 3EEC
  \l 3EED
  \l 3EEE
  \l 3EEF
  \l 3EF0
  \l 3EF1
  \l 3EF2
  \l 3EF3
  \l 3EF4
  \l 3EF5
  \l 3EF6
  \l 3EF7
  \l 3EF8
  \l 3EF9
  \l 3EFA
  \l 3EFB
  \l 3EFC
  \l 3EFD
  \l 3EFE
  \l 3EFF
  \l 3F00
  \l 3F01
  \l 3F02
  \l 3F03
  \l 3F04
  \l 3F05
  \l 3F06
  \l 3F07
  \l 3F08
  \l 3F09
  \l 3F0A
  \l 3F0B
  \l 3F0C
  \l 3F0D
  \l 3F0E
  \l 3F0F
  \l 3F10
  \l 3F11
  \l 3F12
  \l 3F13
  \l 3F14
  \l 3F15
  \l 3F16
  \l 3F17
  \l 3F18
  \l 3F19
  \l 3F1A
  \l 3F1B
  \l 3F1C
  \l 3F1D
  \l 3F1E
  \l 3F1F
  \l 3F20
  \l 3F21
  \l 3F22
  \l 3F23
  \l 3F24
  \l 3F25
  \l 3F26
  \l 3F27
  \l 3F28
  \l 3F29
  \l 3F2A
  \l 3F2B
  \l 3F2C
  \l 3F2D
  \l 3F2E
  \l 3F2F
  \l 3F30
  \l 3F31
  \l 3F32
  \l 3F33
  \l 3F34
  \l 3F35
  \l 3F36
  \l 3F37
  \l 3F38
  \l 3F39
  \l 3F3A
  \l 3F3B
  \l 3F3C
  \l 3F3D
  \l 3F3E
  \l 3F3F
  \l 3F40
  \l 3F41
  \l 3F42
  \l 3F43
  \l 3F44
  \l 3F45
  \l 3F46
  \l 3F47
  \l 3F48
  \l 3F49
  \l 3F4A
  \l 3F4B
  \l 3F4C
  \l 3F4D
  \l 3F4E
  \l 3F4F
  \l 3F50
  \l 3F51
  \l 3F52
  \l 3F53
  \l 3F54
  \l 3F55
  \l 3F56
  \l 3F57
  \l 3F58
  \l 3F59
  \l 3F5A
  \l 3F5B
  \l 3F5C
  \l 3F5D
  \l 3F5E
  \l 3F5F
  \l 3F60
  \l 3F61
  \l 3F62
  \l 3F63
  \l 3F64
  \l 3F65
  \l 3F66
  \l 3F67
  \l 3F68
  \l 3F69
  \l 3F6A
  \l 3F6B
  \l 3F6C
  \l 3F6D
  \l 3F6E
  \l 3F6F
  \l 3F70
  \l 3F71
  \l 3F72
  \l 3F73
  \l 3F74
  \l 3F75
  \l 3F76
  \l 3F77
  \l 3F78
  \l 3F79
  \l 3F7A
  \l 3F7B
  \l 3F7C
  \l 3F7D
  \l 3F7E
  \l 3F7F
  \l 3F80
  \l 3F81
  \l 3F82
  \l 3F83
  \l 3F84
  \l 3F85
  \l 3F86
  \l 3F87
  \l 3F88
  \l 3F89
  \l 3F8A
  \l 3F8B
  \l 3F8C
  \l 3F8D
  \l 3F8E
  \l 3F8F
  \l 3F90
  \l 3F91
  \l 3F92
  \l 3F93
  \l 3F94
  \l 3F95
  \l 3F96
  \l 3F97
  \l 3F98
  \l 3F99
  \l 3F9A
  \l 3F9B
  \l 3F9C
  \l 3F9D
  \l 3F9E
  \l 3F9F
  \l 3FA0
  \l 3FA1
  \l 3FA2
  \l 3FA3
  \l 3FA4
  \l 3FA5
  \l 3FA6
  \l 3FA7
  \l 3FA8
  \l 3FA9
  \l 3FAA
  \l 3FAB
  \l 3FAC
  \l 3FAD
  \l 3FAE
  \l 3FAF
  \l 3FB0
  \l 3FB1
  \l 3FB2
  \l 3FB3
  \l 3FB4
  \l 3FB5
  \l 3FB6
  \l 3FB7
  \l 3FB8
  \l 3FB9
  \l 3FBA
  \l 3FBB
  \l 3FBC
  \l 3FBD
  \l 3FBE
  \l 3FBF
  \l 3FC0
  \l 3FC1
  \l 3FC2
  \l 3FC3
  \l 3FC4
  \l 3FC5
  \l 3FC6
  \l 3FC7
  \l 3FC8
  \l 3FC9
  \l 3FCA
  \l 3FCB
  \l 3FCC
  \l 3FCD
  \l 3FCE
  \l 3FCF
  \l 3FD0
  \l 3FD1
  \l 3FD2
  \l 3FD3
  \l 3FD4
  \l 3FD5
  \l 3FD6
  \l 3FD7
  \l 3FD8
  \l 3FD9
  \l 3FDA
  \l 3FDB
  \l 3FDC
  \l 3FDD
  \l 3FDE
  \l 3FDF
  \l 3FE0
  \l 3FE1
  \l 3FE2
  \l 3FE3
  \l 3FE4
  \l 3FE5
  \l 3FE6
  \l 3FE7
  \l 3FE8
  \l 3FE9
  \l 3FEA
  \l 3FEB
  \l 3FEC
  \l 3FED
  \l 3FEE
  \l 3FEF
  \l 3FF0
  \l 3FF1
  \l 3FF2
  \l 3FF3
  \l 3FF4
  \l 3FF5
  \l 3FF6
  \l 3FF7
  \l 3FF8
  \l 3FF9
  \l 3FFA
  \l 3FFB
  \l 3FFC
  \l 3FFD
  \l 3FFE
  \l 3FFF
  \l 4000
  \l 4001
  \l 4002
  \l 4003
  \l 4004
  \l 4005
  \l 4006
  \l 4007
  \l 4008
  \l 4009
  \l 400A
  \l 400B
  \l 400C
  \l 400D
  \l 400E
  \l 400F
  \l 4010
  \l 4011
  \l 4012
  \l 4013
  \l 4014
  \l 4015
  \l 4016
  \l 4017
  \l 4018
  \l 4019
  \l 401A
  \l 401B
  \l 401C
  \l 401D
  \l 401E
  \l 401F
  \l 4020
  \l 4021
  \l 4022
  \l 4023
  \l 4024
  \l 4025
  \l 4026
  \l 4027
  \l 4028
  \l 4029
  \l 402A
  \l 402B
  \l 402C
  \l 402D
  \l 402E
  \l 402F
  \l 4030
  \l 4031
  \l 4032
  \l 4033
  \l 4034
  \l 4035
  \l 4036
  \l 4037
  \l 4038
  \l 4039
  \l 403A
  \l 403B
  \l 403C
  \l 403D
  \l 403E
  \l 403F
  \l 4040
  \l 4041
  \l 4042
  \l 4043
  \l 4044
  \l 4045
  \l 4046
  \l 4047
  \l 4048
  \l 4049
  \l 404A
  \l 404B
  \l 404C
  \l 404D
  \l 404E
  \l 404F
  \l 4050
  \l 4051
  \l 4052
  \l 4053
  \l 4054
  \l 4055
  \l 4056
  \l 4057
  \l 4058
  \l 4059
  \l 405A
  \l 405B
  \l 405C
  \l 405D
  \l 405E
  \l 405F
  \l 4060
  \l 4061
  \l 4062
  \l 4063
  \l 4064
  \l 4065
  \l 4066
  \l 4067
  \l 4068
  \l 4069
  \l 406A
  \l 406B
  \l 406C
  \l 406D
  \l 406E
  \l 406F
  \l 4070
  \l 4071
  \l 4072
  \l 4073
  \l 4074
  \l 4075
  \l 4076
  \l 4077
  \l 4078
  \l 4079
  \l 407A
  \l 407B
  \l 407C
  \l 407D
  \l 407E
  \l 407F
  \l 4080
  \l 4081
  \l 4082
  \l 4083
  \l 4084
  \l 4085
  \l 4086
  \l 4087
  \l 4088
  \l 4089
  \l 408A
  \l 408B
  \l 408C
  \l 408D
  \l 408E
  \l 408F
  \l 4090
  \l 4091
  \l 4092
  \l 4093
  \l 4094
  \l 4095
  \l 4096
  \l 4097
  \l 4098
  \l 4099
  \l 409A
  \l 409B
  \l 409C
  \l 409D
  \l 409E
  \l 409F
  \l 40A0
  \l 40A1
  \l 40A2
  \l 40A3
  \l 40A4
  \l 40A5
  \l 40A6
  \l 40A7
  \l 40A8
  \l 40A9
  \l 40AA
  \l 40AB
  \l 40AC
  \l 40AD
  \l 40AE
  \l 40AF
  \l 40B0
  \l 40B1
  \l 40B2
  \l 40B3
  \l 40B4
  \l 40B5
  \l 40B6
  \l 40B7
  \l 40B8
  \l 40B9
  \l 40BA
  \l 40BB
  \l 40BC
  \l 40BD
  \l 40BE
  \l 40BF
  \l 40C0
  \l 40C1
  \l 40C2
  \l 40C3
  \l 40C4
  \l 40C5
  \l 40C6
  \l 40C7
  \l 40C8
  \l 40C9
  \l 40CA
  \l 40CB
  \l 40CC
  \l 40CD
  \l 40CE
  \l 40CF
  \l 40D0
  \l 40D1
  \l 40D2
  \l 40D3
  \l 40D4
  \l 40D5
  \l 40D6
  \l 40D7
  \l 40D8
  \l 40D9
  \l 40DA
  \l 40DB
  \l 40DC
  \l 40DD
  \l 40DE
  \l 40DF
  \l 40E0
  \l 40E1
  \l 40E2
  \l 40E3
  \l 40E4
  \l 40E5
  \l 40E6
  \l 40E7
  \l 40E8
  \l 40E9
  \l 40EA
  \l 40EB
  \l 40EC
  \l 40ED
  \l 40EE
  \l 40EF
  \l 40F0
  \l 40F1
  \l 40F2
  \l 40F3
  \l 40F4
  \l 40F5
  \l 40F6
  \l 40F7
  \l 40F8
  \l 40F9
  \l 40FA
  \l 40FB
  \l 40FC
  \l 40FD
  \l 40FE
  \l 40FF
  \l 4100
  \l 4101
  \l 4102
  \l 4103
  \l 4104
  \l 4105
  \l 4106
  \l 4107
  \l 4108
  \l 4109
  \l 410A
  \l 410B
  \l 410C
  \l 410D
  \l 410E
  \l 410F
  \l 4110
  \l 4111
  \l 4112
  \l 4113
  \l 4114
  \l 4115
  \l 4116
  \l 4117
  \l 4118
  \l 4119
  \l 411A
  \l 411B
  \l 411C
  \l 411D
  \l 411E
  \l 411F
  \l 4120
  \l 4121
  \l 4122
  \l 4123
  \l 4124
  \l 4125
  \l 4126
  \l 4127
  \l 4128
  \l 4129
  \l 412A
  \l 412B
  \l 412C
  \l 412D
  \l 412E
  \l 412F
  \l 4130
  \l 4131
  \l 4132
  \l 4133
  \l 4134
  \l 4135
  \l 4136
  \l 4137
  \l 4138
  \l 4139
  \l 413A
  \l 413B
  \l 413C
  \l 413D
  \l 413E
  \l 413F
  \l 4140
  \l 4141
  \l 4142
  \l 4143
  \l 4144
  \l 4145
  \l 4146
  \l 4147
  \l 4148
  \l 4149
  \l 414A
  \l 414B
  \l 414C
  \l 414D
  \l 414E
  \l 414F
  \l 4150
  \l 4151
  \l 4152
  \l 4153
  \l 4154
  \l 4155
  \l 4156
  \l 4157
  \l 4158
  \l 4159
  \l 415A
  \l 415B
  \l 415C
  \l 415D
  \l 415E
  \l 415F
  \l 4160
  \l 4161
  \l 4162
  \l 4163
  \l 4164
  \l 4165
  \l 4166
  \l 4167
  \l 4168
  \l 4169
  \l 416A
  \l 416B
  \l 416C
  \l 416D
  \l 416E
  \l 416F
  \l 4170
  \l 4171
  \l 4172
  \l 4173
  \l 4174
  \l 4175
  \l 4176
  \l 4177
  \l 4178
  \l 4179
  \l 417A
  \l 417B
  \l 417C
  \l 417D
  \l 417E
  \l 417F
  \l 4180
  \l 4181
  \l 4182
  \l 4183
  \l 4184
  \l 4185
  \l 4186
  \l 4187
  \l 4188
  \l 4189
  \l 418A
  \l 418B
  \l 418C
  \l 418D
  \l 418E
  \l 418F
  \l 4190
  \l 4191
  \l 4192
  \l 4193
  \l 4194
  \l 4195
  \l 4196
  \l 4197
  \l 4198
  \l 4199
  \l 419A
  \l 419B
  \l 419C
  \l 419D
  \l 419E
  \l 419F
  \l 41A0
  \l 41A1
  \l 41A2
  \l 41A3
  \l 41A4
  \l 41A5
  \l 41A6
  \l 41A7
  \l 41A8
  \l 41A9
  \l 41AA
  \l 41AB
  \l 41AC
  \l 41AD
  \l 41AE
  \l 41AF
  \l 41B0
  \l 41B1
  \l 41B2
  \l 41B3
  \l 41B4
  \l 41B5
  \l 41B6
  \l 41B7
  \l 41B8
  \l 41B9
  \l 41BA
  \l 41BB
  \l 41BC
  \l 41BD
  \l 41BE
  \l 41BF
  \l 41C0
  \l 41C1
  \l 41C2
  \l 41C3
  \l 41C4
  \l 41C5
  \l 41C6
  \l 41C7
  \l 41C8
  \l 41C9
  \l 41CA
  \l 41CB
  \l 41CC
  \l 41CD
  \l 41CE
  \l 41CF
  \l 41D0
  \l 41D1
  \l 41D2
  \l 41D3
  \l 41D4
  \l 41D5
  \l 41D6
  \l 41D7
  \l 41D8
  \l 41D9
  \l 41DA
  \l 41DB
  \l 41DC
  \l 41DD
  \l 41DE
  \l 41DF
  \l 41E0
  \l 41E1
  \l 41E2
  \l 41E3
  \l 41E4
  \l 41E5
  \l 41E6
  \l 41E7
  \l 41E8
  \l 41E9
  \l 41EA
  \l 41EB
  \l 41EC
  \l 41ED
  \l 41EE
  \l 41EF
  \l 41F0
  \l 41F1
  \l 41F2
  \l 41F3
  \l 41F4
  \l 41F5
  \l 41F6
  \l 41F7
  \l 41F8
  \l 41F9
  \l 41FA
  \l 41FB
  \l 41FC
  \l 41FD
  \l 41FE
  \l 41FF
  \l 4200
  \l 4201
  \l 4202
  \l 4203
  \l 4204
  \l 4205
  \l 4206
  \l 4207
  \l 4208
  \l 4209
  \l 420A
  \l 420B
  \l 420C
  \l 420D
  \l 420E
  \l 420F
  \l 4210
  \l 4211
  \l 4212
  \l 4213
  \l 4214
  \l 4215
  \l 4216
  \l 4217
  \l 4218
  \l 4219
  \l 421A
  \l 421B
  \l 421C
  \l 421D
  \l 421E
  \l 421F
  \l 4220
  \l 4221
  \l 4222
  \l 4223
  \l 4224
  \l 4225
  \l 4226
  \l 4227
  \l 4228
  \l 4229
  \l 422A
  \l 422B
  \l 422C
  \l 422D
  \l 422E
  \l 422F
  \l 4230
  \l 4231
  \l 4232
  \l 4233
  \l 4234
  \l 4235
  \l 4236
  \l 4237
  \l 4238
  \l 4239
  \l 423A
  \l 423B
  \l 423C
  \l 423D
  \l 423E
  \l 423F
  \l 4240
  \l 4241
  \l 4242
  \l 4243
  \l 4244
  \l 4245
  \l 4246
  \l 4247
  \l 4248
  \l 4249
  \l 424A
  \l 424B
  \l 424C
  \l 424D
  \l 424E
  \l 424F
  \l 4250
  \l 4251
  \l 4252
  \l 4253
  \l 4254
  \l 4255
  \l 4256
  \l 4257
  \l 4258
  \l 4259
  \l 425A
  \l 425B
  \l 425C
  \l 425D
  \l 425E
  \l 425F
  \l 4260
  \l 4261
  \l 4262
  \l 4263
  \l 4264
  \l 4265
  \l 4266
  \l 4267
  \l 4268
  \l 4269
  \l 426A
  \l 426B
  \l 426C
  \l 426D
  \l 426E
  \l 426F
  \l 4270
  \l 4271
  \l 4272
  \l 4273
  \l 4274
  \l 4275
  \l 4276
  \l 4277
  \l 4278
  \l 4279
  \l 427A
  \l 427B
  \l 427C
  \l 427D
  \l 427E
  \l 427F
  \l 4280
  \l 4281
  \l 4282
  \l 4283
  \l 4284
  \l 4285
  \l 4286
  \l 4287
  \l 4288
  \l 4289
  \l 428A
  \l 428B
  \l 428C
  \l 428D
  \l 428E
  \l 428F
  \l 4290
  \l 4291
  \l 4292
  \l 4293
  \l 4294
  \l 4295
  \l 4296
  \l 4297
  \l 4298
  \l 4299
  \l 429A
  \l 429B
  \l 429C
  \l 429D
  \l 429E
  \l 429F
  \l 42A0
  \l 42A1
  \l 42A2
  \l 42A3
  \l 42A4
  \l 42A5
  \l 42A6
  \l 42A7
  \l 42A8
  \l 42A9
  \l 42AA
  \l 42AB
  \l 42AC
  \l 42AD
  \l 42AE
  \l 42AF
  \l 42B0
  \l 42B1
  \l 42B2
  \l 42B3
  \l 42B4
  \l 42B5
  \l 42B6
  \l 42B7
  \l 42B8
  \l 42B9
  \l 42BA
  \l 42BB
  \l 42BC
  \l 42BD
  \l 42BE
  \l 42BF
  \l 42C0
  \l 42C1
  \l 42C2
  \l 42C3
  \l 42C4
  \l 42C5
  \l 42C6
  \l 42C7
  \l 42C8
  \l 42C9
  \l 42CA
  \l 42CB
  \l 42CC
  \l 42CD
  \l 42CE
  \l 42CF
  \l 42D0
  \l 42D1
  \l 42D2
  \l 42D3
  \l 42D4
  \l 42D5
  \l 42D6
  \l 42D7
  \l 42D8
  \l 42D9
  \l 42DA
  \l 42DB
  \l 42DC
  \l 42DD
  \l 42DE
  \l 42DF
  \l 42E0
  \l 42E1
  \l 42E2
  \l 42E3
  \l 42E4
  \l 42E5
  \l 42E6
  \l 42E7
  \l 42E8
  \l 42E9
  \l 42EA
  \l 42EB
  \l 42EC
  \l 42ED
  \l 42EE
  \l 42EF
  \l 42F0
  \l 42F1
  \l 42F2
  \l 42F3
  \l 42F4
  \l 42F5
  \l 42F6
  \l 42F7
  \l 42F8
  \l 42F9
  \l 42FA
  \l 42FB
  \l 42FC
  \l 42FD
  \l 42FE
  \l 42FF
  \l 4300
  \l 4301
  \l 4302
  \l 4303
  \l 4304
  \l 4305
  \l 4306
  \l 4307
  \l 4308
  \l 4309
  \l 430A
  \l 430B
  \l 430C
  \l 430D
  \l 430E
  \l 430F
  \l 4310
  \l 4311
  \l 4312
  \l 4313
  \l 4314
  \l 4315
  \l 4316
  \l 4317
  \l 4318
  \l 4319
  \l 431A
  \l 431B
  \l 431C
  \l 431D
  \l 431E
  \l 431F
  \l 4320
  \l 4321
  \l 4322
  \l 4323
  \l 4324
  \l 4325
  \l 4326
  \l 4327
  \l 4328
  \l 4329
  \l 432A
  \l 432B
  \l 432C
  \l 432D
  \l 432E
  \l 432F
  \l 4330
  \l 4331
  \l 4332
  \l 4333
  \l 4334
  \l 4335
  \l 4336
  \l 4337
  \l 4338
  \l 4339
  \l 433A
  \l 433B
  \l 433C
  \l 433D
  \l 433E
  \l 433F
  \l 4340
  \l 4341
  \l 4342
  \l 4343
  \l 4344
  \l 4345
  \l 4346
  \l 4347
  \l 4348
  \l 4349
  \l 434A
  \l 434B
  \l 434C
  \l 434D
  \l 434E
  \l 434F
  \l 4350
  \l 4351
  \l 4352
  \l 4353
  \l 4354
  \l 4355
  \l 4356
  \l 4357
  \l 4358
  \l 4359
  \l 435A
  \l 435B
  \l 435C
  \l 435D
  \l 435E
  \l 435F
  \l 4360
  \l 4361
  \l 4362
  \l 4363
  \l 4364
  \l 4365
  \l 4366
  \l 4367
  \l 4368
  \l 4369
  \l 436A
  \l 436B
  \l 436C
  \l 436D
  \l 436E
  \l 436F
  \l 4370
  \l 4371
  \l 4372
  \l 4373
  \l 4374
  \l 4375
  \l 4376
  \l 4377
  \l 4378
  \l 4379
  \l 437A
  \l 437B
  \l 437C
  \l 437D
  \l 437E
  \l 437F
  \l 4380
  \l 4381
  \l 4382
  \l 4383
  \l 4384
  \l 4385
  \l 4386
  \l 4387
  \l 4388
  \l 4389
  \l 438A
  \l 438B
  \l 438C
  \l 438D
  \l 438E
  \l 438F
  \l 4390
  \l 4391
  \l 4392
  \l 4393
  \l 4394
  \l 4395
  \l 4396
  \l 4397
  \l 4398
  \l 4399
  \l 439A
  \l 439B
  \l 439C
  \l 439D
  \l 439E
  \l 439F
  \l 43A0
  \l 43A1
  \l 43A2
  \l 43A3
  \l 43A4
  \l 43A5
  \l 43A6
  \l 43A7
  \l 43A8
  \l 43A9
  \l 43AA
  \l 43AB
  \l 43AC
  \l 43AD
  \l 43AE
  \l 43AF
  \l 43B0
  \l 43B1
  \l 43B2
  \l 43B3
  \l 43B4
  \l 43B5
  \l 43B6
  \l 43B7
  \l 43B8
  \l 43B9
  \l 43BA
  \l 43BB
  \l 43BC
  \l 43BD
  \l 43BE
  \l 43BF
  \l 43C0
  \l 43C1
  \l 43C2
  \l 43C3
  \l 43C4
  \l 43C5
  \l 43C6
  \l 43C7
  \l 43C8
  \l 43C9
  \l 43CA
  \l 43CB
  \l 43CC
  \l 43CD
  \l 43CE
  \l 43CF
  \l 43D0
  \l 43D1
  \l 43D2
  \l 43D3
  \l 43D4
  \l 43D5
  \l 43D6
  \l 43D7
  \l 43D8
  \l 43D9
  \l 43DA
  \l 43DB
  \l 43DC
  \l 43DD
  \l 43DE
  \l 43DF
  \l 43E0
  \l 43E1
  \l 43E2
  \l 43E3
  \l 43E4
  \l 43E5
  \l 43E6
  \l 43E7
  \l 43E8
  \l 43E9
  \l 43EA
  \l 43EB
  \l 43EC
  \l 43ED
  \l 43EE
  \l 43EF
  \l 43F0
  \l 43F1
  \l 43F2
  \l 43F3
  \l 43F4
  \l 43F5
  \l 43F6
  \l 43F7
  \l 43F8
  \l 43F9
  \l 43FA
  \l 43FB
  \l 43FC
  \l 43FD
  \l 43FE
  \l 43FF
  \l 4400
  \l 4401
  \l 4402
  \l 4403
  \l 4404
  \l 4405
  \l 4406
  \l 4407
  \l 4408
  \l 4409
  \l 440A
  \l 440B
  \l 440C
  \l 440D
  \l 440E
  \l 440F
  \l 4410
  \l 4411
  \l 4412
  \l 4413
  \l 4414
  \l 4415
  \l 4416
  \l 4417
  \l 4418
  \l 4419
  \l 441A
  \l 441B
  \l 441C
  \l 441D
  \l 441E
  \l 441F
  \l 4420
  \l 4421
  \l 4422
  \l 4423
  \l 4424
  \l 4425
  \l 4426
  \l 4427
  \l 4428
  \l 4429
  \l 442A
  \l 442B
  \l 442C
  \l 442D
  \l 442E
  \l 442F
  \l 4430
  \l 4431
  \l 4432
  \l 4433
  \l 4434
  \l 4435
  \l 4436
  \l 4437
  \l 4438
  \l 4439
  \l 443A
  \l 443B
  \l 443C
  \l 443D
  \l 443E
  \l 443F
  \l 4440
  \l 4441
  \l 4442
  \l 4443
  \l 4444
  \l 4445
  \l 4446
  \l 4447
  \l 4448
  \l 4449
  \l 444A
  \l 444B
  \l 444C
  \l 444D
  \l 444E
  \l 444F
  \l 4450
  \l 4451
  \l 4452
  \l 4453
  \l 4454
  \l 4455
  \l 4456
  \l 4457
  \l 4458
  \l 4459
  \l 445A
  \l 445B
  \l 445C
  \l 445D
  \l 445E
  \l 445F
  \l 4460
  \l 4461
  \l 4462
  \l 4463
  \l 4464
  \l 4465
  \l 4466
  \l 4467
  \l 4468
  \l 4469
  \l 446A
  \l 446B
  \l 446C
  \l 446D
  \l 446E
  \l 446F
  \l 4470
  \l 4471
  \l 4472
  \l 4473
  \l 4474
  \l 4475
  \l 4476
  \l 4477
  \l 4478
  \l 4479
  \l 447A
  \l 447B
  \l 447C
  \l 447D
  \l 447E
  \l 447F
  \l 4480
  \l 4481
  \l 4482
  \l 4483
  \l 4484
  \l 4485
  \l 4486
  \l 4487
  \l 4488
  \l 4489
  \l 448A
  \l 448B
  \l 448C
  \l 448D
  \l 448E
  \l 448F
  \l 4490
  \l 4491
  \l 4492
  \l 4493
  \l 4494
  \l 4495
  \l 4496
  \l 4497
  \l 4498
  \l 4499
  \l 449A
  \l 449B
  \l 449C
  \l 449D
  \l 449E
  \l 449F
  \l 44A0
  \l 44A1
  \l 44A2
  \l 44A3
  \l 44A4
  \l 44A5
  \l 44A6
  \l 44A7
  \l 44A8
  \l 44A9
  \l 44AA
  \l 44AB
  \l 44AC
  \l 44AD
  \l 44AE
  \l 44AF
  \l 44B0
  \l 44B1
  \l 44B2
  \l 44B3
  \l 44B4
  \l 44B5
  \l 44B6
  \l 44B7
  \l 44B8
  \l 44B9
  \l 44BA
  \l 44BB
  \l 44BC
  \l 44BD
  \l 44BE
  \l 44BF
  \l 44C0
  \l 44C1
  \l 44C2
  \l 44C3
  \l 44C4
  \l 44C5
  \l 44C6
  \l 44C7
  \l 44C8
  \l 44C9
  \l 44CA
  \l 44CB
  \l 44CC
  \l 44CD
  \l 44CE
  \l 44CF
  \l 44D0
  \l 44D1
  \l 44D2
  \l 44D3
  \l 44D4
  \l 44D5
  \l 44D6
  \l 44D7
  \l 44D8
  \l 44D9
  \l 44DA
  \l 44DB
  \l 44DC
  \l 44DD
  \l 44DE
  \l 44DF
  \l 44E0
  \l 44E1
  \l 44E2
  \l 44E3
  \l 44E4
  \l 44E5
  \l 44E6
  \l 44E7
  \l 44E8
  \l 44E9
  \l 44EA
  \l 44EB
  \l 44EC
  \l 44ED
  \l 44EE
  \l 44EF
  \l 44F0
  \l 44F1
  \l 44F2
  \l 44F3
  \l 44F4
  \l 44F5
  \l 44F6
  \l 44F7
  \l 44F8
  \l 44F9
  \l 44FA
  \l 44FB
  \l 44FC
  \l 44FD
  \l 44FE
  \l 44FF
  \l 4500
  \l 4501
  \l 4502
  \l 4503
  \l 4504
  \l 4505
  \l 4506
  \l 4507
  \l 4508
  \l 4509
  \l 450A
  \l 450B
  \l 450C
  \l 450D
  \l 450E
  \l 450F
  \l 4510
  \l 4511
  \l 4512
  \l 4513
  \l 4514
  \l 4515
  \l 4516
  \l 4517
  \l 4518
  \l 4519
  \l 451A
  \l 451B
  \l 451C
  \l 451D
  \l 451E
  \l 451F
  \l 4520
  \l 4521
  \l 4522
  \l 4523
  \l 4524
  \l 4525
  \l 4526
  \l 4527
  \l 4528
  \l 4529
  \l 452A
  \l 452B
  \l 452C
  \l 452D
  \l 452E
  \l 452F
  \l 4530
  \l 4531
  \l 4532
  \l 4533
  \l 4534
  \l 4535
  \l 4536
  \l 4537
  \l 4538
  \l 4539
  \l 453A
  \l 453B
  \l 453C
  \l 453D
  \l 453E
  \l 453F
  \l 4540
  \l 4541
  \l 4542
  \l 4543
  \l 4544
  \l 4545
  \l 4546
  \l 4547
  \l 4548
  \l 4549
  \l 454A
  \l 454B
  \l 454C
  \l 454D
  \l 454E
  \l 454F
  \l 4550
  \l 4551
  \l 4552
  \l 4553
  \l 4554
  \l 4555
  \l 4556
  \l 4557
  \l 4558
  \l 4559
  \l 455A
  \l 455B
  \l 455C
  \l 455D
  \l 455E
  \l 455F
  \l 4560
  \l 4561
  \l 4562
  \l 4563
  \l 4564
  \l 4565
  \l 4566
  \l 4567
  \l 4568
  \l 4569
  \l 456A
  \l 456B
  \l 456C
  \l 456D
  \l 456E
  \l 456F
  \l 4570
  \l 4571
  \l 4572
  \l 4573
  \l 4574
  \l 4575
  \l 4576
  \l 4577
  \l 4578
  \l 4579
  \l 457A
  \l 457B
  \l 457C
  \l 457D
  \l 457E
  \l 457F
  \l 4580
  \l 4581
  \l 4582
  \l 4583
  \l 4584
  \l 4585
  \l 4586
  \l 4587
  \l 4588
  \l 4589
  \l 458A
  \l 458B
  \l 458C
  \l 458D
  \l 458E
  \l 458F
  \l 4590
  \l 4591
  \l 4592
  \l 4593
  \l 4594
  \l 4595
  \l 4596
  \l 4597
  \l 4598
  \l 4599
  \l 459A
  \l 459B
  \l 459C
  \l 459D
  \l 459E
  \l 459F
  \l 45A0
  \l 45A1
  \l 45A2
  \l 45A3
  \l 45A4
  \l 45A5
  \l 45A6
  \l 45A7
  \l 45A8
  \l 45A9
  \l 45AA
  \l 45AB
  \l 45AC
  \l 45AD
  \l 45AE
  \l 45AF
  \l 45B0
  \l 45B1
  \l 45B2
  \l 45B3
  \l 45B4
  \l 45B5
  \l 45B6
  \l 45B7
  \l 45B8
  \l 45B9
  \l 45BA
  \l 45BB
  \l 45BC
  \l 45BD
  \l 45BE
  \l 45BF
  \l 45C0
  \l 45C1
  \l 45C2
  \l 45C3
  \l 45C4
  \l 45C5
  \l 45C6
  \l 45C7
  \l 45C8
  \l 45C9
  \l 45CA
  \l 45CB
  \l 45CC
  \l 45CD
  \l 45CE
  \l 45CF
  \l 45D0
  \l 45D1
  \l 45D2
  \l 45D3
  \l 45D4
  \l 45D5
  \l 45D6
  \l 45D7
  \l 45D8
  \l 45D9
  \l 45DA
  \l 45DB
  \l 45DC
  \l 45DD
  \l 45DE
  \l 45DF
  \l 45E0
  \l 45E1
  \l 45E2
  \l 45E3
  \l 45E4
  \l 45E5
  \l 45E6
  \l 45E7
  \l 45E8
  \l 45E9
  \l 45EA
  \l 45EB
  \l 45EC
  \l 45ED
  \l 45EE
  \l 45EF
  \l 45F0
  \l 45F1
  \l 45F2
  \l 45F3
  \l 45F4
  \l 45F5
  \l 45F6
  \l 45F7
  \l 45F8
  \l 45F9
  \l 45FA
  \l 45FB
  \l 45FC
  \l 45FD
  \l 45FE
  \l 45FF
  \l 4600
  \l 4601
  \l 4602
  \l 4603
  \l 4604
  \l 4605
  \l 4606
  \l 4607
  \l 4608
  \l 4609
  \l 460A
  \l 460B
  \l 460C
  \l 460D
  \l 460E
  \l 460F
  \l 4610
  \l 4611
  \l 4612
  \l 4613
  \l 4614
  \l 4615
  \l 4616
  \l 4617
  \l 4618
  \l 4619
  \l 461A
  \l 461B
  \l 461C
  \l 461D
  \l 461E
  \l 461F
  \l 4620
  \l 4621
  \l 4622
  \l 4623
  \l 4624
  \l 4625
  \l 4626
  \l 4627
  \l 4628
  \l 4629
  \l 462A
  \l 462B
  \l 462C
  \l 462D
  \l 462E
  \l 462F
  \l 4630
  \l 4631
  \l 4632
  \l 4633
  \l 4634
  \l 4635
  \l 4636
  \l 4637
  \l 4638
  \l 4639
  \l 463A
  \l 463B
  \l 463C
  \l 463D
  \l 463E
  \l 463F
  \l 4640
  \l 4641
  \l 4642
  \l 4643
  \l 4644
  \l 4645
  \l 4646
  \l 4647
  \l 4648
  \l 4649
  \l 464A
  \l 464B
  \l 464C
  \l 464D
  \l 464E
  \l 464F
  \l 4650
  \l 4651
  \l 4652
  \l 4653
  \l 4654
  \l 4655
  \l 4656
  \l 4657
  \l 4658
  \l 4659
  \l 465A
  \l 465B
  \l 465C
  \l 465D
  \l 465E
  \l 465F
  \l 4660
  \l 4661
  \l 4662
  \l 4663
  \l 4664
  \l 4665
  \l 4666
  \l 4667
  \l 4668
  \l 4669
  \l 466A
  \l 466B
  \l 466C
  \l 466D
  \l 466E
  \l 466F
  \l 4670
  \l 4671
  \l 4672
  \l 4673
  \l 4674
  \l 4675
  \l 4676
  \l 4677
  \l 4678
  \l 4679
  \l 467A
  \l 467B
  \l 467C
  \l 467D
  \l 467E
  \l 467F
  \l 4680
  \l 4681
  \l 4682
  \l 4683
  \l 4684
  \l 4685
  \l 4686
  \l 4687
  \l 4688
  \l 4689
  \l 468A
  \l 468B
  \l 468C
  \l 468D
  \l 468E
  \l 468F
  \l 4690
  \l 4691
  \l 4692
  \l 4693
  \l 4694
  \l 4695
  \l 4696
  \l 4697
  \l 4698
  \l 4699
  \l 469A
  \l 469B
  \l 469C
  \l 469D
  \l 469E
  \l 469F
  \l 46A0
  \l 46A1
  \l 46A2
  \l 46A3
  \l 46A4
  \l 46A5
  \l 46A6
  \l 46A7
  \l 46A8
  \l 46A9
  \l 46AA
  \l 46AB
  \l 46AC
  \l 46AD
  \l 46AE
  \l 46AF
  \l 46B0
  \l 46B1
  \l 46B2
  \l 46B3
  \l 46B4
  \l 46B5
  \l 46B6
  \l 46B7
  \l 46B8
  \l 46B9
  \l 46BA
  \l 46BB
  \l 46BC
  \l 46BD
  \l 46BE
  \l 46BF
  \l 46C0
  \l 46C1
  \l 46C2
  \l 46C3
  \l 46C4
  \l 46C5
  \l 46C6
  \l 46C7
  \l 46C8
  \l 46C9
  \l 46CA
  \l 46CB
  \l 46CC
  \l 46CD
  \l 46CE
  \l 46CF
  \l 46D0
  \l 46D1
  \l 46D2
  \l 46D3
  \l 46D4
  \l 46D5
  \l 46D6
  \l 46D7
  \l 46D8
  \l 46D9
  \l 46DA
  \l 46DB
  \l 46DC
  \l 46DD
  \l 46DE
  \l 46DF
  \l 46E0
  \l 46E1
  \l 46E2
  \l 46E3
  \l 46E4
  \l 46E5
  \l 46E6
  \l 46E7
  \l 46E8
  \l 46E9
  \l 46EA
  \l 46EB
  \l 46EC
  \l 46ED
  \l 46EE
  \l 46EF
  \l 46F0
  \l 46F1
  \l 46F2
  \l 46F3
  \l 46F4
  \l 46F5
  \l 46F6
  \l 46F7
  \l 46F8
  \l 46F9
  \l 46FA
  \l 46FB
  \l 46FC
  \l 46FD
  \l 46FE
  \l 46FF
  \l 4700
  \l 4701
  \l 4702
  \l 4703
  \l 4704
  \l 4705
  \l 4706
  \l 4707
  \l 4708
  \l 4709
  \l 470A
  \l 470B
  \l 470C
  \l 470D
  \l 470E
  \l 470F
  \l 4710
  \l 4711
  \l 4712
  \l 4713
  \l 4714
  \l 4715
  \l 4716
  \l 4717
  \l 4718
  \l 4719
  \l 471A
  \l 471B
  \l 471C
  \l 471D
  \l 471E
  \l 471F
  \l 4720
  \l 4721
  \l 4722
  \l 4723
  \l 4724
  \l 4725
  \l 4726
  \l 4727
  \l 4728
  \l 4729
  \l 472A
  \l 472B
  \l 472C
  \l 472D
  \l 472E
  \l 472F
  \l 4730
  \l 4731
  \l 4732
  \l 4733
  \l 4734
  \l 4735
  \l 4736
  \l 4737
  \l 4738
  \l 4739
  \l 473A
  \l 473B
  \l 473C
  \l 473D
  \l 473E
  \l 473F
  \l 4740
  \l 4741
  \l 4742
  \l 4743
  \l 4744
  \l 4745
  \l 4746
  \l 4747
  \l 4748
  \l 4749
  \l 474A
  \l 474B
  \l 474C
  \l 474D
  \l 474E
  \l 474F
  \l 4750
  \l 4751
  \l 4752
  \l 4753
  \l 4754
  \l 4755
  \l 4756
  \l 4757
  \l 4758
  \l 4759
  \l 475A
  \l 475B
  \l 475C
  \l 475D
  \l 475E
  \l 475F
  \l 4760
  \l 4761
  \l 4762
  \l 4763
  \l 4764
  \l 4765
  \l 4766
  \l 4767
  \l 4768
  \l 4769
  \l 476A
  \l 476B
  \l 476C
  \l 476D
  \l 476E
  \l 476F
  \l 4770
  \l 4771
  \l 4772
  \l 4773
  \l 4774
  \l 4775
  \l 4776
  \l 4777
  \l 4778
  \l 4779
  \l 477A
  \l 477B
  \l 477C
  \l 477D
  \l 477E
  \l 477F
  \l 4780
  \l 4781
  \l 4782
  \l 4783
  \l 4784
  \l 4785
  \l 4786
  \l 4787
  \l 4788
  \l 4789
  \l 478A
  \l 478B
  \l 478C
  \l 478D
  \l 478E
  \l 478F
  \l 4790
  \l 4791
  \l 4792
  \l 4793
  \l 4794
  \l 4795
  \l 4796
  \l 4797
  \l 4798
  \l 4799
  \l 479A
  \l 479B
  \l 479C
  \l 479D
  \l 479E
  \l 479F
  \l 47A0
  \l 47A1
  \l 47A2
  \l 47A3
  \l 47A4
  \l 47A5
  \l 47A6
  \l 47A7
  \l 47A8
  \l 47A9
  \l 47AA
  \l 47AB
  \l 47AC
  \l 47AD
  \l 47AE
  \l 47AF
  \l 47B0
  \l 47B1
  \l 47B2
  \l 47B3
  \l 47B4
  \l 47B5
  \l 47B6
  \l 47B7
  \l 47B8
  \l 47B9
  \l 47BA
  \l 47BB
  \l 47BC
  \l 47BD
  \l 47BE
  \l 47BF
  \l 47C0
  \l 47C1
  \l 47C2
  \l 47C3
  \l 47C4
  \l 47C5
  \l 47C6
  \l 47C7
  \l 47C8
  \l 47C9
  \l 47CA
  \l 47CB
  \l 47CC
  \l 47CD
  \l 47CE
  \l 47CF
  \l 47D0
  \l 47D1
  \l 47D2
  \l 47D3
  \l 47D4
  \l 47D5
  \l 47D6
  \l 47D7
  \l 47D8
  \l 47D9
  \l 47DA
  \l 47DB
  \l 47DC
  \l 47DD
  \l 47DE
  \l 47DF
  \l 47E0
  \l 47E1
  \l 47E2
  \l 47E3
  \l 47E4
  \l 47E5
  \l 47E6
  \l 47E7
  \l 47E8
  \l 47E9
  \l 47EA
  \l 47EB
  \l 47EC
  \l 47ED
  \l 47EE
  \l 47EF
  \l 47F0
  \l 47F1
  \l 47F2
  \l 47F3
  \l 47F4
  \l 47F5
  \l 47F6
  \l 47F7
  \l 47F8
  \l 47F9
  \l 47FA
  \l 47FB
  \l 47FC
  \l 47FD
  \l 47FE
  \l 47FF
  \l 4800
  \l 4801
  \l 4802
  \l 4803
  \l 4804
  \l 4805
  \l 4806
  \l 4807
  \l 4808
  \l 4809
  \l 480A
  \l 480B
  \l 480C
  \l 480D
  \l 480E
  \l 480F
  \l 4810
  \l 4811
  \l 4812
  \l 4813
  \l 4814
  \l 4815
  \l 4816
  \l 4817
  \l 4818
  \l 4819
  \l 481A
  \l 481B
  \l 481C
  \l 481D
  \l 481E
  \l 481F
  \l 4820
  \l 4821
  \l 4822
  \l 4823
  \l 4824
  \l 4825
  \l 4826
  \l 4827
  \l 4828
  \l 4829
  \l 482A
  \l 482B
  \l 482C
  \l 482D
  \l 482E
  \l 482F
  \l 4830
  \l 4831
  \l 4832
  \l 4833
  \l 4834
  \l 4835
  \l 4836
  \l 4837
  \l 4838
  \l 4839
  \l 483A
  \l 483B
  \l 483C
  \l 483D
  \l 483E
  \l 483F
  \l 4840
  \l 4841
  \l 4842
  \l 4843
  \l 4844
  \l 4845
  \l 4846
  \l 4847
  \l 4848
  \l 4849
  \l 484A
  \l 484B
  \l 484C
  \l 484D
  \l 484E
  \l 484F
  \l 4850
  \l 4851
  \l 4852
  \l 4853
  \l 4854
  \l 4855
  \l 4856
  \l 4857
  \l 4858
  \l 4859
  \l 485A
  \l 485B
  \l 485C
  \l 485D
  \l 485E
  \l 485F
  \l 4860
  \l 4861
  \l 4862
  \l 4863
  \l 4864
  \l 4865
  \l 4866
  \l 4867
  \l 4868
  \l 4869
  \l 486A
  \l 486B
  \l 486C
  \l 486D
  \l 486E
  \l 486F
  \l 4870
  \l 4871
  \l 4872
  \l 4873
  \l 4874
  \l 4875
  \l 4876
  \l 4877
  \l 4878
  \l 4879
  \l 487A
  \l 487B
  \l 487C
  \l 487D
  \l 487E
  \l 487F
  \l 4880
  \l 4881
  \l 4882
  \l 4883
  \l 4884
  \l 4885
  \l 4886
  \l 4887
  \l 4888
  \l 4889
  \l 488A
  \l 488B
  \l 488C
  \l 488D
  \l 488E
  \l 488F
  \l 4890
  \l 4891
  \l 4892
  \l 4893
  \l 4894
  \l 4895
  \l 4896
  \l 4897
  \l 4898
  \l 4899
  \l 489A
  \l 489B
  \l 489C
  \l 489D
  \l 489E
  \l 489F
  \l 48A0
  \l 48A1
  \l 48A2
  \l 48A3
  \l 48A4
  \l 48A5
  \l 48A6
  \l 48A7
  \l 48A8
  \l 48A9
  \l 48AA
  \l 48AB
  \l 48AC
  \l 48AD
  \l 48AE
  \l 48AF
  \l 48B0
  \l 48B1
  \l 48B2
  \l 48B3
  \l 48B4
  \l 48B5
  \l 48B6
  \l 48B7
  \l 48B8
  \l 48B9
  \l 48BA
  \l 48BB
  \l 48BC
  \l 48BD
  \l 48BE
  \l 48BF
  \l 48C0
  \l 48C1
  \l 48C2
  \l 48C3
  \l 48C4
  \l 48C5
  \l 48C6
  \l 48C7
  \l 48C8
  \l 48C9
  \l 48CA
  \l 48CB
  \l 48CC
  \l 48CD
  \l 48CE
  \l 48CF
  \l 48D0
  \l 48D1
  \l 48D2
  \l 48D3
  \l 48D4
  \l 48D5
  \l 48D6
  \l 48D7
  \l 48D8
  \l 48D9
  \l 48DA
  \l 48DB
  \l 48DC
  \l 48DD
  \l 48DE
  \l 48DF
  \l 48E0
  \l 48E1
  \l 48E2
  \l 48E3
  \l 48E4
  \l 48E5
  \l 48E6
  \l 48E7
  \l 48E8
  \l 48E9
  \l 48EA
  \l 48EB
  \l 48EC
  \l 48ED
  \l 48EE
  \l 48EF
  \l 48F0
  \l 48F1
  \l 48F2
  \l 48F3
  \l 48F4
  \l 48F5
  \l 48F6
  \l 48F7
  \l 48F8
  \l 48F9
  \l 48FA
  \l 48FB
  \l 48FC
  \l 48FD
  \l 48FE
  \l 48FF
  \l 4900
  \l 4901
  \l 4902
  \l 4903
  \l 4904
  \l 4905
  \l 4906
  \l 4907
  \l 4908
  \l 4909
  \l 490A
  \l 490B
  \l 490C
  \l 490D
  \l 490E
  \l 490F
  \l 4910
  \l 4911
  \l 4912
  \l 4913
  \l 4914
  \l 4915
  \l 4916
  \l 4917
  \l 4918
  \l 4919
  \l 491A
  \l 491B
  \l 491C
  \l 491D
  \l 491E
  \l 491F
  \l 4920
  \l 4921
  \l 4922
  \l 4923
  \l 4924
  \l 4925
  \l 4926
  \l 4927
  \l 4928
  \l 4929
  \l 492A
  \l 492B
  \l 492C
  \l 492D
  \l 492E
  \l 492F
  \l 4930
  \l 4931
  \l 4932
  \l 4933
  \l 4934
  \l 4935
  \l 4936
  \l 4937
  \l 4938
  \l 4939
  \l 493A
  \l 493B
  \l 493C
  \l 493D
  \l 493E
  \l 493F
  \l 4940
  \l 4941
  \l 4942
  \l 4943
  \l 4944
  \l 4945
  \l 4946
  \l 4947
  \l 4948
  \l 4949
  \l 494A
  \l 494B
  \l 494C
  \l 494D
  \l 494E
  \l 494F
  \l 4950
  \l 4951
  \l 4952
  \l 4953
  \l 4954
  \l 4955
  \l 4956
  \l 4957
  \l 4958
  \l 4959
  \l 495A
  \l 495B
  \l 495C
  \l 495D
  \l 495E
  \l 495F
  \l 4960
  \l 4961
  \l 4962
  \l 4963
  \l 4964
  \l 4965
  \l 4966
  \l 4967
  \l 4968
  \l 4969
  \l 496A
  \l 496B
  \l 496C
  \l 496D
  \l 496E
  \l 496F
  \l 4970
  \l 4971
  \l 4972
  \l 4973
  \l 4974
  \l 4975
  \l 4976
  \l 4977
  \l 4978
  \l 4979
  \l 497A
  \l 497B
  \l 497C
  \l 497D
  \l 497E
  \l 497F
  \l 4980
  \l 4981
  \l 4982
  \l 4983
  \l 4984
  \l 4985
  \l 4986
  \l 4987
  \l 4988
  \l 4989
  \l 498A
  \l 498B
  \l 498C
  \l 498D
  \l 498E
  \l 498F
  \l 4990
  \l 4991
  \l 4992
  \l 4993
  \l 4994
  \l 4995
  \l 4996
  \l 4997
  \l 4998
  \l 4999
  \l 499A
  \l 499B
  \l 499C
  \l 499D
  \l 499E
  \l 499F
  \l 49A0
  \l 49A1
  \l 49A2
  \l 49A3
  \l 49A4
  \l 49A5
  \l 49A6
  \l 49A7
  \l 49A8
  \l 49A9
  \l 49AA
  \l 49AB
  \l 49AC
  \l 49AD
  \l 49AE
  \l 49AF
  \l 49B0
  \l 49B1
  \l 49B2
  \l 49B3
  \l 49B4
  \l 49B5
  \l 49B6
  \l 49B7
  \l 49B8
  \l 49B9
  \l 49BA
  \l 49BB
  \l 49BC
  \l 49BD
  \l 49BE
  \l 49BF
  \l 49C0
  \l 49C1
  \l 49C2
  \l 49C3
  \l 49C4
  \l 49C5
  \l 49C6
  \l 49C7
  \l 49C8
  \l 49C9
  \l 49CA
  \l 49CB
  \l 49CC
  \l 49CD
  \l 49CE
  \l 49CF
  \l 49D0
  \l 49D1
  \l 49D2
  \l 49D3
  \l 49D4
  \l 49D5
  \l 49D6
  \l 49D7
  \l 49D8
  \l 49D9
  \l 49DA
  \l 49DB
  \l 49DC
  \l 49DD
  \l 49DE
  \l 49DF
  \l 49E0
  \l 49E1
  \l 49E2
  \l 49E3
  \l 49E4
  \l 49E5
  \l 49E6
  \l 49E7
  \l 49E8
  \l 49E9
  \l 49EA
  \l 49EB
  \l 49EC
  \l 49ED
  \l 49EE
  \l 49EF
  \l 49F0
  \l 49F1
  \l 49F2
  \l 49F3
  \l 49F4
  \l 49F5
  \l 49F6
  \l 49F7
  \l 49F8
  \l 49F9
  \l 49FA
  \l 49FB
  \l 49FC
  \l 49FD
  \l 49FE
  \l 49FF
  \l 4A00
  \l 4A01
  \l 4A02
  \l 4A03
  \l 4A04
  \l 4A05
  \l 4A06
  \l 4A07
  \l 4A08
  \l 4A09
  \l 4A0A
  \l 4A0B
  \l 4A0C
  \l 4A0D
  \l 4A0E
  \l 4A0F
  \l 4A10
  \l 4A11
  \l 4A12
  \l 4A13
  \l 4A14
  \l 4A15
  \l 4A16
  \l 4A17
  \l 4A18
  \l 4A19
  \l 4A1A
  \l 4A1B
  \l 4A1C
  \l 4A1D
  \l 4A1E
  \l 4A1F
  \l 4A20
  \l 4A21
  \l 4A22
  \l 4A23
  \l 4A24
  \l 4A25
  \l 4A26
  \l 4A27
  \l 4A28
  \l 4A29
  \l 4A2A
  \l 4A2B
  \l 4A2C
  \l 4A2D
  \l 4A2E
  \l 4A2F
  \l 4A30
  \l 4A31
  \l 4A32
  \l 4A33
  \l 4A34
  \l 4A35
  \l 4A36
  \l 4A37
  \l 4A38
  \l 4A39
  \l 4A3A
  \l 4A3B
  \l 4A3C
  \l 4A3D
  \l 4A3E
  \l 4A3F
  \l 4A40
  \l 4A41
  \l 4A42
  \l 4A43
  \l 4A44
  \l 4A45
  \l 4A46
  \l 4A47
  \l 4A48
  \l 4A49
  \l 4A4A
  \l 4A4B
  \l 4A4C
  \l 4A4D
  \l 4A4E
  \l 4A4F
  \l 4A50
  \l 4A51
  \l 4A52
  \l 4A53
  \l 4A54
  \l 4A55
  \l 4A56
  \l 4A57
  \l 4A58
  \l 4A59
  \l 4A5A
  \l 4A5B
  \l 4A5C
  \l 4A5D
  \l 4A5E
  \l 4A5F
  \l 4A60
  \l 4A61
  \l 4A62
  \l 4A63
  \l 4A64
  \l 4A65
  \l 4A66
  \l 4A67
  \l 4A68
  \l 4A69
  \l 4A6A
  \l 4A6B
  \l 4A6C
  \l 4A6D
  \l 4A6E
  \l 4A6F
  \l 4A70
  \l 4A71
  \l 4A72
  \l 4A73
  \l 4A74
  \l 4A75
  \l 4A76
  \l 4A77
  \l 4A78
  \l 4A79
  \l 4A7A
  \l 4A7B
  \l 4A7C
  \l 4A7D
  \l 4A7E
  \l 4A7F
  \l 4A80
  \l 4A81
  \l 4A82
  \l 4A83
  \l 4A84
  \l 4A85
  \l 4A86
  \l 4A87
  \l 4A88
  \l 4A89
  \l 4A8A
  \l 4A8B
  \l 4A8C
  \l 4A8D
  \l 4A8E
  \l 4A8F
  \l 4A90
  \l 4A91
  \l 4A92
  \l 4A93
  \l 4A94
  \l 4A95
  \l 4A96
  \l 4A97
  \l 4A98
  \l 4A99
  \l 4A9A
  \l 4A9B
  \l 4A9C
  \l 4A9D
  \l 4A9E
  \l 4A9F
  \l 4AA0
  \l 4AA1
  \l 4AA2
  \l 4AA3
  \l 4AA4
  \l 4AA5
  \l 4AA6
  \l 4AA7
  \l 4AA8
  \l 4AA9
  \l 4AAA
  \l 4AAB
  \l 4AAC
  \l 4AAD
  \l 4AAE
  \l 4AAF
  \l 4AB0
  \l 4AB1
  \l 4AB2
  \l 4AB3
  \l 4AB4
  \l 4AB5
  \l 4AB6
  \l 4AB7
  \l 4AB8
  \l 4AB9
  \l 4ABA
  \l 4ABB
  \l 4ABC
  \l 4ABD
  \l 4ABE
  \l 4ABF
  \l 4AC0
  \l 4AC1
  \l 4AC2
  \l 4AC3
  \l 4AC4
  \l 4AC5
  \l 4AC6
  \l 4AC7
  \l 4AC8
  \l 4AC9
  \l 4ACA
  \l 4ACB
  \l 4ACC
  \l 4ACD
  \l 4ACE
  \l 4ACF
  \l 4AD0
  \l 4AD1
  \l 4AD2
  \l 4AD3
  \l 4AD4
  \l 4AD5
  \l 4AD6
  \l 4AD7
  \l 4AD8
  \l 4AD9
  \l 4ADA
  \l 4ADB
  \l 4ADC
  \l 4ADD
  \l 4ADE
  \l 4ADF
  \l 4AE0
  \l 4AE1
  \l 4AE2
  \l 4AE3
  \l 4AE4
  \l 4AE5
  \l 4AE6
  \l 4AE7
  \l 4AE8
  \l 4AE9
  \l 4AEA
  \l 4AEB
  \l 4AEC
  \l 4AED
  \l 4AEE
  \l 4AEF
  \l 4AF0
  \l 4AF1
  \l 4AF2
  \l 4AF3
  \l 4AF4
  \l 4AF5
  \l 4AF6
  \l 4AF7
  \l 4AF8
  \l 4AF9
  \l 4AFA
  \l 4AFB
  \l 4AFC
  \l 4AFD
  \l 4AFE
  \l 4AFF
  \l 4B00
  \l 4B01
  \l 4B02
  \l 4B03
  \l 4B04
  \l 4B05
  \l 4B06
  \l 4B07
  \l 4B08
  \l 4B09
  \l 4B0A
  \l 4B0B
  \l 4B0C
  \l 4B0D
  \l 4B0E
  \l 4B0F
  \l 4B10
  \l 4B11
  \l 4B12
  \l 4B13
  \l 4B14
  \l 4B15
  \l 4B16
  \l 4B17
  \l 4B18
  \l 4B19
  \l 4B1A
  \l 4B1B
  \l 4B1C
  \l 4B1D
  \l 4B1E
  \l 4B1F
  \l 4B20
  \l 4B21
  \l 4B22
  \l 4B23
  \l 4B24
  \l 4B25
  \l 4B26
  \l 4B27
  \l 4B28
  \l 4B29
  \l 4B2A
  \l 4B2B
  \l 4B2C
  \l 4B2D
  \l 4B2E
  \l 4B2F
  \l 4B30
  \l 4B31
  \l 4B32
  \l 4B33
  \l 4B34
  \l 4B35
  \l 4B36
  \l 4B37
  \l 4B38
  \l 4B39
  \l 4B3A
  \l 4B3B
  \l 4B3C
  \l 4B3D
  \l 4B3E
  \l 4B3F
  \l 4B40
  \l 4B41
  \l 4B42
  \l 4B43
  \l 4B44
  \l 4B45
  \l 4B46
  \l 4B47
  \l 4B48
  \l 4B49
  \l 4B4A
  \l 4B4B
  \l 4B4C
  \l 4B4D
  \l 4B4E
  \l 4B4F
  \l 4B50
  \l 4B51
  \l 4B52
  \l 4B53
  \l 4B54
  \l 4B55
  \l 4B56
  \l 4B57
  \l 4B58
  \l 4B59
  \l 4B5A
  \l 4B5B
  \l 4B5C
  \l 4B5D
  \l 4B5E
  \l 4B5F
  \l 4B60
  \l 4B61
  \l 4B62
  \l 4B63
  \l 4B64
  \l 4B65
  \l 4B66
  \l 4B67
  \l 4B68
  \l 4B69
  \l 4B6A
  \l 4B6B
  \l 4B6C
  \l 4B6D
  \l 4B6E
  \l 4B6F
  \l 4B70
  \l 4B71
  \l 4B72
  \l 4B73
  \l 4B74
  \l 4B75
  \l 4B76
  \l 4B77
  \l 4B78
  \l 4B79
  \l 4B7A
  \l 4B7B
  \l 4B7C
  \l 4B7D
  \l 4B7E
  \l 4B7F
  \l 4B80
  \l 4B81
  \l 4B82
  \l 4B83
  \l 4B84
  \l 4B85
  \l 4B86
  \l 4B87
  \l 4B88
  \l 4B89
  \l 4B8A
  \l 4B8B
  \l 4B8C
  \l 4B8D
  \l 4B8E
  \l 4B8F
  \l 4B90
  \l 4B91
  \l 4B92
  \l 4B93
  \l 4B94
  \l 4B95
  \l 4B96
  \l 4B97
  \l 4B98
  \l 4B99
  \l 4B9A
  \l 4B9B
  \l 4B9C
  \l 4B9D
  \l 4B9E
  \l 4B9F
  \l 4BA0
  \l 4BA1
  \l 4BA2
  \l 4BA3
  \l 4BA4
  \l 4BA5
  \l 4BA6
  \l 4BA7
  \l 4BA8
  \l 4BA9
  \l 4BAA
  \l 4BAB
  \l 4BAC
  \l 4BAD
  \l 4BAE
  \l 4BAF
  \l 4BB0
  \l 4BB1
  \l 4BB2
  \l 4BB3
  \l 4BB4
  \l 4BB5
  \l 4BB6
  \l 4BB7
  \l 4BB8
  \l 4BB9
  \l 4BBA
  \l 4BBB
  \l 4BBC
  \l 4BBD
  \l 4BBE
  \l 4BBF
  \l 4BC0
  \l 4BC1
  \l 4BC2
  \l 4BC3
  \l 4BC4
  \l 4BC5
  \l 4BC6
  \l 4BC7
  \l 4BC8
  \l 4BC9
  \l 4BCA
  \l 4BCB
  \l 4BCC
  \l 4BCD
  \l 4BCE
  \l 4BCF
  \l 4BD0
  \l 4BD1
  \l 4BD2
  \l 4BD3
  \l 4BD4
  \l 4BD5
  \l 4BD6
  \l 4BD7
  \l 4BD8
  \l 4BD9
  \l 4BDA
  \l 4BDB
  \l 4BDC
  \l 4BDD
  \l 4BDE
  \l 4BDF
  \l 4BE0
  \l 4BE1
  \l 4BE2
  \l 4BE3
  \l 4BE4
  \l 4BE5
  \l 4BE6
  \l 4BE7
  \l 4BE8
  \l 4BE9
  \l 4BEA
  \l 4BEB
  \l 4BEC
  \l 4BED
  \l 4BEE
  \l 4BEF
  \l 4BF0
  \l 4BF1
  \l 4BF2
  \l 4BF3
  \l 4BF4
  \l 4BF5
  \l 4BF6
  \l 4BF7
  \l 4BF8
  \l 4BF9
  \l 4BFA
  \l 4BFB
  \l 4BFC
  \l 4BFD
  \l 4BFE
  \l 4BFF
  \l 4C00
  \l 4C01
  \l 4C02
  \l 4C03
  \l 4C04
  \l 4C05
  \l 4C06
  \l 4C07
  \l 4C08
  \l 4C09
  \l 4C0A
  \l 4C0B
  \l 4C0C
  \l 4C0D
  \l 4C0E
  \l 4C0F
  \l 4C10
  \l 4C11
  \l 4C12
  \l 4C13
  \l 4C14
  \l 4C15
  \l 4C16
  \l 4C17
  \l 4C18
  \l 4C19
  \l 4C1A
  \l 4C1B
  \l 4C1C
  \l 4C1D
  \l 4C1E
  \l 4C1F
  \l 4C20
  \l 4C21
  \l 4C22
  \l 4C23
  \l 4C24
  \l 4C25
  \l 4C26
  \l 4C27
  \l 4C28
  \l 4C29
  \l 4C2A
  \l 4C2B
  \l 4C2C
  \l 4C2D
  \l 4C2E
  \l 4C2F
  \l 4C30
  \l 4C31
  \l 4C32
  \l 4C33
  \l 4C34
  \l 4C35
  \l 4C36
  \l 4C37
  \l 4C38
  \l 4C39
  \l 4C3A
  \l 4C3B
  \l 4C3C
  \l 4C3D
  \l 4C3E
  \l 4C3F
  \l 4C40
  \l 4C41
  \l 4C42
  \l 4C43
  \l 4C44
  \l 4C45
  \l 4C46
  \l 4C47
  \l 4C48
  \l 4C49
  \l 4C4A
  \l 4C4B
  \l 4C4C
  \l 4C4D
  \l 4C4E
  \l 4C4F
  \l 4C50
  \l 4C51
  \l 4C52
  \l 4C53
  \l 4C54
  \l 4C55
  \l 4C56
  \l 4C57
  \l 4C58
  \l 4C59
  \l 4C5A
  \l 4C5B
  \l 4C5C
  \l 4C5D
  \l 4C5E
  \l 4C5F
  \l 4C60
  \l 4C61
  \l 4C62
  \l 4C63
  \l 4C64
  \l 4C65
  \l 4C66
  \l 4C67
  \l 4C68
  \l 4C69
  \l 4C6A
  \l 4C6B
  \l 4C6C
  \l 4C6D
  \l 4C6E
  \l 4C6F
  \l 4C70
  \l 4C71
  \l 4C72
  \l 4C73
  \l 4C74
  \l 4C75
  \l 4C76
  \l 4C77
  \l 4C78
  \l 4C79
  \l 4C7A
  \l 4C7B
  \l 4C7C
  \l 4C7D
  \l 4C7E
  \l 4C7F
  \l 4C80
  \l 4C81
  \l 4C82
  \l 4C83
  \l 4C84
  \l 4C85
  \l 4C86
  \l 4C87
  \l 4C88
  \l 4C89
  \l 4C8A
  \l 4C8B
  \l 4C8C
  \l 4C8D
  \l 4C8E
  \l 4C8F
  \l 4C90
  \l 4C91
  \l 4C92
  \l 4C93
  \l 4C94
  \l 4C95
  \l 4C96
  \l 4C97
  \l 4C98
  \l 4C99
  \l 4C9A
  \l 4C9B
  \l 4C9C
  \l 4C9D
  \l 4C9E
  \l 4C9F
  \l 4CA0
  \l 4CA1
  \l 4CA2
  \l 4CA3
  \l 4CA4
  \l 4CA5
  \l 4CA6
  \l 4CA7
  \l 4CA8
  \l 4CA9
  \l 4CAA
  \l 4CAB
  \l 4CAC
  \l 4CAD
  \l 4CAE
  \l 4CAF
  \l 4CB0
  \l 4CB1
  \l 4CB2
  \l 4CB3
  \l 4CB4
  \l 4CB5
  \l 4CB6
  \l 4CB7
  \l 4CB8
  \l 4CB9
  \l 4CBA
  \l 4CBB
  \l 4CBC
  \l 4CBD
  \l 4CBE
  \l 4CBF
  \l 4CC0
  \l 4CC1
  \l 4CC2
  \l 4CC3
  \l 4CC4
  \l 4CC5
  \l 4CC6
  \l 4CC7
  \l 4CC8
  \l 4CC9
  \l 4CCA
  \l 4CCB
  \l 4CCC
  \l 4CCD
  \l 4CCE
  \l 4CCF
  \l 4CD0
  \l 4CD1
  \l 4CD2
  \l 4CD3
  \l 4CD4
  \l 4CD5
  \l 4CD6
  \l 4CD7
  \l 4CD8
  \l 4CD9
  \l 4CDA
  \l 4CDB
  \l 4CDC
  \l 4CDD
  \l 4CDE
  \l 4CDF
  \l 4CE0
  \l 4CE1
  \l 4CE2
  \l 4CE3
  \l 4CE4
  \l 4CE5
  \l 4CE6
  \l 4CE7
  \l 4CE8
  \l 4CE9
  \l 4CEA
  \l 4CEB
  \l 4CEC
  \l 4CED
  \l 4CEE
  \l 4CEF
  \l 4CF0
  \l 4CF1
  \l 4CF2
  \l 4CF3
  \l 4CF4
  \l 4CF5
  \l 4CF6
  \l 4CF7
  \l 4CF8
  \l 4CF9
  \l 4CFA
  \l 4CFB
  \l 4CFC
  \l 4CFD
  \l 4CFE
  \l 4CFF
  \l 4D00
  \l 4D01
  \l 4D02
  \l 4D03
  \l 4D04
  \l 4D05
  \l 4D06
  \l 4D07
  \l 4D08
  \l 4D09
  \l 4D0A
  \l 4D0B
  \l 4D0C
  \l 4D0D
  \l 4D0E
  \l 4D0F
  \l 4D10
  \l 4D11
  \l 4D12
  \l 4D13
  \l 4D14
  \l 4D15
  \l 4D16
  \l 4D17
  \l 4D18
  \l 4D19
  \l 4D1A
  \l 4D1B
  \l 4D1C
  \l 4D1D
  \l 4D1E
  \l 4D1F
  \l 4D20
  \l 4D21
  \l 4D22
  \l 4D23
  \l 4D24
  \l 4D25
  \l 4D26
  \l 4D27
  \l 4D28
  \l 4D29
  \l 4D2A
  \l 4D2B
  \l 4D2C
  \l 4D2D
  \l 4D2E
  \l 4D2F
  \l 4D30
  \l 4D31
  \l 4D32
  \l 4D33
  \l 4D34
  \l 4D35
  \l 4D36
  \l 4D37
  \l 4D38
  \l 4D39
  \l 4D3A
  \l 4D3B
  \l 4D3C
  \l 4D3D
  \l 4D3E
  \l 4D3F
  \l 4D40
  \l 4D41
  \l 4D42
  \l 4D43
  \l 4D44
  \l 4D45
  \l 4D46
  \l 4D47
  \l 4D48
  \l 4D49
  \l 4D4A
  \l 4D4B
  \l 4D4C
  \l 4D4D
  \l 4D4E
  \l 4D4F
  \l 4D50
  \l 4D51
  \l 4D52
  \l 4D53
  \l 4D54
  \l 4D55
  \l 4D56
  \l 4D57
  \l 4D58
  \l 4D59
  \l 4D5A
  \l 4D5B
  \l 4D5C
  \l 4D5D
  \l 4D5E
  \l 4D5F
  \l 4D60
  \l 4D61
  \l 4D62
  \l 4D63
  \l 4D64
  \l 4D65
  \l 4D66
  \l 4D67
  \l 4D68
  \l 4D69
  \l 4D6A
  \l 4D6B
  \l 4D6C
  \l 4D6D
  \l 4D6E
  \l 4D6F
  \l 4D70
  \l 4D71
  \l 4D72
  \l 4D73
  \l 4D74
  \l 4D75
  \l 4D76
  \l 4D77
  \l 4D78
  \l 4D79
  \l 4D7A
  \l 4D7B
  \l 4D7C
  \l 4D7D
  \l 4D7E
  \l 4D7F
  \l 4D80
  \l 4D81
  \l 4D82
  \l 4D83
  \l 4D84
  \l 4D85
  \l 4D86
  \l 4D87
  \l 4D88
  \l 4D89
  \l 4D8A
  \l 4D8B
  \l 4D8C
  \l 4D8D
  \l 4D8E
  \l 4D8F
  \l 4D90
  \l 4D91
  \l 4D92
  \l 4D93
  \l 4D94
  \l 4D95
  \l 4D96
  \l 4D97
  \l 4D98
  \l 4D99
  \l 4D9A
  \l 4D9B
  \l 4D9C
  \l 4D9D
  \l 4D9E
  \l 4D9F
  \l 4DA0
  \l 4DA1
  \l 4DA2
  \l 4DA3
  \l 4DA4
  \l 4DA5
  \l 4DA6
  \l 4DA7
  \l 4DA8
  \l 4DA9
  \l 4DAA
  \l 4DAB
  \l 4DAC
  \l 4DAD
  \l 4DAE
  \l 4DAF
  \l 4DB0
  \l 4DB1
  \l 4DB2
  \l 4DB3
  \l 4DB4
  \l 4DB5
  \l 4E00
  \l 4E01
  \l 4E02
  \l 4E03
  \l 4E04
  \l 4E05
  \l 4E06
  \l 4E07
  \l 4E08
  \l 4E09
  \l 4E0A
  \l 4E0B
  \l 4E0C
  \l 4E0D
  \l 4E0E
  \l 4E0F
  \l 4E10
  \l 4E11
  \l 4E12
  \l 4E13
  \l 4E14
  \l 4E15
  \l 4E16
  \l 4E17
  \l 4E18
  \l 4E19
  \l 4E1A
  \l 4E1B
  \l 4E1C
  \l 4E1D
  \l 4E1E
  \l 4E1F
  \l 4E20
  \l 4E21
  \l 4E22
  \l 4E23
  \l 4E24
  \l 4E25
  \l 4E26
  \l 4E27
  \l 4E28
  \l 4E29
  \l 4E2A
  \l 4E2B
  \l 4E2C
  \l 4E2D
  \l 4E2E
  \l 4E2F
  \l 4E30
  \l 4E31
  \l 4E32
  \l 4E33
  \l 4E34
  \l 4E35
  \l 4E36
  \l 4E37
  \l 4E38
  \l 4E39
  \l 4E3A
  \l 4E3B
  \l 4E3C
  \l 4E3D
  \l 4E3E
  \l 4E3F
  \l 4E40
  \l 4E41
  \l 4E42
  \l 4E43
  \l 4E44
  \l 4E45
  \l 4E46
  \l 4E47
  \l 4E48
  \l 4E49
  \l 4E4A
  \l 4E4B
  \l 4E4C
  \l 4E4D
  \l 4E4E
  \l 4E4F
  \l 4E50
  \l 4E51
  \l 4E52
  \l 4E53
  \l 4E54
  \l 4E55
  \l 4E56
  \l 4E57
  \l 4E58
  \l 4E59
  \l 4E5A
  \l 4E5B
  \l 4E5C
  \l 4E5D
  \l 4E5E
  \l 4E5F
  \l 4E60
  \l 4E61
  \l 4E62
  \l 4E63
  \l 4E64
  \l 4E65
  \l 4E66
  \l 4E67
  \l 4E68
  \l 4E69
  \l 4E6A
  \l 4E6B
  \l 4E6C
  \l 4E6D
  \l 4E6E
  \l 4E6F
  \l 4E70
  \l 4E71
  \l 4E72
  \l 4E73
  \l 4E74
  \l 4E75
  \l 4E76
  \l 4E77
  \l 4E78
  \l 4E79
  \l 4E7A
  \l 4E7B
  \l 4E7C
  \l 4E7D
  \l 4E7E
  \l 4E7F
  \l 4E80
  \l 4E81
  \l 4E82
  \l 4E83
  \l 4E84
  \l 4E85
  \l 4E86
  \l 4E87
  \l 4E88
  \l 4E89
  \l 4E8A
  \l 4E8B
  \l 4E8C
  \l 4E8D
  \l 4E8E
  \l 4E8F
  \l 4E90
  \l 4E91
  \l 4E92
  \l 4E93
  \l 4E94
  \l 4E95
  \l 4E96
  \l 4E97
  \l 4E98
  \l 4E99
  \l 4E9A
  \l 4E9B
  \l 4E9C
  \l 4E9D
  \l 4E9E
  \l 4E9F
  \l 4EA0
  \l 4EA1
  \l 4EA2
  \l 4EA3
  \l 4EA4
  \l 4EA5
  \l 4EA6
  \l 4EA7
  \l 4EA8
  \l 4EA9
  \l 4EAA
  \l 4EAB
  \l 4EAC
  \l 4EAD
  \l 4EAE
  \l 4EAF
  \l 4EB0
  \l 4EB1
  \l 4EB2
  \l 4EB3
  \l 4EB4
  \l 4EB5
  \l 4EB6
  \l 4EB7
  \l 4EB8
  \l 4EB9
  \l 4EBA
  \l 4EBB
  \l 4EBC
  \l 4EBD
  \l 4EBE
  \l 4EBF
  \l 4EC0
  \l 4EC1
  \l 4EC2
  \l 4EC3
  \l 4EC4
  \l 4EC5
  \l 4EC6
  \l 4EC7
  \l 4EC8
  \l 4EC9
  \l 4ECA
  \l 4ECB
  \l 4ECC
  \l 4ECD
  \l 4ECE
  \l 4ECF
  \l 4ED0
  \l 4ED1
  \l 4ED2
  \l 4ED3
  \l 4ED4
  \l 4ED5
  \l 4ED6
  \l 4ED7
  \l 4ED8
  \l 4ED9
  \l 4EDA
  \l 4EDB
  \l 4EDC
  \l 4EDD
  \l 4EDE
  \l 4EDF
  \l 4EE0
  \l 4EE1
  \l 4EE2
  \l 4EE3
  \l 4EE4
  \l 4EE5
  \l 4EE6
  \l 4EE7
  \l 4EE8
  \l 4EE9
  \l 4EEA
  \l 4EEB
  \l 4EEC
  \l 4EED
  \l 4EEE
  \l 4EEF
  \l 4EF0
  \l 4EF1
  \l 4EF2
  \l 4EF3
  \l 4EF4
  \l 4EF5
  \l 4EF6
  \l 4EF7
  \l 4EF8
  \l 4EF9
  \l 4EFA
  \l 4EFB
  \l 4EFC
  \l 4EFD
  \l 4EFE
  \l 4EFF
  \l 4F00
  \l 4F01
  \l 4F02
  \l 4F03
  \l 4F04
  \l 4F05
  \l 4F06
  \l 4F07
  \l 4F08
  \l 4F09
  \l 4F0A
  \l 4F0B
  \l 4F0C
  \l 4F0D
  \l 4F0E
  \l 4F0F
  \l 4F10
  \l 4F11
  \l 4F12
  \l 4F13
  \l 4F14
  \l 4F15
  \l 4F16
  \l 4F17
  \l 4F18
  \l 4F19
  \l 4F1A
  \l 4F1B
  \l 4F1C
  \l 4F1D
  \l 4F1E
  \l 4F1F
  \l 4F20
  \l 4F21
  \l 4F22
  \l 4F23
  \l 4F24
  \l 4F25
  \l 4F26
  \l 4F27
  \l 4F28
  \l 4F29
  \l 4F2A
  \l 4F2B
  \l 4F2C
  \l 4F2D
  \l 4F2E
  \l 4F2F
  \l 4F30
  \l 4F31
  \l 4F32
  \l 4F33
  \l 4F34
  \l 4F35
  \l 4F36
  \l 4F37
  \l 4F38
  \l 4F39
  \l 4F3A
  \l 4F3B
  \l 4F3C
  \l 4F3D
  \l 4F3E
  \l 4F3F
  \l 4F40
  \l 4F41
  \l 4F42
  \l 4F43
  \l 4F44
  \l 4F45
  \l 4F46
  \l 4F47
  \l 4F48
  \l 4F49
  \l 4F4A
  \l 4F4B
  \l 4F4C
  \l 4F4D
  \l 4F4E
  \l 4F4F
  \l 4F50
  \l 4F51
  \l 4F52
  \l 4F53
  \l 4F54
  \l 4F55
  \l 4F56
  \l 4F57
  \l 4F58
  \l 4F59
  \l 4F5A
  \l 4F5B
  \l 4F5C
  \l 4F5D
  \l 4F5E
  \l 4F5F
  \l 4F60
  \l 4F61
  \l 4F62
  \l 4F63
  \l 4F64
  \l 4F65
  \l 4F66
  \l 4F67
  \l 4F68
  \l 4F69
  \l 4F6A
  \l 4F6B
  \l 4F6C
  \l 4F6D
  \l 4F6E
  \l 4F6F
  \l 4F70
  \l 4F71
  \l 4F72
  \l 4F73
  \l 4F74
  \l 4F75
  \l 4F76
  \l 4F77
  \l 4F78
  \l 4F79
  \l 4F7A
  \l 4F7B
  \l 4F7C
  \l 4F7D
  \l 4F7E
  \l 4F7F
  \l 4F80
  \l 4F81
  \l 4F82
  \l 4F83
  \l 4F84
  \l 4F85
  \l 4F86
  \l 4F87
  \l 4F88
  \l 4F89
  \l 4F8A
  \l 4F8B
  \l 4F8C
  \l 4F8D
  \l 4F8E
  \l 4F8F
  \l 4F90
  \l 4F91
  \l 4F92
  \l 4F93
  \l 4F94
  \l 4F95
  \l 4F96
  \l 4F97
  \l 4F98
  \l 4F99
  \l 4F9A
  \l 4F9B
  \l 4F9C
  \l 4F9D
  \l 4F9E
  \l 4F9F
  \l 4FA0
  \l 4FA1
  \l 4FA2
  \l 4FA3
  \l 4FA4
  \l 4FA5
  \l 4FA6
  \l 4FA7
  \l 4FA8
  \l 4FA9
  \l 4FAA
  \l 4FAB
  \l 4FAC
  \l 4FAD
  \l 4FAE
  \l 4FAF
  \l 4FB0
  \l 4FB1
  \l 4FB2
  \l 4FB3
  \l 4FB4
  \l 4FB5
  \l 4FB6
  \l 4FB7
  \l 4FB8
  \l 4FB9
  \l 4FBA
  \l 4FBB
  \l 4FBC
  \l 4FBD
  \l 4FBE
  \l 4FBF
  \l 4FC0
  \l 4FC1
  \l 4FC2
  \l 4FC3
  \l 4FC4
  \l 4FC5
  \l 4FC6
  \l 4FC7
  \l 4FC8
  \l 4FC9
  \l 4FCA
  \l 4FCB
  \l 4FCC
  \l 4FCD
  \l 4FCE
  \l 4FCF
  \l 4FD0
  \l 4FD1
  \l 4FD2
  \l 4FD3
  \l 4FD4
  \l 4FD5
  \l 4FD6
  \l 4FD7
  \l 4FD8
  \l 4FD9
  \l 4FDA
  \l 4FDB
  \l 4FDC
  \l 4FDD
  \l 4FDE
  \l 4FDF
  \l 4FE0
  \l 4FE1
  \l 4FE2
  \l 4FE3
  \l 4FE4
  \l 4FE5
  \l 4FE6
  \l 4FE7
  \l 4FE8
  \l 4FE9
  \l 4FEA
  \l 4FEB
  \l 4FEC
  \l 4FED
  \l 4FEE
  \l 4FEF
  \l 4FF0
  \l 4FF1
  \l 4FF2
  \l 4FF3
  \l 4FF4
  \l 4FF5
  \l 4FF6
  \l 4FF7
  \l 4FF8
  \l 4FF9
  \l 4FFA
  \l 4FFB
  \l 4FFC
  \l 4FFD
  \l 4FFE
  \l 4FFF
  \l 5000
  \l 5001
  \l 5002
  \l 5003
  \l 5004
  \l 5005
  \l 5006
  \l 5007
  \l 5008
  \l 5009
  \l 500A
  \l 500B
  \l 500C
  \l 500D
  \l 500E
  \l 500F
  \l 5010
  \l 5011
  \l 5012
  \l 5013
  \l 5014
  \l 5015
  \l 5016
  \l 5017
  \l 5018
  \l 5019
  \l 501A
  \l 501B
  \l 501C
  \l 501D
  \l 501E
  \l 501F
  \l 5020
  \l 5021
  \l 5022
  \l 5023
  \l 5024
  \l 5025
  \l 5026
  \l 5027
  \l 5028
  \l 5029
  \l 502A
  \l 502B
  \l 502C
  \l 502D
  \l 502E
  \l 502F
  \l 5030
  \l 5031
  \l 5032
  \l 5033
  \l 5034
  \l 5035
  \l 5036
  \l 5037
  \l 5038
  \l 5039
  \l 503A
  \l 503B
  \l 503C
  \l 503D
  \l 503E
  \l 503F
  \l 5040
  \l 5041
  \l 5042
  \l 5043
  \l 5044
  \l 5045
  \l 5046
  \l 5047
  \l 5048
  \l 5049
  \l 504A
  \l 504B
  \l 504C
  \l 504D
  \l 504E
  \l 504F
  \l 5050
  \l 5051
  \l 5052
  \l 5053
  \l 5054
  \l 5055
  \l 5056
  \l 5057
  \l 5058
  \l 5059
  \l 505A
  \l 505B
  \l 505C
  \l 505D
  \l 505E
  \l 505F
  \l 5060
  \l 5061
  \l 5062
  \l 5063
  \l 5064
  \l 5065
  \l 5066
  \l 5067
  \l 5068
  \l 5069
  \l 506A
  \l 506B
  \l 506C
  \l 506D
  \l 506E
  \l 506F
  \l 5070
  \l 5071
  \l 5072
  \l 5073
  \l 5074
  \l 5075
  \l 5076
  \l 5077
  \l 5078
  \l 5079
  \l 507A
  \l 507B
  \l 507C
  \l 507D
  \l 507E
  \l 507F
  \l 5080
  \l 5081
  \l 5082
  \l 5083
  \l 5084
  \l 5085
  \l 5086
  \l 5087
  \l 5088
  \l 5089
  \l 508A
  \l 508B
  \l 508C
  \l 508D
  \l 508E
  \l 508F
  \l 5090
  \l 5091
  \l 5092
  \l 5093
  \l 5094
  \l 5095
  \l 5096
  \l 5097
  \l 5098
  \l 5099
  \l 509A
  \l 509B
  \l 509C
  \l 509D
  \l 509E
  \l 509F
  \l 50A0
  \l 50A1
  \l 50A2
  \l 50A3
  \l 50A4
  \l 50A5
  \l 50A6
  \l 50A7
  \l 50A8
  \l 50A9
  \l 50AA
  \l 50AB
  \l 50AC
  \l 50AD
  \l 50AE
  \l 50AF
  \l 50B0
  \l 50B1
  \l 50B2
  \l 50B3
  \l 50B4
  \l 50B5
  \l 50B6
  \l 50B7
  \l 50B8
  \l 50B9
  \l 50BA
  \l 50BB
  \l 50BC
  \l 50BD
  \l 50BE
  \l 50BF
  \l 50C0
  \l 50C1
  \l 50C2
  \l 50C3
  \l 50C4
  \l 50C5
  \l 50C6
  \l 50C7
  \l 50C8
  \l 50C9
  \l 50CA
  \l 50CB
  \l 50CC
  \l 50CD
  \l 50CE
  \l 50CF
  \l 50D0
  \l 50D1
  \l 50D2
  \l 50D3
  \l 50D4
  \l 50D5
  \l 50D6
  \l 50D7
  \l 50D8
  \l 50D9
  \l 50DA
  \l 50DB
  \l 50DC
  \l 50DD
  \l 50DE
  \l 50DF
  \l 50E0
  \l 50E1
  \l 50E2
  \l 50E3
  \l 50E4
  \l 50E5
  \l 50E6
  \l 50E7
  \l 50E8
  \l 50E9
  \l 50EA
  \l 50EB
  \l 50EC
  \l 50ED
  \l 50EE
  \l 50EF
  \l 50F0
  \l 50F1
  \l 50F2
  \l 50F3
  \l 50F4
  \l 50F5
  \l 50F6
  \l 50F7
  \l 50F8
  \l 50F9
  \l 50FA
  \l 50FB
  \l 50FC
  \l 50FD
  \l 50FE
  \l 50FF
  \l 5100
  \l 5101
  \l 5102
  \l 5103
  \l 5104
  \l 5105
  \l 5106
  \l 5107
  \l 5108
  \l 5109
  \l 510A
  \l 510B
  \l 510C
  \l 510D
  \l 510E
  \l 510F
  \l 5110
  \l 5111
  \l 5112
  \l 5113
  \l 5114
  \l 5115
  \l 5116
  \l 5117
  \l 5118
  \l 5119
  \l 511A
  \l 511B
  \l 511C
  \l 511D
  \l 511E
  \l 511F
  \l 5120
  \l 5121
  \l 5122
  \l 5123
  \l 5124
  \l 5125
  \l 5126
  \l 5127
  \l 5128
  \l 5129
  \l 512A
  \l 512B
  \l 512C
  \l 512D
  \l 512E
  \l 512F
  \l 5130
  \l 5131
  \l 5132
  \l 5133
  \l 5134
  \l 5135
  \l 5136
  \l 5137
  \l 5138
  \l 5139
  \l 513A
  \l 513B
  \l 513C
  \l 513D
  \l 513E
  \l 513F
  \l 5140
  \l 5141
  \l 5142
  \l 5143
  \l 5144
  \l 5145
  \l 5146
  \l 5147
  \l 5148
  \l 5149
  \l 514A
  \l 514B
  \l 514C
  \l 514D
  \l 514E
  \l 514F
  \l 5150
  \l 5151
  \l 5152
  \l 5153
  \l 5154
  \l 5155
  \l 5156
  \l 5157
  \l 5158
  \l 5159
  \l 515A
  \l 515B
  \l 515C
  \l 515D
  \l 515E
  \l 515F
  \l 5160
  \l 5161
  \l 5162
  \l 5163
  \l 5164
  \l 5165
  \l 5166
  \l 5167
  \l 5168
  \l 5169
  \l 516A
  \l 516B
  \l 516C
  \l 516D
  \l 516E
  \l 516F
  \l 5170
  \l 5171
  \l 5172
  \l 5173
  \l 5174
  \l 5175
  \l 5176
  \l 5177
  \l 5178
  \l 5179
  \l 517A
  \l 517B
  \l 517C
  \l 517D
  \l 517E
  \l 517F
  \l 5180
  \l 5181
  \l 5182
  \l 5183
  \l 5184
  \l 5185
  \l 5186
  \l 5187
  \l 5188
  \l 5189
  \l 518A
  \l 518B
  \l 518C
  \l 518D
  \l 518E
  \l 518F
  \l 5190
  \l 5191
  \l 5192
  \l 5193
  \l 5194
  \l 5195
  \l 5196
  \l 5197
  \l 5198
  \l 5199
  \l 519A
  \l 519B
  \l 519C
  \l 519D
  \l 519E
  \l 519F
  \l 51A0
  \l 51A1
  \l 51A2
  \l 51A3
  \l 51A4
  \l 51A5
  \l 51A6
  \l 51A7
  \l 51A8
  \l 51A9
  \l 51AA
  \l 51AB
  \l 51AC
  \l 51AD
  \l 51AE
  \l 51AF
  \l 51B0
  \l 51B1
  \l 51B2
  \l 51B3
  \l 51B4
  \l 51B5
  \l 51B6
  \l 51B7
  \l 51B8
  \l 51B9
  \l 51BA
  \l 51BB
  \l 51BC
  \l 51BD
  \l 51BE
  \l 51BF
  \l 51C0
  \l 51C1
  \l 51C2
  \l 51C3
  \l 51C4
  \l 51C5
  \l 51C6
  \l 51C7
  \l 51C8
  \l 51C9
  \l 51CA
  \l 51CB
  \l 51CC
  \l 51CD
  \l 51CE
  \l 51CF
  \l 51D0
  \l 51D1
  \l 51D2
  \l 51D3
  \l 51D4
  \l 51D5
  \l 51D6
  \l 51D7
  \l 51D8
  \l 51D9
  \l 51DA
  \l 51DB
  \l 51DC
  \l 51DD
  \l 51DE
  \l 51DF
  \l 51E0
  \l 51E1
  \l 51E2
  \l 51E3
  \l 51E4
  \l 51E5
  \l 51E6
  \l 51E7
  \l 51E8
  \l 51E9
  \l 51EA
  \l 51EB
  \l 51EC
  \l 51ED
  \l 51EE
  \l 51EF
  \l 51F0
  \l 51F1
  \l 51F2
  \l 51F3
  \l 51F4
  \l 51F5
  \l 51F6
  \l 51F7
  \l 51F8
  \l 51F9
  \l 51FA
  \l 51FB
  \l 51FC
  \l 51FD
  \l 51FE
  \l 51FF
  \l 5200
  \l 5201
  \l 5202
  \l 5203
  \l 5204
  \l 5205
  \l 5206
  \l 5207
  \l 5208
  \l 5209
  \l 520A
  \l 520B
  \l 520C
  \l 520D
  \l 520E
  \l 520F
  \l 5210
  \l 5211
  \l 5212
  \l 5213
  \l 5214
  \l 5215
  \l 5216
  \l 5217
  \l 5218
  \l 5219
  \l 521A
  \l 521B
  \l 521C
  \l 521D
  \l 521E
  \l 521F
  \l 5220
  \l 5221
  \l 5222
  \l 5223
  \l 5224
  \l 5225
  \l 5226
  \l 5227
  \l 5228
  \l 5229
  \l 522A
  \l 522B
  \l 522C
  \l 522D
  \l 522E
  \l 522F
  \l 5230
  \l 5231
  \l 5232
  \l 5233
  \l 5234
  \l 5235
  \l 5236
  \l 5237
  \l 5238
  \l 5239
  \l 523A
  \l 523B
  \l 523C
  \l 523D
  \l 523E
  \l 523F
  \l 5240
  \l 5241
  \l 5242
  \l 5243
  \l 5244
  \l 5245
  \l 5246
  \l 5247
  \l 5248
  \l 5249
  \l 524A
  \l 524B
  \l 524C
  \l 524D
  \l 524E
  \l 524F
  \l 5250
  \l 5251
  \l 5252
  \l 5253
  \l 5254
  \l 5255
  \l 5256
  \l 5257
  \l 5258
  \l 5259
  \l 525A
  \l 525B
  \l 525C
  \l 525D
  \l 525E
  \l 525F
  \l 5260
  \l 5261
  \l 5262
  \l 5263
  \l 5264
  \l 5265
  \l 5266
  \l 5267
  \l 5268
  \l 5269
  \l 526A
  \l 526B
  \l 526C
  \l 526D
  \l 526E
  \l 526F
  \l 5270
  \l 5271
  \l 5272
  \l 5273
  \l 5274
  \l 5275
  \l 5276
  \l 5277
  \l 5278
  \l 5279
  \l 527A
  \l 527B
  \l 527C
  \l 527D
  \l 527E
  \l 527F
  \l 5280
  \l 5281
  \l 5282
  \l 5283
  \l 5284
  \l 5285
  \l 5286
  \l 5287
  \l 5288
  \l 5289
  \l 528A
  \l 528B
  \l 528C
  \l 528D
  \l 528E
  \l 528F
  \l 5290
  \l 5291
  \l 5292
  \l 5293
  \l 5294
  \l 5295
  \l 5296
  \l 5297
  \l 5298
  \l 5299
  \l 529A
  \l 529B
  \l 529C
  \l 529D
  \l 529E
  \l 529F
  \l 52A0
  \l 52A1
  \l 52A2
  \l 52A3
  \l 52A4
  \l 52A5
  \l 52A6
  \l 52A7
  \l 52A8
  \l 52A9
  \l 52AA
  \l 52AB
  \l 52AC
  \l 52AD
  \l 52AE
  \l 52AF
  \l 52B0
  \l 52B1
  \l 52B2
  \l 52B3
  \l 52B4
  \l 52B5
  \l 52B6
  \l 52B7
  \l 52B8
  \l 52B9
  \l 52BA
  \l 52BB
  \l 52BC
  \l 52BD
  \l 52BE
  \l 52BF
  \l 52C0
  \l 52C1
  \l 52C2
  \l 52C3
  \l 52C4
  \l 52C5
  \l 52C6
  \l 52C7
  \l 52C8
  \l 52C9
  \l 52CA
  \l 52CB
  \l 52CC
  \l 52CD
  \l 52CE
  \l 52CF
  \l 52D0
  \l 52D1
  \l 52D2
  \l 52D3
  \l 52D4
  \l 52D5
  \l 52D6
  \l 52D7
  \l 52D8
  \l 52D9
  \l 52DA
  \l 52DB
  \l 52DC
  \l 52DD
  \l 52DE
  \l 52DF
  \l 52E0
  \l 52E1
  \l 52E2
  \l 52E3
  \l 52E4
  \l 52E5
  \l 52E6
  \l 52E7
  \l 52E8
  \l 52E9
  \l 52EA
  \l 52EB
  \l 52EC
  \l 52ED
  \l 52EE
  \l 52EF
  \l 52F0
  \l 52F1
  \l 52F2
  \l 52F3
  \l 52F4
  \l 52F5
  \l 52F6
  \l 52F7
  \l 52F8
  \l 52F9
  \l 52FA
  \l 52FB
  \l 52FC
  \l 52FD
  \l 52FE
  \l 52FF
  \l 5300
  \l 5301
  \l 5302
  \l 5303
  \l 5304
  \l 5305
  \l 5306
  \l 5307
  \l 5308
  \l 5309
  \l 530A
  \l 530B
  \l 530C
  \l 530D
  \l 530E
  \l 530F
  \l 5310
  \l 5311
  \l 5312
  \l 5313
  \l 5314
  \l 5315
  \l 5316
  \l 5317
  \l 5318
  \l 5319
  \l 531A
  \l 531B
  \l 531C
  \l 531D
  \l 531E
  \l 531F
  \l 5320
  \l 5321
  \l 5322
  \l 5323
  \l 5324
  \l 5325
  \l 5326
  \l 5327
  \l 5328
  \l 5329
  \l 532A
  \l 532B
  \l 532C
  \l 532D
  \l 532E
  \l 532F
  \l 5330
  \l 5331
  \l 5332
  \l 5333
  \l 5334
  \l 5335
  \l 5336
  \l 5337
  \l 5338
  \l 5339
  \l 533A
  \l 533B
  \l 533C
  \l 533D
  \l 533E
  \l 533F
  \l 5340
  \l 5341
  \l 5342
  \l 5343
  \l 5344
  \l 5345
  \l 5346
  \l 5347
  \l 5348
  \l 5349
  \l 534A
  \l 534B
  \l 534C
  \l 534D
  \l 534E
  \l 534F
  \l 5350
  \l 5351
  \l 5352
  \l 5353
  \l 5354
  \l 5355
  \l 5356
  \l 5357
  \l 5358
  \l 5359
  \l 535A
  \l 535B
  \l 535C
  \l 535D
  \l 535E
  \l 535F
  \l 5360
  \l 5361
  \l 5362
  \l 5363
  \l 5364
  \l 5365
  \l 5366
  \l 5367
  \l 5368
  \l 5369
  \l 536A
  \l 536B
  \l 536C
  \l 536D
  \l 536E
  \l 536F
  \l 5370
  \l 5371
  \l 5372
  \l 5373
  \l 5374
  \l 5375
  \l 5376
  \l 5377
  \l 5378
  \l 5379
  \l 537A
  \l 537B
  \l 537C
  \l 537D
  \l 537E
  \l 537F
  \l 5380
  \l 5381
  \l 5382
  \l 5383
  \l 5384
  \l 5385
  \l 5386
  \l 5387
  \l 5388
  \l 5389
  \l 538A
  \l 538B
  \l 538C
  \l 538D
  \l 538E
  \l 538F
  \l 5390
  \l 5391
  \l 5392
  \l 5393
  \l 5394
  \l 5395
  \l 5396
  \l 5397
  \l 5398
  \l 5399
  \l 539A
  \l 539B
  \l 539C
  \l 539D
  \l 539E
  \l 539F
  \l 53A0
  \l 53A1
  \l 53A2
  \l 53A3
  \l 53A4
  \l 53A5
  \l 53A6
  \l 53A7
  \l 53A8
  \l 53A9
  \l 53AA
  \l 53AB
  \l 53AC
  \l 53AD
  \l 53AE
  \l 53AF
  \l 53B0
  \l 53B1
  \l 53B2
  \l 53B3
  \l 53B4
  \l 53B5
  \l 53B6
  \l 53B7
  \l 53B8
  \l 53B9
  \l 53BA
  \l 53BB
  \l 53BC
  \l 53BD
  \l 53BE
  \l 53BF
  \l 53C0
  \l 53C1
  \l 53C2
  \l 53C3
  \l 53C4
  \l 53C5
  \l 53C6
  \l 53C7
  \l 53C8
  \l 53C9
  \l 53CA
  \l 53CB
  \l 53CC
  \l 53CD
  \l 53CE
  \l 53CF
  \l 53D0
  \l 53D1
  \l 53D2
  \l 53D3
  \l 53D4
  \l 53D5
  \l 53D6
  \l 53D7
  \l 53D8
  \l 53D9
  \l 53DA
  \l 53DB
  \l 53DC
  \l 53DD
  \l 53DE
  \l 53DF
  \l 53E0
  \l 53E1
  \l 53E2
  \l 53E3
  \l 53E4
  \l 53E5
  \l 53E6
  \l 53E7
  \l 53E8
  \l 53E9
  \l 53EA
  \l 53EB
  \l 53EC
  \l 53ED
  \l 53EE
  \l 53EF
  \l 53F0
  \l 53F1
  \l 53F2
  \l 53F3
  \l 53F4
  \l 53F5
  \l 53F6
  \l 53F7
  \l 53F8
  \l 53F9
  \l 53FA
  \l 53FB
  \l 53FC
  \l 53FD
  \l 53FE
  \l 53FF
  \l 5400
  \l 5401
  \l 5402
  \l 5403
  \l 5404
  \l 5405
  \l 5406
  \l 5407
  \l 5408
  \l 5409
  \l 540A
  \l 540B
  \l 540C
  \l 540D
  \l 540E
  \l 540F
  \l 5410
  \l 5411
  \l 5412
  \l 5413
  \l 5414
  \l 5415
  \l 5416
  \l 5417
  \l 5418
  \l 5419
  \l 541A
  \l 541B
  \l 541C
  \l 541D
  \l 541E
  \l 541F
  \l 5420
  \l 5421
  \l 5422
  \l 5423
  \l 5424
  \l 5425
  \l 5426
  \l 5427
  \l 5428
  \l 5429
  \l 542A
  \l 542B
  \l 542C
  \l 542D
  \l 542E
  \l 542F
  \l 5430
  \l 5431
  \l 5432
  \l 5433
  \l 5434
  \l 5435
  \l 5436
  \l 5437
  \l 5438
  \l 5439
  \l 543A
  \l 543B
  \l 543C
  \l 543D
  \l 543E
  \l 543F
  \l 5440
  \l 5441
  \l 5442
  \l 5443
  \l 5444
  \l 5445
  \l 5446
  \l 5447
  \l 5448
  \l 5449
  \l 544A
  \l 544B
  \l 544C
  \l 544D
  \l 544E
  \l 544F
  \l 5450
  \l 5451
  \l 5452
  \l 5453
  \l 5454
  \l 5455
  \l 5456
  \l 5457
  \l 5458
  \l 5459
  \l 545A
  \l 545B
  \l 545C
  \l 545D
  \l 545E
  \l 545F
  \l 5460
  \l 5461
  \l 5462
  \l 5463
  \l 5464
  \l 5465
  \l 5466
  \l 5467
  \l 5468
  \l 5469
  \l 546A
  \l 546B
  \l 546C
  \l 546D
  \l 546E
  \l 546F
  \l 5470
  \l 5471
  \l 5472
  \l 5473
  \l 5474
  \l 5475
  \l 5476
  \l 5477
  \l 5478
  \l 5479
  \l 547A
  \l 547B
  \l 547C
  \l 547D
  \l 547E
  \l 547F
  \l 5480
  \l 5481
  \l 5482
  \l 5483
  \l 5484
  \l 5485
  \l 5486
  \l 5487
  \l 5488
  \l 5489
  \l 548A
  \l 548B
  \l 548C
  \l 548D
  \l 548E
  \l 548F
  \l 5490
  \l 5491
  \l 5492
  \l 5493
  \l 5494
  \l 5495
  \l 5496
  \l 5497
  \l 5498
  \l 5499
  \l 549A
  \l 549B
  \l 549C
  \l 549D
  \l 549E
  \l 549F
  \l 54A0
  \l 54A1
  \l 54A2
  \l 54A3
  \l 54A4
  \l 54A5
  \l 54A6
  \l 54A7
  \l 54A8
  \l 54A9
  \l 54AA
  \l 54AB
  \l 54AC
  \l 54AD
  \l 54AE
  \l 54AF
  \l 54B0
  \l 54B1
  \l 54B2
  \l 54B3
  \l 54B4
  \l 54B5
  \l 54B6
  \l 54B7
  \l 54B8
  \l 54B9
  \l 54BA
  \l 54BB
  \l 54BC
  \l 54BD
  \l 54BE
  \l 54BF
  \l 54C0
  \l 54C1
  \l 54C2
  \l 54C3
  \l 54C4
  \l 54C5
  \l 54C6
  \l 54C7
  \l 54C8
  \l 54C9
  \l 54CA
  \l 54CB
  \l 54CC
  \l 54CD
  \l 54CE
  \l 54CF
  \l 54D0
  \l 54D1
  \l 54D2
  \l 54D3
  \l 54D4
  \l 54D5
  \l 54D6
  \l 54D7
  \l 54D8
  \l 54D9
  \l 54DA
  \l 54DB
  \l 54DC
  \l 54DD
  \l 54DE
  \l 54DF
  \l 54E0
  \l 54E1
  \l 54E2
  \l 54E3
  \l 54E4
  \l 54E5
  \l 54E6
  \l 54E7
  \l 54E8
  \l 54E9
  \l 54EA
  \l 54EB
  \l 54EC
  \l 54ED
  \l 54EE
  \l 54EF
  \l 54F0
  \l 54F1
  \l 54F2
  \l 54F3
  \l 54F4
  \l 54F5
  \l 54F6
  \l 54F7
  \l 54F8
  \l 54F9
  \l 54FA
  \l 54FB
  \l 54FC
  \l 54FD
  \l 54FE
  \l 54FF
  \l 5500
  \l 5501
  \l 5502
  \l 5503
  \l 5504
  \l 5505
  \l 5506
  \l 5507
  \l 5508
  \l 5509
  \l 550A
  \l 550B
  \l 550C
  \l 550D
  \l 550E
  \l 550F
  \l 5510
  \l 5511
  \l 5512
  \l 5513
  \l 5514
  \l 5515
  \l 5516
  \l 5517
  \l 5518
  \l 5519
  \l 551A
  \l 551B
  \l 551C
  \l 551D
  \l 551E
  \l 551F
  \l 5520
  \l 5521
  \l 5522
  \l 5523
  \l 5524
  \l 5525
  \l 5526
  \l 5527
  \l 5528
  \l 5529
  \l 552A
  \l 552B
  \l 552C
  \l 552D
  \l 552E
  \l 552F
  \l 5530
  \l 5531
  \l 5532
  \l 5533
  \l 5534
  \l 5535
  \l 5536
  \l 5537
  \l 5538
  \l 5539
  \l 553A
  \l 553B
  \l 553C
  \l 553D
  \l 553E
  \l 553F
  \l 5540
  \l 5541
  \l 5542
  \l 5543
  \l 5544
  \l 5545
  \l 5546
  \l 5547
  \l 5548
  \l 5549
  \l 554A
  \l 554B
  \l 554C
  \l 554D
  \l 554E
  \l 554F
  \l 5550
  \l 5551
  \l 5552
  \l 5553
  \l 5554
  \l 5555
  \l 5556
  \l 5557
  \l 5558
  \l 5559
  \l 555A
  \l 555B
  \l 555C
  \l 555D
  \l 555E
  \l 555F
  \l 5560
  \l 5561
  \l 5562
  \l 5563
  \l 5564
  \l 5565
  \l 5566
  \l 5567
  \l 5568
  \l 5569
  \l 556A
  \l 556B
  \l 556C
  \l 556D
  \l 556E
  \l 556F
  \l 5570
  \l 5571
  \l 5572
  \l 5573
  \l 5574
  \l 5575
  \l 5576
  \l 5577
  \l 5578
  \l 5579
  \l 557A
  \l 557B
  \l 557C
  \l 557D
  \l 557E
  \l 557F
  \l 5580
  \l 5581
  \l 5582
  \l 5583
  \l 5584
  \l 5585
  \l 5586
  \l 5587
  \l 5588
  \l 5589
  \l 558A
  \l 558B
  \l 558C
  \l 558D
  \l 558E
  \l 558F
  \l 5590
  \l 5591
  \l 5592
  \l 5593
  \l 5594
  \l 5595
  \l 5596
  \l 5597
  \l 5598
  \l 5599
  \l 559A
  \l 559B
  \l 559C
  \l 559D
  \l 559E
  \l 559F
  \l 55A0
  \l 55A1
  \l 55A2
  \l 55A3
  \l 55A4
  \l 55A5
  \l 55A6
  \l 55A7
  \l 55A8
  \l 55A9
  \l 55AA
  \l 55AB
  \l 55AC
  \l 55AD
  \l 55AE
  \l 55AF
  \l 55B0
  \l 55B1
  \l 55B2
  \l 55B3
  \l 55B4
  \l 55B5
  \l 55B6
  \l 55B7
  \l 55B8
  \l 55B9
  \l 55BA
  \l 55BB
  \l 55BC
  \l 55BD
  \l 55BE
  \l 55BF
  \l 55C0
  \l 55C1
  \l 55C2
  \l 55C3
  \l 55C4
  \l 55C5
  \l 55C6
  \l 55C7
  \l 55C8
  \l 55C9
  \l 55CA
  \l 55CB
  \l 55CC
  \l 55CD
  \l 55CE
  \l 55CF
  \l 55D0
  \l 55D1
  \l 55D2
  \l 55D3
  \l 55D4
  \l 55D5
  \l 55D6
  \l 55D7
  \l 55D8
  \l 55D9
  \l 55DA
  \l 55DB
  \l 55DC
  \l 55DD
  \l 55DE
  \l 55DF
  \l 55E0
  \l 55E1
  \l 55E2
  \l 55E3
  \l 55E4
  \l 55E5
  \l 55E6
  \l 55E7
  \l 55E8
  \l 55E9
  \l 55EA
  \l 55EB
  \l 55EC
  \l 55ED
  \l 55EE
  \l 55EF
  \l 55F0
  \l 55F1
  \l 55F2
  \l 55F3
  \l 55F4
  \l 55F5
  \l 55F6
  \l 55F7
  \l 55F8
  \l 55F9
  \l 55FA
  \l 55FB
  \l 55FC
  \l 55FD
  \l 55FE
  \l 55FF
  \l 5600
  \l 5601
  \l 5602
  \l 5603
  \l 5604
  \l 5605
  \l 5606
  \l 5607
  \l 5608
  \l 5609
  \l 560A
  \l 560B
  \l 560C
  \l 560D
  \l 560E
  \l 560F
  \l 5610
  \l 5611
  \l 5612
  \l 5613
  \l 5614
  \l 5615
  \l 5616
  \l 5617
  \l 5618
  \l 5619
  \l 561A
  \l 561B
  \l 561C
  \l 561D
  \l 561E
  \l 561F
  \l 5620
  \l 5621
  \l 5622
  \l 5623
  \l 5624
  \l 5625
  \l 5626
  \l 5627
  \l 5628
  \l 5629
  \l 562A
  \l 562B
  \l 562C
  \l 562D
  \l 562E
  \l 562F
  \l 5630
  \l 5631
  \l 5632
  \l 5633
  \l 5634
  \l 5635
  \l 5636
  \l 5637
  \l 5638
  \l 5639
  \l 563A
  \l 563B
  \l 563C
  \l 563D
  \l 563E
  \l 563F
  \l 5640
  \l 5641
  \l 5642
  \l 5643
  \l 5644
  \l 5645
  \l 5646
  \l 5647
  \l 5648
  \l 5649
  \l 564A
  \l 564B
  \l 564C
  \l 564D
  \l 564E
  \l 564F
  \l 5650
  \l 5651
  \l 5652
  \l 5653
  \l 5654
  \l 5655
  \l 5656
  \l 5657
  \l 5658
  \l 5659
  \l 565A
  \l 565B
  \l 565C
  \l 565D
  \l 565E
  \l 565F
  \l 5660
  \l 5661
  \l 5662
  \l 5663
  \l 5664
  \l 5665
  \l 5666
  \l 5667
  \l 5668
  \l 5669
  \l 566A
  \l 566B
  \l 566C
  \l 566D
  \l 566E
  \l 566F
  \l 5670
  \l 5671
  \l 5672
  \l 5673
  \l 5674
  \l 5675
  \l 5676
  \l 5677
  \l 5678
  \l 5679
  \l 567A
  \l 567B
  \l 567C
  \l 567D
  \l 567E
  \l 567F
  \l 5680
  \l 5681
  \l 5682
  \l 5683
  \l 5684
  \l 5685
  \l 5686
  \l 5687
  \l 5688
  \l 5689
  \l 568A
  \l 568B
  \l 568C
  \l 568D
  \l 568E
  \l 568F
  \l 5690
  \l 5691
  \l 5692
  \l 5693
  \l 5694
  \l 5695
  \l 5696
  \l 5697
  \l 5698
  \l 5699
  \l 569A
  \l 569B
  \l 569C
  \l 569D
  \l 569E
  \l 569F
  \l 56A0
  \l 56A1
  \l 56A2
  \l 56A3
  \l 56A4
  \l 56A5
  \l 56A6
  \l 56A7
  \l 56A8
  \l 56A9
  \l 56AA
  \l 56AB
  \l 56AC
  \l 56AD
  \l 56AE
  \l 56AF
  \l 56B0
  \l 56B1
  \l 56B2
  \l 56B3
  \l 56B4
  \l 56B5
  \l 56B6
  \l 56B7
  \l 56B8
  \l 56B9
  \l 56BA
  \l 56BB
  \l 56BC
  \l 56BD
  \l 56BE
  \l 56BF
  \l 56C0
  \l 56C1
  \l 56C2
  \l 56C3
  \l 56C4
  \l 56C5
  \l 56C6
  \l 56C7
  \l 56C8
  \l 56C9
  \l 56CA
  \l 56CB
  \l 56CC
  \l 56CD
  \l 56CE
  \l 56CF
  \l 56D0
  \l 56D1
  \l 56D2
  \l 56D3
  \l 56D4
  \l 56D5
  \l 56D6
  \l 56D7
  \l 56D8
  \l 56D9
  \l 56DA
  \l 56DB
  \l 56DC
  \l 56DD
  \l 56DE
  \l 56DF
  \l 56E0
  \l 56E1
  \l 56E2
  \l 56E3
  \l 56E4
  \l 56E5
  \l 56E6
  \l 56E7
  \l 56E8
  \l 56E9
  \l 56EA
  \l 56EB
  \l 56EC
  \l 56ED
  \l 56EE
  \l 56EF
  \l 56F0
  \l 56F1
  \l 56F2
  \l 56F3
  \l 56F4
  \l 56F5
  \l 56F6
  \l 56F7
  \l 56F8
  \l 56F9
  \l 56FA
  \l 56FB
  \l 56FC
  \l 56FD
  \l 56FE
  \l 56FF
  \l 5700
  \l 5701
  \l 5702
  \l 5703
  \l 5704
  \l 5705
  \l 5706
  \l 5707
  \l 5708
  \l 5709
  \l 570A
  \l 570B
  \l 570C
  \l 570D
  \l 570E
  \l 570F
  \l 5710
  \l 5711
  \l 5712
  \l 5713
  \l 5714
  \l 5715
  \l 5716
  \l 5717
  \l 5718
  \l 5719
  \l 571A
  \l 571B
  \l 571C
  \l 571D
  \l 571E
  \l 571F
  \l 5720
  \l 5721
  \l 5722
  \l 5723
  \l 5724
  \l 5725
  \l 5726
  \l 5727
  \l 5728
  \l 5729
  \l 572A
  \l 572B
  \l 572C
  \l 572D
  \l 572E
  \l 572F
  \l 5730
  \l 5731
  \l 5732
  \l 5733
  \l 5734
  \l 5735
  \l 5736
  \l 5737
  \l 5738
  \l 5739
  \l 573A
  \l 573B
  \l 573C
  \l 573D
  \l 573E
  \l 573F
  \l 5740
  \l 5741
  \l 5742
  \l 5743
  \l 5744
  \l 5745
  \l 5746
  \l 5747
  \l 5748
  \l 5749
  \l 574A
  \l 574B
  \l 574C
  \l 574D
  \l 574E
  \l 574F
  \l 5750
  \l 5751
  \l 5752
  \l 5753
  \l 5754
  \l 5755
  \l 5756
  \l 5757
  \l 5758
  \l 5759
  \l 575A
  \l 575B
  \l 575C
  \l 575D
  \l 575E
  \l 575F
  \l 5760
  \l 5761
  \l 5762
  \l 5763
  \l 5764
  \l 5765
  \l 5766
  \l 5767
  \l 5768
  \l 5769
  \l 576A
  \l 576B
  \l 576C
  \l 576D
  \l 576E
  \l 576F
  \l 5770
  \l 5771
  \l 5772
  \l 5773
  \l 5774
  \l 5775
  \l 5776
  \l 5777
  \l 5778
  \l 5779
  \l 577A
  \l 577B
  \l 577C
  \l 577D
  \l 577E
  \l 577F
  \l 5780
  \l 5781
  \l 5782
  \l 5783
  \l 5784
  \l 5785
  \l 5786
  \l 5787
  \l 5788
  \l 5789
  \l 578A
  \l 578B
  \l 578C
  \l 578D
  \l 578E
  \l 578F
  \l 5790
  \l 5791
  \l 5792
  \l 5793
  \l 5794
  \l 5795
  \l 5796
  \l 5797
  \l 5798
  \l 5799
  \l 579A
  \l 579B
  \l 579C
  \l 579D
  \l 579E
  \l 579F
  \l 57A0
  \l 57A1
  \l 57A2
  \l 57A3
  \l 57A4
  \l 57A5
  \l 57A6
  \l 57A7
  \l 57A8
  \l 57A9
  \l 57AA
  \l 57AB
  \l 57AC
  \l 57AD
  \l 57AE
  \l 57AF
  \l 57B0
  \l 57B1
  \l 57B2
  \l 57B3
  \l 57B4
  \l 57B5
  \l 57B6
  \l 57B7
  \l 57B8
  \l 57B9
  \l 57BA
  \l 57BB
  \l 57BC
  \l 57BD
  \l 57BE
  \l 57BF
  \l 57C0
  \l 57C1
  \l 57C2
  \l 57C3
  \l 57C4
  \l 57C5
  \l 57C6
  \l 57C7
  \l 57C8
  \l 57C9
  \l 57CA
  \l 57CB
  \l 57CC
  \l 57CD
  \l 57CE
  \l 57CF
  \l 57D0
  \l 57D1
  \l 57D2
  \l 57D3
  \l 57D4
  \l 57D5
  \l 57D6
  \l 57D7
  \l 57D8
  \l 57D9
  \l 57DA
  \l 57DB
  \l 57DC
  \l 57DD
  \l 57DE
  \l 57DF
  \l 57E0
  \l 57E1
  \l 57E2
  \l 57E3
  \l 57E4
  \l 57E5
  \l 57E6
  \l 57E7
  \l 57E8
  \l 57E9
  \l 57EA
  \l 57EB
  \l 57EC
  \l 57ED
  \l 57EE
  \l 57EF
  \l 57F0
  \l 57F1
  \l 57F2
  \l 57F3
  \l 57F4
  \l 57F5
  \l 57F6
  \l 57F7
  \l 57F8
  \l 57F9
  \l 57FA
  \l 57FB
  \l 57FC
  \l 57FD
  \l 57FE
  \l 57FF
  \l 5800
  \l 5801
  \l 5802
  \l 5803
  \l 5804
  \l 5805
  \l 5806
  \l 5807
  \l 5808
  \l 5809
  \l 580A
  \l 580B
  \l 580C
  \l 580D
  \l 580E
  \l 580F
  \l 5810
  \l 5811
  \l 5812
  \l 5813
  \l 5814
  \l 5815
  \l 5816
  \l 5817
  \l 5818
  \l 5819
  \l 581A
  \l 581B
  \l 581C
  \l 581D
  \l 581E
  \l 581F
  \l 5820
  \l 5821
  \l 5822
  \l 5823
  \l 5824
  \l 5825
  \l 5826
  \l 5827
  \l 5828
  \l 5829
  \l 582A
  \l 582B
  \l 582C
  \l 582D
  \l 582E
  \l 582F
  \l 5830
  \l 5831
  \l 5832
  \l 5833
  \l 5834
  \l 5835
  \l 5836
  \l 5837
  \l 5838
  \l 5839
  \l 583A
  \l 583B
  \l 583C
  \l 583D
  \l 583E
  \l 583F
  \l 5840
  \l 5841
  \l 5842
  \l 5843
  \l 5844
  \l 5845
  \l 5846
  \l 5847
  \l 5848
  \l 5849
  \l 584A
  \l 584B
  \l 584C
  \l 584D
  \l 584E
  \l 584F
  \l 5850
  \l 5851
  \l 5852
  \l 5853
  \l 5854
  \l 5855
  \l 5856
  \l 5857
  \l 5858
  \l 5859
  \l 585A
  \l 585B
  \l 585C
  \l 585D
  \l 585E
  \l 585F
  \l 5860
  \l 5861
  \l 5862
  \l 5863
  \l 5864
  \l 5865
  \l 5866
  \l 5867
  \l 5868
  \l 5869
  \l 586A
  \l 586B
  \l 586C
  \l 586D
  \l 586E
  \l 586F
  \l 5870
  \l 5871
  \l 5872
  \l 5873
  \l 5874
  \l 5875
  \l 5876
  \l 5877
  \l 5878
  \l 5879
  \l 587A
  \l 587B
  \l 587C
  \l 587D
  \l 587E
  \l 587F
  \l 5880
  \l 5881
  \l 5882
  \l 5883
  \l 5884
  \l 5885
  \l 5886
  \l 5887
  \l 5888
  \l 5889
  \l 588A
  \l 588B
  \l 588C
  \l 588D
  \l 588E
  \l 588F
  \l 5890
  \l 5891
  \l 5892
  \l 5893
  \l 5894
  \l 5895
  \l 5896
  \l 5897
  \l 5898
  \l 5899
  \l 589A
  \l 589B
  \l 589C
  \l 589D
  \l 589E
  \l 589F
  \l 58A0
  \l 58A1
  \l 58A2
  \l 58A3
  \l 58A4
  \l 58A5
  \l 58A6
  \l 58A7
  \l 58A8
  \l 58A9
  \l 58AA
  \l 58AB
  \l 58AC
  \l 58AD
  \l 58AE
  \l 58AF
  \l 58B0
  \l 58B1
  \l 58B2
  \l 58B3
  \l 58B4
  \l 58B5
  \l 58B6
  \l 58B7
  \l 58B8
  \l 58B9
  \l 58BA
  \l 58BB
  \l 58BC
  \l 58BD
  \l 58BE
  \l 58BF
  \l 58C0
  \l 58C1
  \l 58C2
  \l 58C3
  \l 58C4
  \l 58C5
  \l 58C6
  \l 58C7
  \l 58C8
  \l 58C9
  \l 58CA
  \l 58CB
  \l 58CC
  \l 58CD
  \l 58CE
  \l 58CF
  \l 58D0
  \l 58D1
  \l 58D2
  \l 58D3
  \l 58D4
  \l 58D5
  \l 58D6
  \l 58D7
  \l 58D8
  \l 58D9
  \l 58DA
  \l 58DB
  \l 58DC
  \l 58DD
  \l 58DE
  \l 58DF
  \l 58E0
  \l 58E1
  \l 58E2
  \l 58E3
  \l 58E4
  \l 58E5
  \l 58E6
  \l 58E7
  \l 58E8
  \l 58E9
  \l 58EA
  \l 58EB
  \l 58EC
  \l 58ED
  \l 58EE
  \l 58EF
  \l 58F0
  \l 58F1
  \l 58F2
  \l 58F3
  \l 58F4
  \l 58F5
  \l 58F6
  \l 58F7
  \l 58F8
  \l 58F9
  \l 58FA
  \l 58FB
  \l 58FC
  \l 58FD
  \l 58FE
  \l 58FF
  \l 5900
  \l 5901
  \l 5902
  \l 5903
  \l 5904
  \l 5905
  \l 5906
  \l 5907
  \l 5908
  \l 5909
  \l 590A
  \l 590B
  \l 590C
  \l 590D
  \l 590E
  \l 590F
  \l 5910
  \l 5911
  \l 5912
  \l 5913
  \l 5914
  \l 5915
  \l 5916
  \l 5917
  \l 5918
  \l 5919
  \l 591A
  \l 591B
  \l 591C
  \l 591D
  \l 591E
  \l 591F
  \l 5920
  \l 5921
  \l 5922
  \l 5923
  \l 5924
  \l 5925
  \l 5926
  \l 5927
  \l 5928
  \l 5929
  \l 592A
  \l 592B
  \l 592C
  \l 592D
  \l 592E
  \l 592F
  \l 5930
  \l 5931
  \l 5932
  \l 5933
  \l 5934
  \l 5935
  \l 5936
  \l 5937
  \l 5938
  \l 5939
  \l 593A
  \l 593B
  \l 593C
  \l 593D
  \l 593E
  \l 593F
  \l 5940
  \l 5941
  \l 5942
  \l 5943
  \l 5944
  \l 5945
  \l 5946
  \l 5947
  \l 5948
  \l 5949
  \l 594A
  \l 594B
  \l 594C
  \l 594D
  \l 594E
  \l 594F
  \l 5950
  \l 5951
  \l 5952
  \l 5953
  \l 5954
  \l 5955
  \l 5956
  \l 5957
  \l 5958
  \l 5959
  \l 595A
  \l 595B
  \l 595C
  \l 595D
  \l 595E
  \l 595F
  \l 5960
  \l 5961
  \l 5962
  \l 5963
  \l 5964
  \l 5965
  \l 5966
  \l 5967
  \l 5968
  \l 5969
  \l 596A
  \l 596B
  \l 596C
  \l 596D
  \l 596E
  \l 596F
  \l 5970
  \l 5971
  \l 5972
  \l 5973
  \l 5974
  \l 5975
  \l 5976
  \l 5977
  \l 5978
  \l 5979
  \l 597A
  \l 597B
  \l 597C
  \l 597D
  \l 597E
  \l 597F
  \l 5980
  \l 5981
  \l 5982
  \l 5983
  \l 5984
  \l 5985
  \l 5986
  \l 5987
  \l 5988
  \l 5989
  \l 598A
  \l 598B
  \l 598C
  \l 598D
  \l 598E
  \l 598F
  \l 5990
  \l 5991
  \l 5992
  \l 5993
  \l 5994
  \l 5995
  \l 5996
  \l 5997
  \l 5998
  \l 5999
  \l 599A
  \l 599B
  \l 599C
  \l 599D
  \l 599E
  \l 599F
  \l 59A0
  \l 59A1
  \l 59A2
  \l 59A3
  \l 59A4
  \l 59A5
  \l 59A6
  \l 59A7
  \l 59A8
  \l 59A9
  \l 59AA
  \l 59AB
  \l 59AC
  \l 59AD
  \l 59AE
  \l 59AF
  \l 59B0
  \l 59B1
  \l 59B2
  \l 59B3
  \l 59B4
  \l 59B5
  \l 59B6
  \l 59B7
  \l 59B8
  \l 59B9
  \l 59BA
  \l 59BB
  \l 59BC
  \l 59BD
  \l 59BE
  \l 59BF
  \l 59C0
  \l 59C1
  \l 59C2
  \l 59C3
  \l 59C4
  \l 59C5
  \l 59C6
  \l 59C7
  \l 59C8
  \l 59C9
  \l 59CA
  \l 59CB
  \l 59CC
  \l 59CD
  \l 59CE
  \l 59CF
  \l 59D0
  \l 59D1
  \l 59D2
  \l 59D3
  \l 59D4
  \l 59D5
  \l 59D6
  \l 59D7
  \l 59D8
  \l 59D9
  \l 59DA
  \l 59DB
  \l 59DC
  \l 59DD
  \l 59DE
  \l 59DF
  \l 59E0
  \l 59E1
  \l 59E2
  \l 59E3
  \l 59E4
  \l 59E5
  \l 59E6
  \l 59E7
  \l 59E8
  \l 59E9
  \l 59EA
  \l 59EB
  \l 59EC
  \l 59ED
  \l 59EE
  \l 59EF
  \l 59F0
  \l 59F1
  \l 59F2
  \l 59F3
  \l 59F4
  \l 59F5
  \l 59F6
  \l 59F7
  \l 59F8
  \l 59F9
  \l 59FA
  \l 59FB
  \l 59FC
  \l 59FD
  \l 59FE
  \l 59FF
  \l 5A00
  \l 5A01
  \l 5A02
  \l 5A03
  \l 5A04
  \l 5A05
  \l 5A06
  \l 5A07
  \l 5A08
  \l 5A09
  \l 5A0A
  \l 5A0B
  \l 5A0C
  \l 5A0D
  \l 5A0E
  \l 5A0F
  \l 5A10
  \l 5A11
  \l 5A12
  \l 5A13
  \l 5A14
  \l 5A15
  \l 5A16
  \l 5A17
  \l 5A18
  \l 5A19
  \l 5A1A
  \l 5A1B
  \l 5A1C
  \l 5A1D
  \l 5A1E
  \l 5A1F
  \l 5A20
  \l 5A21
  \l 5A22
  \l 5A23
  \l 5A24
  \l 5A25
  \l 5A26
  \l 5A27
  \l 5A28
  \l 5A29
  \l 5A2A
  \l 5A2B
  \l 5A2C
  \l 5A2D
  \l 5A2E
  \l 5A2F
  \l 5A30
  \l 5A31
  \l 5A32
  \l 5A33
  \l 5A34
  \l 5A35
  \l 5A36
  \l 5A37
  \l 5A38
  \l 5A39
  \l 5A3A
  \l 5A3B
  \l 5A3C
  \l 5A3D
  \l 5A3E
  \l 5A3F
  \l 5A40
  \l 5A41
  \l 5A42
  \l 5A43
  \l 5A44
  \l 5A45
  \l 5A46
  \l 5A47
  \l 5A48
  \l 5A49
  \l 5A4A
  \l 5A4B
  \l 5A4C
  \l 5A4D
  \l 5A4E
  \l 5A4F
  \l 5A50
  \l 5A51
  \l 5A52
  \l 5A53
  \l 5A54
  \l 5A55
  \l 5A56
  \l 5A57
  \l 5A58
  \l 5A59
  \l 5A5A
  \l 5A5B
  \l 5A5C
  \l 5A5D
  \l 5A5E
  \l 5A5F
  \l 5A60
  \l 5A61
  \l 5A62
  \l 5A63
  \l 5A64
  \l 5A65
  \l 5A66
  \l 5A67
  \l 5A68
  \l 5A69
  \l 5A6A
  \l 5A6B
  \l 5A6C
  \l 5A6D
  \l 5A6E
  \l 5A6F
  \l 5A70
  \l 5A71
  \l 5A72
  \l 5A73
  \l 5A74
  \l 5A75
  \l 5A76
  \l 5A77
  \l 5A78
  \l 5A79
  \l 5A7A
  \l 5A7B
  \l 5A7C
  \l 5A7D
  \l 5A7E
  \l 5A7F
  \l 5A80
  \l 5A81
  \l 5A82
  \l 5A83
  \l 5A84
  \l 5A85
  \l 5A86
  \l 5A87
  \l 5A88
  \l 5A89
  \l 5A8A
  \l 5A8B
  \l 5A8C
  \l 5A8D
  \l 5A8E
  \l 5A8F
  \l 5A90
  \l 5A91
  \l 5A92
  \l 5A93
  \l 5A94
  \l 5A95
  \l 5A96
  \l 5A97
  \l 5A98
  \l 5A99
  \l 5A9A
  \l 5A9B
  \l 5A9C
  \l 5A9D
  \l 5A9E
  \l 5A9F
  \l 5AA0
  \l 5AA1
  \l 5AA2
  \l 5AA3
  \l 5AA4
  \l 5AA5
  \l 5AA6
  \l 5AA7
  \l 5AA8
  \l 5AA9
  \l 5AAA
  \l 5AAB
  \l 5AAC
  \l 5AAD
  \l 5AAE
  \l 5AAF
  \l 5AB0
  \l 5AB1
  \l 5AB2
  \l 5AB3
  \l 5AB4
  \l 5AB5
  \l 5AB6
  \l 5AB7
  \l 5AB8
  \l 5AB9
  \l 5ABA
  \l 5ABB
  \l 5ABC
  \l 5ABD
  \l 5ABE
  \l 5ABF
  \l 5AC0
  \l 5AC1
  \l 5AC2
  \l 5AC3
  \l 5AC4
  \l 5AC5
  \l 5AC6
  \l 5AC7
  \l 5AC8
  \l 5AC9
  \l 5ACA
  \l 5ACB
  \l 5ACC
  \l 5ACD
  \l 5ACE
  \l 5ACF
  \l 5AD0
  \l 5AD1
  \l 5AD2
  \l 5AD3
  \l 5AD4
  \l 5AD5
  \l 5AD6
  \l 5AD7
  \l 5AD8
  \l 5AD9
  \l 5ADA
  \l 5ADB
  \l 5ADC
  \l 5ADD
  \l 5ADE
  \l 5ADF
  \l 5AE0
  \l 5AE1
  \l 5AE2
  \l 5AE3
  \l 5AE4
  \l 5AE5
  \l 5AE6
  \l 5AE7
  \l 5AE8
  \l 5AE9
  \l 5AEA
  \l 5AEB
  \l 5AEC
  \l 5AED
  \l 5AEE
  \l 5AEF
  \l 5AF0
  \l 5AF1
  \l 5AF2
  \l 5AF3
  \l 5AF4
  \l 5AF5
  \l 5AF6
  \l 5AF7
  \l 5AF8
  \l 5AF9
  \l 5AFA
  \l 5AFB
  \l 5AFC
  \l 5AFD
  \l 5AFE
  \l 5AFF
  \l 5B00
  \l 5B01
  \l 5B02
  \l 5B03
  \l 5B04
  \l 5B05
  \l 5B06
  \l 5B07
  \l 5B08
  \l 5B09
  \l 5B0A
  \l 5B0B
  \l 5B0C
  \l 5B0D
  \l 5B0E
  \l 5B0F
  \l 5B10
  \l 5B11
  \l 5B12
  \l 5B13
  \l 5B14
  \l 5B15
  \l 5B16
  \l 5B17
  \l 5B18
  \l 5B19
  \l 5B1A
  \l 5B1B
  \l 5B1C
  \l 5B1D
  \l 5B1E
  \l 5B1F
  \l 5B20
  \l 5B21
  \l 5B22
  \l 5B23
  \l 5B24
  \l 5B25
  \l 5B26
  \l 5B27
  \l 5B28
  \l 5B29
  \l 5B2A
  \l 5B2B
  \l 5B2C
  \l 5B2D
  \l 5B2E
  \l 5B2F
  \l 5B30
  \l 5B31
  \l 5B32
  \l 5B33
  \l 5B34
  \l 5B35
  \l 5B36
  \l 5B37
  \l 5B38
  \l 5B39
  \l 5B3A
  \l 5B3B
  \l 5B3C
  \l 5B3D
  \l 5B3E
  \l 5B3F
  \l 5B40
  \l 5B41
  \l 5B42
  \l 5B43
  \l 5B44
  \l 5B45
  \l 5B46
  \l 5B47
  \l 5B48
  \l 5B49
  \l 5B4A
  \l 5B4B
  \l 5B4C
  \l 5B4D
  \l 5B4E
  \l 5B4F
  \l 5B50
  \l 5B51
  \l 5B52
  \l 5B53
  \l 5B54
  \l 5B55
  \l 5B56
  \l 5B57
  \l 5B58
  \l 5B59
  \l 5B5A
  \l 5B5B
  \l 5B5C
  \l 5B5D
  \l 5B5E
  \l 5B5F
  \l 5B60
  \l 5B61
  \l 5B62
  \l 5B63
  \l 5B64
  \l 5B65
  \l 5B66
  \l 5B67
  \l 5B68
  \l 5B69
  \l 5B6A
  \l 5B6B
  \l 5B6C
  \l 5B6D
  \l 5B6E
  \l 5B6F
  \l 5B70
  \l 5B71
  \l 5B72
  \l 5B73
  \l 5B74
  \l 5B75
  \l 5B76
  \l 5B77
  \l 5B78
  \l 5B79
  \l 5B7A
  \l 5B7B
  \l 5B7C
  \l 5B7D
  \l 5B7E
  \l 5B7F
  \l 5B80
  \l 5B81
  \l 5B82
  \l 5B83
  \l 5B84
  \l 5B85
  \l 5B86
  \l 5B87
  \l 5B88
  \l 5B89
  \l 5B8A
  \l 5B8B
  \l 5B8C
  \l 5B8D
  \l 5B8E
  \l 5B8F
  \l 5B90
  \l 5B91
  \l 5B92
  \l 5B93
  \l 5B94
  \l 5B95
  \l 5B96
  \l 5B97
  \l 5B98
  \l 5B99
  \l 5B9A
  \l 5B9B
  \l 5B9C
  \l 5B9D
  \l 5B9E
  \l 5B9F
  \l 5BA0
  \l 5BA1
  \l 5BA2
  \l 5BA3
  \l 5BA4
  \l 5BA5
  \l 5BA6
  \l 5BA7
  \l 5BA8
  \l 5BA9
  \l 5BAA
  \l 5BAB
  \l 5BAC
  \l 5BAD
  \l 5BAE
  \l 5BAF
  \l 5BB0
  \l 5BB1
  \l 5BB2
  \l 5BB3
  \l 5BB4
  \l 5BB5
  \l 5BB6
  \l 5BB7
  \l 5BB8
  \l 5BB9
  \l 5BBA
  \l 5BBB
  \l 5BBC
  \l 5BBD
  \l 5BBE
  \l 5BBF
  \l 5BC0
  \l 5BC1
  \l 5BC2
  \l 5BC3
  \l 5BC4
  \l 5BC5
  \l 5BC6
  \l 5BC7
  \l 5BC8
  \l 5BC9
  \l 5BCA
  \l 5BCB
  \l 5BCC
  \l 5BCD
  \l 5BCE
  \l 5BCF
  \l 5BD0
  \l 5BD1
  \l 5BD2
  \l 5BD3
  \l 5BD4
  \l 5BD5
  \l 5BD6
  \l 5BD7
  \l 5BD8
  \l 5BD9
  \l 5BDA
  \l 5BDB
  \l 5BDC
  \l 5BDD
  \l 5BDE
  \l 5BDF
  \l 5BE0
  \l 5BE1
  \l 5BE2
  \l 5BE3
  \l 5BE4
  \l 5BE5
  \l 5BE6
  \l 5BE7
  \l 5BE8
  \l 5BE9
  \l 5BEA
  \l 5BEB
  \l 5BEC
  \l 5BED
  \l 5BEE
  \l 5BEF
  \l 5BF0
  \l 5BF1
  \l 5BF2
  \l 5BF3
  \l 5BF4
  \l 5BF5
  \l 5BF6
  \l 5BF7
  \l 5BF8
  \l 5BF9
  \l 5BFA
  \l 5BFB
  \l 5BFC
  \l 5BFD
  \l 5BFE
  \l 5BFF
  \l 5C00
  \l 5C01
  \l 5C02
  \l 5C03
  \l 5C04
  \l 5C05
  \l 5C06
  \l 5C07
  \l 5C08
  \l 5C09
  \l 5C0A
  \l 5C0B
  \l 5C0C
  \l 5C0D
  \l 5C0E
  \l 5C0F
  \l 5C10
  \l 5C11
  \l 5C12
  \l 5C13
  \l 5C14
  \l 5C15
  \l 5C16
  \l 5C17
  \l 5C18
  \l 5C19
  \l 5C1A
  \l 5C1B
  \l 5C1C
  \l 5C1D
  \l 5C1E
  \l 5C1F
  \l 5C20
  \l 5C21
  \l 5C22
  \l 5C23
  \l 5C24
  \l 5C25
  \l 5C26
  \l 5C27
  \l 5C28
  \l 5C29
  \l 5C2A
  \l 5C2B
  \l 5C2C
  \l 5C2D
  \l 5C2E
  \l 5C2F
  \l 5C30
  \l 5C31
  \l 5C32
  \l 5C33
  \l 5C34
  \l 5C35
  \l 5C36
  \l 5C37
  \l 5C38
  \l 5C39
  \l 5C3A
  \l 5C3B
  \l 5C3C
  \l 5C3D
  \l 5C3E
  \l 5C3F
  \l 5C40
  \l 5C41
  \l 5C42
  \l 5C43
  \l 5C44
  \l 5C45
  \l 5C46
  \l 5C47
  \l 5C48
  \l 5C49
  \l 5C4A
  \l 5C4B
  \l 5C4C
  \l 5C4D
  \l 5C4E
  \l 5C4F
  \l 5C50
  \l 5C51
  \l 5C52
  \l 5C53
  \l 5C54
  \l 5C55
  \l 5C56
  \l 5C57
  \l 5C58
  \l 5C59
  \l 5C5A
  \l 5C5B
  \l 5C5C
  \l 5C5D
  \l 5C5E
  \l 5C5F
  \l 5C60
  \l 5C61
  \l 5C62
  \l 5C63
  \l 5C64
  \l 5C65
  \l 5C66
  \l 5C67
  \l 5C68
  \l 5C69
  \l 5C6A
  \l 5C6B
  \l 5C6C
  \l 5C6D
  \l 5C6E
  \l 5C6F
  \l 5C70
  \l 5C71
  \l 5C72
  \l 5C73
  \l 5C74
  \l 5C75
  \l 5C76
  \l 5C77
  \l 5C78
  \l 5C79
  \l 5C7A
  \l 5C7B
  \l 5C7C
  \l 5C7D
  \l 5C7E
  \l 5C7F
  \l 5C80
  \l 5C81
  \l 5C82
  \l 5C83
  \l 5C84
  \l 5C85
  \l 5C86
  \l 5C87
  \l 5C88
  \l 5C89
  \l 5C8A
  \l 5C8B
  \l 5C8C
  \l 5C8D
  \l 5C8E
  \l 5C8F
  \l 5C90
  \l 5C91
  \l 5C92
  \l 5C93
  \l 5C94
  \l 5C95
  \l 5C96
  \l 5C97
  \l 5C98
  \l 5C99
  \l 5C9A
  \l 5C9B
  \l 5C9C
  \l 5C9D
  \l 5C9E
  \l 5C9F
  \l 5CA0
  \l 5CA1
  \l 5CA2
  \l 5CA3
  \l 5CA4
  \l 5CA5
  \l 5CA6
  \l 5CA7
  \l 5CA8
  \l 5CA9
  \l 5CAA
  \l 5CAB
  \l 5CAC
  \l 5CAD
  \l 5CAE
  \l 5CAF
  \l 5CB0
  \l 5CB1
  \l 5CB2
  \l 5CB3
  \l 5CB4
  \l 5CB5
  \l 5CB6
  \l 5CB7
  \l 5CB8
  \l 5CB9
  \l 5CBA
  \l 5CBB
  \l 5CBC
  \l 5CBD
  \l 5CBE
  \l 5CBF
  \l 5CC0
  \l 5CC1
  \l 5CC2
  \l 5CC3
  \l 5CC4
  \l 5CC5
  \l 5CC6
  \l 5CC7
  \l 5CC8
  \l 5CC9
  \l 5CCA
  \l 5CCB
  \l 5CCC
  \l 5CCD
  \l 5CCE
  \l 5CCF
  \l 5CD0
  \l 5CD1
  \l 5CD2
  \l 5CD3
  \l 5CD4
  \l 5CD5
  \l 5CD6
  \l 5CD7
  \l 5CD8
  \l 5CD9
  \l 5CDA
  \l 5CDB
  \l 5CDC
  \l 5CDD
  \l 5CDE
  \l 5CDF
  \l 5CE0
  \l 5CE1
  \l 5CE2
  \l 5CE3
  \l 5CE4
  \l 5CE5
  \l 5CE6
  \l 5CE7
  \l 5CE8
  \l 5CE9
  \l 5CEA
  \l 5CEB
  \l 5CEC
  \l 5CED
  \l 5CEE
  \l 5CEF
  \l 5CF0
  \l 5CF1
  \l 5CF2
  \l 5CF3
  \l 5CF4
  \l 5CF5
  \l 5CF6
  \l 5CF7
  \l 5CF8
  \l 5CF9
  \l 5CFA
  \l 5CFB
  \l 5CFC
  \l 5CFD
  \l 5CFE
  \l 5CFF
  \l 5D00
  \l 5D01
  \l 5D02
  \l 5D03
  \l 5D04
  \l 5D05
  \l 5D06
  \l 5D07
  \l 5D08
  \l 5D09
  \l 5D0A
  \l 5D0B
  \l 5D0C
  \l 5D0D
  \l 5D0E
  \l 5D0F
  \l 5D10
  \l 5D11
  \l 5D12
  \l 5D13
  \l 5D14
  \l 5D15
  \l 5D16
  \l 5D17
  \l 5D18
  \l 5D19
  \l 5D1A
  \l 5D1B
  \l 5D1C
  \l 5D1D
  \l 5D1E
  \l 5D1F
  \l 5D20
  \l 5D21
  \l 5D22
  \l 5D23
  \l 5D24
  \l 5D25
  \l 5D26
  \l 5D27
  \l 5D28
  \l 5D29
  \l 5D2A
  \l 5D2B
  \l 5D2C
  \l 5D2D
  \l 5D2E
  \l 5D2F
  \l 5D30
  \l 5D31
  \l 5D32
  \l 5D33
  \l 5D34
  \l 5D35
  \l 5D36
  \l 5D37
  \l 5D38
  \l 5D39
  \l 5D3A
  \l 5D3B
  \l 5D3C
  \l 5D3D
  \l 5D3E
  \l 5D3F
  \l 5D40
  \l 5D41
  \l 5D42
  \l 5D43
  \l 5D44
  \l 5D45
  \l 5D46
  \l 5D47
  \l 5D48
  \l 5D49
  \l 5D4A
  \l 5D4B
  \l 5D4C
  \l 5D4D
  \l 5D4E
  \l 5D4F
  \l 5D50
  \l 5D51
  \l 5D52
  \l 5D53
  \l 5D54
  \l 5D55
  \l 5D56
  \l 5D57
  \l 5D58
  \l 5D59
  \l 5D5A
  \l 5D5B
  \l 5D5C
  \l 5D5D
  \l 5D5E
  \l 5D5F
  \l 5D60
  \l 5D61
  \l 5D62
  \l 5D63
  \l 5D64
  \l 5D65
  \l 5D66
  \l 5D67
  \l 5D68
  \l 5D69
  \l 5D6A
  \l 5D6B
  \l 5D6C
  \l 5D6D
  \l 5D6E
  \l 5D6F
  \l 5D70
  \l 5D71
  \l 5D72
  \l 5D73
  \l 5D74
  \l 5D75
  \l 5D76
  \l 5D77
  \l 5D78
  \l 5D79
  \l 5D7A
  \l 5D7B
  \l 5D7C
  \l 5D7D
  \l 5D7E
  \l 5D7F
  \l 5D80
  \l 5D81
  \l 5D82
  \l 5D83
  \l 5D84
  \l 5D85
  \l 5D86
  \l 5D87
  \l 5D88
  \l 5D89
  \l 5D8A
  \l 5D8B
  \l 5D8C
  \l 5D8D
  \l 5D8E
  \l 5D8F
  \l 5D90
  \l 5D91
  \l 5D92
  \l 5D93
  \l 5D94
  \l 5D95
  \l 5D96
  \l 5D97
  \l 5D98
  \l 5D99
  \l 5D9A
  \l 5D9B
  \l 5D9C
  \l 5D9D
  \l 5D9E
  \l 5D9F
  \l 5DA0
  \l 5DA1
  \l 5DA2
  \l 5DA3
  \l 5DA4
  \l 5DA5
  \l 5DA6
  \l 5DA7
  \l 5DA8
  \l 5DA9
  \l 5DAA
  \l 5DAB
  \l 5DAC
  \l 5DAD
  \l 5DAE
  \l 5DAF
  \l 5DB0
  \l 5DB1
  \l 5DB2
  \l 5DB3
  \l 5DB4
  \l 5DB5
  \l 5DB6
  \l 5DB7
  \l 5DB8
  \l 5DB9
  \l 5DBA
  \l 5DBB
  \l 5DBC
  \l 5DBD
  \l 5DBE
  \l 5DBF
  \l 5DC0
  \l 5DC1
  \l 5DC2
  \l 5DC3
  \l 5DC4
  \l 5DC5
  \l 5DC6
  \l 5DC7
  \l 5DC8
  \l 5DC9
  \l 5DCA
  \l 5DCB
  \l 5DCC
  \l 5DCD
  \l 5DCE
  \l 5DCF
  \l 5DD0
  \l 5DD1
  \l 5DD2
  \l 5DD3
  \l 5DD4
  \l 5DD5
  \l 5DD6
  \l 5DD7
  \l 5DD8
  \l 5DD9
  \l 5DDA
  \l 5DDB
  \l 5DDC
  \l 5DDD
  \l 5DDE
  \l 5DDF
  \l 5DE0
  \l 5DE1
  \l 5DE2
  \l 5DE3
  \l 5DE4
  \l 5DE5
  \l 5DE6
  \l 5DE7
  \l 5DE8
  \l 5DE9
  \l 5DEA
  \l 5DEB
  \l 5DEC
  \l 5DED
  \l 5DEE
  \l 5DEF
  \l 5DF0
  \l 5DF1
  \l 5DF2
  \l 5DF3
  \l 5DF4
  \l 5DF5
  \l 5DF6
  \l 5DF7
  \l 5DF8
  \l 5DF9
  \l 5DFA
  \l 5DFB
  \l 5DFC
  \l 5DFD
  \l 5DFE
  \l 5DFF
  \l 5E00
  \l 5E01
  \l 5E02
  \l 5E03
  \l 5E04
  \l 5E05
  \l 5E06
  \l 5E07
  \l 5E08
  \l 5E09
  \l 5E0A
  \l 5E0B
  \l 5E0C
  \l 5E0D
  \l 5E0E
  \l 5E0F
  \l 5E10
  \l 5E11
  \l 5E12
  \l 5E13
  \l 5E14
  \l 5E15
  \l 5E16
  \l 5E17
  \l 5E18
  \l 5E19
  \l 5E1A
  \l 5E1B
  \l 5E1C
  \l 5E1D
  \l 5E1E
  \l 5E1F
  \l 5E20
  \l 5E21
  \l 5E22
  \l 5E23
  \l 5E24
  \l 5E25
  \l 5E26
  \l 5E27
  \l 5E28
  \l 5E29
  \l 5E2A
  \l 5E2B
  \l 5E2C
  \l 5E2D
  \l 5E2E
  \l 5E2F
  \l 5E30
  \l 5E31
  \l 5E32
  \l 5E33
  \l 5E34
  \l 5E35
  \l 5E36
  \l 5E37
  \l 5E38
  \l 5E39
  \l 5E3A
  \l 5E3B
  \l 5E3C
  \l 5E3D
  \l 5E3E
  \l 5E3F
  \l 5E40
  \l 5E41
  \l 5E42
  \l 5E43
  \l 5E44
  \l 5E45
  \l 5E46
  \l 5E47
  \l 5E48
  \l 5E49
  \l 5E4A
  \l 5E4B
  \l 5E4C
  \l 5E4D
  \l 5E4E
  \l 5E4F
  \l 5E50
  \l 5E51
  \l 5E52
  \l 5E53
  \l 5E54
  \l 5E55
  \l 5E56
  \l 5E57
  \l 5E58
  \l 5E59
  \l 5E5A
  \l 5E5B
  \l 5E5C
  \l 5E5D
  \l 5E5E
  \l 5E5F
  \l 5E60
  \l 5E61
  \l 5E62
  \l 5E63
  \l 5E64
  \l 5E65
  \l 5E66
  \l 5E67
  \l 5E68
  \l 5E69
  \l 5E6A
  \l 5E6B
  \l 5E6C
  \l 5E6D
  \l 5E6E
  \l 5E6F
  \l 5E70
  \l 5E71
  \l 5E72
  \l 5E73
  \l 5E74
  \l 5E75
  \l 5E76
  \l 5E77
  \l 5E78
  \l 5E79
  \l 5E7A
  \l 5E7B
  \l 5E7C
  \l 5E7D
  \l 5E7E
  \l 5E7F
  \l 5E80
  \l 5E81
  \l 5E82
  \l 5E83
  \l 5E84
  \l 5E85
  \l 5E86
  \l 5E87
  \l 5E88
  \l 5E89
  \l 5E8A
  \l 5E8B
  \l 5E8C
  \l 5E8D
  \l 5E8E
  \l 5E8F
  \l 5E90
  \l 5E91
  \l 5E92
  \l 5E93
  \l 5E94
  \l 5E95
  \l 5E96
  \l 5E97
  \l 5E98
  \l 5E99
  \l 5E9A
  \l 5E9B
  \l 5E9C
  \l 5E9D
  \l 5E9E
  \l 5E9F
  \l 5EA0
  \l 5EA1
  \l 5EA2
  \l 5EA3
  \l 5EA4
  \l 5EA5
  \l 5EA6
  \l 5EA7
  \l 5EA8
  \l 5EA9
  \l 5EAA
  \l 5EAB
  \l 5EAC
  \l 5EAD
  \l 5EAE
  \l 5EAF
  \l 5EB0
  \l 5EB1
  \l 5EB2
  \l 5EB3
  \l 5EB4
  \l 5EB5
  \l 5EB6
  \l 5EB7
  \l 5EB8
  \l 5EB9
  \l 5EBA
  \l 5EBB
  \l 5EBC
  \l 5EBD
  \l 5EBE
  \l 5EBF
  \l 5EC0
  \l 5EC1
  \l 5EC2
  \l 5EC3
  \l 5EC4
  \l 5EC5
  \l 5EC6
  \l 5EC7
  \l 5EC8
  \l 5EC9
  \l 5ECA
  \l 5ECB
  \l 5ECC
  \l 5ECD
  \l 5ECE
  \l 5ECF
  \l 5ED0
  \l 5ED1
  \l 5ED2
  \l 5ED3
  \l 5ED4
  \l 5ED5
  \l 5ED6
  \l 5ED7
  \l 5ED8
  \l 5ED9
  \l 5EDA
  \l 5EDB
  \l 5EDC
  \l 5EDD
  \l 5EDE
  \l 5EDF
  \l 5EE0
  \l 5EE1
  \l 5EE2
  \l 5EE3
  \l 5EE4
  \l 5EE5
  \l 5EE6
  \l 5EE7
  \l 5EE8
  \l 5EE9
  \l 5EEA
  \l 5EEB
  \l 5EEC
  \l 5EED
  \l 5EEE
  \l 5EEF
  \l 5EF0
  \l 5EF1
  \l 5EF2
  \l 5EF3
  \l 5EF4
  \l 5EF5
  \l 5EF6
  \l 5EF7
  \l 5EF8
  \l 5EF9
  \l 5EFA
  \l 5EFB
  \l 5EFC
  \l 5EFD
  \l 5EFE
  \l 5EFF
  \l 5F00
  \l 5F01
  \l 5F02
  \l 5F03
  \l 5F04
  \l 5F05
  \l 5F06
  \l 5F07
  \l 5F08
  \l 5F09
  \l 5F0A
  \l 5F0B
  \l 5F0C
  \l 5F0D
  \l 5F0E
  \l 5F0F
  \l 5F10
  \l 5F11
  \l 5F12
  \l 5F13
  \l 5F14
  \l 5F15
  \l 5F16
  \l 5F17
  \l 5F18
  \l 5F19
  \l 5F1A
  \l 5F1B
  \l 5F1C
  \l 5F1D
  \l 5F1E
  \l 5F1F
  \l 5F20
  \l 5F21
  \l 5F22
  \l 5F23
  \l 5F24
  \l 5F25
  \l 5F26
  \l 5F27
  \l 5F28
  \l 5F29
  \l 5F2A
  \l 5F2B
  \l 5F2C
  \l 5F2D
  \l 5F2E
  \l 5F2F
  \l 5F30
  \l 5F31
  \l 5F32
  \l 5F33
  \l 5F34
  \l 5F35
  \l 5F36
  \l 5F37
  \l 5F38
  \l 5F39
  \l 5F3A
  \l 5F3B
  \l 5F3C
  \l 5F3D
  \l 5F3E
  \l 5F3F
  \l 5F40
  \l 5F41
  \l 5F42
  \l 5F43
  \l 5F44
  \l 5F45
  \l 5F46
  \l 5F47
  \l 5F48
  \l 5F49
  \l 5F4A
  \l 5F4B
  \l 5F4C
  \l 5F4D
  \l 5F4E
  \l 5F4F
  \l 5F50
  \l 5F51
  \l 5F52
  \l 5F53
  \l 5F54
  \l 5F55
  \l 5F56
  \l 5F57
  \l 5F58
  \l 5F59
  \l 5F5A
  \l 5F5B
  \l 5F5C
  \l 5F5D
  \l 5F5E
  \l 5F5F
  \l 5F60
  \l 5F61
  \l 5F62
  \l 5F63
  \l 5F64
  \l 5F65
  \l 5F66
  \l 5F67
  \l 5F68
  \l 5F69
  \l 5F6A
  \l 5F6B
  \l 5F6C
  \l 5F6D
  \l 5F6E
  \l 5F6F
  \l 5F70
  \l 5F71
  \l 5F72
  \l 5F73
  \l 5F74
  \l 5F75
  \l 5F76
  \l 5F77
  \l 5F78
  \l 5F79
  \l 5F7A
  \l 5F7B
  \l 5F7C
  \l 5F7D
  \l 5F7E
  \l 5F7F
  \l 5F80
  \l 5F81
  \l 5F82
  \l 5F83
  \l 5F84
  \l 5F85
  \l 5F86
  \l 5F87
  \l 5F88
  \l 5F89
  \l 5F8A
  \l 5F8B
  \l 5F8C
  \l 5F8D
  \l 5F8E
  \l 5F8F
  \l 5F90
  \l 5F91
  \l 5F92
  \l 5F93
  \l 5F94
  \l 5F95
  \l 5F96
  \l 5F97
  \l 5F98
  \l 5F99
  \l 5F9A
  \l 5F9B
  \l 5F9C
  \l 5F9D
  \l 5F9E
  \l 5F9F
  \l 5FA0
  \l 5FA1
  \l 5FA2
  \l 5FA3
  \l 5FA4
  \l 5FA5
  \l 5FA6
  \l 5FA7
  \l 5FA8
  \l 5FA9
  \l 5FAA
  \l 5FAB
  \l 5FAC
  \l 5FAD
  \l 5FAE
  \l 5FAF
  \l 5FB0
  \l 5FB1
  \l 5FB2
  \l 5FB3
  \l 5FB4
  \l 5FB5
  \l 5FB6
  \l 5FB7
  \l 5FB8
  \l 5FB9
  \l 5FBA
  \l 5FBB
  \l 5FBC
  \l 5FBD
  \l 5FBE
  \l 5FBF
  \l 5FC0
  \l 5FC1
  \l 5FC2
  \l 5FC3
  \l 5FC4
  \l 5FC5
  \l 5FC6
  \l 5FC7
  \l 5FC8
  \l 5FC9
  \l 5FCA
  \l 5FCB
  \l 5FCC
  \l 5FCD
  \l 5FCE
  \l 5FCF
  \l 5FD0
  \l 5FD1
  \l 5FD2
  \l 5FD3
  \l 5FD4
  \l 5FD5
  \l 5FD6
  \l 5FD7
  \l 5FD8
  \l 5FD9
  \l 5FDA
  \l 5FDB
  \l 5FDC
  \l 5FDD
  \l 5FDE
  \l 5FDF
  \l 5FE0
  \l 5FE1
  \l 5FE2
  \l 5FE3
  \l 5FE4
  \l 5FE5
  \l 5FE6
  \l 5FE7
  \l 5FE8
  \l 5FE9
  \l 5FEA
  \l 5FEB
  \l 5FEC
  \l 5FED
  \l 5FEE
  \l 5FEF
  \l 5FF0
  \l 5FF1
  \l 5FF2
  \l 5FF3
  \l 5FF4
  \l 5FF5
  \l 5FF6
  \l 5FF7
  \l 5FF8
  \l 5FF9
  \l 5FFA
  \l 5FFB
  \l 5FFC
  \l 5FFD
  \l 5FFE
  \l 5FFF
  \l 6000
  \l 6001
  \l 6002
  \l 6003
  \l 6004
  \l 6005
  \l 6006
  \l 6007
  \l 6008
  \l 6009
  \l 600A
  \l 600B
  \l 600C
  \l 600D
  \l 600E
  \l 600F
  \l 6010
  \l 6011
  \l 6012
  \l 6013
  \l 6014
  \l 6015
  \l 6016
  \l 6017
  \l 6018
  \l 6019
  \l 601A
  \l 601B
  \l 601C
  \l 601D
  \l 601E
  \l 601F
  \l 6020
  \l 6021
  \l 6022
  \l 6023
  \l 6024
  \l 6025
  \l 6026
  \l 6027
  \l 6028
  \l 6029
  \l 602A
  \l 602B
  \l 602C
  \l 602D
  \l 602E
  \l 602F
  \l 6030
  \l 6031
  \l 6032
  \l 6033
  \l 6034
  \l 6035
  \l 6036
  \l 6037
  \l 6038
  \l 6039
  \l 603A
  \l 603B
  \l 603C
  \l 603D
  \l 603E
  \l 603F
  \l 6040
  \l 6041
  \l 6042
  \l 6043
  \l 6044
  \l 6045
  \l 6046
  \l 6047
  \l 6048
  \l 6049
  \l 604A
  \l 604B
  \l 604C
  \l 604D
  \l 604E
  \l 604F
  \l 6050
  \l 6051
  \l 6052
  \l 6053
  \l 6054
  \l 6055
  \l 6056
  \l 6057
  \l 6058
  \l 6059
  \l 605A
  \l 605B
  \l 605C
  \l 605D
  \l 605E
  \l 605F
  \l 6060
  \l 6061
  \l 6062
  \l 6063
  \l 6064
  \l 6065
  \l 6066
  \l 6067
  \l 6068
  \l 6069
  \l 606A
  \l 606B
  \l 606C
  \l 606D
  \l 606E
  \l 606F
  \l 6070
  \l 6071
  \l 6072
  \l 6073
  \l 6074
  \l 6075
  \l 6076
  \l 6077
  \l 6078
  \l 6079
  \l 607A
  \l 607B
  \l 607C
  \l 607D
  \l 607E
  \l 607F
  \l 6080
  \l 6081
  \l 6082
  \l 6083
  \l 6084
  \l 6085
  \l 6086
  \l 6087
  \l 6088
  \l 6089
  \l 608A
  \l 608B
  \l 608C
  \l 608D
  \l 608E
  \l 608F
  \l 6090
  \l 6091
  \l 6092
  \l 6093
  \l 6094
  \l 6095
  \l 6096
  \l 6097
  \l 6098
  \l 6099
  \l 609A
  \l 609B
  \l 609C
  \l 609D
  \l 609E
  \l 609F
  \l 60A0
  \l 60A1
  \l 60A2
  \l 60A3
  \l 60A4
  \l 60A5
  \l 60A6
  \l 60A7
  \l 60A8
  \l 60A9
  \l 60AA
  \l 60AB
  \l 60AC
  \l 60AD
  \l 60AE
  \l 60AF
  \l 60B0
  \l 60B1
  \l 60B2
  \l 60B3
  \l 60B4
  \l 60B5
  \l 60B6
  \l 60B7
  \l 60B8
  \l 60B9
  \l 60BA
  \l 60BB
  \l 60BC
  \l 60BD
  \l 60BE
  \l 60BF
  \l 60C0
  \l 60C1
  \l 60C2
  \l 60C3
  \l 60C4
  \l 60C5
  \l 60C6
  \l 60C7
  \l 60C8
  \l 60C9
  \l 60CA
  \l 60CB
  \l 60CC
  \l 60CD
  \l 60CE
  \l 60CF
  \l 60D0
  \l 60D1
  \l 60D2
  \l 60D3
  \l 60D4
  \l 60D5
  \l 60D6
  \l 60D7
  \l 60D8
  \l 60D9
  \l 60DA
  \l 60DB
  \l 60DC
  \l 60DD
  \l 60DE
  \l 60DF
  \l 60E0
  \l 60E1
  \l 60E2
  \l 60E3
  \l 60E4
  \l 60E5
  \l 60E6
  \l 60E7
  \l 60E8
  \l 60E9
  \l 60EA
  \l 60EB
  \l 60EC
  \l 60ED
  \l 60EE
  \l 60EF
  \l 60F0
  \l 60F1
  \l 60F2
  \l 60F3
  \l 60F4
  \l 60F5
  \l 60F6
  \l 60F7
  \l 60F8
  \l 60F9
  \l 60FA
  \l 60FB
  \l 60FC
  \l 60FD
  \l 60FE
  \l 60FF
  \l 6100
  \l 6101
  \l 6102
  \l 6103
  \l 6104
  \l 6105
  \l 6106
  \l 6107
  \l 6108
  \l 6109
  \l 610A
  \l 610B
  \l 610C
  \l 610D
  \l 610E
  \l 610F
  \l 6110
  \l 6111
  \l 6112
  \l 6113
  \l 6114
  \l 6115
  \l 6116
  \l 6117
  \l 6118
  \l 6119
  \l 611A
  \l 611B
  \l 611C
  \l 611D
  \l 611E
  \l 611F
  \l 6120
  \l 6121
  \l 6122
  \l 6123
  \l 6124
  \l 6125
  \l 6126
  \l 6127
  \l 6128
  \l 6129
  \l 612A
  \l 612B
  \l 612C
  \l 612D
  \l 612E
  \l 612F
  \l 6130
  \l 6131
  \l 6132
  \l 6133
  \l 6134
  \l 6135
  \l 6136
  \l 6137
  \l 6138
  \l 6139
  \l 613A
  \l 613B
  \l 613C
  \l 613D
  \l 613E
  \l 613F
  \l 6140
  \l 6141
  \l 6142
  \l 6143
  \l 6144
  \l 6145
  \l 6146
  \l 6147
  \l 6148
  \l 6149
  \l 614A
  \l 614B
  \l 614C
  \l 614D
  \l 614E
  \l 614F
  \l 6150
  \l 6151
  \l 6152
  \l 6153
  \l 6154
  \l 6155
  \l 6156
  \l 6157
  \l 6158
  \l 6159
  \l 615A
  \l 615B
  \l 615C
  \l 615D
  \l 615E
  \l 615F
  \l 6160
  \l 6161
  \l 6162
  \l 6163
  \l 6164
  \l 6165
  \l 6166
  \l 6167
  \l 6168
  \l 6169
  \l 616A
  \l 616B
  \l 616C
  \l 616D
  \l 616E
  \l 616F
  \l 6170
  \l 6171
  \l 6172
  \l 6173
  \l 6174
  \l 6175
  \l 6176
  \l 6177
  \l 6178
  \l 6179
  \l 617A
  \l 617B
  \l 617C
  \l 617D
  \l 617E
  \l 617F
  \l 6180
  \l 6181
  \l 6182
  \l 6183
  \l 6184
  \l 6185
  \l 6186
  \l 6187
  \l 6188
  \l 6189
  \l 618A
  \l 618B
  \l 618C
  \l 618D
  \l 618E
  \l 618F
  \l 6190
  \l 6191
  \l 6192
  \l 6193
  \l 6194
  \l 6195
  \l 6196
  \l 6197
  \l 6198
  \l 6199
  \l 619A
  \l 619B
  \l 619C
  \l 619D
  \l 619E
  \l 619F
  \l 61A0
  \l 61A1
  \l 61A2
  \l 61A3
  \l 61A4
  \l 61A5
  \l 61A6
  \l 61A7
  \l 61A8
  \l 61A9
  \l 61AA
  \l 61AB
  \l 61AC
  \l 61AD
  \l 61AE
  \l 61AF
  \l 61B0
  \l 61B1
  \l 61B2
  \l 61B3
  \l 61B4
  \l 61B5
  \l 61B6
  \l 61B7
  \l 61B8
  \l 61B9
  \l 61BA
  \l 61BB
  \l 61BC
  \l 61BD
  \l 61BE
  \l 61BF
  \l 61C0
  \l 61C1
  \l 61C2
  \l 61C3
  \l 61C4
  \l 61C5
  \l 61C6
  \l 61C7
  \l 61C8
  \l 61C9
  \l 61CA
  \l 61CB
  \l 61CC
  \l 61CD
  \l 61CE
  \l 61CF
  \l 61D0
  \l 61D1
  \l 61D2
  \l 61D3
  \l 61D4
  \l 61D5
  \l 61D6
  \l 61D7
  \l 61D8
  \l 61D9
  \l 61DA
  \l 61DB
  \l 61DC
  \l 61DD
  \l 61DE
  \l 61DF
  \l 61E0
  \l 61E1
  \l 61E2
  \l 61E3
  \l 61E4
  \l 61E5
  \l 61E6
  \l 61E7
  \l 61E8
  \l 61E9
  \l 61EA
  \l 61EB
  \l 61EC
  \l 61ED
  \l 61EE
  \l 61EF
  \l 61F0
  \l 61F1
  \l 61F2
  \l 61F3
  \l 61F4
  \l 61F5
  \l 61F6
  \l 61F7
  \l 61F8
  \l 61F9
  \l 61FA
  \l 61FB
  \l 61FC
  \l 61FD
  \l 61FE
  \l 61FF
  \l 6200
  \l 6201
  \l 6202
  \l 6203
  \l 6204
  \l 6205
  \l 6206
  \l 6207
  \l 6208
  \l 6209
  \l 620A
  \l 620B
  \l 620C
  \l 620D
  \l 620E
  \l 620F
  \l 6210
  \l 6211
  \l 6212
  \l 6213
  \l 6214
  \l 6215
  \l 6216
  \l 6217
  \l 6218
  \l 6219
  \l 621A
  \l 621B
  \l 621C
  \l 621D
  \l 621E
  \l 621F
  \l 6220
  \l 6221
  \l 6222
  \l 6223
  \l 6224
  \l 6225
  \l 6226
  \l 6227
  \l 6228
  \l 6229
  \l 622A
  \l 622B
  \l 622C
  \l 622D
  \l 622E
  \l 622F
  \l 6230
  \l 6231
  \l 6232
  \l 6233
  \l 6234
  \l 6235
  \l 6236
  \l 6237
  \l 6238
  \l 6239
  \l 623A
  \l 623B
  \l 623C
  \l 623D
  \l 623E
  \l 623F
  \l 6240
  \l 6241
  \l 6242
  \l 6243
  \l 6244
  \l 6245
  \l 6246
  \l 6247
  \l 6248
  \l 6249
  \l 624A
  \l 624B
  \l 624C
  \l 624D
  \l 624E
  \l 624F
  \l 6250
  \l 6251
  \l 6252
  \l 6253
  \l 6254
  \l 6255
  \l 6256
  \l 6257
  \l 6258
  \l 6259
  \l 625A
  \l 625B
  \l 625C
  \l 625D
  \l 625E
  \l 625F
  \l 6260
  \l 6261
  \l 6262
  \l 6263
  \l 6264
  \l 6265
  \l 6266
  \l 6267
  \l 6268
  \l 6269
  \l 626A
  \l 626B
  \l 626C
  \l 626D
  \l 626E
  \l 626F
  \l 6270
  \l 6271
  \l 6272
  \l 6273
  \l 6274
  \l 6275
  \l 6276
  \l 6277
  \l 6278
  \l 6279
  \l 627A
  \l 627B
  \l 627C
  \l 627D
  \l 627E
  \l 627F
  \l 6280
  \l 6281
  \l 6282
  \l 6283
  \l 6284
  \l 6285
  \l 6286
  \l 6287
  \l 6288
  \l 6289
  \l 628A
  \l 628B
  \l 628C
  \l 628D
  \l 628E
  \l 628F
  \l 6290
  \l 6291
  \l 6292
  \l 6293
  \l 6294
  \l 6295
  \l 6296
  \l 6297
  \l 6298
  \l 6299
  \l 629A
  \l 629B
  \l 629C
  \l 629D
  \l 629E
  \l 629F
  \l 62A0
  \l 62A1
  \l 62A2
  \l 62A3
  \l 62A4
  \l 62A5
  \l 62A6
  \l 62A7
  \l 62A8
  \l 62A9
  \l 62AA
  \l 62AB
  \l 62AC
  \l 62AD
  \l 62AE
  \l 62AF
  \l 62B0
  \l 62B1
  \l 62B2
  \l 62B3
  \l 62B4
  \l 62B5
  \l 62B6
  \l 62B7
  \l 62B8
  \l 62B9
  \l 62BA
  \l 62BB
  \l 62BC
  \l 62BD
  \l 62BE
  \l 62BF
  \l 62C0
  \l 62C1
  \l 62C2
  \l 62C3
  \l 62C4
  \l 62C5
  \l 62C6
  \l 62C7
  \l 62C8
  \l 62C9
  \l 62CA
  \l 62CB
  \l 62CC
  \l 62CD
  \l 62CE
  \l 62CF
  \l 62D0
  \l 62D1
  \l 62D2
  \l 62D3
  \l 62D4
  \l 62D5
  \l 62D6
  \l 62D7
  \l 62D8
  \l 62D9
  \l 62DA
  \l 62DB
  \l 62DC
  \l 62DD
  \l 62DE
  \l 62DF
  \l 62E0
  \l 62E1
  \l 62E2
  \l 62E3
  \l 62E4
  \l 62E5
  \l 62E6
  \l 62E7
  \l 62E8
  \l 62E9
  \l 62EA
  \l 62EB
  \l 62EC
  \l 62ED
  \l 62EE
  \l 62EF
  \l 62F0
  \l 62F1
  \l 62F2
  \l 62F3
  \l 62F4
  \l 62F5
  \l 62F6
  \l 62F7
  \l 62F8
  \l 62F9
  \l 62FA
  \l 62FB
  \l 62FC
  \l 62FD
  \l 62FE
  \l 62FF
  \l 6300
  \l 6301
  \l 6302
  \l 6303
  \l 6304
  \l 6305
  \l 6306
  \l 6307
  \l 6308
  \l 6309
  \l 630A
  \l 630B
  \l 630C
  \l 630D
  \l 630E
  \l 630F
  \l 6310
  \l 6311
  \l 6312
  \l 6313
  \l 6314
  \l 6315
  \l 6316
  \l 6317
  \l 6318
  \l 6319
  \l 631A
  \l 631B
  \l 631C
  \l 631D
  \l 631E
  \l 631F
  \l 6320
  \l 6321
  \l 6322
  \l 6323
  \l 6324
  \l 6325
  \l 6326
  \l 6327
  \l 6328
  \l 6329
  \l 632A
  \l 632B
  \l 632C
  \l 632D
  \l 632E
  \l 632F
  \l 6330
  \l 6331
  \l 6332
  \l 6333
  \l 6334
  \l 6335
  \l 6336
  \l 6337
  \l 6338
  \l 6339
  \l 633A
  \l 633B
  \l 633C
  \l 633D
  \l 633E
  \l 633F
  \l 6340
  \l 6341
  \l 6342
  \l 6343
  \l 6344
  \l 6345
  \l 6346
  \l 6347
  \l 6348
  \l 6349
  \l 634A
  \l 634B
  \l 634C
  \l 634D
  \l 634E
  \l 634F
  \l 6350
  \l 6351
  \l 6352
  \l 6353
  \l 6354
  \l 6355
  \l 6356
  \l 6357
  \l 6358
  \l 6359
  \l 635A
  \l 635B
  \l 635C
  \l 635D
  \l 635E
  \l 635F
  \l 6360
  \l 6361
  \l 6362
  \l 6363
  \l 6364
  \l 6365
  \l 6366
  \l 6367
  \l 6368
  \l 6369
  \l 636A
  \l 636B
  \l 636C
  \l 636D
  \l 636E
  \l 636F
  \l 6370
  \l 6371
  \l 6372
  \l 6373
  \l 6374
  \l 6375
  \l 6376
  \l 6377
  \l 6378
  \l 6379
  \l 637A
  \l 637B
  \l 637C
  \l 637D
  \l 637E
  \l 637F
  \l 6380
  \l 6381
  \l 6382
  \l 6383
  \l 6384
  \l 6385
  \l 6386
  \l 6387
  \l 6388
  \l 6389
  \l 638A
  \l 638B
  \l 638C
  \l 638D
  \l 638E
  \l 638F
  \l 6390
  \l 6391
  \l 6392
  \l 6393
  \l 6394
  \l 6395
  \l 6396
  \l 6397
  \l 6398
  \l 6399
  \l 639A
  \l 639B
  \l 639C
  \l 639D
  \l 639E
  \l 639F
  \l 63A0
  \l 63A1
  \l 63A2
  \l 63A3
  \l 63A4
  \l 63A5
  \l 63A6
  \l 63A7
  \l 63A8
  \l 63A9
  \l 63AA
  \l 63AB
  \l 63AC
  \l 63AD
  \l 63AE
  \l 63AF
  \l 63B0
  \l 63B1
  \l 63B2
  \l 63B3
  \l 63B4
  \l 63B5
  \l 63B6
  \l 63B7
  \l 63B8
  \l 63B9
  \l 63BA
  \l 63BB
  \l 63BC
  \l 63BD
  \l 63BE
  \l 63BF
  \l 63C0
  \l 63C1
  \l 63C2
  \l 63C3
  \l 63C4
  \l 63C5
  \l 63C6
  \l 63C7
  \l 63C8
  \l 63C9
  \l 63CA
  \l 63CB
  \l 63CC
  \l 63CD
  \l 63CE
  \l 63CF
  \l 63D0
  \l 63D1
  \l 63D2
  \l 63D3
  \l 63D4
  \l 63D5
  \l 63D6
  \l 63D7
  \l 63D8
  \l 63D9
  \l 63DA
  \l 63DB
  \l 63DC
  \l 63DD
  \l 63DE
  \l 63DF
  \l 63E0
  \l 63E1
  \l 63E2
  \l 63E3
  \l 63E4
  \l 63E5
  \l 63E6
  \l 63E7
  \l 63E8
  \l 63E9
  \l 63EA
  \l 63EB
  \l 63EC
  \l 63ED
  \l 63EE
  \l 63EF
  \l 63F0
  \l 63F1
  \l 63F2
  \l 63F3
  \l 63F4
  \l 63F5
  \l 63F6
  \l 63F7
  \l 63F8
  \l 63F9
  \l 63FA
  \l 63FB
  \l 63FC
  \l 63FD
  \l 63FE
  \l 63FF
  \l 6400
  \l 6401
  \l 6402
  \l 6403
  \l 6404
  \l 6405
  \l 6406
  \l 6407
  \l 6408
  \l 6409
  \l 640A
  \l 640B
  \l 640C
  \l 640D
  \l 640E
  \l 640F
  \l 6410
  \l 6411
  \l 6412
  \l 6413
  \l 6414
  \l 6415
  \l 6416
  \l 6417
  \l 6418
  \l 6419
  \l 641A
  \l 641B
  \l 641C
  \l 641D
  \l 641E
  \l 641F
  \l 6420
  \l 6421
  \l 6422
  \l 6423
  \l 6424
  \l 6425
  \l 6426
  \l 6427
  \l 6428
  \l 6429
  \l 642A
  \l 642B
  \l 642C
  \l 642D
  \l 642E
  \l 642F
  \l 6430
  \l 6431
  \l 6432
  \l 6433
  \l 6434
  \l 6435
  \l 6436
  \l 6437
  \l 6438
  \l 6439
  \l 643A
  \l 643B
  \l 643C
  \l 643D
  \l 643E
  \l 643F
  \l 6440
  \l 6441
  \l 6442
  \l 6443
  \l 6444
  \l 6445
  \l 6446
  \l 6447
  \l 6448
  \l 6449
  \l 644A
  \l 644B
  \l 644C
  \l 644D
  \l 644E
  \l 644F
  \l 6450
  \l 6451
  \l 6452
  \l 6453
  \l 6454
  \l 6455
  \l 6456
  \l 6457
  \l 6458
  \l 6459
  \l 645A
  \l 645B
  \l 645C
  \l 645D
  \l 645E
  \l 645F
  \l 6460
  \l 6461
  \l 6462
  \l 6463
  \l 6464
  \l 6465
  \l 6466
  \l 6467
  \l 6468
  \l 6469
  \l 646A
  \l 646B
  \l 646C
  \l 646D
  \l 646E
  \l 646F
  \l 6470
  \l 6471
  \l 6472
  \l 6473
  \l 6474
  \l 6475
  \l 6476
  \l 6477
  \l 6478
  \l 6479
  \l 647A
  \l 647B
  \l 647C
  \l 647D
  \l 647E
  \l 647F
  \l 6480
  \l 6481
  \l 6482
  \l 6483
  \l 6484
  \l 6485
  \l 6486
  \l 6487
  \l 6488
  \l 6489
  \l 648A
  \l 648B
  \l 648C
  \l 648D
  \l 648E
  \l 648F
  \l 6490
  \l 6491
  \l 6492
  \l 6493
  \l 6494
  \l 6495
  \l 6496
  \l 6497
  \l 6498
  \l 6499
  \l 649A
  \l 649B
  \l 649C
  \l 649D
  \l 649E
  \l 649F
  \l 64A0
  \l 64A1
  \l 64A2
  \l 64A3
  \l 64A4
  \l 64A5
  \l 64A6
  \l 64A7
  \l 64A8
  \l 64A9
  \l 64AA
  \l 64AB
  \l 64AC
  \l 64AD
  \l 64AE
  \l 64AF
  \l 64B0
  \l 64B1
  \l 64B2
  \l 64B3
  \l 64B4
  \l 64B5
  \l 64B6
  \l 64B7
  \l 64B8
  \l 64B9
  \l 64BA
  \l 64BB
  \l 64BC
  \l 64BD
  \l 64BE
  \l 64BF
  \l 64C0
  \l 64C1
  \l 64C2
  \l 64C3
  \l 64C4
  \l 64C5
  \l 64C6
  \l 64C7
  \l 64C8
  \l 64C9
  \l 64CA
  \l 64CB
  \l 64CC
  \l 64CD
  \l 64CE
  \l 64CF
  \l 64D0
  \l 64D1
  \l 64D2
  \l 64D3
  \l 64D4
  \l 64D5
  \l 64D6
  \l 64D7
  \l 64D8
  \l 64D9
  \l 64DA
  \l 64DB
  \l 64DC
  \l 64DD
  \l 64DE
  \l 64DF
  \l 64E0
  \l 64E1
  \l 64E2
  \l 64E3
  \l 64E4
  \l 64E5
  \l 64E6
  \l 64E7
  \l 64E8
  \l 64E9
  \l 64EA
  \l 64EB
  \l 64EC
  \l 64ED
  \l 64EE
  \l 64EF
  \l 64F0
  \l 64F1
  \l 64F2
  \l 64F3
  \l 64F4
  \l 64F5
  \l 64F6
  \l 64F7
  \l 64F8
  \l 64F9
  \l 64FA
  \l 64FB
  \l 64FC
  \l 64FD
  \l 64FE
  \l 64FF
  \l 6500
  \l 6501
  \l 6502
  \l 6503
  \l 6504
  \l 6505
  \l 6506
  \l 6507
  \l 6508
  \l 6509
  \l 650A
  \l 650B
  \l 650C
  \l 650D
  \l 650E
  \l 650F
  \l 6510
  \l 6511
  \l 6512
  \l 6513
  \l 6514
  \l 6515
  \l 6516
  \l 6517
  \l 6518
  \l 6519
  \l 651A
  \l 651B
  \l 651C
  \l 651D
  \l 651E
  \l 651F
  \l 6520
  \l 6521
  \l 6522
  \l 6523
  \l 6524
  \l 6525
  \l 6526
  \l 6527
  \l 6528
  \l 6529
  \l 652A
  \l 652B
  \l 652C
  \l 652D
  \l 652E
  \l 652F
  \l 6530
  \l 6531
  \l 6532
  \l 6533
  \l 6534
  \l 6535
  \l 6536
  \l 6537
  \l 6538
  \l 6539
  \l 653A
  \l 653B
  \l 653C
  \l 653D
  \l 653E
  \l 653F
  \l 6540
  \l 6541
  \l 6542
  \l 6543
  \l 6544
  \l 6545
  \l 6546
  \l 6547
  \l 6548
  \l 6549
  \l 654A
  \l 654B
  \l 654C
  \l 654D
  \l 654E
  \l 654F
  \l 6550
  \l 6551
  \l 6552
  \l 6553
  \l 6554
  \l 6555
  \l 6556
  \l 6557
  \l 6558
  \l 6559
  \l 655A
  \l 655B
  \l 655C
  \l 655D
  \l 655E
  \l 655F
  \l 6560
  \l 6561
  \l 6562
  \l 6563
  \l 6564
  \l 6565
  \l 6566
  \l 6567
  \l 6568
  \l 6569
  \l 656A
  \l 656B
  \l 656C
  \l 656D
  \l 656E
  \l 656F
  \l 6570
  \l 6571
  \l 6572
  \l 6573
  \l 6574
  \l 6575
  \l 6576
  \l 6577
  \l 6578
  \l 6579
  \l 657A
  \l 657B
  \l 657C
  \l 657D
  \l 657E
  \l 657F
  \l 6580
  \l 6581
  \l 6582
  \l 6583
  \l 6584
  \l 6585
  \l 6586
  \l 6587
  \l 6588
  \l 6589
  \l 658A
  \l 658B
  \l 658C
  \l 658D
  \l 658E
  \l 658F
  \l 6590
  \l 6591
  \l 6592
  \l 6593
  \l 6594
  \l 6595
  \l 6596
  \l 6597
  \l 6598
  \l 6599
  \l 659A
  \l 659B
  \l 659C
  \l 659D
  \l 659E
  \l 659F
  \l 65A0
  \l 65A1
  \l 65A2
  \l 65A3
  \l 65A4
  \l 65A5
  \l 65A6
  \l 65A7
  \l 65A8
  \l 65A9
  \l 65AA
  \l 65AB
  \l 65AC
  \l 65AD
  \l 65AE
  \l 65AF
  \l 65B0
  \l 65B1
  \l 65B2
  \l 65B3
  \l 65B4
  \l 65B5
  \l 65B6
  \l 65B7
  \l 65B8
  \l 65B9
  \l 65BA
  \l 65BB
  \l 65BC
  \l 65BD
  \l 65BE
  \l 65BF
  \l 65C0
  \l 65C1
  \l 65C2
  \l 65C3
  \l 65C4
  \l 65C5
  \l 65C6
  \l 65C7
  \l 65C8
  \l 65C9
  \l 65CA
  \l 65CB
  \l 65CC
  \l 65CD
  \l 65CE
  \l 65CF
  \l 65D0
  \l 65D1
  \l 65D2
  \l 65D3
  \l 65D4
  \l 65D5
  \l 65D6
  \l 65D7
  \l 65D8
  \l 65D9
  \l 65DA
  \l 65DB
  \l 65DC
  \l 65DD
  \l 65DE
  \l 65DF
  \l 65E0
  \l 65E1
  \l 65E2
  \l 65E3
  \l 65E4
  \l 65E5
  \l 65E6
  \l 65E7
  \l 65E8
  \l 65E9
  \l 65EA
  \l 65EB
  \l 65EC
  \l 65ED
  \l 65EE
  \l 65EF
  \l 65F0
  \l 65F1
  \l 65F2
  \l 65F3
  \l 65F4
  \l 65F5
  \l 65F6
  \l 65F7
  \l 65F8
  \l 65F9
  \l 65FA
  \l 65FB
  \l 65FC
  \l 65FD
  \l 65FE
  \l 65FF
  \l 6600
  \l 6601
  \l 6602
  \l 6603
  \l 6604
  \l 6605
  \l 6606
  \l 6607
  \l 6608
  \l 6609
  \l 660A
  \l 660B
  \l 660C
  \l 660D
  \l 660E
  \l 660F
  \l 6610
  \l 6611
  \l 6612
  \l 6613
  \l 6614
  \l 6615
  \l 6616
  \l 6617
  \l 6618
  \l 6619
  \l 661A
  \l 661B
  \l 661C
  \l 661D
  \l 661E
  \l 661F
  \l 6620
  \l 6621
  \l 6622
  \l 6623
  \l 6624
  \l 6625
  \l 6626
  \l 6627
  \l 6628
  \l 6629
  \l 662A
  \l 662B
  \l 662C
  \l 662D
  \l 662E
  \l 662F
  \l 6630
  \l 6631
  \l 6632
  \l 6633
  \l 6634
  \l 6635
  \l 6636
  \l 6637
  \l 6638
  \l 6639
  \l 663A
  \l 663B
  \l 663C
  \l 663D
  \l 663E
  \l 663F
  \l 6640
  \l 6641
  \l 6642
  \l 6643
  \l 6644
  \l 6645
  \l 6646
  \l 6647
  \l 6648
  \l 6649
  \l 664A
  \l 664B
  \l 664C
  \l 664D
  \l 664E
  \l 664F
  \l 6650
  \l 6651
  \l 6652
  \l 6653
  \l 6654
  \l 6655
  \l 6656
  \l 6657
  \l 6658
  \l 6659
  \l 665A
  \l 665B
  \l 665C
  \l 665D
  \l 665E
  \l 665F
  \l 6660
  \l 6661
  \l 6662
  \l 6663
  \l 6664
  \l 6665
  \l 6666
  \l 6667
  \l 6668
  \l 6669
  \l 666A
  \l 666B
  \l 666C
  \l 666D
  \l 666E
  \l 666F
  \l 6670
  \l 6671
  \l 6672
  \l 6673
  \l 6674
  \l 6675
  \l 6676
  \l 6677
  \l 6678
  \l 6679
  \l 667A
  \l 667B
  \l 667C
  \l 667D
  \l 667E
  \l 667F
  \l 6680
  \l 6681
  \l 6682
  \l 6683
  \l 6684
  \l 6685
  \l 6686
  \l 6687
  \l 6688
  \l 6689
  \l 668A
  \l 668B
  \l 668C
  \l 668D
  \l 668E
  \l 668F
  \l 6690
  \l 6691
  \l 6692
  \l 6693
  \l 6694
  \l 6695
  \l 6696
  \l 6697
  \l 6698
  \l 6699
  \l 669A
  \l 669B
  \l 669C
  \l 669D
  \l 669E
  \l 669F
  \l 66A0
  \l 66A1
  \l 66A2
  \l 66A3
  \l 66A4
  \l 66A5
  \l 66A6
  \l 66A7
  \l 66A8
  \l 66A9
  \l 66AA
  \l 66AB
  \l 66AC
  \l 66AD
  \l 66AE
  \l 66AF
  \l 66B0
  \l 66B1
  \l 66B2
  \l 66B3
  \l 66B4
  \l 66B5
  \l 66B6
  \l 66B7
  \l 66B8
  \l 66B9
  \l 66BA
  \l 66BB
  \l 66BC
  \l 66BD
  \l 66BE
  \l 66BF
  \l 66C0
  \l 66C1
  \l 66C2
  \l 66C3
  \l 66C4
  \l 66C5
  \l 66C6
  \l 66C7
  \l 66C8
  \l 66C9
  \l 66CA
  \l 66CB
  \l 66CC
  \l 66CD
  \l 66CE
  \l 66CF
  \l 66D0
  \l 66D1
  \l 66D2
  \l 66D3
  \l 66D4
  \l 66D5
  \l 66D6
  \l 66D7
  \l 66D8
  \l 66D9
  \l 66DA
  \l 66DB
  \l 66DC
  \l 66DD
  \l 66DE
  \l 66DF
  \l 66E0
  \l 66E1
  \l 66E2
  \l 66E3
  \l 66E4
  \l 66E5
  \l 66E6
  \l 66E7
  \l 66E8
  \l 66E9
  \l 66EA
  \l 66EB
  \l 66EC
  \l 66ED
  \l 66EE
  \l 66EF
  \l 66F0
  \l 66F1
  \l 66F2
  \l 66F3
  \l 66F4
  \l 66F5
  \l 66F6
  \l 66F7
  \l 66F8
  \l 66F9
  \l 66FA
  \l 66FB
  \l 66FC
  \l 66FD
  \l 66FE
  \l 66FF
  \l 6700
  \l 6701
  \l 6702
  \l 6703
  \l 6704
  \l 6705
  \l 6706
  \l 6707
  \l 6708
  \l 6709
  \l 670A
  \l 670B
  \l 670C
  \l 670D
  \l 670E
  \l 670F
  \l 6710
  \l 6711
  \l 6712
  \l 6713
  \l 6714
  \l 6715
  \l 6716
  \l 6717
  \l 6718
  \l 6719
  \l 671A
  \l 671B
  \l 671C
  \l 671D
  \l 671E
  \l 671F
  \l 6720
  \l 6721
  \l 6722
  \l 6723
  \l 6724
  \l 6725
  \l 6726
  \l 6727
  \l 6728
  \l 6729
  \l 672A
  \l 672B
  \l 672C
  \l 672D
  \l 672E
  \l 672F
  \l 6730
  \l 6731
  \l 6732
  \l 6733
  \l 6734
  \l 6735
  \l 6736
  \l 6737
  \l 6738
  \l 6739
  \l 673A
  \l 673B
  \l 673C
  \l 673D
  \l 673E
  \l 673F
  \l 6740
  \l 6741
  \l 6742
  \l 6743
  \l 6744
  \l 6745
  \l 6746
  \l 6747
  \l 6748
  \l 6749
  \l 674A
  \l 674B
  \l 674C
  \l 674D
  \l 674E
  \l 674F
  \l 6750
  \l 6751
  \l 6752
  \l 6753
  \l 6754
  \l 6755
  \l 6756
  \l 6757
  \l 6758
  \l 6759
  \l 675A
  \l 675B
  \l 675C
  \l 675D
  \l 675E
  \l 675F
  \l 6760
  \l 6761
  \l 6762
  \l 6763
  \l 6764
  \l 6765
  \l 6766
  \l 6767
  \l 6768
  \l 6769
  \l 676A
  \l 676B
  \l 676C
  \l 676D
  \l 676E
  \l 676F
  \l 6770
  \l 6771
  \l 6772
  \l 6773
  \l 6774
  \l 6775
  \l 6776
  \l 6777
  \l 6778
  \l 6779
  \l 677A
  \l 677B
  \l 677C
  \l 677D
  \l 677E
  \l 677F
  \l 6780
  \l 6781
  \l 6782
  \l 6783
  \l 6784
  \l 6785
  \l 6786
  \l 6787
  \l 6788
  \l 6789
  \l 678A
  \l 678B
  \l 678C
  \l 678D
  \l 678E
  \l 678F
  \l 6790
  \l 6791
  \l 6792
  \l 6793
  \l 6794
  \l 6795
  \l 6796
  \l 6797
  \l 6798
  \l 6799
  \l 679A
  \l 679B
  \l 679C
  \l 679D
  \l 679E
  \l 679F
  \l 67A0
  \l 67A1
  \l 67A2
  \l 67A3
  \l 67A4
  \l 67A5
  \l 67A6
  \l 67A7
  \l 67A8
  \l 67A9
  \l 67AA
  \l 67AB
  \l 67AC
  \l 67AD
  \l 67AE
  \l 67AF
  \l 67B0
  \l 67B1
  \l 67B2
  \l 67B3
  \l 67B4
  \l 67B5
  \l 67B6
  \l 67B7
  \l 67B8
  \l 67B9
  \l 67BA
  \l 67BB
  \l 67BC
  \l 67BD
  \l 67BE
  \l 67BF
  \l 67C0
  \l 67C1
  \l 67C2
  \l 67C3
  \l 67C4
  \l 67C5
  \l 67C6
  \l 67C7
  \l 67C8
  \l 67C9
  \l 67CA
  \l 67CB
  \l 67CC
  \l 67CD
  \l 67CE
  \l 67CF
  \l 67D0
  \l 67D1
  \l 67D2
  \l 67D3
  \l 67D4
  \l 67D5
  \l 67D6
  \l 67D7
  \l 67D8
  \l 67D9
  \l 67DA
  \l 67DB
  \l 67DC
  \l 67DD
  \l 67DE
  \l 67DF
  \l 67E0
  \l 67E1
  \l 67E2
  \l 67E3
  \l 67E4
  \l 67E5
  \l 67E6
  \l 67E7
  \l 67E8
  \l 67E9
  \l 67EA
  \l 67EB
  \l 67EC
  \l 67ED
  \l 67EE
  \l 67EF
  \l 67F0
  \l 67F1
  \l 67F2
  \l 67F3
  \l 67F4
  \l 67F5
  \l 67F6
  \l 67F7
  \l 67F8
  \l 67F9
  \l 67FA
  \l 67FB
  \l 67FC
  \l 67FD
  \l 67FE
  \l 67FF
  \l 6800
  \l 6801
  \l 6802
  \l 6803
  \l 6804
  \l 6805
  \l 6806
  \l 6807
  \l 6808
  \l 6809
  \l 680A
  \l 680B
  \l 680C
  \l 680D
  \l 680E
  \l 680F
  \l 6810
  \l 6811
  \l 6812
  \l 6813
  \l 6814
  \l 6815
  \l 6816
  \l 6817
  \l 6818
  \l 6819
  \l 681A
  \l 681B
  \l 681C
  \l 681D
  \l 681E
  \l 681F
  \l 6820
  \l 6821
  \l 6822
  \l 6823
  \l 6824
  \l 6825
  \l 6826
  \l 6827
  \l 6828
  \l 6829
  \l 682A
  \l 682B
  \l 682C
  \l 682D
  \l 682E
  \l 682F
  \l 6830
  \l 6831
  \l 6832
  \l 6833
  \l 6834
  \l 6835
  \l 6836
  \l 6837
  \l 6838
  \l 6839
  \l 683A
  \l 683B
  \l 683C
  \l 683D
  \l 683E
  \l 683F
  \l 6840
  \l 6841
  \l 6842
  \l 6843
  \l 6844
  \l 6845
  \l 6846
  \l 6847
  \l 6848
  \l 6849
  \l 684A
  \l 684B
  \l 684C
  \l 684D
  \l 684E
  \l 684F
  \l 6850
  \l 6851
  \l 6852
  \l 6853
  \l 6854
  \l 6855
  \l 6856
  \l 6857
  \l 6858
  \l 6859
  \l 685A
  \l 685B
  \l 685C
  \l 685D
  \l 685E
  \l 685F
  \l 6860
  \l 6861
  \l 6862
  \l 6863
  \l 6864
  \l 6865
  \l 6866
  \l 6867
  \l 6868
  \l 6869
  \l 686A
  \l 686B
  \l 686C
  \l 686D
  \l 686E
  \l 686F
  \l 6870
  \l 6871
  \l 6872
  \l 6873
  \l 6874
  \l 6875
  \l 6876
  \l 6877
  \l 6878
  \l 6879
  \l 687A
  \l 687B
  \l 687C
  \l 687D
  \l 687E
  \l 687F
  \l 6880
  \l 6881
  \l 6882
  \l 6883
  \l 6884
  \l 6885
  \l 6886
  \l 6887
  \l 6888
  \l 6889
  \l 688A
  \l 688B
  \l 688C
  \l 688D
  \l 688E
  \l 688F
  \l 6890
  \l 6891
  \l 6892
  \l 6893
  \l 6894
  \l 6895
  \l 6896
  \l 6897
  \l 6898
  \l 6899
  \l 689A
  \l 689B
  \l 689C
  \l 689D
  \l 689E
  \l 689F
  \l 68A0
  \l 68A1
  \l 68A2
  \l 68A3
  \l 68A4
  \l 68A5
  \l 68A6
  \l 68A7
  \l 68A8
  \l 68A9
  \l 68AA
  \l 68AB
  \l 68AC
  \l 68AD
  \l 68AE
  \l 68AF
  \l 68B0
  \l 68B1
  \l 68B2
  \l 68B3
  \l 68B4
  \l 68B5
  \l 68B6
  \l 68B7
  \l 68B8
  \l 68B9
  \l 68BA
  \l 68BB
  \l 68BC
  \l 68BD
  \l 68BE
  \l 68BF
  \l 68C0
  \l 68C1
  \l 68C2
  \l 68C3
  \l 68C4
  \l 68C5
  \l 68C6
  \l 68C7
  \l 68C8
  \l 68C9
  \l 68CA
  \l 68CB
  \l 68CC
  \l 68CD
  \l 68CE
  \l 68CF
  \l 68D0
  \l 68D1
  \l 68D2
  \l 68D3
  \l 68D4
  \l 68D5
  \l 68D6
  \l 68D7
  \l 68D8
  \l 68D9
  \l 68DA
  \l 68DB
  \l 68DC
  \l 68DD
  \l 68DE
  \l 68DF
  \l 68E0
  \l 68E1
  \l 68E2
  \l 68E3
  \l 68E4
  \l 68E5
  \l 68E6
  \l 68E7
  \l 68E8
  \l 68E9
  \l 68EA
  \l 68EB
  \l 68EC
  \l 68ED
  \l 68EE
  \l 68EF
  \l 68F0
  \l 68F1
  \l 68F2
  \l 68F3
  \l 68F4
  \l 68F5
  \l 68F6
  \l 68F7
  \l 68F8
  \l 68F9
  \l 68FA
  \l 68FB
  \l 68FC
  \l 68FD
  \l 68FE
  \l 68FF
  \l 6900
  \l 6901
  \l 6902
  \l 6903
  \l 6904
  \l 6905
  \l 6906
  \l 6907
  \l 6908
  \l 6909
  \l 690A
  \l 690B
  \l 690C
  \l 690D
  \l 690E
  \l 690F
  \l 6910
  \l 6911
  \l 6912
  \l 6913
  \l 6914
  \l 6915
  \l 6916
  \l 6917
  \l 6918
  \l 6919
  \l 691A
  \l 691B
  \l 691C
  \l 691D
  \l 691E
  \l 691F
  \l 6920
  \l 6921
  \l 6922
  \l 6923
  \l 6924
  \l 6925
  \l 6926
  \l 6927
  \l 6928
  \l 6929
  \l 692A
  \l 692B
  \l 692C
  \l 692D
  \l 692E
  \l 692F
  \l 6930
  \l 6931
  \l 6932
  \l 6933
  \l 6934
  \l 6935
  \l 6936
  \l 6937
  \l 6938
  \l 6939
  \l 693A
  \l 693B
  \l 693C
  \l 693D
  \l 693E
  \l 693F
  \l 6940
  \l 6941
  \l 6942
  \l 6943
  \l 6944
  \l 6945
  \l 6946
  \l 6947
  \l 6948
  \l 6949
  \l 694A
  \l 694B
  \l 694C
  \l 694D
  \l 694E
  \l 694F
  \l 6950
  \l 6951
  \l 6952
  \l 6953
  \l 6954
  \l 6955
  \l 6956
  \l 6957
  \l 6958
  \l 6959
  \l 695A
  \l 695B
  \l 695C
  \l 695D
  \l 695E
  \l 695F
  \l 6960
  \l 6961
  \l 6962
  \l 6963
  \l 6964
  \l 6965
  \l 6966
  \l 6967
  \l 6968
  \l 6969
  \l 696A
  \l 696B
  \l 696C
  \l 696D
  \l 696E
  \l 696F
  \l 6970
  \l 6971
  \l 6972
  \l 6973
  \l 6974
  \l 6975
  \l 6976
  \l 6977
  \l 6978
  \l 6979
  \l 697A
  \l 697B
  \l 697C
  \l 697D
  \l 697E
  \l 697F
  \l 6980
  \l 6981
  \l 6982
  \l 6983
  \l 6984
  \l 6985
  \l 6986
  \l 6987
  \l 6988
  \l 6989
  \l 698A
  \l 698B
  \l 698C
  \l 698D
  \l 698E
  \l 698F
  \l 6990
  \l 6991
  \l 6992
  \l 6993
  \l 6994
  \l 6995
  \l 6996
  \l 6997
  \l 6998
  \l 6999
  \l 699A
  \l 699B
  \l 699C
  \l 699D
  \l 699E
  \l 699F
  \l 69A0
  \l 69A1
  \l 69A2
  \l 69A3
  \l 69A4
  \l 69A5
  \l 69A6
  \l 69A7
  \l 69A8
  \l 69A9
  \l 69AA
  \l 69AB
  \l 69AC
  \l 69AD
  \l 69AE
  \l 69AF
  \l 69B0
  \l 69B1
  \l 69B2
  \l 69B3
  \l 69B4
  \l 69B5
  \l 69B6
  \l 69B7
  \l 69B8
  \l 69B9
  \l 69BA
  \l 69BB
  \l 69BC
  \l 69BD
  \l 69BE
  \l 69BF
  \l 69C0
  \l 69C1
  \l 69C2
  \l 69C3
  \l 69C4
  \l 69C5
  \l 69C6
  \l 69C7
  \l 69C8
  \l 69C9
  \l 69CA
  \l 69CB
  \l 69CC
  \l 69CD
  \l 69CE
  \l 69CF
  \l 69D0
  \l 69D1
  \l 69D2
  \l 69D3
  \l 69D4
  \l 69D5
  \l 69D6
  \l 69D7
  \l 69D8
  \l 69D9
  \l 69DA
  \l 69DB
  \l 69DC
  \l 69DD
  \l 69DE
  \l 69DF
  \l 69E0
  \l 69E1
  \l 69E2
  \l 69E3
  \l 69E4
  \l 69E5
  \l 69E6
  \l 69E7
  \l 69E8
  \l 69E9
  \l 69EA
  \l 69EB
  \l 69EC
  \l 69ED
  \l 69EE
  \l 69EF
  \l 69F0
  \l 69F1
  \l 69F2
  \l 69F3
  \l 69F4
  \l 69F5
  \l 69F6
  \l 69F7
  \l 69F8
  \l 69F9
  \l 69FA
  \l 69FB
  \l 69FC
  \l 69FD
  \l 69FE
  \l 69FF
  \l 6A00
  \l 6A01
  \l 6A02
  \l 6A03
  \l 6A04
  \l 6A05
  \l 6A06
  \l 6A07
  \l 6A08
  \l 6A09
  \l 6A0A
  \l 6A0B
  \l 6A0C
  \l 6A0D
  \l 6A0E
  \l 6A0F
  \l 6A10
  \l 6A11
  \l 6A12
  \l 6A13
  \l 6A14
  \l 6A15
  \l 6A16
  \l 6A17
  \l 6A18
  \l 6A19
  \l 6A1A
  \l 6A1B
  \l 6A1C
  \l 6A1D
  \l 6A1E
  \l 6A1F
  \l 6A20
  \l 6A21
  \l 6A22
  \l 6A23
  \l 6A24
  \l 6A25
  \l 6A26
  \l 6A27
  \l 6A28
  \l 6A29
  \l 6A2A
  \l 6A2B
  \l 6A2C
  \l 6A2D
  \l 6A2E
  \l 6A2F
  \l 6A30
  \l 6A31
  \l 6A32
  \l 6A33
  \l 6A34
  \l 6A35
  \l 6A36
  \l 6A37
  \l 6A38
  \l 6A39
  \l 6A3A
  \l 6A3B
  \l 6A3C
  \l 6A3D
  \l 6A3E
  \l 6A3F
  \l 6A40
  \l 6A41
  \l 6A42
  \l 6A43
  \l 6A44
  \l 6A45
  \l 6A46
  \l 6A47
  \l 6A48
  \l 6A49
  \l 6A4A
  \l 6A4B
  \l 6A4C
  \l 6A4D
  \l 6A4E
  \l 6A4F
  \l 6A50
  \l 6A51
  \l 6A52
  \l 6A53
  \l 6A54
  \l 6A55
  \l 6A56
  \l 6A57
  \l 6A58
  \l 6A59
  \l 6A5A
  \l 6A5B
  \l 6A5C
  \l 6A5D
  \l 6A5E
  \l 6A5F
  \l 6A60
  \l 6A61
  \l 6A62
  \l 6A63
  \l 6A64
  \l 6A65
  \l 6A66
  \l 6A67
  \l 6A68
  \l 6A69
  \l 6A6A
  \l 6A6B
  \l 6A6C
  \l 6A6D
  \l 6A6E
  \l 6A6F
  \l 6A70
  \l 6A71
  \l 6A72
  \l 6A73
  \l 6A74
  \l 6A75
  \l 6A76
  \l 6A77
  \l 6A78
  \l 6A79
  \l 6A7A
  \l 6A7B
  \l 6A7C
  \l 6A7D
  \l 6A7E
  \l 6A7F
  \l 6A80
  \l 6A81
  \l 6A82
  \l 6A83
  \l 6A84
  \l 6A85
  \l 6A86
  \l 6A87
  \l 6A88
  \l 6A89
  \l 6A8A
  \l 6A8B
  \l 6A8C
  \l 6A8D
  \l 6A8E
  \l 6A8F
  \l 6A90
  \l 6A91
  \l 6A92
  \l 6A93
  \l 6A94
  \l 6A95
  \l 6A96
  \l 6A97
  \l 6A98
  \l 6A99
  \l 6A9A
  \l 6A9B
  \l 6A9C
  \l 6A9D
  \l 6A9E
  \l 6A9F
  \l 6AA0
  \l 6AA1
  \l 6AA2
  \l 6AA3
  \l 6AA4
  \l 6AA5
  \l 6AA6
  \l 6AA7
  \l 6AA8
  \l 6AA9
  \l 6AAA
  \l 6AAB
  \l 6AAC
  \l 6AAD
  \l 6AAE
  \l 6AAF
  \l 6AB0
  \l 6AB1
  \l 6AB2
  \l 6AB3
  \l 6AB4
  \l 6AB5
  \l 6AB6
  \l 6AB7
  \l 6AB8
  \l 6AB9
  \l 6ABA
  \l 6ABB
  \l 6ABC
  \l 6ABD
  \l 6ABE
  \l 6ABF
  \l 6AC0
  \l 6AC1
  \l 6AC2
  \l 6AC3
  \l 6AC4
  \l 6AC5
  \l 6AC6
  \l 6AC7
  \l 6AC8
  \l 6AC9
  \l 6ACA
  \l 6ACB
  \l 6ACC
  \l 6ACD
  \l 6ACE
  \l 6ACF
  \l 6AD0
  \l 6AD1
  \l 6AD2
  \l 6AD3
  \l 6AD4
  \l 6AD5
  \l 6AD6
  \l 6AD7
  \l 6AD8
  \l 6AD9
  \l 6ADA
  \l 6ADB
  \l 6ADC
  \l 6ADD
  \l 6ADE
  \l 6ADF
  \l 6AE0
  \l 6AE1
  \l 6AE2
  \l 6AE3
  \l 6AE4
  \l 6AE5
  \l 6AE6
  \l 6AE7
  \l 6AE8
  \l 6AE9
  \l 6AEA
  \l 6AEB
  \l 6AEC
  \l 6AED
  \l 6AEE
  \l 6AEF
  \l 6AF0
  \l 6AF1
  \l 6AF2
  \l 6AF3
  \l 6AF4
  \l 6AF5
  \l 6AF6
  \l 6AF7
  \l 6AF8
  \l 6AF9
  \l 6AFA
  \l 6AFB
  \l 6AFC
  \l 6AFD
  \l 6AFE
  \l 6AFF
  \l 6B00
  \l 6B01
  \l 6B02
  \l 6B03
  \l 6B04
  \l 6B05
  \l 6B06
  \l 6B07
  \l 6B08
  \l 6B09
  \l 6B0A
  \l 6B0B
  \l 6B0C
  \l 6B0D
  \l 6B0E
  \l 6B0F
  \l 6B10
  \l 6B11
  \l 6B12
  \l 6B13
  \l 6B14
  \l 6B15
  \l 6B16
  \l 6B17
  \l 6B18
  \l 6B19
  \l 6B1A
  \l 6B1B
  \l 6B1C
  \l 6B1D
  \l 6B1E
  \l 6B1F
  \l 6B20
  \l 6B21
  \l 6B22
  \l 6B23
  \l 6B24
  \l 6B25
  \l 6B26
  \l 6B27
  \l 6B28
  \l 6B29
  \l 6B2A
  \l 6B2B
  \l 6B2C
  \l 6B2D
  \l 6B2E
  \l 6B2F
  \l 6B30
  \l 6B31
  \l 6B32
  \l 6B33
  \l 6B34
  \l 6B35
  \l 6B36
  \l 6B37
  \l 6B38
  \l 6B39
  \l 6B3A
  \l 6B3B
  \l 6B3C
  \l 6B3D
  \l 6B3E
  \l 6B3F
  \l 6B40
  \l 6B41
  \l 6B42
  \l 6B43
  \l 6B44
  \l 6B45
  \l 6B46
  \l 6B47
  \l 6B48
  \l 6B49
  \l 6B4A
  \l 6B4B
  \l 6B4C
  \l 6B4D
  \l 6B4E
  \l 6B4F
  \l 6B50
  \l 6B51
  \l 6B52
  \l 6B53
  \l 6B54
  \l 6B55
  \l 6B56
  \l 6B57
  \l 6B58
  \l 6B59
  \l 6B5A
  \l 6B5B
  \l 6B5C
  \l 6B5D
  \l 6B5E
  \l 6B5F
  \l 6B60
  \l 6B61
  \l 6B62
  \l 6B63
  \l 6B64
  \l 6B65
  \l 6B66
  \l 6B67
  \l 6B68
  \l 6B69
  \l 6B6A
  \l 6B6B
  \l 6B6C
  \l 6B6D
  \l 6B6E
  \l 6B6F
  \l 6B70
  \l 6B71
  \l 6B72
  \l 6B73
  \l 6B74
  \l 6B75
  \l 6B76
  \l 6B77
  \l 6B78
  \l 6B79
  \l 6B7A
  \l 6B7B
  \l 6B7C
  \l 6B7D
  \l 6B7E
  \l 6B7F
  \l 6B80
  \l 6B81
  \l 6B82
  \l 6B83
  \l 6B84
  \l 6B85
  \l 6B86
  \l 6B87
  \l 6B88
  \l 6B89
  \l 6B8A
  \l 6B8B
  \l 6B8C
  \l 6B8D
  \l 6B8E
  \l 6B8F
  \l 6B90
  \l 6B91
  \l 6B92
  \l 6B93
  \l 6B94
  \l 6B95
  \l 6B96
  \l 6B97
  \l 6B98
  \l 6B99
  \l 6B9A
  \l 6B9B
  \l 6B9C
  \l 6B9D
  \l 6B9E
  \l 6B9F
  \l 6BA0
  \l 6BA1
  \l 6BA2
  \l 6BA3
  \l 6BA4
  \l 6BA5
  \l 6BA6
  \l 6BA7
  \l 6BA8
  \l 6BA9
  \l 6BAA
  \l 6BAB
  \l 6BAC
  \l 6BAD
  \l 6BAE
  \l 6BAF
  \l 6BB0
  \l 6BB1
  \l 6BB2
  \l 6BB3
  \l 6BB4
  \l 6BB5
  \l 6BB6
  \l 6BB7
  \l 6BB8
  \l 6BB9
  \l 6BBA
  \l 6BBB
  \l 6BBC
  \l 6BBD
  \l 6BBE
  \l 6BBF
  \l 6BC0
  \l 6BC1
  \l 6BC2
  \l 6BC3
  \l 6BC4
  \l 6BC5
  \l 6BC6
  \l 6BC7
  \l 6BC8
  \l 6BC9
  \l 6BCA
  \l 6BCB
  \l 6BCC
  \l 6BCD
  \l 6BCE
  \l 6BCF
  \l 6BD0
  \l 6BD1
  \l 6BD2
  \l 6BD3
  \l 6BD4
  \l 6BD5
  \l 6BD6
  \l 6BD7
  \l 6BD8
  \l 6BD9
  \l 6BDA
  \l 6BDB
  \l 6BDC
  \l 6BDD
  \l 6BDE
  \l 6BDF
  \l 6BE0
  \l 6BE1
  \l 6BE2
  \l 6BE3
  \l 6BE4
  \l 6BE5
  \l 6BE6
  \l 6BE7
  \l 6BE8
  \l 6BE9
  \l 6BEA
  \l 6BEB
  \l 6BEC
  \l 6BED
  \l 6BEE
  \l 6BEF
  \l 6BF0
  \l 6BF1
  \l 6BF2
  \l 6BF3
  \l 6BF4
  \l 6BF5
  \l 6BF6
  \l 6BF7
  \l 6BF8
  \l 6BF9
  \l 6BFA
  \l 6BFB
  \l 6BFC
  \l 6BFD
  \l 6BFE
  \l 6BFF
  \l 6C00
  \l 6C01
  \l 6C02
  \l 6C03
  \l 6C04
  \l 6C05
  \l 6C06
  \l 6C07
  \l 6C08
  \l 6C09
  \l 6C0A
  \l 6C0B
  \l 6C0C
  \l 6C0D
  \l 6C0E
  \l 6C0F
  \l 6C10
  \l 6C11
  \l 6C12
  \l 6C13
  \l 6C14
  \l 6C15
  \l 6C16
  \l 6C17
  \l 6C18
  \l 6C19
  \l 6C1A
  \l 6C1B
  \l 6C1C
  \l 6C1D
  \l 6C1E
  \l 6C1F
  \l 6C20
  \l 6C21
  \l 6C22
  \l 6C23
  \l 6C24
  \l 6C25
  \l 6C26
  \l 6C27
  \l 6C28
  \l 6C29
  \l 6C2A
  \l 6C2B
  \l 6C2C
  \l 6C2D
  \l 6C2E
  \l 6C2F
  \l 6C30
  \l 6C31
  \l 6C32
  \l 6C33
  \l 6C34
  \l 6C35
  \l 6C36
  \l 6C37
  \l 6C38
  \l 6C39
  \l 6C3A
  \l 6C3B
  \l 6C3C
  \l 6C3D
  \l 6C3E
  \l 6C3F
  \l 6C40
  \l 6C41
  \l 6C42
  \l 6C43
  \l 6C44
  \l 6C45
  \l 6C46
  \l 6C47
  \l 6C48
  \l 6C49
  \l 6C4A
  \l 6C4B
  \l 6C4C
  \l 6C4D
  \l 6C4E
  \l 6C4F
  \l 6C50
  \l 6C51
  \l 6C52
  \l 6C53
  \l 6C54
  \l 6C55
  \l 6C56
  \l 6C57
  \l 6C58
  \l 6C59
  \l 6C5A
  \l 6C5B
  \l 6C5C
  \l 6C5D
  \l 6C5E
  \l 6C5F
  \l 6C60
  \l 6C61
  \l 6C62
  \l 6C63
  \l 6C64
  \l 6C65
  \l 6C66
  \l 6C67
  \l 6C68
  \l 6C69
  \l 6C6A
  \l 6C6B
  \l 6C6C
  \l 6C6D
  \l 6C6E
  \l 6C6F
  \l 6C70
  \l 6C71
  \l 6C72
  \l 6C73
  \l 6C74
  \l 6C75
  \l 6C76
  \l 6C77
  \l 6C78
  \l 6C79
  \l 6C7A
  \l 6C7B
  \l 6C7C
  \l 6C7D
  \l 6C7E
  \l 6C7F
  \l 6C80
  \l 6C81
  \l 6C82
  \l 6C83
  \l 6C84
  \l 6C85
  \l 6C86
  \l 6C87
  \l 6C88
  \l 6C89
  \l 6C8A
  \l 6C8B
  \l 6C8C
  \l 6C8D
  \l 6C8E
  \l 6C8F
  \l 6C90
  \l 6C91
  \l 6C92
  \l 6C93
  \l 6C94
  \l 6C95
  \l 6C96
  \l 6C97
  \l 6C98
  \l 6C99
  \l 6C9A
  \l 6C9B
  \l 6C9C
  \l 6C9D
  \l 6C9E
  \l 6C9F
  \l 6CA0
  \l 6CA1
  \l 6CA2
  \l 6CA3
  \l 6CA4
  \l 6CA5
  \l 6CA6
  \l 6CA7
  \l 6CA8
  \l 6CA9
  \l 6CAA
  \l 6CAB
  \l 6CAC
  \l 6CAD
  \l 6CAE
  \l 6CAF
  \l 6CB0
  \l 6CB1
  \l 6CB2
  \l 6CB3
  \l 6CB4
  \l 6CB5
  \l 6CB6
  \l 6CB7
  \l 6CB8
  \l 6CB9
  \l 6CBA
  \l 6CBB
  \l 6CBC
  \l 6CBD
  \l 6CBE
  \l 6CBF
  \l 6CC0
  \l 6CC1
  \l 6CC2
  \l 6CC3
  \l 6CC4
  \l 6CC5
  \l 6CC6
  \l 6CC7
  \l 6CC8
  \l 6CC9
  \l 6CCA
  \l 6CCB
  \l 6CCC
  \l 6CCD
  \l 6CCE
  \l 6CCF
  \l 6CD0
  \l 6CD1
  \l 6CD2
  \l 6CD3
  \l 6CD4
  \l 6CD5
  \l 6CD6
  \l 6CD7
  \l 6CD8
  \l 6CD9
  \l 6CDA
  \l 6CDB
  \l 6CDC
  \l 6CDD
  \l 6CDE
  \l 6CDF
  \l 6CE0
  \l 6CE1
  \l 6CE2
  \l 6CE3
  \l 6CE4
  \l 6CE5
  \l 6CE6
  \l 6CE7
  \l 6CE8
  \l 6CE9
  \l 6CEA
  \l 6CEB
  \l 6CEC
  \l 6CED
  \l 6CEE
  \l 6CEF
  \l 6CF0
  \l 6CF1
  \l 6CF2
  \l 6CF3
  \l 6CF4
  \l 6CF5
  \l 6CF6
  \l 6CF7
  \l 6CF8
  \l 6CF9
  \l 6CFA
  \l 6CFB
  \l 6CFC
  \l 6CFD
  \l 6CFE
  \l 6CFF
  \l 6D00
  \l 6D01
  \l 6D02
  \l 6D03
  \l 6D04
  \l 6D05
  \l 6D06
  \l 6D07
  \l 6D08
  \l 6D09
  \l 6D0A
  \l 6D0B
  \l 6D0C
  \l 6D0D
  \l 6D0E
  \l 6D0F
  \l 6D10
  \l 6D11
  \l 6D12
  \l 6D13
  \l 6D14
  \l 6D15
  \l 6D16
  \l 6D17
  \l 6D18
  \l 6D19
  \l 6D1A
  \l 6D1B
  \l 6D1C
  \l 6D1D
  \l 6D1E
  \l 6D1F
  \l 6D20
  \l 6D21
  \l 6D22
  \l 6D23
  \l 6D24
  \l 6D25
  \l 6D26
  \l 6D27
  \l 6D28
  \l 6D29
  \l 6D2A
  \l 6D2B
  \l 6D2C
  \l 6D2D
  \l 6D2E
  \l 6D2F
  \l 6D30
  \l 6D31
  \l 6D32
  \l 6D33
  \l 6D34
  \l 6D35
  \l 6D36
  \l 6D37
  \l 6D38
  \l 6D39
  \l 6D3A
  \l 6D3B
  \l 6D3C
  \l 6D3D
  \l 6D3E
  \l 6D3F
  \l 6D40
  \l 6D41
  \l 6D42
  \l 6D43
  \l 6D44
  \l 6D45
  \l 6D46
  \l 6D47
  \l 6D48
  \l 6D49
  \l 6D4A
  \l 6D4B
  \l 6D4C
  \l 6D4D
  \l 6D4E
  \l 6D4F
  \l 6D50
  \l 6D51
  \l 6D52
  \l 6D53
  \l 6D54
  \l 6D55
  \l 6D56
  \l 6D57
  \l 6D58
  \l 6D59
  \l 6D5A
  \l 6D5B
  \l 6D5C
  \l 6D5D
  \l 6D5E
  \l 6D5F
  \l 6D60
  \l 6D61
  \l 6D62
  \l 6D63
  \l 6D64
  \l 6D65
  \l 6D66
  \l 6D67
  \l 6D68
  \l 6D69
  \l 6D6A
  \l 6D6B
  \l 6D6C
  \l 6D6D
  \l 6D6E
  \l 6D6F
  \l 6D70
  \l 6D71
  \l 6D72
  \l 6D73
  \l 6D74
  \l 6D75
  \l 6D76
  \l 6D77
  \l 6D78
  \l 6D79
  \l 6D7A
  \l 6D7B
  \l 6D7C
  \l 6D7D
  \l 6D7E
  \l 6D7F
  \l 6D80
  \l 6D81
  \l 6D82
  \l 6D83
  \l 6D84
  \l 6D85
  \l 6D86
  \l 6D87
  \l 6D88
  \l 6D89
  \l 6D8A
  \l 6D8B
  \l 6D8C
  \l 6D8D
  \l 6D8E
  \l 6D8F
  \l 6D90
  \l 6D91
  \l 6D92
  \l 6D93
  \l 6D94
  \l 6D95
  \l 6D96
  \l 6D97
  \l 6D98
  \l 6D99
  \l 6D9A
  \l 6D9B
  \l 6D9C
  \l 6D9D
  \l 6D9E
  \l 6D9F
  \l 6DA0
  \l 6DA1
  \l 6DA2
  \l 6DA3
  \l 6DA4
  \l 6DA5
  \l 6DA6
  \l 6DA7
  \l 6DA8
  \l 6DA9
  \l 6DAA
  \l 6DAB
  \l 6DAC
  \l 6DAD
  \l 6DAE
  \l 6DAF
  \l 6DB0
  \l 6DB1
  \l 6DB2
  \l 6DB3
  \l 6DB4
  \l 6DB5
  \l 6DB6
  \l 6DB7
  \l 6DB8
  \l 6DB9
  \l 6DBA
  \l 6DBB
  \l 6DBC
  \l 6DBD
  \l 6DBE
  \l 6DBF
  \l 6DC0
  \l 6DC1
  \l 6DC2
  \l 6DC3
  \l 6DC4
  \l 6DC5
  \l 6DC6
  \l 6DC7
  \l 6DC8
  \l 6DC9
  \l 6DCA
  \l 6DCB
  \l 6DCC
  \l 6DCD
  \l 6DCE
  \l 6DCF
  \l 6DD0
  \l 6DD1
  \l 6DD2
  \l 6DD3
  \l 6DD4
  \l 6DD5
  \l 6DD6
  \l 6DD7
  \l 6DD8
  \l 6DD9
  \l 6DDA
  \l 6DDB
  \l 6DDC
  \l 6DDD
  \l 6DDE
  \l 6DDF
  \l 6DE0
  \l 6DE1
  \l 6DE2
  \l 6DE3
  \l 6DE4
  \l 6DE5
  \l 6DE6
  \l 6DE7
  \l 6DE8
  \l 6DE9
  \l 6DEA
  \l 6DEB
  \l 6DEC
  \l 6DED
  \l 6DEE
  \l 6DEF
  \l 6DF0
  \l 6DF1
  \l 6DF2
  \l 6DF3
  \l 6DF4
  \l 6DF5
  \l 6DF6
  \l 6DF7
  \l 6DF8
  \l 6DF9
  \l 6DFA
  \l 6DFB
  \l 6DFC
  \l 6DFD
  \l 6DFE
  \l 6DFF
  \l 6E00
  \l 6E01
  \l 6E02
  \l 6E03
  \l 6E04
  \l 6E05
  \l 6E06
  \l 6E07
  \l 6E08
  \l 6E09
  \l 6E0A
  \l 6E0B
  \l 6E0C
  \l 6E0D
  \l 6E0E
  \l 6E0F
  \l 6E10
  \l 6E11
  \l 6E12
  \l 6E13
  \l 6E14
  \l 6E15
  \l 6E16
  \l 6E17
  \l 6E18
  \l 6E19
  \l 6E1A
  \l 6E1B
  \l 6E1C
  \l 6E1D
  \l 6E1E
  \l 6E1F
  \l 6E20
  \l 6E21
  \l 6E22
  \l 6E23
  \l 6E24
  \l 6E25
  \l 6E26
  \l 6E27
  \l 6E28
  \l 6E29
  \l 6E2A
  \l 6E2B
  \l 6E2C
  \l 6E2D
  \l 6E2E
  \l 6E2F
  \l 6E30
  \l 6E31
  \l 6E32
  \l 6E33
  \l 6E34
  \l 6E35
  \l 6E36
  \l 6E37
  \l 6E38
  \l 6E39
  \l 6E3A
  \l 6E3B
  \l 6E3C
  \l 6E3D
  \l 6E3E
  \l 6E3F
  \l 6E40
  \l 6E41
  \l 6E42
  \l 6E43
  \l 6E44
  \l 6E45
  \l 6E46
  \l 6E47
  \l 6E48
  \l 6E49
  \l 6E4A
  \l 6E4B
  \l 6E4C
  \l 6E4D
  \l 6E4E
  \l 6E4F
  \l 6E50
  \l 6E51
  \l 6E52
  \l 6E53
  \l 6E54
  \l 6E55
  \l 6E56
  \l 6E57
  \l 6E58
  \l 6E59
  \l 6E5A
  \l 6E5B
  \l 6E5C
  \l 6E5D
  \l 6E5E
  \l 6E5F
  \l 6E60
  \l 6E61
  \l 6E62
  \l 6E63
  \l 6E64
  \l 6E65
  \l 6E66
  \l 6E67
  \l 6E68
  \l 6E69
  \l 6E6A
  \l 6E6B
  \l 6E6C
  \l 6E6D
  \l 6E6E
  \l 6E6F
  \l 6E70
  \l 6E71
  \l 6E72
  \l 6E73
  \l 6E74
  \l 6E75
  \l 6E76
  \l 6E77
  \l 6E78
  \l 6E79
  \l 6E7A
  \l 6E7B
  \l 6E7C
  \l 6E7D
  \l 6E7E
  \l 6E7F
  \l 6E80
  \l 6E81
  \l 6E82
  \l 6E83
  \l 6E84
  \l 6E85
  \l 6E86
  \l 6E87
  \l 6E88
  \l 6E89
  \l 6E8A
  \l 6E8B
  \l 6E8C
  \l 6E8D
  \l 6E8E
  \l 6E8F
  \l 6E90
  \l 6E91
  \l 6E92
  \l 6E93
  \l 6E94
  \l 6E95
  \l 6E96
  \l 6E97
  \l 6E98
  \l 6E99
  \l 6E9A
  \l 6E9B
  \l 6E9C
  \l 6E9D
  \l 6E9E
  \l 6E9F
  \l 6EA0
  \l 6EA1
  \l 6EA2
  \l 6EA3
  \l 6EA4
  \l 6EA5
  \l 6EA6
  \l 6EA7
  \l 6EA8
  \l 6EA9
  \l 6EAA
  \l 6EAB
  \l 6EAC
  \l 6EAD
  \l 6EAE
  \l 6EAF
  \l 6EB0
  \l 6EB1
  \l 6EB2
  \l 6EB3
  \l 6EB4
  \l 6EB5
  \l 6EB6
  \l 6EB7
  \l 6EB8
  \l 6EB9
  \l 6EBA
  \l 6EBB
  \l 6EBC
  \l 6EBD
  \l 6EBE
  \l 6EBF
  \l 6EC0
  \l 6EC1
  \l 6EC2
  \l 6EC3
  \l 6EC4
  \l 6EC5
  \l 6EC6
  \l 6EC7
  \l 6EC8
  \l 6EC9
  \l 6ECA
  \l 6ECB
  \l 6ECC
  \l 6ECD
  \l 6ECE
  \l 6ECF
  \l 6ED0
  \l 6ED1
  \l 6ED2
  \l 6ED3
  \l 6ED4
  \l 6ED5
  \l 6ED6
  \l 6ED7
  \l 6ED8
  \l 6ED9
  \l 6EDA
  \l 6EDB
  \l 6EDC
  \l 6EDD
  \l 6EDE
  \l 6EDF
  \l 6EE0
  \l 6EE1
  \l 6EE2
  \l 6EE3
  \l 6EE4
  \l 6EE5
  \l 6EE6
  \l 6EE7
  \l 6EE8
  \l 6EE9
  \l 6EEA
  \l 6EEB
  \l 6EEC
  \l 6EED
  \l 6EEE
  \l 6EEF
  \l 6EF0
  \l 6EF1
  \l 6EF2
  \l 6EF3
  \l 6EF4
  \l 6EF5
  \l 6EF6
  \l 6EF7
  \l 6EF8
  \l 6EF9
  \l 6EFA
  \l 6EFB
  \l 6EFC
  \l 6EFD
  \l 6EFE
  \l 6EFF
  \l 6F00
  \l 6F01
  \l 6F02
  \l 6F03
  \l 6F04
  \l 6F05
  \l 6F06
  \l 6F07
  \l 6F08
  \l 6F09
  \l 6F0A
  \l 6F0B
  \l 6F0C
  \l 6F0D
  \l 6F0E
  \l 6F0F
  \l 6F10
  \l 6F11
  \l 6F12
  \l 6F13
  \l 6F14
  \l 6F15
  \l 6F16
  \l 6F17
  \l 6F18
  \l 6F19
  \l 6F1A
  \l 6F1B
  \l 6F1C
  \l 6F1D
  \l 6F1E
  \l 6F1F
  \l 6F20
  \l 6F21
  \l 6F22
  \l 6F23
  \l 6F24
  \l 6F25
  \l 6F26
  \l 6F27
  \l 6F28
  \l 6F29
  \l 6F2A
  \l 6F2B
  \l 6F2C
  \l 6F2D
  \l 6F2E
  \l 6F2F
  \l 6F30
  \l 6F31
  \l 6F32
  \l 6F33
  \l 6F34
  \l 6F35
  \l 6F36
  \l 6F37
  \l 6F38
  \l 6F39
  \l 6F3A
  \l 6F3B
  \l 6F3C
  \l 6F3D
  \l 6F3E
  \l 6F3F
  \l 6F40
  \l 6F41
  \l 6F42
  \l 6F43
  \l 6F44
  \l 6F45
  \l 6F46
  \l 6F47
  \l 6F48
  \l 6F49
  \l 6F4A
  \l 6F4B
  \l 6F4C
  \l 6F4D
  \l 6F4E
  \l 6F4F
  \l 6F50
  \l 6F51
  \l 6F52
  \l 6F53
  \l 6F54
  \l 6F55
  \l 6F56
  \l 6F57
  \l 6F58
  \l 6F59
  \l 6F5A
  \l 6F5B
  \l 6F5C
  \l 6F5D
  \l 6F5E
  \l 6F5F
  \l 6F60
  \l 6F61
  \l 6F62
  \l 6F63
  \l 6F64
  \l 6F65
  \l 6F66
  \l 6F67
  \l 6F68
  \l 6F69
  \l 6F6A
  \l 6F6B
  \l 6F6C
  \l 6F6D
  \l 6F6E
  \l 6F6F
  \l 6F70
  \l 6F71
  \l 6F72
  \l 6F73
  \l 6F74
  \l 6F75
  \l 6F76
  \l 6F77
  \l 6F78
  \l 6F79
  \l 6F7A
  \l 6F7B
  \l 6F7C
  \l 6F7D
  \l 6F7E
  \l 6F7F
  \l 6F80
  \l 6F81
  \l 6F82
  \l 6F83
  \l 6F84
  \l 6F85
  \l 6F86
  \l 6F87
  \l 6F88
  \l 6F89
  \l 6F8A
  \l 6F8B
  \l 6F8C
  \l 6F8D
  \l 6F8E
  \l 6F8F
  \l 6F90
  \l 6F91
  \l 6F92
  \l 6F93
  \l 6F94
  \l 6F95
  \l 6F96
  \l 6F97
  \l 6F98
  \l 6F99
  \l 6F9A
  \l 6F9B
  \l 6F9C
  \l 6F9D
  \l 6F9E
  \l 6F9F
  \l 6FA0
  \l 6FA1
  \l 6FA2
  \l 6FA3
  \l 6FA4
  \l 6FA5
  \l 6FA6
  \l 6FA7
  \l 6FA8
  \l 6FA9
  \l 6FAA
  \l 6FAB
  \l 6FAC
  \l 6FAD
  \l 6FAE
  \l 6FAF
  \l 6FB0
  \l 6FB1
  \l 6FB2
  \l 6FB3
  \l 6FB4
  \l 6FB5
  \l 6FB6
  \l 6FB7
  \l 6FB8
  \l 6FB9
  \l 6FBA
  \l 6FBB
  \l 6FBC
  \l 6FBD
  \l 6FBE
  \l 6FBF
  \l 6FC0
  \l 6FC1
  \l 6FC2
  \l 6FC3
  \l 6FC4
  \l 6FC5
  \l 6FC6
  \l 6FC7
  \l 6FC8
  \l 6FC9
  \l 6FCA
  \l 6FCB
  \l 6FCC
  \l 6FCD
  \l 6FCE
  \l 6FCF
  \l 6FD0
  \l 6FD1
  \l 6FD2
  \l 6FD3
  \l 6FD4
  \l 6FD5
  \l 6FD6
  \l 6FD7
  \l 6FD8
  \l 6FD9
  \l 6FDA
  \l 6FDB
  \l 6FDC
  \l 6FDD
  \l 6FDE
  \l 6FDF
  \l 6FE0
  \l 6FE1
  \l 6FE2
  \l 6FE3
  \l 6FE4
  \l 6FE5
  \l 6FE6
  \l 6FE7
  \l 6FE8
  \l 6FE9
  \l 6FEA
  \l 6FEB
  \l 6FEC
  \l 6FED
  \l 6FEE
  \l 6FEF
  \l 6FF0
  \l 6FF1
  \l 6FF2
  \l 6FF3
  \l 6FF4
  \l 6FF5
  \l 6FF6
  \l 6FF7
  \l 6FF8
  \l 6FF9
  \l 6FFA
  \l 6FFB
  \l 6FFC
  \l 6FFD
  \l 6FFE
  \l 6FFF
  \l 7000
  \l 7001
  \l 7002
  \l 7003
  \l 7004
  \l 7005
  \l 7006
  \l 7007
  \l 7008
  \l 7009
  \l 700A
  \l 700B
  \l 700C
  \l 700D
  \l 700E
  \l 700F
  \l 7010
  \l 7011
  \l 7012
  \l 7013
  \l 7014
  \l 7015
  \l 7016
  \l 7017
  \l 7018
  \l 7019
  \l 701A
  \l 701B
  \l 701C
  \l 701D
  \l 701E
  \l 701F
  \l 7020
  \l 7021
  \l 7022
  \l 7023
  \l 7024
  \l 7025
  \l 7026
  \l 7027
  \l 7028
  \l 7029
  \l 702A
  \l 702B
  \l 702C
  \l 702D
  \l 702E
  \l 702F
  \l 7030
  \l 7031
  \l 7032
  \l 7033
  \l 7034
  \l 7035
  \l 7036
  \l 7037
  \l 7038
  \l 7039
  \l 703A
  \l 703B
  \l 703C
  \l 703D
  \l 703E
  \l 703F
  \l 7040
  \l 7041
  \l 7042
  \l 7043
  \l 7044
  \l 7045
  \l 7046
  \l 7047
  \l 7048
  \l 7049
  \l 704A
  \l 704B
  \l 704C
  \l 704D
  \l 704E
  \l 704F
  \l 7050
  \l 7051
  \l 7052
  \l 7053
  \l 7054
  \l 7055
  \l 7056
  \l 7057
  \l 7058
  \l 7059
  \l 705A
  \l 705B
  \l 705C
  \l 705D
  \l 705E
  \l 705F
  \l 7060
  \l 7061
  \l 7062
  \l 7063
  \l 7064
  \l 7065
  \l 7066
  \l 7067
  \l 7068
  \l 7069
  \l 706A
  \l 706B
  \l 706C
  \l 706D
  \l 706E
  \l 706F
  \l 7070
  \l 7071
  \l 7072
  \l 7073
  \l 7074
  \l 7075
  \l 7076
  \l 7077
  \l 7078
  \l 7079
  \l 707A
  \l 707B
  \l 707C
  \l 707D
  \l 707E
  \l 707F
  \l 7080
  \l 7081
  \l 7082
  \l 7083
  \l 7084
  \l 7085
  \l 7086
  \l 7087
  \l 7088
  \l 7089
  \l 708A
  \l 708B
  \l 708C
  \l 708D
  \l 708E
  \l 708F
  \l 7090
  \l 7091
  \l 7092
  \l 7093
  \l 7094
  \l 7095
  \l 7096
  \l 7097
  \l 7098
  \l 7099
  \l 709A
  \l 709B
  \l 709C
  \l 709D
  \l 709E
  \l 709F
  \l 70A0
  \l 70A1
  \l 70A2
  \l 70A3
  \l 70A4
  \l 70A5
  \l 70A6
  \l 70A7
  \l 70A8
  \l 70A9
  \l 70AA
  \l 70AB
  \l 70AC
  \l 70AD
  \l 70AE
  \l 70AF
  \l 70B0
  \l 70B1
  \l 70B2
  \l 70B3
  \l 70B4
  \l 70B5
  \l 70B6
  \l 70B7
  \l 70B8
  \l 70B9
  \l 70BA
  \l 70BB
  \l 70BC
  \l 70BD
  \l 70BE
  \l 70BF
  \l 70C0
  \l 70C1
  \l 70C2
  \l 70C3
  \l 70C4
  \l 70C5
  \l 70C6
  \l 70C7
  \l 70C8
  \l 70C9
  \l 70CA
  \l 70CB
  \l 70CC
  \l 70CD
  \l 70CE
  \l 70CF
  \l 70D0
  \l 70D1
  \l 70D2
  \l 70D3
  \l 70D4
  \l 70D5
  \l 70D6
  \l 70D7
  \l 70D8
  \l 70D9
  \l 70DA
  \l 70DB
  \l 70DC
  \l 70DD
  \l 70DE
  \l 70DF
  \l 70E0
  \l 70E1
  \l 70E2
  \l 70E3
  \l 70E4
  \l 70E5
  \l 70E6
  \l 70E7
  \l 70E8
  \l 70E9
  \l 70EA
  \l 70EB
  \l 70EC
  \l 70ED
  \l 70EE
  \l 70EF
  \l 70F0
  \l 70F1
  \l 70F2
  \l 70F3
  \l 70F4
  \l 70F5
  \l 70F6
  \l 70F7
  \l 70F8
  \l 70F9
  \l 70FA
  \l 70FB
  \l 70FC
  \l 70FD
  \l 70FE
  \l 70FF
  \l 7100
  \l 7101
  \l 7102
  \l 7103
  \l 7104
  \l 7105
  \l 7106
  \l 7107
  \l 7108
  \l 7109
  \l 710A
  \l 710B
  \l 710C
  \l 710D
  \l 710E
  \l 710F
  \l 7110
  \l 7111
  \l 7112
  \l 7113
  \l 7114
  \l 7115
  \l 7116
  \l 7117
  \l 7118
  \l 7119
  \l 711A
  \l 711B
  \l 711C
  \l 711D
  \l 711E
  \l 711F
  \l 7120
  \l 7121
  \l 7122
  \l 7123
  \l 7124
  \l 7125
  \l 7126
  \l 7127
  \l 7128
  \l 7129
  \l 712A
  \l 712B
  \l 712C
  \l 712D
  \l 712E
  \l 712F
  \l 7130
  \l 7131
  \l 7132
  \l 7133
  \l 7134
  \l 7135
  \l 7136
  \l 7137
  \l 7138
  \l 7139
  \l 713A
  \l 713B
  \l 713C
  \l 713D
  \l 713E
  \l 713F
  \l 7140
  \l 7141
  \l 7142
  \l 7143
  \l 7144
  \l 7145
  \l 7146
  \l 7147
  \l 7148
  \l 7149
  \l 714A
  \l 714B
  \l 714C
  \l 714D
  \l 714E
  \l 714F
  \l 7150
  \l 7151
  \l 7152
  \l 7153
  \l 7154
  \l 7155
  \l 7156
  \l 7157
  \l 7158
  \l 7159
  \l 715A
  \l 715B
  \l 715C
  \l 715D
  \l 715E
  \l 715F
  \l 7160
  \l 7161
  \l 7162
  \l 7163
  \l 7164
  \l 7165
  \l 7166
  \l 7167
  \l 7168
  \l 7169
  \l 716A
  \l 716B
  \l 716C
  \l 716D
  \l 716E
  \l 716F
  \l 7170
  \l 7171
  \l 7172
  \l 7173
  \l 7174
  \l 7175
  \l 7176
  \l 7177
  \l 7178
  \l 7179
  \l 717A
  \l 717B
  \l 717C
  \l 717D
  \l 717E
  \l 717F
  \l 7180
  \l 7181
  \l 7182
  \l 7183
  \l 7184
  \l 7185
  \l 7186
  \l 7187
  \l 7188
  \l 7189
  \l 718A
  \l 718B
  \l 718C
  \l 718D
  \l 718E
  \l 718F
  \l 7190
  \l 7191
  \l 7192
  \l 7193
  \l 7194
  \l 7195
  \l 7196
  \l 7197
  \l 7198
  \l 7199
  \l 719A
  \l 719B
  \l 719C
  \l 719D
  \l 719E
  \l 719F
  \l 71A0
  \l 71A1
  \l 71A2
  \l 71A3
  \l 71A4
  \l 71A5
  \l 71A6
  \l 71A7
  \l 71A8
  \l 71A9
  \l 71AA
  \l 71AB
  \l 71AC
  \l 71AD
  \l 71AE
  \l 71AF
  \l 71B0
  \l 71B1
  \l 71B2
  \l 71B3
  \l 71B4
  \l 71B5
  \l 71B6
  \l 71B7
  \l 71B8
  \l 71B9
  \l 71BA
  \l 71BB
  \l 71BC
  \l 71BD
  \l 71BE
  \l 71BF
  \l 71C0
  \l 71C1
  \l 71C2
  \l 71C3
  \l 71C4
  \l 71C5
  \l 71C6
  \l 71C7
  \l 71C8
  \l 71C9
  \l 71CA
  \l 71CB
  \l 71CC
  \l 71CD
  \l 71CE
  \l 71CF
  \l 71D0
  \l 71D1
  \l 71D2
  \l 71D3
  \l 71D4
  \l 71D5
  \l 71D6
  \l 71D7
  \l 71D8
  \l 71D9
  \l 71DA
  \l 71DB
  \l 71DC
  \l 71DD
  \l 71DE
  \l 71DF
  \l 71E0
  \l 71E1
  \l 71E2
  \l 71E3
  \l 71E4
  \l 71E5
  \l 71E6
  \l 71E7
  \l 71E8
  \l 71E9
  \l 71EA
  \l 71EB
  \l 71EC
  \l 71ED
  \l 71EE
  \l 71EF
  \l 71F0
  \l 71F1
  \l 71F2
  \l 71F3
  \l 71F4
  \l 71F5
  \l 71F6
  \l 71F7
  \l 71F8
  \l 71F9
  \l 71FA
  \l 71FB
  \l 71FC
  \l 71FD
  \l 71FE
  \l 71FF
  \l 7200
  \l 7201
  \l 7202
  \l 7203
  \l 7204
  \l 7205
  \l 7206
  \l 7207
  \l 7208
  \l 7209
  \l 720A
  \l 720B
  \l 720C
  \l 720D
  \l 720E
  \l 720F
  \l 7210
  \l 7211
  \l 7212
  \l 7213
  \l 7214
  \l 7215
  \l 7216
  \l 7217
  \l 7218
  \l 7219
  \l 721A
  \l 721B
  \l 721C
  \l 721D
  \l 721E
  \l 721F
  \l 7220
  \l 7221
  \l 7222
  \l 7223
  \l 7224
  \l 7225
  \l 7226
  \l 7227
  \l 7228
  \l 7229
  \l 722A
  \l 722B
  \l 722C
  \l 722D
  \l 722E
  \l 722F
  \l 7230
  \l 7231
  \l 7232
  \l 7233
  \l 7234
  \l 7235
  \l 7236
  \l 7237
  \l 7238
  \l 7239
  \l 723A
  \l 723B
  \l 723C
  \l 723D
  \l 723E
  \l 723F
  \l 7240
  \l 7241
  \l 7242
  \l 7243
  \l 7244
  \l 7245
  \l 7246
  \l 7247
  \l 7248
  \l 7249
  \l 724A
  \l 724B
  \l 724C
  \l 724D
  \l 724E
  \l 724F
  \l 7250
  \l 7251
  \l 7252
  \l 7253
  \l 7254
  \l 7255
  \l 7256
  \l 7257
  \l 7258
  \l 7259
  \l 725A
  \l 725B
  \l 725C
  \l 725D
  \l 725E
  \l 725F
  \l 7260
  \l 7261
  \l 7262
  \l 7263
  \l 7264
  \l 7265
  \l 7266
  \l 7267
  \l 7268
  \l 7269
  \l 726A
  \l 726B
  \l 726C
  \l 726D
  \l 726E
  \l 726F
  \l 7270
  \l 7271
  \l 7272
  \l 7273
  \l 7274
  \l 7275
  \l 7276
  \l 7277
  \l 7278
  \l 7279
  \l 727A
  \l 727B
  \l 727C
  \l 727D
  \l 727E
  \l 727F
  \l 7280
  \l 7281
  \l 7282
  \l 7283
  \l 7284
  \l 7285
  \l 7286
  \l 7287
  \l 7288
  \l 7289
  \l 728A
  \l 728B
  \l 728C
  \l 728D
  \l 728E
  \l 728F
  \l 7290
  \l 7291
  \l 7292
  \l 7293
  \l 7294
  \l 7295
  \l 7296
  \l 7297
  \l 7298
  \l 7299
  \l 729A
  \l 729B
  \l 729C
  \l 729D
  \l 729E
  \l 729F
  \l 72A0
  \l 72A1
  \l 72A2
  \l 72A3
  \l 72A4
  \l 72A5
  \l 72A6
  \l 72A7
  \l 72A8
  \l 72A9
  \l 72AA
  \l 72AB
  \l 72AC
  \l 72AD
  \l 72AE
  \l 72AF
  \l 72B0
  \l 72B1
  \l 72B2
  \l 72B3
  \l 72B4
  \l 72B5
  \l 72B6
  \l 72B7
  \l 72B8
  \l 72B9
  \l 72BA
  \l 72BB
  \l 72BC
  \l 72BD
  \l 72BE
  \l 72BF
  \l 72C0
  \l 72C1
  \l 72C2
  \l 72C3
  \l 72C4
  \l 72C5
  \l 72C6
  \l 72C7
  \l 72C8
  \l 72C9
  \l 72CA
  \l 72CB
  \l 72CC
  \l 72CD
  \l 72CE
  \l 72CF
  \l 72D0
  \l 72D1
  \l 72D2
  \l 72D3
  \l 72D4
  \l 72D5
  \l 72D6
  \l 72D7
  \l 72D8
  \l 72D9
  \l 72DA
  \l 72DB
  \l 72DC
  \l 72DD
  \l 72DE
  \l 72DF
  \l 72E0
  \l 72E1
  \l 72E2
  \l 72E3
  \l 72E4
  \l 72E5
  \l 72E6
  \l 72E7
  \l 72E8
  \l 72E9
  \l 72EA
  \l 72EB
  \l 72EC
  \l 72ED
  \l 72EE
  \l 72EF
  \l 72F0
  \l 72F1
  \l 72F2
  \l 72F3
  \l 72F4
  \l 72F5
  \l 72F6
  \l 72F7
  \l 72F8
  \l 72F9
  \l 72FA
  \l 72FB
  \l 72FC
  \l 72FD
  \l 72FE
  \l 72FF
  \l 7300
  \l 7301
  \l 7302
  \l 7303
  \l 7304
  \l 7305
  \l 7306
  \l 7307
  \l 7308
  \l 7309
  \l 730A
  \l 730B
  \l 730C
  \l 730D
  \l 730E
  \l 730F
  \l 7310
  \l 7311
  \l 7312
  \l 7313
  \l 7314
  \l 7315
  \l 7316
  \l 7317
  \l 7318
  \l 7319
  \l 731A
  \l 731B
  \l 731C
  \l 731D
  \l 731E
  \l 731F
  \l 7320
  \l 7321
  \l 7322
  \l 7323
  \l 7324
  \l 7325
  \l 7326
  \l 7327
  \l 7328
  \l 7329
  \l 732A
  \l 732B
  \l 732C
  \l 732D
  \l 732E
  \l 732F
  \l 7330
  \l 7331
  \l 7332
  \l 7333
  \l 7334
  \l 7335
  \l 7336
  \l 7337
  \l 7338
  \l 7339
  \l 733A
  \l 733B
  \l 733C
  \l 733D
  \l 733E
  \l 733F
  \l 7340
  \l 7341
  \l 7342
  \l 7343
  \l 7344
  \l 7345
  \l 7346
  \l 7347
  \l 7348
  \l 7349
  \l 734A
  \l 734B
  \l 734C
  \l 734D
  \l 734E
  \l 734F
  \l 7350
  \l 7351
  \l 7352
  \l 7353
  \l 7354
  \l 7355
  \l 7356
  \l 7357
  \l 7358
  \l 7359
  \l 735A
  \l 735B
  \l 735C
  \l 735D
  \l 735E
  \l 735F
  \l 7360
  \l 7361
  \l 7362
  \l 7363
  \l 7364
  \l 7365
  \l 7366
  \l 7367
  \l 7368
  \l 7369
  \l 736A
  \l 736B
  \l 736C
  \l 736D
  \l 736E
  \l 736F
  \l 7370
  \l 7371
  \l 7372
  \l 7373
  \l 7374
  \l 7375
  \l 7376
  \l 7377
  \l 7378
  \l 7379
  \l 737A
  \l 737B
  \l 737C
  \l 737D
  \l 737E
  \l 737F
  \l 7380
  \l 7381
  \l 7382
  \l 7383
  \l 7384
  \l 7385
  \l 7386
  \l 7387
  \l 7388
  \l 7389
  \l 738A
  \l 738B
  \l 738C
  \l 738D
  \l 738E
  \l 738F
  \l 7390
  \l 7391
  \l 7392
  \l 7393
  \l 7394
  \l 7395
  \l 7396
  \l 7397
  \l 7398
  \l 7399
  \l 739A
  \l 739B
  \l 739C
  \l 739D
  \l 739E
  \l 739F
  \l 73A0
  \l 73A1
  \l 73A2
  \l 73A3
  \l 73A4
  \l 73A5
  \l 73A6
  \l 73A7
  \l 73A8
  \l 73A9
  \l 73AA
  \l 73AB
  \l 73AC
  \l 73AD
  \l 73AE
  \l 73AF
  \l 73B0
  \l 73B1
  \l 73B2
  \l 73B3
  \l 73B4
  \l 73B5
  \l 73B6
  \l 73B7
  \l 73B8
  \l 73B9
  \l 73BA
  \l 73BB
  \l 73BC
  \l 73BD
  \l 73BE
  \l 73BF
  \l 73C0
  \l 73C1
  \l 73C2
  \l 73C3
  \l 73C4
  \l 73C5
  \l 73C6
  \l 73C7
  \l 73C8
  \l 73C9
  \l 73CA
  \l 73CB
  \l 73CC
  \l 73CD
  \l 73CE
  \l 73CF
  \l 73D0
  \l 73D1
  \l 73D2
  \l 73D3
  \l 73D4
  \l 73D5
  \l 73D6
  \l 73D7
  \l 73D8
  \l 73D9
  \l 73DA
  \l 73DB
  \l 73DC
  \l 73DD
  \l 73DE
  \l 73DF
  \l 73E0
  \l 73E1
  \l 73E2
  \l 73E3
  \l 73E4
  \l 73E5
  \l 73E6
  \l 73E7
  \l 73E8
  \l 73E9
  \l 73EA
  \l 73EB
  \l 73EC
  \l 73ED
  \l 73EE
  \l 73EF
  \l 73F0
  \l 73F1
  \l 73F2
  \l 73F3
  \l 73F4
  \l 73F5
  \l 73F6
  \l 73F7
  \l 73F8
  \l 73F9
  \l 73FA
  \l 73FB
  \l 73FC
  \l 73FD
  \l 73FE
  \l 73FF
  \l 7400
  \l 7401
  \l 7402
  \l 7403
  \l 7404
  \l 7405
  \l 7406
  \l 7407
  \l 7408
  \l 7409
  \l 740A
  \l 740B
  \l 740C
  \l 740D
  \l 740E
  \l 740F
  \l 7410
  \l 7411
  \l 7412
  \l 7413
  \l 7414
  \l 7415
  \l 7416
  \l 7417
  \l 7418
  \l 7419
  \l 741A
  \l 741B
  \l 741C
  \l 741D
  \l 741E
  \l 741F
  \l 7420
  \l 7421
  \l 7422
  \l 7423
  \l 7424
  \l 7425
  \l 7426
  \l 7427
  \l 7428
  \l 7429
  \l 742A
  \l 742B
  \l 742C
  \l 742D
  \l 742E
  \l 742F
  \l 7430
  \l 7431
  \l 7432
  \l 7433
  \l 7434
  \l 7435
  \l 7436
  \l 7437
  \l 7438
  \l 7439
  \l 743A
  \l 743B
  \l 743C
  \l 743D
  \l 743E
  \l 743F
  \l 7440
  \l 7441
  \l 7442
  \l 7443
  \l 7444
  \l 7445
  \l 7446
  \l 7447
  \l 7448
  \l 7449
  \l 744A
  \l 744B
  \l 744C
  \l 744D
  \l 744E
  \l 744F
  \l 7450
  \l 7451
  \l 7452
  \l 7453
  \l 7454
  \l 7455
  \l 7456
  \l 7457
  \l 7458
  \l 7459
  \l 745A
  \l 745B
  \l 745C
  \l 745D
  \l 745E
  \l 745F
  \l 7460
  \l 7461
  \l 7462
  \l 7463
  \l 7464
  \l 7465
  \l 7466
  \l 7467
  \l 7468
  \l 7469
  \l 746A
  \l 746B
  \l 746C
  \l 746D
  \l 746E
  \l 746F
  \l 7470
  \l 7471
  \l 7472
  \l 7473
  \l 7474
  \l 7475
  \l 7476
  \l 7477
  \l 7478
  \l 7479
  \l 747A
  \l 747B
  \l 747C
  \l 747D
  \l 747E
  \l 747F
  \l 7480
  \l 7481
  \l 7482
  \l 7483
  \l 7484
  \l 7485
  \l 7486
  \l 7487
  \l 7488
  \l 7489
  \l 748A
  \l 748B
  \l 748C
  \l 748D
  \l 748E
  \l 748F
  \l 7490
  \l 7491
  \l 7492
  \l 7493
  \l 7494
  \l 7495
  \l 7496
  \l 7497
  \l 7498
  \l 7499
  \l 749A
  \l 749B
  \l 749C
  \l 749D
  \l 749E
  \l 749F
  \l 74A0
  \l 74A1
  \l 74A2
  \l 74A3
  \l 74A4
  \l 74A5
  \l 74A6
  \l 74A7
  \l 74A8
  \l 74A9
  \l 74AA
  \l 74AB
  \l 74AC
  \l 74AD
  \l 74AE
  \l 74AF
  \l 74B0
  \l 74B1
  \l 74B2
  \l 74B3
  \l 74B4
  \l 74B5
  \l 74B6
  \l 74B7
  \l 74B8
  \l 74B9
  \l 74BA
  \l 74BB
  \l 74BC
  \l 74BD
  \l 74BE
  \l 74BF
  \l 74C0
  \l 74C1
  \l 74C2
  \l 74C3
  \l 74C4
  \l 74C5
  \l 74C6
  \l 74C7
  \l 74C8
  \l 74C9
  \l 74CA
  \l 74CB
  \l 74CC
  \l 74CD
  \l 74CE
  \l 74CF
  \l 74D0
  \l 74D1
  \l 74D2
  \l 74D3
  \l 74D4
  \l 74D5
  \l 74D6
  \l 74D7
  \l 74D8
  \l 74D9
  \l 74DA
  \l 74DB
  \l 74DC
  \l 74DD
  \l 74DE
  \l 74DF
  \l 74E0
  \l 74E1
  \l 74E2
  \l 74E3
  \l 74E4
  \l 74E5
  \l 74E6
  \l 74E7
  \l 74E8
  \l 74E9
  \l 74EA
  \l 74EB
  \l 74EC
  \l 74ED
  \l 74EE
  \l 74EF
  \l 74F0
  \l 74F1
  \l 74F2
  \l 74F3
  \l 74F4
  \l 74F5
  \l 74F6
  \l 74F7
  \l 74F8
  \l 74F9
  \l 74FA
  \l 74FB
  \l 74FC
  \l 74FD
  \l 74FE
  \l 74FF
  \l 7500
  \l 7501
  \l 7502
  \l 7503
  \l 7504
  \l 7505
  \l 7506
  \l 7507
  \l 7508
  \l 7509
  \l 750A
  \l 750B
  \l 750C
  \l 750D
  \l 750E
  \l 750F
  \l 7510
  \l 7511
  \l 7512
  \l 7513
  \l 7514
  \l 7515
  \l 7516
  \l 7517
  \l 7518
  \l 7519
  \l 751A
  \l 751B
  \l 751C
  \l 751D
  \l 751E
  \l 751F
  \l 7520
  \l 7521
  \l 7522
  \l 7523
  \l 7524
  \l 7525
  \l 7526
  \l 7527
  \l 7528
  \l 7529
  \l 752A
  \l 752B
  \l 752C
  \l 752D
  \l 752E
  \l 752F
  \l 7530
  \l 7531
  \l 7532
  \l 7533
  \l 7534
  \l 7535
  \l 7536
  \l 7537
  \l 7538
  \l 7539
  \l 753A
  \l 753B
  \l 753C
  \l 753D
  \l 753E
  \l 753F
  \l 7540
  \l 7541
  \l 7542
  \l 7543
  \l 7544
  \l 7545
  \l 7546
  \l 7547
  \l 7548
  \l 7549
  \l 754A
  \l 754B
  \l 754C
  \l 754D
  \l 754E
  \l 754F
  \l 7550
  \l 7551
  \l 7552
  \l 7553
  \l 7554
  \l 7555
  \l 7556
  \l 7557
  \l 7558
  \l 7559
  \l 755A
  \l 755B
  \l 755C
  \l 755D
  \l 755E
  \l 755F
  \l 7560
  \l 7561
  \l 7562
  \l 7563
  \l 7564
  \l 7565
  \l 7566
  \l 7567
  \l 7568
  \l 7569
  \l 756A
  \l 756B
  \l 756C
  \l 756D
  \l 756E
  \l 756F
  \l 7570
  \l 7571
  \l 7572
  \l 7573
  \l 7574
  \l 7575
  \l 7576
  \l 7577
  \l 7578
  \l 7579
  \l 757A
  \l 757B
  \l 757C
  \l 757D
  \l 757E
  \l 757F
  \l 7580
  \l 7581
  \l 7582
  \l 7583
  \l 7584
  \l 7585
  \l 7586
  \l 7587
  \l 7588
  \l 7589
  \l 758A
  \l 758B
  \l 758C
  \l 758D
  \l 758E
  \l 758F
  \l 7590
  \l 7591
  \l 7592
  \l 7593
  \l 7594
  \l 7595
  \l 7596
  \l 7597
  \l 7598
  \l 7599
  \l 759A
  \l 759B
  \l 759C
  \l 759D
  \l 759E
  \l 759F
  \l 75A0
  \l 75A1
  \l 75A2
  \l 75A3
  \l 75A4
  \l 75A5
  \l 75A6
  \l 75A7
  \l 75A8
  \l 75A9
  \l 75AA
  \l 75AB
  \l 75AC
  \l 75AD
  \l 75AE
  \l 75AF
  \l 75B0
  \l 75B1
  \l 75B2
  \l 75B3
  \l 75B4
  \l 75B5
  \l 75B6
  \l 75B7
  \l 75B8
  \l 75B9
  \l 75BA
  \l 75BB
  \l 75BC
  \l 75BD
  \l 75BE
  \l 75BF
  \l 75C0
  \l 75C1
  \l 75C2
  \l 75C3
  \l 75C4
  \l 75C5
  \l 75C6
  \l 75C7
  \l 75C8
  \l 75C9
  \l 75CA
  \l 75CB
  \l 75CC
  \l 75CD
  \l 75CE
  \l 75CF
  \l 75D0
  \l 75D1
  \l 75D2
  \l 75D3
  \l 75D4
  \l 75D5
  \l 75D6
  \l 75D7
  \l 75D8
  \l 75D9
  \l 75DA
  \l 75DB
  \l 75DC
  \l 75DD
  \l 75DE
  \l 75DF
  \l 75E0
  \l 75E1
  \l 75E2
  \l 75E3
  \l 75E4
  \l 75E5
  \l 75E6
  \l 75E7
  \l 75E8
  \l 75E9
  \l 75EA
  \l 75EB
  \l 75EC
  \l 75ED
  \l 75EE
  \l 75EF
  \l 75F0
  \l 75F1
  \l 75F2
  \l 75F3
  \l 75F4
  \l 75F5
  \l 75F6
  \l 75F7
  \l 75F8
  \l 75F9
  \l 75FA
  \l 75FB
  \l 75FC
  \l 75FD
  \l 75FE
  \l 75FF
  \l 7600
  \l 7601
  \l 7602
  \l 7603
  \l 7604
  \l 7605
  \l 7606
  \l 7607
  \l 7608
  \l 7609
  \l 760A
  \l 760B
  \l 760C
  \l 760D
  \l 760E
  \l 760F
  \l 7610
  \l 7611
  \l 7612
  \l 7613
  \l 7614
  \l 7615
  \l 7616
  \l 7617
  \l 7618
  \l 7619
  \l 761A
  \l 761B
  \l 761C
  \l 761D
  \l 761E
  \l 761F
  \l 7620
  \l 7621
  \l 7622
  \l 7623
  \l 7624
  \l 7625
  \l 7626
  \l 7627
  \l 7628
  \l 7629
  \l 762A
  \l 762B
  \l 762C
  \l 762D
  \l 762E
  \l 762F
  \l 7630
  \l 7631
  \l 7632
  \l 7633
  \l 7634
  \l 7635
  \l 7636
  \l 7637
  \l 7638
  \l 7639
  \l 763A
  \l 763B
  \l 763C
  \l 763D
  \l 763E
  \l 763F
  \l 7640
  \l 7641
  \l 7642
  \l 7643
  \l 7644
  \l 7645
  \l 7646
  \l 7647
  \l 7648
  \l 7649
  \l 764A
  \l 764B
  \l 764C
  \l 764D
  \l 764E
  \l 764F
  \l 7650
  \l 7651
  \l 7652
  \l 7653
  \l 7654
  \l 7655
  \l 7656
  \l 7657
  \l 7658
  \l 7659
  \l 765A
  \l 765B
  \l 765C
  \l 765D
  \l 765E
  \l 765F
  \l 7660
  \l 7661
  \l 7662
  \l 7663
  \l 7664
  \l 7665
  \l 7666
  \l 7667
  \l 7668
  \l 7669
  \l 766A
  \l 766B
  \l 766C
  \l 766D
  \l 766E
  \l 766F
  \l 7670
  \l 7671
  \l 7672
  \l 7673
  \l 7674
  \l 7675
  \l 7676
  \l 7677
  \l 7678
  \l 7679
  \l 767A
  \l 767B
  \l 767C
  \l 767D
  \l 767E
  \l 767F
  \l 7680
  \l 7681
  \l 7682
  \l 7683
  \l 7684
  \l 7685
  \l 7686
  \l 7687
  \l 7688
  \l 7689
  \l 768A
  \l 768B
  \l 768C
  \l 768D
  \l 768E
  \l 768F
  \l 7690
  \l 7691
  \l 7692
  \l 7693
  \l 7694
  \l 7695
  \l 7696
  \l 7697
  \l 7698
  \l 7699
  \l 769A
  \l 769B
  \l 769C
  \l 769D
  \l 769E
  \l 769F
  \l 76A0
  \l 76A1
  \l 76A2
  \l 76A3
  \l 76A4
  \l 76A5
  \l 76A6
  \l 76A7
  \l 76A8
  \l 76A9
  \l 76AA
  \l 76AB
  \l 76AC
  \l 76AD
  \l 76AE
  \l 76AF
  \l 76B0
  \l 76B1
  \l 76B2
  \l 76B3
  \l 76B4
  \l 76B5
  \l 76B6
  \l 76B7
  \l 76B8
  \l 76B9
  \l 76BA
  \l 76BB
  \l 76BC
  \l 76BD
  \l 76BE
  \l 76BF
  \l 76C0
  \l 76C1
  \l 76C2
  \l 76C3
  \l 76C4
  \l 76C5
  \l 76C6
  \l 76C7
  \l 76C8
  \l 76C9
  \l 76CA
  \l 76CB
  \l 76CC
  \l 76CD
  \l 76CE
  \l 76CF
  \l 76D0
  \l 76D1
  \l 76D2
  \l 76D3
  \l 76D4
  \l 76D5
  \l 76D6
  \l 76D7
  \l 76D8
  \l 76D9
  \l 76DA
  \l 76DB
  \l 76DC
  \l 76DD
  \l 76DE
  \l 76DF
  \l 76E0
  \l 76E1
  \l 76E2
  \l 76E3
  \l 76E4
  \l 76E5
  \l 76E6
  \l 76E7
  \l 76E8
  \l 76E9
  \l 76EA
  \l 76EB
  \l 76EC
  \l 76ED
  \l 76EE
  \l 76EF
  \l 76F0
  \l 76F1
  \l 76F2
  \l 76F3
  \l 76F4
  \l 76F5
  \l 76F6
  \l 76F7
  \l 76F8
  \l 76F9
  \l 76FA
  \l 76FB
  \l 76FC
  \l 76FD
  \l 76FE
  \l 76FF
  \l 7700
  \l 7701
  \l 7702
  \l 7703
  \l 7704
  \l 7705
  \l 7706
  \l 7707
  \l 7708
  \l 7709
  \l 770A
  \l 770B
  \l 770C
  \l 770D
  \l 770E
  \l 770F
  \l 7710
  \l 7711
  \l 7712
  \l 7713
  \l 7714
  \l 7715
  \l 7716
  \l 7717
  \l 7718
  \l 7719
  \l 771A
  \l 771B
  \l 771C
  \l 771D
  \l 771E
  \l 771F
  \l 7720
  \l 7721
  \l 7722
  \l 7723
  \l 7724
  \l 7725
  \l 7726
  \l 7727
  \l 7728
  \l 7729
  \l 772A
  \l 772B
  \l 772C
  \l 772D
  \l 772E
  \l 772F
  \l 7730
  \l 7731
  \l 7732
  \l 7733
  \l 7734
  \l 7735
  \l 7736
  \l 7737
  \l 7738
  \l 7739
  \l 773A
  \l 773B
  \l 773C
  \l 773D
  \l 773E
  \l 773F
  \l 7740
  \l 7741
  \l 7742
  \l 7743
  \l 7744
  \l 7745
  \l 7746
  \l 7747
  \l 7748
  \l 7749
  \l 774A
  \l 774B
  \l 774C
  \l 774D
  \l 774E
  \l 774F
  \l 7750
  \l 7751
  \l 7752
  \l 7753
  \l 7754
  \l 7755
  \l 7756
  \l 7757
  \l 7758
  \l 7759
  \l 775A
  \l 775B
  \l 775C
  \l 775D
  \l 775E
  \l 775F
  \l 7760
  \l 7761
  \l 7762
  \l 7763
  \l 7764
  \l 7765
  \l 7766
  \l 7767
  \l 7768
  \l 7769
  \l 776A
  \l 776B
  \l 776C
  \l 776D
  \l 776E
  \l 776F
  \l 7770
  \l 7771
  \l 7772
  \l 7773
  \l 7774
  \l 7775
  \l 7776
  \l 7777
  \l 7778
  \l 7779
  \l 777A
  \l 777B
  \l 777C
  \l 777D
  \l 777E
  \l 777F
  \l 7780
  \l 7781
  \l 7782
  \l 7783
  \l 7784
  \l 7785
  \l 7786
  \l 7787
  \l 7788
  \l 7789
  \l 778A
  \l 778B
  \l 778C
  \l 778D
  \l 778E
  \l 778F
  \l 7790
  \l 7791
  \l 7792
  \l 7793
  \l 7794
  \l 7795
  \l 7796
  \l 7797
  \l 7798
  \l 7799
  \l 779A
  \l 779B
  \l 779C
  \l 779D
  \l 779E
  \l 779F
  \l 77A0
  \l 77A1
  \l 77A2
  \l 77A3
  \l 77A4
  \l 77A5
  \l 77A6
  \l 77A7
  \l 77A8
  \l 77A9
  \l 77AA
  \l 77AB
  \l 77AC
  \l 77AD
  \l 77AE
  \l 77AF
  \l 77B0
  \l 77B1
  \l 77B2
  \l 77B3
  \l 77B4
  \l 77B5
  \l 77B6
  \l 77B7
  \l 77B8
  \l 77B9
  \l 77BA
  \l 77BB
  \l 77BC
  \l 77BD
  \l 77BE
  \l 77BF
  \l 77C0
  \l 77C1
  \l 77C2
  \l 77C3
  \l 77C4
  \l 77C5
  \l 77C6
  \l 77C7
  \l 77C8
  \l 77C9
  \l 77CA
  \l 77CB
  \l 77CC
  \l 77CD
  \l 77CE
  \l 77CF
  \l 77D0
  \l 77D1
  \l 77D2
  \l 77D3
  \l 77D4
  \l 77D5
  \l 77D6
  \l 77D7
  \l 77D8
  \l 77D9
  \l 77DA
  \l 77DB
  \l 77DC
  \l 77DD
  \l 77DE
  \l 77DF
  \l 77E0
  \l 77E1
  \l 77E2
  \l 77E3
  \l 77E4
  \l 77E5
  \l 77E6
  \l 77E7
  \l 77E8
  \l 77E9
  \l 77EA
  \l 77EB
  \l 77EC
  \l 77ED
  \l 77EE
  \l 77EF
  \l 77F0
  \l 77F1
  \l 77F2
  \l 77F3
  \l 77F4
  \l 77F5
  \l 77F6
  \l 77F7
  \l 77F8
  \l 77F9
  \l 77FA
  \l 77FB
  \l 77FC
  \l 77FD
  \l 77FE
  \l 77FF
  \l 7800
  \l 7801
  \l 7802
  \l 7803
  \l 7804
  \l 7805
  \l 7806
  \l 7807
  \l 7808
  \l 7809
  \l 780A
  \l 780B
  \l 780C
  \l 780D
  \l 780E
  \l 780F
  \l 7810
  \l 7811
  \l 7812
  \l 7813
  \l 7814
  \l 7815
  \l 7816
  \l 7817
  \l 7818
  \l 7819
  \l 781A
  \l 781B
  \l 781C
  \l 781D
  \l 781E
  \l 781F
  \l 7820
  \l 7821
  \l 7822
  \l 7823
  \l 7824
  \l 7825
  \l 7826
  \l 7827
  \l 7828
  \l 7829
  \l 782A
  \l 782B
  \l 782C
  \l 782D
  \l 782E
  \l 782F
  \l 7830
  \l 7831
  \l 7832
  \l 7833
  \l 7834
  \l 7835
  \l 7836
  \l 7837
  \l 7838
  \l 7839
  \l 783A
  \l 783B
  \l 783C
  \l 783D
  \l 783E
  \l 783F
  \l 7840
  \l 7841
  \l 7842
  \l 7843
  \l 7844
  \l 7845
  \l 7846
  \l 7847
  \l 7848
  \l 7849
  \l 784A
  \l 784B
  \l 784C
  \l 784D
  \l 784E
  \l 784F
  \l 7850
  \l 7851
  \l 7852
  \l 7853
  \l 7854
  \l 7855
  \l 7856
  \l 7857
  \l 7858
  \l 7859
  \l 785A
  \l 785B
  \l 785C
  \l 785D
  \l 785E
  \l 785F
  \l 7860
  \l 7861
  \l 7862
  \l 7863
  \l 7864
  \l 7865
  \l 7866
  \l 7867
  \l 7868
  \l 7869
  \l 786A
  \l 786B
  \l 786C
  \l 786D
  \l 786E
  \l 786F
  \l 7870
  \l 7871
  \l 7872
  \l 7873
  \l 7874
  \l 7875
  \l 7876
  \l 7877
  \l 7878
  \l 7879
  \l 787A
  \l 787B
  \l 787C
  \l 787D
  \l 787E
  \l 787F
  \l 7880
  \l 7881
  \l 7882
  \l 7883
  \l 7884
  \l 7885
  \l 7886
  \l 7887
  \l 7888
  \l 7889
  \l 788A
  \l 788B
  \l 788C
  \l 788D
  \l 788E
  \l 788F
  \l 7890
  \l 7891
  \l 7892
  \l 7893
  \l 7894
  \l 7895
  \l 7896
  \l 7897
  \l 7898
  \l 7899
  \l 789A
  \l 789B
  \l 789C
  \l 789D
  \l 789E
  \l 789F
  \l 78A0
  \l 78A1
  \l 78A2
  \l 78A3
  \l 78A4
  \l 78A5
  \l 78A6
  \l 78A7
  \l 78A8
  \l 78A9
  \l 78AA
  \l 78AB
  \l 78AC
  \l 78AD
  \l 78AE
  \l 78AF
  \l 78B0
  \l 78B1
  \l 78B2
  \l 78B3
  \l 78B4
  \l 78B5
  \l 78B6
  \l 78B7
  \l 78B8
  \l 78B9
  \l 78BA
  \l 78BB
  \l 78BC
  \l 78BD
  \l 78BE
  \l 78BF
  \l 78C0
  \l 78C1
  \l 78C2
  \l 78C3
  \l 78C4
  \l 78C5
  \l 78C6
  \l 78C7
  \l 78C8
  \l 78C9
  \l 78CA
  \l 78CB
  \l 78CC
  \l 78CD
  \l 78CE
  \l 78CF
  \l 78D0
  \l 78D1
  \l 78D2
  \l 78D3
  \l 78D4
  \l 78D5
  \l 78D6
  \l 78D7
  \l 78D8
  \l 78D9
  \l 78DA
  \l 78DB
  \l 78DC
  \l 78DD
  \l 78DE
  \l 78DF
  \l 78E0
  \l 78E1
  \l 78E2
  \l 78E3
  \l 78E4
  \l 78E5
  \l 78E6
  \l 78E7
  \l 78E8
  \l 78E9
  \l 78EA
  \l 78EB
  \l 78EC
  \l 78ED
  \l 78EE
  \l 78EF
  \l 78F0
  \l 78F1
  \l 78F2
  \l 78F3
  \l 78F4
  \l 78F5
  \l 78F6
  \l 78F7
  \l 78F8
  \l 78F9
  \l 78FA
  \l 78FB
  \l 78FC
  \l 78FD
  \l 78FE
  \l 78FF
  \l 7900
  \l 7901
  \l 7902
  \l 7903
  \l 7904
  \l 7905
  \l 7906
  \l 7907
  \l 7908
  \l 7909
  \l 790A
  \l 790B
  \l 790C
  \l 790D
  \l 790E
  \l 790F
  \l 7910
  \l 7911
  \l 7912
  \l 7913
  \l 7914
  \l 7915
  \l 7916
  \l 7917
  \l 7918
  \l 7919
  \l 791A
  \l 791B
  \l 791C
  \l 791D
  \l 791E
  \l 791F
  \l 7920
  \l 7921
  \l 7922
  \l 7923
  \l 7924
  \l 7925
  \l 7926
  \l 7927
  \l 7928
  \l 7929
  \l 792A
  \l 792B
  \l 792C
  \l 792D
  \l 792E
  \l 792F
  \l 7930
  \l 7931
  \l 7932
  \l 7933
  \l 7934
  \l 7935
  \l 7936
  \l 7937
  \l 7938
  \l 7939
  \l 793A
  \l 793B
  \l 793C
  \l 793D
  \l 793E
  \l 793F
  \l 7940
  \l 7941
  \l 7942
  \l 7943
  \l 7944
  \l 7945
  \l 7946
  \l 7947
  \l 7948
  \l 7949
  \l 794A
  \l 794B
  \l 794C
  \l 794D
  \l 794E
  \l 794F
  \l 7950
  \l 7951
  \l 7952
  \l 7953
  \l 7954
  \l 7955
  \l 7956
  \l 7957
  \l 7958
  \l 7959
  \l 795A
  \l 795B
  \l 795C
  \l 795D
  \l 795E
  \l 795F
  \l 7960
  \l 7961
  \l 7962
  \l 7963
  \l 7964
  \l 7965
  \l 7966
  \l 7967
  \l 7968
  \l 7969
  \l 796A
  \l 796B
  \l 796C
  \l 796D
  \l 796E
  \l 796F
  \l 7970
  \l 7971
  \l 7972
  \l 7973
  \l 7974
  \l 7975
  \l 7976
  \l 7977
  \l 7978
  \l 7979
  \l 797A
  \l 797B
  \l 797C
  \l 797D
  \l 797E
  \l 797F
  \l 7980
  \l 7981
  \l 7982
  \l 7983
  \l 7984
  \l 7985
  \l 7986
  \l 7987
  \l 7988
  \l 7989
  \l 798A
  \l 798B
  \l 798C
  \l 798D
  \l 798E
  \l 798F
  \l 7990
  \l 7991
  \l 7992
  \l 7993
  \l 7994
  \l 7995
  \l 7996
  \l 7997
  \l 7998
  \l 7999
  \l 799A
  \l 799B
  \l 799C
  \l 799D
  \l 799E
  \l 799F
  \l 79A0
  \l 79A1
  \l 79A2
  \l 79A3
  \l 79A4
  \l 79A5
  \l 79A6
  \l 79A7
  \l 79A8
  \l 79A9
  \l 79AA
  \l 79AB
  \l 79AC
  \l 79AD
  \l 79AE
  \l 79AF
  \l 79B0
  \l 79B1
  \l 79B2
  \l 79B3
  \l 79B4
  \l 79B5
  \l 79B6
  \l 79B7
  \l 79B8
  \l 79B9
  \l 79BA
  \l 79BB
  \l 79BC
  \l 79BD
  \l 79BE
  \l 79BF
  \l 79C0
  \l 79C1
  \l 79C2
  \l 79C3
  \l 79C4
  \l 79C5
  \l 79C6
  \l 79C7
  \l 79C8
  \l 79C9
  \l 79CA
  \l 79CB
  \l 79CC
  \l 79CD
  \l 79CE
  \l 79CF
  \l 79D0
  \l 79D1
  \l 79D2
  \l 79D3
  \l 79D4
  \l 79D5
  \l 79D6
  \l 79D7
  \l 79D8
  \l 79D9
  \l 79DA
  \l 79DB
  \l 79DC
  \l 79DD
  \l 79DE
  \l 79DF
  \l 79E0
  \l 79E1
  \l 79E2
  \l 79E3
  \l 79E4
  \l 79E5
  \l 79E6
  \l 79E7
  \l 79E8
  \l 79E9
  \l 79EA
  \l 79EB
  \l 79EC
  \l 79ED
  \l 79EE
  \l 79EF
  \l 79F0
  \l 79F1
  \l 79F2
  \l 79F3
  \l 79F4
  \l 79F5
  \l 79F6
  \l 79F7
  \l 79F8
  \l 79F9
  \l 79FA
  \l 79FB
  \l 79FC
  \l 79FD
  \l 79FE
  \l 79FF
  \l 7A00
  \l 7A01
  \l 7A02
  \l 7A03
  \l 7A04
  \l 7A05
  \l 7A06
  \l 7A07
  \l 7A08
  \l 7A09
  \l 7A0A
  \l 7A0B
  \l 7A0C
  \l 7A0D
  \l 7A0E
  \l 7A0F
  \l 7A10
  \l 7A11
  \l 7A12
  \l 7A13
  \l 7A14
  \l 7A15
  \l 7A16
  \l 7A17
  \l 7A18
  \l 7A19
  \l 7A1A
  \l 7A1B
  \l 7A1C
  \l 7A1D
  \l 7A1E
  \l 7A1F
  \l 7A20
  \l 7A21
  \l 7A22
  \l 7A23
  \l 7A24
  \l 7A25
  \l 7A26
  \l 7A27
  \l 7A28
  \l 7A29
  \l 7A2A
  \l 7A2B
  \l 7A2C
  \l 7A2D
  \l 7A2E
  \l 7A2F
  \l 7A30
  \l 7A31
  \l 7A32
  \l 7A33
  \l 7A34
  \l 7A35
  \l 7A36
  \l 7A37
  \l 7A38
  \l 7A39
  \l 7A3A
  \l 7A3B
  \l 7A3C
  \l 7A3D
  \l 7A3E
  \l 7A3F
  \l 7A40
  \l 7A41
  \l 7A42
  \l 7A43
  \l 7A44
  \l 7A45
  \l 7A46
  \l 7A47
  \l 7A48
  \l 7A49
  \l 7A4A
  \l 7A4B
  \l 7A4C
  \l 7A4D
  \l 7A4E
  \l 7A4F
  \l 7A50
  \l 7A51
  \l 7A52
  \l 7A53
  \l 7A54
  \l 7A55
  \l 7A56
  \l 7A57
  \l 7A58
  \l 7A59
  \l 7A5A
  \l 7A5B
  \l 7A5C
  \l 7A5D
  \l 7A5E
  \l 7A5F
  \l 7A60
  \l 7A61
  \l 7A62
  \l 7A63
  \l 7A64
  \l 7A65
  \l 7A66
  \l 7A67
  \l 7A68
  \l 7A69
  \l 7A6A
  \l 7A6B
  \l 7A6C
  \l 7A6D
  \l 7A6E
  \l 7A6F
  \l 7A70
  \l 7A71
  \l 7A72
  \l 7A73
  \l 7A74
  \l 7A75
  \l 7A76
  \l 7A77
  \l 7A78
  \l 7A79
  \l 7A7A
  \l 7A7B
  \l 7A7C
  \l 7A7D
  \l 7A7E
  \l 7A7F
  \l 7A80
  \l 7A81
  \l 7A82
  \l 7A83
  \l 7A84
  \l 7A85
  \l 7A86
  \l 7A87
  \l 7A88
  \l 7A89
  \l 7A8A
  \l 7A8B
  \l 7A8C
  \l 7A8D
  \l 7A8E
  \l 7A8F
  \l 7A90
  \l 7A91
  \l 7A92
  \l 7A93
  \l 7A94
  \l 7A95
  \l 7A96
  \l 7A97
  \l 7A98
  \l 7A99
  \l 7A9A
  \l 7A9B
  \l 7A9C
  \l 7A9D
  \l 7A9E
  \l 7A9F
  \l 7AA0
  \l 7AA1
  \l 7AA2
  \l 7AA3
  \l 7AA4
  \l 7AA5
  \l 7AA6
  \l 7AA7
  \l 7AA8
  \l 7AA9
  \l 7AAA
  \l 7AAB
  \l 7AAC
  \l 7AAD
  \l 7AAE
  \l 7AAF
  \l 7AB0
  \l 7AB1
  \l 7AB2
  \l 7AB3
  \l 7AB4
  \l 7AB5
  \l 7AB6
  \l 7AB7
  \l 7AB8
  \l 7AB9
  \l 7ABA
  \l 7ABB
  \l 7ABC
  \l 7ABD
  \l 7ABE
  \l 7ABF
  \l 7AC0
  \l 7AC1
  \l 7AC2
  \l 7AC3
  \l 7AC4
  \l 7AC5
  \l 7AC6
  \l 7AC7
  \l 7AC8
  \l 7AC9
  \l 7ACA
  \l 7ACB
  \l 7ACC
  \l 7ACD
  \l 7ACE
  \l 7ACF
  \l 7AD0
  \l 7AD1
  \l 7AD2
  \l 7AD3
  \l 7AD4
  \l 7AD5
  \l 7AD6
  \l 7AD7
  \l 7AD8
  \l 7AD9
  \l 7ADA
  \l 7ADB
  \l 7ADC
  \l 7ADD
  \l 7ADE
  \l 7ADF
  \l 7AE0
  \l 7AE1
  \l 7AE2
  \l 7AE3
  \l 7AE4
  \l 7AE5
  \l 7AE6
  \l 7AE7
  \l 7AE8
  \l 7AE9
  \l 7AEA
  \l 7AEB
  \l 7AEC
  \l 7AED
  \l 7AEE
  \l 7AEF
  \l 7AF0
  \l 7AF1
  \l 7AF2
  \l 7AF3
  \l 7AF4
  \l 7AF5
  \l 7AF6
  \l 7AF7
  \l 7AF8
  \l 7AF9
  \l 7AFA
  \l 7AFB
  \l 7AFC
  \l 7AFD
  \l 7AFE
  \l 7AFF
  \l 7B00
  \l 7B01
  \l 7B02
  \l 7B03
  \l 7B04
  \l 7B05
  \l 7B06
  \l 7B07
  \l 7B08
  \l 7B09
  \l 7B0A
  \l 7B0B
  \l 7B0C
  \l 7B0D
  \l 7B0E
  \l 7B0F
  \l 7B10
  \l 7B11
  \l 7B12
  \l 7B13
  \l 7B14
  \l 7B15
  \l 7B16
  \l 7B17
  \l 7B18
  \l 7B19
  \l 7B1A
  \l 7B1B
  \l 7B1C
  \l 7B1D
  \l 7B1E
  \l 7B1F
  \l 7B20
  \l 7B21
  \l 7B22
  \l 7B23
  \l 7B24
  \l 7B25
  \l 7B26
  \l 7B27
  \l 7B28
  \l 7B29
  \l 7B2A
  \l 7B2B
  \l 7B2C
  \l 7B2D
  \l 7B2E
  \l 7B2F
  \l 7B30
  \l 7B31
  \l 7B32
  \l 7B33
  \l 7B34
  \l 7B35
  \l 7B36
  \l 7B37
  \l 7B38
  \l 7B39
  \l 7B3A
  \l 7B3B
  \l 7B3C
  \l 7B3D
  \l 7B3E
  \l 7B3F
  \l 7B40
  \l 7B41
  \l 7B42
  \l 7B43
  \l 7B44
  \l 7B45
  \l 7B46
  \l 7B47
  \l 7B48
  \l 7B49
  \l 7B4A
  \l 7B4B
  \l 7B4C
  \l 7B4D
  \l 7B4E
  \l 7B4F
  \l 7B50
  \l 7B51
  \l 7B52
  \l 7B53
  \l 7B54
  \l 7B55
  \l 7B56
  \l 7B57
  \l 7B58
  \l 7B59
  \l 7B5A
  \l 7B5B
  \l 7B5C
  \l 7B5D
  \l 7B5E
  \l 7B5F
  \l 7B60
  \l 7B61
  \l 7B62
  \l 7B63
  \l 7B64
  \l 7B65
  \l 7B66
  \l 7B67
  \l 7B68
  \l 7B69
  \l 7B6A
  \l 7B6B
  \l 7B6C
  \l 7B6D
  \l 7B6E
  \l 7B6F
  \l 7B70
  \l 7B71
  \l 7B72
  \l 7B73
  \l 7B74
  \l 7B75
  \l 7B76
  \l 7B77
  \l 7B78
  \l 7B79
  \l 7B7A
  \l 7B7B
  \l 7B7C
  \l 7B7D
  \l 7B7E
  \l 7B7F
  \l 7B80
  \l 7B81
  \l 7B82
  \l 7B83
  \l 7B84
  \l 7B85
  \l 7B86
  \l 7B87
  \l 7B88
  \l 7B89
  \l 7B8A
  \l 7B8B
  \l 7B8C
  \l 7B8D
  \l 7B8E
  \l 7B8F
  \l 7B90
  \l 7B91
  \l 7B92
  \l 7B93
  \l 7B94
  \l 7B95
  \l 7B96
  \l 7B97
  \l 7B98
  \l 7B99
  \l 7B9A
  \l 7B9B
  \l 7B9C
  \l 7B9D
  \l 7B9E
  \l 7B9F
  \l 7BA0
  \l 7BA1
  \l 7BA2
  \l 7BA3
  \l 7BA4
  \l 7BA5
  \l 7BA6
  \l 7BA7
  \l 7BA8
  \l 7BA9
  \l 7BAA
  \l 7BAB
  \l 7BAC
  \l 7BAD
  \l 7BAE
  \l 7BAF
  \l 7BB0
  \l 7BB1
  \l 7BB2
  \l 7BB3
  \l 7BB4
  \l 7BB5
  \l 7BB6
  \l 7BB7
  \l 7BB8
  \l 7BB9
  \l 7BBA
  \l 7BBB
  \l 7BBC
  \l 7BBD
  \l 7BBE
  \l 7BBF
  \l 7BC0
  \l 7BC1
  \l 7BC2
  \l 7BC3
  \l 7BC4
  \l 7BC5
  \l 7BC6
  \l 7BC7
  \l 7BC8
  \l 7BC9
  \l 7BCA
  \l 7BCB
  \l 7BCC
  \l 7BCD
  \l 7BCE
  \l 7BCF
  \l 7BD0
  \l 7BD1
  \l 7BD2
  \l 7BD3
  \l 7BD4
  \l 7BD5
  \l 7BD6
  \l 7BD7
  \l 7BD8
  \l 7BD9
  \l 7BDA
  \l 7BDB
  \l 7BDC
  \l 7BDD
  \l 7BDE
  \l 7BDF
  \l 7BE0
  \l 7BE1
  \l 7BE2
  \l 7BE3
  \l 7BE4
  \l 7BE5
  \l 7BE6
  \l 7BE7
  \l 7BE8
  \l 7BE9
  \l 7BEA
  \l 7BEB
  \l 7BEC
  \l 7BED
  \l 7BEE
  \l 7BEF
  \l 7BF0
  \l 7BF1
  \l 7BF2
  \l 7BF3
  \l 7BF4
  \l 7BF5
  \l 7BF6
  \l 7BF7
  \l 7BF8
  \l 7BF9
  \l 7BFA
  \l 7BFB
  \l 7BFC
  \l 7BFD
  \l 7BFE
  \l 7BFF
  \l 7C00
  \l 7C01
  \l 7C02
  \l 7C03
  \l 7C04
  \l 7C05
  \l 7C06
  \l 7C07
  \l 7C08
  \l 7C09
  \l 7C0A
  \l 7C0B
  \l 7C0C
  \l 7C0D
  \l 7C0E
  \l 7C0F
  \l 7C10
  \l 7C11
  \l 7C12
  \l 7C13
  \l 7C14
  \l 7C15
  \l 7C16
  \l 7C17
  \l 7C18
  \l 7C19
  \l 7C1A
  \l 7C1B
  \l 7C1C
  \l 7C1D
  \l 7C1E
  \l 7C1F
  \l 7C20
  \l 7C21
  \l 7C22
  \l 7C23
  \l 7C24
  \l 7C25
  \l 7C26
  \l 7C27
  \l 7C28
  \l 7C29
  \l 7C2A
  \l 7C2B
  \l 7C2C
  \l 7C2D
  \l 7C2E
  \l 7C2F
  \l 7C30
  \l 7C31
  \l 7C32
  \l 7C33
  \l 7C34
  \l 7C35
  \l 7C36
  \l 7C37
  \l 7C38
  \l 7C39
  \l 7C3A
  \l 7C3B
  \l 7C3C
  \l 7C3D
  \l 7C3E
  \l 7C3F
  \l 7C40
  \l 7C41
  \l 7C42
  \l 7C43
  \l 7C44
  \l 7C45
  \l 7C46
  \l 7C47
  \l 7C48
  \l 7C49
  \l 7C4A
  \l 7C4B
  \l 7C4C
  \l 7C4D
  \l 7C4E
  \l 7C4F
  \l 7C50
  \l 7C51
  \l 7C52
  \l 7C53
  \l 7C54
  \l 7C55
  \l 7C56
  \l 7C57
  \l 7C58
  \l 7C59
  \l 7C5A
  \l 7C5B
  \l 7C5C
  \l 7C5D
  \l 7C5E
  \l 7C5F
  \l 7C60
  \l 7C61
  \l 7C62
  \l 7C63
  \l 7C64
  \l 7C65
  \l 7C66
  \l 7C67
  \l 7C68
  \l 7C69
  \l 7C6A
  \l 7C6B
  \l 7C6C
  \l 7C6D
  \l 7C6E
  \l 7C6F
  \l 7C70
  \l 7C71
  \l 7C72
  \l 7C73
  \l 7C74
  \l 7C75
  \l 7C76
  \l 7C77
  \l 7C78
  \l 7C79
  \l 7C7A
  \l 7C7B
  \l 7C7C
  \l 7C7D
  \l 7C7E
  \l 7C7F
  \l 7C80
  \l 7C81
  \l 7C82
  \l 7C83
  \l 7C84
  \l 7C85
  \l 7C86
  \l 7C87
  \l 7C88
  \l 7C89
  \l 7C8A
  \l 7C8B
  \l 7C8C
  \l 7C8D
  \l 7C8E
  \l 7C8F
  \l 7C90
  \l 7C91
  \l 7C92
  \l 7C93
  \l 7C94
  \l 7C95
  \l 7C96
  \l 7C97
  \l 7C98
  \l 7C99
  \l 7C9A
  \l 7C9B
  \l 7C9C
  \l 7C9D
  \l 7C9E
  \l 7C9F
  \l 7CA0
  \l 7CA1
  \l 7CA2
  \l 7CA3
  \l 7CA4
  \l 7CA5
  \l 7CA6
  \l 7CA7
  \l 7CA8
  \l 7CA9
  \l 7CAA
  \l 7CAB
  \l 7CAC
  \l 7CAD
  \l 7CAE
  \l 7CAF
  \l 7CB0
  \l 7CB1
  \l 7CB2
  \l 7CB3
  \l 7CB4
  \l 7CB5
  \l 7CB6
  \l 7CB7
  \l 7CB8
  \l 7CB9
  \l 7CBA
  \l 7CBB
  \l 7CBC
  \l 7CBD
  \l 7CBE
  \l 7CBF
  \l 7CC0
  \l 7CC1
  \l 7CC2
  \l 7CC3
  \l 7CC4
  \l 7CC5
  \l 7CC6
  \l 7CC7
  \l 7CC8
  \l 7CC9
  \l 7CCA
  \l 7CCB
  \l 7CCC
  \l 7CCD
  \l 7CCE
  \l 7CCF
  \l 7CD0
  \l 7CD1
  \l 7CD2
  \l 7CD3
  \l 7CD4
  \l 7CD5
  \l 7CD6
  \l 7CD7
  \l 7CD8
  \l 7CD9
  \l 7CDA
  \l 7CDB
  \l 7CDC
  \l 7CDD
  \l 7CDE
  \l 7CDF
  \l 7CE0
  \l 7CE1
  \l 7CE2
  \l 7CE3
  \l 7CE4
  \l 7CE5
  \l 7CE6
  \l 7CE7
  \l 7CE8
  \l 7CE9
  \l 7CEA
  \l 7CEB
  \l 7CEC
  \l 7CED
  \l 7CEE
  \l 7CEF
  \l 7CF0
  \l 7CF1
  \l 7CF2
  \l 7CF3
  \l 7CF4
  \l 7CF5
  \l 7CF6
  \l 7CF7
  \l 7CF8
  \l 7CF9
  \l 7CFA
  \l 7CFB
  \l 7CFC
  \l 7CFD
  \l 7CFE
  \l 7CFF
  \l 7D00
  \l 7D01
  \l 7D02
  \l 7D03
  \l 7D04
  \l 7D05
  \l 7D06
  \l 7D07
  \l 7D08
  \l 7D09
  \l 7D0A
  \l 7D0B
  \l 7D0C
  \l 7D0D
  \l 7D0E
  \l 7D0F
  \l 7D10
  \l 7D11
  \l 7D12
  \l 7D13
  \l 7D14
  \l 7D15
  \l 7D16
  \l 7D17
  \l 7D18
  \l 7D19
  \l 7D1A
  \l 7D1B
  \l 7D1C
  \l 7D1D
  \l 7D1E
  \l 7D1F
  \l 7D20
  \l 7D21
  \l 7D22
  \l 7D23
  \l 7D24
  \l 7D25
  \l 7D26
  \l 7D27
  \l 7D28
  \l 7D29
  \l 7D2A
  \l 7D2B
  \l 7D2C
  \l 7D2D
  \l 7D2E
  \l 7D2F
  \l 7D30
  \l 7D31
  \l 7D32
  \l 7D33
  \l 7D34
  \l 7D35
  \l 7D36
  \l 7D37
  \l 7D38
  \l 7D39
  \l 7D3A
  \l 7D3B
  \l 7D3C
  \l 7D3D
  \l 7D3E
  \l 7D3F
  \l 7D40
  \l 7D41
  \l 7D42
  \l 7D43
  \l 7D44
  \l 7D45
  \l 7D46
  \l 7D47
  \l 7D48
  \l 7D49
  \l 7D4A
  \l 7D4B
  \l 7D4C
  \l 7D4D
  \l 7D4E
  \l 7D4F
  \l 7D50
  \l 7D51
  \l 7D52
  \l 7D53
  \l 7D54
  \l 7D55
  \l 7D56
  \l 7D57
  \l 7D58
  \l 7D59
  \l 7D5A
  \l 7D5B
  \l 7D5C
  \l 7D5D
  \l 7D5E
  \l 7D5F
  \l 7D60
  \l 7D61
  \l 7D62
  \l 7D63
  \l 7D64
  \l 7D65
  \l 7D66
  \l 7D67
  \l 7D68
  \l 7D69
  \l 7D6A
  \l 7D6B
  \l 7D6C
  \l 7D6D
  \l 7D6E
  \l 7D6F
  \l 7D70
  \l 7D71
  \l 7D72
  \l 7D73
  \l 7D74
  \l 7D75
  \l 7D76
  \l 7D77
  \l 7D78
  \l 7D79
  \l 7D7A
  \l 7D7B
  \l 7D7C
  \l 7D7D
  \l 7D7E
  \l 7D7F
  \l 7D80
  \l 7D81
  \l 7D82
  \l 7D83
  \l 7D84
  \l 7D85
  \l 7D86
  \l 7D87
  \l 7D88
  \l 7D89
  \l 7D8A
  \l 7D8B
  \l 7D8C
  \l 7D8D
  \l 7D8E
  \l 7D8F
  \l 7D90
  \l 7D91
  \l 7D92
  \l 7D93
  \l 7D94
  \l 7D95
  \l 7D96
  \l 7D97
  \l 7D98
  \l 7D99
  \l 7D9A
  \l 7D9B
  \l 7D9C
  \l 7D9D
  \l 7D9E
  \l 7D9F
  \l 7DA0
  \l 7DA1
  \l 7DA2
  \l 7DA3
  \l 7DA4
  \l 7DA5
  \l 7DA6
  \l 7DA7
  \l 7DA8
  \l 7DA9
  \l 7DAA
  \l 7DAB
  \l 7DAC
  \l 7DAD
  \l 7DAE
  \l 7DAF
  \l 7DB0
  \l 7DB1
  \l 7DB2
  \l 7DB3
  \l 7DB4
  \l 7DB5
  \l 7DB6
  \l 7DB7
  \l 7DB8
  \l 7DB9
  \l 7DBA
  \l 7DBB
  \l 7DBC
  \l 7DBD
  \l 7DBE
  \l 7DBF
  \l 7DC0
  \l 7DC1
  \l 7DC2
  \l 7DC3
  \l 7DC4
  \l 7DC5
  \l 7DC6
  \l 7DC7
  \l 7DC8
  \l 7DC9
  \l 7DCA
  \l 7DCB
  \l 7DCC
  \l 7DCD
  \l 7DCE
  \l 7DCF
  \l 7DD0
  \l 7DD1
  \l 7DD2
  \l 7DD3
  \l 7DD4
  \l 7DD5
  \l 7DD6
  \l 7DD7
  \l 7DD8
  \l 7DD9
  \l 7DDA
  \l 7DDB
  \l 7DDC
  \l 7DDD
  \l 7DDE
  \l 7DDF
  \l 7DE0
  \l 7DE1
  \l 7DE2
  \l 7DE3
  \l 7DE4
  \l 7DE5
  \l 7DE6
  \l 7DE7
  \l 7DE8
  \l 7DE9
  \l 7DEA
  \l 7DEB
  \l 7DEC
  \l 7DED
  \l 7DEE
  \l 7DEF
  \l 7DF0
  \l 7DF1
  \l 7DF2
  \l 7DF3
  \l 7DF4
  \l 7DF5
  \l 7DF6
  \l 7DF7
  \l 7DF8
  \l 7DF9
  \l 7DFA
  \l 7DFB
  \l 7DFC
  \l 7DFD
  \l 7DFE
  \l 7DFF
  \l 7E00
  \l 7E01
  \l 7E02
  \l 7E03
  \l 7E04
  \l 7E05
  \l 7E06
  \l 7E07
  \l 7E08
  \l 7E09
  \l 7E0A
  \l 7E0B
  \l 7E0C
  \l 7E0D
  \l 7E0E
  \l 7E0F
  \l 7E10
  \l 7E11
  \l 7E12
  \l 7E13
  \l 7E14
  \l 7E15
  \l 7E16
  \l 7E17
  \l 7E18
  \l 7E19
  \l 7E1A
  \l 7E1B
  \l 7E1C
  \l 7E1D
  \l 7E1E
  \l 7E1F
  \l 7E20
  \l 7E21
  \l 7E22
  \l 7E23
  \l 7E24
  \l 7E25
  \l 7E26
  \l 7E27
  \l 7E28
  \l 7E29
  \l 7E2A
  \l 7E2B
  \l 7E2C
  \l 7E2D
  \l 7E2E
  \l 7E2F
  \l 7E30
  \l 7E31
  \l 7E32
  \l 7E33
  \l 7E34
  \l 7E35
  \l 7E36
  \l 7E37
  \l 7E38
  \l 7E39
  \l 7E3A
  \l 7E3B
  \l 7E3C
  \l 7E3D
  \l 7E3E
  \l 7E3F
  \l 7E40
  \l 7E41
  \l 7E42
  \l 7E43
  \l 7E44
  \l 7E45
  \l 7E46
  \l 7E47
  \l 7E48
  \l 7E49
  \l 7E4A
  \l 7E4B
  \l 7E4C
  \l 7E4D
  \l 7E4E
  \l 7E4F
  \l 7E50
  \l 7E51
  \l 7E52
  \l 7E53
  \l 7E54
  \l 7E55
  \l 7E56
  \l 7E57
  \l 7E58
  \l 7E59
  \l 7E5A
  \l 7E5B
  \l 7E5C
  \l 7E5D
  \l 7E5E
  \l 7E5F
  \l 7E60
  \l 7E61
  \l 7E62
  \l 7E63
  \l 7E64
  \l 7E65
  \l 7E66
  \l 7E67
  \l 7E68
  \l 7E69
  \l 7E6A
  \l 7E6B
  \l 7E6C
  \l 7E6D
  \l 7E6E
  \l 7E6F
  \l 7E70
  \l 7E71
  \l 7E72
  \l 7E73
  \l 7E74
  \l 7E75
  \l 7E76
  \l 7E77
  \l 7E78
  \l 7E79
  \l 7E7A
  \l 7E7B
  \l 7E7C
  \l 7E7D
  \l 7E7E
  \l 7E7F
  \l 7E80
  \l 7E81
  \l 7E82
  \l 7E83
  \l 7E84
  \l 7E85
  \l 7E86
  \l 7E87
  \l 7E88
  \l 7E89
  \l 7E8A
  \l 7E8B
  \l 7E8C
  \l 7E8D
  \l 7E8E
  \l 7E8F
  \l 7E90
  \l 7E91
  \l 7E92
  \l 7E93
  \l 7E94
  \l 7E95
  \l 7E96
  \l 7E97
  \l 7E98
  \l 7E99
  \l 7E9A
  \l 7E9B
  \l 7E9C
  \l 7E9D
  \l 7E9E
  \l 7E9F
  \l 7EA0
  \l 7EA1
  \l 7EA2
  \l 7EA3
  \l 7EA4
  \l 7EA5
  \l 7EA6
  \l 7EA7
  \l 7EA8
  \l 7EA9
  \l 7EAA
  \l 7EAB
  \l 7EAC
  \l 7EAD
  \l 7EAE
  \l 7EAF
  \l 7EB0
  \l 7EB1
  \l 7EB2
  \l 7EB3
  \l 7EB4
  \l 7EB5
  \l 7EB6
  \l 7EB7
  \l 7EB8
  \l 7EB9
  \l 7EBA
  \l 7EBB
  \l 7EBC
  \l 7EBD
  \l 7EBE
  \l 7EBF
  \l 7EC0
  \l 7EC1
  \l 7EC2
  \l 7EC3
  \l 7EC4
  \l 7EC5
  \l 7EC6
  \l 7EC7
  \l 7EC8
  \l 7EC9
  \l 7ECA
  \l 7ECB
  \l 7ECC
  \l 7ECD
  \l 7ECE
  \l 7ECF
  \l 7ED0
  \l 7ED1
  \l 7ED2
  \l 7ED3
  \l 7ED4
  \l 7ED5
  \l 7ED6
  \l 7ED7
  \l 7ED8
  \l 7ED9
  \l 7EDA
  \l 7EDB
  \l 7EDC
  \l 7EDD
  \l 7EDE
  \l 7EDF
  \l 7EE0
  \l 7EE1
  \l 7EE2
  \l 7EE3
  \l 7EE4
  \l 7EE5
  \l 7EE6
  \l 7EE7
  \l 7EE8
  \l 7EE9
  \l 7EEA
  \l 7EEB
  \l 7EEC
  \l 7EED
  \l 7EEE
  \l 7EEF
  \l 7EF0
  \l 7EF1
  \l 7EF2
  \l 7EF3
  \l 7EF4
  \l 7EF5
  \l 7EF6
  \l 7EF7
  \l 7EF8
  \l 7EF9
  \l 7EFA
  \l 7EFB
  \l 7EFC
  \l 7EFD
  \l 7EFE
  \l 7EFF
  \l 7F00
  \l 7F01
  \l 7F02
  \l 7F03
  \l 7F04
  \l 7F05
  \l 7F06
  \l 7F07
  \l 7F08
  \l 7F09
  \l 7F0A
  \l 7F0B
  \l 7F0C
  \l 7F0D
  \l 7F0E
  \l 7F0F
  \l 7F10
  \l 7F11
  \l 7F12
  \l 7F13
  \l 7F14
  \l 7F15
  \l 7F16
  \l 7F17
  \l 7F18
  \l 7F19
  \l 7F1A
  \l 7F1B
  \l 7F1C
  \l 7F1D
  \l 7F1E
  \l 7F1F
  \l 7F20
  \l 7F21
  \l 7F22
  \l 7F23
  \l 7F24
  \l 7F25
  \l 7F26
  \l 7F27
  \l 7F28
  \l 7F29
  \l 7F2A
  \l 7F2B
  \l 7F2C
  \l 7F2D
  \l 7F2E
  \l 7F2F
  \l 7F30
  \l 7F31
  \l 7F32
  \l 7F33
  \l 7F34
  \l 7F35
  \l 7F36
  \l 7F37
  \l 7F38
  \l 7F39
  \l 7F3A
  \l 7F3B
  \l 7F3C
  \l 7F3D
  \l 7F3E
  \l 7F3F
  \l 7F40
  \l 7F41
  \l 7F42
  \l 7F43
  \l 7F44
  \l 7F45
  \l 7F46
  \l 7F47
  \l 7F48
  \l 7F49
  \l 7F4A
  \l 7F4B
  \l 7F4C
  \l 7F4D
  \l 7F4E
  \l 7F4F
  \l 7F50
  \l 7F51
  \l 7F52
  \l 7F53
  \l 7F54
  \l 7F55
  \l 7F56
  \l 7F57
  \l 7F58
  \l 7F59
  \l 7F5A
  \l 7F5B
  \l 7F5C
  \l 7F5D
  \l 7F5E
  \l 7F5F
  \l 7F60
  \l 7F61
  \l 7F62
  \l 7F63
  \l 7F64
  \l 7F65
  \l 7F66
  \l 7F67
  \l 7F68
  \l 7F69
  \l 7F6A
  \l 7F6B
  \l 7F6C
  \l 7F6D
  \l 7F6E
  \l 7F6F
  \l 7F70
  \l 7F71
  \l 7F72
  \l 7F73
  \l 7F74
  \l 7F75
  \l 7F76
  \l 7F77
  \l 7F78
  \l 7F79
  \l 7F7A
  \l 7F7B
  \l 7F7C
  \l 7F7D
  \l 7F7E
  \l 7F7F
  \l 7F80
  \l 7F81
  \l 7F82
  \l 7F83
  \l 7F84
  \l 7F85
  \l 7F86
  \l 7F87
  \l 7F88
  \l 7F89
  \l 7F8A
  \l 7F8B
  \l 7F8C
  \l 7F8D
  \l 7F8E
  \l 7F8F
  \l 7F90
  \l 7F91
  \l 7F92
  \l 7F93
  \l 7F94
  \l 7F95
  \l 7F96
  \l 7F97
  \l 7F98
  \l 7F99
  \l 7F9A
  \l 7F9B
  \l 7F9C
  \l 7F9D
  \l 7F9E
  \l 7F9F
  \l 7FA0
  \l 7FA1
  \l 7FA2
  \l 7FA3
  \l 7FA4
  \l 7FA5
  \l 7FA6
  \l 7FA7
  \l 7FA8
  \l 7FA9
  \l 7FAA
  \l 7FAB
  \l 7FAC
  \l 7FAD
  \l 7FAE
  \l 7FAF
  \l 7FB0
  \l 7FB1
  \l 7FB2
  \l 7FB3
  \l 7FB4
  \l 7FB5
  \l 7FB6
  \l 7FB7
  \l 7FB8
  \l 7FB9
  \l 7FBA
  \l 7FBB
  \l 7FBC
  \l 7FBD
  \l 7FBE
  \l 7FBF
  \l 7FC0
  \l 7FC1
  \l 7FC2
  \l 7FC3
  \l 7FC4
  \l 7FC5
  \l 7FC6
  \l 7FC7
  \l 7FC8
  \l 7FC9
  \l 7FCA
  \l 7FCB
  \l 7FCC
  \l 7FCD
  \l 7FCE
  \l 7FCF
  \l 7FD0
  \l 7FD1
  \l 7FD2
  \l 7FD3
  \l 7FD4
  \l 7FD5
  \l 7FD6
  \l 7FD7
  \l 7FD8
  \l 7FD9
  \l 7FDA
  \l 7FDB
  \l 7FDC
  \l 7FDD
  \l 7FDE
  \l 7FDF
  \l 7FE0
  \l 7FE1
  \l 7FE2
  \l 7FE3
  \l 7FE4
  \l 7FE5
  \l 7FE6
  \l 7FE7
  \l 7FE8
  \l 7FE9
  \l 7FEA
  \l 7FEB
  \l 7FEC
  \l 7FED
  \l 7FEE
  \l 7FEF
  \l 7FF0
  \l 7FF1
  \l 7FF2
  \l 7FF3
  \l 7FF4
  \l 7FF5
  \l 7FF6
  \l 7FF7
  \l 7FF8
  \l 7FF9
  \l 7FFA
  \l 7FFB
  \l 7FFC
  \l 7FFD
  \l 7FFE
  \l 7FFF
  \l 8000
  \l 8001
  \l 8002
  \l 8003
  \l 8004
  \l 8005
  \l 8006
  \l 8007
  \l 8008
  \l 8009
  \l 800A
  \l 800B
  \l 800C
  \l 800D
  \l 800E
  \l 800F
  \l 8010
  \l 8011
  \l 8012
  \l 8013
  \l 8014
  \l 8015
  \l 8016
  \l 8017
  \l 8018
  \l 8019
  \l 801A
  \l 801B
  \l 801C
  \l 801D
  \l 801E
  \l 801F
  \l 8020
  \l 8021
  \l 8022
  \l 8023
  \l 8024
  \l 8025
  \l 8026
  \l 8027
  \l 8028
  \l 8029
  \l 802A
  \l 802B
  \l 802C
  \l 802D
  \l 802E
  \l 802F
  \l 8030
  \l 8031
  \l 8032
  \l 8033
  \l 8034
  \l 8035
  \l 8036
  \l 8037
  \l 8038
  \l 8039
  \l 803A
  \l 803B
  \l 803C
  \l 803D
  \l 803E
  \l 803F
  \l 8040
  \l 8041
  \l 8042
  \l 8043
  \l 8044
  \l 8045
  \l 8046
  \l 8047
  \l 8048
  \l 8049
  \l 804A
  \l 804B
  \l 804C
  \l 804D
  \l 804E
  \l 804F
  \l 8050
  \l 8051
  \l 8052
  \l 8053
  \l 8054
  \l 8055
  \l 8056
  \l 8057
  \l 8058
  \l 8059
  \l 805A
  \l 805B
  \l 805C
  \l 805D
  \l 805E
  \l 805F
  \l 8060
  \l 8061
  \l 8062
  \l 8063
  \l 8064
  \l 8065
  \l 8066
  \l 8067
  \l 8068
  \l 8069
  \l 806A
  \l 806B
  \l 806C
  \l 806D
  \l 806E
  \l 806F
  \l 8070
  \l 8071
  \l 8072
  \l 8073
  \l 8074
  \l 8075
  \l 8076
  \l 8077
  \l 8078
  \l 8079
  \l 807A
  \l 807B
  \l 807C
  \l 807D
  \l 807E
  \l 807F
  \l 8080
  \l 8081
  \l 8082
  \l 8083
  \l 8084
  \l 8085
  \l 8086
  \l 8087
  \l 8088
  \l 8089
  \l 808A
  \l 808B
  \l 808C
  \l 808D
  \l 808E
  \l 808F
  \l 8090
  \l 8091
  \l 8092
  \l 8093
  \l 8094
  \l 8095
  \l 8096
  \l 8097
  \l 8098
  \l 8099
  \l 809A
  \l 809B
  \l 809C
  \l 809D
  \l 809E
  \l 809F
  \l 80A0
  \l 80A1
  \l 80A2
  \l 80A3
  \l 80A4
  \l 80A5
  \l 80A6
  \l 80A7
  \l 80A8
  \l 80A9
  \l 80AA
  \l 80AB
  \l 80AC
  \l 80AD
  \l 80AE
  \l 80AF
  \l 80B0
  \l 80B1
  \l 80B2
  \l 80B3
  \l 80B4
  \l 80B5
  \l 80B6
  \l 80B7
  \l 80B8
  \l 80B9
  \l 80BA
  \l 80BB
  \l 80BC
  \l 80BD
  \l 80BE
  \l 80BF
  \l 80C0
  \l 80C1
  \l 80C2
  \l 80C3
  \l 80C4
  \l 80C5
  \l 80C6
  \l 80C7
  \l 80C8
  \l 80C9
  \l 80CA
  \l 80CB
  \l 80CC
  \l 80CD
  \l 80CE
  \l 80CF
  \l 80D0
  \l 80D1
  \l 80D2
  \l 80D3
  \l 80D4
  \l 80D5
  \l 80D6
  \l 80D7
  \l 80D8
  \l 80D9
  \l 80DA
  \l 80DB
  \l 80DC
  \l 80DD
  \l 80DE
  \l 80DF
  \l 80E0
  \l 80E1
  \l 80E2
  \l 80E3
  \l 80E4
  \l 80E5
  \l 80E6
  \l 80E7
  \l 80E8
  \l 80E9
  \l 80EA
  \l 80EB
  \l 80EC
  \l 80ED
  \l 80EE
  \l 80EF
  \l 80F0
  \l 80F1
  \l 80F2
  \l 80F3
  \l 80F4
  \l 80F5
  \l 80F6
  \l 80F7
  \l 80F8
  \l 80F9
  \l 80FA
  \l 80FB
  \l 80FC
  \l 80FD
  \l 80FE
  \l 80FF
  \l 8100
  \l 8101
  \l 8102
  \l 8103
  \l 8104
  \l 8105
  \l 8106
  \l 8107
  \l 8108
  \l 8109
  \l 810A
  \l 810B
  \l 810C
  \l 810D
  \l 810E
  \l 810F
  \l 8110
  \l 8111
  \l 8112
  \l 8113
  \l 8114
  \l 8115
  \l 8116
  \l 8117
  \l 8118
  \l 8119
  \l 811A
  \l 811B
  \l 811C
  \l 811D
  \l 811E
  \l 811F
  \l 8120
  \l 8121
  \l 8122
  \l 8123
  \l 8124
  \l 8125
  \l 8126
  \l 8127
  \l 8128
  \l 8129
  \l 812A
  \l 812B
  \l 812C
  \l 812D
  \l 812E
  \l 812F
  \l 8130
  \l 8131
  \l 8132
  \l 8133
  \l 8134
  \l 8135
  \l 8136
  \l 8137
  \l 8138
  \l 8139
  \l 813A
  \l 813B
  \l 813C
  \l 813D
  \l 813E
  \l 813F
  \l 8140
  \l 8141
  \l 8142
  \l 8143
  \l 8144
  \l 8145
  \l 8146
  \l 8147
  \l 8148
  \l 8149
  \l 814A
  \l 814B
  \l 814C
  \l 814D
  \l 814E
  \l 814F
  \l 8150
  \l 8151
  \l 8152
  \l 8153
  \l 8154
  \l 8155
  \l 8156
  \l 8157
  \l 8158
  \l 8159
  \l 815A
  \l 815B
  \l 815C
  \l 815D
  \l 815E
  \l 815F
  \l 8160
  \l 8161
  \l 8162
  \l 8163
  \l 8164
  \l 8165
  \l 8166
  \l 8167
  \l 8168
  \l 8169
  \l 816A
  \l 816B
  \l 816C
  \l 816D
  \l 816E
  \l 816F
  \l 8170
  \l 8171
  \l 8172
  \l 8173
  \l 8174
  \l 8175
  \l 8176
  \l 8177
  \l 8178
  \l 8179
  \l 817A
  \l 817B
  \l 817C
  \l 817D
  \l 817E
  \l 817F
  \l 8180
  \l 8181
  \l 8182
  \l 8183
  \l 8184
  \l 8185
  \l 8186
  \l 8187
  \l 8188
  \l 8189
  \l 818A
  \l 818B
  \l 818C
  \l 818D
  \l 818E
  \l 818F
  \l 8190
  \l 8191
  \l 8192
  \l 8193
  \l 8194
  \l 8195
  \l 8196
  \l 8197
  \l 8198
  \l 8199
  \l 819A
  \l 819B
  \l 819C
  \l 819D
  \l 819E
  \l 819F
  \l 81A0
  \l 81A1
  \l 81A2
  \l 81A3
  \l 81A4
  \l 81A5
  \l 81A6
  \l 81A7
  \l 81A8
  \l 81A9
  \l 81AA
  \l 81AB
  \l 81AC
  \l 81AD
  \l 81AE
  \l 81AF
  \l 81B0
  \l 81B1
  \l 81B2
  \l 81B3
  \l 81B4
  \l 81B5
  \l 81B6
  \l 81B7
  \l 81B8
  \l 81B9
  \l 81BA
  \l 81BB
  \l 81BC
  \l 81BD
  \l 81BE
  \l 81BF
  \l 81C0
  \l 81C1
  \l 81C2
  \l 81C3
  \l 81C4
  \l 81C5
  \l 81C6
  \l 81C7
  \l 81C8
  \l 81C9
  \l 81CA
  \l 81CB
  \l 81CC
  \l 81CD
  \l 81CE
  \l 81CF
  \l 81D0
  \l 81D1
  \l 81D2
  \l 81D3
  \l 81D4
  \l 81D5
  \l 81D6
  \l 81D7
  \l 81D8
  \l 81D9
  \l 81DA
  \l 81DB
  \l 81DC
  \l 81DD
  \l 81DE
  \l 81DF
  \l 81E0
  \l 81E1
  \l 81E2
  \l 81E3
  \l 81E4
  \l 81E5
  \l 81E6
  \l 81E7
  \l 81E8
  \l 81E9
  \l 81EA
  \l 81EB
  \l 81EC
  \l 81ED
  \l 81EE
  \l 81EF
  \l 81F0
  \l 81F1
  \l 81F2
  \l 81F3
  \l 81F4
  \l 81F5
  \l 81F6
  \l 81F7
  \l 81F8
  \l 81F9
  \l 81FA
  \l 81FB
  \l 81FC
  \l 81FD
  \l 81FE
  \l 81FF
  \l 8200
  \l 8201
  \l 8202
  \l 8203
  \l 8204
  \l 8205
  \l 8206
  \l 8207
  \l 8208
  \l 8209
  \l 820A
  \l 820B
  \l 820C
  \l 820D
  \l 820E
  \l 820F
  \l 8210
  \l 8211
  \l 8212
  \l 8213
  \l 8214
  \l 8215
  \l 8216
  \l 8217
  \l 8218
  \l 8219
  \l 821A
  \l 821B
  \l 821C
  \l 821D
  \l 821E
  \l 821F
  \l 8220
  \l 8221
  \l 8222
  \l 8223
  \l 8224
  \l 8225
  \l 8226
  \l 8227
  \l 8228
  \l 8229
  \l 822A
  \l 822B
  \l 822C
  \l 822D
  \l 822E
  \l 822F
  \l 8230
  \l 8231
  \l 8232
  \l 8233
  \l 8234
  \l 8235
  \l 8236
  \l 8237
  \l 8238
  \l 8239
  \l 823A
  \l 823B
  \l 823C
  \l 823D
  \l 823E
  \l 823F
  \l 8240
  \l 8241
  \l 8242
  \l 8243
  \l 8244
  \l 8245
  \l 8246
  \l 8247
  \l 8248
  \l 8249
  \l 824A
  \l 824B
  \l 824C
  \l 824D
  \l 824E
  \l 824F
  \l 8250
  \l 8251
  \l 8252
  \l 8253
  \l 8254
  \l 8255
  \l 8256
  \l 8257
  \l 8258
  \l 8259
  \l 825A
  \l 825B
  \l 825C
  \l 825D
  \l 825E
  \l 825F
  \l 8260
  \l 8261
  \l 8262
  \l 8263
  \l 8264
  \l 8265
  \l 8266
  \l 8267
  \l 8268
  \l 8269
  \l 826A
  \l 826B
  \l 826C
  \l 826D
  \l 826E
  \l 826F
  \l 8270
  \l 8271
  \l 8272
  \l 8273
  \l 8274
  \l 8275
  \l 8276
  \l 8277
  \l 8278
  \l 8279
  \l 827A
  \l 827B
  \l 827C
  \l 827D
  \l 827E
  \l 827F
  \l 8280
  \l 8281
  \l 8282
  \l 8283
  \l 8284
  \l 8285
  \l 8286
  \l 8287
  \l 8288
  \l 8289
  \l 828A
  \l 828B
  \l 828C
  \l 828D
  \l 828E
  \l 828F
  \l 8290
  \l 8291
  \l 8292
  \l 8293
  \l 8294
  \l 8295
  \l 8296
  \l 8297
  \l 8298
  \l 8299
  \l 829A
  \l 829B
  \l 829C
  \l 829D
  \l 829E
  \l 829F
  \l 82A0
  \l 82A1
  \l 82A2
  \l 82A3
  \l 82A4
  \l 82A5
  \l 82A6
  \l 82A7
  \l 82A8
  \l 82A9
  \l 82AA
  \l 82AB
  \l 82AC
  \l 82AD
  \l 82AE
  \l 82AF
  \l 82B0
  \l 82B1
  \l 82B2
  \l 82B3
  \l 82B4
  \l 82B5
  \l 82B6
  \l 82B7
  \l 82B8
  \l 82B9
  \l 82BA
  \l 82BB
  \l 82BC
  \l 82BD
  \l 82BE
  \l 82BF
  \l 82C0
  \l 82C1
  \l 82C2
  \l 82C3
  \l 82C4
  \l 82C5
  \l 82C6
  \l 82C7
  \l 82C8
  \l 82C9
  \l 82CA
  \l 82CB
  \l 82CC
  \l 82CD
  \l 82CE
  \l 82CF
  \l 82D0
  \l 82D1
  \l 82D2
  \l 82D3
  \l 82D4
  \l 82D5
  \l 82D6
  \l 82D7
  \l 82D8
  \l 82D9
  \l 82DA
  \l 82DB
  \l 82DC
  \l 82DD
  \l 82DE
  \l 82DF
  \l 82E0
  \l 82E1
  \l 82E2
  \l 82E3
  \l 82E4
  \l 82E5
  \l 82E6
  \l 82E7
  \l 82E8
  \l 82E9
  \l 82EA
  \l 82EB
  \l 82EC
  \l 82ED
  \l 82EE
  \l 82EF
  \l 82F0
  \l 82F1
  \l 82F2
  \l 82F3
  \l 82F4
  \l 82F5
  \l 82F6
  \l 82F7
  \l 82F8
  \l 82F9
  \l 82FA
  \l 82FB
  \l 82FC
  \l 82FD
  \l 82FE
  \l 82FF
  \l 8300
  \l 8301
  \l 8302
  \l 8303
  \l 8304
  \l 8305
  \l 8306
  \l 8307
  \l 8308
  \l 8309
  \l 830A
  \l 830B
  \l 830C
  \l 830D
  \l 830E
  \l 830F
  \l 8310
  \l 8311
  \l 8312
  \l 8313
  \l 8314
  \l 8315
  \l 8316
  \l 8317
  \l 8318
  \l 8319
  \l 831A
  \l 831B
  \l 831C
  \l 831D
  \l 831E
  \l 831F
  \l 8320
  \l 8321
  \l 8322
  \l 8323
  \l 8324
  \l 8325
  \l 8326
  \l 8327
  \l 8328
  \l 8329
  \l 832A
  \l 832B
  \l 832C
  \l 832D
  \l 832E
  \l 832F
  \l 8330
  \l 8331
  \l 8332
  \l 8333
  \l 8334
  \l 8335
  \l 8336
  \l 8337
  \l 8338
  \l 8339
  \l 833A
  \l 833B
  \l 833C
  \l 833D
  \l 833E
  \l 833F
  \l 8340
  \l 8341
  \l 8342
  \l 8343
  \l 8344
  \l 8345
  \l 8346
  \l 8347
  \l 8348
  \l 8349
  \l 834A
  \l 834B
  \l 834C
  \l 834D
  \l 834E
  \l 834F
  \l 8350
  \l 8351
  \l 8352
  \l 8353
  \l 8354
  \l 8355
  \l 8356
  \l 8357
  \l 8358
  \l 8359
  \l 835A
  \l 835B
  \l 835C
  \l 835D
  \l 835E
  \l 835F
  \l 8360
  \l 8361
  \l 8362
  \l 8363
  \l 8364
  \l 8365
  \l 8366
  \l 8367
  \l 8368
  \l 8369
  \l 836A
  \l 836B
  \l 836C
  \l 836D
  \l 836E
  \l 836F
  \l 8370
  \l 8371
  \l 8372
  \l 8373
  \l 8374
  \l 8375
  \l 8376
  \l 8377
  \l 8378
  \l 8379
  \l 837A
  \l 837B
  \l 837C
  \l 837D
  \l 837E
  \l 837F
  \l 8380
  \l 8381
  \l 8382
  \l 8383
  \l 8384
  \l 8385
  \l 8386
  \l 8387
  \l 8388
  \l 8389
  \l 838A
  \l 838B
  \l 838C
  \l 838D
  \l 838E
  \l 838F
  \l 8390
  \l 8391
  \l 8392
  \l 8393
  \l 8394
  \l 8395
  \l 8396
  \l 8397
  \l 8398
  \l 8399
  \l 839A
  \l 839B
  \l 839C
  \l 839D
  \l 839E
  \l 839F
  \l 83A0
  \l 83A1
  \l 83A2
  \l 83A3
  \l 83A4
  \l 83A5
  \l 83A6
  \l 83A7
  \l 83A8
  \l 83A9
  \l 83AA
  \l 83AB
  \l 83AC
  \l 83AD
  \l 83AE
  \l 83AF
  \l 83B0
  \l 83B1
  \l 83B2
  \l 83B3
  \l 83B4
  \l 83B5
  \l 83B6
  \l 83B7
  \l 83B8
  \l 83B9
  \l 83BA
  \l 83BB
  \l 83BC
  \l 83BD
  \l 83BE
  \l 83BF
  \l 83C0
  \l 83C1
  \l 83C2
  \l 83C3
  \l 83C4
  \l 83C5
  \l 83C6
  \l 83C7
  \l 83C8
  \l 83C9
  \l 83CA
  \l 83CB
  \l 83CC
  \l 83CD
  \l 83CE
  \l 83CF
  \l 83D0
  \l 83D1
  \l 83D2
  \l 83D3
  \l 83D4
  \l 83D5
  \l 83D6
  \l 83D7
  \l 83D8
  \l 83D9
  \l 83DA
  \l 83DB
  \l 83DC
  \l 83DD
  \l 83DE
  \l 83DF
  \l 83E0
  \l 83E1
  \l 83E2
  \l 83E3
  \l 83E4
  \l 83E5
  \l 83E6
  \l 83E7
  \l 83E8
  \l 83E9
  \l 83EA
  \l 83EB
  \l 83EC
  \l 83ED
  \l 83EE
  \l 83EF
  \l 83F0
  \l 83F1
  \l 83F2
  \l 83F3
  \l 83F4
  \l 83F5
  \l 83F6
  \l 83F7
  \l 83F8
  \l 83F9
  \l 83FA
  \l 83FB
  \l 83FC
  \l 83FD
  \l 83FE
  \l 83FF
  \l 8400
  \l 8401
  \l 8402
  \l 8403
  \l 8404
  \l 8405
  \l 8406
  \l 8407
  \l 8408
  \l 8409
  \l 840A
  \l 840B
  \l 840C
  \l 840D
  \l 840E
  \l 840F
  \l 8410
  \l 8411
  \l 8412
  \l 8413
  \l 8414
  \l 8415
  \l 8416
  \l 8417
  \l 8418
  \l 8419
  \l 841A
  \l 841B
  \l 841C
  \l 841D
  \l 841E
  \l 841F
  \l 8420
  \l 8421
  \l 8422
  \l 8423
  \l 8424
  \l 8425
  \l 8426
  \l 8427
  \l 8428
  \l 8429
  \l 842A
  \l 842B
  \l 842C
  \l 842D
  \l 842E
  \l 842F
  \l 8430
  \l 8431
  \l 8432
  \l 8433
  \l 8434
  \l 8435
  \l 8436
  \l 8437
  \l 8438
  \l 8439
  \l 843A
  \l 843B
  \l 843C
  \l 843D
  \l 843E
  \l 843F
  \l 8440
  \l 8441
  \l 8442
  \l 8443
  \l 8444
  \l 8445
  \l 8446
  \l 8447
  \l 8448
  \l 8449
  \l 844A
  \l 844B
  \l 844C
  \l 844D
  \l 844E
  \l 844F
  \l 8450
  \l 8451
  \l 8452
  \l 8453
  \l 8454
  \l 8455
  \l 8456
  \l 8457
  \l 8458
  \l 8459
  \l 845A
  \l 845B
  \l 845C
  \l 845D
  \l 845E
  \l 845F
  \l 8460
  \l 8461
  \l 8462
  \l 8463
  \l 8464
  \l 8465
  \l 8466
  \l 8467
  \l 8468
  \l 8469
  \l 846A
  \l 846B
  \l 846C
  \l 846D
  \l 846E
  \l 846F
  \l 8470
  \l 8471
  \l 8472
  \l 8473
  \l 8474
  \l 8475
  \l 8476
  \l 8477
  \l 8478
  \l 8479
  \l 847A
  \l 847B
  \l 847C
  \l 847D
  \l 847E
  \l 847F
  \l 8480
  \l 8481
  \l 8482
  \l 8483
  \l 8484
  \l 8485
  \l 8486
  \l 8487
  \l 8488
  \l 8489
  \l 848A
  \l 848B
  \l 848C
  \l 848D
  \l 848E
  \l 848F
  \l 8490
  \l 8491
  \l 8492
  \l 8493
  \l 8494
  \l 8495
  \l 8496
  \l 8497
  \l 8498
  \l 8499
  \l 849A
  \l 849B
  \l 849C
  \l 849D
  \l 849E
  \l 849F
  \l 84A0
  \l 84A1
  \l 84A2
  \l 84A3
  \l 84A4
  \l 84A5
  \l 84A6
  \l 84A7
  \l 84A8
  \l 84A9
  \l 84AA
  \l 84AB
  \l 84AC
  \l 84AD
  \l 84AE
  \l 84AF
  \l 84B0
  \l 84B1
  \l 84B2
  \l 84B3
  \l 84B4
  \l 84B5
  \l 84B6
  \l 84B7
  \l 84B8
  \l 84B9
  \l 84BA
  \l 84BB
  \l 84BC
  \l 84BD
  \l 84BE
  \l 84BF
  \l 84C0
  \l 84C1
  \l 84C2
  \l 84C3
  \l 84C4
  \l 84C5
  \l 84C6
  \l 84C7
  \l 84C8
  \l 84C9
  \l 84CA
  \l 84CB
  \l 84CC
  \l 84CD
  \l 84CE
  \l 84CF
  \l 84D0
  \l 84D1
  \l 84D2
  \l 84D3
  \l 84D4
  \l 84D5
  \l 84D6
  \l 84D7
  \l 84D8
  \l 84D9
  \l 84DA
  \l 84DB
  \l 84DC
  \l 84DD
  \l 84DE
  \l 84DF
  \l 84E0
  \l 84E1
  \l 84E2
  \l 84E3
  \l 84E4
  \l 84E5
  \l 84E6
  \l 84E7
  \l 84E8
  \l 84E9
  \l 84EA
  \l 84EB
  \l 84EC
  \l 84ED
  \l 84EE
  \l 84EF
  \l 84F0
  \l 84F1
  \l 84F2
  \l 84F3
  \l 84F4
  \l 84F5
  \l 84F6
  \l 84F7
  \l 84F8
  \l 84F9
  \l 84FA
  \l 84FB
  \l 84FC
  \l 84FD
  \l 84FE
  \l 84FF
  \l 8500
  \l 8501
  \l 8502
  \l 8503
  \l 8504
  \l 8505
  \l 8506
  \l 8507
  \l 8508
  \l 8509
  \l 850A
  \l 850B
  \l 850C
  \l 850D
  \l 850E
  \l 850F
  \l 8510
  \l 8511
  \l 8512
  \l 8513
  \l 8514
  \l 8515
  \l 8516
  \l 8517
  \l 8518
  \l 8519
  \l 851A
  \l 851B
  \l 851C
  \l 851D
  \l 851E
  \l 851F
  \l 8520
  \l 8521
  \l 8522
  \l 8523
  \l 8524
  \l 8525
  \l 8526
  \l 8527
  \l 8528
  \l 8529
  \l 852A
  \l 852B
  \l 852C
  \l 852D
  \l 852E
  \l 852F
  \l 8530
  \l 8531
  \l 8532
  \l 8533
  \l 8534
  \l 8535
  \l 8536
  \l 8537
  \l 8538
  \l 8539
  \l 853A
  \l 853B
  \l 853C
  \l 853D
  \l 853E
  \l 853F
  \l 8540
  \l 8541
  \l 8542
  \l 8543
  \l 8544
  \l 8545
  \l 8546
  \l 8547
  \l 8548
  \l 8549
  \l 854A
  \l 854B
  \l 854C
  \l 854D
  \l 854E
  \l 854F
  \l 8550
  \l 8551
  \l 8552
  \l 8553
  \l 8554
  \l 8555
  \l 8556
  \l 8557
  \l 8558
  \l 8559
  \l 855A
  \l 855B
  \l 855C
  \l 855D
  \l 855E
  \l 855F
  \l 8560
  \l 8561
  \l 8562
  \l 8563
  \l 8564
  \l 8565
  \l 8566
  \l 8567
  \l 8568
  \l 8569
  \l 856A
  \l 856B
  \l 856C
  \l 856D
  \l 856E
  \l 856F
  \l 8570
  \l 8571
  \l 8572
  \l 8573
  \l 8574
  \l 8575
  \l 8576
  \l 8577
  \l 8578
  \l 8579
  \l 857A
  \l 857B
  \l 857C
  \l 857D
  \l 857E
  \l 857F
  \l 8580
  \l 8581
  \l 8582
  \l 8583
  \l 8584
  \l 8585
  \l 8586
  \l 8587
  \l 8588
  \l 8589
  \l 858A
  \l 858B
  \l 858C
  \l 858D
  \l 858E
  \l 858F
  \l 8590
  \l 8591
  \l 8592
  \l 8593
  \l 8594
  \l 8595
  \l 8596
  \l 8597
  \l 8598
  \l 8599
  \l 859A
  \l 859B
  \l 859C
  \l 859D
  \l 859E
  \l 859F
  \l 85A0
  \l 85A1
  \l 85A2
  \l 85A3
  \l 85A4
  \l 85A5
  \l 85A6
  \l 85A7
  \l 85A8
  \l 85A9
  \l 85AA
  \l 85AB
  \l 85AC
  \l 85AD
  \l 85AE
  \l 85AF
  \l 85B0
  \l 85B1
  \l 85B2
  \l 85B3
  \l 85B4
  \l 85B5
  \l 85B6
  \l 85B7
  \l 85B8
  \l 85B9
  \l 85BA
  \l 85BB
  \l 85BC
  \l 85BD
  \l 85BE
  \l 85BF
  \l 85C0
  \l 85C1
  \l 85C2
  \l 85C3
  \l 85C4
  \l 85C5
  \l 85C6
  \l 85C7
  \l 85C8
  \l 85C9
  \l 85CA
  \l 85CB
  \l 85CC
  \l 85CD
  \l 85CE
  \l 85CF
  \l 85D0
  \l 85D1
  \l 85D2
  \l 85D3
  \l 85D4
  \l 85D5
  \l 85D6
  \l 85D7
  \l 85D8
  \l 85D9
  \l 85DA
  \l 85DB
  \l 85DC
  \l 85DD
  \l 85DE
  \l 85DF
  \l 85E0
  \l 85E1
  \l 85E2
  \l 85E3
  \l 85E4
  \l 85E5
  \l 85E6
  \l 85E7
  \l 85E8
  \l 85E9
  \l 85EA
  \l 85EB
  \l 85EC
  \l 85ED
  \l 85EE
  \l 85EF
  \l 85F0
  \l 85F1
  \l 85F2
  \l 85F3
  \l 85F4
  \l 85F5
  \l 85F6
  \l 85F7
  \l 85F8
  \l 85F9
  \l 85FA
  \l 85FB
  \l 85FC
  \l 85FD
  \l 85FE
  \l 85FF
  \l 8600
  \l 8601
  \l 8602
  \l 8603
  \l 8604
  \l 8605
  \l 8606
  \l 8607
  \l 8608
  \l 8609
  \l 860A
  \l 860B
  \l 860C
  \l 860D
  \l 860E
  \l 860F
  \l 8610
  \l 8611
  \l 8612
  \l 8613
  \l 8614
  \l 8615
  \l 8616
  \l 8617
  \l 8618
  \l 8619
  \l 861A
  \l 861B
  \l 861C
  \l 861D
  \l 861E
  \l 861F
  \l 8620
  \l 8621
  \l 8622
  \l 8623
  \l 8624
  \l 8625
  \l 8626
  \l 8627
  \l 8628
  \l 8629
  \l 862A
  \l 862B
  \l 862C
  \l 862D
  \l 862E
  \l 862F
  \l 8630
  \l 8631
  \l 8632
  \l 8633
  \l 8634
  \l 8635
  \l 8636
  \l 8637
  \l 8638
  \l 8639
  \l 863A
  \l 863B
  \l 863C
  \l 863D
  \l 863E
  \l 863F
  \l 8640
  \l 8641
  \l 8642
  \l 8643
  \l 8644
  \l 8645
  \l 8646
  \l 8647
  \l 8648
  \l 8649
  \l 864A
  \l 864B
  \l 864C
  \l 864D
  \l 864E
  \l 864F
  \l 8650
  \l 8651
  \l 8652
  \l 8653
  \l 8654
  \l 8655
  \l 8656
  \l 8657
  \l 8658
  \l 8659
  \l 865A
  \l 865B
  \l 865C
  \l 865D
  \l 865E
  \l 865F
  \l 8660
  \l 8661
  \l 8662
  \l 8663
  \l 8664
  \l 8665
  \l 8666
  \l 8667
  \l 8668
  \l 8669
  \l 866A
  \l 866B
  \l 866C
  \l 866D
  \l 866E
  \l 866F
  \l 8670
  \l 8671
  \l 8672
  \l 8673
  \l 8674
  \l 8675
  \l 8676
  \l 8677
  \l 8678
  \l 8679
  \l 867A
  \l 867B
  \l 867C
  \l 867D
  \l 867E
  \l 867F
  \l 8680
  \l 8681
  \l 8682
  \l 8683
  \l 8684
  \l 8685
  \l 8686
  \l 8687
  \l 8688
  \l 8689
  \l 868A
  \l 868B
  \l 868C
  \l 868D
  \l 868E
  \l 868F
  \l 8690
  \l 8691
  \l 8692
  \l 8693
  \l 8694
  \l 8695
  \l 8696
  \l 8697
  \l 8698
  \l 8699
  \l 869A
  \l 869B
  \l 869C
  \l 869D
  \l 869E
  \l 869F
  \l 86A0
  \l 86A1
  \l 86A2
  \l 86A3
  \l 86A4
  \l 86A5
  \l 86A6
  \l 86A7
  \l 86A8
  \l 86A9
  \l 86AA
  \l 86AB
  \l 86AC
  \l 86AD
  \l 86AE
  \l 86AF
  \l 86B0
  \l 86B1
  \l 86B2
  \l 86B3
  \l 86B4
  \l 86B5
  \l 86B6
  \l 86B7
  \l 86B8
  \l 86B9
  \l 86BA
  \l 86BB
  \l 86BC
  \l 86BD
  \l 86BE
  \l 86BF
  \l 86C0
  \l 86C1
  \l 86C2
  \l 86C3
  \l 86C4
  \l 86C5
  \l 86C6
  \l 86C7
  \l 86C8
  \l 86C9
  \l 86CA
  \l 86CB
  \l 86CC
  \l 86CD
  \l 86CE
  \l 86CF
  \l 86D0
  \l 86D1
  \l 86D2
  \l 86D3
  \l 86D4
  \l 86D5
  \l 86D6
  \l 86D7
  \l 86D8
  \l 86D9
  \l 86DA
  \l 86DB
  \l 86DC
  \l 86DD
  \l 86DE
  \l 86DF
  \l 86E0
  \l 86E1
  \l 86E2
  \l 86E3
  \l 86E4
  \l 86E5
  \l 86E6
  \l 86E7
  \l 86E8
  \l 86E9
  \l 86EA
  \l 86EB
  \l 86EC
  \l 86ED
  \l 86EE
  \l 86EF
  \l 86F0
  \l 86F1
  \l 86F2
  \l 86F3
  \l 86F4
  \l 86F5
  \l 86F6
  \l 86F7
  \l 86F8
  \l 86F9
  \l 86FA
  \l 86FB
  \l 86FC
  \l 86FD
  \l 86FE
  \l 86FF
  \l 8700
  \l 8701
  \l 8702
  \l 8703
  \l 8704
  \l 8705
  \l 8706
  \l 8707
  \l 8708
  \l 8709
  \l 870A
  \l 870B
  \l 870C
  \l 870D
  \l 870E
  \l 870F
  \l 8710
  \l 8711
  \l 8712
  \l 8713
  \l 8714
  \l 8715
  \l 8716
  \l 8717
  \l 8718
  \l 8719
  \l 871A
  \l 871B
  \l 871C
  \l 871D
  \l 871E
  \l 871F
  \l 8720
  \l 8721
  \l 8722
  \l 8723
  \l 8724
  \l 8725
  \l 8726
  \l 8727
  \l 8728
  \l 8729
  \l 872A
  \l 872B
  \l 872C
  \l 872D
  \l 872E
  \l 872F
  \l 8730
  \l 8731
  \l 8732
  \l 8733
  \l 8734
  \l 8735
  \l 8736
  \l 8737
  \l 8738
  \l 8739
  \l 873A
  \l 873B
  \l 873C
  \l 873D
  \l 873E
  \l 873F
  \l 8740
  \l 8741
  \l 8742
  \l 8743
  \l 8744
  \l 8745
  \l 8746
  \l 8747
  \l 8748
  \l 8749
  \l 874A
  \l 874B
  \l 874C
  \l 874D
  \l 874E
  \l 874F
  \l 8750
  \l 8751
  \l 8752
  \l 8753
  \l 8754
  \l 8755
  \l 8756
  \l 8757
  \l 8758
  \l 8759
  \l 875A
  \l 875B
  \l 875C
  \l 875D
  \l 875E
  \l 875F
  \l 8760
  \l 8761
  \l 8762
  \l 8763
  \l 8764
  \l 8765
  \l 8766
  \l 8767
  \l 8768
  \l 8769
  \l 876A
  \l 876B
  \l 876C
  \l 876D
  \l 876E
  \l 876F
  \l 8770
  \l 8771
  \l 8772
  \l 8773
  \l 8774
  \l 8775
  \l 8776
  \l 8777
  \l 8778
  \l 8779
  \l 877A
  \l 877B
  \l 877C
  \l 877D
  \l 877E
  \l 877F
  \l 8780
  \l 8781
  \l 8782
  \l 8783
  \l 8784
  \l 8785
  \l 8786
  \l 8787
  \l 8788
  \l 8789
  \l 878A
  \l 878B
  \l 878C
  \l 878D
  \l 878E
  \l 878F
  \l 8790
  \l 8791
  \l 8792
  \l 8793
  \l 8794
  \l 8795
  \l 8796
  \l 8797
  \l 8798
  \l 8799
  \l 879A
  \l 879B
  \l 879C
  \l 879D
  \l 879E
  \l 879F
  \l 87A0
  \l 87A1
  \l 87A2
  \l 87A3
  \l 87A4
  \l 87A5
  \l 87A6
  \l 87A7
  \l 87A8
  \l 87A9
  \l 87AA
  \l 87AB
  \l 87AC
  \l 87AD
  \l 87AE
  \l 87AF
  \l 87B0
  \l 87B1
  \l 87B2
  \l 87B3
  \l 87B4
  \l 87B5
  \l 87B6
  \l 87B7
  \l 87B8
  \l 87B9
  \l 87BA
  \l 87BB
  \l 87BC
  \l 87BD
  \l 87BE
  \l 87BF
  \l 87C0
  \l 87C1
  \l 87C2
  \l 87C3
  \l 87C4
  \l 87C5
  \l 87C6
  \l 87C7
  \l 87C8
  \l 87C9
  \l 87CA
  \l 87CB
  \l 87CC
  \l 87CD
  \l 87CE
  \l 87CF
  \l 87D0
  \l 87D1
  \l 87D2
  \l 87D3
  \l 87D4
  \l 87D5
  \l 87D6
  \l 87D7
  \l 87D8
  \l 87D9
  \l 87DA
  \l 87DB
  \l 87DC
  \l 87DD
  \l 87DE
  \l 87DF
  \l 87E0
  \l 87E1
  \l 87E2
  \l 87E3
  \l 87E4
  \l 87E5
  \l 87E6
  \l 87E7
  \l 87E8
  \l 87E9
  \l 87EA
  \l 87EB
  \l 87EC
  \l 87ED
  \l 87EE
  \l 87EF
  \l 87F0
  \l 87F1
  \l 87F2
  \l 87F3
  \l 87F4
  \l 87F5
  \l 87F6
  \l 87F7
  \l 87F8
  \l 87F9
  \l 87FA
  \l 87FB
  \l 87FC
  \l 87FD
  \l 87FE
  \l 87FF
  \l 8800
  \l 8801
  \l 8802
  \l 8803
  \l 8804
  \l 8805
  \l 8806
  \l 8807
  \l 8808
  \l 8809
  \l 880A
  \l 880B
  \l 880C
  \l 880D
  \l 880E
  \l 880F
  \l 8810
  \l 8811
  \l 8812
  \l 8813
  \l 8814
  \l 8815
  \l 8816
  \l 8817
  \l 8818
  \l 8819
  \l 881A
  \l 881B
  \l 881C
  \l 881D
  \l 881E
  \l 881F
  \l 8820
  \l 8821
  \l 8822
  \l 8823
  \l 8824
  \l 8825
  \l 8826
  \l 8827
  \l 8828
  \l 8829
  \l 882A
  \l 882B
  \l 882C
  \l 882D
  \l 882E
  \l 882F
  \l 8830
  \l 8831
  \l 8832
  \l 8833
  \l 8834
  \l 8835
  \l 8836
  \l 8837
  \l 8838
  \l 8839
  \l 883A
  \l 883B
  \l 883C
  \l 883D
  \l 883E
  \l 883F
  \l 8840
  \l 8841
  \l 8842
  \l 8843
  \l 8844
  \l 8845
  \l 8846
  \l 8847
  \l 8848
  \l 8849
  \l 884A
  \l 884B
  \l 884C
  \l 884D
  \l 884E
  \l 884F
  \l 8850
  \l 8851
  \l 8852
  \l 8853
  \l 8854
  \l 8855
  \l 8856
  \l 8857
  \l 8858
  \l 8859
  \l 885A
  \l 885B
  \l 885C
  \l 885D
  \l 885E
  \l 885F
  \l 8860
  \l 8861
  \l 8862
  \l 8863
  \l 8864
  \l 8865
  \l 8866
  \l 8867
  \l 8868
  \l 8869
  \l 886A
  \l 886B
  \l 886C
  \l 886D
  \l 886E
  \l 886F
  \l 8870
  \l 8871
  \l 8872
  \l 8873
  \l 8874
  \l 8875
  \l 8876
  \l 8877
  \l 8878
  \l 8879
  \l 887A
  \l 887B
  \l 887C
  \l 887D
  \l 887E
  \l 887F
  \l 8880
  \l 8881
  \l 8882
  \l 8883
  \l 8884
  \l 8885
  \l 8886
  \l 8887
  \l 8888
  \l 8889
  \l 888A
  \l 888B
  \l 888C
  \l 888D
  \l 888E
  \l 888F
  \l 8890
  \l 8891
  \l 8892
  \l 8893
  \l 8894
  \l 8895
  \l 8896
  \l 8897
  \l 8898
  \l 8899
  \l 889A
  \l 889B
  \l 889C
  \l 889D
  \l 889E
  \l 889F
  \l 88A0
  \l 88A1
  \l 88A2
  \l 88A3
  \l 88A4
  \l 88A5
  \l 88A6
  \l 88A7
  \l 88A8
  \l 88A9
  \l 88AA
  \l 88AB
  \l 88AC
  \l 88AD
  \l 88AE
  \l 88AF
  \l 88B0
  \l 88B1
  \l 88B2
  \l 88B3
  \l 88B4
  \l 88B5
  \l 88B6
  \l 88B7
  \l 88B8
  \l 88B9
  \l 88BA
  \l 88BB
  \l 88BC
  \l 88BD
  \l 88BE
  \l 88BF
  \l 88C0
  \l 88C1
  \l 88C2
  \l 88C3
  \l 88C4
  \l 88C5
  \l 88C6
  \l 88C7
  \l 88C8
  \l 88C9
  \l 88CA
  \l 88CB
  \l 88CC
  \l 88CD
  \l 88CE
  \l 88CF
  \l 88D0
  \l 88D1
  \l 88D2
  \l 88D3
  \l 88D4
  \l 88D5
  \l 88D6
  \l 88D7
  \l 88D8
  \l 88D9
  \l 88DA
  \l 88DB
  \l 88DC
  \l 88DD
  \l 88DE
  \l 88DF
  \l 88E0
  \l 88E1
  \l 88E2
  \l 88E3
  \l 88E4
  \l 88E5
  \l 88E6
  \l 88E7
  \l 88E8
  \l 88E9
  \l 88EA
  \l 88EB
  \l 88EC
  \l 88ED
  \l 88EE
  \l 88EF
  \l 88F0
  \l 88F1
  \l 88F2
  \l 88F3
  \l 88F4
  \l 88F5
  \l 88F6
  \l 88F7
  \l 88F8
  \l 88F9
  \l 88FA
  \l 88FB
  \l 88FC
  \l 88FD
  \l 88FE
  \l 88FF
  \l 8900
  \l 8901
  \l 8902
  \l 8903
  \l 8904
  \l 8905
  \l 8906
  \l 8907
  \l 8908
  \l 8909
  \l 890A
  \l 890B
  \l 890C
  \l 890D
  \l 890E
  \l 890F
  \l 8910
  \l 8911
  \l 8912
  \l 8913
  \l 8914
  \l 8915
  \l 8916
  \l 8917
  \l 8918
  \l 8919
  \l 891A
  \l 891B
  \l 891C
  \l 891D
  \l 891E
  \l 891F
  \l 8920
  \l 8921
  \l 8922
  \l 8923
  \l 8924
  \l 8925
  \l 8926
  \l 8927
  \l 8928
  \l 8929
  \l 892A
  \l 892B
  \l 892C
  \l 892D
  \l 892E
  \l 892F
  \l 8930
  \l 8931
  \l 8932
  \l 8933
  \l 8934
  \l 8935
  \l 8936
  \l 8937
  \l 8938
  \l 8939
  \l 893A
  \l 893B
  \l 893C
  \l 893D
  \l 893E
  \l 893F
  \l 8940
  \l 8941
  \l 8942
  \l 8943
  \l 8944
  \l 8945
  \l 8946
  \l 8947
  \l 8948
  \l 8949
  \l 894A
  \l 894B
  \l 894C
  \l 894D
  \l 894E
  \l 894F
  \l 8950
  \l 8951
  \l 8952
  \l 8953
  \l 8954
  \l 8955
  \l 8956
  \l 8957
  \l 8958
  \l 8959
  \l 895A
  \l 895B
  \l 895C
  \l 895D
  \l 895E
  \l 895F
  \l 8960
  \l 8961
  \l 8962
  \l 8963
  \l 8964
  \l 8965
  \l 8966
  \l 8967
  \l 8968
  \l 8969
  \l 896A
  \l 896B
  \l 896C
  \l 896D
  \l 896E
  \l 896F
  \l 8970
  \l 8971
  \l 8972
  \l 8973
  \l 8974
  \l 8975
  \l 8976
  \l 8977
  \l 8978
  \l 8979
  \l 897A
  \l 897B
  \l 897C
  \l 897D
  \l 897E
  \l 897F
  \l 8980
  \l 8981
  \l 8982
  \l 8983
  \l 8984
  \l 8985
  \l 8986
  \l 8987
  \l 8988
  \l 8989
  \l 898A
  \l 898B
  \l 898C
  \l 898D
  \l 898E
  \l 898F
  \l 8990
  \l 8991
  \l 8992
  \l 8993
  \l 8994
  \l 8995
  \l 8996
  \l 8997
  \l 8998
  \l 8999
  \l 899A
  \l 899B
  \l 899C
  \l 899D
  \l 899E
  \l 899F
  \l 89A0
  \l 89A1
  \l 89A2
  \l 89A3
  \l 89A4
  \l 89A5
  \l 89A6
  \l 89A7
  \l 89A8
  \l 89A9
  \l 89AA
  \l 89AB
  \l 89AC
  \l 89AD
  \l 89AE
  \l 89AF
  \l 89B0
  \l 89B1
  \l 89B2
  \l 89B3
  \l 89B4
  \l 89B5
  \l 89B6
  \l 89B7
  \l 89B8
  \l 89B9
  \l 89BA
  \l 89BB
  \l 89BC
  \l 89BD
  \l 89BE
  \l 89BF
  \l 89C0
  \l 89C1
  \l 89C2
  \l 89C3
  \l 89C4
  \l 89C5
  \l 89C6
  \l 89C7
  \l 89C8
  \l 89C9
  \l 89CA
  \l 89CB
  \l 89CC
  \l 89CD
  \l 89CE
  \l 89CF
  \l 89D0
  \l 89D1
  \l 89D2
  \l 89D3
  \l 89D4
  \l 89D5
  \l 89D6
  \l 89D7
  \l 89D8
  \l 89D9
  \l 89DA
  \l 89DB
  \l 89DC
  \l 89DD
  \l 89DE
  \l 89DF
  \l 89E0
  \l 89E1
  \l 89E2
  \l 89E3
  \l 89E4
  \l 89E5
  \l 89E6
  \l 89E7
  \l 89E8
  \l 89E9
  \l 89EA
  \l 89EB
  \l 89EC
  \l 89ED
  \l 89EE
  \l 89EF
  \l 89F0
  \l 89F1
  \l 89F2
  \l 89F3
  \l 89F4
  \l 89F5
  \l 89F6
  \l 89F7
  \l 89F8
  \l 89F9
  \l 89FA
  \l 89FB
  \l 89FC
  \l 89FD
  \l 89FE
  \l 89FF
  \l 8A00
  \l 8A01
  \l 8A02
  \l 8A03
  \l 8A04
  \l 8A05
  \l 8A06
  \l 8A07
  \l 8A08
  \l 8A09
  \l 8A0A
  \l 8A0B
  \l 8A0C
  \l 8A0D
  \l 8A0E
  \l 8A0F
  \l 8A10
  \l 8A11
  \l 8A12
  \l 8A13
  \l 8A14
  \l 8A15
  \l 8A16
  \l 8A17
  \l 8A18
  \l 8A19
  \l 8A1A
  \l 8A1B
  \l 8A1C
  \l 8A1D
  \l 8A1E
  \l 8A1F
  \l 8A20
  \l 8A21
  \l 8A22
  \l 8A23
  \l 8A24
  \l 8A25
  \l 8A26
  \l 8A27
  \l 8A28
  \l 8A29
  \l 8A2A
  \l 8A2B
  \l 8A2C
  \l 8A2D
  \l 8A2E
  \l 8A2F
  \l 8A30
  \l 8A31
  \l 8A32
  \l 8A33
  \l 8A34
  \l 8A35
  \l 8A36
  \l 8A37
  \l 8A38
  \l 8A39
  \l 8A3A
  \l 8A3B
  \l 8A3C
  \l 8A3D
  \l 8A3E
  \l 8A3F
  \l 8A40
  \l 8A41
  \l 8A42
  \l 8A43
  \l 8A44
  \l 8A45
  \l 8A46
  \l 8A47
  \l 8A48
  \l 8A49
  \l 8A4A
  \l 8A4B
  \l 8A4C
  \l 8A4D
  \l 8A4E
  \l 8A4F
  \l 8A50
  \l 8A51
  \l 8A52
  \l 8A53
  \l 8A54
  \l 8A55
  \l 8A56
  \l 8A57
  \l 8A58
  \l 8A59
  \l 8A5A
  \l 8A5B
  \l 8A5C
  \l 8A5D
  \l 8A5E
  \l 8A5F
  \l 8A60
  \l 8A61
  \l 8A62
  \l 8A63
  \l 8A64
  \l 8A65
  \l 8A66
  \l 8A67
  \l 8A68
  \l 8A69
  \l 8A6A
  \l 8A6B
  \l 8A6C
  \l 8A6D
  \l 8A6E
  \l 8A6F
  \l 8A70
  \l 8A71
  \l 8A72
  \l 8A73
  \l 8A74
  \l 8A75
  \l 8A76
  \l 8A77
  \l 8A78
  \l 8A79
  \l 8A7A
  \l 8A7B
  \l 8A7C
  \l 8A7D
  \l 8A7E
  \l 8A7F
  \l 8A80
  \l 8A81
  \l 8A82
  \l 8A83
  \l 8A84
  \l 8A85
  \l 8A86
  \l 8A87
  \l 8A88
  \l 8A89
  \l 8A8A
  \l 8A8B
  \l 8A8C
  \l 8A8D
  \l 8A8E
  \l 8A8F
  \l 8A90
  \l 8A91
  \l 8A92
  \l 8A93
  \l 8A94
  \l 8A95
  \l 8A96
  \l 8A97
  \l 8A98
  \l 8A99
  \l 8A9A
  \l 8A9B
  \l 8A9C
  \l 8A9D
  \l 8A9E
  \l 8A9F
  \l 8AA0
  \l 8AA1
  \l 8AA2
  \l 8AA3
  \l 8AA4
  \l 8AA5
  \l 8AA6
  \l 8AA7
  \l 8AA8
  \l 8AA9
  \l 8AAA
  \l 8AAB
  \l 8AAC
  \l 8AAD
  \l 8AAE
  \l 8AAF
  \l 8AB0
  \l 8AB1
  \l 8AB2
  \l 8AB3
  \l 8AB4
  \l 8AB5
  \l 8AB6
  \l 8AB7
  \l 8AB8
  \l 8AB9
  \l 8ABA
  \l 8ABB
  \l 8ABC
  \l 8ABD
  \l 8ABE
  \l 8ABF
  \l 8AC0
  \l 8AC1
  \l 8AC2
  \l 8AC3
  \l 8AC4
  \l 8AC5
  \l 8AC6
  \l 8AC7
  \l 8AC8
  \l 8AC9
  \l 8ACA
  \l 8ACB
  \l 8ACC
  \l 8ACD
  \l 8ACE
  \l 8ACF
  \l 8AD0
  \l 8AD1
  \l 8AD2
  \l 8AD3
  \l 8AD4
  \l 8AD5
  \l 8AD6
  \l 8AD7
  \l 8AD8
  \l 8AD9
  \l 8ADA
  \l 8ADB
  \l 8ADC
  \l 8ADD
  \l 8ADE
  \l 8ADF
  \l 8AE0
  \l 8AE1
  \l 8AE2
  \l 8AE3
  \l 8AE4
  \l 8AE5
  \l 8AE6
  \l 8AE7
  \l 8AE8
  \l 8AE9
  \l 8AEA
  \l 8AEB
  \l 8AEC
  \l 8AED
  \l 8AEE
  \l 8AEF
  \l 8AF0
  \l 8AF1
  \l 8AF2
  \l 8AF3
  \l 8AF4
  \l 8AF5
  \l 8AF6
  \l 8AF7
  \l 8AF8
  \l 8AF9
  \l 8AFA
  \l 8AFB
  \l 8AFC
  \l 8AFD
  \l 8AFE
  \l 8AFF
  \l 8B00
  \l 8B01
  \l 8B02
  \l 8B03
  \l 8B04
  \l 8B05
  \l 8B06
  \l 8B07
  \l 8B08
  \l 8B09
  \l 8B0A
  \l 8B0B
  \l 8B0C
  \l 8B0D
  \l 8B0E
  \l 8B0F
  \l 8B10
  \l 8B11
  \l 8B12
  \l 8B13
  \l 8B14
  \l 8B15
  \l 8B16
  \l 8B17
  \l 8B18
  \l 8B19
  \l 8B1A
  \l 8B1B
  \l 8B1C
  \l 8B1D
  \l 8B1E
  \l 8B1F
  \l 8B20
  \l 8B21
  \l 8B22
  \l 8B23
  \l 8B24
  \l 8B25
  \l 8B26
  \l 8B27
  \l 8B28
  \l 8B29
  \l 8B2A
  \l 8B2B
  \l 8B2C
  \l 8B2D
  \l 8B2E
  \l 8B2F
  \l 8B30
  \l 8B31
  \l 8B32
  \l 8B33
  \l 8B34
  \l 8B35
  \l 8B36
  \l 8B37
  \l 8B38
  \l 8B39
  \l 8B3A
  \l 8B3B
  \l 8B3C
  \l 8B3D
  \l 8B3E
  \l 8B3F
  \l 8B40
  \l 8B41
  \l 8B42
  \l 8B43
  \l 8B44
  \l 8B45
  \l 8B46
  \l 8B47
  \l 8B48
  \l 8B49
  \l 8B4A
  \l 8B4B
  \l 8B4C
  \l 8B4D
  \l 8B4E
  \l 8B4F
  \l 8B50
  \l 8B51
  \l 8B52
  \l 8B53
  \l 8B54
  \l 8B55
  \l 8B56
  \l 8B57
  \l 8B58
  \l 8B59
  \l 8B5A
  \l 8B5B
  \l 8B5C
  \l 8B5D
  \l 8B5E
  \l 8B5F
  \l 8B60
  \l 8B61
  \l 8B62
  \l 8B63
  \l 8B64
  \l 8B65
  \l 8B66
  \l 8B67
  \l 8B68
  \l 8B69
  \l 8B6A
  \l 8B6B
  \l 8B6C
  \l 8B6D
  \l 8B6E
  \l 8B6F
  \l 8B70
  \l 8B71
  \l 8B72
  \l 8B73
  \l 8B74
  \l 8B75
  \l 8B76
  \l 8B77
  \l 8B78
  \l 8B79
  \l 8B7A
  \l 8B7B
  \l 8B7C
  \l 8B7D
  \l 8B7E
  \l 8B7F
  \l 8B80
  \l 8B81
  \l 8B82
  \l 8B83
  \l 8B84
  \l 8B85
  \l 8B86
  \l 8B87
  \l 8B88
  \l 8B89
  \l 8B8A
  \l 8B8B
  \l 8B8C
  \l 8B8D
  \l 8B8E
  \l 8B8F
  \l 8B90
  \l 8B91
  \l 8B92
  \l 8B93
  \l 8B94
  \l 8B95
  \l 8B96
  \l 8B97
  \l 8B98
  \l 8B99
  \l 8B9A
  \l 8B9B
  \l 8B9C
  \l 8B9D
  \l 8B9E
  \l 8B9F
  \l 8BA0
  \l 8BA1
  \l 8BA2
  \l 8BA3
  \l 8BA4
  \l 8BA5
  \l 8BA6
  \l 8BA7
  \l 8BA8
  \l 8BA9
  \l 8BAA
  \l 8BAB
  \l 8BAC
  \l 8BAD
  \l 8BAE
  \l 8BAF
  \l 8BB0
  \l 8BB1
  \l 8BB2
  \l 8BB3
  \l 8BB4
  \l 8BB5
  \l 8BB6
  \l 8BB7
  \l 8BB8
  \l 8BB9
  \l 8BBA
  \l 8BBB
  \l 8BBC
  \l 8BBD
  \l 8BBE
  \l 8BBF
  \l 8BC0
  \l 8BC1
  \l 8BC2
  \l 8BC3
  \l 8BC4
  \l 8BC5
  \l 8BC6
  \l 8BC7
  \l 8BC8
  \l 8BC9
  \l 8BCA
  \l 8BCB
  \l 8BCC
  \l 8BCD
  \l 8BCE
  \l 8BCF
  \l 8BD0
  \l 8BD1
  \l 8BD2
  \l 8BD3
  \l 8BD4
  \l 8BD5
  \l 8BD6
  \l 8BD7
  \l 8BD8
  \l 8BD9
  \l 8BDA
  \l 8BDB
  \l 8BDC
  \l 8BDD
  \l 8BDE
  \l 8BDF
  \l 8BE0
  \l 8BE1
  \l 8BE2
  \l 8BE3
  \l 8BE4
  \l 8BE5
  \l 8BE6
  \l 8BE7
  \l 8BE8
  \l 8BE9
  \l 8BEA
  \l 8BEB
  \l 8BEC
  \l 8BED
  \l 8BEE
  \l 8BEF
  \l 8BF0
  \l 8BF1
  \l 8BF2
  \l 8BF3
  \l 8BF4
  \l 8BF5
  \l 8BF6
  \l 8BF7
  \l 8BF8
  \l 8BF9
  \l 8BFA
  \l 8BFB
  \l 8BFC
  \l 8BFD
  \l 8BFE
  \l 8BFF
  \l 8C00
  \l 8C01
  \l 8C02
  \l 8C03
  \l 8C04
  \l 8C05
  \l 8C06
  \l 8C07
  \l 8C08
  \l 8C09
  \l 8C0A
  \l 8C0B
  \l 8C0C
  \l 8C0D
  \l 8C0E
  \l 8C0F
  \l 8C10
  \l 8C11
  \l 8C12
  \l 8C13
  \l 8C14
  \l 8C15
  \l 8C16
  \l 8C17
  \l 8C18
  \l 8C19
  \l 8C1A
  \l 8C1B
  \l 8C1C
  \l 8C1D
  \l 8C1E
  \l 8C1F
  \l 8C20
  \l 8C21
  \l 8C22
  \l 8C23
  \l 8C24
  \l 8C25
  \l 8C26
  \l 8C27
  \l 8C28
  \l 8C29
  \l 8C2A
  \l 8C2B
  \l 8C2C
  \l 8C2D
  \l 8C2E
  \l 8C2F
  \l 8C30
  \l 8C31
  \l 8C32
  \l 8C33
  \l 8C34
  \l 8C35
  \l 8C36
  \l 8C37
  \l 8C38
  \l 8C39
  \l 8C3A
  \l 8C3B
  \l 8C3C
  \l 8C3D
  \l 8C3E
  \l 8C3F
  \l 8C40
  \l 8C41
  \l 8C42
  \l 8C43
  \l 8C44
  \l 8C45
  \l 8C46
  \l 8C47
  \l 8C48
  \l 8C49
  \l 8C4A
  \l 8C4B
  \l 8C4C
  \l 8C4D
  \l 8C4E
  \l 8C4F
  \l 8C50
  \l 8C51
  \l 8C52
  \l 8C53
  \l 8C54
  \l 8C55
  \l 8C56
  \l 8C57
  \l 8C58
  \l 8C59
  \l 8C5A
  \l 8C5B
  \l 8C5C
  \l 8C5D
  \l 8C5E
  \l 8C5F
  \l 8C60
  \l 8C61
  \l 8C62
  \l 8C63
  \l 8C64
  \l 8C65
  \l 8C66
  \l 8C67
  \l 8C68
  \l 8C69
  \l 8C6A
  \l 8C6B
  \l 8C6C
  \l 8C6D
  \l 8C6E
  \l 8C6F
  \l 8C70
  \l 8C71
  \l 8C72
  \l 8C73
  \l 8C74
  \l 8C75
  \l 8C76
  \l 8C77
  \l 8C78
  \l 8C79
  \l 8C7A
  \l 8C7B
  \l 8C7C
  \l 8C7D
  \l 8C7E
  \l 8C7F
  \l 8C80
  \l 8C81
  \l 8C82
  \l 8C83
  \l 8C84
  \l 8C85
  \l 8C86
  \l 8C87
  \l 8C88
  \l 8C89
  \l 8C8A
  \l 8C8B
  \l 8C8C
  \l 8C8D
  \l 8C8E
  \l 8C8F
  \l 8C90
  \l 8C91
  \l 8C92
  \l 8C93
  \l 8C94
  \l 8C95
  \l 8C96
  \l 8C97
  \l 8C98
  \l 8C99
  \l 8C9A
  \l 8C9B
  \l 8C9C
  \l 8C9D
  \l 8C9E
  \l 8C9F
  \l 8CA0
  \l 8CA1
  \l 8CA2
  \l 8CA3
  \l 8CA4
  \l 8CA5
  \l 8CA6
  \l 8CA7
  \l 8CA8
  \l 8CA9
  \l 8CAA
  \l 8CAB
  \l 8CAC
  \l 8CAD
  \l 8CAE
  \l 8CAF
  \l 8CB0
  \l 8CB1
  \l 8CB2
  \l 8CB3
  \l 8CB4
  \l 8CB5
  \l 8CB6
  \l 8CB7
  \l 8CB8
  \l 8CB9
  \l 8CBA
  \l 8CBB
  \l 8CBC
  \l 8CBD
  \l 8CBE
  \l 8CBF
  \l 8CC0
  \l 8CC1
  \l 8CC2
  \l 8CC3
  \l 8CC4
  \l 8CC5
  \l 8CC6
  \l 8CC7
  \l 8CC8
  \l 8CC9
  \l 8CCA
  \l 8CCB
  \l 8CCC
  \l 8CCD
  \l 8CCE
  \l 8CCF
  \l 8CD0
  \l 8CD1
  \l 8CD2
  \l 8CD3
  \l 8CD4
  \l 8CD5
  \l 8CD6
  \l 8CD7
  \l 8CD8
  \l 8CD9
  \l 8CDA
  \l 8CDB
  \l 8CDC
  \l 8CDD
  \l 8CDE
  \l 8CDF
  \l 8CE0
  \l 8CE1
  \l 8CE2
  \l 8CE3
  \l 8CE4
  \l 8CE5
  \l 8CE6
  \l 8CE7
  \l 8CE8
  \l 8CE9
  \l 8CEA
  \l 8CEB
  \l 8CEC
  \l 8CED
  \l 8CEE
  \l 8CEF
  \l 8CF0
  \l 8CF1
  \l 8CF2
  \l 8CF3
  \l 8CF4
  \l 8CF5
  \l 8CF6
  \l 8CF7
  \l 8CF8
  \l 8CF9
  \l 8CFA
  \l 8CFB
  \l 8CFC
  \l 8CFD
  \l 8CFE
  \l 8CFF
  \l 8D00
  \l 8D01
  \l 8D02
  \l 8D03
  \l 8D04
  \l 8D05
  \l 8D06
  \l 8D07
  \l 8D08
  \l 8D09
  \l 8D0A
  \l 8D0B
  \l 8D0C
  \l 8D0D
  \l 8D0E
  \l 8D0F
  \l 8D10
  \l 8D11
  \l 8D12
  \l 8D13
  \l 8D14
  \l 8D15
  \l 8D16
  \l 8D17
  \l 8D18
  \l 8D19
  \l 8D1A
  \l 8D1B
  \l 8D1C
  \l 8D1D
  \l 8D1E
  \l 8D1F
  \l 8D20
  \l 8D21
  \l 8D22
  \l 8D23
  \l 8D24
  \l 8D25
  \l 8D26
  \l 8D27
  \l 8D28
  \l 8D29
  \l 8D2A
  \l 8D2B
  \l 8D2C
  \l 8D2D
  \l 8D2E
  \l 8D2F
  \l 8D30
  \l 8D31
  \l 8D32
  \l 8D33
  \l 8D34
  \l 8D35
  \l 8D36
  \l 8D37
  \l 8D38
  \l 8D39
  \l 8D3A
  \l 8D3B
  \l 8D3C
  \l 8D3D
  \l 8D3E
  \l 8D3F
  \l 8D40
  \l 8D41
  \l 8D42
  \l 8D43
  \l 8D44
  \l 8D45
  \l 8D46
  \l 8D47
  \l 8D48
  \l 8D49
  \l 8D4A
  \l 8D4B
  \l 8D4C
  \l 8D4D
  \l 8D4E
  \l 8D4F
  \l 8D50
  \l 8D51
  \l 8D52
  \l 8D53
  \l 8D54
  \l 8D55
  \l 8D56
  \l 8D57
  \l 8D58
  \l 8D59
  \l 8D5A
  \l 8D5B
  \l 8D5C
  \l 8D5D
  \l 8D5E
  \l 8D5F
  \l 8D60
  \l 8D61
  \l 8D62
  \l 8D63
  \l 8D64
  \l 8D65
  \l 8D66
  \l 8D67
  \l 8D68
  \l 8D69
  \l 8D6A
  \l 8D6B
  \l 8D6C
  \l 8D6D
  \l 8D6E
  \l 8D6F
  \l 8D70
  \l 8D71
  \l 8D72
  \l 8D73
  \l 8D74
  \l 8D75
  \l 8D76
  \l 8D77
  \l 8D78
  \l 8D79
  \l 8D7A
  \l 8D7B
  \l 8D7C
  \l 8D7D
  \l 8D7E
  \l 8D7F
  \l 8D80
  \l 8D81
  \l 8D82
  \l 8D83
  \l 8D84
  \l 8D85
  \l 8D86
  \l 8D87
  \l 8D88
  \l 8D89
  \l 8D8A
  \l 8D8B
  \l 8D8C
  \l 8D8D
  \l 8D8E
  \l 8D8F
  \l 8D90
  \l 8D91
  \l 8D92
  \l 8D93
  \l 8D94
  \l 8D95
  \l 8D96
  \l 8D97
  \l 8D98
  \l 8D99
  \l 8D9A
  \l 8D9B
  \l 8D9C
  \l 8D9D
  \l 8D9E
  \l 8D9F
  \l 8DA0
  \l 8DA1
  \l 8DA2
  \l 8DA3
  \l 8DA4
  \l 8DA5
  \l 8DA6
  \l 8DA7
  \l 8DA8
  \l 8DA9
  \l 8DAA
  \l 8DAB
  \l 8DAC
  \l 8DAD
  \l 8DAE
  \l 8DAF
  \l 8DB0
  \l 8DB1
  \l 8DB2
  \l 8DB3
  \l 8DB4
  \l 8DB5
  \l 8DB6
  \l 8DB7
  \l 8DB8
  \l 8DB9
  \l 8DBA
  \l 8DBB
  \l 8DBC
  \l 8DBD
  \l 8DBE
  \l 8DBF
  \l 8DC0
  \l 8DC1
  \l 8DC2
  \l 8DC3
  \l 8DC4
  \l 8DC5
  \l 8DC6
  \l 8DC7
  \l 8DC8
  \l 8DC9
  \l 8DCA
  \l 8DCB
  \l 8DCC
  \l 8DCD
  \l 8DCE
  \l 8DCF
  \l 8DD0
  \l 8DD1
  \l 8DD2
  \l 8DD3
  \l 8DD4
  \l 8DD5
  \l 8DD6
  \l 8DD7
  \l 8DD8
  \l 8DD9
  \l 8DDA
  \l 8DDB
  \l 8DDC
  \l 8DDD
  \l 8DDE
  \l 8DDF
  \l 8DE0
  \l 8DE1
  \l 8DE2
  \l 8DE3
  \l 8DE4
  \l 8DE5
  \l 8DE6
  \l 8DE7
  \l 8DE8
  \l 8DE9
  \l 8DEA
  \l 8DEB
  \l 8DEC
  \l 8DED
  \l 8DEE
  \l 8DEF
  \l 8DF0
  \l 8DF1
  \l 8DF2
  \l 8DF3
  \l 8DF4
  \l 8DF5
  \l 8DF6
  \l 8DF7
  \l 8DF8
  \l 8DF9
  \l 8DFA
  \l 8DFB
  \l 8DFC
  \l 8DFD
  \l 8DFE
  \l 8DFF
  \l 8E00
  \l 8E01
  \l 8E02
  \l 8E03
  \l 8E04
  \l 8E05
  \l 8E06
  \l 8E07
  \l 8E08
  \l 8E09
  \l 8E0A
  \l 8E0B
  \l 8E0C
  \l 8E0D
  \l 8E0E
  \l 8E0F
  \l 8E10
  \l 8E11
  \l 8E12
  \l 8E13
  \l 8E14
  \l 8E15
  \l 8E16
  \l 8E17
  \l 8E18
  \l 8E19
  \l 8E1A
  \l 8E1B
  \l 8E1C
  \l 8E1D
  \l 8E1E
  \l 8E1F
  \l 8E20
  \l 8E21
  \l 8E22
  \l 8E23
  \l 8E24
  \l 8E25
  \l 8E26
  \l 8E27
  \l 8E28
  \l 8E29
  \l 8E2A
  \l 8E2B
  \l 8E2C
  \l 8E2D
  \l 8E2E
  \l 8E2F
  \l 8E30
  \l 8E31
  \l 8E32
  \l 8E33
  \l 8E34
  \l 8E35
  \l 8E36
  \l 8E37
  \l 8E38
  \l 8E39
  \l 8E3A
  \l 8E3B
  \l 8E3C
  \l 8E3D
  \l 8E3E
  \l 8E3F
  \l 8E40
  \l 8E41
  \l 8E42
  \l 8E43
  \l 8E44
  \l 8E45
  \l 8E46
  \l 8E47
  \l 8E48
  \l 8E49
  \l 8E4A
  \l 8E4B
  \l 8E4C
  \l 8E4D
  \l 8E4E
  \l 8E4F
  \l 8E50
  \l 8E51
  \l 8E52
  \l 8E53
  \l 8E54
  \l 8E55
  \l 8E56
  \l 8E57
  \l 8E58
  \l 8E59
  \l 8E5A
  \l 8E5B
  \l 8E5C
  \l 8E5D
  \l 8E5E
  \l 8E5F
  \l 8E60
  \l 8E61
  \l 8E62
  \l 8E63
  \l 8E64
  \l 8E65
  \l 8E66
  \l 8E67
  \l 8E68
  \l 8E69
  \l 8E6A
  \l 8E6B
  \l 8E6C
  \l 8E6D
  \l 8E6E
  \l 8E6F
  \l 8E70
  \l 8E71
  \l 8E72
  \l 8E73
  \l 8E74
  \l 8E75
  \l 8E76
  \l 8E77
  \l 8E78
  \l 8E79
  \l 8E7A
  \l 8E7B
  \l 8E7C
  \l 8E7D
  \l 8E7E
  \l 8E7F
  \l 8E80
  \l 8E81
  \l 8E82
  \l 8E83
  \l 8E84
  \l 8E85
  \l 8E86
  \l 8E87
  \l 8E88
  \l 8E89
  \l 8E8A
  \l 8E8B
  \l 8E8C
  \l 8E8D
  \l 8E8E
  \l 8E8F
  \l 8E90
  \l 8E91
  \l 8E92
  \l 8E93
  \l 8E94
  \l 8E95
  \l 8E96
  \l 8E97
  \l 8E98
  \l 8E99
  \l 8E9A
  \l 8E9B
  \l 8E9C
  \l 8E9D
  \l 8E9E
  \l 8E9F
  \l 8EA0
  \l 8EA1
  \l 8EA2
  \l 8EA3
  \l 8EA4
  \l 8EA5
  \l 8EA6
  \l 8EA7
  \l 8EA8
  \l 8EA9
  \l 8EAA
  \l 8EAB
  \l 8EAC
  \l 8EAD
  \l 8EAE
  \l 8EAF
  \l 8EB0
  \l 8EB1
  \l 8EB2
  \l 8EB3
  \l 8EB4
  \l 8EB5
  \l 8EB6
  \l 8EB7
  \l 8EB8
  \l 8EB9
  \l 8EBA
  \l 8EBB
  \l 8EBC
  \l 8EBD
  \l 8EBE
  \l 8EBF
  \l 8EC0
  \l 8EC1
  \l 8EC2
  \l 8EC3
  \l 8EC4
  \l 8EC5
  \l 8EC6
  \l 8EC7
  \l 8EC8
  \l 8EC9
  \l 8ECA
  \l 8ECB
  \l 8ECC
  \l 8ECD
  \l 8ECE
  \l 8ECF
  \l 8ED0
  \l 8ED1
  \l 8ED2
  \l 8ED3
  \l 8ED4
  \l 8ED5
  \l 8ED6
  \l 8ED7
  \l 8ED8
  \l 8ED9
  \l 8EDA
  \l 8EDB
  \l 8EDC
  \l 8EDD
  \l 8EDE
  \l 8EDF
  \l 8EE0
  \l 8EE1
  \l 8EE2
  \l 8EE3
  \l 8EE4
  \l 8EE5
  \l 8EE6
  \l 8EE7
  \l 8EE8
  \l 8EE9
  \l 8EEA
  \l 8EEB
  \l 8EEC
  \l 8EED
  \l 8EEE
  \l 8EEF
  \l 8EF0
  \l 8EF1
  \l 8EF2
  \l 8EF3
  \l 8EF4
  \l 8EF5
  \l 8EF6
  \l 8EF7
  \l 8EF8
  \l 8EF9
  \l 8EFA
  \l 8EFB
  \l 8EFC
  \l 8EFD
  \l 8EFE
  \l 8EFF
  \l 8F00
  \l 8F01
  \l 8F02
  \l 8F03
  \l 8F04
  \l 8F05
  \l 8F06
  \l 8F07
  \l 8F08
  \l 8F09
  \l 8F0A
  \l 8F0B
  \l 8F0C
  \l 8F0D
  \l 8F0E
  \l 8F0F
  \l 8F10
  \l 8F11
  \l 8F12
  \l 8F13
  \l 8F14
  \l 8F15
  \l 8F16
  \l 8F17
  \l 8F18
  \l 8F19
  \l 8F1A
  \l 8F1B
  \l 8F1C
  \l 8F1D
  \l 8F1E
  \l 8F1F
  \l 8F20
  \l 8F21
  \l 8F22
  \l 8F23
  \l 8F24
  \l 8F25
  \l 8F26
  \l 8F27
  \l 8F28
  \l 8F29
  \l 8F2A
  \l 8F2B
  \l 8F2C
  \l 8F2D
  \l 8F2E
  \l 8F2F
  \l 8F30
  \l 8F31
  \l 8F32
  \l 8F33
  \l 8F34
  \l 8F35
  \l 8F36
  \l 8F37
  \l 8F38
  \l 8F39
  \l 8F3A
  \l 8F3B
  \l 8F3C
  \l 8F3D
  \l 8F3E
  \l 8F3F
  \l 8F40
  \l 8F41
  \l 8F42
  \l 8F43
  \l 8F44
  \l 8F45
  \l 8F46
  \l 8F47
  \l 8F48
  \l 8F49
  \l 8F4A
  \l 8F4B
  \l 8F4C
  \l 8F4D
  \l 8F4E
  \l 8F4F
  \l 8F50
  \l 8F51
  \l 8F52
  \l 8F53
  \l 8F54
  \l 8F55
  \l 8F56
  \l 8F57
  \l 8F58
  \l 8F59
  \l 8F5A
  \l 8F5B
  \l 8F5C
  \l 8F5D
  \l 8F5E
  \l 8F5F
  \l 8F60
  \l 8F61
  \l 8F62
  \l 8F63
  \l 8F64
  \l 8F65
  \l 8F66
  \l 8F67
  \l 8F68
  \l 8F69
  \l 8F6A
  \l 8F6B
  \l 8F6C
  \l 8F6D
  \l 8F6E
  \l 8F6F
  \l 8F70
  \l 8F71
  \l 8F72
  \l 8F73
  \l 8F74
  \l 8F75
  \l 8F76
  \l 8F77
  \l 8F78
  \l 8F79
  \l 8F7A
  \l 8F7B
  \l 8F7C
  \l 8F7D
  \l 8F7E
  \l 8F7F
  \l 8F80
  \l 8F81
  \l 8F82
  \l 8F83
  \l 8F84
  \l 8F85
  \l 8F86
  \l 8F87
  \l 8F88
  \l 8F89
  \l 8F8A
  \l 8F8B
  \l 8F8C
  \l 8F8D
  \l 8F8E
  \l 8F8F
  \l 8F90
  \l 8F91
  \l 8F92
  \l 8F93
  \l 8F94
  \l 8F95
  \l 8F96
  \l 8F97
  \l 8F98
  \l 8F99
  \l 8F9A
  \l 8F9B
  \l 8F9C
  \l 8F9D
  \l 8F9E
  \l 8F9F
  \l 8FA0
  \l 8FA1
  \l 8FA2
  \l 8FA3
  \l 8FA4
  \l 8FA5
  \l 8FA6
  \l 8FA7
  \l 8FA8
  \l 8FA9
  \l 8FAA
  \l 8FAB
  \l 8FAC
  \l 8FAD
  \l 8FAE
  \l 8FAF
  \l 8FB0
  \l 8FB1
  \l 8FB2
  \l 8FB3
  \l 8FB4
  \l 8FB5
  \l 8FB6
  \l 8FB7
  \l 8FB8
  \l 8FB9
  \l 8FBA
  \l 8FBB
  \l 8FBC
  \l 8FBD
  \l 8FBE
  \l 8FBF
  \l 8FC0
  \l 8FC1
  \l 8FC2
  \l 8FC3
  \l 8FC4
  \l 8FC5
  \l 8FC6
  \l 8FC7
  \l 8FC8
  \l 8FC9
  \l 8FCA
  \l 8FCB
  \l 8FCC
  \l 8FCD
  \l 8FCE
  \l 8FCF
  \l 8FD0
  \l 8FD1
  \l 8FD2
  \l 8FD3
  \l 8FD4
  \l 8FD5
  \l 8FD6
  \l 8FD7
  \l 8FD8
  \l 8FD9
  \l 8FDA
  \l 8FDB
  \l 8FDC
  \l 8FDD
  \l 8FDE
  \l 8FDF
  \l 8FE0
  \l 8FE1
  \l 8FE2
  \l 8FE3
  \l 8FE4
  \l 8FE5
  \l 8FE6
  \l 8FE7
  \l 8FE8
  \l 8FE9
  \l 8FEA
  \l 8FEB
  \l 8FEC
  \l 8FED
  \l 8FEE
  \l 8FEF
  \l 8FF0
  \l 8FF1
  \l 8FF2
  \l 8FF3
  \l 8FF4
  \l 8FF5
  \l 8FF6
  \l 8FF7
  \l 8FF8
  \l 8FF9
  \l 8FFA
  \l 8FFB
  \l 8FFC
  \l 8FFD
  \l 8FFE
  \l 8FFF
  \l 9000
  \l 9001
  \l 9002
  \l 9003
  \l 9004
  \l 9005
  \l 9006
  \l 9007
  \l 9008
  \l 9009
  \l 900A
  \l 900B
  \l 900C
  \l 900D
  \l 900E
  \l 900F
  \l 9010
  \l 9011
  \l 9012
  \l 9013
  \l 9014
  \l 9015
  \l 9016
  \l 9017
  \l 9018
  \l 9019
  \l 901A
  \l 901B
  \l 901C
  \l 901D
  \l 901E
  \l 901F
  \l 9020
  \l 9021
  \l 9022
  \l 9023
  \l 9024
  \l 9025
  \l 9026
  \l 9027
  \l 9028
  \l 9029
  \l 902A
  \l 902B
  \l 902C
  \l 902D
  \l 902E
  \l 902F
  \l 9030
  \l 9031
  \l 9032
  \l 9033
  \l 9034
  \l 9035
  \l 9036
  \l 9037
  \l 9038
  \l 9039
  \l 903A
  \l 903B
  \l 903C
  \l 903D
  \l 903E
  \l 903F
  \l 9040
  \l 9041
  \l 9042
  \l 9043
  \l 9044
  \l 9045
  \l 9046
  \l 9047
  \l 9048
  \l 9049
  \l 904A
  \l 904B
  \l 904C
  \l 904D
  \l 904E
  \l 904F
  \l 9050
  \l 9051
  \l 9052
  \l 9053
  \l 9054
  \l 9055
  \l 9056
  \l 9057
  \l 9058
  \l 9059
  \l 905A
  \l 905B
  \l 905C
  \l 905D
  \l 905E
  \l 905F
  \l 9060
  \l 9061
  \l 9062
  \l 9063
  \l 9064
  \l 9065
  \l 9066
  \l 9067
  \l 9068
  \l 9069
  \l 906A
  \l 906B
  \l 906C
  \l 906D
  \l 906E
  \l 906F
  \l 9070
  \l 9071
  \l 9072
  \l 9073
  \l 9074
  \l 9075
  \l 9076
  \l 9077
  \l 9078
  \l 9079
  \l 907A
  \l 907B
  \l 907C
  \l 907D
  \l 907E
  \l 907F
  \l 9080
  \l 9081
  \l 9082
  \l 9083
  \l 9084
  \l 9085
  \l 9086
  \l 9087
  \l 9088
  \l 9089
  \l 908A
  \l 908B
  \l 908C
  \l 908D
  \l 908E
  \l 908F
  \l 9090
  \l 9091
  \l 9092
  \l 9093
  \l 9094
  \l 9095
  \l 9096
  \l 9097
  \l 9098
  \l 9099
  \l 909A
  \l 909B
  \l 909C
  \l 909D
  \l 909E
  \l 909F
  \l 90A0
  \l 90A1
  \l 90A2
  \l 90A3
  \l 90A4
  \l 90A5
  \l 90A6
  \l 90A7
  \l 90A8
  \l 90A9
  \l 90AA
  \l 90AB
  \l 90AC
  \l 90AD
  \l 90AE
  \l 90AF
  \l 90B0
  \l 90B1
  \l 90B2
  \l 90B3
  \l 90B4
  \l 90B5
  \l 90B6
  \l 90B7
  \l 90B8
  \l 90B9
  \l 90BA
  \l 90BB
  \l 90BC
  \l 90BD
  \l 90BE
  \l 90BF
  \l 90C0
  \l 90C1
  \l 90C2
  \l 90C3
  \l 90C4
  \l 90C5
  \l 90C6
  \l 90C7
  \l 90C8
  \l 90C9
  \l 90CA
  \l 90CB
  \l 90CC
  \l 90CD
  \l 90CE
  \l 90CF
  \l 90D0
  \l 90D1
  \l 90D2
  \l 90D3
  \l 90D4
  \l 90D5
  \l 90D6
  \l 90D7
  \l 90D8
  \l 90D9
  \l 90DA
  \l 90DB
  \l 90DC
  \l 90DD
  \l 90DE
  \l 90DF
  \l 90E0
  \l 90E1
  \l 90E2
  \l 90E3
  \l 90E4
  \l 90E5
  \l 90E6
  \l 90E7
  \l 90E8
  \l 90E9
  \l 90EA
  \l 90EB
  \l 90EC
  \l 90ED
  \l 90EE
  \l 90EF
  \l 90F0
  \l 90F1
  \l 90F2
  \l 90F3
  \l 90F4
  \l 90F5
  \l 90F6
  \l 90F7
  \l 90F8
  \l 90F9
  \l 90FA
  \l 90FB
  \l 90FC
  \l 90FD
  \l 90FE
  \l 90FF
  \l 9100
  \l 9101
  \l 9102
  \l 9103
  \l 9104
  \l 9105
  \l 9106
  \l 9107
  \l 9108
  \l 9109
  \l 910A
  \l 910B
  \l 910C
  \l 910D
  \l 910E
  \l 910F
  \l 9110
  \l 9111
  \l 9112
  \l 9113
  \l 9114
  \l 9115
  \l 9116
  \l 9117
  \l 9118
  \l 9119
  \l 911A
  \l 911B
  \l 911C
  \l 911D
  \l 911E
  \l 911F
  \l 9120
  \l 9121
  \l 9122
  \l 9123
  \l 9124
  \l 9125
  \l 9126
  \l 9127
  \l 9128
  \l 9129
  \l 912A
  \l 912B
  \l 912C
  \l 912D
  \l 912E
  \l 912F
  \l 9130
  \l 9131
  \l 9132
  \l 9133
  \l 9134
  \l 9135
  \l 9136
  \l 9137
  \l 9138
  \l 9139
  \l 913A
  \l 913B
  \l 913C
  \l 913D
  \l 913E
  \l 913F
  \l 9140
  \l 9141
  \l 9142
  \l 9143
  \l 9144
  \l 9145
  \l 9146
  \l 9147
  \l 9148
  \l 9149
  \l 914A
  \l 914B
  \l 914C
  \l 914D
  \l 914E
  \l 914F
  \l 9150
  \l 9151
  \l 9152
  \l 9153
  \l 9154
  \l 9155
  \l 9156
  \l 9157
  \l 9158
  \l 9159
  \l 915A
  \l 915B
  \l 915C
  \l 915D
  \l 915E
  \l 915F
  \l 9160
  \l 9161
  \l 9162
  \l 9163
  \l 9164
  \l 9165
  \l 9166
  \l 9167
  \l 9168
  \l 9169
  \l 916A
  \l 916B
  \l 916C
  \l 916D
  \l 916E
  \l 916F
  \l 9170
  \l 9171
  \l 9172
  \l 9173
  \l 9174
  \l 9175
  \l 9176
  \l 9177
  \l 9178
  \l 9179
  \l 917A
  \l 917B
  \l 917C
  \l 917D
  \l 917E
  \l 917F
  \l 9180
  \l 9181
  \l 9182
  \l 9183
  \l 9184
  \l 9185
  \l 9186
  \l 9187
  \l 9188
  \l 9189
  \l 918A
  \l 918B
  \l 918C
  \l 918D
  \l 918E
  \l 918F
  \l 9190
  \l 9191
  \l 9192
  \l 9193
  \l 9194
  \l 9195
  \l 9196
  \l 9197
  \l 9198
  \l 9199
  \l 919A
  \l 919B
  \l 919C
  \l 919D
  \l 919E
  \l 919F
  \l 91A0
  \l 91A1
  \l 91A2
  \l 91A3
  \l 91A4
  \l 91A5
  \l 91A6
  \l 91A7
  \l 91A8
  \l 91A9
  \l 91AA
  \l 91AB
  \l 91AC
  \l 91AD
  \l 91AE
  \l 91AF
  \l 91B0
  \l 91B1
  \l 91B2
  \l 91B3
  \l 91B4
  \l 91B5
  \l 91B6
  \l 91B7
  \l 91B8
  \l 91B9
  \l 91BA
  \l 91BB
  \l 91BC
  \l 91BD
  \l 91BE
  \l 91BF
  \l 91C0
  \l 91C1
  \l 91C2
  \l 91C3
  \l 91C4
  \l 91C5
  \l 91C6
  \l 91C7
  \l 91C8
  \l 91C9
  \l 91CA
  \l 91CB
  \l 91CC
  \l 91CD
  \l 91CE
  \l 91CF
  \l 91D0
  \l 91D1
  \l 91D2
  \l 91D3
  \l 91D4
  \l 91D5
  \l 91D6
  \l 91D7
  \l 91D8
  \l 91D9
  \l 91DA
  \l 91DB
  \l 91DC
  \l 91DD
  \l 91DE
  \l 91DF
  \l 91E0
  \l 91E1
  \l 91E2
  \l 91E3
  \l 91E4
  \l 91E5
  \l 91E6
  \l 91E7
  \l 91E8
  \l 91E9
  \l 91EA
  \l 91EB
  \l 91EC
  \l 91ED
  \l 91EE
  \l 91EF
  \l 91F0
  \l 91F1
  \l 91F2
  \l 91F3
  \l 91F4
  \l 91F5
  \l 91F6
  \l 91F7
  \l 91F8
  \l 91F9
  \l 91FA
  \l 91FB
  \l 91FC
  \l 91FD
  \l 91FE
  \l 91FF
  \l 9200
  \l 9201
  \l 9202
  \l 9203
  \l 9204
  \l 9205
  \l 9206
  \l 9207
  \l 9208
  \l 9209
  \l 920A
  \l 920B
  \l 920C
  \l 920D
  \l 920E
  \l 920F
  \l 9210
  \l 9211
  \l 9212
  \l 9213
  \l 9214
  \l 9215
  \l 9216
  \l 9217
  \l 9218
  \l 9219
  \l 921A
  \l 921B
  \l 921C
  \l 921D
  \l 921E
  \l 921F
  \l 9220
  \l 9221
  \l 9222
  \l 9223
  \l 9224
  \l 9225
  \l 9226
  \l 9227
  \l 9228
  \l 9229
  \l 922A
  \l 922B
  \l 922C
  \l 922D
  \l 922E
  \l 922F
  \l 9230
  \l 9231
  \l 9232
  \l 9233
  \l 9234
  \l 9235
  \l 9236
  \l 9237
  \l 9238
  \l 9239
  \l 923A
  \l 923B
  \l 923C
  \l 923D
  \l 923E
  \l 923F
  \l 9240
  \l 9241
  \l 9242
  \l 9243
  \l 9244
  \l 9245
  \l 9246
  \l 9247
  \l 9248
  \l 9249
  \l 924A
  \l 924B
  \l 924C
  \l 924D
  \l 924E
  \l 924F
  \l 9250
  \l 9251
  \l 9252
  \l 9253
  \l 9254
  \l 9255
  \l 9256
  \l 9257
  \l 9258
  \l 9259
  \l 925A
  \l 925B
  \l 925C
  \l 925D
  \l 925E
  \l 925F
  \l 9260
  \l 9261
  \l 9262
  \l 9263
  \l 9264
  \l 9265
  \l 9266
  \l 9267
  \l 9268
  \l 9269
  \l 926A
  \l 926B
  \l 926C
  \l 926D
  \l 926E
  \l 926F
  \l 9270
  \l 9271
  \l 9272
  \l 9273
  \l 9274
  \l 9275
  \l 9276
  \l 9277
  \l 9278
  \l 9279
  \l 927A
  \l 927B
  \l 927C
  \l 927D
  \l 927E
  \l 927F
  \l 9280
  \l 9281
  \l 9282
  \l 9283
  \l 9284
  \l 9285
  \l 9286
  \l 9287
  \l 9288
  \l 9289
  \l 928A
  \l 928B
  \l 928C
  \l 928D
  \l 928E
  \l 928F
  \l 9290
  \l 9291
  \l 9292
  \l 9293
  \l 9294
  \l 9295
  \l 9296
  \l 9297
  \l 9298
  \l 9299
  \l 929A
  \l 929B
  \l 929C
  \l 929D
  \l 929E
  \l 929F
  \l 92A0
  \l 92A1
  \l 92A2
  \l 92A3
  \l 92A4
  \l 92A5
  \l 92A6
  \l 92A7
  \l 92A8
  \l 92A9
  \l 92AA
  \l 92AB
  \l 92AC
  \l 92AD
  \l 92AE
  \l 92AF
  \l 92B0
  \l 92B1
  \l 92B2
  \l 92B3
  \l 92B4
  \l 92B5
  \l 92B6
  \l 92B7
  \l 92B8
  \l 92B9
  \l 92BA
  \l 92BB
  \l 92BC
  \l 92BD
  \l 92BE
  \l 92BF
  \l 92C0
  \l 92C1
  \l 92C2
  \l 92C3
  \l 92C4
  \l 92C5
  \l 92C6
  \l 92C7
  \l 92C8
  \l 92C9
  \l 92CA
  \l 92CB
  \l 92CC
  \l 92CD
  \l 92CE
  \l 92CF
  \l 92D0
  \l 92D1
  \l 92D2
  \l 92D3
  \l 92D4
  \l 92D5
  \l 92D6
  \l 92D7
  \l 92D8
  \l 92D9
  \l 92DA
  \l 92DB
  \l 92DC
  \l 92DD
  \l 92DE
  \l 92DF
  \l 92E0
  \l 92E1
  \l 92E2
  \l 92E3
  \l 92E4
  \l 92E5
  \l 92E6
  \l 92E7
  \l 92E8
  \l 92E9
  \l 92EA
  \l 92EB
  \l 92EC
  \l 92ED
  \l 92EE
  \l 92EF
  \l 92F0
  \l 92F1
  \l 92F2
  \l 92F3
  \l 92F4
  \l 92F5
  \l 92F6
  \l 92F7
  \l 92F8
  \l 92F9
  \l 92FA
  \l 92FB
  \l 92FC
  \l 92FD
  \l 92FE
  \l 92FF
  \l 9300
  \l 9301
  \l 9302
  \l 9303
  \l 9304
  \l 9305
  \l 9306
  \l 9307
  \l 9308
  \l 9309
  \l 930A
  \l 930B
  \l 930C
  \l 930D
  \l 930E
  \l 930F
  \l 9310
  \l 9311
  \l 9312
  \l 9313
  \l 9314
  \l 9315
  \l 9316
  \l 9317
  \l 9318
  \l 9319
  \l 931A
  \l 931B
  \l 931C
  \l 931D
  \l 931E
  \l 931F
  \l 9320
  \l 9321
  \l 9322
  \l 9323
  \l 9324
  \l 9325
  \l 9326
  \l 9327
  \l 9328
  \l 9329
  \l 932A
  \l 932B
  \l 932C
  \l 932D
  \l 932E
  \l 932F
  \l 9330
  \l 9331
  \l 9332
  \l 9333
  \l 9334
  \l 9335
  \l 9336
  \l 9337
  \l 9338
  \l 9339
  \l 933A
  \l 933B
  \l 933C
  \l 933D
  \l 933E
  \l 933F
  \l 9340
  \l 9341
  \l 9342
  \l 9343
  \l 9344
  \l 9345
  \l 9346
  \l 9347
  \l 9348
  \l 9349
  \l 934A
  \l 934B
  \l 934C
  \l 934D
  \l 934E
  \l 934F
  \l 9350
  \l 9351
  \l 9352
  \l 9353
  \l 9354
  \l 9355
  \l 9356
  \l 9357
  \l 9358
  \l 9359
  \l 935A
  \l 935B
  \l 935C
  \l 935D
  \l 935E
  \l 935F
  \l 9360
  \l 9361
  \l 9362
  \l 9363
  \l 9364
  \l 9365
  \l 9366
  \l 9367
  \l 9368
  \l 9369
  \l 936A
  \l 936B
  \l 936C
  \l 936D
  \l 936E
  \l 936F
  \l 9370
  \l 9371
  \l 9372
  \l 9373
  \l 9374
  \l 9375
  \l 9376
  \l 9377
  \l 9378
  \l 9379
  \l 937A
  \l 937B
  \l 937C
  \l 937D
  \l 937E
  \l 937F
  \l 9380
  \l 9381
  \l 9382
  \l 9383
  \l 9384
  \l 9385
  \l 9386
  \l 9387
  \l 9388
  \l 9389
  \l 938A
  \l 938B
  \l 938C
  \l 938D
  \l 938E
  \l 938F
  \l 9390
  \l 9391
  \l 9392
  \l 9393
  \l 9394
  \l 9395
  \l 9396
  \l 9397
  \l 9398
  \l 9399
  \l 939A
  \l 939B
  \l 939C
  \l 939D
  \l 939E
  \l 939F
  \l 93A0
  \l 93A1
  \l 93A2
  \l 93A3
  \l 93A4
  \l 93A5
  \l 93A6
  \l 93A7
  \l 93A8
  \l 93A9
  \l 93AA
  \l 93AB
  \l 93AC
  \l 93AD
  \l 93AE
  \l 93AF
  \l 93B0
  \l 93B1
  \l 93B2
  \l 93B3
  \l 93B4
  \l 93B5
  \l 93B6
  \l 93B7
  \l 93B8
  \l 93B9
  \l 93BA
  \l 93BB
  \l 93BC
  \l 93BD
  \l 93BE
  \l 93BF
  \l 93C0
  \l 93C1
  \l 93C2
  \l 93C3
  \l 93C4
  \l 93C5
  \l 93C6
  \l 93C7
  \l 93C8
  \l 93C9
  \l 93CA
  \l 93CB
  \l 93CC
  \l 93CD
  \l 93CE
  \l 93CF
  \l 93D0
  \l 93D1
  \l 93D2
  \l 93D3
  \l 93D4
  \l 93D5
  \l 93D6
  \l 93D7
  \l 93D8
  \l 93D9
  \l 93DA
  \l 93DB
  \l 93DC
  \l 93DD
  \l 93DE
  \l 93DF
  \l 93E0
  \l 93E1
  \l 93E2
  \l 93E3
  \l 93E4
  \l 93E5
  \l 93E6
  \l 93E7
  \l 93E8
  \l 93E9
  \l 93EA
  \l 93EB
  \l 93EC
  \l 93ED
  \l 93EE
  \l 93EF
  \l 93F0
  \l 93F1
  \l 93F2
  \l 93F3
  \l 93F4
  \l 93F5
  \l 93F6
  \l 93F7
  \l 93F8
  \l 93F9
  \l 93FA
  \l 93FB
  \l 93FC
  \l 93FD
  \l 93FE
  \l 93FF
  \l 9400
  \l 9401
  \l 9402
  \l 9403
  \l 9404
  \l 9405
  \l 9406
  \l 9407
  \l 9408
  \l 9409
  \l 940A
  \l 940B
  \l 940C
  \l 940D
  \l 940E
  \l 940F
  \l 9410
  \l 9411
  \l 9412
  \l 9413
  \l 9414
  \l 9415
  \l 9416
  \l 9417
  \l 9418
  \l 9419
  \l 941A
  \l 941B
  \l 941C
  \l 941D
  \l 941E
  \l 941F
  \l 9420
  \l 9421
  \l 9422
  \l 9423
  \l 9424
  \l 9425
  \l 9426
  \l 9427
  \l 9428
  \l 9429
  \l 942A
  \l 942B
  \l 942C
  \l 942D
  \l 942E
  \l 942F
  \l 9430
  \l 9431
  \l 9432
  \l 9433
  \l 9434
  \l 9435
  \l 9436
  \l 9437
  \l 9438
  \l 9439
  \l 943A
  \l 943B
  \l 943C
  \l 943D
  \l 943E
  \l 943F
  \l 9440
  \l 9441
  \l 9442
  \l 9443
  \l 9444
  \l 9445
  \l 9446
  \l 9447
  \l 9448
  \l 9449
  \l 944A
  \l 944B
  \l 944C
  \l 944D
  \l 944E
  \l 944F
  \l 9450
  \l 9451
  \l 9452
  \l 9453
  \l 9454
  \l 9455
  \l 9456
  \l 9457
  \l 9458
  \l 9459
  \l 945A
  \l 945B
  \l 945C
  \l 945D
  \l 945E
  \l 945F
  \l 9460
  \l 9461
  \l 9462
  \l 9463
  \l 9464
  \l 9465
  \l 9466
  \l 9467
  \l 9468
  \l 9469
  \l 946A
  \l 946B
  \l 946C
  \l 946D
  \l 946E
  \l 946F
  \l 9470
  \l 9471
  \l 9472
  \l 9473
  \l 9474
  \l 9475
  \l 9476
  \l 9477
  \l 9478
  \l 9479
  \l 947A
  \l 947B
  \l 947C
  \l 947D
  \l 947E
  \l 947F
  \l 9480
  \l 9481
  \l 9482
  \l 9483
  \l 9484
  \l 9485
  \l 9486
  \l 9487
  \l 9488
  \l 9489
  \l 948A
  \l 948B
  \l 948C
  \l 948D
  \l 948E
  \l 948F
  \l 9490
  \l 9491
  \l 9492
  \l 9493
  \l 9494
  \l 9495
  \l 9496
  \l 9497
  \l 9498
  \l 9499
  \l 949A
  \l 949B
  \l 949C
  \l 949D
  \l 949E
  \l 949F
  \l 94A0
  \l 94A1
  \l 94A2
  \l 94A3
  \l 94A4
  \l 94A5
  \l 94A6
  \l 94A7
  \l 94A8
  \l 94A9
  \l 94AA
  \l 94AB
  \l 94AC
  \l 94AD
  \l 94AE
  \l 94AF
  \l 94B0
  \l 94B1
  \l 94B2
  \l 94B3
  \l 94B4
  \l 94B5
  \l 94B6
  \l 94B7
  \l 94B8
  \l 94B9
  \l 94BA
  \l 94BB
  \l 94BC
  \l 94BD
  \l 94BE
  \l 94BF
  \l 94C0
  \l 94C1
  \l 94C2
  \l 94C3
  \l 94C4
  \l 94C5
  \l 94C6
  \l 94C7
  \l 94C8
  \l 94C9
  \l 94CA
  \l 94CB
  \l 94CC
  \l 94CD
  \l 94CE
  \l 94CF
  \l 94D0
  \l 94D1
  \l 94D2
  \l 94D3
  \l 94D4
  \l 94D5
  \l 94D6
  \l 94D7
  \l 94D8
  \l 94D9
  \l 94DA
  \l 94DB
  \l 94DC
  \l 94DD
  \l 94DE
  \l 94DF
  \l 94E0
  \l 94E1
  \l 94E2
  \l 94E3
  \l 94E4
  \l 94E5
  \l 94E6
  \l 94E7
  \l 94E8
  \l 94E9
  \l 94EA
  \l 94EB
  \l 94EC
  \l 94ED
  \l 94EE
  \l 94EF
  \l 94F0
  \l 94F1
  \l 94F2
  \l 94F3
  \l 94F4
  \l 94F5
  \l 94F6
  \l 94F7
  \l 94F8
  \l 94F9
  \l 94FA
  \l 94FB
  \l 94FC
  \l 94FD
  \l 94FE
  \l 94FF
  \l 9500
  \l 9501
  \l 9502
  \l 9503
  \l 9504
  \l 9505
  \l 9506
  \l 9507
  \l 9508
  \l 9509
  \l 950A
  \l 950B
  \l 950C
  \l 950D
  \l 950E
  \l 950F
  \l 9510
  \l 9511
  \l 9512
  \l 9513
  \l 9514
  \l 9515
  \l 9516
  \l 9517
  \l 9518
  \l 9519
  \l 951A
  \l 951B
  \l 951C
  \l 951D
  \l 951E
  \l 951F
  \l 9520
  \l 9521
  \l 9522
  \l 9523
  \l 9524
  \l 9525
  \l 9526
  \l 9527
  \l 9528
  \l 9529
  \l 952A
  \l 952B
  \l 952C
  \l 952D
  \l 952E
  \l 952F
  \l 9530
  \l 9531
  \l 9532
  \l 9533
  \l 9534
  \l 9535
  \l 9536
  \l 9537
  \l 9538
  \l 9539
  \l 953A
  \l 953B
  \l 953C
  \l 953D
  \l 953E
  \l 953F
  \l 9540
  \l 9541
  \l 9542
  \l 9543
  \l 9544
  \l 9545
  \l 9546
  \l 9547
  \l 9548
  \l 9549
  \l 954A
  \l 954B
  \l 954C
  \l 954D
  \l 954E
  \l 954F
  \l 9550
  \l 9551
  \l 9552
  \l 9553
  \l 9554
  \l 9555
  \l 9556
  \l 9557
  \l 9558
  \l 9559
  \l 955A
  \l 955B
  \l 955C
  \l 955D
  \l 955E
  \l 955F
  \l 9560
  \l 9561
  \l 9562
  \l 9563
  \l 9564
  \l 9565
  \l 9566
  \l 9567
  \l 9568
  \l 9569
  \l 956A
  \l 956B
  \l 956C
  \l 956D
  \l 956E
  \l 956F
  \l 9570
  \l 9571
  \l 9572
  \l 9573
  \l 9574
  \l 9575
  \l 9576
  \l 9577
  \l 9578
  \l 9579
  \l 957A
  \l 957B
  \l 957C
  \l 957D
  \l 957E
  \l 957F
  \l 9580
  \l 9581
  \l 9582
  \l 9583
  \l 9584
  \l 9585
  \l 9586
  \l 9587
  \l 9588
  \l 9589
  \l 958A
  \l 958B
  \l 958C
  \l 958D
  \l 958E
  \l 958F
  \l 9590
  \l 9591
  \l 9592
  \l 9593
  \l 9594
  \l 9595
  \l 9596
  \l 9597
  \l 9598
  \l 9599
  \l 959A
  \l 959B
  \l 959C
  \l 959D
  \l 959E
  \l 959F
  \l 95A0
  \l 95A1
  \l 95A2
  \l 95A3
  \l 95A4
  \l 95A5
  \l 95A6
  \l 95A7
  \l 95A8
  \l 95A9
  \l 95AA
  \l 95AB
  \l 95AC
  \l 95AD
  \l 95AE
  \l 95AF
  \l 95B0
  \l 95B1
  \l 95B2
  \l 95B3
  \l 95B4
  \l 95B5
  \l 95B6
  \l 95B7
  \l 95B8
  \l 95B9
  \l 95BA
  \l 95BB
  \l 95BC
  \l 95BD
  \l 95BE
  \l 95BF
  \l 95C0
  \l 95C1
  \l 95C2
  \l 95C3
  \l 95C4
  \l 95C5
  \l 95C6
  \l 95C7
  \l 95C8
  \l 95C9
  \l 95CA
  \l 95CB
  \l 95CC
  \l 95CD
  \l 95CE
  \l 95CF
  \l 95D0
  \l 95D1
  \l 95D2
  \l 95D3
  \l 95D4
  \l 95D5
  \l 95D6
  \l 95D7
  \l 95D8
  \l 95D9
  \l 95DA
  \l 95DB
  \l 95DC
  \l 95DD
  \l 95DE
  \l 95DF
  \l 95E0
  \l 95E1
  \l 95E2
  \l 95E3
  \l 95E4
  \l 95E5
  \l 95E6
  \l 95E7
  \l 95E8
  \l 95E9
  \l 95EA
  \l 95EB
  \l 95EC
  \l 95ED
  \l 95EE
  \l 95EF
  \l 95F0
  \l 95F1
  \l 95F2
  \l 95F3
  \l 95F4
  \l 95F5
  \l 95F6
  \l 95F7
  \l 95F8
  \l 95F9
  \l 95FA
  \l 95FB
  \l 95FC
  \l 95FD
  \l 95FE
  \l 95FF
  \l 9600
  \l 9601
  \l 9602
  \l 9603
  \l 9604
  \l 9605
  \l 9606
  \l 9607
  \l 9608
  \l 9609
  \l 960A
  \l 960B
  \l 960C
  \l 960D
  \l 960E
  \l 960F
  \l 9610
  \l 9611
  \l 9612
  \l 9613
  \l 9614
  \l 9615
  \l 9616
  \l 9617
  \l 9618
  \l 9619
  \l 961A
  \l 961B
  \l 961C
  \l 961D
  \l 961E
  \l 961F
  \l 9620
  \l 9621
  \l 9622
  \l 9623
  \l 9624
  \l 9625
  \l 9626
  \l 9627
  \l 9628
  \l 9629
  \l 962A
  \l 962B
  \l 962C
  \l 962D
  \l 962E
  \l 962F
  \l 9630
  \l 9631
  \l 9632
  \l 9633
  \l 9634
  \l 9635
  \l 9636
  \l 9637
  \l 9638
  \l 9639
  \l 963A
  \l 963B
  \l 963C
  \l 963D
  \l 963E
  \l 963F
  \l 9640
  \l 9641
  \l 9642
  \l 9643
  \l 9644
  \l 9645
  \l 9646
  \l 9647
  \l 9648
  \l 9649
  \l 964A
  \l 964B
  \l 964C
  \l 964D
  \l 964E
  \l 964F
  \l 9650
  \l 9651
  \l 9652
  \l 9653
  \l 9654
  \l 9655
  \l 9656
  \l 9657
  \l 9658
  \l 9659
  \l 965A
  \l 965B
  \l 965C
  \l 965D
  \l 965E
  \l 965F
  \l 9660
  \l 9661
  \l 9662
  \l 9663
  \l 9664
  \l 9665
  \l 9666
  \l 9667
  \l 9668
  \l 9669
  \l 966A
  \l 966B
  \l 966C
  \l 966D
  \l 966E
  \l 966F
  \l 9670
  \l 9671
  \l 9672
  \l 9673
  \l 9674
  \l 9675
  \l 9676
  \l 9677
  \l 9678
  \l 9679
  \l 967A
  \l 967B
  \l 967C
  \l 967D
  \l 967E
  \l 967F
  \l 9680
  \l 9681
  \l 9682
  \l 9683
  \l 9684
  \l 9685
  \l 9686
  \l 9687
  \l 9688
  \l 9689
  \l 968A
  \l 968B
  \l 968C
  \l 968D
  \l 968E
  \l 968F
  \l 9690
  \l 9691
  \l 9692
  \l 9693
  \l 9694
  \l 9695
  \l 9696
  \l 9697
  \l 9698
  \l 9699
  \l 969A
  \l 969B
  \l 969C
  \l 969D
  \l 969E
  \l 969F
  \l 96A0
  \l 96A1
  \l 96A2
  \l 96A3
  \l 96A4
  \l 96A5
  \l 96A6
  \l 96A7
  \l 96A8
  \l 96A9
  \l 96AA
  \l 96AB
  \l 96AC
  \l 96AD
  \l 96AE
  \l 96AF
  \l 96B0
  \l 96B1
  \l 96B2
  \l 96B3
  \l 96B4
  \l 96B5
  \l 96B6
  \l 96B7
  \l 96B8
  \l 96B9
  \l 96BA
  \l 96BB
  \l 96BC
  \l 96BD
  \l 96BE
  \l 96BF
  \l 96C0
  \l 96C1
  \l 96C2
  \l 96C3
  \l 96C4
  \l 96C5
  \l 96C6
  \l 96C7
  \l 96C8
  \l 96C9
  \l 96CA
  \l 96CB
  \l 96CC
  \l 96CD
  \l 96CE
  \l 96CF
  \l 96D0
  \l 96D1
  \l 96D2
  \l 96D3
  \l 96D4
  \l 96D5
  \l 96D6
  \l 96D7
  \l 96D8
  \l 96D9
  \l 96DA
  \l 96DB
  \l 96DC
  \l 96DD
  \l 96DE
  \l 96DF
  \l 96E0
  \l 96E1
  \l 96E2
  \l 96E3
  \l 96E4
  \l 96E5
  \l 96E6
  \l 96E7
  \l 96E8
  \l 96E9
  \l 96EA
  \l 96EB
  \l 96EC
  \l 96ED
  \l 96EE
  \l 96EF
  \l 96F0
  \l 96F1
  \l 96F2
  \l 96F3
  \l 96F4
  \l 96F5
  \l 96F6
  \l 96F7
  \l 96F8
  \l 96F9
  \l 96FA
  \l 96FB
  \l 96FC
  \l 96FD
  \l 96FE
  \l 96FF
  \l 9700
  \l 9701
  \l 9702
  \l 9703
  \l 9704
  \l 9705
  \l 9706
  \l 9707
  \l 9708
  \l 9709
  \l 970A
  \l 970B
  \l 970C
  \l 970D
  \l 970E
  \l 970F
  \l 9710
  \l 9711
  \l 9712
  \l 9713
  \l 9714
  \l 9715
  \l 9716
  \l 9717
  \l 9718
  \l 9719
  \l 971A
  \l 971B
  \l 971C
  \l 971D
  \l 971E
  \l 971F
  \l 9720
  \l 9721
  \l 9722
  \l 9723
  \l 9724
  \l 9725
  \l 9726
  \l 9727
  \l 9728
  \l 9729
  \l 972A
  \l 972B
  \l 972C
  \l 972D
  \l 972E
  \l 972F
  \l 9730
  \l 9731
  \l 9732
  \l 9733
  \l 9734
  \l 9735
  \l 9736
  \l 9737
  \l 9738
  \l 9739
  \l 973A
  \l 973B
  \l 973C
  \l 973D
  \l 973E
  \l 973F
  \l 9740
  \l 9741
  \l 9742
  \l 9743
  \l 9744
  \l 9745
  \l 9746
  \l 9747
  \l 9748
  \l 9749
  \l 974A
  \l 974B
  \l 974C
  \l 974D
  \l 974E
  \l 974F
  \l 9750
  \l 9751
  \l 9752
  \l 9753
  \l 9754
  \l 9755
  \l 9756
  \l 9757
  \l 9758
  \l 9759
  \l 975A
  \l 975B
  \l 975C
  \l 975D
  \l 975E
  \l 975F
  \l 9760
  \l 9761
  \l 9762
  \l 9763
  \l 9764
  \l 9765
  \l 9766
  \l 9767
  \l 9768
  \l 9769
  \l 976A
  \l 976B
  \l 976C
  \l 976D
  \l 976E
  \l 976F
  \l 9770
  \l 9771
  \l 9772
  \l 9773
  \l 9774
  \l 9775
  \l 9776
  \l 9777
  \l 9778
  \l 9779
  \l 977A
  \l 977B
  \l 977C
  \l 977D
  \l 977E
  \l 977F
  \l 9780
  \l 9781
  \l 9782
  \l 9783
  \l 9784
  \l 9785
  \l 9786
  \l 9787
  \l 9788
  \l 9789
  \l 978A
  \l 978B
  \l 978C
  \l 978D
  \l 978E
  \l 978F
  \l 9790
  \l 9791
  \l 9792
  \l 9793
  \l 9794
  \l 9795
  \l 9796
  \l 9797
  \l 9798
  \l 9799
  \l 979A
  \l 979B
  \l 979C
  \l 979D
  \l 979E
  \l 979F
  \l 97A0
  \l 97A1
  \l 97A2
  \l 97A3
  \l 97A4
  \l 97A5
  \l 97A6
  \l 97A7
  \l 97A8
  \l 97A9
  \l 97AA
  \l 97AB
  \l 97AC
  \l 97AD
  \l 97AE
  \l 97AF
  \l 97B0
  \l 97B1
  \l 97B2
  \l 97B3
  \l 97B4
  \l 97B5
  \l 97B6
  \l 97B7
  \l 97B8
  \l 97B9
  \l 97BA
  \l 97BB
  \l 97BC
  \l 97BD
  \l 97BE
  \l 97BF
  \l 97C0
  \l 97C1
  \l 97C2
  \l 97C3
  \l 97C4
  \l 97C5
  \l 97C6
  \l 97C7
  \l 97C8
  \l 97C9
  \l 97CA
  \l 97CB
  \l 97CC
  \l 97CD
  \l 97CE
  \l 97CF
  \l 97D0
  \l 97D1
  \l 97D2
  \l 97D3
  \l 97D4
  \l 97D5
  \l 97D6
  \l 97D7
  \l 97D8
  \l 97D9
  \l 97DA
  \l 97DB
  \l 97DC
  \l 97DD
  \l 97DE
  \l 97DF
  \l 97E0
  \l 97E1
  \l 97E2
  \l 97E3
  \l 97E4
  \l 97E5
  \l 97E6
  \l 97E7
  \l 97E8
  \l 97E9
  \l 97EA
  \l 97EB
  \l 97EC
  \l 97ED
  \l 97EE
  \l 97EF
  \l 97F0
  \l 97F1
  \l 97F2
  \l 97F3
  \l 97F4
  \l 97F5
  \l 97F6
  \l 97F7
  \l 97F8
  \l 97F9
  \l 97FA
  \l 97FB
  \l 97FC
  \l 97FD
  \l 97FE
  \l 97FF
  \l 9800
  \l 9801
  \l 9802
  \l 9803
  \l 9804
  \l 9805
  \l 9806
  \l 9807
  \l 9808
  \l 9809
  \l 980A
  \l 980B
  \l 980C
  \l 980D
  \l 980E
  \l 980F
  \l 9810
  \l 9811
  \l 9812
  \l 9813
  \l 9814
  \l 9815
  \l 9816
  \l 9817
  \l 9818
  \l 9819
  \l 981A
  \l 981B
  \l 981C
  \l 981D
  \l 981E
  \l 981F
  \l 9820
  \l 9821
  \l 9822
  \l 9823
  \l 9824
  \l 9825
  \l 9826
  \l 9827
  \l 9828
  \l 9829
  \l 982A
  \l 982B
  \l 982C
  \l 982D
  \l 982E
  \l 982F
  \l 9830
  \l 9831
  \l 9832
  \l 9833
  \l 9834
  \l 9835
  \l 9836
  \l 9837
  \l 9838
  \l 9839
  \l 983A
  \l 983B
  \l 983C
  \l 983D
  \l 983E
  \l 983F
  \l 9840
  \l 9841
  \l 9842
  \l 9843
  \l 9844
  \l 9845
  \l 9846
  \l 9847
  \l 9848
  \l 9849
  \l 984A
  \l 984B
  \l 984C
  \l 984D
  \l 984E
  \l 984F
  \l 9850
  \l 9851
  \l 9852
  \l 9853
  \l 9854
  \l 9855
  \l 9856
  \l 9857
  \l 9858
  \l 9859
  \l 985A
  \l 985B
  \l 985C
  \l 985D
  \l 985E
  \l 985F
  \l 9860
  \l 9861
  \l 9862
  \l 9863
  \l 9864
  \l 9865
  \l 9866
  \l 9867
  \l 9868
  \l 9869
  \l 986A
  \l 986B
  \l 986C
  \l 986D
  \l 986E
  \l 986F
  \l 9870
  \l 9871
  \l 9872
  \l 9873
  \l 9874
  \l 9875
  \l 9876
  \l 9877
  \l 9878
  \l 9879
  \l 987A
  \l 987B
  \l 987C
  \l 987D
  \l 987E
  \l 987F
  \l 9880
  \l 9881
  \l 9882
  \l 9883
  \l 9884
  \l 9885
  \l 9886
  \l 9887
  \l 9888
  \l 9889
  \l 988A
  \l 988B
  \l 988C
  \l 988D
  \l 988E
  \l 988F
  \l 9890
  \l 9891
  \l 9892
  \l 9893
  \l 9894
  \l 9895
  \l 9896
  \l 9897
  \l 9898
  \l 9899
  \l 989A
  \l 989B
  \l 989C
  \l 989D
  \l 989E
  \l 989F
  \l 98A0
  \l 98A1
  \l 98A2
  \l 98A3
  \l 98A4
  \l 98A5
  \l 98A6
  \l 98A7
  \l 98A8
  \l 98A9
  \l 98AA
  \l 98AB
  \l 98AC
  \l 98AD
  \l 98AE
  \l 98AF
  \l 98B0
  \l 98B1
  \l 98B2
  \l 98B3
  \l 98B4
  \l 98B5
  \l 98B6
  \l 98B7
  \l 98B8
  \l 98B9
  \l 98BA
  \l 98BB
  \l 98BC
  \l 98BD
  \l 98BE
  \l 98BF
  \l 98C0
  \l 98C1
  \l 98C2
  \l 98C3
  \l 98C4
  \l 98C5
  \l 98C6
  \l 98C7
  \l 98C8
  \l 98C9
  \l 98CA
  \l 98CB
  \l 98CC
  \l 98CD
  \l 98CE
  \l 98CF
  \l 98D0
  \l 98D1
  \l 98D2
  \l 98D3
  \l 98D4
  \l 98D5
  \l 98D6
  \l 98D7
  \l 98D8
  \l 98D9
  \l 98DA
  \l 98DB
  \l 98DC
  \l 98DD
  \l 98DE
  \l 98DF
  \l 98E0
  \l 98E1
  \l 98E2
  \l 98E3
  \l 98E4
  \l 98E5
  \l 98E6
  \l 98E7
  \l 98E8
  \l 98E9
  \l 98EA
  \l 98EB
  \l 98EC
  \l 98ED
  \l 98EE
  \l 98EF
  \l 98F0
  \l 98F1
  \l 98F2
  \l 98F3
  \l 98F4
  \l 98F5
  \l 98F6
  \l 98F7
  \l 98F8
  \l 98F9
  \l 98FA
  \l 98FB
  \l 98FC
  \l 98FD
  \l 98FE
  \l 98FF
  \l 9900
  \l 9901
  \l 9902
  \l 9903
  \l 9904
  \l 9905
  \l 9906
  \l 9907
  \l 9908
  \l 9909
  \l 990A
  \l 990B
  \l 990C
  \l 990D
  \l 990E
  \l 990F
  \l 9910
  \l 9911
  \l 9912
  \l 9913
  \l 9914
  \l 9915
  \l 9916
  \l 9917
  \l 9918
  \l 9919
  \l 991A
  \l 991B
  \l 991C
  \l 991D
  \l 991E
  \l 991F
  \l 9920
  \l 9921
  \l 9922
  \l 9923
  \l 9924
  \l 9925
  \l 9926
  \l 9927
  \l 9928
  \l 9929
  \l 992A
  \l 992B
  \l 992C
  \l 992D
  \l 992E
  \l 992F
  \l 9930
  \l 9931
  \l 9932
  \l 9933
  \l 9934
  \l 9935
  \l 9936
  \l 9937
  \l 9938
  \l 9939
  \l 993A
  \l 993B
  \l 993C
  \l 993D
  \l 993E
  \l 993F
  \l 9940
  \l 9941
  \l 9942
  \l 9943
  \l 9944
  \l 9945
  \l 9946
  \l 9947
  \l 9948
  \l 9949
  \l 994A
  \l 994B
  \l 994C
  \l 994D
  \l 994E
  \l 994F
  \l 9950
  \l 9951
  \l 9952
  \l 9953
  \l 9954
  \l 9955
  \l 9956
  \l 9957
  \l 9958
  \l 9959
  \l 995A
  \l 995B
  \l 995C
  \l 995D
  \l 995E
  \l 995F
  \l 9960
  \l 9961
  \l 9962
  \l 9963
  \l 9964
  \l 9965
  \l 9966
  \l 9967
  \l 9968
  \l 9969
  \l 996A
  \l 996B
  \l 996C
  \l 996D
  \l 996E
  \l 996F
  \l 9970
  \l 9971
  \l 9972
  \l 9973
  \l 9974
  \l 9975
  \l 9976
  \l 9977
  \l 9978
  \l 9979
  \l 997A
  \l 997B
  \l 997C
  \l 997D
  \l 997E
  \l 997F
  \l 9980
  \l 9981
  \l 9982
  \l 9983
  \l 9984
  \l 9985
  \l 9986
  \l 9987
  \l 9988
  \l 9989
  \l 998A
  \l 998B
  \l 998C
  \l 998D
  \l 998E
  \l 998F
  \l 9990
  \l 9991
  \l 9992
  \l 9993
  \l 9994
  \l 9995
  \l 9996
  \l 9997
  \l 9998
  \l 9999
  \l 999A
  \l 999B
  \l 999C
  \l 999D
  \l 999E
  \l 999F
  \l 99A0
  \l 99A1
  \l 99A2
  \l 99A3
  \l 99A4
  \l 99A5
  \l 99A6
  \l 99A7
  \l 99A8
  \l 99A9
  \l 99AA
  \l 99AB
  \l 99AC
  \l 99AD
  \l 99AE
  \l 99AF
  \l 99B0
  \l 99B1
  \l 99B2
  \l 99B3
  \l 99B4
  \l 99B5
  \l 99B6
  \l 99B7
  \l 99B8
  \l 99B9
  \l 99BA
  \l 99BB
  \l 99BC
  \l 99BD
  \l 99BE
  \l 99BF
  \l 99C0
  \l 99C1
  \l 99C2
  \l 99C3
  \l 99C4
  \l 99C5
  \l 99C6
  \l 99C7
  \l 99C8
  \l 99C9
  \l 99CA
  \l 99CB
  \l 99CC
  \l 99CD
  \l 99CE
  \l 99CF
  \l 99D0
  \l 99D1
  \l 99D2
  \l 99D3
  \l 99D4
  \l 99D5
  \l 99D6
  \l 99D7
  \l 99D8
  \l 99D9
  \l 99DA
  \l 99DB
  \l 99DC
  \l 99DD
  \l 99DE
  \l 99DF
  \l 99E0
  \l 99E1
  \l 99E2
  \l 99E3
  \l 99E4
  \l 99E5
  \l 99E6
  \l 99E7
  \l 99E8
  \l 99E9
  \l 99EA
  \l 99EB
  \l 99EC
  \l 99ED
  \l 99EE
  \l 99EF
  \l 99F0
  \l 99F1
  \l 99F2
  \l 99F3
  \l 99F4
  \l 99F5
  \l 99F6
  \l 99F7
  \l 99F8
  \l 99F9
  \l 99FA
  \l 99FB
  \l 99FC
  \l 99FD
  \l 99FE
  \l 99FF
  \l 9A00
  \l 9A01
  \l 9A02
  \l 9A03
  \l 9A04
  \l 9A05
  \l 9A06
  \l 9A07
  \l 9A08
  \l 9A09
  \l 9A0A
  \l 9A0B
  \l 9A0C
  \l 9A0D
  \l 9A0E
  \l 9A0F
  \l 9A10
  \l 9A11
  \l 9A12
  \l 9A13
  \l 9A14
  \l 9A15
  \l 9A16
  \l 9A17
  \l 9A18
  \l 9A19
  \l 9A1A
  \l 9A1B
  \l 9A1C
  \l 9A1D
  \l 9A1E
  \l 9A1F
  \l 9A20
  \l 9A21
  \l 9A22
  \l 9A23
  \l 9A24
  \l 9A25
  \l 9A26
  \l 9A27
  \l 9A28
  \l 9A29
  \l 9A2A
  \l 9A2B
  \l 9A2C
  \l 9A2D
  \l 9A2E
  \l 9A2F
  \l 9A30
  \l 9A31
  \l 9A32
  \l 9A33
  \l 9A34
  \l 9A35
  \l 9A36
  \l 9A37
  \l 9A38
  \l 9A39
  \l 9A3A
  \l 9A3B
  \l 9A3C
  \l 9A3D
  \l 9A3E
  \l 9A3F
  \l 9A40
  \l 9A41
  \l 9A42
  \l 9A43
  \l 9A44
  \l 9A45
  \l 9A46
  \l 9A47
  \l 9A48
  \l 9A49
  \l 9A4A
  \l 9A4B
  \l 9A4C
  \l 9A4D
  \l 9A4E
  \l 9A4F
  \l 9A50
  \l 9A51
  \l 9A52
  \l 9A53
  \l 9A54
  \l 9A55
  \l 9A56
  \l 9A57
  \l 9A58
  \l 9A59
  \l 9A5A
  \l 9A5B
  \l 9A5C
  \l 9A5D
  \l 9A5E
  \l 9A5F
  \l 9A60
  \l 9A61
  \l 9A62
  \l 9A63
  \l 9A64
  \l 9A65
  \l 9A66
  \l 9A67
  \l 9A68
  \l 9A69
  \l 9A6A
  \l 9A6B
  \l 9A6C
  \l 9A6D
  \l 9A6E
  \l 9A6F
  \l 9A70
  \l 9A71
  \l 9A72
  \l 9A73
  \l 9A74
  \l 9A75
  \l 9A76
  \l 9A77
  \l 9A78
  \l 9A79
  \l 9A7A
  \l 9A7B
  \l 9A7C
  \l 9A7D
  \l 9A7E
  \l 9A7F
  \l 9A80
  \l 9A81
  \l 9A82
  \l 9A83
  \l 9A84
  \l 9A85
  \l 9A86
  \l 9A87
  \l 9A88
  \l 9A89
  \l 9A8A
  \l 9A8B
  \l 9A8C
  \l 9A8D
  \l 9A8E
  \l 9A8F
  \l 9A90
  \l 9A91
  \l 9A92
  \l 9A93
  \l 9A94
  \l 9A95
  \l 9A96
  \l 9A97
  \l 9A98
  \l 9A99
  \l 9A9A
  \l 9A9B
  \l 9A9C
  \l 9A9D
  \l 9A9E
  \l 9A9F
  \l 9AA0
  \l 9AA1
  \l 9AA2
  \l 9AA3
  \l 9AA4
  \l 9AA5
  \l 9AA6
  \l 9AA7
  \l 9AA8
  \l 9AA9
  \l 9AAA
  \l 9AAB
  \l 9AAC
  \l 9AAD
  \l 9AAE
  \l 9AAF
  \l 9AB0
  \l 9AB1
  \l 9AB2
  \l 9AB3
  \l 9AB4
  \l 9AB5
  \l 9AB6
  \l 9AB7
  \l 9AB8
  \l 9AB9
  \l 9ABA
  \l 9ABB
  \l 9ABC
  \l 9ABD
  \l 9ABE
  \l 9ABF
  \l 9AC0
  \l 9AC1
  \l 9AC2
  \l 9AC3
  \l 9AC4
  \l 9AC5
  \l 9AC6
  \l 9AC7
  \l 9AC8
  \l 9AC9
  \l 9ACA
  \l 9ACB
  \l 9ACC
  \l 9ACD
  \l 9ACE
  \l 9ACF
  \l 9AD0
  \l 9AD1
  \l 9AD2
  \l 9AD3
  \l 9AD4
  \l 9AD5
  \l 9AD6
  \l 9AD7
  \l 9AD8
  \l 9AD9
  \l 9ADA
  \l 9ADB
  \l 9ADC
  \l 9ADD
  \l 9ADE
  \l 9ADF
  \l 9AE0
  \l 9AE1
  \l 9AE2
  \l 9AE3
  \l 9AE4
  \l 9AE5
  \l 9AE6
  \l 9AE7
  \l 9AE8
  \l 9AE9
  \l 9AEA
  \l 9AEB
  \l 9AEC
  \l 9AED
  \l 9AEE
  \l 9AEF
  \l 9AF0
  \l 9AF1
  \l 9AF2
  \l 9AF3
  \l 9AF4
  \l 9AF5
  \l 9AF6
  \l 9AF7
  \l 9AF8
  \l 9AF9
  \l 9AFA
  \l 9AFB
  \l 9AFC
  \l 9AFD
  \l 9AFE
  \l 9AFF
  \l 9B00
  \l 9B01
  \l 9B02
  \l 9B03
  \l 9B04
  \l 9B05
  \l 9B06
  \l 9B07
  \l 9B08
  \l 9B09
  \l 9B0A
  \l 9B0B
  \l 9B0C
  \l 9B0D
  \l 9B0E
  \l 9B0F
  \l 9B10
  \l 9B11
  \l 9B12
  \l 9B13
  \l 9B14
  \l 9B15
  \l 9B16
  \l 9B17
  \l 9B18
  \l 9B19
  \l 9B1A
  \l 9B1B
  \l 9B1C
  \l 9B1D
  \l 9B1E
  \l 9B1F
  \l 9B20
  \l 9B21
  \l 9B22
  \l 9B23
  \l 9B24
  \l 9B25
  \l 9B26
  \l 9B27
  \l 9B28
  \l 9B29
  \l 9B2A
  \l 9B2B
  \l 9B2C
  \l 9B2D
  \l 9B2E
  \l 9B2F
  \l 9B30
  \l 9B31
  \l 9B32
  \l 9B33
  \l 9B34
  \l 9B35
  \l 9B36
  \l 9B37
  \l 9B38
  \l 9B39
  \l 9B3A
  \l 9B3B
  \l 9B3C
  \l 9B3D
  \l 9B3E
  \l 9B3F
  \l 9B40
  \l 9B41
  \l 9B42
  \l 9B43
  \l 9B44
  \l 9B45
  \l 9B46
  \l 9B47
  \l 9B48
  \l 9B49
  \l 9B4A
  \l 9B4B
  \l 9B4C
  \l 9B4D
  \l 9B4E
  \l 9B4F
  \l 9B50
  \l 9B51
  \l 9B52
  \l 9B53
  \l 9B54
  \l 9B55
  \l 9B56
  \l 9B57
  \l 9B58
  \l 9B59
  \l 9B5A
  \l 9B5B
  \l 9B5C
  \l 9B5D
  \l 9B5E
  \l 9B5F
  \l 9B60
  \l 9B61
  \l 9B62
  \l 9B63
  \l 9B64
  \l 9B65
  \l 9B66
  \l 9B67
  \l 9B68
  \l 9B69
  \l 9B6A
  \l 9B6B
  \l 9B6C
  \l 9B6D
  \l 9B6E
  \l 9B6F
  \l 9B70
  \l 9B71
  \l 9B72
  \l 9B73
  \l 9B74
  \l 9B75
  \l 9B76
  \l 9B77
  \l 9B78
  \l 9B79
  \l 9B7A
  \l 9B7B
  \l 9B7C
  \l 9B7D
  \l 9B7E
  \l 9B7F
  \l 9B80
  \l 9B81
  \l 9B82
  \l 9B83
  \l 9B84
  \l 9B85
  \l 9B86
  \l 9B87
  \l 9B88
  \l 9B89
  \l 9B8A
  \l 9B8B
  \l 9B8C
  \l 9B8D
  \l 9B8E
  \l 9B8F
  \l 9B90
  \l 9B91
  \l 9B92
  \l 9B93
  \l 9B94
  \l 9B95
  \l 9B96
  \l 9B97
  \l 9B98
  \l 9B99
  \l 9B9A
  \l 9B9B
  \l 9B9C
  \l 9B9D
  \l 9B9E
  \l 9B9F
  \l 9BA0
  \l 9BA1
  \l 9BA2
  \l 9BA3
  \l 9BA4
  \l 9BA5
  \l 9BA6
  \l 9BA7
  \l 9BA8
  \l 9BA9
  \l 9BAA
  \l 9BAB
  \l 9BAC
  \l 9BAD
  \l 9BAE
  \l 9BAF
  \l 9BB0
  \l 9BB1
  \l 9BB2
  \l 9BB3
  \l 9BB4
  \l 9BB5
  \l 9BB6
  \l 9BB7
  \l 9BB8
  \l 9BB9
  \l 9BBA
  \l 9BBB
  \l 9BBC
  \l 9BBD
  \l 9BBE
  \l 9BBF
  \l 9BC0
  \l 9BC1
  \l 9BC2
  \l 9BC3
  \l 9BC4
  \l 9BC5
  \l 9BC6
  \l 9BC7
  \l 9BC8
  \l 9BC9
  \l 9BCA
  \l 9BCB
  \l 9BCC
  \l 9BCD
  \l 9BCE
  \l 9BCF
  \l 9BD0
  \l 9BD1
  \l 9BD2
  \l 9BD3
  \l 9BD4
  \l 9BD5
  \l 9BD6
  \l 9BD7
  \l 9BD8
  \l 9BD9
  \l 9BDA
  \l 9BDB
  \l 9BDC
  \l 9BDD
  \l 9BDE
  \l 9BDF
  \l 9BE0
  \l 9BE1
  \l 9BE2
  \l 9BE3
  \l 9BE4
  \l 9BE5
  \l 9BE6
  \l 9BE7
  \l 9BE8
  \l 9BE9
  \l 9BEA
  \l 9BEB
  \l 9BEC
  \l 9BED
  \l 9BEE
  \l 9BEF
  \l 9BF0
  \l 9BF1
  \l 9BF2
  \l 9BF3
  \l 9BF4
  \l 9BF5
  \l 9BF6
  \l 9BF7
  \l 9BF8
  \l 9BF9
  \l 9BFA
  \l 9BFB
  \l 9BFC
  \l 9BFD
  \l 9BFE
  \l 9BFF
  \l 9C00
  \l 9C01
  \l 9C02
  \l 9C03
  \l 9C04
  \l 9C05
  \l 9C06
  \l 9C07
  \l 9C08
  \l 9C09
  \l 9C0A
  \l 9C0B
  \l 9C0C
  \l 9C0D
  \l 9C0E
  \l 9C0F
  \l 9C10
  \l 9C11
  \l 9C12
  \l 9C13
  \l 9C14
  \l 9C15
  \l 9C16
  \l 9C17
  \l 9C18
  \l 9C19
  \l 9C1A
  \l 9C1B
  \l 9C1C
  \l 9C1D
  \l 9C1E
  \l 9C1F
  \l 9C20
  \l 9C21
  \l 9C22
  \l 9C23
  \l 9C24
  \l 9C25
  \l 9C26
  \l 9C27
  \l 9C28
  \l 9C29
  \l 9C2A
  \l 9C2B
  \l 9C2C
  \l 9C2D
  \l 9C2E
  \l 9C2F
  \l 9C30
  \l 9C31
  \l 9C32
  \l 9C33
  \l 9C34
  \l 9C35
  \l 9C36
  \l 9C37
  \l 9C38
  \l 9C39
  \l 9C3A
  \l 9C3B
  \l 9C3C
  \l 9C3D
  \l 9C3E
  \l 9C3F
  \l 9C40
  \l 9C41
  \l 9C42
  \l 9C43
  \l 9C44
  \l 9C45
  \l 9C46
  \l 9C47
  \l 9C48
  \l 9C49
  \l 9C4A
  \l 9C4B
  \l 9C4C
  \l 9C4D
  \l 9C4E
  \l 9C4F
  \l 9C50
  \l 9C51
  \l 9C52
  \l 9C53
  \l 9C54
  \l 9C55
  \l 9C56
  \l 9C57
  \l 9C58
  \l 9C59
  \l 9C5A
  \l 9C5B
  \l 9C5C
  \l 9C5D
  \l 9C5E
  \l 9C5F
  \l 9C60
  \l 9C61
  \l 9C62
  \l 9C63
  \l 9C64
  \l 9C65
  \l 9C66
  \l 9C67
  \l 9C68
  \l 9C69
  \l 9C6A
  \l 9C6B
  \l 9C6C
  \l 9C6D
  \l 9C6E
  \l 9C6F
  \l 9C70
  \l 9C71
  \l 9C72
  \l 9C73
  \l 9C74
  \l 9C75
  \l 9C76
  \l 9C77
  \l 9C78
  \l 9C79
  \l 9C7A
  \l 9C7B
  \l 9C7C
  \l 9C7D
  \l 9C7E
  \l 9C7F
  \l 9C80
  \l 9C81
  \l 9C82
  \l 9C83
  \l 9C84
  \l 9C85
  \l 9C86
  \l 9C87
  \l 9C88
  \l 9C89
  \l 9C8A
  \l 9C8B
  \l 9C8C
  \l 9C8D
  \l 9C8E
  \l 9C8F
  \l 9C90
  \l 9C91
  \l 9C92
  \l 9C93
  \l 9C94
  \l 9C95
  \l 9C96
  \l 9C97
  \l 9C98
  \l 9C99
  \l 9C9A
  \l 9C9B
  \l 9C9C
  \l 9C9D
  \l 9C9E
  \l 9C9F
  \l 9CA0
  \l 9CA1
  \l 9CA2
  \l 9CA3
  \l 9CA4
  \l 9CA5
  \l 9CA6
  \l 9CA7
  \l 9CA8
  \l 9CA9
  \l 9CAA
  \l 9CAB
  \l 9CAC
  \l 9CAD
  \l 9CAE
  \l 9CAF
  \l 9CB0
  \l 9CB1
  \l 9CB2
  \l 9CB3
  \l 9CB4
  \l 9CB5
  \l 9CB6
  \l 9CB7
  \l 9CB8
  \l 9CB9
  \l 9CBA
  \l 9CBB
  \l 9CBC
  \l 9CBD
  \l 9CBE
  \l 9CBF
  \l 9CC0
  \l 9CC1
  \l 9CC2
  \l 9CC3
  \l 9CC4
  \l 9CC5
  \l 9CC6
  \l 9CC7
  \l 9CC8
  \l 9CC9
  \l 9CCA
  \l 9CCB
  \l 9CCC
  \l 9CCD
  \l 9CCE
  \l 9CCF
  \l 9CD0
  \l 9CD1
  \l 9CD2
  \l 9CD3
  \l 9CD4
  \l 9CD5
  \l 9CD6
  \l 9CD7
  \l 9CD8
  \l 9CD9
  \l 9CDA
  \l 9CDB
  \l 9CDC
  \l 9CDD
  \l 9CDE
  \l 9CDF
  \l 9CE0
  \l 9CE1
  \l 9CE2
  \l 9CE3
  \l 9CE4
  \l 9CE5
  \l 9CE6
  \l 9CE7
  \l 9CE8
  \l 9CE9
  \l 9CEA
  \l 9CEB
  \l 9CEC
  \l 9CED
  \l 9CEE
  \l 9CEF
  \l 9CF0
  \l 9CF1
  \l 9CF2
  \l 9CF3
  \l 9CF4
  \l 9CF5
  \l 9CF6
  \l 9CF7
  \l 9CF8
  \l 9CF9
  \l 9CFA
  \l 9CFB
  \l 9CFC
  \l 9CFD
  \l 9CFE
  \l 9CFF
  \l 9D00
  \l 9D01
  \l 9D02
  \l 9D03
  \l 9D04
  \l 9D05
  \l 9D06
  \l 9D07
  \l 9D08
  \l 9D09
  \l 9D0A
  \l 9D0B
  \l 9D0C
  \l 9D0D
  \l 9D0E
  \l 9D0F
  \l 9D10
  \l 9D11
  \l 9D12
  \l 9D13
  \l 9D14
  \l 9D15
  \l 9D16
  \l 9D17
  \l 9D18
  \l 9D19
  \l 9D1A
  \l 9D1B
  \l 9D1C
  \l 9D1D
  \l 9D1E
  \l 9D1F
  \l 9D20
  \l 9D21
  \l 9D22
  \l 9D23
  \l 9D24
  \l 9D25
  \l 9D26
  \l 9D27
  \l 9D28
  \l 9D29
  \l 9D2A
  \l 9D2B
  \l 9D2C
  \l 9D2D
  \l 9D2E
  \l 9D2F
  \l 9D30
  \l 9D31
  \l 9D32
  \l 9D33
  \l 9D34
  \l 9D35
  \l 9D36
  \l 9D37
  \l 9D38
  \l 9D39
  \l 9D3A
  \l 9D3B
  \l 9D3C
  \l 9D3D
  \l 9D3E
  \l 9D3F
  \l 9D40
  \l 9D41
  \l 9D42
  \l 9D43
  \l 9D44
  \l 9D45
  \l 9D46
  \l 9D47
  \l 9D48
  \l 9D49
  \l 9D4A
  \l 9D4B
  \l 9D4C
  \l 9D4D
  \l 9D4E
  \l 9D4F
  \l 9D50
  \l 9D51
  \l 9D52
  \l 9D53
  \l 9D54
  \l 9D55
  \l 9D56
  \l 9D57
  \l 9D58
  \l 9D59
  \l 9D5A
  \l 9D5B
  \l 9D5C
  \l 9D5D
  \l 9D5E
  \l 9D5F
  \l 9D60
  \l 9D61
  \l 9D62
  \l 9D63
  \l 9D64
  \l 9D65
  \l 9D66
  \l 9D67
  \l 9D68
  \l 9D69
  \l 9D6A
  \l 9D6B
  \l 9D6C
  \l 9D6D
  \l 9D6E
  \l 9D6F
  \l 9D70
  \l 9D71
  \l 9D72
  \l 9D73
  \l 9D74
  \l 9D75
  \l 9D76
  \l 9D77
  \l 9D78
  \l 9D79
  \l 9D7A
  \l 9D7B
  \l 9D7C
  \l 9D7D
  \l 9D7E
  \l 9D7F
  \l 9D80
  \l 9D81
  \l 9D82
  \l 9D83
  \l 9D84
  \l 9D85
  \l 9D86
  \l 9D87
  \l 9D88
  \l 9D89
  \l 9D8A
  \l 9D8B
  \l 9D8C
  \l 9D8D
  \l 9D8E
  \l 9D8F
  \l 9D90
  \l 9D91
  \l 9D92
  \l 9D93
  \l 9D94
  \l 9D95
  \l 9D96
  \l 9D97
  \l 9D98
  \l 9D99
  \l 9D9A
  \l 9D9B
  \l 9D9C
  \l 9D9D
  \l 9D9E
  \l 9D9F
  \l 9DA0
  \l 9DA1
  \l 9DA2
  \l 9DA3
  \l 9DA4
  \l 9DA5
  \l 9DA6
  \l 9DA7
  \l 9DA8
  \l 9DA9
  \l 9DAA
  \l 9DAB
  \l 9DAC
  \l 9DAD
  \l 9DAE
  \l 9DAF
  \l 9DB0
  \l 9DB1
  \l 9DB2
  \l 9DB3
  \l 9DB4
  \l 9DB5
  \l 9DB6
  \l 9DB7
  \l 9DB8
  \l 9DB9
  \l 9DBA
  \l 9DBB
  \l 9DBC
  \l 9DBD
  \l 9DBE
  \l 9DBF
  \l 9DC0
  \l 9DC1
  \l 9DC2
  \l 9DC3
  \l 9DC4
  \l 9DC5
  \l 9DC6
  \l 9DC7
  \l 9DC8
  \l 9DC9
  \l 9DCA
  \l 9DCB
  \l 9DCC
  \l 9DCD
  \l 9DCE
  \l 9DCF
  \l 9DD0
  \l 9DD1
  \l 9DD2
  \l 9DD3
  \l 9DD4
  \l 9DD5
  \l 9DD6
  \l 9DD7
  \l 9DD8
  \l 9DD9
  \l 9DDA
  \l 9DDB
  \l 9DDC
  \l 9DDD
  \l 9DDE
  \l 9DDF
  \l 9DE0
  \l 9DE1
  \l 9DE2
  \l 9DE3
  \l 9DE4
  \l 9DE5
  \l 9DE6
  \l 9DE7
  \l 9DE8
  \l 9DE9
  \l 9DEA
  \l 9DEB
  \l 9DEC
  \l 9DED
  \l 9DEE
  \l 9DEF
  \l 9DF0
  \l 9DF1
  \l 9DF2
  \l 9DF3
  \l 9DF4
  \l 9DF5
  \l 9DF6
  \l 9DF7
  \l 9DF8
  \l 9DF9
  \l 9DFA
  \l 9DFB
  \l 9DFC
  \l 9DFD
  \l 9DFE
  \l 9DFF
  \l 9E00
  \l 9E01
  \l 9E02
  \l 9E03
  \l 9E04
  \l 9E05
  \l 9E06
  \l 9E07
  \l 9E08
  \l 9E09
  \l 9E0A
  \l 9E0B
  \l 9E0C
  \l 9E0D
  \l 9E0E
  \l 9E0F
  \l 9E10
  \l 9E11
  \l 9E12
  \l 9E13
  \l 9E14
  \l 9E15
  \l 9E16
  \l 9E17
  \l 9E18
  \l 9E19
  \l 9E1A
  \l 9E1B
  \l 9E1C
  \l 9E1D
  \l 9E1E
  \l 9E1F
  \l 9E20
  \l 9E21
  \l 9E22
  \l 9E23
  \l 9E24
  \l 9E25
  \l 9E26
  \l 9E27
  \l 9E28
  \l 9E29
  \l 9E2A
  \l 9E2B
  \l 9E2C
  \l 9E2D
  \l 9E2E
  \l 9E2F
  \l 9E30
  \l 9E31
  \l 9E32
  \l 9E33
  \l 9E34
  \l 9E35
  \l 9E36
  \l 9E37
  \l 9E38
  \l 9E39
  \l 9E3A
  \l 9E3B
  \l 9E3C
  \l 9E3D
  \l 9E3E
  \l 9E3F
  \l 9E40
  \l 9E41
  \l 9E42
  \l 9E43
  \l 9E44
  \l 9E45
  \l 9E46
  \l 9E47
  \l 9E48
  \l 9E49
  \l 9E4A
  \l 9E4B
  \l 9E4C
  \l 9E4D
  \l 9E4E
  \l 9E4F
  \l 9E50
  \l 9E51
  \l 9E52
  \l 9E53
  \l 9E54
  \l 9E55
  \l 9E56
  \l 9E57
  \l 9E58
  \l 9E59
  \l 9E5A
  \l 9E5B
  \l 9E5C
  \l 9E5D
  \l 9E5E
  \l 9E5F
  \l 9E60
  \l 9E61
  \l 9E62
  \l 9E63
  \l 9E64
  \l 9E65
  \l 9E66
  \l 9E67
  \l 9E68
  \l 9E69
  \l 9E6A
  \l 9E6B
  \l 9E6C
  \l 9E6D
  \l 9E6E
  \l 9E6F
  \l 9E70
  \l 9E71
  \l 9E72
  \l 9E73
  \l 9E74
  \l 9E75
  \l 9E76
  \l 9E77
  \l 9E78
  \l 9E79
  \l 9E7A
  \l 9E7B
  \l 9E7C
  \l 9E7D
  \l 9E7E
  \l 9E7F
  \l 9E80
  \l 9E81
  \l 9E82
  \l 9E83
  \l 9E84
  \l 9E85
  \l 9E86
  \l 9E87
  \l 9E88
  \l 9E89
  \l 9E8A
  \l 9E8B
  \l 9E8C
  \l 9E8D
  \l 9E8E
  \l 9E8F
  \l 9E90
  \l 9E91
  \l 9E92
  \l 9E93
  \l 9E94
  \l 9E95
  \l 9E96
  \l 9E97
  \l 9E98
  \l 9E99
  \l 9E9A
  \l 9E9B
  \l 9E9C
  \l 9E9D
  \l 9E9E
  \l 9E9F
  \l 9EA0
  \l 9EA1
  \l 9EA2
  \l 9EA3
  \l 9EA4
  \l 9EA5
  \l 9EA6
  \l 9EA7
  \l 9EA8
  \l 9EA9
  \l 9EAA
  \l 9EAB
  \l 9EAC
  \l 9EAD
  \l 9EAE
  \l 9EAF
  \l 9EB0
  \l 9EB1
  \l 9EB2
  \l 9EB3
  \l 9EB4
  \l 9EB5
  \l 9EB6
  \l 9EB7
  \l 9EB8
  \l 9EB9
  \l 9EBA
  \l 9EBB
  \l 9EBC
  \l 9EBD
  \l 9EBE
  \l 9EBF
  \l 9EC0
  \l 9EC1
  \l 9EC2
  \l 9EC3
  \l 9EC4
  \l 9EC5
  \l 9EC6
  \l 9EC7
  \l 9EC8
  \l 9EC9
  \l 9ECA
  \l 9ECB
  \l 9ECC
  \l 9ECD
  \l 9ECE
  \l 9ECF
  \l 9ED0
  \l 9ED1
  \l 9ED2
  \l 9ED3
  \l 9ED4
  \l 9ED5
  \l 9ED6
  \l 9ED7
  \l 9ED8
  \l 9ED9
  \l 9EDA
  \l 9EDB
  \l 9EDC
  \l 9EDD
  \l 9EDE
  \l 9EDF
  \l 9EE0
  \l 9EE1
  \l 9EE2
  \l 9EE3
  \l 9EE4
  \l 9EE5
  \l 9EE6
  \l 9EE7
  \l 9EE8
  \l 9EE9
  \l 9EEA
  \l 9EEB
  \l 9EEC
  \l 9EED
  \l 9EEE
  \l 9EEF
  \l 9EF0
  \l 9EF1
  \l 9EF2
  \l 9EF3
  \l 9EF4
  \l 9EF5
  \l 9EF6
  \l 9EF7
  \l 9EF8
  \l 9EF9
  \l 9EFA
  \l 9EFB
  \l 9EFC
  \l 9EFD
  \l 9EFE
  \l 9EFF
  \l 9F00
  \l 9F01
  \l 9F02
  \l 9F03
  \l 9F04
  \l 9F05
  \l 9F06
  \l 9F07
  \l 9F08
  \l 9F09
  \l 9F0A
  \l 9F0B
  \l 9F0C
  \l 9F0D
  \l 9F0E
  \l 9F0F
  \l 9F10
  \l 9F11
  \l 9F12
  \l 9F13
  \l 9F14
  \l 9F15
  \l 9F16
  \l 9F17
  \l 9F18
  \l 9F19
  \l 9F1A
  \l 9F1B
  \l 9F1C
  \l 9F1D
  \l 9F1E
  \l 9F1F
  \l 9F20
  \l 9F21
  \l 9F22
  \l 9F23
  \l 9F24
  \l 9F25
  \l 9F26
  \l 9F27
  \l 9F28
  \l 9F29
  \l 9F2A
  \l 9F2B
  \l 9F2C
  \l 9F2D
  \l 9F2E
  \l 9F2F
  \l 9F30
  \l 9F31
  \l 9F32
  \l 9F33
  \l 9F34
  \l 9F35
  \l 9F36
  \l 9F37
  \l 9F38
  \l 9F39
  \l 9F3A
  \l 9F3B
  \l 9F3C
  \l 9F3D
  \l 9F3E
  \l 9F3F
  \l 9F40
  \l 9F41
  \l 9F42
  \l 9F43
  \l 9F44
  \l 9F45
  \l 9F46
  \l 9F47
  \l 9F48
  \l 9F49
  \l 9F4A
  \l 9F4B
  \l 9F4C
  \l 9F4D
  \l 9F4E
  \l 9F4F
  \l 9F50
  \l 9F51
  \l 9F52
  \l 9F53
  \l 9F54
  \l 9F55
  \l 9F56
  \l 9F57
  \l 9F58
  \l 9F59
  \l 9F5A
  \l 9F5B
  \l 9F5C
  \l 9F5D
  \l 9F5E
  \l 9F5F
  \l 9F60
  \l 9F61
  \l 9F62
  \l 9F63
  \l 9F64
  \l 9F65
  \l 9F66
  \l 9F67
  \l 9F68
  \l 9F69
  \l 9F6A
  \l 9F6B
  \l 9F6C
  \l 9F6D
  \l 9F6E
  \l 9F6F
  \l 9F70
  \l 9F71
  \l 9F72
  \l 9F73
  \l 9F74
  \l 9F75
  \l 9F76
  \l 9F77
  \l 9F78
  \l 9F79
  \l 9F7A
  \l 9F7B
  \l 9F7C
  \l 9F7D
  \l 9F7E
  \l 9F7F
  \l 9F80
  \l 9F81
  \l 9F82
  \l 9F83
  \l 9F84
  \l 9F85
  \l 9F86
  \l 9F87
  \l 9F88
  \l 9F89
  \l 9F8A
  \l 9F8B
  \l 9F8C
  \l 9F8D
  \l 9F8E
  \l 9F8F
  \l 9F90
  \l 9F91
  \l 9F92
  \l 9F93
  \l 9F94
  \l 9F95
  \l 9F96
  \l 9F97
  \l 9F98
  \l 9F99
  \l 9F9A
  \l 9F9B
  \l 9F9C
  \l 9F9D
  \l 9F9E
  \l 9F9F
  \l 9FA0
  \l 9FA1
  \l 9FA2
  \l 9FA3
  \l 9FA4
  \l 9FA5
  \l 9FA6
  \l 9FA7
  \l 9FA8
  \l 9FA9
  \l 9FAA
  \l 9FAB
  \l 9FAC
  \l 9FAD
  \l 9FAE
  \l 9FAF
  \l 9FB0
  \l 9FB1
  \l 9FB2
  \l 9FB3
  \l 9FB4
  \l 9FB5
  \l 9FB6
  \l 9FB7
  \l 9FB8
  \l 9FB9
  \l 9FBA
  \l 9FBB
  \l 9FBC
  \l 9FBD
  \l 9FBE
  \l 9FBF
  \l 9FC0
  \l 9FC1
  \l 9FC2
  \l 9FC3
  \l 9FC4
  \l 9FC5
  \l 9FC6
  \l 9FC7
  \l 9FC8
  \l 9FC9
  \l 9FCA
  \l 9FCB
  \l 9FCC
  \l A000
  \l A001
  \l A002
  \l A003
  \l A004
  \l A005
  \l A006
  \l A007
  \l A008
  \l A009
  \l A00A
  \l A00B
  \l A00C
  \l A00D
  \l A00E
  \l A00F
  \l A010
  \l A011
  \l A012
  \l A013
  \l A014
  \l A015
  \l A016
  \l A017
  \l A018
  \l A019
  \l A01A
  \l A01B
  \l A01C
  \l A01D
  \l A01E
  \l A01F
  \l A020
  \l A021
  \l A022
  \l A023
  \l A024
  \l A025
  \l A026
  \l A027
  \l A028
  \l A029
  \l A02A
  \l A02B
  \l A02C
  \l A02D
  \l A02E
  \l A02F
  \l A030
  \l A031
  \l A032
  \l A033
  \l A034
  \l A035
  \l A036
  \l A037
  \l A038
  \l A039
  \l A03A
  \l A03B
  \l A03C
  \l A03D
  \l A03E
  \l A03F
  \l A040
  \l A041
  \l A042
  \l A043
  \l A044
  \l A045
  \l A046
  \l A047
  \l A048
  \l A049
  \l A04A
  \l A04B
  \l A04C
  \l A04D
  \l A04E
  \l A04F
  \l A050
  \l A051
  \l A052
  \l A053
  \l A054
  \l A055
  \l A056
  \l A057
  \l A058
  \l A059
  \l A05A
  \l A05B
  \l A05C
  \l A05D
  \l A05E
  \l A05F
  \l A060
  \l A061
  \l A062
  \l A063
  \l A064
  \l A065
  \l A066
  \l A067
  \l A068
  \l A069
  \l A06A
  \l A06B
  \l A06C
  \l A06D
  \l A06E
  \l A06F
  \l A070
  \l A071
  \l A072
  \l A073
  \l A074
  \l A075
  \l A076
  \l A077
  \l A078
  \l A079
  \l A07A
  \l A07B
  \l A07C
  \l A07D
  \l A07E
  \l A07F
  \l A080
  \l A081
  \l A082
  \l A083
  \l A084
  \l A085
  \l A086
  \l A087
  \l A088
  \l A089
  \l A08A
  \l A08B
  \l A08C
  \l A08D
  \l A08E
  \l A08F
  \l A090
  \l A091
  \l A092
  \l A093
  \l A094
  \l A095
  \l A096
  \l A097
  \l A098
  \l A099
  \l A09A
  \l A09B
  \l A09C
  \l A09D
  \l A09E
  \l A09F
  \l A0A0
  \l A0A1
  \l A0A2
  \l A0A3
  \l A0A4
  \l A0A5
  \l A0A6
  \l A0A7
  \l A0A8
  \l A0A9
  \l A0AA
  \l A0AB
  \l A0AC
  \l A0AD
  \l A0AE
  \l A0AF
  \l A0B0
  \l A0B1
  \l A0B2
  \l A0B3
  \l A0B4
  \l A0B5
  \l A0B6
  \l A0B7
  \l A0B8
  \l A0B9
  \l A0BA
  \l A0BB
  \l A0BC
  \l A0BD
  \l A0BE
  \l A0BF
  \l A0C0
  \l A0C1
  \l A0C2
  \l A0C3
  \l A0C4
  \l A0C5
  \l A0C6
  \l A0C7
  \l A0C8
  \l A0C9
  \l A0CA
  \l A0CB
  \l A0CC
  \l A0CD
  \l A0CE
  \l A0CF
  \l A0D0
  \l A0D1
  \l A0D2
  \l A0D3
  \l A0D4
  \l A0D5
  \l A0D6
  \l A0D7
  \l A0D8
  \l A0D9
  \l A0DA
  \l A0DB
  \l A0DC
  \l A0DD
  \l A0DE
  \l A0DF
  \l A0E0
  \l A0E1
  \l A0E2
  \l A0E3
  \l A0E4
  \l A0E5
  \l A0E6
  \l A0E7
  \l A0E8
  \l A0E9
  \l A0EA
  \l A0EB
  \l A0EC
  \l A0ED
  \l A0EE
  \l A0EF
  \l A0F0
  \l A0F1
  \l A0F2
  \l A0F3
  \l A0F4
  \l A0F5
  \l A0F6
  \l A0F7
  \l A0F8
  \l A0F9
  \l A0FA
  \l A0FB
  \l A0FC
  \l A0FD
  \l A0FE
  \l A0FF
  \l A100
  \l A101
  \l A102
  \l A103
  \l A104
  \l A105
  \l A106
  \l A107
  \l A108
  \l A109
  \l A10A
  \l A10B
  \l A10C
  \l A10D
  \l A10E
  \l A10F
  \l A110
  \l A111
  \l A112
  \l A113
  \l A114
  \l A115
  \l A116
  \l A117
  \l A118
  \l A119
  \l A11A
  \l A11B
  \l A11C
  \l A11D
  \l A11E
  \l A11F
  \l A120
  \l A121
  \l A122
  \l A123
  \l A124
  \l A125
  \l A126
  \l A127
  \l A128
  \l A129
  \l A12A
  \l A12B
  \l A12C
  \l A12D
  \l A12E
  \l A12F
  \l A130
  \l A131
  \l A132
  \l A133
  \l A134
  \l A135
  \l A136
  \l A137
  \l A138
  \l A139
  \l A13A
  \l A13B
  \l A13C
  \l A13D
  \l A13E
  \l A13F
  \l A140
  \l A141
  \l A142
  \l A143
  \l A144
  \l A145
  \l A146
  \l A147
  \l A148
  \l A149
  \l A14A
  \l A14B
  \l A14C
  \l A14D
  \l A14E
  \l A14F
  \l A150
  \l A151
  \l A152
  \l A153
  \l A154
  \l A155
  \l A156
  \l A157
  \l A158
  \l A159
  \l A15A
  \l A15B
  \l A15C
  \l A15D
  \l A15E
  \l A15F
  \l A160
  \l A161
  \l A162
  \l A163
  \l A164
  \l A165
  \l A166
  \l A167
  \l A168
  \l A169
  \l A16A
  \l A16B
  \l A16C
  \l A16D
  \l A16E
  \l A16F
  \l A170
  \l A171
  \l A172
  \l A173
  \l A174
  \l A175
  \l A176
  \l A177
  \l A178
  \l A179
  \l A17A
  \l A17B
  \l A17C
  \l A17D
  \l A17E
  \l A17F
  \l A180
  \l A181
  \l A182
  \l A183
  \l A184
  \l A185
  \l A186
  \l A187
  \l A188
  \l A189
  \l A18A
  \l A18B
  \l A18C
  \l A18D
  \l A18E
  \l A18F
  \l A190
  \l A191
  \l A192
  \l A193
  \l A194
  \l A195
  \l A196
  \l A197
  \l A198
  \l A199
  \l A19A
  \l A19B
  \l A19C
  \l A19D
  \l A19E
  \l A19F
  \l A1A0
  \l A1A1
  \l A1A2
  \l A1A3
  \l A1A4
  \l A1A5
  \l A1A6
  \l A1A7
  \l A1A8
  \l A1A9
  \l A1AA
  \l A1AB
  \l A1AC
  \l A1AD
  \l A1AE
  \l A1AF
  \l A1B0
  \l A1B1
  \l A1B2
  \l A1B3
  \l A1B4
  \l A1B5
  \l A1B6
  \l A1B7
  \l A1B8
  \l A1B9
  \l A1BA
  \l A1BB
  \l A1BC
  \l A1BD
  \l A1BE
  \l A1BF
  \l A1C0
  \l A1C1
  \l A1C2
  \l A1C3
  \l A1C4
  \l A1C5
  \l A1C6
  \l A1C7
  \l A1C8
  \l A1C9
  \l A1CA
  \l A1CB
  \l A1CC
  \l A1CD
  \l A1CE
  \l A1CF
  \l A1D0
  \l A1D1
  \l A1D2
  \l A1D3
  \l A1D4
  \l A1D5
  \l A1D6
  \l A1D7
  \l A1D8
  \l A1D9
  \l A1DA
  \l A1DB
  \l A1DC
  \l A1DD
  \l A1DE
  \l A1DF
  \l A1E0
  \l A1E1
  \l A1E2
  \l A1E3
  \l A1E4
  \l A1E5
  \l A1E6
  \l A1E7
  \l A1E8
  \l A1E9
  \l A1EA
  \l A1EB
  \l A1EC
  \l A1ED
  \l A1EE
  \l A1EF
  \l A1F0
  \l A1F1
  \l A1F2
  \l A1F3
  \l A1F4
  \l A1F5
  \l A1F6
  \l A1F7
  \l A1F8
  \l A1F9
  \l A1FA
  \l A1FB
  \l A1FC
  \l A1FD
  \l A1FE
  \l A1FF
  \l A200
  \l A201
  \l A202
  \l A203
  \l A204
  \l A205
  \l A206
  \l A207
  \l A208
  \l A209
  \l A20A
  \l A20B
  \l A20C
  \l A20D
  \l A20E
  \l A20F
  \l A210
  \l A211
  \l A212
  \l A213
  \l A214
  \l A215
  \l A216
  \l A217
  \l A218
  \l A219
  \l A21A
  \l A21B
  \l A21C
  \l A21D
  \l A21E
  \l A21F
  \l A220
  \l A221
  \l A222
  \l A223
  \l A224
  \l A225
  \l A226
  \l A227
  \l A228
  \l A229
  \l A22A
  \l A22B
  \l A22C
  \l A22D
  \l A22E
  \l A22F
  \l A230
  \l A231
  \l A232
  \l A233
  \l A234
  \l A235
  \l A236
  \l A237
  \l A238
  \l A239
  \l A23A
  \l A23B
  \l A23C
  \l A23D
  \l A23E
  \l A23F
  \l A240
  \l A241
  \l A242
  \l A243
  \l A244
  \l A245
  \l A246
  \l A247
  \l A248
  \l A249
  \l A24A
  \l A24B
  \l A24C
  \l A24D
  \l A24E
  \l A24F
  \l A250
  \l A251
  \l A252
  \l A253
  \l A254
  \l A255
  \l A256
  \l A257
  \l A258
  \l A259
  \l A25A
  \l A25B
  \l A25C
  \l A25D
  \l A25E
  \l A25F
  \l A260
  \l A261
  \l A262
  \l A263
  \l A264
  \l A265
  \l A266
  \l A267
  \l A268
  \l A269
  \l A26A
  \l A26B
  \l A26C
  \l A26D
  \l A26E
  \l A26F
  \l A270
  \l A271
  \l A272
  \l A273
  \l A274
  \l A275
  \l A276
  \l A277
  \l A278
  \l A279
  \l A27A
  \l A27B
  \l A27C
  \l A27D
  \l A27E
  \l A27F
  \l A280
  \l A281
  \l A282
  \l A283
  \l A284
  \l A285
  \l A286
  \l A287
  \l A288
  \l A289
  \l A28A
  \l A28B
  \l A28C
  \l A28D
  \l A28E
  \l A28F
  \l A290
  \l A291
  \l A292
  \l A293
  \l A294
  \l A295
  \l A296
  \l A297
  \l A298
  \l A299
  \l A29A
  \l A29B
  \l A29C
  \l A29D
  \l A29E
  \l A29F
  \l A2A0
  \l A2A1
  \l A2A2
  \l A2A3
  \l A2A4
  \l A2A5
  \l A2A6
  \l A2A7
  \l A2A8
  \l A2A9
  \l A2AA
  \l A2AB
  \l A2AC
  \l A2AD
  \l A2AE
  \l A2AF
  \l A2B0
  \l A2B1
  \l A2B2
  \l A2B3
  \l A2B4
  \l A2B5
  \l A2B6
  \l A2B7
  \l A2B8
  \l A2B9
  \l A2BA
  \l A2BB
  \l A2BC
  \l A2BD
  \l A2BE
  \l A2BF
  \l A2C0
  \l A2C1
  \l A2C2
  \l A2C3
  \l A2C4
  \l A2C5
  \l A2C6
  \l A2C7
  \l A2C8
  \l A2C9
  \l A2CA
  \l A2CB
  \l A2CC
  \l A2CD
  \l A2CE
  \l A2CF
  \l A2D0
  \l A2D1
  \l A2D2
  \l A2D3
  \l A2D4
  \l A2D5
  \l A2D6
  \l A2D7
  \l A2D8
  \l A2D9
  \l A2DA
  \l A2DB
  \l A2DC
  \l A2DD
  \l A2DE
  \l A2DF
  \l A2E0
  \l A2E1
  \l A2E2
  \l A2E3
  \l A2E4
  \l A2E5
  \l A2E6
  \l A2E7
  \l A2E8
  \l A2E9
  \l A2EA
  \l A2EB
  \l A2EC
  \l A2ED
  \l A2EE
  \l A2EF
  \l A2F0
  \l A2F1
  \l A2F2
  \l A2F3
  \l A2F4
  \l A2F5
  \l A2F6
  \l A2F7
  \l A2F8
  \l A2F9
  \l A2FA
  \l A2FB
  \l A2FC
  \l A2FD
  \l A2FE
  \l A2FF
  \l A300
  \l A301
  \l A302
  \l A303
  \l A304
  \l A305
  \l A306
  \l A307
  \l A308
  \l A309
  \l A30A
  \l A30B
  \l A30C
  \l A30D
  \l A30E
  \l A30F
  \l A310
  \l A311
  \l A312
  \l A313
  \l A314
  \l A315
  \l A316
  \l A317
  \l A318
  \l A319
  \l A31A
  \l A31B
  \l A31C
  \l A31D
  \l A31E
  \l A31F
  \l A320
  \l A321
  \l A322
  \l A323
  \l A324
  \l A325
  \l A326
  \l A327
  \l A328
  \l A329
  \l A32A
  \l A32B
  \l A32C
  \l A32D
  \l A32E
  \l A32F
  \l A330
  \l A331
  \l A332
  \l A333
  \l A334
  \l A335
  \l A336
  \l A337
  \l A338
  \l A339
  \l A33A
  \l A33B
  \l A33C
  \l A33D
  \l A33E
  \l A33F
  \l A340
  \l A341
  \l A342
  \l A343
  \l A344
  \l A345
  \l A346
  \l A347
  \l A348
  \l A349
  \l A34A
  \l A34B
  \l A34C
  \l A34D
  \l A34E
  \l A34F
  \l A350
  \l A351
  \l A352
  \l A353
  \l A354
  \l A355
  \l A356
  \l A357
  \l A358
  \l A359
  \l A35A
  \l A35B
  \l A35C
  \l A35D
  \l A35E
  \l A35F
  \l A360
  \l A361
  \l A362
  \l A363
  \l A364
  \l A365
  \l A366
  \l A367
  \l A368
  \l A369
  \l A36A
  \l A36B
  \l A36C
  \l A36D
  \l A36E
  \l A36F
  \l A370
  \l A371
  \l A372
  \l A373
  \l A374
  \l A375
  \l A376
  \l A377
  \l A378
  \l A379
  \l A37A
  \l A37B
  \l A37C
  \l A37D
  \l A37E
  \l A37F
  \l A380
  \l A381
  \l A382
  \l A383
  \l A384
  \l A385
  \l A386
  \l A387
  \l A388
  \l A389
  \l A38A
  \l A38B
  \l A38C
  \l A38D
  \l A38E
  \l A38F
  \l A390
  \l A391
  \l A392
  \l A393
  \l A394
  \l A395
  \l A396
  \l A397
  \l A398
  \l A399
  \l A39A
  \l A39B
  \l A39C
  \l A39D
  \l A39E
  \l A39F
  \l A3A0
  \l A3A1
  \l A3A2
  \l A3A3
  \l A3A4
  \l A3A5
  \l A3A6
  \l A3A7
  \l A3A8
  \l A3A9
  \l A3AA
  \l A3AB
  \l A3AC
  \l A3AD
  \l A3AE
  \l A3AF
  \l A3B0
  \l A3B1
  \l A3B2
  \l A3B3
  \l A3B4
  \l A3B5
  \l A3B6
  \l A3B7
  \l A3B8
  \l A3B9
  \l A3BA
  \l A3BB
  \l A3BC
  \l A3BD
  \l A3BE
  \l A3BF
  \l A3C0
  \l A3C1
  \l A3C2
  \l A3C3
  \l A3C4
  \l A3C5
  \l A3C6
  \l A3C7
  \l A3C8
  \l A3C9
  \l A3CA
  \l A3CB
  \l A3CC
  \l A3CD
  \l A3CE
  \l A3CF
  \l A3D0
  \l A3D1
  \l A3D2
  \l A3D3
  \l A3D4
  \l A3D5
  \l A3D6
  \l A3D7
  \l A3D8
  \l A3D9
  \l A3DA
  \l A3DB
  \l A3DC
  \l A3DD
  \l A3DE
  \l A3DF
  \l A3E0
  \l A3E1
  \l A3E2
  \l A3E3
  \l A3E4
  \l A3E5
  \l A3E6
  \l A3E7
  \l A3E8
  \l A3E9
  \l A3EA
  \l A3EB
  \l A3EC
  \l A3ED
  \l A3EE
  \l A3EF
  \l A3F0
  \l A3F1
  \l A3F2
  \l A3F3
  \l A3F4
  \l A3F5
  \l A3F6
  \l A3F7
  \l A3F8
  \l A3F9
  \l A3FA
  \l A3FB
  \l A3FC
  \l A3FD
  \l A3FE
  \l A3FF
  \l A400
  \l A401
  \l A402
  \l A403
  \l A404
  \l A405
  \l A406
  \l A407
  \l A408
  \l A409
  \l A40A
  \l A40B
  \l A40C
  \l A40D
  \l A40E
  \l A40F
  \l A410
  \l A411
  \l A412
  \l A413
  \l A414
  \l A415
  \l A416
  \l A417
  \l A418
  \l A419
  \l A41A
  \l A41B
  \l A41C
  \l A41D
  \l A41E
  \l A41F
  \l A420
  \l A421
  \l A422
  \l A423
  \l A424
  \l A425
  \l A426
  \l A427
  \l A428
  \l A429
  \l A42A
  \l A42B
  \l A42C
  \l A42D
  \l A42E
  \l A42F
  \l A430
  \l A431
  \l A432
  \l A433
  \l A434
  \l A435
  \l A436
  \l A437
  \l A438
  \l A439
  \l A43A
  \l A43B
  \l A43C
  \l A43D
  \l A43E
  \l A43F
  \l A440
  \l A441
  \l A442
  \l A443
  \l A444
  \l A445
  \l A446
  \l A447
  \l A448
  \l A449
  \l A44A
  \l A44B
  \l A44C
  \l A44D
  \l A44E
  \l A44F
  \l A450
  \l A451
  \l A452
  \l A453
  \l A454
  \l A455
  \l A456
  \l A457
  \l A458
  \l A459
  \l A45A
  \l A45B
  \l A45C
  \l A45D
  \l A45E
  \l A45F
  \l A460
  \l A461
  \l A462
  \l A463
  \l A464
  \l A465
  \l A466
  \l A467
  \l A468
  \l A469
  \l A46A
  \l A46B
  \l A46C
  \l A46D
  \l A46E
  \l A46F
  \l A470
  \l A471
  \l A472
  \l A473
  \l A474
  \l A475
  \l A476
  \l A477
  \l A478
  \l A479
  \l A47A
  \l A47B
  \l A47C
  \l A47D
  \l A47E
  \l A47F
  \l A480
  \l A481
  \l A482
  \l A483
  \l A484
  \l A485
  \l A486
  \l A487
  \l A488
  \l A489
  \l A48A
  \l A48B
  \l A48C
  \l A4D0
  \l A4D1
  \l A4D2
  \l A4D3
  \l A4D4
  \l A4D5
  \l A4D6
  \l A4D7
  \l A4D8
  \l A4D9
  \l A4DA
  \l A4DB
  \l A4DC
  \l A4DD
  \l A4DE
  \l A4DF
  \l A4E0
  \l A4E1
  \l A4E2
  \l A4E3
  \l A4E4
  \l A4E5
  \l A4E6
  \l A4E7
  \l A4E8
  \l A4E9
  \l A4EA
  \l A4EB
  \l A4EC
  \l A4ED
  \l A4EE
  \l A4EF
  \l A4F0
  \l A4F1
  \l A4F2
  \l A4F3
  \l A4F4
  \l A4F5
  \l A4F6
  \l A4F7
  \l A4F8
  \l A4F9
  \l A4FA
  \l A4FB
  \l A4FC
  \l A4FD
  \l A500
  \l A501
  \l A502
  \l A503
  \l A504
  \l A505
  \l A506
  \l A507
  \l A508
  \l A509
  \l A50A
  \l A50B
  \l A50C
  \l A50D
  \l A50E
  \l A50F
  \l A510
  \l A511
  \l A512
  \l A513
  \l A514
  \l A515
  \l A516
  \l A517
  \l A518
  \l A519
  \l A51A
  \l A51B
  \l A51C
  \l A51D
  \l A51E
  \l A51F
  \l A520
  \l A521
  \l A522
  \l A523
  \l A524
  \l A525
  \l A526
  \l A527
  \l A528
  \l A529
  \l A52A
  \l A52B
  \l A52C
  \l A52D
  \l A52E
  \l A52F
  \l A530
  \l A531
  \l A532
  \l A533
  \l A534
  \l A535
  \l A536
  \l A537
  \l A538
  \l A539
  \l A53A
  \l A53B
  \l A53C
  \l A53D
  \l A53E
  \l A53F
  \l A540
  \l A541
  \l A542
  \l A543
  \l A544
  \l A545
  \l A546
  \l A547
  \l A548
  \l A549
  \l A54A
  \l A54B
  \l A54C
  \l A54D
  \l A54E
  \l A54F
  \l A550
  \l A551
  \l A552
  \l A553
  \l A554
  \l A555
  \l A556
  \l A557
  \l A558
  \l A559
  \l A55A
  \l A55B
  \l A55C
  \l A55D
  \l A55E
  \l A55F
  \l A560
  \l A561
  \l A562
  \l A563
  \l A564
  \l A565
  \l A566
  \l A567
  \l A568
  \l A569
  \l A56A
  \l A56B
  \l A56C
  \l A56D
  \l A56E
  \l A56F
  \l A570
  \l A571
  \l A572
  \l A573
  \l A574
  \l A575
  \l A576
  \l A577
  \l A578
  \l A579
  \l A57A
  \l A57B
  \l A57C
  \l A57D
  \l A57E
  \l A57F
  \l A580
  \l A581
  \l A582
  \l A583
  \l A584
  \l A585
  \l A586
  \l A587
  \l A588
  \l A589
  \l A58A
  \l A58B
  \l A58C
  \l A58D
  \l A58E
  \l A58F
  \l A590
  \l A591
  \l A592
  \l A593
  \l A594
  \l A595
  \l A596
  \l A597
  \l A598
  \l A599
  \l A59A
  \l A59B
  \l A59C
  \l A59D
  \l A59E
  \l A59F
  \l A5A0
  \l A5A1
  \l A5A2
  \l A5A3
  \l A5A4
  \l A5A5
  \l A5A6
  \l A5A7
  \l A5A8
  \l A5A9
  \l A5AA
  \l A5AB
  \l A5AC
  \l A5AD
  \l A5AE
  \l A5AF
  \l A5B0
  \l A5B1
  \l A5B2
  \l A5B3
  \l A5B4
  \l A5B5
  \l A5B6
  \l A5B7
  \l A5B8
  \l A5B9
  \l A5BA
  \l A5BB
  \l A5BC
  \l A5BD
  \l A5BE
  \l A5BF
  \l A5C0
  \l A5C1
  \l A5C2
  \l A5C3
  \l A5C4
  \l A5C5
  \l A5C6
  \l A5C7
  \l A5C8
  \l A5C9
  \l A5CA
  \l A5CB
  \l A5CC
  \l A5CD
  \l A5CE
  \l A5CF
  \l A5D0
  \l A5D1
  \l A5D2
  \l A5D3
  \l A5D4
  \l A5D5
  \l A5D6
  \l A5D7
  \l A5D8
  \l A5D9
  \l A5DA
  \l A5DB
  \l A5DC
  \l A5DD
  \l A5DE
  \l A5DF
  \l A5E0
  \l A5E1
  \l A5E2
  \l A5E3
  \l A5E4
  \l A5E5
  \l A5E6
  \l A5E7
  \l A5E8
  \l A5E9
  \l A5EA
  \l A5EB
  \l A5EC
  \l A5ED
  \l A5EE
  \l A5EF
  \l A5F0
  \l A5F1
  \l A5F2
  \l A5F3
  \l A5F4
  \l A5F5
  \l A5F6
  \l A5F7
  \l A5F8
  \l A5F9
  \l A5FA
  \l A5FB
  \l A5FC
  \l A5FD
  \l A5FE
  \l A5FF
  \l A600
  \l A601
  \l A602
  \l A603
  \l A604
  \l A605
  \l A606
  \l A607
  \l A608
  \l A609
  \l A60A
  \l A60B
  \l A60C
  \l A610
  \l A611
  \l A612
  \l A613
  \l A614
  \l A615
  \l A616
  \l A617
  \l A618
  \l A619
  \l A61A
  \l A61B
  \l A61C
  \l A61D
  \l A61E
  \l A61F
  \l A62A
  \l A62B
  \L A640 A640 A641
  \L A641 A640 A641
  \L A642 A642 A643
  \L A643 A642 A643
  \L A644 A644 A645
  \L A645 A644 A645
  \L A646 A646 A647
  \L A647 A646 A647
  \L A648 A648 A649
  \L A649 A648 A649
  \L A64A A64A A64B
  \L A64B A64A A64B
  \L A64C A64C A64D
  \L A64D A64C A64D
  \L A64E A64E A64F
  \L A64F A64E A64F
  \L A650 A650 A651
  \L A651 A650 A651
  \L A652 A652 A653
  \L A653 A652 A653
  \L A654 A654 A655
  \L A655 A654 A655
  \L A656 A656 A657
  \L A657 A656 A657
  \L A658 A658 A659
  \L A659 A658 A659
  \L A65A A65A A65B
  \L A65B A65A A65B
  \L A65C A65C A65D
  \L A65D A65C A65D
  \L A65E A65E A65F
  \L A65F A65E A65F
  \L A660 A660 A661
  \L A661 A660 A661
  \L A662 A662 A663
  \L A663 A662 A663
  \L A664 A664 A665
  \L A665 A664 A665
  \L A666 A666 A667
  \L A667 A666 A667
  \L A668 A668 A669
  \L A669 A668 A669
  \L A66A A66A A66B
  \L A66B A66A A66B
  \L A66C A66C A66D
  \L A66D A66C A66D
  \l A66E
  \l A66F
  \l A670
  \l A671
  \l A672
  \l A674
  \l A675
  \l A676
  \l A677
  \l A678
  \l A679
  \l A67A
  \l A67B
  \l A67C
  \l A67D
  \l A67F
  \L A680 A680 A681
  \L A681 A680 A681
  \L A682 A682 A683
  \L A683 A682 A683
  \L A684 A684 A685
  \L A685 A684 A685
  \L A686 A686 A687
  \L A687 A686 A687
  \L A688 A688 A689
  \L A689 A688 A689
  \L A68A A68A A68B
  \L A68B A68A A68B
  \L A68C A68C A68D
  \L A68D A68C A68D
  \L A68E A68E A68F
  \L A68F A68E A68F
  \L A690 A690 A691
  \L A691 A690 A691
  \L A692 A692 A693
  \L A693 A692 A693
  \L A694 A694 A695
  \L A695 A694 A695
  \L A696 A696 A697
  \L A697 A696 A697
  \L A698 A698 A699
  \L A699 A698 A699
  \L A69A A69A A69B
  \L A69B A69A A69B
  \l A69C
  \l A69D
  \l A69F
  \l A6A0
  \l A6A1
  \l A6A2
  \l A6A3
  \l A6A4
  \l A6A5
  \l A6A6
  \l A6A7
  \l A6A8
  \l A6A9
  \l A6AA
  \l A6AB
  \l A6AC
  \l A6AD
  \l A6AE
  \l A6AF
  \l A6B0
  \l A6B1
  \l A6B2
  \l A6B3
  \l A6B4
  \l A6B5
  \l A6B6
  \l A6B7
  \l A6B8
  \l A6B9
  \l A6BA
  \l A6BB
  \l A6BC
  \l A6BD
  \l A6BE
  \l A6BF
  \l A6C0
  \l A6C1
  \l A6C2
  \l A6C3
  \l A6C4
  \l A6C5
  \l A6C6
  \l A6C7
  \l A6C8
  \l A6C9
  \l A6CA
  \l A6CB
  \l A6CC
  \l A6CD
  \l A6CE
  \l A6CF
  \l A6D0
  \l A6D1
  \l A6D2
  \l A6D3
  \l A6D4
  \l A6D5
  \l A6D6
  \l A6D7
  \l A6D8
  \l A6D9
  \l A6DA
  \l A6DB
  \l A6DC
  \l A6DD
  \l A6DE
  \l A6DF
  \l A6E0
  \l A6E1
  \l A6E2
  \l A6E3
  \l A6E4
  \l A6E5
  \l A6F0
  \l A6F1
  \l A717
  \l A718
  \l A719
  \l A71A
  \l A71B
  \l A71C
  \l A71D
  \l A71E
  \l A71F
  \L A722 A722 A723
  \L A723 A722 A723
  \L A724 A724 A725
  \L A725 A724 A725
  \L A726 A726 A727
  \L A727 A726 A727
  \L A728 A728 A729
  \L A729 A728 A729
  \L A72A A72A A72B
  \L A72B A72A A72B
  \L A72C A72C A72D
  \L A72D A72C A72D
  \L A72E A72E A72F
  \L A72F A72E A72F
  \l A730
  \l A731
  \L A732 A732 A733
  \L A733 A732 A733
  \L A734 A734 A735
  \L A735 A734 A735
  \L A736 A736 A737
  \L A737 A736 A737
  \L A738 A738 A739
  \L A739 A738 A739
  \L A73A A73A A73B
  \L A73B A73A A73B
  \L A73C A73C A73D
  \L A73D A73C A73D
  \L A73E A73E A73F
  \L A73F A73E A73F
  \L A740 A740 A741
  \L A741 A740 A741
  \L A742 A742 A743
  \L A743 A742 A743
  \L A744 A744 A745
  \L A745 A744 A745
  \L A746 A746 A747
  \L A747 A746 A747
  \L A748 A748 A749
  \L A749 A748 A749
  \L A74A A74A A74B
  \L A74B A74A A74B
  \L A74C A74C A74D
  \L A74D A74C A74D
  \L A74E A74E A74F
  \L A74F A74E A74F
  \L A750 A750 A751
  \L A751 A750 A751
  \L A752 A752 A753
  \L A753 A752 A753
  \L A754 A754 A755
  \L A755 A754 A755
  \L A756 A756 A757
  \L A757 A756 A757
  \L A758 A758 A759
  \L A759 A758 A759
  \L A75A A75A A75B
  \L A75B A75A A75B
  \L A75C A75C A75D
  \L A75D A75C A75D
  \L A75E A75E A75F
  \L A75F A75E A75F
  \L A760 A760 A761
  \L A761 A760 A761
  \L A762 A762 A763
  \L A763 A762 A763
  \L A764 A764 A765
  \L A765 A764 A765
  \L A766 A766 A767
  \L A767 A766 A767
  \L A768 A768 A769
  \L A769 A768 A769
  \L A76A A76A A76B
  \L A76B A76A A76B
  \L A76C A76C A76D
  \L A76D A76C A76D
  \L A76E A76E A76F
  \L A76F A76E A76F
  \l A770
  \l A771
  \l A772
  \l A773
  \l A774
  \l A775
  \l A776
  \l A777
  \l A778
  \L A779 A779 A77A
  \L A77A A779 A77A
  \L A77B A77B A77C
  \L A77C A77B A77C
  \L A77D A77D 1D79
  \L A77E A77E A77F
  \L A77F A77E A77F
  \L A780 A780 A781
  \L A781 A780 A781
  \L A782 A782 A783
  \L A783 A782 A783
  \L A784 A784 A785
  \L A785 A784 A785
  \L A786 A786 A787
  \L A787 A786 A787
  \l A788
  \L A78B A78B A78C
  \L A78C A78B A78C
  \L A78D A78D 0265
  \l A78E
  \L A790 A790 A791
  \L A791 A790 A791
  \L A792 A792 A793
  \L A793 A792 A793
  \l A794
  \l A795
  \L A796 A796 A797
  \L A797 A796 A797
  \L A798 A798 A799
  \L A799 A798 A799
  \L A79A A79A A79B
  \L A79B A79A A79B
  \L A79C A79C A79D
  \L A79D A79C A79D
  \L A79E A79E A79F
  \L A79F A79E A79F
  \L A7A0 A7A0 A7A1
  \L A7A1 A7A0 A7A1
  \L A7A2 A7A2 A7A3
  \L A7A3 A7A2 A7A3
  \L A7A4 A7A4 A7A5
  \L A7A5 A7A4 A7A5
  \L A7A6 A7A6 A7A7
  \L A7A7 A7A6 A7A7
  \L A7A8 A7A8 A7A9
  \L A7A9 A7A8 A7A9
  \L A7AA A7AA 0266
  \L A7AB A7AB 025C
  \L A7AC A7AC 0261
  \L A7AD A7AD 026C
  \L A7B0 A7B0 029E
  \L A7B1 A7B1 0287
  \l A7F7
  \l A7F8
  \l A7F9
  \l A7FA
  \l A7FB
  \l A7FC
  \l A7FD
  \l A7FE
  \l A7FF
  \l A800
  \l A801
  \l A802
  \l A803
  \l A804
  \l A805
  \l A806
  \l A807
  \l A808
  \l A809
  \l A80A
  \l A80B
  \l A80C
  \l A80D
  \l A80E
  \l A80F
  \l A810
  \l A811
  \l A812
  \l A813
  \l A814
  \l A815
  \l A816
  \l A817
  \l A818
  \l A819
  \l A81A
  \l A81B
  \l A81C
  \l A81D
  \l A81E
  \l A81F
  \l A820
  \l A821
  \l A822
  \l A823
  \l A824
  \l A825
  \l A826
  \l A827
  \l A840
  \l A841
  \l A842
  \l A843
  \l A844
  \l A845
  \l A846
  \l A847
  \l A848
  \l A849
  \l A84A
  \l A84B
  \l A84C
  \l A84D
  \l A84E
  \l A84F
  \l A850
  \l A851
  \l A852
  \l A853
  \l A854
  \l A855
  \l A856
  \l A857
  \l A858
  \l A859
  \l A85A
  \l A85B
  \l A85C
  \l A85D
  \l A85E
  \l A85F
  \l A860
  \l A861
  \l A862
  \l A863
  \l A864
  \l A865
  \l A866
  \l A867
  \l A868
  \l A869
  \l A86A
  \l A86B
  \l A86C
  \l A86D
  \l A86E
  \l A86F
  \l A870
  \l A871
  \l A872
  \l A873
  \l A880
  \l A881
  \l A882
  \l A883
  \l A884
  \l A885
  \l A886
  \l A887
  \l A888
  \l A889
  \l A88A
  \l A88B
  \l A88C
  \l A88D
  \l A88E
  \l A88F
  \l A890
  \l A891
  \l A892
  \l A893
  \l A894
  \l A895
  \l A896
  \l A897
  \l A898
  \l A899
  \l A89A
  \l A89B
  \l A89C
  \l A89D
  \l A89E
  \l A89F
  \l A8A0
  \l A8A1
  \l A8A2
  \l A8A3
  \l A8A4
  \l A8A5
  \l A8A6
  \l A8A7
  \l A8A8
  \l A8A9
  \l A8AA
  \l A8AB
  \l A8AC
  \l A8AD
  \l A8AE
  \l A8AF
  \l A8B0
  \l A8B1
  \l A8B2
  \l A8B3
  \l A8B4
  \l A8B5
  \l A8B6
  \l A8B7
  \l A8B8
  \l A8B9
  \l A8BA
  \l A8BB
  \l A8BC
  \l A8BD
  \l A8BE
  \l A8BF
  \l A8C0
  \l A8C1
  \l A8C2
  \l A8C3
  \l A8C4
  \l A8E0
  \l A8E1
  \l A8E2
  \l A8E3
  \l A8E4
  \l A8E5
  \l A8E6
  \l A8E7
  \l A8E8
  \l A8E9
  \l A8EA
  \l A8EB
  \l A8EC
  \l A8ED
  \l A8EE
  \l A8EF
  \l A8F0
  \l A8F1
  \l A8F2
  \l A8F3
  \l A8F4
  \l A8F5
  \l A8F6
  \l A8F7
  \l A8FB
  \l A90A
  \l A90B
  \l A90C
  \l A90D
  \l A90E
  \l A90F
  \l A910
  \l A911
  \l A912
  \l A913
  \l A914
  \l A915
  \l A916
  \l A917
  \l A918
  \l A919
  \l A91A
  \l A91B
  \l A91C
  \l A91D
  \l A91E
  \l A91F
  \l A920
  \l A921
  \l A922
  \l A923
  \l A924
  \l A925
  \l A926
  \l A927
  \l A928
  \l A929
  \l A92A
  \l A92B
  \l A92C
  \l A92D
  \l A930
  \l A931
  \l A932
  \l A933
  \l A934
  \l A935
  \l A936
  \l A937
  \l A938
  \l A939
  \l A93A
  \l A93B
  \l A93C
  \l A93D
  \l A93E
  \l A93F
  \l A940
  \l A941
  \l A942
  \l A943
  \l A944
  \l A945
  \l A946
  \l A947
  \l A948
  \l A949
  \l A94A
  \l A94B
  \l A94C
  \l A94D
  \l A94E
  \l A94F
  \l A950
  \l A951
  \l A952
  \l A953
  \l A960
  \l A961
  \l A962
  \l A963
  \l A964
  \l A965
  \l A966
  \l A967
  \l A968
  \l A969
  \l A96A
  \l A96B
  \l A96C
  \l A96D
  \l A96E
  \l A96F
  \l A970
  \l A971
  \l A972
  \l A973
  \l A974
  \l A975
  \l A976
  \l A977
  \l A978
  \l A979
  \l A97A
  \l A97B
  \l A97C
  \l A980
  \l A981
  \l A982
  \l A983
  \l A984
  \l A985
  \l A986
  \l A987
  \l A988
  \l A989
  \l A98A
  \l A98B
  \l A98C
  \l A98D
  \l A98E
  \l A98F
  \l A990
  \l A991
  \l A992
  \l A993
  \l A994
  \l A995
  \l A996
  \l A997
  \l A998
  \l A999
  \l A99A
  \l A99B
  \l A99C
  \l A99D
  \l A99E
  \l A99F
  \l A9A0
  \l A9A1
  \l A9A2
  \l A9A3
  \l A9A4
  \l A9A5
  \l A9A6
  \l A9A7
  \l A9A8
  \l A9A9
  \l A9AA
  \l A9AB
  \l A9AC
  \l A9AD
  \l A9AE
  \l A9AF
  \l A9B0
  \l A9B1
  \l A9B2
  \l A9B3
  \l A9B4
  \l A9B5
  \l A9B6
  \l A9B7
  \l A9B8
  \l A9B9
  \l A9BA
  \l A9BB
  \l A9BC
  \l A9BD
  \l A9BE
  \l A9BF
  \l A9C0
  \l A9CF
  \l A9E0
  \l A9E1
  \l A9E2
  \l A9E3
  \l A9E4
  \l A9E5
  \l A9E6
  \l A9E7
  \l A9E8
  \l A9E9
  \l A9EA
  \l A9EB
  \l A9EC
  \l A9ED
  \l A9EE
  \l A9EF
  \l A9FA
  \l A9FB
  \l A9FC
  \l A9FD
  \l A9FE
  \l AA00
  \l AA01
  \l AA02
  \l AA03
  \l AA04
  \l AA05
  \l AA06
  \l AA07
  \l AA08
  \l AA09
  \l AA0A
  \l AA0B
  \l AA0C
  \l AA0D
  \l AA0E
  \l AA0F
  \l AA10
  \l AA11
  \l AA12
  \l AA13
  \l AA14
  \l AA15
  \l AA16
  \l AA17
  \l AA18
  \l AA19
  \l AA1A
  \l AA1B
  \l AA1C
  \l AA1D
  \l AA1E
  \l AA1F
  \l AA20
  \l AA21
  \l AA22
  \l AA23
  \l AA24
  \l AA25
  \l AA26
  \l AA27
  \l AA28
  \l AA29
  \l AA2A
  \l AA2B
  \l AA2C
  \l AA2D
  \l AA2E
  \l AA2F
  \l AA30
  \l AA31
  \l AA32
  \l AA33
  \l AA34
  \l AA35
  \l AA36
  \l AA40
  \l AA41
  \l AA42
  \l AA43
  \l AA44
  \l AA45
  \l AA46
  \l AA47
  \l AA48
  \l AA49
  \l AA4A
  \l AA4B
  \l AA4C
  \l AA4D
  \l AA60
  \l AA61
  \l AA62
  \l AA63
  \l AA64
  \l AA65
  \l AA66
  \l AA67
  \l AA68
  \l AA69
  \l AA6A
  \l AA6B
  \l AA6C
  \l AA6D
  \l AA6E
  \l AA6F
  \l AA70
  \l AA71
  \l AA72
  \l AA73
  \l AA74
  \l AA75
  \l AA76
  \l AA7A
  \l AA7B
  \l AA7C
  \l AA7D
  \l AA7E
  \l AA7F
  \l AA80
  \l AA81
  \l AA82
  \l AA83
  \l AA84
  \l AA85
  \l AA86
  \l AA87
  \l AA88
  \l AA89
  \l AA8A
  \l AA8B
  \l AA8C
  \l AA8D
  \l AA8E
  \l AA8F
  \l AA90
  \l AA91
  \l AA92
  \l AA93
  \l AA94
  \l AA95
  \l AA96
  \l AA97
  \l AA98
  \l AA99
  \l AA9A
  \l AA9B
  \l AA9C
  \l AA9D
  \l AA9E
  \l AA9F
  \l AAA0
  \l AAA1
  \l AAA2
  \l AAA3
  \l AAA4
  \l AAA5
  \l AAA6
  \l AAA7
  \l AAA8
  \l AAA9
  \l AAAA
  \l AAAB
  \l AAAC
  \l AAAD
  \l AAAE
  \l AAAF
  \l AAB0
  \l AAB1
  \l AAB2
  \l AAB3
  \l AAB4
  \l AAB5
  \l AAB6
  \l AAB7
  \l AAB8
  \l AAB9
  \l AABA
  \l AABB
  \l AABC
  \l AABD
  \l AABE
  \l AABF
  \l AAC0
  \l AAC1
  \l AAC2
  \l AADB
  \l AADC
  \l AADD
  \l AAE0
  \l AAE1
  \l AAE2
  \l AAE3
  \l AAE4
  \l AAE5
  \l AAE6
  \l AAE7
  \l AAE8
  \l AAE9
  \l AAEA
  \l AAEB
  \l AAEC
  \l AAED
  \l AAEE
  \l AAEF
  \l AAF2
  \l AAF3
  \l AAF4
  \l AAF5
  \l AAF6
  \l AB01
  \l AB02
  \l AB03
  \l AB04
  \l AB05
  \l AB06
  \l AB09
  \l AB0A
  \l AB0B
  \l AB0C
  \l AB0D
  \l AB0E
  \l AB11
  \l AB12
  \l AB13
  \l AB14
  \l AB15
  \l AB16
  \l AB20
  \l AB21
  \l AB22
  \l AB23
  \l AB24
  \l AB25
  \l AB26
  \l AB28
  \l AB29
  \l AB2A
  \l AB2B
  \l AB2C
  \l AB2D
  \l AB2E
  \l AB30
  \l AB31
  \l AB32
  \l AB33
  \l AB34
  \l AB35
  \l AB36
  \l AB37
  \l AB38
  \l AB39
  \l AB3A
  \l AB3B
  \l AB3C
  \l AB3D
  \l AB3E
  \l AB3F
  \l AB40
  \l AB41
  \l AB42
  \l AB43
  \l AB44
  \l AB45
  \l AB46
  \l AB47
  \l AB48
  \l AB49
  \l AB4A
  \l AB4B
  \l AB4C
  \l AB4D
  \l AB4E
  \l AB4F
  \l AB50
  \l AB51
  \l AB52
  \l AB53
  \l AB54
  \l AB55
  \l AB56
  \l AB57
  \l AB58
  \l AB59
  \l AB5A
  \l AB5C
  \l AB5D
  \l AB5E
  \l AB5F
  \l AB64
  \l AB65
  \l ABC0
  \l ABC1
  \l ABC2
  \l ABC3
  \l ABC4
  \l ABC5
  \l ABC6
  \l ABC7
  \l ABC8
  \l ABC9
  \l ABCA
  \l ABCB
  \l ABCC
  \l ABCD
  \l ABCE
  \l ABCF
  \l ABD0
  \l ABD1
  \l ABD2
  \l ABD3
  \l ABD4
  \l ABD5
  \l ABD6
  \l ABD7
  \l ABD8
  \l ABD9
  \l ABDA
  \l ABDB
  \l ABDC
  \l ABDD
  \l ABDE
  \l ABDF
  \l ABE0
  \l ABE1
  \l ABE2
  \l ABE3
  \l ABE4
  \l ABE5
  \l ABE6
  \l ABE7
  \l ABE8
  \l ABE9
  \l ABEA
  \l ABEC
  \l ABED
  \l AC00
  \l AC01
  \l AC02
  \l AC03
  \l AC04
  \l AC05
  \l AC06
  \l AC07
  \l AC08
  \l AC09
  \l AC0A
  \l AC0B
  \l AC0C
  \l AC0D
  \l AC0E
  \l AC0F
  \l AC10
  \l AC11
  \l AC12
  \l AC13
  \l AC14
  \l AC15
  \l AC16
  \l AC17
  \l AC18
  \l AC19
  \l AC1A
  \l AC1B
  \l AC1C
  \l AC1D
  \l AC1E
  \l AC1F
  \l AC20
  \l AC21
  \l AC22
  \l AC23
  \l AC24
  \l AC25
  \l AC26
  \l AC27
  \l AC28
  \l AC29
  \l AC2A
  \l AC2B
  \l AC2C
  \l AC2D
  \l AC2E
  \l AC2F
  \l AC30
  \l AC31
  \l AC32
  \l AC33
  \l AC34
  \l AC35
  \l AC36
  \l AC37
  \l AC38
  \l AC39
  \l AC3A
  \l AC3B
  \l AC3C
  \l AC3D
  \l AC3E
  \l AC3F
  \l AC40
  \l AC41
  \l AC42
  \l AC43
  \l AC44
  \l AC45
  \l AC46
  \l AC47
  \l AC48
  \l AC49
  \l AC4A
  \l AC4B
  \l AC4C
  \l AC4D
  \l AC4E
  \l AC4F
  \l AC50
  \l AC51
  \l AC52
  \l AC53
  \l AC54
  \l AC55
  \l AC56
  \l AC57
  \l AC58
  \l AC59
  \l AC5A
  \l AC5B
  \l AC5C
  \l AC5D
  \l AC5E
  \l AC5F
  \l AC60
  \l AC61
  \l AC62
  \l AC63
  \l AC64
  \l AC65
  \l AC66
  \l AC67
  \l AC68
  \l AC69
  \l AC6A
  \l AC6B
  \l AC6C
  \l AC6D
  \l AC6E
  \l AC6F
  \l AC70
  \l AC71
  \l AC72
  \l AC73
  \l AC74
  \l AC75
  \l AC76
  \l AC77
  \l AC78
  \l AC79
  \l AC7A
  \l AC7B
  \l AC7C
  \l AC7D
  \l AC7E
  \l AC7F
  \l AC80
  \l AC81
  \l AC82
  \l AC83
  \l AC84
  \l AC85
  \l AC86
  \l AC87
  \l AC88
  \l AC89
  \l AC8A
  \l AC8B
  \l AC8C
  \l AC8D
  \l AC8E
  \l AC8F
  \l AC90
  \l AC91
  \l AC92
  \l AC93
  \l AC94
  \l AC95
  \l AC96
  \l AC97
  \l AC98
  \l AC99
  \l AC9A
  \l AC9B
  \l AC9C
  \l AC9D
  \l AC9E
  \l AC9F
  \l ACA0
  \l ACA1
  \l ACA2
  \l ACA3
  \l ACA4
  \l ACA5
  \l ACA6
  \l ACA7
  \l ACA8
  \l ACA9
  \l ACAA
  \l ACAB
  \l ACAC
  \l ACAD
  \l ACAE
  \l ACAF
  \l ACB0
  \l ACB1
  \l ACB2
  \l ACB3
  \l ACB4
  \l ACB5
  \l ACB6
  \l ACB7
  \l ACB8
  \l ACB9
  \l ACBA
  \l ACBB
  \l ACBC
  \l ACBD
  \l ACBE
  \l ACBF
  \l ACC0
  \l ACC1
  \l ACC2
  \l ACC3
  \l ACC4
  \l ACC5
  \l ACC6
  \l ACC7
  \l ACC8
  \l ACC9
  \l ACCA
  \l ACCB
  \l ACCC
  \l ACCD
  \l ACCE
  \l ACCF
  \l ACD0
  \l ACD1
  \l ACD2
  \l ACD3
  \l ACD4
  \l ACD5
  \l ACD6
  \l ACD7
  \l ACD8
  \l ACD9
  \l ACDA
  \l ACDB
  \l ACDC
  \l ACDD
  \l ACDE
  \l ACDF
  \l ACE0
  \l ACE1
  \l ACE2
  \l ACE3
  \l ACE4
  \l ACE5
  \l ACE6
  \l ACE7
  \l ACE8
  \l ACE9
  \l ACEA
  \l ACEB
  \l ACEC
  \l ACED
  \l ACEE
  \l ACEF
  \l ACF0
  \l ACF1
  \l ACF2
  \l ACF3
  \l ACF4
  \l ACF5
  \l ACF6
  \l ACF7
  \l ACF8
  \l ACF9
  \l ACFA
  \l ACFB
  \l ACFC
  \l ACFD
  \l ACFE
  \l ACFF
  \l AD00
  \l AD01
  \l AD02
  \l AD03
  \l AD04
  \l AD05
  \l AD06
  \l AD07
  \l AD08
  \l AD09
  \l AD0A
  \l AD0B
  \l AD0C
  \l AD0D
  \l AD0E
  \l AD0F
  \l AD10
  \l AD11
  \l AD12
  \l AD13
  \l AD14
  \l AD15
  \l AD16
  \l AD17
  \l AD18
  \l AD19
  \l AD1A
  \l AD1B
  \l AD1C
  \l AD1D
  \l AD1E
  \l AD1F
  \l AD20
  \l AD21
  \l AD22
  \l AD23
  \l AD24
  \l AD25
  \l AD26
  \l AD27
  \l AD28
  \l AD29
  \l AD2A
  \l AD2B
  \l AD2C
  \l AD2D
  \l AD2E
  \l AD2F
  \l AD30
  \l AD31
  \l AD32
  \l AD33
  \l AD34
  \l AD35
  \l AD36
  \l AD37
  \l AD38
  \l AD39
  \l AD3A
  \l AD3B
  \l AD3C
  \l AD3D
  \l AD3E
  \l AD3F
  \l AD40
  \l AD41
  \l AD42
  \l AD43
  \l AD44
  \l AD45
  \l AD46
  \l AD47
  \l AD48
  \l AD49
  \l AD4A
  \l AD4B
  \l AD4C
  \l AD4D
  \l AD4E
  \l AD4F
  \l AD50
  \l AD51
  \l AD52
  \l AD53
  \l AD54
  \l AD55
  \l AD56
  \l AD57
  \l AD58
  \l AD59
  \l AD5A
  \l AD5B
  \l AD5C
  \l AD5D
  \l AD5E
  \l AD5F
  \l AD60
  \l AD61
  \l AD62
  \l AD63
  \l AD64
  \l AD65
  \l AD66
  \l AD67
  \l AD68
  \l AD69
  \l AD6A
  \l AD6B
  \l AD6C
  \l AD6D
  \l AD6E
  \l AD6F
  \l AD70
  \l AD71
  \l AD72
  \l AD73
  \l AD74
  \l AD75
  \l AD76
  \l AD77
  \l AD78
  \l AD79
  \l AD7A
  \l AD7B
  \l AD7C
  \l AD7D
  \l AD7E
  \l AD7F
  \l AD80
  \l AD81
  \l AD82
  \l AD83
  \l AD84
  \l AD85
  \l AD86
  \l AD87
  \l AD88
  \l AD89
  \l AD8A
  \l AD8B
  \l AD8C
  \l AD8D
  \l AD8E
  \l AD8F
  \l AD90
  \l AD91
  \l AD92
  \l AD93
  \l AD94
  \l AD95
  \l AD96
  \l AD97
  \l AD98
  \l AD99
  \l AD9A
  \l AD9B
  \l AD9C
  \l AD9D
  \l AD9E
  \l AD9F
  \l ADA0
  \l ADA1
  \l ADA2
  \l ADA3
  \l ADA4
  \l ADA5
  \l ADA6
  \l ADA7
  \l ADA8
  \l ADA9
  \l ADAA
  \l ADAB
  \l ADAC
  \l ADAD
  \l ADAE
  \l ADAF
  \l ADB0
  \l ADB1
  \l ADB2
  \l ADB3
  \l ADB4
  \l ADB5
  \l ADB6
  \l ADB7
  \l ADB8
  \l ADB9
  \l ADBA
  \l ADBB
  \l ADBC
  \l ADBD
  \l ADBE
  \l ADBF
  \l ADC0
  \l ADC1
  \l ADC2
  \l ADC3
  \l ADC4
  \l ADC5
  \l ADC6
  \l ADC7
  \l ADC8
  \l ADC9
  \l ADCA
  \l ADCB
  \l ADCC
  \l ADCD
  \l ADCE
  \l ADCF
  \l ADD0
  \l ADD1
  \l ADD2
  \l ADD3
  \l ADD4
  \l ADD5
  \l ADD6
  \l ADD7
  \l ADD8
  \l ADD9
  \l ADDA
  \l ADDB
  \l ADDC
  \l ADDD
  \l ADDE
  \l ADDF
  \l ADE0
  \l ADE1
  \l ADE2
  \l ADE3
  \l ADE4
  \l ADE5
  \l ADE6
  \l ADE7
  \l ADE8
  \l ADE9
  \l ADEA
  \l ADEB
  \l ADEC
  \l ADED
  \l ADEE
  \l ADEF
  \l ADF0
  \l ADF1
  \l ADF2
  \l ADF3
  \l ADF4
  \l ADF5
  \l ADF6
  \l ADF7
  \l ADF8
  \l ADF9
  \l ADFA
  \l ADFB
  \l ADFC
  \l ADFD
  \l ADFE
  \l ADFF
  \l AE00
  \l AE01
  \l AE02
  \l AE03
  \l AE04
  \l AE05
  \l AE06
  \l AE07
  \l AE08
  \l AE09
  \l AE0A
  \l AE0B
  \l AE0C
  \l AE0D
  \l AE0E
  \l AE0F
  \l AE10
  \l AE11
  \l AE12
  \l AE13
  \l AE14
  \l AE15
  \l AE16
  \l AE17
  \l AE18
  \l AE19
  \l AE1A
  \l AE1B
  \l AE1C
  \l AE1D
  \l AE1E
  \l AE1F
  \l AE20
  \l AE21
  \l AE22
  \l AE23
  \l AE24
  \l AE25
  \l AE26
  \l AE27
  \l AE28
  \l AE29
  \l AE2A
  \l AE2B
  \l AE2C
  \l AE2D
  \l AE2E
  \l AE2F
  \l AE30
  \l AE31
  \l AE32
  \l AE33
  \l AE34
  \l AE35
  \l AE36
  \l AE37
  \l AE38
  \l AE39
  \l AE3A
  \l AE3B
  \l AE3C
  \l AE3D
  \l AE3E
  \l AE3F
  \l AE40
  \l AE41
  \l AE42
  \l AE43
  \l AE44
  \l AE45
  \l AE46
  \l AE47
  \l AE48
  \l AE49
  \l AE4A
  \l AE4B
  \l AE4C
  \l AE4D
  \l AE4E
  \l AE4F
  \l AE50
  \l AE51
  \l AE52
  \l AE53
  \l AE54
  \l AE55
  \l AE56
  \l AE57
  \l AE58
  \l AE59
  \l AE5A
  \l AE5B
  \l AE5C
  \l AE5D
  \l AE5E
  \l AE5F
  \l AE60
  \l AE61
  \l AE62
  \l AE63
  \l AE64
  \l AE65
  \l AE66
  \l AE67
  \l AE68
  \l AE69
  \l AE6A
  \l AE6B
  \l AE6C
  \l AE6D
  \l AE6E
  \l AE6F
  \l AE70
  \l AE71
  \l AE72
  \l AE73
  \l AE74
  \l AE75
  \l AE76
  \l AE77
  \l AE78
  \l AE79
  \l AE7A
  \l AE7B
  \l AE7C
  \l AE7D
  \l AE7E
  \l AE7F
  \l AE80
  \l AE81
  \l AE82
  \l AE83
  \l AE84
  \l AE85
  \l AE86
  \l AE87
  \l AE88
  \l AE89
  \l AE8A
  \l AE8B
  \l AE8C
  \l AE8D
  \l AE8E
  \l AE8F
  \l AE90
  \l AE91
  \l AE92
  \l AE93
  \l AE94
  \l AE95
  \l AE96
  \l AE97
  \l AE98
  \l AE99
  \l AE9A
  \l AE9B
  \l AE9C
  \l AE9D
  \l AE9E
  \l AE9F
  \l AEA0
  \l AEA1
  \l AEA2
  \l AEA3
  \l AEA4
  \l AEA5
  \l AEA6
  \l AEA7
  \l AEA8
  \l AEA9
  \l AEAA
  \l AEAB
  \l AEAC
  \l AEAD
  \l AEAE
  \l AEAF
  \l AEB0
  \l AEB1
  \l AEB2
  \l AEB3
  \l AEB4
  \l AEB5
  \l AEB6
  \l AEB7
  \l AEB8
  \l AEB9
  \l AEBA
  \l AEBB
  \l AEBC
  \l AEBD
  \l AEBE
  \l AEBF
  \l AEC0
  \l AEC1
  \l AEC2
  \l AEC3
  \l AEC4
  \l AEC5
  \l AEC6
  \l AEC7
  \l AEC8
  \l AEC9
  \l AECA
  \l AECB
  \l AECC
  \l AECD
  \l AECE
  \l AECF
  \l AED0
  \l AED1
  \l AED2
  \l AED3
  \l AED4
  \l AED5
  \l AED6
  \l AED7
  \l AED8
  \l AED9
  \l AEDA
  \l AEDB
  \l AEDC
  \l AEDD
  \l AEDE
  \l AEDF
  \l AEE0
  \l AEE1
  \l AEE2
  \l AEE3
  \l AEE4
  \l AEE5
  \l AEE6
  \l AEE7
  \l AEE8
  \l AEE9
  \l AEEA
  \l AEEB
  \l AEEC
  \l AEED
  \l AEEE
  \l AEEF
  \l AEF0
  \l AEF1
  \l AEF2
  \l AEF3
  \l AEF4
  \l AEF5
  \l AEF6
  \l AEF7
  \l AEF8
  \l AEF9
  \l AEFA
  \l AEFB
  \l AEFC
  \l AEFD
  \l AEFE
  \l AEFF
  \l AF00
  \l AF01
  \l AF02
  \l AF03
  \l AF04
  \l AF05
  \l AF06
  \l AF07
  \l AF08
  \l AF09
  \l AF0A
  \l AF0B
  \l AF0C
  \l AF0D
  \l AF0E
  \l AF0F
  \l AF10
  \l AF11
  \l AF12
  \l AF13
  \l AF14
  \l AF15
  \l AF16
  \l AF17
  \l AF18
  \l AF19
  \l AF1A
  \l AF1B
  \l AF1C
  \l AF1D
  \l AF1E
  \l AF1F
  \l AF20
  \l AF21
  \l AF22
  \l AF23
  \l AF24
  \l AF25
  \l AF26
  \l AF27
  \l AF28
  \l AF29
  \l AF2A
  \l AF2B
  \l AF2C
  \l AF2D
  \l AF2E
  \l AF2F
  \l AF30
  \l AF31
  \l AF32
  \l AF33
  \l AF34
  \l AF35
  \l AF36
  \l AF37
  \l AF38
  \l AF39
  \l AF3A
  \l AF3B
  \l AF3C
  \l AF3D
  \l AF3E
  \l AF3F
  \l AF40
  \l AF41
  \l AF42
  \l AF43
  \l AF44
  \l AF45
  \l AF46
  \l AF47
  \l AF48
  \l AF49
  \l AF4A
  \l AF4B
  \l AF4C
  \l AF4D
  \l AF4E
  \l AF4F
  \l AF50
  \l AF51
  \l AF52
  \l AF53
  \l AF54
  \l AF55
  \l AF56
  \l AF57
  \l AF58
  \l AF59
  \l AF5A
  \l AF5B
  \l AF5C
  \l AF5D
  \l AF5E
  \l AF5F
  \l AF60
  \l AF61
  \l AF62
  \l AF63
  \l AF64
  \l AF65
  \l AF66
  \l AF67
  \l AF68
  \l AF69
  \l AF6A
  \l AF6B
  \l AF6C
  \l AF6D
  \l AF6E
  \l AF6F
  \l AF70
  \l AF71
  \l AF72
  \l AF73
  \l AF74
  \l AF75
  \l AF76
  \l AF77
  \l AF78
  \l AF79
  \l AF7A
  \l AF7B
  \l AF7C
  \l AF7D
  \l AF7E
  \l AF7F
  \l AF80
  \l AF81
  \l AF82
  \l AF83
  \l AF84
  \l AF85
  \l AF86
  \l AF87
  \l AF88
  \l AF89
  \l AF8A
  \l AF8B
  \l AF8C
  \l AF8D
  \l AF8E
  \l AF8F
  \l AF90
  \l AF91
  \l AF92
  \l AF93
  \l AF94
  \l AF95
  \l AF96
  \l AF97
  \l AF98
  \l AF99
  \l AF9A
  \l AF9B
  \l AF9C
  \l AF9D
  \l AF9E
  \l AF9F
  \l AFA0
  \l AFA1
  \l AFA2
  \l AFA3
  \l AFA4
  \l AFA5
  \l AFA6
  \l AFA7
  \l AFA8
  \l AFA9
  \l AFAA
  \l AFAB
  \l AFAC
  \l AFAD
  \l AFAE
  \l AFAF
  \l AFB0
  \l AFB1
  \l AFB2
  \l AFB3
  \l AFB4
  \l AFB5
  \l AFB6
  \l AFB7
  \l AFB8
  \l AFB9
  \l AFBA
  \l AFBB
  \l AFBC
  \l AFBD
  \l AFBE
  \l AFBF
  \l AFC0
  \l AFC1
  \l AFC2
  \l AFC3
  \l AFC4
  \l AFC5
  \l AFC6
  \l AFC7
  \l AFC8
  \l AFC9
  \l AFCA
  \l AFCB
  \l AFCC
  \l AFCD
  \l AFCE
  \l AFCF
  \l AFD0
  \l AFD1
  \l AFD2
  \l AFD3
  \l AFD4
  \l AFD5
  \l AFD6
  \l AFD7
  \l AFD8
  \l AFD9
  \l AFDA
  \l AFDB
  \l AFDC
  \l AFDD
  \l AFDE
  \l AFDF
  \l AFE0
  \l AFE1
  \l AFE2
  \l AFE3
  \l AFE4
  \l AFE5
  \l AFE6
  \l AFE7
  \l AFE8
  \l AFE9
  \l AFEA
  \l AFEB
  \l AFEC
  \l AFED
  \l AFEE
  \l AFEF
  \l AFF0
  \l AFF1
  \l AFF2
  \l AFF3
  \l AFF4
  \l AFF5
  \l AFF6
  \l AFF7
  \l AFF8
  \l AFF9
  \l AFFA
  \l AFFB
  \l AFFC
  \l AFFD
  \l AFFE
  \l AFFF
  \l B000
  \l B001
  \l B002
  \l B003
  \l B004
  \l B005
  \l B006
  \l B007
  \l B008
  \l B009
  \l B00A
  \l B00B
  \l B00C
  \l B00D
  \l B00E
  \l B00F
  \l B010
  \l B011
  \l B012
  \l B013
  \l B014
  \l B015
  \l B016
  \l B017
  \l B018
  \l B019
  \l B01A
  \l B01B
  \l B01C
  \l B01D
  \l B01E
  \l B01F
  \l B020
  \l B021
  \l B022
  \l B023
  \l B024
  \l B025
  \l B026
  \l B027
  \l B028
  \l B029
  \l B02A
  \l B02B
  \l B02C
  \l B02D
  \l B02E
  \l B02F
  \l B030
  \l B031
  \l B032
  \l B033
  \l B034
  \l B035
  \l B036
  \l B037
  \l B038
  \l B039
  \l B03A
  \l B03B
  \l B03C
  \l B03D
  \l B03E
  \l B03F
  \l B040
  \l B041
  \l B042
  \l B043
  \l B044
  \l B045
  \l B046
  \l B047
  \l B048
  \l B049
  \l B04A
  \l B04B
  \l B04C
  \l B04D
  \l B04E
  \l B04F
  \l B050
  \l B051
  \l B052
  \l B053
  \l B054
  \l B055
  \l B056
  \l B057
  \l B058
  \l B059
  \l B05A
  \l B05B
  \l B05C
  \l B05D
  \l B05E
  \l B05F
  \l B060
  \l B061
  \l B062
  \l B063
  \l B064
  \l B065
  \l B066
  \l B067
  \l B068
  \l B069
  \l B06A
  \l B06B
  \l B06C
  \l B06D
  \l B06E
  \l B06F
  \l B070
  \l B071
  \l B072
  \l B073
  \l B074
  \l B075
  \l B076
  \l B077
  \l B078
  \l B079
  \l B07A
  \l B07B
  \l B07C
  \l B07D
  \l B07E
  \l B07F
  \l B080
  \l B081
  \l B082
  \l B083
  \l B084
  \l B085
  \l B086
  \l B087
  \l B088
  \l B089
  \l B08A
  \l B08B
  \l B08C
  \l B08D
  \l B08E
  \l B08F
  \l B090
  \l B091
  \l B092
  \l B093
  \l B094
  \l B095
  \l B096
  \l B097
  \l B098
  \l B099
  \l B09A
  \l B09B
  \l B09C
  \l B09D
  \l B09E
  \l B09F
  \l B0A0
  \l B0A1
  \l B0A2
  \l B0A3
  \l B0A4
  \l B0A5
  \l B0A6
  \l B0A7
  \l B0A8
  \l B0A9
  \l B0AA
  \l B0AB
  \l B0AC
  \l B0AD
  \l B0AE
  \l B0AF
  \l B0B0
  \l B0B1
  \l B0B2
  \l B0B3
  \l B0B4
  \l B0B5
  \l B0B6
  \l B0B7
  \l B0B8
  \l B0B9
  \l B0BA
  \l B0BB
  \l B0BC
  \l B0BD
  \l B0BE
  \l B0BF
  \l B0C0
  \l B0C1
  \l B0C2
  \l B0C3
  \l B0C4
  \l B0C5
  \l B0C6
  \l B0C7
  \l B0C8
  \l B0C9
  \l B0CA
  \l B0CB
  \l B0CC
  \l B0CD
  \l B0CE
  \l B0CF
  \l B0D0
  \l B0D1
  \l B0D2
  \l B0D3
  \l B0D4
  \l B0D5
  \l B0D6
  \l B0D7
  \l B0D8
  \l B0D9
  \l B0DA
  \l B0DB
  \l B0DC
  \l B0DD
  \l B0DE
  \l B0DF
  \l B0E0
  \l B0E1
  \l B0E2
  \l B0E3
  \l B0E4
  \l B0E5
  \l B0E6
  \l B0E7
  \l B0E8
  \l B0E9
  \l B0EA
  \l B0EB
  \l B0EC
  \l B0ED
  \l B0EE
  \l B0EF
  \l B0F0
  \l B0F1
  \l B0F2
  \l B0F3
  \l B0F4
  \l B0F5
  \l B0F6
  \l B0F7
  \l B0F8
  \l B0F9
  \l B0FA
  \l B0FB
  \l B0FC
  \l B0FD
  \l B0FE
  \l B0FF
  \l B100
  \l B101
  \l B102
  \l B103
  \l B104
  \l B105
  \l B106
  \l B107
  \l B108
  \l B109
  \l B10A
  \l B10B
  \l B10C
  \l B10D
  \l B10E
  \l B10F
  \l B110
  \l B111
  \l B112
  \l B113
  \l B114
  \l B115
  \l B116
  \l B117
  \l B118
  \l B119
  \l B11A
  \l B11B
  \l B11C
  \l B11D
  \l B11E
  \l B11F
  \l B120
  \l B121
  \l B122
  \l B123
  \l B124
  \l B125
  \l B126
  \l B127
  \l B128
  \l B129
  \l B12A
  \l B12B
  \l B12C
  \l B12D
  \l B12E
  \l B12F
  \l B130
  \l B131
  \l B132
  \l B133
  \l B134
  \l B135
  \l B136
  \l B137
  \l B138
  \l B139
  \l B13A
  \l B13B
  \l B13C
  \l B13D
  \l B13E
  \l B13F
  \l B140
  \l B141
  \l B142
  \l B143
  \l B144
  \l B145
  \l B146
  \l B147
  \l B148
  \l B149
  \l B14A
  \l B14B
  \l B14C
  \l B14D
  \l B14E
  \l B14F
  \l B150
  \l B151
  \l B152
  \l B153
  \l B154
  \l B155
  \l B156
  \l B157
  \l B158
  \l B159
  \l B15A
  \l B15B
  \l B15C
  \l B15D
  \l B15E
  \l B15F
  \l B160
  \l B161
  \l B162
  \l B163
  \l B164
  \l B165
  \l B166
  \l B167
  \l B168
  \l B169
  \l B16A
  \l B16B
  \l B16C
  \l B16D
  \l B16E
  \l B16F
  \l B170
  \l B171
  \l B172
  \l B173
  \l B174
  \l B175
  \l B176
  \l B177
  \l B178
  \l B179
  \l B17A
  \l B17B
  \l B17C
  \l B17D
  \l B17E
  \l B17F
  \l B180
  \l B181
  \l B182
  \l B183
  \l B184
  \l B185
  \l B186
  \l B187
  \l B188
  \l B189
  \l B18A
  \l B18B
  \l B18C
  \l B18D
  \l B18E
  \l B18F
  \l B190
  \l B191
  \l B192
  \l B193
  \l B194
  \l B195
  \l B196
  \l B197
  \l B198
  \l B199
  \l B19A
  \l B19B
  \l B19C
  \l B19D
  \l B19E
  \l B19F
  \l B1A0
  \l B1A1
  \l B1A2
  \l B1A3
  \l B1A4
  \l B1A5
  \l B1A6
  \l B1A7
  \l B1A8
  \l B1A9
  \l B1AA
  \l B1AB
  \l B1AC
  \l B1AD
  \l B1AE
  \l B1AF
  \l B1B0
  \l B1B1
  \l B1B2
  \l B1B3
  \l B1B4
  \l B1B5
  \l B1B6
  \l B1B7
  \l B1B8
  \l B1B9
  \l B1BA
  \l B1BB
  \l B1BC
  \l B1BD
  \l B1BE
  \l B1BF
  \l B1C0
  \l B1C1
  \l B1C2
  \l B1C3
  \l B1C4
  \l B1C5
  \l B1C6
  \l B1C7
  \l B1C8
  \l B1C9
  \l B1CA
  \l B1CB
  \l B1CC
  \l B1CD
  \l B1CE
  \l B1CF
  \l B1D0
  \l B1D1
  \l B1D2
  \l B1D3
  \l B1D4
  \l B1D5
  \l B1D6
  \l B1D7
  \l B1D8
  \l B1D9
  \l B1DA
  \l B1DB
  \l B1DC
  \l B1DD
  \l B1DE
  \l B1DF
  \l B1E0
  \l B1E1
  \l B1E2
  \l B1E3
  \l B1E4
  \l B1E5
  \l B1E6
  \l B1E7
  \l B1E8
  \l B1E9
  \l B1EA
  \l B1EB
  \l B1EC
  \l B1ED
  \l B1EE
  \l B1EF
  \l B1F0
  \l B1F1
  \l B1F2
  \l B1F3
  \l B1F4
  \l B1F5
  \l B1F6
  \l B1F7
  \l B1F8
  \l B1F9
  \l B1FA
  \l B1FB
  \l B1FC
  \l B1FD
  \l B1FE
  \l B1FF
  \l B200
  \l B201
  \l B202
  \l B203
  \l B204
  \l B205
  \l B206
  \l B207
  \l B208
  \l B209
  \l B20A
  \l B20B
  \l B20C
  \l B20D
  \l B20E
  \l B20F
  \l B210
  \l B211
  \l B212
  \l B213
  \l B214
  \l B215
  \l B216
  \l B217
  \l B218
  \l B219
  \l B21A
  \l B21B
  \l B21C
  \l B21D
  \l B21E
  \l B21F
  \l B220
  \l B221
  \l B222
  \l B223
  \l B224
  \l B225
  \l B226
  \l B227
  \l B228
  \l B229
  \l B22A
  \l B22B
  \l B22C
  \l B22D
  \l B22E
  \l B22F
  \l B230
  \l B231
  \l B232
  \l B233
  \l B234
  \l B235
  \l B236
  \l B237
  \l B238
  \l B239
  \l B23A
  \l B23B
  \l B23C
  \l B23D
  \l B23E
  \l B23F
  \l B240
  \l B241
  \l B242
  \l B243
  \l B244
  \l B245
  \l B246
  \l B247
  \l B248
  \l B249
  \l B24A
  \l B24B
  \l B24C
  \l B24D
  \l B24E
  \l B24F
  \l B250
  \l B251
  \l B252
  \l B253
  \l B254
  \l B255
  \l B256
  \l B257
  \l B258
  \l B259
  \l B25A
  \l B25B
  \l B25C
  \l B25D
  \l B25E
  \l B25F
  \l B260
  \l B261
  \l B262
  \l B263
  \l B264
  \l B265
  \l B266
  \l B267
  \l B268
  \l B269
  \l B26A
  \l B26B
  \l B26C
  \l B26D
  \l B26E
  \l B26F
  \l B270
  \l B271
  \l B272
  \l B273
  \l B274
  \l B275
  \l B276
  \l B277
  \l B278
  \l B279
  \l B27A
  \l B27B
  \l B27C
  \l B27D
  \l B27E
  \l B27F
  \l B280
  \l B281
  \l B282
  \l B283
  \l B284
  \l B285
  \l B286
  \l B287
  \l B288
  \l B289
  \l B28A
  \l B28B
  \l B28C
  \l B28D
  \l B28E
  \l B28F
  \l B290
  \l B291
  \l B292
  \l B293
  \l B294
  \l B295
  \l B296
  \l B297
  \l B298
  \l B299
  \l B29A
  \l B29B
  \l B29C
  \l B29D
  \l B29E
  \l B29F
  \l B2A0
  \l B2A1
  \l B2A2
  \l B2A3
  \l B2A4
  \l B2A5
  \l B2A6
  \l B2A7
  \l B2A8
  \l B2A9
  \l B2AA
  \l B2AB
  \l B2AC
  \l B2AD
  \l B2AE
  \l B2AF
  \l B2B0
  \l B2B1
  \l B2B2
  \l B2B3
  \l B2B4
  \l B2B5
  \l B2B6
  \l B2B7
  \l B2B8
  \l B2B9
  \l B2BA
  \l B2BB
  \l B2BC
  \l B2BD
  \l B2BE
  \l B2BF
  \l B2C0
  \l B2C1
  \l B2C2
  \l B2C3
  \l B2C4
  \l B2C5
  \l B2C6
  \l B2C7
  \l B2C8
  \l B2C9
  \l B2CA
  \l B2CB
  \l B2CC
  \l B2CD
  \l B2CE
  \l B2CF
  \l B2D0
  \l B2D1
  \l B2D2
  \l B2D3
  \l B2D4
  \l B2D5
  \l B2D6
  \l B2D7
  \l B2D8
  \l B2D9
  \l B2DA
  \l B2DB
  \l B2DC
  \l B2DD
  \l B2DE
  \l B2DF
  \l B2E0
  \l B2E1
  \l B2E2
  \l B2E3
  \l B2E4
  \l B2E5
  \l B2E6
  \l B2E7
  \l B2E8
  \l B2E9
  \l B2EA
  \l B2EB
  \l B2EC
  \l B2ED
  \l B2EE
  \l B2EF
  \l B2F0
  \l B2F1
  \l B2F2
  \l B2F3
  \l B2F4
  \l B2F5
  \l B2F6
  \l B2F7
  \l B2F8
  \l B2F9
  \l B2FA
  \l B2FB
  \l B2FC
  \l B2FD
  \l B2FE
  \l B2FF
  \l B300
  \l B301
  \l B302
  \l B303
  \l B304
  \l B305
  \l B306
  \l B307
  \l B308
  \l B309
  \l B30A
  \l B30B
  \l B30C
  \l B30D
  \l B30E
  \l B30F
  \l B310
  \l B311
  \l B312
  \l B313
  \l B314
  \l B315
  \l B316
  \l B317
  \l B318
  \l B319
  \l B31A
  \l B31B
  \l B31C
  \l B31D
  \l B31E
  \l B31F
  \l B320
  \l B321
  \l B322
  \l B323
  \l B324
  \l B325
  \l B326
  \l B327
  \l B328
  \l B329
  \l B32A
  \l B32B
  \l B32C
  \l B32D
  \l B32E
  \l B32F
  \l B330
  \l B331
  \l B332
  \l B333
  \l B334
  \l B335
  \l B336
  \l B337
  \l B338
  \l B339
  \l B33A
  \l B33B
  \l B33C
  \l B33D
  \l B33E
  \l B33F
  \l B340
  \l B341
  \l B342
  \l B343
  \l B344
  \l B345
  \l B346
  \l B347
  \l B348
  \l B349
  \l B34A
  \l B34B
  \l B34C
  \l B34D
  \l B34E
  \l B34F
  \l B350
  \l B351
  \l B352
  \l B353
  \l B354
  \l B355
  \l B356
  \l B357
  \l B358
  \l B359
  \l B35A
  \l B35B
  \l B35C
  \l B35D
  \l B35E
  \l B35F
  \l B360
  \l B361
  \l B362
  \l B363
  \l B364
  \l B365
  \l B366
  \l B367
  \l B368
  \l B369
  \l B36A
  \l B36B
  \l B36C
  \l B36D
  \l B36E
  \l B36F
  \l B370
  \l B371
  \l B372
  \l B373
  \l B374
  \l B375
  \l B376
  \l B377
  \l B378
  \l B379
  \l B37A
  \l B37B
  \l B37C
  \l B37D
  \l B37E
  \l B37F
  \l B380
  \l B381
  \l B382
  \l B383
  \l B384
  \l B385
  \l B386
  \l B387
  \l B388
  \l B389
  \l B38A
  \l B38B
  \l B38C
  \l B38D
  \l B38E
  \l B38F
  \l B390
  \l B391
  \l B392
  \l B393
  \l B394
  \l B395
  \l B396
  \l B397
  \l B398
  \l B399
  \l B39A
  \l B39B
  \l B39C
  \l B39D
  \l B39E
  \l B39F
  \l B3A0
  \l B3A1
  \l B3A2
  \l B3A3
  \l B3A4
  \l B3A5
  \l B3A6
  \l B3A7
  \l B3A8
  \l B3A9
  \l B3AA
  \l B3AB
  \l B3AC
  \l B3AD
  \l B3AE
  \l B3AF
  \l B3B0
  \l B3B1
  \l B3B2
  \l B3B3
  \l B3B4
  \l B3B5
  \l B3B6
  \l B3B7
  \l B3B8
  \l B3B9
  \l B3BA
  \l B3BB
  \l B3BC
  \l B3BD
  \l B3BE
  \l B3BF
  \l B3C0
  \l B3C1
  \l B3C2
  \l B3C3
  \l B3C4
  \l B3C5
  \l B3C6
  \l B3C7
  \l B3C8
  \l B3C9
  \l B3CA
  \l B3CB
  \l B3CC
  \l B3CD
  \l B3CE
  \l B3CF
  \l B3D0
  \l B3D1
  \l B3D2
  \l B3D3
  \l B3D4
  \l B3D5
  \l B3D6
  \l B3D7
  \l B3D8
  \l B3D9
  \l B3DA
  \l B3DB
  \l B3DC
  \l B3DD
  \l B3DE
  \l B3DF
  \l B3E0
  \l B3E1
  \l B3E2
  \l B3E3
  \l B3E4
  \l B3E5
  \l B3E6
  \l B3E7
  \l B3E8
  \l B3E9
  \l B3EA
  \l B3EB
  \l B3EC
  \l B3ED
  \l B3EE
  \l B3EF
  \l B3F0
  \l B3F1
  \l B3F2
  \l B3F3
  \l B3F4
  \l B3F5
  \l B3F6
  \l B3F7
  \l B3F8
  \l B3F9
  \l B3FA
  \l B3FB
  \l B3FC
  \l B3FD
  \l B3FE
  \l B3FF
  \l B400
  \l B401
  \l B402
  \l B403
  \l B404
  \l B405
  \l B406
  \l B407
  \l B408
  \l B409
  \l B40A
  \l B40B
  \l B40C
  \l B40D
  \l B40E
  \l B40F
  \l B410
  \l B411
  \l B412
  \l B413
  \l B414
  \l B415
  \l B416
  \l B417
  \l B418
  \l B419
  \l B41A
  \l B41B
  \l B41C
  \l B41D
  \l B41E
  \l B41F
  \l B420
  \l B421
  \l B422
  \l B423
  \l B424
  \l B425
  \l B426
  \l B427
  \l B428
  \l B429
  \l B42A
  \l B42B
  \l B42C
  \l B42D
  \l B42E
  \l B42F
  \l B430
  \l B431
  \l B432
  \l B433
  \l B434
  \l B435
  \l B436
  \l B437
  \l B438
  \l B439
  \l B43A
  \l B43B
  \l B43C
  \l B43D
  \l B43E
  \l B43F
  \l B440
  \l B441
  \l B442
  \l B443
  \l B444
  \l B445
  \l B446
  \l B447
  \l B448
  \l B449
  \l B44A
  \l B44B
  \l B44C
  \l B44D
  \l B44E
  \l B44F
  \l B450
  \l B451
  \l B452
  \l B453
  \l B454
  \l B455
  \l B456
  \l B457
  \l B458
  \l B459
  \l B45A
  \l B45B
  \l B45C
  \l B45D
  \l B45E
  \l B45F
  \l B460
  \l B461
  \l B462
  \l B463
  \l B464
  \l B465
  \l B466
  \l B467
  \l B468
  \l B469
  \l B46A
  \l B46B
  \l B46C
  \l B46D
  \l B46E
  \l B46F
  \l B470
  \l B471
  \l B472
  \l B473
  \l B474
  \l B475
  \l B476
  \l B477
  \l B478
  \l B479
  \l B47A
  \l B47B
  \l B47C
  \l B47D
  \l B47E
  \l B47F
  \l B480
  \l B481
  \l B482
  \l B483
  \l B484
  \l B485
  \l B486
  \l B487
  \l B488
  \l B489
  \l B48A
  \l B48B
  \l B48C
  \l B48D
  \l B48E
  \l B48F
  \l B490
  \l B491
  \l B492
  \l B493
  \l B494
  \l B495
  \l B496
  \l B497
  \l B498
  \l B499
  \l B49A
  \l B49B
  \l B49C
  \l B49D
  \l B49E
  \l B49F
  \l B4A0
  \l B4A1
  \l B4A2
  \l B4A3
  \l B4A4
  \l B4A5
  \l B4A6
  \l B4A7
  \l B4A8
  \l B4A9
  \l B4AA
  \l B4AB
  \l B4AC
  \l B4AD
  \l B4AE
  \l B4AF
  \l B4B0
  \l B4B1
  \l B4B2
  \l B4B3
  \l B4B4
  \l B4B5
  \l B4B6
  \l B4B7
  \l B4B8
  \l B4B9
  \l B4BA
  \l B4BB
  \l B4BC
  \l B4BD
  \l B4BE
  \l B4BF
  \l B4C0
  \l B4C1
  \l B4C2
  \l B4C3
  \l B4C4
  \l B4C5
  \l B4C6
  \l B4C7
  \l B4C8
  \l B4C9
  \l B4CA
  \l B4CB
  \l B4CC
  \l B4CD
  \l B4CE
  \l B4CF
  \l B4D0
  \l B4D1
  \l B4D2
  \l B4D3
  \l B4D4
  \l B4D5
  \l B4D6
  \l B4D7
  \l B4D8
  \l B4D9
  \l B4DA
  \l B4DB
  \l B4DC
  \l B4DD
  \l B4DE
  \l B4DF
  \l B4E0
  \l B4E1
  \l B4E2
  \l B4E3
  \l B4E4
  \l B4E5
  \l B4E6
  \l B4E7
  \l B4E8
  \l B4E9
  \l B4EA
  \l B4EB
  \l B4EC
  \l B4ED
  \l B4EE
  \l B4EF
  \l B4F0
  \l B4F1
  \l B4F2
  \l B4F3
  \l B4F4
  \l B4F5
  \l B4F6
  \l B4F7
  \l B4F8
  \l B4F9
  \l B4FA
  \l B4FB
  \l B4FC
  \l B4FD
  \l B4FE
  \l B4FF
  \l B500
  \l B501
  \l B502
  \l B503
  \l B504
  \l B505
  \l B506
  \l B507
  \l B508
  \l B509
  \l B50A
  \l B50B
  \l B50C
  \l B50D
  \l B50E
  \l B50F
  \l B510
  \l B511
  \l B512
  \l B513
  \l B514
  \l B515
  \l B516
  \l B517
  \l B518
  \l B519
  \l B51A
  \l B51B
  \l B51C
  \l B51D
  \l B51E
  \l B51F
  \l B520
  \l B521
  \l B522
  \l B523
  \l B524
  \l B525
  \l B526
  \l B527
  \l B528
  \l B529
  \l B52A
  \l B52B
  \l B52C
  \l B52D
  \l B52E
  \l B52F
  \l B530
  \l B531
  \l B532
  \l B533
  \l B534
  \l B535
  \l B536
  \l B537
  \l B538
  \l B539
  \l B53A
  \l B53B
  \l B53C
  \l B53D
  \l B53E
  \l B53F
  \l B540
  \l B541
  \l B542
  \l B543
  \l B544
  \l B545
  \l B546
  \l B547
  \l B548
  \l B549
  \l B54A
  \l B54B
  \l B54C
  \l B54D
  \l B54E
  \l B54F
  \l B550
  \l B551
  \l B552
  \l B553
  \l B554
  \l B555
  \l B556
  \l B557
  \l B558
  \l B559
  \l B55A
  \l B55B
  \l B55C
  \l B55D
  \l B55E
  \l B55F
  \l B560
  \l B561
  \l B562
  \l B563
  \l B564
  \l B565
  \l B566
  \l B567
  \l B568
  \l B569
  \l B56A
  \l B56B
  \l B56C
  \l B56D
  \l B56E
  \l B56F
  \l B570
  \l B571
  \l B572
  \l B573
  \l B574
  \l B575
  \l B576
  \l B577
  \l B578
  \l B579
  \l B57A
  \l B57B
  \l B57C
  \l B57D
  \l B57E
  \l B57F
  \l B580
  \l B581
  \l B582
  \l B583
  \l B584
  \l B585
  \l B586
  \l B587
  \l B588
  \l B589
  \l B58A
  \l B58B
  \l B58C
  \l B58D
  \l B58E
  \l B58F
  \l B590
  \l B591
  \l B592
  \l B593
  \l B594
  \l B595
  \l B596
  \l B597
  \l B598
  \l B599
  \l B59A
  \l B59B
  \l B59C
  \l B59D
  \l B59E
  \l B59F
  \l B5A0
  \l B5A1
  \l B5A2
  \l B5A3
  \l B5A4
  \l B5A5
  \l B5A6
  \l B5A7
  \l B5A8
  \l B5A9
  \l B5AA
  \l B5AB
  \l B5AC
  \l B5AD
  \l B5AE
  \l B5AF
  \l B5B0
  \l B5B1
  \l B5B2
  \l B5B3
  \l B5B4
  \l B5B5
  \l B5B6
  \l B5B7
  \l B5B8
  \l B5B9
  \l B5BA
  \l B5BB
  \l B5BC
  \l B5BD
  \l B5BE
  \l B5BF
  \l B5C0
  \l B5C1
  \l B5C2
  \l B5C3
  \l B5C4
  \l B5C5
  \l B5C6
  \l B5C7
  \l B5C8
  \l B5C9
  \l B5CA
  \l B5CB
  \l B5CC
  \l B5CD
  \l B5CE
  \l B5CF
  \l B5D0
  \l B5D1
  \l B5D2
  \l B5D3
  \l B5D4
  \l B5D5
  \l B5D6
  \l B5D7
  \l B5D8
  \l B5D9
  \l B5DA
  \l B5DB
  \l B5DC
  \l B5DD
  \l B5DE
  \l B5DF
  \l B5E0
  \l B5E1
  \l B5E2
  \l B5E3
  \l B5E4
  \l B5E5
  \l B5E6
  \l B5E7
  \l B5E8
  \l B5E9
  \l B5EA
  \l B5EB
  \l B5EC
  \l B5ED
  \l B5EE
  \l B5EF
  \l B5F0
  \l B5F1
  \l B5F2
  \l B5F3
  \l B5F4
  \l B5F5
  \l B5F6
  \l B5F7
  \l B5F8
  \l B5F9
  \l B5FA
  \l B5FB
  \l B5FC
  \l B5FD
  \l B5FE
  \l B5FF
  \l B600
  \l B601
  \l B602
  \l B603
  \l B604
  \l B605
  \l B606
  \l B607
  \l B608
  \l B609
  \l B60A
  \l B60B
  \l B60C
  \l B60D
  \l B60E
  \l B60F
  \l B610
  \l B611
  \l B612
  \l B613
  \l B614
  \l B615
  \l B616
  \l B617
  \l B618
  \l B619
  \l B61A
  \l B61B
  \l B61C
  \l B61D
  \l B61E
  \l B61F
  \l B620
  \l B621
  \l B622
  \l B623
  \l B624
  \l B625
  \l B626
  \l B627
  \l B628
  \l B629
  \l B62A
  \l B62B
  \l B62C
  \l B62D
  \l B62E
  \l B62F
  \l B630
  \l B631
  \l B632
  \l B633
  \l B634
  \l B635
  \l B636
  \l B637
  \l B638
  \l B639
  \l B63A
  \l B63B
  \l B63C
  \l B63D
  \l B63E
  \l B63F
  \l B640
  \l B641
  \l B642
  \l B643
  \l B644
  \l B645
  \l B646
  \l B647
  \l B648
  \l B649
  \l B64A
  \l B64B
  \l B64C
  \l B64D
  \l B64E
  \l B64F
  \l B650
  \l B651
  \l B652
  \l B653
  \l B654
  \l B655
  \l B656
  \l B657
  \l B658
  \l B659
  \l B65A
  \l B65B
  \l B65C
  \l B65D
  \l B65E
  \l B65F
  \l B660
  \l B661
  \l B662
  \l B663
  \l B664
  \l B665
  \l B666
  \l B667
  \l B668
  \l B669
  \l B66A
  \l B66B
  \l B66C
  \l B66D
  \l B66E
  \l B66F
  \l B670
  \l B671
  \l B672
  \l B673
  \l B674
  \l B675
  \l B676
  \l B677
  \l B678
  \l B679
  \l B67A
  \l B67B
  \l B67C
  \l B67D
  \l B67E
  \l B67F
  \l B680
  \l B681
  \l B682
  \l B683
  \l B684
  \l B685
  \l B686
  \l B687
  \l B688
  \l B689
  \l B68A
  \l B68B
  \l B68C
  \l B68D
  \l B68E
  \l B68F
  \l B690
  \l B691
  \l B692
  \l B693
  \l B694
  \l B695
  \l B696
  \l B697
  \l B698
  \l B699
  \l B69A
  \l B69B
  \l B69C
  \l B69D
  \l B69E
  \l B69F
  \l B6A0
  \l B6A1
  \l B6A2
  \l B6A3
  \l B6A4
  \l B6A5
  \l B6A6
  \l B6A7
  \l B6A8
  \l B6A9
  \l B6AA
  \l B6AB
  \l B6AC
  \l B6AD
  \l B6AE
  \l B6AF
  \l B6B0
  \l B6B1
  \l B6B2
  \l B6B3
  \l B6B4
  \l B6B5
  \l B6B6
  \l B6B7
  \l B6B8
  \l B6B9
  \l B6BA
  \l B6BB
  \l B6BC
  \l B6BD
  \l B6BE
  \l B6BF
  \l B6C0
  \l B6C1
  \l B6C2
  \l B6C3
  \l B6C4
  \l B6C5
  \l B6C6
  \l B6C7
  \l B6C8
  \l B6C9
  \l B6CA
  \l B6CB
  \l B6CC
  \l B6CD
  \l B6CE
  \l B6CF
  \l B6D0
  \l B6D1
  \l B6D2
  \l B6D3
  \l B6D4
  \l B6D5
  \l B6D6
  \l B6D7
  \l B6D8
  \l B6D9
  \l B6DA
  \l B6DB
  \l B6DC
  \l B6DD
  \l B6DE
  \l B6DF
  \l B6E0
  \l B6E1
  \l B6E2
  \l B6E3
  \l B6E4
  \l B6E5
  \l B6E6
  \l B6E7
  \l B6E8
  \l B6E9
  \l B6EA
  \l B6EB
  \l B6EC
  \l B6ED
  \l B6EE
  \l B6EF
  \l B6F0
  \l B6F1
  \l B6F2
  \l B6F3
  \l B6F4
  \l B6F5
  \l B6F6
  \l B6F7
  \l B6F8
  \l B6F9
  \l B6FA
  \l B6FB
  \l B6FC
  \l B6FD
  \l B6FE
  \l B6FF
  \l B700
  \l B701
  \l B702
  \l B703
  \l B704
  \l B705
  \l B706
  \l B707
  \l B708
  \l B709
  \l B70A
  \l B70B
  \l B70C
  \l B70D
  \l B70E
  \l B70F
  \l B710
  \l B711
  \l B712
  \l B713
  \l B714
  \l B715
  \l B716
  \l B717
  \l B718
  \l B719
  \l B71A
  \l B71B
  \l B71C
  \l B71D
  \l B71E
  \l B71F
  \l B720
  \l B721
  \l B722
  \l B723
  \l B724
  \l B725
  \l B726
  \l B727
  \l B728
  \l B729
  \l B72A
  \l B72B
  \l B72C
  \l B72D
  \l B72E
  \l B72F
  \l B730
  \l B731
  \l B732
  \l B733
  \l B734
  \l B735
  \l B736
  \l B737
  \l B738
  \l B739
  \l B73A
  \l B73B
  \l B73C
  \l B73D
  \l B73E
  \l B73F
  \l B740
  \l B741
  \l B742
  \l B743
  \l B744
  \l B745
  \l B746
  \l B747
  \l B748
  \l B749
  \l B74A
  \l B74B
  \l B74C
  \l B74D
  \l B74E
  \l B74F
  \l B750
  \l B751
  \l B752
  \l B753
  \l B754
  \l B755
  \l B756
  \l B757
  \l B758
  \l B759
  \l B75A
  \l B75B
  \l B75C
  \l B75D
  \l B75E
  \l B75F
  \l B760
  \l B761
  \l B762
  \l B763
  \l B764
  \l B765
  \l B766
  \l B767
  \l B768
  \l B769
  \l B76A
  \l B76B
  \l B76C
  \l B76D
  \l B76E
  \l B76F
  \l B770
  \l B771
  \l B772
  \l B773
  \l B774
  \l B775
  \l B776
  \l B777
  \l B778
  \l B779
  \l B77A
  \l B77B
  \l B77C
  \l B77D
  \l B77E
  \l B77F
  \l B780
  \l B781
  \l B782
  \l B783
  \l B784
  \l B785
  \l B786
  \l B787
  \l B788
  \l B789
  \l B78A
  \l B78B
  \l B78C
  \l B78D
  \l B78E
  \l B78F
  \l B790
  \l B791
  \l B792
  \l B793
  \l B794
  \l B795
  \l B796
  \l B797
  \l B798
  \l B799
  \l B79A
  \l B79B
  \l B79C
  \l B79D
  \l B79E
  \l B79F
  \l B7A0
  \l B7A1
  \l B7A2
  \l B7A3
  \l B7A4
  \l B7A5
  \l B7A6
  \l B7A7
  \l B7A8
  \l B7A9
  \l B7AA
  \l B7AB
  \l B7AC
  \l B7AD
  \l B7AE
  \l B7AF
  \l B7B0
  \l B7B1
  \l B7B2
  \l B7B3
  \l B7B4
  \l B7B5
  \l B7B6
  \l B7B7
  \l B7B8
  \l B7B9
  \l B7BA
  \l B7BB
  \l B7BC
  \l B7BD
  \l B7BE
  \l B7BF
  \l B7C0
  \l B7C1
  \l B7C2
  \l B7C3
  \l B7C4
  \l B7C5
  \l B7C6
  \l B7C7
  \l B7C8
  \l B7C9
  \l B7CA
  \l B7CB
  \l B7CC
  \l B7CD
  \l B7CE
  \l B7CF
  \l B7D0
  \l B7D1
  \l B7D2
  \l B7D3
  \l B7D4
  \l B7D5
  \l B7D6
  \l B7D7
  \l B7D8
  \l B7D9
  \l B7DA
  \l B7DB
  \l B7DC
  \l B7DD
  \l B7DE
  \l B7DF
  \l B7E0
  \l B7E1
  \l B7E2
  \l B7E3
  \l B7E4
  \l B7E5
  \l B7E6
  \l B7E7
  \l B7E8
  \l B7E9
  \l B7EA
  \l B7EB
  \l B7EC
  \l B7ED
  \l B7EE
  \l B7EF
  \l B7F0
  \l B7F1
  \l B7F2
  \l B7F3
  \l B7F4
  \l B7F5
  \l B7F6
  \l B7F7
  \l B7F8
  \l B7F9
  \l B7FA
  \l B7FB
  \l B7FC
  \l B7FD
  \l B7FE
  \l B7FF
  \l B800
  \l B801
  \l B802
  \l B803
  \l B804
  \l B805
  \l B806
  \l B807
  \l B808
  \l B809
  \l B80A
  \l B80B
  \l B80C
  \l B80D
  \l B80E
  \l B80F
  \l B810
  \l B811
  \l B812
  \l B813
  \l B814
  \l B815
  \l B816
  \l B817
  \l B818
  \l B819
  \l B81A
  \l B81B
  \l B81C
  \l B81D
  \l B81E
  \l B81F
  \l B820
  \l B821
  \l B822
  \l B823
  \l B824
  \l B825
  \l B826
  \l B827
  \l B828
  \l B829
  \l B82A
  \l B82B
  \l B82C
  \l B82D
  \l B82E
  \l B82F
  \l B830
  \l B831
  \l B832
  \l B833
  \l B834
  \l B835
  \l B836
  \l B837
  \l B838
  \l B839
  \l B83A
  \l B83B
  \l B83C
  \l B83D
  \l B83E
  \l B83F
  \l B840
  \l B841
  \l B842
  \l B843
  \l B844
  \l B845
  \l B846
  \l B847
  \l B848
  \l B849
  \l B84A
  \l B84B
  \l B84C
  \l B84D
  \l B84E
  \l B84F
  \l B850
  \l B851
  \l B852
  \l B853
  \l B854
  \l B855
  \l B856
  \l B857
  \l B858
  \l B859
  \l B85A
  \l B85B
  \l B85C
  \l B85D
  \l B85E
  \l B85F
  \l B860
  \l B861
  \l B862
  \l B863
  \l B864
  \l B865
  \l B866
  \l B867
  \l B868
  \l B869
  \l B86A
  \l B86B
  \l B86C
  \l B86D
  \l B86E
  \l B86F
  \l B870
  \l B871
  \l B872
  \l B873
  \l B874
  \l B875
  \l B876
  \l B877
  \l B878
  \l B879
  \l B87A
  \l B87B
  \l B87C
  \l B87D
  \l B87E
  \l B87F
  \l B880
  \l B881
  \l B882
  \l B883
  \l B884
  \l B885
  \l B886
  \l B887
  \l B888
  \l B889
  \l B88A
  \l B88B
  \l B88C
  \l B88D
  \l B88E
  \l B88F
  \l B890
  \l B891
  \l B892
  \l B893
  \l B894
  \l B895
  \l B896
  \l B897
  \l B898
  \l B899
  \l B89A
  \l B89B
  \l B89C
  \l B89D
  \l B89E
  \l B89F
  \l B8A0
  \l B8A1
  \l B8A2
  \l B8A3
  \l B8A4
  \l B8A5
  \l B8A6
  \l B8A7
  \l B8A8
  \l B8A9
  \l B8AA
  \l B8AB
  \l B8AC
  \l B8AD
  \l B8AE
  \l B8AF
  \l B8B0
  \l B8B1
  \l B8B2
  \l B8B3
  \l B8B4
  \l B8B5
  \l B8B6
  \l B8B7
  \l B8B8
  \l B8B9
  \l B8BA
  \l B8BB
  \l B8BC
  \l B8BD
  \l B8BE
  \l B8BF
  \l B8C0
  \l B8C1
  \l B8C2
  \l B8C3
  \l B8C4
  \l B8C5
  \l B8C6
  \l B8C7
  \l B8C8
  \l B8C9
  \l B8CA
  \l B8CB
  \l B8CC
  \l B8CD
  \l B8CE
  \l B8CF
  \l B8D0
  \l B8D1
  \l B8D2
  \l B8D3
  \l B8D4
  \l B8D5
  \l B8D6
  \l B8D7
  \l B8D8
  \l B8D9
  \l B8DA
  \l B8DB
  \l B8DC
  \l B8DD
  \l B8DE
  \l B8DF
  \l B8E0
  \l B8E1
  \l B8E2
  \l B8E3
  \l B8E4
  \l B8E5
  \l B8E6
  \l B8E7
  \l B8E8
  \l B8E9
  \l B8EA
  \l B8EB
  \l B8EC
  \l B8ED
  \l B8EE
  \l B8EF
  \l B8F0
  \l B8F1
  \l B8F2
  \l B8F3
  \l B8F4
  \l B8F5
  \l B8F6
  \l B8F7
  \l B8F8
  \l B8F9
  \l B8FA
  \l B8FB
  \l B8FC
  \l B8FD
  \l B8FE
  \l B8FF
  \l B900
  \l B901
  \l B902
  \l B903
  \l B904
  \l B905
  \l B906
  \l B907
  \l B908
  \l B909
  \l B90A
  \l B90B
  \l B90C
  \l B90D
  \l B90E
  \l B90F
  \l B910
  \l B911
  \l B912
  \l B913
  \l B914
  \l B915
  \l B916
  \l B917
  \l B918
  \l B919
  \l B91A
  \l B91B
  \l B91C
  \l B91D
  \l B91E
  \l B91F
  \l B920
  \l B921
  \l B922
  \l B923
  \l B924
  \l B925
  \l B926
  \l B927
  \l B928
  \l B929
  \l B92A
  \l B92B
  \l B92C
  \l B92D
  \l B92E
  \l B92F
  \l B930
  \l B931
  \l B932
  \l B933
  \l B934
  \l B935
  \l B936
  \l B937
  \l B938
  \l B939
  \l B93A
  \l B93B
  \l B93C
  \l B93D
  \l B93E
  \l B93F
  \l B940
  \l B941
  \l B942
  \l B943
  \l B944
  \l B945
  \l B946
  \l B947
  \l B948
  \l B949
  \l B94A
  \l B94B
  \l B94C
  \l B94D
  \l B94E
  \l B94F
  \l B950
  \l B951
  \l B952
  \l B953
  \l B954
  \l B955
  \l B956
  \l B957
  \l B958
  \l B959
  \l B95A
  \l B95B
  \l B95C
  \l B95D
  \l B95E
  \l B95F
  \l B960
  \l B961
  \l B962
  \l B963
  \l B964
  \l B965
  \l B966
  \l B967
  \l B968
  \l B969
  \l B96A
  \l B96B
  \l B96C
  \l B96D
  \l B96E
  \l B96F
  \l B970
  \l B971
  \l B972
  \l B973
  \l B974
  \l B975
  \l B976
  \l B977
  \l B978
  \l B979
  \l B97A
  \l B97B
  \l B97C
  \l B97D
  \l B97E
  \l B97F
  \l B980
  \l B981
  \l B982
  \l B983
  \l B984
  \l B985
  \l B986
  \l B987
  \l B988
  \l B989
  \l B98A
  \l B98B
  \l B98C
  \l B98D
  \l B98E
  \l B98F
  \l B990
  \l B991
  \l B992
  \l B993
  \l B994
  \l B995
  \l B996
  \l B997
  \l B998
  \l B999
  \l B99A
  \l B99B
  \l B99C
  \l B99D
  \l B99E
  \l B99F
  \l B9A0
  \l B9A1
  \l B9A2
  \l B9A3
  \l B9A4
  \l B9A5
  \l B9A6
  \l B9A7
  \l B9A8
  \l B9A9
  \l B9AA
  \l B9AB
  \l B9AC
  \l B9AD
  \l B9AE
  \l B9AF
  \l B9B0
  \l B9B1
  \l B9B2
  \l B9B3
  \l B9B4
  \l B9B5
  \l B9B6
  \l B9B7
  \l B9B8
  \l B9B9
  \l B9BA
  \l B9BB
  \l B9BC
  \l B9BD
  \l B9BE
  \l B9BF
  \l B9C0
  \l B9C1
  \l B9C2
  \l B9C3
  \l B9C4
  \l B9C5
  \l B9C6
  \l B9C7
  \l B9C8
  \l B9C9
  \l B9CA
  \l B9CB
  \l B9CC
  \l B9CD
  \l B9CE
  \l B9CF
  \l B9D0
  \l B9D1
  \l B9D2
  \l B9D3
  \l B9D4
  \l B9D5
  \l B9D6
  \l B9D7
  \l B9D8
  \l B9D9
  \l B9DA
  \l B9DB
  \l B9DC
  \l B9DD
  \l B9DE
  \l B9DF
  \l B9E0
  \l B9E1
  \l B9E2
  \l B9E3
  \l B9E4
  \l B9E5
  \l B9E6
  \l B9E7
  \l B9E8
  \l B9E9
  \l B9EA
  \l B9EB
  \l B9EC
  \l B9ED
  \l B9EE
  \l B9EF
  \l B9F0
  \l B9F1
  \l B9F2
  \l B9F3
  \l B9F4
  \l B9F5
  \l B9F6
  \l B9F7
  \l B9F8
  \l B9F9
  \l B9FA
  \l B9FB
  \l B9FC
  \l B9FD
  \l B9FE
  \l B9FF
  \l BA00
  \l BA01
  \l BA02
  \l BA03
  \l BA04
  \l BA05
  \l BA06
  \l BA07
  \l BA08
  \l BA09
  \l BA0A
  \l BA0B
  \l BA0C
  \l BA0D
  \l BA0E
  \l BA0F
  \l BA10
  \l BA11
  \l BA12
  \l BA13
  \l BA14
  \l BA15
  \l BA16
  \l BA17
  \l BA18
  \l BA19
  \l BA1A
  \l BA1B
  \l BA1C
  \l BA1D
  \l BA1E
  \l BA1F
  \l BA20
  \l BA21
  \l BA22
  \l BA23
  \l BA24
  \l BA25
  \l BA26
  \l BA27
  \l BA28
  \l BA29
  \l BA2A
  \l BA2B
  \l BA2C
  \l BA2D
  \l BA2E
  \l BA2F
  \l BA30
  \l BA31
  \l BA32
  \l BA33
  \l BA34
  \l BA35
  \l BA36
  \l BA37
  \l BA38
  \l BA39
  \l BA3A
  \l BA3B
  \l BA3C
  \l BA3D
  \l BA3E
  \l BA3F
  \l BA40
  \l BA41
  \l BA42
  \l BA43
  \l BA44
  \l BA45
  \l BA46
  \l BA47
  \l BA48
  \l BA49
  \l BA4A
  \l BA4B
  \l BA4C
  \l BA4D
  \l BA4E
  \l BA4F
  \l BA50
  \l BA51
  \l BA52
  \l BA53
  \l BA54
  \l BA55
  \l BA56
  \l BA57
  \l BA58
  \l BA59
  \l BA5A
  \l BA5B
  \l BA5C
  \l BA5D
  \l BA5E
  \l BA5F
  \l BA60
  \l BA61
  \l BA62
  \l BA63
  \l BA64
  \l BA65
  \l BA66
  \l BA67
  \l BA68
  \l BA69
  \l BA6A
  \l BA6B
  \l BA6C
  \l BA6D
  \l BA6E
  \l BA6F
  \l BA70
  \l BA71
  \l BA72
  \l BA73
  \l BA74
  \l BA75
  \l BA76
  \l BA77
  \l BA78
  \l BA79
  \l BA7A
  \l BA7B
  \l BA7C
  \l BA7D
  \l BA7E
  \l BA7F
  \l BA80
  \l BA81
  \l BA82
  \l BA83
  \l BA84
  \l BA85
  \l BA86
  \l BA87
  \l BA88
  \l BA89
  \l BA8A
  \l BA8B
  \l BA8C
  \l BA8D
  \l BA8E
  \l BA8F
  \l BA90
  \l BA91
  \l BA92
  \l BA93
  \l BA94
  \l BA95
  \l BA96
  \l BA97
  \l BA98
  \l BA99
  \l BA9A
  \l BA9B
  \l BA9C
  \l BA9D
  \l BA9E
  \l BA9F
  \l BAA0
  \l BAA1
  \l BAA2
  \l BAA3
  \l BAA4
  \l BAA5
  \l BAA6
  \l BAA7
  \l BAA8
  \l BAA9
  \l BAAA
  \l BAAB
  \l BAAC
  \l BAAD
  \l BAAE
  \l BAAF
  \l BAB0
  \l BAB1
  \l BAB2
  \l BAB3
  \l BAB4
  \l BAB5
  \l BAB6
  \l BAB7
  \l BAB8
  \l BAB9
  \l BABA
  \l BABB
  \l BABC
  \l BABD
  \l BABE
  \l BABF
  \l BAC0
  \l BAC1
  \l BAC2
  \l BAC3
  \l BAC4
  \l BAC5
  \l BAC6
  \l BAC7
  \l BAC8
  \l BAC9
  \l BACA
  \l BACB
  \l BACC
  \l BACD
  \l BACE
  \l BACF
  \l BAD0
  \l BAD1
  \l BAD2
  \l BAD3
  \l BAD4
  \l BAD5
  \l BAD6
  \l BAD7
  \l BAD8
  \l BAD9
  \l BADA
  \l BADB
  \l BADC
  \l BADD
  \l BADE
  \l BADF
  \l BAE0
  \l BAE1
  \l BAE2
  \l BAE3
  \l BAE4
  \l BAE5
  \l BAE6
  \l BAE7
  \l BAE8
  \l BAE9
  \l BAEA
  \l BAEB
  \l BAEC
  \l BAED
  \l BAEE
  \l BAEF
  \l BAF0
  \l BAF1
  \l BAF2
  \l BAF3
  \l BAF4
  \l BAF5
  \l BAF6
  \l BAF7
  \l BAF8
  \l BAF9
  \l BAFA
  \l BAFB
  \l BAFC
  \l BAFD
  \l BAFE
  \l BAFF
  \l BB00
  \l BB01
  \l BB02
  \l BB03
  \l BB04
  \l BB05
  \l BB06
  \l BB07
  \l BB08
  \l BB09
  \l BB0A
  \l BB0B
  \l BB0C
  \l BB0D
  \l BB0E
  \l BB0F
  \l BB10
  \l BB11
  \l BB12
  \l BB13
  \l BB14
  \l BB15
  \l BB16
  \l BB17
  \l BB18
  \l BB19
  \l BB1A
  \l BB1B
  \l BB1C
  \l BB1D
  \l BB1E
  \l BB1F
  \l BB20
  \l BB21
  \l BB22
  \l BB23
  \l BB24
  \l BB25
  \l BB26
  \l BB27
  \l BB28
  \l BB29
  \l BB2A
  \l BB2B
  \l BB2C
  \l BB2D
  \l BB2E
  \l BB2F
  \l BB30
  \l BB31
  \l BB32
  \l BB33
  \l BB34
  \l BB35
  \l BB36
  \l BB37
  \l BB38
  \l BB39
  \l BB3A
  \l BB3B
  \l BB3C
  \l BB3D
  \l BB3E
  \l BB3F
  \l BB40
  \l BB41
  \l BB42
  \l BB43
  \l BB44
  \l BB45
  \l BB46
  \l BB47
  \l BB48
  \l BB49
  \l BB4A
  \l BB4B
  \l BB4C
  \l BB4D
  \l BB4E
  \l BB4F
  \l BB50
  \l BB51
  \l BB52
  \l BB53
  \l BB54
  \l BB55
  \l BB56
  \l BB57
  \l BB58
  \l BB59
  \l BB5A
  \l BB5B
  \l BB5C
  \l BB5D
  \l BB5E
  \l BB5F
  \l BB60
  \l BB61
  \l BB62
  \l BB63
  \l BB64
  \l BB65
  \l BB66
  \l BB67
  \l BB68
  \l BB69
  \l BB6A
  \l BB6B
  \l BB6C
  \l BB6D
  \l BB6E
  \l BB6F
  \l BB70
  \l BB71
  \l BB72
  \l BB73
  \l BB74
  \l BB75
  \l BB76
  \l BB77
  \l BB78
  \l BB79
  \l BB7A
  \l BB7B
  \l BB7C
  \l BB7D
  \l BB7E
  \l BB7F
  \l BB80
  \l BB81
  \l BB82
  \l BB83
  \l BB84
  \l BB85
  \l BB86
  \l BB87
  \l BB88
  \l BB89
  \l BB8A
  \l BB8B
  \l BB8C
  \l BB8D
  \l BB8E
  \l BB8F
  \l BB90
  \l BB91
  \l BB92
  \l BB93
  \l BB94
  \l BB95
  \l BB96
  \l BB97
  \l BB98
  \l BB99
  \l BB9A
  \l BB9B
  \l BB9C
  \l BB9D
  \l BB9E
  \l BB9F
  \l BBA0
  \l BBA1
  \l BBA2
  \l BBA3
  \l BBA4
  \l BBA5
  \l BBA6
  \l BBA7
  \l BBA8
  \l BBA9
  \l BBAA
  \l BBAB
  \l BBAC
  \l BBAD
  \l BBAE
  \l BBAF
  \l BBB0
  \l BBB1
  \l BBB2
  \l BBB3
  \l BBB4
  \l BBB5
  \l BBB6
  \l BBB7
  \l BBB8
  \l BBB9
  \l BBBA
  \l BBBB
  \l BBBC
  \l BBBD
  \l BBBE
  \l BBBF
  \l BBC0
  \l BBC1
  \l BBC2
  \l BBC3
  \l BBC4
  \l BBC5
  \l BBC6
  \l BBC7
  \l BBC8
  \l BBC9
  \l BBCA
  \l BBCB
  \l BBCC
  \l BBCD
  \l BBCE
  \l BBCF
  \l BBD0
  \l BBD1
  \l BBD2
  \l BBD3
  \l BBD4
  \l BBD5
  \l BBD6
  \l BBD7
  \l BBD8
  \l BBD9
  \l BBDA
  \l BBDB
  \l BBDC
  \l BBDD
  \l BBDE
  \l BBDF
  \l BBE0
  \l BBE1
  \l BBE2
  \l BBE3
  \l BBE4
  \l BBE5
  \l BBE6
  \l BBE7
  \l BBE8
  \l BBE9
  \l BBEA
  \l BBEB
  \l BBEC
  \l BBED
  \l BBEE
  \l BBEF
  \l BBF0
  \l BBF1
  \l BBF2
  \l BBF3
  \l BBF4
  \l BBF5
  \l BBF6
  \l BBF7
  \l BBF8
  \l BBF9
  \l BBFA
  \l BBFB
  \l BBFC
  \l BBFD
  \l BBFE
  \l BBFF
  \l BC00
  \l BC01
  \l BC02
  \l BC03
  \l BC04
  \l BC05
  \l BC06
  \l BC07
  \l BC08
  \l BC09
  \l BC0A
  \l BC0B
  \l BC0C
  \l BC0D
  \l BC0E
  \l BC0F
  \l BC10
  \l BC11
  \l BC12
  \l BC13
  \l BC14
  \l BC15
  \l BC16
  \l BC17
  \l BC18
  \l BC19
  \l BC1A
  \l BC1B
  \l BC1C
  \l BC1D
  \l BC1E
  \l BC1F
  \l BC20
  \l BC21
  \l BC22
  \l BC23
  \l BC24
  \l BC25
  \l BC26
  \l BC27
  \l BC28
  \l BC29
  \l BC2A
  \l BC2B
  \l BC2C
  \l BC2D
  \l BC2E
  \l BC2F
  \l BC30
  \l BC31
  \l BC32
  \l BC33
  \l BC34
  \l BC35
  \l BC36
  \l BC37
  \l BC38
  \l BC39
  \l BC3A
  \l BC3B
  \l BC3C
  \l BC3D
  \l BC3E
  \l BC3F
  \l BC40
  \l BC41
  \l BC42
  \l BC43
  \l BC44
  \l BC45
  \l BC46
  \l BC47
  \l BC48
  \l BC49
  \l BC4A
  \l BC4B
  \l BC4C
  \l BC4D
  \l BC4E
  \l BC4F
  \l BC50
  \l BC51
  \l BC52
  \l BC53
  \l BC54
  \l BC55
  \l BC56
  \l BC57
  \l BC58
  \l BC59
  \l BC5A
  \l BC5B
  \l BC5C
  \l BC5D
  \l BC5E
  \l BC5F
  \l BC60
  \l BC61
  \l BC62
  \l BC63
  \l BC64
  \l BC65
  \l BC66
  \l BC67
  \l BC68
  \l BC69
  \l BC6A
  \l BC6B
  \l BC6C
  \l BC6D
  \l BC6E
  \l BC6F
  \l BC70
  \l BC71
  \l BC72
  \l BC73
  \l BC74
  \l BC75
  \l BC76
  \l BC77
  \l BC78
  \l BC79
  \l BC7A
  \l BC7B
  \l BC7C
  \l BC7D
  \l BC7E
  \l BC7F
  \l BC80
  \l BC81
  \l BC82
  \l BC83
  \l BC84
  \l BC85
  \l BC86
  \l BC87
  \l BC88
  \l BC89
  \l BC8A
  \l BC8B
  \l BC8C
  \l BC8D
  \l BC8E
  \l BC8F
  \l BC90
  \l BC91
  \l BC92
  \l BC93
  \l BC94
  \l BC95
  \l BC96
  \l BC97
  \l BC98
  \l BC99
  \l BC9A
  \l BC9B
  \l BC9C
  \l BC9D
  \l BC9E
  \l BC9F
  \l BCA0
  \l BCA1
  \l BCA2
  \l BCA3
  \l BCA4
  \l BCA5
  \l BCA6
  \l BCA7
  \l BCA8
  \l BCA9
  \l BCAA
  \l BCAB
  \l BCAC
  \l BCAD
  \l BCAE
  \l BCAF
  \l BCB0
  \l BCB1
  \l BCB2
  \l BCB3
  \l BCB4
  \l BCB5
  \l BCB6
  \l BCB7
  \l BCB8
  \l BCB9
  \l BCBA
  \l BCBB
  \l BCBC
  \l BCBD
  \l BCBE
  \l BCBF
  \l BCC0
  \l BCC1
  \l BCC2
  \l BCC3
  \l BCC4
  \l BCC5
  \l BCC6
  \l BCC7
  \l BCC8
  \l BCC9
  \l BCCA
  \l BCCB
  \l BCCC
  \l BCCD
  \l BCCE
  \l BCCF
  \l BCD0
  \l BCD1
  \l BCD2
  \l BCD3
  \l BCD4
  \l BCD5
  \l BCD6
  \l BCD7
  \l BCD8
  \l BCD9
  \l BCDA
  \l BCDB
  \l BCDC
  \l BCDD
  \l BCDE
  \l BCDF
  \l BCE0
  \l BCE1
  \l BCE2
  \l BCE3
  \l BCE4
  \l BCE5
  \l BCE6
  \l BCE7
  \l BCE8
  \l BCE9
  \l BCEA
  \l BCEB
  \l BCEC
  \l BCED
  \l BCEE
  \l BCEF
  \l BCF0
  \l BCF1
  \l BCF2
  \l BCF3
  \l BCF4
  \l BCF5
  \l BCF6
  \l BCF7
  \l BCF8
  \l BCF9
  \l BCFA
  \l BCFB
  \l BCFC
  \l BCFD
  \l BCFE
  \l BCFF
  \l BD00
  \l BD01
  \l BD02
  \l BD03
  \l BD04
  \l BD05
  \l BD06
  \l BD07
  \l BD08
  \l BD09
  \l BD0A
  \l BD0B
  \l BD0C
  \l BD0D
  \l BD0E
  \l BD0F
  \l BD10
  \l BD11
  \l BD12
  \l BD13
  \l BD14
  \l BD15
  \l BD16
  \l BD17
  \l BD18
  \l BD19
  \l BD1A
  \l BD1B
  \l BD1C
  \l BD1D
  \l BD1E
  \l BD1F
  \l BD20
  \l BD21
  \l BD22
  \l BD23
  \l BD24
  \l BD25
  \l BD26
  \l BD27
  \l BD28
  \l BD29
  \l BD2A
  \l BD2B
  \l BD2C
  \l BD2D
  \l BD2E
  \l BD2F
  \l BD30
  \l BD31
  \l BD32
  \l BD33
  \l BD34
  \l BD35
  \l BD36
  \l BD37
  \l BD38
  \l BD39
  \l BD3A
  \l BD3B
  \l BD3C
  \l BD3D
  \l BD3E
  \l BD3F
  \l BD40
  \l BD41
  \l BD42
  \l BD43
  \l BD44
  \l BD45
  \l BD46
  \l BD47
  \l BD48
  \l BD49
  \l BD4A
  \l BD4B
  \l BD4C
  \l BD4D
  \l BD4E
  \l BD4F
  \l BD50
  \l BD51
  \l BD52
  \l BD53
  \l BD54
  \l BD55
  \l BD56
  \l BD57
  \l BD58
  \l BD59
  \l BD5A
  \l BD5B
  \l BD5C
  \l BD5D
  \l BD5E
  \l BD5F
  \l BD60
  \l BD61
  \l BD62
  \l BD63
  \l BD64
  \l BD65
  \l BD66
  \l BD67
  \l BD68
  \l BD69
  \l BD6A
  \l BD6B
  \l BD6C
  \l BD6D
  \l BD6E
  \l BD6F
  \l BD70
  \l BD71
  \l BD72
  \l BD73
  \l BD74
  \l BD75
  \l BD76
  \l BD77
  \l BD78
  \l BD79
  \l BD7A
  \l BD7B
  \l BD7C
  \l BD7D
  \l BD7E
  \l BD7F
  \l BD80
  \l BD81
  \l BD82
  \l BD83
  \l BD84
  \l BD85
  \l BD86
  \l BD87
  \l BD88
  \l BD89
  \l BD8A
  \l BD8B
  \l BD8C
  \l BD8D
  \l BD8E
  \l BD8F
  \l BD90
  \l BD91
  \l BD92
  \l BD93
  \l BD94
  \l BD95
  \l BD96
  \l BD97
  \l BD98
  \l BD99
  \l BD9A
  \l BD9B
  \l BD9C
  \l BD9D
  \l BD9E
  \l BD9F
  \l BDA0
  \l BDA1
  \l BDA2
  \l BDA3
  \l BDA4
  \l BDA5
  \l BDA6
  \l BDA7
  \l BDA8
  \l BDA9
  \l BDAA
  \l BDAB
  \l BDAC
  \l BDAD
  \l BDAE
  \l BDAF
  \l BDB0
  \l BDB1
  \l BDB2
  \l BDB3
  \l BDB4
  \l BDB5
  \l BDB6
  \l BDB7
  \l BDB8
  \l BDB9
  \l BDBA
  \l BDBB
  \l BDBC
  \l BDBD
  \l BDBE
  \l BDBF
  \l BDC0
  \l BDC1
  \l BDC2
  \l BDC3
  \l BDC4
  \l BDC5
  \l BDC6
  \l BDC7
  \l BDC8
  \l BDC9
  \l BDCA
  \l BDCB
  \l BDCC
  \l BDCD
  \l BDCE
  \l BDCF
  \l BDD0
  \l BDD1
  \l BDD2
  \l BDD3
  \l BDD4
  \l BDD5
  \l BDD6
  \l BDD7
  \l BDD8
  \l BDD9
  \l BDDA
  \l BDDB
  \l BDDC
  \l BDDD
  \l BDDE
  \l BDDF
  \l BDE0
  \l BDE1
  \l BDE2
  \l BDE3
  \l BDE4
  \l BDE5
  \l BDE6
  \l BDE7
  \l BDE8
  \l BDE9
  \l BDEA
  \l BDEB
  \l BDEC
  \l BDED
  \l BDEE
  \l BDEF
  \l BDF0
  \l BDF1
  \l BDF2
  \l BDF3
  \l BDF4
  \l BDF5
  \l BDF6
  \l BDF7
  \l BDF8
  \l BDF9
  \l BDFA
  \l BDFB
  \l BDFC
  \l BDFD
  \l BDFE
  \l BDFF
  \l BE00
  \l BE01
  \l BE02
  \l BE03
  \l BE04
  \l BE05
  \l BE06
  \l BE07
  \l BE08
  \l BE09
  \l BE0A
  \l BE0B
  \l BE0C
  \l BE0D
  \l BE0E
  \l BE0F
  \l BE10
  \l BE11
  \l BE12
  \l BE13
  \l BE14
  \l BE15
  \l BE16
  \l BE17
  \l BE18
  \l BE19
  \l BE1A
  \l BE1B
  \l BE1C
  \l BE1D
  \l BE1E
  \l BE1F
  \l BE20
  \l BE21
  \l BE22
  \l BE23
  \l BE24
  \l BE25
  \l BE26
  \l BE27
  \l BE28
  \l BE29
  \l BE2A
  \l BE2B
  \l BE2C
  \l BE2D
  \l BE2E
  \l BE2F
  \l BE30
  \l BE31
  \l BE32
  \l BE33
  \l BE34
  \l BE35
  \l BE36
  \l BE37
  \l BE38
  \l BE39
  \l BE3A
  \l BE3B
  \l BE3C
  \l BE3D
  \l BE3E
  \l BE3F
  \l BE40
  \l BE41
  \l BE42
  \l BE43
  \l BE44
  \l BE45
  \l BE46
  \l BE47
  \l BE48
  \l BE49
  \l BE4A
  \l BE4B
  \l BE4C
  \l BE4D
  \l BE4E
  \l BE4F
  \l BE50
  \l BE51
  \l BE52
  \l BE53
  \l BE54
  \l BE55
  \l BE56
  \l BE57
  \l BE58
  \l BE59
  \l BE5A
  \l BE5B
  \l BE5C
  \l BE5D
  \l BE5E
  \l BE5F
  \l BE60
  \l BE61
  \l BE62
  \l BE63
  \l BE64
  \l BE65
  \l BE66
  \l BE67
  \l BE68
  \l BE69
  \l BE6A
  \l BE6B
  \l BE6C
  \l BE6D
  \l BE6E
  \l BE6F
  \l BE70
  \l BE71
  \l BE72
  \l BE73
  \l BE74
  \l BE75
  \l BE76
  \l BE77
  \l BE78
  \l BE79
  \l BE7A
  \l BE7B
  \l BE7C
  \l BE7D
  \l BE7E
  \l BE7F
  \l BE80
  \l BE81
  \l BE82
  \l BE83
  \l BE84
  \l BE85
  \l BE86
  \l BE87
  \l BE88
  \l BE89
  \l BE8A
  \l BE8B
  \l BE8C
  \l BE8D
  \l BE8E
  \l BE8F
  \l BE90
  \l BE91
  \l BE92
  \l BE93
  \l BE94
  \l BE95
  \l BE96
  \l BE97
  \l BE98
  \l BE99
  \l BE9A
  \l BE9B
  \l BE9C
  \l BE9D
  \l BE9E
  \l BE9F
  \l BEA0
  \l BEA1
  \l BEA2
  \l BEA3
  \l BEA4
  \l BEA5
  \l BEA6
  \l BEA7
  \l BEA8
  \l BEA9
  \l BEAA
  \l BEAB
  \l BEAC
  \l BEAD
  \l BEAE
  \l BEAF
  \l BEB0
  \l BEB1
  \l BEB2
  \l BEB3
  \l BEB4
  \l BEB5
  \l BEB6
  \l BEB7
  \l BEB8
  \l BEB9
  \l BEBA
  \l BEBB
  \l BEBC
  \l BEBD
  \l BEBE
  \l BEBF
  \l BEC0
  \l BEC1
  \l BEC2
  \l BEC3
  \l BEC4
  \l BEC5
  \l BEC6
  \l BEC7
  \l BEC8
  \l BEC9
  \l BECA
  \l BECB
  \l BECC
  \l BECD
  \l BECE
  \l BECF
  \l BED0
  \l BED1
  \l BED2
  \l BED3
  \l BED4
  \l BED5
  \l BED6
  \l BED7
  \l BED8
  \l BED9
  \l BEDA
  \l BEDB
  \l BEDC
  \l BEDD
  \l BEDE
  \l BEDF
  \l BEE0
  \l BEE1
  \l BEE2
  \l BEE3
  \l BEE4
  \l BEE5
  \l BEE6
  \l BEE7
  \l BEE8
  \l BEE9
  \l BEEA
  \l BEEB
  \l BEEC
  \l BEED
  \l BEEE
  \l BEEF
  \l BEF0
  \l BEF1
  \l BEF2
  \l BEF3
  \l BEF4
  \l BEF5
  \l BEF6
  \l BEF7
  \l BEF8
  \l BEF9
  \l BEFA
  \l BEFB
  \l BEFC
  \l BEFD
  \l BEFE
  \l BEFF
  \l BF00
  \l BF01
  \l BF02
  \l BF03
  \l BF04
  \l BF05
  \l BF06
  \l BF07
  \l BF08
  \l BF09
  \l BF0A
  \l BF0B
  \l BF0C
  \l BF0D
  \l BF0E
  \l BF0F
  \l BF10
  \l BF11
  \l BF12
  \l BF13
  \l BF14
  \l BF15
  \l BF16
  \l BF17
  \l BF18
  \l BF19
  \l BF1A
  \l BF1B
  \l BF1C
  \l BF1D
  \l BF1E
  \l BF1F
  \l BF20
  \l BF21
  \l BF22
  \l BF23
  \l BF24
  \l BF25
  \l BF26
  \l BF27
  \l BF28
  \l BF29
  \l BF2A
  \l BF2B
  \l BF2C
  \l BF2D
  \l BF2E
  \l BF2F
  \l BF30
  \l BF31
  \l BF32
  \l BF33
  \l BF34
  \l BF35
  \l BF36
  \l BF37
  \l BF38
  \l BF39
  \l BF3A
  \l BF3B
  \l BF3C
  \l BF3D
  \l BF3E
  \l BF3F
  \l BF40
  \l BF41
  \l BF42
  \l BF43
  \l BF44
  \l BF45
  \l BF46
  \l BF47
  \l BF48
  \l BF49
  \l BF4A
  \l BF4B
  \l BF4C
  \l BF4D
  \l BF4E
  \l BF4F
  \l BF50
  \l BF51
  \l BF52
  \l BF53
  \l BF54
  \l BF55
  \l BF56
  \l BF57
  \l BF58
  \l BF59
  \l BF5A
  \l BF5B
  \l BF5C
  \l BF5D
  \l BF5E
  \l BF5F
  \l BF60
  \l BF61
  \l BF62
  \l BF63
  \l BF64
  \l BF65
  \l BF66
  \l BF67
  \l BF68
  \l BF69
  \l BF6A
  \l BF6B
  \l BF6C
  \l BF6D
  \l BF6E
  \l BF6F
  \l BF70
  \l BF71
  \l BF72
  \l BF73
  \l BF74
  \l BF75
  \l BF76
  \l BF77
  \l BF78
  \l BF79
  \l BF7A
  \l BF7B
  \l BF7C
  \l BF7D
  \l BF7E
  \l BF7F
  \l BF80
  \l BF81
  \l BF82
  \l BF83
  \l BF84
  \l BF85
  \l BF86
  \l BF87
  \l BF88
  \l BF89
  \l BF8A
  \l BF8B
  \l BF8C
  \l BF8D
  \l BF8E
  \l BF8F
  \l BF90
  \l BF91
  \l BF92
  \l BF93
  \l BF94
  \l BF95
  \l BF96
  \l BF97
  \l BF98
  \l BF99
  \l BF9A
  \l BF9B
  \l BF9C
  \l BF9D
  \l BF9E
  \l BF9F
  \l BFA0
  \l BFA1
  \l BFA2
  \l BFA3
  \l BFA4
  \l BFA5
  \l BFA6
  \l BFA7
  \l BFA8
  \l BFA9
  \l BFAA
  \l BFAB
  \l BFAC
  \l BFAD
  \l BFAE
  \l BFAF
  \l BFB0
  \l BFB1
  \l BFB2
  \l BFB3
  \l BFB4
  \l BFB5
  \l BFB6
  \l BFB7
  \l BFB8
  \l BFB9
  \l BFBA
  \l BFBB
  \l BFBC
  \l BFBD
  \l BFBE
  \l BFBF
  \l BFC0
  \l BFC1
  \l BFC2
  \l BFC3
  \l BFC4
  \l BFC5
  \l BFC6
  \l BFC7
  \l BFC8
  \l BFC9
  \l BFCA
  \l BFCB
  \l BFCC
  \l BFCD
  \l BFCE
  \l BFCF
  \l BFD0
  \l BFD1
  \l BFD2
  \l BFD3
  \l BFD4
  \l BFD5
  \l BFD6
  \l BFD7
  \l BFD8
  \l BFD9
  \l BFDA
  \l BFDB
  \l BFDC
  \l BFDD
  \l BFDE
  \l BFDF
  \l BFE0
  \l BFE1
  \l BFE2
  \l BFE3
  \l BFE4
  \l BFE5
  \l BFE6
  \l BFE7
  \l BFE8
  \l BFE9
  \l BFEA
  \l BFEB
  \l BFEC
  \l BFED
  \l BFEE
  \l BFEF
  \l BFF0
  \l BFF1
  \l BFF2
  \l BFF3
  \l BFF4
  \l BFF5
  \l BFF6
  \l BFF7
  \l BFF8
  \l BFF9
  \l BFFA
  \l BFFB
  \l BFFC
  \l BFFD
  \l BFFE
  \l BFFF
  \l C000
  \l C001
  \l C002
  \l C003
  \l C004
  \l C005
  \l C006
  \l C007
  \l C008
  \l C009
  \l C00A
  \l C00B
  \l C00C
  \l C00D
  \l C00E
  \l C00F
  \l C010
  \l C011
  \l C012
  \l C013
  \l C014
  \l C015
  \l C016
  \l C017
  \l C018
  \l C019
  \l C01A
  \l C01B
  \l C01C
  \l C01D
  \l C01E
  \l C01F
  \l C020
  \l C021
  \l C022
  \l C023
  \l C024
  \l C025
  \l C026
  \l C027
  \l C028
  \l C029
  \l C02A
  \l C02B
  \l C02C
  \l C02D
  \l C02E
  \l C02F
  \l C030
  \l C031
  \l C032
  \l C033
  \l C034
  \l C035
  \l C036
  \l C037
  \l C038
  \l C039
  \l C03A
  \l C03B
  \l C03C
  \l C03D
  \l C03E
  \l C03F
  \l C040
  \l C041
  \l C042
  \l C043
  \l C044
  \l C045
  \l C046
  \l C047
  \l C048
  \l C049
  \l C04A
  \l C04B
  \l C04C
  \l C04D
  \l C04E
  \l C04F
  \l C050
  \l C051
  \l C052
  \l C053
  \l C054
  \l C055
  \l C056
  \l C057
  \l C058
  \l C059
  \l C05A
  \l C05B
  \l C05C
  \l C05D
  \l C05E
  \l C05F
  \l C060
  \l C061
  \l C062
  \l C063
  \l C064
  \l C065
  \l C066
  \l C067
  \l C068
  \l C069
  \l C06A
  \l C06B
  \l C06C
  \l C06D
  \l C06E
  \l C06F
  \l C070
  \l C071
  \l C072
  \l C073
  \l C074
  \l C075
  \l C076
  \l C077
  \l C078
  \l C079
  \l C07A
  \l C07B
  \l C07C
  \l C07D
  \l C07E
  \l C07F
  \l C080
  \l C081
  \l C082
  \l C083
  \l C084
  \l C085
  \l C086
  \l C087
  \l C088
  \l C089
  \l C08A
  \l C08B
  \l C08C
  \l C08D
  \l C08E
  \l C08F
  \l C090
  \l C091
  \l C092
  \l C093
  \l C094
  \l C095
  \l C096
  \l C097
  \l C098
  \l C099
  \l C09A
  \l C09B
  \l C09C
  \l C09D
  \l C09E
  \l C09F
  \l C0A0
  \l C0A1
  \l C0A2
  \l C0A3
  \l C0A4
  \l C0A5
  \l C0A6
  \l C0A7
  \l C0A8
  \l C0A9
  \l C0AA
  \l C0AB
  \l C0AC
  \l C0AD
  \l C0AE
  \l C0AF
  \l C0B0
  \l C0B1
  \l C0B2
  \l C0B3
  \l C0B4
  \l C0B5
  \l C0B6
  \l C0B7
  \l C0B8
  \l C0B9
  \l C0BA
  \l C0BB
  \l C0BC
  \l C0BD
  \l C0BE
  \l C0BF
  \l C0C0
  \l C0C1
  \l C0C2
  \l C0C3
  \l C0C4
  \l C0C5
  \l C0C6
  \l C0C7
  \l C0C8
  \l C0C9
  \l C0CA
  \l C0CB
  \l C0CC
  \l C0CD
  \l C0CE
  \l C0CF
  \l C0D0
  \l C0D1
  \l C0D2
  \l C0D3
  \l C0D4
  \l C0D5
  \l C0D6
  \l C0D7
  \l C0D8
  \l C0D9
  \l C0DA
  \l C0DB
  \l C0DC
  \l C0DD
  \l C0DE
  \l C0DF
  \l C0E0
  \l C0E1
  \l C0E2
  \l C0E3
  \l C0E4
  \l C0E5
  \l C0E6
  \l C0E7
  \l C0E8
  \l C0E9
  \l C0EA
  \l C0EB
  \l C0EC
  \l C0ED
  \l C0EE
  \l C0EF
  \l C0F0
  \l C0F1
  \l C0F2
  \l C0F3
  \l C0F4
  \l C0F5
  \l C0F6
  \l C0F7
  \l C0F8
  \l C0F9
  \l C0FA
  \l C0FB
  \l C0FC
  \l C0FD
  \l C0FE
  \l C0FF
  \l C100
  \l C101
  \l C102
  \l C103
  \l C104
  \l C105
  \l C106
  \l C107
  \l C108
  \l C109
  \l C10A
  \l C10B
  \l C10C
  \l C10D
  \l C10E
  \l C10F
  \l C110
  \l C111
  \l C112
  \l C113
  \l C114
  \l C115
  \l C116
  \l C117
  \l C118
  \l C119
  \l C11A
  \l C11B
  \l C11C
  \l C11D
  \l C11E
  \l C11F
  \l C120
  \l C121
  \l C122
  \l C123
  \l C124
  \l C125
  \l C126
  \l C127
  \l C128
  \l C129
  \l C12A
  \l C12B
  \l C12C
  \l C12D
  \l C12E
  \l C12F
  \l C130
  \l C131
  \l C132
  \l C133
  \l C134
  \l C135
  \l C136
  \l C137
  \l C138
  \l C139
  \l C13A
  \l C13B
  \l C13C
  \l C13D
  \l C13E
  \l C13F
  \l C140
  \l C141
  \l C142
  \l C143
  \l C144
  \l C145
  \l C146
  \l C147
  \l C148
  \l C149
  \l C14A
  \l C14B
  \l C14C
  \l C14D
  \l C14E
  \l C14F
  \l C150
  \l C151
  \l C152
  \l C153
  \l C154
  \l C155
  \l C156
  \l C157
  \l C158
  \l C159
  \l C15A
  \l C15B
  \l C15C
  \l C15D
  \l C15E
  \l C15F
  \l C160
  \l C161
  \l C162
  \l C163
  \l C164
  \l C165
  \l C166
  \l C167
  \l C168
  \l C169
  \l C16A
  \l C16B
  \l C16C
  \l C16D
  \l C16E
  \l C16F
  \l C170
  \l C171
  \l C172
  \l C173
  \l C174
  \l C175
  \l C176
  \l C177
  \l C178
  \l C179
  \l C17A
  \l C17B
  \l C17C
  \l C17D
  \l C17E
  \l C17F
  \l C180
  \l C181
  \l C182
  \l C183
  \l C184
  \l C185
  \l C186
  \l C187
  \l C188
  \l C189
  \l C18A
  \l C18B
  \l C18C
  \l C18D
  \l C18E
  \l C18F
  \l C190
  \l C191
  \l C192
  \l C193
  \l C194
  \l C195
  \l C196
  \l C197
  \l C198
  \l C199
  \l C19A
  \l C19B
  \l C19C
  \l C19D
  \l C19E
  \l C19F
  \l C1A0
  \l C1A1
  \l C1A2
  \l C1A3
  \l C1A4
  \l C1A5
  \l C1A6
  \l C1A7
  \l C1A8
  \l C1A9
  \l C1AA
  \l C1AB
  \l C1AC
  \l C1AD
  \l C1AE
  \l C1AF
  \l C1B0
  \l C1B1
  \l C1B2
  \l C1B3
  \l C1B4
  \l C1B5
  \l C1B6
  \l C1B7
  \l C1B8
  \l C1B9
  \l C1BA
  \l C1BB
  \l C1BC
  \l C1BD
  \l C1BE
  \l C1BF
  \l C1C0
  \l C1C1
  \l C1C2
  \l C1C3
  \l C1C4
  \l C1C5
  \l C1C6
  \l C1C7
  \l C1C8
  \l C1C9
  \l C1CA
  \l C1CB
  \l C1CC
  \l C1CD
  \l C1CE
  \l C1CF
  \l C1D0
  \l C1D1
  \l C1D2
  \l C1D3
  \l C1D4
  \l C1D5
  \l C1D6
  \l C1D7
  \l C1D8
  \l C1D9
  \l C1DA
  \l C1DB
  \l C1DC
  \l C1DD
  \l C1DE
  \l C1DF
  \l C1E0
  \l C1E1
  \l C1E2
  \l C1E3
  \l C1E4
  \l C1E5
  \l C1E6
  \l C1E7
  \l C1E8
  \l C1E9
  \l C1EA
  \l C1EB
  \l C1EC
  \l C1ED
  \l C1EE
  \l C1EF
  \l C1F0
  \l C1F1
  \l C1F2
  \l C1F3
  \l C1F4
  \l C1F5
  \l C1F6
  \l C1F7
  \l C1F8
  \l C1F9
  \l C1FA
  \l C1FB
  \l C1FC
  \l C1FD
  \l C1FE
  \l C1FF
  \l C200
  \l C201
  \l C202
  \l C203
  \l C204
  \l C205
  \l C206
  \l C207
  \l C208
  \l C209
  \l C20A
  \l C20B
  \l C20C
  \l C20D
  \l C20E
  \l C20F
  \l C210
  \l C211
  \l C212
  \l C213
  \l C214
  \l C215
  \l C216
  \l C217
  \l C218
  \l C219
  \l C21A
  \l C21B
  \l C21C
  \l C21D
  \l C21E
  \l C21F
  \l C220
  \l C221
  \l C222
  \l C223
  \l C224
  \l C225
  \l C226
  \l C227
  \l C228
  \l C229
  \l C22A
  \l C22B
  \l C22C
  \l C22D
  \l C22E
  \l C22F
  \l C230
  \l C231
  \l C232
  \l C233
  \l C234
  \l C235
  \l C236
  \l C237
  \l C238
  \l C239
  \l C23A
  \l C23B
  \l C23C
  \l C23D
  \l C23E
  \l C23F
  \l C240
  \l C241
  \l C242
  \l C243
  \l C244
  \l C245
  \l C246
  \l C247
  \l C248
  \l C249
  \l C24A
  \l C24B
  \l C24C
  \l C24D
  \l C24E
  \l C24F
  \l C250
  \l C251
  \l C252
  \l C253
  \l C254
  \l C255
  \l C256
  \l C257
  \l C258
  \l C259
  \l C25A
  \l C25B
  \l C25C
  \l C25D
  \l C25E
  \l C25F
  \l C260
  \l C261
  \l C262
  \l C263
  \l C264
  \l C265
  \l C266
  \l C267
  \l C268
  \l C269
  \l C26A
  \l C26B
  \l C26C
  \l C26D
  \l C26E
  \l C26F
  \l C270
  \l C271
  \l C272
  \l C273
  \l C274
  \l C275
  \l C276
  \l C277
  \l C278
  \l C279
  \l C27A
  \l C27B
  \l C27C
  \l C27D
  \l C27E
  \l C27F
  \l C280
  \l C281
  \l C282
  \l C283
  \l C284
  \l C285
  \l C286
  \l C287
  \l C288
  \l C289
  \l C28A
  \l C28B
  \l C28C
  \l C28D
  \l C28E
  \l C28F
  \l C290
  \l C291
  \l C292
  \l C293
  \l C294
  \l C295
  \l C296
  \l C297
  \l C298
  \l C299
  \l C29A
  \l C29B
  \l C29C
  \l C29D
  \l C29E
  \l C29F
  \l C2A0
  \l C2A1
  \l C2A2
  \l C2A3
  \l C2A4
  \l C2A5
  \l C2A6
  \l C2A7
  \l C2A8
  \l C2A9
  \l C2AA
  \l C2AB
  \l C2AC
  \l C2AD
  \l C2AE
  \l C2AF
  \l C2B0
  \l C2B1
  \l C2B2
  \l C2B3
  \l C2B4
  \l C2B5
  \l C2B6
  \l C2B7
  \l C2B8
  \l C2B9
  \l C2BA
  \l C2BB
  \l C2BC
  \l C2BD
  \l C2BE
  \l C2BF
  \l C2C0
  \l C2C1
  \l C2C2
  \l C2C3
  \l C2C4
  \l C2C5
  \l C2C6
  \l C2C7
  \l C2C8
  \l C2C9
  \l C2CA
  \l C2CB
  \l C2CC
  \l C2CD
  \l C2CE
  \l C2CF
  \l C2D0
  \l C2D1
  \l C2D2
  \l C2D3
  \l C2D4
  \l C2D5
  \l C2D6
  \l C2D7
  \l C2D8
  \l C2D9
  \l C2DA
  \l C2DB
  \l C2DC
  \l C2DD
  \l C2DE
  \l C2DF
  \l C2E0
  \l C2E1
  \l C2E2
  \l C2E3
  \l C2E4
  \l C2E5
  \l C2E6
  \l C2E7
  \l C2E8
  \l C2E9
  \l C2EA
  \l C2EB
  \l C2EC
  \l C2ED
  \l C2EE
  \l C2EF
  \l C2F0
  \l C2F1
  \l C2F2
  \l C2F3
  \l C2F4
  \l C2F5
  \l C2F6
  \l C2F7
  \l C2F8
  \l C2F9
  \l C2FA
  \l C2FB
  \l C2FC
  \l C2FD
  \l C2FE
  \l C2FF
  \l C300
  \l C301
  \l C302
  \l C303
  \l C304
  \l C305
  \l C306
  \l C307
  \l C308
  \l C309
  \l C30A
  \l C30B
  \l C30C
  \l C30D
  \l C30E
  \l C30F
  \l C310
  \l C311
  \l C312
  \l C313
  \l C314
  \l C315
  \l C316
  \l C317
  \l C318
  \l C319
  \l C31A
  \l C31B
  \l C31C
  \l C31D
  \l C31E
  \l C31F
  \l C320
  \l C321
  \l C322
  \l C323
  \l C324
  \l C325
  \l C326
  \l C327
  \l C328
  \l C329
  \l C32A
  \l C32B
  \l C32C
  \l C32D
  \l C32E
  \l C32F
  \l C330
  \l C331
  \l C332
  \l C333
  \l C334
  \l C335
  \l C336
  \l C337
  \l C338
  \l C339
  \l C33A
  \l C33B
  \l C33C
  \l C33D
  \l C33E
  \l C33F
  \l C340
  \l C341
  \l C342
  \l C343
  \l C344
  \l C345
  \l C346
  \l C347
  \l C348
  \l C349
  \l C34A
  \l C34B
  \l C34C
  \l C34D
  \l C34E
  \l C34F
  \l C350
  \l C351
  \l C352
  \l C353
  \l C354
  \l C355
  \l C356
  \l C357
  \l C358
  \l C359
  \l C35A
  \l C35B
  \l C35C
  \l C35D
  \l C35E
  \l C35F
  \l C360
  \l C361
  \l C362
  \l C363
  \l C364
  \l C365
  \l C366
  \l C367
  \l C368
  \l C369
  \l C36A
  \l C36B
  \l C36C
  \l C36D
  \l C36E
  \l C36F
  \l C370
  \l C371
  \l C372
  \l C373
  \l C374
  \l C375
  \l C376
  \l C377
  \l C378
  \l C379
  \l C37A
  \l C37B
  \l C37C
  \l C37D
  \l C37E
  \l C37F
  \l C380
  \l C381
  \l C382
  \l C383
  \l C384
  \l C385
  \l C386
  \l C387
  \l C388
  \l C389
  \l C38A
  \l C38B
  \l C38C
  \l C38D
  \l C38E
  \l C38F
  \l C390
  \l C391
  \l C392
  \l C393
  \l C394
  \l C395
  \l C396
  \l C397
  \l C398
  \l C399
  \l C39A
  \l C39B
  \l C39C
  \l C39D
  \l C39E
  \l C39F
  \l C3A0
  \l C3A1
  \l C3A2
  \l C3A3
  \l C3A4
  \l C3A5
  \l C3A6
  \l C3A7
  \l C3A8
  \l C3A9
  \l C3AA
  \l C3AB
  \l C3AC
  \l C3AD
  \l C3AE
  \l C3AF
  \l C3B0
  \l C3B1
  \l C3B2
  \l C3B3
  \l C3B4
  \l C3B5
  \l C3B6
  \l C3B7
  \l C3B8
  \l C3B9
  \l C3BA
  \l C3BB
  \l C3BC
  \l C3BD
  \l C3BE
  \l C3BF
  \l C3C0
  \l C3C1
  \l C3C2
  \l C3C3
  \l C3C4
  \l C3C5
  \l C3C6
  \l C3C7
  \l C3C8
  \l C3C9
  \l C3CA
  \l C3CB
  \l C3CC
  \l C3CD
  \l C3CE
  \l C3CF
  \l C3D0
  \l C3D1
  \l C3D2
  \l C3D3
  \l C3D4
  \l C3D5
  \l C3D6
  \l C3D7
  \l C3D8
  \l C3D9
  \l C3DA
  \l C3DB
  \l C3DC
  \l C3DD
  \l C3DE
  \l C3DF
  \l C3E0
  \l C3E1
  \l C3E2
  \l C3E3
  \l C3E4
  \l C3E5
  \l C3E6
  \l C3E7
  \l C3E8
  \l C3E9
  \l C3EA
  \l C3EB
  \l C3EC
  \l C3ED
  \l C3EE
  \l C3EF
  \l C3F0
  \l C3F1
  \l C3F2
  \l C3F3
  \l C3F4
  \l C3F5
  \l C3F6
  \l C3F7
  \l C3F8
  \l C3F9
  \l C3FA
  \l C3FB
  \l C3FC
  \l C3FD
  \l C3FE
  \l C3FF
  \l C400
  \l C401
  \l C402
  \l C403
  \l C404
  \l C405
  \l C406
  \l C407
  \l C408
  \l C409
  \l C40A
  \l C40B
  \l C40C
  \l C40D
  \l C40E
  \l C40F
  \l C410
  \l C411
  \l C412
  \l C413
  \l C414
  \l C415
  \l C416
  \l C417
  \l C418
  \l C419
  \l C41A
  \l C41B
  \l C41C
  \l C41D
  \l C41E
  \l C41F
  \l C420
  \l C421
  \l C422
  \l C423
  \l C424
  \l C425
  \l C426
  \l C427
  \l C428
  \l C429
  \l C42A
  \l C42B
  \l C42C
  \l C42D
  \l C42E
  \l C42F
  \l C430
  \l C431
  \l C432
  \l C433
  \l C434
  \l C435
  \l C436
  \l C437
  \l C438
  \l C439
  \l C43A
  \l C43B
  \l C43C
  \l C43D
  \l C43E
  \l C43F
  \l C440
  \l C441
  \l C442
  \l C443
  \l C444
  \l C445
  \l C446
  \l C447
  \l C448
  \l C449
  \l C44A
  \l C44B
  \l C44C
  \l C44D
  \l C44E
  \l C44F
  \l C450
  \l C451
  \l C452
  \l C453
  \l C454
  \l C455
  \l C456
  \l C457
  \l C458
  \l C459
  \l C45A
  \l C45B
  \l C45C
  \l C45D
  \l C45E
  \l C45F
  \l C460
  \l C461
  \l C462
  \l C463
  \l C464
  \l C465
  \l C466
  \l C467
  \l C468
  \l C469
  \l C46A
  \l C46B
  \l C46C
  \l C46D
  \l C46E
  \l C46F
  \l C470
  \l C471
  \l C472
  \l C473
  \l C474
  \l C475
  \l C476
  \l C477
  \l C478
  \l C479
  \l C47A
  \l C47B
  \l C47C
  \l C47D
  \l C47E
  \l C47F
  \l C480
  \l C481
  \l C482
  \l C483
  \l C484
  \l C485
  \l C486
  \l C487
  \l C488
  \l C489
  \l C48A
  \l C48B
  \l C48C
  \l C48D
  \l C48E
  \l C48F
  \l C490
  \l C491
  \l C492
  \l C493
  \l C494
  \l C495
  \l C496
  \l C497
  \l C498
  \l C499
  \l C49A
  \l C49B
  \l C49C
  \l C49D
  \l C49E
  \l C49F
  \l C4A0
  \l C4A1
  \l C4A2
  \l C4A3
  \l C4A4
  \l C4A5
  \l C4A6
  \l C4A7
  \l C4A8
  \l C4A9
  \l C4AA
  \l C4AB
  \l C4AC
  \l C4AD
  \l C4AE
  \l C4AF
  \l C4B0
  \l C4B1
  \l C4B2
  \l C4B3
  \l C4B4
  \l C4B5
  \l C4B6
  \l C4B7
  \l C4B8
  \l C4B9
  \l C4BA
  \l C4BB
  \l C4BC
  \l C4BD
  \l C4BE
  \l C4BF
  \l C4C0
  \l C4C1
  \l C4C2
  \l C4C3
  \l C4C4
  \l C4C5
  \l C4C6
  \l C4C7
  \l C4C8
  \l C4C9
  \l C4CA
  \l C4CB
  \l C4CC
  \l C4CD
  \l C4CE
  \l C4CF
  \l C4D0
  \l C4D1
  \l C4D2
  \l C4D3
  \l C4D4
  \l C4D5
  \l C4D6
  \l C4D7
  \l C4D8
  \l C4D9
  \l C4DA
  \l C4DB
  \l C4DC
  \l C4DD
  \l C4DE
  \l C4DF
  \l C4E0
  \l C4E1
  \l C4E2
  \l C4E3
  \l C4E4
  \l C4E5
  \l C4E6
  \l C4E7
  \l C4E8
  \l C4E9
  \l C4EA
  \l C4EB
  \l C4EC
  \l C4ED
  \l C4EE
  \l C4EF
  \l C4F0
  \l C4F1
  \l C4F2
  \l C4F3
  \l C4F4
  \l C4F5
  \l C4F6
  \l C4F7
  \l C4F8
  \l C4F9
  \l C4FA
  \l C4FB
  \l C4FC
  \l C4FD
  \l C4FE
  \l C4FF
  \l C500
  \l C501
  \l C502
  \l C503
  \l C504
  \l C505
  \l C506
  \l C507
  \l C508
  \l C509
  \l C50A
  \l C50B
  \l C50C
  \l C50D
  \l C50E
  \l C50F
  \l C510
  \l C511
  \l C512
  \l C513
  \l C514
  \l C515
  \l C516
  \l C517
  \l C518
  \l C519
  \l C51A
  \l C51B
  \l C51C
  \l C51D
  \l C51E
  \l C51F
  \l C520
  \l C521
  \l C522
  \l C523
  \l C524
  \l C525
  \l C526
  \l C527
  \l C528
  \l C529
  \l C52A
  \l C52B
  \l C52C
  \l C52D
  \l C52E
  \l C52F
  \l C530
  \l C531
  \l C532
  \l C533
  \l C534
  \l C535
  \l C536
  \l C537
  \l C538
  \l C539
  \l C53A
  \l C53B
  \l C53C
  \l C53D
  \l C53E
  \l C53F
  \l C540
  \l C541
  \l C542
  \l C543
  \l C544
  \l C545
  \l C546
  \l C547
  \l C548
  \l C549
  \l C54A
  \l C54B
  \l C54C
  \l C54D
  \l C54E
  \l C54F
  \l C550
  \l C551
  \l C552
  \l C553
  \l C554
  \l C555
  \l C556
  \l C557
  \l C558
  \l C559
  \l C55A
  \l C55B
  \l C55C
  \l C55D
  \l C55E
  \l C55F
  \l C560
  \l C561
  \l C562
  \l C563
  \l C564
  \l C565
  \l C566
  \l C567
  \l C568
  \l C569
  \l C56A
  \l C56B
  \l C56C
  \l C56D
  \l C56E
  \l C56F
  \l C570
  \l C571
  \l C572
  \l C573
  \l C574
  \l C575
  \l C576
  \l C577
  \l C578
  \l C579
  \l C57A
  \l C57B
  \l C57C
  \l C57D
  \l C57E
  \l C57F
  \l C580
  \l C581
  \l C582
  \l C583
  \l C584
  \l C585
  \l C586
  \l C587
  \l C588
  \l C589
  \l C58A
  \l C58B
  \l C58C
  \l C58D
  \l C58E
  \l C58F
  \l C590
  \l C591
  \l C592
  \l C593
  \l C594
  \l C595
  \l C596
  \l C597
  \l C598
  \l C599
  \l C59A
  \l C59B
  \l C59C
  \l C59D
  \l C59E
  \l C59F
  \l C5A0
  \l C5A1
  \l C5A2
  \l C5A3
  \l C5A4
  \l C5A5
  \l C5A6
  \l C5A7
  \l C5A8
  \l C5A9
  \l C5AA
  \l C5AB
  \l C5AC
  \l C5AD
  \l C5AE
  \l C5AF
  \l C5B0
  \l C5B1
  \l C5B2
  \l C5B3
  \l C5B4
  \l C5B5
  \l C5B6
  \l C5B7
  \l C5B8
  \l C5B9
  \l C5BA
  \l C5BB
  \l C5BC
  \l C5BD
  \l C5BE
  \l C5BF
  \l C5C0
  \l C5C1
  \l C5C2
  \l C5C3
  \l C5C4
  \l C5C5
  \l C5C6
  \l C5C7
  \l C5C8
  \l C5C9
  \l C5CA
  \l C5CB
  \l C5CC
  \l C5CD
  \l C5CE
  \l C5CF
  \l C5D0
  \l C5D1
  \l C5D2
  \l C5D3
  \l C5D4
  \l C5D5
  \l C5D6
  \l C5D7
  \l C5D8
  \l C5D9
  \l C5DA
  \l C5DB
  \l C5DC
  \l C5DD
  \l C5DE
  \l C5DF
  \l C5E0
  \l C5E1
  \l C5E2
  \l C5E3
  \l C5E4
  \l C5E5
  \l C5E6
  \l C5E7
  \l C5E8
  \l C5E9
  \l C5EA
  \l C5EB
  \l C5EC
  \l C5ED
  \l C5EE
  \l C5EF
  \l C5F0
  \l C5F1
  \l C5F2
  \l C5F3
  \l C5F4
  \l C5F5
  \l C5F6
  \l C5F7
  \l C5F8
  \l C5F9
  \l C5FA
  \l C5FB
  \l C5FC
  \l C5FD
  \l C5FE
  \l C5FF
  \l C600
  \l C601
  \l C602
  \l C603
  \l C604
  \l C605
  \l C606
  \l C607
  \l C608
  \l C609
  \l C60A
  \l C60B
  \l C60C
  \l C60D
  \l C60E
  \l C60F
  \l C610
  \l C611
  \l C612
  \l C613
  \l C614
  \l C615
  \l C616
  \l C617
  \l C618
  \l C619
  \l C61A
  \l C61B
  \l C61C
  \l C61D
  \l C61E
  \l C61F
  \l C620
  \l C621
  \l C622
  \l C623
  \l C624
  \l C625
  \l C626
  \l C627
  \l C628
  \l C629
  \l C62A
  \l C62B
  \l C62C
  \l C62D
  \l C62E
  \l C62F
  \l C630
  \l C631
  \l C632
  \l C633
  \l C634
  \l C635
  \l C636
  \l C637
  \l C638
  \l C639
  \l C63A
  \l C63B
  \l C63C
  \l C63D
  \l C63E
  \l C63F
  \l C640
  \l C641
  \l C642
  \l C643
  \l C644
  \l C645
  \l C646
  \l C647
  \l C648
  \l C649
  \l C64A
  \l C64B
  \l C64C
  \l C64D
  \l C64E
  \l C64F
  \l C650
  \l C651
  \l C652
  \l C653
  \l C654
  \l C655
  \l C656
  \l C657
  \l C658
  \l C659
  \l C65A
  \l C65B
  \l C65C
  \l C65D
  \l C65E
  \l C65F
  \l C660
  \l C661
  \l C662
  \l C663
  \l C664
  \l C665
  \l C666
  \l C667
  \l C668
  \l C669
  \l C66A
  \l C66B
  \l C66C
  \l C66D
  \l C66E
  \l C66F
  \l C670
  \l C671
  \l C672
  \l C673
  \l C674
  \l C675
  \l C676
  \l C677
  \l C678
  \l C679
  \l C67A
  \l C67B
  \l C67C
  \l C67D
  \l C67E
  \l C67F
  \l C680
  \l C681
  \l C682
  \l C683
  \l C684
  \l C685
  \l C686
  \l C687
  \l C688
  \l C689
  \l C68A
  \l C68B
  \l C68C
  \l C68D
  \l C68E
  \l C68F
  \l C690
  \l C691
  \l C692
  \l C693
  \l C694
  \l C695
  \l C696
  \l C697
  \l C698
  \l C699
  \l C69A
  \l C69B
  \l C69C
  \l C69D
  \l C69E
  \l C69F
  \l C6A0
  \l C6A1
  \l C6A2
  \l C6A3
  \l C6A4
  \l C6A5
  \l C6A6
  \l C6A7
  \l C6A8
  \l C6A9
  \l C6AA
  \l C6AB
  \l C6AC
  \l C6AD
  \l C6AE
  \l C6AF
  \l C6B0
  \l C6B1
  \l C6B2
  \l C6B3
  \l C6B4
  \l C6B5
  \l C6B6
  \l C6B7
  \l C6B8
  \l C6B9
  \l C6BA
  \l C6BB
  \l C6BC
  \l C6BD
  \l C6BE
  \l C6BF
  \l C6C0
  \l C6C1
  \l C6C2
  \l C6C3
  \l C6C4
  \l C6C5
  \l C6C6
  \l C6C7
  \l C6C8
  \l C6C9
  \l C6CA
  \l C6CB
  \l C6CC
  \l C6CD
  \l C6CE
  \l C6CF
  \l C6D0
  \l C6D1
  \l C6D2
  \l C6D3
  \l C6D4
  \l C6D5
  \l C6D6
  \l C6D7
  \l C6D8
  \l C6D9
  \l C6DA
  \l C6DB
  \l C6DC
  \l C6DD
  \l C6DE
  \l C6DF
  \l C6E0
  \l C6E1
  \l C6E2
  \l C6E3
  \l C6E4
  \l C6E5
  \l C6E6
  \l C6E7
  \l C6E8
  \l C6E9
  \l C6EA
  \l C6EB
  \l C6EC
  \l C6ED
  \l C6EE
  \l C6EF
  \l C6F0
  \l C6F1
  \l C6F2
  \l C6F3
  \l C6F4
  \l C6F5
  \l C6F6
  \l C6F7
  \l C6F8
  \l C6F9
  \l C6FA
  \l C6FB
  \l C6FC
  \l C6FD
  \l C6FE
  \l C6FF
  \l C700
  \l C701
  \l C702
  \l C703
  \l C704
  \l C705
  \l C706
  \l C707
  \l C708
  \l C709
  \l C70A
  \l C70B
  \l C70C
  \l C70D
  \l C70E
  \l C70F
  \l C710
  \l C711
  \l C712
  \l C713
  \l C714
  \l C715
  \l C716
  \l C717
  \l C718
  \l C719
  \l C71A
  \l C71B
  \l C71C
  \l C71D
  \l C71E
  \l C71F
  \l C720
  \l C721
  \l C722
  \l C723
  \l C724
  \l C725
  \l C726
  \l C727
  \l C728
  \l C729
  \l C72A
  \l C72B
  \l C72C
  \l C72D
  \l C72E
  \l C72F
  \l C730
  \l C731
  \l C732
  \l C733
  \l C734
  \l C735
  \l C736
  \l C737
  \l C738
  \l C739
  \l C73A
  \l C73B
  \l C73C
  \l C73D
  \l C73E
  \l C73F
  \l C740
  \l C741
  \l C742
  \l C743
  \l C744
  \l C745
  \l C746
  \l C747
  \l C748
  \l C749
  \l C74A
  \l C74B
  \l C74C
  \l C74D
  \l C74E
  \l C74F
  \l C750
  \l C751
  \l C752
  \l C753
  \l C754
  \l C755
  \l C756
  \l C757
  \l C758
  \l C759
  \l C75A
  \l C75B
  \l C75C
  \l C75D
  \l C75E
  \l C75F
  \l C760
  \l C761
  \l C762
  \l C763
  \l C764
  \l C765
  \l C766
  \l C767
  \l C768
  \l C769
  \l C76A
  \l C76B
  \l C76C
  \l C76D
  \l C76E
  \l C76F
  \l C770
  \l C771
  \l C772
  \l C773
  \l C774
  \l C775
  \l C776
  \l C777
  \l C778
  \l C779
  \l C77A
  \l C77B
  \l C77C
  \l C77D
  \l C77E
  \l C77F
  \l C780
  \l C781
  \l C782
  \l C783
  \l C784
  \l C785
  \l C786
  \l C787
  \l C788
  \l C789
  \l C78A
  \l C78B
  \l C78C
  \l C78D
  \l C78E
  \l C78F
  \l C790
  \l C791
  \l C792
  \l C793
  \l C794
  \l C795
  \l C796
  \l C797
  \l C798
  \l C799
  \l C79A
  \l C79B
  \l C79C
  \l C79D
  \l C79E
  \l C79F
  \l C7A0
  \l C7A1
  \l C7A2
  \l C7A3
  \l C7A4
  \l C7A5
  \l C7A6
  \l C7A7
  \l C7A8
  \l C7A9
  \l C7AA
  \l C7AB
  \l C7AC
  \l C7AD
  \l C7AE
  \l C7AF
  \l C7B0
  \l C7B1
  \l C7B2
  \l C7B3
  \l C7B4
  \l C7B5
  \l C7B6
  \l C7B7
  \l C7B8
  \l C7B9
  \l C7BA
  \l C7BB
  \l C7BC
  \l C7BD
  \l C7BE
  \l C7BF
  \l C7C0
  \l C7C1
  \l C7C2
  \l C7C3
  \l C7C4
  \l C7C5
  \l C7C6
  \l C7C7
  \l C7C8
  \l C7C9
  \l C7CA
  \l C7CB
  \l C7CC
  \l C7CD
  \l C7CE
  \l C7CF
  \l C7D0
  \l C7D1
  \l C7D2
  \l C7D3
  \l C7D4
  \l C7D5
  \l C7D6
  \l C7D7
  \l C7D8
  \l C7D9
  \l C7DA
  \l C7DB
  \l C7DC
  \l C7DD
  \l C7DE
  \l C7DF
  \l C7E0
  \l C7E1
  \l C7E2
  \l C7E3
  \l C7E4
  \l C7E5
  \l C7E6
  \l C7E7
  \l C7E8
  \l C7E9
  \l C7EA
  \l C7EB
  \l C7EC
  \l C7ED
  \l C7EE
  \l C7EF
  \l C7F0
  \l C7F1
  \l C7F2
  \l C7F3
  \l C7F4
  \l C7F5
  \l C7F6
  \l C7F7
  \l C7F8
  \l C7F9
  \l C7FA
  \l C7FB
  \l C7FC
  \l C7FD
  \l C7FE
  \l C7FF
  \l C800
  \l C801
  \l C802
  \l C803
  \l C804
  \l C805
  \l C806
  \l C807
  \l C808
  \l C809
  \l C80A
  \l C80B
  \l C80C
  \l C80D
  \l C80E
  \l C80F
  \l C810
  \l C811
  \l C812
  \l C813
  \l C814
  \l C815
  \l C816
  \l C817
  \l C818
  \l C819
  \l C81A
  \l C81B
  \l C81C
  \l C81D
  \l C81E
  \l C81F
  \l C820
  \l C821
  \l C822
  \l C823
  \l C824
  \l C825
  \l C826
  \l C827
  \l C828
  \l C829
  \l C82A
  \l C82B
  \l C82C
  \l C82D
  \l C82E
  \l C82F
  \l C830
  \l C831
  \l C832
  \l C833
  \l C834
  \l C835
  \l C836
  \l C837
  \l C838
  \l C839
  \l C83A
  \l C83B
  \l C83C
  \l C83D
  \l C83E
  \l C83F
  \l C840
  \l C841
  \l C842
  \l C843
  \l C844
  \l C845
  \l C846
  \l C847
  \l C848
  \l C849
  \l C84A
  \l C84B
  \l C84C
  \l C84D
  \l C84E
  \l C84F
  \l C850
  \l C851
  \l C852
  \l C853
  \l C854
  \l C855
  \l C856
  \l C857
  \l C858
  \l C859
  \l C85A
  \l C85B
  \l C85C
  \l C85D
  \l C85E
  \l C85F
  \l C860
  \l C861
  \l C862
  \l C863
  \l C864
  \l C865
  \l C866
  \l C867
  \l C868
  \l C869
  \l C86A
  \l C86B
  \l C86C
  \l C86D
  \l C86E
  \l C86F
  \l C870
  \l C871
  \l C872
  \l C873
  \l C874
  \l C875
  \l C876
  \l C877
  \l C878
  \l C879
  \l C87A
  \l C87B
  \l C87C
  \l C87D
  \l C87E
  \l C87F
  \l C880
  \l C881
  \l C882
  \l C883
  \l C884
  \l C885
  \l C886
  \l C887
  \l C888
  \l C889
  \l C88A
  \l C88B
  \l C88C
  \l C88D
  \l C88E
  \l C88F
  \l C890
  \l C891
  \l C892
  \l C893
  \l C894
  \l C895
  \l C896
  \l C897
  \l C898
  \l C899
  \l C89A
  \l C89B
  \l C89C
  \l C89D
  \l C89E
  \l C89F
  \l C8A0
  \l C8A1
  \l C8A2
  \l C8A3
  \l C8A4
  \l C8A5
  \l C8A6
  \l C8A7
  \l C8A8
  \l C8A9
  \l C8AA
  \l C8AB
  \l C8AC
  \l C8AD
  \l C8AE
  \l C8AF
  \l C8B0
  \l C8B1
  \l C8B2
  \l C8B3
  \l C8B4
  \l C8B5
  \l C8B6
  \l C8B7
  \l C8B8
  \l C8B9
  \l C8BA
  \l C8BB
  \l C8BC
  \l C8BD
  \l C8BE
  \l C8BF
  \l C8C0
  \l C8C1
  \l C8C2
  \l C8C3
  \l C8C4
  \l C8C5
  \l C8C6
  \l C8C7
  \l C8C8
  \l C8C9
  \l C8CA
  \l C8CB
  \l C8CC
  \l C8CD
  \l C8CE
  \l C8CF
  \l C8D0
  \l C8D1
  \l C8D2
  \l C8D3
  \l C8D4
  \l C8D5
  \l C8D6
  \l C8D7
  \l C8D8
  \l C8D9
  \l C8DA
  \l C8DB
  \l C8DC
  \l C8DD
  \l C8DE
  \l C8DF
  \l C8E0
  \l C8E1
  \l C8E2
  \l C8E3
  \l C8E4
  \l C8E5
  \l C8E6
  \l C8E7
  \l C8E8
  \l C8E9
  \l C8EA
  \l C8EB
  \l C8EC
  \l C8ED
  \l C8EE
  \l C8EF
  \l C8F0
  \l C8F1
  \l C8F2
  \l C8F3
  \l C8F4
  \l C8F5
  \l C8F6
  \l C8F7
  \l C8F8
  \l C8F9
  \l C8FA
  \l C8FB
  \l C8FC
  \l C8FD
  \l C8FE
  \l C8FF
  \l C900
  \l C901
  \l C902
  \l C903
  \l C904
  \l C905
  \l C906
  \l C907
  \l C908
  \l C909
  \l C90A
  \l C90B
  \l C90C
  \l C90D
  \l C90E
  \l C90F
  \l C910
  \l C911
  \l C912
  \l C913
  \l C914
  \l C915
  \l C916
  \l C917
  \l C918
  \l C919
  \l C91A
  \l C91B
  \l C91C
  \l C91D
  \l C91E
  \l C91F
  \l C920
  \l C921
  \l C922
  \l C923
  \l C924
  \l C925
  \l C926
  \l C927
  \l C928
  \l C929
  \l C92A
  \l C92B
  \l C92C
  \l C92D
  \l C92E
  \l C92F
  \l C930
  \l C931
  \l C932
  \l C933
  \l C934
  \l C935
  \l C936
  \l C937
  \l C938
  \l C939
  \l C93A
  \l C93B
  \l C93C
  \l C93D
  \l C93E
  \l C93F
  \l C940
  \l C941
  \l C942
  \l C943
  \l C944
  \l C945
  \l C946
  \l C947
  \l C948
  \l C949
  \l C94A
  \l C94B
  \l C94C
  \l C94D
  \l C94E
  \l C94F
  \l C950
  \l C951
  \l C952
  \l C953
  \l C954
  \l C955
  \l C956
  \l C957
  \l C958
  \l C959
  \l C95A
  \l C95B
  \l C95C
  \l C95D
  \l C95E
  \l C95F
  \l C960
  \l C961
  \l C962
  \l C963
  \l C964
  \l C965
  \l C966
  \l C967
  \l C968
  \l C969
  \l C96A
  \l C96B
  \l C96C
  \l C96D
  \l C96E
  \l C96F
  \l C970
  \l C971
  \l C972
  \l C973
  \l C974
  \l C975
  \l C976
  \l C977
  \l C978
  \l C979
  \l C97A
  \l C97B
  \l C97C
  \l C97D
  \l C97E
  \l C97F
  \l C980
  \l C981
  \l C982
  \l C983
  \l C984
  \l C985
  \l C986
  \l C987
  \l C988
  \l C989
  \l C98A
  \l C98B
  \l C98C
  \l C98D
  \l C98E
  \l C98F
  \l C990
  \l C991
  \l C992
  \l C993
  \l C994
  \l C995
  \l C996
  \l C997
  \l C998
  \l C999
  \l C99A
  \l C99B
  \l C99C
  \l C99D
  \l C99E
  \l C99F
  \l C9A0
  \l C9A1
  \l C9A2
  \l C9A3
  \l C9A4
  \l C9A5
  \l C9A6
  \l C9A7
  \l C9A8
  \l C9A9
  \l C9AA
  \l C9AB
  \l C9AC
  \l C9AD
  \l C9AE
  \l C9AF
  \l C9B0
  \l C9B1
  \l C9B2
  \l C9B3
  \l C9B4
  \l C9B5
  \l C9B6
  \l C9B7
  \l C9B8
  \l C9B9
  \l C9BA
  \l C9BB
  \l C9BC
  \l C9BD
  \l C9BE
  \l C9BF
  \l C9C0
  \l C9C1
  \l C9C2
  \l C9C3
  \l C9C4
  \l C9C5
  \l C9C6
  \l C9C7
  \l C9C8
  \l C9C9
  \l C9CA
  \l C9CB
  \l C9CC
  \l C9CD
  \l C9CE
  \l C9CF
  \l C9D0
  \l C9D1
  \l C9D2
  \l C9D3
  \l C9D4
  \l C9D5
  \l C9D6
  \l C9D7
  \l C9D8
  \l C9D9
  \l C9DA
  \l C9DB
  \l C9DC
  \l C9DD
  \l C9DE
  \l C9DF
  \l C9E0
  \l C9E1
  \l C9E2
  \l C9E3
  \l C9E4
  \l C9E5
  \l C9E6
  \l C9E7
  \l C9E8
  \l C9E9
  \l C9EA
  \l C9EB
  \l C9EC
  \l C9ED
  \l C9EE
  \l C9EF
  \l C9F0
  \l C9F1
  \l C9F2
  \l C9F3
  \l C9F4
  \l C9F5
  \l C9F6
  \l C9F7
  \l C9F8
  \l C9F9
  \l C9FA
  \l C9FB
  \l C9FC
  \l C9FD
  \l C9FE
  \l C9FF
  \l CA00
  \l CA01
  \l CA02
  \l CA03
  \l CA04
  \l CA05
  \l CA06
  \l CA07
  \l CA08
  \l CA09
  \l CA0A
  \l CA0B
  \l CA0C
  \l CA0D
  \l CA0E
  \l CA0F
  \l CA10
  \l CA11
  \l CA12
  \l CA13
  \l CA14
  \l CA15
  \l CA16
  \l CA17
  \l CA18
  \l CA19
  \l CA1A
  \l CA1B
  \l CA1C
  \l CA1D
  \l CA1E
  \l CA1F
  \l CA20
  \l CA21
  \l CA22
  \l CA23
  \l CA24
  \l CA25
  \l CA26
  \l CA27
  \l CA28
  \l CA29
  \l CA2A
  \l CA2B
  \l CA2C
  \l CA2D
  \l CA2E
  \l CA2F
  \l CA30
  \l CA31
  \l CA32
  \l CA33
  \l CA34
  \l CA35
  \l CA36
  \l CA37
  \l CA38
  \l CA39
  \l CA3A
  \l CA3B
  \l CA3C
  \l CA3D
  \l CA3E
  \l CA3F
  \l CA40
  \l CA41
  \l CA42
  \l CA43
  \l CA44
  \l CA45
  \l CA46
  \l CA47
  \l CA48
  \l CA49
  \l CA4A
  \l CA4B
  \l CA4C
  \l CA4D
  \l CA4E
  \l CA4F
  \l CA50
  \l CA51
  \l CA52
  \l CA53
  \l CA54
  \l CA55
  \l CA56
  \l CA57
  \l CA58
  \l CA59
  \l CA5A
  \l CA5B
  \l CA5C
  \l CA5D
  \l CA5E
  \l CA5F
  \l CA60
  \l CA61
  \l CA62
  \l CA63
  \l CA64
  \l CA65
  \l CA66
  \l CA67
  \l CA68
  \l CA69
  \l CA6A
  \l CA6B
  \l CA6C
  \l CA6D
  \l CA6E
  \l CA6F
  \l CA70
  \l CA71
  \l CA72
  \l CA73
  \l CA74
  \l CA75
  \l CA76
  \l CA77
  \l CA78
  \l CA79
  \l CA7A
  \l CA7B
  \l CA7C
  \l CA7D
  \l CA7E
  \l CA7F
  \l CA80
  \l CA81
  \l CA82
  \l CA83
  \l CA84
  \l CA85
  \l CA86
  \l CA87
  \l CA88
  \l CA89
  \l CA8A
  \l CA8B
  \l CA8C
  \l CA8D
  \l CA8E
  \l CA8F
  \l CA90
  \l CA91
  \l CA92
  \l CA93
  \l CA94
  \l CA95
  \l CA96
  \l CA97
  \l CA98
  \l CA99
  \l CA9A
  \l CA9B
  \l CA9C
  \l CA9D
  \l CA9E
  \l CA9F
  \l CAA0
  \l CAA1
  \l CAA2
  \l CAA3
  \l CAA4
  \l CAA5
  \l CAA6
  \l CAA7
  \l CAA8
  \l CAA9
  \l CAAA
  \l CAAB
  \l CAAC
  \l CAAD
  \l CAAE
  \l CAAF
  \l CAB0
  \l CAB1
  \l CAB2
  \l CAB3
  \l CAB4
  \l CAB5
  \l CAB6
  \l CAB7
  \l CAB8
  \l CAB9
  \l CABA
  \l CABB
  \l CABC
  \l CABD
  \l CABE
  \l CABF
  \l CAC0
  \l CAC1
  \l CAC2
  \l CAC3
  \l CAC4
  \l CAC5
  \l CAC6
  \l CAC7
  \l CAC8
  \l CAC9
  \l CACA
  \l CACB
  \l CACC
  \l CACD
  \l CACE
  \l CACF
  \l CAD0
  \l CAD1
  \l CAD2
  \l CAD3
  \l CAD4
  \l CAD5
  \l CAD6
  \l CAD7
  \l CAD8
  \l CAD9
  \l CADA
  \l CADB
  \l CADC
  \l CADD
  \l CADE
  \l CADF
  \l CAE0
  \l CAE1
  \l CAE2
  \l CAE3
  \l CAE4
  \l CAE5
  \l CAE6
  \l CAE7
  \l CAE8
  \l CAE9
  \l CAEA
  \l CAEB
  \l CAEC
  \l CAED
  \l CAEE
  \l CAEF
  \l CAF0
  \l CAF1
  \l CAF2
  \l CAF3
  \l CAF4
  \l CAF5
  \l CAF6
  \l CAF7
  \l CAF8
  \l CAF9
  \l CAFA
  \l CAFB
  \l CAFC
  \l CAFD
  \l CAFE
  \l CAFF
  \l CB00
  \l CB01
  \l CB02
  \l CB03
  \l CB04
  \l CB05
  \l CB06
  \l CB07
  \l CB08
  \l CB09
  \l CB0A
  \l CB0B
  \l CB0C
  \l CB0D
  \l CB0E
  \l CB0F
  \l CB10
  \l CB11
  \l CB12
  \l CB13
  \l CB14
  \l CB15
  \l CB16
  \l CB17
  \l CB18
  \l CB19
  \l CB1A
  \l CB1B
  \l CB1C
  \l CB1D
  \l CB1E
  \l CB1F
  \l CB20
  \l CB21
  \l CB22
  \l CB23
  \l CB24
  \l CB25
  \l CB26
  \l CB27
  \l CB28
  \l CB29
  \l CB2A
  \l CB2B
  \l CB2C
  \l CB2D
  \l CB2E
  \l CB2F
  \l CB30
  \l CB31
  \l CB32
  \l CB33
  \l CB34
  \l CB35
  \l CB36
  \l CB37
  \l CB38
  \l CB39
  \l CB3A
  \l CB3B
  \l CB3C
  \l CB3D
  \l CB3E
  \l CB3F
  \l CB40
  \l CB41
  \l CB42
  \l CB43
  \l CB44
  \l CB45
  \l CB46
  \l CB47
  \l CB48
  \l CB49
  \l CB4A
  \l CB4B
  \l CB4C
  \l CB4D
  \l CB4E
  \l CB4F
  \l CB50
  \l CB51
  \l CB52
  \l CB53
  \l CB54
  \l CB55
  \l CB56
  \l CB57
  \l CB58
  \l CB59
  \l CB5A
  \l CB5B
  \l CB5C
  \l CB5D
  \l CB5E
  \l CB5F
  \l CB60
  \l CB61
  \l CB62
  \l CB63
  \l CB64
  \l CB65
  \l CB66
  \l CB67
  \l CB68
  \l CB69
  \l CB6A
  \l CB6B
  \l CB6C
  \l CB6D
  \l CB6E
  \l CB6F
  \l CB70
  \l CB71
  \l CB72
  \l CB73
  \l CB74
  \l CB75
  \l CB76
  \l CB77
  \l CB78
  \l CB79
  \l CB7A
  \l CB7B
  \l CB7C
  \l CB7D
  \l CB7E
  \l CB7F
  \l CB80
  \l CB81
  \l CB82
  \l CB83
  \l CB84
  \l CB85
  \l CB86
  \l CB87
  \l CB88
  \l CB89
  \l CB8A
  \l CB8B
  \l CB8C
  \l CB8D
  \l CB8E
  \l CB8F
  \l CB90
  \l CB91
  \l CB92
  \l CB93
  \l CB94
  \l CB95
  \l CB96
  \l CB97
  \l CB98
  \l CB99
  \l CB9A
  \l CB9B
  \l CB9C
  \l CB9D
  \l CB9E
  \l CB9F
  \l CBA0
  \l CBA1
  \l CBA2
  \l CBA3
  \l CBA4
  \l CBA5
  \l CBA6
  \l CBA7
  \l CBA8
  \l CBA9
  \l CBAA
  \l CBAB
  \l CBAC
  \l CBAD
  \l CBAE
  \l CBAF
  \l CBB0
  \l CBB1
  \l CBB2
  \l CBB3
  \l CBB4
  \l CBB5
  \l CBB6
  \l CBB7
  \l CBB8
  \l CBB9
  \l CBBA
  \l CBBB
  \l CBBC
  \l CBBD
  \l CBBE
  \l CBBF
  \l CBC0
  \l CBC1
  \l CBC2
  \l CBC3
  \l CBC4
  \l CBC5
  \l CBC6
  \l CBC7
  \l CBC8
  \l CBC9
  \l CBCA
  \l CBCB
  \l CBCC
  \l CBCD
  \l CBCE
  \l CBCF
  \l CBD0
  \l CBD1
  \l CBD2
  \l CBD3
  \l CBD4
  \l CBD5
  \l CBD6
  \l CBD7
  \l CBD8
  \l CBD9
  \l CBDA
  \l CBDB
  \l CBDC
  \l CBDD
  \l CBDE
  \l CBDF
  \l CBE0
  \l CBE1
  \l CBE2
  \l CBE3
  \l CBE4
  \l CBE5
  \l CBE6
  \l CBE7
  \l CBE8
  \l CBE9
  \l CBEA
  \l CBEB
  \l CBEC
  \l CBED
  \l CBEE
  \l CBEF
  \l CBF0
  \l CBF1
  \l CBF2
  \l CBF3
  \l CBF4
  \l CBF5
  \l CBF6
  \l CBF7
  \l CBF8
  \l CBF9
  \l CBFA
  \l CBFB
  \l CBFC
  \l CBFD
  \l CBFE
  \l CBFF
  \l CC00
  \l CC01
  \l CC02
  \l CC03
  \l CC04
  \l CC05
  \l CC06
  \l CC07
  \l CC08
  \l CC09
  \l CC0A
  \l CC0B
  \l CC0C
  \l CC0D
  \l CC0E
  \l CC0F
  \l CC10
  \l CC11
  \l CC12
  \l CC13
  \l CC14
  \l CC15
  \l CC16
  \l CC17
  \l CC18
  \l CC19
  \l CC1A
  \l CC1B
  \l CC1C
  \l CC1D
  \l CC1E
  \l CC1F
  \l CC20
  \l CC21
  \l CC22
  \l CC23
  \l CC24
  \l CC25
  \l CC26
  \l CC27
  \l CC28
  \l CC29
  \l CC2A
  \l CC2B
  \l CC2C
  \l CC2D
  \l CC2E
  \l CC2F
  \l CC30
  \l CC31
  \l CC32
  \l CC33
  \l CC34
  \l CC35
  \l CC36
  \l CC37
  \l CC38
  \l CC39
  \l CC3A
  \l CC3B
  \l CC3C
  \l CC3D
  \l CC3E
  \l CC3F
  \l CC40
  \l CC41
  \l CC42
  \l CC43
  \l CC44
  \l CC45
  \l CC46
  \l CC47
  \l CC48
  \l CC49
  \l CC4A
  \l CC4B
  \l CC4C
  \l CC4D
  \l CC4E
  \l CC4F
  \l CC50
  \l CC51
  \l CC52
  \l CC53
  \l CC54
  \l CC55
  \l CC56
  \l CC57
  \l CC58
  \l CC59
  \l CC5A
  \l CC5B
  \l CC5C
  \l CC5D
  \l CC5E
  \l CC5F
  \l CC60
  \l CC61
  \l CC62
  \l CC63
  \l CC64
  \l CC65
  \l CC66
  \l CC67
  \l CC68
  \l CC69
  \l CC6A
  \l CC6B
  \l CC6C
  \l CC6D
  \l CC6E
  \l CC6F
  \l CC70
  \l CC71
  \l CC72
  \l CC73
  \l CC74
  \l CC75
  \l CC76
  \l CC77
  \l CC78
  \l CC79
  \l CC7A
  \l CC7B
  \l CC7C
  \l CC7D
  \l CC7E
  \l CC7F
  \l CC80
  \l CC81
  \l CC82
  \l CC83
  \l CC84
  \l CC85
  \l CC86
  \l CC87
  \l CC88
  \l CC89
  \l CC8A
  \l CC8B
  \l CC8C
  \l CC8D
  \l CC8E
  \l CC8F
  \l CC90
  \l CC91
  \l CC92
  \l CC93
  \l CC94
  \l CC95
  \l CC96
  \l CC97
  \l CC98
  \l CC99
  \l CC9A
  \l CC9B
  \l CC9C
  \l CC9D
  \l CC9E
  \l CC9F
  \l CCA0
  \l CCA1
  \l CCA2
  \l CCA3
  \l CCA4
  \l CCA5
  \l CCA6
  \l CCA7
  \l CCA8
  \l CCA9
  \l CCAA
  \l CCAB
  \l CCAC
  \l CCAD
  \l CCAE
  \l CCAF
  \l CCB0
  \l CCB1
  \l CCB2
  \l CCB3
  \l CCB4
  \l CCB5
  \l CCB6
  \l CCB7
  \l CCB8
  \l CCB9
  \l CCBA
  \l CCBB
  \l CCBC
  \l CCBD
  \l CCBE
  \l CCBF
  \l CCC0
  \l CCC1
  \l CCC2
  \l CCC3
  \l CCC4
  \l CCC5
  \l CCC6
  \l CCC7
  \l CCC8
  \l CCC9
  \l CCCA
  \l CCCB
  \l CCCC
  \l CCCD
  \l CCCE
  \l CCCF
  \l CCD0
  \l CCD1
  \l CCD2
  \l CCD3
  \l CCD4
  \l CCD5
  \l CCD6
  \l CCD7
  \l CCD8
  \l CCD9
  \l CCDA
  \l CCDB
  \l CCDC
  \l CCDD
  \l CCDE
  \l CCDF
  \l CCE0
  \l CCE1
  \l CCE2
  \l CCE3
  \l CCE4
  \l CCE5
  \l CCE6
  \l CCE7
  \l CCE8
  \l CCE9
  \l CCEA
  \l CCEB
  \l CCEC
  \l CCED
  \l CCEE
  \l CCEF
  \l CCF0
  \l CCF1
  \l CCF2
  \l CCF3
  \l CCF4
  \l CCF5
  \l CCF6
  \l CCF7
  \l CCF8
  \l CCF9
  \l CCFA
  \l CCFB
  \l CCFC
  \l CCFD
  \l CCFE
  \l CCFF
  \l CD00
  \l CD01
  \l CD02
  \l CD03
  \l CD04
  \l CD05
  \l CD06
  \l CD07
  \l CD08
  \l CD09
  \l CD0A
  \l CD0B
  \l CD0C
  \l CD0D
  \l CD0E
  \l CD0F
  \l CD10
  \l CD11
  \l CD12
  \l CD13
  \l CD14
  \l CD15
  \l CD16
  \l CD17
  \l CD18
  \l CD19
  \l CD1A
  \l CD1B
  \l CD1C
  \l CD1D
  \l CD1E
  \l CD1F
  \l CD20
  \l CD21
  \l CD22
  \l CD23
  \l CD24
  \l CD25
  \l CD26
  \l CD27
  \l CD28
  \l CD29
  \l CD2A
  \l CD2B
  \l CD2C
  \l CD2D
  \l CD2E
  \l CD2F
  \l CD30
  \l CD31
  \l CD32
  \l CD33
  \l CD34
  \l CD35
  \l CD36
  \l CD37
  \l CD38
  \l CD39
  \l CD3A
  \l CD3B
  \l CD3C
  \l CD3D
  \l CD3E
  \l CD3F
  \l CD40
  \l CD41
  \l CD42
  \l CD43
  \l CD44
  \l CD45
  \l CD46
  \l CD47
  \l CD48
  \l CD49
  \l CD4A
  \l CD4B
  \l CD4C
  \l CD4D
  \l CD4E
  \l CD4F
  \l CD50
  \l CD51
  \l CD52
  \l CD53
  \l CD54
  \l CD55
  \l CD56
  \l CD57
  \l CD58
  \l CD59
  \l CD5A
  \l CD5B
  \l CD5C
  \l CD5D
  \l CD5E
  \l CD5F
  \l CD60
  \l CD61
  \l CD62
  \l CD63
  \l CD64
  \l CD65
  \l CD66
  \l CD67
  \l CD68
  \l CD69
  \l CD6A
  \l CD6B
  \l CD6C
  \l CD6D
  \l CD6E
  \l CD6F
  \l CD70
  \l CD71
  \l CD72
  \l CD73
  \l CD74
  \l CD75
  \l CD76
  \l CD77
  \l CD78
  \l CD79
  \l CD7A
  \l CD7B
  \l CD7C
  \l CD7D
  \l CD7E
  \l CD7F
  \l CD80
  \l CD81
  \l CD82
  \l CD83
  \l CD84
  \l CD85
  \l CD86
  \l CD87
  \l CD88
  \l CD89
  \l CD8A
  \l CD8B
  \l CD8C
  \l CD8D
  \l CD8E
  \l CD8F
  \l CD90
  \l CD91
  \l CD92
  \l CD93
  \l CD94
  \l CD95
  \l CD96
  \l CD97
  \l CD98
  \l CD99
  \l CD9A
  \l CD9B
  \l CD9C
  \l CD9D
  \l CD9E
  \l CD9F
  \l CDA0
  \l CDA1
  \l CDA2
  \l CDA3
  \l CDA4
  \l CDA5
  \l CDA6
  \l CDA7
  \l CDA8
  \l CDA9
  \l CDAA
  \l CDAB
  \l CDAC
  \l CDAD
  \l CDAE
  \l CDAF
  \l CDB0
  \l CDB1
  \l CDB2
  \l CDB3
  \l CDB4
  \l CDB5
  \l CDB6
  \l CDB7
  \l CDB8
  \l CDB9
  \l CDBA
  \l CDBB
  \l CDBC
  \l CDBD
  \l CDBE
  \l CDBF
  \l CDC0
  \l CDC1
  \l CDC2
  \l CDC3
  \l CDC4
  \l CDC5
  \l CDC6
  \l CDC7
  \l CDC8
  \l CDC9
  \l CDCA
  \l CDCB
  \l CDCC
  \l CDCD
  \l CDCE
  \l CDCF
  \l CDD0
  \l CDD1
  \l CDD2
  \l CDD3
  \l CDD4
  \l CDD5
  \l CDD6
  \l CDD7
  \l CDD8
  \l CDD9
  \l CDDA
  \l CDDB
  \l CDDC
  \l CDDD
  \l CDDE
  \l CDDF
  \l CDE0
  \l CDE1
  \l CDE2
  \l CDE3
  \l CDE4
  \l CDE5
  \l CDE6
  \l CDE7
  \l CDE8
  \l CDE9
  \l CDEA
  \l CDEB
  \l CDEC
  \l CDED
  \l CDEE
  \l CDEF
  \l CDF0
  \l CDF1
  \l CDF2
  \l CDF3
  \l CDF4
  \l CDF5
  \l CDF6
  \l CDF7
  \l CDF8
  \l CDF9
  \l CDFA
  \l CDFB
  \l CDFC
  \l CDFD
  \l CDFE
  \l CDFF
  \l CE00
  \l CE01
  \l CE02
  \l CE03
  \l CE04
  \l CE05
  \l CE06
  \l CE07
  \l CE08
  \l CE09
  \l CE0A
  \l CE0B
  \l CE0C
  \l CE0D
  \l CE0E
  \l CE0F
  \l CE10
  \l CE11
  \l CE12
  \l CE13
  \l CE14
  \l CE15
  \l CE16
  \l CE17
  \l CE18
  \l CE19
  \l CE1A
  \l CE1B
  \l CE1C
  \l CE1D
  \l CE1E
  \l CE1F
  \l CE20
  \l CE21
  \l CE22
  \l CE23
  \l CE24
  \l CE25
  \l CE26
  \l CE27
  \l CE28
  \l CE29
  \l CE2A
  \l CE2B
  \l CE2C
  \l CE2D
  \l CE2E
  \l CE2F
  \l CE30
  \l CE31
  \l CE32
  \l CE33
  \l CE34
  \l CE35
  \l CE36
  \l CE37
  \l CE38
  \l CE39
  \l CE3A
  \l CE3B
  \l CE3C
  \l CE3D
  \l CE3E
  \l CE3F
  \l CE40
  \l CE41
  \l CE42
  \l CE43
  \l CE44
  \l CE45
  \l CE46
  \l CE47
  \l CE48
  \l CE49
  \l CE4A
  \l CE4B
  \l CE4C
  \l CE4D
  \l CE4E
  \l CE4F
  \l CE50
  \l CE51
  \l CE52
  \l CE53
  \l CE54
  \l CE55
  \l CE56
  \l CE57
  \l CE58
  \l CE59
  \l CE5A
  \l CE5B
  \l CE5C
  \l CE5D
  \l CE5E
  \l CE5F
  \l CE60
  \l CE61
  \l CE62
  \l CE63
  \l CE64
  \l CE65
  \l CE66
  \l CE67
  \l CE68
  \l CE69
  \l CE6A
  \l CE6B
  \l CE6C
  \l CE6D
  \l CE6E
  \l CE6F
  \l CE70
  \l CE71
  \l CE72
  \l CE73
  \l CE74
  \l CE75
  \l CE76
  \l CE77
  \l CE78
  \l CE79
  \l CE7A
  \l CE7B
  \l CE7C
  \l CE7D
  \l CE7E
  \l CE7F
  \l CE80
  \l CE81
  \l CE82
  \l CE83
  \l CE84
  \l CE85
  \l CE86
  \l CE87
  \l CE88
  \l CE89
  \l CE8A
  \l CE8B
  \l CE8C
  \l CE8D
  \l CE8E
  \l CE8F
  \l CE90
  \l CE91
  \l CE92
  \l CE93
  \l CE94
  \l CE95
  \l CE96
  \l CE97
  \l CE98
  \l CE99
  \l CE9A
  \l CE9B
  \l CE9C
  \l CE9D
  \l CE9E
  \l CE9F
  \l CEA0
  \l CEA1
  \l CEA2
  \l CEA3
  \l CEA4
  \l CEA5
  \l CEA6
  \l CEA7
  \l CEA8
  \l CEA9
  \l CEAA
  \l CEAB
  \l CEAC
  \l CEAD
  \l CEAE
  \l CEAF
  \l CEB0
  \l CEB1
  \l CEB2
  \l CEB3
  \l CEB4
  \l CEB5
  \l CEB6
  \l CEB7
  \l CEB8
  \l CEB9
  \l CEBA
  \l CEBB
  \l CEBC
  \l CEBD
  \l CEBE
  \l CEBF
  \l CEC0
  \l CEC1
  \l CEC2
  \l CEC3
  \l CEC4
  \l CEC5
  \l CEC6
  \l CEC7
  \l CEC8
  \l CEC9
  \l CECA
  \l CECB
  \l CECC
  \l CECD
  \l CECE
  \l CECF
  \l CED0
  \l CED1
  \l CED2
  \l CED3
  \l CED4
  \l CED5
  \l CED6
  \l CED7
  \l CED8
  \l CED9
  \l CEDA
  \l CEDB
  \l CEDC
  \l CEDD
  \l CEDE
  \l CEDF
  \l CEE0
  \l CEE1
  \l CEE2
  \l CEE3
  \l CEE4
  \l CEE5
  \l CEE6
  \l CEE7
  \l CEE8
  \l CEE9
  \l CEEA
  \l CEEB
  \l CEEC
  \l CEED
  \l CEEE
  \l CEEF
  \l CEF0
  \l CEF1
  \l CEF2
  \l CEF3
  \l CEF4
  \l CEF5
  \l CEF6
  \l CEF7
  \l CEF8
  \l CEF9
  \l CEFA
  \l CEFB
  \l CEFC
  \l CEFD
  \l CEFE
  \l CEFF
  \l CF00
  \l CF01
  \l CF02
  \l CF03
  \l CF04
  \l CF05
  \l CF06
  \l CF07
  \l CF08
  \l CF09
  \l CF0A
  \l CF0B
  \l CF0C
  \l CF0D
  \l CF0E
  \l CF0F
  \l CF10
  \l CF11
  \l CF12
  \l CF13
  \l CF14
  \l CF15
  \l CF16
  \l CF17
  \l CF18
  \l CF19
  \l CF1A
  \l CF1B
  \l CF1C
  \l CF1D
  \l CF1E
  \l CF1F
  \l CF20
  \l CF21
  \l CF22
  \l CF23
  \l CF24
  \l CF25
  \l CF26
  \l CF27
  \l CF28
  \l CF29
  \l CF2A
  \l CF2B
  \l CF2C
  \l CF2D
  \l CF2E
  \l CF2F
  \l CF30
  \l CF31
  \l CF32
  \l CF33
  \l CF34
  \l CF35
  \l CF36
  \l CF37
  \l CF38
  \l CF39
  \l CF3A
  \l CF3B
  \l CF3C
  \l CF3D
  \l CF3E
  \l CF3F
  \l CF40
  \l CF41
  \l CF42
  \l CF43
  \l CF44
  \l CF45
  \l CF46
  \l CF47
  \l CF48
  \l CF49
  \l CF4A
  \l CF4B
  \l CF4C
  \l CF4D
  \l CF4E
  \l CF4F
  \l CF50
  \l CF51
  \l CF52
  \l CF53
  \l CF54
  \l CF55
  \l CF56
  \l CF57
  \l CF58
  \l CF59
  \l CF5A
  \l CF5B
  \l CF5C
  \l CF5D
  \l CF5E
  \l CF5F
  \l CF60
  \l CF61
  \l CF62
  \l CF63
  \l CF64
  \l CF65
  \l CF66
  \l CF67
  \l CF68
  \l CF69
  \l CF6A
  \l CF6B
  \l CF6C
  \l CF6D
  \l CF6E
  \l CF6F
  \l CF70
  \l CF71
  \l CF72
  \l CF73
  \l CF74
  \l CF75
  \l CF76
  \l CF77
  \l CF78
  \l CF79
  \l CF7A
  \l CF7B
  \l CF7C
  \l CF7D
  \l CF7E
  \l CF7F
  \l CF80
  \l CF81
  \l CF82
  \l CF83
  \l CF84
  \l CF85
  \l CF86
  \l CF87
  \l CF88
  \l CF89
  \l CF8A
  \l CF8B
  \l CF8C
  \l CF8D
  \l CF8E
  \l CF8F
  \l CF90
  \l CF91
  \l CF92
  \l CF93
  \l CF94
  \l CF95
  \l CF96
  \l CF97
  \l CF98
  \l CF99
  \l CF9A
  \l CF9B
  \l CF9C
  \l CF9D
  \l CF9E
  \l CF9F
  \l CFA0
  \l CFA1
  \l CFA2
  \l CFA3
  \l CFA4
  \l CFA5
  \l CFA6
  \l CFA7
  \l CFA8
  \l CFA9
  \l CFAA
  \l CFAB
  \l CFAC
  \l CFAD
  \l CFAE
  \l CFAF
  \l CFB0
  \l CFB1
  \l CFB2
  \l CFB3
  \l CFB4
  \l CFB5
  \l CFB6
  \l CFB7
  \l CFB8
  \l CFB9
  \l CFBA
  \l CFBB
  \l CFBC
  \l CFBD
  \l CFBE
  \l CFBF
  \l CFC0
  \l CFC1
  \l CFC2
  \l CFC3
  \l CFC4
  \l CFC5
  \l CFC6
  \l CFC7
  \l CFC8
  \l CFC9
  \l CFCA
  \l CFCB
  \l CFCC
  \l CFCD
  \l CFCE
  \l CFCF
  \l CFD0
  \l CFD1
  \l CFD2
  \l CFD3
  \l CFD4
  \l CFD5
  \l CFD6
  \l CFD7
  \l CFD8
  \l CFD9
  \l CFDA
  \l CFDB
  \l CFDC
  \l CFDD
  \l CFDE
  \l CFDF
  \l CFE0
  \l CFE1
  \l CFE2
  \l CFE3
  \l CFE4
  \l CFE5
  \l CFE6
  \l CFE7
  \l CFE8
  \l CFE9
  \l CFEA
  \l CFEB
  \l CFEC
  \l CFED
  \l CFEE
  \l CFEF
  \l CFF0
  \l CFF1
  \l CFF2
  \l CFF3
  \l CFF4
  \l CFF5
  \l CFF6
  \l CFF7
  \l CFF8
  \l CFF9
  \l CFFA
  \l CFFB
  \l CFFC
  \l CFFD
  \l CFFE
  \l CFFF
  \l D000
  \l D001
  \l D002
  \l D003
  \l D004
  \l D005
  \l D006
  \l D007
  \l D008
  \l D009
  \l D00A
  \l D00B
  \l D00C
  \l D00D
  \l D00E
  \l D00F
  \l D010
  \l D011
  \l D012
  \l D013
  \l D014
  \l D015
  \l D016
  \l D017
  \l D018
  \l D019
  \l D01A
  \l D01B
  \l D01C
  \l D01D
  \l D01E
  \l D01F
  \l D020
  \l D021
  \l D022
  \l D023
  \l D024
  \l D025
  \l D026
  \l D027
  \l D028
  \l D029
  \l D02A
  \l D02B
  \l D02C
  \l D02D
  \l D02E
  \l D02F
  \l D030
  \l D031
  \l D032
  \l D033
  \l D034
  \l D035
  \l D036
  \l D037
  \l D038
  \l D039
  \l D03A
  \l D03B
  \l D03C
  \l D03D
  \l D03E
  \l D03F
  \l D040
  \l D041
  \l D042
  \l D043
  \l D044
  \l D045
  \l D046
  \l D047
  \l D048
  \l D049
  \l D04A
  \l D04B
  \l D04C
  \l D04D
  \l D04E
  \l D04F
  \l D050
  \l D051
  \l D052
  \l D053
  \l D054
  \l D055
  \l D056
  \l D057
  \l D058
  \l D059
  \l D05A
  \l D05B
  \l D05C
  \l D05D
  \l D05E
  \l D05F
  \l D060
  \l D061
  \l D062
  \l D063
  \l D064
  \l D065
  \l D066
  \l D067
  \l D068
  \l D069
  \l D06A
  \l D06B
  \l D06C
  \l D06D
  \l D06E
  \l D06F
  \l D070
  \l D071
  \l D072
  \l D073
  \l D074
  \l D075
  \l D076
  \l D077
  \l D078
  \l D079
  \l D07A
  \l D07B
  \l D07C
  \l D07D
  \l D07E
  \l D07F
  \l D080
  \l D081
  \l D082
  \l D083
  \l D084
  \l D085
  \l D086
  \l D087
  \l D088
  \l D089
  \l D08A
  \l D08B
  \l D08C
  \l D08D
  \l D08E
  \l D08F
  \l D090
  \l D091
  \l D092
  \l D093
  \l D094
  \l D095
  \l D096
  \l D097
  \l D098
  \l D099
  \l D09A
  \l D09B
  \l D09C
  \l D09D
  \l D09E
  \l D09F
  \l D0A0
  \l D0A1
  \l D0A2
  \l D0A3
  \l D0A4
  \l D0A5
  \l D0A6
  \l D0A7
  \l D0A8
  \l D0A9
  \l D0AA
  \l D0AB
  \l D0AC
  \l D0AD
  \l D0AE
  \l D0AF
  \l D0B0
  \l D0B1
  \l D0B2
  \l D0B3
  \l D0B4
  \l D0B5
  \l D0B6
  \l D0B7
  \l D0B8
  \l D0B9
  \l D0BA
  \l D0BB
  \l D0BC
  \l D0BD
  \l D0BE
  \l D0BF
  \l D0C0
  \l D0C1
  \l D0C2
  \l D0C3
  \l D0C4
  \l D0C5
  \l D0C6
  \l D0C7
  \l D0C8
  \l D0C9
  \l D0CA
  \l D0CB
  \l D0CC
  \l D0CD
  \l D0CE
  \l D0CF
  \l D0D0
  \l D0D1
  \l D0D2
  \l D0D3
  \l D0D4
  \l D0D5
  \l D0D6
  \l D0D7
  \l D0D8
  \l D0D9
  \l D0DA
  \l D0DB
  \l D0DC
  \l D0DD
  \l D0DE
  \l D0DF
  \l D0E0
  \l D0E1
  \l D0E2
  \l D0E3
  \l D0E4
  \l D0E5
  \l D0E6
  \l D0E7
  \l D0E8
  \l D0E9
  \l D0EA
  \l D0EB
  \l D0EC
  \l D0ED
  \l D0EE
  \l D0EF
  \l D0F0
  \l D0F1
  \l D0F2
  \l D0F3
  \l D0F4
  \l D0F5
  \l D0F6
  \l D0F7
  \l D0F8
  \l D0F9
  \l D0FA
  \l D0FB
  \l D0FC
  \l D0FD
  \l D0FE
  \l D0FF
  \l D100
  \l D101
  \l D102
  \l D103
  \l D104
  \l D105
  \l D106
  \l D107
  \l D108
  \l D109
  \l D10A
  \l D10B
  \l D10C
  \l D10D
  \l D10E
  \l D10F
  \l D110
  \l D111
  \l D112
  \l D113
  \l D114
  \l D115
  \l D116
  \l D117
  \l D118
  \l D119
  \l D11A
  \l D11B
  \l D11C
  \l D11D
  \l D11E
  \l D11F
  \l D120
  \l D121
  \l D122
  \l D123
  \l D124
  \l D125
  \l D126
  \l D127
  \l D128
  \l D129
  \l D12A
  \l D12B
  \l D12C
  \l D12D
  \l D12E
  \l D12F
  \l D130
  \l D131
  \l D132
  \l D133
  \l D134
  \l D135
  \l D136
  \l D137
  \l D138
  \l D139
  \l D13A
  \l D13B
  \l D13C
  \l D13D
  \l D13E
  \l D13F
  \l D140
  \l D141
  \l D142
  \l D143
  \l D144
  \l D145
  \l D146
  \l D147
  \l D148
  \l D149
  \l D14A
  \l D14B
  \l D14C
  \l D14D
  \l D14E
  \l D14F
  \l D150
  \l D151
  \l D152
  \l D153
  \l D154
  \l D155
  \l D156
  \l D157
  \l D158
  \l D159
  \l D15A
  \l D15B
  \l D15C
  \l D15D
  \l D15E
  \l D15F
  \l D160
  \l D161
  \l D162
  \l D163
  \l D164
  \l D165
  \l D166
  \l D167
  \l D168
  \l D169
  \l D16A
  \l D16B
  \l D16C
  \l D16D
  \l D16E
  \l D16F
  \l D170
  \l D171
  \l D172
  \l D173
  \l D174
  \l D175
  \l D176
  \l D177
  \l D178
  \l D179
  \l D17A
  \l D17B
  \l D17C
  \l D17D
  \l D17E
  \l D17F
  \l D180
  \l D181
  \l D182
  \l D183
  \l D184
  \l D185
  \l D186
  \l D187
  \l D188
  \l D189
  \l D18A
  \l D18B
  \l D18C
  \l D18D
  \l D18E
  \l D18F
  \l D190
  \l D191
  \l D192
  \l D193
  \l D194
  \l D195
  \l D196
  \l D197
  \l D198
  \l D199
  \l D19A
  \l D19B
  \l D19C
  \l D19D
  \l D19E
  \l D19F
  \l D1A0
  \l D1A1
  \l D1A2
  \l D1A3
  \l D1A4
  \l D1A5
  \l D1A6
  \l D1A7
  \l D1A8
  \l D1A9
  \l D1AA
  \l D1AB
  \l D1AC
  \l D1AD
  \l D1AE
  \l D1AF
  \l D1B0
  \l D1B1
  \l D1B2
  \l D1B3
  \l D1B4
  \l D1B5
  \l D1B6
  \l D1B7
  \l D1B8
  \l D1B9
  \l D1BA
  \l D1BB
  \l D1BC
  \l D1BD
  \l D1BE
  \l D1BF
  \l D1C0
  \l D1C1
  \l D1C2
  \l D1C3
  \l D1C4
  \l D1C5
  \l D1C6
  \l D1C7
  \l D1C8
  \l D1C9
  \l D1CA
  \l D1CB
  \l D1CC
  \l D1CD
  \l D1CE
  \l D1CF
  \l D1D0
  \l D1D1
  \l D1D2
  \l D1D3
  \l D1D4
  \l D1D5
  \l D1D6
  \l D1D7
  \l D1D8
  \l D1D9
  \l D1DA
  \l D1DB
  \l D1DC
  \l D1DD
  \l D1DE
  \l D1DF
  \l D1E0
  \l D1E1
  \l D1E2
  \l D1E3
  \l D1E4
  \l D1E5
  \l D1E6
  \l D1E7
  \l D1E8
  \l D1E9
  \l D1EA
  \l D1EB
  \l D1EC
  \l D1ED
  \l D1EE
  \l D1EF
  \l D1F0
  \l D1F1
  \l D1F2
  \l D1F3
  \l D1F4
  \l D1F5
  \l D1F6
  \l D1F7
  \l D1F8
  \l D1F9
  \l D1FA
  \l D1FB
  \l D1FC
  \l D1FD
  \l D1FE
  \l D1FF
  \l D200
  \l D201
  \l D202
  \l D203
  \l D204
  \l D205
  \l D206
  \l D207
  \l D208
  \l D209
  \l D20A
  \l D20B
  \l D20C
  \l D20D
  \l D20E
  \l D20F
  \l D210
  \l D211
  \l D212
  \l D213
  \l D214
  \l D215
  \l D216
  \l D217
  \l D218
  \l D219
  \l D21A
  \l D21B
  \l D21C
  \l D21D
  \l D21E
  \l D21F
  \l D220
  \l D221
  \l D222
  \l D223
  \l D224
  \l D225
  \l D226
  \l D227
  \l D228
  \l D229
  \l D22A
  \l D22B
  \l D22C
  \l D22D
  \l D22E
  \l D22F
  \l D230
  \l D231
  \l D232
  \l D233
  \l D234
  \l D235
  \l D236
  \l D237
  \l D238
  \l D239
  \l D23A
  \l D23B
  \l D23C
  \l D23D
  \l D23E
  \l D23F
  \l D240
  \l D241
  \l D242
  \l D243
  \l D244
  \l D245
  \l D246
  \l D247
  \l D248
  \l D249
  \l D24A
  \l D24B
  \l D24C
  \l D24D
  \l D24E
  \l D24F
  \l D250
  \l D251
  \l D252
  \l D253
  \l D254
  \l D255
  \l D256
  \l D257
  \l D258
  \l D259
  \l D25A
  \l D25B
  \l D25C
  \l D25D
  \l D25E
  \l D25F
  \l D260
  \l D261
  \l D262
  \l D263
  \l D264
  \l D265
  \l D266
  \l D267
  \l D268
  \l D269
  \l D26A
  \l D26B
  \l D26C
  \l D26D
  \l D26E
  \l D26F
  \l D270
  \l D271
  \l D272
  \l D273
  \l D274
  \l D275
  \l D276
  \l D277
  \l D278
  \l D279
  \l D27A
  \l D27B
  \l D27C
  \l D27D
  \l D27E
  \l D27F
  \l D280
  \l D281
  \l D282
  \l D283
  \l D284
  \l D285
  \l D286
  \l D287
  \l D288
  \l D289
  \l D28A
  \l D28B
  \l D28C
  \l D28D
  \l D28E
  \l D28F
  \l D290
  \l D291
  \l D292
  \l D293
  \l D294
  \l D295
  \l D296
  \l D297
  \l D298
  \l D299
  \l D29A
  \l D29B
  \l D29C
  \l D29D
  \l D29E
  \l D29F
  \l D2A0
  \l D2A1
  \l D2A2
  \l D2A3
  \l D2A4
  \l D2A5
  \l D2A6
  \l D2A7
  \l D2A8
  \l D2A9
  \l D2AA
  \l D2AB
  \l D2AC
  \l D2AD
  \l D2AE
  \l D2AF
  \l D2B0
  \l D2B1
  \l D2B2
  \l D2B3
  \l D2B4
  \l D2B5
  \l D2B6
  \l D2B7
  \l D2B8
  \l D2B9
  \l D2BA
  \l D2BB
  \l D2BC
  \l D2BD
  \l D2BE
  \l D2BF
  \l D2C0
  \l D2C1
  \l D2C2
  \l D2C3
  \l D2C4
  \l D2C5
  \l D2C6
  \l D2C7
  \l D2C8
  \l D2C9
  \l D2CA
  \l D2CB
  \l D2CC
  \l D2CD
  \l D2CE
  \l D2CF
  \l D2D0
  \l D2D1
  \l D2D2
  \l D2D3
  \l D2D4
  \l D2D5
  \l D2D6
  \l D2D7
  \l D2D8
  \l D2D9
  \l D2DA
  \l D2DB
  \l D2DC
  \l D2DD
  \l D2DE
  \l D2DF
  \l D2E0
  \l D2E1
  \l D2E2
  \l D2E3
  \l D2E4
  \l D2E5
  \l D2E6
  \l D2E7
  \l D2E8
  \l D2E9
  \l D2EA
  \l D2EB
  \l D2EC
  \l D2ED
  \l D2EE
  \l D2EF
  \l D2F0
  \l D2F1
  \l D2F2
  \l D2F3
  \l D2F4
  \l D2F5
  \l D2F6
  \l D2F7
  \l D2F8
  \l D2F9
  \l D2FA
  \l D2FB
  \l D2FC
  \l D2FD
  \l D2FE
  \l D2FF
  \l D300
  \l D301
  \l D302
  \l D303
  \l D304
  \l D305
  \l D306
  \l D307
  \l D308
  \l D309
  \l D30A
  \l D30B
  \l D30C
  \l D30D
  \l D30E
  \l D30F
  \l D310
  \l D311
  \l D312
  \l D313
  \l D314
  \l D315
  \l D316
  \l D317
  \l D318
  \l D319
  \l D31A
  \l D31B
  \l D31C
  \l D31D
  \l D31E
  \l D31F
  \l D320
  \l D321
  \l D322
  \l D323
  \l D324
  \l D325
  \l D326
  \l D327
  \l D328
  \l D329
  \l D32A
  \l D32B
  \l D32C
  \l D32D
  \l D32E
  \l D32F
  \l D330
  \l D331
  \l D332
  \l D333
  \l D334
  \l D335
  \l D336
  \l D337
  \l D338
  \l D339
  \l D33A
  \l D33B
  \l D33C
  \l D33D
  \l D33E
  \l D33F
  \l D340
  \l D341
  \l D342
  \l D343
  \l D344
  \l D345
  \l D346
  \l D347
  \l D348
  \l D349
  \l D34A
  \l D34B
  \l D34C
  \l D34D
  \l D34E
  \l D34F
  \l D350
  \l D351
  \l D352
  \l D353
  \l D354
  \l D355
  \l D356
  \l D357
  \l D358
  \l D359
  \l D35A
  \l D35B
  \l D35C
  \l D35D
  \l D35E
  \l D35F
  \l D360
  \l D361
  \l D362
  \l D363
  \l D364
  \l D365
  \l D366
  \l D367
  \l D368
  \l D369
  \l D36A
  \l D36B
  \l D36C
  \l D36D
  \l D36E
  \l D36F
  \l D370
  \l D371
  \l D372
  \l D373
  \l D374
  \l D375
  \l D376
  \l D377
  \l D378
  \l D379
  \l D37A
  \l D37B
  \l D37C
  \l D37D
  \l D37E
  \l D37F
  \l D380
  \l D381
  \l D382
  \l D383
  \l D384
  \l D385
  \l D386
  \l D387
  \l D388
  \l D389
  \l D38A
  \l D38B
  \l D38C
  \l D38D
  \l D38E
  \l D38F
  \l D390
  \l D391
  \l D392
  \l D393
  \l D394
  \l D395
  \l D396
  \l D397
  \l D398
  \l D399
  \l D39A
  \l D39B
  \l D39C
  \l D39D
  \l D39E
  \l D39F
  \l D3A0
  \l D3A1
  \l D3A2
  \l D3A3
  \l D3A4
  \l D3A5
  \l D3A6
  \l D3A7
  \l D3A8
  \l D3A9
  \l D3AA
  \l D3AB
  \l D3AC
  \l D3AD
  \l D3AE
  \l D3AF
  \l D3B0
  \l D3B1
  \l D3B2
  \l D3B3
  \l D3B4
  \l D3B5
  \l D3B6
  \l D3B7
  \l D3B8
  \l D3B9
  \l D3BA
  \l D3BB
  \l D3BC
  \l D3BD
  \l D3BE
  \l D3BF
  \l D3C0
  \l D3C1
  \l D3C2
  \l D3C3
  \l D3C4
  \l D3C5
  \l D3C6
  \l D3C7
  \l D3C8
  \l D3C9
  \l D3CA
  \l D3CB
  \l D3CC
  \l D3CD
  \l D3CE
  \l D3CF
  \l D3D0
  \l D3D1
  \l D3D2
  \l D3D3
  \l D3D4
  \l D3D5
  \l D3D6
  \l D3D7
  \l D3D8
  \l D3D9
  \l D3DA
  \l D3DB
  \l D3DC
  \l D3DD
  \l D3DE
  \l D3DF
  \l D3E0
  \l D3E1
  \l D3E2
  \l D3E3
  \l D3E4
  \l D3E5
  \l D3E6
  \l D3E7
  \l D3E8
  \l D3E9
  \l D3EA
  \l D3EB
  \l D3EC
  \l D3ED
  \l D3EE
  \l D3EF
  \l D3F0
  \l D3F1
  \l D3F2
  \l D3F3
  \l D3F4
  \l D3F5
  \l D3F6
  \l D3F7
  \l D3F8
  \l D3F9
  \l D3FA
  \l D3FB
  \l D3FC
  \l D3FD
  \l D3FE
  \l D3FF
  \l D400
  \l D401
  \l D402
  \l D403
  \l D404
  \l D405
  \l D406
  \l D407
  \l D408
  \l D409
  \l D40A
  \l D40B
  \l D40C
  \l D40D
  \l D40E
  \l D40F
  \l D410
  \l D411
  \l D412
  \l D413
  \l D414
  \l D415
  \l D416
  \l D417
  \l D418
  \l D419
  \l D41A
  \l D41B
  \l D41C
  \l D41D
  \l D41E
  \l D41F
  \l D420
  \l D421
  \l D422
  \l D423
  \l D424
  \l D425
  \l D426
  \l D427
  \l D428
  \l D429
  \l D42A
  \l D42B
  \l D42C
  \l D42D
  \l D42E
  \l D42F
  \l D430
  \l D431
  \l D432
  \l D433
  \l D434
  \l D435
  \l D436
  \l D437
  \l D438
  \l D439
  \l D43A
  \l D43B
  \l D43C
  \l D43D
  \l D43E
  \l D43F
  \l D440
  \l D441
  \l D442
  \l D443
  \l D444
  \l D445
  \l D446
  \l D447
  \l D448
  \l D449
  \l D44A
  \l D44B
  \l D44C
  \l D44D
  \l D44E
  \l D44F
  \l D450
  \l D451
  \l D452
  \l D453
  \l D454
  \l D455
  \l D456
  \l D457
  \l D458
  \l D459
  \l D45A
  \l D45B
  \l D45C
  \l D45D
  \l D45E
  \l D45F
  \l D460
  \l D461
  \l D462
  \l D463
  \l D464
  \l D465
  \l D466
  \l D467
  \l D468
  \l D469
  \l D46A
  \l D46B
  \l D46C
  \l D46D
  \l D46E
  \l D46F
  \l D470
  \l D471
  \l D472
  \l D473
  \l D474
  \l D475
  \l D476
  \l D477
  \l D478
  \l D479
  \l D47A
  \l D47B
  \l D47C
  \l D47D
  \l D47E
  \l D47F
  \l D480
  \l D481
  \l D482
  \l D483
  \l D484
  \l D485
  \l D486
  \l D487
  \l D488
  \l D489
  \l D48A
  \l D48B
  \l D48C
  \l D48D
  \l D48E
  \l D48F
  \l D490
  \l D491
  \l D492
  \l D493
  \l D494
  \l D495
  \l D496
  \l D497
  \l D498
  \l D499
  \l D49A
  \l D49B
  \l D49C
  \l D49D
  \l D49E
  \l D49F
  \l D4A0
  \l D4A1
  \l D4A2
  \l D4A3
  \l D4A4
  \l D4A5
  \l D4A6
  \l D4A7
  \l D4A8
  \l D4A9
  \l D4AA
  \l D4AB
  \l D4AC
  \l D4AD
  \l D4AE
  \l D4AF
  \l D4B0
  \l D4B1
  \l D4B2
  \l D4B3
  \l D4B4
  \l D4B5
  \l D4B6
  \l D4B7
  \l D4B8
  \l D4B9
  \l D4BA
  \l D4BB
  \l D4BC
  \l D4BD
  \l D4BE
  \l D4BF
  \l D4C0
  \l D4C1
  \l D4C2
  \l D4C3
  \l D4C4
  \l D4C5
  \l D4C6
  \l D4C7
  \l D4C8
  \l D4C9
  \l D4CA
  \l D4CB
  \l D4CC
  \l D4CD
  \l D4CE
  \l D4CF
  \l D4D0
  \l D4D1
  \l D4D2
  \l D4D3
  \l D4D4
  \l D4D5
  \l D4D6
  \l D4D7
  \l D4D8
  \l D4D9
  \l D4DA
  \l D4DB
  \l D4DC
  \l D4DD
  \l D4DE
  \l D4DF
  \l D4E0
  \l D4E1
  \l D4E2
  \l D4E3
  \l D4E4
  \l D4E5
  \l D4E6
  \l D4E7
  \l D4E8
  \l D4E9
  \l D4EA
  \l D4EB
  \l D4EC
  \l D4ED
  \l D4EE
  \l D4EF
  \l D4F0
  \l D4F1
  \l D4F2
  \l D4F3
  \l D4F4
  \l D4F5
  \l D4F6
  \l D4F7
  \l D4F8
  \l D4F9
  \l D4FA
  \l D4FB
  \l D4FC
  \l D4FD
  \l D4FE
  \l D4FF
  \l D500
  \l D501
  \l D502
  \l D503
  \l D504
  \l D505
  \l D506
  \l D507
  \l D508
  \l D509
  \l D50A
  \l D50B
  \l D50C
  \l D50D
  \l D50E
  \l D50F
  \l D510
  \l D511
  \l D512
  \l D513
  \l D514
  \l D515
  \l D516
  \l D517
  \l D518
  \l D519
  \l D51A
  \l D51B
  \l D51C
  \l D51D
  \l D51E
  \l D51F
  \l D520
  \l D521
  \l D522
  \l D523
  \l D524
  \l D525
  \l D526
  \l D527
  \l D528
  \l D529
  \l D52A
  \l D52B
  \l D52C
  \l D52D
  \l D52E
  \l D52F
  \l D530
  \l D531
  \l D532
  \l D533
  \l D534
  \l D535
  \l D536
  \l D537
  \l D538
  \l D539
  \l D53A
  \l D53B
  \l D53C
  \l D53D
  \l D53E
  \l D53F
  \l D540
  \l D541
  \l D542
  \l D543
  \l D544
  \l D545
  \l D546
  \l D547
  \l D548
  \l D549
  \l D54A
  \l D54B
  \l D54C
  \l D54D
  \l D54E
  \l D54F
  \l D550
  \l D551
  \l D552
  \l D553
  \l D554
  \l D555
  \l D556
  \l D557
  \l D558
  \l D559
  \l D55A
  \l D55B
  \l D55C
  \l D55D
  \l D55E
  \l D55F
  \l D560
  \l D561
  \l D562
  \l D563
  \l D564
  \l D565
  \l D566
  \l D567
  \l D568
  \l D569
  \l D56A
  \l D56B
  \l D56C
  \l D56D
  \l D56E
  \l D56F
  \l D570
  \l D571
  \l D572
  \l D573
  \l D574
  \l D575
  \l D576
  \l D577
  \l D578
  \l D579
  \l D57A
  \l D57B
  \l D57C
  \l D57D
  \l D57E
  \l D57F
  \l D580
  \l D581
  \l D582
  \l D583
  \l D584
  \l D585
  \l D586
  \l D587
  \l D588
  \l D589
  \l D58A
  \l D58B
  \l D58C
  \l D58D
  \l D58E
  \l D58F
  \l D590
  \l D591
  \l D592
  \l D593
  \l D594
  \l D595
  \l D596
  \l D597
  \l D598
  \l D599
  \l D59A
  \l D59B
  \l D59C
  \l D59D
  \l D59E
  \l D59F
  \l D5A0
  \l D5A1
  \l D5A2
  \l D5A3
  \l D5A4
  \l D5A5
  \l D5A6
  \l D5A7
  \l D5A8
  \l D5A9
  \l D5AA
  \l D5AB
  \l D5AC
  \l D5AD
  \l D5AE
  \l D5AF
  \l D5B0
  \l D5B1
  \l D5B2
  \l D5B3
  \l D5B4
  \l D5B5
  \l D5B6
  \l D5B7
  \l D5B8
  \l D5B9
  \l D5BA
  \l D5BB
  \l D5BC
  \l D5BD
  \l D5BE
  \l D5BF
  \l D5C0
  \l D5C1
  \l D5C2
  \l D5C3
  \l D5C4
  \l D5C5
  \l D5C6
  \l D5C7
  \l D5C8
  \l D5C9
  \l D5CA
  \l D5CB
  \l D5CC
  \l D5CD
  \l D5CE
  \l D5CF
  \l D5D0
  \l D5D1
  \l D5D2
  \l D5D3
  \l D5D4
  \l D5D5
  \l D5D6
  \l D5D7
  \l D5D8
  \l D5D9
  \l D5DA
  \l D5DB
  \l D5DC
  \l D5DD
  \l D5DE
  \l D5DF
  \l D5E0
  \l D5E1
  \l D5E2
  \l D5E3
  \l D5E4
  \l D5E5
  \l D5E6
  \l D5E7
  \l D5E8
  \l D5E9
  \l D5EA
  \l D5EB
  \l D5EC
  \l D5ED
  \l D5EE
  \l D5EF
  \l D5F0
  \l D5F1
  \l D5F2
  \l D5F3
  \l D5F4
  \l D5F5
  \l D5F6
  \l D5F7
  \l D5F8
  \l D5F9
  \l D5FA
  \l D5FB
  \l D5FC
  \l D5FD
  \l D5FE
  \l D5FF
  \l D600
  \l D601
  \l D602
  \l D603
  \l D604
  \l D605
  \l D606
  \l D607
  \l D608
  \l D609
  \l D60A
  \l D60B
  \l D60C
  \l D60D
  \l D60E
  \l D60F
  \l D610
  \l D611
  \l D612
  \l D613
  \l D614
  \l D615
  \l D616
  \l D617
  \l D618
  \l D619
  \l D61A
  \l D61B
  \l D61C
  \l D61D
  \l D61E
  \l D61F
  \l D620
  \l D621
  \l D622
  \l D623
  \l D624
  \l D625
  \l D626
  \l D627
  \l D628
  \l D629
  \l D62A
  \l D62B
  \l D62C
  \l D62D
  \l D62E
  \l D62F
  \l D630
  \l D631
  \l D632
  \l D633
  \l D634
  \l D635
  \l D636
  \l D637
  \l D638
  \l D639
  \l D63A
  \l D63B
  \l D63C
  \l D63D
  \l D63E
  \l D63F
  \l D640
  \l D641
  \l D642
  \l D643
  \l D644
  \l D645
  \l D646
  \l D647
  \l D648
  \l D649
  \l D64A
  \l D64B
  \l D64C
  \l D64D
  \l D64E
  \l D64F
  \l D650
  \l D651
  \l D652
  \l D653
  \l D654
  \l D655
  \l D656
  \l D657
  \l D658
  \l D659
  \l D65A
  \l D65B
  \l D65C
  \l D65D
  \l D65E
  \l D65F
  \l D660
  \l D661
  \l D662
  \l D663
  \l D664
  \l D665
  \l D666
  \l D667
  \l D668
  \l D669
  \l D66A
  \l D66B
  \l D66C
  \l D66D
  \l D66E
  \l D66F
  \l D670
  \l D671
  \l D672
  \l D673
  \l D674
  \l D675
  \l D676
  \l D677
  \l D678
  \l D679
  \l D67A
  \l D67B
  \l D67C
  \l D67D
  \l D67E
  \l D67F
  \l D680
  \l D681
  \l D682
  \l D683
  \l D684
  \l D685
  \l D686
  \l D687
  \l D688
  \l D689
  \l D68A
  \l D68B
  \l D68C
  \l D68D
  \l D68E
  \l D68F
  \l D690
  \l D691
  \l D692
  \l D693
  \l D694
  \l D695
  \l D696
  \l D697
  \l D698
  \l D699
  \l D69A
  \l D69B
  \l D69C
  \l D69D
  \l D69E
  \l D69F
  \l D6A0
  \l D6A1
  \l D6A2
  \l D6A3
  \l D6A4
  \l D6A5
  \l D6A6
  \l D6A7
  \l D6A8
  \l D6A9
  \l D6AA
  \l D6AB
  \l D6AC
  \l D6AD
  \l D6AE
  \l D6AF
  \l D6B0
  \l D6B1
  \l D6B2
  \l D6B3
  \l D6B4
  \l D6B5
  \l D6B6
  \l D6B7
  \l D6B8
  \l D6B9
  \l D6BA
  \l D6BB
  \l D6BC
  \l D6BD
  \l D6BE
  \l D6BF
  \l D6C0
  \l D6C1
  \l D6C2
  \l D6C3
  \l D6C4
  \l D6C5
  \l D6C6
  \l D6C7
  \l D6C8
  \l D6C9
  \l D6CA
  \l D6CB
  \l D6CC
  \l D6CD
  \l D6CE
  \l D6CF
  \l D6D0
  \l D6D1
  \l D6D2
  \l D6D3
  \l D6D4
  \l D6D5
  \l D6D6
  \l D6D7
  \l D6D8
  \l D6D9
  \l D6DA
  \l D6DB
  \l D6DC
  \l D6DD
  \l D6DE
  \l D6DF
  \l D6E0
  \l D6E1
  \l D6E2
  \l D6E3
  \l D6E4
  \l D6E5
  \l D6E6
  \l D6E7
  \l D6E8
  \l D6E9
  \l D6EA
  \l D6EB
  \l D6EC
  \l D6ED
  \l D6EE
  \l D6EF
  \l D6F0
  \l D6F1
  \l D6F2
  \l D6F3
  \l D6F4
  \l D6F5
  \l D6F6
  \l D6F7
  \l D6F8
  \l D6F9
  \l D6FA
  \l D6FB
  \l D6FC
  \l D6FD
  \l D6FE
  \l D6FF
  \l D700
  \l D701
  \l D702
  \l D703
  \l D704
  \l D705
  \l D706
  \l D707
  \l D708
  \l D709
  \l D70A
  \l D70B
  \l D70C
  \l D70D
  \l D70E
  \l D70F
  \l D710
  \l D711
  \l D712
  \l D713
  \l D714
  \l D715
  \l D716
  \l D717
  \l D718
  \l D719
  \l D71A
  \l D71B
  \l D71C
  \l D71D
  \l D71E
  \l D71F
  \l D720
  \l D721
  \l D722
  \l D723
  \l D724
  \l D725
  \l D726
  \l D727
  \l D728
  \l D729
  \l D72A
  \l D72B
  \l D72C
  \l D72D
  \l D72E
  \l D72F
  \l D730
  \l D731
  \l D732
  \l D733
  \l D734
  \l D735
  \l D736
  \l D737
  \l D738
  \l D739
  \l D73A
  \l D73B
  \l D73C
  \l D73D
  \l D73E
  \l D73F
  \l D740
  \l D741
  \l D742
  \l D743
  \l D744
  \l D745
  \l D746
  \l D747
  \l D748
  \l D749
  \l D74A
  \l D74B
  \l D74C
  \l D74D
  \l D74E
  \l D74F
  \l D750
  \l D751
  \l D752
  \l D753
  \l D754
  \l D755
  \l D756
  \l D757
  \l D758
  \l D759
  \l D75A
  \l D75B
  \l D75C
  \l D75D
  \l D75E
  \l D75F
  \l D760
  \l D761
  \l D762
  \l D763
  \l D764
  \l D765
  \l D766
  \l D767
  \l D768
  \l D769
  \l D76A
  \l D76B
  \l D76C
  \l D76D
  \l D76E
  \l D76F
  \l D770
  \l D771
  \l D772
  \l D773
  \l D774
  \l D775
  \l D776
  \l D777
  \l D778
  \l D779
  \l D77A
  \l D77B
  \l D77C
  \l D77D
  \l D77E
  \l D77F
  \l D780
  \l D781
  \l D782
  \l D783
  \l D784
  \l D785
  \l D786
  \l D787
  \l D788
  \l D789
  \l D78A
  \l D78B
  \l D78C
  \l D78D
  \l D78E
  \l D78F
  \l D790
  \l D791
  \l D792
  \l D793
  \l D794
  \l D795
  \l D796
  \l D797
  \l D798
  \l D799
  \l D79A
  \l D79B
  \l D79C
  \l D79D
  \l D79E
  \l D79F
  \l D7A0
  \l D7A1
  \l D7A2
  \l D7A3
  \l D7B0
  \l D7B1
  \l D7B2
  \l D7B3
  \l D7B4
  \l D7B5
  \l D7B6
  \l D7B7
  \l D7B8
  \l D7B9
  \l D7BA
  \l D7BB
  \l D7BC
  \l D7BD
  \l D7BE
  \l D7BF
  \l D7C0
  \l D7C1
  \l D7C2
  \l D7C3
  \l D7C4
  \l D7C5
  \l D7C6
  \l D7CB
  \l D7CC
  \l D7CD
  \l D7CE
  \l D7CF
  \l D7D0
  \l D7D1
  \l D7D2
  \l D7D3
  \l D7D4
  \l D7D5
  \l D7D6
  \l D7D7
  \l D7D8
  \l D7D9
  \l D7DA
  \l D7DB
  \l D7DC
  \l D7DD
  \l D7DE
  \l D7DF
  \l D7E0
  \l D7E1
  \l D7E2
  \l D7E3
  \l D7E4
  \l D7E5
  \l D7E6
  \l D7E7
  \l D7E8
  \l D7E9
  \l D7EA
  \l D7EB
  \l D7EC
  \l D7ED
  \l D7EE
  \l D7EF
  \l D7F0
  \l D7F1
  \l D7F2
  \l D7F3
  \l D7F4
  \l D7F5
  \l D7F6
  \l D7F7
  \l D7F8
  \l D7F9
  \l D7FA
  \l D7FB
  \l F900
  \l F901
  \l F902
  \l F903
  \l F904
  \l F905
  \l F906
  \l F907
  \l F908
  \l F909
  \l F90A
  \l F90B
  \l F90C
  \l F90D
  \l F90E
  \l F90F
  \l F910
  \l F911
  \l F912
  \l F913
  \l F914
  \l F915
  \l F916
  \l F917
  \l F918
  \l F919
  \l F91A
  \l F91B
  \l F91C
  \l F91D
  \l F91E
  \l F91F
  \l F920
  \l F921
  \l F922
  \l F923
  \l F924
  \l F925
  \l F926
  \l F927
  \l F928
  \l F929
  \l F92A
  \l F92B
  \l F92C
  \l F92D
  \l F92E
  \l F92F
  \l F930
  \l F931
  \l F932
  \l F933
  \l F934
  \l F935
  \l F936
  \l F937
  \l F938
  \l F939
  \l F93A
  \l F93B
  \l F93C
  \l F93D
  \l F93E
  \l F93F
  \l F940
  \l F941
  \l F942
  \l F943
  \l F944
  \l F945
  \l F946
  \l F947
  \l F948
  \l F949
  \l F94A
  \l F94B
  \l F94C
  \l F94D
  \l F94E
  \l F94F
  \l F950
  \l F951
  \l F952
  \l F953
  \l F954
  \l F955
  \l F956
  \l F957
  \l F958
  \l F959
  \l F95A
  \l F95B
  \l F95C
  \l F95D
  \l F95E
  \l F95F
  \l F960
  \l F961
  \l F962
  \l F963
  \l F964
  \l F965
  \l F966
  \l F967
  \l F968
  \l F969
  \l F96A
  \l F96B
  \l F96C
  \l F96D
  \l F96E
  \l F96F
  \l F970
  \l F971
  \l F972
  \l F973
  \l F974
  \l F975
  \l F976
  \l F977
  \l F978
  \l F979
  \l F97A
  \l F97B
  \l F97C
  \l F97D
  \l F97E
  \l F97F
  \l F980
  \l F981
  \l F982
  \l F983
  \l F984
  \l F985
  \l F986
  \l F987
  \l F988
  \l F989
  \l F98A
  \l F98B
  \l F98C
  \l F98D
  \l F98E
  \l F98F
  \l F990
  \l F991
  \l F992
  \l F993
  \l F994
  \l F995
  \l F996
  \l F997
  \l F998
  \l F999
  \l F99A
  \l F99B
  \l F99C
  \l F99D
  \l F99E
  \l F99F
  \l F9A0
  \l F9A1
  \l F9A2
  \l F9A3
  \l F9A4
  \l F9A5
  \l F9A6
  \l F9A7
  \l F9A8
  \l F9A9
  \l F9AA
  \l F9AB
  \l F9AC
  \l F9AD
  \l F9AE
  \l F9AF
  \l F9B0
  \l F9B1
  \l F9B2
  \l F9B3
  \l F9B4
  \l F9B5
  \l F9B6
  \l F9B7
  \l F9B8
  \l F9B9
  \l F9BA
  \l F9BB
  \l F9BC
  \l F9BD
  \l F9BE
  \l F9BF
  \l F9C0
  \l F9C1
  \l F9C2
  \l F9C3
  \l F9C4
  \l F9C5
  \l F9C6
  \l F9C7
  \l F9C8
  \l F9C9
  \l F9CA
  \l F9CB
  \l F9CC
  \l F9CD
  \l F9CE
  \l F9CF
  \l F9D0
  \l F9D1
  \l F9D2
  \l F9D3
  \l F9D4
  \l F9D5
  \l F9D6
  \l F9D7
  \l F9D8
  \l F9D9
  \l F9DA
  \l F9DB
  \l F9DC
  \l F9DD
  \l F9DE
  \l F9DF
  \l F9E0
  \l F9E1
  \l F9E2
  \l F9E3
  \l F9E4
  \l F9E5
  \l F9E6
  \l F9E7
  \l F9E8
  \l F9E9
  \l F9EA
  \l F9EB
  \l F9EC
  \l F9ED
  \l F9EE
  \l F9EF
  \l F9F0
  \l F9F1
  \l F9F2
  \l F9F3
  \l F9F4
  \l F9F5
  \l F9F6
  \l F9F7
  \l F9F8
  \l F9F9
  \l F9FA
  \l F9FB
  \l F9FC
  \l F9FD
  \l F9FE
  \l F9FF
  \l FA00
  \l FA01
  \l FA02
  \l FA03
  \l FA04
  \l FA05
  \l FA06
  \l FA07
  \l FA08
  \l FA09
  \l FA0A
  \l FA0B
  \l FA0C
  \l FA0D
  \l FA0E
  \l FA0F
  \l FA10
  \l FA11
  \l FA12
  \l FA13
  \l FA14
  \l FA15
  \l FA16
  \l FA17
  \l FA18
  \l FA19
  \l FA1A
  \l FA1B
  \l FA1C
  \l FA1D
  \l FA1E
  \l FA1F
  \l FA20
  \l FA21
  \l FA22
  \l FA23
  \l FA24
  \l FA25
  \l FA26
  \l FA27
  \l FA28
  \l FA29
  \l FA2A
  \l FA2B
  \l FA2C
  \l FA2D
  \l FA2E
  \l FA2F
  \l FA30
  \l FA31
  \l FA32
  \l FA33
  \l FA34
  \l FA35
  \l FA36
  \l FA37
  \l FA38
  \l FA39
  \l FA3A
  \l FA3B
  \l FA3C
  \l FA3D
  \l FA3E
  \l FA3F
  \l FA40
  \l FA41
  \l FA42
  \l FA43
  \l FA44
  \l FA45
  \l FA46
  \l FA47
  \l FA48
  \l FA49
  \l FA4A
  \l FA4B
  \l FA4C
  \l FA4D
  \l FA4E
  \l FA4F
  \l FA50
  \l FA51
  \l FA52
  \l FA53
  \l FA54
  \l FA55
  \l FA56
  \l FA57
  \l FA58
  \l FA59
  \l FA5A
  \l FA5B
  \l FA5C
  \l FA5D
  \l FA5E
  \l FA5F
  \l FA60
  \l FA61
  \l FA62
  \l FA63
  \l FA64
  \l FA65
  \l FA66
  \l FA67
  \l FA68
  \l FA69
  \l FA6A
  \l FA6B
  \l FA6C
  \l FA6D
  \l FA70
  \l FA71
  \l FA72
  \l FA73
  \l FA74
  \l FA75
  \l FA76
  \l FA77
  \l FA78
  \l FA79
  \l FA7A
  \l FA7B
  \l FA7C
  \l FA7D
  \l FA7E
  \l FA7F
  \l FA80
  \l FA81
  \l FA82
  \l FA83
  \l FA84
  \l FA85
  \l FA86
  \l FA87
  \l FA88
  \l FA89
  \l FA8A
  \l FA8B
  \l FA8C
  \l FA8D
  \l FA8E
  \l FA8F
  \l FA90
  \l FA91
  \l FA92
  \l FA93
  \l FA94
  \l FA95
  \l FA96
  \l FA97
  \l FA98
  \l FA99
  \l FA9A
  \l FA9B
  \l FA9C
  \l FA9D
  \l FA9E
  \l FA9F
  \l FAA0
  \l FAA1
  \l FAA2
  \l FAA3
  \l FAA4
  \l FAA5
  \l FAA6
  \l FAA7
  \l FAA8
  \l FAA9
  \l FAAA
  \l FAAB
  \l FAAC
  \l FAAD
  \l FAAE
  \l FAAF
  \l FAB0
  \l FAB1
  \l FAB2
  \l FAB3
  \l FAB4
  \l FAB5
  \l FAB6
  \l FAB7
  \l FAB8
  \l FAB9
  \l FABA
  \l FABB
  \l FABC
  \l FABD
  \l FABE
  \l FABF
  \l FAC0
  \l FAC1
  \l FAC2
  \l FAC3
  \l FAC4
  \l FAC5
  \l FAC6
  \l FAC7
  \l FAC8
  \l FAC9
  \l FACA
  \l FACB
  \l FACC
  \l FACD
  \l FACE
  \l FACF
  \l FAD0
  \l FAD1
  \l FAD2
  \l FAD3
  \l FAD4
  \l FAD5
  \l FAD6
  \l FAD7
  \l FAD8
  \l FAD9
  \l FB00
  \l FB01
  \l FB02
  \l FB03
  \l FB04
  \l FB05
  \l FB06
  \l FB13
  \l FB14
  \l FB15
  \l FB16
  \l FB17
  \l FB1D
  \l FB1E
  \l FB1F
  \l FB20
  \l FB21
  \l FB22
  \l FB23
  \l FB24
  \l FB25
  \l FB26
  \l FB27
  \l FB28
  \l FB2A
  \l FB2B
  \l FB2C
  \l FB2D
  \l FB2E
  \l FB2F
  \l FB30
  \l FB31
  \l FB32
  \l FB33
  \l FB34
  \l FB35
  \l FB36
  \l FB38
  \l FB39
  \l FB3A
  \l FB3B
  \l FB3C
  \l FB3E
  \l FB40
  \l FB41
  \l FB43
  \l FB44
  \l FB46
  \l FB47
  \l FB48
  \l FB49
  \l FB4A
  \l FB4B
  \l FB4C
  \l FB4D
  \l FB4E
  \l FB4F
  \l FB50
  \l FB51
  \l FB52
  \l FB53
  \l FB54
  \l FB55
  \l FB56
  \l FB57
  \l FB58
  \l FB59
  \l FB5A
  \l FB5B
  \l FB5C
  \l FB5D
  \l FB5E
  \l FB5F
  \l FB60
  \l FB61
  \l FB62
  \l FB63
  \l FB64
  \l FB65
  \l FB66
  \l FB67
  \l FB68
  \l FB69
  \l FB6A
  \l FB6B
  \l FB6C
  \l FB6D
  \l FB6E
  \l FB6F
  \l FB70
  \l FB71
  \l FB72
  \l FB73
  \l FB74
  \l FB75
  \l FB76
  \l FB77
  \l FB78
  \l FB79
  \l FB7A
  \l FB7B
  \l FB7C
  \l FB7D
  \l FB7E
  \l FB7F
  \l FB80
  \l FB81
  \l FB82
  \l FB83
  \l FB84
  \l FB85
  \l FB86
  \l FB87
  \l FB88
  \l FB89
  \l FB8A
  \l FB8B
  \l FB8C
  \l FB8D
  \l FB8E
  \l FB8F
  \l FB90
  \l FB91
  \l FB92
  \l FB93
  \l FB94
  \l FB95
  \l FB96
  \l FB97
  \l FB98
  \l FB99
  \l FB9A
  \l FB9B
  \l FB9C
  \l FB9D
  \l FB9E
  \l FB9F
  \l FBA0
  \l FBA1
  \l FBA2
  \l FBA3
  \l FBA4
  \l FBA5
  \l FBA6
  \l FBA7
  \l FBA8
  \l FBA9
  \l FBAA
  \l FBAB
  \l FBAC
  \l FBAD
  \l FBAE
  \l FBAF
  \l FBB0
  \l FBB1
  \l FBD3
  \l FBD4
  \l FBD5
  \l FBD6
  \l FBD7
  \l FBD8
  \l FBD9
  \l FBDA
  \l FBDB
  \l FBDC
  \l FBDD
  \l FBDE
  \l FBDF
  \l FBE0
  \l FBE1
  \l FBE2
  \l FBE3
  \l FBE4
  \l FBE5
  \l FBE6
  \l FBE7
  \l FBE8
  \l FBE9
  \l FBEA
  \l FBEB
  \l FBEC
  \l FBED
  \l FBEE
  \l FBEF
  \l FBF0
  \l FBF1
  \l FBF2
  \l FBF3
  \l FBF4
  \l FBF5
  \l FBF6
  \l FBF7
  \l FBF8
  \l FBF9
  \l FBFA
  \l FBFB
  \l FBFC
  \l FBFD
  \l FBFE
  \l FBFF
  \l FC00
  \l FC01
  \l FC02
  \l FC03
  \l FC04
  \l FC05
  \l FC06
  \l FC07
  \l FC08
  \l FC09
  \l FC0A
  \l FC0B
  \l FC0C
  \l FC0D
  \l FC0E
  \l FC0F
  \l FC10
  \l FC11
  \l FC12
  \l FC13
  \l FC14
  \l FC15
  \l FC16
  \l FC17
  \l FC18
  \l FC19
  \l FC1A
  \l FC1B
  \l FC1C
  \l FC1D
  \l FC1E
  \l FC1F
  \l FC20
  \l FC21
  \l FC22
  \l FC23
  \l FC24
  \l FC25
  \l FC26
  \l FC27
  \l FC28
  \l FC29
  \l FC2A
  \l FC2B
  \l FC2C
  \l FC2D
  \l FC2E
  \l FC2F
  \l FC30
  \l FC31
  \l FC32
  \l FC33
  \l FC34
  \l FC35
  \l FC36
  \l FC37
  \l FC38
  \l FC39
  \l FC3A
  \l FC3B
  \l FC3C
  \l FC3D
  \l FC3E
  \l FC3F
  \l FC40
  \l FC41
  \l FC42
  \l FC43
  \l FC44
  \l FC45
  \l FC46
  \l FC47
  \l FC48
  \l FC49
  \l FC4A
  \l FC4B
  \l FC4C
  \l FC4D
  \l FC4E
  \l FC4F
  \l FC50
  \l FC51
  \l FC52
  \l FC53
  \l FC54
  \l FC55
  \l FC56
  \l FC57
  \l FC58
  \l FC59
  \l FC5A
  \l FC5B
  \l FC5C
  \l FC5D
  \l FC5E
  \l FC5F
  \l FC60
  \l FC61
  \l FC62
  \l FC63
  \l FC64
  \l FC65
  \l FC66
  \l FC67
  \l FC68
  \l FC69
  \l FC6A
  \l FC6B
  \l FC6C
  \l FC6D
  \l FC6E
  \l FC6F
  \l FC70
  \l FC71
  \l FC72
  \l FC73
  \l FC74
  \l FC75
  \l FC76
  \l FC77
  \l FC78
  \l FC79
  \l FC7A
  \l FC7B
  \l FC7C
  \l FC7D
  \l FC7E
  \l FC7F
  \l FC80
  \l FC81
  \l FC82
  \l FC83
  \l FC84
  \l FC85
  \l FC86
  \l FC87
  \l FC88
  \l FC89
  \l FC8A
  \l FC8B
  \l FC8C
  \l FC8D
  \l FC8E
  \l FC8F
  \l FC90
  \l FC91
  \l FC92
  \l FC93
  \l FC94
  \l FC95
  \l FC96
  \l FC97
  \l FC98
  \l FC99
  \l FC9A
  \l FC9B
  \l FC9C
  \l FC9D
  \l FC9E
  \l FC9F
  \l FCA0
  \l FCA1
  \l FCA2
  \l FCA3
  \l FCA4
  \l FCA5
  \l FCA6
  \l FCA7
  \l FCA8
  \l FCA9
  \l FCAA
  \l FCAB
  \l FCAC
  \l FCAD
  \l FCAE
  \l FCAF
  \l FCB0
  \l FCB1
  \l FCB2
  \l FCB3
  \l FCB4
  \l FCB5
  \l FCB6
  \l FCB7
  \l FCB8
  \l FCB9
  \l FCBA
  \l FCBB
  \l FCBC
  \l FCBD
  \l FCBE
  \l FCBF
  \l FCC0
  \l FCC1
  \l FCC2
  \l FCC3
  \l FCC4
  \l FCC5
  \l FCC6
  \l FCC7
  \l FCC8
  \l FCC9
  \l FCCA
  \l FCCB
  \l FCCC
  \l FCCD
  \l FCCE
  \l FCCF
  \l FCD0
  \l FCD1
  \l FCD2
  \l FCD3
  \l FCD4
  \l FCD5
  \l FCD6
  \l FCD7
  \l FCD8
  \l FCD9
  \l FCDA
  \l FCDB
  \l FCDC
  \l FCDD
  \l FCDE
  \l FCDF
  \l FCE0
  \l FCE1
  \l FCE2
  \l FCE3
  \l FCE4
  \l FCE5
  \l FCE6
  \l FCE7
  \l FCE8
  \l FCE9
  \l FCEA
  \l FCEB
  \l FCEC
  \l FCED
  \l FCEE
  \l FCEF
  \l FCF0
  \l FCF1
  \l FCF2
  \l FCF3
  \l FCF4
  \l FCF5
  \l FCF6
  \l FCF7
  \l FCF8
  \l FCF9
  \l FCFA
  \l FCFB
  \l FCFC
  \l FCFD
  \l FCFE
  \l FCFF
  \l FD00
  \l FD01
  \l FD02
  \l FD03
  \l FD04
  \l FD05
  \l FD06
  \l FD07
  \l FD08
  \l FD09
  \l FD0A
  \l FD0B
  \l FD0C
  \l FD0D
  \l FD0E
  \l FD0F
  \l FD10
  \l FD11
  \l FD12
  \l FD13
  \l FD14
  \l FD15
  \l FD16
  \l FD17
  \l FD18
  \l FD19
  \l FD1A
  \l FD1B
  \l FD1C
  \l FD1D
  \l FD1E
  \l FD1F
  \l FD20
  \l FD21
  \l FD22
  \l FD23
  \l FD24
  \l FD25
  \l FD26
  \l FD27
  \l FD28
  \l FD29
  \l FD2A
  \l FD2B
  \l FD2C
  \l FD2D
  \l FD2E
  \l FD2F
  \l FD30
  \l FD31
  \l FD32
  \l FD33
  \l FD34
  \l FD35
  \l FD36
  \l FD37
  \l FD38
  \l FD39
  \l FD3A
  \l FD3B
  \l FD3C
  \l FD3D
  \l FD50
  \l FD51
  \l FD52
  \l FD53
  \l FD54
  \l FD55
  \l FD56
  \l FD57
  \l FD58
  \l FD59
  \l FD5A
  \l FD5B
  \l FD5C
  \l FD5D
  \l FD5E
  \l FD5F
  \l FD60
  \l FD61
  \l FD62
  \l FD63
  \l FD64
  \l FD65
  \l FD66
  \l FD67
  \l FD68
  \l FD69
  \l FD6A
  \l FD6B
  \l FD6C
  \l FD6D
  \l FD6E
  \l FD6F
  \l FD70
  \l FD71
  \l FD72
  \l FD73
  \l FD74
  \l FD75
  \l FD76
  \l FD77
  \l FD78
  \l FD79
  \l FD7A
  \l FD7B
  \l FD7C
  \l FD7D
  \l FD7E
  \l FD7F
  \l FD80
  \l FD81
  \l FD82
  \l FD83
  \l FD84
  \l FD85
  \l FD86
  \l FD87
  \l FD88
  \l FD89
  \l FD8A
  \l FD8B
  \l FD8C
  \l FD8D
  \l FD8E
  \l FD8F
  \l FD92
  \l FD93
  \l FD94
  \l FD95
  \l FD96
  \l FD97
  \l FD98
  \l FD99
  \l FD9A
  \l FD9B
  \l FD9C
  \l FD9D
  \l FD9E
  \l FD9F
  \l FDA0
  \l FDA1
  \l FDA2
  \l FDA3
  \l FDA4
  \l FDA5
  \l FDA6
  \l FDA7
  \l FDA8
  \l FDA9
  \l FDAA
  \l FDAB
  \l FDAC
  \l FDAD
  \l FDAE
  \l FDAF
  \l FDB0
  \l FDB1
  \l FDB2
  \l FDB3
  \l FDB4
  \l FDB5
  \l FDB6
  \l FDB7
  \l FDB8
  \l FDB9
  \l FDBA
  \l FDBB
  \l FDBC
  \l FDBD
  \l FDBE
  \l FDBF
  \l FDC0
  \l FDC1
  \l FDC2
  \l FDC3
  \l FDC4
  \l FDC5
  \l FDC6
  \l FDC7
  \l FDF0
  \l FDF1
  \l FDF2
  \l FDF3
  \l FDF4
  \l FDF5
  \l FDF6
  \l FDF7
  \l FDF8
  \l FDF9
  \l FDFA
  \l FDFB
  \l FE00
  \l FE01
  \l FE02
  \l FE03
  \l FE04
  \l FE05
  \l FE06
  \l FE07
  \l FE08
  \l FE09
  \l FE0A
  \l FE0B
  \l FE0C
  \l FE0D
  \l FE0E
  \l FE0F
  \l FE20
  \l FE21
  \l FE22
  \l FE23
  \l FE24
  \l FE25
  \l FE26
  \l FE27
  \l FE28
  \l FE29
  \l FE2A
  \l FE2B
  \l FE2C
  \l FE2D
  \l FE70
  \l FE71
  \l FE72
  \l FE73
  \l FE74
  \l FE76
  \l FE77
  \l FE78
  \l FE79
  \l FE7A
  \l FE7B
  \l FE7C
  \l FE7D
  \l FE7E
  \l FE7F
  \l FE80
  \l FE81
  \l FE82
  \l FE83
  \l FE84
  \l FE85
  \l FE86
  \l FE87
  \l FE88
  \l FE89
  \l FE8A
  \l FE8B
  \l FE8C
  \l FE8D
  \l FE8E
  \l FE8F
  \l FE90
  \l FE91
  \l FE92
  \l FE93
  \l FE94
  \l FE95
  \l FE96
  \l FE97
  \l FE98
  \l FE99
  \l FE9A
  \l FE9B
  \l FE9C
  \l FE9D
  \l FE9E
  \l FE9F
  \l FEA0
  \l FEA1
  \l FEA2
  \l FEA3
  \l FEA4
  \l FEA5
  \l FEA6
  \l FEA7
  \l FEA8
  \l FEA9
  \l FEAA
  \l FEAB
  \l FEAC
  \l FEAD
  \l FEAE
  \l FEAF
  \l FEB0
  \l FEB1
  \l FEB2
  \l FEB3
  \l FEB4
  \l FEB5
  \l FEB6
  \l FEB7
  \l FEB8
  \l FEB9
  \l FEBA
  \l FEBB
  \l FEBC
  \l FEBD
  \l FEBE
  \l FEBF
  \l FEC0
  \l FEC1
  \l FEC2
  \l FEC3
  \l FEC4
  \l FEC5
  \l FEC6
  \l FEC7
  \l FEC8
  \l FEC9
  \l FECA
  \l FECB
  \l FECC
  \l FECD
  \l FECE
  \l FECF
  \l FED0
  \l FED1
  \l FED2
  \l FED3
  \l FED4
  \l FED5
  \l FED6
  \l FED7
  \l FED8
  \l FED9
  \l FEDA
  \l FEDB
  \l FEDC
  \l FEDD
  \l FEDE
  \l FEDF
  \l FEE0
  \l FEE1
  \l FEE2
  \l FEE3
  \l FEE4
  \l FEE5
  \l FEE6
  \l FEE7
  \l FEE8
  \l FEE9
  \l FEEA
  \l FEEB
  \l FEEC
  \l FEED
  \l FEEE
  \l FEEF
  \l FEF0
  \l FEF1
  \l FEF2
  \l FEF3
  \l FEF4
  \l FEF5
  \l FEF6
  \l FEF7
  \l FEF8
  \l FEF9
  \l FEFA
  \l FEFB
  \l FEFC
  \L FF21 FF21 FF41
  \L FF22 FF22 FF42
  \L FF23 FF23 FF43
  \L FF24 FF24 FF44
  \L FF25 FF25 FF45
  \L FF26 FF26 FF46
  \L FF27 FF27 FF47
  \L FF28 FF28 FF48
  \L FF29 FF29 FF49
  \L FF2A FF2A FF4A
  \L FF2B FF2B FF4B
  \L FF2C FF2C FF4C
  \L FF2D FF2D FF4D
  \L FF2E FF2E FF4E
  \L FF2F FF2F FF4F
  \L FF30 FF30 FF50
  \L FF31 FF31 FF51
  \L FF32 FF32 FF52
  \L FF33 FF33 FF53
  \L FF34 FF34 FF54
  \L FF35 FF35 FF55
  \L FF36 FF36 FF56
  \L FF37 FF37 FF57
  \L FF38 FF38 FF58
  \L FF39 FF39 FF59
  \L FF3A FF3A FF5A
  \L FF41 FF21 FF41
  \L FF42 FF22 FF42
  \L FF43 FF23 FF43
  \L FF44 FF24 FF44
  \L FF45 FF25 FF45
  \L FF46 FF26 FF46
  \L FF47 FF27 FF47
  \L FF48 FF28 FF48
  \L FF49 FF29 FF49
  \L FF4A FF2A FF4A
  \L FF4B FF2B FF4B
  \L FF4C FF2C FF4C
  \L FF4D FF2D FF4D
  \L FF4E FF2E FF4E
  \L FF4F FF2F FF4F
  \L FF50 FF30 FF50
  \L FF51 FF31 FF51
  \L FF52 FF32 FF52
  \L FF53 FF33 FF53
  \L FF54 FF34 FF54
  \L FF55 FF35 FF55
  \L FF56 FF36 FF56
  \L FF57 FF37 FF57
  \L FF58 FF38 FF58
  \L FF59 FF39 FF59
  \L FF5A FF3A FF5A
  \l FF66
  \l FF67
  \l FF68
  \l FF69
  \l FF6A
  \l FF6B
  \l FF6C
  \l FF6D
  \l FF6E
  \l FF6F
  \l FF70
  \l FF71
  \l FF72
  \l FF73
  \l FF74
  \l FF75
  \l FF76
  \l FF77
  \l FF78
  \l FF79
  \l FF7A
  \l FF7B
  \l FF7C
  \l FF7D
  \l FF7E
  \l FF7F
  \l FF80
  \l FF81
  \l FF82
  \l FF83
  \l FF84
  \l FF85
  \l FF86
  \l FF87
  \l FF88
  \l FF89
  \l FF8A
  \l FF8B
  \l FF8C
  \l FF8D
  \l FF8E
  \l FF8F
  \l FF90
  \l FF91
  \l FF92
  \l FF93
  \l FF94
  \l FF95
  \l FF96
  \l FF97
  \l FF98
  \l FF99
  \l FF9A
  \l FF9B
  \l FF9C
  \l FF9D
  \l FF9E
  \l FF9F
  \l FFA0
  \l FFA1
  \l FFA2
  \l FFA3
  \l FFA4
  \l FFA5
  \l FFA6
  \l FFA7
  \l FFA8
  \l FFA9
  \l FFAA
  \l FFAB
  \l FFAC
  \l FFAD
  \l FFAE
  \l FFAF
  \l FFB0
  \l FFB1
  \l FFB2
  \l FFB3
  \l FFB4
  \l FFB5
  \l FFB6
  \l FFB7
  \l FFB8
  \l FFB9
  \l FFBA
  \l FFBB
  \l FFBC
  \l FFBD
  \l FFBE
  \l FFC2
  \l FFC3
  \l FFC4
  \l FFC5
  \l FFC6
  \l FFC7
  \l FFCA
  \l FFCB
  \l FFCC
  \l FFCD
  \l FFCE
  \l FFCF
  \l FFD2
  \l FFD3
  \l FFD4
  \l FFD5
  \l FFD6
  \l FFD7
  \l FFDA
  \l FFDB
  \l FFDC
  \l 10000
  \l 10001
  \l 10002
  \l 10003
  \l 10004
  \l 10005
  \l 10006
  \l 10007
  \l 10008
  \l 10009
  \l 1000A
  \l 1000B
  \l 1000D
  \l 1000E
  \l 1000F
  \l 10010
  \l 10011
  \l 10012
  \l 10013
  \l 10014
  \l 10015
  \l 10016
  \l 10017
  \l 10018
  \l 10019
  \l 1001A
  \l 1001B
  \l 1001C
  \l 1001D
  \l 1001E
  \l 1001F
  \l 10020
  \l 10021
  \l 10022
  \l 10023
  \l 10024
  \l 10025
  \l 10026
  \l 10028
  \l 10029
  \l 1002A
  \l 1002B
  \l 1002C
  \l 1002D
  \l 1002E
  \l 1002F
  \l 10030
  \l 10031
  \l 10032
  \l 10033
  \l 10034
  \l 10035
  \l 10036
  \l 10037
  \l 10038
  \l 10039
  \l 1003A
  \l 1003C
  \l 1003D
  \l 1003F
  \l 10040
  \l 10041
  \l 10042
  \l 10043
  \l 10044
  \l 10045
  \l 10046
  \l 10047
  \l 10048
  \l 10049
  \l 1004A
  \l 1004B
  \l 1004C
  \l 1004D
  \l 10050
  \l 10051
  \l 10052
  \l 10053
  \l 10054
  \l 10055
  \l 10056
  \l 10057
  \l 10058
  \l 10059
  \l 1005A
  \l 1005B
  \l 1005C
  \l 1005D
  \l 10080
  \l 10081
  \l 10082
  \l 10083
  \l 10084
  \l 10085
  \l 10086
  \l 10087
  \l 10088
  \l 10089
  \l 1008A
  \l 1008B
  \l 1008C
  \l 1008D
  \l 1008E
  \l 1008F
  \l 10090
  \l 10091
  \l 10092
  \l 10093
  \l 10094
  \l 10095
  \l 10096
  \l 10097
  \l 10098
  \l 10099
  \l 1009A
  \l 1009B
  \l 1009C
  \l 1009D
  \l 1009E
  \l 1009F
  \l 100A0
  \l 100A1
  \l 100A2
  \l 100A3
  \l 100A4
  \l 100A5
  \l 100A6
  \l 100A7
  \l 100A8
  \l 100A9
  \l 100AA
  \l 100AB
  \l 100AC
  \l 100AD
  \l 100AE
  \l 100AF
  \l 100B0
  \l 100B1
  \l 100B2
  \l 100B3
  \l 100B4
  \l 100B5
  \l 100B6
  \l 100B7
  \l 100B8
  \l 100B9
  \l 100BA
  \l 100BB
  \l 100BC
  \l 100BD
  \l 100BE
  \l 100BF
  \l 100C0
  \l 100C1
  \l 100C2
  \l 100C3
  \l 100C4
  \l 100C5
  \l 100C6
  \l 100C7
  \l 100C8
  \l 100C9
  \l 100CA
  \l 100CB
  \l 100CC
  \l 100CD
  \l 100CE
  \l 100CF
  \l 100D0
  \l 100D1
  \l 100D2
  \l 100D3
  \l 100D4
  \l 100D5
  \l 100D6
  \l 100D7
  \l 100D8
  \l 100D9
  \l 100DA
  \l 100DB
  \l 100DC
  \l 100DD
  \l 100DE
  \l 100DF
  \l 100E0
  \l 100E1
  \l 100E2
  \l 100E3
  \l 100E4
  \l 100E5
  \l 100E6
  \l 100E7
  \l 100E8
  \l 100E9
  \l 100EA
  \l 100EB
  \l 100EC
  \l 100ED
  \l 100EE
  \l 100EF
  \l 100F0
  \l 100F1
  \l 100F2
  \l 100F3
  \l 100F4
  \l 100F5
  \l 100F6
  \l 100F7
  \l 100F8
  \l 100F9
  \l 100FA
  \l 101FD
  \l 10280
  \l 10281
  \l 10282
  \l 10283
  \l 10284
  \l 10285
  \l 10286
  \l 10287
  \l 10288
  \l 10289
  \l 1028A
  \l 1028B
  \l 1028C
  \l 1028D
  \l 1028E
  \l 1028F
  \l 10290
  \l 10291
  \l 10292
  \l 10293
  \l 10294
  \l 10295
  \l 10296
  \l 10297
  \l 10298
  \l 10299
  \l 1029A
  \l 1029B
  \l 1029C
  \l 102A0
  \l 102A1
  \l 102A2
  \l 102A3
  \l 102A4
  \l 102A5
  \l 102A6
  \l 102A7
  \l 102A8
  \l 102A9
  \l 102AA
  \l 102AB
  \l 102AC
  \l 102AD
  \l 102AE
  \l 102AF
  \l 102B0
  \l 102B1
  \l 102B2
  \l 102B3
  \l 102B4
  \l 102B5
  \l 102B6
  \l 102B7
  \l 102B8
  \l 102B9
  \l 102BA
  \l 102BB
  \l 102BC
  \l 102BD
  \l 102BE
  \l 102BF
  \l 102C0
  \l 102C1
  \l 102C2
  \l 102C3
  \l 102C4
  \l 102C5
  \l 102C6
  \l 102C7
  \l 102C8
  \l 102C9
  \l 102CA
  \l 102CB
  \l 102CC
  \l 102CD
  \l 102CE
  \l 102CF
  \l 102D0
  \l 102E0
  \l 10300
  \l 10301
  \l 10302
  \l 10303
  \l 10304
  \l 10305
  \l 10306
  \l 10307
  \l 10308
  \l 10309
  \l 1030A
  \l 1030B
  \l 1030C
  \l 1030D
  \l 1030E
  \l 1030F
  \l 10310
  \l 10311
  \l 10312
  \l 10313
  \l 10314
  \l 10315
  \l 10316
  \l 10317
  \l 10318
  \l 10319
  \l 1031A
  \l 1031B
  \l 1031C
  \l 1031D
  \l 1031E
  \l 1031F
  \l 10330
  \l 10331
  \l 10332
  \l 10333
  \l 10334
  \l 10335
  \l 10336
  \l 10337
  \l 10338
  \l 10339
  \l 1033A
  \l 1033B
  \l 1033C
  \l 1033D
  \l 1033E
  \l 1033F
  \l 10340
  \l 10342
  \l 10343
  \l 10344
  \l 10345
  \l 10346
  \l 10347
  \l 10348
  \l 10349
  \l 10350
  \l 10351
  \l 10352
  \l 10353
  \l 10354
  \l 10355
  \l 10356
  \l 10357
  \l 10358
  \l 10359
  \l 1035A
  \l 1035B
  \l 1035C
  \l 1035D
  \l 1035E
  \l 1035F
  \l 10360
  \l 10361
  \l 10362
  \l 10363
  \l 10364
  \l 10365
  \l 10366
  \l 10367
  \l 10368
  \l 10369
  \l 1036A
  \l 1036B
  \l 1036C
  \l 1036D
  \l 1036E
  \l 1036F
  \l 10370
  \l 10371
  \l 10372
  \l 10373
  \l 10374
  \l 10375
  \l 10376
  \l 10377
  \l 10378
  \l 10379
  \l 1037A
  \l 10380
  \l 10381
  \l 10382
  \l 10383
  \l 10384
  \l 10385
  \l 10386
  \l 10387
  \l 10388
  \l 10389
  \l 1038A
  \l 1038B
  \l 1038C
  \l 1038D
  \l 1038E
  \l 1038F
  \l 10390
  \l 10391
  \l 10392
  \l 10393
  \l 10394
  \l 10395
  \l 10396
  \l 10397
  \l 10398
  \l 10399
  \l 1039A
  \l 1039B
  \l 1039C
  \l 1039D
  \l 103A0
  \l 103A1
  \l 103A2
  \l 103A3
  \l 103A4
  \l 103A5
  \l 103A6
  \l 103A7
  \l 103A8
  \l 103A9
  \l 103AA
  \l 103AB
  \l 103AC
  \l 103AD
  \l 103AE
  \l 103AF
  \l 103B0
  \l 103B1
  \l 103B2
  \l 103B3
  \l 103B4
  \l 103B5
  \l 103B6
  \l 103B7
  \l 103B8
  \l 103B9
  \l 103BA
  \l 103BB
  \l 103BC
  \l 103BD
  \l 103BE
  \l 103BF
  \l 103C0
  \l 103C1
  \l 103C2
  \l 103C3
  \l 103C8
  \l 103C9
  \l 103CA
  \l 103CB
  \l 103CC
  \l 103CD
  \l 103CE
  \l 103CF
  \L 10400 10400 10428
  \L 10401 10401 10429
  \L 10402 10402 1042A
  \L 10403 10403 1042B
  \L 10404 10404 1042C
  \L 10405 10405 1042D
  \L 10406 10406 1042E
  \L 10407 10407 1042F
  \L 10408 10408 10430
  \L 10409 10409 10431
  \L 1040A 1040A 10432
  \L 1040B 1040B 10433
  \L 1040C 1040C 10434
  \L 1040D 1040D 10435
  \L 1040E 1040E 10436
  \L 1040F 1040F 10437
  \L 10410 10410 10438
  \L 10411 10411 10439
  \L 10412 10412 1043A
  \L 10413 10413 1043B
  \L 10414 10414 1043C
  \L 10415 10415 1043D
  \L 10416 10416 1043E
  \L 10417 10417 1043F
  \L 10418 10418 10440
  \L 10419 10419 10441
  \L 1041A 1041A 10442
  \L 1041B 1041B 10443
  \L 1041C 1041C 10444
  \L 1041D 1041D 10445
  \L 1041E 1041E 10446
  \L 1041F 1041F 10447
  \L 10420 10420 10448
  \L 10421 10421 10449
  \L 10422 10422 1044A
  \L 10423 10423 1044B
  \L 10424 10424 1044C
  \L 10425 10425 1044D
  \L 10426 10426 1044E
  \L 10427 10427 1044F
  \L 10428 10400 10428
  \L 10429 10401 10429
  \L 1042A 10402 1042A
  \L 1042B 10403 1042B
  \L 1042C 10404 1042C
  \L 1042D 10405 1042D
  \L 1042E 10406 1042E
  \L 1042F 10407 1042F
  \L 10430 10408 10430
  \L 10431 10409 10431
  \L 10432 1040A 10432
  \L 10433 1040B 10433
  \L 10434 1040C 10434
  \L 10435 1040D 10435
  \L 10436 1040E 10436
  \L 10437 1040F 10437
  \L 10438 10410 10438
  \L 10439 10411 10439
  \L 1043A 10412 1043A
  \L 1043B 10413 1043B
  \L 1043C 10414 1043C
  \L 1043D 10415 1043D
  \L 1043E 10416 1043E
  \L 1043F 10417 1043F
  \L 10440 10418 10440
  \L 10441 10419 10441
  \L 10442 1041A 10442
  \L 10443 1041B 10443
  \L 10444 1041C 10444
  \L 10445 1041D 10445
  \L 10446 1041E 10446
  \L 10447 1041F 10447
  \L 10448 10420 10448
  \L 10449 10421 10449
  \L 1044A 10422 1044A
  \L 1044B 10423 1044B
  \L 1044C 10424 1044C
  \L 1044D 10425 1044D
  \L 1044E 10426 1044E
  \L 1044F 10427 1044F
  \l 10450
  \l 10451
  \l 10452
  \l 10453
  \l 10454
  \l 10455
  \l 10456
  \l 10457
  \l 10458
  \l 10459
  \l 1045A
  \l 1045B
  \l 1045C
  \l 1045D
  \l 1045E
  \l 1045F
  \l 10460
  \l 10461
  \l 10462
  \l 10463
  \l 10464
  \l 10465
  \l 10466
  \l 10467
  \l 10468
  \l 10469
  \l 1046A
  \l 1046B
  \l 1046C
  \l 1046D
  \l 1046E
  \l 1046F
  \l 10470
  \l 10471
  \l 10472
  \l 10473
  \l 10474
  \l 10475
  \l 10476
  \l 10477
  \l 10478
  \l 10479
  \l 1047A
  \l 1047B
  \l 1047C
  \l 1047D
  \l 1047E
  \l 1047F
  \l 10480
  \l 10481
  \l 10482
  \l 10483
  \l 10484
  \l 10485
  \l 10486
  \l 10487
  \l 10488
  \l 10489
  \l 1048A
  \l 1048B
  \l 1048C
  \l 1048D
  \l 1048E
  \l 1048F
  \l 10490
  \l 10491
  \l 10492
  \l 10493
  \l 10494
  \l 10495
  \l 10496
  \l 10497
  \l 10498
  \l 10499
  \l 1049A
  \l 1049B
  \l 1049C
  \l 1049D
  \l 10500
  \l 10501
  \l 10502
  \l 10503
  \l 10504
  \l 10505
  \l 10506
  \l 10507
  \l 10508
  \l 10509
  \l 1050A
  \l 1050B
  \l 1050C
  \l 1050D
  \l 1050E
  \l 1050F
  \l 10510
  \l 10511
  \l 10512
  \l 10513
  \l 10514
  \l 10515
  \l 10516
  \l 10517
  \l 10518
  \l 10519
  \l 1051A
  \l 1051B
  \l 1051C
  \l 1051D
  \l 1051E
  \l 1051F
  \l 10520
  \l 10521
  \l 10522
  \l 10523
  \l 10524
  \l 10525
  \l 10526
  \l 10527
  \l 10530
  \l 10531
  \l 10532
  \l 10533
  \l 10534
  \l 10535
  \l 10536
  \l 10537
  \l 10538
  \l 10539
  \l 1053A
  \l 1053B
  \l 1053C
  \l 1053D
  \l 1053E
  \l 1053F
  \l 10540
  \l 10541
  \l 10542
  \l 10543
  \l 10544
  \l 10545
  \l 10546
  \l 10547
  \l 10548
  \l 10549
  \l 1054A
  \l 1054B
  \l 1054C
  \l 1054D
  \l 1054E
  \l 1054F
  \l 10550
  \l 10551
  \l 10552
  \l 10553
  \l 10554
  \l 10555
  \l 10556
  \l 10557
  \l 10558
  \l 10559
  \l 1055A
  \l 1055B
  \l 1055C
  \l 1055D
  \l 1055E
  \l 1055F
  \l 10560
  \l 10561
  \l 10562
  \l 10563
  \l 10600
  \l 10601
  \l 10602
  \l 10603
  \l 10604
  \l 10605
  \l 10606
  \l 10607
  \l 10608
  \l 10609
  \l 1060A
  \l 1060B
  \l 1060C
  \l 1060D
  \l 1060E
  \l 1060F
  \l 10610
  \l 10611
  \l 10612
  \l 10613
  \l 10614
  \l 10615
  \l 10616
  \l 10617
  \l 10618
  \l 10619
  \l 1061A
  \l 1061B
  \l 1061C
  \l 1061D
  \l 1061E
  \l 1061F
  \l 10620
  \l 10621
  \l 10622
  \l 10623
  \l 10624
  \l 10625
  \l 10626
  \l 10627
  \l 10628
  \l 10629
  \l 1062A
  \l 1062B
  \l 1062C
  \l 1062D
  \l 1062E
  \l 1062F
  \l 10630
  \l 10631
  \l 10632
  \l 10633
  \l 10634
  \l 10635
  \l 10636
  \l 10637
  \l 10638
  \l 10639
  \l 1063A
  \l 1063B
  \l 1063C
  \l 1063D
  \l 1063E
  \l 1063F
  \l 10640
  \l 10641
  \l 10642
  \l 10643
  \l 10644
  \l 10645
  \l 10646
  \l 10647
  \l 10648
  \l 10649
  \l 1064A
  \l 1064B
  \l 1064C
  \l 1064D
  \l 1064E
  \l 1064F
  \l 10650
  \l 10651
  \l 10652
  \l 10653
  \l 10654
  \l 10655
  \l 10656
  \l 10657
  \l 10658
  \l 10659
  \l 1065A
  \l 1065B
  \l 1065C
  \l 1065D
  \l 1065E
  \l 1065F
  \l 10660
  \l 10661
  \l 10662
  \l 10663
  \l 10664
  \l 10665
  \l 10666
  \l 10667
  \l 10668
  \l 10669
  \l 1066A
  \l 1066B
  \l 1066C
  \l 1066D
  \l 1066E
  \l 1066F
  \l 10670
  \l 10671
  \l 10672
  \l 10673
  \l 10674
  \l 10675
  \l 10676
  \l 10677
  \l 10678
  \l 10679
  \l 1067A
  \l 1067B
  \l 1067C
  \l 1067D
  \l 1067E
  \l 1067F
  \l 10680
  \l 10681
  \l 10682
  \l 10683
  \l 10684
  \l 10685
  \l 10686
  \l 10687
  \l 10688
  \l 10689
  \l 1068A
  \l 1068B
  \l 1068C
  \l 1068D
  \l 1068E
  \l 1068F
  \l 10690
  \l 10691
  \l 10692
  \l 10693
  \l 10694
  \l 10695
  \l 10696
  \l 10697
  \l 10698
  \l 10699
  \l 1069A
  \l 1069B
  \l 1069C
  \l 1069D
  \l 1069E
  \l 1069F
  \l 106A0
  \l 106A1
  \l 106A2
  \l 106A3
  \l 106A4
  \l 106A5
  \l 106A6
  \l 106A7
  \l 106A8
  \l 106A9
  \l 106AA
  \l 106AB
  \l 106AC
  \l 106AD
  \l 106AE
  \l 106AF
  \l 106B0
  \l 106B1
  \l 106B2
  \l 106B3
  \l 106B4
  \l 106B5
  \l 106B6
  \l 106B7
  \l 106B8
  \l 106B9
  \l 106BA
  \l 106BB
  \l 106BC
  \l 106BD
  \l 106BE
  \l 106BF
  \l 106C0
  \l 106C1
  \l 106C2
  \l 106C3
  \l 106C4
  \l 106C5
  \l 106C6
  \l 106C7
  \l 106C8
  \l 106C9
  \l 106CA
  \l 106CB
  \l 106CC
  \l 106CD
  \l 106CE
  \l 106CF
  \l 106D0
  \l 106D1
  \l 106D2
  \l 106D3
  \l 106D4
  \l 106D5
  \l 106D6
  \l 106D7
  \l 106D8
  \l 106D9
  \l 106DA
  \l 106DB
  \l 106DC
  \l 106DD
  \l 106DE
  \l 106DF
  \l 106E0
  \l 106E1
  \l 106E2
  \l 106E3
  \l 106E4
  \l 106E5
  \l 106E6
  \l 106E7
  \l 106E8
  \l 106E9
  \l 106EA
  \l 106EB
  \l 106EC
  \l 106ED
  \l 106EE
  \l 106EF
  \l 106F0
  \l 106F1
  \l 106F2
  \l 106F3
  \l 106F4
  \l 106F5
  \l 106F6
  \l 106F7
  \l 106F8
  \l 106F9
  \l 106FA
  \l 106FB
  \l 106FC
  \l 106FD
  \l 106FE
  \l 106FF
  \l 10700
  \l 10701
  \l 10702
  \l 10703
  \l 10704
  \l 10705
  \l 10706
  \l 10707
  \l 10708
  \l 10709
  \l 1070A
  \l 1070B
  \l 1070C
  \l 1070D
  \l 1070E
  \l 1070F
  \l 10710
  \l 10711
  \l 10712
  \l 10713
  \l 10714
  \l 10715
  \l 10716
  \l 10717
  \l 10718
  \l 10719
  \l 1071A
  \l 1071B
  \l 1071C
  \l 1071D
  \l 1071E
  \l 1071F
  \l 10720
  \l 10721
  \l 10722
  \l 10723
  \l 10724
  \l 10725
  \l 10726
  \l 10727
  \l 10728
  \l 10729
  \l 1072A
  \l 1072B
  \l 1072C
  \l 1072D
  \l 1072E
  \l 1072F
  \l 10730
  \l 10731
  \l 10732
  \l 10733
  \l 10734
  \l 10735
  \l 10736
  \l 10740
  \l 10741
  \l 10742
  \l 10743
  \l 10744
  \l 10745
  \l 10746
  \l 10747
  \l 10748
  \l 10749
  \l 1074A
  \l 1074B
  \l 1074C
  \l 1074D
  \l 1074E
  \l 1074F
  \l 10750
  \l 10751
  \l 10752
  \l 10753
  \l 10754
  \l 10755
  \l 10760
  \l 10761
  \l 10762
  \l 10763
  \l 10764
  \l 10765
  \l 10766
  \l 10767
  \l 10800
  \l 10801
  \l 10802
  \l 10803
  \l 10804
  \l 10805
  \l 10808
  \l 1080A
  \l 1080B
  \l 1080C
  \l 1080D
  \l 1080E
  \l 1080F
  \l 10810
  \l 10811
  \l 10812
  \l 10813
  \l 10814
  \l 10815
  \l 10816
  \l 10817
  \l 10818
  \l 10819
  \l 1081A
  \l 1081B
  \l 1081C
  \l 1081D
  \l 1081E
  \l 1081F
  \l 10820
  \l 10821
  \l 10822
  \l 10823
  \l 10824
  \l 10825
  \l 10826
  \l 10827
  \l 10828
  \l 10829
  \l 1082A
  \l 1082B
  \l 1082C
  \l 1082D
  \l 1082E
  \l 1082F
  \l 10830
  \l 10831
  \l 10832
  \l 10833
  \l 10834
  \l 10835
  \l 10837
  \l 10838
  \l 1083C
  \l 1083F
  \l 10840
  \l 10841
  \l 10842
  \l 10843
  \l 10844
  \l 10845
  \l 10846
  \l 10847
  \l 10848
  \l 10849
  \l 1084A
  \l 1084B
  \l 1084C
  \l 1084D
  \l 1084E
  \l 1084F
  \l 10850
  \l 10851
  \l 10852
  \l 10853
  \l 10854
  \l 10855
  \l 10860
  \l 10861
  \l 10862
  \l 10863
  \l 10864
  \l 10865
  \l 10866
  \l 10867
  \l 10868
  \l 10869
  \l 1086A
  \l 1086B
  \l 1086C
  \l 1086D
  \l 1086E
  \l 1086F
  \l 10870
  \l 10871
  \l 10872
  \l 10873
  \l 10874
  \l 10875
  \l 10876
  \l 10880
  \l 10881
  \l 10882
  \l 10883
  \l 10884
  \l 10885
  \l 10886
  \l 10887
  \l 10888
  \l 10889
  \l 1088A
  \l 1088B
  \l 1088C
  \l 1088D
  \l 1088E
  \l 1088F
  \l 10890
  \l 10891
  \l 10892
  \l 10893
  \l 10894
  \l 10895
  \l 10896
  \l 10897
  \l 10898
  \l 10899
  \l 1089A
  \l 1089B
  \l 1089C
  \l 1089D
  \l 1089E
  \l 10900
  \l 10901
  \l 10902
  \l 10903
  \l 10904
  \l 10905
  \l 10906
  \l 10907
  \l 10908
  \l 10909
  \l 1090A
  \l 1090B
  \l 1090C
  \l 1090D
  \l 1090E
  \l 1090F
  \l 10910
  \l 10911
  \l 10912
  \l 10913
  \l 10914
  \l 10915
  \l 10920
  \l 10921
  \l 10922
  \l 10923
  \l 10924
  \l 10925
  \l 10926
  \l 10927
  \l 10928
  \l 10929
  \l 1092A
  \l 1092B
  \l 1092C
  \l 1092D
  \l 1092E
  \l 1092F
  \l 10930
  \l 10931
  \l 10932
  \l 10933
  \l 10934
  \l 10935
  \l 10936
  \l 10937
  \l 10938
  \l 10939
  \l 10980
  \l 10981
  \l 10982
  \l 10983
  \l 10984
  \l 10985
  \l 10986
  \l 10987
  \l 10988
  \l 10989
  \l 1098A
  \l 1098B
  \l 1098C
  \l 1098D
  \l 1098E
  \l 1098F
  \l 10990
  \l 10991
  \l 10992
  \l 10993
  \l 10994
  \l 10995
  \l 10996
  \l 10997
  \l 10998
  \l 10999
  \l 1099A
  \l 1099B
  \l 1099C
  \l 1099D
  \l 1099E
  \l 1099F
  \l 109A0
  \l 109A1
  \l 109A2
  \l 109A3
  \l 109A4
  \l 109A5
  \l 109A6
  \l 109A7
  \l 109A8
  \l 109A9
  \l 109AA
  \l 109AB
  \l 109AC
  \l 109AD
  \l 109AE
  \l 109AF
  \l 109B0
  \l 109B1
  \l 109B2
  \l 109B3
  \l 109B4
  \l 109B5
  \l 109B6
  \l 109B7
  \l 109BE
  \l 109BF
  \l 10A00
  \l 10A01
  \l 10A02
  \l 10A03
  \l 10A05
  \l 10A06
  \l 10A0C
  \l 10A0D
  \l 10A0E
  \l 10A0F
  \l 10A10
  \l 10A11
  \l 10A12
  \l 10A13
  \l 10A15
  \l 10A16
  \l 10A17
  \l 10A19
  \l 10A1A
  \l 10A1B
  \l 10A1C
  \l 10A1D
  \l 10A1E
  \l 10A1F
  \l 10A20
  \l 10A21
  \l 10A22
  \l 10A23
  \l 10A24
  \l 10A25
  \l 10A26
  \l 10A27
  \l 10A28
  \l 10A29
  \l 10A2A
  \l 10A2B
  \l 10A2C
  \l 10A2D
  \l 10A2E
  \l 10A2F
  \l 10A30
  \l 10A31
  \l 10A32
  \l 10A33
  \l 10A38
  \l 10A39
  \l 10A3A
  \l 10A3F
  \l 10A60
  \l 10A61
  \l 10A62
  \l 10A63
  \l 10A64
  \l 10A65
  \l 10A66
  \l 10A67
  \l 10A68
  \l 10A69
  \l 10A6A
  \l 10A6B
  \l 10A6C
  \l 10A6D
  \l 10A6E
  \l 10A6F
  \l 10A70
  \l 10A71
  \l 10A72
  \l 10A73
  \l 10A74
  \l 10A75
  \l 10A76
  \l 10A77
  \l 10A78
  \l 10A79
  \l 10A7A
  \l 10A7B
  \l 10A7C
  \l 10A80
  \l 10A81
  \l 10A82
  \l 10A83
  \l 10A84
  \l 10A85
  \l 10A86
  \l 10A87
  \l 10A88
  \l 10A89
  \l 10A8A
  \l 10A8B
  \l 10A8C
  \l 10A8D
  \l 10A8E
  \l 10A8F
  \l 10A90
  \l 10A91
  \l 10A92
  \l 10A93
  \l 10A94
  \l 10A95
  \l 10A96
  \l 10A97
  \l 10A98
  \l 10A99
  \l 10A9A
  \l 10A9B
  \l 10A9C
  \l 10AC0
  \l 10AC1
  \l 10AC2
  \l 10AC3
  \l 10AC4
  \l 10AC5
  \l 10AC6
  \l 10AC7
  \l 10AC9
  \l 10ACA
  \l 10ACB
  \l 10ACC
  \l 10ACD
  \l 10ACE
  \l 10ACF
  \l 10AD0
  \l 10AD1
  \l 10AD2
  \l 10AD3
  \l 10AD4
  \l 10AD5
  \l 10AD6
  \l 10AD7
  \l 10AD8
  \l 10AD9
  \l 10ADA
  \l 10ADB
  \l 10ADC
  \l 10ADD
  \l 10ADE
  \l 10ADF
  \l 10AE0
  \l 10AE1
  \l 10AE2
  \l 10AE3
  \l 10AE4
  \l 10AE5
  \l 10AE6
  \l 10B00
  \l 10B01
  \l 10B02
  \l 10B03
  \l 10B04
  \l 10B05
  \l 10B06
  \l 10B07
  \l 10B08
  \l 10B09
  \l 10B0A
  \l 10B0B
  \l 10B0C
  \l 10B0D
  \l 10B0E
  \l 10B0F
  \l 10B10
  \l 10B11
  \l 10B12
  \l 10B13
  \l 10B14
  \l 10B15
  \l 10B16
  \l 10B17
  \l 10B18
  \l 10B19
  \l 10B1A
  \l 10B1B
  \l 10B1C
  \l 10B1D
  \l 10B1E
  \l 10B1F
  \l 10B20
  \l 10B21
  \l 10B22
  \l 10B23
  \l 10B24
  \l 10B25
  \l 10B26
  \l 10B27
  \l 10B28
  \l 10B29
  \l 10B2A
  \l 10B2B
  \l 10B2C
  \l 10B2D
  \l 10B2E
  \l 10B2F
  \l 10B30
  \l 10B31
  \l 10B32
  \l 10B33
  \l 10B34
  \l 10B35
  \l 10B40
  \l 10B41
  \l 10B42
  \l 10B43
  \l 10B44
  \l 10B45
  \l 10B46
  \l 10B47
  \l 10B48
  \l 10B49
  \l 10B4A
  \l 10B4B
  \l 10B4C
  \l 10B4D
  \l 10B4E
  \l 10B4F
  \l 10B50
  \l 10B51
  \l 10B52
  \l 10B53
  \l 10B54
  \l 10B55
  \l 10B60
  \l 10B61
  \l 10B62
  \l 10B63
  \l 10B64
  \l 10B65
  \l 10B66
  \l 10B67
  \l 10B68
  \l 10B69
  \l 10B6A
  \l 10B6B
  \l 10B6C
  \l 10B6D
  \l 10B6E
  \l 10B6F
  \l 10B70
  \l 10B71
  \l 10B72
  \l 10B80
  \l 10B81
  \l 10B82
  \l 10B83
  \l 10B84
  \l 10B85
  \l 10B86
  \l 10B87
  \l 10B88
  \l 10B89
  \l 10B8A
  \l 10B8B
  \l 10B8C
  \l 10B8D
  \l 10B8E
  \l 10B8F
  \l 10B90
  \l 10B91
  \l 10C00
  \l 10C01
  \l 10C02
  \l 10C03
  \l 10C04
  \l 10C05
  \l 10C06
  \l 10C07
  \l 10C08
  \l 10C09
  \l 10C0A
  \l 10C0B
  \l 10C0C
  \l 10C0D
  \l 10C0E
  \l 10C0F
  \l 10C10
  \l 10C11
  \l 10C12
  \l 10C13
  \l 10C14
  \l 10C15
  \l 10C16
  \l 10C17
  \l 10C18
  \l 10C19
  \l 10C1A
  \l 10C1B
  \l 10C1C
  \l 10C1D
  \l 10C1E
  \l 10C1F
  \l 10C20
  \l 10C21
  \l 10C22
  \l 10C23
  \l 10C24
  \l 10C25
  \l 10C26
  \l 10C27
  \l 10C28
  \l 10C29
  \l 10C2A
  \l 10C2B
  \l 10C2C
  \l 10C2D
  \l 10C2E
  \l 10C2F
  \l 10C30
  \l 10C31
  \l 10C32
  \l 10C33
  \l 10C34
  \l 10C35
  \l 10C36
  \l 10C37
  \l 10C38
  \l 10C39
  \l 10C3A
  \l 10C3B
  \l 10C3C
  \l 10C3D
  \l 10C3E
  \l 10C3F
  \l 10C40
  \l 10C41
  \l 10C42
  \l 10C43
  \l 10C44
  \l 10C45
  \l 10C46
  \l 10C47
  \l 10C48
  \l 11000
  \l 11001
  \l 11002
  \l 11003
  \l 11004
  \l 11005
  \l 11006
  \l 11007
  \l 11008
  \l 11009
  \l 1100A
  \l 1100B
  \l 1100C
  \l 1100D
  \l 1100E
  \l 1100F
  \l 11010
  \l 11011
  \l 11012
  \l 11013
  \l 11014
  \l 11015
  \l 11016
  \l 11017
  \l 11018
  \l 11019
  \l 1101A
  \l 1101B
  \l 1101C
  \l 1101D
  \l 1101E
  \l 1101F
  \l 11020
  \l 11021
  \l 11022
  \l 11023
  \l 11024
  \l 11025
  \l 11026
  \l 11027
  \l 11028
  \l 11029
  \l 1102A
  \l 1102B
  \l 1102C
  \l 1102D
  \l 1102E
  \l 1102F
  \l 11030
  \l 11031
  \l 11032
  \l 11033
  \l 11034
  \l 11035
  \l 11036
  \l 11037
  \l 11038
  \l 11039
  \l 1103A
  \l 1103B
  \l 1103C
  \l 1103D
  \l 1103E
  \l 1103F
  \l 11040
  \l 11041
  \l 11042
  \l 11043
  \l 11044
  \l 11045
  \l 11046
  \l 1107F
  \l 11080
  \l 11081
  \l 11082
  \l 11083
  \l 11084
  \l 11085
  \l 11086
  \l 11087
  \l 11088
  \l 11089
  \l 1108A
  \l 1108B
  \l 1108C
  \l 1108D
  \l 1108E
  \l 1108F
  \l 11090
  \l 11091
  \l 11092
  \l 11093
  \l 11094
  \l 11095
  \l 11096
  \l 11097
  \l 11098
  \l 11099
  \l 1109A
  \l 1109B
  \l 1109C
  \l 1109D
  \l 1109E
  \l 1109F
  \l 110A0
  \l 110A1
  \l 110A2
  \l 110A3
  \l 110A4
  \l 110A5
  \l 110A6
  \l 110A7
  \l 110A8
  \l 110A9
  \l 110AA
  \l 110AB
  \l 110AC
  \l 110AD
  \l 110AE
  \l 110AF
  \l 110B0
  \l 110B1
  \l 110B2
  \l 110B3
  \l 110B4
  \l 110B5
  \l 110B6
  \l 110B7
  \l 110B8
  \l 110B9
  \l 110BA
  \l 110D0
  \l 110D1
  \l 110D2
  \l 110D3
  \l 110D4
  \l 110D5
  \l 110D6
  \l 110D7
  \l 110D8
  \l 110D9
  \l 110DA
  \l 110DB
  \l 110DC
  \l 110DD
  \l 110DE
  \l 110DF
  \l 110E0
  \l 110E1
  \l 110E2
  \l 110E3
  \l 110E4
  \l 110E5
  \l 110E6
  \l 110E7
  \l 110E8
  \l 11100
  \l 11101
  \l 11102
  \l 11103
  \l 11104
  \l 11105
  \l 11106
  \l 11107
  \l 11108
  \l 11109
  \l 1110A
  \l 1110B
  \l 1110C
  \l 1110D
  \l 1110E
  \l 1110F
  \l 11110
  \l 11111
  \l 11112
  \l 11113
  \l 11114
  \l 11115
  \l 11116
  \l 11117
  \l 11118
  \l 11119
  \l 1111A
  \l 1111B
  \l 1111C
  \l 1111D
  \l 1111E
  \l 1111F
  \l 11120
  \l 11121
  \l 11122
  \l 11123
  \l 11124
  \l 11125
  \l 11126
  \l 11127
  \l 11128
  \l 11129
  \l 1112A
  \l 1112B
  \l 1112C
  \l 1112D
  \l 1112E
  \l 1112F
  \l 11130
  \l 11131
  \l 11132
  \l 11133
  \l 11134
  \l 11150
  \l 11151
  \l 11152
  \l 11153
  \l 11154
  \l 11155
  \l 11156
  \l 11157
  \l 11158
  \l 11159
  \l 1115A
  \l 1115B
  \l 1115C
  \l 1115D
  \l 1115E
  \l 1115F
  \l 11160
  \l 11161
  \l 11162
  \l 11163
  \l 11164
  \l 11165
  \l 11166
  \l 11167
  \l 11168
  \l 11169
  \l 1116A
  \l 1116B
  \l 1116C
  \l 1116D
  \l 1116E
  \l 1116F
  \l 11170
  \l 11171
  \l 11172
  \l 11173
  \l 11176
  \l 11180
  \l 11181
  \l 11182
  \l 11183
  \l 11184
  \l 11185
  \l 11186
  \l 11187
  \l 11188
  \l 11189
  \l 1118A
  \l 1118B
  \l 1118C
  \l 1118D
  \l 1118E
  \l 1118F
  \l 11190
  \l 11191
  \l 11192
  \l 11193
  \l 11194
  \l 11195
  \l 11196
  \l 11197
  \l 11198
  \l 11199
  \l 1119A
  \l 1119B
  \l 1119C
  \l 1119D
  \l 1119E
  \l 1119F
  \l 111A0
  \l 111A1
  \l 111A2
  \l 111A3
  \l 111A4
  \l 111A5
  \l 111A6
  \l 111A7
  \l 111A8
  \l 111A9
  \l 111AA
  \l 111AB
  \l 111AC
  \l 111AD
  \l 111AE
  \l 111AF
  \l 111B0
  \l 111B1
  \l 111B2
  \l 111B3
  \l 111B4
  \l 111B5
  \l 111B6
  \l 111B7
  \l 111B8
  \l 111B9
  \l 111BA
  \l 111BB
  \l 111BC
  \l 111BD
  \l 111BE
  \l 111BF
  \l 111C0
  \l 111C1
  \l 111C2
  \l 111C3
  \l 111C4
  \l 111DA
  \l 11200
  \l 11201
  \l 11202
  \l 11203
  \l 11204
  \l 11205
  \l 11206
  \l 11207
  \l 11208
  \l 11209
  \l 1120A
  \l 1120B
  \l 1120C
  \l 1120D
  \l 1120E
  \l 1120F
  \l 11210
  \l 11211
  \l 11213
  \l 11214
  \l 11215
  \l 11216
  \l 11217
  \l 11218
  \l 11219
  \l 1121A
  \l 1121B
  \l 1121C
  \l 1121D
  \l 1121E
  \l 1121F
  \l 11220
  \l 11221
  \l 11222
  \l 11223
  \l 11224
  \l 11225
  \l 11226
  \l 11227
  \l 11228
  \l 11229
  \l 1122A
  \l 1122B
  \l 1122C
  \l 1122D
  \l 1122E
  \l 1122F
  \l 11230
  \l 11231
  \l 11232
  \l 11233
  \l 11234
  \l 11235
  \l 11236
  \l 11237
  \l 112B0
  \l 112B1
  \l 112B2
  \l 112B3
  \l 112B4
  \l 112B5
  \l 112B6
  \l 112B7
  \l 112B8
  \l 112B9
  \l 112BA
  \l 112BB
  \l 112BC
  \l 112BD
  \l 112BE
  \l 112BF
  \l 112C0
  \l 112C1
  \l 112C2
  \l 112C3
  \l 112C4
  \l 112C5
  \l 112C6
  \l 112C7
  \l 112C8
  \l 112C9
  \l 112CA
  \l 112CB
  \l 112CC
  \l 112CD
  \l 112CE
  \l 112CF
  \l 112D0
  \l 112D1
  \l 112D2
  \l 112D3
  \l 112D4
  \l 112D5
  \l 112D6
  \l 112D7
  \l 112D8
  \l 112D9
  \l 112DA
  \l 112DB
  \l 112DC
  \l 112DD
  \l 112DE
  \l 112DF
  \l 112E0
  \l 112E1
  \l 112E2
  \l 112E3
  \l 112E4
  \l 112E5
  \l 112E6
  \l 112E7
  \l 112E8
  \l 112E9
  \l 112EA
  \l 11301
  \l 11302
  \l 11303
  \l 11305
  \l 11306
  \l 11307
  \l 11308
  \l 11309
  \l 1130A
  \l 1130B
  \l 1130C
  \l 1130F
  \l 11310
  \l 11313
  \l 11314
  \l 11315
  \l 11316
  \l 11317
  \l 11318
  \l 11319
  \l 1131A
  \l 1131B
  \l 1131C
  \l 1131D
  \l 1131E
  \l 1131F
  \l 11320
  \l 11321
  \l 11322
  \l 11323
  \l 11324
  \l 11325
  \l 11326
  \l 11327
  \l 11328
  \l 1132A
  \l 1132B
  \l 1132C
  \l 1132D
  \l 1132E
  \l 1132F
  \l 11330
  \l 11332
  \l 11333
  \l 11335
  \l 11336
  \l 11337
  \l 11338
  \l 11339
  \l 1133C
  \l 1133D
  \l 1133E
  \l 1133F
  \l 11340
  \l 11341
  \l 11342
  \l 11343
  \l 11344
  \l 11347
  \l 11348
  \l 1134B
  \l 1134C
  \l 1134D
  \l 11357
  \l 1135D
  \l 1135E
  \l 1135F
  \l 11360
  \l 11361
  \l 11362
  \l 11363
  \l 11366
  \l 11367
  \l 11368
  \l 11369
  \l 1136A
  \l 1136B
  \l 1136C
  \l 11370
  \l 11371
  \l 11372
  \l 11373
  \l 11374
  \l 11480
  \l 11481
  \l 11482
  \l 11483
  \l 11484
  \l 11485
  \l 11486
  \l 11487
  \l 11488
  \l 11489
  \l 1148A
  \l 1148B
  \l 1148C
  \l 1148D
  \l 1148E
  \l 1148F
  \l 11490
  \l 11491
  \l 11492
  \l 11493
  \l 11494
  \l 11495
  \l 11496
  \l 11497
  \l 11498
  \l 11499
  \l 1149A
  \l 1149B
  \l 1149C
  \l 1149D
  \l 1149E
  \l 1149F
  \l 114A0
  \l 114A1
  \l 114A2
  \l 114A3
  \l 114A4
  \l 114A5
  \l 114A6
  \l 114A7
  \l 114A8
  \l 114A9
  \l 114AA
  \l 114AB
  \l 114AC
  \l 114AD
  \l 114AE
  \l 114AF
  \l 114B0
  \l 114B1
  \l 114B2
  \l 114B3
  \l 114B4
  \l 114B5
  \l 114B6
  \l 114B7
  \l 114B8
  \l 114B9
  \l 114BA
  \l 114BB
  \l 114BC
  \l 114BD
  \l 114BE
  \l 114BF
  \l 114C0
  \l 114C1
  \l 114C2
  \l 114C3
  \l 114C4
  \l 114C5
  \l 114C7
  \l 11580
  \l 11581
  \l 11582
  \l 11583
  \l 11584
  \l 11585
  \l 11586
  \l 11587
  \l 11588
  \l 11589
  \l 1158A
  \l 1158B
  \l 1158C
  \l 1158D
  \l 1158E
  \l 1158F
  \l 11590
  \l 11591
  \l 11592
  \l 11593
  \l 11594
  \l 11595
  \l 11596
  \l 11597
  \l 11598
  \l 11599
  \l 1159A
  \l 1159B
  \l 1159C
  \l 1159D
  \l 1159E
  \l 1159F
  \l 115A0
  \l 115A1
  \l 115A2
  \l 115A3
  \l 115A4
  \l 115A5
  \l 115A6
  \l 115A7
  \l 115A8
  \l 115A9
  \l 115AA
  \l 115AB
  \l 115AC
  \l 115AD
  \l 115AE
  \l 115AF
  \l 115B0
  \l 115B1
  \l 115B2
  \l 115B3
  \l 115B4
  \l 115B5
  \l 115B8
  \l 115B9
  \l 115BA
  \l 115BB
  \l 115BC
  \l 115BD
  \l 115BE
  \l 115BF
  \l 115C0
  \l 11600
  \l 11601
  \l 11602
  \l 11603
  \l 11604
  \l 11605
  \l 11606
  \l 11607
  \l 11608
  \l 11609
  \l 1160A
  \l 1160B
  \l 1160C
  \l 1160D
  \l 1160E
  \l 1160F
  \l 11610
  \l 11611
  \l 11612
  \l 11613
  \l 11614
  \l 11615
  \l 11616
  \l 11617
  \l 11618
  \l 11619
  \l 1161A
  \l 1161B
  \l 1161C
  \l 1161D
  \l 1161E
  \l 1161F
  \l 11620
  \l 11621
  \l 11622
  \l 11623
  \l 11624
  \l 11625
  \l 11626
  \l 11627
  \l 11628
  \l 11629
  \l 1162A
  \l 1162B
  \l 1162C
  \l 1162D
  \l 1162E
  \l 1162F
  \l 11630
  \l 11631
  \l 11632
  \l 11633
  \l 11634
  \l 11635
  \l 11636
  \l 11637
  \l 11638
  \l 11639
  \l 1163A
  \l 1163B
  \l 1163C
  \l 1163D
  \l 1163E
  \l 1163F
  \l 11640
  \l 11644
  \l 11680
  \l 11681
  \l 11682
  \l 11683
  \l 11684
  \l 11685
  \l 11686
  \l 11687
  \l 11688
  \l 11689
  \l 1168A
  \l 1168B
  \l 1168C
  \l 1168D
  \l 1168E
  \l 1168F
  \l 11690
  \l 11691
  \l 11692
  \l 11693
  \l 11694
  \l 11695
  \l 11696
  \l 11697
  \l 11698
  \l 11699
  \l 1169A
  \l 1169B
  \l 1169C
  \l 1169D
  \l 1169E
  \l 1169F
  \l 116A0
  \l 116A1
  \l 116A2
  \l 116A3
  \l 116A4
  \l 116A5
  \l 116A6
  \l 116A7
  \l 116A8
  \l 116A9
  \l 116AA
  \l 116AB
  \l 116AC
  \l 116AD
  \l 116AE
  \l 116AF
  \l 116B0
  \l 116B1
  \l 116B2
  \l 116B3
  \l 116B4
  \l 116B5
  \l 116B6
  \l 116B7
  \L 118A0 118A0 118C0
  \L 118A1 118A1 118C1
  \L 118A2 118A2 118C2
  \L 118A3 118A3 118C3
  \L 118A4 118A4 118C4
  \L 118A5 118A5 118C5
  \L 118A6 118A6 118C6
  \L 118A7 118A7 118C7
  \L 118A8 118A8 118C8
  \L 118A9 118A9 118C9
  \L 118AA 118AA 118CA
  \L 118AB 118AB 118CB
  \L 118AC 118AC 118CC
  \L 118AD 118AD 118CD
  \L 118AE 118AE 118CE
  \L 118AF 118AF 118CF
  \L 118B0 118B0 118D0
  \L 118B1 118B1 118D1
  \L 118B2 118B2 118D2
  \L 118B3 118B3 118D3
  \L 118B4 118B4 118D4
  \L 118B5 118B5 118D5
  \L 118B6 118B6 118D6
  \L 118B7 118B7 118D7
  \L 118B8 118B8 118D8
  \L 118B9 118B9 118D9
  \L 118BA 118BA 118DA
  \L 118BB 118BB 118DB
  \L 118BC 118BC 118DC
  \L 118BD 118BD 118DD
  \L 118BE 118BE 118DE
  \L 118BF 118BF 118DF
  \L 118C0 118A0 118C0
  \L 118C1 118A1 118C1
  \L 118C2 118A2 118C2
  \L 118C3 118A3 118C3
  \L 118C4 118A4 118C4
  \L 118C5 118A5 118C5
  \L 118C6 118A6 118C6
  \L 118C7 118A7 118C7
  \L 118C8 118A8 118C8
  \L 118C9 118A9 118C9
  \L 118CA 118AA 118CA
  \L 118CB 118AB 118CB
  \L 118CC 118AC 118CC
  \L 118CD 118AD 118CD
  \L 118CE 118AE 118CE
  \L 118CF 118AF 118CF
  \L 118D0 118B0 118D0
  \L 118D1 118B1 118D1
  \L 118D2 118B2 118D2
  \L 118D3 118B3 118D3
  \L 118D4 118B4 118D4
  \L 118D5 118B5 118D5
  \L 118D6 118B6 118D6
  \L 118D7 118B7 118D7
  \L 118D8 118B8 118D8
  \L 118D9 118B9 118D9
  \L 118DA 118BA 118DA
  \L 118DB 118BB 118DB
  \L 118DC 118BC 118DC
  \L 118DD 118BD 118DD
  \L 118DE 118BE 118DE
  \L 118DF 118BF 118DF
  \l 118FF
  \l 11AC0
  \l 11AC1
  \l 11AC2
  \l 11AC3
  \l 11AC4
  \l 11AC5
  \l 11AC6
  \l 11AC7
  \l 11AC8
  \l 11AC9
  \l 11ACA
  \l 11ACB
  \l 11ACC
  \l 11ACD
  \l 11ACE
  \l 11ACF
  \l 11AD0
  \l 11AD1
  \l 11AD2
  \l 11AD3
  \l 11AD4
  \l 11AD5
  \l 11AD6
  \l 11AD7
  \l 11AD8
  \l 11AD9
  \l 11ADA
  \l 11ADB
  \l 11ADC
  \l 11ADD
  \l 11ADE
  \l 11ADF
  \l 11AE0
  \l 11AE1
  \l 11AE2
  \l 11AE3
  \l 11AE4
  \l 11AE5
  \l 11AE6
  \l 11AE7
  \l 11AE8
  \l 11AE9
  \l 11AEA
  \l 11AEB
  \l 11AEC
  \l 11AED
  \l 11AEE
  \l 11AEF
  \l 11AF0
  \l 11AF1
  \l 11AF2
  \l 11AF3
  \l 11AF4
  \l 11AF5
  \l 11AF6
  \l 11AF7
  \l 11AF8
  \l 12000
  \l 12001
  \l 12002
  \l 12003
  \l 12004
  \l 12005
  \l 12006
  \l 12007
  \l 12008
  \l 12009
  \l 1200A
  \l 1200B
  \l 1200C
  \l 1200D
  \l 1200E
  \l 1200F
  \l 12010
  \l 12011
  \l 12012
  \l 12013
  \l 12014
  \l 12015
  \l 12016
  \l 12017
  \l 12018
  \l 12019
  \l 1201A
  \l 1201B
  \l 1201C
  \l 1201D
  \l 1201E
  \l 1201F
  \l 12020
  \l 12021
  \l 12022
  \l 12023
  \l 12024
  \l 12025
  \l 12026
  \l 12027
  \l 12028
  \l 12029
  \l 1202A
  \l 1202B
  \l 1202C
  \l 1202D
  \l 1202E
  \l 1202F
  \l 12030
  \l 12031
  \l 12032
  \l 12033
  \l 12034
  \l 12035
  \l 12036
  \l 12037
  \l 12038
  \l 12039
  \l 1203A
  \l 1203B
  \l 1203C
  \l 1203D
  \l 1203E
  \l 1203F
  \l 12040
  \l 12041
  \l 12042
  \l 12043
  \l 12044
  \l 12045
  \l 12046
  \l 12047
  \l 12048
  \l 12049
  \l 1204A
  \l 1204B
  \l 1204C
  \l 1204D
  \l 1204E
  \l 1204F
  \l 12050
  \l 12051
  \l 12052
  \l 12053
  \l 12054
  \l 12055
  \l 12056
  \l 12057
  \l 12058
  \l 12059
  \l 1205A
  \l 1205B
  \l 1205C
  \l 1205D
  \l 1205E
  \l 1205F
  \l 12060
  \l 12061
  \l 12062
  \l 12063
  \l 12064
  \l 12065
  \l 12066
  \l 12067
  \l 12068
  \l 12069
  \l 1206A
  \l 1206B
  \l 1206C
  \l 1206D
  \l 1206E
  \l 1206F
  \l 12070
  \l 12071
  \l 12072
  \l 12073
  \l 12074
  \l 12075
  \l 12076
  \l 12077
  \l 12078
  \l 12079
  \l 1207A
  \l 1207B
  \l 1207C
  \l 1207D
  \l 1207E
  \l 1207F
  \l 12080
  \l 12081
  \l 12082
  \l 12083
  \l 12084
  \l 12085
  \l 12086
  \l 12087
  \l 12088
  \l 12089
  \l 1208A
  \l 1208B
  \l 1208C
  \l 1208D
  \l 1208E
  \l 1208F
  \l 12090
  \l 12091
  \l 12092
  \l 12093
  \l 12094
  \l 12095
  \l 12096
  \l 12097
  \l 12098
  \l 12099
  \l 1209A
  \l 1209B
  \l 1209C
  \l 1209D
  \l 1209E
  \l 1209F
  \l 120A0
  \l 120A1
  \l 120A2
  \l 120A3
  \l 120A4
  \l 120A5
  \l 120A6
  \l 120A7
  \l 120A8
  \l 120A9
  \l 120AA
  \l 120AB
  \l 120AC
  \l 120AD
  \l 120AE
  \l 120AF
  \l 120B0
  \l 120B1
  \l 120B2
  \l 120B3
  \l 120B4
  \l 120B5
  \l 120B6
  \l 120B7
  \l 120B8
  \l 120B9
  \l 120BA
  \l 120BB
  \l 120BC
  \l 120BD
  \l 120BE
  \l 120BF
  \l 120C0
  \l 120C1
  \l 120C2
  \l 120C3
  \l 120C4
  \l 120C5
  \l 120C6
  \l 120C7
  \l 120C8
  \l 120C9
  \l 120CA
  \l 120CB
  \l 120CC
  \l 120CD
  \l 120CE
  \l 120CF
  \l 120D0
  \l 120D1
  \l 120D2
  \l 120D3
  \l 120D4
  \l 120D5
  \l 120D6
  \l 120D7
  \l 120D8
  \l 120D9
  \l 120DA
  \l 120DB
  \l 120DC
  \l 120DD
  \l 120DE
  \l 120DF
  \l 120E0
  \l 120E1
  \l 120E2
  \l 120E3
  \l 120E4
  \l 120E5
  \l 120E6
  \l 120E7
  \l 120E8
  \l 120E9
  \l 120EA
  \l 120EB
  \l 120EC
  \l 120ED
  \l 120EE
  \l 120EF
  \l 120F0
  \l 120F1
  \l 120F2
  \l 120F3
  \l 120F4
  \l 120F5
  \l 120F6
  \l 120F7
  \l 120F8
  \l 120F9
  \l 120FA
  \l 120FB
  \l 120FC
  \l 120FD
  \l 120FE
  \l 120FF
  \l 12100
  \l 12101
  \l 12102
  \l 12103
  \l 12104
  \l 12105
  \l 12106
  \l 12107
  \l 12108
  \l 12109
  \l 1210A
  \l 1210B
  \l 1210C
  \l 1210D
  \l 1210E
  \l 1210F
  \l 12110
  \l 12111
  \l 12112
  \l 12113
  \l 12114
  \l 12115
  \l 12116
  \l 12117
  \l 12118
  \l 12119
  \l 1211A
  \l 1211B
  \l 1211C
  \l 1211D
  \l 1211E
  \l 1211F
  \l 12120
  \l 12121
  \l 12122
  \l 12123
  \l 12124
  \l 12125
  \l 12126
  \l 12127
  \l 12128
  \l 12129
  \l 1212A
  \l 1212B
  \l 1212C
  \l 1212D
  \l 1212E
  \l 1212F
  \l 12130
  \l 12131
  \l 12132
  \l 12133
  \l 12134
  \l 12135
  \l 12136
  \l 12137
  \l 12138
  \l 12139
  \l 1213A
  \l 1213B
  \l 1213C
  \l 1213D
  \l 1213E
  \l 1213F
  \l 12140
  \l 12141
  \l 12142
  \l 12143
  \l 12144
  \l 12145
  \l 12146
  \l 12147
  \l 12148
  \l 12149
  \l 1214A
  \l 1214B
  \l 1214C
  \l 1214D
  \l 1214E
  \l 1214F
  \l 12150
  \l 12151
  \l 12152
  \l 12153
  \l 12154
  \l 12155
  \l 12156
  \l 12157
  \l 12158
  \l 12159
  \l 1215A
  \l 1215B
  \l 1215C
  \l 1215D
  \l 1215E
  \l 1215F
  \l 12160
  \l 12161
  \l 12162
  \l 12163
  \l 12164
  \l 12165
  \l 12166
  \l 12167
  \l 12168
  \l 12169
  \l 1216A
  \l 1216B
  \l 1216C
  \l 1216D
  \l 1216E
  \l 1216F
  \l 12170
  \l 12171
  \l 12172
  \l 12173
  \l 12174
  \l 12175
  \l 12176
  \l 12177
  \l 12178
  \l 12179
  \l 1217A
  \l 1217B
  \l 1217C
  \l 1217D
  \l 1217E
  \l 1217F
  \l 12180
  \l 12181
  \l 12182
  \l 12183
  \l 12184
  \l 12185
  \l 12186
  \l 12187
  \l 12188
  \l 12189
  \l 1218A
  \l 1218B
  \l 1218C
  \l 1218D
  \l 1218E
  \l 1218F
  \l 12190
  \l 12191
  \l 12192
  \l 12193
  \l 12194
  \l 12195
  \l 12196
  \l 12197
  \l 12198
  \l 12199
  \l 1219A
  \l 1219B
  \l 1219C
  \l 1219D
  \l 1219E
  \l 1219F
  \l 121A0
  \l 121A1
  \l 121A2
  \l 121A3
  \l 121A4
  \l 121A5
  \l 121A6
  \l 121A7
  \l 121A8
  \l 121A9
  \l 121AA
  \l 121AB
  \l 121AC
  \l 121AD
  \l 121AE
  \l 121AF
  \l 121B0
  \l 121B1
  \l 121B2
  \l 121B3
  \l 121B4
  \l 121B5
  \l 121B6
  \l 121B7
  \l 121B8
  \l 121B9
  \l 121BA
  \l 121BB
  \l 121BC
  \l 121BD
  \l 121BE
  \l 121BF
  \l 121C0
  \l 121C1
  \l 121C2
  \l 121C3
  \l 121C4
  \l 121C5
  \l 121C6
  \l 121C7
  \l 121C8
  \l 121C9
  \l 121CA
  \l 121CB
  \l 121CC
  \l 121CD
  \l 121CE
  \l 121CF
  \l 121D0
  \l 121D1
  \l 121D2
  \l 121D3
  \l 121D4
  \l 121D5
  \l 121D6
  \l 121D7
  \l 121D8
  \l 121D9
  \l 121DA
  \l 121DB
  \l 121DC
  \l 121DD
  \l 121DE
  \l 121DF
  \l 121E0
  \l 121E1
  \l 121E2
  \l 121E3
  \l 121E4
  \l 121E5
  \l 121E6
  \l 121E7
  \l 121E8
  \l 121E9
  \l 121EA
  \l 121EB
  \l 121EC
  \l 121ED
  \l 121EE
  \l 121EF
  \l 121F0
  \l 121F1
  \l 121F2
  \l 121F3
  \l 121F4
  \l 121F5
  \l 121F6
  \l 121F7
  \l 121F8
  \l 121F9
  \l 121FA
  \l 121FB
  \l 121FC
  \l 121FD
  \l 121FE
  \l 121FF
  \l 12200
  \l 12201
  \l 12202
  \l 12203
  \l 12204
  \l 12205
  \l 12206
  \l 12207
  \l 12208
  \l 12209
  \l 1220A
  \l 1220B
  \l 1220C
  \l 1220D
  \l 1220E
  \l 1220F
  \l 12210
  \l 12211
  \l 12212
  \l 12213
  \l 12214
  \l 12215
  \l 12216
  \l 12217
  \l 12218
  \l 12219
  \l 1221A
  \l 1221B
  \l 1221C
  \l 1221D
  \l 1221E
  \l 1221F
  \l 12220
  \l 12221
  \l 12222
  \l 12223
  \l 12224
  \l 12225
  \l 12226
  \l 12227
  \l 12228
  \l 12229
  \l 1222A
  \l 1222B
  \l 1222C
  \l 1222D
  \l 1222E
  \l 1222F
  \l 12230
  \l 12231
  \l 12232
  \l 12233
  \l 12234
  \l 12235
  \l 12236
  \l 12237
  \l 12238
  \l 12239
  \l 1223A
  \l 1223B
  \l 1223C
  \l 1223D
  \l 1223E
  \l 1223F
  \l 12240
  \l 12241
  \l 12242
  \l 12243
  \l 12244
  \l 12245
  \l 12246
  \l 12247
  \l 12248
  \l 12249
  \l 1224A
  \l 1224B
  \l 1224C
  \l 1224D
  \l 1224E
  \l 1224F
  \l 12250
  \l 12251
  \l 12252
  \l 12253
  \l 12254
  \l 12255
  \l 12256
  \l 12257
  \l 12258
  \l 12259
  \l 1225A
  \l 1225B
  \l 1225C
  \l 1225D
  \l 1225E
  \l 1225F
  \l 12260
  \l 12261
  \l 12262
  \l 12263
  \l 12264
  \l 12265
  \l 12266
  \l 12267
  \l 12268
  \l 12269
  \l 1226A
  \l 1226B
  \l 1226C
  \l 1226D
  \l 1226E
  \l 1226F
  \l 12270
  \l 12271
  \l 12272
  \l 12273
  \l 12274
  \l 12275
  \l 12276
  \l 12277
  \l 12278
  \l 12279
  \l 1227A
  \l 1227B
  \l 1227C
  \l 1227D
  \l 1227E
  \l 1227F
  \l 12280
  \l 12281
  \l 12282
  \l 12283
  \l 12284
  \l 12285
  \l 12286
  \l 12287
  \l 12288
  \l 12289
  \l 1228A
  \l 1228B
  \l 1228C
  \l 1228D
  \l 1228E
  \l 1228F
  \l 12290
  \l 12291
  \l 12292
  \l 12293
  \l 12294
  \l 12295
  \l 12296
  \l 12297
  \l 12298
  \l 12299
  \l 1229A
  \l 1229B
  \l 1229C
  \l 1229D
  \l 1229E
  \l 1229F
  \l 122A0
  \l 122A1
  \l 122A2
  \l 122A3
  \l 122A4
  \l 122A5
  \l 122A6
  \l 122A7
  \l 122A8
  \l 122A9
  \l 122AA
  \l 122AB
  \l 122AC
  \l 122AD
  \l 122AE
  \l 122AF
  \l 122B0
  \l 122B1
  \l 122B2
  \l 122B3
  \l 122B4
  \l 122B5
  \l 122B6
  \l 122B7
  \l 122B8
  \l 122B9
  \l 122BA
  \l 122BB
  \l 122BC
  \l 122BD
  \l 122BE
  \l 122BF
  \l 122C0
  \l 122C1
  \l 122C2
  \l 122C3
  \l 122C4
  \l 122C5
  \l 122C6
  \l 122C7
  \l 122C8
  \l 122C9
  \l 122CA
  \l 122CB
  \l 122CC
  \l 122CD
  \l 122CE
  \l 122CF
  \l 122D0
  \l 122D1
  \l 122D2
  \l 122D3
  \l 122D4
  \l 122D5
  \l 122D6
  \l 122D7
  \l 122D8
  \l 122D9
  \l 122DA
  \l 122DB
  \l 122DC
  \l 122DD
  \l 122DE
  \l 122DF
  \l 122E0
  \l 122E1
  \l 122E2
  \l 122E3
  \l 122E4
  \l 122E5
  \l 122E6
  \l 122E7
  \l 122E8
  \l 122E9
  \l 122EA
  \l 122EB
  \l 122EC
  \l 122ED
  \l 122EE
  \l 122EF
  \l 122F0
  \l 122F1
  \l 122F2
  \l 122F3
  \l 122F4
  \l 122F5
  \l 122F6
  \l 122F7
  \l 122F8
  \l 122F9
  \l 122FA
  \l 122FB
  \l 122FC
  \l 122FD
  \l 122FE
  \l 122FF
  \l 12300
  \l 12301
  \l 12302
  \l 12303
  \l 12304
  \l 12305
  \l 12306
  \l 12307
  \l 12308
  \l 12309
  \l 1230A
  \l 1230B
  \l 1230C
  \l 1230D
  \l 1230E
  \l 1230F
  \l 12310
  \l 12311
  \l 12312
  \l 12313
  \l 12314
  \l 12315
  \l 12316
  \l 12317
  \l 12318
  \l 12319
  \l 1231A
  \l 1231B
  \l 1231C
  \l 1231D
  \l 1231E
  \l 1231F
  \l 12320
  \l 12321
  \l 12322
  \l 12323
  \l 12324
  \l 12325
  \l 12326
  \l 12327
  \l 12328
  \l 12329
  \l 1232A
  \l 1232B
  \l 1232C
  \l 1232D
  \l 1232E
  \l 1232F
  \l 12330
  \l 12331
  \l 12332
  \l 12333
  \l 12334
  \l 12335
  \l 12336
  \l 12337
  \l 12338
  \l 12339
  \l 1233A
  \l 1233B
  \l 1233C
  \l 1233D
  \l 1233E
  \l 1233F
  \l 12340
  \l 12341
  \l 12342
  \l 12343
  \l 12344
  \l 12345
  \l 12346
  \l 12347
  \l 12348
  \l 12349
  \l 1234A
  \l 1234B
  \l 1234C
  \l 1234D
  \l 1234E
  \l 1234F
  \l 12350
  \l 12351
  \l 12352
  \l 12353
  \l 12354
  \l 12355
  \l 12356
  \l 12357
  \l 12358
  \l 12359
  \l 1235A
  \l 1235B
  \l 1235C
  \l 1235D
  \l 1235E
  \l 1235F
  \l 12360
  \l 12361
  \l 12362
  \l 12363
  \l 12364
  \l 12365
  \l 12366
  \l 12367
  \l 12368
  \l 12369
  \l 1236A
  \l 1236B
  \l 1236C
  \l 1236D
  \l 1236E
  \l 1236F
  \l 12370
  \l 12371
  \l 12372
  \l 12373
  \l 12374
  \l 12375
  \l 12376
  \l 12377
  \l 12378
  \l 12379
  \l 1237A
  \l 1237B
  \l 1237C
  \l 1237D
  \l 1237E
  \l 1237F
  \l 12380
  \l 12381
  \l 12382
  \l 12383
  \l 12384
  \l 12385
  \l 12386
  \l 12387
  \l 12388
  \l 12389
  \l 1238A
  \l 1238B
  \l 1238C
  \l 1238D
  \l 1238E
  \l 1238F
  \l 12390
  \l 12391
  \l 12392
  \l 12393
  \l 12394
  \l 12395
  \l 12396
  \l 12397
  \l 12398
  \l 13000
  \l 13001
  \l 13002
  \l 13003
  \l 13004
  \l 13005
  \l 13006
  \l 13007
  \l 13008
  \l 13009
  \l 1300A
  \l 1300B
  \l 1300C
  \l 1300D
  \l 1300E
  \l 1300F
  \l 13010
  \l 13011
  \l 13012
  \l 13013
  \l 13014
  \l 13015
  \l 13016
  \l 13017
  \l 13018
  \l 13019
  \l 1301A
  \l 1301B
  \l 1301C
  \l 1301D
  \l 1301E
  \l 1301F
  \l 13020
  \l 13021
  \l 13022
  \l 13023
  \l 13024
  \l 13025
  \l 13026
  \l 13027
  \l 13028
  \l 13029
  \l 1302A
  \l 1302B
  \l 1302C
  \l 1302D
  \l 1302E
  \l 1302F
  \l 13030
  \l 13031
  \l 13032
  \l 13033
  \l 13034
  \l 13035
  \l 13036
  \l 13037
  \l 13038
  \l 13039
  \l 1303A
  \l 1303B
  \l 1303C
  \l 1303D
  \l 1303E
  \l 1303F
  \l 13040
  \l 13041
  \l 13042
  \l 13043
  \l 13044
  \l 13045
  \l 13046
  \l 13047
  \l 13048
  \l 13049
  \l 1304A
  \l 1304B
  \l 1304C
  \l 1304D
  \l 1304E
  \l 1304F
  \l 13050
  \l 13051
  \l 13052
  \l 13053
  \l 13054
  \l 13055
  \l 13056
  \l 13057
  \l 13058
  \l 13059
  \l 1305A
  \l 1305B
  \l 1305C
  \l 1305D
  \l 1305E
  \l 1305F
  \l 13060
  \l 13061
  \l 13062
  \l 13063
  \l 13064
  \l 13065
  \l 13066
  \l 13067
  \l 13068
  \l 13069
  \l 1306A
  \l 1306B
  \l 1306C
  \l 1306D
  \l 1306E
  \l 1306F
  \l 13070
  \l 13071
  \l 13072
  \l 13073
  \l 13074
  \l 13075
  \l 13076
  \l 13077
  \l 13078
  \l 13079
  \l 1307A
  \l 1307B
  \l 1307C
  \l 1307D
  \l 1307E
  \l 1307F
  \l 13080
  \l 13081
  \l 13082
  \l 13083
  \l 13084
  \l 13085
  \l 13086
  \l 13087
  \l 13088
  \l 13089
  \l 1308A
  \l 1308B
  \l 1308C
  \l 1308D
  \l 1308E
  \l 1308F
  \l 13090
  \l 13091
  \l 13092
  \l 13093
  \l 13094
  \l 13095
  \l 13096
  \l 13097
  \l 13098
  \l 13099
  \l 1309A
  \l 1309B
  \l 1309C
  \l 1309D
  \l 1309E
  \l 1309F
  \l 130A0
  \l 130A1
  \l 130A2
  \l 130A3
  \l 130A4
  \l 130A5
  \l 130A6
  \l 130A7
  \l 130A8
  \l 130A9
  \l 130AA
  \l 130AB
  \l 130AC
  \l 130AD
  \l 130AE
  \l 130AF
  \l 130B0
  \l 130B1
  \l 130B2
  \l 130B3
  \l 130B4
  \l 130B5
  \l 130B6
  \l 130B7
  \l 130B8
  \l 130B9
  \l 130BA
  \l 130BB
  \l 130BC
  \l 130BD
  \l 130BE
  \l 130BF
  \l 130C0
  \l 130C1
  \l 130C2
  \l 130C3
  \l 130C4
  \l 130C5
  \l 130C6
  \l 130C7
  \l 130C8
  \l 130C9
  \l 130CA
  \l 130CB
  \l 130CC
  \l 130CD
  \l 130CE
  \l 130CF
  \l 130D0
  \l 130D1
  \l 130D2
  \l 130D3
  \l 130D4
  \l 130D5
  \l 130D6
  \l 130D7
  \l 130D8
  \l 130D9
  \l 130DA
  \l 130DB
  \l 130DC
  \l 130DD
  \l 130DE
  \l 130DF
  \l 130E0
  \l 130E1
  \l 130E2
  \l 130E3
  \l 130E4
  \l 130E5
  \l 130E6
  \l 130E7
  \l 130E8
  \l 130E9
  \l 130EA
  \l 130EB
  \l 130EC
  \l 130ED
  \l 130EE
  \l 130EF
  \l 130F0
  \l 130F1
  \l 130F2
  \l 130F3
  \l 130F4
  \l 130F5
  \l 130F6
  \l 130F7
  \l 130F8
  \l 130F9
  \l 130FA
  \l 130FB
  \l 130FC
  \l 130FD
  \l 130FE
  \l 130FF
  \l 13100
  \l 13101
  \l 13102
  \l 13103
  \l 13104
  \l 13105
  \l 13106
  \l 13107
  \l 13108
  \l 13109
  \l 1310A
  \l 1310B
  \l 1310C
  \l 1310D
  \l 1310E
  \l 1310F
  \l 13110
  \l 13111
  \l 13112
  \l 13113
  \l 13114
  \l 13115
  \l 13116
  \l 13117
  \l 13118
  \l 13119
  \l 1311A
  \l 1311B
  \l 1311C
  \l 1311D
  \l 1311E
  \l 1311F
  \l 13120
  \l 13121
  \l 13122
  \l 13123
  \l 13124
  \l 13125
  \l 13126
  \l 13127
  \l 13128
  \l 13129
  \l 1312A
  \l 1312B
  \l 1312C
  \l 1312D
  \l 1312E
  \l 1312F
  \l 13130
  \l 13131
  \l 13132
  \l 13133
  \l 13134
  \l 13135
  \l 13136
  \l 13137
  \l 13138
  \l 13139
  \l 1313A
  \l 1313B
  \l 1313C
  \l 1313D
  \l 1313E
  \l 1313F
  \l 13140
  \l 13141
  \l 13142
  \l 13143
  \l 13144
  \l 13145
  \l 13146
  \l 13147
  \l 13148
  \l 13149
  \l 1314A
  \l 1314B
  \l 1314C
  \l 1314D
  \l 1314E
  \l 1314F
  \l 13150
  \l 13151
  \l 13152
  \l 13153
  \l 13154
  \l 13155
  \l 13156
  \l 13157
  \l 13158
  \l 13159
  \l 1315A
  \l 1315B
  \l 1315C
  \l 1315D
  \l 1315E
  \l 1315F
  \l 13160
  \l 13161
  \l 13162
  \l 13163
  \l 13164
  \l 13165
  \l 13166
  \l 13167
  \l 13168
  \l 13169
  \l 1316A
  \l 1316B
  \l 1316C
  \l 1316D
  \l 1316E
  \l 1316F
  \l 13170
  \l 13171
  \l 13172
  \l 13173
  \l 13174
  \l 13175
  \l 13176
  \l 13177
  \l 13178
  \l 13179
  \l 1317A
  \l 1317B
  \l 1317C
  \l 1317D
  \l 1317E
  \l 1317F
  \l 13180
  \l 13181
  \l 13182
  \l 13183
  \l 13184
  \l 13185
  \l 13186
  \l 13187
  \l 13188
  \l 13189
  \l 1318A
  \l 1318B
  \l 1318C
  \l 1318D
  \l 1318E
  \l 1318F
  \l 13190
  \l 13191
  \l 13192
  \l 13193
  \l 13194
  \l 13195
  \l 13196
  \l 13197
  \l 13198
  \l 13199
  \l 1319A
  \l 1319B
  \l 1319C
  \l 1319D
  \l 1319E
  \l 1319F
  \l 131A0
  \l 131A1
  \l 131A2
  \l 131A3
  \l 131A4
  \l 131A5
  \l 131A6
  \l 131A7
  \l 131A8
  \l 131A9
  \l 131AA
  \l 131AB
  \l 131AC
  \l 131AD
  \l 131AE
  \l 131AF
  \l 131B0
  \l 131B1
  \l 131B2
  \l 131B3
  \l 131B4
  \l 131B5
  \l 131B6
  \l 131B7
  \l 131B8
  \l 131B9
  \l 131BA
  \l 131BB
  \l 131BC
  \l 131BD
  \l 131BE
  \l 131BF
  \l 131C0
  \l 131C1
  \l 131C2
  \l 131C3
  \l 131C4
  \l 131C5
  \l 131C6
  \l 131C7
  \l 131C8
  \l 131C9
  \l 131CA
  \l 131CB
  \l 131CC
  \l 131CD
  \l 131CE
  \l 131CF
  \l 131D0
  \l 131D1
  \l 131D2
  \l 131D3
  \l 131D4
  \l 131D5
  \l 131D6
  \l 131D7
  \l 131D8
  \l 131D9
  \l 131DA
  \l 131DB
  \l 131DC
  \l 131DD
  \l 131DE
  \l 131DF
  \l 131E0
  \l 131E1
  \l 131E2
  \l 131E3
  \l 131E4
  \l 131E5
  \l 131E6
  \l 131E7
  \l 131E8
  \l 131E9
  \l 131EA
  \l 131EB
  \l 131EC
  \l 131ED
  \l 131EE
  \l 131EF
  \l 131F0
  \l 131F1
  \l 131F2
  \l 131F3
  \l 131F4
  \l 131F5
  \l 131F6
  \l 131F7
  \l 131F8
  \l 131F9
  \l 131FA
  \l 131FB
  \l 131FC
  \l 131FD
  \l 131FE
  \l 131FF
  \l 13200
  \l 13201
  \l 13202
  \l 13203
  \l 13204
  \l 13205
  \l 13206
  \l 13207
  \l 13208
  \l 13209
  \l 1320A
  \l 1320B
  \l 1320C
  \l 1320D
  \l 1320E
  \l 1320F
  \l 13210
  \l 13211
  \l 13212
  \l 13213
  \l 13214
  \l 13215
  \l 13216
  \l 13217
  \l 13218
  \l 13219
  \l 1321A
  \l 1321B
  \l 1321C
  \l 1321D
  \l 1321E
  \l 1321F
  \l 13220
  \l 13221
  \l 13222
  \l 13223
  \l 13224
  \l 13225
  \l 13226
  \l 13227
  \l 13228
  \l 13229
  \l 1322A
  \l 1322B
  \l 1322C
  \l 1322D
  \l 1322E
  \l 1322F
  \l 13230
  \l 13231
  \l 13232
  \l 13233
  \l 13234
  \l 13235
  \l 13236
  \l 13237
  \l 13238
  \l 13239
  \l 1323A
  \l 1323B
  \l 1323C
  \l 1323D
  \l 1323E
  \l 1323F
  \l 13240
  \l 13241
  \l 13242
  \l 13243
  \l 13244
  \l 13245
  \l 13246
  \l 13247
  \l 13248
  \l 13249
  \l 1324A
  \l 1324B
  \l 1324C
  \l 1324D
  \l 1324E
  \l 1324F
  \l 13250
  \l 13251
  \l 13252
  \l 13253
  \l 13254
  \l 13255
  \l 13256
  \l 13257
  \l 13258
  \l 13259
  \l 1325A
  \l 1325B
  \l 1325C
  \l 1325D
  \l 1325E
  \l 1325F
  \l 13260
  \l 13261
  \l 13262
  \l 13263
  \l 13264
  \l 13265
  \l 13266
  \l 13267
  \l 13268
  \l 13269
  \l 1326A
  \l 1326B
  \l 1326C
  \l 1326D
  \l 1326E
  \l 1326F
  \l 13270
  \l 13271
  \l 13272
  \l 13273
  \l 13274
  \l 13275
  \l 13276
  \l 13277
  \l 13278
  \l 13279
  \l 1327A
  \l 1327B
  \l 1327C
  \l 1327D
  \l 1327E
  \l 1327F
  \l 13280
  \l 13281
  \l 13282
  \l 13283
  \l 13284
  \l 13285
  \l 13286
  \l 13287
  \l 13288
  \l 13289
  \l 1328A
  \l 1328B
  \l 1328C
  \l 1328D
  \l 1328E
  \l 1328F
  \l 13290
  \l 13291
  \l 13292
  \l 13293
  \l 13294
  \l 13295
  \l 13296
  \l 13297
  \l 13298
  \l 13299
  \l 1329A
  \l 1329B
  \l 1329C
  \l 1329D
  \l 1329E
  \l 1329F
  \l 132A0
  \l 132A1
  \l 132A2
  \l 132A3
  \l 132A4
  \l 132A5
  \l 132A6
  \l 132A7
  \l 132A8
  \l 132A9
  \l 132AA
  \l 132AB
  \l 132AC
  \l 132AD
  \l 132AE
  \l 132AF
  \l 132B0
  \l 132B1
  \l 132B2
  \l 132B3
  \l 132B4
  \l 132B5
  \l 132B6
  \l 132B7
  \l 132B8
  \l 132B9
  \l 132BA
  \l 132BB
  \l 132BC
  \l 132BD
  \l 132BE
  \l 132BF
  \l 132C0
  \l 132C1
  \l 132C2
  \l 132C3
  \l 132C4
  \l 132C5
  \l 132C6
  \l 132C7
  \l 132C8
  \l 132C9
  \l 132CA
  \l 132CB
  \l 132CC
  \l 132CD
  \l 132CE
  \l 132CF
  \l 132D0
  \l 132D1
  \l 132D2
  \l 132D3
  \l 132D4
  \l 132D5
  \l 132D6
  \l 132D7
  \l 132D8
  \l 132D9
  \l 132DA
  \l 132DB
  \l 132DC
  \l 132DD
  \l 132DE
  \l 132DF
  \l 132E0
  \l 132E1
  \l 132E2
  \l 132E3
  \l 132E4
  \l 132E5
  \l 132E6
  \l 132E7
  \l 132E8
  \l 132E9
  \l 132EA
  \l 132EB
  \l 132EC
  \l 132ED
  \l 132EE
  \l 132EF
  \l 132F0
  \l 132F1
  \l 132F2
  \l 132F3
  \l 132F4
  \l 132F5
  \l 132F6
  \l 132F7
  \l 132F8
  \l 132F9
  \l 132FA
  \l 132FB
  \l 132FC
  \l 132FD
  \l 132FE
  \l 132FF
  \l 13300
  \l 13301
  \l 13302
  \l 13303
  \l 13304
  \l 13305
  \l 13306
  \l 13307
  \l 13308
  \l 13309
  \l 1330A
  \l 1330B
  \l 1330C
  \l 1330D
  \l 1330E
  \l 1330F
  \l 13310
  \l 13311
  \l 13312
  \l 13313
  \l 13314
  \l 13315
  \l 13316
  \l 13317
  \l 13318
  \l 13319
  \l 1331A
  \l 1331B
  \l 1331C
  \l 1331D
  \l 1331E
  \l 1331F
  \l 13320
  \l 13321
  \l 13322
  \l 13323
  \l 13324
  \l 13325
  \l 13326
  \l 13327
  \l 13328
  \l 13329
  \l 1332A
  \l 1332B
  \l 1332C
  \l 1332D
  \l 1332E
  \l 1332F
  \l 13330
  \l 13331
  \l 13332
  \l 13333
  \l 13334
  \l 13335
  \l 13336
  \l 13337
  \l 13338
  \l 13339
  \l 1333A
  \l 1333B
  \l 1333C
  \l 1333D
  \l 1333E
  \l 1333F
  \l 13340
  \l 13341
  \l 13342
  \l 13343
  \l 13344
  \l 13345
  \l 13346
  \l 13347
  \l 13348
  \l 13349
  \l 1334A
  \l 1334B
  \l 1334C
  \l 1334D
  \l 1334E
  \l 1334F
  \l 13350
  \l 13351
  \l 13352
  \l 13353
  \l 13354
  \l 13355
  \l 13356
  \l 13357
  \l 13358
  \l 13359
  \l 1335A
  \l 1335B
  \l 1335C
  \l 1335D
  \l 1335E
  \l 1335F
  \l 13360
  \l 13361
  \l 13362
  \l 13363
  \l 13364
  \l 13365
  \l 13366
  \l 13367
  \l 13368
  \l 13369
  \l 1336A
  \l 1336B
  \l 1336C
  \l 1336D
  \l 1336E
  \l 1336F
  \l 13370
  \l 13371
  \l 13372
  \l 13373
  \l 13374
  \l 13375
  \l 13376
  \l 13377
  \l 13378
  \l 13379
  \l 1337A
  \l 1337B
  \l 1337C
  \l 1337D
  \l 1337E
  \l 1337F
  \l 13380
  \l 13381
  \l 13382
  \l 13383
  \l 13384
  \l 13385
  \l 13386
  \l 13387
  \l 13388
  \l 13389
  \l 1338A
  \l 1338B
  \l 1338C
  \l 1338D
  \l 1338E
  \l 1338F
  \l 13390
  \l 13391
  \l 13392
  \l 13393
  \l 13394
  \l 13395
  \l 13396
  \l 13397
  \l 13398
  \l 13399
  \l 1339A
  \l 1339B
  \l 1339C
  \l 1339D
  \l 1339E
  \l 1339F
  \l 133A0
  \l 133A1
  \l 133A2
  \l 133A3
  \l 133A4
  \l 133A5
  \l 133A6
  \l 133A7
  \l 133A8
  \l 133A9
  \l 133AA
  \l 133AB
  \l 133AC
  \l 133AD
  \l 133AE
  \l 133AF
  \l 133B0
  \l 133B1
  \l 133B2
  \l 133B3
  \l 133B4
  \l 133B5
  \l 133B6
  \l 133B7
  \l 133B8
  \l 133B9
  \l 133BA
  \l 133BB
  \l 133BC
  \l 133BD
  \l 133BE
  \l 133BF
  \l 133C0
  \l 133C1
  \l 133C2
  \l 133C3
  \l 133C4
  \l 133C5
  \l 133C6
  \l 133C7
  \l 133C8
  \l 133C9
  \l 133CA
  \l 133CB
  \l 133CC
  \l 133CD
  \l 133CE
  \l 133CF
  \l 133D0
  \l 133D1
  \l 133D2
  \l 133D3
  \l 133D4
  \l 133D5
  \l 133D6
  \l 133D7
  \l 133D8
  \l 133D9
  \l 133DA
  \l 133DB
  \l 133DC
  \l 133DD
  \l 133DE
  \l 133DF
  \l 133E0
  \l 133E1
  \l 133E2
  \l 133E3
  \l 133E4
  \l 133E5
  \l 133E6
  \l 133E7
  \l 133E8
  \l 133E9
  \l 133EA
  \l 133EB
  \l 133EC
  \l 133ED
  \l 133EE
  \l 133EF
  \l 133F0
  \l 133F1
  \l 133F2
  \l 133F3
  \l 133F4
  \l 133F5
  \l 133F6
  \l 133F7
  \l 133F8
  \l 133F9
  \l 133FA
  \l 133FB
  \l 133FC
  \l 133FD
  \l 133FE
  \l 133FF
  \l 13400
  \l 13401
  \l 13402
  \l 13403
  \l 13404
  \l 13405
  \l 13406
  \l 13407
  \l 13408
  \l 13409
  \l 1340A
  \l 1340B
  \l 1340C
  \l 1340D
  \l 1340E
  \l 1340F
  \l 13410
  \l 13411
  \l 13412
  \l 13413
  \l 13414
  \l 13415
  \l 13416
  \l 13417
  \l 13418
  \l 13419
  \l 1341A
  \l 1341B
  \l 1341C
  \l 1341D
  \l 1341E
  \l 1341F
  \l 13420
  \l 13421
  \l 13422
  \l 13423
  \l 13424
  \l 13425
  \l 13426
  \l 13427
  \l 13428
  \l 13429
  \l 1342A
  \l 1342B
  \l 1342C
  \l 1342D
  \l 1342E
  \l 16800
  \l 16801
  \l 16802
  \l 16803
  \l 16804
  \l 16805
  \l 16806
  \l 16807
  \l 16808
  \l 16809
  \l 1680A
  \l 1680B
  \l 1680C
  \l 1680D
  \l 1680E
  \l 1680F
  \l 16810
  \l 16811
  \l 16812
  \l 16813
  \l 16814
  \l 16815
  \l 16816
  \l 16817
  \l 16818
  \l 16819
  \l 1681A
  \l 1681B
  \l 1681C
  \l 1681D
  \l 1681E
  \l 1681F
  \l 16820
  \l 16821
  \l 16822
  \l 16823
  \l 16824
  \l 16825
  \l 16826
  \l 16827
  \l 16828
  \l 16829
  \l 1682A
  \l 1682B
  \l 1682C
  \l 1682D
  \l 1682E
  \l 1682F
  \l 16830
  \l 16831
  \l 16832
  \l 16833
  \l 16834
  \l 16835
  \l 16836
  \l 16837
  \l 16838
  \l 16839
  \l 1683A
  \l 1683B
  \l 1683C
  \l 1683D
  \l 1683E
  \l 1683F
  \l 16840
  \l 16841
  \l 16842
  \l 16843
  \l 16844
  \l 16845
  \l 16846
  \l 16847
  \l 16848
  \l 16849
  \l 1684A
  \l 1684B
  \l 1684C
  \l 1684D
  \l 1684E
  \l 1684F
  \l 16850
  \l 16851
  \l 16852
  \l 16853
  \l 16854
  \l 16855
  \l 16856
  \l 16857
  \l 16858
  \l 16859
  \l 1685A
  \l 1685B
  \l 1685C
  \l 1685D
  \l 1685E
  \l 1685F
  \l 16860
  \l 16861
  \l 16862
  \l 16863
  \l 16864
  \l 16865
  \l 16866
  \l 16867
  \l 16868
  \l 16869
  \l 1686A
  \l 1686B
  \l 1686C
  \l 1686D
  \l 1686E
  \l 1686F
  \l 16870
  \l 16871
  \l 16872
  \l 16873
  \l 16874
  \l 16875
  \l 16876
  \l 16877
  \l 16878
  \l 16879
  \l 1687A
  \l 1687B
  \l 1687C
  \l 1687D
  \l 1687E
  \l 1687F
  \l 16880
  \l 16881
  \l 16882
  \l 16883
  \l 16884
  \l 16885
  \l 16886
  \l 16887
  \l 16888
  \l 16889
  \l 1688A
  \l 1688B
  \l 1688C
  \l 1688D
  \l 1688E
  \l 1688F
  \l 16890
  \l 16891
  \l 16892
  \l 16893
  \l 16894
  \l 16895
  \l 16896
  \l 16897
  \l 16898
  \l 16899
  \l 1689A
  \l 1689B
  \l 1689C
  \l 1689D
  \l 1689E
  \l 1689F
  \l 168A0
  \l 168A1
  \l 168A2
  \l 168A3
  \l 168A4
  \l 168A5
  \l 168A6
  \l 168A7
  \l 168A8
  \l 168A9
  \l 168AA
  \l 168AB
  \l 168AC
  \l 168AD
  \l 168AE
  \l 168AF
  \l 168B0
  \l 168B1
  \l 168B2
  \l 168B3
  \l 168B4
  \l 168B5
  \l 168B6
  \l 168B7
  \l 168B8
  \l 168B9
  \l 168BA
  \l 168BB
  \l 168BC
  \l 168BD
  \l 168BE
  \l 168BF
  \l 168C0
  \l 168C1
  \l 168C2
  \l 168C3
  \l 168C4
  \l 168C5
  \l 168C6
  \l 168C7
  \l 168C8
  \l 168C9
  \l 168CA
  \l 168CB
  \l 168CC
  \l 168CD
  \l 168CE
  \l 168CF
  \l 168D0
  \l 168D1
  \l 168D2
  \l 168D3
  \l 168D4
  \l 168D5
  \l 168D6
  \l 168D7
  \l 168D8
  \l 168D9
  \l 168DA
  \l 168DB
  \l 168DC
  \l 168DD
  \l 168DE
  \l 168DF
  \l 168E0
  \l 168E1
  \l 168E2
  \l 168E3
  \l 168E4
  \l 168E5
  \l 168E6
  \l 168E7
  \l 168E8
  \l 168E9
  \l 168EA
  \l 168EB
  \l 168EC
  \l 168ED
  \l 168EE
  \l 168EF
  \l 168F0
  \l 168F1
  \l 168F2
  \l 168F3
  \l 168F4
  \l 168F5
  \l 168F6
  \l 168F7
  \l 168F8
  \l 168F9
  \l 168FA
  \l 168FB
  \l 168FC
  \l 168FD
  \l 168FE
  \l 168FF
  \l 16900
  \l 16901
  \l 16902
  \l 16903
  \l 16904
  \l 16905
  \l 16906
  \l 16907
  \l 16908
  \l 16909
  \l 1690A
  \l 1690B
  \l 1690C
  \l 1690D
  \l 1690E
  \l 1690F
  \l 16910
  \l 16911
  \l 16912
  \l 16913
  \l 16914
  \l 16915
  \l 16916
  \l 16917
  \l 16918
  \l 16919
  \l 1691A
  \l 1691B
  \l 1691C
  \l 1691D
  \l 1691E
  \l 1691F
  \l 16920
  \l 16921
  \l 16922
  \l 16923
  \l 16924
  \l 16925
  \l 16926
  \l 16927
  \l 16928
  \l 16929
  \l 1692A
  \l 1692B
  \l 1692C
  \l 1692D
  \l 1692E
  \l 1692F
  \l 16930
  \l 16931
  \l 16932
  \l 16933
  \l 16934
  \l 16935
  \l 16936
  \l 16937
  \l 16938
  \l 16939
  \l 1693A
  \l 1693B
  \l 1693C
  \l 1693D
  \l 1693E
  \l 1693F
  \l 16940
  \l 16941
  \l 16942
  \l 16943
  \l 16944
  \l 16945
  \l 16946
  \l 16947
  \l 16948
  \l 16949
  \l 1694A
  \l 1694B
  \l 1694C
  \l 1694D
  \l 1694E
  \l 1694F
  \l 16950
  \l 16951
  \l 16952
  \l 16953
  \l 16954
  \l 16955
  \l 16956
  \l 16957
  \l 16958
  \l 16959
  \l 1695A
  \l 1695B
  \l 1695C
  \l 1695D
  \l 1695E
  \l 1695F
  \l 16960
  \l 16961
  \l 16962
  \l 16963
  \l 16964
  \l 16965
  \l 16966
  \l 16967
  \l 16968
  \l 16969
  \l 1696A
  \l 1696B
  \l 1696C
  \l 1696D
  \l 1696E
  \l 1696F
  \l 16970
  \l 16971
  \l 16972
  \l 16973
  \l 16974
  \l 16975
  \l 16976
  \l 16977
  \l 16978
  \l 16979
  \l 1697A
  \l 1697B
  \l 1697C
  \l 1697D
  \l 1697E
  \l 1697F
  \l 16980
  \l 16981
  \l 16982
  \l 16983
  \l 16984
  \l 16985
  \l 16986
  \l 16987
  \l 16988
  \l 16989
  \l 1698A
  \l 1698B
  \l 1698C
  \l 1698D
  \l 1698E
  \l 1698F
  \l 16990
  \l 16991
  \l 16992
  \l 16993
  \l 16994
  \l 16995
  \l 16996
  \l 16997
  \l 16998
  \l 16999
  \l 1699A
  \l 1699B
  \l 1699C
  \l 1699D
  \l 1699E
  \l 1699F
  \l 169A0
  \l 169A1
  \l 169A2
  \l 169A3
  \l 169A4
  \l 169A5
  \l 169A6
  \l 169A7
  \l 169A8
  \l 169A9
  \l 169AA
  \l 169AB
  \l 169AC
  \l 169AD
  \l 169AE
  \l 169AF
  \l 169B0
  \l 169B1
  \l 169B2
  \l 169B3
  \l 169B4
  \l 169B5
  \l 169B6
  \l 169B7
  \l 169B8
  \l 169B9
  \l 169BA
  \l 169BB
  \l 169BC
  \l 169BD
  \l 169BE
  \l 169BF
  \l 169C0
  \l 169C1
  \l 169C2
  \l 169C3
  \l 169C4
  \l 169C5
  \l 169C6
  \l 169C7
  \l 169C8
  \l 169C9
  \l 169CA
  \l 169CB
  \l 169CC
  \l 169CD
  \l 169CE
  \l 169CF
  \l 169D0
  \l 169D1
  \l 169D2
  \l 169D3
  \l 169D4
  \l 169D5
  \l 169D6
  \l 169D7
  \l 169D8
  \l 169D9
  \l 169DA
  \l 169DB
  \l 169DC
  \l 169DD
  \l 169DE
  \l 169DF
  \l 169E0
  \l 169E1
  \l 169E2
  \l 169E3
  \l 169E4
  \l 169E5
  \l 169E6
  \l 169E7
  \l 169E8
  \l 169E9
  \l 169EA
  \l 169EB
  \l 169EC
  \l 169ED
  \l 169EE
  \l 169EF
  \l 169F0
  \l 169F1
  \l 169F2
  \l 169F3
  \l 169F4
  \l 169F5
  \l 169F6
  \l 169F7
  \l 169F8
  \l 169F9
  \l 169FA
  \l 169FB
  \l 169FC
  \l 169FD
  \l 169FE
  \l 169FF
  \l 16A00
  \l 16A01
  \l 16A02
  \l 16A03
  \l 16A04
  \l 16A05
  \l 16A06
  \l 16A07
  \l 16A08
  \l 16A09
  \l 16A0A
  \l 16A0B
  \l 16A0C
  \l 16A0D
  \l 16A0E
  \l 16A0F
  \l 16A10
  \l 16A11
  \l 16A12
  \l 16A13
  \l 16A14
  \l 16A15
  \l 16A16
  \l 16A17
  \l 16A18
  \l 16A19
  \l 16A1A
  \l 16A1B
  \l 16A1C
  \l 16A1D
  \l 16A1E
  \l 16A1F
  \l 16A20
  \l 16A21
  \l 16A22
  \l 16A23
  \l 16A24
  \l 16A25
  \l 16A26
  \l 16A27
  \l 16A28
  \l 16A29
  \l 16A2A
  \l 16A2B
  \l 16A2C
  \l 16A2D
  \l 16A2E
  \l 16A2F
  \l 16A30
  \l 16A31
  \l 16A32
  \l 16A33
  \l 16A34
  \l 16A35
  \l 16A36
  \l 16A37
  \l 16A38
  \l 16A40
  \l 16A41
  \l 16A42
  \l 16A43
  \l 16A44
  \l 16A45
  \l 16A46
  \l 16A47
  \l 16A48
  \l 16A49
  \l 16A4A
  \l 16A4B
  \l 16A4C
  \l 16A4D
  \l 16A4E
  \l 16A4F
  \l 16A50
  \l 16A51
  \l 16A52
  \l 16A53
  \l 16A54
  \l 16A55
  \l 16A56
  \l 16A57
  \l 16A58
  \l 16A59
  \l 16A5A
  \l 16A5B
  \l 16A5C
  \l 16A5D
  \l 16A5E
  \l 16AD0
  \l 16AD1
  \l 16AD2
  \l 16AD3
  \l 16AD4
  \l 16AD5
  \l 16AD6
  \l 16AD7
  \l 16AD8
  \l 16AD9
  \l 16ADA
  \l 16ADB
  \l 16ADC
  \l 16ADD
  \l 16ADE
  \l 16ADF
  \l 16AE0
  \l 16AE1
  \l 16AE2
  \l 16AE3
  \l 16AE4
  \l 16AE5
  \l 16AE6
  \l 16AE7
  \l 16AE8
  \l 16AE9
  \l 16AEA
  \l 16AEB
  \l 16AEC
  \l 16AED
  \l 16AF0
  \l 16AF1
  \l 16AF2
  \l 16AF3
  \l 16AF4
  \l 16B00
  \l 16B01
  \l 16B02
  \l 16B03
  \l 16B04
  \l 16B05
  \l 16B06
  \l 16B07
  \l 16B08
  \l 16B09
  \l 16B0A
  \l 16B0B
  \l 16B0C
  \l 16B0D
  \l 16B0E
  \l 16B0F
  \l 16B10
  \l 16B11
  \l 16B12
  \l 16B13
  \l 16B14
  \l 16B15
  \l 16B16
  \l 16B17
  \l 16B18
  \l 16B19
  \l 16B1A
  \l 16B1B
  \l 16B1C
  \l 16B1D
  \l 16B1E
  \l 16B1F
  \l 16B20
  \l 16B21
  \l 16B22
  \l 16B23
  \l 16B24
  \l 16B25
  \l 16B26
  \l 16B27
  \l 16B28
  \l 16B29
  \l 16B2A
  \l 16B2B
  \l 16B2C
  \l 16B2D
  \l 16B2E
  \l 16B2F
  \l 16B30
  \l 16B31
  \l 16B32
  \l 16B33
  \l 16B34
  \l 16B35
  \l 16B36
  \l 16B40
  \l 16B41
  \l 16B42
  \l 16B43
  \l 16B63
  \l 16B64
  \l 16B65
  \l 16B66
  \l 16B67
  \l 16B68
  \l 16B69
  \l 16B6A
  \l 16B6B
  \l 16B6C
  \l 16B6D
  \l 16B6E
  \l 16B6F
  \l 16B70
  \l 16B71
  \l 16B72
  \l 16B73
  \l 16B74
  \l 16B75
  \l 16B76
  \l 16B77
  \l 16B7D
  \l 16B7E
  \l 16B7F
  \l 16B80
  \l 16B81
  \l 16B82
  \l 16B83
  \l 16B84
  \l 16B85
  \l 16B86
  \l 16B87
  \l 16B88
  \l 16B89
  \l 16B8A
  \l 16B8B
  \l 16B8C
  \l 16B8D
  \l 16B8E
  \l 16B8F
  \l 16F00
  \l 16F01
  \l 16F02
  \l 16F03
  \l 16F04
  \l 16F05
  \l 16F06
  \l 16F07
  \l 16F08
  \l 16F09
  \l 16F0A
  \l 16F0B
  \l 16F0C
  \l 16F0D
  \l 16F0E
  \l 16F0F
  \l 16F10
  \l 16F11
  \l 16F12
  \l 16F13
  \l 16F14
  \l 16F15
  \l 16F16
  \l 16F17
  \l 16F18
  \l 16F19
  \l 16F1A
  \l 16F1B
  \l 16F1C
  \l 16F1D
  \l 16F1E
  \l 16F1F
  \l 16F20
  \l 16F21
  \l 16F22
  \l 16F23
  \l 16F24
  \l 16F25
  \l 16F26
  \l 16F27
  \l 16F28
  \l 16F29
  \l 16F2A
  \l 16F2B
  \l 16F2C
  \l 16F2D
  \l 16F2E
  \l 16F2F
  \l 16F30
  \l 16F31
  \l 16F32
  \l 16F33
  \l 16F34
  \l 16F35
  \l 16F36
  \l 16F37
  \l 16F38
  \l 16F39
  \l 16F3A
  \l 16F3B
  \l 16F3C
  \l 16F3D
  \l 16F3E
  \l 16F3F
  \l 16F40
  \l 16F41
  \l 16F42
  \l 16F43
  \l 16F44
  \l 16F50
  \l 16F51
  \l 16F52
  \l 16F53
  \l 16F54
  \l 16F55
  \l 16F56
  \l 16F57
  \l 16F58
  \l 16F59
  \l 16F5A
  \l 16F5B
  \l 16F5C
  \l 16F5D
  \l 16F5E
  \l 16F5F
  \l 16F60
  \l 16F61
  \l 16F62
  \l 16F63
  \l 16F64
  \l 16F65
  \l 16F66
  \l 16F67
  \l 16F68
  \l 16F69
  \l 16F6A
  \l 16F6B
  \l 16F6C
  \l 16F6D
  \l 16F6E
  \l 16F6F
  \l 16F70
  \l 16F71
  \l 16F72
  \l 16F73
  \l 16F74
  \l 16F75
  \l 16F76
  \l 16F77
  \l 16F78
  \l 16F79
  \l 16F7A
  \l 16F7B
  \l 16F7C
  \l 16F7D
  \l 16F7E
  \l 16F8F
  \l 16F90
  \l 16F91
  \l 16F92
  \l 16F93
  \l 16F94
  \l 16F95
  \l 16F96
  \l 16F97
  \l 16F98
  \l 16F99
  \l 16F9A
  \l 16F9B
  \l 16F9C
  \l 16F9D
  \l 16F9E
  \l 16F9F
  \l 1B000
  \l 1B001
  \l 1BC00
  \l 1BC01
  \l 1BC02
  \l 1BC03
  \l 1BC04
  \l 1BC05
  \l 1BC06
  \l 1BC07
  \l 1BC08
  \l 1BC09
  \l 1BC0A
  \l 1BC0B
  \l 1BC0C
  \l 1BC0D
  \l 1BC0E
  \l 1BC0F
  \l 1BC10
  \l 1BC11
  \l 1BC12
  \l 1BC13
  \l 1BC14
  \l 1BC15
  \l 1BC16
  \l 1BC17
  \l 1BC18
  \l 1BC19
  \l 1BC1A
  \l 1BC1B
  \l 1BC1C
  \l 1BC1D
  \l 1BC1E
  \l 1BC1F
  \l 1BC20
  \l 1BC21
  \l 1BC22
  \l 1BC23
  \l 1BC24
  \l 1BC25
  \l 1BC26
  \l 1BC27
  \l 1BC28
  \l 1BC29
  \l 1BC2A
  \l 1BC2B
  \l 1BC2C
  \l 1BC2D
  \l 1BC2E
  \l 1BC2F
  \l 1BC30
  \l 1BC31
  \l 1BC32
  \l 1BC33
  \l 1BC34
  \l 1BC35
  \l 1BC36
  \l 1BC37
  \l 1BC38
  \l 1BC39
  \l 1BC3A
  \l 1BC3B
  \l 1BC3C
  \l 1BC3D
  \l 1BC3E
  \l 1BC3F
  \l 1BC40
  \l 1BC41
  \l 1BC42
  \l 1BC43
  \l 1BC44
  \l 1BC45
  \l 1BC46
  \l 1BC47
  \l 1BC48
  \l 1BC49
  \l 1BC4A
  \l 1BC4B
  \l 1BC4C
  \l 1BC4D
  \l 1BC4E
  \l 1BC4F
  \l 1BC50
  \l 1BC51
  \l 1BC52
  \l 1BC53
  \l 1BC54
  \l 1BC55
  \l 1BC56
  \l 1BC57
  \l 1BC58
  \l 1BC59
  \l 1BC5A
  \l 1BC5B
  \l 1BC5C
  \l 1BC5D
  \l 1BC5E
  \l 1BC5F
  \l 1BC60
  \l 1BC61
  \l 1BC62
  \l 1BC63
  \l 1BC64
  \l 1BC65
  \l 1BC66
  \l 1BC67
  \l 1BC68
  \l 1BC69
  \l 1BC6A
  \l 1BC70
  \l 1BC71
  \l 1BC72
  \l 1BC73
  \l 1BC74
  \l 1BC75
  \l 1BC76
  \l 1BC77
  \l 1BC78
  \l 1BC79
  \l 1BC7A
  \l 1BC7B
  \l 1BC7C
  \l 1BC80
  \l 1BC81
  \l 1BC82
  \l 1BC83
  \l 1BC84
  \l 1BC85
  \l 1BC86
  \l 1BC87
  \l 1BC88
  \l 1BC90
  \l 1BC91
  \l 1BC92
  \l 1BC93
  \l 1BC94
  \l 1BC95
  \l 1BC96
  \l 1BC97
  \l 1BC98
  \l 1BC99
  \l 1BC9D
  \l 1BC9E
  \l 1D165
  \l 1D166
  \l 1D167
  \l 1D168
  \l 1D169
  \l 1D16D
  \l 1D16E
  \l 1D16F
  \l 1D170
  \l 1D171
  \l 1D172
  \l 1D17B
  \l 1D17C
  \l 1D17D
  \l 1D17E
  \l 1D17F
  \l 1D180
  \l 1D181
  \l 1D182
  \l 1D185
  \l 1D186
  \l 1D187
  \l 1D188
  \l 1D189
  \l 1D18A
  \l 1D18B
  \l 1D1AA
  \l 1D1AB
  \l 1D1AC
  \l 1D1AD
  \l 1D242
  \l 1D243
  \l 1D244
  \l 1D400
  \l 1D401
  \l 1D402
  \l 1D403
  \l 1D404
  \l 1D405
  \l 1D406
  \l 1D407
  \l 1D408
  \l 1D409
  \l 1D40A
  \l 1D40B
  \l 1D40C
  \l 1D40D
  \l 1D40E
  \l 1D40F
  \l 1D410
  \l 1D411
  \l 1D412
  \l 1D413
  \l 1D414
  \l 1D415
  \l 1D416
  \l 1D417
  \l 1D418
  \l 1D419
  \l 1D41A
  \l 1D41B
  \l 1D41C
  \l 1D41D
  \l 1D41E
  \l 1D41F
  \l 1D420
  \l 1D421
  \l 1D422
  \l 1D423
  \l 1D424
  \l 1D425
  \l 1D426
  \l 1D427
  \l 1D428
  \l 1D429
  \l 1D42A
  \l 1D42B
  \l 1D42C
  \l 1D42D
  \l 1D42E
  \l 1D42F
  \l 1D430
  \l 1D431
  \l 1D432
  \l 1D433
  \l 1D434
  \l 1D435
  \l 1D436
  \l 1D437
  \l 1D438
  \l 1D439
  \l 1D43A
  \l 1D43B
  \l 1D43C
  \l 1D43D
  \l 1D43E
  \l 1D43F
  \l 1D440
  \l 1D441
  \l 1D442
  \l 1D443
  \l 1D444
  \l 1D445
  \l 1D446
  \l 1D447
  \l 1D448
  \l 1D449
  \l 1D44A
  \l 1D44B
  \l 1D44C
  \l 1D44D
  \l 1D44E
  \l 1D44F
  \l 1D450
  \l 1D451
  \l 1D452
  \l 1D453
  \l 1D454
  \l 1D456
  \l 1D457
  \l 1D458
  \l 1D459
  \l 1D45A
  \l 1D45B
  \l 1D45C
  \l 1D45D
  \l 1D45E
  \l 1D45F
  \l 1D460
  \l 1D461
  \l 1D462
  \l 1D463
  \l 1D464
  \l 1D465
  \l 1D466
  \l 1D467
  \l 1D468
  \l 1D469
  \l 1D46A
  \l 1D46B
  \l 1D46C
  \l 1D46D
  \l 1D46E
  \l 1D46F
  \l 1D470
  \l 1D471
  \l 1D472
  \l 1D473
  \l 1D474
  \l 1D475
  \l 1D476
  \l 1D477
  \l 1D478
  \l 1D479
  \l 1D47A
  \l 1D47B
  \l 1D47C
  \l 1D47D
  \l 1D47E
  \l 1D47F
  \l 1D480
  \l 1D481
  \l 1D482
  \l 1D483
  \l 1D484
  \l 1D485
  \l 1D486
  \l 1D487
  \l 1D488
  \l 1D489
  \l 1D48A
  \l 1D48B
  \l 1D48C
  \l 1D48D
  \l 1D48E
  \l 1D48F
  \l 1D490
  \l 1D491
  \l 1D492
  \l 1D493
  \l 1D494
  \l 1D495
  \l 1D496
  \l 1D497
  \l 1D498
  \l 1D499
  \l 1D49A
  \l 1D49B
  \l 1D49C
  \l 1D49E
  \l 1D49F
  \l 1D4A2
  \l 1D4A5
  \l 1D4A6
  \l 1D4A9
  \l 1D4AA
  \l 1D4AB
  \l 1D4AC
  \l 1D4AE
  \l 1D4AF
  \l 1D4B0
  \l 1D4B1
  \l 1D4B2
  \l 1D4B3
  \l 1D4B4
  \l 1D4B5
  \l 1D4B6
  \l 1D4B7
  \l 1D4B8
  \l 1D4B9
  \l 1D4BB
  \l 1D4BD
  \l 1D4BE
  \l 1D4BF
  \l 1D4C0
  \l 1D4C1
  \l 1D4C2
  \l 1D4C3
  \l 1D4C5
  \l 1D4C6
  \l 1D4C7
  \l 1D4C8
  \l 1D4C9
  \l 1D4CA
  \l 1D4CB
  \l 1D4CC
  \l 1D4CD
  \l 1D4CE
  \l 1D4CF
  \l 1D4D0
  \l 1D4D1
  \l 1D4D2
  \l 1D4D3
  \l 1D4D4
  \l 1D4D5
  \l 1D4D6
  \l 1D4D7
  \l 1D4D8
  \l 1D4D9
  \l 1D4DA
  \l 1D4DB
  \l 1D4DC
  \l 1D4DD
  \l 1D4DE
  \l 1D4DF
  \l 1D4E0
  \l 1D4E1
  \l 1D4E2
  \l 1D4E3
  \l 1D4E4
  \l 1D4E5
  \l 1D4E6
  \l 1D4E7
  \l 1D4E8
  \l 1D4E9
  \l 1D4EA
  \l 1D4EB
  \l 1D4EC
  \l 1D4ED
  \l 1D4EE
  \l 1D4EF
  \l 1D4F0
  \l 1D4F1
  \l 1D4F2
  \l 1D4F3
  \l 1D4F4
  \l 1D4F5
  \l 1D4F6
  \l 1D4F7
  \l 1D4F8
  \l 1D4F9
  \l 1D4FA
  \l 1D4FB
  \l 1D4FC
  \l 1D4FD
  \l 1D4FE
  \l 1D4FF
  \l 1D500
  \l 1D501
  \l 1D502
  \l 1D503
  \l 1D504
  \l 1D505
  \l 1D507
  \l 1D508
  \l 1D509
  \l 1D50A
  \l 1D50D
  \l 1D50E
  \l 1D50F
  \l 1D510
  \l 1D511
  \l 1D512
  \l 1D513
  \l 1D514
  \l 1D516
  \l 1D517
  \l 1D518
  \l 1D519
  \l 1D51A
  \l 1D51B
  \l 1D51C
  \l 1D51E
  \l 1D51F
  \l 1D520
  \l 1D521
  \l 1D522
  \l 1D523
  \l 1D524
  \l 1D525
  \l 1D526
  \l 1D527
  \l 1D528
  \l 1D529
  \l 1D52A
  \l 1D52B
  \l 1D52C
  \l 1D52D
  \l 1D52E
  \l 1D52F
  \l 1D530
  \l 1D531
  \l 1D532
  \l 1D533
  \l 1D534
  \l 1D535
  \l 1D536
  \l 1D537
  \l 1D538
  \l 1D539
  \l 1D53B
  \l 1D53C
  \l 1D53D
  \l 1D53E
  \l 1D540
  \l 1D541
  \l 1D542
  \l 1D543
  \l 1D544
  \l 1D546
  \l 1D54A
  \l 1D54B
  \l 1D54C
  \l 1D54D
  \l 1D54E
  \l 1D54F
  \l 1D550
  \l 1D552
  \l 1D553
  \l 1D554
  \l 1D555
  \l 1D556
  \l 1D557
  \l 1D558
  \l 1D559
  \l 1D55A
  \l 1D55B
  \l 1D55C
  \l 1D55D
  \l 1D55E
  \l 1D55F
  \l 1D560
  \l 1D561
  \l 1D562
  \l 1D563
  \l 1D564
  \l 1D565
  \l 1D566
  \l 1D567
  \l 1D568
  \l 1D569
  \l 1D56A
  \l 1D56B
  \l 1D56C
  \l 1D56D
  \l 1D56E
  \l 1D56F
  \l 1D570
  \l 1D571
  \l 1D572
  \l 1D573
  \l 1D574
  \l 1D575
  \l 1D576
  \l 1D577
  \l 1D578
  \l 1D579
  \l 1D57A
  \l 1D57B
  \l 1D57C
  \l 1D57D
  \l 1D57E
  \l 1D57F
  \l 1D580
  \l 1D581
  \l 1D582
  \l 1D583
  \l 1D584
  \l 1D585
  \l 1D586
  \l 1D587
  \l 1D588
  \l 1D589
  \l 1D58A
  \l 1D58B
  \l 1D58C
  \l 1D58D
  \l 1D58E
  \l 1D58F
  \l 1D590
  \l 1D591
  \l 1D592
  \l 1D593
  \l 1D594
  \l 1D595
  \l 1D596
  \l 1D597
  \l 1D598
  \l 1D599
  \l 1D59A
  \l 1D59B
  \l 1D59C
  \l 1D59D
  \l 1D59E
  \l 1D59F
  \l 1D5A0
  \l 1D5A1
  \l 1D5A2
  \l 1D5A3
  \l 1D5A4
  \l 1D5A5
  \l 1D5A6
  \l 1D5A7
  \l 1D5A8
  \l 1D5A9
  \l 1D5AA
  \l 1D5AB
  \l 1D5AC
  \l 1D5AD
  \l 1D5AE
  \l 1D5AF
  \l 1D5B0
  \l 1D5B1
  \l 1D5B2
  \l 1D5B3
  \l 1D5B4
  \l 1D5B5
  \l 1D5B6
  \l 1D5B7
  \l 1D5B8
  \l 1D5B9
  \l 1D5BA
  \l 1D5BB
  \l 1D5BC
  \l 1D5BD
  \l 1D5BE
  \l 1D5BF
  \l 1D5C0
  \l 1D5C1
  \l 1D5C2
  \l 1D5C3
  \l 1D5C4
  \l 1D5C5
  \l 1D5C6
  \l 1D5C7
  \l 1D5C8
  \l 1D5C9
  \l 1D5CA
  \l 1D5CB
  \l 1D5CC
  \l 1D5CD
  \l 1D5CE
  \l 1D5CF
  \l 1D5D0
  \l 1D5D1
  \l 1D5D2
  \l 1D5D3
  \l 1D5D4
  \l 1D5D5
  \l 1D5D6
  \l 1D5D7
  \l 1D5D8
  \l 1D5D9
  \l 1D5DA
  \l 1D5DB
  \l 1D5DC
  \l 1D5DD
  \l 1D5DE
  \l 1D5DF
  \l 1D5E0
  \l 1D5E1
  \l 1D5E2
  \l 1D5E3
  \l 1D5E4
  \l 1D5E5
  \l 1D5E6
  \l 1D5E7
  \l 1D5E8
  \l 1D5E9
  \l 1D5EA
  \l 1D5EB
  \l 1D5EC
  \l 1D5ED
  \l 1D5EE
  \l 1D5EF
  \l 1D5F0
  \l 1D5F1
  \l 1D5F2
  \l 1D5F3
  \l 1D5F4
  \l 1D5F5
  \l 1D5F6
  \l 1D5F7
  \l 1D5F8
  \l 1D5F9
  \l 1D5FA
  \l 1D5FB
  \l 1D5FC
  \l 1D5FD
  \l 1D5FE
  \l 1D5FF
  \l 1D600
  \l 1D601
  \l 1D602
  \l 1D603
  \l 1D604
  \l 1D605
  \l 1D606
  \l 1D607
  \l 1D608
  \l 1D609
  \l 1D60A
  \l 1D60B
  \l 1D60C
  \l 1D60D
  \l 1D60E
  \l 1D60F
  \l 1D610
  \l 1D611
  \l 1D612
  \l 1D613
  \l 1D614
  \l 1D615
  \l 1D616
  \l 1D617
  \l 1D618
  \l 1D619
  \l 1D61A
  \l 1D61B
  \l 1D61C
  \l 1D61D
  \l 1D61E
  \l 1D61F
  \l 1D620
  \l 1D621
  \l 1D622
  \l 1D623
  \l 1D624
  \l 1D625
  \l 1D626
  \l 1D627
  \l 1D628
  \l 1D629
  \l 1D62A
  \l 1D62B
  \l 1D62C
  \l 1D62D
  \l 1D62E
  \l 1D62F
  \l 1D630
  \l 1D631
  \l 1D632
  \l 1D633
  \l 1D634
  \l 1D635
  \l 1D636
  \l 1D637
  \l 1D638
  \l 1D639
  \l 1D63A
  \l 1D63B
  \l 1D63C
  \l 1D63D
  \l 1D63E
  \l 1D63F
  \l 1D640
  \l 1D641
  \l 1D642
  \l 1D643
  \l 1D644
  \l 1D645
  \l 1D646
  \l 1D647
  \l 1D648
  \l 1D649
  \l 1D64A
  \l 1D64B
  \l 1D64C
  \l 1D64D
  \l 1D64E
  \l 1D64F
  \l 1D650
  \l 1D651
  \l 1D652
  \l 1D653
  \l 1D654
  \l 1D655
  \l 1D656
  \l 1D657
  \l 1D658
  \l 1D659
  \l 1D65A
  \l 1D65B
  \l 1D65C
  \l 1D65D
  \l 1D65E
  \l 1D65F
  \l 1D660
  \l 1D661
  \l 1D662
  \l 1D663
  \l 1D664
  \l 1D665
  \l 1D666
  \l 1D667
  \l 1D668
  \l 1D669
  \l 1D66A
  \l 1D66B
  \l 1D66C
  \l 1D66D
  \l 1D66E
  \l 1D66F
  \l 1D670
  \l 1D671
  \l 1D672
  \l 1D673
  \l 1D674
  \l 1D675
  \l 1D676
  \l 1D677
  \l 1D678
  \l 1D679
  \l 1D67A
  \l 1D67B
  \l 1D67C
  \l 1D67D
  \l 1D67E
  \l 1D67F
  \l 1D680
  \l 1D681
  \l 1D682
  \l 1D683
  \l 1D684
  \l 1D685
  \l 1D686
  \l 1D687
  \l 1D688
  \l 1D689
  \l 1D68A
  \l 1D68B
  \l 1D68C
  \l 1D68D
  \l 1D68E
  \l 1D68F
  \l 1D690
  \l 1D691
  \l 1D692
  \l 1D693
  \l 1D694
  \l 1D695
  \l 1D696
  \l 1D697
  \l 1D698
  \l 1D699
  \l 1D69A
  \l 1D69B
  \l 1D69C
  \l 1D69D
  \l 1D69E
  \l 1D69F
  \l 1D6A0
  \l 1D6A1
  \l 1D6A2
  \l 1D6A3
  \l 1D6A4
  \l 1D6A5
  \l 1D6A8
  \l 1D6A9
  \l 1D6AA
  \l 1D6AB
  \l 1D6AC
  \l 1D6AD
  \l 1D6AE
  \l 1D6AF
  \l 1D6B0
  \l 1D6B1
  \l 1D6B2
  \l 1D6B3
  \l 1D6B4
  \l 1D6B5
  \l 1D6B6
  \l 1D6B7
  \l 1D6B8
  \l 1D6B9
  \l 1D6BA
  \l 1D6BB
  \l 1D6BC
  \l 1D6BD
  \l 1D6BE
  \l 1D6BF
  \l 1D6C0
  \l 1D6C2
  \l 1D6C3
  \l 1D6C4
  \l 1D6C5
  \l 1D6C6
  \l 1D6C7
  \l 1D6C8
  \l 1D6C9
  \l 1D6CA
  \l 1D6CB
  \l 1D6CC
  \l 1D6CD
  \l 1D6CE
  \l 1D6CF
  \l 1D6D0
  \l 1D6D1
  \l 1D6D2
  \l 1D6D3
  \l 1D6D4
  \l 1D6D5
  \l 1D6D6
  \l 1D6D7
  \l 1D6D8
  \l 1D6D9
  \l 1D6DA
  \l 1D6DC
  \l 1D6DD
  \l 1D6DE
  \l 1D6DF
  \l 1D6E0
  \l 1D6E1
  \l 1D6E2
  \l 1D6E3
  \l 1D6E4
  \l 1D6E5
  \l 1D6E6
  \l 1D6E7
  \l 1D6E8
  \l 1D6E9
  \l 1D6EA
  \l 1D6EB
  \l 1D6EC
  \l 1D6ED
  \l 1D6EE
  \l 1D6EF
  \l 1D6F0
  \l 1D6F1
  \l 1D6F2
  \l 1D6F3
  \l 1D6F4
  \l 1D6F5
  \l 1D6F6
  \l 1D6F7
  \l 1D6F8
  \l 1D6F9
  \l 1D6FA
  \l 1D6FC
  \l 1D6FD
  \l 1D6FE
  \l 1D6FF
  \l 1D700
  \l 1D701
  \l 1D702
  \l 1D703
  \l 1D704
  \l 1D705
  \l 1D706
  \l 1D707
  \l 1D708
  \l 1D709
  \l 1D70A
  \l 1D70B
  \l 1D70C
  \l 1D70D
  \l 1D70E
  \l 1D70F
  \l 1D710
  \l 1D711
  \l 1D712
  \l 1D713
  \l 1D714
  \l 1D716
  \l 1D717
  \l 1D718
  \l 1D719
  \l 1D71A
  \l 1D71B
  \l 1D71C
  \l 1D71D
  \l 1D71E
  \l 1D71F
  \l 1D720
  \l 1D721
  \l 1D722
  \l 1D723
  \l 1D724
  \l 1D725
  \l 1D726
  \l 1D727
  \l 1D728
  \l 1D729
  \l 1D72A
  \l 1D72B
  \l 1D72C
  \l 1D72D
  \l 1D72E
  \l 1D72F
  \l 1D730
  \l 1D731
  \l 1D732
  \l 1D733
  \l 1D734
  \l 1D736
  \l 1D737
  \l 1D738
  \l 1D739
  \l 1D73A
  \l 1D73B
  \l 1D73C
  \l 1D73D
  \l 1D73E
  \l 1D73F
  \l 1D740
  \l 1D741
  \l 1D742
  \l 1D743
  \l 1D744
  \l 1D745
  \l 1D746
  \l 1D747
  \l 1D748
  \l 1D749
  \l 1D74A
  \l 1D74B
  \l 1D74C
  \l 1D74D
  \l 1D74E
  \l 1D750
  \l 1D751
  \l 1D752
  \l 1D753
  \l 1D754
  \l 1D755
  \l 1D756
  \l 1D757
  \l 1D758
  \l 1D759
  \l 1D75A
  \l 1D75B
  \l 1D75C
  \l 1D75D
  \l 1D75E
  \l 1D75F
  \l 1D760
  \l 1D761
  \l 1D762
  \l 1D763
  \l 1D764
  \l 1D765
  \l 1D766
  \l 1D767
  \l 1D768
  \l 1D769
  \l 1D76A
  \l 1D76B
  \l 1D76C
  \l 1D76D
  \l 1D76E
  \l 1D770
  \l 1D771
  \l 1D772
  \l 1D773
  \l 1D774
  \l 1D775
  \l 1D776
  \l 1D777
  \l 1D778
  \l 1D779
  \l 1D77A
  \l 1D77B
  \l 1D77C
  \l 1D77D
  \l 1D77E
  \l 1D77F
  \l 1D780
  \l 1D781
  \l 1D782
  \l 1D783
  \l 1D784
  \l 1D785
  \l 1D786
  \l 1D787
  \l 1D788
  \l 1D78A
  \l 1D78B
  \l 1D78C
  \l 1D78D
  \l 1D78E
  \l 1D78F
  \l 1D790
  \l 1D791
  \l 1D792
  \l 1D793
  \l 1D794
  \l 1D795
  \l 1D796
  \l 1D797
  \l 1D798
  \l 1D799
  \l 1D79A
  \l 1D79B
  \l 1D79C
  \l 1D79D
  \l 1D79E
  \l 1D79F
  \l 1D7A0
  \l 1D7A1
  \l 1D7A2
  \l 1D7A3
  \l 1D7A4
  \l 1D7A5
  \l 1D7A6
  \l 1D7A7
  \l 1D7A8
  \l 1D7AA
  \l 1D7AB
  \l 1D7AC
  \l 1D7AD
  \l 1D7AE
  \l 1D7AF
  \l 1D7B0
  \l 1D7B1
  \l 1D7B2
  \l 1D7B3
  \l 1D7B4
  \l 1D7B5
  \l 1D7B6
  \l 1D7B7
  \l 1D7B8
  \l 1D7B9
  \l 1D7BA
  \l 1D7BB
  \l 1D7BC
  \l 1D7BD
  \l 1D7BE
  \l 1D7BF
  \l 1D7C0
  \l 1D7C1
  \l 1D7C2
  \l 1D7C4
  \l 1D7C5
  \l 1D7C6
  \l 1D7C7
  \l 1D7C8
  \l 1D7C9
  \l 1D7CA
  \l 1D7CB
  \l 1E800
  \l 1E801
  \l 1E802
  \l 1E803
  \l 1E804
  \l 1E805
  \l 1E806
  \l 1E807
  \l 1E808
  \l 1E809
  \l 1E80A
  \l 1E80B
  \l 1E80C
  \l 1E80D
  \l 1E80E
  \l 1E80F
  \l 1E810
  \l 1E811
  \l 1E812
  \l 1E813
  \l 1E814
  \l 1E815
  \l 1E816
  \l 1E817
  \l 1E818
  \l 1E819
  \l 1E81A
  \l 1E81B
  \l 1E81C
  \l 1E81D
  \l 1E81E
  \l 1E81F
  \l 1E820
  \l 1E821
  \l 1E822
  \l 1E823
  \l 1E824
  \l 1E825
  \l 1E826
  \l 1E827
  \l 1E828
  \l 1E829
  \l 1E82A
  \l 1E82B
  \l 1E82C
  \l 1E82D
  \l 1E82E
  \l 1E82F
  \l 1E830
  \l 1E831
  \l 1E832
  \l 1E833
  \l 1E834
  \l 1E835
  \l 1E836
  \l 1E837
  \l 1E838
  \l 1E839
  \l 1E83A
  \l 1E83B
  \l 1E83C
  \l 1E83D
  \l 1E83E
  \l 1E83F
  \l 1E840
  \l 1E841
  \l 1E842
  \l 1E843
  \l 1E844
  \l 1E845
  \l 1E846
  \l 1E847
  \l 1E848
  \l 1E849
  \l 1E84A
  \l 1E84B
  \l 1E84C
  \l 1E84D
  \l 1E84E
  \l 1E84F
  \l 1E850
  \l 1E851
  \l 1E852
  \l 1E853
  \l 1E854
  \l 1E855
  \l 1E856
  \l 1E857
  \l 1E858
  \l 1E859
  \l 1E85A
  \l 1E85B
  \l 1E85C
  \l 1E85D
  \l 1E85E
  \l 1E85F
  \l 1E860
  \l 1E861
  \l 1E862
  \l 1E863
  \l 1E864
  \l 1E865
  \l 1E866
  \l 1E867
  \l 1E868
  \l 1E869
  \l 1E86A
  \l 1E86B
  \l 1E86C
  \l 1E86D
  \l 1E86E
  \l 1E86F
  \l 1E870
  \l 1E871
  \l 1E872
  \l 1E873
  \l 1E874
  \l 1E875
  \l 1E876
  \l 1E877
  \l 1E878
  \l 1E879
  \l 1E87A
  \l 1E87B
  \l 1E87C
  \l 1E87D
  \l 1E87E
  \l 1E87F
  \l 1E880
  \l 1E881
  \l 1E882
  \l 1E883
  \l 1E884
  \l 1E885
  \l 1E886
  \l 1E887
  \l 1E888
  \l 1E889
  \l 1E88A
  \l 1E88B
  \l 1E88C
  \l 1E88D
  \l 1E88E
  \l 1E88F
  \l 1E890
  \l 1E891
  \l 1E892
  \l 1E893
  \l 1E894
  \l 1E895
  \l 1E896
  \l 1E897
  \l 1E898
  \l 1E899
  \l 1E89A
  \l 1E89B
  \l 1E89C
  \l 1E89D
  \l 1E89E
  \l 1E89F
  \l 1E8A0
  \l 1E8A1
  \l 1E8A2
  \l 1E8A3
  \l 1E8A4
  \l 1E8A5
  \l 1E8A6
  \l 1E8A7
  \l 1E8A8
  \l 1E8A9
  \l 1E8AA
  \l 1E8AB
  \l 1E8AC
  \l 1E8AD
  \l 1E8AE
  \l 1E8AF
  \l 1E8B0
  \l 1E8B1
  \l 1E8B2
  \l 1E8B3
  \l 1E8B4
  \l 1E8B5
  \l 1E8B6
  \l 1E8B7
  \l 1E8B8
  \l 1E8B9
  \l 1E8BA
  \l 1E8BB
  \l 1E8BC
  \l 1E8BD
  \l 1E8BE
  \l 1E8BF
  \l 1E8C0
  \l 1E8C1
  \l 1E8C2
  \l 1E8C3
  \l 1E8C4
  \l 1E8D0
  \l 1E8D1
  \l 1E8D2
  \l 1E8D3
  \l 1E8D4
  \l 1E8D5
  \l 1E8D6
  \l 1EE00
  \l 1EE01
  \l 1EE02
  \l 1EE03
  \l 1EE05
  \l 1EE06
  \l 1EE07
  \l 1EE08
  \l 1EE09
  \l 1EE0A
  \l 1EE0B
  \l 1EE0C
  \l 1EE0D
  \l 1EE0E
  \l 1EE0F
  \l 1EE10
  \l 1EE11
  \l 1EE12
  \l 1EE13
  \l 1EE14
  \l 1EE15
  \l 1EE16
  \l 1EE17
  \l 1EE18
  \l 1EE19
  \l 1EE1A
  \l 1EE1B
  \l 1EE1C
  \l 1EE1D
  \l 1EE1E
  \l 1EE1F
  \l 1EE21
  \l 1EE22
  \l 1EE24
  \l 1EE27
  \l 1EE29
  \l 1EE2A
  \l 1EE2B
  \l 1EE2C
  \l 1EE2D
  \l 1EE2E
  \l 1EE2F
  \l 1EE30
  \l 1EE31
  \l 1EE32
  \l 1EE34
  \l 1EE35
  \l 1EE36
  \l 1EE37
  \l 1EE39
  \l 1EE3B
  \l 1EE42
  \l 1EE47
  \l 1EE49
  \l 1EE4B
  \l 1EE4D
  \l 1EE4E
  \l 1EE4F
  \l 1EE51
  \l 1EE52
  \l 1EE54
  \l 1EE57
  \l 1EE59
  \l 1EE5B
  \l 1EE5D
  \l 1EE5F
  \l 1EE61
  \l 1EE62
  \l 1EE64
  \l 1EE67
  \l 1EE68
  \l 1EE69
  \l 1EE6A
  \l 1EE6C
  \l 1EE6D
  \l 1EE6E
  \l 1EE6F
  \l 1EE70
  \l 1EE71
  \l 1EE72
  \l 1EE74
  \l 1EE75
  \l 1EE76
  \l 1EE77
  \l 1EE79
  \l 1EE7A
  \l 1EE7B
  \l 1EE7C
  \l 1EE7E
  \l 1EE80
  \l 1EE81
  \l 1EE82
  \l 1EE83
  \l 1EE84
  \l 1EE85
  \l 1EE86
  \l 1EE87
  \l 1EE88
  \l 1EE89
  \l 1EE8B
  \l 1EE8C
  \l 1EE8D
  \l 1EE8E
  \l 1EE8F
  \l 1EE90
  \l 1EE91
  \l 1EE92
  \l 1EE93
  \l 1EE94
  \l 1EE95
  \l 1EE96
  \l 1EE97
  \l 1EE98
  \l 1EE99
  \l 1EE9A
  \l 1EE9B
  \l 1EEA1
  \l 1EEA2
  \l 1EEA3
  \l 1EEA5
  \l 1EEA6
  \l 1EEA7
  \l 1EEA8
  \l 1EEA9
  \l 1EEAB
  \l 1EEAC
  \l 1EEAD
  \l 1EEAE
  \l 1EEAF
  \l 1EEB0
  \l 1EEB1
  \l 1EEB2
  \l 1EEB3
  \l 1EEB4
  \l 1EEB5
  \l 1EEB6
  \l 1EEB7
  \l 1EEB8
  \l 1EEB9
  \l 1EEBA
  \l 1EEBB
  \l 20000
  \l 20001
  \l 20002
  \l 20003
  \l 20004
  \l 20005
  \l 20006
  \l 20007
  \l 20008
  \l 20009
  \l 2000A
  \l 2000B
  \l 2000C
  \l 2000D
  \l 2000E
  \l 2000F
  \l 20010
  \l 20011
  \l 20012
  \l 20013
  \l 20014
  \l 20015
  \l 20016
  \l 20017
  \l 20018
  \l 20019
  \l 2001A
  \l 2001B
  \l 2001C
  \l 2001D
  \l 2001E
  \l 2001F
  \l 20020
  \l 20021
  \l 20022
  \l 20023
  \l 20024
  \l 20025
  \l 20026
  \l 20027
  \l 20028
  \l 20029
  \l 2002A
  \l 2002B
  \l 2002C
  \l 2002D
  \l 2002E
  \l 2002F
  \l 20030
  \l 20031
  \l 20032
  \l 20033
  \l 20034
  \l 20035
  \l 20036
  \l 20037
  \l 20038
  \l 20039
  \l 2003A
  \l 2003B
  \l 2003C
  \l 2003D
  \l 2003E
  \l 2003F
  \l 20040
  \l 20041
  \l 20042
  \l 20043
  \l 20044
  \l 20045
  \l 20046
  \l 20047
  \l 20048
  \l 20049
  \l 2004A
  \l 2004B
  \l 2004C
  \l 2004D
  \l 2004E
  \l 2004F
  \l 20050
  \l 20051
  \l 20052
  \l 20053
  \l 20054
  \l 20055
  \l 20056
  \l 20057
  \l 20058
  \l 20059
  \l 2005A
  \l 2005B
  \l 2005C
  \l 2005D
  \l 2005E
  \l 2005F
  \l 20060
  \l 20061
  \l 20062
  \l 20063
  \l 20064
  \l 20065
  \l 20066
  \l 20067
  \l 20068
  \l 20069
  \l 2006A
  \l 2006B
  \l 2006C
  \l 2006D
  \l 2006E
  \l 2006F
  \l 20070
  \l 20071
  \l 20072
  \l 20073
  \l 20074
  \l 20075
  \l 20076
  \l 20077
  \l 20078
  \l 20079
  \l 2007A
  \l 2007B
  \l 2007C
  \l 2007D
  \l 2007E
  \l 2007F
  \l 20080
  \l 20081
  \l 20082
  \l 20083
  \l 20084
  \l 20085
  \l 20086
  \l 20087
  \l 20088
  \l 20089
  \l 2008A
  \l 2008B
  \l 2008C
  \l 2008D
  \l 2008E
  \l 2008F
  \l 20090
  \l 20091
  \l 20092
  \l 20093
  \l 20094
  \l 20095
  \l 20096
  \l 20097
  \l 20098
  \l 20099
  \l 2009A
  \l 2009B
  \l 2009C
  \l 2009D
  \l 2009E
  \l 2009F
  \l 200A0
  \l 200A1
  \l 200A2
  \l 200A3
  \l 200A4
  \l 200A5
  \l 200A6
  \l 200A7
  \l 200A8
  \l 200A9
  \l 200AA
  \l 200AB
  \l 200AC
  \l 200AD
  \l 200AE
  \l 200AF
  \l 200B0
  \l 200B1
  \l 200B2
  \l 200B3
  \l 200B4
  \l 200B5
  \l 200B6
  \l 200B7
  \l 200B8
  \l 200B9
  \l 200BA
  \l 200BB
  \l 200BC
  \l 200BD
  \l 200BE
  \l 200BF
  \l 200C0
  \l 200C1
  \l 200C2
  \l 200C3
  \l 200C4
  \l 200C5
  \l 200C6
  \l 200C7
  \l 200C8
  \l 200C9
  \l 200CA
  \l 200CB
  \l 200CC
  \l 200CD
  \l 200CE
  \l 200CF
  \l 200D0
  \l 200D1
  \l 200D2
  \l 200D3
  \l 200D4
  \l 200D5
  \l 200D6
  \l 200D7
  \l 200D8
  \l 200D9
  \l 200DA
  \l 200DB
  \l 200DC
  \l 200DD
  \l 200DE
  \l 200DF
  \l 200E0
  \l 200E1
  \l 200E2
  \l 200E3
  \l 200E4
  \l 200E5
  \l 200E6
  \l 200E7
  \l 200E8
  \l 200E9
  \l 200EA
  \l 200EB
  \l 200EC
  \l 200ED
  \l 200EE
  \l 200EF
  \l 200F0
  \l 200F1
  \l 200F2
  \l 200F3
  \l 200F4
  \l 200F5
  \l 200F6
  \l 200F7
  \l 200F8
  \l 200F9
  \l 200FA
  \l 200FB
  \l 200FC
  \l 200FD
  \l 200FE
  \l 200FF
  \l 20100
  \l 20101
  \l 20102
  \l 20103
  \l 20104
  \l 20105
  \l 20106
  \l 20107
  \l 20108
  \l 20109
  \l 2010A
  \l 2010B
  \l 2010C
  \l 2010D
  \l 2010E
  \l 2010F
  \l 20110
  \l 20111
  \l 20112
  \l 20113
  \l 20114
  \l 20115
  \l 20116
  \l 20117
  \l 20118
  \l 20119
  \l 2011A
  \l 2011B
  \l 2011C
  \l 2011D
  \l 2011E
  \l 2011F
  \l 20120
  \l 20121
  \l 20122
  \l 20123
  \l 20124
  \l 20125
  \l 20126
  \l 20127
  \l 20128
  \l 20129
  \l 2012A
  \l 2012B
  \l 2012C
  \l 2012D
  \l 2012E
  \l 2012F
  \l 20130
  \l 20131
  \l 20132
  \l 20133
  \l 20134
  \l 20135
  \l 20136
  \l 20137
  \l 20138
  \l 20139
  \l 2013A
  \l 2013B
  \l 2013C
  \l 2013D
  \l 2013E
  \l 2013F
  \l 20140
  \l 20141
  \l 20142
  \l 20143
  \l 20144
  \l 20145
  \l 20146
  \l 20147
  \l 20148
  \l 20149
  \l 2014A
  \l 2014B
  \l 2014C
  \l 2014D
  \l 2014E
  \l 2014F
  \l 20150
  \l 20151
  \l 20152
  \l 20153
  \l 20154
  \l 20155
  \l 20156
  \l 20157
  \l 20158
  \l 20159
  \l 2015A
  \l 2015B
  \l 2015C
  \l 2015D
  \l 2015E
  \l 2015F
  \l 20160
  \l 20161
  \l 20162
  \l 20163
  \l 20164
  \l 20165
  \l 20166
  \l 20167
  \l 20168
  \l 20169
  \l 2016A
  \l 2016B
  \l 2016C
  \l 2016D
  \l 2016E
  \l 2016F
  \l 20170
  \l 20171
  \l 20172
  \l 20173
  \l 20174
  \l 20175
  \l 20176
  \l 20177
  \l 20178
  \l 20179
  \l 2017A
  \l 2017B
  \l 2017C
  \l 2017D
  \l 2017E
  \l 2017F
  \l 20180
  \l 20181
  \l 20182
  \l 20183
  \l 20184
  \l 20185
  \l 20186
  \l 20187
  \l 20188
  \l 20189
  \l 2018A
  \l 2018B
  \l 2018C
  \l 2018D
  \l 2018E
  \l 2018F
  \l 20190
  \l 20191
  \l 20192
  \l 20193
  \l 20194
  \l 20195
  \l 20196
  \l 20197
  \l 20198
  \l 20199
  \l 2019A
  \l 2019B
  \l 2019C
  \l 2019D
  \l 2019E
  \l 2019F
  \l 201A0
  \l 201A1
  \l 201A2
  \l 201A3
  \l 201A4
  \l 201A5
  \l 201A6
  \l 201A7
  \l 201A8
  \l 201A9
  \l 201AA
  \l 201AB
  \l 201AC
  \l 201AD
  \l 201AE
  \l 201AF
  \l 201B0
  \l 201B1
  \l 201B2
  \l 201B3
  \l 201B4
  \l 201B5
  \l 201B6
  \l 201B7
  \l 201B8
  \l 201B9
  \l 201BA
  \l 201BB
  \l 201BC
  \l 201BD
  \l 201BE
  \l 201BF
  \l 201C0
  \l 201C1
  \l 201C2
  \l 201C3
  \l 201C4
  \l 201C5
  \l 201C6
  \l 201C7
  \l 201C8
  \l 201C9
  \l 201CA
  \l 201CB
  \l 201CC
  \l 201CD
  \l 201CE
  \l 201CF
  \l 201D0
  \l 201D1
  \l 201D2
  \l 201D3
  \l 201D4
  \l 201D5
  \l 201D6
  \l 201D7
  \l 201D8
  \l 201D9
  \l 201DA
  \l 201DB
  \l 201DC
  \l 201DD
  \l 201DE
  \l 201DF
  \l 201E0
  \l 201E1
  \l 201E2
  \l 201E3
  \l 201E4
  \l 201E5
  \l 201E6
  \l 201E7
  \l 201E8
  \l 201E9
  \l 201EA
  \l 201EB
  \l 201EC
  \l 201ED
  \l 201EE
  \l 201EF
  \l 201F0
  \l 201F1
  \l 201F2
  \l 201F3
  \l 201F4
  \l 201F5
  \l 201F6
  \l 201F7
  \l 201F8
  \l 201F9
  \l 201FA
  \l 201FB
  \l 201FC
  \l 201FD
  \l 201FE
  \l 201FF
  \l 20200
  \l 20201
  \l 20202
  \l 20203
  \l 20204
  \l 20205
  \l 20206
  \l 20207
  \l 20208
  \l 20209
  \l 2020A
  \l 2020B
  \l 2020C
  \l 2020D
  \l 2020E
  \l 2020F
  \l 20210
  \l 20211
  \l 20212
  \l 20213
  \l 20214
  \l 20215
  \l 20216
  \l 20217
  \l 20218
  \l 20219
  \l 2021A
  \l 2021B
  \l 2021C
  \l 2021D
  \l 2021E
  \l 2021F
  \l 20220
  \l 20221
  \l 20222
  \l 20223
  \l 20224
  \l 20225
  \l 20226
  \l 20227
  \l 20228
  \l 20229
  \l 2022A
  \l 2022B
  \l 2022C
  \l 2022D
  \l 2022E
  \l 2022F
  \l 20230
  \l 20231
  \l 20232
  \l 20233
  \l 20234
  \l 20235
  \l 20236
  \l 20237
  \l 20238
  \l 20239
  \l 2023A
  \l 2023B
  \l 2023C
  \l 2023D
  \l 2023E
  \l 2023F
  \l 20240
  \l 20241
  \l 20242
  \l 20243
  \l 20244
  \l 20245
  \l 20246
  \l 20247
  \l 20248
  \l 20249
  \l 2024A
  \l 2024B
  \l 2024C
  \l 2024D
  \l 2024E
  \l 2024F
  \l 20250
  \l 20251
  \l 20252
  \l 20253
  \l 20254
  \l 20255
  \l 20256
  \l 20257
  \l 20258
  \l 20259
  \l 2025A
  \l 2025B
  \l 2025C
  \l 2025D
  \l 2025E
  \l 2025F
  \l 20260
  \l 20261
  \l 20262
  \l 20263
  \l 20264
  \l 20265
  \l 20266
  \l 20267
  \l 20268
  \l 20269
  \l 2026A
  \l 2026B
  \l 2026C
  \l 2026D
  \l 2026E
  \l 2026F
  \l 20270
  \l 20271
  \l 20272
  \l 20273
  \l 20274
  \l 20275
  \l 20276
  \l 20277
  \l 20278
  \l 20279
  \l 2027A
  \l 2027B
  \l 2027C
  \l 2027D
  \l 2027E
  \l 2027F
  \l 20280
  \l 20281
  \l 20282
  \l 20283
  \l 20284
  \l 20285
  \l 20286
  \l 20287
  \l 20288
  \l 20289
  \l 2028A
  \l 2028B
  \l 2028C
  \l 2028D
  \l 2028E
  \l 2028F
  \l 20290
  \l 20291
  \l 20292
  \l 20293
  \l 20294
  \l 20295
  \l 20296
  \l 20297
  \l 20298
  \l 20299
  \l 2029A
  \l 2029B
  \l 2029C
  \l 2029D
  \l 2029E
  \l 2029F
  \l 202A0
  \l 202A1
  \l 202A2
  \l 202A3
  \l 202A4
  \l 202A5
  \l 202A6
  \l 202A7
  \l 202A8
  \l 202A9
  \l 202AA
  \l 202AB
  \l 202AC
  \l 202AD
  \l 202AE
  \l 202AF
  \l 202B0
  \l 202B1
  \l 202B2
  \l 202B3
  \l 202B4
  \l 202B5
  \l 202B6
  \l 202B7
  \l 202B8
  \l 202B9
  \l 202BA
  \l 202BB
  \l 202BC
  \l 202BD
  \l 202BE
  \l 202BF
  \l 202C0
  \l 202C1
  \l 202C2
  \l 202C3
  \l 202C4
  \l 202C5
  \l 202C6
  \l 202C7
  \l 202C8
  \l 202C9
  \l 202CA
  \l 202CB
  \l 202CC
  \l 202CD
  \l 202CE
  \l 202CF
  \l 202D0
  \l 202D1
  \l 202D2
  \l 202D3
  \l 202D4
  \l 202D5
  \l 202D6
  \l 202D7
  \l 202D8
  \l 202D9
  \l 202DA
  \l 202DB
  \l 202DC
  \l 202DD
  \l 202DE
  \l 202DF
  \l 202E0
  \l 202E1
  \l 202E2
  \l 202E3
  \l 202E4
  \l 202E5
  \l 202E6
  \l 202E7
  \l 202E8
  \l 202E9
  \l 202EA
  \l 202EB
  \l 202EC
  \l 202ED
  \l 202EE
  \l 202EF
  \l 202F0
  \l 202F1
  \l 202F2
  \l 202F3
  \l 202F4
  \l 202F5
  \l 202F6
  \l 202F7
  \l 202F8
  \l 202F9
  \l 202FA
  \l 202FB
  \l 202FC
  \l 202FD
  \l 202FE
  \l 202FF
  \l 20300
  \l 20301
  \l 20302
  \l 20303
  \l 20304
  \l 20305
  \l 20306
  \l 20307
  \l 20308
  \l 20309
  \l 2030A
  \l 2030B
  \l 2030C
  \l 2030D
  \l 2030E
  \l 2030F
  \l 20310
  \l 20311
  \l 20312
  \l 20313
  \l 20314
  \l 20315
  \l 20316
  \l 20317
  \l 20318
  \l 20319
  \l 2031A
  \l 2031B
  \l 2031C
  \l 2031D
  \l 2031E
  \l 2031F
  \l 20320
  \l 20321
  \l 20322
  \l 20323
  \l 20324
  \l 20325
  \l 20326
  \l 20327
  \l 20328
  \l 20329
  \l 2032A
  \l 2032B
  \l 2032C
  \l 2032D
  \l 2032E
  \l 2032F
  \l 20330
  \l 20331
  \l 20332
  \l 20333
  \l 20334
  \l 20335
  \l 20336
  \l 20337
  \l 20338
  \l 20339
  \l 2033A
  \l 2033B
  \l 2033C
  \l 2033D
  \l 2033E
  \l 2033F
  \l 20340
  \l 20341
  \l 20342
  \l 20343
  \l 20344
  \l 20345
  \l 20346
  \l 20347
  \l 20348
  \l 20349
  \l 2034A
  \l 2034B
  \l 2034C
  \l 2034D
  \l 2034E
  \l 2034F
  \l 20350
  \l 20351
  \l 20352
  \l 20353
  \l 20354
  \l 20355
  \l 20356
  \l 20357
  \l 20358
  \l 20359
  \l 2035A
  \l 2035B
  \l 2035C
  \l 2035D
  \l 2035E
  \l 2035F
  \l 20360
  \l 20361
  \l 20362
  \l 20363
  \l 20364
  \l 20365
  \l 20366
  \l 20367
  \l 20368
  \l 20369
  \l 2036A
  \l 2036B
  \l 2036C
  \l 2036D
  \l 2036E
  \l 2036F
  \l 20370
  \l 20371
  \l 20372
  \l 20373
  \l 20374
  \l 20375
  \l 20376
  \l 20377
  \l 20378
  \l 20379
  \l 2037A
  \l 2037B
  \l 2037C
  \l 2037D
  \l 2037E
  \l 2037F
  \l 20380
  \l 20381
  \l 20382
  \l 20383
  \l 20384
  \l 20385
  \l 20386
  \l 20387
  \l 20388
  \l 20389
  \l 2038A
  \l 2038B
  \l 2038C
  \l 2038D
  \l 2038E
  \l 2038F
  \l 20390
  \l 20391
  \l 20392
  \l 20393
  \l 20394
  \l 20395
  \l 20396
  \l 20397
  \l 20398
  \l 20399
  \l 2039A
  \l 2039B
  \l 2039C
  \l 2039D
  \l 2039E
  \l 2039F
  \l 203A0
  \l 203A1
  \l 203A2
  \l 203A3
  \l 203A4
  \l 203A5
  \l 203A6
  \l 203A7
  \l 203A8
  \l 203A9
  \l 203AA
  \l 203AB
  \l 203AC
  \l 203AD
  \l 203AE
  \l 203AF
  \l 203B0
  \l 203B1
  \l 203B2
  \l 203B3
  \l 203B4
  \l 203B5
  \l 203B6
  \l 203B7
  \l 203B8
  \l 203B9
  \l 203BA
  \l 203BB
  \l 203BC
  \l 203BD
  \l 203BE
  \l 203BF
  \l 203C0
  \l 203C1
  \l 203C2
  \l 203C3
  \l 203C4
  \l 203C5
  \l 203C6
  \l 203C7
  \l 203C8
  \l 203C9
  \l 203CA
  \l 203CB
  \l 203CC
  \l 203CD
  \l 203CE
  \l 203CF
  \l 203D0
  \l 203D1
  \l 203D2
  \l 203D3
  \l 203D4
  \l 203D5
  \l 203D6
  \l 203D7
  \l 203D8
  \l 203D9
  \l 203DA
  \l 203DB
  \l 203DC
  \l 203DD
  \l 203DE
  \l 203DF
  \l 203E0
  \l 203E1
  \l 203E2
  \l 203E3
  \l 203E4
  \l 203E5
  \l 203E6
  \l 203E7
  \l 203E8
  \l 203E9
  \l 203EA
  \l 203EB
  \l 203EC
  \l 203ED
  \l 203EE
  \l 203EF
  \l 203F0
  \l 203F1
  \l 203F2
  \l 203F3
  \l 203F4
  \l 203F5
  \l 203F6
  \l 203F7
  \l 203F8
  \l 203F9
  \l 203FA
  \l 203FB
  \l 203FC
  \l 203FD
  \l 203FE
  \l 203FF
  \l 20400
  \l 20401
  \l 20402
  \l 20403
  \l 20404
  \l 20405
  \l 20406
  \l 20407
  \l 20408
  \l 20409
  \l 2040A
  \l 2040B
  \l 2040C
  \l 2040D
  \l 2040E
  \l 2040F
  \l 20410
  \l 20411
  \l 20412
  \l 20413
  \l 20414
  \l 20415
  \l 20416
  \l 20417
  \l 20418
  \l 20419
  \l 2041A
  \l 2041B
  \l 2041C
  \l 2041D
  \l 2041E
  \l 2041F
  \l 20420
  \l 20421
  \l 20422
  \l 20423
  \l 20424
  \l 20425
  \l 20426
  \l 20427
  \l 20428
  \l 20429
  \l 2042A
  \l 2042B
  \l 2042C
  \l 2042D
  \l 2042E
  \l 2042F
  \l 20430
  \l 20431
  \l 20432
  \l 20433
  \l 20434
  \l 20435
  \l 20436
  \l 20437
  \l 20438
  \l 20439
  \l 2043A
  \l 2043B
  \l 2043C
  \l 2043D
  \l 2043E
  \l 2043F
  \l 20440
  \l 20441
  \l 20442
  \l 20443
  \l 20444
  \l 20445
  \l 20446
  \l 20447
  \l 20448
  \l 20449
  \l 2044A
  \l 2044B
  \l 2044C
  \l 2044D
  \l 2044E
  \l 2044F
  \l 20450
  \l 20451
  \l 20452
  \l 20453
  \l 20454
  \l 20455
  \l 20456
  \l 20457
  \l 20458
  \l 20459
  \l 2045A
  \l 2045B
  \l 2045C
  \l 2045D
  \l 2045E
  \l 2045F
  \l 20460
  \l 20461
  \l 20462
  \l 20463
  \l 20464
  \l 20465
  \l 20466
  \l 20467
  \l 20468
  \l 20469
  \l 2046A
  \l 2046B
  \l 2046C
  \l 2046D
  \l 2046E
  \l 2046F
  \l 20470
  \l 20471
  \l 20472
  \l 20473
  \l 20474
  \l 20475
  \l 20476
  \l 20477
  \l 20478
  \l 20479
  \l 2047A
  \l 2047B
  \l 2047C
  \l 2047D
  \l 2047E
  \l 2047F
  \l 20480
  \l 20481
  \l 20482
  \l 20483
  \l 20484
  \l 20485
  \l 20486
  \l 20487
  \l 20488
  \l 20489
  \l 2048A
  \l 2048B
  \l 2048C
  \l 2048D
  \l 2048E
  \l 2048F
  \l 20490
  \l 20491
  \l 20492
  \l 20493
  \l 20494
  \l 20495
  \l 20496
  \l 20497
  \l 20498
  \l 20499
  \l 2049A
  \l 2049B
  \l 2049C
  \l 2049D
  \l 2049E
  \l 2049F
  \l 204A0
  \l 204A1
  \l 204A2
  \l 204A3
  \l 204A4
  \l 204A5
  \l 204A6
  \l 204A7
  \l 204A8
  \l 204A9
  \l 204AA
  \l 204AB
  \l 204AC
  \l 204AD
  \l 204AE
  \l 204AF
  \l 204B0
  \l 204B1
  \l 204B2
  \l 204B3
  \l 204B4
  \l 204B5
  \l 204B6
  \l 204B7
  \l 204B8
  \l 204B9
  \l 204BA
  \l 204BB
  \l 204BC
  \l 204BD
  \l 204BE
  \l 204BF
  \l 204C0
  \l 204C1
  \l 204C2
  \l 204C3
  \l 204C4
  \l 204C5
  \l 204C6
  \l 204C7
  \l 204C8
  \l 204C9
  \l 204CA
  \l 204CB
  \l 204CC
  \l 204CD
  \l 204CE
  \l 204CF
  \l 204D0
  \l 204D1
  \l 204D2
  \l 204D3
  \l 204D4
  \l 204D5
  \l 204D6
  \l 204D7
  \l 204D8
  \l 204D9
  \l 204DA
  \l 204DB
  \l 204DC
  \l 204DD
  \l 204DE
  \l 204DF
  \l 204E0
  \l 204E1
  \l 204E2
  \l 204E3
  \l 204E4
  \l 204E5
  \l 204E6
  \l 204E7
  \l 204E8
  \l 204E9
  \l 204EA
  \l 204EB
  \l 204EC
  \l 204ED
  \l 204EE
  \l 204EF
  \l 204F0
  \l 204F1
  \l 204F2
  \l 204F3
  \l 204F4
  \l 204F5
  \l 204F6
  \l 204F7
  \l 204F8
  \l 204F9
  \l 204FA
  \l 204FB
  \l 204FC
  \l 204FD
  \l 204FE
  \l 204FF
  \l 20500
  \l 20501
  \l 20502
  \l 20503
  \l 20504
  \l 20505
  \l 20506
  \l 20507
  \l 20508
  \l 20509
  \l 2050A
  \l 2050B
  \l 2050C
  \l 2050D
  \l 2050E
  \l 2050F
  \l 20510
  \l 20511
  \l 20512
  \l 20513
  \l 20514
  \l 20515
  \l 20516
  \l 20517
  \l 20518
  \l 20519
  \l 2051A
  \l 2051B
  \l 2051C
  \l 2051D
  \l 2051E
  \l 2051F
  \l 20520
  \l 20521
  \l 20522
  \l 20523
  \l 20524
  \l 20525
  \l 20526
  \l 20527
  \l 20528
  \l 20529
  \l 2052A
  \l 2052B
  \l 2052C
  \l 2052D
  \l 2052E
  \l 2052F
  \l 20530
  \l 20531
  \l 20532
  \l 20533
  \l 20534
  \l 20535
  \l 20536
  \l 20537
  \l 20538
  \l 20539
  \l 2053A
  \l 2053B
  \l 2053C
  \l 2053D
  \l 2053E
  \l 2053F
  \l 20540
  \l 20541
  \l 20542
  \l 20543
  \l 20544
  \l 20545
  \l 20546
  \l 20547
  \l 20548
  \l 20549
  \l 2054A
  \l 2054B
  \l 2054C
  \l 2054D
  \l 2054E
  \l 2054F
  \l 20550
  \l 20551
  \l 20552
  \l 20553
  \l 20554
  \l 20555
  \l 20556
  \l 20557
  \l 20558
  \l 20559
  \l 2055A
  \l 2055B
  \l 2055C
  \l 2055D
  \l 2055E
  \l 2055F
  \l 20560
  \l 20561
  \l 20562
  \l 20563
  \l 20564
  \l 20565
  \l 20566
  \l 20567
  \l 20568
  \l 20569
  \l 2056A
  \l 2056B
  \l 2056C
  \l 2056D
  \l 2056E
  \l 2056F
  \l 20570
  \l 20571
  \l 20572
  \l 20573
  \l 20574
  \l 20575
  \l 20576
  \l 20577
  \l 20578
  \l 20579
  \l 2057A
  \l 2057B
  \l 2057C
  \l 2057D
  \l 2057E
  \l 2057F
  \l 20580
  \l 20581
  \l 20582
  \l 20583
  \l 20584
  \l 20585
  \l 20586
  \l 20587
  \l 20588
  \l 20589
  \l 2058A
  \l 2058B
  \l 2058C
  \l 2058D
  \l 2058E
  \l 2058F
  \l 20590
  \l 20591
  \l 20592
  \l 20593
  \l 20594
  \l 20595
  \l 20596
  \l 20597
  \l 20598
  \l 20599
  \l 2059A
  \l 2059B
  \l 2059C
  \l 2059D
  \l 2059E
  \l 2059F
  \l 205A0
  \l 205A1
  \l 205A2
  \l 205A3
  \l 205A4
  \l 205A5
  \l 205A6
  \l 205A7
  \l 205A8
  \l 205A9
  \l 205AA
  \l 205AB
  \l 205AC
  \l 205AD
  \l 205AE
  \l 205AF
  \l 205B0
  \l 205B1
  \l 205B2
  \l 205B3
  \l 205B4
  \l 205B5
  \l 205B6
  \l 205B7
  \l 205B8
  \l 205B9
  \l 205BA
  \l 205BB
  \l 205BC
  \l 205BD
  \l 205BE
  \l 205BF
  \l 205C0
  \l 205C1
  \l 205C2
  \l 205C3
  \l 205C4
  \l 205C5
  \l 205C6
  \l 205C7
  \l 205C8
  \l 205C9
  \l 205CA
  \l 205CB
  \l 205CC
  \l 205CD
  \l 205CE
  \l 205CF
  \l 205D0
  \l 205D1
  \l 205D2
  \l 205D3
  \l 205D4
  \l 205D5
  \l 205D6
  \l 205D7
  \l 205D8
  \l 205D9
  \l 205DA
  \l 205DB
  \l 205DC
  \l 205DD
  \l 205DE
  \l 205DF
  \l 205E0
  \l 205E1
  \l 205E2
  \l 205E3
  \l 205E4
  \l 205E5
  \l 205E6
  \l 205E7
  \l 205E8
  \l 205E9
  \l 205EA
  \l 205EB
  \l 205EC
  \l 205ED
  \l 205EE
  \l 205EF
  \l 205F0
  \l 205F1
  \l 205F2
  \l 205F3
  \l 205F4
  \l 205F5
  \l 205F6
  \l 205F7
  \l 205F8
  \l 205F9
  \l 205FA
  \l 205FB
  \l 205FC
  \l 205FD
  \l 205FE
  \l 205FF
  \l 20600
  \l 20601
  \l 20602
  \l 20603
  \l 20604
  \l 20605
  \l 20606
  \l 20607
  \l 20608
  \l 20609
  \l 2060A
  \l 2060B
  \l 2060C
  \l 2060D
  \l 2060E
  \l 2060F
  \l 20610
  \l 20611
  \l 20612
  \l 20613
  \l 20614
  \l 20615
  \l 20616
  \l 20617
  \l 20618
  \l 20619
  \l 2061A
  \l 2061B
  \l 2061C
  \l 2061D
  \l 2061E
  \l 2061F
  \l 20620
  \l 20621
  \l 20622
  \l 20623
  \l 20624
  \l 20625
  \l 20626
  \l 20627
  \l 20628
  \l 20629
  \l 2062A
  \l 2062B
  \l 2062C
  \l 2062D
  \l 2062E
  \l 2062F
  \l 20630
  \l 20631
  \l 20632
  \l 20633
  \l 20634
  \l 20635
  \l 20636
  \l 20637
  \l 20638
  \l 20639
  \l 2063A
  \l 2063B
  \l 2063C
  \l 2063D
  \l 2063E
  \l 2063F
  \l 20640
  \l 20641
  \l 20642
  \l 20643
  \l 20644
  \l 20645
  \l 20646
  \l 20647
  \l 20648
  \l 20649
  \l 2064A
  \l 2064B
  \l 2064C
  \l 2064D
  \l 2064E
  \l 2064F
  \l 20650
  \l 20651
  \l 20652
  \l 20653
  \l 20654
  \l 20655
  \l 20656
  \l 20657
  \l 20658
  \l 20659
  \l 2065A
  \l 2065B
  \l 2065C
  \l 2065D
  \l 2065E
  \l 2065F
  \l 20660
  \l 20661
  \l 20662
  \l 20663
  \l 20664
  \l 20665
  \l 20666
  \l 20667
  \l 20668
  \l 20669
  \l 2066A
  \l 2066B
  \l 2066C
  \l 2066D
  \l 2066E
  \l 2066F
  \l 20670
  \l 20671
  \l 20672
  \l 20673
  \l 20674
  \l 20675
  \l 20676
  \l 20677
  \l 20678
  \l 20679
  \l 2067A
  \l 2067B
  \l 2067C
  \l 2067D
  \l 2067E
  \l 2067F
  \l 20680
  \l 20681
  \l 20682
  \l 20683
  \l 20684
  \l 20685
  \l 20686
  \l 20687
  \l 20688
  \l 20689
  \l 2068A
  \l 2068B
  \l 2068C
  \l 2068D
  \l 2068E
  \l 2068F
  \l 20690
  \l 20691
  \l 20692
  \l 20693
  \l 20694
  \l 20695
  \l 20696
  \l 20697
  \l 20698
  \l 20699
  \l 2069A
  \l 2069B
  \l 2069C
  \l 2069D
  \l 2069E
  \l 2069F
  \l 206A0
  \l 206A1
  \l 206A2
  \l 206A3
  \l 206A4
  \l 206A5
  \l 206A6
  \l 206A7
  \l 206A8
  \l 206A9
  \l 206AA
  \l 206AB
  \l 206AC
  \l 206AD
  \l 206AE
  \l 206AF
  \l 206B0
  \l 206B1
  \l 206B2
  \l 206B3
  \l 206B4
  \l 206B5
  \l 206B6
  \l 206B7
  \l 206B8
  \l 206B9
  \l 206BA
  \l 206BB
  \l 206BC
  \l 206BD
  \l 206BE
  \l 206BF
  \l 206C0
  \l 206C1
  \l 206C2
  \l 206C3
  \l 206C4
  \l 206C5
  \l 206C6
  \l 206C7
  \l 206C8
  \l 206C9
  \l 206CA
  \l 206CB
  \l 206CC
  \l 206CD
  \l 206CE
  \l 206CF
  \l 206D0
  \l 206D1
  \l 206D2
  \l 206D3
  \l 206D4
  \l 206D5
  \l 206D6
  \l 206D7
  \l 206D8
  \l 206D9
  \l 206DA
  \l 206DB
  \l 206DC
  \l 206DD
  \l 206DE
  \l 206DF
  \l 206E0
  \l 206E1
  \l 206E2
  \l 206E3
  \l 206E4
  \l 206E5
  \l 206E6
  \l 206E7
  \l 206E8
  \l 206E9
  \l 206EA
  \l 206EB
  \l 206EC
  \l 206ED
  \l 206EE
  \l 206EF
  \l 206F0
  \l 206F1
  \l 206F2
  \l 206F3
  \l 206F4
  \l 206F5
  \l 206F6
  \l 206F7
  \l 206F8
  \l 206F9
  \l 206FA
  \l 206FB
  \l 206FC
  \l 206FD
  \l 206FE
  \l 206FF
  \l 20700
  \l 20701
  \l 20702
  \l 20703
  \l 20704
  \l 20705
  \l 20706
  \l 20707
  \l 20708
  \l 20709
  \l 2070A
  \l 2070B
  \l 2070C
  \l 2070D
  \l 2070E
  \l 2070F
  \l 20710
  \l 20711
  \l 20712
  \l 20713
  \l 20714
  \l 20715
  \l 20716
  \l 20717
  \l 20718
  \l 20719
  \l 2071A
  \l 2071B
  \l 2071C
  \l 2071D
  \l 2071E
  \l 2071F
  \l 20720
  \l 20721
  \l 20722
  \l 20723
  \l 20724
  \l 20725
  \l 20726
  \l 20727
  \l 20728
  \l 20729
  \l 2072A
  \l 2072B
  \l 2072C
  \l 2072D
  \l 2072E
  \l 2072F
  \l 20730
  \l 20731
  \l 20732
  \l 20733
  \l 20734
  \l 20735
  \l 20736
  \l 20737
  \l 20738
  \l 20739
  \l 2073A
  \l 2073B
  \l 2073C
  \l 2073D
  \l 2073E
  \l 2073F
  \l 20740
  \l 20741
  \l 20742
  \l 20743
  \l 20744
  \l 20745
  \l 20746
  \l 20747
  \l 20748
  \l 20749
  \l 2074A
  \l 2074B
  \l 2074C
  \l 2074D
  \l 2074E
  \l 2074F
  \l 20750
  \l 20751
  \l 20752
  \l 20753
  \l 20754
  \l 20755
  \l 20756
  \l 20757
  \l 20758
  \l 20759
  \l 2075A
  \l 2075B
  \l 2075C
  \l 2075D
  \l 2075E
  \l 2075F
  \l 20760
  \l 20761
  \l 20762
  \l 20763
  \l 20764
  \l 20765
  \l 20766
  \l 20767
  \l 20768
  \l 20769
  \l 2076A
  \l 2076B
  \l 2076C
  \l 2076D
  \l 2076E
  \l 2076F
  \l 20770
  \l 20771
  \l 20772
  \l 20773
  \l 20774
  \l 20775
  \l 20776
  \l 20777
  \l 20778
  \l 20779
  \l 2077A
  \l 2077B
  \l 2077C
  \l 2077D
  \l 2077E
  \l 2077F
  \l 20780
  \l 20781
  \l 20782
  \l 20783
  \l 20784
  \l 20785
  \l 20786
  \l 20787
  \l 20788
  \l 20789
  \l 2078A
  \l 2078B
  \l 2078C
  \l 2078D
  \l 2078E
  \l 2078F
  \l 20790
  \l 20791
  \l 20792
  \l 20793
  \l 20794
  \l 20795
  \l 20796
  \l 20797
  \l 20798
  \l 20799
  \l 2079A
  \l 2079B
  \l 2079C
  \l 2079D
  \l 2079E
  \l 2079F
  \l 207A0
  \l 207A1
  \l 207A2
  \l 207A3
  \l 207A4
  \l 207A5
  \l 207A6
  \l 207A7
  \l 207A8
  \l 207A9
  \l 207AA
  \l 207AB
  \l 207AC
  \l 207AD
  \l 207AE
  \l 207AF
  \l 207B0
  \l 207B1
  \l 207B2
  \l 207B3
  \l 207B4
  \l 207B5
  \l 207B6
  \l 207B7
  \l 207B8
  \l 207B9
  \l 207BA
  \l 207BB
  \l 207BC
  \l 207BD
  \l 207BE
  \l 207BF
  \l 207C0
  \l 207C1
  \l 207C2
  \l 207C3
  \l 207C4
  \l 207C5
  \l 207C6
  \l 207C7
  \l 207C8
  \l 207C9
  \l 207CA
  \l 207CB
  \l 207CC
  \l 207CD
  \l 207CE
  \l 207CF
  \l 207D0
  \l 207D1
  \l 207D2
  \l 207D3
  \l 207D4
  \l 207D5
  \l 207D6
  \l 207D7
  \l 207D8
  \l 207D9
  \l 207DA
  \l 207DB
  \l 207DC
  \l 207DD
  \l 207DE
  \l 207DF
  \l 207E0
  \l 207E1
  \l 207E2
  \l 207E3
  \l 207E4
  \l 207E5
  \l 207E6
  \l 207E7
  \l 207E8
  \l 207E9
  \l 207EA
  \l 207EB
  \l 207EC
  \l 207ED
  \l 207EE
  \l 207EF
  \l 207F0
  \l 207F1
  \l 207F2
  \l 207F3
  \l 207F4
  \l 207F5
  \l 207F6
  \l 207F7
  \l 207F8
  \l 207F9
  \l 207FA
  \l 207FB
  \l 207FC
  \l 207FD
  \l 207FE
  \l 207FF
  \l 20800
  \l 20801
  \l 20802
  \l 20803
  \l 20804
  \l 20805
  \l 20806
  \l 20807
  \l 20808
  \l 20809
  \l 2080A
  \l 2080B
  \l 2080C
  \l 2080D
  \l 2080E
  \l 2080F
  \l 20810
  \l 20811
  \l 20812
  \l 20813
  \l 20814
  \l 20815
  \l 20816
  \l 20817
  \l 20818
  \l 20819
  \l 2081A
  \l 2081B
  \l 2081C
  \l 2081D
  \l 2081E
  \l 2081F
  \l 20820
  \l 20821
  \l 20822
  \l 20823
  \l 20824
  \l 20825
  \l 20826
  \l 20827
  \l 20828
  \l 20829
  \l 2082A
  \l 2082B
  \l 2082C
  \l 2082D
  \l 2082E
  \l 2082F
  \l 20830
  \l 20831
  \l 20832
  \l 20833
  \l 20834
  \l 20835
  \l 20836
  \l 20837
  \l 20838
  \l 20839
  \l 2083A
  \l 2083B
  \l 2083C
  \l 2083D
  \l 2083E
  \l 2083F
  \l 20840
  \l 20841
  \l 20842
  \l 20843
  \l 20844
  \l 20845
  \l 20846
  \l 20847
  \l 20848
  \l 20849
  \l 2084A
  \l 2084B
  \l 2084C
  \l 2084D
  \l 2084E
  \l 2084F
  \l 20850
  \l 20851
  \l 20852
  \l 20853
  \l 20854
  \l 20855
  \l 20856
  \l 20857
  \l 20858
  \l 20859
  \l 2085A
  \l 2085B
  \l 2085C
  \l 2085D
  \l 2085E
  \l 2085F
  \l 20860
  \l 20861
  \l 20862
  \l 20863
  \l 20864
  \l 20865
  \l 20866
  \l 20867
  \l 20868
  \l 20869
  \l 2086A
  \l 2086B
  \l 2086C
  \l 2086D
  \l 2086E
  \l 2086F
  \l 20870
  \l 20871
  \l 20872
  \l 20873
  \l 20874
  \l 20875
  \l 20876
  \l 20877
  \l 20878
  \l 20879
  \l 2087A
  \l 2087B
  \l 2087C
  \l 2087D
  \l 2087E
  \l 2087F
  \l 20880
  \l 20881
  \l 20882
  \l 20883
  \l 20884
  \l 20885
  \l 20886
  \l 20887
  \l 20888
  \l 20889
  \l 2088A
  \l 2088B
  \l 2088C
  \l 2088D
  \l 2088E
  \l 2088F
  \l 20890
  \l 20891
  \l 20892
  \l 20893
  \l 20894
  \l 20895
  \l 20896
  \l 20897
  \l 20898
  \l 20899
  \l 2089A
  \l 2089B
  \l 2089C
  \l 2089D
  \l 2089E
  \l 2089F
  \l 208A0
  \l 208A1
  \l 208A2
  \l 208A3
  \l 208A4
  \l 208A5
  \l 208A6
  \l 208A7
  \l 208A8
  \l 208A9
  \l 208AA
  \l 208AB
  \l 208AC
  \l 208AD
  \l 208AE
  \l 208AF
  \l 208B0
  \l 208B1
  \l 208B2
  \l 208B3
  \l 208B4
  \l 208B5
  \l 208B6
  \l 208B7
  \l 208B8
  \l 208B9
  \l 208BA
  \l 208BB
  \l 208BC
  \l 208BD
  \l 208BE
  \l 208BF
  \l 208C0
  \l 208C1
  \l 208C2
  \l 208C3
  \l 208C4
  \l 208C5
  \l 208C6
  \l 208C7
  \l 208C8
  \l 208C9
  \l 208CA
  \l 208CB
  \l 208CC
  \l 208CD
  \l 208CE
  \l 208CF
  \l 208D0
  \l 208D1
  \l 208D2
  \l 208D3
  \l 208D4
  \l 208D5
  \l 208D6
  \l 208D7
  \l 208D8
  \l 208D9
  \l 208DA
  \l 208DB
  \l 208DC
  \l 208DD
  \l 208DE
  \l 208DF
  \l 208E0
  \l 208E1
  \l 208E2
  \l 208E3
  \l 208E4
  \l 208E5
  \l 208E6
  \l 208E7
  \l 208E8
  \l 208E9
  \l 208EA
  \l 208EB
  \l 208EC
  \l 208ED
  \l 208EE
  \l 208EF
  \l 208F0
  \l 208F1
  \l 208F2
  \l 208F3
  \l 208F4
  \l 208F5
  \l 208F6
  \l 208F7
  \l 208F8
  \l 208F9
  \l 208FA
  \l 208FB
  \l 208FC
  \l 208FD
  \l 208FE
  \l 208FF
  \l 20900
  \l 20901
  \l 20902
  \l 20903
  \l 20904
  \l 20905
  \l 20906
  \l 20907
  \l 20908
  \l 20909
  \l 2090A
  \l 2090B
  \l 2090C
  \l 2090D
  \l 2090E
  \l 2090F
  \l 20910
  \l 20911
  \l 20912
  \l 20913
  \l 20914
  \l 20915
  \l 20916
  \l 20917
  \l 20918
  \l 20919
  \l 2091A
  \l 2091B
  \l 2091C
  \l 2091D
  \l 2091E
  \l 2091F
  \l 20920
  \l 20921
  \l 20922
  \l 20923
  \l 20924
  \l 20925
  \l 20926
  \l 20927
  \l 20928
  \l 20929
  \l 2092A
  \l 2092B
  \l 2092C
  \l 2092D
  \l 2092E
  \l 2092F
  \l 20930
  \l 20931
  \l 20932
  \l 20933
  \l 20934
  \l 20935
  \l 20936
  \l 20937
  \l 20938
  \l 20939
  \l 2093A
  \l 2093B
  \l 2093C
  \l 2093D
  \l 2093E
  \l 2093F
  \l 20940
  \l 20941
  \l 20942
  \l 20943
  \l 20944
  \l 20945
  \l 20946
  \l 20947
  \l 20948
  \l 20949
  \l 2094A
  \l 2094B
  \l 2094C
  \l 2094D
  \l 2094E
  \l 2094F
  \l 20950
  \l 20951
  \l 20952
  \l 20953
  \l 20954
  \l 20955
  \l 20956
  \l 20957
  \l 20958
  \l 20959
  \l 2095A
  \l 2095B
  \l 2095C
  \l 2095D
  \l 2095E
  \l 2095F
  \l 20960
  \l 20961
  \l 20962
  \l 20963
  \l 20964
  \l 20965
  \l 20966
  \l 20967
  \l 20968
  \l 20969
  \l 2096A
  \l 2096B
  \l 2096C
  \l 2096D
  \l 2096E
  \l 2096F
  \l 20970
  \l 20971
  \l 20972
  \l 20973
  \l 20974
  \l 20975
  \l 20976
  \l 20977
  \l 20978
  \l 20979
  \l 2097A
  \l 2097B
  \l 2097C
  \l 2097D
  \l 2097E
  \l 2097F
  \l 20980
  \l 20981
  \l 20982
  \l 20983
  \l 20984
  \l 20985
  \l 20986
  \l 20987
  \l 20988
  \l 20989
  \l 2098A
  \l 2098B
  \l 2098C
  \l 2098D
  \l 2098E
  \l 2098F
  \l 20990
  \l 20991
  \l 20992
  \l 20993
  \l 20994
  \l 20995
  \l 20996
  \l 20997
  \l 20998
  \l 20999
  \l 2099A
  \l 2099B
  \l 2099C
  \l 2099D
  \l 2099E
  \l 2099F
  \l 209A0
  \l 209A1
  \l 209A2
  \l 209A3
  \l 209A4
  \l 209A5
  \l 209A6
  \l 209A7
  \l 209A8
  \l 209A9
  \l 209AA
  \l 209AB
  \l 209AC
  \l 209AD
  \l 209AE
  \l 209AF
  \l 209B0
  \l 209B1
  \l 209B2
  \l 209B3
  \l 209B4
  \l 209B5
  \l 209B6
  \l 209B7
  \l 209B8
  \l 209B9
  \l 209BA
  \l 209BB
  \l 209BC
  \l 209BD
  \l 209BE
  \l 209BF
  \l 209C0
  \l 209C1
  \l 209C2
  \l 209C3
  \l 209C4
  \l 209C5
  \l 209C6
  \l 209C7
  \l 209C8
  \l 209C9
  \l 209CA
  \l 209CB
  \l 209CC
  \l 209CD
  \l 209CE
  \l 209CF
  \l 209D0
  \l 209D1
  \l 209D2
  \l 209D3
  \l 209D4
  \l 209D5
  \l 209D6
  \l 209D7
  \l 209D8
  \l 209D9
  \l 209DA
  \l 209DB
  \l 209DC
  \l 209DD
  \l 209DE
  \l 209DF
  \l 209E0
  \l 209E1
  \l 209E2
  \l 209E3
  \l 209E4
  \l 209E5
  \l 209E6
  \l 209E7
  \l 209E8
  \l 209E9
  \l 209EA
  \l 209EB
  \l 209EC
  \l 209ED
  \l 209EE
  \l 209EF
  \l 209F0
  \l 209F1
  \l 209F2
  \l 209F3
  \l 209F4
  \l 209F5
  \l 209F6
  \l 209F7
  \l 209F8
  \l 209F9
  \l 209FA
  \l 209FB
  \l 209FC
  \l 209FD
  \l 209FE
  \l 209FF
  \l 20A00
  \l 20A01
  \l 20A02
  \l 20A03
  \l 20A04
  \l 20A05
  \l 20A06
  \l 20A07
  \l 20A08
  \l 20A09
  \l 20A0A
  \l 20A0B
  \l 20A0C
  \l 20A0D
  \l 20A0E
  \l 20A0F
  \l 20A10
  \l 20A11
  \l 20A12
  \l 20A13
  \l 20A14
  \l 20A15
  \l 20A16
  \l 20A17
  \l 20A18
  \l 20A19
  \l 20A1A
  \l 20A1B
  \l 20A1C
  \l 20A1D
  \l 20A1E
  \l 20A1F
  \l 20A20
  \l 20A21
  \l 20A22
  \l 20A23
  \l 20A24
  \l 20A25
  \l 20A26
  \l 20A27
  \l 20A28
  \l 20A29
  \l 20A2A
  \l 20A2B
  \l 20A2C
  \l 20A2D
  \l 20A2E
  \l 20A2F
  \l 20A30
  \l 20A31
  \l 20A32
  \l 20A33
  \l 20A34
  \l 20A35
  \l 20A36
  \l 20A37
  \l 20A38
  \l 20A39
  \l 20A3A
  \l 20A3B
  \l 20A3C
  \l 20A3D
  \l 20A3E
  \l 20A3F
  \l 20A40
  \l 20A41
  \l 20A42
  \l 20A43
  \l 20A44
  \l 20A45
  \l 20A46
  \l 20A47
  \l 20A48
  \l 20A49
  \l 20A4A
  \l 20A4B
  \l 20A4C
  \l 20A4D
  \l 20A4E
  \l 20A4F
  \l 20A50
  \l 20A51
  \l 20A52
  \l 20A53
  \l 20A54
  \l 20A55
  \l 20A56
  \l 20A57
  \l 20A58
  \l 20A59
  \l 20A5A
  \l 20A5B
  \l 20A5C
  \l 20A5D
  \l 20A5E
  \l 20A5F
  \l 20A60
  \l 20A61
  \l 20A62
  \l 20A63
  \l 20A64
  \l 20A65
  \l 20A66
  \l 20A67
  \l 20A68
  \l 20A69
  \l 20A6A
  \l 20A6B
  \l 20A6C
  \l 20A6D
  \l 20A6E
  \l 20A6F
  \l 20A70
  \l 20A71
  \l 20A72
  \l 20A73
  \l 20A74
  \l 20A75
  \l 20A76
  \l 20A77
  \l 20A78
  \l 20A79
  \l 20A7A
  \l 20A7B
  \l 20A7C
  \l 20A7D
  \l 20A7E
  \l 20A7F
  \l 20A80
  \l 20A81
  \l 20A82
  \l 20A83
  \l 20A84
  \l 20A85
  \l 20A86
  \l 20A87
  \l 20A88
  \l 20A89
  \l 20A8A
  \l 20A8B
  \l 20A8C
  \l 20A8D
  \l 20A8E
  \l 20A8F
  \l 20A90
  \l 20A91
  \l 20A92
  \l 20A93
  \l 20A94
  \l 20A95
  \l 20A96
  \l 20A97
  \l 20A98
  \l 20A99
  \l 20A9A
  \l 20A9B
  \l 20A9C
  \l 20A9D
  \l 20A9E
  \l 20A9F
  \l 20AA0
  \l 20AA1
  \l 20AA2
  \l 20AA3
  \l 20AA4
  \l 20AA5
  \l 20AA6
  \l 20AA7
  \l 20AA8
  \l 20AA9
  \l 20AAA
  \l 20AAB
  \l 20AAC
  \l 20AAD
  \l 20AAE
  \l 20AAF
  \l 20AB0
  \l 20AB1
  \l 20AB2
  \l 20AB3
  \l 20AB4
  \l 20AB5
  \l 20AB6
  \l 20AB7
  \l 20AB8
  \l 20AB9
  \l 20ABA
  \l 20ABB
  \l 20ABC
  \l 20ABD
  \l 20ABE
  \l 20ABF
  \l 20AC0
  \l 20AC1
  \l 20AC2
  \l 20AC3
  \l 20AC4
  \l 20AC5
  \l 20AC6
  \l 20AC7
  \l 20AC8
  \l 20AC9
  \l 20ACA
  \l 20ACB
  \l 20ACC
  \l 20ACD
  \l 20ACE
  \l 20ACF
  \l 20AD0
  \l 20AD1
  \l 20AD2
  \l 20AD3
  \l 20AD4
  \l 20AD5
  \l 20AD6
  \l 20AD7
  \l 20AD8
  \l 20AD9
  \l 20ADA
  \l 20ADB
  \l 20ADC
  \l 20ADD
  \l 20ADE
  \l 20ADF
  \l 20AE0
  \l 20AE1
  \l 20AE2
  \l 20AE3
  \l 20AE4
  \l 20AE5
  \l 20AE6
  \l 20AE7
  \l 20AE8
  \l 20AE9
  \l 20AEA
  \l 20AEB
  \l 20AEC
  \l 20AED
  \l 20AEE
  \l 20AEF
  \l 20AF0
  \l 20AF1
  \l 20AF2
  \l 20AF3
  \l 20AF4
  \l 20AF5
  \l 20AF6
  \l 20AF7
  \l 20AF8
  \l 20AF9
  \l 20AFA
  \l 20AFB
  \l 20AFC
  \l 20AFD
  \l 20AFE
  \l 20AFF
  \l 20B00
  \l 20B01
  \l 20B02
  \l 20B03
  \l 20B04
  \l 20B05
  \l 20B06
  \l 20B07
  \l 20B08
  \l 20B09
  \l 20B0A
  \l 20B0B
  \l 20B0C
  \l 20B0D
  \l 20B0E
  \l 20B0F
  \l 20B10
  \l 20B11
  \l 20B12
  \l 20B13
  \l 20B14
  \l 20B15
  \l 20B16
  \l 20B17
  \l 20B18
  \l 20B19
  \l 20B1A
  \l 20B1B
  \l 20B1C
  \l 20B1D
  \l 20B1E
  \l 20B1F
  \l 20B20
  \l 20B21
  \l 20B22
  \l 20B23
  \l 20B24
  \l 20B25
  \l 20B26
  \l 20B27
  \l 20B28
  \l 20B29
  \l 20B2A
  \l 20B2B
  \l 20B2C
  \l 20B2D
  \l 20B2E
  \l 20B2F
  \l 20B30
  \l 20B31
  \l 20B32
  \l 20B33
  \l 20B34
  \l 20B35
  \l 20B36
  \l 20B37
  \l 20B38
  \l 20B39
  \l 20B3A
  \l 20B3B
  \l 20B3C
  \l 20B3D
  \l 20B3E
  \l 20B3F
  \l 20B40
  \l 20B41
  \l 20B42
  \l 20B43
  \l 20B44
  \l 20B45
  \l 20B46
  \l 20B47
  \l 20B48
  \l 20B49
  \l 20B4A
  \l 20B4B
  \l 20B4C
  \l 20B4D
  \l 20B4E
  \l 20B4F
  \l 20B50
  \l 20B51
  \l 20B52
  \l 20B53
  \l 20B54
  \l 20B55
  \l 20B56
  \l 20B57
  \l 20B58
  \l 20B59
  \l 20B5A
  \l 20B5B
  \l 20B5C
  \l 20B5D
  \l 20B5E
  \l 20B5F
  \l 20B60
  \l 20B61
  \l 20B62
  \l 20B63
  \l 20B64
  \l 20B65
  \l 20B66
  \l 20B67
  \l 20B68
  \l 20B69
  \l 20B6A
  \l 20B6B
  \l 20B6C
  \l 20B6D
  \l 20B6E
  \l 20B6F
  \l 20B70
  \l 20B71
  \l 20B72
  \l 20B73
  \l 20B74
  \l 20B75
  \l 20B76
  \l 20B77
  \l 20B78
  \l 20B79
  \l 20B7A
  \l 20B7B
  \l 20B7C
  \l 20B7D
  \l 20B7E
  \l 20B7F
  \l 20B80
  \l 20B81
  \l 20B82
  \l 20B83
  \l 20B84
  \l 20B85
  \l 20B86
  \l 20B87
  \l 20B88
  \l 20B89
  \l 20B8A
  \l 20B8B
  \l 20B8C
  \l 20B8D
  \l 20B8E
  \l 20B8F
  \l 20B90
  \l 20B91
  \l 20B92
  \l 20B93
  \l 20B94
  \l 20B95
  \l 20B96
  \l 20B97
  \l 20B98
  \l 20B99
  \l 20B9A
  \l 20B9B
  \l 20B9C
  \l 20B9D
  \l 20B9E
  \l 20B9F
  \l 20BA0
  \l 20BA1
  \l 20BA2
  \l 20BA3
  \l 20BA4
  \l 20BA5
  \l 20BA6
  \l 20BA7
  \l 20BA8
  \l 20BA9
  \l 20BAA
  \l 20BAB
  \l 20BAC
  \l 20BAD
  \l 20BAE
  \l 20BAF
  \l 20BB0
  \l 20BB1
  \l 20BB2
  \l 20BB3
  \l 20BB4
  \l 20BB5
  \l 20BB6
  \l 20BB7
  \l 20BB8
  \l 20BB9
  \l 20BBA
  \l 20BBB
  \l 20BBC
  \l 20BBD
  \l 20BBE
  \l 20BBF
  \l 20BC0
  \l 20BC1
  \l 20BC2
  \l 20BC3
  \l 20BC4
  \l 20BC5
  \l 20BC6
  \l 20BC7
  \l 20BC8
  \l 20BC9
  \l 20BCA
  \l 20BCB
  \l 20BCC
  \l 20BCD
  \l 20BCE
  \l 20BCF
  \l 20BD0
  \l 20BD1
  \l 20BD2
  \l 20BD3
  \l 20BD4
  \l 20BD5
  \l 20BD6
  \l 20BD7
  \l 20BD8
  \l 20BD9
  \l 20BDA
  \l 20BDB
  \l 20BDC
  \l 20BDD
  \l 20BDE
  \l 20BDF
  \l 20BE0
  \l 20BE1
  \l 20BE2
  \l 20BE3
  \l 20BE4
  \l 20BE5
  \l 20BE6
  \l 20BE7
  \l 20BE8
  \l 20BE9
  \l 20BEA
  \l 20BEB
  \l 20BEC
  \l 20BED
  \l 20BEE
  \l 20BEF
  \l 20BF0
  \l 20BF1
  \l 20BF2
  \l 20BF3
  \l 20BF4
  \l 20BF5
  \l 20BF6
  \l 20BF7
  \l 20BF8
  \l 20BF9
  \l 20BFA
  \l 20BFB
  \l 20BFC
  \l 20BFD
  \l 20BFE
  \l 20BFF
  \l 20C00
  \l 20C01
  \l 20C02
  \l 20C03
  \l 20C04
  \l 20C05
  \l 20C06
  \l 20C07
  \l 20C08
  \l 20C09
  \l 20C0A
  \l 20C0B
  \l 20C0C
  \l 20C0D
  \l 20C0E
  \l 20C0F
  \l 20C10
  \l 20C11
  \l 20C12
  \l 20C13
  \l 20C14
  \l 20C15
  \l 20C16
  \l 20C17
  \l 20C18
  \l 20C19
  \l 20C1A
  \l 20C1B
  \l 20C1C
  \l 20C1D
  \l 20C1E
  \l 20C1F
  \l 20C20
  \l 20C21
  \l 20C22
  \l 20C23
  \l 20C24
  \l 20C25
  \l 20C26
  \l 20C27
  \l 20C28
  \l 20C29
  \l 20C2A
  \l 20C2B
  \l 20C2C
  \l 20C2D
  \l 20C2E
  \l 20C2F
  \l 20C30
  \l 20C31
  \l 20C32
  \l 20C33
  \l 20C34
  \l 20C35
  \l 20C36
  \l 20C37
  \l 20C38
  \l 20C39
  \l 20C3A
  \l 20C3B
  \l 20C3C
  \l 20C3D
  \l 20C3E
  \l 20C3F
  \l 20C40
  \l 20C41
  \l 20C42
  \l 20C43
  \l 20C44
  \l 20C45
  \l 20C46
  \l 20C47
  \l 20C48
  \l 20C49
  \l 20C4A
  \l 20C4B
  \l 20C4C
  \l 20C4D
  \l 20C4E
  \l 20C4F
  \l 20C50
  \l 20C51
  \l 20C52
  \l 20C53
  \l 20C54
  \l 20C55
  \l 20C56
  \l 20C57
  \l 20C58
  \l 20C59
  \l 20C5A
  \l 20C5B
  \l 20C5C
  \l 20C5D
  \l 20C5E
  \l 20C5F
  \l 20C60
  \l 20C61
  \l 20C62
  \l 20C63
  \l 20C64
  \l 20C65
  \l 20C66
  \l 20C67
  \l 20C68
  \l 20C69
  \l 20C6A
  \l 20C6B
  \l 20C6C
  \l 20C6D
  \l 20C6E
  \l 20C6F
  \l 20C70
  \l 20C71
  \l 20C72
  \l 20C73
  \l 20C74
  \l 20C75
  \l 20C76
  \l 20C77
  \l 20C78
  \l 20C79
  \l 20C7A
  \l 20C7B
  \l 20C7C
  \l 20C7D
  \l 20C7E
  \l 20C7F
  \l 20C80
  \l 20C81
  \l 20C82
  \l 20C83
  \l 20C84
  \l 20C85
  \l 20C86
  \l 20C87
  \l 20C88
  \l 20C89
  \l 20C8A
  \l 20C8B
  \l 20C8C
  \l 20C8D
  \l 20C8E
  \l 20C8F
  \l 20C90
  \l 20C91
  \l 20C92
  \l 20C93
  \l 20C94
  \l 20C95
  \l 20C96
  \l 20C97
  \l 20C98
  \l 20C99
  \l 20C9A
  \l 20C9B
  \l 20C9C
  \l 20C9D
  \l 20C9E
  \l 20C9F
  \l 20CA0
  \l 20CA1
  \l 20CA2
  \l 20CA3
  \l 20CA4
  \l 20CA5
  \l 20CA6
  \l 20CA7
  \l 20CA8
  \l 20CA9
  \l 20CAA
  \l 20CAB
  \l 20CAC
  \l 20CAD
  \l 20CAE
  \l 20CAF
  \l 20CB0
  \l 20CB1
  \l 20CB2
  \l 20CB3
  \l 20CB4
  \l 20CB5
  \l 20CB6
  \l 20CB7
  \l 20CB8
  \l 20CB9
  \l 20CBA
  \l 20CBB
  \l 20CBC
  \l 20CBD
  \l 20CBE
  \l 20CBF
  \l 20CC0
  \l 20CC1
  \l 20CC2
  \l 20CC3
  \l 20CC4
  \l 20CC5
  \l 20CC6
  \l 20CC7
  \l 20CC8
  \l 20CC9
  \l 20CCA
  \l 20CCB
  \l 20CCC
  \l 20CCD
  \l 20CCE
  \l 20CCF
  \l 20CD0
  \l 20CD1
  \l 20CD2
  \l 20CD3
  \l 20CD4
  \l 20CD5
  \l 20CD6
  \l 20CD7
  \l 20CD8
  \l 20CD9
  \l 20CDA
  \l 20CDB
  \l 20CDC
  \l 20CDD
  \l 20CDE
  \l 20CDF
  \l 20CE0
  \l 20CE1
  \l 20CE2
  \l 20CE3
  \l 20CE4
  \l 20CE5
  \l 20CE6
  \l 20CE7
  \l 20CE8
  \l 20CE9
  \l 20CEA
  \l 20CEB
  \l 20CEC
  \l 20CED
  \l 20CEE
  \l 20CEF
  \l 20CF0
  \l 20CF1
  \l 20CF2
  \l 20CF3
  \l 20CF4
  \l 20CF5
  \l 20CF6
  \l 20CF7
  \l 20CF8
  \l 20CF9
  \l 20CFA
  \l 20CFB
  \l 20CFC
  \l 20CFD
  \l 20CFE
  \l 20CFF
  \l 20D00
  \l 20D01
  \l 20D02
  \l 20D03
  \l 20D04
  \l 20D05
  \l 20D06
  \l 20D07
  \l 20D08
  \l 20D09
  \l 20D0A
  \l 20D0B
  \l 20D0C
  \l 20D0D
  \l 20D0E
  \l 20D0F
  \l 20D10
  \l 20D11
  \l 20D12
  \l 20D13
  \l 20D14
  \l 20D15
  \l 20D16
  \l 20D17
  \l 20D18
  \l 20D19
  \l 20D1A
  \l 20D1B
  \l 20D1C
  \l 20D1D
  \l 20D1E
  \l 20D1F
  \l 20D20
  \l 20D21
  \l 20D22
  \l 20D23
  \l 20D24
  \l 20D25
  \l 20D26
  \l 20D27
  \l 20D28
  \l 20D29
  \l 20D2A
  \l 20D2B
  \l 20D2C
  \l 20D2D
  \l 20D2E
  \l 20D2F
  \l 20D30
  \l 20D31
  \l 20D32
  \l 20D33
  \l 20D34
  \l 20D35
  \l 20D36
  \l 20D37
  \l 20D38
  \l 20D39
  \l 20D3A
  \l 20D3B
  \l 20D3C
  \l 20D3D
  \l 20D3E
  \l 20D3F
  \l 20D40
  \l 20D41
  \l 20D42
  \l 20D43
  \l 20D44
  \l 20D45
  \l 20D46
  \l 20D47
  \l 20D48
  \l 20D49
  \l 20D4A
  \l 20D4B
  \l 20D4C
  \l 20D4D
  \l 20D4E
  \l 20D4F
  \l 20D50
  \l 20D51
  \l 20D52
  \l 20D53
  \l 20D54
  \l 20D55
  \l 20D56
  \l 20D57
  \l 20D58
  \l 20D59
  \l 20D5A
  \l 20D5B
  \l 20D5C
  \l 20D5D
  \l 20D5E
  \l 20D5F
  \l 20D60
  \l 20D61
  \l 20D62
  \l 20D63
  \l 20D64
  \l 20D65
  \l 20D66
  \l 20D67
  \l 20D68
  \l 20D69
  \l 20D6A
  \l 20D6B
  \l 20D6C
  \l 20D6D
  \l 20D6E
  \l 20D6F
  \l 20D70
  \l 20D71
  \l 20D72
  \l 20D73
  \l 20D74
  \l 20D75
  \l 20D76
  \l 20D77
  \l 20D78
  \l 20D79
  \l 20D7A
  \l 20D7B
  \l 20D7C
  \l 20D7D
  \l 20D7E
  \l 20D7F
  \l 20D80
  \l 20D81
  \l 20D82
  \l 20D83
  \l 20D84
  \l 20D85
  \l 20D86
  \l 20D87
  \l 20D88
  \l 20D89
  \l 20D8A
  \l 20D8B
  \l 20D8C
  \l 20D8D
  \l 20D8E
  \l 20D8F
  \l 20D90
  \l 20D91
  \l 20D92
  \l 20D93
  \l 20D94
  \l 20D95
  \l 20D96
  \l 20D97
  \l 20D98
  \l 20D99
  \l 20D9A
  \l 20D9B
  \l 20D9C
  \l 20D9D
  \l 20D9E
  \l 20D9F
  \l 20DA0
  \l 20DA1
  \l 20DA2
  \l 20DA3
  \l 20DA4
  \l 20DA5
  \l 20DA6
  \l 20DA7
  \l 20DA8
  \l 20DA9
  \l 20DAA
  \l 20DAB
  \l 20DAC
  \l 20DAD
  \l 20DAE
  \l 20DAF
  \l 20DB0
  \l 20DB1
  \l 20DB2
  \l 20DB3
  \l 20DB4
  \l 20DB5
  \l 20DB6
  \l 20DB7
  \l 20DB8
  \l 20DB9
  \l 20DBA
  \l 20DBB
  \l 20DBC
  \l 20DBD
  \l 20DBE
  \l 20DBF
  \l 20DC0
  \l 20DC1
  \l 20DC2
  \l 20DC3
  \l 20DC4
  \l 20DC5
  \l 20DC6
  \l 20DC7
  \l 20DC8
  \l 20DC9
  \l 20DCA
  \l 20DCB
  \l 20DCC
  \l 20DCD
  \l 20DCE
  \l 20DCF
  \l 20DD0
  \l 20DD1
  \l 20DD2
  \l 20DD3
  \l 20DD4
  \l 20DD5
  \l 20DD6
  \l 20DD7
  \l 20DD8
  \l 20DD9
  \l 20DDA
  \l 20DDB
  \l 20DDC
  \l 20DDD
  \l 20DDE
  \l 20DDF
  \l 20DE0
  \l 20DE1
  \l 20DE2
  \l 20DE3
  \l 20DE4
  \l 20DE5
  \l 20DE6
  \l 20DE7
  \l 20DE8
  \l 20DE9
  \l 20DEA
  \l 20DEB
  \l 20DEC
  \l 20DED
  \l 20DEE
  \l 20DEF
  \l 20DF0
  \l 20DF1
  \l 20DF2
  \l 20DF3
  \l 20DF4
  \l 20DF5
  \l 20DF6
  \l 20DF7
  \l 20DF8
  \l 20DF9
  \l 20DFA
  \l 20DFB
  \l 20DFC
  \l 20DFD
  \l 20DFE
  \l 20DFF
  \l 20E00
  \l 20E01
  \l 20E02
  \l 20E03
  \l 20E04
  \l 20E05
  \l 20E06
  \l 20E07
  \l 20E08
  \l 20E09
  \l 20E0A
  \l 20E0B
  \l 20E0C
  \l 20E0D
  \l 20E0E
  \l 20E0F
  \l 20E10
  \l 20E11
  \l 20E12
  \l 20E13
  \l 20E14
  \l 20E15
  \l 20E16
  \l 20E17
  \l 20E18
  \l 20E19
  \l 20E1A
  \l 20E1B
  \l 20E1C
  \l 20E1D
  \l 20E1E
  \l 20E1F
  \l 20E20
  \l 20E21
  \l 20E22
  \l 20E23
  \l 20E24
  \l 20E25
  \l 20E26
  \l 20E27
  \l 20E28
  \l 20E29
  \l 20E2A
  \l 20E2B
  \l 20E2C
  \l 20E2D
  \l 20E2E
  \l 20E2F
  \l 20E30
  \l 20E31
  \l 20E32
  \l 20E33
  \l 20E34
  \l 20E35
  \l 20E36
  \l 20E37
  \l 20E38
  \l 20E39
  \l 20E3A
  \l 20E3B
  \l 20E3C
  \l 20E3D
  \l 20E3E
  \l 20E3F
  \l 20E40
  \l 20E41
  \l 20E42
  \l 20E43
  \l 20E44
  \l 20E45
  \l 20E46
  \l 20E47
  \l 20E48
  \l 20E49
  \l 20E4A
  \l 20E4B
  \l 20E4C
  \l 20E4D
  \l 20E4E
  \l 20E4F
  \l 20E50
  \l 20E51
  \l 20E52
  \l 20E53
  \l 20E54
  \l 20E55
  \l 20E56
  \l 20E57
  \l 20E58
  \l 20E59
  \l 20E5A
  \l 20E5B
  \l 20E5C
  \l 20E5D
  \l 20E5E
  \l 20E5F
  \l 20E60
  \l 20E61
  \l 20E62
  \l 20E63
  \l 20E64
  \l 20E65
  \l 20E66
  \l 20E67
  \l 20E68
  \l 20E69
  \l 20E6A
  \l 20E6B
  \l 20E6C
  \l 20E6D
  \l 20E6E
  \l 20E6F
  \l 20E70
  \l 20E71
  \l 20E72
  \l 20E73
  \l 20E74
  \l 20E75
  \l 20E76
  \l 20E77
  \l 20E78
  \l 20E79
  \l 20E7A
  \l 20E7B
  \l 20E7C
  \l 20E7D
  \l 20E7E
  \l 20E7F
  \l 20E80
  \l 20E81
  \l 20E82
  \l 20E83
  \l 20E84
  \l 20E85
  \l 20E86
  \l 20E87
  \l 20E88
  \l 20E89
  \l 20E8A
  \l 20E8B
  \l 20E8C
  \l 20E8D
  \l 20E8E
  \l 20E8F
  \l 20E90
  \l 20E91
  \l 20E92
  \l 20E93
  \l 20E94
  \l 20E95
  \l 20E96
  \l 20E97
  \l 20E98
  \l 20E99
  \l 20E9A
  \l 20E9B
  \l 20E9C
  \l 20E9D
  \l 20E9E
  \l 20E9F
  \l 20EA0
  \l 20EA1
  \l 20EA2
  \l 20EA3
  \l 20EA4
  \l 20EA5
  \l 20EA6
  \l 20EA7
  \l 20EA8
  \l 20EA9
  \l 20EAA
  \l 20EAB
  \l 20EAC
  \l 20EAD
  \l 20EAE
  \l 20EAF
  \l 20EB0
  \l 20EB1
  \l 20EB2
  \l 20EB3
  \l 20EB4
  \l 20EB5
  \l 20EB6
  \l 20EB7
  \l 20EB8
  \l 20EB9
  \l 20EBA
  \l 20EBB
  \l 20EBC
  \l 20EBD
  \l 20EBE
  \l 20EBF
  \l 20EC0
  \l 20EC1
  \l 20EC2
  \l 20EC3
  \l 20EC4
  \l 20EC5
  \l 20EC6
  \l 20EC7
  \l 20EC8
  \l 20EC9
  \l 20ECA
  \l 20ECB
  \l 20ECC
  \l 20ECD
  \l 20ECE
  \l 20ECF
  \l 20ED0
  \l 20ED1
  \l 20ED2
  \l 20ED3
  \l 20ED4
  \l 20ED5
  \l 20ED6
  \l 20ED7
  \l 20ED8
  \l 20ED9
  \l 20EDA
  \l 20EDB
  \l 20EDC
  \l 20EDD
  \l 20EDE
  \l 20EDF
  \l 20EE0
  \l 20EE1
  \l 20EE2
  \l 20EE3
  \l 20EE4
  \l 20EE5
  \l 20EE6
  \l 20EE7
  \l 20EE8
  \l 20EE9
  \l 20EEA
  \l 20EEB
  \l 20EEC
  \l 20EED
  \l 20EEE
  \l 20EEF
  \l 20EF0
  \l 20EF1
  \l 20EF2
  \l 20EF3
  \l 20EF4
  \l 20EF5
  \l 20EF6
  \l 20EF7
  \l 20EF8
  \l 20EF9
  \l 20EFA
  \l 20EFB
  \l 20EFC
  \l 20EFD
  \l 20EFE
  \l 20EFF
  \l 20F00
  \l 20F01
  \l 20F02
  \l 20F03
  \l 20F04
  \l 20F05
  \l 20F06
  \l 20F07
  \l 20F08
  \l 20F09
  \l 20F0A
  \l 20F0B
  \l 20F0C
  \l 20F0D
  \l 20F0E
  \l 20F0F
  \l 20F10
  \l 20F11
  \l 20F12
  \l 20F13
  \l 20F14
  \l 20F15
  \l 20F16
  \l 20F17
  \l 20F18
  \l 20F19
  \l 20F1A
  \l 20F1B
  \l 20F1C
  \l 20F1D
  \l 20F1E
  \l 20F1F
  \l 20F20
  \l 20F21
  \l 20F22
  \l 20F23
  \l 20F24
  \l 20F25
  \l 20F26
  \l 20F27
  \l 20F28
  \l 20F29
  \l 20F2A
  \l 20F2B
  \l 20F2C
  \l 20F2D
  \l 20F2E
  \l 20F2F
  \l 20F30
  \l 20F31
  \l 20F32
  \l 20F33
  \l 20F34
  \l 20F35
  \l 20F36
  \l 20F37
  \l 20F38
  \l 20F39
  \l 20F3A
  \l 20F3B
  \l 20F3C
  \l 20F3D
  \l 20F3E
  \l 20F3F
  \l 20F40
  \l 20F41
  \l 20F42
  \l 20F43
  \l 20F44
  \l 20F45
  \l 20F46
  \l 20F47
  \l 20F48
  \l 20F49
  \l 20F4A
  \l 20F4B
  \l 20F4C
  \l 20F4D
  \l 20F4E
  \l 20F4F
  \l 20F50
  \l 20F51
  \l 20F52
  \l 20F53
  \l 20F54
  \l 20F55
  \l 20F56
  \l 20F57
  \l 20F58
  \l 20F59
  \l 20F5A
  \l 20F5B
  \l 20F5C
  \l 20F5D
  \l 20F5E
  \l 20F5F
  \l 20F60
  \l 20F61
  \l 20F62
  \l 20F63
  \l 20F64
  \l 20F65
  \l 20F66
  \l 20F67
  \l 20F68
  \l 20F69
  \l 20F6A
  \l 20F6B
  \l 20F6C
  \l 20F6D
  \l 20F6E
  \l 20F6F
  \l 20F70
  \l 20F71
  \l 20F72
  \l 20F73
  \l 20F74
  \l 20F75
  \l 20F76
  \l 20F77
  \l 20F78
  \l 20F79
  \l 20F7A
  \l 20F7B
  \l 20F7C
  \l 20F7D
  \l 20F7E
  \l 20F7F
  \l 20F80
  \l 20F81
  \l 20F82
  \l 20F83
  \l 20F84
  \l 20F85
  \l 20F86
  \l 20F87
  \l 20F88
  \l 20F89
  \l 20F8A
  \l 20F8B
  \l 20F8C
  \l 20F8D
  \l 20F8E
  \l 20F8F
  \l 20F90
  \l 20F91
  \l 20F92
  \l 20F93
  \l 20F94
  \l 20F95
  \l 20F96
  \l 20F97
  \l 20F98
  \l 20F99
  \l 20F9A
  \l 20F9B
  \l 20F9C
  \l 20F9D
  \l 20F9E
  \l 20F9F
  \l 20FA0
  \l 20FA1
  \l 20FA2
  \l 20FA3
  \l 20FA4
  \l 20FA5
  \l 20FA6
  \l 20FA7
  \l 20FA8
  \l 20FA9
  \l 20FAA
  \l 20FAB
  \l 20FAC
  \l 20FAD
  \l 20FAE
  \l 20FAF
  \l 20FB0
  \l 20FB1
  \l 20FB2
  \l 20FB3
  \l 20FB4
  \l 20FB5
  \l 20FB6
  \l 20FB7
  \l 20FB8
  \l 20FB9
  \l 20FBA
  \l 20FBB
  \l 20FBC
  \l 20FBD
  \l 20FBE
  \l 20FBF
  \l 20FC0
  \l 20FC1
  \l 20FC2
  \l 20FC3
  \l 20FC4
  \l 20FC5
  \l 20FC6
  \l 20FC7
  \l 20FC8
  \l 20FC9
  \l 20FCA
  \l 20FCB
  \l 20FCC
  \l 20FCD
  \l 20FCE
  \l 20FCF
  \l 20FD0
  \l 20FD1
  \l 20FD2
  \l 20FD3
  \l 20FD4
  \l 20FD5
  \l 20FD6
  \l 20FD7
  \l 20FD8
  \l 20FD9
  \l 20FDA
  \l 20FDB
  \l 20FDC
  \l 20FDD
  \l 20FDE
  \l 20FDF
  \l 20FE0
  \l 20FE1
  \l 20FE2
  \l 20FE3
  \l 20FE4
  \l 20FE5
  \l 20FE6
  \l 20FE7
  \l 20FE8
  \l 20FE9
  \l 20FEA
  \l 20FEB
  \l 20FEC
  \l 20FED
  \l 20FEE
  \l 20FEF
  \l 20FF0
  \l 20FF1
  \l 20FF2
  \l 20FF3
  \l 20FF4
  \l 20FF5
  \l 20FF6
  \l 20FF7
  \l 20FF8
  \l 20FF9
  \l 20FFA
  \l 20FFB
  \l 20FFC
  \l 20FFD
  \l 20FFE
  \l 20FFF
  \l 21000
  \l 21001
  \l 21002
  \l 21003
  \l 21004
  \l 21005
  \l 21006
  \l 21007
  \l 21008
  \l 21009
  \l 2100A
  \l 2100B
  \l 2100C
  \l 2100D
  \l 2100E
  \l 2100F
  \l 21010
  \l 21011
  \l 21012
  \l 21013
  \l 21014
  \l 21015
  \l 21016
  \l 21017
  \l 21018
  \l 21019
  \l 2101A
  \l 2101B
  \l 2101C
  \l 2101D
  \l 2101E
  \l 2101F
  \l 21020
  \l 21021
  \l 21022
  \l 21023
  \l 21024
  \l 21025
  \l 21026
  \l 21027
  \l 21028
  \l 21029
  \l 2102A
  \l 2102B
  \l 2102C
  \l 2102D
  \l 2102E
  \l 2102F
  \l 21030
  \l 21031
  \l 21032
  \l 21033
  \l 21034
  \l 21035
  \l 21036
  \l 21037
  \l 21038
  \l 21039
  \l 2103A
  \l 2103B
  \l 2103C
  \l 2103D
  \l 2103E
  \l 2103F
  \l 21040
  \l 21041
  \l 21042
  \l 21043
  \l 21044
  \l 21045
  \l 21046
  \l 21047
  \l 21048
  \l 21049
  \l 2104A
  \l 2104B
  \l 2104C
  \l 2104D
  \l 2104E
  \l 2104F
  \l 21050
  \l 21051
  \l 21052
  \l 21053
  \l 21054
  \l 21055
  \l 21056
  \l 21057
  \l 21058
  \l 21059
  \l 2105A
  \l 2105B
  \l 2105C
  \l 2105D
  \l 2105E
  \l 2105F
  \l 21060
  \l 21061
  \l 21062
  \l 21063
  \l 21064
  \l 21065
  \l 21066
  \l 21067
  \l 21068
  \l 21069
  \l 2106A
  \l 2106B
  \l 2106C
  \l 2106D
  \l 2106E
  \l 2106F
  \l 21070
  \l 21071
  \l 21072
  \l 21073
  \l 21074
  \l 21075
  \l 21076
  \l 21077
  \l 21078
  \l 21079
  \l 2107A
  \l 2107B
  \l 2107C
  \l 2107D
  \l 2107E
  \l 2107F
  \l 21080
  \l 21081
  \l 21082
  \l 21083
  \l 21084
  \l 21085
  \l 21086
  \l 21087
  \l 21088
  \l 21089
  \l 2108A
  \l 2108B
  \l 2108C
  \l 2108D
  \l 2108E
  \l 2108F
  \l 21090
  \l 21091
  \l 21092
  \l 21093
  \l 21094
  \l 21095
  \l 21096
  \l 21097
  \l 21098
  \l 21099
  \l 2109A
  \l 2109B
  \l 2109C
  \l 2109D
  \l 2109E
  \l 2109F
  \l 210A0
  \l 210A1
  \l 210A2
  \l 210A3
  \l 210A4
  \l 210A5
  \l 210A6
  \l 210A7
  \l 210A8
  \l 210A9
  \l 210AA
  \l 210AB
  \l 210AC
  \l 210AD
  \l 210AE
  \l 210AF
  \l 210B0
  \l 210B1
  \l 210B2
  \l 210B3
  \l 210B4
  \l 210B5
  \l 210B6
  \l 210B7
  \l 210B8
  \l 210B9
  \l 210BA
  \l 210BB
  \l 210BC
  \l 210BD
  \l 210BE
  \l 210BF
  \l 210C0
  \l 210C1
  \l 210C2
  \l 210C3
  \l 210C4
  \l 210C5
  \l 210C6
  \l 210C7
  \l 210C8
  \l 210C9
  \l 210CA
  \l 210CB
  \l 210CC
  \l 210CD
  \l 210CE
  \l 210CF
  \l 210D0
  \l 210D1
  \l 210D2
  \l 210D3
  \l 210D4
  \l 210D5
  \l 210D6
  \l 210D7
  \l 210D8
  \l 210D9
  \l 210DA
  \l 210DB
  \l 210DC
  \l 210DD
  \l 210DE
  \l 210DF
  \l 210E0
  \l 210E1
  \l 210E2
  \l 210E3
  \l 210E4
  \l 210E5
  \l 210E6
  \l 210E7
  \l 210E8
  \l 210E9
  \l 210EA
  \l 210EB
  \l 210EC
  \l 210ED
  \l 210EE
  \l 210EF
  \l 210F0
  \l 210F1
  \l 210F2
  \l 210F3
  \l 210F4
  \l 210F5
  \l 210F6
  \l 210F7
  \l 210F8
  \l 210F9
  \l 210FA
  \l 210FB
  \l 210FC
  \l 210FD
  \l 210FE
  \l 210FF
  \l 21100
  \l 21101
  \l 21102
  \l 21103
  \l 21104
  \l 21105
  \l 21106
  \l 21107
  \l 21108
  \l 21109
  \l 2110A
  \l 2110B
  \l 2110C
  \l 2110D
  \l 2110E
  \l 2110F
  \l 21110
  \l 21111
  \l 21112
  \l 21113
  \l 21114
  \l 21115
  \l 21116
  \l 21117
  \l 21118
  \l 21119
  \l 2111A
  \l 2111B
  \l 2111C
  \l 2111D
  \l 2111E
  \l 2111F
  \l 21120
  \l 21121
  \l 21122
  \l 21123
  \l 21124
  \l 21125
  \l 21126
  \l 21127
  \l 21128
  \l 21129
  \l 2112A
  \l 2112B
  \l 2112C
  \l 2112D
  \l 2112E
  \l 2112F
  \l 21130
  \l 21131
  \l 21132
  \l 21133
  \l 21134
  \l 21135
  \l 21136
  \l 21137
  \l 21138
  \l 21139
  \l 2113A
  \l 2113B
  \l 2113C
  \l 2113D
  \l 2113E
  \l 2113F
  \l 21140
  \l 21141
  \l 21142
  \l 21143
  \l 21144
  \l 21145
  \l 21146
  \l 21147
  \l 21148
  \l 21149
  \l 2114A
  \l 2114B
  \l 2114C
  \l 2114D
  \l 2114E
  \l 2114F
  \l 21150
  \l 21151
  \l 21152
  \l 21153
  \l 21154
  \l 21155
  \l 21156
  \l 21157
  \l 21158
  \l 21159
  \l 2115A
  \l 2115B
  \l 2115C
  \l 2115D
  \l 2115E
  \l 2115F
  \l 21160
  \l 21161
  \l 21162
  \l 21163
  \l 21164
  \l 21165
  \l 21166
  \l 21167
  \l 21168
  \l 21169
  \l 2116A
  \l 2116B
  \l 2116C
  \l 2116D
  \l 2116E
  \l 2116F
  \l 21170
  \l 21171
  \l 21172
  \l 21173
  \l 21174
  \l 21175
  \l 21176
  \l 21177
  \l 21178
  \l 21179
  \l 2117A
  \l 2117B
  \l 2117C
  \l 2117D
  \l 2117E
  \l 2117F
  \l 21180
  \l 21181
  \l 21182
  \l 21183
  \l 21184
  \l 21185
  \l 21186
  \l 21187
  \l 21188
  \l 21189
  \l 2118A
  \l 2118B
  \l 2118C
  \l 2118D
  \l 2118E
  \l 2118F
  \l 21190
  \l 21191
  \l 21192
  \l 21193
  \l 21194
  \l 21195
  \l 21196
  \l 21197
  \l 21198
  \l 21199
  \l 2119A
  \l 2119B
  \l 2119C
  \l 2119D
  \l 2119E
  \l 2119F
  \l 211A0
  \l 211A1
  \l 211A2
  \l 211A3
  \l 211A4
  \l 211A5
  \l 211A6
  \l 211A7
  \l 211A8
  \l 211A9
  \l 211AA
  \l 211AB
  \l 211AC
  \l 211AD
  \l 211AE
  \l 211AF
  \l 211B0
  \l 211B1
  \l 211B2
  \l 211B3
  \l 211B4
  \l 211B5
  \l 211B6
  \l 211B7
  \l 211B8
  \l 211B9
  \l 211BA
  \l 211BB
  \l 211BC
  \l 211BD
  \l 211BE
  \l 211BF
  \l 211C0
  \l 211C1
  \l 211C2
  \l 211C3
  \l 211C4
  \l 211C5
  \l 211C6
  \l 211C7
  \l 211C8
  \l 211C9
  \l 211CA
  \l 211CB
  \l 211CC
  \l 211CD
  \l 211CE
  \l 211CF
  \l 211D0
  \l 211D1
  \l 211D2
  \l 211D3
  \l 211D4
  \l 211D5
  \l 211D6
  \l 211D7
  \l 211D8
  \l 211D9
  \l 211DA
  \l 211DB
  \l 211DC
  \l 211DD
  \l 211DE
  \l 211DF
  \l 211E0
  \l 211E1
  \l 211E2
  \l 211E3
  \l 211E4
  \l 211E5
  \l 211E6
  \l 211E7
  \l 211E8
  \l 211E9
  \l 211EA
  \l 211EB
  \l 211EC
  \l 211ED
  \l 211EE
  \l 211EF
  \l 211F0
  \l 211F1
  \l 211F2
  \l 211F3
  \l 211F4
  \l 211F5
  \l 211F6
  \l 211F7
  \l 211F8
  \l 211F9
  \l 211FA
  \l 211FB
  \l 211FC
  \l 211FD
  \l 211FE
  \l 211FF
  \l 21200
  \l 21201
  \l 21202
  \l 21203
  \l 21204
  \l 21205
  \l 21206
  \l 21207
  \l 21208
  \l 21209
  \l 2120A
  \l 2120B
  \l 2120C
  \l 2120D
  \l 2120E
  \l 2120F
  \l 21210
  \l 21211
  \l 21212
  \l 21213
  \l 21214
  \l 21215
  \l 21216
  \l 21217
  \l 21218
  \l 21219
  \l 2121A
  \l 2121B
  \l 2121C
  \l 2121D
  \l 2121E
  \l 2121F
  \l 21220
  \l 21221
  \l 21222
  \l 21223
  \l 21224
  \l 21225
  \l 21226
  \l 21227
  \l 21228
  \l 21229
  \l 2122A
  \l 2122B
  \l 2122C
  \l 2122D
  \l 2122E
  \l 2122F
  \l 21230
  \l 21231
  \l 21232
  \l 21233
  \l 21234
  \l 21235
  \l 21236
  \l 21237
  \l 21238
  \l 21239
  \l 2123A
  \l 2123B
  \l 2123C
  \l 2123D
  \l 2123E
  \l 2123F
  \l 21240
  \l 21241
  \l 21242
  \l 21243
  \l 21244
  \l 21245
  \l 21246
  \l 21247
  \l 21248
  \l 21249
  \l 2124A
  \l 2124B
  \l 2124C
  \l 2124D
  \l 2124E
  \l 2124F
  \l 21250
  \l 21251
  \l 21252
  \l 21253
  \l 21254
  \l 21255
  \l 21256
  \l 21257
  \l 21258
  \l 21259
  \l 2125A
  \l 2125B
  \l 2125C
  \l 2125D
  \l 2125E
  \l 2125F
  \l 21260
  \l 21261
  \l 21262
  \l 21263
  \l 21264
  \l 21265
  \l 21266
  \l 21267
  \l 21268
  \l 21269
  \l 2126A
  \l 2126B
  \l 2126C
  \l 2126D
  \l 2126E
  \l 2126F
  \l 21270
  \l 21271
  \l 21272
  \l 21273
  \l 21274
  \l 21275
  \l 21276
  \l 21277
  \l 21278
  \l 21279
  \l 2127A
  \l 2127B
  \l 2127C
  \l 2127D
  \l 2127E
  \l 2127F
  \l 21280
  \l 21281
  \l 21282
  \l 21283
  \l 21284
  \l 21285
  \l 21286
  \l 21287
  \l 21288
  \l 21289
  \l 2128A
  \l 2128B
  \l 2128C
  \l 2128D
  \l 2128E
  \l 2128F
  \l 21290
  \l 21291
  \l 21292
  \l 21293
  \l 21294
  \l 21295
  \l 21296
  \l 21297
  \l 21298
  \l 21299
  \l 2129A
  \l 2129B
  \l 2129C
  \l 2129D
  \l 2129E
  \l 2129F
  \l 212A0
  \l 212A1
  \l 212A2
  \l 212A3
  \l 212A4
  \l 212A5
  \l 212A6
  \l 212A7
  \l 212A8
  \l 212A9
  \l 212AA
  \l 212AB
  \l 212AC
  \l 212AD
  \l 212AE
  \l 212AF
  \l 212B0
  \l 212B1
  \l 212B2
  \l 212B3
  \l 212B4
  \l 212B5
  \l 212B6
  \l 212B7
  \l 212B8
  \l 212B9
  \l 212BA
  \l 212BB
  \l 212BC
  \l 212BD
  \l 212BE
  \l 212BF
  \l 212C0
  \l 212C1
  \l 212C2
  \l 212C3
  \l 212C4
  \l 212C5
  \l 212C6
  \l 212C7
  \l 212C8
  \l 212C9
  \l 212CA
  \l 212CB
  \l 212CC
  \l 212CD
  \l 212CE
  \l 212CF
  \l 212D0
  \l 212D1
  \l 212D2
  \l 212D3
  \l 212D4
  \l 212D5
  \l 212D6
  \l 212D7
  \l 212D8
  \l 212D9
  \l 212DA
  \l 212DB
  \l 212DC
  \l 212DD
  \l 212DE
  \l 212DF
  \l 212E0
  \l 212E1
  \l 212E2
  \l 212E3
  \l 212E4
  \l 212E5
  \l 212E6
  \l 212E7
  \l 212E8
  \l 212E9
  \l 212EA
  \l 212EB
  \l 212EC
  \l 212ED
  \l 212EE
  \l 212EF
  \l 212F0
  \l 212F1
  \l 212F2
  \l 212F3
  \l 212F4
  \l 212F5
  \l 212F6
  \l 212F7
  \l 212F8
  \l 212F9
  \l 212FA
  \l 212FB
  \l 212FC
  \l 212FD
  \l 212FE
  \l 212FF
  \l 21300
  \l 21301
  \l 21302
  \l 21303
  \l 21304
  \l 21305
  \l 21306
  \l 21307
  \l 21308
  \l 21309
  \l 2130A
  \l 2130B
  \l 2130C
  \l 2130D
  \l 2130E
  \l 2130F
  \l 21310
  \l 21311
  \l 21312
  \l 21313
  \l 21314
  \l 21315
  \l 21316
  \l 21317
  \l 21318
  \l 21319
  \l 2131A
  \l 2131B
  \l 2131C
  \l 2131D
  \l 2131E
  \l 2131F
  \l 21320
  \l 21321
  \l 21322
  \l 21323
  \l 21324
  \l 21325
  \l 21326
  \l 21327
  \l 21328
  \l 21329
  \l 2132A
  \l 2132B
  \l 2132C
  \l 2132D
  \l 2132E
  \l 2132F
  \l 21330
  \l 21331
  \l 21332
  \l 21333
  \l 21334
  \l 21335
  \l 21336
  \l 21337
  \l 21338
  \l 21339
  \l 2133A
  \l 2133B
  \l 2133C
  \l 2133D
  \l 2133E
  \l 2133F
  \l 21340
  \l 21341
  \l 21342
  \l 21343
  \l 21344
  \l 21345
  \l 21346
  \l 21347
  \l 21348
  \l 21349
  \l 2134A
  \l 2134B
  \l 2134C
  \l 2134D
  \l 2134E
  \l 2134F
  \l 21350
  \l 21351
  \l 21352
  \l 21353
  \l 21354
  \l 21355
  \l 21356
  \l 21357
  \l 21358
  \l 21359
  \l 2135A
  \l 2135B
  \l 2135C
  \l 2135D
  \l 2135E
  \l 2135F
  \l 21360
  \l 21361
  \l 21362
  \l 21363
  \l 21364
  \l 21365
  \l 21366
  \l 21367
  \l 21368
  \l 21369
  \l 2136A
  \l 2136B
  \l 2136C
  \l 2136D
  \l 2136E
  \l 2136F
  \l 21370
  \l 21371
  \l 21372
  \l 21373
  \l 21374
  \l 21375
  \l 21376
  \l 21377
  \l 21378
  \l 21379
  \l 2137A
  \l 2137B
  \l 2137C
  \l 2137D
  \l 2137E
  \l 2137F
  \l 21380
  \l 21381
  \l 21382
  \l 21383
  \l 21384
  \l 21385
  \l 21386
  \l 21387
  \l 21388
  \l 21389
  \l 2138A
  \l 2138B
  \l 2138C
  \l 2138D
  \l 2138E
  \l 2138F
  \l 21390
  \l 21391
  \l 21392
  \l 21393
  \l 21394
  \l 21395
  \l 21396
  \l 21397
  \l 21398
  \l 21399
  \l 2139A
  \l 2139B
  \l 2139C
  \l 2139D
  \l 2139E
  \l 2139F
  \l 213A0
  \l 213A1
  \l 213A2
  \l 213A3
  \l 213A4
  \l 213A5
  \l 213A6
  \l 213A7
  \l 213A8
  \l 213A9
  \l 213AA
  \l 213AB
  \l 213AC
  \l 213AD
  \l 213AE
  \l 213AF
  \l 213B0
  \l 213B1
  \l 213B2
  \l 213B3
  \l 213B4
  \l 213B5
  \l 213B6
  \l 213B7
  \l 213B8
  \l 213B9
  \l 213BA
  \l 213BB
  \l 213BC
  \l 213BD
  \l 213BE
  \l 213BF
  \l 213C0
  \l 213C1
  \l 213C2
  \l 213C3
  \l 213C4
  \l 213C5
  \l 213C6
  \l 213C7
  \l 213C8
  \l 213C9
  \l 213CA
  \l 213CB
  \l 213CC
  \l 213CD
  \l 213CE
  \l 213CF
  \l 213D0
  \l 213D1
  \l 213D2
  \l 213D3
  \l 213D4
  \l 213D5
  \l 213D6
  \l 213D7
  \l 213D8
  \l 213D9
  \l 213DA
  \l 213DB
  \l 213DC
  \l 213DD
  \l 213DE
  \l 213DF
  \l 213E0
  \l 213E1
  \l 213E2
  \l 213E3
  \l 213E4
  \l 213E5
  \l 213E6
  \l 213E7
  \l 213E8
  \l 213E9
  \l 213EA
  \l 213EB
  \l 213EC
  \l 213ED
  \l 213EE
  \l 213EF
  \l 213F0
  \l 213F1
  \l 213F2
  \l 213F3
  \l 213F4
  \l 213F5
  \l 213F6
  \l 213F7
  \l 213F8
  \l 213F9
  \l 213FA
  \l 213FB
  \l 213FC
  \l 213FD
  \l 213FE
  \l 213FF
  \l 21400
  \l 21401
  \l 21402
  \l 21403
  \l 21404
  \l 21405
  \l 21406
  \l 21407
  \l 21408
  \l 21409
  \l 2140A
  \l 2140B
  \l 2140C
  \l 2140D
  \l 2140E
  \l 2140F
  \l 21410
  \l 21411
  \l 21412
  \l 21413
  \l 21414
  \l 21415
  \l 21416
  \l 21417
  \l 21418
  \l 21419
  \l 2141A
  \l 2141B
  \l 2141C
  \l 2141D
  \l 2141E
  \l 2141F
  \l 21420
  \l 21421
  \l 21422
  \l 21423
  \l 21424
  \l 21425
  \l 21426
  \l 21427
  \l 21428
  \l 21429
  \l 2142A
  \l 2142B
  \l 2142C
  \l 2142D
  \l 2142E
  \l 2142F
  \l 21430
  \l 21431
  \l 21432
  \l 21433
  \l 21434
  \l 21435
  \l 21436
  \l 21437
  \l 21438
  \l 21439
  \l 2143A
  \l 2143B
  \l 2143C
  \l 2143D
  \l 2143E
  \l 2143F
  \l 21440
  \l 21441
  \l 21442
  \l 21443
  \l 21444
  \l 21445
  \l 21446
  \l 21447
  \l 21448
  \l 21449
  \l 2144A
  \l 2144B
  \l 2144C
  \l 2144D
  \l 2144E
  \l 2144F
  \l 21450
  \l 21451
  \l 21452
  \l 21453
  \l 21454
  \l 21455
  \l 21456
  \l 21457
  \l 21458
  \l 21459
  \l 2145A
  \l 2145B
  \l 2145C
  \l 2145D
  \l 2145E
  \l 2145F
  \l 21460
  \l 21461
  \l 21462
  \l 21463
  \l 21464
  \l 21465
  \l 21466
  \l 21467
  \l 21468
  \l 21469
  \l 2146A
  \l 2146B
  \l 2146C
  \l 2146D
  \l 2146E
  \l 2146F
  \l 21470
  \l 21471
  \l 21472
  \l 21473
  \l 21474
  \l 21475
  \l 21476
  \l 21477
  \l 21478
  \l 21479
  \l 2147A
  \l 2147B
  \l 2147C
  \l 2147D
  \l 2147E
  \l 2147F
  \l 21480
  \l 21481
  \l 21482
  \l 21483
  \l 21484
  \l 21485
  \l 21486
  \l 21487
  \l 21488
  \l 21489
  \l 2148A
  \l 2148B
  \l 2148C
  \l 2148D
  \l 2148E
  \l 2148F
  \l 21490
  \l 21491
  \l 21492
  \l 21493
  \l 21494
  \l 21495
  \l 21496
  \l 21497
  \l 21498
  \l 21499
  \l 2149A
  \l 2149B
  \l 2149C
  \l 2149D
  \l 2149E
  \l 2149F
  \l 214A0
  \l 214A1
  \l 214A2
  \l 214A3
  \l 214A4
  \l 214A5
  \l 214A6
  \l 214A7
  \l 214A8
  \l 214A9
  \l 214AA
  \l 214AB
  \l 214AC
  \l 214AD
  \l 214AE
  \l 214AF
  \l 214B0
  \l 214B1
  \l 214B2
  \l 214B3
  \l 214B4
  \l 214B5
  \l 214B6
  \l 214B7
  \l 214B8
  \l 214B9
  \l 214BA
  \l 214BB
  \l 214BC
  \l 214BD
  \l 214BE
  \l 214BF
  \l 214C0
  \l 214C1
  \l 214C2
  \l 214C3
  \l 214C4
  \l 214C5
  \l 214C6
  \l 214C7
  \l 214C8
  \l 214C9
  \l 214CA
  \l 214CB
  \l 214CC
  \l 214CD
  \l 214CE
  \l 214CF
  \l 214D0
  \l 214D1
  \l 214D2
  \l 214D3
  \l 214D4
  \l 214D5
  \l 214D6
  \l 214D7
  \l 214D8
  \l 214D9
  \l 214DA
  \l 214DB
  \l 214DC
  \l 214DD
  \l 214DE
  \l 214DF
  \l 214E0
  \l 214E1
  \l 214E2
  \l 214E3
  \l 214E4
  \l 214E5
  \l 214E6
  \l 214E7
  \l 214E8
  \l 214E9
  \l 214EA
  \l 214EB
  \l 214EC
  \l 214ED
  \l 214EE
  \l 214EF
  \l 214F0
  \l 214F1
  \l 214F2
  \l 214F3
  \l 214F4
  \l 214F5
  \l 214F6
  \l 214F7
  \l 214F8
  \l 214F9
  \l 214FA
  \l 214FB
  \l 214FC
  \l 214FD
  \l 214FE
  \l 214FF
  \l 21500
  \l 21501
  \l 21502
  \l 21503
  \l 21504
  \l 21505
  \l 21506
  \l 21507
  \l 21508
  \l 21509
  \l 2150A
  \l 2150B
  \l 2150C
  \l 2150D
  \l 2150E
  \l 2150F
  \l 21510
  \l 21511
  \l 21512
  \l 21513
  \l 21514
  \l 21515
  \l 21516
  \l 21517
  \l 21518
  \l 21519
  \l 2151A
  \l 2151B
  \l 2151C
  \l 2151D
  \l 2151E
  \l 2151F
  \l 21520
  \l 21521
  \l 21522
  \l 21523
  \l 21524
  \l 21525
  \l 21526
  \l 21527
  \l 21528
  \l 21529
  \l 2152A
  \l 2152B
  \l 2152C
  \l 2152D
  \l 2152E
  \l 2152F
  \l 21530
  \l 21531
  \l 21532
  \l 21533
  \l 21534
  \l 21535
  \l 21536
  \l 21537
  \l 21538
  \l 21539
  \l 2153A
  \l 2153B
  \l 2153C
  \l 2153D
  \l 2153E
  \l 2153F
  \l 21540
  \l 21541
  \l 21542
  \l 21543
  \l 21544
  \l 21545
  \l 21546
  \l 21547
  \l 21548
  \l 21549
  \l 2154A
  \l 2154B
  \l 2154C
  \l 2154D
  \l 2154E
  \l 2154F
  \l 21550
  \l 21551
  \l 21552
  \l 21553
  \l 21554
  \l 21555
  \l 21556
  \l 21557
  \l 21558
  \l 21559
  \l 2155A
  \l 2155B
  \l 2155C
  \l 2155D
  \l 2155E
  \l 2155F
  \l 21560
  \l 21561
  \l 21562
  \l 21563
  \l 21564
  \l 21565
  \l 21566
  \l 21567
  \l 21568
  \l 21569
  \l 2156A
  \l 2156B
  \l 2156C
  \l 2156D
  \l 2156E
  \l 2156F
  \l 21570
  \l 21571
  \l 21572
  \l 21573
  \l 21574
  \l 21575
  \l 21576
  \l 21577
  \l 21578
  \l 21579
  \l 2157A
  \l 2157B
  \l 2157C
  \l 2157D
  \l 2157E
  \l 2157F
  \l 21580
  \l 21581
  \l 21582
  \l 21583
  \l 21584
  \l 21585
  \l 21586
  \l 21587
  \l 21588
  \l 21589
  \l 2158A
  \l 2158B
  \l 2158C
  \l 2158D
  \l 2158E
  \l 2158F
  \l 21590
  \l 21591
  \l 21592
  \l 21593
  \l 21594
  \l 21595
  \l 21596
  \l 21597
  \l 21598
  \l 21599
  \l 2159A
  \l 2159B
  \l 2159C
  \l 2159D
  \l 2159E
  \l 2159F
  \l 215A0
  \l 215A1
  \l 215A2
  \l 215A3
  \l 215A4
  \l 215A5
  \l 215A6
  \l 215A7
  \l 215A8
  \l 215A9
  \l 215AA
  \l 215AB
  \l 215AC
  \l 215AD
  \l 215AE
  \l 215AF
  \l 215B0
  \l 215B1
  \l 215B2
  \l 215B3
  \l 215B4
  \l 215B5
  \l 215B6
  \l 215B7
  \l 215B8
  \l 215B9
  \l 215BA
  \l 215BB
  \l 215BC
  \l 215BD
  \l 215BE
  \l 215BF
  \l 215C0
  \l 215C1
  \l 215C2
  \l 215C3
  \l 215C4
  \l 215C5
  \l 215C6
  \l 215C7
  \l 215C8
  \l 215C9
  \l 215CA
  \l 215CB
  \l 215CC
  \l 215CD
  \l 215CE
  \l 215CF
  \l 215D0
  \l 215D1
  \l 215D2
  \l 215D3
  \l 215D4
  \l 215D5
  \l 215D6
  \l 215D7
  \l 215D8
  \l 215D9
  \l 215DA
  \l 215DB
  \l 215DC
  \l 215DD
  \l 215DE
  \l 215DF
  \l 215E0
  \l 215E1
  \l 215E2
  \l 215E3
  \l 215E4
  \l 215E5
  \l 215E6
  \l 215E7
  \l 215E8
  \l 215E9
  \l 215EA
  \l 215EB
  \l 215EC
  \l 215ED
  \l 215EE
  \l 215EF
  \l 215F0
  \l 215F1
  \l 215F2
  \l 215F3
  \l 215F4
  \l 215F5
  \l 215F6
  \l 215F7
  \l 215F8
  \l 215F9
  \l 215FA
  \l 215FB
  \l 215FC
  \l 215FD
  \l 215FE
  \l 215FF
  \l 21600
  \l 21601
  \l 21602
  \l 21603
  \l 21604
  \l 21605
  \l 21606
  \l 21607
  \l 21608
  \l 21609
  \l 2160A
  \l 2160B
  \l 2160C
  \l 2160D
  \l 2160E
  \l 2160F
  \l 21610
  \l 21611
  \l 21612
  \l 21613
  \l 21614
  \l 21615
  \l 21616
  \l 21617
  \l 21618
  \l 21619
  \l 2161A
  \l 2161B
  \l 2161C
  \l 2161D
  \l 2161E
  \l 2161F
  \l 21620
  \l 21621
  \l 21622
  \l 21623
  \l 21624
  \l 21625
  \l 21626
  \l 21627
  \l 21628
  \l 21629
  \l 2162A
  \l 2162B
  \l 2162C
  \l 2162D
  \l 2162E
  \l 2162F
  \l 21630
  \l 21631
  \l 21632
  \l 21633
  \l 21634
  \l 21635
  \l 21636
  \l 21637
  \l 21638
  \l 21639
  \l 2163A
  \l 2163B
  \l 2163C
  \l 2163D
  \l 2163E
  \l 2163F
  \l 21640
  \l 21641
  \l 21642
  \l 21643
  \l 21644
  \l 21645
  \l 21646
  \l 21647
  \l 21648
  \l 21649
  \l 2164A
  \l 2164B
  \l 2164C
  \l 2164D
  \l 2164E
  \l 2164F
  \l 21650
  \l 21651
  \l 21652
  \l 21653
  \l 21654
  \l 21655
  \l 21656
  \l 21657
  \l 21658
  \l 21659
  \l 2165A
  \l 2165B
  \l 2165C
  \l 2165D
  \l 2165E
  \l 2165F
  \l 21660
  \l 21661
  \l 21662
  \l 21663
  \l 21664
  \l 21665
  \l 21666
  \l 21667
  \l 21668
  \l 21669
  \l 2166A
  \l 2166B
  \l 2166C
  \l 2166D
  \l 2166E
  \l 2166F
  \l 21670
  \l 21671
  \l 21672
  \l 21673
  \l 21674
  \l 21675
  \l 21676
  \l 21677
  \l 21678
  \l 21679
  \l 2167A
  \l 2167B
  \l 2167C
  \l 2167D
  \l 2167E
  \l 2167F
  \l 21680
  \l 21681
  \l 21682
  \l 21683
  \l 21684
  \l 21685
  \l 21686
  \l 21687
  \l 21688
  \l 21689
  \l 2168A
  \l 2168B
  \l 2168C
  \l 2168D
  \l 2168E
  \l 2168F
  \l 21690
  \l 21691
  \l 21692
  \l 21693
  \l 21694
  \l 21695
  \l 21696
  \l 21697
  \l 21698
  \l 21699
  \l 2169A
  \l 2169B
  \l 2169C
  \l 2169D
  \l 2169E
  \l 2169F
  \l 216A0
  \l 216A1
  \l 216A2
  \l 216A3
  \l 216A4
  \l 216A5
  \l 216A6
  \l 216A7
  \l 216A8
  \l 216A9
  \l 216AA
  \l 216AB
  \l 216AC
  \l 216AD
  \l 216AE
  \l 216AF
  \l 216B0
  \l 216B1
  \l 216B2
  \l 216B3
  \l 216B4
  \l 216B5
  \l 216B6
  \l 216B7
  \l 216B8
  \l 216B9
  \l 216BA
  \l 216BB
  \l 216BC
  \l 216BD
  \l 216BE
  \l 216BF
  \l 216C0
  \l 216C1
  \l 216C2
  \l 216C3
  \l 216C4
  \l 216C5
  \l 216C6
  \l 216C7
  \l 216C8
  \l 216C9
  \l 216CA
  \l 216CB
  \l 216CC
  \l 216CD
  \l 216CE
  \l 216CF
  \l 216D0
  \l 216D1
  \l 216D2
  \l 216D3
  \l 216D4
  \l 216D5
  \l 216D6
  \l 216D7
  \l 216D8
  \l 216D9
  \l 216DA
  \l 216DB
  \l 216DC
  \l 216DD
  \l 216DE
  \l 216DF
  \l 216E0
  \l 216E1
  \l 216E2
  \l 216E3
  \l 216E4
  \l 216E5
  \l 216E6
  \l 216E7
  \l 216E8
  \l 216E9
  \l 216EA
  \l 216EB
  \l 216EC
  \l 216ED
  \l 216EE
  \l 216EF
  \l 216F0
  \l 216F1
  \l 216F2
  \l 216F3
  \l 216F4
  \l 216F5
  \l 216F6
  \l 216F7
  \l 216F8
  \l 216F9
  \l 216FA
  \l 216FB
  \l 216FC
  \l 216FD
  \l 216FE
  \l 216FF
  \l 21700
  \l 21701
  \l 21702
  \l 21703
  \l 21704
  \l 21705
  \l 21706
  \l 21707
  \l 21708
  \l 21709
  \l 2170A
  \l 2170B
  \l 2170C
  \l 2170D
  \l 2170E
  \l 2170F
  \l 21710
  \l 21711
  \l 21712
  \l 21713
  \l 21714
  \l 21715
  \l 21716
  \l 21717
  \l 21718
  \l 21719
  \l 2171A
  \l 2171B
  \l 2171C
  \l 2171D
  \l 2171E
  \l 2171F
  \l 21720
  \l 21721
  \l 21722
  \l 21723
  \l 21724
  \l 21725
  \l 21726
  \l 21727
  \l 21728
  \l 21729
  \l 2172A
  \l 2172B
  \l 2172C
  \l 2172D
  \l 2172E
  \l 2172F
  \l 21730
  \l 21731
  \l 21732
  \l 21733
  \l 21734
  \l 21735
  \l 21736
  \l 21737
  \l 21738
  \l 21739
  \l 2173A
  \l 2173B
  \l 2173C
  \l 2173D
  \l 2173E
  \l 2173F
  \l 21740
  \l 21741
  \l 21742
  \l 21743
  \l 21744
  \l 21745
  \l 21746
  \l 21747
  \l 21748
  \l 21749
  \l 2174A
  \l 2174B
  \l 2174C
  \l 2174D
  \l 2174E
  \l 2174F
  \l 21750
  \l 21751
  \l 21752
  \l 21753
  \l 21754
  \l 21755
  \l 21756
  \l 21757
  \l 21758
  \l 21759
  \l 2175A
  \l 2175B
  \l 2175C
  \l 2175D
  \l 2175E
  \l 2175F
  \l 21760
  \l 21761
  \l 21762
  \l 21763
  \l 21764
  \l 21765
  \l 21766
  \l 21767
  \l 21768
  \l 21769
  \l 2176A
  \l 2176B
  \l 2176C
  \l 2176D
  \l 2176E
  \l 2176F
  \l 21770
  \l 21771
  \l 21772
  \l 21773
  \l 21774
  \l 21775
  \l 21776
  \l 21777
  \l 21778
  \l 21779
  \l 2177A
  \l 2177B
  \l 2177C
  \l 2177D
  \l 2177E
  \l 2177F
  \l 21780
  \l 21781
  \l 21782
  \l 21783
  \l 21784
  \l 21785
  \l 21786
  \l 21787
  \l 21788
  \l 21789
  \l 2178A
  \l 2178B
  \l 2178C
  \l 2178D
  \l 2178E
  \l 2178F
  \l 21790
  \l 21791
  \l 21792
  \l 21793
  \l 21794
  \l 21795
  \l 21796
  \l 21797
  \l 21798
  \l 21799
  \l 2179A
  \l 2179B
  \l 2179C
  \l 2179D
  \l 2179E
  \l 2179F
  \l 217A0
  \l 217A1
  \l 217A2
  \l 217A3
  \l 217A4
  \l 217A5
  \l 217A6
  \l 217A7
  \l 217A8
  \l 217A9
  \l 217AA
  \l 217AB
  \l 217AC
  \l 217AD
  \l 217AE
  \l 217AF
  \l 217B0
  \l 217B1
  \l 217B2
  \l 217B3
  \l 217B4
  \l 217B5
  \l 217B6
  \l 217B7
  \l 217B8
  \l 217B9
  \l 217BA
  \l 217BB
  \l 217BC
  \l 217BD
  \l 217BE
  \l 217BF
  \l 217C0
  \l 217C1
  \l 217C2
  \l 217C3
  \l 217C4
  \l 217C5
  \l 217C6
  \l 217C7
  \l 217C8
  \l 217C9
  \l 217CA
  \l 217CB
  \l 217CC
  \l 217CD
  \l 217CE
  \l 217CF
  \l 217D0
  \l 217D1
  \l 217D2
  \l 217D3
  \l 217D4
  \l 217D5
  \l 217D6
  \l 217D7
  \l 217D8
  \l 217D9
  \l 217DA
  \l 217DB
  \l 217DC
  \l 217DD
  \l 217DE
  \l 217DF
  \l 217E0
  \l 217E1
  \l 217E2
  \l 217E3
  \l 217E4
  \l 217E5
  \l 217E6
  \l 217E7
  \l 217E8
  \l 217E9
  \l 217EA
  \l 217EB
  \l 217EC
  \l 217ED
  \l 217EE
  \l 217EF
  \l 217F0
  \l 217F1
  \l 217F2
  \l 217F3
  \l 217F4
  \l 217F5
  \l 217F6
  \l 217F7
  \l 217F8
  \l 217F9
  \l 217FA
  \l 217FB
  \l 217FC
  \l 217FD
  \l 217FE
  \l 217FF
  \l 21800
  \l 21801
  \l 21802
  \l 21803
  \l 21804
  \l 21805
  \l 21806
  \l 21807
  \l 21808
  \l 21809
  \l 2180A
  \l 2180B
  \l 2180C
  \l 2180D
  \l 2180E
  \l 2180F
  \l 21810
  \l 21811
  \l 21812
  \l 21813
  \l 21814
  \l 21815
  \l 21816
  \l 21817
  \l 21818
  \l 21819
  \l 2181A
  \l 2181B
  \l 2181C
  \l 2181D
  \l 2181E
  \l 2181F
  \l 21820
  \l 21821
  \l 21822
  \l 21823
  \l 21824
  \l 21825
  \l 21826
  \l 21827
  \l 21828
  \l 21829
  \l 2182A
  \l 2182B
  \l 2182C
  \l 2182D
  \l 2182E
  \l 2182F
  \l 21830
  \l 21831
  \l 21832
  \l 21833
  \l 21834
  \l 21835
  \l 21836
  \l 21837
  \l 21838
  \l 21839
  \l 2183A
  \l 2183B
  \l 2183C
  \l 2183D
  \l 2183E
  \l 2183F
  \l 21840
  \l 21841
  \l 21842
  \l 21843
  \l 21844
  \l 21845
  \l 21846
  \l 21847
  \l 21848
  \l 21849
  \l 2184A
  \l 2184B
  \l 2184C
  \l 2184D
  \l 2184E
  \l 2184F
  \l 21850
  \l 21851
  \l 21852
  \l 21853
  \l 21854
  \l 21855
  \l 21856
  \l 21857
  \l 21858
  \l 21859
  \l 2185A
  \l 2185B
  \l 2185C
  \l 2185D
  \l 2185E
  \l 2185F
  \l 21860
  \l 21861
  \l 21862
  \l 21863
  \l 21864
  \l 21865
  \l 21866
  \l 21867
  \l 21868
  \l 21869
  \l 2186A
  \l 2186B
  \l 2186C
  \l 2186D
  \l 2186E
  \l 2186F
  \l 21870
  \l 21871
  \l 21872
  \l 21873
  \l 21874
  \l 21875
  \l 21876
  \l 21877
  \l 21878
  \l 21879
  \l 2187A
  \l 2187B
  \l 2187C
  \l 2187D
  \l 2187E
  \l 2187F
  \l 21880
  \l 21881
  \l 21882
  \l 21883
  \l 21884
  \l 21885
  \l 21886
  \l 21887
  \l 21888
  \l 21889
  \l 2188A
  \l 2188B
  \l 2188C
  \l 2188D
  \l 2188E
  \l 2188F
  \l 21890
  \l 21891
  \l 21892
  \l 21893
  \l 21894
  \l 21895
  \l 21896
  \l 21897
  \l 21898
  \l 21899
  \l 2189A
  \l 2189B
  \l 2189C
  \l 2189D
  \l 2189E
  \l 2189F
  \l 218A0
  \l 218A1
  \l 218A2
  \l 218A3
  \l 218A4
  \l 218A5
  \l 218A6
  \l 218A7
  \l 218A8
  \l 218A9
  \l 218AA
  \l 218AB
  \l 218AC
  \l 218AD
  \l 218AE
  \l 218AF
  \l 218B0
  \l 218B1
  \l 218B2
  \l 218B3
  \l 218B4
  \l 218B5
  \l 218B6
  \l 218B7
  \l 218B8
  \l 218B9
  \l 218BA
  \l 218BB
  \l 218BC
  \l 218BD
  \l 218BE
  \l 218BF
  \l 218C0
  \l 218C1
  \l 218C2
  \l 218C3
  \l 218C4
  \l 218C5
  \l 218C6
  \l 218C7
  \l 218C8
  \l 218C9
  \l 218CA
  \l 218CB
  \l 218CC
  \l 218CD
  \l 218CE
  \l 218CF
  \l 218D0
  \l 218D1
  \l 218D2
  \l 218D3
  \l 218D4
  \l 218D5
  \l 218D6
  \l 218D7
  \l 218D8
  \l 218D9
  \l 218DA
  \l 218DB
  \l 218DC
  \l 218DD
  \l 218DE
  \l 218DF
  \l 218E0
  \l 218E1
  \l 218E2
  \l 218E3
  \l 218E4
  \l 218E5
  \l 218E6
  \l 218E7
  \l 218E8
  \l 218E9
  \l 218EA
  \l 218EB
  \l 218EC
  \l 218ED
  \l 218EE
  \l 218EF
  \l 218F0
  \l 218F1
  \l 218F2
  \l 218F3
  \l 218F4
  \l 218F5
  \l 218F6
  \l 218F7
  \l 218F8
  \l 218F9
  \l 218FA
  \l 218FB
  \l 218FC
  \l 218FD
  \l 218FE
  \l 218FF
  \l 21900
  \l 21901
  \l 21902
  \l 21903
  \l 21904
  \l 21905
  \l 21906
  \l 21907
  \l 21908
  \l 21909
  \l 2190A
  \l 2190B
  \l 2190C
  \l 2190D
  \l 2190E
  \l 2190F
  \l 21910
  \l 21911
  \l 21912
  \l 21913
  \l 21914
  \l 21915
  \l 21916
  \l 21917
  \l 21918
  \l 21919
  \l 2191A
  \l 2191B
  \l 2191C
  \l 2191D
  \l 2191E
  \l 2191F
  \l 21920
  \l 21921
  \l 21922
  \l 21923
  \l 21924
  \l 21925
  \l 21926
  \l 21927
  \l 21928
  \l 21929
  \l 2192A
  \l 2192B
  \l 2192C
  \l 2192D
  \l 2192E
  \l 2192F
  \l 21930
  \l 21931
  \l 21932
  \l 21933
  \l 21934
  \l 21935
  \l 21936
  \l 21937
  \l 21938
  \l 21939
  \l 2193A
  \l 2193B
  \l 2193C
  \l 2193D
  \l 2193E
  \l 2193F
  \l 21940
  \l 21941
  \l 21942
  \l 21943
  \l 21944
  \l 21945
  \l 21946
  \l 21947
  \l 21948
  \l 21949
  \l 2194A
  \l 2194B
  \l 2194C
  \l 2194D
  \l 2194E
  \l 2194F
  \l 21950
  \l 21951
  \l 21952
  \l 21953
  \l 21954
  \l 21955
  \l 21956
  \l 21957
  \l 21958
  \l 21959
  \l 2195A
  \l 2195B
  \l 2195C
  \l 2195D
  \l 2195E
  \l 2195F
  \l 21960
  \l 21961
  \l 21962
  \l 21963
  \l 21964
  \l 21965
  \l 21966
  \l 21967
  \l 21968
  \l 21969
  \l 2196A
  \l 2196B
  \l 2196C
  \l 2196D
  \l 2196E
  \l 2196F
  \l 21970
  \l 21971
  \l 21972
  \l 21973
  \l 21974
  \l 21975
  \l 21976
  \l 21977
  \l 21978
  \l 21979
  \l 2197A
  \l 2197B
  \l 2197C
  \l 2197D
  \l 2197E
  \l 2197F
  \l 21980
  \l 21981
  \l 21982
  \l 21983
  \l 21984
  \l 21985
  \l 21986
  \l 21987
  \l 21988
  \l 21989
  \l 2198A
  \l 2198B
  \l 2198C
  \l 2198D
  \l 2198E
  \l 2198F
  \l 21990
  \l 21991
  \l 21992
  \l 21993
  \l 21994
  \l 21995
  \l 21996
  \l 21997
  \l 21998
  \l 21999
  \l 2199A
  \l 2199B
  \l 2199C
  \l 2199D
  \l 2199E
  \l 2199F
  \l 219A0
  \l 219A1
  \l 219A2
  \l 219A3
  \l 219A4
  \l 219A5
  \l 219A6
  \l 219A7
  \l 219A8
  \l 219A9
  \l 219AA
  \l 219AB
  \l 219AC
  \l 219AD
  \l 219AE
  \l 219AF
  \l 219B0
  \l 219B1
  \l 219B2
  \l 219B3
  \l 219B4
  \l 219B5
  \l 219B6
  \l 219B7
  \l 219B8
  \l 219B9
  \l 219BA
  \l 219BB
  \l 219BC
  \l 219BD
  \l 219BE
  \l 219BF
  \l 219C0
  \l 219C1
  \l 219C2
  \l 219C3
  \l 219C4
  \l 219C5
  \l 219C6
  \l 219C7
  \l 219C8
  \l 219C9
  \l 219CA
  \l 219CB
  \l 219CC
  \l 219CD
  \l 219CE
  \l 219CF
  \l 219D0
  \l 219D1
  \l 219D2
  \l 219D3
  \l 219D4
  \l 219D5
  \l 219D6
  \l 219D7
  \l 219D8
  \l 219D9
  \l 219DA
  \l 219DB
  \l 219DC
  \l 219DD
  \l 219DE
  \l 219DF
  \l 219E0
  \l 219E1
  \l 219E2
  \l 219E3
  \l 219E4
  \l 219E5
  \l 219E6
  \l 219E7
  \l 219E8
  \l 219E9
  \l 219EA
  \l 219EB
  \l 219EC
  \l 219ED
  \l 219EE
  \l 219EF
  \l 219F0
  \l 219F1
  \l 219F2
  \l 219F3
  \l 219F4
  \l 219F5
  \l 219F6
  \l 219F7
  \l 219F8
  \l 219F9
  \l 219FA
  \l 219FB
  \l 219FC
  \l 219FD
  \l 219FE
  \l 219FF
  \l 21A00
  \l 21A01
  \l 21A02
  \l 21A03
  \l 21A04
  \l 21A05
  \l 21A06
  \l 21A07
  \l 21A08
  \l 21A09
  \l 21A0A
  \l 21A0B
  \l 21A0C
  \l 21A0D
  \l 21A0E
  \l 21A0F
  \l 21A10
  \l 21A11
  \l 21A12
  \l 21A13
  \l 21A14
  \l 21A15
  \l 21A16
  \l 21A17
  \l 21A18
  \l 21A19
  \l 21A1A
  \l 21A1B
  \l 21A1C
  \l 21A1D
  \l 21A1E
  \l 21A1F
  \l 21A20
  \l 21A21
  \l 21A22
  \l 21A23
  \l 21A24
  \l 21A25
  \l 21A26
  \l 21A27
  \l 21A28
  \l 21A29
  \l 21A2A
  \l 21A2B
  \l 21A2C
  \l 21A2D
  \l 21A2E
  \l 21A2F
  \l 21A30
  \l 21A31
  \l 21A32
  \l 21A33
  \l 21A34
  \l 21A35
  \l 21A36
  \l 21A37
  \l 21A38
  \l 21A39
  \l 21A3A
  \l 21A3B
  \l 21A3C
  \l 21A3D
  \l 21A3E
  \l 21A3F
  \l 21A40
  \l 21A41
  \l 21A42
  \l 21A43
  \l 21A44
  \l 21A45
  \l 21A46
  \l 21A47
  \l 21A48
  \l 21A49
  \l 21A4A
  \l 21A4B
  \l 21A4C
  \l 21A4D
  \l 21A4E
  \l 21A4F
  \l 21A50
  \l 21A51
  \l 21A52
  \l 21A53
  \l 21A54
  \l 21A55
  \l 21A56
  \l 21A57
  \l 21A58
  \l 21A59
  \l 21A5A
  \l 21A5B
  \l 21A5C
  \l 21A5D
  \l 21A5E
  \l 21A5F
  \l 21A60
  \l 21A61
  \l 21A62
  \l 21A63
  \l 21A64
  \l 21A65
  \l 21A66
  \l 21A67
  \l 21A68
  \l 21A69
  \l 21A6A
  \l 21A6B
  \l 21A6C
  \l 21A6D
  \l 21A6E
  \l 21A6F
  \l 21A70
  \l 21A71
  \l 21A72
  \l 21A73
  \l 21A74
  \l 21A75
  \l 21A76
  \l 21A77
  \l 21A78
  \l 21A79
  \l 21A7A
  \l 21A7B
  \l 21A7C
  \l 21A7D
  \l 21A7E
  \l 21A7F
  \l 21A80
  \l 21A81
  \l 21A82
  \l 21A83
  \l 21A84
  \l 21A85
  \l 21A86
  \l 21A87
  \l 21A88
  \l 21A89
  \l 21A8A
  \l 21A8B
  \l 21A8C
  \l 21A8D
  \l 21A8E
  \l 21A8F
  \l 21A90
  \l 21A91
  \l 21A92
  \l 21A93
  \l 21A94
  \l 21A95
  \l 21A96
  \l 21A97
  \l 21A98
  \l 21A99
  \l 21A9A
  \l 21A9B
  \l 21A9C
  \l 21A9D
  \l 21A9E
  \l 21A9F
  \l 21AA0
  \l 21AA1
  \l 21AA2
  \l 21AA3
  \l 21AA4
  \l 21AA5
  \l 21AA6
  \l 21AA7
  \l 21AA8
  \l 21AA9
  \l 21AAA
  \l 21AAB
  \l 21AAC
  \l 21AAD
  \l 21AAE
  \l 21AAF
  \l 21AB0
  \l 21AB1
  \l 21AB2
  \l 21AB3
  \l 21AB4
  \l 21AB5
  \l 21AB6
  \l 21AB7
  \l 21AB8
  \l 21AB9
  \l 21ABA
  \l 21ABB
  \l 21ABC
  \l 21ABD
  \l 21ABE
  \l 21ABF
  \l 21AC0
  \l 21AC1
  \l 21AC2
  \l 21AC3
  \l 21AC4
  \l 21AC5
  \l 21AC6
  \l 21AC7
  \l 21AC8
  \l 21AC9
  \l 21ACA
  \l 21ACB
  \l 21ACC
  \l 21ACD
  \l 21ACE
  \l 21ACF
  \l 21AD0
  \l 21AD1
  \l 21AD2
  \l 21AD3
  \l 21AD4
  \l 21AD5
  \l 21AD6
  \l 21AD7
  \l 21AD8
  \l 21AD9
  \l 21ADA
  \l 21ADB
  \l 21ADC
  \l 21ADD
  \l 21ADE
  \l 21ADF
  \l 21AE0
  \l 21AE1
  \l 21AE2
  \l 21AE3
  \l 21AE4
  \l 21AE5
  \l 21AE6
  \l 21AE7
  \l 21AE8
  \l 21AE9
  \l 21AEA
  \l 21AEB
  \l 21AEC
  \l 21AED
  \l 21AEE
  \l 21AEF
  \l 21AF0
  \l 21AF1
  \l 21AF2
  \l 21AF3
  \l 21AF4
  \l 21AF5
  \l 21AF6
  \l 21AF7
  \l 21AF8
  \l 21AF9
  \l 21AFA
  \l 21AFB
  \l 21AFC
  \l 21AFD
  \l 21AFE
  \l 21AFF
  \l 21B00
  \l 21B01
  \l 21B02
  \l 21B03
  \l 21B04
  \l 21B05
  \l 21B06
  \l 21B07
  \l 21B08
  \l 21B09
  \l 21B0A
  \l 21B0B
  \l 21B0C
  \l 21B0D
  \l 21B0E
  \l 21B0F
  \l 21B10
  \l 21B11
  \l 21B12
  \l 21B13
  \l 21B14
  \l 21B15
  \l 21B16
  \l 21B17
  \l 21B18
  \l 21B19
  \l 21B1A
  \l 21B1B
  \l 21B1C
  \l 21B1D
  \l 21B1E
  \l 21B1F
  \l 21B20
  \l 21B21
  \l 21B22
  \l 21B23
  \l 21B24
  \l 21B25
  \l 21B26
  \l 21B27
  \l 21B28
  \l 21B29
  \l 21B2A
  \l 21B2B
  \l 21B2C
  \l 21B2D
  \l 21B2E
  \l 21B2F
  \l 21B30
  \l 21B31
  \l 21B32
  \l 21B33
  \l 21B34
  \l 21B35
  \l 21B36
  \l 21B37
  \l 21B38
  \l 21B39
  \l 21B3A
  \l 21B3B
  \l 21B3C
  \l 21B3D
  \l 21B3E
  \l 21B3F
  \l 21B40
  \l 21B41
  \l 21B42
  \l 21B43
  \l 21B44
  \l 21B45
  \l 21B46
  \l 21B47
  \l 21B48
  \l 21B49
  \l 21B4A
  \l 21B4B
  \l 21B4C
  \l 21B4D
  \l 21B4E
  \l 21B4F
  \l 21B50
  \l 21B51
  \l 21B52
  \l 21B53
  \l 21B54
  \l 21B55
  \l 21B56
  \l 21B57
  \l 21B58
  \l 21B59
  \l 21B5A
  \l 21B5B
  \l 21B5C
  \l 21B5D
  \l 21B5E
  \l 21B5F
  \l 21B60
  \l 21B61
  \l 21B62
  \l 21B63
  \l 21B64
  \l 21B65
  \l 21B66
  \l 21B67
  \l 21B68
  \l 21B69
  \l 21B6A
  \l 21B6B
  \l 21B6C
  \l 21B6D
  \l 21B6E
  \l 21B6F
  \l 21B70
  \l 21B71
  \l 21B72
  \l 21B73
  \l 21B74
  \l 21B75
  \l 21B76
  \l 21B77
  \l 21B78
  \l 21B79
  \l 21B7A
  \l 21B7B
  \l 21B7C
  \l 21B7D
  \l 21B7E
  \l 21B7F
  \l 21B80
  \l 21B81
  \l 21B82
  \l 21B83
  \l 21B84
  \l 21B85
  \l 21B86
  \l 21B87
  \l 21B88
  \l 21B89
  \l 21B8A
  \l 21B8B
  \l 21B8C
  \l 21B8D
  \l 21B8E
  \l 21B8F
  \l 21B90
  \l 21B91
  \l 21B92
  \l 21B93
  \l 21B94
  \l 21B95
  \l 21B96
  \l 21B97
  \l 21B98
  \l 21B99
  \l 21B9A
  \l 21B9B
  \l 21B9C
  \l 21B9D
  \l 21B9E
  \l 21B9F
  \l 21BA0
  \l 21BA1
  \l 21BA2
  \l 21BA3
  \l 21BA4
  \l 21BA5
  \l 21BA6
  \l 21BA7
  \l 21BA8
  \l 21BA9
  \l 21BAA
  \l 21BAB
  \l 21BAC
  \l 21BAD
  \l 21BAE
  \l 21BAF
  \l 21BB0
  \l 21BB1
  \l 21BB2
  \l 21BB3
  \l 21BB4
  \l 21BB5
  \l 21BB6
  \l 21BB7
  \l 21BB8
  \l 21BB9
  \l 21BBA
  \l 21BBB
  \l 21BBC
  \l 21BBD
  \l 21BBE
  \l 21BBF
  \l 21BC0
  \l 21BC1
  \l 21BC2
  \l 21BC3
  \l 21BC4
  \l 21BC5
  \l 21BC6
  \l 21BC7
  \l 21BC8
  \l 21BC9
  \l 21BCA
  \l 21BCB
  \l 21BCC
  \l 21BCD
  \l 21BCE
  \l 21BCF
  \l 21BD0
  \l 21BD1
  \l 21BD2
  \l 21BD3
  \l 21BD4
  \l 21BD5
  \l 21BD6
  \l 21BD7
  \l 21BD8
  \l 21BD9
  \l 21BDA
  \l 21BDB
  \l 21BDC
  \l 21BDD
  \l 21BDE
  \l 21BDF
  \l 21BE0
  \l 21BE1
  \l 21BE2
  \l 21BE3
  \l 21BE4
  \l 21BE5
  \l 21BE6
  \l 21BE7
  \l 21BE8
  \l 21BE9
  \l 21BEA
  \l 21BEB
  \l 21BEC
  \l 21BED
  \l 21BEE
  \l 21BEF
  \l 21BF0
  \l 21BF1
  \l 21BF2
  \l 21BF3
  \l 21BF4
  \l 21BF5
  \l 21BF6
  \l 21BF7
  \l 21BF8
  \l 21BF9
  \l 21BFA
  \l 21BFB
  \l 21BFC
  \l 21BFD
  \l 21BFE
  \l 21BFF
  \l 21C00
  \l 21C01
  \l 21C02
  \l 21C03
  \l 21C04
  \l 21C05
  \l 21C06
  \l 21C07
  \l 21C08
  \l 21C09
  \l 21C0A
  \l 21C0B
  \l 21C0C
  \l 21C0D
  \l 21C0E
  \l 21C0F
  \l 21C10
  \l 21C11
  \l 21C12
  \l 21C13
  \l 21C14
  \l 21C15
  \l 21C16
  \l 21C17
  \l 21C18
  \l 21C19
  \l 21C1A
  \l 21C1B
  \l 21C1C
  \l 21C1D
  \l 21C1E
  \l 21C1F
  \l 21C20
  \l 21C21
  \l 21C22
  \l 21C23
  \l 21C24
  \l 21C25
  \l 21C26
  \l 21C27
  \l 21C28
  \l 21C29
  \l 21C2A
  \l 21C2B
  \l 21C2C
  \l 21C2D
  \l 21C2E
  \l 21C2F
  \l 21C30
  \l 21C31
  \l 21C32
  \l 21C33
  \l 21C34
  \l 21C35
  \l 21C36
  \l 21C37
  \l 21C38
  \l 21C39
  \l 21C3A
  \l 21C3B
  \l 21C3C
  \l 21C3D
  \l 21C3E
  \l 21C3F
  \l 21C40
  \l 21C41
  \l 21C42
  \l 21C43
  \l 21C44
  \l 21C45
  \l 21C46
  \l 21C47
  \l 21C48
  \l 21C49
  \l 21C4A
  \l 21C4B
  \l 21C4C
  \l 21C4D
  \l 21C4E
  \l 21C4F
  \l 21C50
  \l 21C51
  \l 21C52
  \l 21C53
  \l 21C54
  \l 21C55
  \l 21C56
  \l 21C57
  \l 21C58
  \l 21C59
  \l 21C5A
  \l 21C5B
  \l 21C5C
  \l 21C5D
  \l 21C5E
  \l 21C5F
  \l 21C60
  \l 21C61
  \l 21C62
  \l 21C63
  \l 21C64
  \l 21C65
  \l 21C66
  \l 21C67
  \l 21C68
  \l 21C69
  \l 21C6A
  \l 21C6B
  \l 21C6C
  \l 21C6D
  \l 21C6E
  \l 21C6F
  \l 21C70
  \l 21C71
  \l 21C72
  \l 21C73
  \l 21C74
  \l 21C75
  \l 21C76
  \l 21C77
  \l 21C78
  \l 21C79
  \l 21C7A
  \l 21C7B
  \l 21C7C
  \l 21C7D
  \l 21C7E
  \l 21C7F
  \l 21C80
  \l 21C81
  \l 21C82
  \l 21C83
  \l 21C84
  \l 21C85
  \l 21C86
  \l 21C87
  \l 21C88
  \l 21C89
  \l 21C8A
  \l 21C8B
  \l 21C8C
  \l 21C8D
  \l 21C8E
  \l 21C8F
  \l 21C90
  \l 21C91
  \l 21C92
  \l 21C93
  \l 21C94
  \l 21C95
  \l 21C96
  \l 21C97
  \l 21C98
  \l 21C99
  \l 21C9A
  \l 21C9B
  \l 21C9C
  \l 21C9D
  \l 21C9E
  \l 21C9F
  \l 21CA0
  \l 21CA1
  \l 21CA2
  \l 21CA3
  \l 21CA4
  \l 21CA5
  \l 21CA6
  \l 21CA7
  \l 21CA8
  \l 21CA9
  \l 21CAA
  \l 21CAB
  \l 21CAC
  \l 21CAD
  \l 21CAE
  \l 21CAF
  \l 21CB0
  \l 21CB1
  \l 21CB2
  \l 21CB3
  \l 21CB4
  \l 21CB5
  \l 21CB6
  \l 21CB7
  \l 21CB8
  \l 21CB9
  \l 21CBA
  \l 21CBB
  \l 21CBC
  \l 21CBD
  \l 21CBE
  \l 21CBF
  \l 21CC0
  \l 21CC1
  \l 21CC2
  \l 21CC3
  \l 21CC4
  \l 21CC5
  \l 21CC6
  \l 21CC7
  \l 21CC8
  \l 21CC9
  \l 21CCA
  \l 21CCB
  \l 21CCC
  \l 21CCD
  \l 21CCE
  \l 21CCF
  \l 21CD0
  \l 21CD1
  \l 21CD2
  \l 21CD3
  \l 21CD4
  \l 21CD5
  \l 21CD6
  \l 21CD7
  \l 21CD8
  \l 21CD9
  \l 21CDA
  \l 21CDB
  \l 21CDC
  \l 21CDD
  \l 21CDE
  \l 21CDF
  \l 21CE0
  \l 21CE1
  \l 21CE2
  \l 21CE3
  \l 21CE4
  \l 21CE5
  \l 21CE6
  \l 21CE7
  \l 21CE8
  \l 21CE9
  \l 21CEA
  \l 21CEB
  \l 21CEC
  \l 21CED
  \l 21CEE
  \l 21CEF
  \l 21CF0
  \l 21CF1
  \l 21CF2
  \l 21CF3
  \l 21CF4
  \l 21CF5
  \l 21CF6
  \l 21CF7
  \l 21CF8
  \l 21CF9
  \l 21CFA
  \l 21CFB
  \l 21CFC
  \l 21CFD
  \l 21CFE
  \l 21CFF
  \l 21D00
  \l 21D01
  \l 21D02
  \l 21D03
  \l 21D04
  \l 21D05
  \l 21D06
  \l 21D07
  \l 21D08
  \l 21D09
  \l 21D0A
  \l 21D0B
  \l 21D0C
  \l 21D0D
  \l 21D0E
  \l 21D0F
  \l 21D10
  \l 21D11
  \l 21D12
  \l 21D13
  \l 21D14
  \l 21D15
  \l 21D16
  \l 21D17
  \l 21D18
  \l 21D19
  \l 21D1A
  \l 21D1B
  \l 21D1C
  \l 21D1D
  \l 21D1E
  \l 21D1F
  \l 21D20
  \l 21D21
  \l 21D22
  \l 21D23
  \l 21D24
  \l 21D25
  \l 21D26
  \l 21D27
  \l 21D28
  \l 21D29
  \l 21D2A
  \l 21D2B
  \l 21D2C
  \l 21D2D
  \l 21D2E
  \l 21D2F
  \l 21D30
  \l 21D31
  \l 21D32
  \l 21D33
  \l 21D34
  \l 21D35
  \l 21D36
  \l 21D37
  \l 21D38
  \l 21D39
  \l 21D3A
  \l 21D3B
  \l 21D3C
  \l 21D3D
  \l 21D3E
  \l 21D3F
  \l 21D40
  \l 21D41
  \l 21D42
  \l 21D43
  \l 21D44
  \l 21D45
  \l 21D46
  \l 21D47
  \l 21D48
  \l 21D49
  \l 21D4A
  \l 21D4B
  \l 21D4C
  \l 21D4D
  \l 21D4E
  \l 21D4F
  \l 21D50
  \l 21D51
  \l 21D52
  \l 21D53
  \l 21D54
  \l 21D55
  \l 21D56
  \l 21D57
  \l 21D58
  \l 21D59
  \l 21D5A
  \l 21D5B
  \l 21D5C
  \l 21D5D
  \l 21D5E
  \l 21D5F
  \l 21D60
  \l 21D61
  \l 21D62
  \l 21D63
  \l 21D64
  \l 21D65
  \l 21D66
  \l 21D67
  \l 21D68
  \l 21D69
  \l 21D6A
  \l 21D6B
  \l 21D6C
  \l 21D6D
  \l 21D6E
  \l 21D6F
  \l 21D70
  \l 21D71
  \l 21D72
  \l 21D73
  \l 21D74
  \l 21D75
  \l 21D76
  \l 21D77
  \l 21D78
  \l 21D79
  \l 21D7A
  \l 21D7B
  \l 21D7C
  \l 21D7D
  \l 21D7E
  \l 21D7F
  \l 21D80
  \l 21D81
  \l 21D82
  \l 21D83
  \l 21D84
  \l 21D85
  \l 21D86
  \l 21D87
  \l 21D88
  \l 21D89
  \l 21D8A
  \l 21D8B
  \l 21D8C
  \l 21D8D
  \l 21D8E
  \l 21D8F
  \l 21D90
  \l 21D91
  \l 21D92
  \l 21D93
  \l 21D94
  \l 21D95
  \l 21D96
  \l 21D97
  \l 21D98
  \l 21D99
  \l 21D9A
  \l 21D9B
  \l 21D9C
  \l 21D9D
  \l 21D9E
  \l 21D9F
  \l 21DA0
  \l 21DA1
  \l 21DA2
  \l 21DA3
  \l 21DA4
  \l 21DA5
  \l 21DA6
  \l 21DA7
  \l 21DA8
  \l 21DA9
  \l 21DAA
  \l 21DAB
  \l 21DAC
  \l 21DAD
  \l 21DAE
  \l 21DAF
  \l 21DB0
  \l 21DB1
  \l 21DB2
  \l 21DB3
  \l 21DB4
  \l 21DB5
  \l 21DB6
  \l 21DB7
  \l 21DB8
  \l 21DB9
  \l 21DBA
  \l 21DBB
  \l 21DBC
  \l 21DBD
  \l 21DBE
  \l 21DBF
  \l 21DC0
  \l 21DC1
  \l 21DC2
  \l 21DC3
  \l 21DC4
  \l 21DC5
  \l 21DC6
  \l 21DC7
  \l 21DC8
  \l 21DC9
  \l 21DCA
  \l 21DCB
  \l 21DCC
  \l 21DCD
  \l 21DCE
  \l 21DCF
  \l 21DD0
  \l 21DD1
  \l 21DD2
  \l 21DD3
  \l 21DD4
  \l 21DD5
  \l 21DD6
  \l 21DD7
  \l 21DD8
  \l 21DD9
  \l 21DDA
  \l 21DDB
  \l 21DDC
  \l 21DDD
  \l 21DDE
  \l 21DDF
  \l 21DE0
  \l 21DE1
  \l 21DE2
  \l 21DE3
  \l 21DE4
  \l 21DE5
  \l 21DE6
  \l 21DE7
  \l 21DE8
  \l 21DE9
  \l 21DEA
  \l 21DEB
  \l 21DEC
  \l 21DED
  \l 21DEE
  \l 21DEF
  \l 21DF0
  \l 21DF1
  \l 21DF2
  \l 21DF3
  \l 21DF4
  \l 21DF5
  \l 21DF6
  \l 21DF7
  \l 21DF8
  \l 21DF9
  \l 21DFA
  \l 21DFB
  \l 21DFC
  \l 21DFD
  \l 21DFE
  \l 21DFF
  \l 21E00
  \l 21E01
  \l 21E02
  \l 21E03
  \l 21E04
  \l 21E05
  \l 21E06
  \l 21E07
  \l 21E08
  \l 21E09
  \l 21E0A
  \l 21E0B
  \l 21E0C
  \l 21E0D
  \l 21E0E
  \l 21E0F
  \l 21E10
  \l 21E11
  \l 21E12
  \l 21E13
  \l 21E14
  \l 21E15
  \l 21E16
  \l 21E17
  \l 21E18
  \l 21E19
  \l 21E1A
  \l 21E1B
  \l 21E1C
  \l 21E1D
  \l 21E1E
  \l 21E1F
  \l 21E20
  \l 21E21
  \l 21E22
  \l 21E23
  \l 21E24
  \l 21E25
  \l 21E26
  \l 21E27
  \l 21E28
  \l 21E29
  \l 21E2A
  \l 21E2B
  \l 21E2C
  \l 21E2D
  \l 21E2E
  \l 21E2F
  \l 21E30
  \l 21E31
  \l 21E32
  \l 21E33
  \l 21E34
  \l 21E35
  \l 21E36
  \l 21E37
  \l 21E38
  \l 21E39
  \l 21E3A
  \l 21E3B
  \l 21E3C
  \l 21E3D
  \l 21E3E
  \l 21E3F
  \l 21E40
  \l 21E41
  \l 21E42
  \l 21E43
  \l 21E44
  \l 21E45
  \l 21E46
  \l 21E47
  \l 21E48
  \l 21E49
  \l 21E4A
  \l 21E4B
  \l 21E4C
  \l 21E4D
  \l 21E4E
  \l 21E4F
  \l 21E50
  \l 21E51
  \l 21E52
  \l 21E53
  \l 21E54
  \l 21E55
  \l 21E56
  \l 21E57
  \l 21E58
  \l 21E59
  \l 21E5A
  \l 21E5B
  \l 21E5C
  \l 21E5D
  \l 21E5E
  \l 21E5F
  \l 21E60
  \l 21E61
  \l 21E62
  \l 21E63
  \l 21E64
  \l 21E65
  \l 21E66
  \l 21E67
  \l 21E68
  \l 21E69
  \l 21E6A
  \l 21E6B
  \l 21E6C
  \l 21E6D
  \l 21E6E
  \l 21E6F
  \l 21E70
  \l 21E71
  \l 21E72
  \l 21E73
  \l 21E74
  \l 21E75
  \l 21E76
  \l 21E77
  \l 21E78
  \l 21E79
  \l 21E7A
  \l 21E7B
  \l 21E7C
  \l 21E7D
  \l 21E7E
  \l 21E7F
  \l 21E80
  \l 21E81
  \l 21E82
  \l 21E83
  \l 21E84
  \l 21E85
  \l 21E86
  \l 21E87
  \l 21E88
  \l 21E89
  \l 21E8A
  \l 21E8B
  \l 21E8C
  \l 21E8D
  \l 21E8E
  \l 21E8F
  \l 21E90
  \l 21E91
  \l 21E92
  \l 21E93
  \l 21E94
  \l 21E95
  \l 21E96
  \l 21E97
  \l 21E98
  \l 21E99
  \l 21E9A
  \l 21E9B
  \l 21E9C
  \l 21E9D
  \l 21E9E
  \l 21E9F
  \l 21EA0
  \l 21EA1
  \l 21EA2
  \l 21EA3
  \l 21EA4
  \l 21EA5
  \l 21EA6
  \l 21EA7
  \l 21EA8
  \l 21EA9
  \l 21EAA
  \l 21EAB
  \l 21EAC
  \l 21EAD
  \l 21EAE
  \l 21EAF
  \l 21EB0
  \l 21EB1
  \l 21EB2
  \l 21EB3
  \l 21EB4
  \l 21EB5
  \l 21EB6
  \l 21EB7
  \l 21EB8
  \l 21EB9
  \l 21EBA
  \l 21EBB
  \l 21EBC
  \l 21EBD
  \l 21EBE
  \l 21EBF
  \l 21EC0
  \l 21EC1
  \l 21EC2
  \l 21EC3
  \l 21EC4
  \l 21EC5
  \l 21EC6
  \l 21EC7
  \l 21EC8
  \l 21EC9
  \l 21ECA
  \l 21ECB
  \l 21ECC
  \l 21ECD
  \l 21ECE
  \l 21ECF
  \l 21ED0
  \l 21ED1
  \l 21ED2
  \l 21ED3
  \l 21ED4
  \l 21ED5
  \l 21ED6
  \l 21ED7
  \l 21ED8
  \l 21ED9
  \l 21EDA
  \l 21EDB
  \l 21EDC
  \l 21EDD
  \l 21EDE
  \l 21EDF
  \l 21EE0
  \l 21EE1
  \l 21EE2
  \l 21EE3
  \l 21EE4
  \l 21EE5
  \l 21EE6
  \l 21EE7
  \l 21EE8
  \l 21EE9
  \l 21EEA
  \l 21EEB
  \l 21EEC
  \l 21EED
  \l 21EEE
  \l 21EEF
  \l 21EF0
  \l 21EF1
  \l 21EF2
  \l 21EF3
  \l 21EF4
  \l 21EF5
  \l 21EF6
  \l 21EF7
  \l 21EF8
  \l 21EF9
  \l 21EFA
  \l 21EFB
  \l 21EFC
  \l 21EFD
  \l 21EFE
  \l 21EFF
  \l 21F00
  \l 21F01
  \l 21F02
  \l 21F03
  \l 21F04
  \l 21F05
  \l 21F06
  \l 21F07
  \l 21F08
  \l 21F09
  \l 21F0A
  \l 21F0B
  \l 21F0C
  \l 21F0D
  \l 21F0E
  \l 21F0F
  \l 21F10
  \l 21F11
  \l 21F12
  \l 21F13
  \l 21F14
  \l 21F15
  \l 21F16
  \l 21F17
  \l 21F18
  \l 21F19
  \l 21F1A
  \l 21F1B
  \l 21F1C
  \l 21F1D
  \l 21F1E
  \l 21F1F
  \l 21F20
  \l 21F21
  \l 21F22
  \l 21F23
  \l 21F24
  \l 21F25
  \l 21F26
  \l 21F27
  \l 21F28
  \l 21F29
  \l 21F2A
  \l 21F2B
  \l 21F2C
  \l 21F2D
  \l 21F2E
  \l 21F2F
  \l 21F30
  \l 21F31
  \l 21F32
  \l 21F33
  \l 21F34
  \l 21F35
  \l 21F36
  \l 21F37
  \l 21F38
  \l 21F39
  \l 21F3A
  \l 21F3B
  \l 21F3C
  \l 21F3D
  \l 21F3E
  \l 21F3F
  \l 21F40
  \l 21F41
  \l 21F42
  \l 21F43
  \l 21F44
  \l 21F45
  \l 21F46
  \l 21F47
  \l 21F48
  \l 21F49
  \l 21F4A
  \l 21F4B
  \l 21F4C
  \l 21F4D
  \l 21F4E
  \l 21F4F
  \l 21F50
  \l 21F51
  \l 21F52
  \l 21F53
  \l 21F54
  \l 21F55
  \l 21F56
  \l 21F57
  \l 21F58
  \l 21F59
  \l 21F5A
  \l 21F5B
  \l 21F5C
  \l 21F5D
  \l 21F5E
  \l 21F5F
  \l 21F60
  \l 21F61
  \l 21F62
  \l 21F63
  \l 21F64
  \l 21F65
  \l 21F66
  \l 21F67
  \l 21F68
  \l 21F69
  \l 21F6A
  \l 21F6B
  \l 21F6C
  \l 21F6D
  \l 21F6E
  \l 21F6F
  \l 21F70
  \l 21F71
  \l 21F72
  \l 21F73
  \l 21F74
  \l 21F75
  \l 21F76
  \l 21F77
  \l 21F78
  \l 21F79
  \l 21F7A
  \l 21F7B
  \l 21F7C
  \l 21F7D
  \l 21F7E
  \l 21F7F
  \l 21F80
  \l 21F81
  \l 21F82
  \l 21F83
  \l 21F84
  \l 21F85
  \l 21F86
  \l 21F87
  \l 21F88
  \l 21F89
  \l 21F8A
  \l 21F8B
  \l 21F8C
  \l 21F8D
  \l 21F8E
  \l 21F8F
  \l 21F90
  \l 21F91
  \l 21F92
  \l 21F93
  \l 21F94
  \l 21F95
  \l 21F96
  \l 21F97
  \l 21F98
  \l 21F99
  \l 21F9A
  \l 21F9B
  \l 21F9C
  \l 21F9D
  \l 21F9E
  \l 21F9F
  \l 21FA0
  \l 21FA1
  \l 21FA2
  \l 21FA3
  \l 21FA4
  \l 21FA5
  \l 21FA6
  \l 21FA7
  \l 21FA8
  \l 21FA9
  \l 21FAA
  \l 21FAB
  \l 21FAC
  \l 21FAD
  \l 21FAE
  \l 21FAF
  \l 21FB0
  \l 21FB1
  \l 21FB2
  \l 21FB3
  \l 21FB4
  \l 21FB5
  \l 21FB6
  \l 21FB7
  \l 21FB8
  \l 21FB9
  \l 21FBA
  \l 21FBB
  \l 21FBC
  \l 21FBD
  \l 21FBE
  \l 21FBF
  \l 21FC0
  \l 21FC1
  \l 21FC2
  \l 21FC3
  \l 21FC4
  \l 21FC5
  \l 21FC6
  \l 21FC7
  \l 21FC8
  \l 21FC9
  \l 21FCA
  \l 21FCB
  \l 21FCC
  \l 21FCD
  \l 21FCE
  \l 21FCF
  \l 21FD0
  \l 21FD1
  \l 21FD2
  \l 21FD3
  \l 21FD4
  \l 21FD5
  \l 21FD6
  \l 21FD7
  \l 21FD8
  \l 21FD9
  \l 21FDA
  \l 21FDB
  \l 21FDC
  \l 21FDD
  \l 21FDE
  \l 21FDF
  \l 21FE0
  \l 21FE1
  \l 21FE2
  \l 21FE3
  \l 21FE4
  \l 21FE5
  \l 21FE6
  \l 21FE7
  \l 21FE8
  \l 21FE9
  \l 21FEA
  \l 21FEB
  \l 21FEC
  \l 21FED
  \l 21FEE
  \l 21FEF
  \l 21FF0
  \l 21FF1
  \l 21FF2
  \l 21FF3
  \l 21FF4
  \l 21FF5
  \l 21FF6
  \l 21FF7
  \l 21FF8
  \l 21FF9
  \l 21FFA
  \l 21FFB
  \l 21FFC
  \l 21FFD
  \l 21FFE
  \l 21FFF
  \l 22000
  \l 22001
  \l 22002
  \l 22003
  \l 22004
  \l 22005
  \l 22006
  \l 22007
  \l 22008
  \l 22009
  \l 2200A
  \l 2200B
  \l 2200C
  \l 2200D
  \l 2200E
  \l 2200F
  \l 22010
  \l 22011
  \l 22012
  \l 22013
  \l 22014
  \l 22015
  \l 22016
  \l 22017
  \l 22018
  \l 22019
  \l 2201A
  \l 2201B
  \l 2201C
  \l 2201D
  \l 2201E
  \l 2201F
  \l 22020
  \l 22021
  \l 22022
  \l 22023
  \l 22024
  \l 22025
  \l 22026
  \l 22027
  \l 22028
  \l 22029
  \l 2202A
  \l 2202B
  \l 2202C
  \l 2202D
  \l 2202E
  \l 2202F
  \l 22030
  \l 22031
  \l 22032
  \l 22033
  \l 22034
  \l 22035
  \l 22036
  \l 22037
  \l 22038
  \l 22039
  \l 2203A
  \l 2203B
  \l 2203C
  \l 2203D
  \l 2203E
  \l 2203F
  \l 22040
  \l 22041
  \l 22042
  \l 22043
  \l 22044
  \l 22045
  \l 22046
  \l 22047
  \l 22048
  \l 22049
  \l 2204A
  \l 2204B
  \l 2204C
  \l 2204D
  \l 2204E
  \l 2204F
  \l 22050
  \l 22051
  \l 22052
  \l 22053
  \l 22054
  \l 22055
  \l 22056
  \l 22057
  \l 22058
  \l 22059
  \l 2205A
  \l 2205B
  \l 2205C
  \l 2205D
  \l 2205E
  \l 2205F
  \l 22060
  \l 22061
  \l 22062
  \l 22063
  \l 22064
  \l 22065
  \l 22066
  \l 22067
  \l 22068
  \l 22069
  \l 2206A
  \l 2206B
  \l 2206C
  \l 2206D
  \l 2206E
  \l 2206F
  \l 22070
  \l 22071
  \l 22072
  \l 22073
  \l 22074
  \l 22075
  \l 22076
  \l 22077
  \l 22078
  \l 22079
  \l 2207A
  \l 2207B
  \l 2207C
  \l 2207D
  \l 2207E
  \l 2207F
  \l 22080
  \l 22081
  \l 22082
  \l 22083
  \l 22084
  \l 22085
  \l 22086
  \l 22087
  \l 22088
  \l 22089
  \l 2208A
  \l 2208B
  \l 2208C
  \l 2208D
  \l 2208E
  \l 2208F
  \l 22090
  \l 22091
  \l 22092
  \l 22093
  \l 22094
  \l 22095
  \l 22096
  \l 22097
  \l 22098
  \l 22099
  \l 2209A
  \l 2209B
  \l 2209C
  \l 2209D
  \l 2209E
  \l 2209F
  \l 220A0
  \l 220A1
  \l 220A2
  \l 220A3
  \l 220A4
  \l 220A5
  \l 220A6
  \l 220A7
  \l 220A8
  \l 220A9
  \l 220AA
  \l 220AB
  \l 220AC
  \l 220AD
  \l 220AE
  \l 220AF
  \l 220B0
  \l 220B1
  \l 220B2
  \l 220B3
  \l 220B4
  \l 220B5
  \l 220B6
  \l 220B7
  \l 220B8
  \l 220B9
  \l 220BA
  \l 220BB
  \l 220BC
  \l 220BD
  \l 220BE
  \l 220BF
  \l 220C0
  \l 220C1
  \l 220C2
  \l 220C3
  \l 220C4
  \l 220C5
  \l 220C6
  \l 220C7
  \l 220C8
  \l 220C9
  \l 220CA
  \l 220CB
  \l 220CC
  \l 220CD
  \l 220CE
  \l 220CF
  \l 220D0
  \l 220D1
  \l 220D2
  \l 220D3
  \l 220D4
  \l 220D5
  \l 220D6
  \l 220D7
  \l 220D8
  \l 220D9
  \l 220DA
  \l 220DB
  \l 220DC
  \l 220DD
  \l 220DE
  \l 220DF
  \l 220E0
  \l 220E1
  \l 220E2
  \l 220E3
  \l 220E4
  \l 220E5
  \l 220E6
  \l 220E7
  \l 220E8
  \l 220E9
  \l 220EA
  \l 220EB
  \l 220EC
  \l 220ED
  \l 220EE
  \l 220EF
  \l 220F0
  \l 220F1
  \l 220F2
  \l 220F3
  \l 220F4
  \l 220F5
  \l 220F6
  \l 220F7
  \l 220F8
  \l 220F9
  \l 220FA
  \l 220FB
  \l 220FC
  \l 220FD
  \l 220FE
  \l 220FF
  \l 22100
  \l 22101
  \l 22102
  \l 22103
  \l 22104
  \l 22105
  \l 22106
  \l 22107
  \l 22108
  \l 22109
  \l 2210A
  \l 2210B
  \l 2210C
  \l 2210D
  \l 2210E
  \l 2210F
  \l 22110
  \l 22111
  \l 22112
  \l 22113
  \l 22114
  \l 22115
  \l 22116
  \l 22117
  \l 22118
  \l 22119
  \l 2211A
  \l 2211B
  \l 2211C
  \l 2211D
  \l 2211E
  \l 2211F
  \l 22120
  \l 22121
  \l 22122
  \l 22123
  \l 22124
  \l 22125
  \l 22126
  \l 22127
  \l 22128
  \l 22129
  \l 2212A
  \l 2212B
  \l 2212C
  \l 2212D
  \l 2212E
  \l 2212F
  \l 22130
  \l 22131
  \l 22132
  \l 22133
  \l 22134
  \l 22135
  \l 22136
  \l 22137
  \l 22138
  \l 22139
  \l 2213A
  \l 2213B
  \l 2213C
  \l 2213D
  \l 2213E
  \l 2213F
  \l 22140
  \l 22141
  \l 22142
  \l 22143
  \l 22144
  \l 22145
  \l 22146
  \l 22147
  \l 22148
  \l 22149
  \l 2214A
  \l 2214B
  \l 2214C
  \l 2214D
  \l 2214E
  \l 2214F
  \l 22150
  \l 22151
  \l 22152
  \l 22153
  \l 22154
  \l 22155
  \l 22156
  \l 22157
  \l 22158
  \l 22159
  \l 2215A
  \l 2215B
  \l 2215C
  \l 2215D
  \l 2215E
  \l 2215F
  \l 22160
  \l 22161
  \l 22162
  \l 22163
  \l 22164
  \l 22165
  \l 22166
  \l 22167
  \l 22168
  \l 22169
  \l 2216A
  \l 2216B
  \l 2216C
  \l 2216D
  \l 2216E
  \l 2216F
  \l 22170
  \l 22171
  \l 22172
  \l 22173
  \l 22174
  \l 22175
  \l 22176
  \l 22177
  \l 22178
  \l 22179
  \l 2217A
  \l 2217B
  \l 2217C
  \l 2217D
  \l 2217E
  \l 2217F
  \l 22180
  \l 22181
  \l 22182
  \l 22183
  \l 22184
  \l 22185
  \l 22186
  \l 22187
  \l 22188
  \l 22189
  \l 2218A
  \l 2218B
  \l 2218C
  \l 2218D
  \l 2218E
  \l 2218F
  \l 22190
  \l 22191
  \l 22192
  \l 22193
  \l 22194
  \l 22195
  \l 22196
  \l 22197
  \l 22198
  \l 22199
  \l 2219A
  \l 2219B
  \l 2219C
  \l 2219D
  \l 2219E
  \l 2219F
  \l 221A0
  \l 221A1
  \l 221A2
  \l 221A3
  \l 221A4
  \l 221A5
  \l 221A6
  \l 221A7
  \l 221A8
  \l 221A9
  \l 221AA
  \l 221AB
  \l 221AC
  \l 221AD
  \l 221AE
  \l 221AF
  \l 221B0
  \l 221B1
  \l 221B2
  \l 221B3
  \l 221B4
  \l 221B5
  \l 221B6
  \l 221B7
  \l 221B8
  \l 221B9
  \l 221BA
  \l 221BB
  \l 221BC
  \l 221BD
  \l 221BE
  \l 221BF
  \l 221C0
  \l 221C1
  \l 221C2
  \l 221C3
  \l 221C4
  \l 221C5
  \l 221C6
  \l 221C7
  \l 221C8
  \l 221C9
  \l 221CA
  \l 221CB
  \l 221CC
  \l 221CD
  \l 221CE
  \l 221CF
  \l 221D0
  \l 221D1
  \l 221D2
  \l 221D3
  \l 221D4
  \l 221D5
  \l 221D6
  \l 221D7
  \l 221D8
  \l 221D9
  \l 221DA
  \l 221DB
  \l 221DC
  \l 221DD
  \l 221DE
  \l 221DF
  \l 221E0
  \l 221E1
  \l 221E2
  \l 221E3
  \l 221E4
  \l 221E5
  \l 221E6
  \l 221E7
  \l 221E8
  \l 221E9
  \l 221EA
  \l 221EB
  \l 221EC
  \l 221ED
  \l 221EE
  \l 221EF
  \l 221F0
  \l 221F1
  \l 221F2
  \l 221F3
  \l 221F4
  \l 221F5
  \l 221F6
  \l 221F7
  \l 221F8
  \l 221F9
  \l 221FA
  \l 221FB
  \l 221FC
  \l 221FD
  \l 221FE
  \l 221FF
  \l 22200
  \l 22201
  \l 22202
  \l 22203
  \l 22204
  \l 22205
  \l 22206
  \l 22207
  \l 22208
  \l 22209
  \l 2220A
  \l 2220B
  \l 2220C
  \l 2220D
  \l 2220E
  \l 2220F
  \l 22210
  \l 22211
  \l 22212
  \l 22213
  \l 22214
  \l 22215
  \l 22216
  \l 22217
  \l 22218
  \l 22219
  \l 2221A
  \l 2221B
  \l 2221C
  \l 2221D
  \l 2221E
  \l 2221F
  \l 22220
  \l 22221
  \l 22222
  \l 22223
  \l 22224
  \l 22225
  \l 22226
  \l 22227
  \l 22228
  \l 22229
  \l 2222A
  \l 2222B
  \l 2222C
  \l 2222D
  \l 2222E
  \l 2222F
  \l 22230
  \l 22231
  \l 22232
  \l 22233
  \l 22234
  \l 22235
  \l 22236
  \l 22237
  \l 22238
  \l 22239
  \l 2223A
  \l 2223B
  \l 2223C
  \l 2223D
  \l 2223E
  \l 2223F
  \l 22240
  \l 22241
  \l 22242
  \l 22243
  \l 22244
  \l 22245
  \l 22246
  \l 22247
  \l 22248
  \l 22249
  \l 2224A
  \l 2224B
  \l 2224C
  \l 2224D
  \l 2224E
  \l 2224F
  \l 22250
  \l 22251
  \l 22252
  \l 22253
  \l 22254
  \l 22255
  \l 22256
  \l 22257
  \l 22258
  \l 22259
  \l 2225A
  \l 2225B
  \l 2225C
  \l 2225D
  \l 2225E
  \l 2225F
  \l 22260
  \l 22261
  \l 22262
  \l 22263
  \l 22264
  \l 22265
  \l 22266
  \l 22267
  \l 22268
  \l 22269
  \l 2226A
  \l 2226B
  \l 2226C
  \l 2226D
  \l 2226E
  \l 2226F
  \l 22270
  \l 22271
  \l 22272
  \l 22273
  \l 22274
  \l 22275
  \l 22276
  \l 22277
  \l 22278
  \l 22279
  \l 2227A
  \l 2227B
  \l 2227C
  \l 2227D
  \l 2227E
  \l 2227F
  \l 22280
  \l 22281
  \l 22282
  \l 22283
  \l 22284
  \l 22285
  \l 22286
  \l 22287
  \l 22288
  \l 22289
  \l 2228A
  \l 2228B
  \l 2228C
  \l 2228D
  \l 2228E
  \l 2228F
  \l 22290
  \l 22291
  \l 22292
  \l 22293
  \l 22294
  \l 22295
  \l 22296
  \l 22297
  \l 22298
  \l 22299
  \l 2229A
  \l 2229B
  \l 2229C
  \l 2229D
  \l 2229E
  \l 2229F
  \l 222A0
  \l 222A1
  \l 222A2
  \l 222A3
  \l 222A4
  \l 222A5
  \l 222A6
  \l 222A7
  \l 222A8
  \l 222A9
  \l 222AA
  \l 222AB
  \l 222AC
  \l 222AD
  \l 222AE
  \l 222AF
  \l 222B0
  \l 222B1
  \l 222B2
  \l 222B3
  \l 222B4
  \l 222B5
  \l 222B6
  \l 222B7
  \l 222B8
  \l 222B9
  \l 222BA
  \l 222BB
  \l 222BC
  \l 222BD
  \l 222BE
  \l 222BF
  \l 222C0
  \l 222C1
  \l 222C2
  \l 222C3
  \l 222C4
  \l 222C5
  \l 222C6
  \l 222C7
  \l 222C8
  \l 222C9
  \l 222CA
  \l 222CB
  \l 222CC
  \l 222CD
  \l 222CE
  \l 222CF
  \l 222D0
  \l 222D1
  \l 222D2
  \l 222D3
  \l 222D4
  \l 222D5
  \l 222D6
  \l 222D7
  \l 222D8
  \l 222D9
  \l 222DA
  \l 222DB
  \l 222DC
  \l 222DD
  \l 222DE
  \l 222DF
  \l 222E0
  \l 222E1
  \l 222E2
  \l 222E3
  \l 222E4
  \l 222E5
  \l 222E6
  \l 222E7
  \l 222E8
  \l 222E9
  \l 222EA
  \l 222EB
  \l 222EC
  \l 222ED
  \l 222EE
  \l 222EF
  \l 222F0
  \l 222F1
  \l 222F2
  \l 222F3
  \l 222F4
  \l 222F5
  \l 222F6
  \l 222F7
  \l 222F8
  \l 222F9
  \l 222FA
  \l 222FB
  \l 222FC
  \l 222FD
  \l 222FE
  \l 222FF
  \l 22300
  \l 22301
  \l 22302
  \l 22303
  \l 22304
  \l 22305
  \l 22306
  \l 22307
  \l 22308
  \l 22309
  \l 2230A
  \l 2230B
  \l 2230C
  \l 2230D
  \l 2230E
  \l 2230F
  \l 22310
  \l 22311
  \l 22312
  \l 22313
  \l 22314
  \l 22315
  \l 22316
  \l 22317
  \l 22318
  \l 22319
  \l 2231A
  \l 2231B
  \l 2231C
  \l 2231D
  \l 2231E
  \l 2231F
  \l 22320
  \l 22321
  \l 22322
  \l 22323
  \l 22324
  \l 22325
  \l 22326
  \l 22327
  \l 22328
  \l 22329
  \l 2232A
  \l 2232B
  \l 2232C
  \l 2232D
  \l 2232E
  \l 2232F
  \l 22330
  \l 22331
  \l 22332
  \l 22333
  \l 22334
  \l 22335
  \l 22336
  \l 22337
  \l 22338
  \l 22339
  \l 2233A
  \l 2233B
  \l 2233C
  \l 2233D
  \l 2233E
  \l 2233F
  \l 22340
  \l 22341
  \l 22342
  \l 22343
  \l 22344
  \l 22345
  \l 22346
  \l 22347
  \l 22348
  \l 22349
  \l 2234A
  \l 2234B
  \l 2234C
  \l 2234D
  \l 2234E
  \l 2234F
  \l 22350
  \l 22351
  \l 22352
  \l 22353
  \l 22354
  \l 22355
  \l 22356
  \l 22357
  \l 22358
  \l 22359
  \l 2235A
  \l 2235B
  \l 2235C
  \l 2235D
  \l 2235E
  \l 2235F
  \l 22360
  \l 22361
  \l 22362
  \l 22363
  \l 22364
  \l 22365
  \l 22366
  \l 22367
  \l 22368
  \l 22369
  \l 2236A
  \l 2236B
  \l 2236C
  \l 2236D
  \l 2236E
  \l 2236F
  \l 22370
  \l 22371
  \l 22372
  \l 22373
  \l 22374
  \l 22375
  \l 22376
  \l 22377
  \l 22378
  \l 22379
  \l 2237A
  \l 2237B
  \l 2237C
  \l 2237D
  \l 2237E
  \l 2237F
  \l 22380
  \l 22381
  \l 22382
  \l 22383
  \l 22384
  \l 22385
  \l 22386
  \l 22387
  \l 22388
  \l 22389
  \l 2238A
  \l 2238B
  \l 2238C
  \l 2238D
  \l 2238E
  \l 2238F
  \l 22390
  \l 22391
  \l 22392
  \l 22393
  \l 22394
  \l 22395
  \l 22396
  \l 22397
  \l 22398
  \l 22399
  \l 2239A
  \l 2239B
  \l 2239C
  \l 2239D
  \l 2239E
  \l 2239F
  \l 223A0
  \l 223A1
  \l 223A2
  \l 223A3
  \l 223A4
  \l 223A5
  \l 223A6
  \l 223A7
  \l 223A8
  \l 223A9
  \l 223AA
  \l 223AB
  \l 223AC
  \l 223AD
  \l 223AE
  \l 223AF
  \l 223B0
  \l 223B1
  \l 223B2
  \l 223B3
  \l 223B4
  \l 223B5
  \l 223B6
  \l 223B7
  \l 223B8
  \l 223B9
  \l 223BA
  \l 223BB
  \l 223BC
  \l 223BD
  \l 223BE
  \l 223BF
  \l 223C0
  \l 223C1
  \l 223C2
  \l 223C3
  \l 223C4
  \l 223C5
  \l 223C6
  \l 223C7
  \l 223C8
  \l 223C9
  \l 223CA
  \l 223CB
  \l 223CC
  \l 223CD
  \l 223CE
  \l 223CF
  \l 223D0
  \l 223D1
  \l 223D2
  \l 223D3
  \l 223D4
  \l 223D5
  \l 223D6
  \l 223D7
  \l 223D8
  \l 223D9
  \l 223DA
  \l 223DB
  \l 223DC
  \l 223DD
  \l 223DE
  \l 223DF
  \l 223E0
  \l 223E1
  \l 223E2
  \l 223E3
  \l 223E4
  \l 223E5
  \l 223E6
  \l 223E7
  \l 223E8
  \l 223E9
  \l 223EA
  \l 223EB
  \l 223EC
  \l 223ED
  \l 223EE
  \l 223EF
  \l 223F0
  \l 223F1
  \l 223F2
  \l 223F3
  \l 223F4
  \l 223F5
  \l 223F6
  \l 223F7
  \l 223F8
  \l 223F9
  \l 223FA
  \l 223FB
  \l 223FC
  \l 223FD
  \l 223FE
  \l 223FF
  \l 22400
  \l 22401
  \l 22402
  \l 22403
  \l 22404
  \l 22405
  \l 22406
  \l 22407
  \l 22408
  \l 22409
  \l 2240A
  \l 2240B
  \l 2240C
  \l 2240D
  \l 2240E
  \l 2240F
  \l 22410
  \l 22411
  \l 22412
  \l 22413
  \l 22414
  \l 22415
  \l 22416
  \l 22417
  \l 22418
  \l 22419
  \l 2241A
  \l 2241B
  \l 2241C
  \l 2241D
  \l 2241E
  \l 2241F
  \l 22420
  \l 22421
  \l 22422
  \l 22423
  \l 22424
  \l 22425
  \l 22426
  \l 22427
  \l 22428
  \l 22429
  \l 2242A
  \l 2242B
  \l 2242C
  \l 2242D
  \l 2242E
  \l 2242F
  \l 22430
  \l 22431
  \l 22432
  \l 22433
  \l 22434
  \l 22435
  \l 22436
  \l 22437
  \l 22438
  \l 22439
  \l 2243A
  \l 2243B
  \l 2243C
  \l 2243D
  \l 2243E
  \l 2243F
  \l 22440
  \l 22441
  \l 22442
  \l 22443
  \l 22444
  \l 22445
  \l 22446
  \l 22447
  \l 22448
  \l 22449
  \l 2244A
  \l 2244B
  \l 2244C
  \l 2244D
  \l 2244E
  \l 2244F
  \l 22450
  \l 22451
  \l 22452
  \l 22453
  \l 22454
  \l 22455
  \l 22456
  \l 22457
  \l 22458
  \l 22459
  \l 2245A
  \l 2245B
  \l 2245C
  \l 2245D
  \l 2245E
  \l 2245F
  \l 22460
  \l 22461
  \l 22462
  \l 22463
  \l 22464
  \l 22465
  \l 22466
  \l 22467
  \l 22468
  \l 22469
  \l 2246A
  \l 2246B
  \l 2246C
  \l 2246D
  \l 2246E
  \l 2246F
  \l 22470
  \l 22471
  \l 22472
  \l 22473
  \l 22474
  \l 22475
  \l 22476
  \l 22477
  \l 22478
  \l 22479
  \l 2247A
  \l 2247B
  \l 2247C
  \l 2247D
  \l 2247E
  \l 2247F
  \l 22480
  \l 22481
  \l 22482
  \l 22483
  \l 22484
  \l 22485
  \l 22486
  \l 22487
  \l 22488
  \l 22489
  \l 2248A
  \l 2248B
  \l 2248C
  \l 2248D
  \l 2248E
  \l 2248F
  \l 22490
  \l 22491
  \l 22492
  \l 22493
  \l 22494
  \l 22495
  \l 22496
  \l 22497
  \l 22498
  \l 22499
  \l 2249A
  \l 2249B
  \l 2249C
  \l 2249D
  \l 2249E
  \l 2249F
  \l 224A0
  \l 224A1
  \l 224A2
  \l 224A3
  \l 224A4
  \l 224A5
  \l 224A6
  \l 224A7
  \l 224A8
  \l 224A9
  \l 224AA
  \l 224AB
  \l 224AC
  \l 224AD
  \l 224AE
  \l 224AF
  \l 224B0
  \l 224B1
  \l 224B2
  \l 224B3
  \l 224B4
  \l 224B5
  \l 224B6
  \l 224B7
  \l 224B8
  \l 224B9
  \l 224BA
  \l 224BB
  \l 224BC
  \l 224BD
  \l 224BE
  \l 224BF
  \l 224C0
  \l 224C1
  \l 224C2
  \l 224C3
  \l 224C4
  \l 224C5
  \l 224C6
  \l 224C7
  \l 224C8
  \l 224C9
  \l 224CA
  \l 224CB
  \l 224CC
  \l 224CD
  \l 224CE
  \l 224CF
  \l 224D0
  \l 224D1
  \l 224D2
  \l 224D3
  \l 224D4
  \l 224D5
  \l 224D6
  \l 224D7
  \l 224D8
  \l 224D9
  \l 224DA
  \l 224DB
  \l 224DC
  \l 224DD
  \l 224DE
  \l 224DF
  \l 224E0
  \l 224E1
  \l 224E2
  \l 224E3
  \l 224E4
  \l 224E5
  \l 224E6
  \l 224E7
  \l 224E8
  \l 224E9
  \l 224EA
  \l 224EB
  \l 224EC
  \l 224ED
  \l 224EE
  \l 224EF
  \l 224F0
  \l 224F1
  \l 224F2
  \l 224F3
  \l 224F4
  \l 224F5
  \l 224F6
  \l 224F7
  \l 224F8
  \l 224F9
  \l 224FA
  \l 224FB
  \l 224FC
  \l 224FD
  \l 224FE
  \l 224FF
  \l 22500
  \l 22501
  \l 22502
  \l 22503
  \l 22504
  \l 22505
  \l 22506
  \l 22507
  \l 22508
  \l 22509
  \l 2250A
  \l 2250B
  \l 2250C
  \l 2250D
  \l 2250E
  \l 2250F
  \l 22510
  \l 22511
  \l 22512
  \l 22513
  \l 22514
  \l 22515
  \l 22516
  \l 22517
  \l 22518
  \l 22519
  \l 2251A
  \l 2251B
  \l 2251C
  \l 2251D
  \l 2251E
  \l 2251F
  \l 22520
  \l 22521
  \l 22522
  \l 22523
  \l 22524
  \l 22525
  \l 22526
  \l 22527
  \l 22528
  \l 22529
  \l 2252A
  \l 2252B
  \l 2252C
  \l 2252D
  \l 2252E
  \l 2252F
  \l 22530
  \l 22531
  \l 22532
  \l 22533
  \l 22534
  \l 22535
  \l 22536
  \l 22537
  \l 22538
  \l 22539
  \l 2253A
  \l 2253B
  \l 2253C
  \l 2253D
  \l 2253E
  \l 2253F
  \l 22540
  \l 22541
  \l 22542
  \l 22543
  \l 22544
  \l 22545
  \l 22546
  \l 22547
  \l 22548
  \l 22549
  \l 2254A
  \l 2254B
  \l 2254C
  \l 2254D
  \l 2254E
  \l 2254F
  \l 22550
  \l 22551
  \l 22552
  \l 22553
  \l 22554
  \l 22555
  \l 22556
  \l 22557
  \l 22558
  \l 22559
  \l 2255A
  \l 2255B
  \l 2255C
  \l 2255D
  \l 2255E
  \l 2255F
  \l 22560
  \l 22561
  \l 22562
  \l 22563
  \l 22564
  \l 22565
  \l 22566
  \l 22567
  \l 22568
  \l 22569
  \l 2256A
  \l 2256B
  \l 2256C
  \l 2256D
  \l 2256E
  \l 2256F
  \l 22570
  \l 22571
  \l 22572
  \l 22573
  \l 22574
  \l 22575
  \l 22576
  \l 22577
  \l 22578
  \l 22579
  \l 2257A
  \l 2257B
  \l 2257C
  \l 2257D
  \l 2257E
  \l 2257F
  \l 22580
  \l 22581
  \l 22582
  \l 22583
  \l 22584
  \l 22585
  \l 22586
  \l 22587
  \l 22588
  \l 22589
  \l 2258A
  \l 2258B
  \l 2258C
  \l 2258D
  \l 2258E
  \l 2258F
  \l 22590
  \l 22591
  \l 22592
  \l 22593
  \l 22594
  \l 22595
  \l 22596
  \l 22597
  \l 22598
  \l 22599
  \l 2259A
  \l 2259B
  \l 2259C
  \l 2259D
  \l 2259E
  \l 2259F
  \l 225A0
  \l 225A1
  \l 225A2
  \l 225A3
  \l 225A4
  \l 225A5
  \l 225A6
  \l 225A7
  \l 225A8
  \l 225A9
  \l 225AA
  \l 225AB
  \l 225AC
  \l 225AD
  \l 225AE
  \l 225AF
  \l 225B0
  \l 225B1
  \l 225B2
  \l 225B3
  \l 225B4
  \l 225B5
  \l 225B6
  \l 225B7
  \l 225B8
  \l 225B9
  \l 225BA
  \l 225BB
  \l 225BC
  \l 225BD
  \l 225BE
  \l 225BF
  \l 225C0
  \l 225C1
  \l 225C2
  \l 225C3
  \l 225C4
  \l 225C5
  \l 225C6
  \l 225C7
  \l 225C8
  \l 225C9
  \l 225CA
  \l 225CB
  \l 225CC
  \l 225CD
  \l 225CE
  \l 225CF
  \l 225D0
  \l 225D1
  \l 225D2
  \l 225D3
  \l 225D4
  \l 225D5
  \l 225D6
  \l 225D7
  \l 225D8
  \l 225D9
  \l 225DA
  \l 225DB
  \l 225DC
  \l 225DD
  \l 225DE
  \l 225DF
  \l 225E0
  \l 225E1
  \l 225E2
  \l 225E3
  \l 225E4
  \l 225E5
  \l 225E6
  \l 225E7
  \l 225E8
  \l 225E9
  \l 225EA
  \l 225EB
  \l 225EC
  \l 225ED
  \l 225EE
  \l 225EF
  \l 225F0
  \l 225F1
  \l 225F2
  \l 225F3
  \l 225F4
  \l 225F5
  \l 225F6
  \l 225F7
  \l 225F8
  \l 225F9
  \l 225FA
  \l 225FB
  \l 225FC
  \l 225FD
  \l 225FE
  \l 225FF
  \l 22600
  \l 22601
  \l 22602
  \l 22603
  \l 22604
  \l 22605
  \l 22606
  \l 22607
  \l 22608
  \l 22609
  \l 2260A
  \l 2260B
  \l 2260C
  \l 2260D
  \l 2260E
  \l 2260F
  \l 22610
  \l 22611
  \l 22612
  \l 22613
  \l 22614
  \l 22615
  \l 22616
  \l 22617
  \l 22618
  \l 22619
  \l 2261A
  \l 2261B
  \l 2261C
  \l 2261D
  \l 2261E
  \l 2261F
  \l 22620
  \l 22621
  \l 22622
  \l 22623
  \l 22624
  \l 22625
  \l 22626
  \l 22627
  \l 22628
  \l 22629
  \l 2262A
  \l 2262B
  \l 2262C
  \l 2262D
  \l 2262E
  \l 2262F
  \l 22630
  \l 22631
  \l 22632
  \l 22633
  \l 22634
  \l 22635
  \l 22636
  \l 22637
  \l 22638
  \l 22639
  \l 2263A
  \l 2263B
  \l 2263C
  \l 2263D
  \l 2263E
  \l 2263F
  \l 22640
  \l 22641
  \l 22642
  \l 22643
  \l 22644
  \l 22645
  \l 22646
  \l 22647
  \l 22648
  \l 22649
  \l 2264A
  \l 2264B
  \l 2264C
  \l 2264D
  \l 2264E
  \l 2264F
  \l 22650
  \l 22651
  \l 22652
  \l 22653
  \l 22654
  \l 22655
  \l 22656
  \l 22657
  \l 22658
  \l 22659
  \l 2265A
  \l 2265B
  \l 2265C
  \l 2265D
  \l 2265E
  \l 2265F
  \l 22660
  \l 22661
  \l 22662
  \l 22663
  \l 22664
  \l 22665
  \l 22666
  \l 22667
  \l 22668
  \l 22669
  \l 2266A
  \l 2266B
  \l 2266C
  \l 2266D
  \l 2266E
  \l 2266F
  \l 22670
  \l 22671
  \l 22672
  \l 22673
  \l 22674
  \l 22675
  \l 22676
  \l 22677
  \l 22678
  \l 22679
  \l 2267A
  \l 2267B
  \l 2267C
  \l 2267D
  \l 2267E
  \l 2267F
  \l 22680
  \l 22681
  \l 22682
  \l 22683
  \l 22684
  \l 22685
  \l 22686
  \l 22687
  \l 22688
  \l 22689
  \l 2268A
  \l 2268B
  \l 2268C
  \l 2268D
  \l 2268E
  \l 2268F
  \l 22690
  \l 22691
  \l 22692
  \l 22693
  \l 22694
  \l 22695
  \l 22696
  \l 22697
  \l 22698
  \l 22699
  \l 2269A
  \l 2269B
  \l 2269C
  \l 2269D
  \l 2269E
  \l 2269F
  \l 226A0
  \l 226A1
  \l 226A2
  \l 226A3
  \l 226A4
  \l 226A5
  \l 226A6
  \l 226A7
  \l 226A8
  \l 226A9
  \l 226AA
  \l 226AB
  \l 226AC
  \l 226AD
  \l 226AE
  \l 226AF
  \l 226B0
  \l 226B1
  \l 226B2
  \l 226B3
  \l 226B4
  \l 226B5
  \l 226B6
  \l 226B7
  \l 226B8
  \l 226B9
  \l 226BA
  \l 226BB
  \l 226BC
  \l 226BD
  \l 226BE
  \l 226BF
  \l 226C0
  \l 226C1
  \l 226C2
  \l 226C3
  \l 226C4
  \l 226C5
  \l 226C6
  \l 226C7
  \l 226C8
  \l 226C9
  \l 226CA
  \l 226CB
  \l 226CC
  \l 226CD
  \l 226CE
  \l 226CF
  \l 226D0
  \l 226D1
  \l 226D2
  \l 226D3
  \l 226D4
  \l 226D5
  \l 226D6
  \l 226D7
  \l 226D8
  \l 226D9
  \l 226DA
  \l 226DB
  \l 226DC
  \l 226DD
  \l 226DE
  \l 226DF
  \l 226E0
  \l 226E1
  \l 226E2
  \l 226E3
  \l 226E4
  \l 226E5
  \l 226E6
  \l 226E7
  \l 226E8
  \l 226E9
  \l 226EA
  \l 226EB
  \l 226EC
  \l 226ED
  \l 226EE
  \l 226EF
  \l 226F0
  \l 226F1
  \l 226F2
  \l 226F3
  \l 226F4
  \l 226F5
  \l 226F6
  \l 226F7
  \l 226F8
  \l 226F9
  \l 226FA
  \l 226FB
  \l 226FC
  \l 226FD
  \l 226FE
  \l 226FF
  \l 22700
  \l 22701
  \l 22702
  \l 22703
  \l 22704
  \l 22705
  \l 22706
  \l 22707
  \l 22708
  \l 22709
  \l 2270A
  \l 2270B
  \l 2270C
  \l 2270D
  \l 2270E
  \l 2270F
  \l 22710
  \l 22711
  \l 22712
  \l 22713
  \l 22714
  \l 22715
  \l 22716
  \l 22717
  \l 22718
  \l 22719
  \l 2271A
  \l 2271B
  \l 2271C
  \l 2271D
  \l 2271E
  \l 2271F
  \l 22720
  \l 22721
  \l 22722
  \l 22723
  \l 22724
  \l 22725
  \l 22726
  \l 22727
  \l 22728
  \l 22729
  \l 2272A
  \l 2272B
  \l 2272C
  \l 2272D
  \l 2272E
  \l 2272F
  \l 22730
  \l 22731
  \l 22732
  \l 22733
  \l 22734
  \l 22735
  \l 22736
  \l 22737
  \l 22738
  \l 22739
  \l 2273A
  \l 2273B
  \l 2273C
  \l 2273D
  \l 2273E
  \l 2273F
  \l 22740
  \l 22741
  \l 22742
  \l 22743
  \l 22744
  \l 22745
  \l 22746
  \l 22747
  \l 22748
  \l 22749
  \l 2274A
  \l 2274B
  \l 2274C
  \l 2274D
  \l 2274E
  \l 2274F
  \l 22750
  \l 22751
  \l 22752
  \l 22753
  \l 22754
  \l 22755
  \l 22756
  \l 22757
  \l 22758
  \l 22759
  \l 2275A
  \l 2275B
  \l 2275C
  \l 2275D
  \l 2275E
  \l 2275F
  \l 22760
  \l 22761
  \l 22762
  \l 22763
  \l 22764
  \l 22765
  \l 22766
  \l 22767
  \l 22768
  \l 22769
  \l 2276A
  \l 2276B
  \l 2276C
  \l 2276D
  \l 2276E
  \l 2276F
  \l 22770
  \l 22771
  \l 22772
  \l 22773
  \l 22774
  \l 22775
  \l 22776
  \l 22777
  \l 22778
  \l 22779
  \l 2277A
  \l 2277B
  \l 2277C
  \l 2277D
  \l 2277E
  \l 2277F
  \l 22780
  \l 22781
  \l 22782
  \l 22783
  \l 22784
  \l 22785
  \l 22786
  \l 22787
  \l 22788
  \l 22789
  \l 2278A
  \l 2278B
  \l 2278C
  \l 2278D
  \l 2278E
  \l 2278F
  \l 22790
  \l 22791
  \l 22792
  \l 22793
  \l 22794
  \l 22795
  \l 22796
  \l 22797
  \l 22798
  \l 22799
  \l 2279A
  \l 2279B
  \l 2279C
  \l 2279D
  \l 2279E
  \l 2279F
  \l 227A0
  \l 227A1
  \l 227A2
  \l 227A3
  \l 227A4
  \l 227A5
  \l 227A6
  \l 227A7
  \l 227A8
  \l 227A9
  \l 227AA
  \l 227AB
  \l 227AC
  \l 227AD
  \l 227AE
  \l 227AF
  \l 227B0
  \l 227B1
  \l 227B2
  \l 227B3
  \l 227B4
  \l 227B5
  \l 227B6
  \l 227B7
  \l 227B8
  \l 227B9
  \l 227BA
  \l 227BB
  \l 227BC
  \l 227BD
  \l 227BE
  \l 227BF
  \l 227C0
  \l 227C1
  \l 227C2
  \l 227C3
  \l 227C4
  \l 227C5
  \l 227C6
  \l 227C7
  \l 227C8
  \l 227C9
  \l 227CA
  \l 227CB
  \l 227CC
  \l 227CD
  \l 227CE
  \l 227CF
  \l 227D0
  \l 227D1
  \l 227D2
  \l 227D3
  \l 227D4
  \l 227D5
  \l 227D6
  \l 227D7
  \l 227D8
  \l 227D9
  \l 227DA
  \l 227DB
  \l 227DC
  \l 227DD
  \l 227DE
  \l 227DF
  \l 227E0
  \l 227E1
  \l 227E2
  \l 227E3
  \l 227E4
  \l 227E5
  \l 227E6
  \l 227E7
  \l 227E8
  \l 227E9
  \l 227EA
  \l 227EB
  \l 227EC
  \l 227ED
  \l 227EE
  \l 227EF
  \l 227F0
  \l 227F1
  \l 227F2
  \l 227F3
  \l 227F4
  \l 227F5
  \l 227F6
  \l 227F7
  \l 227F8
  \l 227F9
  \l 227FA
  \l 227FB
  \l 227FC
  \l 227FD
  \l 227FE
  \l 227FF
  \l 22800
  \l 22801
  \l 22802
  \l 22803
  \l 22804
  \l 22805
  \l 22806
  \l 22807
  \l 22808
  \l 22809
  \l 2280A
  \l 2280B
  \l 2280C
  \l 2280D
  \l 2280E
  \l 2280F
  \l 22810
  \l 22811
  \l 22812
  \l 22813
  \l 22814
  \l 22815
  \l 22816
  \l 22817
  \l 22818
  \l 22819
  \l 2281A
  \l 2281B
  \l 2281C
  \l 2281D
  \l 2281E
  \l 2281F
  \l 22820
  \l 22821
  \l 22822
  \l 22823
  \l 22824
  \l 22825
  \l 22826
  \l 22827
  \l 22828
  \l 22829
  \l 2282A
  \l 2282B
  \l 2282C
  \l 2282D
  \l 2282E
  \l 2282F
  \l 22830
  \l 22831
  \l 22832
  \l 22833
  \l 22834
  \l 22835
  \l 22836
  \l 22837
  \l 22838
  \l 22839
  \l 2283A
  \l 2283B
  \l 2283C
  \l 2283D
  \l 2283E
  \l 2283F
  \l 22840
  \l 22841
  \l 22842
  \l 22843
  \l 22844
  \l 22845
  \l 22846
  \l 22847
  \l 22848
  \l 22849
  \l 2284A
  \l 2284B
  \l 2284C
  \l 2284D
  \l 2284E
  \l 2284F
  \l 22850
  \l 22851
  \l 22852
  \l 22853
  \l 22854
  \l 22855
  \l 22856
  \l 22857
  \l 22858
  \l 22859
  \l 2285A
  \l 2285B
  \l 2285C
  \l 2285D
  \l 2285E
  \l 2285F
  \l 22860
  \l 22861
  \l 22862
  \l 22863
  \l 22864
  \l 22865
  \l 22866
  \l 22867
  \l 22868
  \l 22869
  \l 2286A
  \l 2286B
  \l 2286C
  \l 2286D
  \l 2286E
  \l 2286F
  \l 22870
  \l 22871
  \l 22872
  \l 22873
  \l 22874
  \l 22875
  \l 22876
  \l 22877
  \l 22878
  \l 22879
  \l 2287A
  \l 2287B
  \l 2287C
  \l 2287D
  \l 2287E
  \l 2287F
  \l 22880
  \l 22881
  \l 22882
  \l 22883
  \l 22884
  \l 22885
  \l 22886
  \l 22887
  \l 22888
  \l 22889
  \l 2288A
  \l 2288B
  \l 2288C
  \l 2288D
  \l 2288E
  \l 2288F
  \l 22890
  \l 22891
  \l 22892
  \l 22893
  \l 22894
  \l 22895
  \l 22896
  \l 22897
  \l 22898
  \l 22899
  \l 2289A
  \l 2289B
  \l 2289C
  \l 2289D
  \l 2289E
  \l 2289F
  \l 228A0
  \l 228A1
  \l 228A2
  \l 228A3
  \l 228A4
  \l 228A5
  \l 228A6
  \l 228A7
  \l 228A8
  \l 228A9
  \l 228AA
  \l 228AB
  \l 228AC
  \l 228AD
  \l 228AE
  \l 228AF
  \l 228B0
  \l 228B1
  \l 228B2
  \l 228B3
  \l 228B4
  \l 228B5
  \l 228B6
  \l 228B7
  \l 228B8
  \l 228B9
  \l 228BA
  \l 228BB
  \l 228BC
  \l 228BD
  \l 228BE
  \l 228BF
  \l 228C0
  \l 228C1
  \l 228C2
  \l 228C3
  \l 228C4
  \l 228C5
  \l 228C6
  \l 228C7
  \l 228C8
  \l 228C9
  \l 228CA
  \l 228CB
  \l 228CC
  \l 228CD
  \l 228CE
  \l 228CF
  \l 228D0
  \l 228D1
  \l 228D2
  \l 228D3
  \l 228D4
  \l 228D5
  \l 228D6
  \l 228D7
  \l 228D8
  \l 228D9
  \l 228DA
  \l 228DB
  \l 228DC
  \l 228DD
  \l 228DE
  \l 228DF
  \l 228E0
  \l 228E1
  \l 228E2
  \l 228E3
  \l 228E4
  \l 228E5
  \l 228E6
  \l 228E7
  \l 228E8
  \l 228E9
  \l 228EA
  \l 228EB
  \l 228EC
  \l 228ED
  \l 228EE
  \l 228EF
  \l 228F0
  \l 228F1
  \l 228F2
  \l 228F3
  \l 228F4
  \l 228F5
  \l 228F6
  \l 228F7
  \l 228F8
  \l 228F9
  \l 228FA
  \l 228FB
  \l 228FC
  \l 228FD
  \l 228FE
  \l 228FF
  \l 22900
  \l 22901
  \l 22902
  \l 22903
  \l 22904
  \l 22905
  \l 22906
  \l 22907
  \l 22908
  \l 22909
  \l 2290A
  \l 2290B
  \l 2290C
  \l 2290D
  \l 2290E
  \l 2290F
  \l 22910
  \l 22911
  \l 22912
  \l 22913
  \l 22914
  \l 22915
  \l 22916
  \l 22917
  \l 22918
  \l 22919
  \l 2291A
  \l 2291B
  \l 2291C
  \l 2291D
  \l 2291E
  \l 2291F
  \l 22920
  \l 22921
  \l 22922
  \l 22923
  \l 22924
  \l 22925
  \l 22926
  \l 22927
  \l 22928
  \l 22929
  \l 2292A
  \l 2292B
  \l 2292C
  \l 2292D
  \l 2292E
  \l 2292F
  \l 22930
  \l 22931
  \l 22932
  \l 22933
  \l 22934
  \l 22935
  \l 22936
  \l 22937
  \l 22938
  \l 22939
  \l 2293A
  \l 2293B
  \l 2293C
  \l 2293D
  \l 2293E
  \l 2293F
  \l 22940
  \l 22941
  \l 22942
  \l 22943
  \l 22944
  \l 22945
  \l 22946
  \l 22947
  \l 22948
  \l 22949
  \l 2294A
  \l 2294B
  \l 2294C
  \l 2294D
  \l 2294E
  \l 2294F
  \l 22950
  \l 22951
  \l 22952
  \l 22953
  \l 22954
  \l 22955
  \l 22956
  \l 22957
  \l 22958
  \l 22959
  \l 2295A
  \l 2295B
  \l 2295C
  \l 2295D
  \l 2295E
  \l 2295F
  \l 22960
  \l 22961
  \l 22962
  \l 22963
  \l 22964
  \l 22965
  \l 22966
  \l 22967
  \l 22968
  \l 22969
  \l 2296A
  \l 2296B
  \l 2296C
  \l 2296D
  \l 2296E
  \l 2296F
  \l 22970
  \l 22971
  \l 22972
  \l 22973
  \l 22974
  \l 22975
  \l 22976
  \l 22977
  \l 22978
  \l 22979
  \l 2297A
  \l 2297B
  \l 2297C
  \l 2297D
  \l 2297E
  \l 2297F
  \l 22980
  \l 22981
  \l 22982
  \l 22983
  \l 22984
  \l 22985
  \l 22986
  \l 22987
  \l 22988
  \l 22989
  \l 2298A
  \l 2298B
  \l 2298C
  \l 2298D
  \l 2298E
  \l 2298F
  \l 22990
  \l 22991
  \l 22992
  \l 22993
  \l 22994
  \l 22995
  \l 22996
  \l 22997
  \l 22998
  \l 22999
  \l 2299A
  \l 2299B
  \l 2299C
  \l 2299D
  \l 2299E
  \l 2299F
  \l 229A0
  \l 229A1
  \l 229A2
  \l 229A3
  \l 229A4
  \l 229A5
  \l 229A6
  \l 229A7
  \l 229A8
  \l 229A9
  \l 229AA
  \l 229AB
  \l 229AC
  \l 229AD
  \l 229AE
  \l 229AF
  \l 229B0
  \l 229B1
  \l 229B2
  \l 229B3
  \l 229B4
  \l 229B5
  \l 229B6
  \l 229B7
  \l 229B8
  \l 229B9
  \l 229BA
  \l 229BB
  \l 229BC
  \l 229BD
  \l 229BE
  \l 229BF
  \l 229C0
  \l 229C1
  \l 229C2
  \l 229C3
  \l 229C4
  \l 229C5
  \l 229C6
  \l 229C7
  \l 229C8
  \l 229C9
  \l 229CA
  \l 229CB
  \l 229CC
  \l 229CD
  \l 229CE
  \l 229CF
  \l 229D0
  \l 229D1
  \l 229D2
  \l 229D3
  \l 229D4
  \l 229D5
  \l 229D6
  \l 229D7
  \l 229D8
  \l 229D9
  \l 229DA
  \l 229DB
  \l 229DC
  \l 229DD
  \l 229DE
  \l 229DF
  \l 229E0
  \l 229E1
  \l 229E2
  \l 229E3
  \l 229E4
  \l 229E5
  \l 229E6
  \l 229E7
  \l 229E8
  \l 229E9
  \l 229EA
  \l 229EB
  \l 229EC
  \l 229ED
  \l 229EE
  \l 229EF
  \l 229F0
  \l 229F1
  \l 229F2
  \l 229F3
  \l 229F4
  \l 229F5
  \l 229F6
  \l 229F7
  \l 229F8
  \l 229F9
  \l 229FA
  \l 229FB
  \l 229FC
  \l 229FD
  \l 229FE
  \l 229FF
  \l 22A00
  \l 22A01
  \l 22A02
  \l 22A03
  \l 22A04
  \l 22A05
  \l 22A06
  \l 22A07
  \l 22A08
  \l 22A09
  \l 22A0A
  \l 22A0B
  \l 22A0C
  \l 22A0D
  \l 22A0E
  \l 22A0F
  \l 22A10
  \l 22A11
  \l 22A12
  \l 22A13
  \l 22A14
  \l 22A15
  \l 22A16
  \l 22A17
  \l 22A18
  \l 22A19
  \l 22A1A
  \l 22A1B
  \l 22A1C
  \l 22A1D
  \l 22A1E
  \l 22A1F
  \l 22A20
  \l 22A21
  \l 22A22
  \l 22A23
  \l 22A24
  \l 22A25
  \l 22A26
  \l 22A27
  \l 22A28
  \l 22A29
  \l 22A2A
  \l 22A2B
  \l 22A2C
  \l 22A2D
  \l 22A2E
  \l 22A2F
  \l 22A30
  \l 22A31
  \l 22A32
  \l 22A33
  \l 22A34
  \l 22A35
  \l 22A36
  \l 22A37
  \l 22A38
  \l 22A39
  \l 22A3A
  \l 22A3B
  \l 22A3C
  \l 22A3D
  \l 22A3E
  \l 22A3F
  \l 22A40
  \l 22A41
  \l 22A42
  \l 22A43
  \l 22A44
  \l 22A45
  \l 22A46
  \l 22A47
  \l 22A48
  \l 22A49
  \l 22A4A
  \l 22A4B
  \l 22A4C
  \l 22A4D
  \l 22A4E
  \l 22A4F
  \l 22A50
  \l 22A51
  \l 22A52
  \l 22A53
  \l 22A54
  \l 22A55
  \l 22A56
  \l 22A57
  \l 22A58
  \l 22A59
  \l 22A5A
  \l 22A5B
  \l 22A5C
  \l 22A5D
  \l 22A5E
  \l 22A5F
  \l 22A60
  \l 22A61
  \l 22A62
  \l 22A63
  \l 22A64
  \l 22A65
  \l 22A66
  \l 22A67
  \l 22A68
  \l 22A69
  \l 22A6A
  \l 22A6B
  \l 22A6C
  \l 22A6D
  \l 22A6E
  \l 22A6F
  \l 22A70
  \l 22A71
  \l 22A72
  \l 22A73
  \l 22A74
  \l 22A75
  \l 22A76
  \l 22A77
  \l 22A78
  \l 22A79
  \l 22A7A
  \l 22A7B
  \l 22A7C
  \l 22A7D
  \l 22A7E
  \l 22A7F
  \l 22A80
  \l 22A81
  \l 22A82
  \l 22A83
  \l 22A84
  \l 22A85
  \l 22A86
  \l 22A87
  \l 22A88
  \l 22A89
  \l 22A8A
  \l 22A8B
  \l 22A8C
  \l 22A8D
  \l 22A8E
  \l 22A8F
  \l 22A90
  \l 22A91
  \l 22A92
  \l 22A93
  \l 22A94
  \l 22A95
  \l 22A96
  \l 22A97
  \l 22A98
  \l 22A99
  \l 22A9A
  \l 22A9B
  \l 22A9C
  \l 22A9D
  \l 22A9E
  \l 22A9F
  \l 22AA0
  \l 22AA1
  \l 22AA2
  \l 22AA3
  \l 22AA4
  \l 22AA5
  \l 22AA6
  \l 22AA7
  \l 22AA8
  \l 22AA9
  \l 22AAA
  \l 22AAB
  \l 22AAC
  \l 22AAD
  \l 22AAE
  \l 22AAF
  \l 22AB0
  \l 22AB1
  \l 22AB2
  \l 22AB3
  \l 22AB4
  \l 22AB5
  \l 22AB6
  \l 22AB7
  \l 22AB8
  \l 22AB9
  \l 22ABA
  \l 22ABB
  \l 22ABC
  \l 22ABD
  \l 22ABE
  \l 22ABF
  \l 22AC0
  \l 22AC1
  \l 22AC2
  \l 22AC3
  \l 22AC4
  \l 22AC5
  \l 22AC6
  \l 22AC7
  \l 22AC8
  \l 22AC9
  \l 22ACA
  \l 22ACB
  \l 22ACC
  \l 22ACD
  \l 22ACE
  \l 22ACF
  \l 22AD0
  \l 22AD1
  \l 22AD2
  \l 22AD3
  \l 22AD4
  \l 22AD5
  \l 22AD6
  \l 22AD7
  \l 22AD8
  \l 22AD9
  \l 22ADA
  \l 22ADB
  \l 22ADC
  \l 22ADD
  \l 22ADE
  \l 22ADF
  \l 22AE0
  \l 22AE1
  \l 22AE2
  \l 22AE3
  \l 22AE4
  \l 22AE5
  \l 22AE6
  \l 22AE7
  \l 22AE8
  \l 22AE9
  \l 22AEA
  \l 22AEB
  \l 22AEC
  \l 22AED
  \l 22AEE
  \l 22AEF
  \l 22AF0
  \l 22AF1
  \l 22AF2
  \l 22AF3
  \l 22AF4
  \l 22AF5
  \l 22AF6
  \l 22AF7
  \l 22AF8
  \l 22AF9
  \l 22AFA
  \l 22AFB
  \l 22AFC
  \l 22AFD
  \l 22AFE
  \l 22AFF
  \l 22B00
  \l 22B01
  \l 22B02
  \l 22B03
  \l 22B04
  \l 22B05
  \l 22B06
  \l 22B07
  \l 22B08
  \l 22B09
  \l 22B0A
  \l 22B0B
  \l 22B0C
  \l 22B0D
  \l 22B0E
  \l 22B0F
  \l 22B10
  \l 22B11
  \l 22B12
  \l 22B13
  \l 22B14
  \l 22B15
  \l 22B16
  \l 22B17
  \l 22B18
  \l 22B19
  \l 22B1A
  \l 22B1B
  \l 22B1C
  \l 22B1D
  \l 22B1E
  \l 22B1F
  \l 22B20
  \l 22B21
  \l 22B22
  \l 22B23
  \l 22B24
  \l 22B25
  \l 22B26
  \l 22B27
  \l 22B28
  \l 22B29
  \l 22B2A
  \l 22B2B
  \l 22B2C
  \l 22B2D
  \l 22B2E
  \l 22B2F
  \l 22B30
  \l 22B31
  \l 22B32
  \l 22B33
  \l 22B34
  \l 22B35
  \l 22B36
  \l 22B37
  \l 22B38
  \l 22B39
  \l 22B3A
  \l 22B3B
  \l 22B3C
  \l 22B3D
  \l 22B3E
  \l 22B3F
  \l 22B40
  \l 22B41
  \l 22B42
  \l 22B43
  \l 22B44
  \l 22B45
  \l 22B46
  \l 22B47
  \l 22B48
  \l 22B49
  \l 22B4A
  \l 22B4B
  \l 22B4C
  \l 22B4D
  \l 22B4E
  \l 22B4F
  \l 22B50
  \l 22B51
  \l 22B52
  \l 22B53
  \l 22B54
  \l 22B55
  \l 22B56
  \l 22B57
  \l 22B58
  \l 22B59
  \l 22B5A
  \l 22B5B
  \l 22B5C
  \l 22B5D
  \l 22B5E
  \l 22B5F
  \l 22B60
  \l 22B61
  \l 22B62
  \l 22B63
  \l 22B64
  \l 22B65
  \l 22B66
  \l 22B67
  \l 22B68
  \l 22B69
  \l 22B6A
  \l 22B6B
  \l 22B6C
  \l 22B6D
  \l 22B6E
  \l 22B6F
  \l 22B70
  \l 22B71
  \l 22B72
  \l 22B73
  \l 22B74
  \l 22B75
  \l 22B76
  \l 22B77
  \l 22B78
  \l 22B79
  \l 22B7A
  \l 22B7B
  \l 22B7C
  \l 22B7D
  \l 22B7E
  \l 22B7F
  \l 22B80
  \l 22B81
  \l 22B82
  \l 22B83
  \l 22B84
  \l 22B85
  \l 22B86
  \l 22B87
  \l 22B88
  \l 22B89
  \l 22B8A
  \l 22B8B
  \l 22B8C
  \l 22B8D
  \l 22B8E
  \l 22B8F
  \l 22B90
  \l 22B91
  \l 22B92
  \l 22B93
  \l 22B94
  \l 22B95
  \l 22B96
  \l 22B97
  \l 22B98
  \l 22B99
  \l 22B9A
  \l 22B9B
  \l 22B9C
  \l 22B9D
  \l 22B9E
  \l 22B9F
  \l 22BA0
  \l 22BA1
  \l 22BA2
  \l 22BA3
  \l 22BA4
  \l 22BA5
  \l 22BA6
  \l 22BA7
  \l 22BA8
  \l 22BA9
  \l 22BAA
  \l 22BAB
  \l 22BAC
  \l 22BAD
  \l 22BAE
  \l 22BAF
  \l 22BB0
  \l 22BB1
  \l 22BB2
  \l 22BB3
  \l 22BB4
  \l 22BB5
  \l 22BB6
  \l 22BB7
  \l 22BB8
  \l 22BB9
  \l 22BBA
  \l 22BBB
  \l 22BBC
  \l 22BBD
  \l 22BBE
  \l 22BBF
  \l 22BC0
  \l 22BC1
  \l 22BC2
  \l 22BC3
  \l 22BC4
  \l 22BC5
  \l 22BC6
  \l 22BC7
  \l 22BC8
  \l 22BC9
  \l 22BCA
  \l 22BCB
  \l 22BCC
  \l 22BCD
  \l 22BCE
  \l 22BCF
  \l 22BD0
  \l 22BD1
  \l 22BD2
  \l 22BD3
  \l 22BD4
  \l 22BD5
  \l 22BD6
  \l 22BD7
  \l 22BD8
  \l 22BD9
  \l 22BDA
  \l 22BDB
  \l 22BDC
  \l 22BDD
  \l 22BDE
  \l 22BDF
  \l 22BE0
  \l 22BE1
  \l 22BE2
  \l 22BE3
  \l 22BE4
  \l 22BE5
  \l 22BE6
  \l 22BE7
  \l 22BE8
  \l 22BE9
  \l 22BEA
  \l 22BEB
  \l 22BEC
  \l 22BED
  \l 22BEE
  \l 22BEF
  \l 22BF0
  \l 22BF1
  \l 22BF2
  \l 22BF3
  \l 22BF4
  \l 22BF5
  \l 22BF6
  \l 22BF7
  \l 22BF8
  \l 22BF9
  \l 22BFA
  \l 22BFB
  \l 22BFC
  \l 22BFD
  \l 22BFE
  \l 22BFF
  \l 22C00
  \l 22C01
  \l 22C02
  \l 22C03
  \l 22C04
  \l 22C05
  \l 22C06
  \l 22C07
  \l 22C08
  \l 22C09
  \l 22C0A
  \l 22C0B
  \l 22C0C
  \l 22C0D
  \l 22C0E
  \l 22C0F
  \l 22C10
  \l 22C11
  \l 22C12
  \l 22C13
  \l 22C14
  \l 22C15
  \l 22C16
  \l 22C17
  \l 22C18
  \l 22C19
  \l 22C1A
  \l 22C1B
  \l 22C1C
  \l 22C1D
  \l 22C1E
  \l 22C1F
  \l 22C20
  \l 22C21
  \l 22C22
  \l 22C23
  \l 22C24
  \l 22C25
  \l 22C26
  \l 22C27
  \l 22C28
  \l 22C29
  \l 22C2A
  \l 22C2B
  \l 22C2C
  \l 22C2D
  \l 22C2E
  \l 22C2F
  \l 22C30
  \l 22C31
  \l 22C32
  \l 22C33
  \l 22C34
  \l 22C35
  \l 22C36
  \l 22C37
  \l 22C38
  \l 22C39
  \l 22C3A
  \l 22C3B
  \l 22C3C
  \l 22C3D
  \l 22C3E
  \l 22C3F
  \l 22C40
  \l 22C41
  \l 22C42
  \l 22C43
  \l 22C44
  \l 22C45
  \l 22C46
  \l 22C47
  \l 22C48
  \l 22C49
  \l 22C4A
  \l 22C4B
  \l 22C4C
  \l 22C4D
  \l 22C4E
  \l 22C4F
  \l 22C50
  \l 22C51
  \l 22C52
  \l 22C53
  \l 22C54
  \l 22C55
  \l 22C56
  \l 22C57
  \l 22C58
  \l 22C59
  \l 22C5A
  \l 22C5B
  \l 22C5C
  \l 22C5D
  \l 22C5E
  \l 22C5F
  \l 22C60
  \l 22C61
  \l 22C62
  \l 22C63
  \l 22C64
  \l 22C65
  \l 22C66
  \l 22C67
  \l 22C68
  \l 22C69
  \l 22C6A
  \l 22C6B
  \l 22C6C
  \l 22C6D
  \l 22C6E
  \l 22C6F
  \l 22C70
  \l 22C71
  \l 22C72
  \l 22C73
  \l 22C74
  \l 22C75
  \l 22C76
  \l 22C77
  \l 22C78
  \l 22C79
  \l 22C7A
  \l 22C7B
  \l 22C7C
  \l 22C7D
  \l 22C7E
  \l 22C7F
  \l 22C80
  \l 22C81
  \l 22C82
  \l 22C83
  \l 22C84
  \l 22C85
  \l 22C86
  \l 22C87
  \l 22C88
  \l 22C89
  \l 22C8A
  \l 22C8B
  \l 22C8C
  \l 22C8D
  \l 22C8E
  \l 22C8F
  \l 22C90
  \l 22C91
  \l 22C92
  \l 22C93
  \l 22C94
  \l 22C95
  \l 22C96
  \l 22C97
  \l 22C98
  \l 22C99
  \l 22C9A
  \l 22C9B
  \l 22C9C
  \l 22C9D
  \l 22C9E
  \l 22C9F
  \l 22CA0
  \l 22CA1
  \l 22CA2
  \l 22CA3
  \l 22CA4
  \l 22CA5
  \l 22CA6
  \l 22CA7
  \l 22CA8
  \l 22CA9
  \l 22CAA
  \l 22CAB
  \l 22CAC
  \l 22CAD
  \l 22CAE
  \l 22CAF
  \l 22CB0
  \l 22CB1
  \l 22CB2
  \l 22CB3
  \l 22CB4
  \l 22CB5
  \l 22CB6
  \l 22CB7
  \l 22CB8
  \l 22CB9
  \l 22CBA
  \l 22CBB
  \l 22CBC
  \l 22CBD
  \l 22CBE
  \l 22CBF
  \l 22CC0
  \l 22CC1
  \l 22CC2
  \l 22CC3
  \l 22CC4
  \l 22CC5
  \l 22CC6
  \l 22CC7
  \l 22CC8
  \l 22CC9
  \l 22CCA
  \l 22CCB
  \l 22CCC
  \l 22CCD
  \l 22CCE
  \l 22CCF
  \l 22CD0
  \l 22CD1
  \l 22CD2
  \l 22CD3
  \l 22CD4
  \l 22CD5
  \l 22CD6
  \l 22CD7
  \l 22CD8
  \l 22CD9
  \l 22CDA
  \l 22CDB
  \l 22CDC
  \l 22CDD
  \l 22CDE
  \l 22CDF
  \l 22CE0
  \l 22CE1
  \l 22CE2
  \l 22CE3
  \l 22CE4
  \l 22CE5
  \l 22CE6
  \l 22CE7
  \l 22CE8
  \l 22CE9
  \l 22CEA
  \l 22CEB
  \l 22CEC
  \l 22CED
  \l 22CEE
  \l 22CEF
  \l 22CF0
  \l 22CF1
  \l 22CF2
  \l 22CF3
  \l 22CF4
  \l 22CF5
  \l 22CF6
  \l 22CF7
  \l 22CF8
  \l 22CF9
  \l 22CFA
  \l 22CFB
  \l 22CFC
  \l 22CFD
  \l 22CFE
  \l 22CFF
  \l 22D00
  \l 22D01
  \l 22D02
  \l 22D03
  \l 22D04
  \l 22D05
  \l 22D06
  \l 22D07
  \l 22D08
  \l 22D09
  \l 22D0A
  \l 22D0B
  \l 22D0C
  \l 22D0D
  \l 22D0E
  \l 22D0F
  \l 22D10
  \l 22D11
  \l 22D12
  \l 22D13
  \l 22D14
  \l 22D15
  \l 22D16
  \l 22D17
  \l 22D18
  \l 22D19
  \l 22D1A
  \l 22D1B
  \l 22D1C
  \l 22D1D
  \l 22D1E
  \l 22D1F
  \l 22D20
  \l 22D21
  \l 22D22
  \l 22D23
  \l 22D24
  \l 22D25
  \l 22D26
  \l 22D27
  \l 22D28
  \l 22D29
  \l 22D2A
  \l 22D2B
  \l 22D2C
  \l 22D2D
  \l 22D2E
  \l 22D2F
  \l 22D30
  \l 22D31
  \l 22D32
  \l 22D33
  \l 22D34
  \l 22D35
  \l 22D36
  \l 22D37
  \l 22D38
  \l 22D39
  \l 22D3A
  \l 22D3B
  \l 22D3C
  \l 22D3D
  \l 22D3E
  \l 22D3F
  \l 22D40
  \l 22D41
  \l 22D42
  \l 22D43
  \l 22D44
  \l 22D45
  \l 22D46
  \l 22D47
  \l 22D48
  \l 22D49
  \l 22D4A
  \l 22D4B
  \l 22D4C
  \l 22D4D
  \l 22D4E
  \l 22D4F
  \l 22D50
  \l 22D51
  \l 22D52
  \l 22D53
  \l 22D54
  \l 22D55
  \l 22D56
  \l 22D57
  \l 22D58
  \l 22D59
  \l 22D5A
  \l 22D5B
  \l 22D5C
  \l 22D5D
  \l 22D5E
  \l 22D5F
  \l 22D60
  \l 22D61
  \l 22D62
  \l 22D63
  \l 22D64
  \l 22D65
  \l 22D66
  \l 22D67
  \l 22D68
  \l 22D69
  \l 22D6A
  \l 22D6B
  \l 22D6C
  \l 22D6D
  \l 22D6E
  \l 22D6F
  \l 22D70
  \l 22D71
  \l 22D72
  \l 22D73
  \l 22D74
  \l 22D75
  \l 22D76
  \l 22D77
  \l 22D78
  \l 22D79
  \l 22D7A
  \l 22D7B
  \l 22D7C
  \l 22D7D
  \l 22D7E
  \l 22D7F
  \l 22D80
  \l 22D81
  \l 22D82
  \l 22D83
  \l 22D84
  \l 22D85
  \l 22D86
  \l 22D87
  \l 22D88
  \l 22D89
  \l 22D8A
  \l 22D8B
  \l 22D8C
  \l 22D8D
  \l 22D8E
  \l 22D8F
  \l 22D90
  \l 22D91
  \l 22D92
  \l 22D93
  \l 22D94
  \l 22D95
  \l 22D96
  \l 22D97
  \l 22D98
  \l 22D99
  \l 22D9A
  \l 22D9B
  \l 22D9C
  \l 22D9D
  \l 22D9E
  \l 22D9F
  \l 22DA0
  \l 22DA1
  \l 22DA2
  \l 22DA3
  \l 22DA4
  \l 22DA5
  \l 22DA6
  \l 22DA7
  \l 22DA8
  \l 22DA9
  \l 22DAA
  \l 22DAB
  \l 22DAC
  \l 22DAD
  \l 22DAE
  \l 22DAF
  \l 22DB0
  \l 22DB1
  \l 22DB2
  \l 22DB3
  \l 22DB4
  \l 22DB5
  \l 22DB6
  \l 22DB7
  \l 22DB8
  \l 22DB9
  \l 22DBA
  \l 22DBB
  \l 22DBC
  \l 22DBD
  \l 22DBE
  \l 22DBF
  \l 22DC0
  \l 22DC1
  \l 22DC2
  \l 22DC3
  \l 22DC4
  \l 22DC5
  \l 22DC6
  \l 22DC7
  \l 22DC8
  \l 22DC9
  \l 22DCA
  \l 22DCB
  \l 22DCC
  \l 22DCD
  \l 22DCE
  \l 22DCF
  \l 22DD0
  \l 22DD1
  \l 22DD2
  \l 22DD3
  \l 22DD4
  \l 22DD5
  \l 22DD6
  \l 22DD7
  \l 22DD8
  \l 22DD9
  \l 22DDA
  \l 22DDB
  \l 22DDC
  \l 22DDD
  \l 22DDE
  \l 22DDF
  \l 22DE0
  \l 22DE1
  \l 22DE2
  \l 22DE3
  \l 22DE4
  \l 22DE5
  \l 22DE6
  \l 22DE7
  \l 22DE8
  \l 22DE9
  \l 22DEA
  \l 22DEB
  \l 22DEC
  \l 22DED
  \l 22DEE
  \l 22DEF
  \l 22DF0
  \l 22DF1
  \l 22DF2
  \l 22DF3
  \l 22DF4
  \l 22DF5
  \l 22DF6
  \l 22DF7
  \l 22DF8
  \l 22DF9
  \l 22DFA
  \l 22DFB
  \l 22DFC
  \l 22DFD
  \l 22DFE
  \l 22DFF
  \l 22E00
  \l 22E01
  \l 22E02
  \l 22E03
  \l 22E04
  \l 22E05
  \l 22E06
  \l 22E07
  \l 22E08
  \l 22E09
  \l 22E0A
  \l 22E0B
  \l 22E0C
  \l 22E0D
  \l 22E0E
  \l 22E0F
  \l 22E10
  \l 22E11
  \l 22E12
  \l 22E13
  \l 22E14
  \l 22E15
  \l 22E16
  \l 22E17
  \l 22E18
  \l 22E19
  \l 22E1A
  \l 22E1B
  \l 22E1C
  \l 22E1D
  \l 22E1E
  \l 22E1F
  \l 22E20
  \l 22E21
  \l 22E22
  \l 22E23
  \l 22E24
  \l 22E25
  \l 22E26
  \l 22E27
  \l 22E28
  \l 22E29
  \l 22E2A
  \l 22E2B
  \l 22E2C
  \l 22E2D
  \l 22E2E
  \l 22E2F
  \l 22E30
  \l 22E31
  \l 22E32
  \l 22E33
  \l 22E34
  \l 22E35
  \l 22E36
  \l 22E37
  \l 22E38
  \l 22E39
  \l 22E3A
  \l 22E3B
  \l 22E3C
  \l 22E3D
  \l 22E3E
  \l 22E3F
  \l 22E40
  \l 22E41
  \l 22E42
  \l 22E43
  \l 22E44
  \l 22E45
  \l 22E46
  \l 22E47
  \l 22E48
  \l 22E49
  \l 22E4A
  \l 22E4B
  \l 22E4C
  \l 22E4D
  \l 22E4E
  \l 22E4F
  \l 22E50
  \l 22E51
  \l 22E52
  \l 22E53
  \l 22E54
  \l 22E55
  \l 22E56
  \l 22E57
  \l 22E58
  \l 22E59
  \l 22E5A
  \l 22E5B
  \l 22E5C
  \l 22E5D
  \l 22E5E
  \l 22E5F
  \l 22E60
  \l 22E61
  \l 22E62
  \l 22E63
  \l 22E64
  \l 22E65
  \l 22E66
  \l 22E67
  \l 22E68
  \l 22E69
  \l 22E6A
  \l 22E6B
  \l 22E6C
  \l 22E6D
  \l 22E6E
  \l 22E6F
  \l 22E70
  \l 22E71
  \l 22E72
  \l 22E73
  \l 22E74
  \l 22E75
  \l 22E76
  \l 22E77
  \l 22E78
  \l 22E79
  \l 22E7A
  \l 22E7B
  \l 22E7C
  \l 22E7D
  \l 22E7E
  \l 22E7F
  \l 22E80
  \l 22E81
  \l 22E82
  \l 22E83
  \l 22E84
  \l 22E85
  \l 22E86
  \l 22E87
  \l 22E88
  \l 22E89
  \l 22E8A
  \l 22E8B
  \l 22E8C
  \l 22E8D
  \l 22E8E
  \l 22E8F
  \l 22E90
  \l 22E91
  \l 22E92
  \l 22E93
  \l 22E94
  \l 22E95
  \l 22E96
  \l 22E97
  \l 22E98
  \l 22E99
  \l 22E9A
  \l 22E9B
  \l 22E9C
  \l 22E9D
  \l 22E9E
  \l 22E9F
  \l 22EA0
  \l 22EA1
  \l 22EA2
  \l 22EA3
  \l 22EA4
  \l 22EA5
  \l 22EA6
  \l 22EA7
  \l 22EA8
  \l 22EA9
  \l 22EAA
  \l 22EAB
  \l 22EAC
  \l 22EAD
  \l 22EAE
  \l 22EAF
  \l 22EB0
  \l 22EB1
  \l 22EB2
  \l 22EB3
  \l 22EB4
  \l 22EB5
  \l 22EB6
  \l 22EB7
  \l 22EB8
  \l 22EB9
  \l 22EBA
  \l 22EBB
  \l 22EBC
  \l 22EBD
  \l 22EBE
  \l 22EBF
  \l 22EC0
  \l 22EC1
  \l 22EC2
  \l 22EC3
  \l 22EC4
  \l 22EC5
  \l 22EC6
  \l 22EC7
  \l 22EC8
  \l 22EC9
  \l 22ECA
  \l 22ECB
  \l 22ECC
  \l 22ECD
  \l 22ECE
  \l 22ECF
  \l 22ED0
  \l 22ED1
  \l 22ED2
  \l 22ED3
  \l 22ED4
  \l 22ED5
  \l 22ED6
  \l 22ED7
  \l 22ED8
  \l 22ED9
  \l 22EDA
  \l 22EDB
  \l 22EDC
  \l 22EDD
  \l 22EDE
  \l 22EDF
  \l 22EE0
  \l 22EE1
  \l 22EE2
  \l 22EE3
  \l 22EE4
  \l 22EE5
  \l 22EE6
  \l 22EE7
  \l 22EE8
  \l 22EE9
  \l 22EEA
  \l 22EEB
  \l 22EEC
  \l 22EED
  \l 22EEE
  \l 22EEF
  \l 22EF0
  \l 22EF1
  \l 22EF2
  \l 22EF3
  \l 22EF4
  \l 22EF5
  \l 22EF6
  \l 22EF7
  \l 22EF8
  \l 22EF9
  \l 22EFA
  \l 22EFB
  \l 22EFC
  \l 22EFD
  \l 22EFE
  \l 22EFF
  \l 22F00
  \l 22F01
  \l 22F02
  \l 22F03
  \l 22F04
  \l 22F05
  \l 22F06
  \l 22F07
  \l 22F08
  \l 22F09
  \l 22F0A
  \l 22F0B
  \l 22F0C
  \l 22F0D
  \l 22F0E
  \l 22F0F
  \l 22F10
  \l 22F11
  \l 22F12
  \l 22F13
  \l 22F14
  \l 22F15
  \l 22F16
  \l 22F17
  \l 22F18
  \l 22F19
  \l 22F1A
  \l 22F1B
  \l 22F1C
  \l 22F1D
  \l 22F1E
  \l 22F1F
  \l 22F20
  \l 22F21
  \l 22F22
  \l 22F23
  \l 22F24
  \l 22F25
  \l 22F26
  \l 22F27
  \l 22F28
  \l 22F29
  \l 22F2A
  \l 22F2B
  \l 22F2C
  \l 22F2D
  \l 22F2E
  \l 22F2F
  \l 22F30
  \l 22F31
  \l 22F32
  \l 22F33
  \l 22F34
  \l 22F35
  \l 22F36
  \l 22F37
  \l 22F38
  \l 22F39
  \l 22F3A
  \l 22F3B
  \l 22F3C
  \l 22F3D
  \l 22F3E
  \l 22F3F
  \l 22F40
  \l 22F41
  \l 22F42
  \l 22F43
  \l 22F44
  \l 22F45
  \l 22F46
  \l 22F47
  \l 22F48
  \l 22F49
  \l 22F4A
  \l 22F4B
  \l 22F4C
  \l 22F4D
  \l 22F4E
  \l 22F4F
  \l 22F50
  \l 22F51
  \l 22F52
  \l 22F53
  \l 22F54
  \l 22F55
  \l 22F56
  \l 22F57
  \l 22F58
  \l 22F59
  \l 22F5A
  \l 22F5B
  \l 22F5C
  \l 22F5D
  \l 22F5E
  \l 22F5F
  \l 22F60
  \l 22F61
  \l 22F62
  \l 22F63
  \l 22F64
  \l 22F65
  \l 22F66
  \l 22F67
  \l 22F68
  \l 22F69
  \l 22F6A
  \l 22F6B
  \l 22F6C
  \l 22F6D
  \l 22F6E
  \l 22F6F
  \l 22F70
  \l 22F71
  \l 22F72
  \l 22F73
  \l 22F74
  \l 22F75
  \l 22F76
  \l 22F77
  \l 22F78
  \l 22F79
  \l 22F7A
  \l 22F7B
  \l 22F7C
  \l 22F7D
  \l 22F7E
  \l 22F7F
  \l 22F80
  \l 22F81
  \l 22F82
  \l 22F83
  \l 22F84
  \l 22F85
  \l 22F86
  \l 22F87
  \l 22F88
  \l 22F89
  \l 22F8A
  \l 22F8B
  \l 22F8C
  \l 22F8D
  \l 22F8E
  \l 22F8F
  \l 22F90
  \l 22F91
  \l 22F92
  \l 22F93
  \l 22F94
  \l 22F95
  \l 22F96
  \l 22F97
  \l 22F98
  \l 22F99
  \l 22F9A
  \l 22F9B
  \l 22F9C
  \l 22F9D
  \l 22F9E
  \l 22F9F
  \l 22FA0
  \l 22FA1
  \l 22FA2
  \l 22FA3
  \l 22FA4
  \l 22FA5
  \l 22FA6
  \l 22FA7
  \l 22FA8
  \l 22FA9
  \l 22FAA
  \l 22FAB
  \l 22FAC
  \l 22FAD
  \l 22FAE
  \l 22FAF
  \l 22FB0
  \l 22FB1
  \l 22FB2
  \l 22FB3
  \l 22FB4
  \l 22FB5
  \l 22FB6
  \l 22FB7
  \l 22FB8
  \l 22FB9
  \l 22FBA
  \l 22FBB
  \l 22FBC
  \l 22FBD
  \l 22FBE
  \l 22FBF
  \l 22FC0
  \l 22FC1
  \l 22FC2
  \l 22FC3
  \l 22FC4
  \l 22FC5
  \l 22FC6
  \l 22FC7
  \l 22FC8
  \l 22FC9
  \l 22FCA
  \l 22FCB
  \l 22FCC
  \l 22FCD
  \l 22FCE
  \l 22FCF
  \l 22FD0
  \l 22FD1
  \l 22FD2
  \l 22FD3
  \l 22FD4
  \l 22FD5
  \l 22FD6
  \l 22FD7
  \l 22FD8
  \l 22FD9
  \l 22FDA
  \l 22FDB
  \l 22FDC
  \l 22FDD
  \l 22FDE
  \l 22FDF
  \l 22FE0
  \l 22FE1
  \l 22FE2
  \l 22FE3
  \l 22FE4
  \l 22FE5
  \l 22FE6
  \l 22FE7
  \l 22FE8
  \l 22FE9
  \l 22FEA
  \l 22FEB
  \l 22FEC
  \l 22FED
  \l 22FEE
  \l 22FEF
  \l 22FF0
  \l 22FF1
  \l 22FF2
  \l 22FF3
  \l 22FF4
  \l 22FF5
  \l 22FF6
  \l 22FF7
  \l 22FF8
  \l 22FF9
  \l 22FFA
  \l 22FFB
  \l 22FFC
  \l 22FFD
  \l 22FFE
  \l 22FFF
  \l 23000
  \l 23001
  \l 23002
  \l 23003
  \l 23004
  \l 23005
  \l 23006
  \l 23007
  \l 23008
  \l 23009
  \l 2300A
  \l 2300B
  \l 2300C
  \l 2300D
  \l 2300E
  \l 2300F
  \l 23010
  \l 23011
  \l 23012
  \l 23013
  \l 23014
  \l 23015
  \l 23016
  \l 23017
  \l 23018
  \l 23019
  \l 2301A
  \l 2301B
  \l 2301C
  \l 2301D
  \l 2301E
  \l 2301F
  \l 23020
  \l 23021
  \l 23022
  \l 23023
  \l 23024
  \l 23025
  \l 23026
  \l 23027
  \l 23028
  \l 23029
  \l 2302A
  \l 2302B
  \l 2302C
  \l 2302D
  \l 2302E
  \l 2302F
  \l 23030
  \l 23031
  \l 23032
  \l 23033
  \l 23034
  \l 23035
  \l 23036
  \l 23037
  \l 23038
  \l 23039
  \l 2303A
  \l 2303B
  \l 2303C
  \l 2303D
  \l 2303E
  \l 2303F
  \l 23040
  \l 23041
  \l 23042
  \l 23043
  \l 23044
  \l 23045
  \l 23046
  \l 23047
  \l 23048
  \l 23049
  \l 2304A
  \l 2304B
  \l 2304C
  \l 2304D
  \l 2304E
  \l 2304F
  \l 23050
  \l 23051
  \l 23052
  \l 23053
  \l 23054
  \l 23055
  \l 23056
  \l 23057
  \l 23058
  \l 23059
  \l 2305A
  \l 2305B
  \l 2305C
  \l 2305D
  \l 2305E
  \l 2305F
  \l 23060
  \l 23061
  \l 23062
  \l 23063
  \l 23064
  \l 23065
  \l 23066
  \l 23067
  \l 23068
  \l 23069
  \l 2306A
  \l 2306B
  \l 2306C
  \l 2306D
  \l 2306E
  \l 2306F
  \l 23070
  \l 23071
  \l 23072
  \l 23073
  \l 23074
  \l 23075
  \l 23076
  \l 23077
  \l 23078
  \l 23079
  \l 2307A
  \l 2307B
  \l 2307C
  \l 2307D
  \l 2307E
  \l 2307F
  \l 23080
  \l 23081
  \l 23082
  \l 23083
  \l 23084
  \l 23085
  \l 23086
  \l 23087
  \l 23088
  \l 23089
  \l 2308A
  \l 2308B
  \l 2308C
  \l 2308D
  \l 2308E
  \l 2308F
  \l 23090
  \l 23091
  \l 23092
  \l 23093
  \l 23094
  \l 23095
  \l 23096
  \l 23097
  \l 23098
  \l 23099
  \l 2309A
  \l 2309B
  \l 2309C
  \l 2309D
  \l 2309E
  \l 2309F
  \l 230A0
  \l 230A1
  \l 230A2
  \l 230A3
  \l 230A4
  \l 230A5
  \l 230A6
  \l 230A7
  \l 230A8
  \l 230A9
  \l 230AA
  \l 230AB
  \l 230AC
  \l 230AD
  \l 230AE
  \l 230AF
  \l 230B0
  \l 230B1
  \l 230B2
  \l 230B3
  \l 230B4
  \l 230B5
  \l 230B6
  \l 230B7
  \l 230B8
  \l 230B9
  \l 230BA
  \l 230BB
  \l 230BC
  \l 230BD
  \l 230BE
  \l 230BF
  \l 230C0
  \l 230C1
  \l 230C2
  \l 230C3
  \l 230C4
  \l 230C5
  \l 230C6
  \l 230C7
  \l 230C8
  \l 230C9
  \l 230CA
  \l 230CB
  \l 230CC
  \l 230CD
  \l 230CE
  \l 230CF
  \l 230D0
  \l 230D1
  \l 230D2
  \l 230D3
  \l 230D4
  \l 230D5
  \l 230D6
  \l 230D7
  \l 230D8
  \l 230D9
  \l 230DA
  \l 230DB
  \l 230DC
  \l 230DD
  \l 230DE
  \l 230DF
  \l 230E0
  \l 230E1
  \l 230E2
  \l 230E3
  \l 230E4
  \l 230E5
  \l 230E6
  \l 230E7
  \l 230E8
  \l 230E9
  \l 230EA
  \l 230EB
  \l 230EC
  \l 230ED
  \l 230EE
  \l 230EF
  \l 230F0
  \l 230F1
  \l 230F2
  \l 230F3
  \l 230F4
  \l 230F5
  \l 230F6
  \l 230F7
  \l 230F8
  \l 230F9
  \l 230FA
  \l 230FB
  \l 230FC
  \l 230FD
  \l 230FE
  \l 230FF
  \l 23100
  \l 23101
  \l 23102
  \l 23103
  \l 23104
  \l 23105
  \l 23106
  \l 23107
  \l 23108
  \l 23109
  \l 2310A
  \l 2310B
  \l 2310C
  \l 2310D
  \l 2310E
  \l 2310F
  \l 23110
  \l 23111
  \l 23112
  \l 23113
  \l 23114
  \l 23115
  \l 23116
  \l 23117
  \l 23118
  \l 23119
  \l 2311A
  \l 2311B
  \l 2311C
  \l 2311D
  \l 2311E
  \l 2311F
  \l 23120
  \l 23121
  \l 23122
  \l 23123
  \l 23124
  \l 23125
  \l 23126
  \l 23127
  \l 23128
  \l 23129
  \l 2312A
  \l 2312B
  \l 2312C
  \l 2312D
  \l 2312E
  \l 2312F
  \l 23130
  \l 23131
  \l 23132
  \l 23133
  \l 23134
  \l 23135
  \l 23136
  \l 23137
  \l 23138
  \l 23139
  \l 2313A
  \l 2313B
  \l 2313C
  \l 2313D
  \l 2313E
  \l 2313F
  \l 23140
  \l 23141
  \l 23142
  \l 23143
  \l 23144
  \l 23145
  \l 23146
  \l 23147
  \l 23148
  \l 23149
  \l 2314A
  \l 2314B
  \l 2314C
  \l 2314D
  \l 2314E
  \l 2314F
  \l 23150
  \l 23151
  \l 23152
  \l 23153
  \l 23154
  \l 23155
  \l 23156
  \l 23157
  \l 23158
  \l 23159
  \l 2315A
  \l 2315B
  \l 2315C
  \l 2315D
  \l 2315E
  \l 2315F
  \l 23160
  \l 23161
  \l 23162
  \l 23163
  \l 23164
  \l 23165
  \l 23166
  \l 23167
  \l 23168
  \l 23169
  \l 2316A
  \l 2316B
  \l 2316C
  \l 2316D
  \l 2316E
  \l 2316F
  \l 23170
  \l 23171
  \l 23172
  \l 23173
  \l 23174
  \l 23175
  \l 23176
  \l 23177
  \l 23178
  \l 23179
  \l 2317A
  \l 2317B
  \l 2317C
  \l 2317D
  \l 2317E
  \l 2317F
  \l 23180
  \l 23181
  \l 23182
  \l 23183
  \l 23184
  \l 23185
  \l 23186
  \l 23187
  \l 23188
  \l 23189
  \l 2318A
  \l 2318B
  \l 2318C
  \l 2318D
  \l 2318E
  \l 2318F
  \l 23190
  \l 23191
  \l 23192
  \l 23193
  \l 23194
  \l 23195
  \l 23196
  \l 23197
  \l 23198
  \l 23199
  \l 2319A
  \l 2319B
  \l 2319C
  \l 2319D
  \l 2319E
  \l 2319F
  \l 231A0
  \l 231A1
  \l 231A2
  \l 231A3
  \l 231A4
  \l 231A5
  \l 231A6
  \l 231A7
  \l 231A8
  \l 231A9
  \l 231AA
  \l 231AB
  \l 231AC
  \l 231AD
  \l 231AE
  \l 231AF
  \l 231B0
  \l 231B1
  \l 231B2
  \l 231B3
  \l 231B4
  \l 231B5
  \l 231B6
  \l 231B7
  \l 231B8
  \l 231B9
  \l 231BA
  \l 231BB
  \l 231BC
  \l 231BD
  \l 231BE
  \l 231BF
  \l 231C0
  \l 231C1
  \l 231C2
  \l 231C3
  \l 231C4
  \l 231C5
  \l 231C6
  \l 231C7
  \l 231C8
  \l 231C9
  \l 231CA
  \l 231CB
  \l 231CC
  \l 231CD
  \l 231CE
  \l 231CF
  \l 231D0
  \l 231D1
  \l 231D2
  \l 231D3
  \l 231D4
  \l 231D5
  \l 231D6
  \l 231D7
  \l 231D8
  \l 231D9
  \l 231DA
  \l 231DB
  \l 231DC
  \l 231DD
  \l 231DE
  \l 231DF
  \l 231E0
  \l 231E1
  \l 231E2
  \l 231E3
  \l 231E4
  \l 231E5
  \l 231E6
  \l 231E7
  \l 231E8
  \l 231E9
  \l 231EA
  \l 231EB
  \l 231EC
  \l 231ED
  \l 231EE
  \l 231EF
  \l 231F0
  \l 231F1
  \l 231F2
  \l 231F3
  \l 231F4
  \l 231F5
  \l 231F6
  \l 231F7
  \l 231F8
  \l 231F9
  \l 231FA
  \l 231FB
  \l 231FC
  \l 231FD
  \l 231FE
  \l 231FF
  \l 23200
  \l 23201
  \l 23202
  \l 23203
  \l 23204
  \l 23205
  \l 23206
  \l 23207
  \l 23208
  \l 23209
  \l 2320A
  \l 2320B
  \l 2320C
  \l 2320D
  \l 2320E
  \l 2320F
  \l 23210
  \l 23211
  \l 23212
  \l 23213
  \l 23214
  \l 23215
  \l 23216
  \l 23217
  \l 23218
  \l 23219
  \l 2321A
  \l 2321B
  \l 2321C
  \l 2321D
  \l 2321E
  \l 2321F
  \l 23220
  \l 23221
  \l 23222
  \l 23223
  \l 23224
  \l 23225
  \l 23226
  \l 23227
  \l 23228
  \l 23229
  \l 2322A
  \l 2322B
  \l 2322C
  \l 2322D
  \l 2322E
  \l 2322F
  \l 23230
  \l 23231
  \l 23232
  \l 23233
  \l 23234
  \l 23235
  \l 23236
  \l 23237
  \l 23238
  \l 23239
  \l 2323A
  \l 2323B
  \l 2323C
  \l 2323D
  \l 2323E
  \l 2323F
  \l 23240
  \l 23241
  \l 23242
  \l 23243
  \l 23244
  \l 23245
  \l 23246
  \l 23247
  \l 23248
  \l 23249
  \l 2324A
  \l 2324B
  \l 2324C
  \l 2324D
  \l 2324E
  \l 2324F
  \l 23250
  \l 23251
  \l 23252
  \l 23253
  \l 23254
  \l 23255
  \l 23256
  \l 23257
  \l 23258
  \l 23259
  \l 2325A
  \l 2325B
  \l 2325C
  \l 2325D
  \l 2325E
  \l 2325F
  \l 23260
  \l 23261
  \l 23262
  \l 23263
  \l 23264
  \l 23265
  \l 23266
  \l 23267
  \l 23268
  \l 23269
  \l 2326A
  \l 2326B
  \l 2326C
  \l 2326D
  \l 2326E
  \l 2326F
  \l 23270
  \l 23271
  \l 23272
  \l 23273
  \l 23274
  \l 23275
  \l 23276
  \l 23277
  \l 23278
  \l 23279
  \l 2327A
  \l 2327B
  \l 2327C
  \l 2327D
  \l 2327E
  \l 2327F
  \l 23280
  \l 23281
  \l 23282
  \l 23283
  \l 23284
  \l 23285
  \l 23286
  \l 23287
  \l 23288
  \l 23289
  \l 2328A
  \l 2328B
  \l 2328C
  \l 2328D
  \l 2328E
  \l 2328F
  \l 23290
  \l 23291
  \l 23292
  \l 23293
  \l 23294
  \l 23295
  \l 23296
  \l 23297
  \l 23298
  \l 23299
  \l 2329A
  \l 2329B
  \l 2329C
  \l 2329D
  \l 2329E
  \l 2329F
  \l 232A0
  \l 232A1
  \l 232A2
  \l 232A3
  \l 232A4
  \l 232A5
  \l 232A6
  \l 232A7
  \l 232A8
  \l 232A9
  \l 232AA
  \l 232AB
  \l 232AC
  \l 232AD
  \l 232AE
  \l 232AF
  \l 232B0
  \l 232B1
  \l 232B2
  \l 232B3
  \l 232B4
  \l 232B5
  \l 232B6
  \l 232B7
  \l 232B8
  \l 232B9
  \l 232BA
  \l 232BB
  \l 232BC
  \l 232BD
  \l 232BE
  \l 232BF
  \l 232C0
  \l 232C1
  \l 232C2
  \l 232C3
  \l 232C4
  \l 232C5
  \l 232C6
  \l 232C7
  \l 232C8
  \l 232C9
  \l 232CA
  \l 232CB
  \l 232CC
  \l 232CD
  \l 232CE
  \l 232CF
  \l 232D0
  \l 232D1
  \l 232D2
  \l 232D3
  \l 232D4
  \l 232D5
  \l 232D6
  \l 232D7
  \l 232D8
  \l 232D9
  \l 232DA
  \l 232DB
  \l 232DC
  \l 232DD
  \l 232DE
  \l 232DF
  \l 232E0
  \l 232E1
  \l 232E2
  \l 232E3
  \l 232E4
  \l 232E5
  \l 232E6
  \l 232E7
  \l 232E8
  \l 232E9
  \l 232EA
  \l 232EB
  \l 232EC
  \l 232ED
  \l 232EE
  \l 232EF
  \l 232F0
  \l 232F1
  \l 232F2
  \l 232F3
  \l 232F4
  \l 232F5
  \l 232F6
  \l 232F7
  \l 232F8
  \l 232F9
  \l 232FA
  \l 232FB
  \l 232FC
  \l 232FD
  \l 232FE
  \l 232FF
  \l 23300
  \l 23301
  \l 23302
  \l 23303
  \l 23304
  \l 23305
  \l 23306
  \l 23307
  \l 23308
  \l 23309
  \l 2330A
  \l 2330B
  \l 2330C
  \l 2330D
  \l 2330E
  \l 2330F
  \l 23310
  \l 23311
  \l 23312
  \l 23313
  \l 23314
  \l 23315
  \l 23316
  \l 23317
  \l 23318
  \l 23319
  \l 2331A
  \l 2331B
  \l 2331C
  \l 2331D
  \l 2331E
  \l 2331F
  \l 23320
  \l 23321
  \l 23322
  \l 23323
  \l 23324
  \l 23325
  \l 23326
  \l 23327
  \l 23328
  \l 23329
  \l 2332A
  \l 2332B
  \l 2332C
  \l 2332D
  \l 2332E
  \l 2332F
  \l 23330
  \l 23331
  \l 23332
  \l 23333
  \l 23334
  \l 23335
  \l 23336
  \l 23337
  \l 23338
  \l 23339
  \l 2333A
  \l 2333B
  \l 2333C
  \l 2333D
  \l 2333E
  \l 2333F
  \l 23340
  \l 23341
  \l 23342
  \l 23343
  \l 23344
  \l 23345
  \l 23346
  \l 23347
  \l 23348
  \l 23349
  \l 2334A
  \l 2334B
  \l 2334C
  \l 2334D
  \l 2334E
  \l 2334F
  \l 23350
  \l 23351
  \l 23352
  \l 23353
  \l 23354
  \l 23355
  \l 23356
  \l 23357
  \l 23358
  \l 23359
  \l 2335A
  \l 2335B
  \l 2335C
  \l 2335D
  \l 2335E
  \l 2335F
  \l 23360
  \l 23361
  \l 23362
  \l 23363
  \l 23364
  \l 23365
  \l 23366
  \l 23367
  \l 23368
  \l 23369
  \l 2336A
  \l 2336B
  \l 2336C
  \l 2336D
  \l 2336E
  \l 2336F
  \l 23370
  \l 23371
  \l 23372
  \l 23373
  \l 23374
  \l 23375
  \l 23376
  \l 23377
  \l 23378
  \l 23379
  \l 2337A
  \l 2337B
  \l 2337C
  \l 2337D
  \l 2337E
  \l 2337F
  \l 23380
  \l 23381
  \l 23382
  \l 23383
  \l 23384
  \l 23385
  \l 23386
  \l 23387
  \l 23388
  \l 23389
  \l 2338A
  \l 2338B
  \l 2338C
  \l 2338D
  \l 2338E
  \l 2338F
  \l 23390
  \l 23391
  \l 23392
  \l 23393
  \l 23394
  \l 23395
  \l 23396
  \l 23397
  \l 23398
  \l 23399
  \l 2339A
  \l 2339B
  \l 2339C
  \l 2339D
  \l 2339E
  \l 2339F
  \l 233A0
  \l 233A1
  \l 233A2
  \l 233A3
  \l 233A4
  \l 233A5
  \l 233A6
  \l 233A7
  \l 233A8
  \l 233A9
  \l 233AA
  \l 233AB
  \l 233AC
  \l 233AD
  \l 233AE
  \l 233AF
  \l 233B0
  \l 233B1
  \l 233B2
  \l 233B3
  \l 233B4
  \l 233B5
  \l 233B6
  \l 233B7
  \l 233B8
  \l 233B9
  \l 233BA
  \l 233BB
  \l 233BC
  \l 233BD
  \l 233BE
  \l 233BF
  \l 233C0
  \l 233C1
  \l 233C2
  \l 233C3
  \l 233C4
  \l 233C5
  \l 233C6
  \l 233C7
  \l 233C8
  \l 233C9
  \l 233CA
  \l 233CB
  \l 233CC
  \l 233CD
  \l 233CE
  \l 233CF
  \l 233D0
  \l 233D1
  \l 233D2
  \l 233D3
  \l 233D4
  \l 233D5
  \l 233D6
  \l 233D7
  \l 233D8
  \l 233D9
  \l 233DA
  \l 233DB
  \l 233DC
  \l 233DD
  \l 233DE
  \l 233DF
  \l 233E0
  \l 233E1
  \l 233E2
  \l 233E3
  \l 233E4
  \l 233E5
  \l 233E6
  \l 233E7
  \l 233E8
  \l 233E9
  \l 233EA
  \l 233EB
  \l 233EC
  \l 233ED
  \l 233EE
  \l 233EF
  \l 233F0
  \l 233F1
  \l 233F2
  \l 233F3
  \l 233F4
  \l 233F5
  \l 233F6
  \l 233F7
  \l 233F8
  \l 233F9
  \l 233FA
  \l 233FB
  \l 233FC
  \l 233FD
  \l 233FE
  \l 233FF
  \l 23400
  \l 23401
  \l 23402
  \l 23403
  \l 23404
  \l 23405
  \l 23406
  \l 23407
  \l 23408
  \l 23409
  \l 2340A
  \l 2340B
  \l 2340C
  \l 2340D
  \l 2340E
  \l 2340F
  \l 23410
  \l 23411
  \l 23412
  \l 23413
  \l 23414
  \l 23415
  \l 23416
  \l 23417
  \l 23418
  \l 23419
  \l 2341A
  \l 2341B
  \l 2341C
  \l 2341D
  \l 2341E
  \l 2341F
  \l 23420
  \l 23421
  \l 23422
  \l 23423
  \l 23424
  \l 23425
  \l 23426
  \l 23427
  \l 23428
  \l 23429
  \l 2342A
  \l 2342B
  \l 2342C
  \l 2342D
  \l 2342E
  \l 2342F
  \l 23430
  \l 23431
  \l 23432
  \l 23433
  \l 23434
  \l 23435
  \l 23436
  \l 23437
  \l 23438
  \l 23439
  \l 2343A
  \l 2343B
  \l 2343C
  \l 2343D
  \l 2343E
  \l 2343F
  \l 23440
  \l 23441
  \l 23442
  \l 23443
  \l 23444
  \l 23445
  \l 23446
  \l 23447
  \l 23448
  \l 23449
  \l 2344A
  \l 2344B
  \l 2344C
  \l 2344D
  \l 2344E
  \l 2344F
  \l 23450
  \l 23451
  \l 23452
  \l 23453
  \l 23454
  \l 23455
  \l 23456
  \l 23457
  \l 23458
  \l 23459
  \l 2345A
  \l 2345B
  \l 2345C
  \l 2345D
  \l 2345E
  \l 2345F
  \l 23460
  \l 23461
  \l 23462
  \l 23463
  \l 23464
  \l 23465
  \l 23466
  \l 23467
  \l 23468
  \l 23469
  \l 2346A
  \l 2346B
  \l 2346C
  \l 2346D
  \l 2346E
  \l 2346F
  \l 23470
  \l 23471
  \l 23472
  \l 23473
  \l 23474
  \l 23475
  \l 23476
  \l 23477
  \l 23478
  \l 23479
  \l 2347A
  \l 2347B
  \l 2347C
  \l 2347D
  \l 2347E
  \l 2347F
  \l 23480
  \l 23481
  \l 23482
  \l 23483
  \l 23484
  \l 23485
  \l 23486
  \l 23487
  \l 23488
  \l 23489
  \l 2348A
  \l 2348B
  \l 2348C
  \l 2348D
  \l 2348E
  \l 2348F
  \l 23490
  \l 23491
  \l 23492
  \l 23493
  \l 23494
  \l 23495
  \l 23496
  \l 23497
  \l 23498
  \l 23499
  \l 2349A
  \l 2349B
  \l 2349C
  \l 2349D
  \l 2349E
  \l 2349F
  \l 234A0
  \l 234A1
  \l 234A2
  \l 234A3
  \l 234A4
  \l 234A5
  \l 234A6
  \l 234A7
  \l 234A8
  \l 234A9
  \l 234AA
  \l 234AB
  \l 234AC
  \l 234AD
  \l 234AE
  \l 234AF
  \l 234B0
  \l 234B1
  \l 234B2
  \l 234B3
  \l 234B4
  \l 234B5
  \l 234B6
  \l 234B7
  \l 234B8
  \l 234B9
  \l 234BA
  \l 234BB
  \l 234BC
  \l 234BD
  \l 234BE
  \l 234BF
  \l 234C0
  \l 234C1
  \l 234C2
  \l 234C3
  \l 234C4
  \l 234C5
  \l 234C6
  \l 234C7
  \l 234C8
  \l 234C9
  \l 234CA
  \l 234CB
  \l 234CC
  \l 234CD
  \l 234CE
  \l 234CF
  \l 234D0
  \l 234D1
  \l 234D2
  \l 234D3
  \l 234D4
  \l 234D5
  \l 234D6
  \l 234D7
  \l 234D8
  \l 234D9
  \l 234DA
  \l 234DB
  \l 234DC
  \l 234DD
  \l 234DE
  \l 234DF
  \l 234E0
  \l 234E1
  \l 234E2
  \l 234E3
  \l 234E4
  \l 234E5
  \l 234E6
  \l 234E7
  \l 234E8
  \l 234E9
  \l 234EA
  \l 234EB
  \l 234EC
  \l 234ED
  \l 234EE
  \l 234EF
  \l 234F0
  \l 234F1
  \l 234F2
  \l 234F3
  \l 234F4
  \l 234F5
  \l 234F6
  \l 234F7
  \l 234F8
  \l 234F9
  \l 234FA
  \l 234FB
  \l 234FC
  \l 234FD
  \l 234FE
  \l 234FF
  \l 23500
  \l 23501
  \l 23502
  \l 23503
  \l 23504
  \l 23505
  \l 23506
  \l 23507
  \l 23508
  \l 23509
  \l 2350A
  \l 2350B
  \l 2350C
  \l 2350D
  \l 2350E
  \l 2350F
  \l 23510
  \l 23511
  \l 23512
  \l 23513
  \l 23514
  \l 23515
  \l 23516
  \l 23517
  \l 23518
  \l 23519
  \l 2351A
  \l 2351B
  \l 2351C
  \l 2351D
  \l 2351E
  \l 2351F
  \l 23520
  \l 23521
  \l 23522
  \l 23523
  \l 23524
  \l 23525
  \l 23526
  \l 23527
  \l 23528
  \l 23529
  \l 2352A
  \l 2352B
  \l 2352C
  \l 2352D
  \l 2352E
  \l 2352F
  \l 23530
  \l 23531
  \l 23532
  \l 23533
  \l 23534
  \l 23535
  \l 23536
  \l 23537
  \l 23538
  \l 23539
  \l 2353A
  \l 2353B
  \l 2353C
  \l 2353D
  \l 2353E
  \l 2353F
  \l 23540
  \l 23541
  \l 23542
  \l 23543
  \l 23544
  \l 23545
  \l 23546
  \l 23547
  \l 23548
  \l 23549
  \l 2354A
  \l 2354B
  \l 2354C
  \l 2354D
  \l 2354E
  \l 2354F
  \l 23550
  \l 23551
  \l 23552
  \l 23553
  \l 23554
  \l 23555
  \l 23556
  \l 23557
  \l 23558
  \l 23559
  \l 2355A
  \l 2355B
  \l 2355C
  \l 2355D
  \l 2355E
  \l 2355F
  \l 23560
  \l 23561
  \l 23562
  \l 23563
  \l 23564
  \l 23565
  \l 23566
  \l 23567
  \l 23568
  \l 23569
  \l 2356A
  \l 2356B
  \l 2356C
  \l 2356D
  \l 2356E
  \l 2356F
  \l 23570
  \l 23571
  \l 23572
  \l 23573
  \l 23574
  \l 23575
  \l 23576
  \l 23577
  \l 23578
  \l 23579
  \l 2357A
  \l 2357B
  \l 2357C
  \l 2357D
  \l 2357E
  \l 2357F
  \l 23580
  \l 23581
  \l 23582
  \l 23583
  \l 23584
  \l 23585
  \l 23586
  \l 23587
  \l 23588
  \l 23589
  \l 2358A
  \l 2358B
  \l 2358C
  \l 2358D
  \l 2358E
  \l 2358F
  \l 23590
  \l 23591
  \l 23592
  \l 23593
  \l 23594
  \l 23595
  \l 23596
  \l 23597
  \l 23598
  \l 23599
  \l 2359A
  \l 2359B
  \l 2359C
  \l 2359D
  \l 2359E
  \l 2359F
  \l 235A0
  \l 235A1
  \l 235A2
  \l 235A3
  \l 235A4
  \l 235A5
  \l 235A6
  \l 235A7
  \l 235A8
  \l 235A9
  \l 235AA
  \l 235AB
  \l 235AC
  \l 235AD
  \l 235AE
  \l 235AF
  \l 235B0
  \l 235B1
  \l 235B2
  \l 235B3
  \l 235B4
  \l 235B5
  \l 235B6
  \l 235B7
  \l 235B8
  \l 235B9
  \l 235BA
  \l 235BB
  \l 235BC
  \l 235BD
  \l 235BE
  \l 235BF
  \l 235C0
  \l 235C1
  \l 235C2
  \l 235C3
  \l 235C4
  \l 235C5
  \l 235C6
  \l 235C7
  \l 235C8
  \l 235C9
  \l 235CA
  \l 235CB
  \l 235CC
  \l 235CD
  \l 235CE
  \l 235CF
  \l 235D0
  \l 235D1
  \l 235D2
  \l 235D3
  \l 235D4
  \l 235D5
  \l 235D6
  \l 235D7
  \l 235D8
  \l 235D9
  \l 235DA
  \l 235DB
  \l 235DC
  \l 235DD
  \l 235DE
  \l 235DF
  \l 235E0
  \l 235E1
  \l 235E2
  \l 235E3
  \l 235E4
  \l 235E5
  \l 235E6
  \l 235E7
  \l 235E8
  \l 235E9
  \l 235EA
  \l 235EB
  \l 235EC
  \l 235ED
  \l 235EE
  \l 235EF
  \l 235F0
  \l 235F1
  \l 235F2
  \l 235F3
  \l 235F4
  \l 235F5
  \l 235F6
  \l 235F7
  \l 235F8
  \l 235F9
  \l 235FA
  \l 235FB
  \l 235FC
  \l 235FD
  \l 235FE
  \l 235FF
  \l 23600
  \l 23601
  \l 23602
  \l 23603
  \l 23604
  \l 23605
  \l 23606
  \l 23607
  \l 23608
  \l 23609
  \l 2360A
  \l 2360B
  \l 2360C
  \l 2360D
  \l 2360E
  \l 2360F
  \l 23610
  \l 23611
  \l 23612
  \l 23613
  \l 23614
  \l 23615
  \l 23616
  \l 23617
  \l 23618
  \l 23619
  \l 2361A
  \l 2361B
  \l 2361C
  \l 2361D
  \l 2361E
  \l 2361F
  \l 23620
  \l 23621
  \l 23622
  \l 23623
  \l 23624
  \l 23625
  \l 23626
  \l 23627
  \l 23628
  \l 23629
  \l 2362A
  \l 2362B
  \l 2362C
  \l 2362D
  \l 2362E
  \l 2362F
  \l 23630
  \l 23631
  \l 23632
  \l 23633
  \l 23634
  \l 23635
  \l 23636
  \l 23637
  \l 23638
  \l 23639
  \l 2363A
  \l 2363B
  \l 2363C
  \l 2363D
  \l 2363E
  \l 2363F
  \l 23640
  \l 23641
  \l 23642
  \l 23643
  \l 23644
  \l 23645
  \l 23646
  \l 23647
  \l 23648
  \l 23649
  \l 2364A
  \l 2364B
  \l 2364C
  \l 2364D
  \l 2364E
  \l 2364F
  \l 23650
  \l 23651
  \l 23652
  \l 23653
  \l 23654
  \l 23655
  \l 23656
  \l 23657
  \l 23658
  \l 23659
  \l 2365A
  \l 2365B
  \l 2365C
  \l 2365D
  \l 2365E
  \l 2365F
  \l 23660
  \l 23661
  \l 23662
  \l 23663
  \l 23664
  \l 23665
  \l 23666
  \l 23667
  \l 23668
  \l 23669
  \l 2366A
  \l 2366B
  \l 2366C
  \l 2366D
  \l 2366E
  \l 2366F
  \l 23670
  \l 23671
  \l 23672
  \l 23673
  \l 23674
  \l 23675
  \l 23676
  \l 23677
  \l 23678
  \l 23679
  \l 2367A
  \l 2367B
  \l 2367C
  \l 2367D
  \l 2367E
  \l 2367F
  \l 23680
  \l 23681
  \l 23682
  \l 23683
  \l 23684
  \l 23685
  \l 23686
  \l 23687
  \l 23688
  \l 23689
  \l 2368A
  \l 2368B
  \l 2368C
  \l 2368D
  \l 2368E
  \l 2368F
  \l 23690
  \l 23691
  \l 23692
  \l 23693
  \l 23694
  \l 23695
  \l 23696
  \l 23697
  \l 23698
  \l 23699
  \l 2369A
  \l 2369B
  \l 2369C
  \l 2369D
  \l 2369E
  \l 2369F
  \l 236A0
  \l 236A1
  \l 236A2
  \l 236A3
  \l 236A4
  \l 236A5
  \l 236A6
  \l 236A7
  \l 236A8
  \l 236A9
  \l 236AA
  \l 236AB
  \l 236AC
  \l 236AD
  \l 236AE
  \l 236AF
  \l 236B0
  \l 236B1
  \l 236B2
  \l 236B3
  \l 236B4
  \l 236B5
  \l 236B6
  \l 236B7
  \l 236B8
  \l 236B9
  \l 236BA
  \l 236BB
  \l 236BC
  \l 236BD
  \l 236BE
  \l 236BF
  \l 236C0
  \l 236C1
  \l 236C2
  \l 236C3
  \l 236C4
  \l 236C5
  \l 236C6
  \l 236C7
  \l 236C8
  \l 236C9
  \l 236CA
  \l 236CB
  \l 236CC
  \l 236CD
  \l 236CE
  \l 236CF
  \l 236D0
  \l 236D1
  \l 236D2
  \l 236D3
  \l 236D4
  \l 236D5
  \l 236D6
  \l 236D7
  \l 236D8
  \l 236D9
  \l 236DA
  \l 236DB
  \l 236DC
  \l 236DD
  \l 236DE
  \l 236DF
  \l 236E0
  \l 236E1
  \l 236E2
  \l 236E3
  \l 236E4
  \l 236E5
  \l 236E6
  \l 236E7
  \l 236E8
  \l 236E9
  \l 236EA
  \l 236EB
  \l 236EC
  \l 236ED
  \l 236EE
  \l 236EF
  \l 236F0
  \l 236F1
  \l 236F2
  \l 236F3
  \l 236F4
  \l 236F5
  \l 236F6
  \l 236F7
  \l 236F8
  \l 236F9
  \l 236FA
  \l 236FB
  \l 236FC
  \l 236FD
  \l 236FE
  \l 236FF
  \l 23700
  \l 23701
  \l 23702
  \l 23703
  \l 23704
  \l 23705
  \l 23706
  \l 23707
  \l 23708
  \l 23709
  \l 2370A
  \l 2370B
  \l 2370C
  \l 2370D
  \l 2370E
  \l 2370F
  \l 23710
  \l 23711
  \l 23712
  \l 23713
  \l 23714
  \l 23715
  \l 23716
  \l 23717
  \l 23718
  \l 23719
  \l 2371A
  \l 2371B
  \l 2371C
  \l 2371D
  \l 2371E
  \l 2371F
  \l 23720
  \l 23721
  \l 23722
  \l 23723
  \l 23724
  \l 23725
  \l 23726
  \l 23727
  \l 23728
  \l 23729
  \l 2372A
  \l 2372B
  \l 2372C
  \l 2372D
  \l 2372E
  \l 2372F
  \l 23730
  \l 23731
  \l 23732
  \l 23733
  \l 23734
  \l 23735
  \l 23736
  \l 23737
  \l 23738
  \l 23739
  \l 2373A
  \l 2373B
  \l 2373C
  \l 2373D
  \l 2373E
  \l 2373F
  \l 23740
  \l 23741
  \l 23742
  \l 23743
  \l 23744
  \l 23745
  \l 23746
  \l 23747
  \l 23748
  \l 23749
  \l 2374A
  \l 2374B
  \l 2374C
  \l 2374D
  \l 2374E
  \l 2374F
  \l 23750
  \l 23751
  \l 23752
  \l 23753
  \l 23754
  \l 23755
  \l 23756
  \l 23757
  \l 23758
  \l 23759
  \l 2375A
  \l 2375B
  \l 2375C
  \l 2375D
  \l 2375E
  \l 2375F
  \l 23760
  \l 23761
  \l 23762
  \l 23763
  \l 23764
  \l 23765
  \l 23766
  \l 23767
  \l 23768
  \l 23769
  \l 2376A
  \l 2376B
  \l 2376C
  \l 2376D
  \l 2376E
  \l 2376F
  \l 23770
  \l 23771
  \l 23772
  \l 23773
  \l 23774
  \l 23775
  \l 23776
  \l 23777
  \l 23778
  \l 23779
  \l 2377A
  \l 2377B
  \l 2377C
  \l 2377D
  \l 2377E
  \l 2377F
  \l 23780
  \l 23781
  \l 23782
  \l 23783
  \l 23784
  \l 23785
  \l 23786
  \l 23787
  \l 23788
  \l 23789
  \l 2378A
  \l 2378B
  \l 2378C
  \l 2378D
  \l 2378E
  \l 2378F
  \l 23790
  \l 23791
  \l 23792
  \l 23793
  \l 23794
  \l 23795
  \l 23796
  \l 23797
  \l 23798
  \l 23799
  \l 2379A
  \l 2379B
  \l 2379C
  \l 2379D
  \l 2379E
  \l 2379F
  \l 237A0
  \l 237A1
  \l 237A2
  \l 237A3
  \l 237A4
  \l 237A5
  \l 237A6
  \l 237A7
  \l 237A8
  \l 237A9
  \l 237AA
  \l 237AB
  \l 237AC
  \l 237AD
  \l 237AE
  \l 237AF
  \l 237B0
  \l 237B1
  \l 237B2
  \l 237B3
  \l 237B4
  \l 237B5
  \l 237B6
  \l 237B7
  \l 237B8
  \l 237B9
  \l 237BA
  \l 237BB
  \l 237BC
  \l 237BD
  \l 237BE
  \l 237BF
  \l 237C0
  \l 237C1
  \l 237C2
  \l 237C3
  \l 237C4
  \l 237C5
  \l 237C6
  \l 237C7
  \l 237C8
  \l 237C9
  \l 237CA
  \l 237CB
  \l 237CC
  \l 237CD
  \l 237CE
  \l 237CF
  \l 237D0
  \l 237D1
  \l 237D2
  \l 237D3
  \l 237D4
  \l 237D5
  \l 237D6
  \l 237D7
  \l 237D8
  \l 237D9
  \l 237DA
  \l 237DB
  \l 237DC
  \l 237DD
  \l 237DE
  \l 237DF
  \l 237E0
  \l 237E1
  \l 237E2
  \l 237E3
  \l 237E4
  \l 237E5
  \l 237E6
  \l 237E7
  \l 237E8
  \l 237E9
  \l 237EA
  \l 237EB
  \l 237EC
  \l 237ED
  \l 237EE
  \l 237EF
  \l 237F0
  \l 237F1
  \l 237F2
  \l 237F3
  \l 237F4
  \l 237F5
  \l 237F6
  \l 237F7
  \l 237F8
  \l 237F9
  \l 237FA
  \l 237FB
  \l 237FC
  \l 237FD
  \l 237FE
  \l 237FF
  \l 23800
  \l 23801
  \l 23802
  \l 23803
  \l 23804
  \l 23805
  \l 23806
  \l 23807
  \l 23808
  \l 23809
  \l 2380A
  \l 2380B
  \l 2380C
  \l 2380D
  \l 2380E
  \l 2380F
  \l 23810
  \l 23811
  \l 23812
  \l 23813
  \l 23814
  \l 23815
  \l 23816
  \l 23817
  \l 23818
  \l 23819
  \l 2381A
  \l 2381B
  \l 2381C
  \l 2381D
  \l 2381E
  \l 2381F
  \l 23820
  \l 23821
  \l 23822
  \l 23823
  \l 23824
  \l 23825
  \l 23826
  \l 23827
  \l 23828
  \l 23829
  \l 2382A
  \l 2382B
  \l 2382C
  \l 2382D
  \l 2382E
  \l 2382F
  \l 23830
  \l 23831
  \l 23832
  \l 23833
  \l 23834
  \l 23835
  \l 23836
  \l 23837
  \l 23838
  \l 23839
  \l 2383A
  \l 2383B
  \l 2383C
  \l 2383D
  \l 2383E
  \l 2383F
  \l 23840
  \l 23841
  \l 23842
  \l 23843
  \l 23844
  \l 23845
  \l 23846
  \l 23847
  \l 23848
  \l 23849
  \l 2384A
  \l 2384B
  \l 2384C
  \l 2384D
  \l 2384E
  \l 2384F
  \l 23850
  \l 23851
  \l 23852
  \l 23853
  \l 23854
  \l 23855
  \l 23856
  \l 23857
  \l 23858
  \l 23859
  \l 2385A
  \l 2385B
  \l 2385C
  \l 2385D
  \l 2385E
  \l 2385F
  \l 23860
  \l 23861
  \l 23862
  \l 23863
  \l 23864
  \l 23865
  \l 23866
  \l 23867
  \l 23868
  \l 23869
  \l 2386A
  \l 2386B
  \l 2386C
  \l 2386D
  \l 2386E
  \l 2386F
  \l 23870
  \l 23871
  \l 23872
  \l 23873
  \l 23874
  \l 23875
  \l 23876
  \l 23877
  \l 23878
  \l 23879
  \l 2387A
  \l 2387B
  \l 2387C
  \l 2387D
  \l 2387E
  \l 2387F
  \l 23880
  \l 23881
  \l 23882
  \l 23883
  \l 23884
  \l 23885
  \l 23886
  \l 23887
  \l 23888
  \l 23889
  \l 2388A
  \l 2388B
  \l 2388C
  \l 2388D
  \l 2388E
  \l 2388F
  \l 23890
  \l 23891
  \l 23892
  \l 23893
  \l 23894
  \l 23895
  \l 23896
  \l 23897
  \l 23898
  \l 23899
  \l 2389A
  \l 2389B
  \l 2389C
  \l 2389D
  \l 2389E
  \l 2389F
  \l 238A0
  \l 238A1
  \l 238A2
  \l 238A3
  \l 238A4
  \l 238A5
  \l 238A6
  \l 238A7
  \l 238A8
  \l 238A9
  \l 238AA
  \l 238AB
  \l 238AC
  \l 238AD
  \l 238AE
  \l 238AF
  \l 238B0
  \l 238B1
  \l 238B2
  \l 238B3
  \l 238B4
  \l 238B5
  \l 238B6
  \l 238B7
  \l 238B8
  \l 238B9
  \l 238BA
  \l 238BB
  \l 238BC
  \l 238BD
  \l 238BE
  \l 238BF
  \l 238C0
  \l 238C1
  \l 238C2
  \l 238C3
  \l 238C4
  \l 238C5
  \l 238C6
  \l 238C7
  \l 238C8
  \l 238C9
  \l 238CA
  \l 238CB
  \l 238CC
  \l 238CD
  \l 238CE
  \l 238CF
  \l 238D0
  \l 238D1
  \l 238D2
  \l 238D3
  \l 238D4
  \l 238D5
  \l 238D6
  \l 238D7
  \l 238D8
  \l 238D9
  \l 238DA
  \l 238DB
  \l 238DC
  \l 238DD
  \l 238DE
  \l 238DF
  \l 238E0
  \l 238E1
  \l 238E2
  \l 238E3
  \l 238E4
  \l 238E5
  \l 238E6
  \l 238E7
  \l 238E8
  \l 238E9
  \l 238EA
  \l 238EB
  \l 238EC
  \l 238ED
  \l 238EE
  \l 238EF
  \l 238F0
  \l 238F1
  \l 238F2
  \l 238F3
  \l 238F4
  \l 238F5
  \l 238F6
  \l 238F7
  \l 238F8
  \l 238F9
  \l 238FA
  \l 238FB
  \l 238FC
  \l 238FD
  \l 238FE
  \l 238FF
  \l 23900
  \l 23901
  \l 23902
  \l 23903
  \l 23904
  \l 23905
  \l 23906
  \l 23907
  \l 23908
  \l 23909
  \l 2390A
  \l 2390B
  \l 2390C
  \l 2390D
  \l 2390E
  \l 2390F
  \l 23910
  \l 23911
  \l 23912
  \l 23913
  \l 23914
  \l 23915
  \l 23916
  \l 23917
  \l 23918
  \l 23919
  \l 2391A
  \l 2391B
  \l 2391C
  \l 2391D
  \l 2391E
  \l 2391F
  \l 23920
  \l 23921
  \l 23922
  \l 23923
  \l 23924
  \l 23925
  \l 23926
  \l 23927
  \l 23928
  \l 23929
  \l 2392A
  \l 2392B
  \l 2392C
  \l 2392D
  \l 2392E
  \l 2392F
  \l 23930
  \l 23931
  \l 23932
  \l 23933
  \l 23934
  \l 23935
  \l 23936
  \l 23937
  \l 23938
  \l 23939
  \l 2393A
  \l 2393B
  \l 2393C
  \l 2393D
  \l 2393E
  \l 2393F
  \l 23940
  \l 23941
  \l 23942
  \l 23943
  \l 23944
  \l 23945
  \l 23946
  \l 23947
  \l 23948
  \l 23949
  \l 2394A
  \l 2394B
  \l 2394C
  \l 2394D
  \l 2394E
  \l 2394F
  \l 23950
  \l 23951
  \l 23952
  \l 23953
  \l 23954
  \l 23955
  \l 23956
  \l 23957
  \l 23958
  \l 23959
  \l 2395A
  \l 2395B
  \l 2395C
  \l 2395D
  \l 2395E
  \l 2395F
  \l 23960
  \l 23961
  \l 23962
  \l 23963
  \l 23964
  \l 23965
  \l 23966
  \l 23967
  \l 23968
  \l 23969
  \l 2396A
  \l 2396B
  \l 2396C
  \l 2396D
  \l 2396E
  \l 2396F
  \l 23970
  \l 23971
  \l 23972
  \l 23973
  \l 23974
  \l 23975
  \l 23976
  \l 23977
  \l 23978
  \l 23979
  \l 2397A
  \l 2397B
  \l 2397C
  \l 2397D
  \l 2397E
  \l 2397F
  \l 23980
  \l 23981
  \l 23982
  \l 23983
  \l 23984
  \l 23985
  \l 23986
  \l 23987
  \l 23988
  \l 23989
  \l 2398A
  \l 2398B
  \l 2398C
  \l 2398D
  \l 2398E
  \l 2398F
  \l 23990
  \l 23991
  \l 23992
  \l 23993
  \l 23994
  \l 23995
  \l 23996
  \l 23997
  \l 23998
  \l 23999
  \l 2399A
  \l 2399B
  \l 2399C
  \l 2399D
  \l 2399E
  \l 2399F
  \l 239A0
  \l 239A1
  \l 239A2
  \l 239A3
  \l 239A4
  \l 239A5
  \l 239A6
  \l 239A7
  \l 239A8
  \l 239A9
  \l 239AA
  \l 239AB
  \l 239AC
  \l 239AD
  \l 239AE
  \l 239AF
  \l 239B0
  \l 239B1
  \l 239B2
  \l 239B3
  \l 239B4
  \l 239B5
  \l 239B6
  \l 239B7
  \l 239B8
  \l 239B9
  \l 239BA
  \l 239BB
  \l 239BC
  \l 239BD
  \l 239BE
  \l 239BF
  \l 239C0
  \l 239C1
  \l 239C2
  \l 239C3
  \l 239C4
  \l 239C5
  \l 239C6
  \l 239C7
  \l 239C8
  \l 239C9
  \l 239CA
  \l 239CB
  \l 239CC
  \l 239CD
  \l 239CE
  \l 239CF
  \l 239D0
  \l 239D1
  \l 239D2
  \l 239D3
  \l 239D4
  \l 239D5
  \l 239D6
  \l 239D7
  \l 239D8
  \l 239D9
  \l 239DA
  \l 239DB
  \l 239DC
  \l 239DD
  \l 239DE
  \l 239DF
  \l 239E0
  \l 239E1
  \l 239E2
  \l 239E3
  \l 239E4
  \l 239E5
  \l 239E6
  \l 239E7
  \l 239E8
  \l 239E9
  \l 239EA
  \l 239EB
  \l 239EC
  \l 239ED
  \l 239EE
  \l 239EF
  \l 239F0
  \l 239F1
  \l 239F2
  \l 239F3
  \l 239F4
  \l 239F5
  \l 239F6
  \l 239F7
  \l 239F8
  \l 239F9
  \l 239FA
  \l 239FB
  \l 239FC
  \l 239FD
  \l 239FE
  \l 239FF
  \l 23A00
  \l 23A01
  \l 23A02
  \l 23A03
  \l 23A04
  \l 23A05
  \l 23A06
  \l 23A07
  \l 23A08
  \l 23A09
  \l 23A0A
  \l 23A0B
  \l 23A0C
  \l 23A0D
  \l 23A0E
  \l 23A0F
  \l 23A10
  \l 23A11
  \l 23A12
  \l 23A13
  \l 23A14
  \l 23A15
  \l 23A16
  \l 23A17
  \l 23A18
  \l 23A19
  \l 23A1A
  \l 23A1B
  \l 23A1C
  \l 23A1D
  \l 23A1E
  \l 23A1F
  \l 23A20
  \l 23A21
  \l 23A22
  \l 23A23
  \l 23A24
  \l 23A25
  \l 23A26
  \l 23A27
  \l 23A28
  \l 23A29
  \l 23A2A
  \l 23A2B
  \l 23A2C
  \l 23A2D
  \l 23A2E
  \l 23A2F
  \l 23A30
  \l 23A31
  \l 23A32
  \l 23A33
  \l 23A34
  \l 23A35
  \l 23A36
  \l 23A37
  \l 23A38
  \l 23A39
  \l 23A3A
  \l 23A3B
  \l 23A3C
  \l 23A3D
  \l 23A3E
  \l 23A3F
  \l 23A40
  \l 23A41
  \l 23A42
  \l 23A43
  \l 23A44
  \l 23A45
  \l 23A46
  \l 23A47
  \l 23A48
  \l 23A49
  \l 23A4A
  \l 23A4B
  \l 23A4C
  \l 23A4D
  \l 23A4E
  \l 23A4F
  \l 23A50
  \l 23A51
  \l 23A52
  \l 23A53
  \l 23A54
  \l 23A55
  \l 23A56
  \l 23A57
  \l 23A58
  \l 23A59
  \l 23A5A
  \l 23A5B
  \l 23A5C
  \l 23A5D
  \l 23A5E
  \l 23A5F
  \l 23A60
  \l 23A61
  \l 23A62
  \l 23A63
  \l 23A64
  \l 23A65
  \l 23A66
  \l 23A67
  \l 23A68
  \l 23A69
  \l 23A6A
  \l 23A6B
  \l 23A6C
  \l 23A6D
  \l 23A6E
  \l 23A6F
  \l 23A70
  \l 23A71
  \l 23A72
  \l 23A73
  \l 23A74
  \l 23A75
  \l 23A76
  \l 23A77
  \l 23A78
  \l 23A79
  \l 23A7A
  \l 23A7B
  \l 23A7C
  \l 23A7D
  \l 23A7E
  \l 23A7F
  \l 23A80
  \l 23A81
  \l 23A82
  \l 23A83
  \l 23A84
  \l 23A85
  \l 23A86
  \l 23A87
  \l 23A88
  \l 23A89
  \l 23A8A
  \l 23A8B
  \l 23A8C
  \l 23A8D
  \l 23A8E
  \l 23A8F
  \l 23A90
  \l 23A91
  \l 23A92
  \l 23A93
  \l 23A94
  \l 23A95
  \l 23A96
  \l 23A97
  \l 23A98
  \l 23A99
  \l 23A9A
  \l 23A9B
  \l 23A9C
  \l 23A9D
  \l 23A9E
  \l 23A9F
  \l 23AA0
  \l 23AA1
  \l 23AA2
  \l 23AA3
  \l 23AA4
  \l 23AA5
  \l 23AA6
  \l 23AA7
  \l 23AA8
  \l 23AA9
  \l 23AAA
  \l 23AAB
  \l 23AAC
  \l 23AAD
  \l 23AAE
  \l 23AAF
  \l 23AB0
  \l 23AB1
  \l 23AB2
  \l 23AB3
  \l 23AB4
  \l 23AB5
  \l 23AB6
  \l 23AB7
  \l 23AB8
  \l 23AB9
  \l 23ABA
  \l 23ABB
  \l 23ABC
  \l 23ABD
  \l 23ABE
  \l 23ABF
  \l 23AC0
  \l 23AC1
  \l 23AC2
  \l 23AC3
  \l 23AC4
  \l 23AC5
  \l 23AC6
  \l 23AC7
  \l 23AC8
  \l 23AC9
  \l 23ACA
  \l 23ACB
  \l 23ACC
  \l 23ACD
  \l 23ACE
  \l 23ACF
  \l 23AD0
  \l 23AD1
  \l 23AD2
  \l 23AD3
  \l 23AD4
  \l 23AD5
  \l 23AD6
  \l 23AD7
  \l 23AD8
  \l 23AD9
  \l 23ADA
  \l 23ADB
  \l 23ADC
  \l 23ADD
  \l 23ADE
  \l 23ADF
  \l 23AE0
  \l 23AE1
  \l 23AE2
  \l 23AE3
  \l 23AE4
  \l 23AE5
  \l 23AE6
  \l 23AE7
  \l 23AE8
  \l 23AE9
  \l 23AEA
  \l 23AEB
  \l 23AEC
  \l 23AED
  \l 23AEE
  \l 23AEF
  \l 23AF0
  \l 23AF1
  \l 23AF2
  \l 23AF3
  \l 23AF4
  \l 23AF5
  \l 23AF6
  \l 23AF7
  \l 23AF8
  \l 23AF9
  \l 23AFA
  \l 23AFB
  \l 23AFC
  \l 23AFD
  \l 23AFE
  \l 23AFF
  \l 23B00
  \l 23B01
  \l 23B02
  \l 23B03
  \l 23B04
  \l 23B05
  \l 23B06
  \l 23B07
  \l 23B08
  \l 23B09
  \l 23B0A
  \l 23B0B
  \l 23B0C
  \l 23B0D
  \l 23B0E
  \l 23B0F
  \l 23B10
  \l 23B11
  \l 23B12
  \l 23B13
  \l 23B14
  \l 23B15
  \l 23B16
  \l 23B17
  \l 23B18
  \l 23B19
  \l 23B1A
  \l 23B1B
  \l 23B1C
  \l 23B1D
  \l 23B1E
  \l 23B1F
  \l 23B20
  \l 23B21
  \l 23B22
  \l 23B23
  \l 23B24
  \l 23B25
  \l 23B26
  \l 23B27
  \l 23B28
  \l 23B29
  \l 23B2A
  \l 23B2B
  \l 23B2C
  \l 23B2D
  \l 23B2E
  \l 23B2F
  \l 23B30
  \l 23B31
  \l 23B32
  \l 23B33
  \l 23B34
  \l 23B35
  \l 23B36
  \l 23B37
  \l 23B38
  \l 23B39
  \l 23B3A
  \l 23B3B
  \l 23B3C
  \l 23B3D
  \l 23B3E
  \l 23B3F
  \l 23B40
  \l 23B41
  \l 23B42
  \l 23B43
  \l 23B44
  \l 23B45
  \l 23B46
  \l 23B47
  \l 23B48
  \l 23B49
  \l 23B4A
  \l 23B4B
  \l 23B4C
  \l 23B4D
  \l 23B4E
  \l 23B4F
  \l 23B50
  \l 23B51
  \l 23B52
  \l 23B53
  \l 23B54
  \l 23B55
  \l 23B56
  \l 23B57
  \l 23B58
  \l 23B59
  \l 23B5A
  \l 23B5B
  \l 23B5C
  \l 23B5D
  \l 23B5E
  \l 23B5F
  \l 23B60
  \l 23B61
  \l 23B62
  \l 23B63
  \l 23B64
  \l 23B65
  \l 23B66
  \l 23B67
  \l 23B68
  \l 23B69
  \l 23B6A
  \l 23B6B
  \l 23B6C
  \l 23B6D
  \l 23B6E
  \l 23B6F
  \l 23B70
  \l 23B71
  \l 23B72
  \l 23B73
  \l 23B74
  \l 23B75
  \l 23B76
  \l 23B77
  \l 23B78
  \l 23B79
  \l 23B7A
  \l 23B7B
  \l 23B7C
  \l 23B7D
  \l 23B7E
  \l 23B7F
  \l 23B80
  \l 23B81
  \l 23B82
  \l 23B83
  \l 23B84
  \l 23B85
  \l 23B86
  \l 23B87
  \l 23B88
  \l 23B89
  \l 23B8A
  \l 23B8B
  \l 23B8C
  \l 23B8D
  \l 23B8E
  \l 23B8F
  \l 23B90
  \l 23B91
  \l 23B92
  \l 23B93
  \l 23B94
  \l 23B95
  \l 23B96
  \l 23B97
  \l 23B98
  \l 23B99
  \l 23B9A
  \l 23B9B
  \l 23B9C
  \l 23B9D
  \l 23B9E
  \l 23B9F
  \l 23BA0
  \l 23BA1
  \l 23BA2
  \l 23BA3
  \l 23BA4
  \l 23BA5
  \l 23BA6
  \l 23BA7
  \l 23BA8
  \l 23BA9
  \l 23BAA
  \l 23BAB
  \l 23BAC
  \l 23BAD
  \l 23BAE
  \l 23BAF
  \l 23BB0
  \l 23BB1
  \l 23BB2
  \l 23BB3
  \l 23BB4
  \l 23BB5
  \l 23BB6
  \l 23BB7
  \l 23BB8
  \l 23BB9
  \l 23BBA
  \l 23BBB
  \l 23BBC
  \l 23BBD
  \l 23BBE
  \l 23BBF
  \l 23BC0
  \l 23BC1
  \l 23BC2
  \l 23BC3
  \l 23BC4
  \l 23BC5
  \l 23BC6
  \l 23BC7
  \l 23BC8
  \l 23BC9
  \l 23BCA
  \l 23BCB
  \l 23BCC
  \l 23BCD
  \l 23BCE
  \l 23BCF
  \l 23BD0
  \l 23BD1
  \l 23BD2
  \l 23BD3
  \l 23BD4
  \l 23BD5
  \l 23BD6
  \l 23BD7
  \l 23BD8
  \l 23BD9
  \l 23BDA
  \l 23BDB
  \l 23BDC
  \l 23BDD
  \l 23BDE
  \l 23BDF
  \l 23BE0
  \l 23BE1
  \l 23BE2
  \l 23BE3
  \l 23BE4
  \l 23BE5
  \l 23BE6
  \l 23BE7
  \l 23BE8
  \l 23BE9
  \l 23BEA
  \l 23BEB
  \l 23BEC
  \l 23BED
  \l 23BEE
  \l 23BEF
  \l 23BF0
  \l 23BF1
  \l 23BF2
  \l 23BF3
  \l 23BF4
  \l 23BF5
  \l 23BF6
  \l 23BF7
  \l 23BF8
  \l 23BF9
  \l 23BFA
  \l 23BFB
  \l 23BFC
  \l 23BFD
  \l 23BFE
  \l 23BFF
  \l 23C00
  \l 23C01
  \l 23C02
  \l 23C03
  \l 23C04
  \l 23C05
  \l 23C06
  \l 23C07
  \l 23C08
  \l 23C09
  \l 23C0A
  \l 23C0B
  \l 23C0C
  \l 23C0D
  \l 23C0E
  \l 23C0F
  \l 23C10
  \l 23C11
  \l 23C12
  \l 23C13
  \l 23C14
  \l 23C15
  \l 23C16
  \l 23C17
  \l 23C18
  \l 23C19
  \l 23C1A
  \l 23C1B
  \l 23C1C
  \l 23C1D
  \l 23C1E
  \l 23C1F
  \l 23C20
  \l 23C21
  \l 23C22
  \l 23C23
  \l 23C24
  \l 23C25
  \l 23C26
  \l 23C27
  \l 23C28
  \l 23C29
  \l 23C2A
  \l 23C2B
  \l 23C2C
  \l 23C2D
  \l 23C2E
  \l 23C2F
  \l 23C30
  \l 23C31
  \l 23C32
  \l 23C33
  \l 23C34
  \l 23C35
  \l 23C36
  \l 23C37
  \l 23C38
  \l 23C39
  \l 23C3A
  \l 23C3B
  \l 23C3C
  \l 23C3D
  \l 23C3E
  \l 23C3F
  \l 23C40
  \l 23C41
  \l 23C42
  \l 23C43
  \l 23C44
  \l 23C45
  \l 23C46
  \l 23C47
  \l 23C48
  \l 23C49
  \l 23C4A
  \l 23C4B
  \l 23C4C
  \l 23C4D
  \l 23C4E
  \l 23C4F
  \l 23C50
  \l 23C51
  \l 23C52
  \l 23C53
  \l 23C54
  \l 23C55
  \l 23C56
  \l 23C57
  \l 23C58
  \l 23C59
  \l 23C5A
  \l 23C5B
  \l 23C5C
  \l 23C5D
  \l 23C5E
  \l 23C5F
  \l 23C60
  \l 23C61
  \l 23C62
  \l 23C63
  \l 23C64
  \l 23C65
  \l 23C66
  \l 23C67
  \l 23C68
  \l 23C69
  \l 23C6A
  \l 23C6B
  \l 23C6C
  \l 23C6D
  \l 23C6E
  \l 23C6F
  \l 23C70
  \l 23C71
  \l 23C72
  \l 23C73
  \l 23C74
  \l 23C75
  \l 23C76
  \l 23C77
  \l 23C78
  \l 23C79
  \l 23C7A
  \l 23C7B
  \l 23C7C
  \l 23C7D
  \l 23C7E
  \l 23C7F
  \l 23C80
  \l 23C81
  \l 23C82
  \l 23C83
  \l 23C84
  \l 23C85
  \l 23C86
  \l 23C87
  \l 23C88
  \l 23C89
  \l 23C8A
  \l 23C8B
  \l 23C8C
  \l 23C8D
  \l 23C8E
  \l 23C8F
  \l 23C90
  \l 23C91
  \l 23C92
  \l 23C93
  \l 23C94
  \l 23C95
  \l 23C96
  \l 23C97
  \l 23C98
  \l 23C99
  \l 23C9A
  \l 23C9B
  \l 23C9C
  \l 23C9D
  \l 23C9E
  \l 23C9F
  \l 23CA0
  \l 23CA1
  \l 23CA2
  \l 23CA3
  \l 23CA4
  \l 23CA5
  \l 23CA6
  \l 23CA7
  \l 23CA8
  \l 23CA9
  \l 23CAA
  \l 23CAB
  \l 23CAC
  \l 23CAD
  \l 23CAE
  \l 23CAF
  \l 23CB0
  \l 23CB1
  \l 23CB2
  \l 23CB3
  \l 23CB4
  \l 23CB5
  \l 23CB6
  \l 23CB7
  \l 23CB8
  \l 23CB9
  \l 23CBA
  \l 23CBB
  \l 23CBC
  \l 23CBD
  \l 23CBE
  \l 23CBF
  \l 23CC0
  \l 23CC1
  \l 23CC2
  \l 23CC3
  \l 23CC4
  \l 23CC5
  \l 23CC6
  \l 23CC7
  \l 23CC8
  \l 23CC9
  \l 23CCA
  \l 23CCB
  \l 23CCC
  \l 23CCD
  \l 23CCE
  \l 23CCF
  \l 23CD0
  \l 23CD1
  \l 23CD2
  \l 23CD3
  \l 23CD4
  \l 23CD5
  \l 23CD6
  \l 23CD7
  \l 23CD8
  \l 23CD9
  \l 23CDA
  \l 23CDB
  \l 23CDC
  \l 23CDD
  \l 23CDE
  \l 23CDF
  \l 23CE0
  \l 23CE1
  \l 23CE2
  \l 23CE3
  \l 23CE4
  \l 23CE5
  \l 23CE6
  \l 23CE7
  \l 23CE8
  \l 23CE9
  \l 23CEA
  \l 23CEB
  \l 23CEC
  \l 23CED
  \l 23CEE
  \l 23CEF
  \l 23CF0
  \l 23CF1
  \l 23CF2
  \l 23CF3
  \l 23CF4
  \l 23CF5
  \l 23CF6
  \l 23CF7
  \l 23CF8
  \l 23CF9
  \l 23CFA
  \l 23CFB
  \l 23CFC
  \l 23CFD
  \l 23CFE
  \l 23CFF
  \l 23D00
  \l 23D01
  \l 23D02
  \l 23D03
  \l 23D04
  \l 23D05
  \l 23D06
  \l 23D07
  \l 23D08
  \l 23D09
  \l 23D0A
  \l 23D0B
  \l 23D0C
  \l 23D0D
  \l 23D0E
  \l 23D0F
  \l 23D10
  \l 23D11
  \l 23D12
  \l 23D13
  \l 23D14
  \l 23D15
  \l 23D16
  \l 23D17
  \l 23D18
  \l 23D19
  \l 23D1A
  \l 23D1B
  \l 23D1C
  \l 23D1D
  \l 23D1E
  \l 23D1F
  \l 23D20
  \l 23D21
  \l 23D22
  \l 23D23
  \l 23D24
  \l 23D25
  \l 23D26
  \l 23D27
  \l 23D28
  \l 23D29
  \l 23D2A
  \l 23D2B
  \l 23D2C
  \l 23D2D
  \l 23D2E
  \l 23D2F
  \l 23D30
  \l 23D31
  \l 23D32
  \l 23D33
  \l 23D34
  \l 23D35
  \l 23D36
  \l 23D37
  \l 23D38
  \l 23D39
  \l 23D3A
  \l 23D3B
  \l 23D3C
  \l 23D3D
  \l 23D3E
  \l 23D3F
  \l 23D40
  \l 23D41
  \l 23D42
  \l 23D43
  \l 23D44
  \l 23D45
  \l 23D46
  \l 23D47
  \l 23D48
  \l 23D49
  \l 23D4A
  \l 23D4B
  \l 23D4C
  \l 23D4D
  \l 23D4E
  \l 23D4F
  \l 23D50
  \l 23D51
  \l 23D52
  \l 23D53
  \l 23D54
  \l 23D55
  \l 23D56
  \l 23D57
  \l 23D58
  \l 23D59
  \l 23D5A
  \l 23D5B
  \l 23D5C
  \l 23D5D
  \l 23D5E
  \l 23D5F
  \l 23D60
  \l 23D61
  \l 23D62
  \l 23D63
  \l 23D64
  \l 23D65
  \l 23D66
  \l 23D67
  \l 23D68
  \l 23D69
  \l 23D6A
  \l 23D6B
  \l 23D6C
  \l 23D6D
  \l 23D6E
  \l 23D6F
  \l 23D70
  \l 23D71
  \l 23D72
  \l 23D73
  \l 23D74
  \l 23D75
  \l 23D76
  \l 23D77
  \l 23D78
  \l 23D79
  \l 23D7A
  \l 23D7B
  \l 23D7C
  \l 23D7D
  \l 23D7E
  \l 23D7F
  \l 23D80
  \l 23D81
  \l 23D82
  \l 23D83
  \l 23D84
  \l 23D85
  \l 23D86
  \l 23D87
  \l 23D88
  \l 23D89
  \l 23D8A
  \l 23D8B
  \l 23D8C
  \l 23D8D
  \l 23D8E
  \l 23D8F
  \l 23D90
  \l 23D91
  \l 23D92
  \l 23D93
  \l 23D94
  \l 23D95
  \l 23D96
  \l 23D97
  \l 23D98
  \l 23D99
  \l 23D9A
  \l 23D9B
  \l 23D9C
  \l 23D9D
  \l 23D9E
  \l 23D9F
  \l 23DA0
  \l 23DA1
  \l 23DA2
  \l 23DA3
  \l 23DA4
  \l 23DA5
  \l 23DA6
  \l 23DA7
  \l 23DA8
  \l 23DA9
  \l 23DAA
  \l 23DAB
  \l 23DAC
  \l 23DAD
  \l 23DAE
  \l 23DAF
  \l 23DB0
  \l 23DB1
  \l 23DB2
  \l 23DB3
  \l 23DB4
  \l 23DB5
  \l 23DB6
  \l 23DB7
  \l 23DB8
  \l 23DB9
  \l 23DBA
  \l 23DBB
  \l 23DBC
  \l 23DBD
  \l 23DBE
  \l 23DBF
  \l 23DC0
  \l 23DC1
  \l 23DC2
  \l 23DC3
  \l 23DC4
  \l 23DC5
  \l 23DC6
  \l 23DC7
  \l 23DC8
  \l 23DC9
  \l 23DCA
  \l 23DCB
  \l 23DCC
  \l 23DCD
  \l 23DCE
  \l 23DCF
  \l 23DD0
  \l 23DD1
  \l 23DD2
  \l 23DD3
  \l 23DD4
  \l 23DD5
  \l 23DD6
  \l 23DD7
  \l 23DD8
  \l 23DD9
  \l 23DDA
  \l 23DDB
  \l 23DDC
  \l 23DDD
  \l 23DDE
  \l 23DDF
  \l 23DE0
  \l 23DE1
  \l 23DE2
  \l 23DE3
  \l 23DE4
  \l 23DE5
  \l 23DE6
  \l 23DE7
  \l 23DE8
  \l 23DE9
  \l 23DEA
  \l 23DEB
  \l 23DEC
  \l 23DED
  \l 23DEE
  \l 23DEF
  \l 23DF0
  \l 23DF1
  \l 23DF2
  \l 23DF3
  \l 23DF4
  \l 23DF5
  \l 23DF6
  \l 23DF7
  \l 23DF8
  \l 23DF9
  \l 23DFA
  \l 23DFB
  \l 23DFC
  \l 23DFD
  \l 23DFE
  \l 23DFF
  \l 23E00
  \l 23E01
  \l 23E02
  \l 23E03
  \l 23E04
  \l 23E05
  \l 23E06
  \l 23E07
  \l 23E08
  \l 23E09
  \l 23E0A
  \l 23E0B
  \l 23E0C
  \l 23E0D
  \l 23E0E
  \l 23E0F
  \l 23E10
  \l 23E11
  \l 23E12
  \l 23E13
  \l 23E14
  \l 23E15
  \l 23E16
  \l 23E17
  \l 23E18
  \l 23E19
  \l 23E1A
  \l 23E1B
  \l 23E1C
  \l 23E1D
  \l 23E1E
  \l 23E1F
  \l 23E20
  \l 23E21
  \l 23E22
  \l 23E23
  \l 23E24
  \l 23E25
  \l 23E26
  \l 23E27
  \l 23E28
  \l 23E29
  \l 23E2A
  \l 23E2B
  \l 23E2C
  \l 23E2D
  \l 23E2E
  \l 23E2F
  \l 23E30
  \l 23E31
  \l 23E32
  \l 23E33
  \l 23E34
  \l 23E35
  \l 23E36
  \l 23E37
  \l 23E38
  \l 23E39
  \l 23E3A
  \l 23E3B
  \l 23E3C
  \l 23E3D
  \l 23E3E
  \l 23E3F
  \l 23E40
  \l 23E41
  \l 23E42
  \l 23E43
  \l 23E44
  \l 23E45
  \l 23E46
  \l 23E47
  \l 23E48
  \l 23E49
  \l 23E4A
  \l 23E4B
  \l 23E4C
  \l 23E4D
  \l 23E4E
  \l 23E4F
  \l 23E50
  \l 23E51
  \l 23E52
  \l 23E53
  \l 23E54
  \l 23E55
  \l 23E56
  \l 23E57
  \l 23E58
  \l 23E59
  \l 23E5A
  \l 23E5B
  \l 23E5C
  \l 23E5D
  \l 23E5E
  \l 23E5F
  \l 23E60
  \l 23E61
  \l 23E62
  \l 23E63
  \l 23E64
  \l 23E65
  \l 23E66
  \l 23E67
  \l 23E68
  \l 23E69
  \l 23E6A
  \l 23E6B
  \l 23E6C
  \l 23E6D
  \l 23E6E
  \l 23E6F
  \l 23E70
  \l 23E71
  \l 23E72
  \l 23E73
  \l 23E74
  \l 23E75
  \l 23E76
  \l 23E77
  \l 23E78
  \l 23E79
  \l 23E7A
  \l 23E7B
  \l 23E7C
  \l 23E7D
  \l 23E7E
  \l 23E7F
  \l 23E80
  \l 23E81
  \l 23E82
  \l 23E83
  \l 23E84
  \l 23E85
  \l 23E86
  \l 23E87
  \l 23E88
  \l 23E89
  \l 23E8A
  \l 23E8B
  \l 23E8C
  \l 23E8D
  \l 23E8E
  \l 23E8F
  \l 23E90
  \l 23E91
  \l 23E92
  \l 23E93
  \l 23E94
  \l 23E95
  \l 23E96
  \l 23E97
  \l 23E98
  \l 23E99
  \l 23E9A
  \l 23E9B
  \l 23E9C
  \l 23E9D
  \l 23E9E
  \l 23E9F
  \l 23EA0
  \l 23EA1
  \l 23EA2
  \l 23EA3
  \l 23EA4
  \l 23EA5
  \l 23EA6
  \l 23EA7
  \l 23EA8
  \l 23EA9
  \l 23EAA
  \l 23EAB
  \l 23EAC
  \l 23EAD
  \l 23EAE
  \l 23EAF
  \l 23EB0
  \l 23EB1
  \l 23EB2
  \l 23EB3
  \l 23EB4
  \l 23EB5
  \l 23EB6
  \l 23EB7
  \l 23EB8
  \l 23EB9
  \l 23EBA
  \l 23EBB
  \l 23EBC
  \l 23EBD
  \l 23EBE
  \l 23EBF
  \l 23EC0
  \l 23EC1
  \l 23EC2
  \l 23EC3
  \l 23EC4
  \l 23EC5
  \l 23EC6
  \l 23EC7
  \l 23EC8
  \l 23EC9
  \l 23ECA
  \l 23ECB
  \l 23ECC
  \l 23ECD
  \l 23ECE
  \l 23ECF
  \l 23ED0
  \l 23ED1
  \l 23ED2
  \l 23ED3
  \l 23ED4
  \l 23ED5
  \l 23ED6
  \l 23ED7
  \l 23ED8
  \l 23ED9
  \l 23EDA
  \l 23EDB
  \l 23EDC
  \l 23EDD
  \l 23EDE
  \l 23EDF
  \l 23EE0
  \l 23EE1
  \l 23EE2
  \l 23EE3
  \l 23EE4
  \l 23EE5
  \l 23EE6
  \l 23EE7
  \l 23EE8
  \l 23EE9
  \l 23EEA
  \l 23EEB
  \l 23EEC
  \l 23EED
  \l 23EEE
  \l 23EEF
  \l 23EF0
  \l 23EF1
  \l 23EF2
  \l 23EF3
  \l 23EF4
  \l 23EF5
  \l 23EF6
  \l 23EF7
  \l 23EF8
  \l 23EF9
  \l 23EFA
  \l 23EFB
  \l 23EFC
  \l 23EFD
  \l 23EFE
  \l 23EFF
  \l 23F00
  \l 23F01
  \l 23F02
  \l 23F03
  \l 23F04
  \l 23F05
  \l 23F06
  \l 23F07
  \l 23F08
  \l 23F09
  \l 23F0A
  \l 23F0B
  \l 23F0C
  \l 23F0D
  \l 23F0E
  \l 23F0F
  \l 23F10
  \l 23F11
  \l 23F12
  \l 23F13
  \l 23F14
  \l 23F15
  \l 23F16
  \l 23F17
  \l 23F18
  \l 23F19
  \l 23F1A
  \l 23F1B
  \l 23F1C
  \l 23F1D
  \l 23F1E
  \l 23F1F
  \l 23F20
  \l 23F21
  \l 23F22
  \l 23F23
  \l 23F24
  \l 23F25
  \l 23F26
  \l 23F27
  \l 23F28
  \l 23F29
  \l 23F2A
  \l 23F2B
  \l 23F2C
  \l 23F2D
  \l 23F2E
  \l 23F2F
  \l 23F30
  \l 23F31
  \l 23F32
  \l 23F33
  \l 23F34
  \l 23F35
  \l 23F36
  \l 23F37
  \l 23F38
  \l 23F39
  \l 23F3A
  \l 23F3B
  \l 23F3C
  \l 23F3D
  \l 23F3E
  \l 23F3F
  \l 23F40
  \l 23F41
  \l 23F42
  \l 23F43
  \l 23F44
  \l 23F45
  \l 23F46
  \l 23F47
  \l 23F48
  \l 23F49
  \l 23F4A
  \l 23F4B
  \l 23F4C
  \l 23F4D
  \l 23F4E
  \l 23F4F
  \l 23F50
  \l 23F51
  \l 23F52
  \l 23F53
  \l 23F54
  \l 23F55
  \l 23F56
  \l 23F57
  \l 23F58
  \l 23F59
  \l 23F5A
  \l 23F5B
  \l 23F5C
  \l 23F5D
  \l 23F5E
  \l 23F5F
  \l 23F60
  \l 23F61
  \l 23F62
  \l 23F63
  \l 23F64
  \l 23F65
  \l 23F66
  \l 23F67
  \l 23F68
  \l 23F69
  \l 23F6A
  \l 23F6B
  \l 23F6C
  \l 23F6D
  \l 23F6E
  \l 23F6F
  \l 23F70
  \l 23F71
  \l 23F72
  \l 23F73
  \l 23F74
  \l 23F75
  \l 23F76
  \l 23F77
  \l 23F78
  \l 23F79
  \l 23F7A
  \l 23F7B
  \l 23F7C
  \l 23F7D
  \l 23F7E
  \l 23F7F
  \l 23F80
  \l 23F81
  \l 23F82
  \l 23F83
  \l 23F84
  \l 23F85
  \l 23F86
  \l 23F87
  \l 23F88
  \l 23F89
  \l 23F8A
  \l 23F8B
  \l 23F8C
  \l 23F8D
  \l 23F8E
  \l 23F8F
  \l 23F90
  \l 23F91
  \l 23F92
  \l 23F93
  \l 23F94
  \l 23F95
  \l 23F96
  \l 23F97
  \l 23F98
  \l 23F99
  \l 23F9A
  \l 23F9B
  \l 23F9C
  \l 23F9D
  \l 23F9E
  \l 23F9F
  \l 23FA0
  \l 23FA1
  \l 23FA2
  \l 23FA3
  \l 23FA4
  \l 23FA5
  \l 23FA6
  \l 23FA7
  \l 23FA8
  \l 23FA9
  \l 23FAA
  \l 23FAB
  \l 23FAC
  \l 23FAD
  \l 23FAE
  \l 23FAF
  \l 23FB0
  \l 23FB1
  \l 23FB2
  \l 23FB3
  \l 23FB4
  \l 23FB5
  \l 23FB6
  \l 23FB7
  \l 23FB8
  \l 23FB9
  \l 23FBA
  \l 23FBB
  \l 23FBC
  \l 23FBD
  \l 23FBE
  \l 23FBF
  \l 23FC0
  \l 23FC1
  \l 23FC2
  \l 23FC3
  \l 23FC4
  \l 23FC5
  \l 23FC6
  \l 23FC7
  \l 23FC8
  \l 23FC9
  \l 23FCA
  \l 23FCB
  \l 23FCC
  \l 23FCD
  \l 23FCE
  \l 23FCF
  \l 23FD0
  \l 23FD1
  \l 23FD2
  \l 23FD3
  \l 23FD4
  \l 23FD5
  \l 23FD6
  \l 23FD7
  \l 23FD8
  \l 23FD9
  \l 23FDA
  \l 23FDB
  \l 23FDC
  \l 23FDD
  \l 23FDE
  \l 23FDF
  \l 23FE0
  \l 23FE1
  \l 23FE2
  \l 23FE3
  \l 23FE4
  \l 23FE5
  \l 23FE6
  \l 23FE7
  \l 23FE8
  \l 23FE9
  \l 23FEA
  \l 23FEB
  \l 23FEC
  \l 23FED
  \l 23FEE
  \l 23FEF
  \l 23FF0
  \l 23FF1
  \l 23FF2
  \l 23FF3
  \l 23FF4
  \l 23FF5
  \l 23FF6
  \l 23FF7
  \l 23FF8
  \l 23FF9
  \l 23FFA
  \l 23FFB
  \l 23FFC
  \l 23FFD
  \l 23FFE
  \l 23FFF
  \l 24000
  \l 24001
  \l 24002
  \l 24003
  \l 24004
  \l 24005
  \l 24006
  \l 24007
  \l 24008
  \l 24009
  \l 2400A
  \l 2400B
  \l 2400C
  \l 2400D
  \l 2400E
  \l 2400F
  \l 24010
  \l 24011
  \l 24012
  \l 24013
  \l 24014
  \l 24015
  \l 24016
  \l 24017
  \l 24018
  \l 24019
  \l 2401A
  \l 2401B
  \l 2401C
  \l 2401D
  \l 2401E
  \l 2401F
  \l 24020
  \l 24021
  \l 24022
  \l 24023
  \l 24024
  \l 24025
  \l 24026
  \l 24027
  \l 24028
  \l 24029
  \l 2402A
  \l 2402B
  \l 2402C
  \l 2402D
  \l 2402E
  \l 2402F
  \l 24030
  \l 24031
  \l 24032
  \l 24033
  \l 24034
  \l 24035
  \l 24036
  \l 24037
  \l 24038
  \l 24039
  \l 2403A
  \l 2403B
  \l 2403C
  \l 2403D
  \l 2403E
  \l 2403F
  \l 24040
  \l 24041
  \l 24042
  \l 24043
  \l 24044
  \l 24045
  \l 24046
  \l 24047
  \l 24048
  \l 24049
  \l 2404A
  \l 2404B
  \l 2404C
  \l 2404D
  \l 2404E
  \l 2404F
  \l 24050
  \l 24051
  \l 24052
  \l 24053
  \l 24054
  \l 24055
  \l 24056
  \l 24057
  \l 24058
  \l 24059
  \l 2405A
  \l 2405B
  \l 2405C
  \l 2405D
  \l 2405E
  \l 2405F
  \l 24060
  \l 24061
  \l 24062
  \l 24063
  \l 24064
  \l 24065
  \l 24066
  \l 24067
  \l 24068
  \l 24069
  \l 2406A
  \l 2406B
  \l 2406C
  \l 2406D
  \l 2406E
  \l 2406F
  \l 24070
  \l 24071
  \l 24072
  \l 24073
  \l 24074
  \l 24075
  \l 24076
  \l 24077
  \l 24078
  \l 24079
  \l 2407A
  \l 2407B
  \l 2407C
  \l 2407D
  \l 2407E
  \l 2407F
  \l 24080
  \l 24081
  \l 24082
  \l 24083
  \l 24084
  \l 24085
  \l 24086
  \l 24087
  \l 24088
  \l 24089
  \l 2408A
  \l 2408B
  \l 2408C
  \l 2408D
  \l 2408E
  \l 2408F
  \l 24090
  \l 24091
  \l 24092
  \l 24093
  \l 24094
  \l 24095
  \l 24096
  \l 24097
  \l 24098
  \l 24099
  \l 2409A
  \l 2409B
  \l 2409C
  \l 2409D
  \l 2409E
  \l 2409F
  \l 240A0
  \l 240A1
  \l 240A2
  \l 240A3
  \l 240A4
  \l 240A5
  \l 240A6
  \l 240A7
  \l 240A8
  \l 240A9
  \l 240AA
  \l 240AB
  \l 240AC
  \l 240AD
  \l 240AE
  \l 240AF
  \l 240B0
  \l 240B1
  \l 240B2
  \l 240B3
  \l 240B4
  \l 240B5
  \l 240B6
  \l 240B7
  \l 240B8
  \l 240B9
  \l 240BA
  \l 240BB
  \l 240BC
  \l 240BD
  \l 240BE
  \l 240BF
  \l 240C0
  \l 240C1
  \l 240C2
  \l 240C3
  \l 240C4
  \l 240C5
  \l 240C6
  \l 240C7
  \l 240C8
  \l 240C9
  \l 240CA
  \l 240CB
  \l 240CC
  \l 240CD
  \l 240CE
  \l 240CF
  \l 240D0
  \l 240D1
  \l 240D2
  \l 240D3
  \l 240D4
  \l 240D5
  \l 240D6
  \l 240D7
  \l 240D8
  \l 240D9
  \l 240DA
  \l 240DB
  \l 240DC
  \l 240DD
  \l 240DE
  \l 240DF
  \l 240E0
  \l 240E1
  \l 240E2
  \l 240E3
  \l 240E4
  \l 240E5
  \l 240E6
  \l 240E7
  \l 240E8
  \l 240E9
  \l 240EA
  \l 240EB
  \l 240EC
  \l 240ED
  \l 240EE
  \l 240EF
  \l 240F0
  \l 240F1
  \l 240F2
  \l 240F3
  \l 240F4
  \l 240F5
  \l 240F6
  \l 240F7
  \l 240F8
  \l 240F9
  \l 240FA
  \l 240FB
  \l 240FC
  \l 240FD
  \l 240FE
  \l 240FF
  \l 24100
  \l 24101
  \l 24102
  \l 24103
  \l 24104
  \l 24105
  \l 24106
  \l 24107
  \l 24108
  \l 24109
  \l 2410A
  \l 2410B
  \l 2410C
  \l 2410D
  \l 2410E
  \l 2410F
  \l 24110
  \l 24111
  \l 24112
  \l 24113
  \l 24114
  \l 24115
  \l 24116
  \l 24117
  \l 24118
  \l 24119
  \l 2411A
  \l 2411B
  \l 2411C
  \l 2411D
  \l 2411E
  \l 2411F
  \l 24120
  \l 24121
  \l 24122
  \l 24123
  \l 24124
  \l 24125
  \l 24126
  \l 24127
  \l 24128
  \l 24129
  \l 2412A
  \l 2412B
  \l 2412C
  \l 2412D
  \l 2412E
  \l 2412F
  \l 24130
  \l 24131
  \l 24132
  \l 24133
  \l 24134
  \l 24135
  \l 24136
  \l 24137
  \l 24138
  \l 24139
  \l 2413A
  \l 2413B
  \l 2413C
  \l 2413D
  \l 2413E
  \l 2413F
  \l 24140
  \l 24141
  \l 24142
  \l 24143
  \l 24144
  \l 24145
  \l 24146
  \l 24147
  \l 24148
  \l 24149
  \l 2414A
  \l 2414B
  \l 2414C
  \l 2414D
  \l 2414E
  \l 2414F
  \l 24150
  \l 24151
  \l 24152
  \l 24153
  \l 24154
  \l 24155
  \l 24156
  \l 24157
  \l 24158
  \l 24159
  \l 2415A
  \l 2415B
  \l 2415C
  \l 2415D
  \l 2415E
  \l 2415F
  \l 24160
  \l 24161
  \l 24162
  \l 24163
  \l 24164
  \l 24165
  \l 24166
  \l 24167
  \l 24168
  \l 24169
  \l 2416A
  \l 2416B
  \l 2416C
  \l 2416D
  \l 2416E
  \l 2416F
  \l 24170
  \l 24171
  \l 24172
  \l 24173
  \l 24174
  \l 24175
  \l 24176
  \l 24177
  \l 24178
  \l 24179
  \l 2417A
  \l 2417B
  \l 2417C
  \l 2417D
  \l 2417E
  \l 2417F
  \l 24180
  \l 24181
  \l 24182
  \l 24183
  \l 24184
  \l 24185
  \l 24186
  \l 24187
  \l 24188
  \l 24189
  \l 2418A
  \l 2418B
  \l 2418C
  \l 2418D
  \l 2418E
  \l 2418F
  \l 24190
  \l 24191
  \l 24192
  \l 24193
  \l 24194
  \l 24195
  \l 24196
  \l 24197
  \l 24198
  \l 24199
  \l 2419A
  \l 2419B
  \l 2419C
  \l 2419D
  \l 2419E
  \l 2419F
  \l 241A0
  \l 241A1
  \l 241A2
  \l 241A3
  \l 241A4
  \l 241A5
  \l 241A6
  \l 241A7
  \l 241A8
  \l 241A9
  \l 241AA
  \l 241AB
  \l 241AC
  \l 241AD
  \l 241AE
  \l 241AF
  \l 241B0
  \l 241B1
  \l 241B2
  \l 241B3
  \l 241B4
  \l 241B5
  \l 241B6
  \l 241B7
  \l 241B8
  \l 241B9
  \l 241BA
  \l 241BB
  \l 241BC
  \l 241BD
  \l 241BE
  \l 241BF
  \l 241C0
  \l 241C1
  \l 241C2
  \l 241C3
  \l 241C4
  \l 241C5
  \l 241C6
  \l 241C7
  \l 241C8
  \l 241C9
  \l 241CA
  \l 241CB
  \l 241CC
  \l 241CD
  \l 241CE
  \l 241CF
  \l 241D0
  \l 241D1
  \l 241D2
  \l 241D3
  \l 241D4
  \l 241D5
  \l 241D6
  \l 241D7
  \l 241D8
  \l 241D9
  \l 241DA
  \l 241DB
  \l 241DC
  \l 241DD
  \l 241DE
  \l 241DF
  \l 241E0
  \l 241E1
  \l 241E2
  \l 241E3
  \l 241E4
  \l 241E5
  \l 241E6
  \l 241E7
  \l 241E8
  \l 241E9
  \l 241EA
  \l 241EB
  \l 241EC
  \l 241ED
  \l 241EE
  \l 241EF
  \l 241F0
  \l 241F1
  \l 241F2
  \l 241F3
  \l 241F4
  \l 241F5
  \l 241F6
  \l 241F7
  \l 241F8
  \l 241F9
  \l 241FA
  \l 241FB
  \l 241FC
  \l 241FD
  \l 241FE
  \l 241FF
  \l 24200
  \l 24201
  \l 24202
  \l 24203
  \l 24204
  \l 24205
  \l 24206
  \l 24207
  \l 24208
  \l 24209
  \l 2420A
  \l 2420B
  \l 2420C
  \l 2420D
  \l 2420E
  \l 2420F
  \l 24210
  \l 24211
  \l 24212
  \l 24213
  \l 24214
  \l 24215
  \l 24216
  \l 24217
  \l 24218
  \l 24219
  \l 2421A
  \l 2421B
  \l 2421C
  \l 2421D
  \l 2421E
  \l 2421F
  \l 24220
  \l 24221
  \l 24222
  \l 24223
  \l 24224
  \l 24225
  \l 24226
  \l 24227
  \l 24228
  \l 24229
  \l 2422A
  \l 2422B
  \l 2422C
  \l 2422D
  \l 2422E
  \l 2422F
  \l 24230
  \l 24231
  \l 24232
  \l 24233
  \l 24234
  \l 24235
  \l 24236
  \l 24237
  \l 24238
  \l 24239
  \l 2423A
  \l 2423B
  \l 2423C
  \l 2423D
  \l 2423E
  \l 2423F
  \l 24240
  \l 24241
  \l 24242
  \l 24243
  \l 24244
  \l 24245
  \l 24246
  \l 24247
  \l 24248
  \l 24249
  \l 2424A
  \l 2424B
  \l 2424C
  \l 2424D
  \l 2424E
  \l 2424F
  \l 24250
  \l 24251
  \l 24252
  \l 24253
  \l 24254
  \l 24255
  \l 24256
  \l 24257
  \l 24258
  \l 24259
  \l 2425A
  \l 2425B
  \l 2425C
  \l 2425D
  \l 2425E
  \l 2425F
  \l 24260
  \l 24261
  \l 24262
  \l 24263
  \l 24264
  \l 24265
  \l 24266
  \l 24267
  \l 24268
  \l 24269
  \l 2426A
  \l 2426B
  \l 2426C
  \l 2426D
  \l 2426E
  \l 2426F
  \l 24270
  \l 24271
  \l 24272
  \l 24273
  \l 24274
  \l 24275
  \l 24276
  \l 24277
  \l 24278
  \l 24279
  \l 2427A
  \l 2427B
  \l 2427C
  \l 2427D
  \l 2427E
  \l 2427F
  \l 24280
  \l 24281
  \l 24282
  \l 24283
  \l 24284
  \l 24285
  \l 24286
  \l 24287
  \l 24288
  \l 24289
  \l 2428A
  \l 2428B
  \l 2428C
  \l 2428D
  \l 2428E
  \l 2428F
  \l 24290
  \l 24291
  \l 24292
  \l 24293
  \l 24294
  \l 24295
  \l 24296
  \l 24297
  \l 24298
  \l 24299
  \l 2429A
  \l 2429B
  \l 2429C
  \l 2429D
  \l 2429E
  \l 2429F
  \l 242A0
  \l 242A1
  \l 242A2
  \l 242A3
  \l 242A4
  \l 242A5
  \l 242A6
  \l 242A7
  \l 242A8
  \l 242A9
  \l 242AA
  \l 242AB
  \l 242AC
  \l 242AD
  \l 242AE
  \l 242AF
  \l 242B0
  \l 242B1
  \l 242B2
  \l 242B3
  \l 242B4
  \l 242B5
  \l 242B6
  \l 242B7
  \l 242B8
  \l 242B9
  \l 242BA
  \l 242BB
  \l 242BC
  \l 242BD
  \l 242BE
  \l 242BF
  \l 242C0
  \l 242C1
  \l 242C2
  \l 242C3
  \l 242C4
  \l 242C5
  \l 242C6
  \l 242C7
  \l 242C8
  \l 242C9
  \l 242CA
  \l 242CB
  \l 242CC
  \l 242CD
  \l 242CE
  \l 242CF
  \l 242D0
  \l 242D1
  \l 242D2
  \l 242D3
  \l 242D4
  \l 242D5
  \l 242D6
  \l 242D7
  \l 242D8
  \l 242D9
  \l 242DA
  \l 242DB
  \l 242DC
  \l 242DD
  \l 242DE
  \l 242DF
  \l 242E0
  \l 242E1
  \l 242E2
  \l 242E3
  \l 242E4
  \l 242E5
  \l 242E6
  \l 242E7
  \l 242E8
  \l 242E9
  \l 242EA
  \l 242EB
  \l 242EC
  \l 242ED
  \l 242EE
  \l 242EF
  \l 242F0
  \l 242F1
  \l 242F2
  \l 242F3
  \l 242F4
  \l 242F5
  \l 242F6
  \l 242F7
  \l 242F8
  \l 242F9
  \l 242FA
  \l 242FB
  \l 242FC
  \l 242FD
  \l 242FE
  \l 242FF
  \l 24300
  \l 24301
  \l 24302
  \l 24303
  \l 24304
  \l 24305
  \l 24306
  \l 24307
  \l 24308
  \l 24309
  \l 2430A
  \l 2430B
  \l 2430C
  \l 2430D
  \l 2430E
  \l 2430F
  \l 24310
  \l 24311
  \l 24312
  \l 24313
  \l 24314
  \l 24315
  \l 24316
  \l 24317
  \l 24318
  \l 24319
  \l 2431A
  \l 2431B
  \l 2431C
  \l 2431D
  \l 2431E
  \l 2431F
  \l 24320
  \l 24321
  \l 24322
  \l 24323
  \l 24324
  \l 24325
  \l 24326
  \l 24327
  \l 24328
  \l 24329
  \l 2432A
  \l 2432B
  \l 2432C
  \l 2432D
  \l 2432E
  \l 2432F
  \l 24330
  \l 24331
  \l 24332
  \l 24333
  \l 24334
  \l 24335
  \l 24336
  \l 24337
  \l 24338
  \l 24339
  \l 2433A
  \l 2433B
  \l 2433C
  \l 2433D
  \l 2433E
  \l 2433F
  \l 24340
  \l 24341
  \l 24342
  \l 24343
  \l 24344
  \l 24345
  \l 24346
  \l 24347
  \l 24348
  \l 24349
  \l 2434A
  \l 2434B
  \l 2434C
  \l 2434D
  \l 2434E
  \l 2434F
  \l 24350
  \l 24351
  \l 24352
  \l 24353
  \l 24354
  \l 24355
  \l 24356
  \l 24357
  \l 24358
  \l 24359
  \l 2435A
  \l 2435B
  \l 2435C
  \l 2435D
  \l 2435E
  \l 2435F
  \l 24360
  \l 24361
  \l 24362
  \l 24363
  \l 24364
  \l 24365
  \l 24366
  \l 24367
  \l 24368
  \l 24369
  \l 2436A
  \l 2436B
  \l 2436C
  \l 2436D
  \l 2436E
  \l 2436F
  \l 24370
  \l 24371
  \l 24372
  \l 24373
  \l 24374
  \l 24375
  \l 24376
  \l 24377
  \l 24378
  \l 24379
  \l 2437A
  \l 2437B
  \l 2437C
  \l 2437D
  \l 2437E
  \l 2437F
  \l 24380
  \l 24381
  \l 24382
  \l 24383
  \l 24384
  \l 24385
  \l 24386
  \l 24387
  \l 24388
  \l 24389
  \l 2438A
  \l 2438B
  \l 2438C
  \l 2438D
  \l 2438E
  \l 2438F
  \l 24390
  \l 24391
  \l 24392
  \l 24393
  \l 24394
  \l 24395
  \l 24396
  \l 24397
  \l 24398
  \l 24399
  \l 2439A
  \l 2439B
  \l 2439C
  \l 2439D
  \l 2439E
  \l 2439F
  \l 243A0
  \l 243A1
  \l 243A2
  \l 243A3
  \l 243A4
  \l 243A5
  \l 243A6
  \l 243A7
  \l 243A8
  \l 243A9
  \l 243AA
  \l 243AB
  \l 243AC
  \l 243AD
  \l 243AE
  \l 243AF
  \l 243B0
  \l 243B1
  \l 243B2
  \l 243B3
  \l 243B4
  \l 243B5
  \l 243B6
  \l 243B7
  \l 243B8
  \l 243B9
  \l 243BA
  \l 243BB
  \l 243BC
  \l 243BD
  \l 243BE
  \l 243BF
  \l 243C0
  \l 243C1
  \l 243C2
  \l 243C3
  \l 243C4
  \l 243C5
  \l 243C6
  \l 243C7
  \l 243C8
  \l 243C9
  \l 243CA
  \l 243CB
  \l 243CC
  \l 243CD
  \l 243CE
  \l 243CF
  \l 243D0
  \l 243D1
  \l 243D2
  \l 243D3
  \l 243D4
  \l 243D5
  \l 243D6
  \l 243D7
  \l 243D8
  \l 243D9
  \l 243DA
  \l 243DB
  \l 243DC
  \l 243DD
  \l 243DE
  \l 243DF
  \l 243E0
  \l 243E1
  \l 243E2
  \l 243E3
  \l 243E4
  \l 243E5
  \l 243E6
  \l 243E7
  \l 243E8
  \l 243E9
  \l 243EA
  \l 243EB
  \l 243EC
  \l 243ED
  \l 243EE
  \l 243EF
  \l 243F0
  \l 243F1
  \l 243F2
  \l 243F3
  \l 243F4
  \l 243F5
  \l 243F6
  \l 243F7
  \l 243F8
  \l 243F9
  \l 243FA
  \l 243FB
  \l 243FC
  \l 243FD
  \l 243FE
  \l 243FF
  \l 24400
  \l 24401
  \l 24402
  \l 24403
  \l 24404
  \l 24405
  \l 24406
  \l 24407
  \l 24408
  \l 24409
  \l 2440A
  \l 2440B
  \l 2440C
  \l 2440D
  \l 2440E
  \l 2440F
  \l 24410
  \l 24411
  \l 24412
  \l 24413
  \l 24414
  \l 24415
  \l 24416
  \l 24417
  \l 24418
  \l 24419
  \l 2441A
  \l 2441B
  \l 2441C
  \l 2441D
  \l 2441E
  \l 2441F
  \l 24420
  \l 24421
  \l 24422
  \l 24423
  \l 24424
  \l 24425
  \l 24426
  \l 24427
  \l 24428
  \l 24429
  \l 2442A
  \l 2442B
  \l 2442C
  \l 2442D
  \l 2442E
  \l 2442F
  \l 24430
  \l 24431
  \l 24432
  \l 24433
  \l 24434
  \l 24435
  \l 24436
  \l 24437
  \l 24438
  \l 24439
  \l 2443A
  \l 2443B
  \l 2443C
  \l 2443D
  \l 2443E
  \l 2443F
  \l 24440
  \l 24441
  \l 24442
  \l 24443
  \l 24444
  \l 24445
  \l 24446
  \l 24447
  \l 24448
  \l 24449
  \l 2444A
  \l 2444B
  \l 2444C
  \l 2444D
  \l 2444E
  \l 2444F
  \l 24450
  \l 24451
  \l 24452
  \l 24453
  \l 24454
  \l 24455
  \l 24456
  \l 24457
  \l 24458
  \l 24459
  \l 2445A
  \l 2445B
  \l 2445C
  \l 2445D
  \l 2445E
  \l 2445F
  \l 24460
  \l 24461
  \l 24462
  \l 24463
  \l 24464
  \l 24465
  \l 24466
  \l 24467
  \l 24468
  \l 24469
  \l 2446A
  \l 2446B
  \l 2446C
  \l 2446D
  \l 2446E
  \l 2446F
  \l 24470
  \l 24471
  \l 24472
  \l 24473
  \l 24474
  \l 24475
  \l 24476
  \l 24477
  \l 24478
  \l 24479
  \l 2447A
  \l 2447B
  \l 2447C
  \l 2447D
  \l 2447E
  \l 2447F
  \l 24480
  \l 24481
  \l 24482
  \l 24483
  \l 24484
  \l 24485
  \l 24486
  \l 24487
  \l 24488
  \l 24489
  \l 2448A
  \l 2448B
  \l 2448C
  \l 2448D
  \l 2448E
  \l 2448F
  \l 24490
  \l 24491
  \l 24492
  \l 24493
  \l 24494
  \l 24495
  \l 24496
  \l 24497
  \l 24498
  \l 24499
  \l 2449A
  \l 2449B
  \l 2449C
  \l 2449D
  \l 2449E
  \l 2449F
  \l 244A0
  \l 244A1
  \l 244A2
  \l 244A3
  \l 244A4
  \l 244A5
  \l 244A6
  \l 244A7
  \l 244A8
  \l 244A9
  \l 244AA
  \l 244AB
  \l 244AC
  \l 244AD
  \l 244AE
  \l 244AF
  \l 244B0
  \l 244B1
  \l 244B2
  \l 244B3
  \l 244B4
  \l 244B5
  \l 244B6
  \l 244B7
  \l 244B8
  \l 244B9
  \l 244BA
  \l 244BB
  \l 244BC
  \l 244BD
  \l 244BE
  \l 244BF
  \l 244C0
  \l 244C1
  \l 244C2
  \l 244C3
  \l 244C4
  \l 244C5
  \l 244C6
  \l 244C7
  \l 244C8
  \l 244C9
  \l 244CA
  \l 244CB
  \l 244CC
  \l 244CD
  \l 244CE
  \l 244CF
  \l 244D0
  \l 244D1
  \l 244D2
  \l 244D3
  \l 244D4
  \l 244D5
  \l 244D6
  \l 244D7
  \l 244D8
  \l 244D9
  \l 244DA
  \l 244DB
  \l 244DC
  \l 244DD
  \l 244DE
  \l 244DF
  \l 244E0
  \l 244E1
  \l 244E2
  \l 244E3
  \l 244E4
  \l 244E5
  \l 244E6
  \l 244E7
  \l 244E8
  \l 244E9
  \l 244EA
  \l 244EB
  \l 244EC
  \l 244ED
  \l 244EE
  \l 244EF
  \l 244F0
  \l 244F1
  \l 244F2
  \l 244F3
  \l 244F4
  \l 244F5
  \l 244F6
  \l 244F7
  \l 244F8
  \l 244F9
  \l 244FA
  \l 244FB
  \l 244FC
  \l 244FD
  \l 244FE
  \l 244FF
  \l 24500
  \l 24501
  \l 24502
  \l 24503
  \l 24504
  \l 24505
  \l 24506
  \l 24507
  \l 24508
  \l 24509
  \l 2450A
  \l 2450B
  \l 2450C
  \l 2450D
  \l 2450E
  \l 2450F
  \l 24510
  \l 24511
  \l 24512
  \l 24513
  \l 24514
  \l 24515
  \l 24516
  \l 24517
  \l 24518
  \l 24519
  \l 2451A
  \l 2451B
  \l 2451C
  \l 2451D
  \l 2451E
  \l 2451F
  \l 24520
  \l 24521
  \l 24522
  \l 24523
  \l 24524
  \l 24525
  \l 24526
  \l 24527
  \l 24528
  \l 24529
  \l 2452A
  \l 2452B
  \l 2452C
  \l 2452D
  \l 2452E
  \l 2452F
  \l 24530
  \l 24531
  \l 24532
  \l 24533
  \l 24534
  \l 24535
  \l 24536
  \l 24537
  \l 24538
  \l 24539
  \l 2453A
  \l 2453B
  \l 2453C
  \l 2453D
  \l 2453E
  \l 2453F
  \l 24540
  \l 24541
  \l 24542
  \l 24543
  \l 24544
  \l 24545
  \l 24546
  \l 24547
  \l 24548
  \l 24549
  \l 2454A
  \l 2454B
  \l 2454C
  \l 2454D
  \l 2454E
  \l 2454F
  \l 24550
  \l 24551
  \l 24552
  \l 24553
  \l 24554
  \l 24555
  \l 24556
  \l 24557
  \l 24558
  \l 24559
  \l 2455A
  \l 2455B
  \l 2455C
  \l 2455D
  \l 2455E
  \l 2455F
  \l 24560
  \l 24561
  \l 24562
  \l 24563
  \l 24564
  \l 24565
  \l 24566
  \l 24567
  \l 24568
  \l 24569
  \l 2456A
  \l 2456B
  \l 2456C
  \l 2456D
  \l 2456E
  \l 2456F
  \l 24570
  \l 24571
  \l 24572
  \l 24573
  \l 24574
  \l 24575
  \l 24576
  \l 24577
  \l 24578
  \l 24579
  \l 2457A
  \l 2457B
  \l 2457C
  \l 2457D
  \l 2457E
  \l 2457F
  \l 24580
  \l 24581
  \l 24582
  \l 24583
  \l 24584
  \l 24585
  \l 24586
  \l 24587
  \l 24588
  \l 24589
  \l 2458A
  \l 2458B
  \l 2458C
  \l 2458D
  \l 2458E
  \l 2458F
  \l 24590
  \l 24591
  \l 24592
  \l 24593
  \l 24594
  \l 24595
  \l 24596
  \l 24597
  \l 24598
  \l 24599
  \l 2459A
  \l 2459B
  \l 2459C
  \l 2459D
  \l 2459E
  \l 2459F
  \l 245A0
  \l 245A1
  \l 245A2
  \l 245A3
  \l 245A4
  \l 245A5
  \l 245A6
  \l 245A7
  \l 245A8
  \l 245A9
  \l 245AA
  \l 245AB
  \l 245AC
  \l 245AD
  \l 245AE
  \l 245AF
  \l 245B0
  \l 245B1
  \l 245B2
  \l 245B3
  \l 245B4
  \l 245B5
  \l 245B6
  \l 245B7
  \l 245B8
  \l 245B9
  \l 245BA
  \l 245BB
  \l 245BC
  \l 245BD
  \l 245BE
  \l 245BF
  \l 245C0
  \l 245C1
  \l 245C2
  \l 245C3
  \l 245C4
  \l 245C5
  \l 245C6
  \l 245C7
  \l 245C8
  \l 245C9
  \l 245CA
  \l 245CB
  \l 245CC
  \l 245CD
  \l 245CE
  \l 245CF
  \l 245D0
  \l 245D1
  \l 245D2
  \l 245D3
  \l 245D4
  \l 245D5
  \l 245D6
  \l 245D7
  \l 245D8
  \l 245D9
  \l 245DA
  \l 245DB
  \l 245DC
  \l 245DD
  \l 245DE
  \l 245DF
  \l 245E0
  \l 245E1
  \l 245E2
  \l 245E3
  \l 245E4
  \l 245E5
  \l 245E6
  \l 245E7
  \l 245E8
  \l 245E9
  \l 245EA
  \l 245EB
  \l 245EC
  \l 245ED
  \l 245EE
  \l 245EF
  \l 245F0
  \l 245F1
  \l 245F2
  \l 245F3
  \l 245F4
  \l 245F5
  \l 245F6
  \l 245F7
  \l 245F8
  \l 245F9
  \l 245FA
  \l 245FB
  \l 245FC
  \l 245FD
  \l 245FE
  \l 245FF
  \l 24600
  \l 24601
  \l 24602
  \l 24603
  \l 24604
  \l 24605
  \l 24606
  \l 24607
  \l 24608
  \l 24609
  \l 2460A
  \l 2460B
  \l 2460C
  \l 2460D
  \l 2460E
  \l 2460F
  \l 24610
  \l 24611
  \l 24612
  \l 24613
  \l 24614
  \l 24615
  \l 24616
  \l 24617
  \l 24618
  \l 24619
  \l 2461A
  \l 2461B
  \l 2461C
  \l 2461D
  \l 2461E
  \l 2461F
  \l 24620
  \l 24621
  \l 24622
  \l 24623
  \l 24624
  \l 24625
  \l 24626
  \l 24627
  \l 24628
  \l 24629
  \l 2462A
  \l 2462B
  \l 2462C
  \l 2462D
  \l 2462E
  \l 2462F
  \l 24630
  \l 24631
  \l 24632
  \l 24633
  \l 24634
  \l 24635
  \l 24636
  \l 24637
  \l 24638
  \l 24639
  \l 2463A
  \l 2463B
  \l 2463C
  \l 2463D
  \l 2463E
  \l 2463F
  \l 24640
  \l 24641
  \l 24642
  \l 24643
  \l 24644
  \l 24645
  \l 24646
  \l 24647
  \l 24648
  \l 24649
  \l 2464A
  \l 2464B
  \l 2464C
  \l 2464D
  \l 2464E
  \l 2464F
  \l 24650
  \l 24651
  \l 24652
  \l 24653
  \l 24654
  \l 24655
  \l 24656
  \l 24657
  \l 24658
  \l 24659
  \l 2465A
  \l 2465B
  \l 2465C
  \l 2465D
  \l 2465E
  \l 2465F
  \l 24660
  \l 24661
  \l 24662
  \l 24663
  \l 24664
  \l 24665
  \l 24666
  \l 24667
  \l 24668
  \l 24669
  \l 2466A
  \l 2466B
  \l 2466C
  \l 2466D
  \l 2466E
  \l 2466F
  \l 24670
  \l 24671
  \l 24672
  \l 24673
  \l 24674
  \l 24675
  \l 24676
  \l 24677
  \l 24678
  \l 24679
  \l 2467A
  \l 2467B
  \l 2467C
  \l 2467D
  \l 2467E
  \l 2467F
  \l 24680
  \l 24681
  \l 24682
  \l 24683
  \l 24684
  \l 24685
  \l 24686
  \l 24687
  \l 24688
  \l 24689
  \l 2468A
  \l 2468B
  \l 2468C
  \l 2468D
  \l 2468E
  \l 2468F
  \l 24690
  \l 24691
  \l 24692
  \l 24693
  \l 24694
  \l 24695
  \l 24696
  \l 24697
  \l 24698
  \l 24699
  \l 2469A
  \l 2469B
  \l 2469C
  \l 2469D
  \l 2469E
  \l 2469F
  \l 246A0
  \l 246A1
  \l 246A2
  \l 246A3
  \l 246A4
  \l 246A5
  \l 246A6
  \l 246A7
  \l 246A8
  \l 246A9
  \l 246AA
  \l 246AB
  \l 246AC
  \l 246AD
  \l 246AE
  \l 246AF
  \l 246B0
  \l 246B1
  \l 246B2
  \l 246B3
  \l 246B4
  \l 246B5
  \l 246B6
  \l 246B7
  \l 246B8
  \l 246B9
  \l 246BA
  \l 246BB
  \l 246BC
  \l 246BD
  \l 246BE
  \l 246BF
  \l 246C0
  \l 246C1
  \l 246C2
  \l 246C3
  \l 246C4
  \l 246C5
  \l 246C6
  \l 246C7
  \l 246C8
  \l 246C9
  \l 246CA
  \l 246CB
  \l 246CC
  \l 246CD
  \l 246CE
  \l 246CF
  \l 246D0
  \l 246D1
  \l 246D2
  \l 246D3
  \l 246D4
  \l 246D5
  \l 246D6
  \l 246D7
  \l 246D8
  \l 246D9
  \l 246DA
  \l 246DB
  \l 246DC
  \l 246DD
  \l 246DE
  \l 246DF
  \l 246E0
  \l 246E1
  \l 246E2
  \l 246E3
  \l 246E4
  \l 246E5
  \l 246E6
  \l 246E7
  \l 246E8
  \l 246E9
  \l 246EA
  \l 246EB
  \l 246EC
  \l 246ED
  \l 246EE
  \l 246EF
  \l 246F0
  \l 246F1
  \l 246F2
  \l 246F3
  \l 246F4
  \l 246F5
  \l 246F6
  \l 246F7
  \l 246F8
  \l 246F9
  \l 246FA
  \l 246FB
  \l 246FC
  \l 246FD
  \l 246FE
  \l 246FF
  \l 24700
  \l 24701
  \l 24702
  \l 24703
  \l 24704
  \l 24705
  \l 24706
  \l 24707
  \l 24708
  \l 24709
  \l 2470A
  \l 2470B
  \l 2470C
  \l 2470D
  \l 2470E
  \l 2470F
  \l 24710
  \l 24711
  \l 24712
  \l 24713
  \l 24714
  \l 24715
  \l 24716
  \l 24717
  \l 24718
  \l 24719
  \l 2471A
  \l 2471B
  \l 2471C
  \l 2471D
  \l 2471E
  \l 2471F
  \l 24720
  \l 24721
  \l 24722
  \l 24723
  \l 24724
  \l 24725
  \l 24726
  \l 24727
  \l 24728
  \l 24729
  \l 2472A
  \l 2472B
  \l 2472C
  \l 2472D
  \l 2472E
  \l 2472F
  \l 24730
  \l 24731
  \l 24732
  \l 24733
  \l 24734
  \l 24735
  \l 24736
  \l 24737
  \l 24738
  \l 24739
  \l 2473A
  \l 2473B
  \l 2473C
  \l 2473D
  \l 2473E
  \l 2473F
  \l 24740
  \l 24741
  \l 24742
  \l 24743
  \l 24744
  \l 24745
  \l 24746
  \l 24747
  \l 24748
  \l 24749
  \l 2474A
  \l 2474B
  \l 2474C
  \l 2474D
  \l 2474E
  \l 2474F
  \l 24750
  \l 24751
  \l 24752
  \l 24753
  \l 24754
  \l 24755
  \l 24756
  \l 24757
  \l 24758
  \l 24759
  \l 2475A
  \l 2475B
  \l 2475C
  \l 2475D
  \l 2475E
  \l 2475F
  \l 24760
  \l 24761
  \l 24762
  \l 24763
  \l 24764
  \l 24765
  \l 24766
  \l 24767
  \l 24768
  \l 24769
  \l 2476A
  \l 2476B
  \l 2476C
  \l 2476D
  \l 2476E
  \l 2476F
  \l 24770
  \l 24771
  \l 24772
  \l 24773
  \l 24774
  \l 24775
  \l 24776
  \l 24777
  \l 24778
  \l 24779
  \l 2477A
  \l 2477B
  \l 2477C
  \l 2477D
  \l 2477E
  \l 2477F
  \l 24780
  \l 24781
  \l 24782
  \l 24783
  \l 24784
  \l 24785
  \l 24786
  \l 24787
  \l 24788
  \l 24789
  \l 2478A
  \l 2478B
  \l 2478C
  \l 2478D
  \l 2478E
  \l 2478F
  \l 24790
  \l 24791
  \l 24792
  \l 24793
  \l 24794
  \l 24795
  \l 24796
  \l 24797
  \l 24798
  \l 24799
  \l 2479A
  \l 2479B
  \l 2479C
  \l 2479D
  \l 2479E
  \l 2479F
  \l 247A0
  \l 247A1
  \l 247A2
  \l 247A3
  \l 247A4
  \l 247A5
  \l 247A6
  \l 247A7
  \l 247A8
  \l 247A9
  \l 247AA
  \l 247AB
  \l 247AC
  \l 247AD
  \l 247AE
  \l 247AF
  \l 247B0
  \l 247B1
  \l 247B2
  \l 247B3
  \l 247B4
  \l 247B5
  \l 247B6
  \l 247B7
  \l 247B8
  \l 247B9
  \l 247BA
  \l 247BB
  \l 247BC
  \l 247BD
  \l 247BE
  \l 247BF
  \l 247C0
  \l 247C1
  \l 247C2
  \l 247C3
  \l 247C4
  \l 247C5
  \l 247C6
  \l 247C7
  \l 247C8
  \l 247C9
  \l 247CA
  \l 247CB
  \l 247CC
  \l 247CD
  \l 247CE
  \l 247CF
  \l 247D0
  \l 247D1
  \l 247D2
  \l 247D3
  \l 247D4
  \l 247D5
  \l 247D6
  \l 247D7
  \l 247D8
  \l 247D9
  \l 247DA
  \l 247DB
  \l 247DC
  \l 247DD
  \l 247DE
  \l 247DF
  \l 247E0
  \l 247E1
  \l 247E2
  \l 247E3
  \l 247E4
  \l 247E5
  \l 247E6
  \l 247E7
  \l 247E8
  \l 247E9
  \l 247EA
  \l 247EB
  \l 247EC
  \l 247ED
  \l 247EE
  \l 247EF
  \l 247F0
  \l 247F1
  \l 247F2
  \l 247F3
  \l 247F4
  \l 247F5
  \l 247F6
  \l 247F7
  \l 247F8
  \l 247F9
  \l 247FA
  \l 247FB
  \l 247FC
  \l 247FD
  \l 247FE
  \l 247FF
  \l 24800
  \l 24801
  \l 24802
  \l 24803
  \l 24804
  \l 24805
  \l 24806
  \l 24807
  \l 24808
  \l 24809
  \l 2480A
  \l 2480B
  \l 2480C
  \l 2480D
  \l 2480E
  \l 2480F
  \l 24810
  \l 24811
  \l 24812
  \l 24813
  \l 24814
  \l 24815
  \l 24816
  \l 24817
  \l 24818
  \l 24819
  \l 2481A
  \l 2481B
  \l 2481C
  \l 2481D
  \l 2481E
  \l 2481F
  \l 24820
  \l 24821
  \l 24822
  \l 24823
  \l 24824
  \l 24825
  \l 24826
  \l 24827
  \l 24828
  \l 24829
  \l 2482A
  \l 2482B
  \l 2482C
  \l 2482D
  \l 2482E
  \l 2482F
  \l 24830
  \l 24831
  \l 24832
  \l 24833
  \l 24834
  \l 24835
  \l 24836
  \l 24837
  \l 24838
  \l 24839
  \l 2483A
  \l 2483B
  \l 2483C
  \l 2483D
  \l 2483E
  \l 2483F
  \l 24840
  \l 24841
  \l 24842
  \l 24843
  \l 24844
  \l 24845
  \l 24846
  \l 24847
  \l 24848
  \l 24849
  \l 2484A
  \l 2484B
  \l 2484C
  \l 2484D
  \l 2484E
  \l 2484F
  \l 24850
  \l 24851
  \l 24852
  \l 24853
  \l 24854
  \l 24855
  \l 24856
  \l 24857
  \l 24858
  \l 24859
  \l 2485A
  \l 2485B
  \l 2485C
  \l 2485D
  \l 2485E
  \l 2485F
  \l 24860
  \l 24861
  \l 24862
  \l 24863
  \l 24864
  \l 24865
  \l 24866
  \l 24867
  \l 24868
  \l 24869
  \l 2486A
  \l 2486B
  \l 2486C
  \l 2486D
  \l 2486E
  \l 2486F
  \l 24870
  \l 24871
  \l 24872
  \l 24873
  \l 24874
  \l 24875
  \l 24876
  \l 24877
  \l 24878
  \l 24879
  \l 2487A
  \l 2487B
  \l 2487C
  \l 2487D
  \l 2487E
  \l 2487F
  \l 24880
  \l 24881
  \l 24882
  \l 24883
  \l 24884
  \l 24885
  \l 24886
  \l 24887
  \l 24888
  \l 24889
  \l 2488A
  \l 2488B
  \l 2488C
  \l 2488D
  \l 2488E
  \l 2488F
  \l 24890
  \l 24891
  \l 24892
  \l 24893
  \l 24894
  \l 24895
  \l 24896
  \l 24897
  \l 24898
  \l 24899
  \l 2489A
  \l 2489B
  \l 2489C
  \l 2489D
  \l 2489E
  \l 2489F
  \l 248A0
  \l 248A1
  \l 248A2
  \l 248A3
  \l 248A4
  \l 248A5
  \l 248A6
  \l 248A7
  \l 248A8
  \l 248A9
  \l 248AA
  \l 248AB
  \l 248AC
  \l 248AD
  \l 248AE
  \l 248AF
  \l 248B0
  \l 248B1
  \l 248B2
  \l 248B3
  \l 248B4
  \l 248B5
  \l 248B6
  \l 248B7
  \l 248B8
  \l 248B9
  \l 248BA
  \l 248BB
  \l 248BC
  \l 248BD
  \l 248BE
  \l 248BF
  \l 248C0
  \l 248C1
  \l 248C2
  \l 248C3
  \l 248C4
  \l 248C5
  \l 248C6
  \l 248C7
  \l 248C8
  \l 248C9
  \l 248CA
  \l 248CB
  \l 248CC
  \l 248CD
  \l 248CE
  \l 248CF
  \l 248D0
  \l 248D1
  \l 248D2
  \l 248D3
  \l 248D4
  \l 248D5
  \l 248D6
  \l 248D7
  \l 248D8
  \l 248D9
  \l 248DA
  \l 248DB
  \l 248DC
  \l 248DD
  \l 248DE
  \l 248DF
  \l 248E0
  \l 248E1
  \l 248E2
  \l 248E3
  \l 248E4
  \l 248E5
  \l 248E6
  \l 248E7
  \l 248E8
  \l 248E9
  \l 248EA
  \l 248EB
  \l 248EC
  \l 248ED
  \l 248EE
  \l 248EF
  \l 248F0
  \l 248F1
  \l 248F2
  \l 248F3
  \l 248F4
  \l 248F5
  \l 248F6
  \l 248F7
  \l 248F8
  \l 248F9
  \l 248FA
  \l 248FB
  \l 248FC
  \l 248FD
  \l 248FE
  \l 248FF
  \l 24900
  \l 24901
  \l 24902
  \l 24903
  \l 24904
  \l 24905
  \l 24906
  \l 24907
  \l 24908
  \l 24909
  \l 2490A
  \l 2490B
  \l 2490C
  \l 2490D
  \l 2490E
  \l 2490F
  \l 24910
  \l 24911
  \l 24912
  \l 24913
  \l 24914
  \l 24915
  \l 24916
  \l 24917
  \l 24918
  \l 24919
  \l 2491A
  \l 2491B
  \l 2491C
  \l 2491D
  \l 2491E
  \l 2491F
  \l 24920
  \l 24921
  \l 24922
  \l 24923
  \l 24924
  \l 24925
  \l 24926
  \l 24927
  \l 24928
  \l 24929
  \l 2492A
  \l 2492B
  \l 2492C
  \l 2492D
  \l 2492E
  \l 2492F
  \l 24930
  \l 24931
  \l 24932
  \l 24933
  \l 24934
  \l 24935
  \l 24936
  \l 24937
  \l 24938
  \l 24939
  \l 2493A
  \l 2493B
  \l 2493C
  \l 2493D
  \l 2493E
  \l 2493F
  \l 24940
  \l 24941
  \l 24942
  \l 24943
  \l 24944
  \l 24945
  \l 24946
  \l 24947
  \l 24948
  \l 24949
  \l 2494A
  \l 2494B
  \l 2494C
  \l 2494D
  \l 2494E
  \l 2494F
  \l 24950
  \l 24951
  \l 24952
  \l 24953
  \l 24954
  \l 24955
  \l 24956
  \l 24957
  \l 24958
  \l 24959
  \l 2495A
  \l 2495B
  \l 2495C
  \l 2495D
  \l 2495E
  \l 2495F
  \l 24960
  \l 24961
  \l 24962
  \l 24963
  \l 24964
  \l 24965
  \l 24966
  \l 24967
  \l 24968
  \l 24969
  \l 2496A
  \l 2496B
  \l 2496C
  \l 2496D
  \l 2496E
  \l 2496F
  \l 24970
  \l 24971
  \l 24972
  \l 24973
  \l 24974
  \l 24975
  \l 24976
  \l 24977
  \l 24978
  \l 24979
  \l 2497A
  \l 2497B
  \l 2497C
  \l 2497D
  \l 2497E
  \l 2497F
  \l 24980
  \l 24981
  \l 24982
  \l 24983
  \l 24984
  \l 24985
  \l 24986
  \l 24987
  \l 24988
  \l 24989
  \l 2498A
  \l 2498B
  \l 2498C
  \l 2498D
  \l 2498E
  \l 2498F
  \l 24990
  \l 24991
  \l 24992
  \l 24993
  \l 24994
  \l 24995
  \l 24996
  \l 24997
  \l 24998
  \l 24999
  \l 2499A
  \l 2499B
  \l 2499C
  \l 2499D
  \l 2499E
  \l 2499F
  \l 249A0
  \l 249A1
  \l 249A2
  \l 249A3
  \l 249A4
  \l 249A5
  \l 249A6
  \l 249A7
  \l 249A8
  \l 249A9
  \l 249AA
  \l 249AB
  \l 249AC
  \l 249AD
  \l 249AE
  \l 249AF
  \l 249B0
  \l 249B1
  \l 249B2
  \l 249B3
  \l 249B4
  \l 249B5
  \l 249B6
  \l 249B7
  \l 249B8
  \l 249B9
  \l 249BA
  \l 249BB
  \l 249BC
  \l 249BD
  \l 249BE
  \l 249BF
  \l 249C0
  \l 249C1
  \l 249C2
  \l 249C3
  \l 249C4
  \l 249C5
  \l 249C6
  \l 249C7
  \l 249C8
  \l 249C9
  \l 249CA
  \l 249CB
  \l 249CC
  \l 249CD
  \l 249CE
  \l 249CF
  \l 249D0
  \l 249D1
  \l 249D2
  \l 249D3
  \l 249D4
  \l 249D5
  \l 249D6
  \l 249D7
  \l 249D8
  \l 249D9
  \l 249DA
  \l 249DB
  \l 249DC
  \l 249DD
  \l 249DE
  \l 249DF
  \l 249E0
  \l 249E1
  \l 249E2
  \l 249E3
  \l 249E4
  \l 249E5
  \l 249E6
  \l 249E7
  \l 249E8
  \l 249E9
  \l 249EA
  \l 249EB
  \l 249EC
  \l 249ED
  \l 249EE
  \l 249EF
  \l 249F0
  \l 249F1
  \l 249F2
  \l 249F3
  \l 249F4
  \l 249F5
  \l 249F6
  \l 249F7
  \l 249F8
  \l 249F9
  \l 249FA
  \l 249FB
  \l 249FC
  \l 249FD
  \l 249FE
  \l 249FF
  \l 24A00
  \l 24A01
  \l 24A02
  \l 24A03
  \l 24A04
  \l 24A05
  \l 24A06
  \l 24A07
  \l 24A08
  \l 24A09
  \l 24A0A
  \l 24A0B
  \l 24A0C
  \l 24A0D
  \l 24A0E
  \l 24A0F
  \l 24A10
  \l 24A11
  \l 24A12
  \l 24A13
  \l 24A14
  \l 24A15
  \l 24A16
  \l 24A17
  \l 24A18
  \l 24A19
  \l 24A1A
  \l 24A1B
  \l 24A1C
  \l 24A1D
  \l 24A1E
  \l 24A1F
  \l 24A20
  \l 24A21
  \l 24A22
  \l 24A23
  \l 24A24
  \l 24A25
  \l 24A26
  \l 24A27
  \l 24A28
  \l 24A29
  \l 24A2A
  \l 24A2B
  \l 24A2C
  \l 24A2D
  \l 24A2E
  \l 24A2F
  \l 24A30
  \l 24A31
  \l 24A32
  \l 24A33
  \l 24A34
  \l 24A35
  \l 24A36
  \l 24A37
  \l 24A38
  \l 24A39
  \l 24A3A
  \l 24A3B
  \l 24A3C
  \l 24A3D
  \l 24A3E
  \l 24A3F
  \l 24A40
  \l 24A41
  \l 24A42
  \l 24A43
  \l 24A44
  \l 24A45
  \l 24A46
  \l 24A47
  \l 24A48
  \l 24A49
  \l 24A4A
  \l 24A4B
  \l 24A4C
  \l 24A4D
  \l 24A4E
  \l 24A4F
  \l 24A50
  \l 24A51
  \l 24A52
  \l 24A53
  \l 24A54
  \l 24A55
  \l 24A56
  \l 24A57
  \l 24A58
  \l 24A59
  \l 24A5A
  \l 24A5B
  \l 24A5C
  \l 24A5D
  \l 24A5E
  \l 24A5F
  \l 24A60
  \l 24A61
  \l 24A62
  \l 24A63
  \l 24A64
  \l 24A65
  \l 24A66
  \l 24A67
  \l 24A68
  \l 24A69
  \l 24A6A
  \l 24A6B
  \l 24A6C
  \l 24A6D
  \l 24A6E
  \l 24A6F
  \l 24A70
  \l 24A71
  \l 24A72
  \l 24A73
  \l 24A74
  \l 24A75
  \l 24A76
  \l 24A77
  \l 24A78
  \l 24A79
  \l 24A7A
  \l 24A7B
  \l 24A7C
  \l 24A7D
  \l 24A7E
  \l 24A7F
  \l 24A80
  \l 24A81
  \l 24A82
  \l 24A83
  \l 24A84
  \l 24A85
  \l 24A86
  \l 24A87
  \l 24A88
  \l 24A89
  \l 24A8A
  \l 24A8B
  \l 24A8C
  \l 24A8D
  \l 24A8E
  \l 24A8F
  \l 24A90
  \l 24A91
  \l 24A92
  \l 24A93
  \l 24A94
  \l 24A95
  \l 24A96
  \l 24A97
  \l 24A98
  \l 24A99
  \l 24A9A
  \l 24A9B
  \l 24A9C
  \l 24A9D
  \l 24A9E
  \l 24A9F
  \l 24AA0
  \l 24AA1
  \l 24AA2
  \l 24AA3
  \l 24AA4
  \l 24AA5
  \l 24AA6
  \l 24AA7
  \l 24AA8
  \l 24AA9
  \l 24AAA
  \l 24AAB
  \l 24AAC
  \l 24AAD
  \l 24AAE
  \l 24AAF
  \l 24AB0
  \l 24AB1
  \l 24AB2
  \l 24AB3
  \l 24AB4
  \l 24AB5
  \l 24AB6
  \l 24AB7
  \l 24AB8
  \l 24AB9
  \l 24ABA
  \l 24ABB
  \l 24ABC
  \l 24ABD
  \l 24ABE
  \l 24ABF
  \l 24AC0
  \l 24AC1
  \l 24AC2
  \l 24AC3
  \l 24AC4
  \l 24AC5
  \l 24AC6
  \l 24AC7
  \l 24AC8
  \l 24AC9
  \l 24ACA
  \l 24ACB
  \l 24ACC
  \l 24ACD
  \l 24ACE
  \l 24ACF
  \l 24AD0
  \l 24AD1
  \l 24AD2
  \l 24AD3
  \l 24AD4
  \l 24AD5
  \l 24AD6
  \l 24AD7
  \l 24AD8
  \l 24AD9
  \l 24ADA
  \l 24ADB
  \l 24ADC
  \l 24ADD
  \l 24ADE
  \l 24ADF
  \l 24AE0
  \l 24AE1
  \l 24AE2
  \l 24AE3
  \l 24AE4
  \l 24AE5
  \l 24AE6
  \l 24AE7
  \l 24AE8
  \l 24AE9
  \l 24AEA
  \l 24AEB
  \l 24AEC
  \l 24AED
  \l 24AEE
  \l 24AEF
  \l 24AF0
  \l 24AF1
  \l 24AF2
  \l 24AF3
  \l 24AF4
  \l 24AF5
  \l 24AF6
  \l 24AF7
  \l 24AF8
  \l 24AF9
  \l 24AFA
  \l 24AFB
  \l 24AFC
  \l 24AFD
  \l 24AFE
  \l 24AFF
  \l 24B00
  \l 24B01
  \l 24B02
  \l 24B03
  \l 24B04
  \l 24B05
  \l 24B06
  \l 24B07
  \l 24B08
  \l 24B09
  \l 24B0A
  \l 24B0B
  \l 24B0C
  \l 24B0D
  \l 24B0E
  \l 24B0F
  \l 24B10
  \l 24B11
  \l 24B12
  \l 24B13
  \l 24B14
  \l 24B15
  \l 24B16
  \l 24B17
  \l 24B18
  \l 24B19
  \l 24B1A
  \l 24B1B
  \l 24B1C
  \l 24B1D
  \l 24B1E
  \l 24B1F
  \l 24B20
  \l 24B21
  \l 24B22
  \l 24B23
  \l 24B24
  \l 24B25
  \l 24B26
  \l 24B27
  \l 24B28
  \l 24B29
  \l 24B2A
  \l 24B2B
  \l 24B2C
  \l 24B2D
  \l 24B2E
  \l 24B2F
  \l 24B30
  \l 24B31
  \l 24B32
  \l 24B33
  \l 24B34
  \l 24B35
  \l 24B36
  \l 24B37
  \l 24B38
  \l 24B39
  \l 24B3A
  \l 24B3B
  \l 24B3C
  \l 24B3D
  \l 24B3E
  \l 24B3F
  \l 24B40
  \l 24B41
  \l 24B42
  \l 24B43
  \l 24B44
  \l 24B45
  \l 24B46
  \l 24B47
  \l 24B48
  \l 24B49
  \l 24B4A
  \l 24B4B
  \l 24B4C
  \l 24B4D
  \l 24B4E
  \l 24B4F
  \l 24B50
  \l 24B51
  \l 24B52
  \l 24B53
  \l 24B54
  \l 24B55
  \l 24B56
  \l 24B57
  \l 24B58
  \l 24B59
  \l 24B5A
  \l 24B5B
  \l 24B5C
  \l 24B5D
  \l 24B5E
  \l 24B5F
  \l 24B60
  \l 24B61
  \l 24B62
  \l 24B63
  \l 24B64
  \l 24B65
  \l 24B66
  \l 24B67
  \l 24B68
  \l 24B69
  \l 24B6A
  \l 24B6B
  \l 24B6C
  \l 24B6D
  \l 24B6E
  \l 24B6F
  \l 24B70
  \l 24B71
  \l 24B72
  \l 24B73
  \l 24B74
  \l 24B75
  \l 24B76
  \l 24B77
  \l 24B78
  \l 24B79
  \l 24B7A
  \l 24B7B
  \l 24B7C
  \l 24B7D
  \l 24B7E
  \l 24B7F
  \l 24B80
  \l 24B81
  \l 24B82
  \l 24B83
  \l 24B84
  \l 24B85
  \l 24B86
  \l 24B87
  \l 24B88
  \l 24B89
  \l 24B8A
  \l 24B8B
  \l 24B8C
  \l 24B8D
  \l 24B8E
  \l 24B8F
  \l 24B90
  \l 24B91
  \l 24B92
  \l 24B93
  \l 24B94
  \l 24B95
  \l 24B96
  \l 24B97
  \l 24B98
  \l 24B99
  \l 24B9A
  \l 24B9B
  \l 24B9C
  \l 24B9D
  \l 24B9E
  \l 24B9F
  \l 24BA0
  \l 24BA1
  \l 24BA2
  \l 24BA3
  \l 24BA4
  \l 24BA5
  \l 24BA6
  \l 24BA7
  \l 24BA8
  \l 24BA9
  \l 24BAA
  \l 24BAB
  \l 24BAC
  \l 24BAD
  \l 24BAE
  \l 24BAF
  \l 24BB0
  \l 24BB1
  \l 24BB2
  \l 24BB3
  \l 24BB4
  \l 24BB5
  \l 24BB6
  \l 24BB7
  \l 24BB8
  \l 24BB9
  \l 24BBA
  \l 24BBB
  \l 24BBC
  \l 24BBD
  \l 24BBE
  \l 24BBF
  \l 24BC0
  \l 24BC1
  \l 24BC2
  \l 24BC3
  \l 24BC4
  \l 24BC5
  \l 24BC6
  \l 24BC7
  \l 24BC8
  \l 24BC9
  \l 24BCA
  \l 24BCB
  \l 24BCC
  \l 24BCD
  \l 24BCE
  \l 24BCF
  \l 24BD0
  \l 24BD1
  \l 24BD2
  \l 24BD3
  \l 24BD4
  \l 24BD5
  \l 24BD6
  \l 24BD7
  \l 24BD8
  \l 24BD9
  \l 24BDA
  \l 24BDB
  \l 24BDC
  \l 24BDD
  \l 24BDE
  \l 24BDF
  \l 24BE0
  \l 24BE1
  \l 24BE2
  \l 24BE3
  \l 24BE4
  \l 24BE5
  \l 24BE6
  \l 24BE7
  \l 24BE8
  \l 24BE9
  \l 24BEA
  \l 24BEB
  \l 24BEC
  \l 24BED
  \l 24BEE
  \l 24BEF
  \l 24BF0
  \l 24BF1
  \l 24BF2
  \l 24BF3
  \l 24BF4
  \l 24BF5
  \l 24BF6
  \l 24BF7
  \l 24BF8
  \l 24BF9
  \l 24BFA
  \l 24BFB
  \l 24BFC
  \l 24BFD
  \l 24BFE
  \l 24BFF
  \l 24C00
  \l 24C01
  \l 24C02
  \l 24C03
  \l 24C04
  \l 24C05
  \l 24C06
  \l 24C07
  \l 24C08
  \l 24C09
  \l 24C0A
  \l 24C0B
  \l 24C0C
  \l 24C0D
  \l 24C0E
  \l 24C0F
  \l 24C10
  \l 24C11
  \l 24C12
  \l 24C13
  \l 24C14
  \l 24C15
  \l 24C16
  \l 24C17
  \l 24C18
  \l 24C19
  \l 24C1A
  \l 24C1B
  \l 24C1C
  \l 24C1D
  \l 24C1E
  \l 24C1F
  \l 24C20
  \l 24C21
  \l 24C22
  \l 24C23
  \l 24C24
  \l 24C25
  \l 24C26
  \l 24C27
  \l 24C28
  \l 24C29
  \l 24C2A
  \l 24C2B
  \l 24C2C
  \l 24C2D
  \l 24C2E
  \l 24C2F
  \l 24C30
  \l 24C31
  \l 24C32
  \l 24C33
  \l 24C34
  \l 24C35
  \l 24C36
  \l 24C37
  \l 24C38
  \l 24C39
  \l 24C3A
  \l 24C3B
  \l 24C3C
  \l 24C3D
  \l 24C3E
  \l 24C3F
  \l 24C40
  \l 24C41
  \l 24C42
  \l 24C43
  \l 24C44
  \l 24C45
  \l 24C46
  \l 24C47
  \l 24C48
  \l 24C49
  \l 24C4A
  \l 24C4B
  \l 24C4C
  \l 24C4D
  \l 24C4E
  \l 24C4F
  \l 24C50
  \l 24C51
  \l 24C52
  \l 24C53
  \l 24C54
  \l 24C55
  \l 24C56
  \l 24C57
  \l 24C58
  \l 24C59
  \l 24C5A
  \l 24C5B
  \l 24C5C
  \l 24C5D
  \l 24C5E
  \l 24C5F
  \l 24C60
  \l 24C61
  \l 24C62
  \l 24C63
  \l 24C64
  \l 24C65
  \l 24C66
  \l 24C67
  \l 24C68
  \l 24C69
  \l 24C6A
  \l 24C6B
  \l 24C6C
  \l 24C6D
  \l 24C6E
  \l 24C6F
  \l 24C70
  \l 24C71
  \l 24C72
  \l 24C73
  \l 24C74
  \l 24C75
  \l 24C76
  \l 24C77
  \l 24C78
  \l 24C79
  \l 24C7A
  \l 24C7B
  \l 24C7C
  \l 24C7D
  \l 24C7E
  \l 24C7F
  \l 24C80
  \l 24C81
  \l 24C82
  \l 24C83
  \l 24C84
  \l 24C85
  \l 24C86
  \l 24C87
  \l 24C88
  \l 24C89
  \l 24C8A
  \l 24C8B
  \l 24C8C
  \l 24C8D
  \l 24C8E
  \l 24C8F
  \l 24C90
  \l 24C91
  \l 24C92
  \l 24C93
  \l 24C94
  \l 24C95
  \l 24C96
  \l 24C97
  \l 24C98
  \l 24C99
  \l 24C9A
  \l 24C9B
  \l 24C9C
  \l 24C9D
  \l 24C9E
  \l 24C9F
  \l 24CA0
  \l 24CA1
  \l 24CA2
  \l 24CA3
  \l 24CA4
  \l 24CA5
  \l 24CA6
  \l 24CA7
  \l 24CA8
  \l 24CA9
  \l 24CAA
  \l 24CAB
  \l 24CAC
  \l 24CAD
  \l 24CAE
  \l 24CAF
  \l 24CB0
  \l 24CB1
  \l 24CB2
  \l 24CB3
  \l 24CB4
  \l 24CB5
  \l 24CB6
  \l 24CB7
  \l 24CB8
  \l 24CB9
  \l 24CBA
  \l 24CBB
  \l 24CBC
  \l 24CBD
  \l 24CBE
  \l 24CBF
  \l 24CC0
  \l 24CC1
  \l 24CC2
  \l 24CC3
  \l 24CC4
  \l 24CC5
  \l 24CC6
  \l 24CC7
  \l 24CC8
  \l 24CC9
  \l 24CCA
  \l 24CCB
  \l 24CCC
  \l 24CCD
  \l 24CCE
  \l 24CCF
  \l 24CD0
  \l 24CD1
  \l 24CD2
  \l 24CD3
  \l 24CD4
  \l 24CD5
  \l 24CD6
  \l 24CD7
  \l 24CD8
  \l 24CD9
  \l 24CDA
  \l 24CDB
  \l 24CDC
  \l 24CDD
  \l 24CDE
  \l 24CDF
  \l 24CE0
  \l 24CE1
  \l 24CE2
  \l 24CE3
  \l 24CE4
  \l 24CE5
  \l 24CE6
  \l 24CE7
  \l 24CE8
  \l 24CE9
  \l 24CEA
  \l 24CEB
  \l 24CEC
  \l 24CED
  \l 24CEE
  \l 24CEF
  \l 24CF0
  \l 24CF1
  \l 24CF2
  \l 24CF3
  \l 24CF4
  \l 24CF5
  \l 24CF6
  \l 24CF7
  \l 24CF8
  \l 24CF9
  \l 24CFA
  \l 24CFB
  \l 24CFC
  \l 24CFD
  \l 24CFE
  \l 24CFF
  \l 24D00
  \l 24D01
  \l 24D02
  \l 24D03
  \l 24D04
  \l 24D05
  \l 24D06
  \l 24D07
  \l 24D08
  \l 24D09
  \l 24D0A
  \l 24D0B
  \l 24D0C
  \l 24D0D
  \l 24D0E
  \l 24D0F
  \l 24D10
  \l 24D11
  \l 24D12
  \l 24D13
  \l 24D14
  \l 24D15
  \l 24D16
  \l 24D17
  \l 24D18
  \l 24D19
  \l 24D1A
  \l 24D1B
  \l 24D1C
  \l 24D1D
  \l 24D1E
  \l 24D1F
  \l 24D20
  \l 24D21
  \l 24D22
  \l 24D23
  \l 24D24
  \l 24D25
  \l 24D26
  \l 24D27
  \l 24D28
  \l 24D29
  \l 24D2A
  \l 24D2B
  \l 24D2C
  \l 24D2D
  \l 24D2E
  \l 24D2F
  \l 24D30
  \l 24D31
  \l 24D32
  \l 24D33
  \l 24D34
  \l 24D35
  \l 24D36
  \l 24D37
  \l 24D38
  \l 24D39
  \l 24D3A
  \l 24D3B
  \l 24D3C
  \l 24D3D
  \l 24D3E
  \l 24D3F
  \l 24D40
  \l 24D41
  \l 24D42
  \l 24D43
  \l 24D44
  \l 24D45
  \l 24D46
  \l 24D47
  \l 24D48
  \l 24D49
  \l 24D4A
  \l 24D4B
  \l 24D4C
  \l 24D4D
  \l 24D4E
  \l 24D4F
  \l 24D50
  \l 24D51
  \l 24D52
  \l 24D53
  \l 24D54
  \l 24D55
  \l 24D56
  \l 24D57
  \l 24D58
  \l 24D59
  \l 24D5A
  \l 24D5B
  \l 24D5C
  \l 24D5D
  \l 24D5E
  \l 24D5F
  \l 24D60
  \l 24D61
  \l 24D62
  \l 24D63
  \l 24D64
  \l 24D65
  \l 24D66
  \l 24D67
  \l 24D68
  \l 24D69
  \l 24D6A
  \l 24D6B
  \l 24D6C
  \l 24D6D
  \l 24D6E
  \l 24D6F
  \l 24D70
  \l 24D71
  \l 24D72
  \l 24D73
  \l 24D74
  \l 24D75
  \l 24D76
  \l 24D77
  \l 24D78
  \l 24D79
  \l 24D7A
  \l 24D7B
  \l 24D7C
  \l 24D7D
  \l 24D7E
  \l 24D7F
  \l 24D80
  \l 24D81
  \l 24D82
  \l 24D83
  \l 24D84
  \l 24D85
  \l 24D86
  \l 24D87
  \l 24D88
  \l 24D89
  \l 24D8A
  \l 24D8B
  \l 24D8C
  \l 24D8D
  \l 24D8E
  \l 24D8F
  \l 24D90
  \l 24D91
  \l 24D92
  \l 24D93
  \l 24D94
  \l 24D95
  \l 24D96
  \l 24D97
  \l 24D98
  \l 24D99
  \l 24D9A
  \l 24D9B
  \l 24D9C
  \l 24D9D
  \l 24D9E
  \l 24D9F
  \l 24DA0
  \l 24DA1
  \l 24DA2
  \l 24DA3
  \l 24DA4
  \l 24DA5
  \l 24DA6
  \l 24DA7
  \l 24DA8
  \l 24DA9
  \l 24DAA
  \l 24DAB
  \l 24DAC
  \l 24DAD
  \l 24DAE
  \l 24DAF
  \l 24DB0
  \l 24DB1
  \l 24DB2
  \l 24DB3
  \l 24DB4
  \l 24DB5
  \l 24DB6
  \l 24DB7
  \l 24DB8
  \l 24DB9
  \l 24DBA
  \l 24DBB
  \l 24DBC
  \l 24DBD
  \l 24DBE
  \l 24DBF
  \l 24DC0
  \l 24DC1
  \l 24DC2
  \l 24DC3
  \l 24DC4
  \l 24DC5
  \l 24DC6
  \l 24DC7
  \l 24DC8
  \l 24DC9
  \l 24DCA
  \l 24DCB
  \l 24DCC
  \l 24DCD
  \l 24DCE
  \l 24DCF
  \l 24DD0
  \l 24DD1
  \l 24DD2
  \l 24DD3
  \l 24DD4
  \l 24DD5
  \l 24DD6
  \l 24DD7
  \l 24DD8
  \l 24DD9
  \l 24DDA
  \l 24DDB
  \l 24DDC
  \l 24DDD
  \l 24DDE
  \l 24DDF
  \l 24DE0
  \l 24DE1
  \l 24DE2
  \l 24DE3
  \l 24DE4
  \l 24DE5
  \l 24DE6
  \l 24DE7
  \l 24DE8
  \l 24DE9
  \l 24DEA
  \l 24DEB
  \l 24DEC
  \l 24DED
  \l 24DEE
  \l 24DEF
  \l 24DF0
  \l 24DF1
  \l 24DF2
  \l 24DF3
  \l 24DF4
  \l 24DF5
  \l 24DF6
  \l 24DF7
  \l 24DF8
  \l 24DF9
  \l 24DFA
  \l 24DFB
  \l 24DFC
  \l 24DFD
  \l 24DFE
  \l 24DFF
  \l 24E00
  \l 24E01
  \l 24E02
  \l 24E03
  \l 24E04
  \l 24E05
  \l 24E06
  \l 24E07
  \l 24E08
  \l 24E09
  \l 24E0A
  \l 24E0B
  \l 24E0C
  \l 24E0D
  \l 24E0E
  \l 24E0F
  \l 24E10
  \l 24E11
  \l 24E12
  \l 24E13
  \l 24E14
  \l 24E15
  \l 24E16
  \l 24E17
  \l 24E18
  \l 24E19
  \l 24E1A
  \l 24E1B
  \l 24E1C
  \l 24E1D
  \l 24E1E
  \l 24E1F
  \l 24E20
  \l 24E21
  \l 24E22
  \l 24E23
  \l 24E24
  \l 24E25
  \l 24E26
  \l 24E27
  \l 24E28
  \l 24E29
  \l 24E2A
  \l 24E2B
  \l 24E2C
  \l 24E2D
  \l 24E2E
  \l 24E2F
  \l 24E30
  \l 24E31
  \l 24E32
  \l 24E33
  \l 24E34
  \l 24E35
  \l 24E36
  \l 24E37
  \l 24E38
  \l 24E39
  \l 24E3A
  \l 24E3B
  \l 24E3C
  \l 24E3D
  \l 24E3E
  \l 24E3F
  \l 24E40
  \l 24E41
  \l 24E42
  \l 24E43
  \l 24E44
  \l 24E45
  \l 24E46
  \l 24E47
  \l 24E48
  \l 24E49
  \l 24E4A
  \l 24E4B
  \l 24E4C
  \l 24E4D
  \l 24E4E
  \l 24E4F
  \l 24E50
  \l 24E51
  \l 24E52
  \l 24E53
  \l 24E54
  \l 24E55
  \l 24E56
  \l 24E57
  \l 24E58
  \l 24E59
  \l 24E5A
  \l 24E5B
  \l 24E5C
  \l 24E5D
  \l 24E5E
  \l 24E5F
  \l 24E60
  \l 24E61
  \l 24E62
  \l 24E63
  \l 24E64
  \l 24E65
  \l 24E66
  \l 24E67
  \l 24E68
  \l 24E69
  \l 24E6A
  \l 24E6B
  \l 24E6C
  \l 24E6D
  \l 24E6E
  \l 24E6F
  \l 24E70
  \l 24E71
  \l 24E72
  \l 24E73
  \l 24E74
  \l 24E75
  \l 24E76
  \l 24E77
  \l 24E78
  \l 24E79
  \l 24E7A
  \l 24E7B
  \l 24E7C
  \l 24E7D
  \l 24E7E
  \l 24E7F
  \l 24E80
  \l 24E81
  \l 24E82
  \l 24E83
  \l 24E84
  \l 24E85
  \l 24E86
  \l 24E87
  \l 24E88
  \l 24E89
  \l 24E8A
  \l 24E8B
  \l 24E8C
  \l 24E8D
  \l 24E8E
  \l 24E8F
  \l 24E90
  \l 24E91
  \l 24E92
  \l 24E93
  \l 24E94
  \l 24E95
  \l 24E96
  \l 24E97
  \l 24E98
  \l 24E99
  \l 24E9A
  \l 24E9B
  \l 24E9C
  \l 24E9D
  \l 24E9E
  \l 24E9F
  \l 24EA0
  \l 24EA1
  \l 24EA2
  \l 24EA3
  \l 24EA4
  \l 24EA5
  \l 24EA6
  \l 24EA7
  \l 24EA8
  \l 24EA9
  \l 24EAA
  \l 24EAB
  \l 24EAC
  \l 24EAD
  \l 24EAE
  \l 24EAF
  \l 24EB0
  \l 24EB1
  \l 24EB2
  \l 24EB3
  \l 24EB4
  \l 24EB5
  \l 24EB6
  \l 24EB7
  \l 24EB8
  \l 24EB9
  \l 24EBA
  \l 24EBB
  \l 24EBC
  \l 24EBD
  \l 24EBE
  \l 24EBF
  \l 24EC0
  \l 24EC1
  \l 24EC2
  \l 24EC3
  \l 24EC4
  \l 24EC5
  \l 24EC6
  \l 24EC7
  \l 24EC8
  \l 24EC9
  \l 24ECA
  \l 24ECB
  \l 24ECC
  \l 24ECD
  \l 24ECE
  \l 24ECF
  \l 24ED0
  \l 24ED1
  \l 24ED2
  \l 24ED3
  \l 24ED4
  \l 24ED5
  \l 24ED6
  \l 24ED7
  \l 24ED8
  \l 24ED9
  \l 24EDA
  \l 24EDB
  \l 24EDC
  \l 24EDD
  \l 24EDE
  \l 24EDF
  \l 24EE0
  \l 24EE1
  \l 24EE2
  \l 24EE3
  \l 24EE4
  \l 24EE5
  \l 24EE6
  \l 24EE7
  \l 24EE8
  \l 24EE9
  \l 24EEA
  \l 24EEB
  \l 24EEC
  \l 24EED
  \l 24EEE
  \l 24EEF
  \l 24EF0
  \l 24EF1
  \l 24EF2
  \l 24EF3
  \l 24EF4
  \l 24EF5
  \l 24EF6
  \l 24EF7
  \l 24EF8
  \l 24EF9
  \l 24EFA
  \l 24EFB
  \l 24EFC
  \l 24EFD
  \l 24EFE
  \l 24EFF
  \l 24F00
  \l 24F01
  \l 24F02
  \l 24F03
  \l 24F04
  \l 24F05
  \l 24F06
  \l 24F07
  \l 24F08
  \l 24F09
  \l 24F0A
  \l 24F0B
  \l 24F0C
  \l 24F0D
  \l 24F0E
  \l 24F0F
  \l 24F10
  \l 24F11
  \l 24F12
  \l 24F13
  \l 24F14
  \l 24F15
  \l 24F16
  \l 24F17
  \l 24F18
  \l 24F19
  \l 24F1A
  \l 24F1B
  \l 24F1C
  \l 24F1D
  \l 24F1E
  \l 24F1F
  \l 24F20
  \l 24F21
  \l 24F22
  \l 24F23
  \l 24F24
  \l 24F25
  \l 24F26
  \l 24F27
  \l 24F28
  \l 24F29
  \l 24F2A
  \l 24F2B
  \l 24F2C
  \l 24F2D
  \l 24F2E
  \l 24F2F
  \l 24F30
  \l 24F31
  \l 24F32
  \l 24F33
  \l 24F34
  \l 24F35
  \l 24F36
  \l 24F37
  \l 24F38
  \l 24F39
  \l 24F3A
  \l 24F3B
  \l 24F3C
  \l 24F3D
  \l 24F3E
  \l 24F3F
  \l 24F40
  \l 24F41
  \l 24F42
  \l 24F43
  \l 24F44
  \l 24F45
  \l 24F46
  \l 24F47
  \l 24F48
  \l 24F49
  \l 24F4A
  \l 24F4B
  \l 24F4C
  \l 24F4D
  \l 24F4E
  \l 24F4F
  \l 24F50
  \l 24F51
  \l 24F52
  \l 24F53
  \l 24F54
  \l 24F55
  \l 24F56
  \l 24F57
  \l 24F58
  \l 24F59
  \l 24F5A
  \l 24F5B
  \l 24F5C
  \l 24F5D
  \l 24F5E
  \l 24F5F
  \l 24F60
  \l 24F61
  \l 24F62
  \l 24F63
  \l 24F64
  \l 24F65
  \l 24F66
  \l 24F67
  \l 24F68
  \l 24F69
  \l 24F6A
  \l 24F6B
  \l 24F6C
  \l 24F6D
  \l 24F6E
  \l 24F6F
  \l 24F70
  \l 24F71
  \l 24F72
  \l 24F73
  \l 24F74
  \l 24F75
  \l 24F76
  \l 24F77
  \l 24F78
  \l 24F79
  \l 24F7A
  \l 24F7B
  \l 24F7C
  \l 24F7D
  \l 24F7E
  \l 24F7F
  \l 24F80
  \l 24F81
  \l 24F82
  \l 24F83
  \l 24F84
  \l 24F85
  \l 24F86
  \l 24F87
  \l 24F88
  \l 24F89
  \l 24F8A
  \l 24F8B
  \l 24F8C
  \l 24F8D
  \l 24F8E
  \l 24F8F
  \l 24F90
  \l 24F91
  \l 24F92
  \l 24F93
  \l 24F94
  \l 24F95
  \l 24F96
  \l 24F97
  \l 24F98
  \l 24F99
  \l 24F9A
  \l 24F9B
  \l 24F9C
  \l 24F9D
  \l 24F9E
  \l 24F9F
  \l 24FA0
  \l 24FA1
  \l 24FA2
  \l 24FA3
  \l 24FA4
  \l 24FA5
  \l 24FA6
  \l 24FA7
  \l 24FA8
  \l 24FA9
  \l 24FAA
  \l 24FAB
  \l 24FAC
  \l 24FAD
  \l 24FAE
  \l 24FAF
  \l 24FB0
  \l 24FB1
  \l 24FB2
  \l 24FB3
  \l 24FB4
  \l 24FB5
  \l 24FB6
  \l 24FB7
  \l 24FB8
  \l 24FB9
  \l 24FBA
  \l 24FBB
  \l 24FBC
  \l 24FBD
  \l 24FBE
  \l 24FBF
  \l 24FC0
  \l 24FC1
  \l 24FC2
  \l 24FC3
  \l 24FC4
  \l 24FC5
  \l 24FC6
  \l 24FC7
  \l 24FC8
  \l 24FC9
  \l 24FCA
  \l 24FCB
  \l 24FCC
  \l 24FCD
  \l 24FCE
  \l 24FCF
  \l 24FD0
  \l 24FD1
  \l 24FD2
  \l 24FD3
  \l 24FD4
  \l 24FD5
  \l 24FD6
  \l 24FD7
  \l 24FD8
  \l 24FD9
  \l 24FDA
  \l 24FDB
  \l 24FDC
  \l 24FDD
  \l 24FDE
  \l 24FDF
  \l 24FE0
  \l 24FE1
  \l 24FE2
  \l 24FE3
  \l 24FE4
  \l 24FE5
  \l 24FE6
  \l 24FE7
  \l 24FE8
  \l 24FE9
  \l 24FEA
  \l 24FEB
  \l 24FEC
  \l 24FED
  \l 24FEE
  \l 24FEF
  \l 24FF0
  \l 24FF1
  \l 24FF2
  \l 24FF3
  \l 24FF4
  \l 24FF5
  \l 24FF6
  \l 24FF7
  \l 24FF8
  \l 24FF9
  \l 24FFA
  \l 24FFB
  \l 24FFC
  \l 24FFD
  \l 24FFE
  \l 24FFF
  \l 25000
  \l 25001
  \l 25002
  \l 25003
  \l 25004
  \l 25005
  \l 25006
  \l 25007
  \l 25008
  \l 25009
  \l 2500A
  \l 2500B
  \l 2500C
  \l 2500D
  \l 2500E
  \l 2500F
  \l 25010
  \l 25011
  \l 25012
  \l 25013
  \l 25014
  \l 25015
  \l 25016
  \l 25017
  \l 25018
  \l 25019
  \l 2501A
  \l 2501B
  \l 2501C
  \l 2501D
  \l 2501E
  \l 2501F
  \l 25020
  \l 25021
  \l 25022
  \l 25023
  \l 25024
  \l 25025
  \l 25026
  \l 25027
  \l 25028
  \l 25029
  \l 2502A
  \l 2502B
  \l 2502C
  \l 2502D
  \l 2502E
  \l 2502F
  \l 25030
  \l 25031
  \l 25032
  \l 25033
  \l 25034
  \l 25035
  \l 25036
  \l 25037
  \l 25038
  \l 25039
  \l 2503A
  \l 2503B
  \l 2503C
  \l 2503D
  \l 2503E
  \l 2503F
  \l 25040
  \l 25041
  \l 25042
  \l 25043
  \l 25044
  \l 25045
  \l 25046
  \l 25047
  \l 25048
  \l 25049
  \l 2504A
  \l 2504B
  \l 2504C
  \l 2504D
  \l 2504E
  \l 2504F
  \l 25050
  \l 25051
  \l 25052
  \l 25053
  \l 25054
  \l 25055
  \l 25056
  \l 25057
  \l 25058
  \l 25059
  \l 2505A
  \l 2505B
  \l 2505C
  \l 2505D
  \l 2505E
  \l 2505F
  \l 25060
  \l 25061
  \l 25062
  \l 25063
  \l 25064
  \l 25065
  \l 25066
  \l 25067
  \l 25068
  \l 25069
  \l 2506A
  \l 2506B
  \l 2506C
  \l 2506D
  \l 2506E
  \l 2506F
  \l 25070
  \l 25071
  \l 25072
  \l 25073
  \l 25074
  \l 25075
  \l 25076
  \l 25077
  \l 25078
  \l 25079
  \l 2507A
  \l 2507B
  \l 2507C
  \l 2507D
  \l 2507E
  \l 2507F
  \l 25080
  \l 25081
  \l 25082
  \l 25083
  \l 25084
  \l 25085
  \l 25086
  \l 25087
  \l 25088
  \l 25089
  \l 2508A
  \l 2508B
  \l 2508C
  \l 2508D
  \l 2508E
  \l 2508F
  \l 25090
  \l 25091
  \l 25092
  \l 25093
  \l 25094
  \l 25095
  \l 25096
  \l 25097
  \l 25098
  \l 25099
  \l 2509A
  \l 2509B
  \l 2509C
  \l 2509D
  \l 2509E
  \l 2509F
  \l 250A0
  \l 250A1
  \l 250A2
  \l 250A3
  \l 250A4
  \l 250A5
  \l 250A6
  \l 250A7
  \l 250A8
  \l 250A9
  \l 250AA
  \l 250AB
  \l 250AC
  \l 250AD
  \l 250AE
  \l 250AF
  \l 250B0
  \l 250B1
  \l 250B2
  \l 250B3
  \l 250B4
  \l 250B5
  \l 250B6
  \l 250B7
  \l 250B8
  \l 250B9
  \l 250BA
  \l 250BB
  \l 250BC
  \l 250BD
  \l 250BE
  \l 250BF
  \l 250C0
  \l 250C1
  \l 250C2
  \l 250C3
  \l 250C4
  \l 250C5
  \l 250C6
  \l 250C7
  \l 250C8
  \l 250C9
  \l 250CA
  \l 250CB
  \l 250CC
  \l 250CD
  \l 250CE
  \l 250CF
  \l 250D0
  \l 250D1
  \l 250D2
  \l 250D3
  \l 250D4
  \l 250D5
  \l 250D6
  \l 250D7
  \l 250D8
  \l 250D9
  \l 250DA
  \l 250DB
  \l 250DC
  \l 250DD
  \l 250DE
  \l 250DF
  \l 250E0
  \l 250E1
  \l 250E2
  \l 250E3
  \l 250E4
  \l 250E5
  \l 250E6
  \l 250E7
  \l 250E8
  \l 250E9
  \l 250EA
  \l 250EB
  \l 250EC
  \l 250ED
  \l 250EE
  \l 250EF
  \l 250F0
  \l 250F1
  \l 250F2
  \l 250F3
  \l 250F4
  \l 250F5
  \l 250F6
  \l 250F7
  \l 250F8
  \l 250F9
  \l 250FA
  \l 250FB
  \l 250FC
  \l 250FD
  \l 250FE
  \l 250FF
  \l 25100
  \l 25101
  \l 25102
  \l 25103
  \l 25104
  \l 25105
  \l 25106
  \l 25107
  \l 25108
  \l 25109
  \l 2510A
  \l 2510B
  \l 2510C
  \l 2510D
  \l 2510E
  \l 2510F
  \l 25110
  \l 25111
  \l 25112
  \l 25113
  \l 25114
  \l 25115
  \l 25116
  \l 25117
  \l 25118
  \l 25119
  \l 2511A
  \l 2511B
  \l 2511C
  \l 2511D
  \l 2511E
  \l 2511F
  \l 25120
  \l 25121
  \l 25122
  \l 25123
  \l 25124
  \l 25125
  \l 25126
  \l 25127
  \l 25128
  \l 25129
  \l 2512A
  \l 2512B
  \l 2512C
  \l 2512D
  \l 2512E
  \l 2512F
  \l 25130
  \l 25131
  \l 25132
  \l 25133
  \l 25134
  \l 25135
  \l 25136
  \l 25137
  \l 25138
  \l 25139
  \l 2513A
  \l 2513B
  \l 2513C
  \l 2513D
  \l 2513E
  \l 2513F
  \l 25140
  \l 25141
  \l 25142
  \l 25143
  \l 25144
  \l 25145
  \l 25146
  \l 25147
  \l 25148
  \l 25149
  \l 2514A
  \l 2514B
  \l 2514C
  \l 2514D
  \l 2514E
  \l 2514F
  \l 25150
  \l 25151
  \l 25152
  \l 25153
  \l 25154
  \l 25155
  \l 25156
  \l 25157
  \l 25158
  \l 25159
  \l 2515A
  \l 2515B
  \l 2515C
  \l 2515D
  \l 2515E
  \l 2515F
  \l 25160
  \l 25161
  \l 25162
  \l 25163
  \l 25164
  \l 25165
  \l 25166
  \l 25167
  \l 25168
  \l 25169
  \l 2516A
  \l 2516B
  \l 2516C
  \l 2516D
  \l 2516E
  \l 2516F
  \l 25170
  \l 25171
  \l 25172
  \l 25173
  \l 25174
  \l 25175
  \l 25176
  \l 25177
  \l 25178
  \l 25179
  \l 2517A
  \l 2517B
  \l 2517C
  \l 2517D
  \l 2517E
  \l 2517F
  \l 25180
  \l 25181
  \l 25182
  \l 25183
  \l 25184
  \l 25185
  \l 25186
  \l 25187
  \l 25188
  \l 25189
  \l 2518A
  \l 2518B
  \l 2518C
  \l 2518D
  \l 2518E
  \l 2518F
  \l 25190
  \l 25191
  \l 25192
  \l 25193
  \l 25194
  \l 25195
  \l 25196
  \l 25197
  \l 25198
  \l 25199
  \l 2519A
  \l 2519B
  \l 2519C
  \l 2519D
  \l 2519E
  \l 2519F
  \l 251A0
  \l 251A1
  \l 251A2
  \l 251A3
  \l 251A4
  \l 251A5
  \l 251A6
  \l 251A7
  \l 251A8
  \l 251A9
  \l 251AA
  \l 251AB
  \l 251AC
  \l 251AD
  \l 251AE
  \l 251AF
  \l 251B0
  \l 251B1
  \l 251B2
  \l 251B3
  \l 251B4
  \l 251B5
  \l 251B6
  \l 251B7
  \l 251B8
  \l 251B9
  \l 251BA
  \l 251BB
  \l 251BC
  \l 251BD
  \l 251BE
  \l 251BF
  \l 251C0
  \l 251C1
  \l 251C2
  \l 251C3
  \l 251C4
  \l 251C5
  \l 251C6
  \l 251C7
  \l 251C8
  \l 251C9
  \l 251CA
  \l 251CB
  \l 251CC
  \l 251CD
  \l 251CE
  \l 251CF
  \l 251D0
  \l 251D1
  \l 251D2
  \l 251D3
  \l 251D4
  \l 251D5
  \l 251D6
  \l 251D7
  \l 251D8
  \l 251D9
  \l 251DA
  \l 251DB
  \l 251DC
  \l 251DD
  \l 251DE
  \l 251DF
  \l 251E0
  \l 251E1
  \l 251E2
  \l 251E3
  \l 251E4
  \l 251E5
  \l 251E6
  \l 251E7
  \l 251E8
  \l 251E9
  \l 251EA
  \l 251EB
  \l 251EC
  \l 251ED
  \l 251EE
  \l 251EF
  \l 251F0
  \l 251F1
  \l 251F2
  \l 251F3
  \l 251F4
  \l 251F5
  \l 251F6
  \l 251F7
  \l 251F8
  \l 251F9
  \l 251FA
  \l 251FB
  \l 251FC
  \l 251FD
  \l 251FE
  \l 251FF
  \l 25200
  \l 25201
  \l 25202
  \l 25203
  \l 25204
  \l 25205
  \l 25206
  \l 25207
  \l 25208
  \l 25209
  \l 2520A
  \l 2520B
  \l 2520C
  \l 2520D
  \l 2520E
  \l 2520F
  \l 25210
  \l 25211
  \l 25212
  \l 25213
  \l 25214
  \l 25215
  \l 25216
  \l 25217
  \l 25218
  \l 25219
  \l 2521A
  \l 2521B
  \l 2521C
  \l 2521D
  \l 2521E
  \l 2521F
  \l 25220
  \l 25221
  \l 25222
  \l 25223
  \l 25224
  \l 25225
  \l 25226
  \l 25227
  \l 25228
  \l 25229
  \l 2522A
  \l 2522B
  \l 2522C
  \l 2522D
  \l 2522E
  \l 2522F
  \l 25230
  \l 25231
  \l 25232
  \l 25233
  \l 25234
  \l 25235
  \l 25236
  \l 25237
  \l 25238
  \l 25239
  \l 2523A
  \l 2523B
  \l 2523C
  \l 2523D
  \l 2523E
  \l 2523F
  \l 25240
  \l 25241
  \l 25242
  \l 25243
  \l 25244
  \l 25245
  \l 25246
  \l 25247
  \l 25248
  \l 25249
  \l 2524A
  \l 2524B
  \l 2524C
  \l 2524D
  \l 2524E
  \l 2524F
  \l 25250
  \l 25251
  \l 25252
  \l 25253
  \l 25254
  \l 25255
  \l 25256
  \l 25257
  \l 25258
  \l 25259
  \l 2525A
  \l 2525B
  \l 2525C
  \l 2525D
  \l 2525E
  \l 2525F
  \l 25260
  \l 25261
  \l 25262
  \l 25263
  \l 25264
  \l 25265
  \l 25266
  \l 25267
  \l 25268
  \l 25269
  \l 2526A
  \l 2526B
  \l 2526C
  \l 2526D
  \l 2526E
  \l 2526F
  \l 25270
  \l 25271
  \l 25272
  \l 25273
  \l 25274
  \l 25275
  \l 25276
  \l 25277
  \l 25278
  \l 25279
  \l 2527A
  \l 2527B
  \l 2527C
  \l 2527D
  \l 2527E
  \l 2527F
  \l 25280
  \l 25281
  \l 25282
  \l 25283
  \l 25284
  \l 25285
  \l 25286
  \l 25287
  \l 25288
  \l 25289
  \l 2528A
  \l 2528B
  \l 2528C
  \l 2528D
  \l 2528E
  \l 2528F
  \l 25290
  \l 25291
  \l 25292
  \l 25293
  \l 25294
  \l 25295
  \l 25296
  \l 25297
  \l 25298
  \l 25299
  \l 2529A
  \l 2529B
  \l 2529C
  \l 2529D
  \l 2529E
  \l 2529F
  \l 252A0
  \l 252A1
  \l 252A2
  \l 252A3
  \l 252A4
  \l 252A5
  \l 252A6
  \l 252A7
  \l 252A8
  \l 252A9
  \l 252AA
  \l 252AB
  \l 252AC
  \l 252AD
  \l 252AE
  \l 252AF
  \l 252B0
  \l 252B1
  \l 252B2
  \l 252B3
  \l 252B4
  \l 252B5
  \l 252B6
  \l 252B7
  \l 252B8
  \l 252B9
  \l 252BA
  \l 252BB
  \l 252BC
  \l 252BD
  \l 252BE
  \l 252BF
  \l 252C0
  \l 252C1
  \l 252C2
  \l 252C3
  \l 252C4
  \l 252C5
  \l 252C6
  \l 252C7
  \l 252C8
  \l 252C9
  \l 252CA
  \l 252CB
  \l 252CC
  \l 252CD
  \l 252CE
  \l 252CF
  \l 252D0
  \l 252D1
  \l 252D2
  \l 252D3
  \l 252D4
  \l 252D5
  \l 252D6
  \l 252D7
  \l 252D8
  \l 252D9
  \l 252DA
  \l 252DB
  \l 252DC
  \l 252DD
  \l 252DE
  \l 252DF
  \l 252E0
  \l 252E1
  \l 252E2
  \l 252E3
  \l 252E4
  \l 252E5
  \l 252E6
  \l 252E7
  \l 252E8
  \l 252E9
  \l 252EA
  \l 252EB
  \l 252EC
  \l 252ED
  \l 252EE
  \l 252EF
  \l 252F0
  \l 252F1
  \l 252F2
  \l 252F3
  \l 252F4
  \l 252F5
  \l 252F6
  \l 252F7
  \l 252F8
  \l 252F9
  \l 252FA
  \l 252FB
  \l 252FC
  \l 252FD
  \l 252FE
  \l 252FF
  \l 25300
  \l 25301
  \l 25302
  \l 25303
  \l 25304
  \l 25305
  \l 25306
  \l 25307
  \l 25308
  \l 25309
  \l 2530A
  \l 2530B
  \l 2530C
  \l 2530D
  \l 2530E
  \l 2530F
  \l 25310
  \l 25311
  \l 25312
  \l 25313
  \l 25314
  \l 25315
  \l 25316
  \l 25317
  \l 25318
  \l 25319
  \l 2531A
  \l 2531B
  \l 2531C
  \l 2531D
  \l 2531E
  \l 2531F
  \l 25320
  \l 25321
  \l 25322
  \l 25323
  \l 25324
  \l 25325
  \l 25326
  \l 25327
  \l 25328
  \l 25329
  \l 2532A
  \l 2532B
  \l 2532C
  \l 2532D
  \l 2532E
  \l 2532F
  \l 25330
  \l 25331
  \l 25332
  \l 25333
  \l 25334
  \l 25335
  \l 25336
  \l 25337
  \l 25338
  \l 25339
  \l 2533A
  \l 2533B
  \l 2533C
  \l 2533D
  \l 2533E
  \l 2533F
  \l 25340
  \l 25341
  \l 25342
  \l 25343
  \l 25344
  \l 25345
  \l 25346
  \l 25347
  \l 25348
  \l 25349
  \l 2534A
  \l 2534B
  \l 2534C
  \l 2534D
  \l 2534E
  \l 2534F
  \l 25350
  \l 25351
  \l 25352
  \l 25353
  \l 25354
  \l 25355
  \l 25356
  \l 25357
  \l 25358
  \l 25359
  \l 2535A
  \l 2535B
  \l 2535C
  \l 2535D
  \l 2535E
  \l 2535F
  \l 25360
  \l 25361
  \l 25362
  \l 25363
  \l 25364
  \l 25365
  \l 25366
  \l 25367
  \l 25368
  \l 25369
  \l 2536A
  \l 2536B
  \l 2536C
  \l 2536D
  \l 2536E
  \l 2536F
  \l 25370
  \l 25371
  \l 25372
  \l 25373
  \l 25374
  \l 25375
  \l 25376
  \l 25377
  \l 25378
  \l 25379
  \l 2537A
  \l 2537B
  \l 2537C
  \l 2537D
  \l 2537E
  \l 2537F
  \l 25380
  \l 25381
  \l 25382
  \l 25383
  \l 25384
  \l 25385
  \l 25386
  \l 25387
  \l 25388
  \l 25389
  \l 2538A
  \l 2538B
  \l 2538C
  \l 2538D
  \l 2538E
  \l 2538F
  \l 25390
  \l 25391
  \l 25392
  \l 25393
  \l 25394
  \l 25395
  \l 25396
  \l 25397
  \l 25398
  \l 25399
  \l 2539A
  \l 2539B
  \l 2539C
  \l 2539D
  \l 2539E
  \l 2539F
  \l 253A0
  \l 253A1
  \l 253A2
  \l 253A3
  \l 253A4
  \l 253A5
  \l 253A6
  \l 253A7
  \l 253A8
  \l 253A9
  \l 253AA
  \l 253AB
  \l 253AC
  \l 253AD
  \l 253AE
  \l 253AF
  \l 253B0
  \l 253B1
  \l 253B2
  \l 253B3
  \l 253B4
  \l 253B5
  \l 253B6
  \l 253B7
  \l 253B8
  \l 253B9
  \l 253BA
  \l 253BB
  \l 253BC
  \l 253BD
  \l 253BE
  \l 253BF
  \l 253C0
  \l 253C1
  \l 253C2
  \l 253C3
  \l 253C4
  \l 253C5
  \l 253C6
  \l 253C7
  \l 253C8
  \l 253C9
  \l 253CA
  \l 253CB
  \l 253CC
  \l 253CD
  \l 253CE
  \l 253CF
  \l 253D0
  \l 253D1
  \l 253D2
  \l 253D3
  \l 253D4
  \l 253D5
  \l 253D6
  \l 253D7
  \l 253D8
  \l 253D9
  \l 253DA
  \l 253DB
  \l 253DC
  \l 253DD
  \l 253DE
  \l 253DF
  \l 253E0
  \l 253E1
  \l 253E2
  \l 253E3
  \l 253E4
  \l 253E5
  \l 253E6
  \l 253E7
  \l 253E8
  \l 253E9
  \l 253EA
  \l 253EB
  \l 253EC
  \l 253ED
  \l 253EE
  \l 253EF
  \l 253F0
  \l 253F1
  \l 253F2
  \l 253F3
  \l 253F4
  \l 253F5
  \l 253F6
  \l 253F7
  \l 253F8
  \l 253F9
  \l 253FA
  \l 253FB
  \l 253FC
  \l 253FD
  \l 253FE
  \l 253FF
  \l 25400
  \l 25401
  \l 25402
  \l 25403
  \l 25404
  \l 25405
  \l 25406
  \l 25407
  \l 25408
  \l 25409
  \l 2540A
  \l 2540B
  \l 2540C
  \l 2540D
  \l 2540E
  \l 2540F
  \l 25410
  \l 25411
  \l 25412
  \l 25413
  \l 25414
  \l 25415
  \l 25416
  \l 25417
  \l 25418
  \l 25419
  \l 2541A
  \l 2541B
  \l 2541C
  \l 2541D
  \l 2541E
  \l 2541F
  \l 25420
  \l 25421
  \l 25422
  \l 25423
  \l 25424
  \l 25425
  \l 25426
  \l 25427
  \l 25428
  \l 25429
  \l 2542A
  \l 2542B
  \l 2542C
  \l 2542D
  \l 2542E
  \l 2542F
  \l 25430
  \l 25431
  \l 25432
  \l 25433
  \l 25434
  \l 25435
  \l 25436
  \l 25437
  \l 25438
  \l 25439
  \l 2543A
  \l 2543B
  \l 2543C
  \l 2543D
  \l 2543E
  \l 2543F
  \l 25440
  \l 25441
  \l 25442
  \l 25443
  \l 25444
  \l 25445
  \l 25446
  \l 25447
  \l 25448
  \l 25449
  \l 2544A
  \l 2544B
  \l 2544C
  \l 2544D
  \l 2544E
  \l 2544F
  \l 25450
  \l 25451
  \l 25452
  \l 25453
  \l 25454
  \l 25455
  \l 25456
  \l 25457
  \l 25458
  \l 25459
  \l 2545A
  \l 2545B
  \l 2545C
  \l 2545D
  \l 2545E
  \l 2545F
  \l 25460
  \l 25461
  \l 25462
  \l 25463
  \l 25464
  \l 25465
  \l 25466
  \l 25467
  \l 25468
  \l 25469
  \l 2546A
  \l 2546B
  \l 2546C
  \l 2546D
  \l 2546E
  \l 2546F
  \l 25470
  \l 25471
  \l 25472
  \l 25473
  \l 25474
  \l 25475
  \l 25476
  \l 25477
  \l 25478
  \l 25479
  \l 2547A
  \l 2547B
  \l 2547C
  \l 2547D
  \l 2547E
  \l 2547F
  \l 25480
  \l 25481
  \l 25482
  \l 25483
  \l 25484
  \l 25485
  \l 25486
  \l 25487
  \l 25488
  \l 25489
  \l 2548A
  \l 2548B
  \l 2548C
  \l 2548D
  \l 2548E
  \l 2548F
  \l 25490
  \l 25491
  \l 25492
  \l 25493
  \l 25494
  \l 25495
  \l 25496
  \l 25497
  \l 25498
  \l 25499
  \l 2549A
  \l 2549B
  \l 2549C
  \l 2549D
  \l 2549E
  \l 2549F
  \l 254A0
  \l 254A1
  \l 254A2
  \l 254A3
  \l 254A4
  \l 254A5
  \l 254A6
  \l 254A7
  \l 254A8
  \l 254A9
  \l 254AA
  \l 254AB
  \l 254AC
  \l 254AD
  \l 254AE
  \l 254AF
  \l 254B0
  \l 254B1
  \l 254B2
  \l 254B3
  \l 254B4
  \l 254B5
  \l 254B6
  \l 254B7
  \l 254B8
  \l 254B9
  \l 254BA
  \l 254BB
  \l 254BC
  \l 254BD
  \l 254BE
  \l 254BF
  \l 254C0
  \l 254C1
  \l 254C2
  \l 254C3
  \l 254C4
  \l 254C5
  \l 254C6
  \l 254C7
  \l 254C8
  \l 254C9
  \l 254CA
  \l 254CB
  \l 254CC
  \l 254CD
  \l 254CE
  \l 254CF
  \l 254D0
  \l 254D1
  \l 254D2
  \l 254D3
  \l 254D4
  \l 254D5
  \l 254D6
  \l 254D7
  \l 254D8
  \l 254D9
  \l 254DA
  \l 254DB
  \l 254DC
  \l 254DD
  \l 254DE
  \l 254DF
  \l 254E0
  \l 254E1
  \l 254E2
  \l 254E3
  \l 254E4
  \l 254E5
  \l 254E6
  \l 254E7
  \l 254E8
  \l 254E9
  \l 254EA
  \l 254EB
  \l 254EC
  \l 254ED
  \l 254EE
  \l 254EF
  \l 254F0
  \l 254F1
  \l 254F2
  \l 254F3
  \l 254F4
  \l 254F5
  \l 254F6
  \l 254F7
  \l 254F8
  \l 254F9
  \l 254FA
  \l 254FB
  \l 254FC
  \l 254FD
  \l 254FE
  \l 254FF
  \l 25500
  \l 25501
  \l 25502
  \l 25503
  \l 25504
  \l 25505
  \l 25506
  \l 25507
  \l 25508
  \l 25509
  \l 2550A
  \l 2550B
  \l 2550C
  \l 2550D
  \l 2550E
  \l 2550F
  \l 25510
  \l 25511
  \l 25512
  \l 25513
  \l 25514
  \l 25515
  \l 25516
  \l 25517
  \l 25518
  \l 25519
  \l 2551A
  \l 2551B
  \l 2551C
  \l 2551D
  \l 2551E
  \l 2551F
  \l 25520
  \l 25521
  \l 25522
  \l 25523
  \l 25524
  \l 25525
  \l 25526
  \l 25527
  \l 25528
  \l 25529
  \l 2552A
  \l 2552B
  \l 2552C
  \l 2552D
  \l 2552E
  \l 2552F
  \l 25530
  \l 25531
  \l 25532
  \l 25533
  \l 25534
  \l 25535
  \l 25536
  \l 25537
  \l 25538
  \l 25539
  \l 2553A
  \l 2553B
  \l 2553C
  \l 2553D
  \l 2553E
  \l 2553F
  \l 25540
  \l 25541
  \l 25542
  \l 25543
  \l 25544
  \l 25545
  \l 25546
  \l 25547
  \l 25548
  \l 25549
  \l 2554A
  \l 2554B
  \l 2554C
  \l 2554D
  \l 2554E
  \l 2554F
  \l 25550
  \l 25551
  \l 25552
  \l 25553
  \l 25554
  \l 25555
  \l 25556
  \l 25557
  \l 25558
  \l 25559
  \l 2555A
  \l 2555B
  \l 2555C
  \l 2555D
  \l 2555E
  \l 2555F
  \l 25560
  \l 25561
  \l 25562
  \l 25563
  \l 25564
  \l 25565
  \l 25566
  \l 25567
  \l 25568
  \l 25569
  \l 2556A
  \l 2556B
  \l 2556C
  \l 2556D
  \l 2556E
  \l 2556F
  \l 25570
  \l 25571
  \l 25572
  \l 25573
  \l 25574
  \l 25575
  \l 25576
  \l 25577
  \l 25578
  \l 25579
  \l 2557A
  \l 2557B
  \l 2557C
  \l 2557D
  \l 2557E
  \l 2557F
  \l 25580
  \l 25581
  \l 25582
  \l 25583
  \l 25584
  \l 25585
  \l 25586
  \l 25587
  \l 25588
  \l 25589
  \l 2558A
  \l 2558B
  \l 2558C
  \l 2558D
  \l 2558E
  \l 2558F
  \l 25590
  \l 25591
  \l 25592
  \l 25593
  \l 25594
  \l 25595
  \l 25596
  \l 25597
  \l 25598
  \l 25599
  \l 2559A
  \l 2559B
  \l 2559C
  \l 2559D
  \l 2559E
  \l 2559F
  \l 255A0
  \l 255A1
  \l 255A2
  \l 255A3
  \l 255A4
  \l 255A5
  \l 255A6
  \l 255A7
  \l 255A8
  \l 255A9
  \l 255AA
  \l 255AB
  \l 255AC
  \l 255AD
  \l 255AE
  \l 255AF
  \l 255B0
  \l 255B1
  \l 255B2
  \l 255B3
  \l 255B4
  \l 255B5
  \l 255B6
  \l 255B7
  \l 255B8
  \l 255B9
  \l 255BA
  \l 255BB
  \l 255BC
  \l 255BD
  \l 255BE
  \l 255BF
  \l 255C0
  \l 255C1
  \l 255C2
  \l 255C3
  \l 255C4
  \l 255C5
  \l 255C6
  \l 255C7
  \l 255C8
  \l 255C9
  \l 255CA
  \l 255CB
  \l 255CC
  \l 255CD
  \l 255CE
  \l 255CF
  \l 255D0
  \l 255D1
  \l 255D2
  \l 255D3
  \l 255D4
  \l 255D5
  \l 255D6
  \l 255D7
  \l 255D8
  \l 255D9
  \l 255DA
  \l 255DB
  \l 255DC
  \l 255DD
  \l 255DE
  \l 255DF
  \l 255E0
  \l 255E1
  \l 255E2
  \l 255E3
  \l 255E4
  \l 255E5
  \l 255E6
  \l 255E7
  \l 255E8
  \l 255E9
  \l 255EA
  \l 255EB
  \l 255EC
  \l 255ED
  \l 255EE
  \l 255EF
  \l 255F0
  \l 255F1
  \l 255F2
  \l 255F3
  \l 255F4
  \l 255F5
  \l 255F6
  \l 255F7
  \l 255F8
  \l 255F9
  \l 255FA
  \l 255FB
  \l 255FC
  \l 255FD
  \l 255FE
  \l 255FF
  \l 25600
  \l 25601
  \l 25602
  \l 25603
  \l 25604
  \l 25605
  \l 25606
  \l 25607
  \l 25608
  \l 25609
  \l 2560A
  \l 2560B
  \l 2560C
  \l 2560D
  \l 2560E
  \l 2560F
  \l 25610
  \l 25611
  \l 25612
  \l 25613
  \l 25614
  \l 25615
  \l 25616
  \l 25617
  \l 25618
  \l 25619
  \l 2561A
  \l 2561B
  \l 2561C
  \l 2561D
  \l 2561E
  \l 2561F
  \l 25620
  \l 25621
  \l 25622
  \l 25623
  \l 25624
  \l 25625
  \l 25626
  \l 25627
  \l 25628
  \l 25629
  \l 2562A
  \l 2562B
  \l 2562C
  \l 2562D
  \l 2562E
  \l 2562F
  \l 25630
  \l 25631
  \l 25632
  \l 25633
  \l 25634
  \l 25635
  \l 25636
  \l 25637
  \l 25638
  \l 25639
  \l 2563A
  \l 2563B
  \l 2563C
  \l 2563D
  \l 2563E
  \l 2563F
  \l 25640
  \l 25641
  \l 25642
  \l 25643
  \l 25644
  \l 25645
  \l 25646
  \l 25647
  \l 25648
  \l 25649
  \l 2564A
  \l 2564B
  \l 2564C
  \l 2564D
  \l 2564E
  \l 2564F
  \l 25650
  \l 25651
  \l 25652
  \l 25653
  \l 25654
  \l 25655
  \l 25656
  \l 25657
  \l 25658
  \l 25659
  \l 2565A
  \l 2565B
  \l 2565C
  \l 2565D
  \l 2565E
  \l 2565F
  \l 25660
  \l 25661
  \l 25662
  \l 25663
  \l 25664
  \l 25665
  \l 25666
  \l 25667
  \l 25668
  \l 25669
  \l 2566A
  \l 2566B
  \l 2566C
  \l 2566D
  \l 2566E
  \l 2566F
  \l 25670
  \l 25671
  \l 25672
  \l 25673
  \l 25674
  \l 25675
  \l 25676
  \l 25677
  \l 25678
  \l 25679
  \l 2567A
  \l 2567B
  \l 2567C
  \l 2567D
  \l 2567E
  \l 2567F
  \l 25680
  \l 25681
  \l 25682
  \l 25683
  \l 25684
  \l 25685
  \l 25686
  \l 25687
  \l 25688
  \l 25689
  \l 2568A
  \l 2568B
  \l 2568C
  \l 2568D
  \l 2568E
  \l 2568F
  \l 25690
  \l 25691
  \l 25692
  \l 25693
  \l 25694
  \l 25695
  \l 25696
  \l 25697
  \l 25698
  \l 25699
  \l 2569A
  \l 2569B
  \l 2569C
  \l 2569D
  \l 2569E
  \l 2569F
  \l 256A0
  \l 256A1
  \l 256A2
  \l 256A3
  \l 256A4
  \l 256A5
  \l 256A6
  \l 256A7
  \l 256A8
  \l 256A9
  \l 256AA
  \l 256AB
  \l 256AC
  \l 256AD
  \l 256AE
  \l 256AF
  \l 256B0
  \l 256B1
  \l 256B2
  \l 256B3
  \l 256B4
  \l 256B5
  \l 256B6
  \l 256B7
  \l 256B8
  \l 256B9
  \l 256BA
  \l 256BB
  \l 256BC
  \l 256BD
  \l 256BE
  \l 256BF
  \l 256C0
  \l 256C1
  \l 256C2
  \l 256C3
  \l 256C4
  \l 256C5
  \l 256C6
  \l 256C7
  \l 256C8
  \l 256C9
  \l 256CA
  \l 256CB
  \l 256CC
  \l 256CD
  \l 256CE
  \l 256CF
  \l 256D0
  \l 256D1
  \l 256D2
  \l 256D3
  \l 256D4
  \l 256D5
  \l 256D6
  \l 256D7
  \l 256D8
  \l 256D9
  \l 256DA
  \l 256DB
  \l 256DC
  \l 256DD
  \l 256DE
  \l 256DF
  \l 256E0
  \l 256E1
  \l 256E2
  \l 256E3
  \l 256E4
  \l 256E5
  \l 256E6
  \l 256E7
  \l 256E8
  \l 256E9
  \l 256EA
  \l 256EB
  \l 256EC
  \l 256ED
  \l 256EE
  \l 256EF
  \l 256F0
  \l 256F1
  \l 256F2
  \l 256F3
  \l 256F4
  \l 256F5
  \l 256F6
  \l 256F7
  \l 256F8
  \l 256F9
  \l 256FA
  \l 256FB
  \l 256FC
  \l 256FD
  \l 256FE
  \l 256FF
  \l 25700
  \l 25701
  \l 25702
  \l 25703
  \l 25704
  \l 25705
  \l 25706
  \l 25707
  \l 25708
  \l 25709
  \l 2570A
  \l 2570B
  \l 2570C
  \l 2570D
  \l 2570E
  \l 2570F
  \l 25710
  \l 25711
  \l 25712
  \l 25713
  \l 25714
  \l 25715
  \l 25716
  \l 25717
  \l 25718
  \l 25719
  \l 2571A
  \l 2571B
  \l 2571C
  \l 2571D
  \l 2571E
  \l 2571F
  \l 25720
  \l 25721
  \l 25722
  \l 25723
  \l 25724
  \l 25725
  \l 25726
  \l 25727
  \l 25728
  \l 25729
  \l 2572A
  \l 2572B
  \l 2572C
  \l 2572D
  \l 2572E
  \l 2572F
  \l 25730
  \l 25731
  \l 25732
  \l 25733
  \l 25734
  \l 25735
  \l 25736
  \l 25737
  \l 25738
  \l 25739
  \l 2573A
  \l 2573B
  \l 2573C
  \l 2573D
  \l 2573E
  \l 2573F
  \l 25740
  \l 25741
  \l 25742
  \l 25743
  \l 25744
  \l 25745
  \l 25746
  \l 25747
  \l 25748
  \l 25749
  \l 2574A
  \l 2574B
  \l 2574C
  \l 2574D
  \l 2574E
  \l 2574F
  \l 25750
  \l 25751
  \l 25752
  \l 25753
  \l 25754
  \l 25755
  \l 25756
  \l 25757
  \l 25758
  \l 25759
  \l 2575A
  \l 2575B
  \l 2575C
  \l 2575D
  \l 2575E
  \l 2575F
  \l 25760
  \l 25761
  \l 25762
  \l 25763
  \l 25764
  \l 25765
  \l 25766
  \l 25767
  \l 25768
  \l 25769
  \l 2576A
  \l 2576B
  \l 2576C
  \l 2576D
  \l 2576E
  \l 2576F
  \l 25770
  \l 25771
  \l 25772
  \l 25773
  \l 25774
  \l 25775
  \l 25776
  \l 25777
  \l 25778
  \l 25779
  \l 2577A
  \l 2577B
  \l 2577C
  \l 2577D
  \l 2577E
  \l 2577F
  \l 25780
  \l 25781
  \l 25782
  \l 25783
  \l 25784
  \l 25785
  \l 25786
  \l 25787
  \l 25788
  \l 25789
  \l 2578A
  \l 2578B
  \l 2578C
  \l 2578D
  \l 2578E
  \l 2578F
  \l 25790
  \l 25791
  \l 25792
  \l 25793
  \l 25794
  \l 25795
  \l 25796
  \l 25797
  \l 25798
  \l 25799
  \l 2579A
  \l 2579B
  \l 2579C
  \l 2579D
  \l 2579E
  \l 2579F
  \l 257A0
  \l 257A1
  \l 257A2
  \l 257A3
  \l 257A4
  \l 257A5
  \l 257A6
  \l 257A7
  \l 257A8
  \l 257A9
  \l 257AA
  \l 257AB
  \l 257AC
  \l 257AD
  \l 257AE
  \l 257AF
  \l 257B0
  \l 257B1
  \l 257B2
  \l 257B3
  \l 257B4
  \l 257B5
  \l 257B6
  \l 257B7
  \l 257B8
  \l 257B9
  \l 257BA
  \l 257BB
  \l 257BC
  \l 257BD
  \l 257BE
  \l 257BF
  \l 257C0
  \l 257C1
  \l 257C2
  \l 257C3
  \l 257C4
  \l 257C5
  \l 257C6
  \l 257C7
  \l 257C8
  \l 257C9
  \l 257CA
  \l 257CB
  \l 257CC
  \l 257CD
  \l 257CE
  \l 257CF
  \l 257D0
  \l 257D1
  \l 257D2
  \l 257D3
  \l 257D4
  \l 257D5
  \l 257D6
  \l 257D7
  \l 257D8
  \l 257D9
  \l 257DA
  \l 257DB
  \l 257DC
  \l 257DD
  \l 257DE
  \l 257DF
  \l 257E0
  \l 257E1
  \l 257E2
  \l 257E3
  \l 257E4
  \l 257E5
  \l 257E6
  \l 257E7
  \l 257E8
  \l 257E9
  \l 257EA
  \l 257EB
  \l 257EC
  \l 257ED
  \l 257EE
  \l 257EF
  \l 257F0
  \l 257F1
  \l 257F2
  \l 257F3
  \l 257F4
  \l 257F5
  \l 257F6
  \l 257F7
  \l 257F8
  \l 257F9
  \l 257FA
  \l 257FB
  \l 257FC
  \l 257FD
  \l 257FE
  \l 257FF
  \l 25800
  \l 25801
  \l 25802
  \l 25803
  \l 25804
  \l 25805
  \l 25806
  \l 25807
  \l 25808
  \l 25809
  \l 2580A
  \l 2580B
  \l 2580C
  \l 2580D
  \l 2580E
  \l 2580F
  \l 25810
  \l 25811
  \l 25812
  \l 25813
  \l 25814
  \l 25815
  \l 25816
  \l 25817
  \l 25818
  \l 25819
  \l 2581A
  \l 2581B
  \l 2581C
  \l 2581D
  \l 2581E
  \l 2581F
  \l 25820
  \l 25821
  \l 25822
  \l 25823
  \l 25824
  \l 25825
  \l 25826
  \l 25827
  \l 25828
  \l 25829
  \l 2582A
  \l 2582B
  \l 2582C
  \l 2582D
  \l 2582E
  \l 2582F
  \l 25830
  \l 25831
  \l 25832
  \l 25833
  \l 25834
  \l 25835
  \l 25836
  \l 25837
  \l 25838
  \l 25839
  \l 2583A
  \l 2583B
  \l 2583C
  \l 2583D
  \l 2583E
  \l 2583F
  \l 25840
  \l 25841
  \l 25842
  \l 25843
  \l 25844
  \l 25845
  \l 25846
  \l 25847
  \l 25848
  \l 25849
  \l 2584A
  \l 2584B
  \l 2584C
  \l 2584D
  \l 2584E
  \l 2584F
  \l 25850
  \l 25851
  \l 25852
  \l 25853
  \l 25854
  \l 25855
  \l 25856
  \l 25857
  \l 25858
  \l 25859
  \l 2585A
  \l 2585B
  \l 2585C
  \l 2585D
  \l 2585E
  \l 2585F
  \l 25860
  \l 25861
  \l 25862
  \l 25863
  \l 25864
  \l 25865
  \l 25866
  \l 25867
  \l 25868
  \l 25869
  \l 2586A
  \l 2586B
  \l 2586C
  \l 2586D
  \l 2586E
  \l 2586F
  \l 25870
  \l 25871
  \l 25872
  \l 25873
  \l 25874
  \l 25875
  \l 25876
  \l 25877
  \l 25878
  \l 25879
  \l 2587A
  \l 2587B
  \l 2587C
  \l 2587D
  \l 2587E
  \l 2587F
  \l 25880
  \l 25881
  \l 25882
  \l 25883
  \l 25884
  \l 25885
  \l 25886
  \l 25887
  \l 25888
  \l 25889
  \l 2588A
  \l 2588B
  \l 2588C
  \l 2588D
  \l 2588E
  \l 2588F
  \l 25890
  \l 25891
  \l 25892
  \l 25893
  \l 25894
  \l 25895
  \l 25896
  \l 25897
  \l 25898
  \l 25899
  \l 2589A
  \l 2589B
  \l 2589C
  \l 2589D
  \l 2589E
  \l 2589F
  \l 258A0
  \l 258A1
  \l 258A2
  \l 258A3
  \l 258A4
  \l 258A5
  \l 258A6
  \l 258A7
  \l 258A8
  \l 258A9
  \l 258AA
  \l 258AB
  \l 258AC
  \l 258AD
  \l 258AE
  \l 258AF
  \l 258B0
  \l 258B1
  \l 258B2
  \l 258B3
  \l 258B4
  \l 258B5
  \l 258B6
  \l 258B7
  \l 258B8
  \l 258B9
  \l 258BA
  \l 258BB
  \l 258BC
  \l 258BD
  \l 258BE
  \l 258BF
  \l 258C0
  \l 258C1
  \l 258C2
  \l 258C3
  \l 258C4
  \l 258C5
  \l 258C6
  \l 258C7
  \l 258C8
  \l 258C9
  \l 258CA
  \l 258CB
  \l 258CC
  \l 258CD
  \l 258CE
  \l 258CF
  \l 258D0
  \l 258D1
  \l 258D2
  \l 258D3
  \l 258D4
  \l 258D5
  \l 258D6
  \l 258D7
  \l 258D8
  \l 258D9
  \l 258DA
  \l 258DB
  \l 258DC
  \l 258DD
  \l 258DE
  \l 258DF
  \l 258E0
  \l 258E1
  \l 258E2
  \l 258E3
  \l 258E4
  \l 258E5
  \l 258E6
  \l 258E7
  \l 258E8
  \l 258E9
  \l 258EA
  \l 258EB
  \l 258EC
  \l 258ED
  \l 258EE
  \l 258EF
  \l 258F0
  \l 258F1
  \l 258F2
  \l 258F3
  \l 258F4
  \l 258F5
  \l 258F6
  \l 258F7
  \l 258F8
  \l 258F9
  \l 258FA
  \l 258FB
  \l 258FC
  \l 258FD
  \l 258FE
  \l 258FF
  \l 25900
  \l 25901
  \l 25902
  \l 25903
  \l 25904
  \l 25905
  \l 25906
  \l 25907
  \l 25908
  \l 25909
  \l 2590A
  \l 2590B
  \l 2590C
  \l 2590D
  \l 2590E
  \l 2590F
  \l 25910
  \l 25911
  \l 25912
  \l 25913
  \l 25914
  \l 25915
  \l 25916
  \l 25917
  \l 25918
  \l 25919
  \l 2591A
  \l 2591B
  \l 2591C
  \l 2591D
  \l 2591E
  \l 2591F
  \l 25920
  \l 25921
  \l 25922
  \l 25923
  \l 25924
  \l 25925
  \l 25926
  \l 25927
  \l 25928
  \l 25929
  \l 2592A
  \l 2592B
  \l 2592C
  \l 2592D
  \l 2592E
  \l 2592F
  \l 25930
  \l 25931
  \l 25932
  \l 25933
  \l 25934
  \l 25935
  \l 25936
  \l 25937
  \l 25938
  \l 25939
  \l 2593A
  \l 2593B
  \l 2593C
  \l 2593D
  \l 2593E
  \l 2593F
  \l 25940
  \l 25941
  \l 25942
  \l 25943
  \l 25944
  \l 25945
  \l 25946
  \l 25947
  \l 25948
  \l 25949
  \l 2594A
  \l 2594B
  \l 2594C
  \l 2594D
  \l 2594E
  \l 2594F
  \l 25950
  \l 25951
  \l 25952
  \l 25953
  \l 25954
  \l 25955
  \l 25956
  \l 25957
  \l 25958
  \l 25959
  \l 2595A
  \l 2595B
  \l 2595C
  \l 2595D
  \l 2595E
  \l 2595F
  \l 25960
  \l 25961
  \l 25962
  \l 25963
  \l 25964
  \l 25965
  \l 25966
  \l 25967
  \l 25968
  \l 25969
  \l 2596A
  \l 2596B
  \l 2596C
  \l 2596D
  \l 2596E
  \l 2596F
  \l 25970
  \l 25971
  \l 25972
  \l 25973
  \l 25974
  \l 25975
  \l 25976
  \l 25977
  \l 25978
  \l 25979
  \l 2597A
  \l 2597B
  \l 2597C
  \l 2597D
  \l 2597E
  \l 2597F
  \l 25980
  \l 25981
  \l 25982
  \l 25983
  \l 25984
  \l 25985
  \l 25986
  \l 25987
  \l 25988
  \l 25989
  \l 2598A
  \l 2598B
  \l 2598C
  \l 2598D
  \l 2598E
  \l 2598F
  \l 25990
  \l 25991
  \l 25992
  \l 25993
  \l 25994
  \l 25995
  \l 25996
  \l 25997
  \l 25998
  \l 25999
  \l 2599A
  \l 2599B
  \l 2599C
  \l 2599D
  \l 2599E
  \l 2599F
  \l 259A0
  \l 259A1
  \l 259A2
  \l 259A3
  \l 259A4
  \l 259A5
  \l 259A6
  \l 259A7
  \l 259A8
  \l 259A9
  \l 259AA
  \l 259AB
  \l 259AC
  \l 259AD
  \l 259AE
  \l 259AF
  \l 259B0
  \l 259B1
  \l 259B2
  \l 259B3
  \l 259B4
  \l 259B5
  \l 259B6
  \l 259B7
  \l 259B8
  \l 259B9
  \l 259BA
  \l 259BB
  \l 259BC
  \l 259BD
  \l 259BE
  \l 259BF
  \l 259C0
  \l 259C1
  \l 259C2
  \l 259C3
  \l 259C4
  \l 259C5
  \l 259C6
  \l 259C7
  \l 259C8
  \l 259C9
  \l 259CA
  \l 259CB
  \l 259CC
  \l 259CD
  \l 259CE
  \l 259CF
  \l 259D0
  \l 259D1
  \l 259D2
  \l 259D3
  \l 259D4
  \l 259D5
  \l 259D6
  \l 259D7
  \l 259D8
  \l 259D9
  \l 259DA
  \l 259DB
  \l 259DC
  \l 259DD
  \l 259DE
  \l 259DF
  \l 259E0
  \l 259E1
  \l 259E2
  \l 259E3
  \l 259E4
  \l 259E5
  \l 259E6
  \l 259E7
  \l 259E8
  \l 259E9
  \l 259EA
  \l 259EB
  \l 259EC
  \l 259ED
  \l 259EE
  \l 259EF
  \l 259F0
  \l 259F1
  \l 259F2
  \l 259F3
  \l 259F4
  \l 259F5
  \l 259F6
  \l 259F7
  \l 259F8
  \l 259F9
  \l 259FA
  \l 259FB
  \l 259FC
  \l 259FD
  \l 259FE
  \l 259FF
  \l 25A00
  \l 25A01
  \l 25A02
  \l 25A03
  \l 25A04
  \l 25A05
  \l 25A06
  \l 25A07
  \l 25A08
  \l 25A09
  \l 25A0A
  \l 25A0B
  \l 25A0C
  \l 25A0D
  \l 25A0E
  \l 25A0F
  \l 25A10
  \l 25A11
  \l 25A12
  \l 25A13
  \l 25A14
  \l 25A15
  \l 25A16
  \l 25A17
  \l 25A18
  \l 25A19
  \l 25A1A
  \l 25A1B
  \l 25A1C
  \l 25A1D
  \l 25A1E
  \l 25A1F
  \l 25A20
  \l 25A21
  \l 25A22
  \l 25A23
  \l 25A24
  \l 25A25
  \l 25A26
  \l 25A27
  \l 25A28
  \l 25A29
  \l 25A2A
  \l 25A2B
  \l 25A2C
  \l 25A2D
  \l 25A2E
  \l 25A2F
  \l 25A30
  \l 25A31
  \l 25A32
  \l 25A33
  \l 25A34
  \l 25A35
  \l 25A36
  \l 25A37
  \l 25A38
  \l 25A39
  \l 25A3A
  \l 25A3B
  \l 25A3C
  \l 25A3D
  \l 25A3E
  \l 25A3F
  \l 25A40
  \l 25A41
  \l 25A42
  \l 25A43
  \l 25A44
  \l 25A45
  \l 25A46
  \l 25A47
  \l 25A48
  \l 25A49
  \l 25A4A
  \l 25A4B
  \l 25A4C
  \l 25A4D
  \l 25A4E
  \l 25A4F
  \l 25A50
  \l 25A51
  \l 25A52
  \l 25A53
  \l 25A54
  \l 25A55
  \l 25A56
  \l 25A57
  \l 25A58
  \l 25A59
  \l 25A5A
  \l 25A5B
  \l 25A5C
  \l 25A5D
  \l 25A5E
  \l 25A5F
  \l 25A60
  \l 25A61
  \l 25A62
  \l 25A63
  \l 25A64
  \l 25A65
  \l 25A66
  \l 25A67
  \l 25A68
  \l 25A69
  \l 25A6A
  \l 25A6B
  \l 25A6C
  \l 25A6D
  \l 25A6E
  \l 25A6F
  \l 25A70
  \l 25A71
  \l 25A72
  \l 25A73
  \l 25A74
  \l 25A75
  \l 25A76
  \l 25A77
  \l 25A78
  \l 25A79
  \l 25A7A
  \l 25A7B
  \l 25A7C
  \l 25A7D
  \l 25A7E
  \l 25A7F
  \l 25A80
  \l 25A81
  \l 25A82
  \l 25A83
  \l 25A84
  \l 25A85
  \l 25A86
  \l 25A87
  \l 25A88
  \l 25A89
  \l 25A8A
  \l 25A8B
  \l 25A8C
  \l 25A8D
  \l 25A8E
  \l 25A8F
  \l 25A90
  \l 25A91
  \l 25A92
  \l 25A93
  \l 25A94
  \l 25A95
  \l 25A96
  \l 25A97
  \l 25A98
  \l 25A99
  \l 25A9A
  \l 25A9B
  \l 25A9C
  \l 25A9D
  \l 25A9E
  \l 25A9F
  \l 25AA0
  \l 25AA1
  \l 25AA2
  \l 25AA3
  \l 25AA4
  \l 25AA5
  \l 25AA6
  \l 25AA7
  \l 25AA8
  \l 25AA9
  \l 25AAA
  \l 25AAB
  \l 25AAC
  \l 25AAD
  \l 25AAE
  \l 25AAF
  \l 25AB0
  \l 25AB1
  \l 25AB2
  \l 25AB3
  \l 25AB4
  \l 25AB5
  \l 25AB6
  \l 25AB7
  \l 25AB8
  \l 25AB9
  \l 25ABA
  \l 25ABB
  \l 25ABC
  \l 25ABD
  \l 25ABE
  \l 25ABF
  \l 25AC0
  \l 25AC1
  \l 25AC2
  \l 25AC3
  \l 25AC4
  \l 25AC5
  \l 25AC6
  \l 25AC7
  \l 25AC8
  \l 25AC9
  \l 25ACA
  \l 25ACB
  \l 25ACC
  \l 25ACD
  \l 25ACE
  \l 25ACF
  \l 25AD0
  \l 25AD1
  \l 25AD2
  \l 25AD3
  \l 25AD4
  \l 25AD5
  \l 25AD6
  \l 25AD7
  \l 25AD8
  \l 25AD9
  \l 25ADA
  \l 25ADB
  \l 25ADC
  \l 25ADD
  \l 25ADE
  \l 25ADF
  \l 25AE0
  \l 25AE1
  \l 25AE2
  \l 25AE3
  \l 25AE4
  \l 25AE5
  \l 25AE6
  \l 25AE7
  \l 25AE8
  \l 25AE9
  \l 25AEA
  \l 25AEB
  \l 25AEC
  \l 25AED
  \l 25AEE
  \l 25AEF
  \l 25AF0
  \l 25AF1
  \l 25AF2
  \l 25AF3
  \l 25AF4
  \l 25AF5
  \l 25AF6
  \l 25AF7
  \l 25AF8
  \l 25AF9
  \l 25AFA
  \l 25AFB
  \l 25AFC
  \l 25AFD
  \l 25AFE
  \l 25AFF
  \l 25B00
  \l 25B01
  \l 25B02
  \l 25B03
  \l 25B04
  \l 25B05
  \l 25B06
  \l 25B07
  \l 25B08
  \l 25B09
  \l 25B0A
  \l 25B0B
  \l 25B0C
  \l 25B0D
  \l 25B0E
  \l 25B0F
  \l 25B10
  \l 25B11
  \l 25B12
  \l 25B13
  \l 25B14
  \l 25B15
  \l 25B16
  \l 25B17
  \l 25B18
  \l 25B19
  \l 25B1A
  \l 25B1B
  \l 25B1C
  \l 25B1D
  \l 25B1E
  \l 25B1F
  \l 25B20
  \l 25B21
  \l 25B22
  \l 25B23
  \l 25B24
  \l 25B25
  \l 25B26
  \l 25B27
  \l 25B28
  \l 25B29
  \l 25B2A
  \l 25B2B
  \l 25B2C
  \l 25B2D
  \l 25B2E
  \l 25B2F
  \l 25B30
  \l 25B31
  \l 25B32
  \l 25B33
  \l 25B34
  \l 25B35
  \l 25B36
  \l 25B37
  \l 25B38
  \l 25B39
  \l 25B3A
  \l 25B3B
  \l 25B3C
  \l 25B3D
  \l 25B3E
  \l 25B3F
  \l 25B40
  \l 25B41
  \l 25B42
  \l 25B43
  \l 25B44
  \l 25B45
  \l 25B46
  \l 25B47
  \l 25B48
  \l 25B49
  \l 25B4A
  \l 25B4B
  \l 25B4C
  \l 25B4D
  \l 25B4E
  \l 25B4F
  \l 25B50
  \l 25B51
  \l 25B52
  \l 25B53
  \l 25B54
  \l 25B55
  \l 25B56
  \l 25B57
  \l 25B58
  \l 25B59
  \l 25B5A
  \l 25B5B
  \l 25B5C
  \l 25B5D
  \l 25B5E
  \l 25B5F
  \l 25B60
  \l 25B61
  \l 25B62
  \l 25B63
  \l 25B64
  \l 25B65
  \l 25B66
  \l 25B67
  \l 25B68
  \l 25B69
  \l 25B6A
  \l 25B6B
  \l 25B6C
  \l 25B6D
  \l 25B6E
  \l 25B6F
  \l 25B70
  \l 25B71
  \l 25B72
  \l 25B73
  \l 25B74
  \l 25B75
  \l 25B76
  \l 25B77
  \l 25B78
  \l 25B79
  \l 25B7A
  \l 25B7B
  \l 25B7C
  \l 25B7D
  \l 25B7E
  \l 25B7F
  \l 25B80
  \l 25B81
  \l 25B82
  \l 25B83
  \l 25B84
  \l 25B85
  \l 25B86
  \l 25B87
  \l 25B88
  \l 25B89
  \l 25B8A
  \l 25B8B
  \l 25B8C
  \l 25B8D
  \l 25B8E
  \l 25B8F
  \l 25B90
  \l 25B91
  \l 25B92
  \l 25B93
  \l 25B94
  \l 25B95
  \l 25B96
  \l 25B97
  \l 25B98
  \l 25B99
  \l 25B9A
  \l 25B9B
  \l 25B9C
  \l 25B9D
  \l 25B9E
  \l 25B9F
  \l 25BA0
  \l 25BA1
  \l 25BA2
  \l 25BA3
  \l 25BA4
  \l 25BA5
  \l 25BA6
  \l 25BA7
  \l 25BA8
  \l 25BA9
  \l 25BAA
  \l 25BAB
  \l 25BAC
  \l 25BAD
  \l 25BAE
  \l 25BAF
  \l 25BB0
  \l 25BB1
  \l 25BB2
  \l 25BB3
  \l 25BB4
  \l 25BB5
  \l 25BB6
  \l 25BB7
  \l 25BB8
  \l 25BB9
  \l 25BBA
  \l 25BBB
  \l 25BBC
  \l 25BBD
  \l 25BBE
  \l 25BBF
  \l 25BC0
  \l 25BC1
  \l 25BC2
  \l 25BC3
  \l 25BC4
  \l 25BC5
  \l 25BC6
  \l 25BC7
  \l 25BC8
  \l 25BC9
  \l 25BCA
  \l 25BCB
  \l 25BCC
  \l 25BCD
  \l 25BCE
  \l 25BCF
  \l 25BD0
  \l 25BD1
  \l 25BD2
  \l 25BD3
  \l 25BD4
  \l 25BD5
  \l 25BD6
  \l 25BD7
  \l 25BD8
  \l 25BD9
  \l 25BDA
  \l 25BDB
  \l 25BDC
  \l 25BDD
  \l 25BDE
  \l 25BDF
  \l 25BE0
  \l 25BE1
  \l 25BE2
  \l 25BE3
  \l 25BE4
  \l 25BE5
  \l 25BE6
  \l 25BE7
  \l 25BE8
  \l 25BE9
  \l 25BEA
  \l 25BEB
  \l 25BEC
  \l 25BED
  \l 25BEE
  \l 25BEF
  \l 25BF0
  \l 25BF1
  \l 25BF2
  \l 25BF3
  \l 25BF4
  \l 25BF5
  \l 25BF6
  \l 25BF7
  \l 25BF8
  \l 25BF9
  \l 25BFA
  \l 25BFB
  \l 25BFC
  \l 25BFD
  \l 25BFE
  \l 25BFF
  \l 25C00
  \l 25C01
  \l 25C02
  \l 25C03
  \l 25C04
  \l 25C05
  \l 25C06
  \l 25C07
  \l 25C08
  \l 25C09
  \l 25C0A
  \l 25C0B
  \l 25C0C
  \l 25C0D
  \l 25C0E
  \l 25C0F
  \l 25C10
  \l 25C11
  \l 25C12
  \l 25C13
  \l 25C14
  \l 25C15
  \l 25C16
  \l 25C17
  \l 25C18
  \l 25C19
  \l 25C1A
  \l 25C1B
  \l 25C1C
  \l 25C1D
  \l 25C1E
  \l 25C1F
  \l 25C20
  \l 25C21
  \l 25C22
  \l 25C23
  \l 25C24
  \l 25C25
  \l 25C26
  \l 25C27
  \l 25C28
  \l 25C29
  \l 25C2A
  \l 25C2B
  \l 25C2C
  \l 25C2D
  \l 25C2E
  \l 25C2F
  \l 25C30
  \l 25C31
  \l 25C32
  \l 25C33
  \l 25C34
  \l 25C35
  \l 25C36
  \l 25C37
  \l 25C38
  \l 25C39
  \l 25C3A
  \l 25C3B
  \l 25C3C
  \l 25C3D
  \l 25C3E
  \l 25C3F
  \l 25C40
  \l 25C41
  \l 25C42
  \l 25C43
  \l 25C44
  \l 25C45
  \l 25C46
  \l 25C47
  \l 25C48
  \l 25C49
  \l 25C4A
  \l 25C4B
  \l 25C4C
  \l 25C4D
  \l 25C4E
  \l 25C4F
  \l 25C50
  \l 25C51
  \l 25C52
  \l 25C53
  \l 25C54
  \l 25C55
  \l 25C56
  \l 25C57
  \l 25C58
  \l 25C59
  \l 25C5A
  \l 25C5B
  \l 25C5C
  \l 25C5D
  \l 25C5E
  \l 25C5F
  \l 25C60
  \l 25C61
  \l 25C62
  \l 25C63
  \l 25C64
  \l 25C65
  \l 25C66
  \l 25C67
  \l 25C68
  \l 25C69
  \l 25C6A
  \l 25C6B
  \l 25C6C
  \l 25C6D
  \l 25C6E
  \l 25C6F
  \l 25C70
  \l 25C71
  \l 25C72
  \l 25C73
  \l 25C74
  \l 25C75
  \l 25C76
  \l 25C77
  \l 25C78
  \l 25C79
  \l 25C7A
  \l 25C7B
  \l 25C7C
  \l 25C7D
  \l 25C7E
  \l 25C7F
  \l 25C80
  \l 25C81
  \l 25C82
  \l 25C83
  \l 25C84
  \l 25C85
  \l 25C86
  \l 25C87
  \l 25C88
  \l 25C89
  \l 25C8A
  \l 25C8B
  \l 25C8C
  \l 25C8D
  \l 25C8E
  \l 25C8F
  \l 25C90
  \l 25C91
  \l 25C92
  \l 25C93
  \l 25C94
  \l 25C95
  \l 25C96
  \l 25C97
  \l 25C98
  \l 25C99
  \l 25C9A
  \l 25C9B
  \l 25C9C
  \l 25C9D
  \l 25C9E
  \l 25C9F
  \l 25CA0
  \l 25CA1
  \l 25CA2
  \l 25CA3
  \l 25CA4
  \l 25CA5
  \l 25CA6
  \l 25CA7
  \l 25CA8
  \l 25CA9
  \l 25CAA
  \l 25CAB
  \l 25CAC
  \l 25CAD
  \l 25CAE
  \l 25CAF
  \l 25CB0
  \l 25CB1
  \l 25CB2
  \l 25CB3
  \l 25CB4
  \l 25CB5
  \l 25CB6
  \l 25CB7
  \l 25CB8
  \l 25CB9
  \l 25CBA
  \l 25CBB
  \l 25CBC
  \l 25CBD
  \l 25CBE
  \l 25CBF
  \l 25CC0
  \l 25CC1
  \l 25CC2
  \l 25CC3
  \l 25CC4
  \l 25CC5
  \l 25CC6
  \l 25CC7
  \l 25CC8
  \l 25CC9
  \l 25CCA
  \l 25CCB
  \l 25CCC
  \l 25CCD
  \l 25CCE
  \l 25CCF
  \l 25CD0
  \l 25CD1
  \l 25CD2
  \l 25CD3
  \l 25CD4
  \l 25CD5
  \l 25CD6
  \l 25CD7
  \l 25CD8
  \l 25CD9
  \l 25CDA
  \l 25CDB
  \l 25CDC
  \l 25CDD
  \l 25CDE
  \l 25CDF
  \l 25CE0
  \l 25CE1
  \l 25CE2
  \l 25CE3
  \l 25CE4
  \l 25CE5
  \l 25CE6
  \l 25CE7
  \l 25CE8
  \l 25CE9
  \l 25CEA
  \l 25CEB
  \l 25CEC
  \l 25CED
  \l 25CEE
  \l 25CEF
  \l 25CF0
  \l 25CF1
  \l 25CF2
  \l 25CF3
  \l 25CF4
  \l 25CF5
  \l 25CF6
  \l 25CF7
  \l 25CF8
  \l 25CF9
  \l 25CFA
  \l 25CFB
  \l 25CFC
  \l 25CFD
  \l 25CFE
  \l 25CFF
  \l 25D00
  \l 25D01
  \l 25D02
  \l 25D03
  \l 25D04
  \l 25D05
  \l 25D06
  \l 25D07
  \l 25D08
  \l 25D09
  \l 25D0A
  \l 25D0B
  \l 25D0C
  \l 25D0D
  \l 25D0E
  \l 25D0F
  \l 25D10
  \l 25D11
  \l 25D12
  \l 25D13
  \l 25D14
  \l 25D15
  \l 25D16
  \l 25D17
  \l 25D18
  \l 25D19
  \l 25D1A
  \l 25D1B
  \l 25D1C
  \l 25D1D
  \l 25D1E
  \l 25D1F
  \l 25D20
  \l 25D21
  \l 25D22
  \l 25D23
  \l 25D24
  \l 25D25
  \l 25D26
  \l 25D27
  \l 25D28
  \l 25D29
  \l 25D2A
  \l 25D2B
  \l 25D2C
  \l 25D2D
  \l 25D2E
  \l 25D2F
  \l 25D30
  \l 25D31
  \l 25D32
  \l 25D33
  \l 25D34
  \l 25D35
  \l 25D36
  \l 25D37
  \l 25D38
  \l 25D39
  \l 25D3A
  \l 25D3B
  \l 25D3C
  \l 25D3D
  \l 25D3E
  \l 25D3F
  \l 25D40
  \l 25D41
  \l 25D42
  \l 25D43
  \l 25D44
  \l 25D45
  \l 25D46
  \l 25D47
  \l 25D48
  \l 25D49
  \l 25D4A
  \l 25D4B
  \l 25D4C
  \l 25D4D
  \l 25D4E
  \l 25D4F
  \l 25D50
  \l 25D51
  \l 25D52
  \l 25D53
  \l 25D54
  \l 25D55
  \l 25D56
  \l 25D57
  \l 25D58
  \l 25D59
  \l 25D5A
  \l 25D5B
  \l 25D5C
  \l 25D5D
  \l 25D5E
  \l 25D5F
  \l 25D60
  \l 25D61
  \l 25D62
  \l 25D63
  \l 25D64
  \l 25D65
  \l 25D66
  \l 25D67
  \l 25D68
  \l 25D69
  \l 25D6A
  \l 25D6B
  \l 25D6C
  \l 25D6D
  \l 25D6E
  \l 25D6F
  \l 25D70
  \l 25D71
  \l 25D72
  \l 25D73
  \l 25D74
  \l 25D75
  \l 25D76
  \l 25D77
  \l 25D78
  \l 25D79
  \l 25D7A
  \l 25D7B
  \l 25D7C
  \l 25D7D
  \l 25D7E
  \l 25D7F
  \l 25D80
  \l 25D81
  \l 25D82
  \l 25D83
  \l 25D84
  \l 25D85
  \l 25D86
  \l 25D87
  \l 25D88
  \l 25D89
  \l 25D8A
  \l 25D8B
  \l 25D8C
  \l 25D8D
  \l 25D8E
  \l 25D8F
  \l 25D90
  \l 25D91
  \l 25D92
  \l 25D93
  \l 25D94
  \l 25D95
  \l 25D96
  \l 25D97
  \l 25D98
  \l 25D99
  \l 25D9A
  \l 25D9B
  \l 25D9C
  \l 25D9D
  \l 25D9E
  \l 25D9F
  \l 25DA0
  \l 25DA1
  \l 25DA2
  \l 25DA3
  \l 25DA4
  \l 25DA5
  \l 25DA6
  \l 25DA7
  \l 25DA8
  \l 25DA9
  \l 25DAA
  \l 25DAB
  \l 25DAC
  \l 25DAD
  \l 25DAE
  \l 25DAF
  \l 25DB0
  \l 25DB1
  \l 25DB2
  \l 25DB3
  \l 25DB4
  \l 25DB5
  \l 25DB6
  \l 25DB7
  \l 25DB8
  \l 25DB9
  \l 25DBA
  \l 25DBB
  \l 25DBC
  \l 25DBD
  \l 25DBE
  \l 25DBF
  \l 25DC0
  \l 25DC1
  \l 25DC2
  \l 25DC3
  \l 25DC4
  \l 25DC5
  \l 25DC6
  \l 25DC7
  \l 25DC8
  \l 25DC9
  \l 25DCA
  \l 25DCB
  \l 25DCC
  \l 25DCD
  \l 25DCE
  \l 25DCF
  \l 25DD0
  \l 25DD1
  \l 25DD2
  \l 25DD3
  \l 25DD4
  \l 25DD5
  \l 25DD6
  \l 25DD7
  \l 25DD8
  \l 25DD9
  \l 25DDA
  \l 25DDB
  \l 25DDC
  \l 25DDD
  \l 25DDE
  \l 25DDF
  \l 25DE0
  \l 25DE1
  \l 25DE2
  \l 25DE3
  \l 25DE4
  \l 25DE5
  \l 25DE6
  \l 25DE7
  \l 25DE8
  \l 25DE9
  \l 25DEA
  \l 25DEB
  \l 25DEC
  \l 25DED
  \l 25DEE
  \l 25DEF
  \l 25DF0
  \l 25DF1
  \l 25DF2
  \l 25DF3
  \l 25DF4
  \l 25DF5
  \l 25DF6
  \l 25DF7
  \l 25DF8
  \l 25DF9
  \l 25DFA
  \l 25DFB
  \l 25DFC
  \l 25DFD
  \l 25DFE
  \l 25DFF
  \l 25E00
  \l 25E01
  \l 25E02
  \l 25E03
  \l 25E04
  \l 25E05
  \l 25E06
  \l 25E07
  \l 25E08
  \l 25E09
  \l 25E0A
  \l 25E0B
  \l 25E0C
  \l 25E0D
  \l 25E0E
  \l 25E0F
  \l 25E10
  \l 25E11
  \l 25E12
  \l 25E13
  \l 25E14
  \l 25E15
  \l 25E16
  \l 25E17
  \l 25E18
  \l 25E19
  \l 25E1A
  \l 25E1B
  \l 25E1C
  \l 25E1D
  \l 25E1E
  \l 25E1F
  \l 25E20
  \l 25E21
  \l 25E22
  \l 25E23
  \l 25E24
  \l 25E25
  \l 25E26
  \l 25E27
  \l 25E28
  \l 25E29
  \l 25E2A
  \l 25E2B
  \l 25E2C
  \l 25E2D
  \l 25E2E
  \l 25E2F
  \l 25E30
  \l 25E31
  \l 25E32
  \l 25E33
  \l 25E34
  \l 25E35
  \l 25E36
  \l 25E37
  \l 25E38
  \l 25E39
  \l 25E3A
  \l 25E3B
  \l 25E3C
  \l 25E3D
  \l 25E3E
  \l 25E3F
  \l 25E40
  \l 25E41
  \l 25E42
  \l 25E43
  \l 25E44
  \l 25E45
  \l 25E46
  \l 25E47
  \l 25E48
  \l 25E49
  \l 25E4A
  \l 25E4B
  \l 25E4C
  \l 25E4D
  \l 25E4E
  \l 25E4F
  \l 25E50
  \l 25E51
  \l 25E52
  \l 25E53
  \l 25E54
  \l 25E55
  \l 25E56
  \l 25E57
  \l 25E58
  \l 25E59
  \l 25E5A
  \l 25E5B
  \l 25E5C
  \l 25E5D
  \l 25E5E
  \l 25E5F
  \l 25E60
  \l 25E61
  \l 25E62
  \l 25E63
  \l 25E64
  \l 25E65
  \l 25E66
  \l 25E67
  \l 25E68
  \l 25E69
  \l 25E6A
  \l 25E6B
  \l 25E6C
  \l 25E6D
  \l 25E6E
  \l 25E6F
  \l 25E70
  \l 25E71
  \l 25E72
  \l 25E73
  \l 25E74
  \l 25E75
  \l 25E76
  \l 25E77
  \l 25E78
  \l 25E79
  \l 25E7A
  \l 25E7B
  \l 25E7C
  \l 25E7D
  \l 25E7E
  \l 25E7F
  \l 25E80
  \l 25E81
  \l 25E82
  \l 25E83
  \l 25E84
  \l 25E85
  \l 25E86
  \l 25E87
  \l 25E88
  \l 25E89
  \l 25E8A
  \l 25E8B
  \l 25E8C
  \l 25E8D
  \l 25E8E
  \l 25E8F
  \l 25E90
  \l 25E91
  \l 25E92
  \l 25E93
  \l 25E94
  \l 25E95
  \l 25E96
  \l 25E97
  \l 25E98
  \l 25E99
  \l 25E9A
  \l 25E9B
  \l 25E9C
  \l 25E9D
  \l 25E9E
  \l 25E9F
  \l 25EA0
  \l 25EA1
  \l 25EA2
  \l 25EA3
  \l 25EA4
  \l 25EA5
  \l 25EA6
  \l 25EA7
  \l 25EA8
  \l 25EA9
  \l 25EAA
  \l 25EAB
  \l 25EAC
  \l 25EAD
  \l 25EAE
  \l 25EAF
  \l 25EB0
  \l 25EB1
  \l 25EB2
  \l 25EB3
  \l 25EB4
  \l 25EB5
  \l 25EB6
  \l 25EB7
  \l 25EB8
  \l 25EB9
  \l 25EBA
  \l 25EBB
  \l 25EBC
  \l 25EBD
  \l 25EBE
  \l 25EBF
  \l 25EC0
  \l 25EC1
  \l 25EC2
  \l 25EC3
  \l 25EC4
  \l 25EC5
  \l 25EC6
  \l 25EC7
  \l 25EC8
  \l 25EC9
  \l 25ECA
  \l 25ECB
  \l 25ECC
  \l 25ECD
  \l 25ECE
  \l 25ECF
  \l 25ED0
  \l 25ED1
  \l 25ED2
  \l 25ED3
  \l 25ED4
  \l 25ED5
  \l 25ED6
  \l 25ED7
  \l 25ED8
  \l 25ED9
  \l 25EDA
  \l 25EDB
  \l 25EDC
  \l 25EDD
  \l 25EDE
  \l 25EDF
  \l 25EE0
  \l 25EE1
  \l 25EE2
  \l 25EE3
  \l 25EE4
  \l 25EE5
  \l 25EE6
  \l 25EE7
  \l 25EE8
  \l 25EE9
  \l 25EEA
  \l 25EEB
  \l 25EEC
  \l 25EED
  \l 25EEE
  \l 25EEF
  \l 25EF0
  \l 25EF1
  \l 25EF2
  \l 25EF3
  \l 25EF4
  \l 25EF5
  \l 25EF6
  \l 25EF7
  \l 25EF8
  \l 25EF9
  \l 25EFA
  \l 25EFB
  \l 25EFC
  \l 25EFD
  \l 25EFE
  \l 25EFF
  \l 25F00
  \l 25F01
  \l 25F02
  \l 25F03
  \l 25F04
  \l 25F05
  \l 25F06
  \l 25F07
  \l 25F08
  \l 25F09
  \l 25F0A
  \l 25F0B
  \l 25F0C
  \l 25F0D
  \l 25F0E
  \l 25F0F
  \l 25F10
  \l 25F11
  \l 25F12
  \l 25F13
  \l 25F14
  \l 25F15
  \l 25F16
  \l 25F17
  \l 25F18
  \l 25F19
  \l 25F1A
  \l 25F1B
  \l 25F1C
  \l 25F1D
  \l 25F1E
  \l 25F1F
  \l 25F20
  \l 25F21
  \l 25F22
  \l 25F23
  \l 25F24
  \l 25F25
  \l 25F26
  \l 25F27
  \l 25F28
  \l 25F29
  \l 25F2A
  \l 25F2B
  \l 25F2C
  \l 25F2D
  \l 25F2E
  \l 25F2F
  \l 25F30
  \l 25F31
  \l 25F32
  \l 25F33
  \l 25F34
  \l 25F35
  \l 25F36
  \l 25F37
  \l 25F38
  \l 25F39
  \l 25F3A
  \l 25F3B
  \l 25F3C
  \l 25F3D
  \l 25F3E
  \l 25F3F
  \l 25F40
  \l 25F41
  \l 25F42
  \l 25F43
  \l 25F44
  \l 25F45
  \l 25F46
  \l 25F47
  \l 25F48
  \l 25F49
  \l 25F4A
  \l 25F4B
  \l 25F4C
  \l 25F4D
  \l 25F4E
  \l 25F4F
  \l 25F50
  \l 25F51
  \l 25F52
  \l 25F53
  \l 25F54
  \l 25F55
  \l 25F56
  \l 25F57
  \l 25F58
  \l 25F59
  \l 25F5A
  \l 25F5B
  \l 25F5C
  \l 25F5D
  \l 25F5E
  \l 25F5F
  \l 25F60
  \l 25F61
  \l 25F62
  \l 25F63
  \l 25F64
  \l 25F65
  \l 25F66
  \l 25F67
  \l 25F68
  \l 25F69
  \l 25F6A
  \l 25F6B
  \l 25F6C
  \l 25F6D
  \l 25F6E
  \l 25F6F
  \l 25F70
  \l 25F71
  \l 25F72
  \l 25F73
  \l 25F74
  \l 25F75
  \l 25F76
  \l 25F77
  \l 25F78
  \l 25F79
  \l 25F7A
  \l 25F7B
  \l 25F7C
  \l 25F7D
  \l 25F7E
  \l 25F7F
  \l 25F80
  \l 25F81
  \l 25F82
  \l 25F83
  \l 25F84
  \l 25F85
  \l 25F86
  \l 25F87
  \l 25F88
  \l 25F89
  \l 25F8A
  \l 25F8B
  \l 25F8C
  \l 25F8D
  \l 25F8E
  \l 25F8F
  \l 25F90
  \l 25F91
  \l 25F92
  \l 25F93
  \l 25F94
  \l 25F95
  \l 25F96
  \l 25F97
  \l 25F98
  \l 25F99
  \l 25F9A
  \l 25F9B
  \l 25F9C
  \l 25F9D
  \l 25F9E
  \l 25F9F
  \l 25FA0
  \l 25FA1
  \l 25FA2
  \l 25FA3
  \l 25FA4
  \l 25FA5
  \l 25FA6
  \l 25FA7
  \l 25FA8
  \l 25FA9
  \l 25FAA
  \l 25FAB
  \l 25FAC
  \l 25FAD
  \l 25FAE
  \l 25FAF
  \l 25FB0
  \l 25FB1
  \l 25FB2
  \l 25FB3
  \l 25FB4
  \l 25FB5
  \l 25FB6
  \l 25FB7
  \l 25FB8
  \l 25FB9
  \l 25FBA
  \l 25FBB
  \l 25FBC
  \l 25FBD
  \l 25FBE
  \l 25FBF
  \l 25FC0
  \l 25FC1
  \l 25FC2
  \l 25FC3
  \l 25FC4
  \l 25FC5
  \l 25FC6
  \l 25FC7
  \l 25FC8
  \l 25FC9
  \l 25FCA
  \l 25FCB
  \l 25FCC
  \l 25FCD
  \l 25FCE
  \l 25FCF
  \l 25FD0
  \l 25FD1
  \l 25FD2
  \l 25FD3
  \l 25FD4
  \l 25FD5
  \l 25FD6
  \l 25FD7
  \l 25FD8
  \l 25FD9
  \l 25FDA
  \l 25FDB
  \l 25FDC
  \l 25FDD
  \l 25FDE
  \l 25FDF
  \l 25FE0
  \l 25FE1
  \l 25FE2
  \l 25FE3
  \l 25FE4
  \l 25FE5
  \l 25FE6
  \l 25FE7
  \l 25FE8
  \l 25FE9
  \l 25FEA
  \l 25FEB
  \l 25FEC
  \l 25FED
  \l 25FEE
  \l 25FEF
  \l 25FF0
  \l 25FF1
  \l 25FF2
  \l 25FF3
  \l 25FF4
  \l 25FF5
  \l 25FF6
  \l 25FF7
  \l 25FF8
  \l 25FF9
  \l 25FFA
  \l 25FFB
  \l 25FFC
  \l 25FFD
  \l 25FFE
  \l 25FFF
  \l 26000
  \l 26001
  \l 26002
  \l 26003
  \l 26004
  \l 26005
  \l 26006
  \l 26007
  \l 26008
  \l 26009
  \l 2600A
  \l 2600B
  \l 2600C
  \l 2600D
  \l 2600E
  \l 2600F
  \l 26010
  \l 26011
  \l 26012
  \l 26013
  \l 26014
  \l 26015
  \l 26016
  \l 26017
  \l 26018
  \l 26019
  \l 2601A
  \l 2601B
  \l 2601C
  \l 2601D
  \l 2601E
  \l 2601F
  \l 26020
  \l 26021
  \l 26022
  \l 26023
  \l 26024
  \l 26025
  \l 26026
  \l 26027
  \l 26028
  \l 26029
  \l 2602A
  \l 2602B
  \l 2602C
  \l 2602D
  \l 2602E
  \l 2602F
  \l 26030
  \l 26031
  \l 26032
  \l 26033
  \l 26034
  \l 26035
  \l 26036
  \l 26037
  \l 26038
  \l 26039
  \l 2603A
  \l 2603B
  \l 2603C
  \l 2603D
  \l 2603E
  \l 2603F
  \l 26040
  \l 26041
  \l 26042
  \l 26043
  \l 26044
  \l 26045
  \l 26046
  \l 26047
  \l 26048
  \l 26049
  \l 2604A
  \l 2604B
  \l 2604C
  \l 2604D
  \l 2604E
  \l 2604F
  \l 26050
  \l 26051
  \l 26052
  \l 26053
  \l 26054
  \l 26055
  \l 26056
  \l 26057
  \l 26058
  \l 26059
  \l 2605A
  \l 2605B
  \l 2605C
  \l 2605D
  \l 2605E
  \l 2605F
  \l 26060
  \l 26061
  \l 26062
  \l 26063
  \l 26064
  \l 26065
  \l 26066
  \l 26067
  \l 26068
  \l 26069
  \l 2606A
  \l 2606B
  \l 2606C
  \l 2606D
  \l 2606E
  \l 2606F
  \l 26070
  \l 26071
  \l 26072
  \l 26073
  \l 26074
  \l 26075
  \l 26076
  \l 26077
  \l 26078
  \l 26079
  \l 2607A
  \l 2607B
  \l 2607C
  \l 2607D
  \l 2607E
  \l 2607F
  \l 26080
  \l 26081
  \l 26082
  \l 26083
  \l 26084
  \l 26085
  \l 26086
  \l 26087
  \l 26088
  \l 26089
  \l 2608A
  \l 2608B
  \l 2608C
  \l 2608D
  \l 2608E
  \l 2608F
  \l 26090
  \l 26091
  \l 26092
  \l 26093
  \l 26094
  \l 26095
  \l 26096
  \l 26097
  \l 26098
  \l 26099
  \l 2609A
  \l 2609B
  \l 2609C
  \l 2609D
  \l 2609E
  \l 2609F
  \l 260A0
  \l 260A1
  \l 260A2
  \l 260A3
  \l 260A4
  \l 260A5
  \l 260A6
  \l 260A7
  \l 260A8
  \l 260A9
  \l 260AA
  \l 260AB
  \l 260AC
  \l 260AD
  \l 260AE
  \l 260AF
  \l 260B0
  \l 260B1
  \l 260B2
  \l 260B3
  \l 260B4
  \l 260B5
  \l 260B6
  \l 260B7
  \l 260B8
  \l 260B9
  \l 260BA
  \l 260BB
  \l 260BC
  \l 260BD
  \l 260BE
  \l 260BF
  \l 260C0
  \l 260C1
  \l 260C2
  \l 260C3
  \l 260C4
  \l 260C5
  \l 260C6
  \l 260C7
  \l 260C8
  \l 260C9
  \l 260CA
  \l 260CB
  \l 260CC
  \l 260CD
  \l 260CE
  \l 260CF
  \l 260D0
  \l 260D1
  \l 260D2
  \l 260D3
  \l 260D4
  \l 260D5
  \l 260D6
  \l 260D7
  \l 260D8
  \l 260D9
  \l 260DA
  \l 260DB
  \l 260DC
  \l 260DD
  \l 260DE
  \l 260DF
  \l 260E0
  \l 260E1
  \l 260E2
  \l 260E3
  \l 260E4
  \l 260E5
  \l 260E6
  \l 260E7
  \l 260E8
  \l 260E9
  \l 260EA
  \l 260EB
  \l 260EC
  \l 260ED
  \l 260EE
  \l 260EF
  \l 260F0
  \l 260F1
  \l 260F2
  \l 260F3
  \l 260F4
  \l 260F5
  \l 260F6
  \l 260F7
  \l 260F8
  \l 260F9
  \l 260FA
  \l 260FB
  \l 260FC
  \l 260FD
  \l 260FE
  \l 260FF
  \l 26100
  \l 26101
  \l 26102
  \l 26103
  \l 26104
  \l 26105
  \l 26106
  \l 26107
  \l 26108
  \l 26109
  \l 2610A
  \l 2610B
  \l 2610C
  \l 2610D
  \l 2610E
  \l 2610F
  \l 26110
  \l 26111
  \l 26112
  \l 26113
  \l 26114
  \l 26115
  \l 26116
  \l 26117
  \l 26118
  \l 26119
  \l 2611A
  \l 2611B
  \l 2611C
  \l 2611D
  \l 2611E
  \l 2611F
  \l 26120
  \l 26121
  \l 26122
  \l 26123
  \l 26124
  \l 26125
  \l 26126
  \l 26127
  \l 26128
  \l 26129
  \l 2612A
  \l 2612B
  \l 2612C
  \l 2612D
  \l 2612E
  \l 2612F
  \l 26130
  \l 26131
  \l 26132
  \l 26133
  \l 26134
  \l 26135
  \l 26136
  \l 26137
  \l 26138
  \l 26139
  \l 2613A
  \l 2613B
  \l 2613C
  \l 2613D
  \l 2613E
  \l 2613F
  \l 26140
  \l 26141
  \l 26142
  \l 26143
  \l 26144
  \l 26145
  \l 26146
  \l 26147
  \l 26148
  \l 26149
  \l 2614A
  \l 2614B
  \l 2614C
  \l 2614D
  \l 2614E
  \l 2614F
  \l 26150
  \l 26151
  \l 26152
  \l 26153
  \l 26154
  \l 26155
  \l 26156
  \l 26157
  \l 26158
  \l 26159
  \l 2615A
  \l 2615B
  \l 2615C
  \l 2615D
  \l 2615E
  \l 2615F
  \l 26160
  \l 26161
  \l 26162
  \l 26163
  \l 26164
  \l 26165
  \l 26166
  \l 26167
  \l 26168
  \l 26169
  \l 2616A
  \l 2616B
  \l 2616C
  \l 2616D
  \l 2616E
  \l 2616F
  \l 26170
  \l 26171
  \l 26172
  \l 26173
  \l 26174
  \l 26175
  \l 26176
  \l 26177
  \l 26178
  \l 26179
  \l 2617A
  \l 2617B
  \l 2617C
  \l 2617D
  \l 2617E
  \l 2617F
  \l 26180
  \l 26181
  \l 26182
  \l 26183
  \l 26184
  \l 26185
  \l 26186
  \l 26187
  \l 26188
  \l 26189
  \l 2618A
  \l 2618B
  \l 2618C
  \l 2618D
  \l 2618E
  \l 2618F
  \l 26190
  \l 26191
  \l 26192
  \l 26193
  \l 26194
  \l 26195
  \l 26196
  \l 26197
  \l 26198
  \l 26199
  \l 2619A
  \l 2619B
  \l 2619C
  \l 2619D
  \l 2619E
  \l 2619F
  \l 261A0
  \l 261A1
  \l 261A2
  \l 261A3
  \l 261A4
  \l 261A5
  \l 261A6
  \l 261A7
  \l 261A8
  \l 261A9
  \l 261AA
  \l 261AB
  \l 261AC
  \l 261AD
  \l 261AE
  \l 261AF
  \l 261B0
  \l 261B1
  \l 261B2
  \l 261B3
  \l 261B4
  \l 261B5
  \l 261B6
  \l 261B7
  \l 261B8
  \l 261B9
  \l 261BA
  \l 261BB
  \l 261BC
  \l 261BD
  \l 261BE
  \l 261BF
  \l 261C0
  \l 261C1
  \l 261C2
  \l 261C3
  \l 261C4
  \l 261C5
  \l 261C6
  \l 261C7
  \l 261C8
  \l 261C9
  \l 261CA
  \l 261CB
  \l 261CC
  \l 261CD
  \l 261CE
  \l 261CF
  \l 261D0
  \l 261D1
  \l 261D2
  \l 261D3
  \l 261D4
  \l 261D5
  \l 261D6
  \l 261D7
  \l 261D8
  \l 261D9
  \l 261DA
  \l 261DB
  \l 261DC
  \l 261DD
  \l 261DE
  \l 261DF
  \l 261E0
  \l 261E1
  \l 261E2
  \l 261E3
  \l 261E4
  \l 261E5
  \l 261E6
  \l 261E7
  \l 261E8
  \l 261E9
  \l 261EA
  \l 261EB
  \l 261EC
  \l 261ED
  \l 261EE
  \l 261EF
  \l 261F0
  \l 261F1
  \l 261F2
  \l 261F3
  \l 261F4
  \l 261F5
  \l 261F6
  \l 261F7
  \l 261F8
  \l 261F9
  \l 261FA
  \l 261FB
  \l 261FC
  \l 261FD
  \l 261FE
  \l 261FF
  \l 26200
  \l 26201
  \l 26202
  \l 26203
  \l 26204
  \l 26205
  \l 26206
  \l 26207
  \l 26208
  \l 26209
  \l 2620A
  \l 2620B
  \l 2620C
  \l 2620D
  \l 2620E
  \l 2620F
  \l 26210
  \l 26211
  \l 26212
  \l 26213
  \l 26214
  \l 26215
  \l 26216
  \l 26217
  \l 26218
  \l 26219
  \l 2621A
  \l 2621B
  \l 2621C
  \l 2621D
  \l 2621E
  \l 2621F
  \l 26220
  \l 26221
  \l 26222
  \l 26223
  \l 26224
  \l 26225
  \l 26226
  \l 26227
  \l 26228
  \l 26229
  \l 2622A
  \l 2622B
  \l 2622C
  \l 2622D
  \l 2622E
  \l 2622F
  \l 26230
  \l 26231
  \l 26232
  \l 26233
  \l 26234
  \l 26235
  \l 26236
  \l 26237
  \l 26238
  \l 26239
  \l 2623A
  \l 2623B
  \l 2623C
  \l 2623D
  \l 2623E
  \l 2623F
  \l 26240
  \l 26241
  \l 26242
  \l 26243
  \l 26244
  \l 26245
  \l 26246
  \l 26247
  \l 26248
  \l 26249
  \l 2624A
  \l 2624B
  \l 2624C
  \l 2624D
  \l 2624E
  \l 2624F
  \l 26250
  \l 26251
  \l 26252
  \l 26253
  \l 26254
  \l 26255
  \l 26256
  \l 26257
  \l 26258
  \l 26259
  \l 2625A
  \l 2625B
  \l 2625C
  \l 2625D
  \l 2625E
  \l 2625F
  \l 26260
  \l 26261
  \l 26262
  \l 26263
  \l 26264
  \l 26265
  \l 26266
  \l 26267
  \l 26268
  \l 26269
  \l 2626A
  \l 2626B
  \l 2626C
  \l 2626D
  \l 2626E
  \l 2626F
  \l 26270
  \l 26271
  \l 26272
  \l 26273
  \l 26274
  \l 26275
  \l 26276
  \l 26277
  \l 26278
  \l 26279
  \l 2627A
  \l 2627B
  \l 2627C
  \l 2627D
  \l 2627E
  \l 2627F
  \l 26280
  \l 26281
  \l 26282
  \l 26283
  \l 26284
  \l 26285
  \l 26286
  \l 26287
  \l 26288
  \l 26289
  \l 2628A
  \l 2628B
  \l 2628C
  \l 2628D
  \l 2628E
  \l 2628F
  \l 26290
  \l 26291
  \l 26292
  \l 26293
  \l 26294
  \l 26295
  \l 26296
  \l 26297
  \l 26298
  \l 26299
  \l 2629A
  \l 2629B
  \l 2629C
  \l 2629D
  \l 2629E
  \l 2629F
  \l 262A0
  \l 262A1
  \l 262A2
  \l 262A3
  \l 262A4
  \l 262A5
  \l 262A6
  \l 262A7
  \l 262A8
  \l 262A9
  \l 262AA
  \l 262AB
  \l 262AC
  \l 262AD
  \l 262AE
  \l 262AF
  \l 262B0
  \l 262B1
  \l 262B2
  \l 262B3
  \l 262B4
  \l 262B5
  \l 262B6
  \l 262B7
  \l 262B8
  \l 262B9
  \l 262BA
  \l 262BB
  \l 262BC
  \l 262BD
  \l 262BE
  \l 262BF
  \l 262C0
  \l 262C1
  \l 262C2
  \l 262C3
  \l 262C4
  \l 262C5
  \l 262C6
  \l 262C7
  \l 262C8
  \l 262C9
  \l 262CA
  \l 262CB
  \l 262CC
  \l 262CD
  \l 262CE
  \l 262CF
  \l 262D0
  \l 262D1
  \l 262D2
  \l 262D3
  \l 262D4
  \l 262D5
  \l 262D6
  \l 262D7
  \l 262D8
  \l 262D9
  \l 262DA
  \l 262DB
  \l 262DC
  \l 262DD
  \l 262DE
  \l 262DF
  \l 262E0
  \l 262E1
  \l 262E2
  \l 262E3
  \l 262E4
  \l 262E5
  \l 262E6
  \l 262E7
  \l 262E8
  \l 262E9
  \l 262EA
  \l 262EB
  \l 262EC
  \l 262ED
  \l 262EE
  \l 262EF
  \l 262F0
  \l 262F1
  \l 262F2
  \l 262F3
  \l 262F4
  \l 262F5
  \l 262F6
  \l 262F7
  \l 262F8
  \l 262F9
  \l 262FA
  \l 262FB
  \l 262FC
  \l 262FD
  \l 262FE
  \l 262FF
  \l 26300
  \l 26301
  \l 26302
  \l 26303
  \l 26304
  \l 26305
  \l 26306
  \l 26307
  \l 26308
  \l 26309
  \l 2630A
  \l 2630B
  \l 2630C
  \l 2630D
  \l 2630E
  \l 2630F
  \l 26310
  \l 26311
  \l 26312
  \l 26313
  \l 26314
  \l 26315
  \l 26316
  \l 26317
  \l 26318
  \l 26319
  \l 2631A
  \l 2631B
  \l 2631C
  \l 2631D
  \l 2631E
  \l 2631F
  \l 26320
  \l 26321
  \l 26322
  \l 26323
  \l 26324
  \l 26325
  \l 26326
  \l 26327
  \l 26328
  \l 26329
  \l 2632A
  \l 2632B
  \l 2632C
  \l 2632D
  \l 2632E
  \l 2632F
  \l 26330
  \l 26331
  \l 26332
  \l 26333
  \l 26334
  \l 26335
  \l 26336
  \l 26337
  \l 26338
  \l 26339
  \l 2633A
  \l 2633B
  \l 2633C
  \l 2633D
  \l 2633E
  \l 2633F
  \l 26340
  \l 26341
  \l 26342
  \l 26343
  \l 26344
  \l 26345
  \l 26346
  \l 26347
  \l 26348
  \l 26349
  \l 2634A
  \l 2634B
  \l 2634C
  \l 2634D
  \l 2634E
  \l 2634F
  \l 26350
  \l 26351
  \l 26352
  \l 26353
  \l 26354
  \l 26355
  \l 26356
  \l 26357
  \l 26358
  \l 26359
  \l 2635A
  \l 2635B
  \l 2635C
  \l 2635D
  \l 2635E
  \l 2635F
  \l 26360
  \l 26361
  \l 26362
  \l 26363
  \l 26364
  \l 26365
  \l 26366
  \l 26367
  \l 26368
  \l 26369
  \l 2636A
  \l 2636B
  \l 2636C
  \l 2636D
  \l 2636E
  \l 2636F
  \l 26370
  \l 26371
  \l 26372
  \l 26373
  \l 26374
  \l 26375
  \l 26376
  \l 26377
  \l 26378
  \l 26379
  \l 2637A
  \l 2637B
  \l 2637C
  \l 2637D
  \l 2637E
  \l 2637F
  \l 26380
  \l 26381
  \l 26382
  \l 26383
  \l 26384
  \l 26385
  \l 26386
  \l 26387
  \l 26388
  \l 26389
  \l 2638A
  \l 2638B
  \l 2638C
  \l 2638D
  \l 2638E
  \l 2638F
  \l 26390
  \l 26391
  \l 26392
  \l 26393
  \l 26394
  \l 26395
  \l 26396
  \l 26397
  \l 26398
  \l 26399
  \l 2639A
  \l 2639B
  \l 2639C
  \l 2639D
  \l 2639E
  \l 2639F
  \l 263A0
  \l 263A1
  \l 263A2
  \l 263A3
  \l 263A4
  \l 263A5
  \l 263A6
  \l 263A7
  \l 263A8
  \l 263A9
  \l 263AA
  \l 263AB
  \l 263AC
  \l 263AD
  \l 263AE
  \l 263AF
  \l 263B0
  \l 263B1
  \l 263B2
  \l 263B3
  \l 263B4
  \l 263B5
  \l 263B6
  \l 263B7
  \l 263B8
  \l 263B9
  \l 263BA
  \l 263BB
  \l 263BC
  \l 263BD
  \l 263BE
  \l 263BF
  \l 263C0
  \l 263C1
  \l 263C2
  \l 263C3
  \l 263C4
  \l 263C5
  \l 263C6
  \l 263C7
  \l 263C8
  \l 263C9
  \l 263CA
  \l 263CB
  \l 263CC
  \l 263CD
  \l 263CE
  \l 263CF
  \l 263D0
  \l 263D1
  \l 263D2
  \l 263D3
  \l 263D4
  \l 263D5
  \l 263D6
  \l 263D7
  \l 263D8
  \l 263D9
  \l 263DA
  \l 263DB
  \l 263DC
  \l 263DD
  \l 263DE
  \l 263DF
  \l 263E0
  \l 263E1
  \l 263E2
  \l 263E3
  \l 263E4
  \l 263E5
  \l 263E6
  \l 263E7
  \l 263E8
  \l 263E9
  \l 263EA
  \l 263EB
  \l 263EC
  \l 263ED
  \l 263EE
  \l 263EF
  \l 263F0
  \l 263F1
  \l 263F2
  \l 263F3
  \l 263F4
  \l 263F5
  \l 263F6
  \l 263F7
  \l 263F8
  \l 263F9
  \l 263FA
  \l 263FB
  \l 263FC
  \l 263FD
  \l 263FE
  \l 263FF
  \l 26400
  \l 26401
  \l 26402
  \l 26403
  \l 26404
  \l 26405
  \l 26406
  \l 26407
  \l 26408
  \l 26409
  \l 2640A
  \l 2640B
  \l 2640C
  \l 2640D
  \l 2640E
  \l 2640F
  \l 26410
  \l 26411
  \l 26412
  \l 26413
  \l 26414
  \l 26415
  \l 26416
  \l 26417
  \l 26418
  \l 26419
  \l 2641A
  \l 2641B
  \l 2641C
  \l 2641D
  \l 2641E
  \l 2641F
  \l 26420
  \l 26421
  \l 26422
  \l 26423
  \l 26424
  \l 26425
  \l 26426
  \l 26427
  \l 26428
  \l 26429
  \l 2642A
  \l 2642B
  \l 2642C
  \l 2642D
  \l 2642E
  \l 2642F
  \l 26430
  \l 26431
  \l 26432
  \l 26433
  \l 26434
  \l 26435
  \l 26436
  \l 26437
  \l 26438
  \l 26439
  \l 2643A
  \l 2643B
  \l 2643C
  \l 2643D
  \l 2643E
  \l 2643F
  \l 26440
  \l 26441
  \l 26442
  \l 26443
  \l 26444
  \l 26445
  \l 26446
  \l 26447
  \l 26448
  \l 26449
  \l 2644A
  \l 2644B
  \l 2644C
  \l 2644D
  \l 2644E
  \l 2644F
  \l 26450
  \l 26451
  \l 26452
  \l 26453
  \l 26454
  \l 26455
  \l 26456
  \l 26457
  \l 26458
  \l 26459
  \l 2645A
  \l 2645B
  \l 2645C
  \l 2645D
  \l 2645E
  \l 2645F
  \l 26460
  \l 26461
  \l 26462
  \l 26463
  \l 26464
  \l 26465
  \l 26466
  \l 26467
  \l 26468
  \l 26469
  \l 2646A
  \l 2646B
  \l 2646C
  \l 2646D
  \l 2646E
  \l 2646F
  \l 26470
  \l 26471
  \l 26472
  \l 26473
  \l 26474
  \l 26475
  \l 26476
  \l 26477
  \l 26478
  \l 26479
  \l 2647A
  \l 2647B
  \l 2647C
  \l 2647D
  \l 2647E
  \l 2647F
  \l 26480
  \l 26481
  \l 26482
  \l 26483
  \l 26484
  \l 26485
  \l 26486
  \l 26487
  \l 26488
  \l 26489
  \l 2648A
  \l 2648B
  \l 2648C
  \l 2648D
  \l 2648E
  \l 2648F
  \l 26490
  \l 26491
  \l 26492
  \l 26493
  \l 26494
  \l 26495
  \l 26496
  \l 26497
  \l 26498
  \l 26499
  \l 2649A
  \l 2649B
  \l 2649C
  \l 2649D
  \l 2649E
  \l 2649F
  \l 264A0
  \l 264A1
  \l 264A2
  \l 264A3
  \l 264A4
  \l 264A5
  \l 264A6
  \l 264A7
  \l 264A8
  \l 264A9
  \l 264AA
  \l 264AB
  \l 264AC
  \l 264AD
  \l 264AE
  \l 264AF
  \l 264B0
  \l 264B1
  \l 264B2
  \l 264B3
  \l 264B4
  \l 264B5
  \l 264B6
  \l 264B7
  \l 264B8
  \l 264B9
  \l 264BA
  \l 264BB
  \l 264BC
  \l 264BD
  \l 264BE
  \l 264BF
  \l 264C0
  \l 264C1
  \l 264C2
  \l 264C3
  \l 264C4
  \l 264C5
  \l 264C6
  \l 264C7
  \l 264C8
  \l 264C9
  \l 264CA
  \l 264CB
  \l 264CC
  \l 264CD
  \l 264CE
  \l 264CF
  \l 264D0
  \l 264D1
  \l 264D2
  \l 264D3
  \l 264D4
  \l 264D5
  \l 264D6
  \l 264D7
  \l 264D8
  \l 264D9
  \l 264DA
  \l 264DB
  \l 264DC
  \l 264DD
  \l 264DE
  \l 264DF
  \l 264E0
  \l 264E1
  \l 264E2
  \l 264E3
  \l 264E4
  \l 264E5
  \l 264E6
  \l 264E7
  \l 264E8
  \l 264E9
  \l 264EA
  \l 264EB
  \l 264EC
  \l 264ED
  \l 264EE
  \l 264EF
  \l 264F0
  \l 264F1
  \l 264F2
  \l 264F3
  \l 264F4
  \l 264F5
  \l 264F6
  \l 264F7
  \l 264F8
  \l 264F9
  \l 264FA
  \l 264FB
  \l 264FC
  \l 264FD
  \l 264FE
  \l 264FF
  \l 26500
  \l 26501
  \l 26502
  \l 26503
  \l 26504
  \l 26505
  \l 26506
  \l 26507
  \l 26508
  \l 26509
  \l 2650A
  \l 2650B
  \l 2650C
  \l 2650D
  \l 2650E
  \l 2650F
  \l 26510
  \l 26511
  \l 26512
  \l 26513
  \l 26514
  \l 26515
  \l 26516
  \l 26517
  \l 26518
  \l 26519
  \l 2651A
  \l 2651B
  \l 2651C
  \l 2651D
  \l 2651E
  \l 2651F
  \l 26520
  \l 26521
  \l 26522
  \l 26523
  \l 26524
  \l 26525
  \l 26526
  \l 26527
  \l 26528
  \l 26529
  \l 2652A
  \l 2652B
  \l 2652C
  \l 2652D
  \l 2652E
  \l 2652F
  \l 26530
  \l 26531
  \l 26532
  \l 26533
  \l 26534
  \l 26535
  \l 26536
  \l 26537
  \l 26538
  \l 26539
  \l 2653A
  \l 2653B
  \l 2653C
  \l 2653D
  \l 2653E
  \l 2653F
  \l 26540
  \l 26541
  \l 26542
  \l 26543
  \l 26544
  \l 26545
  \l 26546
  \l 26547
  \l 26548
  \l 26549
  \l 2654A
  \l 2654B
  \l 2654C
  \l 2654D
  \l 2654E
  \l 2654F
  \l 26550
  \l 26551
  \l 26552
  \l 26553
  \l 26554
  \l 26555
  \l 26556
  \l 26557
  \l 26558
  \l 26559
  \l 2655A
  \l 2655B
  \l 2655C
  \l 2655D
  \l 2655E
  \l 2655F
  \l 26560
  \l 26561
  \l 26562
  \l 26563
  \l 26564
  \l 26565
  \l 26566
  \l 26567
  \l 26568
  \l 26569
  \l 2656A
  \l 2656B
  \l 2656C
  \l 2656D
  \l 2656E
  \l 2656F
  \l 26570
  \l 26571
  \l 26572
  \l 26573
  \l 26574
  \l 26575
  \l 26576
  \l 26577
  \l 26578
  \l 26579
  \l 2657A
  \l 2657B
  \l 2657C
  \l 2657D
  \l 2657E
  \l 2657F
  \l 26580
  \l 26581
  \l 26582
  \l 26583
  \l 26584
  \l 26585
  \l 26586
  \l 26587
  \l 26588
  \l 26589
  \l 2658A
  \l 2658B
  \l 2658C
  \l 2658D
  \l 2658E
  \l 2658F
  \l 26590
  \l 26591
  \l 26592
  \l 26593
  \l 26594
  \l 26595
  \l 26596
  \l 26597
  \l 26598
  \l 26599
  \l 2659A
  \l 2659B
  \l 2659C
  \l 2659D
  \l 2659E
  \l 2659F
  \l 265A0
  \l 265A1
  \l 265A2
  \l 265A3
  \l 265A4
  \l 265A5
  \l 265A6
  \l 265A7
  \l 265A8
  \l 265A9
  \l 265AA
  \l 265AB
  \l 265AC
  \l 265AD
  \l 265AE
  \l 265AF
  \l 265B0
  \l 265B1
  \l 265B2
  \l 265B3
  \l 265B4
  \l 265B5
  \l 265B6
  \l 265B7
  \l 265B8
  \l 265B9
  \l 265BA
  \l 265BB
  \l 265BC
  \l 265BD
  \l 265BE
  \l 265BF
  \l 265C0
  \l 265C1
  \l 265C2
  \l 265C3
  \l 265C4
  \l 265C5
  \l 265C6
  \l 265C7
  \l 265C8
  \l 265C9
  \l 265CA
  \l 265CB
  \l 265CC
  \l 265CD
  \l 265CE
  \l 265CF
  \l 265D0
  \l 265D1
  \l 265D2
  \l 265D3
  \l 265D4
  \l 265D5
  \l 265D6
  \l 265D7
  \l 265D8
  \l 265D9
  \l 265DA
  \l 265DB
  \l 265DC
  \l 265DD
  \l 265DE
  \l 265DF
  \l 265E0
  \l 265E1
  \l 265E2
  \l 265E3
  \l 265E4
  \l 265E5
  \l 265E6
  \l 265E7
  \l 265E8
  \l 265E9
  \l 265EA
  \l 265EB
  \l 265EC
  \l 265ED
  \l 265EE
  \l 265EF
  \l 265F0
  \l 265F1
  \l 265F2
  \l 265F3
  \l 265F4
  \l 265F5
  \l 265F6
  \l 265F7
  \l 265F8
  \l 265F9
  \l 265FA
  \l 265FB
  \l 265FC
  \l 265FD
  \l 265FE
  \l 265FF
  \l 26600
  \l 26601
  \l 26602
  \l 26603
  \l 26604
  \l 26605
  \l 26606
  \l 26607
  \l 26608
  \l 26609
  \l 2660A
  \l 2660B
  \l 2660C
  \l 2660D
  \l 2660E
  \l 2660F
  \l 26610
  \l 26611
  \l 26612
  \l 26613
  \l 26614
  \l 26615
  \l 26616
  \l 26617
  \l 26618
  \l 26619
  \l 2661A
  \l 2661B
  \l 2661C
  \l 2661D
  \l 2661E
  \l 2661F
  \l 26620
  \l 26621
  \l 26622
  \l 26623
  \l 26624
  \l 26625
  \l 26626
  \l 26627
  \l 26628
  \l 26629
  \l 2662A
  \l 2662B
  \l 2662C
  \l 2662D
  \l 2662E
  \l 2662F
  \l 26630
  \l 26631
  \l 26632
  \l 26633
  \l 26634
  \l 26635
  \l 26636
  \l 26637
  \l 26638
  \l 26639
  \l 2663A
  \l 2663B
  \l 2663C
  \l 2663D
  \l 2663E
  \l 2663F
  \l 26640
  \l 26641
  \l 26642
  \l 26643
  \l 26644
  \l 26645
  \l 26646
  \l 26647
  \l 26648
  \l 26649
  \l 2664A
  \l 2664B
  \l 2664C
  \l 2664D
  \l 2664E
  \l 2664F
  \l 26650
  \l 26651
  \l 26652
  \l 26653
  \l 26654
  \l 26655
  \l 26656
  \l 26657
  \l 26658
  \l 26659
  \l 2665A
  \l 2665B
  \l 2665C
  \l 2665D
  \l 2665E
  \l 2665F
  \l 26660
  \l 26661
  \l 26662
  \l 26663
  \l 26664
  \l 26665
  \l 26666
  \l 26667
  \l 26668
  \l 26669
  \l 2666A
  \l 2666B
  \l 2666C
  \l 2666D
  \l 2666E
  \l 2666F
  \l 26670
  \l 26671
  \l 26672
  \l 26673
  \l 26674
  \l 26675
  \l 26676
  \l 26677
  \l 26678
  \l 26679
  \l 2667A
  \l 2667B
  \l 2667C
  \l 2667D
  \l 2667E
  \l 2667F
  \l 26680
  \l 26681
  \l 26682
  \l 26683
  \l 26684
  \l 26685
  \l 26686
  \l 26687
  \l 26688
  \l 26689
  \l 2668A
  \l 2668B
  \l 2668C
  \l 2668D
  \l 2668E
  \l 2668F
  \l 26690
  \l 26691
  \l 26692
  \l 26693
  \l 26694
  \l 26695
  \l 26696
  \l 26697
  \l 26698
  \l 26699
  \l 2669A
  \l 2669B
  \l 2669C
  \l 2669D
  \l 2669E
  \l 2669F
  \l 266A0
  \l 266A1
  \l 266A2
  \l 266A3
  \l 266A4
  \l 266A5
  \l 266A6
  \l 266A7
  \l 266A8
  \l 266A9
  \l 266AA
  \l 266AB
  \l 266AC
  \l 266AD
  \l 266AE
  \l 266AF
  \l 266B0
  \l 266B1
  \l 266B2
  \l 266B3
  \l 266B4
  \l 266B5
  \l 266B6
  \l 266B7
  \l 266B8
  \l 266B9
  \l 266BA
  \l 266BB
  \l 266BC
  \l 266BD
  \l 266BE
  \l 266BF
  \l 266C0
  \l 266C1
  \l 266C2
  \l 266C3
  \l 266C4
  \l 266C5
  \l 266C6
  \l 266C7
  \l 266C8
  \l 266C9
  \l 266CA
  \l 266CB
  \l 266CC
  \l 266CD
  \l 266CE
  \l 266CF
  \l 266D0
  \l 266D1
  \l 266D2
  \l 266D3
  \l 266D4
  \l 266D5
  \l 266D6
  \l 266D7
  \l 266D8
  \l 266D9
  \l 266DA
  \l 266DB
  \l 266DC
  \l 266DD
  \l 266DE
  \l 266DF
  \l 266E0
  \l 266E1
  \l 266E2
  \l 266E3
  \l 266E4
  \l 266E5
  \l 266E6
  \l 266E7
  \l 266E8
  \l 266E9
  \l 266EA
  \l 266EB
  \l 266EC
  \l 266ED
  \l 266EE
  \l 266EF
  \l 266F0
  \l 266F1
  \l 266F2
  \l 266F3
  \l 266F4
  \l 266F5
  \l 266F6
  \l 266F7
  \l 266F8
  \l 266F9
  \l 266FA
  \l 266FB
  \l 266FC
  \l 266FD
  \l 266FE
  \l 266FF
  \l 26700
  \l 26701
  \l 26702
  \l 26703
  \l 26704
  \l 26705
  \l 26706
  \l 26707
  \l 26708
  \l 26709
  \l 2670A
  \l 2670B
  \l 2670C
  \l 2670D
  \l 2670E
  \l 2670F
  \l 26710
  \l 26711
  \l 26712
  \l 26713
  \l 26714
  \l 26715
  \l 26716
  \l 26717
  \l 26718
  \l 26719
  \l 2671A
  \l 2671B
  \l 2671C
  \l 2671D
  \l 2671E
  \l 2671F
  \l 26720
  \l 26721
  \l 26722
  \l 26723
  \l 26724
  \l 26725
  \l 26726
  \l 26727
  \l 26728
  \l 26729
  \l 2672A
  \l 2672B
  \l 2672C
  \l 2672D
  \l 2672E
  \l 2672F
  \l 26730
  \l 26731
  \l 26732
  \l 26733
  \l 26734
  \l 26735
  \l 26736
  \l 26737
  \l 26738
  \l 26739
  \l 2673A
  \l 2673B
  \l 2673C
  \l 2673D
  \l 2673E
  \l 2673F
  \l 26740
  \l 26741
  \l 26742
  \l 26743
  \l 26744
  \l 26745
  \l 26746
  \l 26747
  \l 26748
  \l 26749
  \l 2674A
  \l 2674B
  \l 2674C
  \l 2674D
  \l 2674E
  \l 2674F
  \l 26750
  \l 26751
  \l 26752
  \l 26753
  \l 26754
  \l 26755
  \l 26756
  \l 26757
  \l 26758
  \l 26759
  \l 2675A
  \l 2675B
  \l 2675C
  \l 2675D
  \l 2675E
  \l 2675F
  \l 26760
  \l 26761
  \l 26762
  \l 26763
  \l 26764
  \l 26765
  \l 26766
  \l 26767
  \l 26768
  \l 26769
  \l 2676A
  \l 2676B
  \l 2676C
  \l 2676D
  \l 2676E
  \l 2676F
  \l 26770
  \l 26771
  \l 26772
  \l 26773
  \l 26774
  \l 26775
  \l 26776
  \l 26777
  \l 26778
  \l 26779
  \l 2677A
  \l 2677B
  \l 2677C
  \l 2677D
  \l 2677E
  \l 2677F
  \l 26780
  \l 26781
  \l 26782
  \l 26783
  \l 26784
  \l 26785
  \l 26786
  \l 26787
  \l 26788
  \l 26789
  \l 2678A
  \l 2678B
  \l 2678C
  \l 2678D
  \l 2678E
  \l 2678F
  \l 26790
  \l 26791
  \l 26792
  \l 26793
  \l 26794
  \l 26795
  \l 26796
  \l 26797
  \l 26798
  \l 26799
  \l 2679A
  \l 2679B
  \l 2679C
  \l 2679D
  \l 2679E
  \l 2679F
  \l 267A0
  \l 267A1
  \l 267A2
  \l 267A3
  \l 267A4
  \l 267A5
  \l 267A6
  \l 267A7
  \l 267A8
  \l 267A9
  \l 267AA
  \l 267AB
  \l 267AC
  \l 267AD
  \l 267AE
  \l 267AF
  \l 267B0
  \l 267B1
  \l 267B2
  \l 267B3
  \l 267B4
  \l 267B5
  \l 267B6
  \l 267B7
  \l 267B8
  \l 267B9
  \l 267BA
  \l 267BB
  \l 267BC
  \l 267BD
  \l 267BE
  \l 267BF
  \l 267C0
  \l 267C1
  \l 267C2
  \l 267C3
  \l 267C4
  \l 267C5
  \l 267C6
  \l 267C7
  \l 267C8
  \l 267C9
  \l 267CA
  \l 267CB
  \l 267CC
  \l 267CD
  \l 267CE
  \l 267CF
  \l 267D0
  \l 267D1
  \l 267D2
  \l 267D3
  \l 267D4
  \l 267D5
  \l 267D6
  \l 267D7
  \l 267D8
  \l 267D9
  \l 267DA
  \l 267DB
  \l 267DC
  \l 267DD
  \l 267DE
  \l 267DF
  \l 267E0
  \l 267E1
  \l 267E2
  \l 267E3
  \l 267E4
  \l 267E5
  \l 267E6
  \l 267E7
  \l 267E8
  \l 267E9
  \l 267EA
  \l 267EB
  \l 267EC
  \l 267ED
  \l 267EE
  \l 267EF
  \l 267F0
  \l 267F1
  \l 267F2
  \l 267F3
  \l 267F4
  \l 267F5
  \l 267F6
  \l 267F7
  \l 267F8
  \l 267F9
  \l 267FA
  \l 267FB
  \l 267FC
  \l 267FD
  \l 267FE
  \l 267FF
  \l 26800
  \l 26801
  \l 26802
  \l 26803
  \l 26804
  \l 26805
  \l 26806
  \l 26807
  \l 26808
  \l 26809
  \l 2680A
  \l 2680B
  \l 2680C
  \l 2680D
  \l 2680E
  \l 2680F
  \l 26810
  \l 26811
  \l 26812
  \l 26813
  \l 26814
  \l 26815
  \l 26816
  \l 26817
  \l 26818
  \l 26819
  \l 2681A
  \l 2681B
  \l 2681C
  \l 2681D
  \l 2681E
  \l 2681F
  \l 26820
  \l 26821
  \l 26822
  \l 26823
  \l 26824
  \l 26825
  \l 26826
  \l 26827
  \l 26828
  \l 26829
  \l 2682A
  \l 2682B
  \l 2682C
  \l 2682D
  \l 2682E
  \l 2682F
  \l 26830
  \l 26831
  \l 26832
  \l 26833
  \l 26834
  \l 26835
  \l 26836
  \l 26837
  \l 26838
  \l 26839
  \l 2683A
  \l 2683B
  \l 2683C
  \l 2683D
  \l 2683E
  \l 2683F
  \l 26840
  \l 26841
  \l 26842
  \l 26843
  \l 26844
  \l 26845
  \l 26846
  \l 26847
  \l 26848
  \l 26849
  \l 2684A
  \l 2684B
  \l 2684C
  \l 2684D
  \l 2684E
  \l 2684F
  \l 26850
  \l 26851
  \l 26852
  \l 26853
  \l 26854
  \l 26855
  \l 26856
  \l 26857
  \l 26858
  \l 26859
  \l 2685A
  \l 2685B
  \l 2685C
  \l 2685D
  \l 2685E
  \l 2685F
  \l 26860
  \l 26861
  \l 26862
  \l 26863
  \l 26864
  \l 26865
  \l 26866
  \l 26867
  \l 26868
  \l 26869
  \l 2686A
  \l 2686B
  \l 2686C
  \l 2686D
  \l 2686E
  \l 2686F
  \l 26870
  \l 26871
  \l 26872
  \l 26873
  \l 26874
  \l 26875
  \l 26876
  \l 26877
  \l 26878
  \l 26879
  \l 2687A
  \l 2687B
  \l 2687C
  \l 2687D
  \l 2687E
  \l 2687F
  \l 26880
  \l 26881
  \l 26882
  \l 26883
  \l 26884
  \l 26885
  \l 26886
  \l 26887
  \l 26888
  \l 26889
  \l 2688A
  \l 2688B
  \l 2688C
  \l 2688D
  \l 2688E
  \l 2688F
  \l 26890
  \l 26891
  \l 26892
  \l 26893
  \l 26894
  \l 26895
  \l 26896
  \l 26897
  \l 26898
  \l 26899
  \l 2689A
  \l 2689B
  \l 2689C
  \l 2689D
  \l 2689E
  \l 2689F
  \l 268A0
  \l 268A1
  \l 268A2
  \l 268A3
  \l 268A4
  \l 268A5
  \l 268A6
  \l 268A7
  \l 268A8
  \l 268A9
  \l 268AA
  \l 268AB
  \l 268AC
  \l 268AD
  \l 268AE
  \l 268AF
  \l 268B0
  \l 268B1
  \l 268B2
  \l 268B3
  \l 268B4
  \l 268B5
  \l 268B6
  \l 268B7
  \l 268B8
  \l 268B9
  \l 268BA
  \l 268BB
  \l 268BC
  \l 268BD
  \l 268BE
  \l 268BF
  \l 268C0
  \l 268C1
  \l 268C2
  \l 268C3
  \l 268C4
  \l 268C5
  \l 268C6
  \l 268C7
  \l 268C8
  \l 268C9
  \l 268CA
  \l 268CB
  \l 268CC
  \l 268CD
  \l 268CE
  \l 268CF
  \l 268D0
  \l 268D1
  \l 268D2
  \l 268D3
  \l 268D4
  \l 268D5
  \l 268D6
  \l 268D7
  \l 268D8
  \l 268D9
  \l 268DA
  \l 268DB
  \l 268DC
  \l 268DD
  \l 268DE
  \l 268DF
  \l 268E0
  \l 268E1
  \l 268E2
  \l 268E3
  \l 268E4
  \l 268E5
  \l 268E6
  \l 268E7
  \l 268E8
  \l 268E9
  \l 268EA
  \l 268EB
  \l 268EC
  \l 268ED
  \l 268EE
  \l 268EF
  \l 268F0
  \l 268F1
  \l 268F2
  \l 268F3
  \l 268F4
  \l 268F5
  \l 268F6
  \l 268F7
  \l 268F8
  \l 268F9
  \l 268FA
  \l 268FB
  \l 268FC
  \l 268FD
  \l 268FE
  \l 268FF
  \l 26900
  \l 26901
  \l 26902
  \l 26903
  \l 26904
  \l 26905
  \l 26906
  \l 26907
  \l 26908
  \l 26909
  \l 2690A
  \l 2690B
  \l 2690C
  \l 2690D
  \l 2690E
  \l 2690F
  \l 26910
  \l 26911
  \l 26912
  \l 26913
  \l 26914
  \l 26915
  \l 26916
  \l 26917
  \l 26918
  \l 26919
  \l 2691A
  \l 2691B
  \l 2691C
  \l 2691D
  \l 2691E
  \l 2691F
  \l 26920
  \l 26921
  \l 26922
  \l 26923
  \l 26924
  \l 26925
  \l 26926
  \l 26927
  \l 26928
  \l 26929
  \l 2692A
  \l 2692B
  \l 2692C
  \l 2692D
  \l 2692E
  \l 2692F
  \l 26930
  \l 26931
  \l 26932
  \l 26933
  \l 26934
  \l 26935
  \l 26936
  \l 26937
  \l 26938
  \l 26939
  \l 2693A
  \l 2693B
  \l 2693C
  \l 2693D
  \l 2693E
  \l 2693F
  \l 26940
  \l 26941
  \l 26942
  \l 26943
  \l 26944
  \l 26945
  \l 26946
  \l 26947
  \l 26948
  \l 26949
  \l 2694A
  \l 2694B
  \l 2694C
  \l 2694D
  \l 2694E
  \l 2694F
  \l 26950
  \l 26951
  \l 26952
  \l 26953
  \l 26954
  \l 26955
  \l 26956
  \l 26957
  \l 26958
  \l 26959
  \l 2695A
  \l 2695B
  \l 2695C
  \l 2695D
  \l 2695E
  \l 2695F
  \l 26960
  \l 26961
  \l 26962
  \l 26963
  \l 26964
  \l 26965
  \l 26966
  \l 26967
  \l 26968
  \l 26969
  \l 2696A
  \l 2696B
  \l 2696C
  \l 2696D
  \l 2696E
  \l 2696F
  \l 26970
  \l 26971
  \l 26972
  \l 26973
  \l 26974
  \l 26975
  \l 26976
  \l 26977
  \l 26978
  \l 26979
  \l 2697A
  \l 2697B
  \l 2697C
  \l 2697D
  \l 2697E
  \l 2697F
  \l 26980
  \l 26981
  \l 26982
  \l 26983
  \l 26984
  \l 26985
  \l 26986
  \l 26987
  \l 26988
  \l 26989
  \l 2698A
  \l 2698B
  \l 2698C
  \l 2698D
  \l 2698E
  \l 2698F
  \l 26990
  \l 26991
  \l 26992
  \l 26993
  \l 26994
  \l 26995
  \l 26996
  \l 26997
  \l 26998
  \l 26999
  \l 2699A
  \l 2699B
  \l 2699C
  \l 2699D
  \l 2699E
  \l 2699F
  \l 269A0
  \l 269A1
  \l 269A2
  \l 269A3
  \l 269A4
  \l 269A5
  \l 269A6
  \l 269A7
  \l 269A8
  \l 269A9
  \l 269AA
  \l 269AB
  \l 269AC
  \l 269AD
  \l 269AE
  \l 269AF
  \l 269B0
  \l 269B1
  \l 269B2
  \l 269B3
  \l 269B4
  \l 269B5
  \l 269B6
  \l 269B7
  \l 269B8
  \l 269B9
  \l 269BA
  \l 269BB
  \l 269BC
  \l 269BD
  \l 269BE
  \l 269BF
  \l 269C0
  \l 269C1
  \l 269C2
  \l 269C3
  \l 269C4
  \l 269C5
  \l 269C6
  \l 269C7
  \l 269C8
  \l 269C9
  \l 269CA
  \l 269CB
  \l 269CC
  \l 269CD
  \l 269CE
  \l 269CF
  \l 269D0
  \l 269D1
  \l 269D2
  \l 269D3
  \l 269D4
  \l 269D5
  \l 269D6
  \l 269D7
  \l 269D8
  \l 269D9
  \l 269DA
  \l 269DB
  \l 269DC
  \l 269DD
  \l 269DE
  \l 269DF
  \l 269E0
  \l 269E1
  \l 269E2
  \l 269E3
  \l 269E4
  \l 269E5
  \l 269E6
  \l 269E7
  \l 269E8
  \l 269E9
  \l 269EA
  \l 269EB
  \l 269EC
  \l 269ED
  \l 269EE
  \l 269EF
  \l 269F0
  \l 269F1
  \l 269F2
  \l 269F3
  \l 269F4
  \l 269F5
  \l 269F6
  \l 269F7
  \l 269F8
  \l 269F9
  \l 269FA
  \l 269FB
  \l 269FC
  \l 269FD
  \l 269FE
  \l 269FF
  \l 26A00
  \l 26A01
  \l 26A02
  \l 26A03
  \l 26A04
  \l 26A05
  \l 26A06
  \l 26A07
  \l 26A08
  \l 26A09
  \l 26A0A
  \l 26A0B
  \l 26A0C
  \l 26A0D
  \l 26A0E
  \l 26A0F
  \l 26A10
  \l 26A11
  \l 26A12
  \l 26A13
  \l 26A14
  \l 26A15
  \l 26A16
  \l 26A17
  \l 26A18
  \l 26A19
  \l 26A1A
  \l 26A1B
  \l 26A1C
  \l 26A1D
  \l 26A1E
  \l 26A1F
  \l 26A20
  \l 26A21
  \l 26A22
  \l 26A23
  \l 26A24
  \l 26A25
  \l 26A26
  \l 26A27
  \l 26A28
  \l 26A29
  \l 26A2A
  \l 26A2B
  \l 26A2C
  \l 26A2D
  \l 26A2E
  \l 26A2F
  \l 26A30
  \l 26A31
  \l 26A32
  \l 26A33
  \l 26A34
  \l 26A35
  \l 26A36
  \l 26A37
  \l 26A38
  \l 26A39
  \l 26A3A
  \l 26A3B
  \l 26A3C
  \l 26A3D
  \l 26A3E
  \l 26A3F
  \l 26A40
  \l 26A41
  \l 26A42
  \l 26A43
  \l 26A44
  \l 26A45
  \l 26A46
  \l 26A47
  \l 26A48
  \l 26A49
  \l 26A4A
  \l 26A4B
  \l 26A4C
  \l 26A4D
  \l 26A4E
  \l 26A4F
  \l 26A50
  \l 26A51
  \l 26A52
  \l 26A53
  \l 26A54
  \l 26A55
  \l 26A56
  \l 26A57
  \l 26A58
  \l 26A59
  \l 26A5A
  \l 26A5B
  \l 26A5C
  \l 26A5D
  \l 26A5E
  \l 26A5F
  \l 26A60
  \l 26A61
  \l 26A62
  \l 26A63
  \l 26A64
  \l 26A65
  \l 26A66
  \l 26A67
  \l 26A68
  \l 26A69
  \l 26A6A
  \l 26A6B
  \l 26A6C
  \l 26A6D
  \l 26A6E
  \l 26A6F
  \l 26A70
  \l 26A71
  \l 26A72
  \l 26A73
  \l 26A74
  \l 26A75
  \l 26A76
  \l 26A77
  \l 26A78
  \l 26A79
  \l 26A7A
  \l 26A7B
  \l 26A7C
  \l 26A7D
  \l 26A7E
  \l 26A7F
  \l 26A80
  \l 26A81
  \l 26A82
  \l 26A83
  \l 26A84
  \l 26A85
  \l 26A86
  \l 26A87
  \l 26A88
  \l 26A89
  \l 26A8A
  \l 26A8B
  \l 26A8C
  \l 26A8D
  \l 26A8E
  \l 26A8F
  \l 26A90
  \l 26A91
  \l 26A92
  \l 26A93
  \l 26A94
  \l 26A95
  \l 26A96
  \l 26A97
  \l 26A98
  \l 26A99
  \l 26A9A
  \l 26A9B
  \l 26A9C
  \l 26A9D
  \l 26A9E
  \l 26A9F
  \l 26AA0
  \l 26AA1
  \l 26AA2
  \l 26AA3
  \l 26AA4
  \l 26AA5
  \l 26AA6
  \l 26AA7
  \l 26AA8
  \l 26AA9
  \l 26AAA
  \l 26AAB
  \l 26AAC
  \l 26AAD
  \l 26AAE
  \l 26AAF
  \l 26AB0
  \l 26AB1
  \l 26AB2
  \l 26AB3
  \l 26AB4
  \l 26AB5
  \l 26AB6
  \l 26AB7
  \l 26AB8
  \l 26AB9
  \l 26ABA
  \l 26ABB
  \l 26ABC
  \l 26ABD
  \l 26ABE
  \l 26ABF
  \l 26AC0
  \l 26AC1
  \l 26AC2
  \l 26AC3
  \l 26AC4
  \l 26AC5
  \l 26AC6
  \l 26AC7
  \l 26AC8
  \l 26AC9
  \l 26ACA
  \l 26ACB
  \l 26ACC
  \l 26ACD
  \l 26ACE
  \l 26ACF
  \l 26AD0
  \l 26AD1
  \l 26AD2
  \l 26AD3
  \l 26AD4
  \l 26AD5
  \l 26AD6
  \l 26AD7
  \l 26AD8
  \l 26AD9
  \l 26ADA
  \l 26ADB
  \l 26ADC
  \l 26ADD
  \l 26ADE
  \l 26ADF
  \l 26AE0
  \l 26AE1
  \l 26AE2
  \l 26AE3
  \l 26AE4
  \l 26AE5
  \l 26AE6
  \l 26AE7
  \l 26AE8
  \l 26AE9
  \l 26AEA
  \l 26AEB
  \l 26AEC
  \l 26AED
  \l 26AEE
  \l 26AEF
  \l 26AF0
  \l 26AF1
  \l 26AF2
  \l 26AF3
  \l 26AF4
  \l 26AF5
  \l 26AF6
  \l 26AF7
  \l 26AF8
  \l 26AF9
  \l 26AFA
  \l 26AFB
  \l 26AFC
  \l 26AFD
  \l 26AFE
  \l 26AFF
  \l 26B00
  \l 26B01
  \l 26B02
  \l 26B03
  \l 26B04
  \l 26B05
  \l 26B06
  \l 26B07
  \l 26B08
  \l 26B09
  \l 26B0A
  \l 26B0B
  \l 26B0C
  \l 26B0D
  \l 26B0E
  \l 26B0F
  \l 26B10
  \l 26B11
  \l 26B12
  \l 26B13
  \l 26B14
  \l 26B15
  \l 26B16
  \l 26B17
  \l 26B18
  \l 26B19
  \l 26B1A
  \l 26B1B
  \l 26B1C
  \l 26B1D
  \l 26B1E
  \l 26B1F
  \l 26B20
  \l 26B21
  \l 26B22
  \l 26B23
  \l 26B24
  \l 26B25
  \l 26B26
  \l 26B27
  \l 26B28
  \l 26B29
  \l 26B2A
  \l 26B2B
  \l 26B2C
  \l 26B2D
  \l 26B2E
  \l 26B2F
  \l 26B30
  \l 26B31
  \l 26B32
  \l 26B33
  \l 26B34
  \l 26B35
  \l 26B36
  \l 26B37
  \l 26B38
  \l 26B39
  \l 26B3A
  \l 26B3B
  \l 26B3C
  \l 26B3D
  \l 26B3E
  \l 26B3F
  \l 26B40
  \l 26B41
  \l 26B42
  \l 26B43
  \l 26B44
  \l 26B45
  \l 26B46
  \l 26B47
  \l 26B48
  \l 26B49
  \l 26B4A
  \l 26B4B
  \l 26B4C
  \l 26B4D
  \l 26B4E
  \l 26B4F
  \l 26B50
  \l 26B51
  \l 26B52
  \l 26B53
  \l 26B54
  \l 26B55
  \l 26B56
  \l 26B57
  \l 26B58
  \l 26B59
  \l 26B5A
  \l 26B5B
  \l 26B5C
  \l 26B5D
  \l 26B5E
  \l 26B5F
  \l 26B60
  \l 26B61
  \l 26B62
  \l 26B63
  \l 26B64
  \l 26B65
  \l 26B66
  \l 26B67
  \l 26B68
  \l 26B69
  \l 26B6A
  \l 26B6B
  \l 26B6C
  \l 26B6D
  \l 26B6E
  \l 26B6F
  \l 26B70
  \l 26B71
  \l 26B72
  \l 26B73
  \l 26B74
  \l 26B75
  \l 26B76
  \l 26B77
  \l 26B78
  \l 26B79
  \l 26B7A
  \l 26B7B
  \l 26B7C
  \l 26B7D
  \l 26B7E
  \l 26B7F
  \l 26B80
  \l 26B81
  \l 26B82
  \l 26B83
  \l 26B84
  \l 26B85
  \l 26B86
  \l 26B87
  \l 26B88
  \l 26B89
  \l 26B8A
  \l 26B8B
  \l 26B8C
  \l 26B8D
  \l 26B8E
  \l 26B8F
  \l 26B90
  \l 26B91
  \l 26B92
  \l 26B93
  \l 26B94
  \l 26B95
  \l 26B96
  \l 26B97
  \l 26B98
  \l 26B99
  \l 26B9A
  \l 26B9B
  \l 26B9C
  \l 26B9D
  \l 26B9E
  \l 26B9F
  \l 26BA0
  \l 26BA1
  \l 26BA2
  \l 26BA3
  \l 26BA4
  \l 26BA5
  \l 26BA6
  \l 26BA7
  \l 26BA8
  \l 26BA9
  \l 26BAA
  \l 26BAB
  \l 26BAC
  \l 26BAD
  \l 26BAE
  \l 26BAF
  \l 26BB0
  \l 26BB1
  \l 26BB2
  \l 26BB3
  \l 26BB4
  \l 26BB5
  \l 26BB6
  \l 26BB7
  \l 26BB8
  \l 26BB9
  \l 26BBA
  \l 26BBB
  \l 26BBC
  \l 26BBD
  \l 26BBE
  \l 26BBF
  \l 26BC0
  \l 26BC1
  \l 26BC2
  \l 26BC3
  \l 26BC4
  \l 26BC5
  \l 26BC6
  \l 26BC7
  \l 26BC8
  \l 26BC9
  \l 26BCA
  \l 26BCB
  \l 26BCC
  \l 26BCD
  \l 26BCE
  \l 26BCF
  \l 26BD0
  \l 26BD1
  \l 26BD2
  \l 26BD3
  \l 26BD4
  \l 26BD5
  \l 26BD6
  \l 26BD7
  \l 26BD8
  \l 26BD9
  \l 26BDA
  \l 26BDB
  \l 26BDC
  \l 26BDD
  \l 26BDE
  \l 26BDF
  \l 26BE0
  \l 26BE1
  \l 26BE2
  \l 26BE3
  \l 26BE4
  \l 26BE5
  \l 26BE6
  \l 26BE7
  \l 26BE8
  \l 26BE9
  \l 26BEA
  \l 26BEB
  \l 26BEC
  \l 26BED
  \l 26BEE
  \l 26BEF
  \l 26BF0
  \l 26BF1
  \l 26BF2
  \l 26BF3
  \l 26BF4
  \l 26BF5
  \l 26BF6
  \l 26BF7
  \l 26BF8
  \l 26BF9
  \l 26BFA
  \l 26BFB
  \l 26BFC
  \l 26BFD
  \l 26BFE
  \l 26BFF
  \l 26C00
  \l 26C01
  \l 26C02
  \l 26C03
  \l 26C04
  \l 26C05
  \l 26C06
  \l 26C07
  \l 26C08
  \l 26C09
  \l 26C0A
  \l 26C0B
  \l 26C0C
  \l 26C0D
  \l 26C0E
  \l 26C0F
  \l 26C10
  \l 26C11
  \l 26C12
  \l 26C13
  \l 26C14
  \l 26C15
  \l 26C16
  \l 26C17
  \l 26C18
  \l 26C19
  \l 26C1A
  \l 26C1B
  \l 26C1C
  \l 26C1D
  \l 26C1E
  \l 26C1F
  \l 26C20
  \l 26C21
  \l 26C22
  \l 26C23
  \l 26C24
  \l 26C25
  \l 26C26
  \l 26C27
  \l 26C28
  \l 26C29
  \l 26C2A
  \l 26C2B
  \l 26C2C
  \l 26C2D
  \l 26C2E
  \l 26C2F
  \l 26C30
  \l 26C31
  \l 26C32
  \l 26C33
  \l 26C34
  \l 26C35
  \l 26C36
  \l 26C37
  \l 26C38
  \l 26C39
  \l 26C3A
  \l 26C3B
  \l 26C3C
  \l 26C3D
  \l 26C3E
  \l 26C3F
  \l 26C40
  \l 26C41
  \l 26C42
  \l 26C43
  \l 26C44
  \l 26C45
  \l 26C46
  \l 26C47
  \l 26C48
  \l 26C49
  \l 26C4A
  \l 26C4B
  \l 26C4C
  \l 26C4D
  \l 26C4E
  \l 26C4F
  \l 26C50
  \l 26C51
  \l 26C52
  \l 26C53
  \l 26C54
  \l 26C55
  \l 26C56
  \l 26C57
  \l 26C58
  \l 26C59
  \l 26C5A
  \l 26C5B
  \l 26C5C
  \l 26C5D
  \l 26C5E
  \l 26C5F
  \l 26C60
  \l 26C61
  \l 26C62
  \l 26C63
  \l 26C64
  \l 26C65
  \l 26C66
  \l 26C67
  \l 26C68
  \l 26C69
  \l 26C6A
  \l 26C6B
  \l 26C6C
  \l 26C6D
  \l 26C6E
  \l 26C6F
  \l 26C70
  \l 26C71
  \l 26C72
  \l 26C73
  \l 26C74
  \l 26C75
  \l 26C76
  \l 26C77
  \l 26C78
  \l 26C79
  \l 26C7A
  \l 26C7B
  \l 26C7C
  \l 26C7D
  \l 26C7E
  \l 26C7F
  \l 26C80
  \l 26C81
  \l 26C82
  \l 26C83
  \l 26C84
  \l 26C85
  \l 26C86
  \l 26C87
  \l 26C88
  \l 26C89
  \l 26C8A
  \l 26C8B
  \l 26C8C
  \l 26C8D
  \l 26C8E
  \l 26C8F
  \l 26C90
  \l 26C91
  \l 26C92
  \l 26C93
  \l 26C94
  \l 26C95
  \l 26C96
  \l 26C97
  \l 26C98
  \l 26C99
  \l 26C9A
  \l 26C9B
  \l 26C9C
  \l 26C9D
  \l 26C9E
  \l 26C9F
  \l 26CA0
  \l 26CA1
  \l 26CA2
  \l 26CA3
  \l 26CA4
  \l 26CA5
  \l 26CA6
  \l 26CA7
  \l 26CA8
  \l 26CA9
  \l 26CAA
  \l 26CAB
  \l 26CAC
  \l 26CAD
  \l 26CAE
  \l 26CAF
  \l 26CB0
  \l 26CB1
  \l 26CB2
  \l 26CB3
  \l 26CB4
  \l 26CB5
  \l 26CB6
  \l 26CB7
  \l 26CB8
  \l 26CB9
  \l 26CBA
  \l 26CBB
  \l 26CBC
  \l 26CBD
  \l 26CBE
  \l 26CBF
  \l 26CC0
  \l 26CC1
  \l 26CC2
  \l 26CC3
  \l 26CC4
  \l 26CC5
  \l 26CC6
  \l 26CC7
  \l 26CC8
  \l 26CC9
  \l 26CCA
  \l 26CCB
  \l 26CCC
  \l 26CCD
  \l 26CCE
  \l 26CCF
  \l 26CD0
  \l 26CD1
  \l 26CD2
  \l 26CD3
  \l 26CD4
  \l 26CD5
  \l 26CD6
  \l 26CD7
  \l 26CD8
  \l 26CD9
  \l 26CDA
  \l 26CDB
  \l 26CDC
  \l 26CDD
  \l 26CDE
  \l 26CDF
  \l 26CE0
  \l 26CE1
  \l 26CE2
  \l 26CE3
  \l 26CE4
  \l 26CE5
  \l 26CE6
  \l 26CE7
  \l 26CE8
  \l 26CE9
  \l 26CEA
  \l 26CEB
  \l 26CEC
  \l 26CED
  \l 26CEE
  \l 26CEF
  \l 26CF0
  \l 26CF1
  \l 26CF2
  \l 26CF3
  \l 26CF4
  \l 26CF5
  \l 26CF6
  \l 26CF7
  \l 26CF8
  \l 26CF9
  \l 26CFA
  \l 26CFB
  \l 26CFC
  \l 26CFD
  \l 26CFE
  \l 26CFF
  \l 26D00
  \l 26D01
  \l 26D02
  \l 26D03
  \l 26D04
  \l 26D05
  \l 26D06
  \l 26D07
  \l 26D08
  \l 26D09
  \l 26D0A
  \l 26D0B
  \l 26D0C
  \l 26D0D
  \l 26D0E
  \l 26D0F
  \l 26D10
  \l 26D11
  \l 26D12
  \l 26D13
  \l 26D14
  \l 26D15
  \l 26D16
  \l 26D17
  \l 26D18
  \l 26D19
  \l 26D1A
  \l 26D1B
  \l 26D1C
  \l 26D1D
  \l 26D1E
  \l 26D1F
  \l 26D20
  \l 26D21
  \l 26D22
  \l 26D23
  \l 26D24
  \l 26D25
  \l 26D26
  \l 26D27
  \l 26D28
  \l 26D29
  \l 26D2A
  \l 26D2B
  \l 26D2C
  \l 26D2D
  \l 26D2E
  \l 26D2F
  \l 26D30
  \l 26D31
  \l 26D32
  \l 26D33
  \l 26D34
  \l 26D35
  \l 26D36
  \l 26D37
  \l 26D38
  \l 26D39
  \l 26D3A
  \l 26D3B
  \l 26D3C
  \l 26D3D
  \l 26D3E
  \l 26D3F
  \l 26D40
  \l 26D41
  \l 26D42
  \l 26D43
  \l 26D44
  \l 26D45
  \l 26D46
  \l 26D47
  \l 26D48
  \l 26D49
  \l 26D4A
  \l 26D4B
  \l 26D4C
  \l 26D4D
  \l 26D4E
  \l 26D4F
  \l 26D50
  \l 26D51
  \l 26D52
  \l 26D53
  \l 26D54
  \l 26D55
  \l 26D56
  \l 26D57
  \l 26D58
  \l 26D59
  \l 26D5A
  \l 26D5B
  \l 26D5C
  \l 26D5D
  \l 26D5E
  \l 26D5F
  \l 26D60
  \l 26D61
  \l 26D62
  \l 26D63
  \l 26D64
  \l 26D65
  \l 26D66
  \l 26D67
  \l 26D68
  \l 26D69
  \l 26D6A
  \l 26D6B
  \l 26D6C
  \l 26D6D
  \l 26D6E
  \l 26D6F
  \l 26D70
  \l 26D71
  \l 26D72
  \l 26D73
  \l 26D74
  \l 26D75
  \l 26D76
  \l 26D77
  \l 26D78
  \l 26D79
  \l 26D7A
  \l 26D7B
  \l 26D7C
  \l 26D7D
  \l 26D7E
  \l 26D7F
  \l 26D80
  \l 26D81
  \l 26D82
  \l 26D83
  \l 26D84
  \l 26D85
  \l 26D86
  \l 26D87
  \l 26D88
  \l 26D89
  \l 26D8A
  \l 26D8B
  \l 26D8C
  \l 26D8D
  \l 26D8E
  \l 26D8F
  \l 26D90
  \l 26D91
  \l 26D92
  \l 26D93
  \l 26D94
  \l 26D95
  \l 26D96
  \l 26D97
  \l 26D98
  \l 26D99
  \l 26D9A
  \l 26D9B
  \l 26D9C
  \l 26D9D
  \l 26D9E
  \l 26D9F
  \l 26DA0
  \l 26DA1
  \l 26DA2
  \l 26DA3
  \l 26DA4
  \l 26DA5
  \l 26DA6
  \l 26DA7
  \l 26DA8
  \l 26DA9
  \l 26DAA
  \l 26DAB
  \l 26DAC
  \l 26DAD
  \l 26DAE
  \l 26DAF
  \l 26DB0
  \l 26DB1
  \l 26DB2
  \l 26DB3
  \l 26DB4
  \l 26DB5
  \l 26DB6
  \l 26DB7
  \l 26DB8
  \l 26DB9
  \l 26DBA
  \l 26DBB
  \l 26DBC
  \l 26DBD
  \l 26DBE
  \l 26DBF
  \l 26DC0
  \l 26DC1
  \l 26DC2
  \l 26DC3
  \l 26DC4
  \l 26DC5
  \l 26DC6
  \l 26DC7
  \l 26DC8
  \l 26DC9
  \l 26DCA
  \l 26DCB
  \l 26DCC
  \l 26DCD
  \l 26DCE
  \l 26DCF
  \l 26DD0
  \l 26DD1
  \l 26DD2
  \l 26DD3
  \l 26DD4
  \l 26DD5
  \l 26DD6
  \l 26DD7
  \l 26DD8
  \l 26DD9
  \l 26DDA
  \l 26DDB
  \l 26DDC
  \l 26DDD
  \l 26DDE
  \l 26DDF
  \l 26DE0
  \l 26DE1
  \l 26DE2
  \l 26DE3
  \l 26DE4
  \l 26DE5
  \l 26DE6
  \l 26DE7
  \l 26DE8
  \l 26DE9
  \l 26DEA
  \l 26DEB
  \l 26DEC
  \l 26DED
  \l 26DEE
  \l 26DEF
  \l 26DF0
  \l 26DF1
  \l 26DF2
  \l 26DF3
  \l 26DF4
  \l 26DF5
  \l 26DF6
  \l 26DF7
  \l 26DF8
  \l 26DF9
  \l 26DFA
  \l 26DFB
  \l 26DFC
  \l 26DFD
  \l 26DFE
  \l 26DFF
  \l 26E00
  \l 26E01
  \l 26E02
  \l 26E03
  \l 26E04
  \l 26E05
  \l 26E06
  \l 26E07
  \l 26E08
  \l 26E09
  \l 26E0A
  \l 26E0B
  \l 26E0C
  \l 26E0D
  \l 26E0E
  \l 26E0F
  \l 26E10
  \l 26E11
  \l 26E12
  \l 26E13
  \l 26E14
  \l 26E15
  \l 26E16
  \l 26E17
  \l 26E18
  \l 26E19
  \l 26E1A
  \l 26E1B
  \l 26E1C
  \l 26E1D
  \l 26E1E
  \l 26E1F
  \l 26E20
  \l 26E21
  \l 26E22
  \l 26E23
  \l 26E24
  \l 26E25
  \l 26E26
  \l 26E27
  \l 26E28
  \l 26E29
  \l 26E2A
  \l 26E2B
  \l 26E2C
  \l 26E2D
  \l 26E2E
  \l 26E2F
  \l 26E30
  \l 26E31
  \l 26E32
  \l 26E33
  \l 26E34
  \l 26E35
  \l 26E36
  \l 26E37
  \l 26E38
  \l 26E39
  \l 26E3A
  \l 26E3B
  \l 26E3C
  \l 26E3D
  \l 26E3E
  \l 26E3F
  \l 26E40
  \l 26E41
  \l 26E42
  \l 26E43
  \l 26E44
  \l 26E45
  \l 26E46
  \l 26E47
  \l 26E48
  \l 26E49
  \l 26E4A
  \l 26E4B
  \l 26E4C
  \l 26E4D
  \l 26E4E
  \l 26E4F
  \l 26E50
  \l 26E51
  \l 26E52
  \l 26E53
  \l 26E54
  \l 26E55
  \l 26E56
  \l 26E57
  \l 26E58
  \l 26E59
  \l 26E5A
  \l 26E5B
  \l 26E5C
  \l 26E5D
  \l 26E5E
  \l 26E5F
  \l 26E60
  \l 26E61
  \l 26E62
  \l 26E63
  \l 26E64
  \l 26E65
  \l 26E66
  \l 26E67
  \l 26E68
  \l 26E69
  \l 26E6A
  \l 26E6B
  \l 26E6C
  \l 26E6D
  \l 26E6E
  \l 26E6F
  \l 26E70
  \l 26E71
  \l 26E72
  \l 26E73
  \l 26E74
  \l 26E75
  \l 26E76
  \l 26E77
  \l 26E78
  \l 26E79
  \l 26E7A
  \l 26E7B
  \l 26E7C
  \l 26E7D
  \l 26E7E
  \l 26E7F
  \l 26E80
  \l 26E81
  \l 26E82
  \l 26E83
  \l 26E84
  \l 26E85
  \l 26E86
  \l 26E87
  \l 26E88
  \l 26E89
  \l 26E8A
  \l 26E8B
  \l 26E8C
  \l 26E8D
  \l 26E8E
  \l 26E8F
  \l 26E90
  \l 26E91
  \l 26E92
  \l 26E93
  \l 26E94
  \l 26E95
  \l 26E96
  \l 26E97
  \l 26E98
  \l 26E99
  \l 26E9A
  \l 26E9B
  \l 26E9C
  \l 26E9D
  \l 26E9E
  \l 26E9F
  \l 26EA0
  \l 26EA1
  \l 26EA2
  \l 26EA3
  \l 26EA4
  \l 26EA5
  \l 26EA6
  \l 26EA7
  \l 26EA8
  \l 26EA9
  \l 26EAA
  \l 26EAB
  \l 26EAC
  \l 26EAD
  \l 26EAE
  \l 26EAF
  \l 26EB0
  \l 26EB1
  \l 26EB2
  \l 26EB3
  \l 26EB4
  \l 26EB5
  \l 26EB6
  \l 26EB7
  \l 26EB8
  \l 26EB9
  \l 26EBA
  \l 26EBB
  \l 26EBC
  \l 26EBD
  \l 26EBE
  \l 26EBF
  \l 26EC0
  \l 26EC1
  \l 26EC2
  \l 26EC3
  \l 26EC4
  \l 26EC5
  \l 26EC6
  \l 26EC7
  \l 26EC8
  \l 26EC9
  \l 26ECA
  \l 26ECB
  \l 26ECC
  \l 26ECD
  \l 26ECE
  \l 26ECF
  \l 26ED0
  \l 26ED1
  \l 26ED2
  \l 26ED3
  \l 26ED4
  \l 26ED5
  \l 26ED6
  \l 26ED7
  \l 26ED8
  \l 26ED9
  \l 26EDA
  \l 26EDB
  \l 26EDC
  \l 26EDD
  \l 26EDE
  \l 26EDF
  \l 26EE0
  \l 26EE1
  \l 26EE2
  \l 26EE3
  \l 26EE4
  \l 26EE5
  \l 26EE6
  \l 26EE7
  \l 26EE8
  \l 26EE9
  \l 26EEA
  \l 26EEB
  \l 26EEC
  \l 26EED
  \l 26EEE
  \l 26EEF
  \l 26EF0
  \l 26EF1
  \l 26EF2
  \l 26EF3
  \l 26EF4
  \l 26EF5
  \l 26EF6
  \l 26EF7
  \l 26EF8
  \l 26EF9
  \l 26EFA
  \l 26EFB
  \l 26EFC
  \l 26EFD
  \l 26EFE
  \l 26EFF
  \l 26F00
  \l 26F01
  \l 26F02
  \l 26F03
  \l 26F04
  \l 26F05
  \l 26F06
  \l 26F07
  \l 26F08
  \l 26F09
  \l 26F0A
  \l 26F0B
  \l 26F0C
  \l 26F0D
  \l 26F0E
  \l 26F0F
  \l 26F10
  \l 26F11
  \l 26F12
  \l 26F13
  \l 26F14
  \l 26F15
  \l 26F16
  \l 26F17
  \l 26F18
  \l 26F19
  \l 26F1A
  \l 26F1B
  \l 26F1C
  \l 26F1D
  \l 26F1E
  \l 26F1F
  \l 26F20
  \l 26F21
  \l 26F22
  \l 26F23
  \l 26F24
  \l 26F25
  \l 26F26
  \l 26F27
  \l 26F28
  \l 26F29
  \l 26F2A
  \l 26F2B
  \l 26F2C
  \l 26F2D
  \l 26F2E
  \l 26F2F
  \l 26F30
  \l 26F31
  \l 26F32
  \l 26F33
  \l 26F34
  \l 26F35
  \l 26F36
  \l 26F37
  \l 26F38
  \l 26F39
  \l 26F3A
  \l 26F3B
  \l 26F3C
  \l 26F3D
  \l 26F3E
  \l 26F3F
  \l 26F40
  \l 26F41
  \l 26F42
  \l 26F43
  \l 26F44
  \l 26F45
  \l 26F46
  \l 26F47
  \l 26F48
  \l 26F49
  \l 26F4A
  \l 26F4B
  \l 26F4C
  \l 26F4D
  \l 26F4E
  \l 26F4F
  \l 26F50
  \l 26F51
  \l 26F52
  \l 26F53
  \l 26F54
  \l 26F55
  \l 26F56
  \l 26F57
  \l 26F58
  \l 26F59
  \l 26F5A
  \l 26F5B
  \l 26F5C
  \l 26F5D
  \l 26F5E
  \l 26F5F
  \l 26F60
  \l 26F61
  \l 26F62
  \l 26F63
  \l 26F64
  \l 26F65
  \l 26F66
  \l 26F67
  \l 26F68
  \l 26F69
  \l 26F6A
  \l 26F6B
  \l 26F6C
  \l 26F6D
  \l 26F6E
  \l 26F6F
  \l 26F70
  \l 26F71
  \l 26F72
  \l 26F73
  \l 26F74
  \l 26F75
  \l 26F76
  \l 26F77
  \l 26F78
  \l 26F79
  \l 26F7A
  \l 26F7B
  \l 26F7C
  \l 26F7D
  \l 26F7E
  \l 26F7F
  \l 26F80
  \l 26F81
  \l 26F82
  \l 26F83
  \l 26F84
  \l 26F85
  \l 26F86
  \l 26F87
  \l 26F88
  \l 26F89
  \l 26F8A
  \l 26F8B
  \l 26F8C
  \l 26F8D
  \l 26F8E
  \l 26F8F
  \l 26F90
  \l 26F91
  \l 26F92
  \l 26F93
  \l 26F94
  \l 26F95
  \l 26F96
  \l 26F97
  \l 26F98
  \l 26F99
  \l 26F9A
  \l 26F9B
  \l 26F9C
  \l 26F9D
  \l 26F9E
  \l 26F9F
  \l 26FA0
  \l 26FA1
  \l 26FA2
  \l 26FA3
  \l 26FA4
  \l 26FA5
  \l 26FA6
  \l 26FA7
  \l 26FA8
  \l 26FA9
  \l 26FAA
  \l 26FAB
  \l 26FAC
  \l 26FAD
  \l 26FAE
  \l 26FAF
  \l 26FB0
  \l 26FB1
  \l 26FB2
  \l 26FB3
  \l 26FB4
  \l 26FB5
  \l 26FB6
  \l 26FB7
  \l 26FB8
  \l 26FB9
  \l 26FBA
  \l 26FBB
  \l 26FBC
  \l 26FBD
  \l 26FBE
  \l 26FBF
  \l 26FC0
  \l 26FC1
  \l 26FC2
  \l 26FC3
  \l 26FC4
  \l 26FC5
  \l 26FC6
  \l 26FC7
  \l 26FC8
  \l 26FC9
  \l 26FCA
  \l 26FCB
  \l 26FCC
  \l 26FCD
  \l 26FCE
  \l 26FCF
  \l 26FD0
  \l 26FD1
  \l 26FD2
  \l 26FD3
  \l 26FD4
  \l 26FD5
  \l 26FD6
  \l 26FD7
  \l 26FD8
  \l 26FD9
  \l 26FDA
  \l 26FDB
  \l 26FDC
  \l 26FDD
  \l 26FDE
  \l 26FDF
  \l 26FE0
  \l 26FE1
  \l 26FE2
  \l 26FE3
  \l 26FE4
  \l 26FE5
  \l 26FE6
  \l 26FE7
  \l 26FE8
  \l 26FE9
  \l 26FEA
  \l 26FEB
  \l 26FEC
  \l 26FED
  \l 26FEE
  \l 26FEF
  \l 26FF0
  \l 26FF1
  \l 26FF2
  \l 26FF3
  \l 26FF4
  \l 26FF5
  \l 26FF6
  \l 26FF7
  \l 26FF8
  \l 26FF9
  \l 26FFA
  \l 26FFB
  \l 26FFC
  \l 26FFD
  \l 26FFE
  \l 26FFF
  \l 27000
  \l 27001
  \l 27002
  \l 27003
  \l 27004
  \l 27005
  \l 27006
  \l 27007
  \l 27008
  \l 27009
  \l 2700A
  \l 2700B
  \l 2700C
  \l 2700D
  \l 2700E
  \l 2700F
  \l 27010
  \l 27011
  \l 27012
  \l 27013
  \l 27014
  \l 27015
  \l 27016
  \l 27017
  \l 27018
  \l 27019
  \l 2701A
  \l 2701B
  \l 2701C
  \l 2701D
  \l 2701E
  \l 2701F
  \l 27020
  \l 27021
  \l 27022
  \l 27023
  \l 27024
  \l 27025
  \l 27026
  \l 27027
  \l 27028
  \l 27029
  \l 2702A
  \l 2702B
  \l 2702C
  \l 2702D
  \l 2702E
  \l 2702F
  \l 27030
  \l 27031
  \l 27032
  \l 27033
  \l 27034
  \l 27035
  \l 27036
  \l 27037
  \l 27038
  \l 27039
  \l 2703A
  \l 2703B
  \l 2703C
  \l 2703D
  \l 2703E
  \l 2703F
  \l 27040
  \l 27041
  \l 27042
  \l 27043
  \l 27044
  \l 27045
  \l 27046
  \l 27047
  \l 27048
  \l 27049
  \l 2704A
  \l 2704B
  \l 2704C
  \l 2704D
  \l 2704E
  \l 2704F
  \l 27050
  \l 27051
  \l 27052
  \l 27053
  \l 27054
  \l 27055
  \l 27056
  \l 27057
  \l 27058
  \l 27059
  \l 2705A
  \l 2705B
  \l 2705C
  \l 2705D
  \l 2705E
  \l 2705F
  \l 27060
  \l 27061
  \l 27062
  \l 27063
  \l 27064
  \l 27065
  \l 27066
  \l 27067
  \l 27068
  \l 27069
  \l 2706A
  \l 2706B
  \l 2706C
  \l 2706D
  \l 2706E
  \l 2706F
  \l 27070
  \l 27071
  \l 27072
  \l 27073
  \l 27074
  \l 27075
  \l 27076
  \l 27077
  \l 27078
  \l 27079
  \l 2707A
  \l 2707B
  \l 2707C
  \l 2707D
  \l 2707E
  \l 2707F
  \l 27080
  \l 27081
  \l 27082
  \l 27083
  \l 27084
  \l 27085
  \l 27086
  \l 27087
  \l 27088
  \l 27089
  \l 2708A
  \l 2708B
  \l 2708C
  \l 2708D
  \l 2708E
  \l 2708F
  \l 27090
  \l 27091
  \l 27092
  \l 27093
  \l 27094
  \l 27095
  \l 27096
  \l 27097
  \l 27098
  \l 27099
  \l 2709A
  \l 2709B
  \l 2709C
  \l 2709D
  \l 2709E
  \l 2709F
  \l 270A0
  \l 270A1
  \l 270A2
  \l 270A3
  \l 270A4
  \l 270A5
  \l 270A6
  \l 270A7
  \l 270A8
  \l 270A9
  \l 270AA
  \l 270AB
  \l 270AC
  \l 270AD
  \l 270AE
  \l 270AF
  \l 270B0
  \l 270B1
  \l 270B2
  \l 270B3
  \l 270B4
  \l 270B5
  \l 270B6
  \l 270B7
  \l 270B8
  \l 270B9
  \l 270BA
  \l 270BB
  \l 270BC
  \l 270BD
  \l 270BE
  \l 270BF
  \l 270C0
  \l 270C1
  \l 270C2
  \l 270C3
  \l 270C4
  \l 270C5
  \l 270C6
  \l 270C7
  \l 270C8
  \l 270C9
  \l 270CA
  \l 270CB
  \l 270CC
  \l 270CD
  \l 270CE
  \l 270CF
  \l 270D0
  \l 270D1
  \l 270D2
  \l 270D3
  \l 270D4
  \l 270D5
  \l 270D6
  \l 270D7
  \l 270D8
  \l 270D9
  \l 270DA
  \l 270DB
  \l 270DC
  \l 270DD
  \l 270DE
  \l 270DF
  \l 270E0
  \l 270E1
  \l 270E2
  \l 270E3
  \l 270E4
  \l 270E5
  \l 270E6
  \l 270E7
  \l 270E8
  \l 270E9
  \l 270EA
  \l 270EB
  \l 270EC
  \l 270ED
  \l 270EE
  \l 270EF
  \l 270F0
  \l 270F1
  \l 270F2
  \l 270F3
  \l 270F4
  \l 270F5
  \l 270F6
  \l 270F7
  \l 270F8
  \l 270F9
  \l 270FA
  \l 270FB
  \l 270FC
  \l 270FD
  \l 270FE
  \l 270FF
  \l 27100
  \l 27101
  \l 27102
  \l 27103
  \l 27104
  \l 27105
  \l 27106
  \l 27107
  \l 27108
  \l 27109
  \l 2710A
  \l 2710B
  \l 2710C
  \l 2710D
  \l 2710E
  \l 2710F
  \l 27110
  \l 27111
  \l 27112
  \l 27113
  \l 27114
  \l 27115
  \l 27116
  \l 27117
  \l 27118
  \l 27119
  \l 2711A
  \l 2711B
  \l 2711C
  \l 2711D
  \l 2711E
  \l 2711F
  \l 27120
  \l 27121
  \l 27122
  \l 27123
  \l 27124
  \l 27125
  \l 27126
  \l 27127
  \l 27128
  \l 27129
  \l 2712A
  \l 2712B
  \l 2712C
  \l 2712D
  \l 2712E
  \l 2712F
  \l 27130
  \l 27131
  \l 27132
  \l 27133
  \l 27134
  \l 27135
  \l 27136
  \l 27137
  \l 27138
  \l 27139
  \l 2713A
  \l 2713B
  \l 2713C
  \l 2713D
  \l 2713E
  \l 2713F
  \l 27140
  \l 27141
  \l 27142
  \l 27143
  \l 27144
  \l 27145
  \l 27146
  \l 27147
  \l 27148
  \l 27149
  \l 2714A
  \l 2714B
  \l 2714C
  \l 2714D
  \l 2714E
  \l 2714F
  \l 27150
  \l 27151
  \l 27152
  \l 27153
  \l 27154
  \l 27155
  \l 27156
  \l 27157
  \l 27158
  \l 27159
  \l 2715A
  \l 2715B
  \l 2715C
  \l 2715D
  \l 2715E
  \l 2715F
  \l 27160
  \l 27161
  \l 27162
  \l 27163
  \l 27164
  \l 27165
  \l 27166
  \l 27167
  \l 27168
  \l 27169
  \l 2716A
  \l 2716B
  \l 2716C
  \l 2716D
  \l 2716E
  \l 2716F
  \l 27170
  \l 27171
  \l 27172
  \l 27173
  \l 27174
  \l 27175
  \l 27176
  \l 27177
  \l 27178
  \l 27179
  \l 2717A
  \l 2717B
  \l 2717C
  \l 2717D
  \l 2717E
  \l 2717F
  \l 27180
  \l 27181
  \l 27182
  \l 27183
  \l 27184
  \l 27185
  \l 27186
  \l 27187
  \l 27188
  \l 27189
  \l 2718A
  \l 2718B
  \l 2718C
  \l 2718D
  \l 2718E
  \l 2718F
  \l 27190
  \l 27191
  \l 27192
  \l 27193
  \l 27194
  \l 27195
  \l 27196
  \l 27197
  \l 27198
  \l 27199
  \l 2719A
  \l 2719B
  \l 2719C
  \l 2719D
  \l 2719E
  \l 2719F
  \l 271A0
  \l 271A1
  \l 271A2
  \l 271A3
  \l 271A4
  \l 271A5
  \l 271A6
  \l 271A7
  \l 271A8
  \l 271A9
  \l 271AA
  \l 271AB
  \l 271AC
  \l 271AD
  \l 271AE
  \l 271AF
  \l 271B0
  \l 271B1
  \l 271B2
  \l 271B3
  \l 271B4
  \l 271B5
  \l 271B6
  \l 271B7
  \l 271B8
  \l 271B9
  \l 271BA
  \l 271BB
  \l 271BC
  \l 271BD
  \l 271BE
  \l 271BF
  \l 271C0
  \l 271C1
  \l 271C2
  \l 271C3
  \l 271C4
  \l 271C5
  \l 271C6
  \l 271C7
  \l 271C8
  \l 271C9
  \l 271CA
  \l 271CB
  \l 271CC
  \l 271CD
  \l 271CE
  \l 271CF
  \l 271D0
  \l 271D1
  \l 271D2
  \l 271D3
  \l 271D4
  \l 271D5
  \l 271D6
  \l 271D7
  \l 271D8
  \l 271D9
  \l 271DA
  \l 271DB
  \l 271DC
  \l 271DD
  \l 271DE
  \l 271DF
  \l 271E0
  \l 271E1
  \l 271E2
  \l 271E3
  \l 271E4
  \l 271E5
  \l 271E6
  \l 271E7
  \l 271E8
  \l 271E9
  \l 271EA
  \l 271EB
  \l 271EC
  \l 271ED
  \l 271EE
  \l 271EF
  \l 271F0
  \l 271F1
  \l 271F2
  \l 271F3
  \l 271F4
  \l 271F5
  \l 271F6
  \l 271F7
  \l 271F8
  \l 271F9
  \l 271FA
  \l 271FB
  \l 271FC
  \l 271FD
  \l 271FE
  \l 271FF
  \l 27200
  \l 27201
  \l 27202
  \l 27203
  \l 27204
  \l 27205
  \l 27206
  \l 27207
  \l 27208
  \l 27209
  \l 2720A
  \l 2720B
  \l 2720C
  \l 2720D
  \l 2720E
  \l 2720F
  \l 27210
  \l 27211
  \l 27212
  \l 27213
  \l 27214
  \l 27215
  \l 27216
  \l 27217
  \l 27218
  \l 27219
  \l 2721A
  \l 2721B
  \l 2721C
  \l 2721D
  \l 2721E
  \l 2721F
  \l 27220
  \l 27221
  \l 27222
  \l 27223
  \l 27224
  \l 27225
  \l 27226
  \l 27227
  \l 27228
  \l 27229
  \l 2722A
  \l 2722B
  \l 2722C
  \l 2722D
  \l 2722E
  \l 2722F
  \l 27230
  \l 27231
  \l 27232
  \l 27233
  \l 27234
  \l 27235
  \l 27236
  \l 27237
  \l 27238
  \l 27239
  \l 2723A
  \l 2723B
  \l 2723C
  \l 2723D
  \l 2723E
  \l 2723F
  \l 27240
  \l 27241
  \l 27242
  \l 27243
  \l 27244
  \l 27245
  \l 27246
  \l 27247
  \l 27248
  \l 27249
  \l 2724A
  \l 2724B
  \l 2724C
  \l 2724D
  \l 2724E
  \l 2724F
  \l 27250
  \l 27251
  \l 27252
  \l 27253
  \l 27254
  \l 27255
  \l 27256
  \l 27257
  \l 27258
  \l 27259
  \l 2725A
  \l 2725B
  \l 2725C
  \l 2725D
  \l 2725E
  \l 2725F
  \l 27260
  \l 27261
  \l 27262
  \l 27263
  \l 27264
  \l 27265
  \l 27266
  \l 27267
  \l 27268
  \l 27269
  \l 2726A
  \l 2726B
  \l 2726C
  \l 2726D
  \l 2726E
  \l 2726F
  \l 27270
  \l 27271
  \l 27272
  \l 27273
  \l 27274
  \l 27275
  \l 27276
  \l 27277
  \l 27278
  \l 27279
  \l 2727A
  \l 2727B
  \l 2727C
  \l 2727D
  \l 2727E
  \l 2727F
  \l 27280
  \l 27281
  \l 27282
  \l 27283
  \l 27284
  \l 27285
  \l 27286
  \l 27287
  \l 27288
  \l 27289
  \l 2728A
  \l 2728B
  \l 2728C
  \l 2728D
  \l 2728E
  \l 2728F
  \l 27290
  \l 27291
  \l 27292
  \l 27293
  \l 27294
  \l 27295
  \l 27296
  \l 27297
  \l 27298
  \l 27299
  \l 2729A
  \l 2729B
  \l 2729C
  \l 2729D
  \l 2729E
  \l 2729F
  \l 272A0
  \l 272A1
  \l 272A2
  \l 272A3
  \l 272A4
  \l 272A5
  \l 272A6
  \l 272A7
  \l 272A8
  \l 272A9
  \l 272AA
  \l 272AB
  \l 272AC
  \l 272AD
  \l 272AE
  \l 272AF
  \l 272B0
  \l 272B1
  \l 272B2
  \l 272B3
  \l 272B4
  \l 272B5
  \l 272B6
  \l 272B7
  \l 272B8
  \l 272B9
  \l 272BA
  \l 272BB
  \l 272BC
  \l 272BD
  \l 272BE
  \l 272BF
  \l 272C0
  \l 272C1
  \l 272C2
  \l 272C3
  \l 272C4
  \l 272C5
  \l 272C6
  \l 272C7
  \l 272C8
  \l 272C9
  \l 272CA
  \l 272CB
  \l 272CC
  \l 272CD
  \l 272CE
  \l 272CF
  \l 272D0
  \l 272D1
  \l 272D2
  \l 272D3
  \l 272D4
  \l 272D5
  \l 272D6
  \l 272D7
  \l 272D8
  \l 272D9
  \l 272DA
  \l 272DB
  \l 272DC
  \l 272DD
  \l 272DE
  \l 272DF
  \l 272E0
  \l 272E1
  \l 272E2
  \l 272E3
  \l 272E4
  \l 272E5
  \l 272E6
  \l 272E7
  \l 272E8
  \l 272E9
  \l 272EA
  \l 272EB
  \l 272EC
  \l 272ED
  \l 272EE
  \l 272EF
  \l 272F0
  \l 272F1
  \l 272F2
  \l 272F3
  \l 272F4
  \l 272F5
  \l 272F6
  \l 272F7
  \l 272F8
  \l 272F9
  \l 272FA
  \l 272FB
  \l 272FC
  \l 272FD
  \l 272FE
  \l 272FF
  \l 27300
  \l 27301
  \l 27302
  \l 27303
  \l 27304
  \l 27305
  \l 27306
  \l 27307
  \l 27308
  \l 27309
  \l 2730A
  \l 2730B
  \l 2730C
  \l 2730D
  \l 2730E
  \l 2730F
  \l 27310
  \l 27311
  \l 27312
  \l 27313
  \l 27314
  \l 27315
  \l 27316
  \l 27317
  \l 27318
  \l 27319
  \l 2731A
  \l 2731B
  \l 2731C
  \l 2731D
  \l 2731E
  \l 2731F
  \l 27320
  \l 27321
  \l 27322
  \l 27323
  \l 27324
  \l 27325
  \l 27326
  \l 27327
  \l 27328
  \l 27329
  \l 2732A
  \l 2732B
  \l 2732C
  \l 2732D
  \l 2732E
  \l 2732F
  \l 27330
  \l 27331
  \l 27332
  \l 27333
  \l 27334
  \l 27335
  \l 27336
  \l 27337
  \l 27338
  \l 27339
  \l 2733A
  \l 2733B
  \l 2733C
  \l 2733D
  \l 2733E
  \l 2733F
  \l 27340
  \l 27341
  \l 27342
  \l 27343
  \l 27344
  \l 27345
  \l 27346
  \l 27347
  \l 27348
  \l 27349
  \l 2734A
  \l 2734B
  \l 2734C
  \l 2734D
  \l 2734E
  \l 2734F
  \l 27350
  \l 27351
  \l 27352
  \l 27353
  \l 27354
  \l 27355
  \l 27356
  \l 27357
  \l 27358
  \l 27359
  \l 2735A
  \l 2735B
  \l 2735C
  \l 2735D
  \l 2735E
  \l 2735F
  \l 27360
  \l 27361
  \l 27362
  \l 27363
  \l 27364
  \l 27365
  \l 27366
  \l 27367
  \l 27368
  \l 27369
  \l 2736A
  \l 2736B
  \l 2736C
  \l 2736D
  \l 2736E
  \l 2736F
  \l 27370
  \l 27371
  \l 27372
  \l 27373
  \l 27374
  \l 27375
  \l 27376
  \l 27377
  \l 27378
  \l 27379
  \l 2737A
  \l 2737B
  \l 2737C
  \l 2737D
  \l 2737E
  \l 2737F
  \l 27380
  \l 27381
  \l 27382
  \l 27383
  \l 27384
  \l 27385
  \l 27386
  \l 27387
  \l 27388
  \l 27389
  \l 2738A
  \l 2738B
  \l 2738C
  \l 2738D
  \l 2738E
  \l 2738F
  \l 27390
  \l 27391
  \l 27392
  \l 27393
  \l 27394
  \l 27395
  \l 27396
  \l 27397
  \l 27398
  \l 27399
  \l 2739A
  \l 2739B
  \l 2739C
  \l 2739D
  \l 2739E
  \l 2739F
  \l 273A0
  \l 273A1
  \l 273A2
  \l 273A3
  \l 273A4
  \l 273A5
  \l 273A6
  \l 273A7
  \l 273A8
  \l 273A9
  \l 273AA
  \l 273AB
  \l 273AC
  \l 273AD
  \l 273AE
  \l 273AF
  \l 273B0
  \l 273B1
  \l 273B2
  \l 273B3
  \l 273B4
  \l 273B5
  \l 273B6
  \l 273B7
  \l 273B8
  \l 273B9
  \l 273BA
  \l 273BB
  \l 273BC
  \l 273BD
  \l 273BE
  \l 273BF
  \l 273C0
  \l 273C1
  \l 273C2
  \l 273C3
  \l 273C4
  \l 273C5
  \l 273C6
  \l 273C7
  \l 273C8
  \l 273C9
  \l 273CA
  \l 273CB
  \l 273CC
  \l 273CD
  \l 273CE
  \l 273CF
  \l 273D0
  \l 273D1
  \l 273D2
  \l 273D3
  \l 273D4
  \l 273D5
  \l 273D6
  \l 273D7
  \l 273D8
  \l 273D9
  \l 273DA
  \l 273DB
  \l 273DC
  \l 273DD
  \l 273DE
  \l 273DF
  \l 273E0
  \l 273E1
  \l 273E2
  \l 273E3
  \l 273E4
  \l 273E5
  \l 273E6
  \l 273E7
  \l 273E8
  \l 273E9
  \l 273EA
  \l 273EB
  \l 273EC
  \l 273ED
  \l 273EE
  \l 273EF
  \l 273F0
  \l 273F1
  \l 273F2
  \l 273F3
  \l 273F4
  \l 273F5
  \l 273F6
  \l 273F7
  \l 273F8
  \l 273F9
  \l 273FA
  \l 273FB
  \l 273FC
  \l 273FD
  \l 273FE
  \l 273FF
  \l 27400
  \l 27401
  \l 27402
  \l 27403
  \l 27404
  \l 27405
  \l 27406
  \l 27407
  \l 27408
  \l 27409
  \l 2740A
  \l 2740B
  \l 2740C
  \l 2740D
  \l 2740E
  \l 2740F
  \l 27410
  \l 27411
  \l 27412
  \l 27413
  \l 27414
  \l 27415
  \l 27416
  \l 27417
  \l 27418
  \l 27419
  \l 2741A
  \l 2741B
  \l 2741C
  \l 2741D
  \l 2741E
  \l 2741F
  \l 27420
  \l 27421
  \l 27422
  \l 27423
  \l 27424
  \l 27425
  \l 27426
  \l 27427
  \l 27428
  \l 27429
  \l 2742A
  \l 2742B
  \l 2742C
  \l 2742D
  \l 2742E
  \l 2742F
  \l 27430
  \l 27431
  \l 27432
  \l 27433
  \l 27434
  \l 27435
  \l 27436
  \l 27437
  \l 27438
  \l 27439
  \l 2743A
  \l 2743B
  \l 2743C
  \l 2743D
  \l 2743E
  \l 2743F
  \l 27440
  \l 27441
  \l 27442
  \l 27443
  \l 27444
  \l 27445
  \l 27446
  \l 27447
  \l 27448
  \l 27449
  \l 2744A
  \l 2744B
  \l 2744C
  \l 2744D
  \l 2744E
  \l 2744F
  \l 27450
  \l 27451
  \l 27452
  \l 27453
  \l 27454
  \l 27455
  \l 27456
  \l 27457
  \l 27458
  \l 27459
  \l 2745A
  \l 2745B
  \l 2745C
  \l 2745D
  \l 2745E
  \l 2745F
  \l 27460
  \l 27461
  \l 27462
  \l 27463
  \l 27464
  \l 27465
  \l 27466
  \l 27467
  \l 27468
  \l 27469
  \l 2746A
  \l 2746B
  \l 2746C
  \l 2746D
  \l 2746E
  \l 2746F
  \l 27470
  \l 27471
  \l 27472
  \l 27473
  \l 27474
  \l 27475
  \l 27476
  \l 27477
  \l 27478
  \l 27479
  \l 2747A
  \l 2747B
  \l 2747C
  \l 2747D
  \l 2747E
  \l 2747F
  \l 27480
  \l 27481
  \l 27482
  \l 27483
  \l 27484
  \l 27485
  \l 27486
  \l 27487
  \l 27488
  \l 27489
  \l 2748A
  \l 2748B
  \l 2748C
  \l 2748D
  \l 2748E
  \l 2748F
  \l 27490
  \l 27491
  \l 27492
  \l 27493
  \l 27494
  \l 27495
  \l 27496
  \l 27497
  \l 27498
  \l 27499
  \l 2749A
  \l 2749B
  \l 2749C
  \l 2749D
  \l 2749E
  \l 2749F
  \l 274A0
  \l 274A1
  \l 274A2
  \l 274A3
  \l 274A4
  \l 274A5
  \l 274A6
  \l 274A7
  \l 274A8
  \l 274A9
  \l 274AA
  \l 274AB
  \l 274AC
  \l 274AD
  \l 274AE
  \l 274AF
  \l 274B0
  \l 274B1
  \l 274B2
  \l 274B3
  \l 274B4
  \l 274B5
  \l 274B6
  \l 274B7
  \l 274B8
  \l 274B9
  \l 274BA
  \l 274BB
  \l 274BC
  \l 274BD
  \l 274BE
  \l 274BF
  \l 274C0
  \l 274C1
  \l 274C2
  \l 274C3
  \l 274C4
  \l 274C5
  \l 274C6
  \l 274C7
  \l 274C8
  \l 274C9
  \l 274CA
  \l 274CB
  \l 274CC
  \l 274CD
  \l 274CE
  \l 274CF
  \l 274D0
  \l 274D1
  \l 274D2
  \l 274D3
  \l 274D4
  \l 274D5
  \l 274D6
  \l 274D7
  \l 274D8
  \l 274D9
  \l 274DA
  \l 274DB
  \l 274DC
  \l 274DD
  \l 274DE
  \l 274DF
  \l 274E0
  \l 274E1
  \l 274E2
  \l 274E3
  \l 274E4
  \l 274E5
  \l 274E6
  \l 274E7
  \l 274E8
  \l 274E9
  \l 274EA
  \l 274EB
  \l 274EC
  \l 274ED
  \l 274EE
  \l 274EF
  \l 274F0
  \l 274F1
  \l 274F2
  \l 274F3
  \l 274F4
  \l 274F5
  \l 274F6
  \l 274F7
  \l 274F8
  \l 274F9
  \l 274FA
  \l 274FB
  \l 274FC
  \l 274FD
  \l 274FE
  \l 274FF
  \l 27500
  \l 27501
  \l 27502
  \l 27503
  \l 27504
  \l 27505
  \l 27506
  \l 27507
  \l 27508
  \l 27509
  \l 2750A
  \l 2750B
  \l 2750C
  \l 2750D
  \l 2750E
  \l 2750F
  \l 27510
  \l 27511
  \l 27512
  \l 27513
  \l 27514
  \l 27515
  \l 27516
  \l 27517
  \l 27518
  \l 27519
  \l 2751A
  \l 2751B
  \l 2751C
  \l 2751D
  \l 2751E
  \l 2751F
  \l 27520
  \l 27521
  \l 27522
  \l 27523
  \l 27524
  \l 27525
  \l 27526
  \l 27527
  \l 27528
  \l 27529
  \l 2752A
  \l 2752B
  \l 2752C
  \l 2752D
  \l 2752E
  \l 2752F
  \l 27530
  \l 27531
  \l 27532
  \l 27533
  \l 27534
  \l 27535
  \l 27536
  \l 27537
  \l 27538
  \l 27539
  \l 2753A
  \l 2753B
  \l 2753C
  \l 2753D
  \l 2753E
  \l 2753F
  \l 27540
  \l 27541
  \l 27542
  \l 27543
  \l 27544
  \l 27545
  \l 27546
  \l 27547
  \l 27548
  \l 27549
  \l 2754A
  \l 2754B
  \l 2754C
  \l 2754D
  \l 2754E
  \l 2754F
  \l 27550
  \l 27551
  \l 27552
  \l 27553
  \l 27554
  \l 27555
  \l 27556
  \l 27557
  \l 27558
  \l 27559
  \l 2755A
  \l 2755B
  \l 2755C
  \l 2755D
  \l 2755E
  \l 2755F
  \l 27560
  \l 27561
  \l 27562
  \l 27563
  \l 27564
  \l 27565
  \l 27566
  \l 27567
  \l 27568
  \l 27569
  \l 2756A
  \l 2756B
  \l 2756C
  \l 2756D
  \l 2756E
  \l 2756F
  \l 27570
  \l 27571
  \l 27572
  \l 27573
  \l 27574
  \l 27575
  \l 27576
  \l 27577
  \l 27578
  \l 27579
  \l 2757A
  \l 2757B
  \l 2757C
  \l 2757D
  \l 2757E
  \l 2757F
  \l 27580
  \l 27581
  \l 27582
  \l 27583
  \l 27584
  \l 27585
  \l 27586
  \l 27587
  \l 27588
  \l 27589
  \l 2758A
  \l 2758B
  \l 2758C
  \l 2758D
  \l 2758E
  \l 2758F
  \l 27590
  \l 27591
  \l 27592
  \l 27593
  \l 27594
  \l 27595
  \l 27596
  \l 27597
  \l 27598
  \l 27599
  \l 2759A
  \l 2759B
  \l 2759C
  \l 2759D
  \l 2759E
  \l 2759F
  \l 275A0
  \l 275A1
  \l 275A2
  \l 275A3
  \l 275A4
  \l 275A5
  \l 275A6
  \l 275A7
  \l 275A8
  \l 275A9
  \l 275AA
  \l 275AB
  \l 275AC
  \l 275AD
  \l 275AE
  \l 275AF
  \l 275B0
  \l 275B1
  \l 275B2
  \l 275B3
  \l 275B4
  \l 275B5
  \l 275B6
  \l 275B7
  \l 275B8
  \l 275B9
  \l 275BA
  \l 275BB
  \l 275BC
  \l 275BD
  \l 275BE
  \l 275BF
  \l 275C0
  \l 275C1
  \l 275C2
  \l 275C3
  \l 275C4
  \l 275C5
  \l 275C6
  \l 275C7
  \l 275C8
  \l 275C9
  \l 275CA
  \l 275CB
  \l 275CC
  \l 275CD
  \l 275CE
  \l 275CF
  \l 275D0
  \l 275D1
  \l 275D2
  \l 275D3
  \l 275D4
  \l 275D5
  \l 275D6
  \l 275D7
  \l 275D8
  \l 275D9
  \l 275DA
  \l 275DB
  \l 275DC
  \l 275DD
  \l 275DE
  \l 275DF
  \l 275E0
  \l 275E1
  \l 275E2
  \l 275E3
  \l 275E4
  \l 275E5
  \l 275E6
  \l 275E7
  \l 275E8
  \l 275E9
  \l 275EA
  \l 275EB
  \l 275EC
  \l 275ED
  \l 275EE
  \l 275EF
  \l 275F0
  \l 275F1
  \l 275F2
  \l 275F3
  \l 275F4
  \l 275F5
  \l 275F6
  \l 275F7
  \l 275F8
  \l 275F9
  \l 275FA
  \l 275FB
  \l 275FC
  \l 275FD
  \l 275FE
  \l 275FF
  \l 27600
  \l 27601
  \l 27602
  \l 27603
  \l 27604
  \l 27605
  \l 27606
  \l 27607
  \l 27608
  \l 27609
  \l 2760A
  \l 2760B
  \l 2760C
  \l 2760D
  \l 2760E
  \l 2760F
  \l 27610
  \l 27611
  \l 27612
  \l 27613
  \l 27614
  \l 27615
  \l 27616
  \l 27617
  \l 27618
  \l 27619
  \l 2761A
  \l 2761B
  \l 2761C
  \l 2761D
  \l 2761E
  \l 2761F
  \l 27620
  \l 27621
  \l 27622
  \l 27623
  \l 27624
  \l 27625
  \l 27626
  \l 27627
  \l 27628
  \l 27629
  \l 2762A
  \l 2762B
  \l 2762C
  \l 2762D
  \l 2762E
  \l 2762F
  \l 27630
  \l 27631
  \l 27632
  \l 27633
  \l 27634
  \l 27635
  \l 27636
  \l 27637
  \l 27638
  \l 27639
  \l 2763A
  \l 2763B
  \l 2763C
  \l 2763D
  \l 2763E
  \l 2763F
  \l 27640
  \l 27641
  \l 27642
  \l 27643
  \l 27644
  \l 27645
  \l 27646
  \l 27647
  \l 27648
  \l 27649
  \l 2764A
  \l 2764B
  \l 2764C
  \l 2764D
  \l 2764E
  \l 2764F
  \l 27650
  \l 27651
  \l 27652
  \l 27653
  \l 27654
  \l 27655
  \l 27656
  \l 27657
  \l 27658
  \l 27659
  \l 2765A
  \l 2765B
  \l 2765C
  \l 2765D
  \l 2765E
  \l 2765F
  \l 27660
  \l 27661
  \l 27662
  \l 27663
  \l 27664
  \l 27665
  \l 27666
  \l 27667
  \l 27668
  \l 27669
  \l 2766A
  \l 2766B
  \l 2766C
  \l 2766D
  \l 2766E
  \l 2766F
  \l 27670
  \l 27671
  \l 27672
  \l 27673
  \l 27674
  \l 27675
  \l 27676
  \l 27677
  \l 27678
  \l 27679
  \l 2767A
  \l 2767B
  \l 2767C
  \l 2767D
  \l 2767E
  \l 2767F
  \l 27680
  \l 27681
  \l 27682
  \l 27683
  \l 27684
  \l 27685
  \l 27686
  \l 27687
  \l 27688
  \l 27689
  \l 2768A
  \l 2768B
  \l 2768C
  \l 2768D
  \l 2768E
  \l 2768F
  \l 27690
  \l 27691
  \l 27692
  \l 27693
  \l 27694
  \l 27695
  \l 27696
  \l 27697
  \l 27698
  \l 27699
  \l 2769A
  \l 2769B
  \l 2769C
  \l 2769D
  \l 2769E
  \l 2769F
  \l 276A0
  \l 276A1
  \l 276A2
  \l 276A3
  \l 276A4
  \l 276A5
  \l 276A6
  \l 276A7
  \l 276A8
  \l 276A9
  \l 276AA
  \l 276AB
  \l 276AC
  \l 276AD
  \l 276AE
  \l 276AF
  \l 276B0
  \l 276B1
  \l 276B2
  \l 276B3
  \l 276B4
  \l 276B5
  \l 276B6
  \l 276B7
  \l 276B8
  \l 276B9
  \l 276BA
  \l 276BB
  \l 276BC
  \l 276BD
  \l 276BE
  \l 276BF
  \l 276C0
  \l 276C1
  \l 276C2
  \l 276C3
  \l 276C4
  \l 276C5
  \l 276C6
  \l 276C7
  \l 276C8
  \l 276C9
  \l 276CA
  \l 276CB
  \l 276CC
  \l 276CD
  \l 276CE
  \l 276CF
  \l 276D0
  \l 276D1
  \l 276D2
  \l 276D3
  \l 276D4
  \l 276D5
  \l 276D6
  \l 276D7
  \l 276D8
  \l 276D9
  \l 276DA
  \l 276DB
  \l 276DC
  \l 276DD
  \l 276DE
  \l 276DF
  \l 276E0
  \l 276E1
  \l 276E2
  \l 276E3
  \l 276E4
  \l 276E5
  \l 276E6
  \l 276E7
  \l 276E8
  \l 276E9
  \l 276EA
  \l 276EB
  \l 276EC
  \l 276ED
  \l 276EE
  \l 276EF
  \l 276F0
  \l 276F1
  \l 276F2
  \l 276F3
  \l 276F4
  \l 276F5
  \l 276F6
  \l 276F7
  \l 276F8
  \l 276F9
  \l 276FA
  \l 276FB
  \l 276FC
  \l 276FD
  \l 276FE
  \l 276FF
  \l 27700
  \l 27701
  \l 27702
  \l 27703
  \l 27704
  \l 27705
  \l 27706
  \l 27707
  \l 27708
  \l 27709
  \l 2770A
  \l 2770B
  \l 2770C
  \l 2770D
  \l 2770E
  \l 2770F
  \l 27710
  \l 27711
  \l 27712
  \l 27713
  \l 27714
  \l 27715
  \l 27716
  \l 27717
  \l 27718
  \l 27719
  \l 2771A
  \l 2771B
  \l 2771C
  \l 2771D
  \l 2771E
  \l 2771F
  \l 27720
  \l 27721
  \l 27722
  \l 27723
  \l 27724
  \l 27725
  \l 27726
  \l 27727
  \l 27728
  \l 27729
  \l 2772A
  \l 2772B
  \l 2772C
  \l 2772D
  \l 2772E
  \l 2772F
  \l 27730
  \l 27731
  \l 27732
  \l 27733
  \l 27734
  \l 27735
  \l 27736
  \l 27737
  \l 27738
  \l 27739
  \l 2773A
  \l 2773B
  \l 2773C
  \l 2773D
  \l 2773E
  \l 2773F
  \l 27740
  \l 27741
  \l 27742
  \l 27743
  \l 27744
  \l 27745
  \l 27746
  \l 27747
  \l 27748
  \l 27749
  \l 2774A
  \l 2774B
  \l 2774C
  \l 2774D
  \l 2774E
  \l 2774F
  \l 27750
  \l 27751
  \l 27752
  \l 27753
  \l 27754
  \l 27755
  \l 27756
  \l 27757
  \l 27758
  \l 27759
  \l 2775A
  \l 2775B
  \l 2775C
  \l 2775D
  \l 2775E
  \l 2775F
  \l 27760
  \l 27761
  \l 27762
  \l 27763
  \l 27764
  \l 27765
  \l 27766
  \l 27767
  \l 27768
  \l 27769
  \l 2776A
  \l 2776B
  \l 2776C
  \l 2776D
  \l 2776E
  \l 2776F
  \l 27770
  \l 27771
  \l 27772
  \l 27773
  \l 27774
  \l 27775
  \l 27776
  \l 27777
  \l 27778
  \l 27779
  \l 2777A
  \l 2777B
  \l 2777C
  \l 2777D
  \l 2777E
  \l 2777F
  \l 27780
  \l 27781
  \l 27782
  \l 27783
  \l 27784
  \l 27785
  \l 27786
  \l 27787
  \l 27788
  \l 27789
  \l 2778A
  \l 2778B
  \l 2778C
  \l 2778D
  \l 2778E
  \l 2778F
  \l 27790
  \l 27791
  \l 27792
  \l 27793
  \l 27794
  \l 27795
  \l 27796
  \l 27797
  \l 27798
  \l 27799
  \l 2779A
  \l 2779B
  \l 2779C
  \l 2779D
  \l 2779E
  \l 2779F
  \l 277A0
  \l 277A1
  \l 277A2
  \l 277A3
  \l 277A4
  \l 277A5
  \l 277A6
  \l 277A7
  \l 277A8
  \l 277A9
  \l 277AA
  \l 277AB
  \l 277AC
  \l 277AD
  \l 277AE
  \l 277AF
  \l 277B0
  \l 277B1
  \l 277B2
  \l 277B3
  \l 277B4
  \l 277B5
  \l 277B6
  \l 277B7
  \l 277B8
  \l 277B9
  \l 277BA
  \l 277BB
  \l 277BC
  \l 277BD
  \l 277BE
  \l 277BF
  \l 277C0
  \l 277C1
  \l 277C2
  \l 277C3
  \l 277C4
  \l 277C5
  \l 277C6
  \l 277C7
  \l 277C8
  \l 277C9
  \l 277CA
  \l 277CB
  \l 277CC
  \l 277CD
  \l 277CE
  \l 277CF
  \l 277D0
  \l 277D1
  \l 277D2
  \l 277D3
  \l 277D4
  \l 277D5
  \l 277D6
  \l 277D7
  \l 277D8
  \l 277D9
  \l 277DA
  \l 277DB
  \l 277DC
  \l 277DD
  \l 277DE
  \l 277DF
  \l 277E0
  \l 277E1
  \l 277E2
  \l 277E3
  \l 277E4
  \l 277E5
  \l 277E6
  \l 277E7
  \l 277E8
  \l 277E9
  \l 277EA
  \l 277EB
  \l 277EC
  \l 277ED
  \l 277EE
  \l 277EF
  \l 277F0
  \l 277F1
  \l 277F2
  \l 277F3
  \l 277F4
  \l 277F5
  \l 277F6
  \l 277F7
  \l 277F8
  \l 277F9
  \l 277FA
  \l 277FB
  \l 277FC
  \l 277FD
  \l 277FE
  \l 277FF
  \l 27800
  \l 27801
  \l 27802
  \l 27803
  \l 27804
  \l 27805
  \l 27806
  \l 27807
  \l 27808
  \l 27809
  \l 2780A
  \l 2780B
  \l 2780C
  \l 2780D
  \l 2780E
  \l 2780F
  \l 27810
  \l 27811
  \l 27812
  \l 27813
  \l 27814
  \l 27815
  \l 27816
  \l 27817
  \l 27818
  \l 27819
  \l 2781A
  \l 2781B
  \l 2781C
  \l 2781D
  \l 2781E
  \l 2781F
  \l 27820
  \l 27821
  \l 27822
  \l 27823
  \l 27824
  \l 27825
  \l 27826
  \l 27827
  \l 27828
  \l 27829
  \l 2782A
  \l 2782B
  \l 2782C
  \l 2782D
  \l 2782E
  \l 2782F
  \l 27830
  \l 27831
  \l 27832
  \l 27833
  \l 27834
  \l 27835
  \l 27836
  \l 27837
  \l 27838
  \l 27839
  \l 2783A
  \l 2783B
  \l 2783C
  \l 2783D
  \l 2783E
  \l 2783F
  \l 27840
  \l 27841
  \l 27842
  \l 27843
  \l 27844
  \l 27845
  \l 27846
  \l 27847
  \l 27848
  \l 27849
  \l 2784A
  \l 2784B
  \l 2784C
  \l 2784D
  \l 2784E
  \l 2784F
  \l 27850
  \l 27851
  \l 27852
  \l 27853
  \l 27854
  \l 27855
  \l 27856
  \l 27857
  \l 27858
  \l 27859
  \l 2785A
  \l 2785B
  \l 2785C
  \l 2785D
  \l 2785E
  \l 2785F
  \l 27860
  \l 27861
  \l 27862
  \l 27863
  \l 27864
  \l 27865
  \l 27866
  \l 27867
  \l 27868
  \l 27869
  \l 2786A
  \l 2786B
  \l 2786C
  \l 2786D
  \l 2786E
  \l 2786F
  \l 27870
  \l 27871
  \l 27872
  \l 27873
  \l 27874
  \l 27875
  \l 27876
  \l 27877
  \l 27878
  \l 27879
  \l 2787A
  \l 2787B
  \l 2787C
  \l 2787D
  \l 2787E
  \l 2787F
  \l 27880
  \l 27881
  \l 27882
  \l 27883
  \l 27884
  \l 27885
  \l 27886
  \l 27887
  \l 27888
  \l 27889
  \l 2788A
  \l 2788B
  \l 2788C
  \l 2788D
  \l 2788E
  \l 2788F
  \l 27890
  \l 27891
  \l 27892
  \l 27893
  \l 27894
  \l 27895
  \l 27896
  \l 27897
  \l 27898
  \l 27899
  \l 2789A
  \l 2789B
  \l 2789C
  \l 2789D
  \l 2789E
  \l 2789F
  \l 278A0
  \l 278A1
  \l 278A2
  \l 278A3
  \l 278A4
  \l 278A5
  \l 278A6
  \l 278A7
  \l 278A8
  \l 278A9
  \l 278AA
  \l 278AB
  \l 278AC
  \l 278AD
  \l 278AE
  \l 278AF
  \l 278B0
  \l 278B1
  \l 278B2
  \l 278B3
  \l 278B4
  \l 278B5
  \l 278B6
  \l 278B7
  \l 278B8
  \l 278B9
  \l 278BA
  \l 278BB
  \l 278BC
  \l 278BD
  \l 278BE
  \l 278BF
  \l 278C0
  \l 278C1
  \l 278C2
  \l 278C3
  \l 278C4
  \l 278C5
  \l 278C6
  \l 278C7
  \l 278C8
  \l 278C9
  \l 278CA
  \l 278CB
  \l 278CC
  \l 278CD
  \l 278CE
  \l 278CF
  \l 278D0
  \l 278D1
  \l 278D2
  \l 278D3
  \l 278D4
  \l 278D5
  \l 278D6
  \l 278D7
  \l 278D8
  \l 278D9
  \l 278DA
  \l 278DB
  \l 278DC
  \l 278DD
  \l 278DE
  \l 278DF
  \l 278E0
  \l 278E1
  \l 278E2
  \l 278E3
  \l 278E4
  \l 278E5
  \l 278E6
  \l 278E7
  \l 278E8
  \l 278E9
  \l 278EA
  \l 278EB
  \l 278EC
  \l 278ED
  \l 278EE
  \l 278EF
  \l 278F0
  \l 278F1
  \l 278F2
  \l 278F3
  \l 278F4
  \l 278F5
  \l 278F6
  \l 278F7
  \l 278F8
  \l 278F9
  \l 278FA
  \l 278FB
  \l 278FC
  \l 278FD
  \l 278FE
  \l 278FF
  \l 27900
  \l 27901
  \l 27902
  \l 27903
  \l 27904
  \l 27905
  \l 27906
  \l 27907
  \l 27908
  \l 27909
  \l 2790A
  \l 2790B
  \l 2790C
  \l 2790D
  \l 2790E
  \l 2790F
  \l 27910
  \l 27911
  \l 27912
  \l 27913
  \l 27914
  \l 27915
  \l 27916
  \l 27917
  \l 27918
  \l 27919
  \l 2791A
  \l 2791B
  \l 2791C
  \l 2791D
  \l 2791E
  \l 2791F
  \l 27920
  \l 27921
  \l 27922
  \l 27923
  \l 27924
  \l 27925
  \l 27926
  \l 27927
  \l 27928
  \l 27929
  \l 2792A
  \l 2792B
  \l 2792C
  \l 2792D
  \l 2792E
  \l 2792F
  \l 27930
  \l 27931
  \l 27932
  \l 27933
  \l 27934
  \l 27935
  \l 27936
  \l 27937
  \l 27938
  \l 27939
  \l 2793A
  \l 2793B
  \l 2793C
  \l 2793D
  \l 2793E
  \l 2793F
  \l 27940
  \l 27941
  \l 27942
  \l 27943
  \l 27944
  \l 27945
  \l 27946
  \l 27947
  \l 27948
  \l 27949
  \l 2794A
  \l 2794B
  \l 2794C
  \l 2794D
  \l 2794E
  \l 2794F
  \l 27950
  \l 27951
  \l 27952
  \l 27953
  \l 27954
  \l 27955
  \l 27956
  \l 27957
  \l 27958
  \l 27959
  \l 2795A
  \l 2795B
  \l 2795C
  \l 2795D
  \l 2795E
  \l 2795F
  \l 27960
  \l 27961
  \l 27962
  \l 27963
  \l 27964
  \l 27965
  \l 27966
  \l 27967
  \l 27968
  \l 27969
  \l 2796A
  \l 2796B
  \l 2796C
  \l 2796D
  \l 2796E
  \l 2796F
  \l 27970
  \l 27971
  \l 27972
  \l 27973
  \l 27974
  \l 27975
  \l 27976
  \l 27977
  \l 27978
  \l 27979
  \l 2797A
  \l 2797B
  \l 2797C
  \l 2797D
  \l 2797E
  \l 2797F
  \l 27980
  \l 27981
  \l 27982
  \l 27983
  \l 27984
  \l 27985
  \l 27986
  \l 27987
  \l 27988
  \l 27989
  \l 2798A
  \l 2798B
  \l 2798C
  \l 2798D
  \l 2798E
  \l 2798F
  \l 27990
  \l 27991
  \l 27992
  \l 27993
  \l 27994
  \l 27995
  \l 27996
  \l 27997
  \l 27998
  \l 27999
  \l 2799A
  \l 2799B
  \l 2799C
  \l 2799D
  \l 2799E
  \l 2799F
  \l 279A0
  \l 279A1
  \l 279A2
  \l 279A3
  \l 279A4
  \l 279A5
  \l 279A6
  \l 279A7
  \l 279A8
  \l 279A9
  \l 279AA
  \l 279AB
  \l 279AC
  \l 279AD
  \l 279AE
  \l 279AF
  \l 279B0
  \l 279B1
  \l 279B2
  \l 279B3
  \l 279B4
  \l 279B5
  \l 279B6
  \l 279B7
  \l 279B8
  \l 279B9
  \l 279BA
  \l 279BB
  \l 279BC
  \l 279BD
  \l 279BE
  \l 279BF
  \l 279C0
  \l 279C1
  \l 279C2
  \l 279C3
  \l 279C4
  \l 279C5
  \l 279C6
  \l 279C7
  \l 279C8
  \l 279C9
  \l 279CA
  \l 279CB
  \l 279CC
  \l 279CD
  \l 279CE
  \l 279CF
  \l 279D0
  \l 279D1
  \l 279D2
  \l 279D3
  \l 279D4
  \l 279D5
  \l 279D6
  \l 279D7
  \l 279D8
  \l 279D9
  \l 279DA
  \l 279DB
  \l 279DC
  \l 279DD
  \l 279DE
  \l 279DF
  \l 279E0
  \l 279E1
  \l 279E2
  \l 279E3
  \l 279E4
  \l 279E5
  \l 279E6
  \l 279E7
  \l 279E8
  \l 279E9
  \l 279EA
  \l 279EB
  \l 279EC
  \l 279ED
  \l 279EE
  \l 279EF
  \l 279F0
  \l 279F1
  \l 279F2
  \l 279F3
  \l 279F4
  \l 279F5
  \l 279F6
  \l 279F7
  \l 279F8
  \l 279F9
  \l 279FA
  \l 279FB
  \l 279FC
  \l 279FD
  \l 279FE
  \l 279FF
  \l 27A00
  \l 27A01
  \l 27A02
  \l 27A03
  \l 27A04
  \l 27A05
  \l 27A06
  \l 27A07
  \l 27A08
  \l 27A09
  \l 27A0A
  \l 27A0B
  \l 27A0C
  \l 27A0D
  \l 27A0E
  \l 27A0F
  \l 27A10
  \l 27A11
  \l 27A12
  \l 27A13
  \l 27A14
  \l 27A15
  \l 27A16
  \l 27A17
  \l 27A18
  \l 27A19
  \l 27A1A
  \l 27A1B
  \l 27A1C
  \l 27A1D
  \l 27A1E
  \l 27A1F
  \l 27A20
  \l 27A21
  \l 27A22
  \l 27A23
  \l 27A24
  \l 27A25
  \l 27A26
  \l 27A27
  \l 27A28
  \l 27A29
  \l 27A2A
  \l 27A2B
  \l 27A2C
  \l 27A2D
  \l 27A2E
  \l 27A2F
  \l 27A30
  \l 27A31
  \l 27A32
  \l 27A33
  \l 27A34
  \l 27A35
  \l 27A36
  \l 27A37
  \l 27A38
  \l 27A39
  \l 27A3A
  \l 27A3B
  \l 27A3C
  \l 27A3D
  \l 27A3E
  \l 27A3F
  \l 27A40
  \l 27A41
  \l 27A42
  \l 27A43
  \l 27A44
  \l 27A45
  \l 27A46
  \l 27A47
  \l 27A48
  \l 27A49
  \l 27A4A
  \l 27A4B
  \l 27A4C
  \l 27A4D
  \l 27A4E
  \l 27A4F
  \l 27A50
  \l 27A51
  \l 27A52
  \l 27A53
  \l 27A54
  \l 27A55
  \l 27A56
  \l 27A57
  \l 27A58
  \l 27A59
  \l 27A5A
  \l 27A5B
  \l 27A5C
  \l 27A5D
  \l 27A5E
  \l 27A5F
  \l 27A60
  \l 27A61
  \l 27A62
  \l 27A63
  \l 27A64
  \l 27A65
  \l 27A66
  \l 27A67
  \l 27A68
  \l 27A69
  \l 27A6A
  \l 27A6B
  \l 27A6C
  \l 27A6D
  \l 27A6E
  \l 27A6F
  \l 27A70
  \l 27A71
  \l 27A72
  \l 27A73
  \l 27A74
  \l 27A75
  \l 27A76
  \l 27A77
  \l 27A78
  \l 27A79
  \l 27A7A
  \l 27A7B
  \l 27A7C
  \l 27A7D
  \l 27A7E
  \l 27A7F
  \l 27A80
  \l 27A81
  \l 27A82
  \l 27A83
  \l 27A84
  \l 27A85
  \l 27A86
  \l 27A87
  \l 27A88
  \l 27A89
  \l 27A8A
  \l 27A8B
  \l 27A8C
  \l 27A8D
  \l 27A8E
  \l 27A8F
  \l 27A90
  \l 27A91
  \l 27A92
  \l 27A93
  \l 27A94
  \l 27A95
  \l 27A96
  \l 27A97
  \l 27A98
  \l 27A99
  \l 27A9A
  \l 27A9B
  \l 27A9C
  \l 27A9D
  \l 27A9E
  \l 27A9F
  \l 27AA0
  \l 27AA1
  \l 27AA2
  \l 27AA3
  \l 27AA4
  \l 27AA5
  \l 27AA6
  \l 27AA7
  \l 27AA8
  \l 27AA9
  \l 27AAA
  \l 27AAB
  \l 27AAC
  \l 27AAD
  \l 27AAE
  \l 27AAF
  \l 27AB0
  \l 27AB1
  \l 27AB2
  \l 27AB3
  \l 27AB4
  \l 27AB5
  \l 27AB6
  \l 27AB7
  \l 27AB8
  \l 27AB9
  \l 27ABA
  \l 27ABB
  \l 27ABC
  \l 27ABD
  \l 27ABE
  \l 27ABF
  \l 27AC0
  \l 27AC1
  \l 27AC2
  \l 27AC3
  \l 27AC4
  \l 27AC5
  \l 27AC6
  \l 27AC7
  \l 27AC8
  \l 27AC9
  \l 27ACA
  \l 27ACB
  \l 27ACC
  \l 27ACD
  \l 27ACE
  \l 27ACF
  \l 27AD0
  \l 27AD1
  \l 27AD2
  \l 27AD3
  \l 27AD4
  \l 27AD5
  \l 27AD6
  \l 27AD7
  \l 27AD8
  \l 27AD9
  \l 27ADA
  \l 27ADB
  \l 27ADC
  \l 27ADD
  \l 27ADE
  \l 27ADF
  \l 27AE0
  \l 27AE1
  \l 27AE2
  \l 27AE3
  \l 27AE4
  \l 27AE5
  \l 27AE6
  \l 27AE7
  \l 27AE8
  \l 27AE9
  \l 27AEA
  \l 27AEB
  \l 27AEC
  \l 27AED
  \l 27AEE
  \l 27AEF
  \l 27AF0
  \l 27AF1
  \l 27AF2
  \l 27AF3
  \l 27AF4
  \l 27AF5
  \l 27AF6
  \l 27AF7
  \l 27AF8
  \l 27AF9
  \l 27AFA
  \l 27AFB
  \l 27AFC
  \l 27AFD
  \l 27AFE
  \l 27AFF
  \l 27B00
  \l 27B01
  \l 27B02
  \l 27B03
  \l 27B04
  \l 27B05
  \l 27B06
  \l 27B07
  \l 27B08
  \l 27B09
  \l 27B0A
  \l 27B0B
  \l 27B0C
  \l 27B0D
  \l 27B0E
  \l 27B0F
  \l 27B10
  \l 27B11
  \l 27B12
  \l 27B13
  \l 27B14
  \l 27B15
  \l 27B16
  \l 27B17
  \l 27B18
  \l 27B19
  \l 27B1A
  \l 27B1B
  \l 27B1C
  \l 27B1D
  \l 27B1E
  \l 27B1F
  \l 27B20
  \l 27B21
  \l 27B22
  \l 27B23
  \l 27B24
  \l 27B25
  \l 27B26
  \l 27B27
  \l 27B28
  \l 27B29
  \l 27B2A
  \l 27B2B
  \l 27B2C
  \l 27B2D
  \l 27B2E
  \l 27B2F
  \l 27B30
  \l 27B31
  \l 27B32
  \l 27B33
  \l 27B34
  \l 27B35
  \l 27B36
  \l 27B37
  \l 27B38
  \l 27B39
  \l 27B3A
  \l 27B3B
  \l 27B3C
  \l 27B3D
  \l 27B3E
  \l 27B3F
  \l 27B40
  \l 27B41
  \l 27B42
  \l 27B43
  \l 27B44
  \l 27B45
  \l 27B46
  \l 27B47
  \l 27B48
  \l 27B49
  \l 27B4A
  \l 27B4B
  \l 27B4C
  \l 27B4D
  \l 27B4E
  \l 27B4F
  \l 27B50
  \l 27B51
  \l 27B52
  \l 27B53
  \l 27B54
  \l 27B55
  \l 27B56
  \l 27B57
  \l 27B58
  \l 27B59
  \l 27B5A
  \l 27B5B
  \l 27B5C
  \l 27B5D
  \l 27B5E
  \l 27B5F
  \l 27B60
  \l 27B61
  \l 27B62
  \l 27B63
  \l 27B64
  \l 27B65
  \l 27B66
  \l 27B67
  \l 27B68
  \l 27B69
  \l 27B6A
  \l 27B6B
  \l 27B6C
  \l 27B6D
  \l 27B6E
  \l 27B6F
  \l 27B70
  \l 27B71
  \l 27B72
  \l 27B73
  \l 27B74
  \l 27B75
  \l 27B76
  \l 27B77
  \l 27B78
  \l 27B79
  \l 27B7A
  \l 27B7B
  \l 27B7C
  \l 27B7D
  \l 27B7E
  \l 27B7F
  \l 27B80
  \l 27B81
  \l 27B82
  \l 27B83
  \l 27B84
  \l 27B85
  \l 27B86
  \l 27B87
  \l 27B88
  \l 27B89
  \l 27B8A
  \l 27B8B
  \l 27B8C
  \l 27B8D
  \l 27B8E
  \l 27B8F
  \l 27B90
  \l 27B91
  \l 27B92
  \l 27B93
  \l 27B94
  \l 27B95
  \l 27B96
  \l 27B97
  \l 27B98
  \l 27B99
  \l 27B9A
  \l 27B9B
  \l 27B9C
  \l 27B9D
  \l 27B9E
  \l 27B9F
  \l 27BA0
  \l 27BA1
  \l 27BA2
  \l 27BA3
  \l 27BA4
  \l 27BA5
  \l 27BA6
  \l 27BA7
  \l 27BA8
  \l 27BA9
  \l 27BAA
  \l 27BAB
  \l 27BAC
  \l 27BAD
  \l 27BAE
  \l 27BAF
  \l 27BB0
  \l 27BB1
  \l 27BB2
  \l 27BB3
  \l 27BB4
  \l 27BB5
  \l 27BB6
  \l 27BB7
  \l 27BB8
  \l 27BB9
  \l 27BBA
  \l 27BBB
  \l 27BBC
  \l 27BBD
  \l 27BBE
  \l 27BBF
  \l 27BC0
  \l 27BC1
  \l 27BC2
  \l 27BC3
  \l 27BC4
  \l 27BC5
  \l 27BC6
  \l 27BC7
  \l 27BC8
  \l 27BC9
  \l 27BCA
  \l 27BCB
  \l 27BCC
  \l 27BCD
  \l 27BCE
  \l 27BCF
  \l 27BD0
  \l 27BD1
  \l 27BD2
  \l 27BD3
  \l 27BD4
  \l 27BD5
  \l 27BD6
  \l 27BD7
  \l 27BD8
  \l 27BD9
  \l 27BDA
  \l 27BDB
  \l 27BDC
  \l 27BDD
  \l 27BDE
  \l 27BDF
  \l 27BE0
  \l 27BE1
  \l 27BE2
  \l 27BE3
  \l 27BE4
  \l 27BE5
  \l 27BE6
  \l 27BE7
  \l 27BE8
  \l 27BE9
  \l 27BEA
  \l 27BEB
  \l 27BEC
  \l 27BED
  \l 27BEE
  \l 27BEF
  \l 27BF0
  \l 27BF1
  \l 27BF2
  \l 27BF3
  \l 27BF4
  \l 27BF5
  \l 27BF6
  \l 27BF7
  \l 27BF8
  \l 27BF9
  \l 27BFA
  \l 27BFB
  \l 27BFC
  \l 27BFD
  \l 27BFE
  \l 27BFF
  \l 27C00
  \l 27C01
  \l 27C02
  \l 27C03
  \l 27C04
  \l 27C05
  \l 27C06
  \l 27C07
  \l 27C08
  \l 27C09
  \l 27C0A
  \l 27C0B
  \l 27C0C
  \l 27C0D
  \l 27C0E
  \l 27C0F
  \l 27C10
  \l 27C11
  \l 27C12
  \l 27C13
  \l 27C14
  \l 27C15
  \l 27C16
  \l 27C17
  \l 27C18
  \l 27C19
  \l 27C1A
  \l 27C1B
  \l 27C1C
  \l 27C1D
  \l 27C1E
  \l 27C1F
  \l 27C20
  \l 27C21
  \l 27C22
  \l 27C23
  \l 27C24
  \l 27C25
  \l 27C26
  \l 27C27
  \l 27C28
  \l 27C29
  \l 27C2A
  \l 27C2B
  \l 27C2C
  \l 27C2D
  \l 27C2E
  \l 27C2F
  \l 27C30
  \l 27C31
  \l 27C32
  \l 27C33
  \l 27C34
  \l 27C35
  \l 27C36
  \l 27C37
  \l 27C38
  \l 27C39
  \l 27C3A
  \l 27C3B
  \l 27C3C
  \l 27C3D
  \l 27C3E
  \l 27C3F
  \l 27C40
  \l 27C41
  \l 27C42
  \l 27C43
  \l 27C44
  \l 27C45
  \l 27C46
  \l 27C47
  \l 27C48
  \l 27C49
  \l 27C4A
  \l 27C4B
  \l 27C4C
  \l 27C4D
  \l 27C4E
  \l 27C4F
  \l 27C50
  \l 27C51
  \l 27C52
  \l 27C53
  \l 27C54
  \l 27C55
  \l 27C56
  \l 27C57
  \l 27C58
  \l 27C59
  \l 27C5A
  \l 27C5B
  \l 27C5C
  \l 27C5D
  \l 27C5E
  \l 27C5F
  \l 27C60
  \l 27C61
  \l 27C62
  \l 27C63
  \l 27C64
  \l 27C65
  \l 27C66
  \l 27C67
  \l 27C68
  \l 27C69
  \l 27C6A
  \l 27C6B
  \l 27C6C
  \l 27C6D
  \l 27C6E
  \l 27C6F
  \l 27C70
  \l 27C71
  \l 27C72
  \l 27C73
  \l 27C74
  \l 27C75
  \l 27C76
  \l 27C77
  \l 27C78
  \l 27C79
  \l 27C7A
  \l 27C7B
  \l 27C7C
  \l 27C7D
  \l 27C7E
  \l 27C7F
  \l 27C80
  \l 27C81
  \l 27C82
  \l 27C83
  \l 27C84
  \l 27C85
  \l 27C86
  \l 27C87
  \l 27C88
  \l 27C89
  \l 27C8A
  \l 27C8B
  \l 27C8C
  \l 27C8D
  \l 27C8E
  \l 27C8F
  \l 27C90
  \l 27C91
  \l 27C92
  \l 27C93
  \l 27C94
  \l 27C95
  \l 27C96
  \l 27C97
  \l 27C98
  \l 27C99
  \l 27C9A
  \l 27C9B
  \l 27C9C
  \l 27C9D
  \l 27C9E
  \l 27C9F
  \l 27CA0
  \l 27CA1
  \l 27CA2
  \l 27CA3
  \l 27CA4
  \l 27CA5
  \l 27CA6
  \l 27CA7
  \l 27CA8
  \l 27CA9
  \l 27CAA
  \l 27CAB
  \l 27CAC
  \l 27CAD
  \l 27CAE
  \l 27CAF
  \l 27CB0
  \l 27CB1
  \l 27CB2
  \l 27CB3
  \l 27CB4
  \l 27CB5
  \l 27CB6
  \l 27CB7
  \l 27CB8
  \l 27CB9
  \l 27CBA
  \l 27CBB
  \l 27CBC
  \l 27CBD
  \l 27CBE
  \l 27CBF
  \l 27CC0
  \l 27CC1
  \l 27CC2
  \l 27CC3
  \l 27CC4
  \l 27CC5
  \l 27CC6
  \l 27CC7
  \l 27CC8
  \l 27CC9
  \l 27CCA
  \l 27CCB
  \l 27CCC
  \l 27CCD
  \l 27CCE
  \l 27CCF
  \l 27CD0
  \l 27CD1
  \l 27CD2
  \l 27CD3
  \l 27CD4
  \l 27CD5
  \l 27CD6
  \l 27CD7
  \l 27CD8
  \l 27CD9
  \l 27CDA
  \l 27CDB
  \l 27CDC
  \l 27CDD
  \l 27CDE
  \l 27CDF
  \l 27CE0
  \l 27CE1
  \l 27CE2
  \l 27CE3
  \l 27CE4
  \l 27CE5
  \l 27CE6
  \l 27CE7
  \l 27CE8
  \l 27CE9
  \l 27CEA
  \l 27CEB
  \l 27CEC
  \l 27CED
  \l 27CEE
  \l 27CEF
  \l 27CF0
  \l 27CF1
  \l 27CF2
  \l 27CF3
  \l 27CF4
  \l 27CF5
  \l 27CF6
  \l 27CF7
  \l 27CF8
  \l 27CF9
  \l 27CFA
  \l 27CFB
  \l 27CFC
  \l 27CFD
  \l 27CFE
  \l 27CFF
  \l 27D00
  \l 27D01
  \l 27D02
  \l 27D03
  \l 27D04
  \l 27D05
  \l 27D06
  \l 27D07
  \l 27D08
  \l 27D09
  \l 27D0A
  \l 27D0B
  \l 27D0C
  \l 27D0D
  \l 27D0E
  \l 27D0F
  \l 27D10
  \l 27D11
  \l 27D12
  \l 27D13
  \l 27D14
  \l 27D15
  \l 27D16
  \l 27D17
  \l 27D18
  \l 27D19
  \l 27D1A
  \l 27D1B
  \l 27D1C
  \l 27D1D
  \l 27D1E
  \l 27D1F
  \l 27D20
  \l 27D21
  \l 27D22
  \l 27D23
  \l 27D24
  \l 27D25
  \l 27D26
  \l 27D27
  \l 27D28
  \l 27D29
  \l 27D2A
  \l 27D2B
  \l 27D2C
  \l 27D2D
  \l 27D2E
  \l 27D2F
  \l 27D30
  \l 27D31
  \l 27D32
  \l 27D33
  \l 27D34
  \l 27D35
  \l 27D36
  \l 27D37
  \l 27D38
  \l 27D39
  \l 27D3A
  \l 27D3B
  \l 27D3C
  \l 27D3D
  \l 27D3E
  \l 27D3F
  \l 27D40
  \l 27D41
  \l 27D42
  \l 27D43
  \l 27D44
  \l 27D45
  \l 27D46
  \l 27D47
  \l 27D48
  \l 27D49
  \l 27D4A
  \l 27D4B
  \l 27D4C
  \l 27D4D
  \l 27D4E
  \l 27D4F
  \l 27D50
  \l 27D51
  \l 27D52
  \l 27D53
  \l 27D54
  \l 27D55
  \l 27D56
  \l 27D57
  \l 27D58
  \l 27D59
  \l 27D5A
  \l 27D5B
  \l 27D5C
  \l 27D5D
  \l 27D5E
  \l 27D5F
  \l 27D60
  \l 27D61
  \l 27D62
  \l 27D63
  \l 27D64
  \l 27D65
  \l 27D66
  \l 27D67
  \l 27D68
  \l 27D69
  \l 27D6A
  \l 27D6B
  \l 27D6C
  \l 27D6D
  \l 27D6E
  \l 27D6F
  \l 27D70
  \l 27D71
  \l 27D72
  \l 27D73
  \l 27D74
  \l 27D75
  \l 27D76
  \l 27D77
  \l 27D78
  \l 27D79
  \l 27D7A
  \l 27D7B
  \l 27D7C
  \l 27D7D
  \l 27D7E
  \l 27D7F
  \l 27D80
  \l 27D81
  \l 27D82
  \l 27D83
  \l 27D84
  \l 27D85
  \l 27D86
  \l 27D87
  \l 27D88
  \l 27D89
  \l 27D8A
  \l 27D8B
  \l 27D8C
  \l 27D8D
  \l 27D8E
  \l 27D8F
  \l 27D90
  \l 27D91
  \l 27D92
  \l 27D93
  \l 27D94
  \l 27D95
  \l 27D96
  \l 27D97
  \l 27D98
  \l 27D99
  \l 27D9A
  \l 27D9B
  \l 27D9C
  \l 27D9D
  \l 27D9E
  \l 27D9F
  \l 27DA0
  \l 27DA1
  \l 27DA2
  \l 27DA3
  \l 27DA4
  \l 27DA5
  \l 27DA6
  \l 27DA7
  \l 27DA8
  \l 27DA9
  \l 27DAA
  \l 27DAB
  \l 27DAC
  \l 27DAD
  \l 27DAE
  \l 27DAF
  \l 27DB0
  \l 27DB1
  \l 27DB2
  \l 27DB3
  \l 27DB4
  \l 27DB5
  \l 27DB6
  \l 27DB7
  \l 27DB8
  \l 27DB9
  \l 27DBA
  \l 27DBB
  \l 27DBC
  \l 27DBD
  \l 27DBE
  \l 27DBF
  \l 27DC0
  \l 27DC1
  \l 27DC2
  \l 27DC3
  \l 27DC4
  \l 27DC5
  \l 27DC6
  \l 27DC7
  \l 27DC8
  \l 27DC9
  \l 27DCA
  \l 27DCB
  \l 27DCC
  \l 27DCD
  \l 27DCE
  \l 27DCF
  \l 27DD0
  \l 27DD1
  \l 27DD2
  \l 27DD3
  \l 27DD4
  \l 27DD5
  \l 27DD6
  \l 27DD7
  \l 27DD8
  \l 27DD9
  \l 27DDA
  \l 27DDB
  \l 27DDC
  \l 27DDD
  \l 27DDE
  \l 27DDF
  \l 27DE0
  \l 27DE1
  \l 27DE2
  \l 27DE3
  \l 27DE4
  \l 27DE5
  \l 27DE6
  \l 27DE7
  \l 27DE8
  \l 27DE9
  \l 27DEA
  \l 27DEB
  \l 27DEC
  \l 27DED
  \l 27DEE
  \l 27DEF
  \l 27DF0
  \l 27DF1
  \l 27DF2
  \l 27DF3
  \l 27DF4
  \l 27DF5
  \l 27DF6
  \l 27DF7
  \l 27DF8
  \l 27DF9
  \l 27DFA
  \l 27DFB
  \l 27DFC
  \l 27DFD
  \l 27DFE
  \l 27DFF
  \l 27E00
  \l 27E01
  \l 27E02
  \l 27E03
  \l 27E04
  \l 27E05
  \l 27E06
  \l 27E07
  \l 27E08
  \l 27E09
  \l 27E0A
  \l 27E0B
  \l 27E0C
  \l 27E0D
  \l 27E0E
  \l 27E0F
  \l 27E10
  \l 27E11
  \l 27E12
  \l 27E13
  \l 27E14
  \l 27E15
  \l 27E16
  \l 27E17
  \l 27E18
  \l 27E19
  \l 27E1A
  \l 27E1B
  \l 27E1C
  \l 27E1D
  \l 27E1E
  \l 27E1F
  \l 27E20
  \l 27E21
  \l 27E22
  \l 27E23
  \l 27E24
  \l 27E25
  \l 27E26
  \l 27E27
  \l 27E28
  \l 27E29
  \l 27E2A
  \l 27E2B
  \l 27E2C
  \l 27E2D
  \l 27E2E
  \l 27E2F
  \l 27E30
  \l 27E31
  \l 27E32
  \l 27E33
  \l 27E34
  \l 27E35
  \l 27E36
  \l 27E37
  \l 27E38
  \l 27E39
  \l 27E3A
  \l 27E3B
  \l 27E3C
  \l 27E3D
  \l 27E3E
  \l 27E3F
  \l 27E40
  \l 27E41
  \l 27E42
  \l 27E43
  \l 27E44
  \l 27E45
  \l 27E46
  \l 27E47
  \l 27E48
  \l 27E49
  \l 27E4A
  \l 27E4B
  \l 27E4C
  \l 27E4D
  \l 27E4E
  \l 27E4F
  \l 27E50
  \l 27E51
  \l 27E52
  \l 27E53
  \l 27E54
  \l 27E55
  \l 27E56
  \l 27E57
  \l 27E58
  \l 27E59
  \l 27E5A
  \l 27E5B
  \l 27E5C
  \l 27E5D
  \l 27E5E
  \l 27E5F
  \l 27E60
  \l 27E61
  \l 27E62
  \l 27E63
  \l 27E64
  \l 27E65
  \l 27E66
  \l 27E67
  \l 27E68
  \l 27E69
  \l 27E6A
  \l 27E6B
  \l 27E6C
  \l 27E6D
  \l 27E6E
  \l 27E6F
  \l 27E70
  \l 27E71
  \l 27E72
  \l 27E73
  \l 27E74
  \l 27E75
  \l 27E76
  \l 27E77
  \l 27E78
  \l 27E79
  \l 27E7A
  \l 27E7B
  \l 27E7C
  \l 27E7D
  \l 27E7E
  \l 27E7F
  \l 27E80
  \l 27E81
  \l 27E82
  \l 27E83
  \l 27E84
  \l 27E85
  \l 27E86
  \l 27E87
  \l 27E88
  \l 27E89
  \l 27E8A
  \l 27E8B
  \l 27E8C
  \l 27E8D
  \l 27E8E
  \l 27E8F
  \l 27E90
  \l 27E91
  \l 27E92
  \l 27E93
  \l 27E94
  \l 27E95
  \l 27E96
  \l 27E97
  \l 27E98
  \l 27E99
  \l 27E9A
  \l 27E9B
  \l 27E9C
  \l 27E9D
  \l 27E9E
  \l 27E9F
  \l 27EA0
  \l 27EA1
  \l 27EA2
  \l 27EA3
  \l 27EA4
  \l 27EA5
  \l 27EA6
  \l 27EA7
  \l 27EA8
  \l 27EA9
  \l 27EAA
  \l 27EAB
  \l 27EAC
  \l 27EAD
  \l 27EAE
  \l 27EAF
  \l 27EB0
  \l 27EB1
  \l 27EB2
  \l 27EB3
  \l 27EB4
  \l 27EB5
  \l 27EB6
  \l 27EB7
  \l 27EB8
  \l 27EB9
  \l 27EBA
  \l 27EBB
  \l 27EBC
  \l 27EBD
  \l 27EBE
  \l 27EBF
  \l 27EC0
  \l 27EC1
  \l 27EC2
  \l 27EC3
  \l 27EC4
  \l 27EC5
  \l 27EC6
  \l 27EC7
  \l 27EC8
  \l 27EC9
  \l 27ECA
  \l 27ECB
  \l 27ECC
  \l 27ECD
  \l 27ECE
  \l 27ECF
  \l 27ED0
  \l 27ED1
  \l 27ED2
  \l 27ED3
  \l 27ED4
  \l 27ED5
  \l 27ED6
  \l 27ED7
  \l 27ED8
  \l 27ED9
  \l 27EDA
  \l 27EDB
  \l 27EDC
  \l 27EDD
  \l 27EDE
  \l 27EDF
  \l 27EE0
  \l 27EE1
  \l 27EE2
  \l 27EE3
  \l 27EE4
  \l 27EE5
  \l 27EE6
  \l 27EE7
  \l 27EE8
  \l 27EE9
  \l 27EEA
  \l 27EEB
  \l 27EEC
  \l 27EED
  \l 27EEE
  \l 27EEF
  \l 27EF0
  \l 27EF1
  \l 27EF2
  \l 27EF3
  \l 27EF4
  \l 27EF5
  \l 27EF6
  \l 27EF7
  \l 27EF8
  \l 27EF9
  \l 27EFA
  \l 27EFB
  \l 27EFC
  \l 27EFD
  \l 27EFE
  \l 27EFF
  \l 27F00
  \l 27F01
  \l 27F02
  \l 27F03
  \l 27F04
  \l 27F05
  \l 27F06
  \l 27F07
  \l 27F08
  \l 27F09
  \l 27F0A
  \l 27F0B
  \l 27F0C
  \l 27F0D
  \l 27F0E
  \l 27F0F
  \l 27F10
  \l 27F11
  \l 27F12
  \l 27F13
  \l 27F14
  \l 27F15
  \l 27F16
  \l 27F17
  \l 27F18
  \l 27F19
  \l 27F1A
  \l 27F1B
  \l 27F1C
  \l 27F1D
  \l 27F1E
  \l 27F1F
  \l 27F20
  \l 27F21
  \l 27F22
  \l 27F23
  \l 27F24
  \l 27F25
  \l 27F26
  \l 27F27
  \l 27F28
  \l 27F29
  \l 27F2A
  \l 27F2B
  \l 27F2C
  \l 27F2D
  \l 27F2E
  \l 27F2F
  \l 27F30
  \l 27F31
  \l 27F32
  \l 27F33
  \l 27F34
  \l 27F35
  \l 27F36
  \l 27F37
  \l 27F38
  \l 27F39
  \l 27F3A
  \l 27F3B
  \l 27F3C
  \l 27F3D
  \l 27F3E
  \l 27F3F
  \l 27F40
  \l 27F41
  \l 27F42
  \l 27F43
  \l 27F44
  \l 27F45
  \l 27F46
  \l 27F47
  \l 27F48
  \l 27F49
  \l 27F4A
  \l 27F4B
  \l 27F4C
  \l 27F4D
  \l 27F4E
  \l 27F4F
  \l 27F50
  \l 27F51
  \l 27F52
  \l 27F53
  \l 27F54
  \l 27F55
  \l 27F56
  \l 27F57
  \l 27F58
  \l 27F59
  \l 27F5A
  \l 27F5B
  \l 27F5C
  \l 27F5D
  \l 27F5E
  \l 27F5F
  \l 27F60
  \l 27F61
  \l 27F62
  \l 27F63
  \l 27F64
  \l 27F65
  \l 27F66
  \l 27F67
  \l 27F68
  \l 27F69
  \l 27F6A
  \l 27F6B
  \l 27F6C
  \l 27F6D
  \l 27F6E
  \l 27F6F
  \l 27F70
  \l 27F71
  \l 27F72
  \l 27F73
  \l 27F74
  \l 27F75
  \l 27F76
  \l 27F77
  \l 27F78
  \l 27F79
  \l 27F7A
  \l 27F7B
  \l 27F7C
  \l 27F7D
  \l 27F7E
  \l 27F7F
  \l 27F80
  \l 27F81
  \l 27F82
  \l 27F83
  \l 27F84
  \l 27F85
  \l 27F86
  \l 27F87
  \l 27F88
  \l 27F89
  \l 27F8A
  \l 27F8B
  \l 27F8C
  \l 27F8D
  \l 27F8E
  \l 27F8F
  \l 27F90
  \l 27F91
  \l 27F92
  \l 27F93
  \l 27F94
  \l 27F95
  \l 27F96
  \l 27F97
  \l 27F98
  \l 27F99
  \l 27F9A
  \l 27F9B
  \l 27F9C
  \l 27F9D
  \l 27F9E
  \l 27F9F
  \l 27FA0
  \l 27FA1
  \l 27FA2
  \l 27FA3
  \l 27FA4
  \l 27FA5
  \l 27FA6
  \l 27FA7
  \l 27FA8
  \l 27FA9
  \l 27FAA
  \l 27FAB
  \l 27FAC
  \l 27FAD
  \l 27FAE
  \l 27FAF
  \l 27FB0
  \l 27FB1
  \l 27FB2
  \l 27FB3
  \l 27FB4
  \l 27FB5
  \l 27FB6
  \l 27FB7
  \l 27FB8
  \l 27FB9
  \l 27FBA
  \l 27FBB
  \l 27FBC
  \l 27FBD
  \l 27FBE
  \l 27FBF
  \l 27FC0
  \l 27FC1
  \l 27FC2
  \l 27FC3
  \l 27FC4
  \l 27FC5
  \l 27FC6
  \l 27FC7
  \l 27FC8
  \l 27FC9
  \l 27FCA
  \l 27FCB
  \l 27FCC
  \l 27FCD
  \l 27FCE
  \l 27FCF
  \l 27FD0
  \l 27FD1
  \l 27FD2
  \l 27FD3
  \l 27FD4
  \l 27FD5
  \l 27FD6
  \l 27FD7
  \l 27FD8
  \l 27FD9
  \l 27FDA
  \l 27FDB
  \l 27FDC
  \l 27FDD
  \l 27FDE
  \l 27FDF
  \l 27FE0
  \l 27FE1
  \l 27FE2
  \l 27FE3
  \l 27FE4
  \l 27FE5
  \l 27FE6
  \l 27FE7
  \l 27FE8
  \l 27FE9
  \l 27FEA
  \l 27FEB
  \l 27FEC
  \l 27FED
  \l 27FEE
  \l 27FEF
  \l 27FF0
  \l 27FF1
  \l 27FF2
  \l 27FF3
  \l 27FF4
  \l 27FF5
  \l 27FF6
  \l 27FF7
  \l 27FF8
  \l 27FF9
  \l 27FFA
  \l 27FFB
  \l 27FFC
  \l 27FFD
  \l 27FFE
  \l 27FFF
  \l 28000
  \l 28001
  \l 28002
  \l 28003
  \l 28004
  \l 28005
  \l 28006
  \l 28007
  \l 28008
  \l 28009
  \l 2800A
  \l 2800B
  \l 2800C
  \l 2800D
  \l 2800E
  \l 2800F
  \l 28010
  \l 28011
  \l 28012
  \l 28013
  \l 28014
  \l 28015
  \l 28016
  \l 28017
  \l 28018
  \l 28019
  \l 2801A
  \l 2801B
  \l 2801C
  \l 2801D
  \l 2801E
  \l 2801F
  \l 28020
  \l 28021
  \l 28022
  \l 28023
  \l 28024
  \l 28025
  \l 28026
  \l 28027
  \l 28028
  \l 28029
  \l 2802A
  \l 2802B
  \l 2802C
  \l 2802D
  \l 2802E
  \l 2802F
  \l 28030
  \l 28031
  \l 28032
  \l 28033
  \l 28034
  \l 28035
  \l 28036
  \l 28037
  \l 28038
  \l 28039
  \l 2803A
  \l 2803B
  \l 2803C
  \l 2803D
  \l 2803E
  \l 2803F
  \l 28040
  \l 28041
  \l 28042
  \l 28043
  \l 28044
  \l 28045
  \l 28046
  \l 28047
  \l 28048
  \l 28049
  \l 2804A
  \l 2804B
  \l 2804C
  \l 2804D
  \l 2804E
  \l 2804F
  \l 28050
  \l 28051
  \l 28052
  \l 28053
  \l 28054
  \l 28055
  \l 28056
  \l 28057
  \l 28058
  \l 28059
  \l 2805A
  \l 2805B
  \l 2805C
  \l 2805D
  \l 2805E
  \l 2805F
  \l 28060
  \l 28061
  \l 28062
  \l 28063
  \l 28064
  \l 28065
  \l 28066
  \l 28067
  \l 28068
  \l 28069
  \l 2806A
  \l 2806B
  \l 2806C
  \l 2806D
  \l 2806E
  \l 2806F
  \l 28070
  \l 28071
  \l 28072
  \l 28073
  \l 28074
  \l 28075
  \l 28076
  \l 28077
  \l 28078
  \l 28079
  \l 2807A
  \l 2807B
  \l 2807C
  \l 2807D
  \l 2807E
  \l 2807F
  \l 28080
  \l 28081
  \l 28082
  \l 28083
  \l 28084
  \l 28085
  \l 28086
  \l 28087
  \l 28088
  \l 28089
  \l 2808A
  \l 2808B
  \l 2808C
  \l 2808D
  \l 2808E
  \l 2808F
  \l 28090
  \l 28091
  \l 28092
  \l 28093
  \l 28094
  \l 28095
  \l 28096
  \l 28097
  \l 28098
  \l 28099
  \l 2809A
  \l 2809B
  \l 2809C
  \l 2809D
  \l 2809E
  \l 2809F
  \l 280A0
  \l 280A1
  \l 280A2
  \l 280A3
  \l 280A4
  \l 280A5
  \l 280A6
  \l 280A7
  \l 280A8
  \l 280A9
  \l 280AA
  \l 280AB
  \l 280AC
  \l 280AD
  \l 280AE
  \l 280AF
  \l 280B0
  \l 280B1
  \l 280B2
  \l 280B3
  \l 280B4
  \l 280B5
  \l 280B6
  \l 280B7
  \l 280B8
  \l 280B9
  \l 280BA
  \l 280BB
  \l 280BC
  \l 280BD
  \l 280BE
  \l 280BF
  \l 280C0
  \l 280C1
  \l 280C2
  \l 280C3
  \l 280C4
  \l 280C5
  \l 280C6
  \l 280C7
  \l 280C8
  \l 280C9
  \l 280CA
  \l 280CB
  \l 280CC
  \l 280CD
  \l 280CE
  \l 280CF
  \l 280D0
  \l 280D1
  \l 280D2
  \l 280D3
  \l 280D4
  \l 280D5
  \l 280D6
  \l 280D7
  \l 280D8
  \l 280D9
  \l 280DA
  \l 280DB
  \l 280DC
  \l 280DD
  \l 280DE
  \l 280DF
  \l 280E0
  \l 280E1
  \l 280E2
  \l 280E3
  \l 280E4
  \l 280E5
  \l 280E6
  \l 280E7
  \l 280E8
  \l 280E9
  \l 280EA
  \l 280EB
  \l 280EC
  \l 280ED
  \l 280EE
  \l 280EF
  \l 280F0
  \l 280F1
  \l 280F2
  \l 280F3
  \l 280F4
  \l 280F5
  \l 280F6
  \l 280F7
  \l 280F8
  \l 280F9
  \l 280FA
  \l 280FB
  \l 280FC
  \l 280FD
  \l 280FE
  \l 280FF
  \l 28100
  \l 28101
  \l 28102
  \l 28103
  \l 28104
  \l 28105
  \l 28106
  \l 28107
  \l 28108
  \l 28109
  \l 2810A
  \l 2810B
  \l 2810C
  \l 2810D
  \l 2810E
  \l 2810F
  \l 28110
  \l 28111
  \l 28112
  \l 28113
  \l 28114
  \l 28115
  \l 28116
  \l 28117
  \l 28118
  \l 28119
  \l 2811A
  \l 2811B
  \l 2811C
  \l 2811D
  \l 2811E
  \l 2811F
  \l 28120
  \l 28121
  \l 28122
  \l 28123
  \l 28124
  \l 28125
  \l 28126
  \l 28127
  \l 28128
  \l 28129
  \l 2812A
  \l 2812B
  \l 2812C
  \l 2812D
  \l 2812E
  \l 2812F
  \l 28130
  \l 28131
  \l 28132
  \l 28133
  \l 28134
  \l 28135
  \l 28136
  \l 28137
  \l 28138
  \l 28139
  \l 2813A
  \l 2813B
  \l 2813C
  \l 2813D
  \l 2813E
  \l 2813F
  \l 28140
  \l 28141
  \l 28142
  \l 28143
  \l 28144
  \l 28145
  \l 28146
  \l 28147
  \l 28148
  \l 28149
  \l 2814A
  \l 2814B
  \l 2814C
  \l 2814D
  \l 2814E
  \l 2814F
  \l 28150
  \l 28151
  \l 28152
  \l 28153
  \l 28154
  \l 28155
  \l 28156
  \l 28157
  \l 28158
  \l 28159
  \l 2815A
  \l 2815B
  \l 2815C
  \l 2815D
  \l 2815E
  \l 2815F
  \l 28160
  \l 28161
  \l 28162
  \l 28163
  \l 28164
  \l 28165
  \l 28166
  \l 28167
  \l 28168
  \l 28169
  \l 2816A
  \l 2816B
  \l 2816C
  \l 2816D
  \l 2816E
  \l 2816F
  \l 28170
  \l 28171
  \l 28172
  \l 28173
  \l 28174
  \l 28175
  \l 28176
  \l 28177
  \l 28178
  \l 28179
  \l 2817A
  \l 2817B
  \l 2817C
  \l 2817D
  \l 2817E
  \l 2817F
  \l 28180
  \l 28181
  \l 28182
  \l 28183
  \l 28184
  \l 28185
  \l 28186
  \l 28187
  \l 28188
  \l 28189
  \l 2818A
  \l 2818B
  \l 2818C
  \l 2818D
  \l 2818E
  \l 2818F
  \l 28190
  \l 28191
  \l 28192
  \l 28193
  \l 28194
  \l 28195
  \l 28196
  \l 28197
  \l 28198
  \l 28199
  \l 2819A
  \l 2819B
  \l 2819C
  \l 2819D
  \l 2819E
  \l 2819F
  \l 281A0
  \l 281A1
  \l 281A2
  \l 281A3
  \l 281A4
  \l 281A5
  \l 281A6
  \l 281A7
  \l 281A8
  \l 281A9
  \l 281AA
  \l 281AB
  \l 281AC
  \l 281AD
  \l 281AE
  \l 281AF
  \l 281B0
  \l 281B1
  \l 281B2
  \l 281B3
  \l 281B4
  \l 281B5
  \l 281B6
  \l 281B7
  \l 281B8
  \l 281B9
  \l 281BA
  \l 281BB
  \l 281BC
  \l 281BD
  \l 281BE
  \l 281BF
  \l 281C0
  \l 281C1
  \l 281C2
  \l 281C3
  \l 281C4
  \l 281C5
  \l 281C6
  \l 281C7
  \l 281C8
  \l 281C9
  \l 281CA
  \l 281CB
  \l 281CC
  \l 281CD
  \l 281CE
  \l 281CF
  \l 281D0
  \l 281D1
  \l 281D2
  \l 281D3
  \l 281D4
  \l 281D5
  \l 281D6
  \l 281D7
  \l 281D8
  \l 281D9
  \l 281DA
  \l 281DB
  \l 281DC
  \l 281DD
  \l 281DE
  \l 281DF
  \l 281E0
  \l 281E1
  \l 281E2
  \l 281E3
  \l 281E4
  \l 281E5
  \l 281E6
  \l 281E7
  \l 281E8
  \l 281E9
  \l 281EA
  \l 281EB
  \l 281EC
  \l 281ED
  \l 281EE
  \l 281EF
  \l 281F0
  \l 281F1
  \l 281F2
  \l 281F3
  \l 281F4
  \l 281F5
  \l 281F6
  \l 281F7
  \l 281F8
  \l 281F9
  \l 281FA
  \l 281FB
  \l 281FC
  \l 281FD
  \l 281FE
  \l 281FF
  \l 28200
  \l 28201
  \l 28202
  \l 28203
  \l 28204
  \l 28205
  \l 28206
  \l 28207
  \l 28208
  \l 28209
  \l 2820A
  \l 2820B
  \l 2820C
  \l 2820D
  \l 2820E
  \l 2820F
  \l 28210
  \l 28211
  \l 28212
  \l 28213
  \l 28214
  \l 28215
  \l 28216
  \l 28217
  \l 28218
  \l 28219
  \l 2821A
  \l 2821B
  \l 2821C
  \l 2821D
  \l 2821E
  \l 2821F
  \l 28220
  \l 28221
  \l 28222
  \l 28223
  \l 28224
  \l 28225
  \l 28226
  \l 28227
  \l 28228
  \l 28229
  \l 2822A
  \l 2822B
  \l 2822C
  \l 2822D
  \l 2822E
  \l 2822F
  \l 28230
  \l 28231
  \l 28232
  \l 28233
  \l 28234
  \l 28235
  \l 28236
  \l 28237
  \l 28238
  \l 28239
  \l 2823A
  \l 2823B
  \l 2823C
  \l 2823D
  \l 2823E
  \l 2823F
  \l 28240
  \l 28241
  \l 28242
  \l 28243
  \l 28244
  \l 28245
  \l 28246
  \l 28247
  \l 28248
  \l 28249
  \l 2824A
  \l 2824B
  \l 2824C
  \l 2824D
  \l 2824E
  \l 2824F
  \l 28250
  \l 28251
  \l 28252
  \l 28253
  \l 28254
  \l 28255
  \l 28256
  \l 28257
  \l 28258
  \l 28259
  \l 2825A
  \l 2825B
  \l 2825C
  \l 2825D
  \l 2825E
  \l 2825F
  \l 28260
  \l 28261
  \l 28262
  \l 28263
  \l 28264
  \l 28265
  \l 28266
  \l 28267
  \l 28268
  \l 28269
  \l 2826A
  \l 2826B
  \l 2826C
  \l 2826D
  \l 2826E
  \l 2826F
  \l 28270
  \l 28271
  \l 28272
  \l 28273
  \l 28274
  \l 28275
  \l 28276
  \l 28277
  \l 28278
  \l 28279
  \l 2827A
  \l 2827B
  \l 2827C
  \l 2827D
  \l 2827E
  \l 2827F
  \l 28280
  \l 28281
  \l 28282
  \l 28283
  \l 28284
  \l 28285
  \l 28286
  \l 28287
  \l 28288
  \l 28289
  \l 2828A
  \l 2828B
  \l 2828C
  \l 2828D
  \l 2828E
  \l 2828F
  \l 28290
  \l 28291
  \l 28292
  \l 28293
  \l 28294
  \l 28295
  \l 28296
  \l 28297
  \l 28298
  \l 28299
  \l 2829A
  \l 2829B
  \l 2829C
  \l 2829D
  \l 2829E
  \l 2829F
  \l 282A0
  \l 282A1
  \l 282A2
  \l 282A3
  \l 282A4
  \l 282A5
  \l 282A6
  \l 282A7
  \l 282A8
  \l 282A9
  \l 282AA
  \l 282AB
  \l 282AC
  \l 282AD
  \l 282AE
  \l 282AF
  \l 282B0
  \l 282B1
  \l 282B2
  \l 282B3
  \l 282B4
  \l 282B5
  \l 282B6
  \l 282B7
  \l 282B8
  \l 282B9
  \l 282BA
  \l 282BB
  \l 282BC
  \l 282BD
  \l 282BE
  \l 282BF
  \l 282C0
  \l 282C1
  \l 282C2
  \l 282C3
  \l 282C4
  \l 282C5
  \l 282C6
  \l 282C7
  \l 282C8
  \l 282C9
  \l 282CA
  \l 282CB
  \l 282CC
  \l 282CD
  \l 282CE
  \l 282CF
  \l 282D0
  \l 282D1
  \l 282D2
  \l 282D3
  \l 282D4
  \l 282D5
  \l 282D6
  \l 282D7
  \l 282D8
  \l 282D9
  \l 282DA
  \l 282DB
  \l 282DC
  \l 282DD
  \l 282DE
  \l 282DF
  \l 282E0
  \l 282E1
  \l 282E2
  \l 282E3
  \l 282E4
  \l 282E5
  \l 282E6
  \l 282E7
  \l 282E8
  \l 282E9
  \l 282EA
  \l 282EB
  \l 282EC
  \l 282ED
  \l 282EE
  \l 282EF
  \l 282F0
  \l 282F1
  \l 282F2
  \l 282F3
  \l 282F4
  \l 282F5
  \l 282F6
  \l 282F7
  \l 282F8
  \l 282F9
  \l 282FA
  \l 282FB
  \l 282FC
  \l 282FD
  \l 282FE
  \l 282FF
  \l 28300
  \l 28301
  \l 28302
  \l 28303
  \l 28304
  \l 28305
  \l 28306
  \l 28307
  \l 28308
  \l 28309
  \l 2830A
  \l 2830B
  \l 2830C
  \l 2830D
  \l 2830E
  \l 2830F
  \l 28310
  \l 28311
  \l 28312
  \l 28313
  \l 28314
  \l 28315
  \l 28316
  \l 28317
  \l 28318
  \l 28319
  \l 2831A
  \l 2831B
  \l 2831C
  \l 2831D
  \l 2831E
  \l 2831F
  \l 28320
  \l 28321
  \l 28322
  \l 28323
  \l 28324
  \l 28325
  \l 28326
  \l 28327
  \l 28328
  \l 28329
  \l 2832A
  \l 2832B
  \l 2832C
  \l 2832D
  \l 2832E
  \l 2832F
  \l 28330
  \l 28331
  \l 28332
  \l 28333
  \l 28334
  \l 28335
  \l 28336
  \l 28337
  \l 28338
  \l 28339
  \l 2833A
  \l 2833B
  \l 2833C
  \l 2833D
  \l 2833E
  \l 2833F
  \l 28340
  \l 28341
  \l 28342
  \l 28343
  \l 28344
  \l 28345
  \l 28346
  \l 28347
  \l 28348
  \l 28349
  \l 2834A
  \l 2834B
  \l 2834C
  \l 2834D
  \l 2834E
  \l 2834F
  \l 28350
  \l 28351
  \l 28352
  \l 28353
  \l 28354
  \l 28355
  \l 28356
  \l 28357
  \l 28358
  \l 28359
  \l 2835A
  \l 2835B
  \l 2835C
  \l 2835D
  \l 2835E
  \l 2835F
  \l 28360
  \l 28361
  \l 28362
  \l 28363
  \l 28364
  \l 28365
  \l 28366
  \l 28367
  \l 28368
  \l 28369
  \l 2836A
  \l 2836B
  \l 2836C
  \l 2836D
  \l 2836E
  \l 2836F
  \l 28370
  \l 28371
  \l 28372
  \l 28373
  \l 28374
  \l 28375
  \l 28376
  \l 28377
  \l 28378
  \l 28379
  \l 2837A
  \l 2837B
  \l 2837C
  \l 2837D
  \l 2837E
  \l 2837F
  \l 28380
  \l 28381
  \l 28382
  \l 28383
  \l 28384
  \l 28385
  \l 28386
  \l 28387
  \l 28388
  \l 28389
  \l 2838A
  \l 2838B
  \l 2838C
  \l 2838D
  \l 2838E
  \l 2838F
  \l 28390
  \l 28391
  \l 28392
  \l 28393
  \l 28394
  \l 28395
  \l 28396
  \l 28397
  \l 28398
  \l 28399
  \l 2839A
  \l 2839B
  \l 2839C
  \l 2839D
  \l 2839E
  \l 2839F
  \l 283A0
  \l 283A1
  \l 283A2
  \l 283A3
  \l 283A4
  \l 283A5
  \l 283A6
  \l 283A7
  \l 283A8
  \l 283A9
  \l 283AA
  \l 283AB
  \l 283AC
  \l 283AD
  \l 283AE
  \l 283AF
  \l 283B0
  \l 283B1
  \l 283B2
  \l 283B3
  \l 283B4
  \l 283B5
  \l 283B6
  \l 283B7
  \l 283B8
  \l 283B9
  \l 283BA
  \l 283BB
  \l 283BC
  \l 283BD
  \l 283BE
  \l 283BF
  \l 283C0
  \l 283C1
  \l 283C2
  \l 283C3
  \l 283C4
  \l 283C5
  \l 283C6
  \l 283C7
  \l 283C8
  \l 283C9
  \l 283CA
  \l 283CB
  \l 283CC
  \l 283CD
  \l 283CE
  \l 283CF
  \l 283D0
  \l 283D1
  \l 283D2
  \l 283D3
  \l 283D4
  \l 283D5
  \l 283D6
  \l 283D7
  \l 283D8
  \l 283D9
  \l 283DA
  \l 283DB
  \l 283DC
  \l 283DD
  \l 283DE
  \l 283DF
  \l 283E0
  \l 283E1
  \l 283E2
  \l 283E3
  \l 283E4
  \l 283E5
  \l 283E6
  \l 283E7
  \l 283E8
  \l 283E9
  \l 283EA
  \l 283EB
  \l 283EC
  \l 283ED
  \l 283EE
  \l 283EF
  \l 283F0
  \l 283F1
  \l 283F2
  \l 283F3
  \l 283F4
  \l 283F5
  \l 283F6
  \l 283F7
  \l 283F8
  \l 283F9
  \l 283FA
  \l 283FB
  \l 283FC
  \l 283FD
  \l 283FE
  \l 283FF
  \l 28400
  \l 28401
  \l 28402
  \l 28403
  \l 28404
  \l 28405
  \l 28406
  \l 28407
  \l 28408
  \l 28409
  \l 2840A
  \l 2840B
  \l 2840C
  \l 2840D
  \l 2840E
  \l 2840F
  \l 28410
  \l 28411
  \l 28412
  \l 28413
  \l 28414
  \l 28415
  \l 28416
  \l 28417
  \l 28418
  \l 28419
  \l 2841A
  \l 2841B
  \l 2841C
  \l 2841D
  \l 2841E
  \l 2841F
  \l 28420
  \l 28421
  \l 28422
  \l 28423
  \l 28424
  \l 28425
  \l 28426
  \l 28427
  \l 28428
  \l 28429
  \l 2842A
  \l 2842B
  \l 2842C
  \l 2842D
  \l 2842E
  \l 2842F
  \l 28430
  \l 28431
  \l 28432
  \l 28433
  \l 28434
  \l 28435
  \l 28436
  \l 28437
  \l 28438
  \l 28439
  \l 2843A
  \l 2843B
  \l 2843C
  \l 2843D
  \l 2843E
  \l 2843F
  \l 28440
  \l 28441
  \l 28442
  \l 28443
  \l 28444
  \l 28445
  \l 28446
  \l 28447
  \l 28448
  \l 28449
  \l 2844A
  \l 2844B
  \l 2844C
  \l 2844D
  \l 2844E
  \l 2844F
  \l 28450
  \l 28451
  \l 28452
  \l 28453
  \l 28454
  \l 28455
  \l 28456
  \l 28457
  \l 28458
  \l 28459
  \l 2845A
  \l 2845B
  \l 2845C
  \l 2845D
  \l 2845E
  \l 2845F
  \l 28460
  \l 28461
  \l 28462
  \l 28463
  \l 28464
  \l 28465
  \l 28466
  \l 28467
  \l 28468
  \l 28469
  \l 2846A
  \l 2846B
  \l 2846C
  \l 2846D
  \l 2846E
  \l 2846F
  \l 28470
  \l 28471
  \l 28472
  \l 28473
  \l 28474
  \l 28475
  \l 28476
  \l 28477
  \l 28478
  \l 28479
  \l 2847A
  \l 2847B
  \l 2847C
  \l 2847D
  \l 2847E
  \l 2847F
  \l 28480
  \l 28481
  \l 28482
  \l 28483
  \l 28484
  \l 28485
  \l 28486
  \l 28487
  \l 28488
  \l 28489
  \l 2848A
  \l 2848B
  \l 2848C
  \l 2848D
  \l 2848E
  \l 2848F
  \l 28490
  \l 28491
  \l 28492
  \l 28493
  \l 28494
  \l 28495
  \l 28496
  \l 28497
  \l 28498
  \l 28499
  \l 2849A
  \l 2849B
  \l 2849C
  \l 2849D
  \l 2849E
  \l 2849F
  \l 284A0
  \l 284A1
  \l 284A2
  \l 284A3
  \l 284A4
  \l 284A5
  \l 284A6
  \l 284A7
  \l 284A8
  \l 284A9
  \l 284AA
  \l 284AB
  \l 284AC
  \l 284AD
  \l 284AE
  \l 284AF
  \l 284B0
  \l 284B1
  \l 284B2
  \l 284B3
  \l 284B4
  \l 284B5
  \l 284B6
  \l 284B7
  \l 284B8
  \l 284B9
  \l 284BA
  \l 284BB
  \l 284BC
  \l 284BD
  \l 284BE
  \l 284BF
  \l 284C0
  \l 284C1
  \l 284C2
  \l 284C3
  \l 284C4
  \l 284C5
  \l 284C6
  \l 284C7
  \l 284C8
  \l 284C9
  \l 284CA
  \l 284CB
  \l 284CC
  \l 284CD
  \l 284CE
  \l 284CF
  \l 284D0
  \l 284D1
  \l 284D2
  \l 284D3
  \l 284D4
  \l 284D5
  \l 284D6
  \l 284D7
  \l 284D8
  \l 284D9
  \l 284DA
  \l 284DB
  \l 284DC
  \l 284DD
  \l 284DE
  \l 284DF
  \l 284E0
  \l 284E1
  \l 284E2
  \l 284E3
  \l 284E4
  \l 284E5
  \l 284E6
  \l 284E7
  \l 284E8
  \l 284E9
  \l 284EA
  \l 284EB
  \l 284EC
  \l 284ED
  \l 284EE
  \l 284EF
  \l 284F0
  \l 284F1
  \l 284F2
  \l 284F3
  \l 284F4
  \l 284F5
  \l 284F6
  \l 284F7
  \l 284F8
  \l 284F9
  \l 284FA
  \l 284FB
  \l 284FC
  \l 284FD
  \l 284FE
  \l 284FF
  \l 28500
  \l 28501
  \l 28502
  \l 28503
  \l 28504
  \l 28505
  \l 28506
  \l 28507
  \l 28508
  \l 28509
  \l 2850A
  \l 2850B
  \l 2850C
  \l 2850D
  \l 2850E
  \l 2850F
  \l 28510
  \l 28511
  \l 28512
  \l 28513
  \l 28514
  \l 28515
  \l 28516
  \l 28517
  \l 28518
  \l 28519
  \l 2851A
  \l 2851B
  \l 2851C
  \l 2851D
  \l 2851E
  \l 2851F
  \l 28520
  \l 28521
  \l 28522
  \l 28523
  \l 28524
  \l 28525
  \l 28526
  \l 28527
  \l 28528
  \l 28529
  \l 2852A
  \l 2852B
  \l 2852C
  \l 2852D
  \l 2852E
  \l 2852F
  \l 28530
  \l 28531
  \l 28532
  \l 28533
  \l 28534
  \l 28535
  \l 28536
  \l 28537
  \l 28538
  \l 28539
  \l 2853A
  \l 2853B
  \l 2853C
  \l 2853D
  \l 2853E
  \l 2853F
  \l 28540
  \l 28541
  \l 28542
  \l 28543
  \l 28544
  \l 28545
  \l 28546
  \l 28547
  \l 28548
  \l 28549
  \l 2854A
  \l 2854B
  \l 2854C
  \l 2854D
  \l 2854E
  \l 2854F
  \l 28550
  \l 28551
  \l 28552
  \l 28553
  \l 28554
  \l 28555
  \l 28556
  \l 28557
  \l 28558
  \l 28559
  \l 2855A
  \l 2855B
  \l 2855C
  \l 2855D
  \l 2855E
  \l 2855F
  \l 28560
  \l 28561
  \l 28562
  \l 28563
  \l 28564
  \l 28565
  \l 28566
  \l 28567
  \l 28568
  \l 28569
  \l 2856A
  \l 2856B
  \l 2856C
  \l 2856D
  \l 2856E
  \l 2856F
  \l 28570
  \l 28571
  \l 28572
  \l 28573
  \l 28574
  \l 28575
  \l 28576
  \l 28577
  \l 28578
  \l 28579
  \l 2857A
  \l 2857B
  \l 2857C
  \l 2857D
  \l 2857E
  \l 2857F
  \l 28580
  \l 28581
  \l 28582
  \l 28583
  \l 28584
  \l 28585
  \l 28586
  \l 28587
  \l 28588
  \l 28589
  \l 2858A
  \l 2858B
  \l 2858C
  \l 2858D
  \l 2858E
  \l 2858F
  \l 28590
  \l 28591
  \l 28592
  \l 28593
  \l 28594
  \l 28595
  \l 28596
  \l 28597
  \l 28598
  \l 28599
  \l 2859A
  \l 2859B
  \l 2859C
  \l 2859D
  \l 2859E
  \l 2859F
  \l 285A0
  \l 285A1
  \l 285A2
  \l 285A3
  \l 285A4
  \l 285A5
  \l 285A6
  \l 285A7
  \l 285A8
  \l 285A9
  \l 285AA
  \l 285AB
  \l 285AC
  \l 285AD
  \l 285AE
  \l 285AF
  \l 285B0
  \l 285B1
  \l 285B2
  \l 285B3
  \l 285B4
  \l 285B5
  \l 285B6
  \l 285B7
  \l 285B8
  \l 285B9
  \l 285BA
  \l 285BB
  \l 285BC
  \l 285BD
  \l 285BE
  \l 285BF
  \l 285C0
  \l 285C1
  \l 285C2
  \l 285C3
  \l 285C4
  \l 285C5
  \l 285C6
  \l 285C7
  \l 285C8
  \l 285C9
  \l 285CA
  \l 285CB
  \l 285CC
  \l 285CD
  \l 285CE
  \l 285CF
  \l 285D0
  \l 285D1
  \l 285D2
  \l 285D3
  \l 285D4
  \l 285D5
  \l 285D6
  \l 285D7
  \l 285D8
  \l 285D9
  \l 285DA
  \l 285DB
  \l 285DC
  \l 285DD
  \l 285DE
  \l 285DF
  \l 285E0
  \l 285E1
  \l 285E2
  \l 285E3
  \l 285E4
  \l 285E5
  \l 285E6
  \l 285E7
  \l 285E8
  \l 285E9
  \l 285EA
  \l 285EB
  \l 285EC
  \l 285ED
  \l 285EE
  \l 285EF
  \l 285F0
  \l 285F1
  \l 285F2
  \l 285F3
  \l 285F4
  \l 285F5
  \l 285F6
  \l 285F7
  \l 285F8
  \l 285F9
  \l 285FA
  \l 285FB
  \l 285FC
  \l 285FD
  \l 285FE
  \l 285FF
  \l 28600
  \l 28601
  \l 28602
  \l 28603
  \l 28604
  \l 28605
  \l 28606
  \l 28607
  \l 28608
  \l 28609
  \l 2860A
  \l 2860B
  \l 2860C
  \l 2860D
  \l 2860E
  \l 2860F
  \l 28610
  \l 28611
  \l 28612
  \l 28613
  \l 28614
  \l 28615
  \l 28616
  \l 28617
  \l 28618
  \l 28619
  \l 2861A
  \l 2861B
  \l 2861C
  \l 2861D
  \l 2861E
  \l 2861F
  \l 28620
  \l 28621
  \l 28622
  \l 28623
  \l 28624
  \l 28625
  \l 28626
  \l 28627
  \l 28628
  \l 28629
  \l 2862A
  \l 2862B
  \l 2862C
  \l 2862D
  \l 2862E
  \l 2862F
  \l 28630
  \l 28631
  \l 28632
  \l 28633
  \l 28634
  \l 28635
  \l 28636
  \l 28637
  \l 28638
  \l 28639
  \l 2863A
  \l 2863B
  \l 2863C
  \l 2863D
  \l 2863E
  \l 2863F
  \l 28640
  \l 28641
  \l 28642
  \l 28643
  \l 28644
  \l 28645
  \l 28646
  \l 28647
  \l 28648
  \l 28649
  \l 2864A
  \l 2864B
  \l 2864C
  \l 2864D
  \l 2864E
  \l 2864F
  \l 28650
  \l 28651
  \l 28652
  \l 28653
  \l 28654
  \l 28655
  \l 28656
  \l 28657
  \l 28658
  \l 28659
  \l 2865A
  \l 2865B
  \l 2865C
  \l 2865D
  \l 2865E
  \l 2865F
  \l 28660
  \l 28661
  \l 28662
  \l 28663
  \l 28664
  \l 28665
  \l 28666
  \l 28667
  \l 28668
  \l 28669
  \l 2866A
  \l 2866B
  \l 2866C
  \l 2866D
  \l 2866E
  \l 2866F
  \l 28670
  \l 28671
  \l 28672
  \l 28673
  \l 28674
  \l 28675
  \l 28676
  \l 28677
  \l 28678
  \l 28679
  \l 2867A
  \l 2867B
  \l 2867C
  \l 2867D
  \l 2867E
  \l 2867F
  \l 28680
  \l 28681
  \l 28682
  \l 28683
  \l 28684
  \l 28685
  \l 28686
  \l 28687
  \l 28688
  \l 28689
  \l 2868A
  \l 2868B
  \l 2868C
  \l 2868D
  \l 2868E
  \l 2868F
  \l 28690
  \l 28691
  \l 28692
  \l 28693
  \l 28694
  \l 28695
  \l 28696
  \l 28697
  \l 28698
  \l 28699
  \l 2869A
  \l 2869B
  \l 2869C
  \l 2869D
  \l 2869E
  \l 2869F
  \l 286A0
  \l 286A1
  \l 286A2
  \l 286A3
  \l 286A4
  \l 286A5
  \l 286A6
  \l 286A7
  \l 286A8
  \l 286A9
  \l 286AA
  \l 286AB
  \l 286AC
  \l 286AD
  \l 286AE
  \l 286AF
  \l 286B0
  \l 286B1
  \l 286B2
  \l 286B3
  \l 286B4
  \l 286B5
  \l 286B6
  \l 286B7
  \l 286B8
  \l 286B9
  \l 286BA
  \l 286BB
  \l 286BC
  \l 286BD
  \l 286BE
  \l 286BF
  \l 286C0
  \l 286C1
  \l 286C2
  \l 286C3
  \l 286C4
  \l 286C5
  \l 286C6
  \l 286C7
  \l 286C8
  \l 286C9
  \l 286CA
  \l 286CB
  \l 286CC
  \l 286CD
  \l 286CE
  \l 286CF
  \l 286D0
  \l 286D1
  \l 286D2
  \l 286D3
  \l 286D4
  \l 286D5
  \l 286D6
  \l 286D7
  \l 286D8
  \l 286D9
  \l 286DA
  \l 286DB
  \l 286DC
  \l 286DD
  \l 286DE
  \l 286DF
  \l 286E0
  \l 286E1
  \l 286E2
  \l 286E3
  \l 286E4
  \l 286E5
  \l 286E6
  \l 286E7
  \l 286E8
  \l 286E9
  \l 286EA
  \l 286EB
  \l 286EC
  \l 286ED
  \l 286EE
  \l 286EF
  \l 286F0
  \l 286F1
  \l 286F2
  \l 286F3
  \l 286F4
  \l 286F5
  \l 286F6
  \l 286F7
  \l 286F8
  \l 286F9
  \l 286FA
  \l 286FB
  \l 286FC
  \l 286FD
  \l 286FE
  \l 286FF
  \l 28700
  \l 28701
  \l 28702
  \l 28703
  \l 28704
  \l 28705
  \l 28706
  \l 28707
  \l 28708
  \l 28709
  \l 2870A
  \l 2870B
  \l 2870C
  \l 2870D
  \l 2870E
  \l 2870F
  \l 28710
  \l 28711
  \l 28712
  \l 28713
  \l 28714
  \l 28715
  \l 28716
  \l 28717
  \l 28718
  \l 28719
  \l 2871A
  \l 2871B
  \l 2871C
  \l 2871D
  \l 2871E
  \l 2871F
  \l 28720
  \l 28721
  \l 28722
  \l 28723
  \l 28724
  \l 28725
  \l 28726
  \l 28727
  \l 28728
  \l 28729
  \l 2872A
  \l 2872B
  \l 2872C
  \l 2872D
  \l 2872E
  \l 2872F
  \l 28730
  \l 28731
  \l 28732
  \l 28733
  \l 28734
  \l 28735
  \l 28736
  \l 28737
  \l 28738
  \l 28739
  \l 2873A
  \l 2873B
  \l 2873C
  \l 2873D
  \l 2873E
  \l 2873F
  \l 28740
  \l 28741
  \l 28742
  \l 28743
  \l 28744
  \l 28745
  \l 28746
  \l 28747
  \l 28748
  \l 28749
  \l 2874A
  \l 2874B
  \l 2874C
  \l 2874D
  \l 2874E
  \l 2874F
  \l 28750
  \l 28751
  \l 28752
  \l 28753
  \l 28754
  \l 28755
  \l 28756
  \l 28757
  \l 28758
  \l 28759
  \l 2875A
  \l 2875B
  \l 2875C
  \l 2875D
  \l 2875E
  \l 2875F
  \l 28760
  \l 28761
  \l 28762
  \l 28763
  \l 28764
  \l 28765
  \l 28766
  \l 28767
  \l 28768
  \l 28769
  \l 2876A
  \l 2876B
  \l 2876C
  \l 2876D
  \l 2876E
  \l 2876F
  \l 28770
  \l 28771
  \l 28772
  \l 28773
  \l 28774
  \l 28775
  \l 28776
  \l 28777
  \l 28778
  \l 28779
  \l 2877A
  \l 2877B
  \l 2877C
  \l 2877D
  \l 2877E
  \l 2877F
  \l 28780
  \l 28781
  \l 28782
  \l 28783
  \l 28784
  \l 28785
  \l 28786
  \l 28787
  \l 28788
  \l 28789
  \l 2878A
  \l 2878B
  \l 2878C
  \l 2878D
  \l 2878E
  \l 2878F
  \l 28790
  \l 28791
  \l 28792
  \l 28793
  \l 28794
  \l 28795
  \l 28796
  \l 28797
  \l 28798
  \l 28799
  \l 2879A
  \l 2879B
  \l 2879C
  \l 2879D
  \l 2879E
  \l 2879F
  \l 287A0
  \l 287A1
  \l 287A2
  \l 287A3
  \l 287A4
  \l 287A5
  \l 287A6
  \l 287A7
  \l 287A8
  \l 287A9
  \l 287AA
  \l 287AB
  \l 287AC
  \l 287AD
  \l 287AE
  \l 287AF
  \l 287B0
  \l 287B1
  \l 287B2
  \l 287B3
  \l 287B4
  \l 287B5
  \l 287B6
  \l 287B7
  \l 287B8
  \l 287B9
  \l 287BA
  \l 287BB
  \l 287BC
  \l 287BD
  \l 287BE
  \l 287BF
  \l 287C0
  \l 287C1
  \l 287C2
  \l 287C3
  \l 287C4
  \l 287C5
  \l 287C6
  \l 287C7
  \l 287C8
  \l 287C9
  \l 287CA
  \l 287CB
  \l 287CC
  \l 287CD
  \l 287CE
  \l 287CF
  \l 287D0
  \l 287D1
  \l 287D2
  \l 287D3
  \l 287D4
  \l 287D5
  \l 287D6
  \l 287D7
  \l 287D8
  \l 287D9
  \l 287DA
  \l 287DB
  \l 287DC
  \l 287DD
  \l 287DE
  \l 287DF
  \l 287E0
  \l 287E1
  \l 287E2
  \l 287E3
  \l 287E4
  \l 287E5
  \l 287E6
  \l 287E7
  \l 287E8
  \l 287E9
  \l 287EA
  \l 287EB
  \l 287EC
  \l 287ED
  \l 287EE
  \l 287EF
  \l 287F0
  \l 287F1
  \l 287F2
  \l 287F3
  \l 287F4
  \l 287F5
  \l 287F6
  \l 287F7
  \l 287F8
  \l 287F9
  \l 287FA
  \l 287FB
  \l 287FC
  \l 287FD
  \l 287FE
  \l 287FF
  \l 28800
  \l 28801
  \l 28802
  \l 28803
  \l 28804
  \l 28805
  \l 28806
  \l 28807
  \l 28808
  \l 28809
  \l 2880A
  \l 2880B
  \l 2880C
  \l 2880D
  \l 2880E
  \l 2880F
  \l 28810
  \l 28811
  \l 28812
  \l 28813
  \l 28814
  \l 28815
  \l 28816
  \l 28817
  \l 28818
  \l 28819
  \l 2881A
  \l 2881B
  \l 2881C
  \l 2881D
  \l 2881E
  \l 2881F
  \l 28820
  \l 28821
  \l 28822
  \l 28823
  \l 28824
  \l 28825
  \l 28826
  \l 28827
  \l 28828
  \l 28829
  \l 2882A
  \l 2882B
  \l 2882C
  \l 2882D
  \l 2882E
  \l 2882F
  \l 28830
  \l 28831
  \l 28832
  \l 28833
  \l 28834
  \l 28835
  \l 28836
  \l 28837
  \l 28838
  \l 28839
  \l 2883A
  \l 2883B
  \l 2883C
  \l 2883D
  \l 2883E
  \l 2883F
  \l 28840
  \l 28841
  \l 28842
  \l 28843
  \l 28844
  \l 28845
  \l 28846
  \l 28847
  \l 28848
  \l 28849
  \l 2884A
  \l 2884B
  \l 2884C
  \l 2884D
  \l 2884E
  \l 2884F
  \l 28850
  \l 28851
  \l 28852
  \l 28853
  \l 28854
  \l 28855
  \l 28856
  \l 28857
  \l 28858
  \l 28859
  \l 2885A
  \l 2885B
  \l 2885C
  \l 2885D
  \l 2885E
  \l 2885F
  \l 28860
  \l 28861
  \l 28862
  \l 28863
  \l 28864
  \l 28865
  \l 28866
  \l 28867
  \l 28868
  \l 28869
  \l 2886A
  \l 2886B
  \l 2886C
  \l 2886D
  \l 2886E
  \l 2886F
  \l 28870
  \l 28871
  \l 28872
  \l 28873
  \l 28874
  \l 28875
  \l 28876
  \l 28877
  \l 28878
  \l 28879
  \l 2887A
  \l 2887B
  \l 2887C
  \l 2887D
  \l 2887E
  \l 2887F
  \l 28880
  \l 28881
  \l 28882
  \l 28883
  \l 28884
  \l 28885
  \l 28886
  \l 28887
  \l 28888
  \l 28889
  \l 2888A
  \l 2888B
  \l 2888C
  \l 2888D
  \l 2888E
  \l 2888F
  \l 28890
  \l 28891
  \l 28892
  \l 28893
  \l 28894
  \l 28895
  \l 28896
  \l 28897
  \l 28898
  \l 28899
  \l 2889A
  \l 2889B
  \l 2889C
  \l 2889D
  \l 2889E
  \l 2889F
  \l 288A0
  \l 288A1
  \l 288A2
  \l 288A3
  \l 288A4
  \l 288A5
  \l 288A6
  \l 288A7
  \l 288A8
  \l 288A9
  \l 288AA
  \l 288AB
  \l 288AC
  \l 288AD
  \l 288AE
  \l 288AF
  \l 288B0
  \l 288B1
  \l 288B2
  \l 288B3
  \l 288B4
  \l 288B5
  \l 288B6
  \l 288B7
  \l 288B8
  \l 288B9
  \l 288BA
  \l 288BB
  \l 288BC
  \l 288BD
  \l 288BE
  \l 288BF
  \l 288C0
  \l 288C1
  \l 288C2
  \l 288C3
  \l 288C4
  \l 288C5
  \l 288C6
  \l 288C7
  \l 288C8
  \l 288C9
  \l 288CA
  \l 288CB
  \l 288CC
  \l 288CD
  \l 288CE
  \l 288CF
  \l 288D0
  \l 288D1
  \l 288D2
  \l 288D3
  \l 288D4
  \l 288D5
  \l 288D6
  \l 288D7
  \l 288D8
  \l 288D9
  \l 288DA
  \l 288DB
  \l 288DC
  \l 288DD
  \l 288DE
  \l 288DF
  \l 288E0
  \l 288E1
  \l 288E2
  \l 288E3
  \l 288E4
  \l 288E5
  \l 288E6
  \l 288E7
  \l 288E8
  \l 288E9
  \l 288EA
  \l 288EB
  \l 288EC
  \l 288ED
  \l 288EE
  \l 288EF
  \l 288F0
  \l 288F1
  \l 288F2
  \l 288F3
  \l 288F4
  \l 288F5
  \l 288F6
  \l 288F7
  \l 288F8
  \l 288F9
  \l 288FA
  \l 288FB
  \l 288FC
  \l 288FD
  \l 288FE
  \l 288FF
  \l 28900
  \l 28901
  \l 28902
  \l 28903
  \l 28904
  \l 28905
  \l 28906
  \l 28907
  \l 28908
  \l 28909
  \l 2890A
  \l 2890B
  \l 2890C
  \l 2890D
  \l 2890E
  \l 2890F
  \l 28910
  \l 28911
  \l 28912
  \l 28913
  \l 28914
  \l 28915
  \l 28916
  \l 28917
  \l 28918
  \l 28919
  \l 2891A
  \l 2891B
  \l 2891C
  \l 2891D
  \l 2891E
  \l 2891F
  \l 28920
  \l 28921
  \l 28922
  \l 28923
  \l 28924
  \l 28925
  \l 28926
  \l 28927
  \l 28928
  \l 28929
  \l 2892A
  \l 2892B
  \l 2892C
  \l 2892D
  \l 2892E
  \l 2892F
  \l 28930
  \l 28931
  \l 28932
  \l 28933
  \l 28934
  \l 28935
  \l 28936
  \l 28937
  \l 28938
  \l 28939
  \l 2893A
  \l 2893B
  \l 2893C
  \l 2893D
  \l 2893E
  \l 2893F
  \l 28940
  \l 28941
  \l 28942
  \l 28943
  \l 28944
  \l 28945
  \l 28946
  \l 28947
  \l 28948
  \l 28949
  \l 2894A
  \l 2894B
  \l 2894C
  \l 2894D
  \l 2894E
  \l 2894F
  \l 28950
  \l 28951
  \l 28952
  \l 28953
  \l 28954
  \l 28955
  \l 28956
  \l 28957
  \l 28958
  \l 28959
  \l 2895A
  \l 2895B
  \l 2895C
  \l 2895D
  \l 2895E
  \l 2895F
  \l 28960
  \l 28961
  \l 28962
  \l 28963
  \l 28964
  \l 28965
  \l 28966
  \l 28967
  \l 28968
  \l 28969
  \l 2896A
  \l 2896B
  \l 2896C
  \l 2896D
  \l 2896E
  \l 2896F
  \l 28970
  \l 28971
  \l 28972
  \l 28973
  \l 28974
  \l 28975
  \l 28976
  \l 28977
  \l 28978
  \l 28979
  \l 2897A
  \l 2897B
  \l 2897C
  \l 2897D
  \l 2897E
  \l 2897F
  \l 28980
  \l 28981
  \l 28982
  \l 28983
  \l 28984
  \l 28985
  \l 28986
  \l 28987
  \l 28988
  \l 28989
  \l 2898A
  \l 2898B
  \l 2898C
  \l 2898D
  \l 2898E
  \l 2898F
  \l 28990
  \l 28991
  \l 28992
  \l 28993
  \l 28994
  \l 28995
  \l 28996
  \l 28997
  \l 28998
  \l 28999
  \l 2899A
  \l 2899B
  \l 2899C
  \l 2899D
  \l 2899E
  \l 2899F
  \l 289A0
  \l 289A1
  \l 289A2
  \l 289A3
  \l 289A4
  \l 289A5
  \l 289A6
  \l 289A7
  \l 289A8
  \l 289A9
  \l 289AA
  \l 289AB
  \l 289AC
  \l 289AD
  \l 289AE
  \l 289AF
  \l 289B0
  \l 289B1
  \l 289B2
  \l 289B3
  \l 289B4
  \l 289B5
  \l 289B6
  \l 289B7
  \l 289B8
  \l 289B9
  \l 289BA
  \l 289BB
  \l 289BC
  \l 289BD
  \l 289BE
  \l 289BF
  \l 289C0
  \l 289C1
  \l 289C2
  \l 289C3
  \l 289C4
  \l 289C5
  \l 289C6
  \l 289C7
  \l 289C8
  \l 289C9
  \l 289CA
  \l 289CB
  \l 289CC
  \l 289CD
  \l 289CE
  \l 289CF
  \l 289D0
  \l 289D1
  \l 289D2
  \l 289D3
  \l 289D4
  \l 289D5
  \l 289D6
  \l 289D7
  \l 289D8
  \l 289D9
  \l 289DA
  \l 289DB
  \l 289DC
  \l 289DD
  \l 289DE
  \l 289DF
  \l 289E0
  \l 289E1
  \l 289E2
  \l 289E3
  \l 289E4
  \l 289E5
  \l 289E6
  \l 289E7
  \l 289E8
  \l 289E9
  \l 289EA
  \l 289EB
  \l 289EC
  \l 289ED
  \l 289EE
  \l 289EF
  \l 289F0
  \l 289F1
  \l 289F2
  \l 289F3
  \l 289F4
  \l 289F5
  \l 289F6
  \l 289F7
  \l 289F8
  \l 289F9
  \l 289FA
  \l 289FB
  \l 289FC
  \l 289FD
  \l 289FE
  \l 289FF
  \l 28A00
  \l 28A01
  \l 28A02
  \l 28A03
  \l 28A04
  \l 28A05
  \l 28A06
  \l 28A07
  \l 28A08
  \l 28A09
  \l 28A0A
  \l 28A0B
  \l 28A0C
  \l 28A0D
  \l 28A0E
  \l 28A0F
  \l 28A10
  \l 28A11
  \l 28A12
  \l 28A13
  \l 28A14
  \l 28A15
  \l 28A16
  \l 28A17
  \l 28A18
  \l 28A19
  \l 28A1A
  \l 28A1B
  \l 28A1C
  \l 28A1D
  \l 28A1E
  \l 28A1F
  \l 28A20
  \l 28A21
  \l 28A22
  \l 28A23
  \l 28A24
  \l 28A25
  \l 28A26
  \l 28A27
  \l 28A28
  \l 28A29
  \l 28A2A
  \l 28A2B
  \l 28A2C
  \l 28A2D
  \l 28A2E
  \l 28A2F
  \l 28A30
  \l 28A31
  \l 28A32
  \l 28A33
  \l 28A34
  \l 28A35
  \l 28A36
  \l 28A37
  \l 28A38
  \l 28A39
  \l 28A3A
  \l 28A3B
  \l 28A3C
  \l 28A3D
  \l 28A3E
  \l 28A3F
  \l 28A40
  \l 28A41
  \l 28A42
  \l 28A43
  \l 28A44
  \l 28A45
  \l 28A46
  \l 28A47
  \l 28A48
  \l 28A49
  \l 28A4A
  \l 28A4B
  \l 28A4C
  \l 28A4D
  \l 28A4E
  \l 28A4F
  \l 28A50
  \l 28A51
  \l 28A52
  \l 28A53
  \l 28A54
  \l 28A55
  \l 28A56
  \l 28A57
  \l 28A58
  \l 28A59
  \l 28A5A
  \l 28A5B
  \l 28A5C
  \l 28A5D
  \l 28A5E
  \l 28A5F
  \l 28A60
  \l 28A61
  \l 28A62
  \l 28A63
  \l 28A64
  \l 28A65
  \l 28A66
  \l 28A67
  \l 28A68
  \l 28A69
  \l 28A6A
  \l 28A6B
  \l 28A6C
  \l 28A6D
  \l 28A6E
  \l 28A6F
  \l 28A70
  \l 28A71
  \l 28A72
  \l 28A73
  \l 28A74
  \l 28A75
  \l 28A76
  \l 28A77
  \l 28A78
  \l 28A79
  \l 28A7A
  \l 28A7B
  \l 28A7C
  \l 28A7D
  \l 28A7E
  \l 28A7F
  \l 28A80
  \l 28A81
  \l 28A82
  \l 28A83
  \l 28A84
  \l 28A85
  \l 28A86
  \l 28A87
  \l 28A88
  \l 28A89
  \l 28A8A
  \l 28A8B
  \l 28A8C
  \l 28A8D
  \l 28A8E
  \l 28A8F
  \l 28A90
  \l 28A91
  \l 28A92
  \l 28A93
  \l 28A94
  \l 28A95
  \l 28A96
  \l 28A97
  \l 28A98
  \l 28A99
  \l 28A9A
  \l 28A9B
  \l 28A9C
  \l 28A9D
  \l 28A9E
  \l 28A9F
  \l 28AA0
  \l 28AA1
  \l 28AA2
  \l 28AA3
  \l 28AA4
  \l 28AA5
  \l 28AA6
  \l 28AA7
  \l 28AA8
  \l 28AA9
  \l 28AAA
  \l 28AAB
  \l 28AAC
  \l 28AAD
  \l 28AAE
  \l 28AAF
  \l 28AB0
  \l 28AB1
  \l 28AB2
  \l 28AB3
  \l 28AB4
  \l 28AB5
  \l 28AB6
  \l 28AB7
  \l 28AB8
  \l 28AB9
  \l 28ABA
  \l 28ABB
  \l 28ABC
  \l 28ABD
  \l 28ABE
  \l 28ABF
  \l 28AC0
  \l 28AC1
  \l 28AC2
  \l 28AC3
  \l 28AC4
  \l 28AC5
  \l 28AC6
  \l 28AC7
  \l 28AC8
  \l 28AC9
  \l 28ACA
  \l 28ACB
  \l 28ACC
  \l 28ACD
  \l 28ACE
  \l 28ACF
  \l 28AD0
  \l 28AD1
  \l 28AD2
  \l 28AD3
  \l 28AD4
  \l 28AD5
  \l 28AD6
  \l 28AD7
  \l 28AD8
  \l 28AD9
  \l 28ADA
  \l 28ADB
  \l 28ADC
  \l 28ADD
  \l 28ADE
  \l 28ADF
  \l 28AE0
  \l 28AE1
  \l 28AE2
  \l 28AE3
  \l 28AE4
  \l 28AE5
  \l 28AE6
  \l 28AE7
  \l 28AE8
  \l 28AE9
  \l 28AEA
  \l 28AEB
  \l 28AEC
  \l 28AED
  \l 28AEE
  \l 28AEF
  \l 28AF0
  \l 28AF1
  \l 28AF2
  \l 28AF3
  \l 28AF4
  \l 28AF5
  \l 28AF6
  \l 28AF7
  \l 28AF8
  \l 28AF9
  \l 28AFA
  \l 28AFB
  \l 28AFC
  \l 28AFD
  \l 28AFE
  \l 28AFF
  \l 28B00
  \l 28B01
  \l 28B02
  \l 28B03
  \l 28B04
  \l 28B05
  \l 28B06
  \l 28B07
  \l 28B08
  \l 28B09
  \l 28B0A
  \l 28B0B
  \l 28B0C
  \l 28B0D
  \l 28B0E
  \l 28B0F
  \l 28B10
  \l 28B11
  \l 28B12
  \l 28B13
  \l 28B14
  \l 28B15
  \l 28B16
  \l 28B17
  \l 28B18
  \l 28B19
  \l 28B1A
  \l 28B1B
  \l 28B1C
  \l 28B1D
  \l 28B1E
  \l 28B1F
  \l 28B20
  \l 28B21
  \l 28B22
  \l 28B23
  \l 28B24
  \l 28B25
  \l 28B26
  \l 28B27
  \l 28B28
  \l 28B29
  \l 28B2A
  \l 28B2B
  \l 28B2C
  \l 28B2D
  \l 28B2E
  \l 28B2F
  \l 28B30
  \l 28B31
  \l 28B32
  \l 28B33
  \l 28B34
  \l 28B35
  \l 28B36
  \l 28B37
  \l 28B38
  \l 28B39
  \l 28B3A
  \l 28B3B
  \l 28B3C
  \l 28B3D
  \l 28B3E
  \l 28B3F
  \l 28B40
  \l 28B41
  \l 28B42
  \l 28B43
  \l 28B44
  \l 28B45
  \l 28B46
  \l 28B47
  \l 28B48
  \l 28B49
  \l 28B4A
  \l 28B4B
  \l 28B4C
  \l 28B4D
  \l 28B4E
  \l 28B4F
  \l 28B50
  \l 28B51
  \l 28B52
  \l 28B53
  \l 28B54
  \l 28B55
  \l 28B56
  \l 28B57
  \l 28B58
  \l 28B59
  \l 28B5A
  \l 28B5B
  \l 28B5C
  \l 28B5D
  \l 28B5E
  \l 28B5F
  \l 28B60
  \l 28B61
  \l 28B62
  \l 28B63
  \l 28B64
  \l 28B65
  \l 28B66
  \l 28B67
  \l 28B68
  \l 28B69
  \l 28B6A
  \l 28B6B
  \l 28B6C
  \l 28B6D
  \l 28B6E
  \l 28B6F
  \l 28B70
  \l 28B71
  \l 28B72
  \l 28B73
  \l 28B74
  \l 28B75
  \l 28B76
  \l 28B77
  \l 28B78
  \l 28B79
  \l 28B7A
  \l 28B7B
  \l 28B7C
  \l 28B7D
  \l 28B7E
  \l 28B7F
  \l 28B80
  \l 28B81
  \l 28B82
  \l 28B83
  \l 28B84
  \l 28B85
  \l 28B86
  \l 28B87
  \l 28B88
  \l 28B89
  \l 28B8A
  \l 28B8B
  \l 28B8C
  \l 28B8D
  \l 28B8E
  \l 28B8F
  \l 28B90
  \l 28B91
  \l 28B92
  \l 28B93
  \l 28B94
  \l 28B95
  \l 28B96
  \l 28B97
  \l 28B98
  \l 28B99
  \l 28B9A
  \l 28B9B
  \l 28B9C
  \l 28B9D
  \l 28B9E
  \l 28B9F
  \l 28BA0
  \l 28BA1
  \l 28BA2
  \l 28BA3
  \l 28BA4
  \l 28BA5
  \l 28BA6
  \l 28BA7
  \l 28BA8
  \l 28BA9
  \l 28BAA
  \l 28BAB
  \l 28BAC
  \l 28BAD
  \l 28BAE
  \l 28BAF
  \l 28BB0
  \l 28BB1
  \l 28BB2
  \l 28BB3
  \l 28BB4
  \l 28BB5
  \l 28BB6
  \l 28BB7
  \l 28BB8
  \l 28BB9
  \l 28BBA
  \l 28BBB
  \l 28BBC
  \l 28BBD
  \l 28BBE
  \l 28BBF
  \l 28BC0
  \l 28BC1
  \l 28BC2
  \l 28BC3
  \l 28BC4
  \l 28BC5
  \l 28BC6
  \l 28BC7
  \l 28BC8
  \l 28BC9
  \l 28BCA
  \l 28BCB
  \l 28BCC
  \l 28BCD
  \l 28BCE
  \l 28BCF
  \l 28BD0
  \l 28BD1
  \l 28BD2
  \l 28BD3
  \l 28BD4
  \l 28BD5
  \l 28BD6
  \l 28BD7
  \l 28BD8
  \l 28BD9
  \l 28BDA
  \l 28BDB
  \l 28BDC
  \l 28BDD
  \l 28BDE
  \l 28BDF
  \l 28BE0
  \l 28BE1
  \l 28BE2
  \l 28BE3
  \l 28BE4
  \l 28BE5
  \l 28BE6
  \l 28BE7
  \l 28BE8
  \l 28BE9
  \l 28BEA
  \l 28BEB
  \l 28BEC
  \l 28BED
  \l 28BEE
  \l 28BEF
  \l 28BF0
  \l 28BF1
  \l 28BF2
  \l 28BF3
  \l 28BF4
  \l 28BF5
  \l 28BF6
  \l 28BF7
  \l 28BF8
  \l 28BF9
  \l 28BFA
  \l 28BFB
  \l 28BFC
  \l 28BFD
  \l 28BFE
  \l 28BFF
  \l 28C00
  \l 28C01
  \l 28C02
  \l 28C03
  \l 28C04
  \l 28C05
  \l 28C06
  \l 28C07
  \l 28C08
  \l 28C09
  \l 28C0A
  \l 28C0B
  \l 28C0C
  \l 28C0D
  \l 28C0E
  \l 28C0F
  \l 28C10
  \l 28C11
  \l 28C12
  \l 28C13
  \l 28C14
  \l 28C15
  \l 28C16
  \l 28C17
  \l 28C18
  \l 28C19
  \l 28C1A
  \l 28C1B
  \l 28C1C
  \l 28C1D
  \l 28C1E
  \l 28C1F
  \l 28C20
  \l 28C21
  \l 28C22
  \l 28C23
  \l 28C24
  \l 28C25
  \l 28C26
  \l 28C27
  \l 28C28
  \l 28C29
  \l 28C2A
  \l 28C2B
  \l 28C2C
  \l 28C2D
  \l 28C2E
  \l 28C2F
  \l 28C30
  \l 28C31
  \l 28C32
  \l 28C33
  \l 28C34
  \l 28C35
  \l 28C36
  \l 28C37
  \l 28C38
  \l 28C39
  \l 28C3A
  \l 28C3B
  \l 28C3C
  \l 28C3D
  \l 28C3E
  \l 28C3F
  \l 28C40
  \l 28C41
  \l 28C42
  \l 28C43
  \l 28C44
  \l 28C45
  \l 28C46
  \l 28C47
  \l 28C48
  \l 28C49
  \l 28C4A
  \l 28C4B
  \l 28C4C
  \l 28C4D
  \l 28C4E
  \l 28C4F
  \l 28C50
  \l 28C51
  \l 28C52
  \l 28C53
  \l 28C54
  \l 28C55
  \l 28C56
  \l 28C57
  \l 28C58
  \l 28C59
  \l 28C5A
  \l 28C5B
  \l 28C5C
  \l 28C5D
  \l 28C5E
  \l 28C5F
  \l 28C60
  \l 28C61
  \l 28C62
  \l 28C63
  \l 28C64
  \l 28C65
  \l 28C66
  \l 28C67
  \l 28C68
  \l 28C69
  \l 28C6A
  \l 28C6B
  \l 28C6C
  \l 28C6D
  \l 28C6E
  \l 28C6F
  \l 28C70
  \l 28C71
  \l 28C72
  \l 28C73
  \l 28C74
  \l 28C75
  \l 28C76
  \l 28C77
  \l 28C78
  \l 28C79
  \l 28C7A
  \l 28C7B
  \l 28C7C
  \l 28C7D
  \l 28C7E
  \l 28C7F
  \l 28C80
  \l 28C81
  \l 28C82
  \l 28C83
  \l 28C84
  \l 28C85
  \l 28C86
  \l 28C87
  \l 28C88
  \l 28C89
  \l 28C8A
  \l 28C8B
  \l 28C8C
  \l 28C8D
  \l 28C8E
  \l 28C8F
  \l 28C90
  \l 28C91
  \l 28C92
  \l 28C93
  \l 28C94
  \l 28C95
  \l 28C96
  \l 28C97
  \l 28C98
  \l 28C99
  \l 28C9A
  \l 28C9B
  \l 28C9C
  \l 28C9D
  \l 28C9E
  \l 28C9F
  \l 28CA0
  \l 28CA1
  \l 28CA2
  \l 28CA3
  \l 28CA4
  \l 28CA5
  \l 28CA6
  \l 28CA7
  \l 28CA8
  \l 28CA9
  \l 28CAA
  \l 28CAB
  \l 28CAC
  \l 28CAD
  \l 28CAE
  \l 28CAF
  \l 28CB0
  \l 28CB1
  \l 28CB2
  \l 28CB3
  \l 28CB4
  \l 28CB5
  \l 28CB6
  \l 28CB7
  \l 28CB8
  \l 28CB9
  \l 28CBA
  \l 28CBB
  \l 28CBC
  \l 28CBD
  \l 28CBE
  \l 28CBF
  \l 28CC0
  \l 28CC1
  \l 28CC2
  \l 28CC3
  \l 28CC4
  \l 28CC5
  \l 28CC6
  \l 28CC7
  \l 28CC8
  \l 28CC9
  \l 28CCA
  \l 28CCB
  \l 28CCC
  \l 28CCD
  \l 28CCE
  \l 28CCF
  \l 28CD0
  \l 28CD1
  \l 28CD2
  \l 28CD3
  \l 28CD4
  \l 28CD5
  \l 28CD6
  \l 28CD7
  \l 28CD8
  \l 28CD9
  \l 28CDA
  \l 28CDB
  \l 28CDC
  \l 28CDD
  \l 28CDE
  \l 28CDF
  \l 28CE0
  \l 28CE1
  \l 28CE2
  \l 28CE3
  \l 28CE4
  \l 28CE5
  \l 28CE6
  \l 28CE7
  \l 28CE8
  \l 28CE9
  \l 28CEA
  \l 28CEB
  \l 28CEC
  \l 28CED
  \l 28CEE
  \l 28CEF
  \l 28CF0
  \l 28CF1
  \l 28CF2
  \l 28CF3
  \l 28CF4
  \l 28CF5
  \l 28CF6
  \l 28CF7
  \l 28CF8
  \l 28CF9
  \l 28CFA
  \l 28CFB
  \l 28CFC
  \l 28CFD
  \l 28CFE
  \l 28CFF
  \l 28D00
  \l 28D01
  \l 28D02
  \l 28D03
  \l 28D04
  \l 28D05
  \l 28D06
  \l 28D07
  \l 28D08
  \l 28D09
  \l 28D0A
  \l 28D0B
  \l 28D0C
  \l 28D0D
  \l 28D0E
  \l 28D0F
  \l 28D10
  \l 28D11
  \l 28D12
  \l 28D13
  \l 28D14
  \l 28D15
  \l 28D16
  \l 28D17
  \l 28D18
  \l 28D19
  \l 28D1A
  \l 28D1B
  \l 28D1C
  \l 28D1D
  \l 28D1E
  \l 28D1F
  \l 28D20
  \l 28D21
  \l 28D22
  \l 28D23
  \l 28D24
  \l 28D25
  \l 28D26
  \l 28D27
  \l 28D28
  \l 28D29
  \l 28D2A
  \l 28D2B
  \l 28D2C
  \l 28D2D
  \l 28D2E
  \l 28D2F
  \l 28D30
  \l 28D31
  \l 28D32
  \l 28D33
  \l 28D34
  \l 28D35
  \l 28D36
  \l 28D37
  \l 28D38
  \l 28D39
  \l 28D3A
  \l 28D3B
  \l 28D3C
  \l 28D3D
  \l 28D3E
  \l 28D3F
  \l 28D40
  \l 28D41
  \l 28D42
  \l 28D43
  \l 28D44
  \l 28D45
  \l 28D46
  \l 28D47
  \l 28D48
  \l 28D49
  \l 28D4A
  \l 28D4B
  \l 28D4C
  \l 28D4D
  \l 28D4E
  \l 28D4F
  \l 28D50
  \l 28D51
  \l 28D52
  \l 28D53
  \l 28D54
  \l 28D55
  \l 28D56
  \l 28D57
  \l 28D58
  \l 28D59
  \l 28D5A
  \l 28D5B
  \l 28D5C
  \l 28D5D
  \l 28D5E
  \l 28D5F
  \l 28D60
  \l 28D61
  \l 28D62
  \l 28D63
  \l 28D64
  \l 28D65
  \l 28D66
  \l 28D67
  \l 28D68
  \l 28D69
  \l 28D6A
  \l 28D6B
  \l 28D6C
  \l 28D6D
  \l 28D6E
  \l 28D6F
  \l 28D70
  \l 28D71
  \l 28D72
  \l 28D73
  \l 28D74
  \l 28D75
  \l 28D76
  \l 28D77
  \l 28D78
  \l 28D79
  \l 28D7A
  \l 28D7B
  \l 28D7C
  \l 28D7D
  \l 28D7E
  \l 28D7F
  \l 28D80
  \l 28D81
  \l 28D82
  \l 28D83
  \l 28D84
  \l 28D85
  \l 28D86
  \l 28D87
  \l 28D88
  \l 28D89
  \l 28D8A
  \l 28D8B
  \l 28D8C
  \l 28D8D
  \l 28D8E
  \l 28D8F
  \l 28D90
  \l 28D91
  \l 28D92
  \l 28D93
  \l 28D94
  \l 28D95
  \l 28D96
  \l 28D97
  \l 28D98
  \l 28D99
  \l 28D9A
  \l 28D9B
  \l 28D9C
  \l 28D9D
  \l 28D9E
  \l 28D9F
  \l 28DA0
  \l 28DA1
  \l 28DA2
  \l 28DA3
  \l 28DA4
  \l 28DA5
  \l 28DA6
  \l 28DA7
  \l 28DA8
  \l 28DA9
  \l 28DAA
  \l 28DAB
  \l 28DAC
  \l 28DAD
  \l 28DAE
  \l 28DAF
  \l 28DB0
  \l 28DB1
  \l 28DB2
  \l 28DB3
  \l 28DB4
  \l 28DB5
  \l 28DB6
  \l 28DB7
  \l 28DB8
  \l 28DB9
  \l 28DBA
  \l 28DBB
  \l 28DBC
  \l 28DBD
  \l 28DBE
  \l 28DBF
  \l 28DC0
  \l 28DC1
  \l 28DC2
  \l 28DC3
  \l 28DC4
  \l 28DC5
  \l 28DC6
  \l 28DC7
  \l 28DC8
  \l 28DC9
  \l 28DCA
  \l 28DCB
  \l 28DCC
  \l 28DCD
  \l 28DCE
  \l 28DCF
  \l 28DD0
  \l 28DD1
  \l 28DD2
  \l 28DD3
  \l 28DD4
  \l 28DD5
  \l 28DD6
  \l 28DD7
  \l 28DD8
  \l 28DD9
  \l 28DDA
  \l 28DDB
  \l 28DDC
  \l 28DDD
  \l 28DDE
  \l 28DDF
  \l 28DE0
  \l 28DE1
  \l 28DE2
  \l 28DE3
  \l 28DE4
  \l 28DE5
  \l 28DE6
  \l 28DE7
  \l 28DE8
  \l 28DE9
  \l 28DEA
  \l 28DEB
  \l 28DEC
  \l 28DED
  \l 28DEE
  \l 28DEF
  \l 28DF0
  \l 28DF1
  \l 28DF2
  \l 28DF3
  \l 28DF4
  \l 28DF5
  \l 28DF6
  \l 28DF7
  \l 28DF8
  \l 28DF9
  \l 28DFA
  \l 28DFB
  \l 28DFC
  \l 28DFD
  \l 28DFE
  \l 28DFF
  \l 28E00
  \l 28E01
  \l 28E02
  \l 28E03
  \l 28E04
  \l 28E05
  \l 28E06
  \l 28E07
  \l 28E08
  \l 28E09
  \l 28E0A
  \l 28E0B
  \l 28E0C
  \l 28E0D
  \l 28E0E
  \l 28E0F
  \l 28E10
  \l 28E11
  \l 28E12
  \l 28E13
  \l 28E14
  \l 28E15
  \l 28E16
  \l 28E17
  \l 28E18
  \l 28E19
  \l 28E1A
  \l 28E1B
  \l 28E1C
  \l 28E1D
  \l 28E1E
  \l 28E1F
  \l 28E20
  \l 28E21
  \l 28E22
  \l 28E23
  \l 28E24
  \l 28E25
  \l 28E26
  \l 28E27
  \l 28E28
  \l 28E29
  \l 28E2A
  \l 28E2B
  \l 28E2C
  \l 28E2D
  \l 28E2E
  \l 28E2F
  \l 28E30
  \l 28E31
  \l 28E32
  \l 28E33
  \l 28E34
  \l 28E35
  \l 28E36
  \l 28E37
  \l 28E38
  \l 28E39
  \l 28E3A
  \l 28E3B
  \l 28E3C
  \l 28E3D
  \l 28E3E
  \l 28E3F
  \l 28E40
  \l 28E41
  \l 28E42
  \l 28E43
  \l 28E44
  \l 28E45
  \l 28E46
  \l 28E47
  \l 28E48
  \l 28E49
  \l 28E4A
  \l 28E4B
  \l 28E4C
  \l 28E4D
  \l 28E4E
  \l 28E4F
  \l 28E50
  \l 28E51
  \l 28E52
  \l 28E53
  \l 28E54
  \l 28E55
  \l 28E56
  \l 28E57
  \l 28E58
  \l 28E59
  \l 28E5A
  \l 28E5B
  \l 28E5C
  \l 28E5D
  \l 28E5E
  \l 28E5F
  \l 28E60
  \l 28E61
  \l 28E62
  \l 28E63
  \l 28E64
  \l 28E65
  \l 28E66
  \l 28E67
  \l 28E68
  \l 28E69
  \l 28E6A
  \l 28E6B
  \l 28E6C
  \l 28E6D
  \l 28E6E
  \l 28E6F
  \l 28E70
  \l 28E71
  \l 28E72
  \l 28E73
  \l 28E74
  \l 28E75
  \l 28E76
  \l 28E77
  \l 28E78
  \l 28E79
  \l 28E7A
  \l 28E7B
  \l 28E7C
  \l 28E7D
  \l 28E7E
  \l 28E7F
  \l 28E80
  \l 28E81
  \l 28E82
  \l 28E83
  \l 28E84
  \l 28E85
  \l 28E86
  \l 28E87
  \l 28E88
  \l 28E89
  \l 28E8A
  \l 28E8B
  \l 28E8C
  \l 28E8D
  \l 28E8E
  \l 28E8F
  \l 28E90
  \l 28E91
  \l 28E92
  \l 28E93
  \l 28E94
  \l 28E95
  \l 28E96
  \l 28E97
  \l 28E98
  \l 28E99
  \l 28E9A
  \l 28E9B
  \l 28E9C
  \l 28E9D
  \l 28E9E
  \l 28E9F
  \l 28EA0
  \l 28EA1
  \l 28EA2
  \l 28EA3
  \l 28EA4
  \l 28EA5
  \l 28EA6
  \l 28EA7
  \l 28EA8
  \l 28EA9
  \l 28EAA
  \l 28EAB
  \l 28EAC
  \l 28EAD
  \l 28EAE
  \l 28EAF
  \l 28EB0
  \l 28EB1
  \l 28EB2
  \l 28EB3
  \l 28EB4
  \l 28EB5
  \l 28EB6
  \l 28EB7
  \l 28EB8
  \l 28EB9
  \l 28EBA
  \l 28EBB
  \l 28EBC
  \l 28EBD
  \l 28EBE
  \l 28EBF
  \l 28EC0
  \l 28EC1
  \l 28EC2
  \l 28EC3
  \l 28EC4
  \l 28EC5
  \l 28EC6
  \l 28EC7
  \l 28EC8
  \l 28EC9
  \l 28ECA
  \l 28ECB
  \l 28ECC
  \l 28ECD
  \l 28ECE
  \l 28ECF
  \l 28ED0
  \l 28ED1
  \l 28ED2
  \l 28ED3
  \l 28ED4
  \l 28ED5
  \l 28ED6
  \l 28ED7
  \l 28ED8
  \l 28ED9
  \l 28EDA
  \l 28EDB
  \l 28EDC
  \l 28EDD
  \l 28EDE
  \l 28EDF
  \l 28EE0
  \l 28EE1
  \l 28EE2
  \l 28EE3
  \l 28EE4
  \l 28EE5
  \l 28EE6
  \l 28EE7
  \l 28EE8
  \l 28EE9
  \l 28EEA
  \l 28EEB
  \l 28EEC
  \l 28EED
  \l 28EEE
  \l 28EEF
  \l 28EF0
  \l 28EF1
  \l 28EF2
  \l 28EF3
  \l 28EF4
  \l 28EF5
  \l 28EF6
  \l 28EF7
  \l 28EF8
  \l 28EF9
  \l 28EFA
  \l 28EFB
  \l 28EFC
  \l 28EFD
  \l 28EFE
  \l 28EFF
  \l 28F00
  \l 28F01
  \l 28F02
  \l 28F03
  \l 28F04
  \l 28F05
  \l 28F06
  \l 28F07
  \l 28F08
  \l 28F09
  \l 28F0A
  \l 28F0B
  \l 28F0C
  \l 28F0D
  \l 28F0E
  \l 28F0F
  \l 28F10
  \l 28F11
  \l 28F12
  \l 28F13
  \l 28F14
  \l 28F15
  \l 28F16
  \l 28F17
  \l 28F18
  \l 28F19
  \l 28F1A
  \l 28F1B
  \l 28F1C
  \l 28F1D
  \l 28F1E
  \l 28F1F
  \l 28F20
  \l 28F21
  \l 28F22
  \l 28F23
  \l 28F24
  \l 28F25
  \l 28F26
  \l 28F27
  \l 28F28
  \l 28F29
  \l 28F2A
  \l 28F2B
  \l 28F2C
  \l 28F2D
  \l 28F2E
  \l 28F2F
  \l 28F30
  \l 28F31
  \l 28F32
  \l 28F33
  \l 28F34
  \l 28F35
  \l 28F36
  \l 28F37
  \l 28F38
  \l 28F39
  \l 28F3A
  \l 28F3B
  \l 28F3C
  \l 28F3D
  \l 28F3E
  \l 28F3F
  \l 28F40
  \l 28F41
  \l 28F42
  \l 28F43
  \l 28F44
  \l 28F45
  \l 28F46
  \l 28F47
  \l 28F48
  \l 28F49
  \l 28F4A
  \l 28F4B
  \l 28F4C
  \l 28F4D
  \l 28F4E
  \l 28F4F
  \l 28F50
  \l 28F51
  \l 28F52
  \l 28F53
  \l 28F54
  \l 28F55
  \l 28F56
  \l 28F57
  \l 28F58
  \l 28F59
  \l 28F5A
  \l 28F5B
  \l 28F5C
  \l 28F5D
  \l 28F5E
  \l 28F5F
  \l 28F60
  \l 28F61
  \l 28F62
  \l 28F63
  \l 28F64
  \l 28F65
  \l 28F66
  \l 28F67
  \l 28F68
  \l 28F69
  \l 28F6A
  \l 28F6B
  \l 28F6C
  \l 28F6D
  \l 28F6E
  \l 28F6F
  \l 28F70
  \l 28F71
  \l 28F72
  \l 28F73
  \l 28F74
  \l 28F75
  \l 28F76
  \l 28F77
  \l 28F78
  \l 28F79
  \l 28F7A
  \l 28F7B
  \l 28F7C
  \l 28F7D
  \l 28F7E
  \l 28F7F
  \l 28F80
  \l 28F81
  \l 28F82
  \l 28F83
  \l 28F84
  \l 28F85
  \l 28F86
  \l 28F87
  \l 28F88
  \l 28F89
  \l 28F8A
  \l 28F8B
  \l 28F8C
  \l 28F8D
  \l 28F8E
  \l 28F8F
  \l 28F90
  \l 28F91
  \l 28F92
  \l 28F93
  \l 28F94
  \l 28F95
  \l 28F96
  \l 28F97
  \l 28F98
  \l 28F99
  \l 28F9A
  \l 28F9B
  \l 28F9C
  \l 28F9D
  \l 28F9E
  \l 28F9F
  \l 28FA0
  \l 28FA1
  \l 28FA2
  \l 28FA3
  \l 28FA4
  \l 28FA5
  \l 28FA6
  \l 28FA7
  \l 28FA8
  \l 28FA9
  \l 28FAA
  \l 28FAB
  \l 28FAC
  \l 28FAD
  \l 28FAE
  \l 28FAF
  \l 28FB0
  \l 28FB1
  \l 28FB2
  \l 28FB3
  \l 28FB4
  \l 28FB5
  \l 28FB6
  \l 28FB7
  \l 28FB8
  \l 28FB9
  \l 28FBA
  \l 28FBB
  \l 28FBC
  \l 28FBD
  \l 28FBE
  \l 28FBF
  \l 28FC0
  \l 28FC1
  \l 28FC2
  \l 28FC3
  \l 28FC4
  \l 28FC5
  \l 28FC6
  \l 28FC7
  \l 28FC8
  \l 28FC9
  \l 28FCA
  \l 28FCB
  \l 28FCC
  \l 28FCD
  \l 28FCE
  \l 28FCF
  \l 28FD0
  \l 28FD1
  \l 28FD2
  \l 28FD3
  \l 28FD4
  \l 28FD5
  \l 28FD6
  \l 28FD7
  \l 28FD8
  \l 28FD9
  \l 28FDA
  \l 28FDB
  \l 28FDC
  \l 28FDD
  \l 28FDE
  \l 28FDF
  \l 28FE0
  \l 28FE1
  \l 28FE2
  \l 28FE3
  \l 28FE4
  \l 28FE5
  \l 28FE6
  \l 28FE7
  \l 28FE8
  \l 28FE9
  \l 28FEA
  \l 28FEB
  \l 28FEC
  \l 28FED
  \l 28FEE
  \l 28FEF
  \l 28FF0
  \l 28FF1
  \l 28FF2
  \l 28FF3
  \l 28FF4
  \l 28FF5
  \l 28FF6
  \l 28FF7
  \l 28FF8
  \l 28FF9
  \l 28FFA
  \l 28FFB
  \l 28FFC
  \l 28FFD
  \l 28FFE
  \l 28FFF
  \l 29000
  \l 29001
  \l 29002
  \l 29003
  \l 29004
  \l 29005
  \l 29006
  \l 29007
  \l 29008
  \l 29009
  \l 2900A
  \l 2900B
  \l 2900C
  \l 2900D
  \l 2900E
  \l 2900F
  \l 29010
  \l 29011
  \l 29012
  \l 29013
  \l 29014
  \l 29015
  \l 29016
  \l 29017
  \l 29018
  \l 29019
  \l 2901A
  \l 2901B
  \l 2901C
  \l 2901D
  \l 2901E
  \l 2901F
  \l 29020
  \l 29021
  \l 29022
  \l 29023
  \l 29024
  \l 29025
  \l 29026
  \l 29027
  \l 29028
  \l 29029
  \l 2902A
  \l 2902B
  \l 2902C
  \l 2902D
  \l 2902E
  \l 2902F
  \l 29030
  \l 29031
  \l 29032
  \l 29033
  \l 29034
  \l 29035
  \l 29036
  \l 29037
  \l 29038
  \l 29039
  \l 2903A
  \l 2903B
  \l 2903C
  \l 2903D
  \l 2903E
  \l 2903F
  \l 29040
  \l 29041
  \l 29042
  \l 29043
  \l 29044
  \l 29045
  \l 29046
  \l 29047
  \l 29048
  \l 29049
  \l 2904A
  \l 2904B
  \l 2904C
  \l 2904D
  \l 2904E
  \l 2904F
  \l 29050
  \l 29051
  \l 29052
  \l 29053
  \l 29054
  \l 29055
  \l 29056
  \l 29057
  \l 29058
  \l 29059
  \l 2905A
  \l 2905B
  \l 2905C
  \l 2905D
  \l 2905E
  \l 2905F
  \l 29060
  \l 29061
  \l 29062
  \l 29063
  \l 29064
  \l 29065
  \l 29066
  \l 29067
  \l 29068
  \l 29069
  \l 2906A
  \l 2906B
  \l 2906C
  \l 2906D
  \l 2906E
  \l 2906F
  \l 29070
  \l 29071
  \l 29072
  \l 29073
  \l 29074
  \l 29075
  \l 29076
  \l 29077
  \l 29078
  \l 29079
  \l 2907A
  \l 2907B
  \l 2907C
  \l 2907D
  \l 2907E
  \l 2907F
  \l 29080
  \l 29081
  \l 29082
  \l 29083
  \l 29084
  \l 29085
  \l 29086
  \l 29087
  \l 29088
  \l 29089
  \l 2908A
  \l 2908B
  \l 2908C
  \l 2908D
  \l 2908E
  \l 2908F
  \l 29090
  \l 29091
  \l 29092
  \l 29093
  \l 29094
  \l 29095
  \l 29096
  \l 29097
  \l 29098
  \l 29099
  \l 2909A
  \l 2909B
  \l 2909C
  \l 2909D
  \l 2909E
  \l 2909F
  \l 290A0
  \l 290A1
  \l 290A2
  \l 290A3
  \l 290A4
  \l 290A5
  \l 290A6
  \l 290A7
  \l 290A8
  \l 290A9
  \l 290AA
  \l 290AB
  \l 290AC
  \l 290AD
  \l 290AE
  \l 290AF
  \l 290B0
  \l 290B1
  \l 290B2
  \l 290B3
  \l 290B4
  \l 290B5
  \l 290B6
  \l 290B7
  \l 290B8
  \l 290B9
  \l 290BA
  \l 290BB
  \l 290BC
  \l 290BD
  \l 290BE
  \l 290BF
  \l 290C0
  \l 290C1
  \l 290C2
  \l 290C3
  \l 290C4
  \l 290C5
  \l 290C6
  \l 290C7
  \l 290C8
  \l 290C9
  \l 290CA
  \l 290CB
  \l 290CC
  \l 290CD
  \l 290CE
  \l 290CF
  \l 290D0
  \l 290D1
  \l 290D2
  \l 290D3
  \l 290D4
  \l 290D5
  \l 290D6
  \l 290D7
  \l 290D8
  \l 290D9
  \l 290DA
  \l 290DB
  \l 290DC
  \l 290DD
  \l 290DE
  \l 290DF
  \l 290E0
  \l 290E1
  \l 290E2
  \l 290E3
  \l 290E4
  \l 290E5
  \l 290E6
  \l 290E7
  \l 290E8
  \l 290E9
  \l 290EA
  \l 290EB
  \l 290EC
  \l 290ED
  \l 290EE
  \l 290EF
  \l 290F0
  \l 290F1
  \l 290F2
  \l 290F3
  \l 290F4
  \l 290F5
  \l 290F6
  \l 290F7
  \l 290F8
  \l 290F9
  \l 290FA
  \l 290FB
  \l 290FC
  \l 290FD
  \l 290FE
  \l 290FF
  \l 29100
  \l 29101
  \l 29102
  \l 29103
  \l 29104
  \l 29105
  \l 29106
  \l 29107
  \l 29108
  \l 29109
  \l 2910A
  \l 2910B
  \l 2910C
  \l 2910D
  \l 2910E
  \l 2910F
  \l 29110
  \l 29111
  \l 29112
  \l 29113
  \l 29114
  \l 29115
  \l 29116
  \l 29117
  \l 29118
  \l 29119
  \l 2911A
  \l 2911B
  \l 2911C
  \l 2911D
  \l 2911E
  \l 2911F
  \l 29120
  \l 29121
  \l 29122
  \l 29123
  \l 29124
  \l 29125
  \l 29126
  \l 29127
  \l 29128
  \l 29129
  \l 2912A
  \l 2912B
  \l 2912C
  \l 2912D
  \l 2912E
  \l 2912F
  \l 29130
  \l 29131
  \l 29132
  \l 29133
  \l 29134
  \l 29135
  \l 29136
  \l 29137
  \l 29138
  \l 29139
  \l 2913A
  \l 2913B
  \l 2913C
  \l 2913D
  \l 2913E
  \l 2913F
  \l 29140
  \l 29141
  \l 29142
  \l 29143
  \l 29144
  \l 29145
  \l 29146
  \l 29147
  \l 29148
  \l 29149
  \l 2914A
  \l 2914B
  \l 2914C
  \l 2914D
  \l 2914E
  \l 2914F
  \l 29150
  \l 29151
  \l 29152
  \l 29153
  \l 29154
  \l 29155
  \l 29156
  \l 29157
  \l 29158
  \l 29159
  \l 2915A
  \l 2915B
  \l 2915C
  \l 2915D
  \l 2915E
  \l 2915F
  \l 29160
  \l 29161
  \l 29162
  \l 29163
  \l 29164
  \l 29165
  \l 29166
  \l 29167
  \l 29168
  \l 29169
  \l 2916A
  \l 2916B
  \l 2916C
  \l 2916D
  \l 2916E
  \l 2916F
  \l 29170
  \l 29171
  \l 29172
  \l 29173
  \l 29174
  \l 29175
  \l 29176
  \l 29177
  \l 29178
  \l 29179
  \l 2917A
  \l 2917B
  \l 2917C
  \l 2917D
  \l 2917E
  \l 2917F
  \l 29180
  \l 29181
  \l 29182
  \l 29183
  \l 29184
  \l 29185
  \l 29186
  \l 29187
  \l 29188
  \l 29189
  \l 2918A
  \l 2918B
  \l 2918C
  \l 2918D
  \l 2918E
  \l 2918F
  \l 29190
  \l 29191
  \l 29192
  \l 29193
  \l 29194
  \l 29195
  \l 29196
  \l 29197
  \l 29198
  \l 29199
  \l 2919A
  \l 2919B
  \l 2919C
  \l 2919D
  \l 2919E
  \l 2919F
  \l 291A0
  \l 291A1
  \l 291A2
  \l 291A3
  \l 291A4
  \l 291A5
  \l 291A6
  \l 291A7
  \l 291A8
  \l 291A9
  \l 291AA
  \l 291AB
  \l 291AC
  \l 291AD
  \l 291AE
  \l 291AF
  \l 291B0
  \l 291B1
  \l 291B2
  \l 291B3
  \l 291B4
  \l 291B5
  \l 291B6
  \l 291B7
  \l 291B8
  \l 291B9
  \l 291BA
  \l 291BB
  \l 291BC
  \l 291BD
  \l 291BE
  \l 291BF
  \l 291C0
  \l 291C1
  \l 291C2
  \l 291C3
  \l 291C4
  \l 291C5
  \l 291C6
  \l 291C7
  \l 291C8
  \l 291C9
  \l 291CA
  \l 291CB
  \l 291CC
  \l 291CD
  \l 291CE
  \l 291CF
  \l 291D0
  \l 291D1
  \l 291D2
  \l 291D3
  \l 291D4
  \l 291D5
  \l 291D6
  \l 291D7
  \l 291D8
  \l 291D9
  \l 291DA
  \l 291DB
  \l 291DC
  \l 291DD
  \l 291DE
  \l 291DF
  \l 291E0
  \l 291E1
  \l 291E2
  \l 291E3
  \l 291E4
  \l 291E5
  \l 291E6
  \l 291E7
  \l 291E8
  \l 291E9
  \l 291EA
  \l 291EB
  \l 291EC
  \l 291ED
  \l 291EE
  \l 291EF
  \l 291F0
  \l 291F1
  \l 291F2
  \l 291F3
  \l 291F4
  \l 291F5
  \l 291F6
  \l 291F7
  \l 291F8
  \l 291F9
  \l 291FA
  \l 291FB
  \l 291FC
  \l 291FD
  \l 291FE
  \l 291FF
  \l 29200
  \l 29201
  \l 29202
  \l 29203
  \l 29204
  \l 29205
  \l 29206
  \l 29207
  \l 29208
  \l 29209
  \l 2920A
  \l 2920B
  \l 2920C
  \l 2920D
  \l 2920E
  \l 2920F
  \l 29210
  \l 29211
  \l 29212
  \l 29213
  \l 29214
  \l 29215
  \l 29216
  \l 29217
  \l 29218
  \l 29219
  \l 2921A
  \l 2921B
  \l 2921C
  \l 2921D
  \l 2921E
  \l 2921F
  \l 29220
  \l 29221
  \l 29222
  \l 29223
  \l 29224
  \l 29225
  \l 29226
  \l 29227
  \l 29228
  \l 29229
  \l 2922A
  \l 2922B
  \l 2922C
  \l 2922D
  \l 2922E
  \l 2922F
  \l 29230
  \l 29231
  \l 29232
  \l 29233
  \l 29234
  \l 29235
  \l 29236
  \l 29237
  \l 29238
  \l 29239
  \l 2923A
  \l 2923B
  \l 2923C
  \l 2923D
  \l 2923E
  \l 2923F
  \l 29240
  \l 29241
  \l 29242
  \l 29243
  \l 29244
  \l 29245
  \l 29246
  \l 29247
  \l 29248
  \l 29249
  \l 2924A
  \l 2924B
  \l 2924C
  \l 2924D
  \l 2924E
  \l 2924F
  \l 29250
  \l 29251
  \l 29252
  \l 29253
  \l 29254
  \l 29255
  \l 29256
  \l 29257
  \l 29258
  \l 29259
  \l 2925A
  \l 2925B
  \l 2925C
  \l 2925D
  \l 2925E
  \l 2925F
  \l 29260
  \l 29261
  \l 29262
  \l 29263
  \l 29264
  \l 29265
  \l 29266
  \l 29267
  \l 29268
  \l 29269
  \l 2926A
  \l 2926B
  \l 2926C
  \l 2926D
  \l 2926E
  \l 2926F
  \l 29270
  \l 29271
  \l 29272
  \l 29273
  \l 29274
  \l 29275
  \l 29276
  \l 29277
  \l 29278
  \l 29279
  \l 2927A
  \l 2927B
  \l 2927C
  \l 2927D
  \l 2927E
  \l 2927F
  \l 29280
  \l 29281
  \l 29282
  \l 29283
  \l 29284
  \l 29285
  \l 29286
  \l 29287
  \l 29288
  \l 29289
  \l 2928A
  \l 2928B
  \l 2928C
  \l 2928D
  \l 2928E
  \l 2928F
  \l 29290
  \l 29291
  \l 29292
  \l 29293
  \l 29294
  \l 29295
  \l 29296
  \l 29297
  \l 29298
  \l 29299
  \l 2929A
  \l 2929B
  \l 2929C
  \l 2929D
  \l 2929E
  \l 2929F
  \l 292A0
  \l 292A1
  \l 292A2
  \l 292A3
  \l 292A4
  \l 292A5
  \l 292A6
  \l 292A7
  \l 292A8
  \l 292A9
  \l 292AA
  \l 292AB
  \l 292AC
  \l 292AD
  \l 292AE
  \l 292AF
  \l 292B0
  \l 292B1
  \l 292B2
  \l 292B3
  \l 292B4
  \l 292B5
  \l 292B6
  \l 292B7
  \l 292B8
  \l 292B9
  \l 292BA
  \l 292BB
  \l 292BC
  \l 292BD
  \l 292BE
  \l 292BF
  \l 292C0
  \l 292C1
  \l 292C2
  \l 292C3
  \l 292C4
  \l 292C5
  \l 292C6
  \l 292C7
  \l 292C8
  \l 292C9
  \l 292CA
  \l 292CB
  \l 292CC
  \l 292CD
  \l 292CE
  \l 292CF
  \l 292D0
  \l 292D1
  \l 292D2
  \l 292D3
  \l 292D4
  \l 292D5
  \l 292D6
  \l 292D7
  \l 292D8
  \l 292D9
  \l 292DA
  \l 292DB
  \l 292DC
  \l 292DD
  \l 292DE
  \l 292DF
  \l 292E0
  \l 292E1
  \l 292E2
  \l 292E3
  \l 292E4
  \l 292E5
  \l 292E6
  \l 292E7
  \l 292E8
  \l 292E9
  \l 292EA
  \l 292EB
  \l 292EC
  \l 292ED
  \l 292EE
  \l 292EF
  \l 292F0
  \l 292F1
  \l 292F2
  \l 292F3
  \l 292F4
  \l 292F5
  \l 292F6
  \l 292F7
  \l 292F8
  \l 292F9
  \l 292FA
  \l 292FB
  \l 292FC
  \l 292FD
  \l 292FE
  \l 292FF
  \l 29300
  \l 29301
  \l 29302
  \l 29303
  \l 29304
  \l 29305
  \l 29306
  \l 29307
  \l 29308
  \l 29309
  \l 2930A
  \l 2930B
  \l 2930C
  \l 2930D
  \l 2930E
  \l 2930F
  \l 29310
  \l 29311
  \l 29312
  \l 29313
  \l 29314
  \l 29315
  \l 29316
  \l 29317
  \l 29318
  \l 29319
  \l 2931A
  \l 2931B
  \l 2931C
  \l 2931D
  \l 2931E
  \l 2931F
  \l 29320
  \l 29321
  \l 29322
  \l 29323
  \l 29324
  \l 29325
  \l 29326
  \l 29327
  \l 29328
  \l 29329
  \l 2932A
  \l 2932B
  \l 2932C
  \l 2932D
  \l 2932E
  \l 2932F
  \l 29330
  \l 29331
  \l 29332
  \l 29333
  \l 29334
  \l 29335
  \l 29336
  \l 29337
  \l 29338
  \l 29339
  \l 2933A
  \l 2933B
  \l 2933C
  \l 2933D
  \l 2933E
  \l 2933F
  \l 29340
  \l 29341
  \l 29342
  \l 29343
  \l 29344
  \l 29345
  \l 29346
  \l 29347
  \l 29348
  \l 29349
  \l 2934A
  \l 2934B
  \l 2934C
  \l 2934D
  \l 2934E
  \l 2934F
  \l 29350
  \l 29351
  \l 29352
  \l 29353
  \l 29354
  \l 29355
  \l 29356
  \l 29357
  \l 29358
  \l 29359
  \l 2935A
  \l 2935B
  \l 2935C
  \l 2935D
  \l 2935E
  \l 2935F
  \l 29360
  \l 29361
  \l 29362
  \l 29363
  \l 29364
  \l 29365
  \l 29366
  \l 29367
  \l 29368
  \l 29369
  \l 2936A
  \l 2936B
  \l 2936C
  \l 2936D
  \l 2936E
  \l 2936F
  \l 29370
  \l 29371
  \l 29372
  \l 29373
  \l 29374
  \l 29375
  \l 29376
  \l 29377
  \l 29378
  \l 29379
  \l 2937A
  \l 2937B
  \l 2937C
  \l 2937D
  \l 2937E
  \l 2937F
  \l 29380
  \l 29381
  \l 29382
  \l 29383
  \l 29384
  \l 29385
  \l 29386
  \l 29387
  \l 29388
  \l 29389
  \l 2938A
  \l 2938B
  \l 2938C
  \l 2938D
  \l 2938E
  \l 2938F
  \l 29390
  \l 29391
  \l 29392
  \l 29393
  \l 29394
  \l 29395
  \l 29396
  \l 29397
  \l 29398
  \l 29399
  \l 2939A
  \l 2939B
  \l 2939C
  \l 2939D
  \l 2939E
  \l 2939F
  \l 293A0
  \l 293A1
  \l 293A2
  \l 293A3
  \l 293A4
  \l 293A5
  \l 293A6
  \l 293A7
  \l 293A8
  \l 293A9
  \l 293AA
  \l 293AB
  \l 293AC
  \l 293AD
  \l 293AE
  \l 293AF
  \l 293B0
  \l 293B1
  \l 293B2
  \l 293B3
  \l 293B4
  \l 293B5
  \l 293B6
  \l 293B7
  \l 293B8
  \l 293B9
  \l 293BA
  \l 293BB
  \l 293BC
  \l 293BD
  \l 293BE
  \l 293BF
  \l 293C0
  \l 293C1
  \l 293C2
  \l 293C3
  \l 293C4
  \l 293C5
  \l 293C6
  \l 293C7
  \l 293C8
  \l 293C9
  \l 293CA
  \l 293CB
  \l 293CC
  \l 293CD
  \l 293CE
  \l 293CF
  \l 293D0
  \l 293D1
  \l 293D2
  \l 293D3
  \l 293D4
  \l 293D5
  \l 293D6
  \l 293D7
  \l 293D8
  \l 293D9
  \l 293DA
  \l 293DB
  \l 293DC
  \l 293DD
  \l 293DE
  \l 293DF
  \l 293E0
  \l 293E1
  \l 293E2
  \l 293E3
  \l 293E4
  \l 293E5
  \l 293E6
  \l 293E7
  \l 293E8
  \l 293E9
  \l 293EA
  \l 293EB
  \l 293EC
  \l 293ED
  \l 293EE
  \l 293EF
  \l 293F0
  \l 293F1
  \l 293F2
  \l 293F3
  \l 293F4
  \l 293F5
  \l 293F6
  \l 293F7
  \l 293F8
  \l 293F9
  \l 293FA
  \l 293FB
  \l 293FC
  \l 293FD
  \l 293FE
  \l 293FF
  \l 29400
  \l 29401
  \l 29402
  \l 29403
  \l 29404
  \l 29405
  \l 29406
  \l 29407
  \l 29408
  \l 29409
  \l 2940A
  \l 2940B
  \l 2940C
  \l 2940D
  \l 2940E
  \l 2940F
  \l 29410
  \l 29411
  \l 29412
  \l 29413
  \l 29414
  \l 29415
  \l 29416
  \l 29417
  \l 29418
  \l 29419
  \l 2941A
  \l 2941B
  \l 2941C
  \l 2941D
  \l 2941E
  \l 2941F
  \l 29420
  \l 29421
  \l 29422
  \l 29423
  \l 29424
  \l 29425
  \l 29426
  \l 29427
  \l 29428
  \l 29429
  \l 2942A
  \l 2942B
  \l 2942C
  \l 2942D
  \l 2942E
  \l 2942F
  \l 29430
  \l 29431
  \l 29432
  \l 29433
  \l 29434
  \l 29435
  \l 29436
  \l 29437
  \l 29438
  \l 29439
  \l 2943A
  \l 2943B
  \l 2943C
  \l 2943D
  \l 2943E
  \l 2943F
  \l 29440
  \l 29441
  \l 29442
  \l 29443
  \l 29444
  \l 29445
  \l 29446
  \l 29447
  \l 29448
  \l 29449
  \l 2944A
  \l 2944B
  \l 2944C
  \l 2944D
  \l 2944E
  \l 2944F
  \l 29450
  \l 29451
  \l 29452
  \l 29453
  \l 29454
  \l 29455
  \l 29456
  \l 29457
  \l 29458
  \l 29459
  \l 2945A
  \l 2945B
  \l 2945C
  \l 2945D
  \l 2945E
  \l 2945F
  \l 29460
  \l 29461
  \l 29462
  \l 29463
  \l 29464
  \l 29465
  \l 29466
  \l 29467
  \l 29468
  \l 29469
  \l 2946A
  \l 2946B
  \l 2946C
  \l 2946D
  \l 2946E
  \l 2946F
  \l 29470
  \l 29471
  \l 29472
  \l 29473
  \l 29474
  \l 29475
  \l 29476
  \l 29477
  \l 29478
  \l 29479
  \l 2947A
  \l 2947B
  \l 2947C
  \l 2947D
  \l 2947E
  \l 2947F
  \l 29480
  \l 29481
  \l 29482
  \l 29483
  \l 29484
  \l 29485
  \l 29486
  \l 29487
  \l 29488
  \l 29489
  \l 2948A
  \l 2948B
  \l 2948C
  \l 2948D
  \l 2948E
  \l 2948F
  \l 29490
  \l 29491
  \l 29492
  \l 29493
  \l 29494
  \l 29495
  \l 29496
  \l 29497
  \l 29498
  \l 29499
  \l 2949A
  \l 2949B
  \l 2949C
  \l 2949D
  \l 2949E
  \l 2949F
  \l 294A0
  \l 294A1
  \l 294A2
  \l 294A3
  \l 294A4
  \l 294A5
  \l 294A6
  \l 294A7
  \l 294A8
  \l 294A9
  \l 294AA
  \l 294AB
  \l 294AC
  \l 294AD
  \l 294AE
  \l 294AF
  \l 294B0
  \l 294B1
  \l 294B2
  \l 294B3
  \l 294B4
  \l 294B5
  \l 294B6
  \l 294B7
  \l 294B8
  \l 294B9
  \l 294BA
  \l 294BB
  \l 294BC
  \l 294BD
  \l 294BE
  \l 294BF
  \l 294C0
  \l 294C1
  \l 294C2
  \l 294C3
  \l 294C4
  \l 294C5
  \l 294C6
  \l 294C7
  \l 294C8
  \l 294C9
  \l 294CA
  \l 294CB
  \l 294CC
  \l 294CD
  \l 294CE
  \l 294CF
  \l 294D0
  \l 294D1
  \l 294D2
  \l 294D3
  \l 294D4
  \l 294D5
  \l 294D6
  \l 294D7
  \l 294D8
  \l 294D9
  \l 294DA
  \l 294DB
  \l 294DC
  \l 294DD
  \l 294DE
  \l 294DF
  \l 294E0
  \l 294E1
  \l 294E2
  \l 294E3
  \l 294E4
  \l 294E5
  \l 294E6
  \l 294E7
  \l 294E8
  \l 294E9
  \l 294EA
  \l 294EB
  \l 294EC
  \l 294ED
  \l 294EE
  \l 294EF
  \l 294F0
  \l 294F1
  \l 294F2
  \l 294F3
  \l 294F4
  \l 294F5
  \l 294F6
  \l 294F7
  \l 294F8
  \l 294F9
  \l 294FA
  \l 294FB
  \l 294FC
  \l 294FD
  \l 294FE
  \l 294FF
  \l 29500
  \l 29501
  \l 29502
  \l 29503
  \l 29504
  \l 29505
  \l 29506
  \l 29507
  \l 29508
  \l 29509
  \l 2950A
  \l 2950B
  \l 2950C
  \l 2950D
  \l 2950E
  \l 2950F
  \l 29510
  \l 29511
  \l 29512
  \l 29513
  \l 29514
  \l 29515
  \l 29516
  \l 29517
  \l 29518
  \l 29519
  \l 2951A
  \l 2951B
  \l 2951C
  \l 2951D
  \l 2951E
  \l 2951F
  \l 29520
  \l 29521
  \l 29522
  \l 29523
  \l 29524
  \l 29525
  \l 29526
  \l 29527
  \l 29528
  \l 29529
  \l 2952A
  \l 2952B
  \l 2952C
  \l 2952D
  \l 2952E
  \l 2952F
  \l 29530
  \l 29531
  \l 29532
  \l 29533
  \l 29534
  \l 29535
  \l 29536
  \l 29537
  \l 29538
  \l 29539
  \l 2953A
  \l 2953B
  \l 2953C
  \l 2953D
  \l 2953E
  \l 2953F
  \l 29540
  \l 29541
  \l 29542
  \l 29543
  \l 29544
  \l 29545
  \l 29546
  \l 29547
  \l 29548
  \l 29549
  \l 2954A
  \l 2954B
  \l 2954C
  \l 2954D
  \l 2954E
  \l 2954F
  \l 29550
  \l 29551
  \l 29552
  \l 29553
  \l 29554
  \l 29555
  \l 29556
  \l 29557
  \l 29558
  \l 29559
  \l 2955A
  \l 2955B
  \l 2955C
  \l 2955D
  \l 2955E
  \l 2955F
  \l 29560
  \l 29561
  \l 29562
  \l 29563
  \l 29564
  \l 29565
  \l 29566
  \l 29567
  \l 29568
  \l 29569
  \l 2956A
  \l 2956B
  \l 2956C
  \l 2956D
  \l 2956E
  \l 2956F
  \l 29570
  \l 29571
  \l 29572
  \l 29573
  \l 29574
  \l 29575
  \l 29576
  \l 29577
  \l 29578
  \l 29579
  \l 2957A
  \l 2957B
  \l 2957C
  \l 2957D
  \l 2957E
  \l 2957F
  \l 29580
  \l 29581
  \l 29582
  \l 29583
  \l 29584
  \l 29585
  \l 29586
  \l 29587
  \l 29588
  \l 29589
  \l 2958A
  \l 2958B
  \l 2958C
  \l 2958D
  \l 2958E
  \l 2958F
  \l 29590
  \l 29591
  \l 29592
  \l 29593
  \l 29594
  \l 29595
  \l 29596
  \l 29597
  \l 29598
  \l 29599
  \l 2959A
  \l 2959B
  \l 2959C
  \l 2959D
  \l 2959E
  \l 2959F
  \l 295A0
  \l 295A1
  \l 295A2
  \l 295A3
  \l 295A4
  \l 295A5
  \l 295A6
  \l 295A7
  \l 295A8
  \l 295A9
  \l 295AA
  \l 295AB
  \l 295AC
  \l 295AD
  \l 295AE
  \l 295AF
  \l 295B0
  \l 295B1
  \l 295B2
  \l 295B3
  \l 295B4
  \l 295B5
  \l 295B6
  \l 295B7
  \l 295B8
  \l 295B9
  \l 295BA
  \l 295BB
  \l 295BC
  \l 295BD
  \l 295BE
  \l 295BF
  \l 295C0
  \l 295C1
  \l 295C2
  \l 295C3
  \l 295C4
  \l 295C5
  \l 295C6
  \l 295C7
  \l 295C8
  \l 295C9
  \l 295CA
  \l 295CB
  \l 295CC
  \l 295CD
  \l 295CE
  \l 295CF
  \l 295D0
  \l 295D1
  \l 295D2
  \l 295D3
  \l 295D4
  \l 295D5
  \l 295D6
  \l 295D7
  \l 295D8
  \l 295D9
  \l 295DA
  \l 295DB
  \l 295DC
  \l 295DD
  \l 295DE
  \l 295DF
  \l 295E0
  \l 295E1
  \l 295E2
  \l 295E3
  \l 295E4
  \l 295E5
  \l 295E6
  \l 295E7
  \l 295E8
  \l 295E9
  \l 295EA
  \l 295EB
  \l 295EC
  \l 295ED
  \l 295EE
  \l 295EF
  \l 295F0
  \l 295F1
  \l 295F2
  \l 295F3
  \l 295F4
  \l 295F5
  \l 295F6
  \l 295F7
  \l 295F8
  \l 295F9
  \l 295FA
  \l 295FB
  \l 295FC
  \l 295FD
  \l 295FE
  \l 295FF
  \l 29600
  \l 29601
  \l 29602
  \l 29603
  \l 29604
  \l 29605
  \l 29606
  \l 29607
  \l 29608
  \l 29609
  \l 2960A
  \l 2960B
  \l 2960C
  \l 2960D
  \l 2960E
  \l 2960F
  \l 29610
  \l 29611
  \l 29612
  \l 29613
  \l 29614
  \l 29615
  \l 29616
  \l 29617
  \l 29618
  \l 29619
  \l 2961A
  \l 2961B
  \l 2961C
  \l 2961D
  \l 2961E
  \l 2961F
  \l 29620
  \l 29621
  \l 29622
  \l 29623
  \l 29624
  \l 29625
  \l 29626
  \l 29627
  \l 29628
  \l 29629
  \l 2962A
  \l 2962B
  \l 2962C
  \l 2962D
  \l 2962E
  \l 2962F
  \l 29630
  \l 29631
  \l 29632
  \l 29633
  \l 29634
  \l 29635
  \l 29636
  \l 29637
  \l 29638
  \l 29639
  \l 2963A
  \l 2963B
  \l 2963C
  \l 2963D
  \l 2963E
  \l 2963F
  \l 29640
  \l 29641
  \l 29642
  \l 29643
  \l 29644
  \l 29645
  \l 29646
  \l 29647
  \l 29648
  \l 29649
  \l 2964A
  \l 2964B
  \l 2964C
  \l 2964D
  \l 2964E
  \l 2964F
  \l 29650
  \l 29651
  \l 29652
  \l 29653
  \l 29654
  \l 29655
  \l 29656
  \l 29657
  \l 29658
  \l 29659
  \l 2965A
  \l 2965B
  \l 2965C
  \l 2965D
  \l 2965E
  \l 2965F
  \l 29660
  \l 29661
  \l 29662
  \l 29663
  \l 29664
  \l 29665
  \l 29666
  \l 29667
  \l 29668
  \l 29669
  \l 2966A
  \l 2966B
  \l 2966C
  \l 2966D
  \l 2966E
  \l 2966F
  \l 29670
  \l 29671
  \l 29672
  \l 29673
  \l 29674
  \l 29675
  \l 29676
  \l 29677
  \l 29678
  \l 29679
  \l 2967A
  \l 2967B
  \l 2967C
  \l 2967D
  \l 2967E
  \l 2967F
  \l 29680
  \l 29681
  \l 29682
  \l 29683
  \l 29684
  \l 29685
  \l 29686
  \l 29687
  \l 29688
  \l 29689
  \l 2968A
  \l 2968B
  \l 2968C
  \l 2968D
  \l 2968E
  \l 2968F
  \l 29690
  \l 29691
  \l 29692
  \l 29693
  \l 29694
  \l 29695
  \l 29696
  \l 29697
  \l 29698
  \l 29699
  \l 2969A
  \l 2969B
  \l 2969C
  \l 2969D
  \l 2969E
  \l 2969F
  \l 296A0
  \l 296A1
  \l 296A2
  \l 296A3
  \l 296A4
  \l 296A5
  \l 296A6
  \l 296A7
  \l 296A8
  \l 296A9
  \l 296AA
  \l 296AB
  \l 296AC
  \l 296AD
  \l 296AE
  \l 296AF
  \l 296B0
  \l 296B1
  \l 296B2
  \l 296B3
  \l 296B4
  \l 296B5
  \l 296B6
  \l 296B7
  \l 296B8
  \l 296B9
  \l 296BA
  \l 296BB
  \l 296BC
  \l 296BD
  \l 296BE
  \l 296BF
  \l 296C0
  \l 296C1
  \l 296C2
  \l 296C3
  \l 296C4
  \l 296C5
  \l 296C6
  \l 296C7
  \l 296C8
  \l 296C9
  \l 296CA
  \l 296CB
  \l 296CC
  \l 296CD
  \l 296CE
  \l 296CF
  \l 296D0
  \l 296D1
  \l 296D2
  \l 296D3
  \l 296D4
  \l 296D5
  \l 296D6
  \l 296D7
  \l 296D8
  \l 296D9
  \l 296DA
  \l 296DB
  \l 296DC
  \l 296DD
  \l 296DE
  \l 296DF
  \l 296E0
  \l 296E1
  \l 296E2
  \l 296E3
  \l 296E4
  \l 296E5
  \l 296E6
  \l 296E7
  \l 296E8
  \l 296E9
  \l 296EA
  \l 296EB
  \l 296EC
  \l 296ED
  \l 296EE
  \l 296EF
  \l 296F0
  \l 296F1
  \l 296F2
  \l 296F3
  \l 296F4
  \l 296F5
  \l 296F6
  \l 296F7
  \l 296F8
  \l 296F9
  \l 296FA
  \l 296FB
  \l 296FC
  \l 296FD
  \l 296FE
  \l 296FF
  \l 29700
  \l 29701
  \l 29702
  \l 29703
  \l 29704
  \l 29705
  \l 29706
  \l 29707
  \l 29708
  \l 29709
  \l 2970A
  \l 2970B
  \l 2970C
  \l 2970D
  \l 2970E
  \l 2970F
  \l 29710
  \l 29711
  \l 29712
  \l 29713
  \l 29714
  \l 29715
  \l 29716
  \l 29717
  \l 29718
  \l 29719
  \l 2971A
  \l 2971B
  \l 2971C
  \l 2971D
  \l 2971E
  \l 2971F
  \l 29720
  \l 29721
  \l 29722
  \l 29723
  \l 29724
  \l 29725
  \l 29726
  \l 29727
  \l 29728
  \l 29729
  \l 2972A
  \l 2972B
  \l 2972C
  \l 2972D
  \l 2972E
  \l 2972F
  \l 29730
  \l 29731
  \l 29732
  \l 29733
  \l 29734
  \l 29735
  \l 29736
  \l 29737
  \l 29738
  \l 29739
  \l 2973A
  \l 2973B
  \l 2973C
  \l 2973D
  \l 2973E
  \l 2973F
  \l 29740
  \l 29741
  \l 29742
  \l 29743
  \l 29744
  \l 29745
  \l 29746
  \l 29747
  \l 29748
  \l 29749
  \l 2974A
  \l 2974B
  \l 2974C
  \l 2974D
  \l 2974E
  \l 2974F
  \l 29750
  \l 29751
  \l 29752
  \l 29753
  \l 29754
  \l 29755
  \l 29756
  \l 29757
  \l 29758
  \l 29759
  \l 2975A
  \l 2975B
  \l 2975C
  \l 2975D
  \l 2975E
  \l 2975F
  \l 29760
  \l 29761
  \l 29762
  \l 29763
  \l 29764
  \l 29765
  \l 29766
  \l 29767
  \l 29768
  \l 29769
  \l 2976A
  \l 2976B
  \l 2976C
  \l 2976D
  \l 2976E
  \l 2976F
  \l 29770
  \l 29771
  \l 29772
  \l 29773
  \l 29774
  \l 29775
  \l 29776
  \l 29777
  \l 29778
  \l 29779
  \l 2977A
  \l 2977B
  \l 2977C
  \l 2977D
  \l 2977E
  \l 2977F
  \l 29780
  \l 29781
  \l 29782
  \l 29783
  \l 29784
  \l 29785
  \l 29786
  \l 29787
  \l 29788
  \l 29789
  \l 2978A
  \l 2978B
  \l 2978C
  \l 2978D
  \l 2978E
  \l 2978F
  \l 29790
  \l 29791
  \l 29792
  \l 29793
  \l 29794
  \l 29795
  \l 29796
  \l 29797
  \l 29798
  \l 29799
  \l 2979A
  \l 2979B
  \l 2979C
  \l 2979D
  \l 2979E
  \l 2979F
  \l 297A0
  \l 297A1
  \l 297A2
  \l 297A3
  \l 297A4
  \l 297A5
  \l 297A6
  \l 297A7
  \l 297A8
  \l 297A9
  \l 297AA
  \l 297AB
  \l 297AC
  \l 297AD
  \l 297AE
  \l 297AF
  \l 297B0
  \l 297B1
  \l 297B2
  \l 297B3
  \l 297B4
  \l 297B5
  \l 297B6
  \l 297B7
  \l 297B8
  \l 297B9
  \l 297BA
  \l 297BB
  \l 297BC
  \l 297BD
  \l 297BE
  \l 297BF
  \l 297C0
  \l 297C1
  \l 297C2
  \l 297C3
  \l 297C4
  \l 297C5
  \l 297C6
  \l 297C7
  \l 297C8
  \l 297C9
  \l 297CA
  \l 297CB
  \l 297CC
  \l 297CD
  \l 297CE
  \l 297CF
  \l 297D0
  \l 297D1
  \l 297D2
  \l 297D3
  \l 297D4
  \l 297D5
  \l 297D6
  \l 297D7
  \l 297D8
  \l 297D9
  \l 297DA
  \l 297DB
  \l 297DC
  \l 297DD
  \l 297DE
  \l 297DF
  \l 297E0
  \l 297E1
  \l 297E2
  \l 297E3
  \l 297E4
  \l 297E5
  \l 297E6
  \l 297E7
  \l 297E8
  \l 297E9
  \l 297EA
  \l 297EB
  \l 297EC
  \l 297ED
  \l 297EE
  \l 297EF
  \l 297F0
  \l 297F1
  \l 297F2
  \l 297F3
  \l 297F4
  \l 297F5
  \l 297F6
  \l 297F7
  \l 297F8
  \l 297F9
  \l 297FA
  \l 297FB
  \l 297FC
  \l 297FD
  \l 297FE
  \l 297FF
  \l 29800
  \l 29801
  \l 29802
  \l 29803
  \l 29804
  \l 29805
  \l 29806
  \l 29807
  \l 29808
  \l 29809
  \l 2980A
  \l 2980B
  \l 2980C
  \l 2980D
  \l 2980E
  \l 2980F
  \l 29810
  \l 29811
  \l 29812
  \l 29813
  \l 29814
  \l 29815
  \l 29816
  \l 29817
  \l 29818
  \l 29819
  \l 2981A
  \l 2981B
  \l 2981C
  \l 2981D
  \l 2981E
  \l 2981F
  \l 29820
  \l 29821
  \l 29822
  \l 29823
  \l 29824
  \l 29825
  \l 29826
  \l 29827
  \l 29828
  \l 29829
  \l 2982A
  \l 2982B
  \l 2982C
  \l 2982D
  \l 2982E
  \l 2982F
  \l 29830
  \l 29831
  \l 29832
  \l 29833
  \l 29834
  \l 29835
  \l 29836
  \l 29837
  \l 29838
  \l 29839
  \l 2983A
  \l 2983B
  \l 2983C
  \l 2983D
  \l 2983E
  \l 2983F
  \l 29840
  \l 29841
  \l 29842
  \l 29843
  \l 29844
  \l 29845
  \l 29846
  \l 29847
  \l 29848
  \l 29849
  \l 2984A
  \l 2984B
  \l 2984C
  \l 2984D
  \l 2984E
  \l 2984F
  \l 29850
  \l 29851
  \l 29852
  \l 29853
  \l 29854
  \l 29855
  \l 29856
  \l 29857
  \l 29858
  \l 29859
  \l 2985A
  \l 2985B
  \l 2985C
  \l 2985D
  \l 2985E
  \l 2985F
  \l 29860
  \l 29861
  \l 29862
  \l 29863
  \l 29864
  \l 29865
  \l 29866
  \l 29867
  \l 29868
  \l 29869
  \l 2986A
  \l 2986B
  \l 2986C
  \l 2986D
  \l 2986E
  \l 2986F
  \l 29870
  \l 29871
  \l 29872
  \l 29873
  \l 29874
  \l 29875
  \l 29876
  \l 29877
  \l 29878
  \l 29879
  \l 2987A
  \l 2987B
  \l 2987C
  \l 2987D
  \l 2987E
  \l 2987F
  \l 29880
  \l 29881
  \l 29882
  \l 29883
  \l 29884
  \l 29885
  \l 29886
  \l 29887
  \l 29888
  \l 29889
  \l 2988A
  \l 2988B
  \l 2988C
  \l 2988D
  \l 2988E
  \l 2988F
  \l 29890
  \l 29891
  \l 29892
  \l 29893
  \l 29894
  \l 29895
  \l 29896
  \l 29897
  \l 29898
  \l 29899
  \l 2989A
  \l 2989B
  \l 2989C
  \l 2989D
  \l 2989E
  \l 2989F
  \l 298A0
  \l 298A1
  \l 298A2
  \l 298A3
  \l 298A4
  \l 298A5
  \l 298A6
  \l 298A7
  \l 298A8
  \l 298A9
  \l 298AA
  \l 298AB
  \l 298AC
  \l 298AD
  \l 298AE
  \l 298AF
  \l 298B0
  \l 298B1
  \l 298B2
  \l 298B3
  \l 298B4
  \l 298B5
  \l 298B6
  \l 298B7
  \l 298B8
  \l 298B9
  \l 298BA
  \l 298BB
  \l 298BC
  \l 298BD
  \l 298BE
  \l 298BF
  \l 298C0
  \l 298C1
  \l 298C2
  \l 298C3
  \l 298C4
  \l 298C5
  \l 298C6
  \l 298C7
  \l 298C8
  \l 298C9
  \l 298CA
  \l 298CB
  \l 298CC
  \l 298CD
  \l 298CE
  \l 298CF
  \l 298D0
  \l 298D1
  \l 298D2
  \l 298D3
  \l 298D4
  \l 298D5
  \l 298D6
  \l 298D7
  \l 298D8
  \l 298D9
  \l 298DA
  \l 298DB
  \l 298DC
  \l 298DD
  \l 298DE
  \l 298DF
  \l 298E0
  \l 298E1
  \l 298E2
  \l 298E3
  \l 298E4
  \l 298E5
  \l 298E6
  \l 298E7
  \l 298E8
  \l 298E9
  \l 298EA
  \l 298EB
  \l 298EC
  \l 298ED
  \l 298EE
  \l 298EF
  \l 298F0
  \l 298F1
  \l 298F2
  \l 298F3
  \l 298F4
  \l 298F5
  \l 298F6
  \l 298F7
  \l 298F8
  \l 298F9
  \l 298FA
  \l 298FB
  \l 298FC
  \l 298FD
  \l 298FE
  \l 298FF
  \l 29900
  \l 29901
  \l 29902
  \l 29903
  \l 29904
  \l 29905
  \l 29906
  \l 29907
  \l 29908
  \l 29909
  \l 2990A
  \l 2990B
  \l 2990C
  \l 2990D
  \l 2990E
  \l 2990F
  \l 29910
  \l 29911
  \l 29912
  \l 29913
  \l 29914
  \l 29915
  \l 29916
  \l 29917
  \l 29918
  \l 29919
  \l 2991A
  \l 2991B
  \l 2991C
  \l 2991D
  \l 2991E
  \l 2991F
  \l 29920
  \l 29921
  \l 29922
  \l 29923
  \l 29924
  \l 29925
  \l 29926
  \l 29927
  \l 29928
  \l 29929
  \l 2992A
  \l 2992B
  \l 2992C
  \l 2992D
  \l 2992E
  \l 2992F
  \l 29930
  \l 29931
  \l 29932
  \l 29933
  \l 29934
  \l 29935
  \l 29936
  \l 29937
  \l 29938
  \l 29939
  \l 2993A
  \l 2993B
  \l 2993C
  \l 2993D
  \l 2993E
  \l 2993F
  \l 29940
  \l 29941
  \l 29942
  \l 29943
  \l 29944
  \l 29945
  \l 29946
  \l 29947
  \l 29948
  \l 29949
  \l 2994A
  \l 2994B
  \l 2994C
  \l 2994D
  \l 2994E
  \l 2994F
  \l 29950
  \l 29951
  \l 29952
  \l 29953
  \l 29954
  \l 29955
  \l 29956
  \l 29957
  \l 29958
  \l 29959
  \l 2995A
  \l 2995B
  \l 2995C
  \l 2995D
  \l 2995E
  \l 2995F
  \l 29960
  \l 29961
  \l 29962
  \l 29963
  \l 29964
  \l 29965
  \l 29966
  \l 29967
  \l 29968
  \l 29969
  \l 2996A
  \l 2996B
  \l 2996C
  \l 2996D
  \l 2996E
  \l 2996F
  \l 29970
  \l 29971
  \l 29972
  \l 29973
  \l 29974
  \l 29975
  \l 29976
  \l 29977
  \l 29978
  \l 29979
  \l 2997A
  \l 2997B
  \l 2997C
  \l 2997D
  \l 2997E
  \l 2997F
  \l 29980
  \l 29981
  \l 29982
  \l 29983
  \l 29984
  \l 29985
  \l 29986
  \l 29987
  \l 29988
  \l 29989
  \l 2998A
  \l 2998B
  \l 2998C
  \l 2998D
  \l 2998E
  \l 2998F
  \l 29990
  \l 29991
  \l 29992
  \l 29993
  \l 29994
  \l 29995
  \l 29996
  \l 29997
  \l 29998
  \l 29999
  \l 2999A
  \l 2999B
  \l 2999C
  \l 2999D
  \l 2999E
  \l 2999F
  \l 299A0
  \l 299A1
  \l 299A2
  \l 299A3
  \l 299A4
  \l 299A5
  \l 299A6
  \l 299A7
  \l 299A8
  \l 299A9
  \l 299AA
  \l 299AB
  \l 299AC
  \l 299AD
  \l 299AE
  \l 299AF
  \l 299B0
  \l 299B1
  \l 299B2
  \l 299B3
  \l 299B4
  \l 299B5
  \l 299B6
  \l 299B7
  \l 299B8
  \l 299B9
  \l 299BA
  \l 299BB
  \l 299BC
  \l 299BD
  \l 299BE
  \l 299BF
  \l 299C0
  \l 299C1
  \l 299C2
  \l 299C3
  \l 299C4
  \l 299C5
  \l 299C6
  \l 299C7
  \l 299C8
  \l 299C9
  \l 299CA
  \l 299CB
  \l 299CC
  \l 299CD
  \l 299CE
  \l 299CF
  \l 299D0
  \l 299D1
  \l 299D2
  \l 299D3
  \l 299D4
  \l 299D5
  \l 299D6
  \l 299D7
  \l 299D8
  \l 299D9
  \l 299DA
  \l 299DB
  \l 299DC
  \l 299DD
  \l 299DE
  \l 299DF
  \l 299E0
  \l 299E1
  \l 299E2
  \l 299E3
  \l 299E4
  \l 299E5
  \l 299E6
  \l 299E7
  \l 299E8
  \l 299E9
  \l 299EA
  \l 299EB
  \l 299EC
  \l 299ED
  \l 299EE
  \l 299EF
  \l 299F0
  \l 299F1
  \l 299F2
  \l 299F3
  \l 299F4
  \l 299F5
  \l 299F6
  \l 299F7
  \l 299F8
  \l 299F9
  \l 299FA
  \l 299FB
  \l 299FC
  \l 299FD
  \l 299FE
  \l 299FF
  \l 29A00
  \l 29A01
  \l 29A02
  \l 29A03
  \l 29A04
  \l 29A05
  \l 29A06
  \l 29A07
  \l 29A08
  \l 29A09
  \l 29A0A
  \l 29A0B
  \l 29A0C
  \l 29A0D
  \l 29A0E
  \l 29A0F
  \l 29A10
  \l 29A11
  \l 29A12
  \l 29A13
  \l 29A14
  \l 29A15
  \l 29A16
  \l 29A17
  \l 29A18
  \l 29A19
  \l 29A1A
  \l 29A1B
  \l 29A1C
  \l 29A1D
  \l 29A1E
  \l 29A1F
  \l 29A20
  \l 29A21
  \l 29A22
  \l 29A23
  \l 29A24
  \l 29A25
  \l 29A26
  \l 29A27
  \l 29A28
  \l 29A29
  \l 29A2A
  \l 29A2B
  \l 29A2C
  \l 29A2D
  \l 29A2E
  \l 29A2F
  \l 29A30
  \l 29A31
  \l 29A32
  \l 29A33
  \l 29A34
  \l 29A35
  \l 29A36
  \l 29A37
  \l 29A38
  \l 29A39
  \l 29A3A
  \l 29A3B
  \l 29A3C
  \l 29A3D
  \l 29A3E
  \l 29A3F
  \l 29A40
  \l 29A41
  \l 29A42
  \l 29A43
  \l 29A44
  \l 29A45
  \l 29A46
  \l 29A47
  \l 29A48
  \l 29A49
  \l 29A4A
  \l 29A4B
  \l 29A4C
  \l 29A4D
  \l 29A4E
  \l 29A4F
  \l 29A50
  \l 29A51
  \l 29A52
  \l 29A53
  \l 29A54
  \l 29A55
  \l 29A56
  \l 29A57
  \l 29A58
  \l 29A59
  \l 29A5A
  \l 29A5B
  \l 29A5C
  \l 29A5D
  \l 29A5E
  \l 29A5F
  \l 29A60
  \l 29A61
  \l 29A62
  \l 29A63
  \l 29A64
  \l 29A65
  \l 29A66
  \l 29A67
  \l 29A68
  \l 29A69
  \l 29A6A
  \l 29A6B
  \l 29A6C
  \l 29A6D
  \l 29A6E
  \l 29A6F
  \l 29A70
  \l 29A71
  \l 29A72
  \l 29A73
  \l 29A74
  \l 29A75
  \l 29A76
  \l 29A77
  \l 29A78
  \l 29A79
  \l 29A7A
  \l 29A7B
  \l 29A7C
  \l 29A7D
  \l 29A7E
  \l 29A7F
  \l 29A80
  \l 29A81
  \l 29A82
  \l 29A83
  \l 29A84
  \l 29A85
  \l 29A86
  \l 29A87
  \l 29A88
  \l 29A89
  \l 29A8A
  \l 29A8B
  \l 29A8C
  \l 29A8D
  \l 29A8E
  \l 29A8F
  \l 29A90
  \l 29A91
  \l 29A92
  \l 29A93
  \l 29A94
  \l 29A95
  \l 29A96
  \l 29A97
  \l 29A98
  \l 29A99
  \l 29A9A
  \l 29A9B
  \l 29A9C
  \l 29A9D
  \l 29A9E
  \l 29A9F
  \l 29AA0
  \l 29AA1
  \l 29AA2
  \l 29AA3
  \l 29AA4
  \l 29AA5
  \l 29AA6
  \l 29AA7
  \l 29AA8
  \l 29AA9
  \l 29AAA
  \l 29AAB
  \l 29AAC
  \l 29AAD
  \l 29AAE
  \l 29AAF
  \l 29AB0
  \l 29AB1
  \l 29AB2
  \l 29AB3
  \l 29AB4
  \l 29AB5
  \l 29AB6
  \l 29AB7
  \l 29AB8
  \l 29AB9
  \l 29ABA
  \l 29ABB
  \l 29ABC
  \l 29ABD
  \l 29ABE
  \l 29ABF
  \l 29AC0
  \l 29AC1
  \l 29AC2
  \l 29AC3
  \l 29AC4
  \l 29AC5
  \l 29AC6
  \l 29AC7
  \l 29AC8
  \l 29AC9
  \l 29ACA
  \l 29ACB
  \l 29ACC
  \l 29ACD
  \l 29ACE
  \l 29ACF
  \l 29AD0
  \l 29AD1
  \l 29AD2
  \l 29AD3
  \l 29AD4
  \l 29AD5
  \l 29AD6
  \l 29AD7
  \l 29AD8
  \l 29AD9
  \l 29ADA
  \l 29ADB
  \l 29ADC
  \l 29ADD
  \l 29ADE
  \l 29ADF
  \l 29AE0
  \l 29AE1
  \l 29AE2
  \l 29AE3
  \l 29AE4
  \l 29AE5
  \l 29AE6
  \l 29AE7
  \l 29AE8
  \l 29AE9
  \l 29AEA
  \l 29AEB
  \l 29AEC
  \l 29AED
  \l 29AEE
  \l 29AEF
  \l 29AF0
  \l 29AF1
  \l 29AF2
  \l 29AF3
  \l 29AF4
  \l 29AF5
  \l 29AF6
  \l 29AF7
  \l 29AF8
  \l 29AF9
  \l 29AFA
  \l 29AFB
  \l 29AFC
  \l 29AFD
  \l 29AFE
  \l 29AFF
  \l 29B00
  \l 29B01
  \l 29B02
  \l 29B03
  \l 29B04
  \l 29B05
  \l 29B06
  \l 29B07
  \l 29B08
  \l 29B09
  \l 29B0A
  \l 29B0B
  \l 29B0C
  \l 29B0D
  \l 29B0E
  \l 29B0F
  \l 29B10
  \l 29B11
  \l 29B12
  \l 29B13
  \l 29B14
  \l 29B15
  \l 29B16
  \l 29B17
  \l 29B18
  \l 29B19
  \l 29B1A
  \l 29B1B
  \l 29B1C
  \l 29B1D
  \l 29B1E
  \l 29B1F
  \l 29B20
  \l 29B21
  \l 29B22
  \l 29B23
  \l 29B24
  \l 29B25
  \l 29B26
  \l 29B27
  \l 29B28
  \l 29B29
  \l 29B2A
  \l 29B2B
  \l 29B2C
  \l 29B2D
  \l 29B2E
  \l 29B2F
  \l 29B30
  \l 29B31
  \l 29B32
  \l 29B33
  \l 29B34
  \l 29B35
  \l 29B36
  \l 29B37
  \l 29B38
  \l 29B39
  \l 29B3A
  \l 29B3B
  \l 29B3C
  \l 29B3D
  \l 29B3E
  \l 29B3F
  \l 29B40
  \l 29B41
  \l 29B42
  \l 29B43
  \l 29B44
  \l 29B45
  \l 29B46
  \l 29B47
  \l 29B48
  \l 29B49
  \l 29B4A
  \l 29B4B
  \l 29B4C
  \l 29B4D
  \l 29B4E
  \l 29B4F
  \l 29B50
  \l 29B51
  \l 29B52
  \l 29B53
  \l 29B54
  \l 29B55
  \l 29B56
  \l 29B57
  \l 29B58
  \l 29B59
  \l 29B5A
  \l 29B5B
  \l 29B5C
  \l 29B5D
  \l 29B5E
  \l 29B5F
  \l 29B60
  \l 29B61
  \l 29B62
  \l 29B63
  \l 29B64
  \l 29B65
  \l 29B66
  \l 29B67
  \l 29B68
  \l 29B69
  \l 29B6A
  \l 29B6B
  \l 29B6C
  \l 29B6D
  \l 29B6E
  \l 29B6F
  \l 29B70
  \l 29B71
  \l 29B72
  \l 29B73
  \l 29B74
  \l 29B75
  \l 29B76
  \l 29B77
  \l 29B78
  \l 29B79
  \l 29B7A
  \l 29B7B
  \l 29B7C
  \l 29B7D
  \l 29B7E
  \l 29B7F
  \l 29B80
  \l 29B81
  \l 29B82
  \l 29B83
  \l 29B84
  \l 29B85
  \l 29B86
  \l 29B87
  \l 29B88
  \l 29B89
  \l 29B8A
  \l 29B8B
  \l 29B8C
  \l 29B8D
  \l 29B8E
  \l 29B8F
  \l 29B90
  \l 29B91
  \l 29B92
  \l 29B93
  \l 29B94
  \l 29B95
  \l 29B96
  \l 29B97
  \l 29B98
  \l 29B99
  \l 29B9A
  \l 29B9B
  \l 29B9C
  \l 29B9D
  \l 29B9E
  \l 29B9F
  \l 29BA0
  \l 29BA1
  \l 29BA2
  \l 29BA3
  \l 29BA4
  \l 29BA5
  \l 29BA6
  \l 29BA7
  \l 29BA8
  \l 29BA9
  \l 29BAA
  \l 29BAB
  \l 29BAC
  \l 29BAD
  \l 29BAE
  \l 29BAF
  \l 29BB0
  \l 29BB1
  \l 29BB2
  \l 29BB3
  \l 29BB4
  \l 29BB5
  \l 29BB6
  \l 29BB7
  \l 29BB8
  \l 29BB9
  \l 29BBA
  \l 29BBB
  \l 29BBC
  \l 29BBD
  \l 29BBE
  \l 29BBF
  \l 29BC0
  \l 29BC1
  \l 29BC2
  \l 29BC3
  \l 29BC4
  \l 29BC5
  \l 29BC6
  \l 29BC7
  \l 29BC8
  \l 29BC9
  \l 29BCA
  \l 29BCB
  \l 29BCC
  \l 29BCD
  \l 29BCE
  \l 29BCF
  \l 29BD0
  \l 29BD1
  \l 29BD2
  \l 29BD3
  \l 29BD4
  \l 29BD5
  \l 29BD6
  \l 29BD7
  \l 29BD8
  \l 29BD9
  \l 29BDA
  \l 29BDB
  \l 29BDC
  \l 29BDD
  \l 29BDE
  \l 29BDF
  \l 29BE0
  \l 29BE1
  \l 29BE2
  \l 29BE3
  \l 29BE4
  \l 29BE5
  \l 29BE6
  \l 29BE7
  \l 29BE8
  \l 29BE9
  \l 29BEA
  \l 29BEB
  \l 29BEC
  \l 29BED
  \l 29BEE
  \l 29BEF
  \l 29BF0
  \l 29BF1
  \l 29BF2
  \l 29BF3
  \l 29BF4
  \l 29BF5
  \l 29BF6
  \l 29BF7
  \l 29BF8
  \l 29BF9
  \l 29BFA
  \l 29BFB
  \l 29BFC
  \l 29BFD
  \l 29BFE
  \l 29BFF
  \l 29C00
  \l 29C01
  \l 29C02
  \l 29C03
  \l 29C04
  \l 29C05
  \l 29C06
  \l 29C07
  \l 29C08
  \l 29C09
  \l 29C0A
  \l 29C0B
  \l 29C0C
  \l 29C0D
  \l 29C0E
  \l 29C0F
  \l 29C10
  \l 29C11
  \l 29C12
  \l 29C13
  \l 29C14
  \l 29C15
  \l 29C16
  \l 29C17
  \l 29C18
  \l 29C19
  \l 29C1A
  \l 29C1B
  \l 29C1C
  \l 29C1D
  \l 29C1E
  \l 29C1F
  \l 29C20
  \l 29C21
  \l 29C22
  \l 29C23
  \l 29C24
  \l 29C25
  \l 29C26
  \l 29C27
  \l 29C28
  \l 29C29
  \l 29C2A
  \l 29C2B
  \l 29C2C
  \l 29C2D
  \l 29C2E
  \l 29C2F
  \l 29C30
  \l 29C31
  \l 29C32
  \l 29C33
  \l 29C34
  \l 29C35
  \l 29C36
  \l 29C37
  \l 29C38
  \l 29C39
  \l 29C3A
  \l 29C3B
  \l 29C3C
  \l 29C3D
  \l 29C3E
  \l 29C3F
  \l 29C40
  \l 29C41
  \l 29C42
  \l 29C43
  \l 29C44
  \l 29C45
  \l 29C46
  \l 29C47
  \l 29C48
  \l 29C49
  \l 29C4A
  \l 29C4B
  \l 29C4C
  \l 29C4D
  \l 29C4E
  \l 29C4F
  \l 29C50
  \l 29C51
  \l 29C52
  \l 29C53
  \l 29C54
  \l 29C55
  \l 29C56
  \l 29C57
  \l 29C58
  \l 29C59
  \l 29C5A
  \l 29C5B
  \l 29C5C
  \l 29C5D
  \l 29C5E
  \l 29C5F
  \l 29C60
  \l 29C61
  \l 29C62
  \l 29C63
  \l 29C64
  \l 29C65
  \l 29C66
  \l 29C67
  \l 29C68
  \l 29C69
  \l 29C6A
  \l 29C6B
  \l 29C6C
  \l 29C6D
  \l 29C6E
  \l 29C6F
  \l 29C70
  \l 29C71
  \l 29C72
  \l 29C73
  \l 29C74
  \l 29C75
  \l 29C76
  \l 29C77
  \l 29C78
  \l 29C79
  \l 29C7A
  \l 29C7B
  \l 29C7C
  \l 29C7D
  \l 29C7E
  \l 29C7F
  \l 29C80
  \l 29C81
  \l 29C82
  \l 29C83
  \l 29C84
  \l 29C85
  \l 29C86
  \l 29C87
  \l 29C88
  \l 29C89
  \l 29C8A
  \l 29C8B
  \l 29C8C
  \l 29C8D
  \l 29C8E
  \l 29C8F
  \l 29C90
  \l 29C91
  \l 29C92
  \l 29C93
  \l 29C94
  \l 29C95
  \l 29C96
  \l 29C97
  \l 29C98
  \l 29C99
  \l 29C9A
  \l 29C9B
  \l 29C9C
  \l 29C9D
  \l 29C9E
  \l 29C9F
  \l 29CA0
  \l 29CA1
  \l 29CA2
  \l 29CA3
  \l 29CA4
  \l 29CA5
  \l 29CA6
  \l 29CA7
  \l 29CA8
  \l 29CA9
  \l 29CAA
  \l 29CAB
  \l 29CAC
  \l 29CAD
  \l 29CAE
  \l 29CAF
  \l 29CB0
  \l 29CB1
  \l 29CB2
  \l 29CB3
  \l 29CB4
  \l 29CB5
  \l 29CB6
  \l 29CB7
  \l 29CB8
  \l 29CB9
  \l 29CBA
  \l 29CBB
  \l 29CBC
  \l 29CBD
  \l 29CBE
  \l 29CBF
  \l 29CC0
  \l 29CC1
  \l 29CC2
  \l 29CC3
  \l 29CC4
  \l 29CC5
  \l 29CC6
  \l 29CC7
  \l 29CC8
  \l 29CC9
  \l 29CCA
  \l 29CCB
  \l 29CCC
  \l 29CCD
  \l 29CCE
  \l 29CCF
  \l 29CD0
  \l 29CD1
  \l 29CD2
  \l 29CD3
  \l 29CD4
  \l 29CD5
  \l 29CD6
  \l 29CD7
  \l 29CD8
  \l 29CD9
  \l 29CDA
  \l 29CDB
  \l 29CDC
  \l 29CDD
  \l 29CDE
  \l 29CDF
  \l 29CE0
  \l 29CE1
  \l 29CE2
  \l 29CE3
  \l 29CE4
  \l 29CE5
  \l 29CE6
  \l 29CE7
  \l 29CE8
  \l 29CE9
  \l 29CEA
  \l 29CEB
  \l 29CEC
  \l 29CED
  \l 29CEE
  \l 29CEF
  \l 29CF0
  \l 29CF1
  \l 29CF2
  \l 29CF3
  \l 29CF4
  \l 29CF5
  \l 29CF6
  \l 29CF7
  \l 29CF8
  \l 29CF9
  \l 29CFA
  \l 29CFB
  \l 29CFC
  \l 29CFD
  \l 29CFE
  \l 29CFF
  \l 29D00
  \l 29D01
  \l 29D02
  \l 29D03
  \l 29D04
  \l 29D05
  \l 29D06
  \l 29D07
  \l 29D08
  \l 29D09
  \l 29D0A
  \l 29D0B
  \l 29D0C
  \l 29D0D
  \l 29D0E
  \l 29D0F
  \l 29D10
  \l 29D11
  \l 29D12
  \l 29D13
  \l 29D14
  \l 29D15
  \l 29D16
  \l 29D17
  \l 29D18
  \l 29D19
  \l 29D1A
  \l 29D1B
  \l 29D1C
  \l 29D1D
  \l 29D1E
  \l 29D1F
  \l 29D20
  \l 29D21
  \l 29D22
  \l 29D23
  \l 29D24
  \l 29D25
  \l 29D26
  \l 29D27
  \l 29D28
  \l 29D29
  \l 29D2A
  \l 29D2B
  \l 29D2C
  \l 29D2D
  \l 29D2E
  \l 29D2F
  \l 29D30
  \l 29D31
  \l 29D32
  \l 29D33
  \l 29D34
  \l 29D35
  \l 29D36
  \l 29D37
  \l 29D38
  \l 29D39
  \l 29D3A
  \l 29D3B
  \l 29D3C
  \l 29D3D
  \l 29D3E
  \l 29D3F
  \l 29D40
  \l 29D41
  \l 29D42
  \l 29D43
  \l 29D44
  \l 29D45
  \l 29D46
  \l 29D47
  \l 29D48
  \l 29D49
  \l 29D4A
  \l 29D4B
  \l 29D4C
  \l 29D4D
  \l 29D4E
  \l 29D4F
  \l 29D50
  \l 29D51
  \l 29D52
  \l 29D53
  \l 29D54
  \l 29D55
  \l 29D56
  \l 29D57
  \l 29D58
  \l 29D59
  \l 29D5A
  \l 29D5B
  \l 29D5C
  \l 29D5D
  \l 29D5E
  \l 29D5F
  \l 29D60
  \l 29D61
  \l 29D62
  \l 29D63
  \l 29D64
  \l 29D65
  \l 29D66
  \l 29D67
  \l 29D68
  \l 29D69
  \l 29D6A
  \l 29D6B
  \l 29D6C
  \l 29D6D
  \l 29D6E
  \l 29D6F
  \l 29D70
  \l 29D71
  \l 29D72
  \l 29D73
  \l 29D74
  \l 29D75
  \l 29D76
  \l 29D77
  \l 29D78
  \l 29D79
  \l 29D7A
  \l 29D7B
  \l 29D7C
  \l 29D7D
  \l 29D7E
  \l 29D7F
  \l 29D80
  \l 29D81
  \l 29D82
  \l 29D83
  \l 29D84
  \l 29D85
  \l 29D86
  \l 29D87
  \l 29D88
  \l 29D89
  \l 29D8A
  \l 29D8B
  \l 29D8C
  \l 29D8D
  \l 29D8E
  \l 29D8F
  \l 29D90
  \l 29D91
  \l 29D92
  \l 29D93
  \l 29D94
  \l 29D95
  \l 29D96
  \l 29D97
  \l 29D98
  \l 29D99
  \l 29D9A
  \l 29D9B
  \l 29D9C
  \l 29D9D
  \l 29D9E
  \l 29D9F
  \l 29DA0
  \l 29DA1
  \l 29DA2
  \l 29DA3
  \l 29DA4
  \l 29DA5
  \l 29DA6
  \l 29DA7
  \l 29DA8
  \l 29DA9
  \l 29DAA
  \l 29DAB
  \l 29DAC
  \l 29DAD
  \l 29DAE
  \l 29DAF
  \l 29DB0
  \l 29DB1
  \l 29DB2
  \l 29DB3
  \l 29DB4
  \l 29DB5
  \l 29DB6
  \l 29DB7
  \l 29DB8
  \l 29DB9
  \l 29DBA
  \l 29DBB
  \l 29DBC
  \l 29DBD
  \l 29DBE
  \l 29DBF
  \l 29DC0
  \l 29DC1
  \l 29DC2
  \l 29DC3
  \l 29DC4
  \l 29DC5
  \l 29DC6
  \l 29DC7
  \l 29DC8
  \l 29DC9
  \l 29DCA
  \l 29DCB
  \l 29DCC
  \l 29DCD
  \l 29DCE
  \l 29DCF
  \l 29DD0
  \l 29DD1
  \l 29DD2
  \l 29DD3
  \l 29DD4
  \l 29DD5
  \l 29DD6
  \l 29DD7
  \l 29DD8
  \l 29DD9
  \l 29DDA
  \l 29DDB
  \l 29DDC
  \l 29DDD
  \l 29DDE
  \l 29DDF
  \l 29DE0
  \l 29DE1
  \l 29DE2
  \l 29DE3
  \l 29DE4
  \l 29DE5
  \l 29DE6
  \l 29DE7
  \l 29DE8
  \l 29DE9
  \l 29DEA
  \l 29DEB
  \l 29DEC
  \l 29DED
  \l 29DEE
  \l 29DEF
  \l 29DF0
  \l 29DF1
  \l 29DF2
  \l 29DF3
  \l 29DF4
  \l 29DF5
  \l 29DF6
  \l 29DF7
  \l 29DF8
  \l 29DF9
  \l 29DFA
  \l 29DFB
  \l 29DFC
  \l 29DFD
  \l 29DFE
  \l 29DFF
  \l 29E00
  \l 29E01
  \l 29E02
  \l 29E03
  \l 29E04
  \l 29E05
  \l 29E06
  \l 29E07
  \l 29E08
  \l 29E09
  \l 29E0A
  \l 29E0B
  \l 29E0C
  \l 29E0D
  \l 29E0E
  \l 29E0F
  \l 29E10
  \l 29E11
  \l 29E12
  \l 29E13
  \l 29E14
  \l 29E15
  \l 29E16
  \l 29E17
  \l 29E18
  \l 29E19
  \l 29E1A
  \l 29E1B
  \l 29E1C
  \l 29E1D
  \l 29E1E
  \l 29E1F
  \l 29E20
  \l 29E21
  \l 29E22
  \l 29E23
  \l 29E24
  \l 29E25
  \l 29E26
  \l 29E27
  \l 29E28
  \l 29E29
  \l 29E2A
  \l 29E2B
  \l 29E2C
  \l 29E2D
  \l 29E2E
  \l 29E2F
  \l 29E30
  \l 29E31
  \l 29E32
  \l 29E33
  \l 29E34
  \l 29E35
  \l 29E36
  \l 29E37
  \l 29E38
  \l 29E39
  \l 29E3A
  \l 29E3B
  \l 29E3C
  \l 29E3D
  \l 29E3E
  \l 29E3F
  \l 29E40
  \l 29E41
  \l 29E42
  \l 29E43
  \l 29E44
  \l 29E45
  \l 29E46
  \l 29E47
  \l 29E48
  \l 29E49
  \l 29E4A
  \l 29E4B
  \l 29E4C
  \l 29E4D
  \l 29E4E
  \l 29E4F
  \l 29E50
  \l 29E51
  \l 29E52
  \l 29E53
  \l 29E54
  \l 29E55
  \l 29E56
  \l 29E57
  \l 29E58
  \l 29E59
  \l 29E5A
  \l 29E5B
  \l 29E5C
  \l 29E5D
  \l 29E5E
  \l 29E5F
  \l 29E60
  \l 29E61
  \l 29E62
  \l 29E63
  \l 29E64
  \l 29E65
  \l 29E66
  \l 29E67
  \l 29E68
  \l 29E69
  \l 29E6A
  \l 29E6B
  \l 29E6C
  \l 29E6D
  \l 29E6E
  \l 29E6F
  \l 29E70
  \l 29E71
  \l 29E72
  \l 29E73
  \l 29E74
  \l 29E75
  \l 29E76
  \l 29E77
  \l 29E78
  \l 29E79
  \l 29E7A
  \l 29E7B
  \l 29E7C
  \l 29E7D
  \l 29E7E
  \l 29E7F
  \l 29E80
  \l 29E81
  \l 29E82
  \l 29E83
  \l 29E84
  \l 29E85
  \l 29E86
  \l 29E87
  \l 29E88
  \l 29E89
  \l 29E8A
  \l 29E8B
  \l 29E8C
  \l 29E8D
  \l 29E8E
  \l 29E8F
  \l 29E90
  \l 29E91
  \l 29E92
  \l 29E93
  \l 29E94
  \l 29E95
  \l 29E96
  \l 29E97
  \l 29E98
  \l 29E99
  \l 29E9A
  \l 29E9B
  \l 29E9C
  \l 29E9D
  \l 29E9E
  \l 29E9F
  \l 29EA0
  \l 29EA1
  \l 29EA2
  \l 29EA3
  \l 29EA4
  \l 29EA5
  \l 29EA6
  \l 29EA7
  \l 29EA8
  \l 29EA9
  \l 29EAA
  \l 29EAB
  \l 29EAC
  \l 29EAD
  \l 29EAE
  \l 29EAF
  \l 29EB0
  \l 29EB1
  \l 29EB2
  \l 29EB3
  \l 29EB4
  \l 29EB5
  \l 29EB6
  \l 29EB7
  \l 29EB8
  \l 29EB9
  \l 29EBA
  \l 29EBB
  \l 29EBC
  \l 29EBD
  \l 29EBE
  \l 29EBF
  \l 29EC0
  \l 29EC1
  \l 29EC2
  \l 29EC3
  \l 29EC4
  \l 29EC5
  \l 29EC6
  \l 29EC7
  \l 29EC8
  \l 29EC9
  \l 29ECA
  \l 29ECB
  \l 29ECC
  \l 29ECD
  \l 29ECE
  \l 29ECF
  \l 29ED0
  \l 29ED1
  \l 29ED2
  \l 29ED3
  \l 29ED4
  \l 29ED5
  \l 29ED6
  \l 29ED7
  \l 29ED8
  \l 29ED9
  \l 29EDA
  \l 29EDB
  \l 29EDC
  \l 29EDD
  \l 29EDE
  \l 29EDF
  \l 29EE0
  \l 29EE1
  \l 29EE2
  \l 29EE3
  \l 29EE4
  \l 29EE5
  \l 29EE6
  \l 29EE7
  \l 29EE8
  \l 29EE9
  \l 29EEA
  \l 29EEB
  \l 29EEC
  \l 29EED
  \l 29EEE
  \l 29EEF
  \l 29EF0
  \l 29EF1
  \l 29EF2
  \l 29EF3
  \l 29EF4
  \l 29EF5
  \l 29EF6
  \l 29EF7
  \l 29EF8
  \l 29EF9
  \l 29EFA
  \l 29EFB
  \l 29EFC
  \l 29EFD
  \l 29EFE
  \l 29EFF
  \l 29F00
  \l 29F01
  \l 29F02
  \l 29F03
  \l 29F04
  \l 29F05
  \l 29F06
  \l 29F07
  \l 29F08
  \l 29F09
  \l 29F0A
  \l 29F0B
  \l 29F0C
  \l 29F0D
  \l 29F0E
  \l 29F0F
  \l 29F10
  \l 29F11
  \l 29F12
  \l 29F13
  \l 29F14
  \l 29F15
  \l 29F16
  \l 29F17
  \l 29F18
  \l 29F19
  \l 29F1A
  \l 29F1B
  \l 29F1C
  \l 29F1D
  \l 29F1E
  \l 29F1F
  \l 29F20
  \l 29F21
  \l 29F22
  \l 29F23
  \l 29F24
  \l 29F25
  \l 29F26
  \l 29F27
  \l 29F28
  \l 29F29
  \l 29F2A
  \l 29F2B
  \l 29F2C
  \l 29F2D
  \l 29F2E
  \l 29F2F
  \l 29F30
  \l 29F31
  \l 29F32
  \l 29F33
  \l 29F34
  \l 29F35
  \l 29F36
  \l 29F37
  \l 29F38
  \l 29F39
  \l 29F3A
  \l 29F3B
  \l 29F3C
  \l 29F3D
  \l 29F3E
  \l 29F3F
  \l 29F40
  \l 29F41
  \l 29F42
  \l 29F43
  \l 29F44
  \l 29F45
  \l 29F46
  \l 29F47
  \l 29F48
  \l 29F49
  \l 29F4A
  \l 29F4B
  \l 29F4C
  \l 29F4D
  \l 29F4E
  \l 29F4F
  \l 29F50
  \l 29F51
  \l 29F52
  \l 29F53
  \l 29F54
  \l 29F55
  \l 29F56
  \l 29F57
  \l 29F58
  \l 29F59
  \l 29F5A
  \l 29F5B
  \l 29F5C
  \l 29F5D
  \l 29F5E
  \l 29F5F
  \l 29F60
  \l 29F61
  \l 29F62
  \l 29F63
  \l 29F64
  \l 29F65
  \l 29F66
  \l 29F67
  \l 29F68
  \l 29F69
  \l 29F6A
  \l 29F6B
  \l 29F6C
  \l 29F6D
  \l 29F6E
  \l 29F6F
  \l 29F70
  \l 29F71
  \l 29F72
  \l 29F73
  \l 29F74
  \l 29F75
  \l 29F76
  \l 29F77
  \l 29F78
  \l 29F79
  \l 29F7A
  \l 29F7B
  \l 29F7C
  \l 29F7D
  \l 29F7E
  \l 29F7F
  \l 29F80
  \l 29F81
  \l 29F82
  \l 29F83
  \l 29F84
  \l 29F85
  \l 29F86
  \l 29F87
  \l 29F88
  \l 29F89
  \l 29F8A
  \l 29F8B
  \l 29F8C
  \l 29F8D
  \l 29F8E
  \l 29F8F
  \l 29F90
  \l 29F91
  \l 29F92
  \l 29F93
  \l 29F94
  \l 29F95
  \l 29F96
  \l 29F97
  \l 29F98
  \l 29F99
  \l 29F9A
  \l 29F9B
  \l 29F9C
  \l 29F9D
  \l 29F9E
  \l 29F9F
  \l 29FA0
  \l 29FA1
  \l 29FA2
  \l 29FA3
  \l 29FA4
  \l 29FA5
  \l 29FA6
  \l 29FA7
  \l 29FA8
  \l 29FA9
  \l 29FAA
  \l 29FAB
  \l 29FAC
  \l 29FAD
  \l 29FAE
  \l 29FAF
  \l 29FB0
  \l 29FB1
  \l 29FB2
  \l 29FB3
  \l 29FB4
  \l 29FB5
  \l 29FB6
  \l 29FB7
  \l 29FB8
  \l 29FB9
  \l 29FBA
  \l 29FBB
  \l 29FBC
  \l 29FBD
  \l 29FBE
  \l 29FBF
  \l 29FC0
  \l 29FC1
  \l 29FC2
  \l 29FC3
  \l 29FC4
  \l 29FC5
  \l 29FC6
  \l 29FC7
  \l 29FC8
  \l 29FC9
  \l 29FCA
  \l 29FCB
  \l 29FCC
  \l 29FCD
  \l 29FCE
  \l 29FCF
  \l 29FD0
  \l 29FD1
  \l 29FD2
  \l 29FD3
  \l 29FD4
  \l 29FD5
  \l 29FD6
  \l 29FD7
  \l 29FD8
  \l 29FD9
  \l 29FDA
  \l 29FDB
  \l 29FDC
  \l 29FDD
  \l 29FDE
  \l 29FDF
  \l 29FE0
  \l 29FE1
  \l 29FE2
  \l 29FE3
  \l 29FE4
  \l 29FE5
  \l 29FE6
  \l 29FE7
  \l 29FE8
  \l 29FE9
  \l 29FEA
  \l 29FEB
  \l 29FEC
  \l 29FED
  \l 29FEE
  \l 29FEF
  \l 29FF0
  \l 29FF1
  \l 29FF2
  \l 29FF3
  \l 29FF4
  \l 29FF5
  \l 29FF6
  \l 29FF7
  \l 29FF8
  \l 29FF9
  \l 29FFA
  \l 29FFB
  \l 29FFC
  \l 29FFD
  \l 29FFE
  \l 29FFF
  \l 2A000
  \l 2A001
  \l 2A002
  \l 2A003
  \l 2A004
  \l 2A005
  \l 2A006
  \l 2A007
  \l 2A008
  \l 2A009
  \l 2A00A
  \l 2A00B
  \l 2A00C
  \l 2A00D
  \l 2A00E
  \l 2A00F
  \l 2A010
  \l 2A011
  \l 2A012
  \l 2A013
  \l 2A014
  \l 2A015
  \l 2A016
  \l 2A017
  \l 2A018
  \l 2A019
  \l 2A01A
  \l 2A01B
  \l 2A01C
  \l 2A01D
  \l 2A01E
  \l 2A01F
  \l 2A020
  \l 2A021
  \l 2A022
  \l 2A023
  \l 2A024
  \l 2A025
  \l 2A026
  \l 2A027
  \l 2A028
  \l 2A029
  \l 2A02A
  \l 2A02B
  \l 2A02C
  \l 2A02D
  \l 2A02E
  \l 2A02F
  \l 2A030
  \l 2A031
  \l 2A032
  \l 2A033
  \l 2A034
  \l 2A035
  \l 2A036
  \l 2A037
  \l 2A038
  \l 2A039
  \l 2A03A
  \l 2A03B
  \l 2A03C
  \l 2A03D
  \l 2A03E
  \l 2A03F
  \l 2A040
  \l 2A041
  \l 2A042
  \l 2A043
  \l 2A044
  \l 2A045
  \l 2A046
  \l 2A047
  \l 2A048
  \l 2A049
  \l 2A04A
  \l 2A04B
  \l 2A04C
  \l 2A04D
  \l 2A04E
  \l 2A04F
  \l 2A050
  \l 2A051
  \l 2A052
  \l 2A053
  \l 2A054
  \l 2A055
  \l 2A056
  \l 2A057
  \l 2A058
  \l 2A059
  \l 2A05A
  \l 2A05B
  \l 2A05C
  \l 2A05D
  \l 2A05E
  \l 2A05F
  \l 2A060
  \l 2A061
  \l 2A062
  \l 2A063
  \l 2A064
  \l 2A065
  \l 2A066
  \l 2A067
  \l 2A068
  \l 2A069
  \l 2A06A
  \l 2A06B
  \l 2A06C
  \l 2A06D
  \l 2A06E
  \l 2A06F
  \l 2A070
  \l 2A071
  \l 2A072
  \l 2A073
  \l 2A074
  \l 2A075
  \l 2A076
  \l 2A077
  \l 2A078
  \l 2A079
  \l 2A07A
  \l 2A07B
  \l 2A07C
  \l 2A07D
  \l 2A07E
  \l 2A07F
  \l 2A080
  \l 2A081
  \l 2A082
  \l 2A083
  \l 2A084
  \l 2A085
  \l 2A086
  \l 2A087
  \l 2A088
  \l 2A089
  \l 2A08A
  \l 2A08B
  \l 2A08C
  \l 2A08D
  \l 2A08E
  \l 2A08F
  \l 2A090
  \l 2A091
  \l 2A092
  \l 2A093
  \l 2A094
  \l 2A095
  \l 2A096
  \l 2A097
  \l 2A098
  \l 2A099
  \l 2A09A
  \l 2A09B
  \l 2A09C
  \l 2A09D
  \l 2A09E
  \l 2A09F
  \l 2A0A0
  \l 2A0A1
  \l 2A0A2
  \l 2A0A3
  \l 2A0A4
  \l 2A0A5
  \l 2A0A6
  \l 2A0A7
  \l 2A0A8
  \l 2A0A9
  \l 2A0AA
  \l 2A0AB
  \l 2A0AC
  \l 2A0AD
  \l 2A0AE
  \l 2A0AF
  \l 2A0B0
  \l 2A0B1
  \l 2A0B2
  \l 2A0B3
  \l 2A0B4
  \l 2A0B5
  \l 2A0B6
  \l 2A0B7
  \l 2A0B8
  \l 2A0B9
  \l 2A0BA
  \l 2A0BB
  \l 2A0BC
  \l 2A0BD
  \l 2A0BE
  \l 2A0BF
  \l 2A0C0
  \l 2A0C1
  \l 2A0C2
  \l 2A0C3
  \l 2A0C4
  \l 2A0C5
  \l 2A0C6
  \l 2A0C7
  \l 2A0C8
  \l 2A0C9
  \l 2A0CA
  \l 2A0CB
  \l 2A0CC
  \l 2A0CD
  \l 2A0CE
  \l 2A0CF
  \l 2A0D0
  \l 2A0D1
  \l 2A0D2
  \l 2A0D3
  \l 2A0D4
  \l 2A0D5
  \l 2A0D6
  \l 2A0D7
  \l 2A0D8
  \l 2A0D9
  \l 2A0DA
  \l 2A0DB
  \l 2A0DC
  \l 2A0DD
  \l 2A0DE
  \l 2A0DF
  \l 2A0E0
  \l 2A0E1
  \l 2A0E2
  \l 2A0E3
  \l 2A0E4
  \l 2A0E5
  \l 2A0E6
  \l 2A0E7
  \l 2A0E8
  \l 2A0E9
  \l 2A0EA
  \l 2A0EB
  \l 2A0EC
  \l 2A0ED
  \l 2A0EE
  \l 2A0EF
  \l 2A0F0
  \l 2A0F1
  \l 2A0F2
  \l 2A0F3
  \l 2A0F4
  \l 2A0F5
  \l 2A0F6
  \l 2A0F7
  \l 2A0F8
  \l 2A0F9
  \l 2A0FA
  \l 2A0FB
  \l 2A0FC
  \l 2A0FD
  \l 2A0FE
  \l 2A0FF
  \l 2A100
  \l 2A101
  \l 2A102
  \l 2A103
  \l 2A104
  \l 2A105
  \l 2A106
  \l 2A107
  \l 2A108
  \l 2A109
  \l 2A10A
  \l 2A10B
  \l 2A10C
  \l 2A10D
  \l 2A10E
  \l 2A10F
  \l 2A110
  \l 2A111
  \l 2A112
  \l 2A113
  \l 2A114
  \l 2A115
  \l 2A116
  \l 2A117
  \l 2A118
  \l 2A119
  \l 2A11A
  \l 2A11B
  \l 2A11C
  \l 2A11D
  \l 2A11E
  \l 2A11F
  \l 2A120
  \l 2A121
  \l 2A122
  \l 2A123
  \l 2A124
  \l 2A125
  \l 2A126
  \l 2A127
  \l 2A128
  \l 2A129
  \l 2A12A
  \l 2A12B
  \l 2A12C
  \l 2A12D
  \l 2A12E
  \l 2A12F
  \l 2A130
  \l 2A131
  \l 2A132
  \l 2A133
  \l 2A134
  \l 2A135
  \l 2A136
  \l 2A137
  \l 2A138
  \l 2A139
  \l 2A13A
  \l 2A13B
  \l 2A13C
  \l 2A13D
  \l 2A13E
  \l 2A13F
  \l 2A140
  \l 2A141
  \l 2A142
  \l 2A143
  \l 2A144
  \l 2A145
  \l 2A146
  \l 2A147
  \l 2A148
  \l 2A149
  \l 2A14A
  \l 2A14B
  \l 2A14C
  \l 2A14D
  \l 2A14E
  \l 2A14F
  \l 2A150
  \l 2A151
  \l 2A152
  \l 2A153
  \l 2A154
  \l 2A155
  \l 2A156
  \l 2A157
  \l 2A158
  \l 2A159
  \l 2A15A
  \l 2A15B
  \l 2A15C
  \l 2A15D
  \l 2A15E
  \l 2A15F
  \l 2A160
  \l 2A161
  \l 2A162
  \l 2A163
  \l 2A164
  \l 2A165
  \l 2A166
  \l 2A167
  \l 2A168
  \l 2A169
  \l 2A16A
  \l 2A16B
  \l 2A16C
  \l 2A16D
  \l 2A16E
  \l 2A16F
  \l 2A170
  \l 2A171
  \l 2A172
  \l 2A173
  \l 2A174
  \l 2A175
  \l 2A176
  \l 2A177
  \l 2A178
  \l 2A179
  \l 2A17A
  \l 2A17B
  \l 2A17C
  \l 2A17D
  \l 2A17E
  \l 2A17F
  \l 2A180
  \l 2A181
  \l 2A182
  \l 2A183
  \l 2A184
  \l 2A185
  \l 2A186
  \l 2A187
  \l 2A188
  \l 2A189
  \l 2A18A
  \l 2A18B
  \l 2A18C
  \l 2A18D
  \l 2A18E
  \l 2A18F
  \l 2A190
  \l 2A191
  \l 2A192
  \l 2A193
  \l 2A194
  \l 2A195
  \l 2A196
  \l 2A197
  \l 2A198
  \l 2A199
  \l 2A19A
  \l 2A19B
  \l 2A19C
  \l 2A19D
  \l 2A19E
  \l 2A19F
  \l 2A1A0
  \l 2A1A1
  \l 2A1A2
  \l 2A1A3
  \l 2A1A4
  \l 2A1A5
  \l 2A1A6
  \l 2A1A7
  \l 2A1A8
  \l 2A1A9
  \l 2A1AA
  \l 2A1AB
  \l 2A1AC
  \l 2A1AD
  \l 2A1AE
  \l 2A1AF
  \l 2A1B0
  \l 2A1B1
  \l 2A1B2
  \l 2A1B3
  \l 2A1B4
  \l 2A1B5
  \l 2A1B6
  \l 2A1B7
  \l 2A1B8
  \l 2A1B9
  \l 2A1BA
  \l 2A1BB
  \l 2A1BC
  \l 2A1BD
  \l 2A1BE
  \l 2A1BF
  \l 2A1C0
  \l 2A1C1
  \l 2A1C2
  \l 2A1C3
  \l 2A1C4
  \l 2A1C5
  \l 2A1C6
  \l 2A1C7
  \l 2A1C8
  \l 2A1C9
  \l 2A1CA
  \l 2A1CB
  \l 2A1CC
  \l 2A1CD
  \l 2A1CE
  \l 2A1CF
  \l 2A1D0
  \l 2A1D1
  \l 2A1D2
  \l 2A1D3
  \l 2A1D4
  \l 2A1D5
  \l 2A1D6
  \l 2A1D7
  \l 2A1D8
  \l 2A1D9
  \l 2A1DA
  \l 2A1DB
  \l 2A1DC
  \l 2A1DD
  \l 2A1DE
  \l 2A1DF
  \l 2A1E0
  \l 2A1E1
  \l 2A1E2
  \l 2A1E3
  \l 2A1E4
  \l 2A1E5
  \l 2A1E6
  \l 2A1E7
  \l 2A1E8
  \l 2A1E9
  \l 2A1EA
  \l 2A1EB
  \l 2A1EC
  \l 2A1ED
  \l 2A1EE
  \l 2A1EF
  \l 2A1F0
  \l 2A1F1
  \l 2A1F2
  \l 2A1F3
  \l 2A1F4
  \l 2A1F5
  \l 2A1F6
  \l 2A1F7
  \l 2A1F8
  \l 2A1F9
  \l 2A1FA
  \l 2A1FB
  \l 2A1FC
  \l 2A1FD
  \l 2A1FE
  \l 2A1FF
  \l 2A200
  \l 2A201
  \l 2A202
  \l 2A203
  \l 2A204
  \l 2A205
  \l 2A206
  \l 2A207
  \l 2A208
  \l 2A209
  \l 2A20A
  \l 2A20B
  \l 2A20C
  \l 2A20D
  \l 2A20E
  \l 2A20F
  \l 2A210
  \l 2A211
  \l 2A212
  \l 2A213
  \l 2A214
  \l 2A215
  \l 2A216
  \l 2A217
  \l 2A218
  \l 2A219
  \l 2A21A
  \l 2A21B
  \l 2A21C
  \l 2A21D
  \l 2A21E
  \l 2A21F
  \l 2A220
  \l 2A221
  \l 2A222
  \l 2A223
  \l 2A224
  \l 2A225
  \l 2A226
  \l 2A227
  \l 2A228
  \l 2A229
  \l 2A22A
  \l 2A22B
  \l 2A22C
  \l 2A22D
  \l 2A22E
  \l 2A22F
  \l 2A230
  \l 2A231
  \l 2A232
  \l 2A233
  \l 2A234
  \l 2A235
  \l 2A236
  \l 2A237
  \l 2A238
  \l 2A239
  \l 2A23A
  \l 2A23B
  \l 2A23C
  \l 2A23D
  \l 2A23E
  \l 2A23F
  \l 2A240
  \l 2A241
  \l 2A242
  \l 2A243
  \l 2A244
  \l 2A245
  \l 2A246
  \l 2A247
  \l 2A248
  \l 2A249
  \l 2A24A
  \l 2A24B
  \l 2A24C
  \l 2A24D
  \l 2A24E
  \l 2A24F
  \l 2A250
  \l 2A251
  \l 2A252
  \l 2A253
  \l 2A254
  \l 2A255
  \l 2A256
  \l 2A257
  \l 2A258
  \l 2A259
  \l 2A25A
  \l 2A25B
  \l 2A25C
  \l 2A25D
  \l 2A25E
  \l 2A25F
  \l 2A260
  \l 2A261
  \l 2A262
  \l 2A263
  \l 2A264
  \l 2A265
  \l 2A266
  \l 2A267
  \l 2A268
  \l 2A269
  \l 2A26A
  \l 2A26B
  \l 2A26C
  \l 2A26D
  \l 2A26E
  \l 2A26F
  \l 2A270
  \l 2A271
  \l 2A272
  \l 2A273
  \l 2A274
  \l 2A275
  \l 2A276
  \l 2A277
  \l 2A278
  \l 2A279
  \l 2A27A
  \l 2A27B
  \l 2A27C
  \l 2A27D
  \l 2A27E
  \l 2A27F
  \l 2A280
  \l 2A281
  \l 2A282
  \l 2A283
  \l 2A284
  \l 2A285
  \l 2A286
  \l 2A287
  \l 2A288
  \l 2A289
  \l 2A28A
  \l 2A28B
  \l 2A28C
  \l 2A28D
  \l 2A28E
  \l 2A28F
  \l 2A290
  \l 2A291
  \l 2A292
  \l 2A293
  \l 2A294
  \l 2A295
  \l 2A296
  \l 2A297
  \l 2A298
  \l 2A299
  \l 2A29A
  \l 2A29B
  \l 2A29C
  \l 2A29D
  \l 2A29E
  \l 2A29F
  \l 2A2A0
  \l 2A2A1
  \l 2A2A2
  \l 2A2A3
  \l 2A2A4
  \l 2A2A5
  \l 2A2A6
  \l 2A2A7
  \l 2A2A8
  \l 2A2A9
  \l 2A2AA
  \l 2A2AB
  \l 2A2AC
  \l 2A2AD
  \l 2A2AE
  \l 2A2AF
  \l 2A2B0
  \l 2A2B1
  \l 2A2B2
  \l 2A2B3
  \l 2A2B4
  \l 2A2B5
  \l 2A2B6
  \l 2A2B7
  \l 2A2B8
  \l 2A2B9
  \l 2A2BA
  \l 2A2BB
  \l 2A2BC
  \l 2A2BD
  \l 2A2BE
  \l 2A2BF
  \l 2A2C0
  \l 2A2C1
  \l 2A2C2
  \l 2A2C3
  \l 2A2C4
  \l 2A2C5
  \l 2A2C6
  \l 2A2C7
  \l 2A2C8
  \l 2A2C9
  \l 2A2CA
  \l 2A2CB
  \l 2A2CC
  \l 2A2CD
  \l 2A2CE
  \l 2A2CF
  \l 2A2D0
  \l 2A2D1
  \l 2A2D2
  \l 2A2D3
  \l 2A2D4
  \l 2A2D5
  \l 2A2D6
  \l 2A2D7
  \l 2A2D8
  \l 2A2D9
  \l 2A2DA
  \l 2A2DB
  \l 2A2DC
  \l 2A2DD
  \l 2A2DE
  \l 2A2DF
  \l 2A2E0
  \l 2A2E1
  \l 2A2E2
  \l 2A2E3
  \l 2A2E4
  \l 2A2E5
  \l 2A2E6
  \l 2A2E7
  \l 2A2E8
  \l 2A2E9
  \l 2A2EA
  \l 2A2EB
  \l 2A2EC
  \l 2A2ED
  \l 2A2EE
  \l 2A2EF
  \l 2A2F0
  \l 2A2F1
  \l 2A2F2
  \l 2A2F3
  \l 2A2F4
  \l 2A2F5
  \l 2A2F6
  \l 2A2F7
  \l 2A2F8
  \l 2A2F9
  \l 2A2FA
  \l 2A2FB
  \l 2A2FC
  \l 2A2FD
  \l 2A2FE
  \l 2A2FF
  \l 2A300
  \l 2A301
  \l 2A302
  \l 2A303
  \l 2A304
  \l 2A305
  \l 2A306
  \l 2A307
  \l 2A308
  \l 2A309
  \l 2A30A
  \l 2A30B
  \l 2A30C
  \l 2A30D
  \l 2A30E
  \l 2A30F
  \l 2A310
  \l 2A311
  \l 2A312
  \l 2A313
  \l 2A314
  \l 2A315
  \l 2A316
  \l 2A317
  \l 2A318
  \l 2A319
  \l 2A31A
  \l 2A31B
  \l 2A31C
  \l 2A31D
  \l 2A31E
  \l 2A31F
  \l 2A320
  \l 2A321
  \l 2A322
  \l 2A323
  \l 2A324
  \l 2A325
  \l 2A326
  \l 2A327
  \l 2A328
  \l 2A329
  \l 2A32A
  \l 2A32B
  \l 2A32C
  \l 2A32D
  \l 2A32E
  \l 2A32F
  \l 2A330
  \l 2A331
  \l 2A332
  \l 2A333
  \l 2A334
  \l 2A335
  \l 2A336
  \l 2A337
  \l 2A338
  \l 2A339
  \l 2A33A
  \l 2A33B
  \l 2A33C
  \l 2A33D
  \l 2A33E
  \l 2A33F
  \l 2A340
  \l 2A341
  \l 2A342
  \l 2A343
  \l 2A344
  \l 2A345
  \l 2A346
  \l 2A347
  \l 2A348
  \l 2A349
  \l 2A34A
  \l 2A34B
  \l 2A34C
  \l 2A34D
  \l 2A34E
  \l 2A34F
  \l 2A350
  \l 2A351
  \l 2A352
  \l 2A353
  \l 2A354
  \l 2A355
  \l 2A356
  \l 2A357
  \l 2A358
  \l 2A359
  \l 2A35A
  \l 2A35B
  \l 2A35C
  \l 2A35D
  \l 2A35E
  \l 2A35F
  \l 2A360
  \l 2A361
  \l 2A362
  \l 2A363
  \l 2A364
  \l 2A365
  \l 2A366
  \l 2A367
  \l 2A368
  \l 2A369
  \l 2A36A
  \l 2A36B
  \l 2A36C
  \l 2A36D
  \l 2A36E
  \l 2A36F
  \l 2A370
  \l 2A371
  \l 2A372
  \l 2A373
  \l 2A374
  \l 2A375
  \l 2A376
  \l 2A377
  \l 2A378
  \l 2A379
  \l 2A37A
  \l 2A37B
  \l 2A37C
  \l 2A37D
  \l 2A37E
  \l 2A37F
  \l 2A380
  \l 2A381
  \l 2A382
  \l 2A383
  \l 2A384
  \l 2A385
  \l 2A386
  \l 2A387
  \l 2A388
  \l 2A389
  \l 2A38A
  \l 2A38B
  \l 2A38C
  \l 2A38D
  \l 2A38E
  \l 2A38F
  \l 2A390
  \l 2A391
  \l 2A392
  \l 2A393
  \l 2A394
  \l 2A395
  \l 2A396
  \l 2A397
  \l 2A398
  \l 2A399
  \l 2A39A
  \l 2A39B
  \l 2A39C
  \l 2A39D
  \l 2A39E
  \l 2A39F
  \l 2A3A0
  \l 2A3A1
  \l 2A3A2
  \l 2A3A3
  \l 2A3A4
  \l 2A3A5
  \l 2A3A6
  \l 2A3A7
  \l 2A3A8
  \l 2A3A9
  \l 2A3AA
  \l 2A3AB
  \l 2A3AC
  \l 2A3AD
  \l 2A3AE
  \l 2A3AF
  \l 2A3B0
  \l 2A3B1
  \l 2A3B2
  \l 2A3B3
  \l 2A3B4
  \l 2A3B5
  \l 2A3B6
  \l 2A3B7
  \l 2A3B8
  \l 2A3B9
  \l 2A3BA
  \l 2A3BB
  \l 2A3BC
  \l 2A3BD
  \l 2A3BE
  \l 2A3BF
  \l 2A3C0
  \l 2A3C1
  \l 2A3C2
  \l 2A3C3
  \l 2A3C4
  \l 2A3C5
  \l 2A3C6
  \l 2A3C7
  \l 2A3C8
  \l 2A3C9
  \l 2A3CA
  \l 2A3CB
  \l 2A3CC
  \l 2A3CD
  \l 2A3CE
  \l 2A3CF
  \l 2A3D0
  \l 2A3D1
  \l 2A3D2
  \l 2A3D3
  \l 2A3D4
  \l 2A3D5
  \l 2A3D6
  \l 2A3D7
  \l 2A3D8
  \l 2A3D9
  \l 2A3DA
  \l 2A3DB
  \l 2A3DC
  \l 2A3DD
  \l 2A3DE
  \l 2A3DF
  \l 2A3E0
  \l 2A3E1
  \l 2A3E2
  \l 2A3E3
  \l 2A3E4
  \l 2A3E5
  \l 2A3E6
  \l 2A3E7
  \l 2A3E8
  \l 2A3E9
  \l 2A3EA
  \l 2A3EB
  \l 2A3EC
  \l 2A3ED
  \l 2A3EE
  \l 2A3EF
  \l 2A3F0
  \l 2A3F1
  \l 2A3F2
  \l 2A3F3
  \l 2A3F4
  \l 2A3F5
  \l 2A3F6
  \l 2A3F7
  \l 2A3F8
  \l 2A3F9
  \l 2A3FA
  \l 2A3FB
  \l 2A3FC
  \l 2A3FD
  \l 2A3FE
  \l 2A3FF
  \l 2A400
  \l 2A401
  \l 2A402
  \l 2A403
  \l 2A404
  \l 2A405
  \l 2A406
  \l 2A407
  \l 2A408
  \l 2A409
  \l 2A40A
  \l 2A40B
  \l 2A40C
  \l 2A40D
  \l 2A40E
  \l 2A40F
  \l 2A410
  \l 2A411
  \l 2A412
  \l 2A413
  \l 2A414
  \l 2A415
  \l 2A416
  \l 2A417
  \l 2A418
  \l 2A419
  \l 2A41A
  \l 2A41B
  \l 2A41C
  \l 2A41D
  \l 2A41E
  \l 2A41F
  \l 2A420
  \l 2A421
  \l 2A422
  \l 2A423
  \l 2A424
  \l 2A425
  \l 2A426
  \l 2A427
  \l 2A428
  \l 2A429
  \l 2A42A
  \l 2A42B
  \l 2A42C
  \l 2A42D
  \l 2A42E
  \l 2A42F
  \l 2A430
  \l 2A431
  \l 2A432
  \l 2A433
  \l 2A434
  \l 2A435
  \l 2A436
  \l 2A437
  \l 2A438
  \l 2A439
  \l 2A43A
  \l 2A43B
  \l 2A43C
  \l 2A43D
  \l 2A43E
  \l 2A43F
  \l 2A440
  \l 2A441
  \l 2A442
  \l 2A443
  \l 2A444
  \l 2A445
  \l 2A446
  \l 2A447
  \l 2A448
  \l 2A449
  \l 2A44A
  \l 2A44B
  \l 2A44C
  \l 2A44D
  \l 2A44E
  \l 2A44F
  \l 2A450
  \l 2A451
  \l 2A452
  \l 2A453
  \l 2A454
  \l 2A455
  \l 2A456
  \l 2A457
  \l 2A458
  \l 2A459
  \l 2A45A
  \l 2A45B
  \l 2A45C
  \l 2A45D
  \l 2A45E
  \l 2A45F
  \l 2A460
  \l 2A461
  \l 2A462
  \l 2A463
  \l 2A464
  \l 2A465
  \l 2A466
  \l 2A467
  \l 2A468
  \l 2A469
  \l 2A46A
  \l 2A46B
  \l 2A46C
  \l 2A46D
  \l 2A46E
  \l 2A46F
  \l 2A470
  \l 2A471
  \l 2A472
  \l 2A473
  \l 2A474
  \l 2A475
  \l 2A476
  \l 2A477
  \l 2A478
  \l 2A479
  \l 2A47A
  \l 2A47B
  \l 2A47C
  \l 2A47D
  \l 2A47E
  \l 2A47F
  \l 2A480
  \l 2A481
  \l 2A482
  \l 2A483
  \l 2A484
  \l 2A485
  \l 2A486
  \l 2A487
  \l 2A488
  \l 2A489
  \l 2A48A
  \l 2A48B
  \l 2A48C
  \l 2A48D
  \l 2A48E
  \l 2A48F
  \l 2A490
  \l 2A491
  \l 2A492
  \l 2A493
  \l 2A494
  \l 2A495
  \l 2A496
  \l 2A497
  \l 2A498
  \l 2A499
  \l 2A49A
  \l 2A49B
  \l 2A49C
  \l 2A49D
  \l 2A49E
  \l 2A49F
  \l 2A4A0
  \l 2A4A1
  \l 2A4A2
  \l 2A4A3
  \l 2A4A4
  \l 2A4A5
  \l 2A4A6
  \l 2A4A7
  \l 2A4A8
  \l 2A4A9
  \l 2A4AA
  \l 2A4AB
  \l 2A4AC
  \l 2A4AD
  \l 2A4AE
  \l 2A4AF
  \l 2A4B0
  \l 2A4B1
  \l 2A4B2
  \l 2A4B3
  \l 2A4B4
  \l 2A4B5
  \l 2A4B6
  \l 2A4B7
  \l 2A4B8
  \l 2A4B9
  \l 2A4BA
  \l 2A4BB
  \l 2A4BC
  \l 2A4BD
  \l 2A4BE
  \l 2A4BF
  \l 2A4C0
  \l 2A4C1
  \l 2A4C2
  \l 2A4C3
  \l 2A4C4
  \l 2A4C5
  \l 2A4C6
  \l 2A4C7
  \l 2A4C8
  \l 2A4C9
  \l 2A4CA
  \l 2A4CB
  \l 2A4CC
  \l 2A4CD
  \l 2A4CE
  \l 2A4CF
  \l 2A4D0
  \l 2A4D1
  \l 2A4D2
  \l 2A4D3
  \l 2A4D4
  \l 2A4D5
  \l 2A4D6
  \l 2A4D7
  \l 2A4D8
  \l 2A4D9
  \l 2A4DA
  \l 2A4DB
  \l 2A4DC
  \l 2A4DD
  \l 2A4DE
  \l 2A4DF
  \l 2A4E0
  \l 2A4E1
  \l 2A4E2
  \l 2A4E3
  \l 2A4E4
  \l 2A4E5
  \l 2A4E6
  \l 2A4E7
  \l 2A4E8
  \l 2A4E9
  \l 2A4EA
  \l 2A4EB
  \l 2A4EC
  \l 2A4ED
  \l 2A4EE
  \l 2A4EF
  \l 2A4F0
  \l 2A4F1
  \l 2A4F2
  \l 2A4F3
  \l 2A4F4
  \l 2A4F5
  \l 2A4F6
  \l 2A4F7
  \l 2A4F8
  \l 2A4F9
  \l 2A4FA
  \l 2A4FB
  \l 2A4FC
  \l 2A4FD
  \l 2A4FE
  \l 2A4FF
  \l 2A500
  \l 2A501
  \l 2A502
  \l 2A503
  \l 2A504
  \l 2A505
  \l 2A506
  \l 2A507
  \l 2A508
  \l 2A509
  \l 2A50A
  \l 2A50B
  \l 2A50C
  \l 2A50D
  \l 2A50E
  \l 2A50F
  \l 2A510
  \l 2A511
  \l 2A512
  \l 2A513
  \l 2A514
  \l 2A515
  \l 2A516
  \l 2A517
  \l 2A518
  \l 2A519
  \l 2A51A
  \l 2A51B
  \l 2A51C
  \l 2A51D
  \l 2A51E
  \l 2A51F
  \l 2A520
  \l 2A521
  \l 2A522
  \l 2A523
  \l 2A524
  \l 2A525
  \l 2A526
  \l 2A527
  \l 2A528
  \l 2A529
  \l 2A52A
  \l 2A52B
  \l 2A52C
  \l 2A52D
  \l 2A52E
  \l 2A52F
  \l 2A530
  \l 2A531
  \l 2A532
  \l 2A533
  \l 2A534
  \l 2A535
  \l 2A536
  \l 2A537
  \l 2A538
  \l 2A539
  \l 2A53A
  \l 2A53B
  \l 2A53C
  \l 2A53D
  \l 2A53E
  \l 2A53F
  \l 2A540
  \l 2A541
  \l 2A542
  \l 2A543
  \l 2A544
  \l 2A545
  \l 2A546
  \l 2A547
  \l 2A548
  \l 2A549
  \l 2A54A
  \l 2A54B
  \l 2A54C
  \l 2A54D
  \l 2A54E
  \l 2A54F
  \l 2A550
  \l 2A551
  \l 2A552
  \l 2A553
  \l 2A554
  \l 2A555
  \l 2A556
  \l 2A557
  \l 2A558
  \l 2A559
  \l 2A55A
  \l 2A55B
  \l 2A55C
  \l 2A55D
  \l 2A55E
  \l 2A55F
  \l 2A560
  \l 2A561
  \l 2A562
  \l 2A563
  \l 2A564
  \l 2A565
  \l 2A566
  \l 2A567
  \l 2A568
  \l 2A569
  \l 2A56A
  \l 2A56B
  \l 2A56C
  \l 2A56D
  \l 2A56E
  \l 2A56F
  \l 2A570
  \l 2A571
  \l 2A572
  \l 2A573
  \l 2A574
  \l 2A575
  \l 2A576
  \l 2A577
  \l 2A578
  \l 2A579
  \l 2A57A
  \l 2A57B
  \l 2A57C
  \l 2A57D
  \l 2A57E
  \l 2A57F
  \l 2A580
  \l 2A581
  \l 2A582
  \l 2A583
  \l 2A584
  \l 2A585
  \l 2A586
  \l 2A587
  \l 2A588
  \l 2A589
  \l 2A58A
  \l 2A58B
  \l 2A58C
  \l 2A58D
  \l 2A58E
  \l 2A58F
  \l 2A590
  \l 2A591
  \l 2A592
  \l 2A593
  \l 2A594
  \l 2A595
  \l 2A596
  \l 2A597
  \l 2A598
  \l 2A599
  \l 2A59A
  \l 2A59B
  \l 2A59C
  \l 2A59D
  \l 2A59E
  \l 2A59F
  \l 2A5A0
  \l 2A5A1
  \l 2A5A2
  \l 2A5A3
  \l 2A5A4
  \l 2A5A5
  \l 2A5A6
  \l 2A5A7
  \l 2A5A8
  \l 2A5A9
  \l 2A5AA
  \l 2A5AB
  \l 2A5AC
  \l 2A5AD
  \l 2A5AE
  \l 2A5AF
  \l 2A5B0
  \l 2A5B1
  \l 2A5B2
  \l 2A5B3
  \l 2A5B4
  \l 2A5B5
  \l 2A5B6
  \l 2A5B7
  \l 2A5B8
  \l 2A5B9
  \l 2A5BA
  \l 2A5BB
  \l 2A5BC
  \l 2A5BD
  \l 2A5BE
  \l 2A5BF
  \l 2A5C0
  \l 2A5C1
  \l 2A5C2
  \l 2A5C3
  \l 2A5C4
  \l 2A5C5
  \l 2A5C6
  \l 2A5C7
  \l 2A5C8
  \l 2A5C9
  \l 2A5CA
  \l 2A5CB
  \l 2A5CC
  \l 2A5CD
  \l 2A5CE
  \l 2A5CF
  \l 2A5D0
  \l 2A5D1
  \l 2A5D2
  \l 2A5D3
  \l 2A5D4
  \l 2A5D5
  \l 2A5D6
  \l 2A5D7
  \l 2A5D8
  \l 2A5D9
  \l 2A5DA
  \l 2A5DB
  \l 2A5DC
  \l 2A5DD
  \l 2A5DE
  \l 2A5DF
  \l 2A5E0
  \l 2A5E1
  \l 2A5E2
  \l 2A5E3
  \l 2A5E4
  \l 2A5E5
  \l 2A5E6
  \l 2A5E7
  \l 2A5E8
  \l 2A5E9
  \l 2A5EA
  \l 2A5EB
  \l 2A5EC
  \l 2A5ED
  \l 2A5EE
  \l 2A5EF
  \l 2A5F0
  \l 2A5F1
  \l 2A5F2
  \l 2A5F3
  \l 2A5F4
  \l 2A5F5
  \l 2A5F6
  \l 2A5F7
  \l 2A5F8
  \l 2A5F9
  \l 2A5FA
  \l 2A5FB
  \l 2A5FC
  \l 2A5FD
  \l 2A5FE
  \l 2A5FF
  \l 2A600
  \l 2A601
  \l 2A602
  \l 2A603
  \l 2A604
  \l 2A605
  \l 2A606
  \l 2A607
  \l 2A608
  \l 2A609
  \l 2A60A
  \l 2A60B
  \l 2A60C
  \l 2A60D
  \l 2A60E
  \l 2A60F
  \l 2A610
  \l 2A611
  \l 2A612
  \l 2A613
  \l 2A614
  \l 2A615
  \l 2A616
  \l 2A617
  \l 2A618
  \l 2A619
  \l 2A61A
  \l 2A61B
  \l 2A61C
  \l 2A61D
  \l 2A61E
  \l 2A61F
  \l 2A620
  \l 2A621
  \l 2A622
  \l 2A623
  \l 2A624
  \l 2A625
  \l 2A626
  \l 2A627
  \l 2A628
  \l 2A629
  \l 2A62A
  \l 2A62B
  \l 2A62C
  \l 2A62D
  \l 2A62E
  \l 2A62F
  \l 2A630
  \l 2A631
  \l 2A632
  \l 2A633
  \l 2A634
  \l 2A635
  \l 2A636
  \l 2A637
  \l 2A638
  \l 2A639
  \l 2A63A
  \l 2A63B
  \l 2A63C
  \l 2A63D
  \l 2A63E
  \l 2A63F
  \l 2A640
  \l 2A641
  \l 2A642
  \l 2A643
  \l 2A644
  \l 2A645
  \l 2A646
  \l 2A647
  \l 2A648
  \l 2A649
  \l 2A64A
  \l 2A64B
  \l 2A64C
  \l 2A64D
  \l 2A64E
  \l 2A64F
  \l 2A650
  \l 2A651
  \l 2A652
  \l 2A653
  \l 2A654
  \l 2A655
  \l 2A656
  \l 2A657
  \l 2A658
  \l 2A659
  \l 2A65A
  \l 2A65B
  \l 2A65C
  \l 2A65D
  \l 2A65E
  \l 2A65F
  \l 2A660
  \l 2A661
  \l 2A662
  \l 2A663
  \l 2A664
  \l 2A665
  \l 2A666
  \l 2A667
  \l 2A668
  \l 2A669
  \l 2A66A
  \l 2A66B
  \l 2A66C
  \l 2A66D
  \l 2A66E
  \l 2A66F
  \l 2A670
  \l 2A671
  \l 2A672
  \l 2A673
  \l 2A674
  \l 2A675
  \l 2A676
  \l 2A677
  \l 2A678
  \l 2A679
  \l 2A67A
  \l 2A67B
  \l 2A67C
  \l 2A67D
  \l 2A67E
  \l 2A67F
  \l 2A680
  \l 2A681
  \l 2A682
  \l 2A683
  \l 2A684
  \l 2A685
  \l 2A686
  \l 2A687
  \l 2A688
  \l 2A689
  \l 2A68A
  \l 2A68B
  \l 2A68C
  \l 2A68D
  \l 2A68E
  \l 2A68F
  \l 2A690
  \l 2A691
  \l 2A692
  \l 2A693
  \l 2A694
  \l 2A695
  \l 2A696
  \l 2A697
  \l 2A698
  \l 2A699
  \l 2A69A
  \l 2A69B
  \l 2A69C
  \l 2A69D
  \l 2A69E
  \l 2A69F
  \l 2A6A0
  \l 2A6A1
  \l 2A6A2
  \l 2A6A3
  \l 2A6A4
  \l 2A6A5
  \l 2A6A6
  \l 2A6A7
  \l 2A6A8
  \l 2A6A9
  \l 2A6AA
  \l 2A6AB
  \l 2A6AC
  \l 2A6AD
  \l 2A6AE
  \l 2A6AF
  \l 2A6B0
  \l 2A6B1
  \l 2A6B2
  \l 2A6B3
  \l 2A6B4
  \l 2A6B5
  \l 2A6B6
  \l 2A6B7
  \l 2A6B8
  \l 2A6B9
  \l 2A6BA
  \l 2A6BB
  \l 2A6BC
  \l 2A6BD
  \l 2A6BE
  \l 2A6BF
  \l 2A6C0
  \l 2A6C1
  \l 2A6C2
  \l 2A6C3
  \l 2A6C4
  \l 2A6C5
  \l 2A6C6
  \l 2A6C7
  \l 2A6C8
  \l 2A6C9
  \l 2A6CA
  \l 2A6CB
  \l 2A6CC
  \l 2A6CD
  \l 2A6CE
  \l 2A6CF
  \l 2A6D0
  \l 2A6D1
  \l 2A6D2
  \l 2A6D3
  \l 2A6D4
  \l 2A6D5
  \l 2A6D6
  \l 2A700
  \l 2A701
  \l 2A702
  \l 2A703
  \l 2A704
  \l 2A705
  \l 2A706
  \l 2A707
  \l 2A708
  \l 2A709
  \l 2A70A
  \l 2A70B
  \l 2A70C
  \l 2A70D
  \l 2A70E
  \l 2A70F
  \l 2A710
  \l 2A711
  \l 2A712
  \l 2A713
  \l 2A714
  \l 2A715
  \l 2A716
  \l 2A717
  \l 2A718
  \l 2A719
  \l 2A71A
  \l 2A71B
  \l 2A71C
  \l 2A71D
  \l 2A71E
  \l 2A71F
  \l 2A720
  \l 2A721
  \l 2A722
  \l 2A723
  \l 2A724
  \l 2A725
  \l 2A726
  \l 2A727
  \l 2A728
  \l 2A729
  \l 2A72A
  \l 2A72B
  \l 2A72C
  \l 2A72D
  \l 2A72E
  \l 2A72F
  \l 2A730
  \l 2A731
  \l 2A732
  \l 2A733
  \l 2A734
  \l 2A735
  \l 2A736
  \l 2A737
  \l 2A738
  \l 2A739
  \l 2A73A
  \l 2A73B
  \l 2A73C
  \l 2A73D
  \l 2A73E
  \l 2A73F
  \l 2A740
  \l 2A741
  \l 2A742
  \l 2A743
  \l 2A744
  \l 2A745
  \l 2A746
  \l 2A747
  \l 2A748
  \l 2A749
  \l 2A74A
  \l 2A74B
  \l 2A74C
  \l 2A74D
  \l 2A74E
  \l 2A74F
  \l 2A750
  \l 2A751
  \l 2A752
  \l 2A753
  \l 2A754
  \l 2A755
  \l 2A756
  \l 2A757
  \l 2A758
  \l 2A759
  \l 2A75A
  \l 2A75B
  \l 2A75C
  \l 2A75D
  \l 2A75E
  \l 2A75F
  \l 2A760
  \l 2A761
  \l 2A762
  \l 2A763
  \l 2A764
  \l 2A765
  \l 2A766
  \l 2A767
  \l 2A768
  \l 2A769
  \l 2A76A
  \l 2A76B
  \l 2A76C
  \l 2A76D
  \l 2A76E
  \l 2A76F
  \l 2A770
  \l 2A771
  \l 2A772
  \l 2A773
  \l 2A774
  \l 2A775
  \l 2A776
  \l 2A777
  \l 2A778
  \l 2A779
  \l 2A77A
  \l 2A77B
  \l 2A77C
  \l 2A77D
  \l 2A77E
  \l 2A77F
  \l 2A780
  \l 2A781
  \l 2A782
  \l 2A783
  \l 2A784
  \l 2A785
  \l 2A786
  \l 2A787
  \l 2A788
  \l 2A789
  \l 2A78A
  \l 2A78B
  \l 2A78C
  \l 2A78D
  \l 2A78E
  \l 2A78F
  \l 2A790
  \l 2A791
  \l 2A792
  \l 2A793
  \l 2A794
  \l 2A795
  \l 2A796
  \l 2A797
  \l 2A798
  \l 2A799
  \l 2A79A
  \l 2A79B
  \l 2A79C
  \l 2A79D
  \l 2A79E
  \l 2A79F
  \l 2A7A0
  \l 2A7A1
  \l 2A7A2
  \l 2A7A3
  \l 2A7A4
  \l 2A7A5
  \l 2A7A6
  \l 2A7A7
  \l 2A7A8
  \l 2A7A9
  \l 2A7AA
  \l 2A7AB
  \l 2A7AC
  \l 2A7AD
  \l 2A7AE
  \l 2A7AF
  \l 2A7B0
  \l 2A7B1
  \l 2A7B2
  \l 2A7B3
  \l 2A7B4
  \l 2A7B5
  \l 2A7B6
  \l 2A7B7
  \l 2A7B8
  \l 2A7B9
  \l 2A7BA
  \l 2A7BB
  \l 2A7BC
  \l 2A7BD
  \l 2A7BE
  \l 2A7BF
  \l 2A7C0
  \l 2A7C1
  \l 2A7C2
  \l 2A7C3
  \l 2A7C4
  \l 2A7C5
  \l 2A7C6
  \l 2A7C7
  \l 2A7C8
  \l 2A7C9
  \l 2A7CA
  \l 2A7CB
  \l 2A7CC
  \l 2A7CD
  \l 2A7CE
  \l 2A7CF
  \l 2A7D0
  \l 2A7D1
  \l 2A7D2
  \l 2A7D3
  \l 2A7D4
  \l 2A7D5
  \l 2A7D6
  \l 2A7D7
  \l 2A7D8
  \l 2A7D9
  \l 2A7DA
  \l 2A7DB
  \l 2A7DC
  \l 2A7DD
  \l 2A7DE
  \l 2A7DF
  \l 2A7E0
  \l 2A7E1
  \l 2A7E2
  \l 2A7E3
  \l 2A7E4
  \l 2A7E5
  \l 2A7E6
  \l 2A7E7
  \l 2A7E8
  \l 2A7E9
  \l 2A7EA
  \l 2A7EB
  \l 2A7EC
  \l 2A7ED
  \l 2A7EE
  \l 2A7EF
  \l 2A7F0
  \l 2A7F1
  \l 2A7F2
  \l 2A7F3
  \l 2A7F4
  \l 2A7F5
  \l 2A7F6
  \l 2A7F7
  \l 2A7F8
  \l 2A7F9
  \l 2A7FA
  \l 2A7FB
  \l 2A7FC
  \l 2A7FD
  \l 2A7FE
  \l 2A7FF
  \l 2A800
  \l 2A801
  \l 2A802
  \l 2A803
  \l 2A804
  \l 2A805
  \l 2A806
  \l 2A807
  \l 2A808
  \l 2A809
  \l 2A80A
  \l 2A80B
  \l 2A80C
  \l 2A80D
  \l 2A80E
  \l 2A80F
  \l 2A810
  \l 2A811
  \l 2A812
  \l 2A813
  \l 2A814
  \l 2A815
  \l 2A816
  \l 2A817
  \l 2A818
  \l 2A819
  \l 2A81A
  \l 2A81B
  \l 2A81C
  \l 2A81D
  \l 2A81E
  \l 2A81F
  \l 2A820
  \l 2A821
  \l 2A822
  \l 2A823
  \l 2A824
  \l 2A825
  \l 2A826
  \l 2A827
  \l 2A828
  \l 2A829
  \l 2A82A
  \l 2A82B
  \l 2A82C
  \l 2A82D
  \l 2A82E
  \l 2A82F
  \l 2A830
  \l 2A831
  \l 2A832
  \l 2A833
  \l 2A834
  \l 2A835
  \l 2A836
  \l 2A837
  \l 2A838
  \l 2A839
  \l 2A83A
  \l 2A83B
  \l 2A83C
  \l 2A83D
  \l 2A83E
  \l 2A83F
  \l 2A840
  \l 2A841
  \l 2A842
  \l 2A843
  \l 2A844
  \l 2A845
  \l 2A846
  \l 2A847
  \l 2A848
  \l 2A849
  \l 2A84A
  \l 2A84B
  \l 2A84C
  \l 2A84D
  \l 2A84E
  \l 2A84F
  \l 2A850
  \l 2A851
  \l 2A852
  \l 2A853
  \l 2A854
  \l 2A855
  \l 2A856
  \l 2A857
  \l 2A858
  \l 2A859
  \l 2A85A
  \l 2A85B
  \l 2A85C
  \l 2A85D
  \l 2A85E
  \l 2A85F
  \l 2A860
  \l 2A861
  \l 2A862
  \l 2A863
  \l 2A864
  \l 2A865
  \l 2A866
  \l 2A867
  \l 2A868
  \l 2A869
  \l 2A86A
  \l 2A86B
  \l 2A86C
  \l 2A86D
  \l 2A86E
  \l 2A86F
  \l 2A870
  \l 2A871
  \l 2A872
  \l 2A873
  \l 2A874
  \l 2A875
  \l 2A876
  \l 2A877
  \l 2A878
  \l 2A879
  \l 2A87A
  \l 2A87B
  \l 2A87C
  \l 2A87D
  \l 2A87E
  \l 2A87F
  \l 2A880
  \l 2A881
  \l 2A882
  \l 2A883
  \l 2A884
  \l 2A885
  \l 2A886
  \l 2A887
  \l 2A888
  \l 2A889
  \l 2A88A
  \l 2A88B
  \l 2A88C
  \l 2A88D
  \l 2A88E
  \l 2A88F
  \l 2A890
  \l 2A891
  \l 2A892
  \l 2A893
  \l 2A894
  \l 2A895
  \l 2A896
  \l 2A897
  \l 2A898
  \l 2A899
  \l 2A89A
  \l 2A89B
  \l 2A89C
  \l 2A89D
  \l 2A89E
  \l 2A89F
  \l 2A8A0
  \l 2A8A1
  \l 2A8A2
  \l 2A8A3
  \l 2A8A4
  \l 2A8A5
  \l 2A8A6
  \l 2A8A7
  \l 2A8A8
  \l 2A8A9
  \l 2A8AA
  \l 2A8AB
  \l 2A8AC
  \l 2A8AD
  \l 2A8AE
  \l 2A8AF
  \l 2A8B0
  \l 2A8B1
  \l 2A8B2
  \l 2A8B3
  \l 2A8B4
  \l 2A8B5
  \l 2A8B6
  \l 2A8B7
  \l 2A8B8
  \l 2A8B9
  \l 2A8BA
  \l 2A8BB
  \l 2A8BC
  \l 2A8BD
  \l 2A8BE
  \l 2A8BF
  \l 2A8C0
  \l 2A8C1
  \l 2A8C2
  \l 2A8C3
  \l 2A8C4
  \l 2A8C5
  \l 2A8C6
  \l 2A8C7
  \l 2A8C8
  \l 2A8C9
  \l 2A8CA
  \l 2A8CB
  \l 2A8CC
  \l 2A8CD
  \l 2A8CE
  \l 2A8CF
  \l 2A8D0
  \l 2A8D1
  \l 2A8D2
  \l 2A8D3
  \l 2A8D4
  \l 2A8D5
  \l 2A8D6
  \l 2A8D7
  \l 2A8D8
  \l 2A8D9
  \l 2A8DA
  \l 2A8DB
  \l 2A8DC
  \l 2A8DD
  \l 2A8DE
  \l 2A8DF
  \l 2A8E0
  \l 2A8E1
  \l 2A8E2
  \l 2A8E3
  \l 2A8E4
  \l 2A8E5
  \l 2A8E6
  \l 2A8E7
  \l 2A8E8
  \l 2A8E9
  \l 2A8EA
  \l 2A8EB
  \l 2A8EC
  \l 2A8ED
  \l 2A8EE
  \l 2A8EF
  \l 2A8F0
  \l 2A8F1
  \l 2A8F2
  \l 2A8F3
  \l 2A8F4
  \l 2A8F5
  \l 2A8F6
  \l 2A8F7
  \l 2A8F8
  \l 2A8F9
  \l 2A8FA
  \l 2A8FB
  \l 2A8FC
  \l 2A8FD
  \l 2A8FE
  \l 2A8FF
  \l 2A900
  \l 2A901
  \l 2A902
  \l 2A903
  \l 2A904
  \l 2A905
  \l 2A906
  \l 2A907
  \l 2A908
  \l 2A909
  \l 2A90A
  \l 2A90B
  \l 2A90C
  \l 2A90D
  \l 2A90E
  \l 2A90F
  \l 2A910
  \l 2A911
  \l 2A912
  \l 2A913
  \l 2A914
  \l 2A915
  \l 2A916
  \l 2A917
  \l 2A918
  \l 2A919
  \l 2A91A
  \l 2A91B
  \l 2A91C
  \l 2A91D
  \l 2A91E
  \l 2A91F
  \l 2A920
  \l 2A921
  \l 2A922
  \l 2A923
  \l 2A924
  \l 2A925
  \l 2A926
  \l 2A927
  \l 2A928
  \l 2A929
  \l 2A92A
  \l 2A92B
  \l 2A92C
  \l 2A92D
  \l 2A92E
  \l 2A92F
  \l 2A930
  \l 2A931
  \l 2A932
  \l 2A933
  \l 2A934
  \l 2A935
  \l 2A936
  \l 2A937
  \l 2A938
  \l 2A939
  \l 2A93A
  \l 2A93B
  \l 2A93C
  \l 2A93D
  \l 2A93E
  \l 2A93F
  \l 2A940
  \l 2A941
  \l 2A942
  \l 2A943
  \l 2A944
  \l 2A945
  \l 2A946
  \l 2A947
  \l 2A948
  \l 2A949
  \l 2A94A
  \l 2A94B
  \l 2A94C
  \l 2A94D
  \l 2A94E
  \l 2A94F
  \l 2A950
  \l 2A951
  \l 2A952
  \l 2A953
  \l 2A954
  \l 2A955
  \l 2A956
  \l 2A957
  \l 2A958
  \l 2A959
  \l 2A95A
  \l 2A95B
  \l 2A95C
  \l 2A95D
  \l 2A95E
  \l 2A95F
  \l 2A960
  \l 2A961
  \l 2A962
  \l 2A963
  \l 2A964
  \l 2A965
  \l 2A966
  \l 2A967
  \l 2A968
  \l 2A969
  \l 2A96A
  \l 2A96B
  \l 2A96C
  \l 2A96D
  \l 2A96E
  \l 2A96F
  \l 2A970
  \l 2A971
  \l 2A972
  \l 2A973
  \l 2A974
  \l 2A975
  \l 2A976
  \l 2A977
  \l 2A978
  \l 2A979
  \l 2A97A
  \l 2A97B
  \l 2A97C
  \l 2A97D
  \l 2A97E
  \l 2A97F
  \l 2A980
  \l 2A981
  \l 2A982
  \l 2A983
  \l 2A984
  \l 2A985
  \l 2A986
  \l 2A987
  \l 2A988
  \l 2A989
  \l 2A98A
  \l 2A98B
  \l 2A98C
  \l 2A98D
  \l 2A98E
  \l 2A98F
  \l 2A990
  \l 2A991
  \l 2A992
  \l 2A993
  \l 2A994
  \l 2A995
  \l 2A996
  \l 2A997
  \l 2A998
  \l 2A999
  \l 2A99A
  \l 2A99B
  \l 2A99C
  \l 2A99D
  \l 2A99E
  \l 2A99F
  \l 2A9A0
  \l 2A9A1
  \l 2A9A2
  \l 2A9A3
  \l 2A9A4
  \l 2A9A5
  \l 2A9A6
  \l 2A9A7
  \l 2A9A8
  \l 2A9A9
  \l 2A9AA
  \l 2A9AB
  \l 2A9AC
  \l 2A9AD
  \l 2A9AE
  \l 2A9AF
  \l 2A9B0
  \l 2A9B1
  \l 2A9B2
  \l 2A9B3
  \l 2A9B4
  \l 2A9B5
  \l 2A9B6
  \l 2A9B7
  \l 2A9B8
  \l 2A9B9
  \l 2A9BA
  \l 2A9BB
  \l 2A9BC
  \l 2A9BD
  \l 2A9BE
  \l 2A9BF
  \l 2A9C0
  \l 2A9C1
  \l 2A9C2
  \l 2A9C3
  \l 2A9C4
  \l 2A9C5
  \l 2A9C6
  \l 2A9C7
  \l 2A9C8
  \l 2A9C9
  \l 2A9CA
  \l 2A9CB
  \l 2A9CC
  \l 2A9CD
  \l 2A9CE
  \l 2A9CF
  \l 2A9D0
  \l 2A9D1
  \l 2A9D2
  \l 2A9D3
  \l 2A9D4
  \l 2A9D5
  \l 2A9D6
  \l 2A9D7
  \l 2A9D8
  \l 2A9D9
  \l 2A9DA
  \l 2A9DB
  \l 2A9DC
  \l 2A9DD
  \l 2A9DE
  \l 2A9DF
  \l 2A9E0
  \l 2A9E1
  \l 2A9E2
  \l 2A9E3
  \l 2A9E4
  \l 2A9E5
  \l 2A9E6
  \l 2A9E7
  \l 2A9E8
  \l 2A9E9
  \l 2A9EA
  \l 2A9EB
  \l 2A9EC
  \l 2A9ED
  \l 2A9EE
  \l 2A9EF
  \l 2A9F0
  \l 2A9F1
  \l 2A9F2
  \l 2A9F3
  \l 2A9F4
  \l 2A9F5
  \l 2A9F6
  \l 2A9F7
  \l 2A9F8
  \l 2A9F9
  \l 2A9FA
  \l 2A9FB
  \l 2A9FC
  \l 2A9FD
  \l 2A9FE
  \l 2A9FF
  \l 2AA00
  \l 2AA01
  \l 2AA02
  \l 2AA03
  \l 2AA04
  \l 2AA05
  \l 2AA06
  \l 2AA07
  \l 2AA08
  \l 2AA09
  \l 2AA0A
  \l 2AA0B
  \l 2AA0C
  \l 2AA0D
  \l 2AA0E
  \l 2AA0F
  \l 2AA10
  \l 2AA11
  \l 2AA12
  \l 2AA13
  \l 2AA14
  \l 2AA15
  \l 2AA16
  \l 2AA17
  \l 2AA18
  \l 2AA19
  \l 2AA1A
  \l 2AA1B
  \l 2AA1C
  \l 2AA1D
  \l 2AA1E
  \l 2AA1F
  \l 2AA20
  \l 2AA21
  \l 2AA22
  \l 2AA23
  \l 2AA24
  \l 2AA25
  \l 2AA26
  \l 2AA27
  \l 2AA28
  \l 2AA29
  \l 2AA2A
  \l 2AA2B
  \l 2AA2C
  \l 2AA2D
  \l 2AA2E
  \l 2AA2F
  \l 2AA30
  \l 2AA31
  \l 2AA32
  \l 2AA33
  \l 2AA34
  \l 2AA35
  \l 2AA36
  \l 2AA37
  \l 2AA38
  \l 2AA39
  \l 2AA3A
  \l 2AA3B
  \l 2AA3C
  \l 2AA3D
  \l 2AA3E
  \l 2AA3F
  \l 2AA40
  \l 2AA41
  \l 2AA42
  \l 2AA43
  \l 2AA44
  \l 2AA45
  \l 2AA46
  \l 2AA47
  \l 2AA48
  \l 2AA49
  \l 2AA4A
  \l 2AA4B
  \l 2AA4C
  \l 2AA4D
  \l 2AA4E
  \l 2AA4F
  \l 2AA50
  \l 2AA51
  \l 2AA52
  \l 2AA53
  \l 2AA54
  \l 2AA55
  \l 2AA56
  \l 2AA57
  \l 2AA58
  \l 2AA59
  \l 2AA5A
  \l 2AA5B
  \l 2AA5C
  \l 2AA5D
  \l 2AA5E
  \l 2AA5F
  \l 2AA60
  \l 2AA61
  \l 2AA62
  \l 2AA63
  \l 2AA64
  \l 2AA65
  \l 2AA66
  \l 2AA67
  \l 2AA68
  \l 2AA69
  \l 2AA6A
  \l 2AA6B
  \l 2AA6C
  \l 2AA6D
  \l 2AA6E
  \l 2AA6F
  \l 2AA70
  \l 2AA71
  \l 2AA72
  \l 2AA73
  \l 2AA74
  \l 2AA75
  \l 2AA76
  \l 2AA77
  \l 2AA78
  \l 2AA79
  \l 2AA7A
  \l 2AA7B
  \l 2AA7C
  \l 2AA7D
  \l 2AA7E
  \l 2AA7F
  \l 2AA80
  \l 2AA81
  \l 2AA82
  \l 2AA83
  \l 2AA84
  \l 2AA85
  \l 2AA86
  \l 2AA87
  \l 2AA88
  \l 2AA89
  \l 2AA8A
  \l 2AA8B
  \l 2AA8C
  \l 2AA8D
  \l 2AA8E
  \l 2AA8F
  \l 2AA90
  \l 2AA91
  \l 2AA92
  \l 2AA93
  \l 2AA94
  \l 2AA95
  \l 2AA96
  \l 2AA97
  \l 2AA98
  \l 2AA99
  \l 2AA9A
  \l 2AA9B
  \l 2AA9C
  \l 2AA9D
  \l 2AA9E
  \l 2AA9F
  \l 2AAA0
  \l 2AAA1
  \l 2AAA2
  \l 2AAA3
  \l 2AAA4
  \l 2AAA5
  \l 2AAA6
  \l 2AAA7
  \l 2AAA8
  \l 2AAA9
  \l 2AAAA
  \l 2AAAB
  \l 2AAAC
  \l 2AAAD
  \l 2AAAE
  \l 2AAAF
  \l 2AAB0
  \l 2AAB1
  \l 2AAB2
  \l 2AAB3
  \l 2AAB4
  \l 2AAB5
  \l 2AAB6
  \l 2AAB7
  \l 2AAB8
  \l 2AAB9
  \l 2AABA
  \l 2AABB
  \l 2AABC
  \l 2AABD
  \l 2AABE
  \l 2AABF
  \l 2AAC0
  \l 2AAC1
  \l 2AAC2
  \l 2AAC3
  \l 2AAC4
  \l 2AAC5
  \l 2AAC6
  \l 2AAC7
  \l 2AAC8
  \l 2AAC9
  \l 2AACA
  \l 2AACB
  \l 2AACC
  \l 2AACD
  \l 2AACE
  \l 2AACF
  \l 2AAD0
  \l 2AAD1
  \l 2AAD2
  \l 2AAD3
  \l 2AAD4
  \l 2AAD5
  \l 2AAD6
  \l 2AAD7
  \l 2AAD8
  \l 2AAD9
  \l 2AADA
  \l 2AADB
  \l 2AADC
  \l 2AADD
  \l 2AADE
  \l 2AADF
  \l 2AAE0
  \l 2AAE1
  \l 2AAE2
  \l 2AAE3
  \l 2AAE4
  \l 2AAE5
  \l 2AAE6
  \l 2AAE7
  \l 2AAE8
  \l 2AAE9
  \l 2AAEA
  \l 2AAEB
  \l 2AAEC
  \l 2AAED
  \l 2AAEE
  \l 2AAEF
  \l 2AAF0
  \l 2AAF1
  \l 2AAF2
  \l 2AAF3
  \l 2AAF4
  \l 2AAF5
  \l 2AAF6
  \l 2AAF7
  \l 2AAF8
  \l 2AAF9
  \l 2AAFA
  \l 2AAFB
  \l 2AAFC
  \l 2AAFD
  \l 2AAFE
  \l 2AAFF
  \l 2AB00
  \l 2AB01
  \l 2AB02
  \l 2AB03
  \l 2AB04
  \l 2AB05
  \l 2AB06
  \l 2AB07
  \l 2AB08
  \l 2AB09
  \l 2AB0A
  \l 2AB0B
  \l 2AB0C
  \l 2AB0D
  \l 2AB0E
  \l 2AB0F
  \l 2AB10
  \l 2AB11
  \l 2AB12
  \l 2AB13
  \l 2AB14
  \l 2AB15
  \l 2AB16
  \l 2AB17
  \l 2AB18
  \l 2AB19
  \l 2AB1A
  \l 2AB1B
  \l 2AB1C
  \l 2AB1D
  \l 2AB1E
  \l 2AB1F
  \l 2AB20
  \l 2AB21
  \l 2AB22
  \l 2AB23
  \l 2AB24
  \l 2AB25
  \l 2AB26
  \l 2AB27
  \l 2AB28
  \l 2AB29
  \l 2AB2A
  \l 2AB2B
  \l 2AB2C
  \l 2AB2D
  \l 2AB2E
  \l 2AB2F
  \l 2AB30
  \l 2AB31
  \l 2AB32
  \l 2AB33
  \l 2AB34
  \l 2AB35
  \l 2AB36
  \l 2AB37
  \l 2AB38
  \l 2AB39
  \l 2AB3A
  \l 2AB3B
  \l 2AB3C
  \l 2AB3D
  \l 2AB3E
  \l 2AB3F
  \l 2AB40
  \l 2AB41
  \l 2AB42
  \l 2AB43
  \l 2AB44
  \l 2AB45
  \l 2AB46
  \l 2AB47
  \l 2AB48
  \l 2AB49
  \l 2AB4A
  \l 2AB4B
  \l 2AB4C
  \l 2AB4D
  \l 2AB4E
  \l 2AB4F
  \l 2AB50
  \l 2AB51
  \l 2AB52
  \l 2AB53
  \l 2AB54
  \l 2AB55
  \l 2AB56
  \l 2AB57
  \l 2AB58
  \l 2AB59
  \l 2AB5A
  \l 2AB5B
  \l 2AB5C
  \l 2AB5D
  \l 2AB5E
  \l 2AB5F
  \l 2AB60
  \l 2AB61
  \l 2AB62
  \l 2AB63
  \l 2AB64
  \l 2AB65
  \l 2AB66
  \l 2AB67
  \l 2AB68
  \l 2AB69
  \l 2AB6A
  \l 2AB6B
  \l 2AB6C
  \l 2AB6D
  \l 2AB6E
  \l 2AB6F
  \l 2AB70
  \l 2AB71
  \l 2AB72
  \l 2AB73
  \l 2AB74
  \l 2AB75
  \l 2AB76
  \l 2AB77
  \l 2AB78
  \l 2AB79
  \l 2AB7A
  \l 2AB7B
  \l 2AB7C
  \l 2AB7D
  \l 2AB7E
  \l 2AB7F
  \l 2AB80
  \l 2AB81
  \l 2AB82
  \l 2AB83
  \l 2AB84
  \l 2AB85
  \l 2AB86
  \l 2AB87
  \l 2AB88
  \l 2AB89
  \l 2AB8A
  \l 2AB8B
  \l 2AB8C
  \l 2AB8D
  \l 2AB8E
  \l 2AB8F
  \l 2AB90
  \l 2AB91
  \l 2AB92
  \l 2AB93
  \l 2AB94
  \l 2AB95
  \l 2AB96
  \l 2AB97
  \l 2AB98
  \l 2AB99
  \l 2AB9A
  \l 2AB9B
  \l 2AB9C
  \l 2AB9D
  \l 2AB9E
  \l 2AB9F
  \l 2ABA0
  \l 2ABA1
  \l 2ABA2
  \l 2ABA3
  \l 2ABA4
  \l 2ABA5
  \l 2ABA6
  \l 2ABA7
  \l 2ABA8
  \l 2ABA9
  \l 2ABAA
  \l 2ABAB
  \l 2ABAC
  \l 2ABAD
  \l 2ABAE
  \l 2ABAF
  \l 2ABB0
  \l 2ABB1
  \l 2ABB2
  \l 2ABB3
  \l 2ABB4
  \l 2ABB5
  \l 2ABB6
  \l 2ABB7
  \l 2ABB8
  \l 2ABB9
  \l 2ABBA
  \l 2ABBB
  \l 2ABBC
  \l 2ABBD
  \l 2ABBE
  \l 2ABBF
  \l 2ABC0
  \l 2ABC1
  \l 2ABC2
  \l 2ABC3
  \l 2ABC4
  \l 2ABC5
  \l 2ABC6
  \l 2ABC7
  \l 2ABC8
  \l 2ABC9
  \l 2ABCA
  \l 2ABCB
  \l 2ABCC
  \l 2ABCD
  \l 2ABCE
  \l 2ABCF
  \l 2ABD0
  \l 2ABD1
  \l 2ABD2
  \l 2ABD3
  \l 2ABD4
  \l 2ABD5
  \l 2ABD6
  \l 2ABD7
  \l 2ABD8
  \l 2ABD9
  \l 2ABDA
  \l 2ABDB
  \l 2ABDC
  \l 2ABDD
  \l 2ABDE
  \l 2ABDF
  \l 2ABE0
  \l 2ABE1
  \l 2ABE2
  \l 2ABE3
  \l 2ABE4
  \l 2ABE5
  \l 2ABE6
  \l 2ABE7
  \l 2ABE8
  \l 2ABE9
  \l 2ABEA
  \l 2ABEB
  \l 2ABEC
  \l 2ABED
  \l 2ABEE
  \l 2ABEF
  \l 2ABF0
  \l 2ABF1
  \l 2ABF2
  \l 2ABF3
  \l 2ABF4
  \l 2ABF5
  \l 2ABF6
  \l 2ABF7
  \l 2ABF8
  \l 2ABF9
  \l 2ABFA
  \l 2ABFB
  \l 2ABFC
  \l 2ABFD
  \l 2ABFE
  \l 2ABFF
  \l 2AC00
  \l 2AC01
  \l 2AC02
  \l 2AC03
  \l 2AC04
  \l 2AC05
  \l 2AC06
  \l 2AC07
  \l 2AC08
  \l 2AC09
  \l 2AC0A
  \l 2AC0B
  \l 2AC0C
  \l 2AC0D
  \l 2AC0E
  \l 2AC0F
  \l 2AC10
  \l 2AC11
  \l 2AC12
  \l 2AC13
  \l 2AC14
  \l 2AC15
  \l 2AC16
  \l 2AC17
  \l 2AC18
  \l 2AC19
  \l 2AC1A
  \l 2AC1B
  \l 2AC1C
  \l 2AC1D
  \l 2AC1E
  \l 2AC1F
  \l 2AC20
  \l 2AC21
  \l 2AC22
  \l 2AC23
  \l 2AC24
  \l 2AC25
  \l 2AC26
  \l 2AC27
  \l 2AC28
  \l 2AC29
  \l 2AC2A
  \l 2AC2B
  \l 2AC2C
  \l 2AC2D
  \l 2AC2E
  \l 2AC2F
  \l 2AC30
  \l 2AC31
  \l 2AC32
  \l 2AC33
  \l 2AC34
  \l 2AC35
  \l 2AC36
  \l 2AC37
  \l 2AC38
  \l 2AC39
  \l 2AC3A
  \l 2AC3B
  \l 2AC3C
  \l 2AC3D
  \l 2AC3E
  \l 2AC3F
  \l 2AC40
  \l 2AC41
  \l 2AC42
  \l 2AC43
  \l 2AC44
  \l 2AC45
  \l 2AC46
  \l 2AC47
  \l 2AC48
  \l 2AC49
  \l 2AC4A
  \l 2AC4B
  \l 2AC4C
  \l 2AC4D
  \l 2AC4E
  \l 2AC4F
  \l 2AC50
  \l 2AC51
  \l 2AC52
  \l 2AC53
  \l 2AC54
  \l 2AC55
  \l 2AC56
  \l 2AC57
  \l 2AC58
  \l 2AC59
  \l 2AC5A
  \l 2AC5B
  \l 2AC5C
  \l 2AC5D
  \l 2AC5E
  \l 2AC5F
  \l 2AC60
  \l 2AC61
  \l 2AC62
  \l 2AC63
  \l 2AC64
  \l 2AC65
  \l 2AC66
  \l 2AC67
  \l 2AC68
  \l 2AC69
  \l 2AC6A
  \l 2AC6B
  \l 2AC6C
  \l 2AC6D
  \l 2AC6E
  \l 2AC6F
  \l 2AC70
  \l 2AC71
  \l 2AC72
  \l 2AC73
  \l 2AC74
  \l 2AC75
  \l 2AC76
  \l 2AC77
  \l 2AC78
  \l 2AC79
  \l 2AC7A
  \l 2AC7B
  \l 2AC7C
  \l 2AC7D
  \l 2AC7E
  \l 2AC7F
  \l 2AC80
  \l 2AC81
  \l 2AC82
  \l 2AC83
  \l 2AC84
  \l 2AC85
  \l 2AC86
  \l 2AC87
  \l 2AC88
  \l 2AC89
  \l 2AC8A
  \l 2AC8B
  \l 2AC8C
  \l 2AC8D
  \l 2AC8E
  \l 2AC8F
  \l 2AC90
  \l 2AC91
  \l 2AC92
  \l 2AC93
  \l 2AC94
  \l 2AC95
  \l 2AC96
  \l 2AC97
  \l 2AC98
  \l 2AC99
  \l 2AC9A
  \l 2AC9B
  \l 2AC9C
  \l 2AC9D
  \l 2AC9E
  \l 2AC9F
  \l 2ACA0
  \l 2ACA1
  \l 2ACA2
  \l 2ACA3
  \l 2ACA4
  \l 2ACA5
  \l 2ACA6
  \l 2ACA7
  \l 2ACA8
  \l 2ACA9
  \l 2ACAA
  \l 2ACAB
  \l 2ACAC
  \l 2ACAD
  \l 2ACAE
  \l 2ACAF
  \l 2ACB0
  \l 2ACB1
  \l 2ACB2
  \l 2ACB3
  \l 2ACB4
  \l 2ACB5
  \l 2ACB6
  \l 2ACB7
  \l 2ACB8
  \l 2ACB9
  \l 2ACBA
  \l 2ACBB
  \l 2ACBC
  \l 2ACBD
  \l 2ACBE
  \l 2ACBF
  \l 2ACC0
  \l 2ACC1
  \l 2ACC2
  \l 2ACC3
  \l 2ACC4
  \l 2ACC5
  \l 2ACC6
  \l 2ACC7
  \l 2ACC8
  \l 2ACC9
  \l 2ACCA
  \l 2ACCB
  \l 2ACCC
  \l 2ACCD
  \l 2ACCE
  \l 2ACCF
  \l 2ACD0
  \l 2ACD1
  \l 2ACD2
  \l 2ACD3
  \l 2ACD4
  \l 2ACD5
  \l 2ACD6
  \l 2ACD7
  \l 2ACD8
  \l 2ACD9
  \l 2ACDA
  \l 2ACDB
  \l 2ACDC
  \l 2ACDD
  \l 2ACDE
  \l 2ACDF
  \l 2ACE0
  \l 2ACE1
  \l 2ACE2
  \l 2ACE3
  \l 2ACE4
  \l 2ACE5
  \l 2ACE6
  \l 2ACE7
  \l 2ACE8
  \l 2ACE9
  \l 2ACEA
  \l 2ACEB
  \l 2ACEC
  \l 2ACED
  \l 2ACEE
  \l 2ACEF
  \l 2ACF0
  \l 2ACF1
  \l 2ACF2
  \l 2ACF3
  \l 2ACF4
  \l 2ACF5
  \l 2ACF6
  \l 2ACF7
  \l 2ACF8
  \l 2ACF9
  \l 2ACFA
  \l 2ACFB
  \l 2ACFC
  \l 2ACFD
  \l 2ACFE
  \l 2ACFF
  \l 2AD00
  \l 2AD01
  \l 2AD02
  \l 2AD03
  \l 2AD04
  \l 2AD05
  \l 2AD06
  \l 2AD07
  \l 2AD08
  \l 2AD09
  \l 2AD0A
  \l 2AD0B
  \l 2AD0C
  \l 2AD0D
  \l 2AD0E
  \l 2AD0F
  \l 2AD10
  \l 2AD11
  \l 2AD12
  \l 2AD13
  \l 2AD14
  \l 2AD15
  \l 2AD16
  \l 2AD17
  \l 2AD18
  \l 2AD19
  \l 2AD1A
  \l 2AD1B
  \l 2AD1C
  \l 2AD1D
  \l 2AD1E
  \l 2AD1F
  \l 2AD20
  \l 2AD21
  \l 2AD22
  \l 2AD23
  \l 2AD24
  \l 2AD25
  \l 2AD26
  \l 2AD27
  \l 2AD28
  \l 2AD29
  \l 2AD2A
  \l 2AD2B
  \l 2AD2C
  \l 2AD2D
  \l 2AD2E
  \l 2AD2F
  \l 2AD30
  \l 2AD31
  \l 2AD32
  \l 2AD33
  \l 2AD34
  \l 2AD35
  \l 2AD36
  \l 2AD37
  \l 2AD38
  \l 2AD39
  \l 2AD3A
  \l 2AD3B
  \l 2AD3C
  \l 2AD3D
  \l 2AD3E
  \l 2AD3F
  \l 2AD40
  \l 2AD41
  \l 2AD42
  \l 2AD43
  \l 2AD44
  \l 2AD45
  \l 2AD46
  \l 2AD47
  \l 2AD48
  \l 2AD49
  \l 2AD4A
  \l 2AD4B
  \l 2AD4C
  \l 2AD4D
  \l 2AD4E
  \l 2AD4F
  \l 2AD50
  \l 2AD51
  \l 2AD52
  \l 2AD53
  \l 2AD54
  \l 2AD55
  \l 2AD56
  \l 2AD57
  \l 2AD58
  \l 2AD59
  \l 2AD5A
  \l 2AD5B
  \l 2AD5C
  \l 2AD5D
  \l 2AD5E
  \l 2AD5F
  \l 2AD60
  \l 2AD61
  \l 2AD62
  \l 2AD63
  \l 2AD64
  \l 2AD65
  \l 2AD66
  \l 2AD67
  \l 2AD68
  \l 2AD69
  \l 2AD6A
  \l 2AD6B
  \l 2AD6C
  \l 2AD6D
  \l 2AD6E
  \l 2AD6F
  \l 2AD70
  \l 2AD71
  \l 2AD72
  \l 2AD73
  \l 2AD74
  \l 2AD75
  \l 2AD76
  \l 2AD77
  \l 2AD78
  \l 2AD79
  \l 2AD7A
  \l 2AD7B
  \l 2AD7C
  \l 2AD7D
  \l 2AD7E
  \l 2AD7F
  \l 2AD80
  \l 2AD81
  \l 2AD82
  \l 2AD83
  \l 2AD84
  \l 2AD85
  \l 2AD86
  \l 2AD87
  \l 2AD88
  \l 2AD89
  \l 2AD8A
  \l 2AD8B
  \l 2AD8C
  \l 2AD8D
  \l 2AD8E
  \l 2AD8F
  \l 2AD90
  \l 2AD91
  \l 2AD92
  \l 2AD93
  \l 2AD94
  \l 2AD95
  \l 2AD96
  \l 2AD97
  \l 2AD98
  \l 2AD99
  \l 2AD9A
  \l 2AD9B
  \l 2AD9C
  \l 2AD9D
  \l 2AD9E
  \l 2AD9F
  \l 2ADA0
  \l 2ADA1
  \l 2ADA2
  \l 2ADA3
  \l 2ADA4
  \l 2ADA5
  \l 2ADA6
  \l 2ADA7
  \l 2ADA8
  \l 2ADA9
  \l 2ADAA
  \l 2ADAB
  \l 2ADAC
  \l 2ADAD
  \l 2ADAE
  \l 2ADAF
  \l 2ADB0
  \l 2ADB1
  \l 2ADB2
  \l 2ADB3
  \l 2ADB4
  \l 2ADB5
  \l 2ADB6
  \l 2ADB7
  \l 2ADB8
  \l 2ADB9
  \l 2ADBA
  \l 2ADBB
  \l 2ADBC
  \l 2ADBD
  \l 2ADBE
  \l 2ADBF
  \l 2ADC0
  \l 2ADC1
  \l 2ADC2
  \l 2ADC3
  \l 2ADC4
  \l 2ADC5
  \l 2ADC6
  \l 2ADC7
  \l 2ADC8
  \l 2ADC9
  \l 2ADCA
  \l 2ADCB
  \l 2ADCC
  \l 2ADCD
  \l 2ADCE
  \l 2ADCF
  \l 2ADD0
  \l 2ADD1
  \l 2ADD2
  \l 2ADD3
  \l 2ADD4
  \l 2ADD5
  \l 2ADD6
  \l 2ADD7
  \l 2ADD8
  \l 2ADD9
  \l 2ADDA
  \l 2ADDB
  \l 2ADDC
  \l 2ADDD
  \l 2ADDE
  \l 2ADDF
  \l 2ADE0
  \l 2ADE1
  \l 2ADE2
  \l 2ADE3
  \l 2ADE4
  \l 2ADE5
  \l 2ADE6
  \l 2ADE7
  \l 2ADE8
  \l 2ADE9
  \l 2ADEA
  \l 2ADEB
  \l 2ADEC
  \l 2ADED
  \l 2ADEE
  \l 2ADEF
  \l 2ADF0
  \l 2ADF1
  \l 2ADF2
  \l 2ADF3
  \l 2ADF4
  \l 2ADF5
  \l 2ADF6
  \l 2ADF7
  \l 2ADF8
  \l 2ADF9
  \l 2ADFA
  \l 2ADFB
  \l 2ADFC
  \l 2ADFD
  \l 2ADFE
  \l 2ADFF
  \l 2AE00
  \l 2AE01
  \l 2AE02
  \l 2AE03
  \l 2AE04
  \l 2AE05
  \l 2AE06
  \l 2AE07
  \l 2AE08
  \l 2AE09
  \l 2AE0A
  \l 2AE0B
  \l 2AE0C
  \l 2AE0D
  \l 2AE0E
  \l 2AE0F
  \l 2AE10
  \l 2AE11
  \l 2AE12
  \l 2AE13
  \l 2AE14
  \l 2AE15
  \l 2AE16
  \l 2AE17
  \l 2AE18
  \l 2AE19
  \l 2AE1A
  \l 2AE1B
  \l 2AE1C
  \l 2AE1D
  \l 2AE1E
  \l 2AE1F
  \l 2AE20
  \l 2AE21
  \l 2AE22
  \l 2AE23
  \l 2AE24
  \l 2AE25
  \l 2AE26
  \l 2AE27
  \l 2AE28
  \l 2AE29
  \l 2AE2A
  \l 2AE2B
  \l 2AE2C
  \l 2AE2D
  \l 2AE2E
  \l 2AE2F
  \l 2AE30
  \l 2AE31
  \l 2AE32
  \l 2AE33
  \l 2AE34
  \l 2AE35
  \l 2AE36
  \l 2AE37
  \l 2AE38
  \l 2AE39
  \l 2AE3A
  \l 2AE3B
  \l 2AE3C
  \l 2AE3D
  \l 2AE3E
  \l 2AE3F
  \l 2AE40
  \l 2AE41
  \l 2AE42
  \l 2AE43
  \l 2AE44
  \l 2AE45
  \l 2AE46
  \l 2AE47
  \l 2AE48
  \l 2AE49
  \l 2AE4A
  \l 2AE4B
  \l 2AE4C
  \l 2AE4D
  \l 2AE4E
  \l 2AE4F
  \l 2AE50
  \l 2AE51
  \l 2AE52
  \l 2AE53
  \l 2AE54
  \l 2AE55
  \l 2AE56
  \l 2AE57
  \l 2AE58
  \l 2AE59
  \l 2AE5A
  \l 2AE5B
  \l 2AE5C
  \l 2AE5D
  \l 2AE5E
  \l 2AE5F
  \l 2AE60
  \l 2AE61
  \l 2AE62
  \l 2AE63
  \l 2AE64
  \l 2AE65
  \l 2AE66
  \l 2AE67
  \l 2AE68
  \l 2AE69
  \l 2AE6A
  \l 2AE6B
  \l 2AE6C
  \l 2AE6D
  \l 2AE6E
  \l 2AE6F
  \l 2AE70
  \l 2AE71
  \l 2AE72
  \l 2AE73
  \l 2AE74
  \l 2AE75
  \l 2AE76
  \l 2AE77
  \l 2AE78
  \l 2AE79
  \l 2AE7A
  \l 2AE7B
  \l 2AE7C
  \l 2AE7D
  \l 2AE7E
  \l 2AE7F
  \l 2AE80
  \l 2AE81
  \l 2AE82
  \l 2AE83
  \l 2AE84
  \l 2AE85
  \l 2AE86
  \l 2AE87
  \l 2AE88
  \l 2AE89
  \l 2AE8A
  \l 2AE8B
  \l 2AE8C
  \l 2AE8D
  \l 2AE8E
  \l 2AE8F
  \l 2AE90
  \l 2AE91
  \l 2AE92
  \l 2AE93
  \l 2AE94
  \l 2AE95
  \l 2AE96
  \l 2AE97
  \l 2AE98
  \l 2AE99
  \l 2AE9A
  \l 2AE9B
  \l 2AE9C
  \l 2AE9D
  \l 2AE9E
  \l 2AE9F
  \l 2AEA0
  \l 2AEA1
  \l 2AEA2
  \l 2AEA3
  \l 2AEA4
  \l 2AEA5
  \l 2AEA6
  \l 2AEA7
  \l 2AEA8
  \l 2AEA9
  \l 2AEAA
  \l 2AEAB
  \l 2AEAC
  \l 2AEAD
  \l 2AEAE
  \l 2AEAF
  \l 2AEB0
  \l 2AEB1
  \l 2AEB2
  \l 2AEB3
  \l 2AEB4
  \l 2AEB5
  \l 2AEB6
  \l 2AEB7
  \l 2AEB8
  \l 2AEB9
  \l 2AEBA
  \l 2AEBB
  \l 2AEBC
  \l 2AEBD
  \l 2AEBE
  \l 2AEBF
  \l 2AEC0
  \l 2AEC1
  \l 2AEC2
  \l 2AEC3
  \l 2AEC4
  \l 2AEC5
  \l 2AEC6
  \l 2AEC7
  \l 2AEC8
  \l 2AEC9
  \l 2AECA
  \l 2AECB
  \l 2AECC
  \l 2AECD
  \l 2AECE
  \l 2AECF
  \l 2AED0
  \l 2AED1
  \l 2AED2
  \l 2AED3
  \l 2AED4
  \l 2AED5
  \l 2AED6
  \l 2AED7
  \l 2AED8
  \l 2AED9
  \l 2AEDA
  \l 2AEDB
  \l 2AEDC
  \l 2AEDD
  \l 2AEDE
  \l 2AEDF
  \l 2AEE0
  \l 2AEE1
  \l 2AEE2
  \l 2AEE3
  \l 2AEE4
  \l 2AEE5
  \l 2AEE6
  \l 2AEE7
  \l 2AEE8
  \l 2AEE9
  \l 2AEEA
  \l 2AEEB
  \l 2AEEC
  \l 2AEED
  \l 2AEEE
  \l 2AEEF
  \l 2AEF0
  \l 2AEF1
  \l 2AEF2
  \l 2AEF3
  \l 2AEF4
  \l 2AEF5
  \l 2AEF6
  \l 2AEF7
  \l 2AEF8
  \l 2AEF9
  \l 2AEFA
  \l 2AEFB
  \l 2AEFC
  \l 2AEFD
  \l 2AEFE
  \l 2AEFF
  \l 2AF00
  \l 2AF01
  \l 2AF02
  \l 2AF03
  \l 2AF04
  \l 2AF05
  \l 2AF06
  \l 2AF07
  \l 2AF08
  \l 2AF09
  \l 2AF0A
  \l 2AF0B
  \l 2AF0C
  \l 2AF0D
  \l 2AF0E
  \l 2AF0F
  \l 2AF10
  \l 2AF11
  \l 2AF12
  \l 2AF13
  \l 2AF14
  \l 2AF15
  \l 2AF16
  \l 2AF17
  \l 2AF18
  \l 2AF19
  \l 2AF1A
  \l 2AF1B
  \l 2AF1C
  \l 2AF1D
  \l 2AF1E
  \l 2AF1F
  \l 2AF20
  \l 2AF21
  \l 2AF22
  \l 2AF23
  \l 2AF24
  \l 2AF25
  \l 2AF26
  \l 2AF27
  \l 2AF28
  \l 2AF29
  \l 2AF2A
  \l 2AF2B
  \l 2AF2C
  \l 2AF2D
  \l 2AF2E
  \l 2AF2F
  \l 2AF30
  \l 2AF31
  \l 2AF32
  \l 2AF33
  \l 2AF34
  \l 2AF35
  \l 2AF36
  \l 2AF37
  \l 2AF38
  \l 2AF39
  \l 2AF3A
  \l 2AF3B
  \l 2AF3C
  \l 2AF3D
  \l 2AF3E
  \l 2AF3F
  \l 2AF40
  \l 2AF41
  \l 2AF42
  \l 2AF43
  \l 2AF44
  \l 2AF45
  \l 2AF46
  \l 2AF47
  \l 2AF48
  \l 2AF49
  \l 2AF4A
  \l 2AF4B
  \l 2AF4C
  \l 2AF4D
  \l 2AF4E
  \l 2AF4F
  \l 2AF50
  \l 2AF51
  \l 2AF52
  \l 2AF53
  \l 2AF54
  \l 2AF55
  \l 2AF56
  \l 2AF57
  \l 2AF58
  \l 2AF59
  \l 2AF5A
  \l 2AF5B
  \l 2AF5C
  \l 2AF5D
  \l 2AF5E
  \l 2AF5F
  \l 2AF60
  \l 2AF61
  \l 2AF62
  \l 2AF63
  \l 2AF64
  \l 2AF65
  \l 2AF66
  \l 2AF67
  \l 2AF68
  \l 2AF69
  \l 2AF6A
  \l 2AF6B
  \l 2AF6C
  \l 2AF6D
  \l 2AF6E
  \l 2AF6F
  \l 2AF70
  \l 2AF71
  \l 2AF72
  \l 2AF73
  \l 2AF74
  \l 2AF75
  \l 2AF76
  \l 2AF77
  \l 2AF78
  \l 2AF79
  \l 2AF7A
  \l 2AF7B
  \l 2AF7C
  \l 2AF7D
  \l 2AF7E
  \l 2AF7F
  \l 2AF80
  \l 2AF81
  \l 2AF82
  \l 2AF83
  \l 2AF84
  \l 2AF85
  \l 2AF86
  \l 2AF87
  \l 2AF88
  \l 2AF89
  \l 2AF8A
  \l 2AF8B
  \l 2AF8C
  \l 2AF8D
  \l 2AF8E
  \l 2AF8F
  \l 2AF90
  \l 2AF91
  \l 2AF92
  \l 2AF93
  \l 2AF94
  \l 2AF95
  \l 2AF96
  \l 2AF97
  \l 2AF98
  \l 2AF99
  \l 2AF9A
  \l 2AF9B
  \l 2AF9C
  \l 2AF9D
  \l 2AF9E
  \l 2AF9F
  \l 2AFA0
  \l 2AFA1
  \l 2AFA2
  \l 2AFA3
  \l 2AFA4
  \l 2AFA5
  \l 2AFA6
  \l 2AFA7
  \l 2AFA8
  \l 2AFA9
  \l 2AFAA
  \l 2AFAB
  \l 2AFAC
  \l 2AFAD
  \l 2AFAE
  \l 2AFAF
  \l 2AFB0
  \l 2AFB1
  \l 2AFB2
  \l 2AFB3
  \l 2AFB4
  \l 2AFB5
  \l 2AFB6
  \l 2AFB7
  \l 2AFB8
  \l 2AFB9
  \l 2AFBA
  \l 2AFBB
  \l 2AFBC
  \l 2AFBD
  \l 2AFBE
  \l 2AFBF
  \l 2AFC0
  \l 2AFC1
  \l 2AFC2
  \l 2AFC3
  \l 2AFC4
  \l 2AFC5
  \l 2AFC6
  \l 2AFC7
  \l 2AFC8
  \l 2AFC9
  \l 2AFCA
  \l 2AFCB
  \l 2AFCC
  \l 2AFCD
  \l 2AFCE
  \l 2AFCF
  \l 2AFD0
  \l 2AFD1
  \l 2AFD2
  \l 2AFD3
  \l 2AFD4
  \l 2AFD5
  \l 2AFD6
  \l 2AFD7
  \l 2AFD8
  \l 2AFD9
  \l 2AFDA
  \l 2AFDB
  \l 2AFDC
  \l 2AFDD
  \l 2AFDE
  \l 2AFDF
  \l 2AFE0
  \l 2AFE1
  \l 2AFE2
  \l 2AFE3
  \l 2AFE4
  \l 2AFE5
  \l 2AFE6
  \l 2AFE7
  \l 2AFE8
  \l 2AFE9
  \l 2AFEA
  \l 2AFEB
  \l 2AFEC
  \l 2AFED
  \l 2AFEE
  \l 2AFEF
  \l 2AFF0
  \l 2AFF1
  \l 2AFF2
  \l 2AFF3
  \l 2AFF4
  \l 2AFF5
  \l 2AFF6
  \l 2AFF7
  \l 2AFF8
  \l 2AFF9
  \l 2AFFA
  \l 2AFFB
  \l 2AFFC
  \l 2AFFD
  \l 2AFFE
  \l 2AFFF
  \l 2B000
  \l 2B001
  \l 2B002
  \l 2B003
  \l 2B004
  \l 2B005
  \l 2B006
  \l 2B007
  \l 2B008
  \l 2B009
  \l 2B00A
  \l 2B00B
  \l 2B00C
  \l 2B00D
  \l 2B00E
  \l 2B00F
  \l 2B010
  \l 2B011
  \l 2B012
  \l 2B013
  \l 2B014
  \l 2B015
  \l 2B016
  \l 2B017
  \l 2B018
  \l 2B019
  \l 2B01A
  \l 2B01B
  \l 2B01C
  \l 2B01D
  \l 2B01E
  \l 2B01F
  \l 2B020
  \l 2B021
  \l 2B022
  \l 2B023
  \l 2B024
  \l 2B025
  \l 2B026
  \l 2B027
  \l 2B028
  \l 2B029
  \l 2B02A
  \l 2B02B
  \l 2B02C
  \l 2B02D
  \l 2B02E
  \l 2B02F
  \l 2B030
  \l 2B031
  \l 2B032
  \l 2B033
  \l 2B034
  \l 2B035
  \l 2B036
  \l 2B037
  \l 2B038
  \l 2B039
  \l 2B03A
  \l 2B03B
  \l 2B03C
  \l 2B03D
  \l 2B03E
  \l 2B03F
  \l 2B040
  \l 2B041
  \l 2B042
  \l 2B043
  \l 2B044
  \l 2B045
  \l 2B046
  \l 2B047
  \l 2B048
  \l 2B049
  \l 2B04A
  \l 2B04B
  \l 2B04C
  \l 2B04D
  \l 2B04E
  \l 2B04F
  \l 2B050
  \l 2B051
  \l 2B052
  \l 2B053
  \l 2B054
  \l 2B055
  \l 2B056
  \l 2B057
  \l 2B058
  \l 2B059
  \l 2B05A
  \l 2B05B
  \l 2B05C
  \l 2B05D
  \l 2B05E
  \l 2B05F
  \l 2B060
  \l 2B061
  \l 2B062
  \l 2B063
  \l 2B064
  \l 2B065
  \l 2B066
  \l 2B067
  \l 2B068
  \l 2B069
  \l 2B06A
  \l 2B06B
  \l 2B06C
  \l 2B06D
  \l 2B06E
  \l 2B06F
  \l 2B070
  \l 2B071
  \l 2B072
  \l 2B073
  \l 2B074
  \l 2B075
  \l 2B076
  \l 2B077
  \l 2B078
  \l 2B079
  \l 2B07A
  \l 2B07B
  \l 2B07C
  \l 2B07D
  \l 2B07E
  \l 2B07F
  \l 2B080
  \l 2B081
  \l 2B082
  \l 2B083
  \l 2B084
  \l 2B085
  \l 2B086
  \l 2B087
  \l 2B088
  \l 2B089
  \l 2B08A
  \l 2B08B
  \l 2B08C
  \l 2B08D
  \l 2B08E
  \l 2B08F
  \l 2B090
  \l 2B091
  \l 2B092
  \l 2B093
  \l 2B094
  \l 2B095
  \l 2B096
  \l 2B097
  \l 2B098
  \l 2B099
  \l 2B09A
  \l 2B09B
  \l 2B09C
  \l 2B09D
  \l 2B09E
  \l 2B09F
  \l 2B0A0
  \l 2B0A1
  \l 2B0A2
  \l 2B0A3
  \l 2B0A4
  \l 2B0A5
  \l 2B0A6
  \l 2B0A7
  \l 2B0A8
  \l 2B0A9
  \l 2B0AA
  \l 2B0AB
  \l 2B0AC
  \l 2B0AD
  \l 2B0AE
  \l 2B0AF
  \l 2B0B0
  \l 2B0B1
  \l 2B0B2
  \l 2B0B3
  \l 2B0B4
  \l 2B0B5
  \l 2B0B6
  \l 2B0B7
  \l 2B0B8
  \l 2B0B9
  \l 2B0BA
  \l 2B0BB
  \l 2B0BC
  \l 2B0BD
  \l 2B0BE
  \l 2B0BF
  \l 2B0C0
  \l 2B0C1
  \l 2B0C2
  \l 2B0C3
  \l 2B0C4
  \l 2B0C5
  \l 2B0C6
  \l 2B0C7
  \l 2B0C8
  \l 2B0C9
  \l 2B0CA
  \l 2B0CB
  \l 2B0CC
  \l 2B0CD
  \l 2B0CE
  \l 2B0CF
  \l 2B0D0
  \l 2B0D1
  \l 2B0D2
  \l 2B0D3
  \l 2B0D4
  \l 2B0D5
  \l 2B0D6
  \l 2B0D7
  \l 2B0D8
  \l 2B0D9
  \l 2B0DA
  \l 2B0DB
  \l 2B0DC
  \l 2B0DD
  \l 2B0DE
  \l 2B0DF
  \l 2B0E0
  \l 2B0E1
  \l 2B0E2
  \l 2B0E3
  \l 2B0E4
  \l 2B0E5
  \l 2B0E6
  \l 2B0E7
  \l 2B0E8
  \l 2B0E9
  \l 2B0EA
  \l 2B0EB
  \l 2B0EC
  \l 2B0ED
  \l 2B0EE
  \l 2B0EF
  \l 2B0F0
  \l 2B0F1
  \l 2B0F2
  \l 2B0F3
  \l 2B0F4
  \l 2B0F5
  \l 2B0F6
  \l 2B0F7
  \l 2B0F8
  \l 2B0F9
  \l 2B0FA
  \l 2B0FB
  \l 2B0FC
  \l 2B0FD
  \l 2B0FE
  \l 2B0FF
  \l 2B100
  \l 2B101
  \l 2B102
  \l 2B103
  \l 2B104
  \l 2B105
  \l 2B106
  \l 2B107
  \l 2B108
  \l 2B109
  \l 2B10A
  \l 2B10B
  \l 2B10C
  \l 2B10D
  \l 2B10E
  \l 2B10F
  \l 2B110
  \l 2B111
  \l 2B112
  \l 2B113
  \l 2B114
  \l 2B115
  \l 2B116
  \l 2B117
  \l 2B118
  \l 2B119
  \l 2B11A
  \l 2B11B
  \l 2B11C
  \l 2B11D
  \l 2B11E
  \l 2B11F
  \l 2B120
  \l 2B121
  \l 2B122
  \l 2B123
  \l 2B124
  \l 2B125
  \l 2B126
  \l 2B127
  \l 2B128
  \l 2B129
  \l 2B12A
  \l 2B12B
  \l 2B12C
  \l 2B12D
  \l 2B12E
  \l 2B12F
  \l 2B130
  \l 2B131
  \l 2B132
  \l 2B133
  \l 2B134
  \l 2B135
  \l 2B136
  \l 2B137
  \l 2B138
  \l 2B139
  \l 2B13A
  \l 2B13B
  \l 2B13C
  \l 2B13D
  \l 2B13E
  \l 2B13F
  \l 2B140
  \l 2B141
  \l 2B142
  \l 2B143
  \l 2B144
  \l 2B145
  \l 2B146
  \l 2B147
  \l 2B148
  \l 2B149
  \l 2B14A
  \l 2B14B
  \l 2B14C
  \l 2B14D
  \l 2B14E
  \l 2B14F
  \l 2B150
  \l 2B151
  \l 2B152
  \l 2B153
  \l 2B154
  \l 2B155
  \l 2B156
  \l 2B157
  \l 2B158
  \l 2B159
  \l 2B15A
  \l 2B15B
  \l 2B15C
  \l 2B15D
  \l 2B15E
  \l 2B15F
  \l 2B160
  \l 2B161
  \l 2B162
  \l 2B163
  \l 2B164
  \l 2B165
  \l 2B166
  \l 2B167
  \l 2B168
  \l 2B169
  \l 2B16A
  \l 2B16B
  \l 2B16C
  \l 2B16D
  \l 2B16E
  \l 2B16F
  \l 2B170
  \l 2B171
  \l 2B172
  \l 2B173
  \l 2B174
  \l 2B175
  \l 2B176
  \l 2B177
  \l 2B178
  \l 2B179
  \l 2B17A
  \l 2B17B
  \l 2B17C
  \l 2B17D
  \l 2B17E
  \l 2B17F
  \l 2B180
  \l 2B181
  \l 2B182
  \l 2B183
  \l 2B184
  \l 2B185
  \l 2B186
  \l 2B187
  \l 2B188
  \l 2B189
  \l 2B18A
  \l 2B18B
  \l 2B18C
  \l 2B18D
  \l 2B18E
  \l 2B18F
  \l 2B190
  \l 2B191
  \l 2B192
  \l 2B193
  \l 2B194
  \l 2B195
  \l 2B196
  \l 2B197
  \l 2B198
  \l 2B199
  \l 2B19A
  \l 2B19B
  \l 2B19C
  \l 2B19D
  \l 2B19E
  \l 2B19F
  \l 2B1A0
  \l 2B1A1
  \l 2B1A2
  \l 2B1A3
  \l 2B1A4
  \l 2B1A5
  \l 2B1A6
  \l 2B1A7
  \l 2B1A8
  \l 2B1A9
  \l 2B1AA
  \l 2B1AB
  \l 2B1AC
  \l 2B1AD
  \l 2B1AE
  \l 2B1AF
  \l 2B1B0
  \l 2B1B1
  \l 2B1B2
  \l 2B1B3
  \l 2B1B4
  \l 2B1B5
  \l 2B1B6
  \l 2B1B7
  \l 2B1B8
  \l 2B1B9
  \l 2B1BA
  \l 2B1BB
  \l 2B1BC
  \l 2B1BD
  \l 2B1BE
  \l 2B1BF
  \l 2B1C0
  \l 2B1C1
  \l 2B1C2
  \l 2B1C3
  \l 2B1C4
  \l 2B1C5
  \l 2B1C6
  \l 2B1C7
  \l 2B1C8
  \l 2B1C9
  \l 2B1CA
  \l 2B1CB
  \l 2B1CC
  \l 2B1CD
  \l 2B1CE
  \l 2B1CF
  \l 2B1D0
  \l 2B1D1
  \l 2B1D2
  \l 2B1D3
  \l 2B1D4
  \l 2B1D5
  \l 2B1D6
  \l 2B1D7
  \l 2B1D8
  \l 2B1D9
  \l 2B1DA
  \l 2B1DB
  \l 2B1DC
  \l 2B1DD
  \l 2B1DE
  \l 2B1DF
  \l 2B1E0
  \l 2B1E1
  \l 2B1E2
  \l 2B1E3
  \l 2B1E4
  \l 2B1E5
  \l 2B1E6
  \l 2B1E7
  \l 2B1E8
  \l 2B1E9
  \l 2B1EA
  \l 2B1EB
  \l 2B1EC
  \l 2B1ED
  \l 2B1EE
  \l 2B1EF
  \l 2B1F0
  \l 2B1F1
  \l 2B1F2
  \l 2B1F3
  \l 2B1F4
  \l 2B1F5
  \l 2B1F6
  \l 2B1F7
  \l 2B1F8
  \l 2B1F9
  \l 2B1FA
  \l 2B1FB
  \l 2B1FC
  \l 2B1FD
  \l 2B1FE
  \l 2B1FF
  \l 2B200
  \l 2B201
  \l 2B202
  \l 2B203
  \l 2B204
  \l 2B205
  \l 2B206
  \l 2B207
  \l 2B208
  \l 2B209
  \l 2B20A
  \l 2B20B
  \l 2B20C
  \l 2B20D
  \l 2B20E
  \l 2B20F
  \l 2B210
  \l 2B211
  \l 2B212
  \l 2B213
  \l 2B214
  \l 2B215
  \l 2B216
  \l 2B217
  \l 2B218
  \l 2B219
  \l 2B21A
  \l 2B21B
  \l 2B21C
  \l 2B21D
  \l 2B21E
  \l 2B21F
  \l 2B220
  \l 2B221
  \l 2B222
  \l 2B223
  \l 2B224
  \l 2B225
  \l 2B226
  \l 2B227
  \l 2B228
  \l 2B229
  \l 2B22A
  \l 2B22B
  \l 2B22C
  \l 2B22D
  \l 2B22E
  \l 2B22F
  \l 2B230
  \l 2B231
  \l 2B232
  \l 2B233
  \l 2B234
  \l 2B235
  \l 2B236
  \l 2B237
  \l 2B238
  \l 2B239
  \l 2B23A
  \l 2B23B
  \l 2B23C
  \l 2B23D
  \l 2B23E
  \l 2B23F
  \l 2B240
  \l 2B241
  \l 2B242
  \l 2B243
  \l 2B244
  \l 2B245
  \l 2B246
  \l 2B247
  \l 2B248
  \l 2B249
  \l 2B24A
  \l 2B24B
  \l 2B24C
  \l 2B24D
  \l 2B24E
  \l 2B24F
  \l 2B250
  \l 2B251
  \l 2B252
  \l 2B253
  \l 2B254
  \l 2B255
  \l 2B256
  \l 2B257
  \l 2B258
  \l 2B259
  \l 2B25A
  \l 2B25B
  \l 2B25C
  \l 2B25D
  \l 2B25E
  \l 2B25F
  \l 2B260
  \l 2B261
  \l 2B262
  \l 2B263
  \l 2B264
  \l 2B265
  \l 2B266
  \l 2B267
  \l 2B268
  \l 2B269
  \l 2B26A
  \l 2B26B
  \l 2B26C
  \l 2B26D
  \l 2B26E
  \l 2B26F
  \l 2B270
  \l 2B271
  \l 2B272
  \l 2B273
  \l 2B274
  \l 2B275
  \l 2B276
  \l 2B277
  \l 2B278
  \l 2B279
  \l 2B27A
  \l 2B27B
  \l 2B27C
  \l 2B27D
  \l 2B27E
  \l 2B27F
  \l 2B280
  \l 2B281
  \l 2B282
  \l 2B283
  \l 2B284
  \l 2B285
  \l 2B286
  \l 2B287
  \l 2B288
  \l 2B289
  \l 2B28A
  \l 2B28B
  \l 2B28C
  \l 2B28D
  \l 2B28E
  \l 2B28F
  \l 2B290
  \l 2B291
  \l 2B292
  \l 2B293
  \l 2B294
  \l 2B295
  \l 2B296
  \l 2B297
  \l 2B298
  \l 2B299
  \l 2B29A
  \l 2B29B
  \l 2B29C
  \l 2B29D
  \l 2B29E
  \l 2B29F
  \l 2B2A0
  \l 2B2A1
  \l 2B2A2
  \l 2B2A3
  \l 2B2A4
  \l 2B2A5
  \l 2B2A6
  \l 2B2A7
  \l 2B2A8
  \l 2B2A9
  \l 2B2AA
  \l 2B2AB
  \l 2B2AC
  \l 2B2AD
  \l 2B2AE
  \l 2B2AF
  \l 2B2B0
  \l 2B2B1
  \l 2B2B2
  \l 2B2B3
  \l 2B2B4
  \l 2B2B5
  \l 2B2B6
  \l 2B2B7
  \l 2B2B8
  \l 2B2B9
  \l 2B2BA
  \l 2B2BB
  \l 2B2BC
  \l 2B2BD
  \l 2B2BE
  \l 2B2BF
  \l 2B2C0
  \l 2B2C1
  \l 2B2C2
  \l 2B2C3
  \l 2B2C4
  \l 2B2C5
  \l 2B2C6
  \l 2B2C7
  \l 2B2C8
  \l 2B2C9
  \l 2B2CA
  \l 2B2CB
  \l 2B2CC
  \l 2B2CD
  \l 2B2CE
  \l 2B2CF
  \l 2B2D0
  \l 2B2D1
  \l 2B2D2
  \l 2B2D3
  \l 2B2D4
  \l 2B2D5
  \l 2B2D6
  \l 2B2D7
  \l 2B2D8
  \l 2B2D9
  \l 2B2DA
  \l 2B2DB
  \l 2B2DC
  \l 2B2DD
  \l 2B2DE
  \l 2B2DF
  \l 2B2E0
  \l 2B2E1
  \l 2B2E2
  \l 2B2E3
  \l 2B2E4
  \l 2B2E5
  \l 2B2E6
  \l 2B2E7
  \l 2B2E8
  \l 2B2E9
  \l 2B2EA
  \l 2B2EB
  \l 2B2EC
  \l 2B2ED
  \l 2B2EE
  \l 2B2EF
  \l 2B2F0
  \l 2B2F1
  \l 2B2F2
  \l 2B2F3
  \l 2B2F4
  \l 2B2F5
  \l 2B2F6
  \l 2B2F7
  \l 2B2F8
  \l 2B2F9
  \l 2B2FA
  \l 2B2FB
  \l 2B2FC
  \l 2B2FD
  \l 2B2FE
  \l 2B2FF
  \l 2B300
  \l 2B301
  \l 2B302
  \l 2B303
  \l 2B304
  \l 2B305
  \l 2B306
  \l 2B307
  \l 2B308
  \l 2B309
  \l 2B30A
  \l 2B30B
  \l 2B30C
  \l 2B30D
  \l 2B30E
  \l 2B30F
  \l 2B310
  \l 2B311
  \l 2B312
  \l 2B313
  \l 2B314
  \l 2B315
  \l 2B316
  \l 2B317
  \l 2B318
  \l 2B319
  \l 2B31A
  \l 2B31B
  \l 2B31C
  \l 2B31D
  \l 2B31E
  \l 2B31F
  \l 2B320
  \l 2B321
  \l 2B322
  \l 2B323
  \l 2B324
  \l 2B325
  \l 2B326
  \l 2B327
  \l 2B328
  \l 2B329
  \l 2B32A
  \l 2B32B
  \l 2B32C
  \l 2B32D
  \l 2B32E
  \l 2B32F
  \l 2B330
  \l 2B331
  \l 2B332
  \l 2B333
  \l 2B334
  \l 2B335
  \l 2B336
  \l 2B337
  \l 2B338
  \l 2B339
  \l 2B33A
  \l 2B33B
  \l 2B33C
  \l 2B33D
  \l 2B33E
  \l 2B33F
  \l 2B340
  \l 2B341
  \l 2B342
  \l 2B343
  \l 2B344
  \l 2B345
  \l 2B346
  \l 2B347
  \l 2B348
  \l 2B349
  \l 2B34A
  \l 2B34B
  \l 2B34C
  \l 2B34D
  \l 2B34E
  \l 2B34F
  \l 2B350
  \l 2B351
  \l 2B352
  \l 2B353
  \l 2B354
  \l 2B355
  \l 2B356
  \l 2B357
  \l 2B358
  \l 2B359
  \l 2B35A
  \l 2B35B
  \l 2B35C
  \l 2B35D
  \l 2B35E
  \l 2B35F
  \l 2B360
  \l 2B361
  \l 2B362
  \l 2B363
  \l 2B364
  \l 2B365
  \l 2B366
  \l 2B367
  \l 2B368
  \l 2B369
  \l 2B36A
  \l 2B36B
  \l 2B36C
  \l 2B36D
  \l 2B36E
  \l 2B36F
  \l 2B370
  \l 2B371
  \l 2B372
  \l 2B373
  \l 2B374
  \l 2B375
  \l 2B376
  \l 2B377
  \l 2B378
  \l 2B379
  \l 2B37A
  \l 2B37B
  \l 2B37C
  \l 2B37D
  \l 2B37E
  \l 2B37F
  \l 2B380
  \l 2B381
  \l 2B382
  \l 2B383
  \l 2B384
  \l 2B385
  \l 2B386
  \l 2B387
  \l 2B388
  \l 2B389
  \l 2B38A
  \l 2B38B
  \l 2B38C
  \l 2B38D
  \l 2B38E
  \l 2B38F
  \l 2B390
  \l 2B391
  \l 2B392
  \l 2B393
  \l 2B394
  \l 2B395
  \l 2B396
  \l 2B397
  \l 2B398
  \l 2B399
  \l 2B39A
  \l 2B39B
  \l 2B39C
  \l 2B39D
  \l 2B39E
  \l 2B39F
  \l 2B3A0
  \l 2B3A1
  \l 2B3A2
  \l 2B3A3
  \l 2B3A4
  \l 2B3A5
  \l 2B3A6
  \l 2B3A7
  \l 2B3A8
  \l 2B3A9
  \l 2B3AA
  \l 2B3AB
  \l 2B3AC
  \l 2B3AD
  \l 2B3AE
  \l 2B3AF
  \l 2B3B0
  \l 2B3B1
  \l 2B3B2
  \l 2B3B3
  \l 2B3B4
  \l 2B3B5
  \l 2B3B6
  \l 2B3B7
  \l 2B3B8
  \l 2B3B9
  \l 2B3BA
  \l 2B3BB
  \l 2B3BC
  \l 2B3BD
  \l 2B3BE
  \l 2B3BF
  \l 2B3C0
  \l 2B3C1
  \l 2B3C2
  \l 2B3C3
  \l 2B3C4
  \l 2B3C5
  \l 2B3C6
  \l 2B3C7
  \l 2B3C8
  \l 2B3C9
  \l 2B3CA
  \l 2B3CB
  \l 2B3CC
  \l 2B3CD
  \l 2B3CE
  \l 2B3CF
  \l 2B3D0
  \l 2B3D1
  \l 2B3D2
  \l 2B3D3
  \l 2B3D4
  \l 2B3D5
  \l 2B3D6
  \l 2B3D7
  \l 2B3D8
  \l 2B3D9
  \l 2B3DA
  \l 2B3DB
  \l 2B3DC
  \l 2B3DD
  \l 2B3DE
  \l 2B3DF
  \l 2B3E0
  \l 2B3E1
  \l 2B3E2
  \l 2B3E3
  \l 2B3E4
  \l 2B3E5
  \l 2B3E6
  \l 2B3E7
  \l 2B3E8
  \l 2B3E9
  \l 2B3EA
  \l 2B3EB
  \l 2B3EC
  \l 2B3ED
  \l 2B3EE
  \l 2B3EF
  \l 2B3F0
  \l 2B3F1
  \l 2B3F2
  \l 2B3F3
  \l 2B3F4
  \l 2B3F5
  \l 2B3F6
  \l 2B3F7
  \l 2B3F8
  \l 2B3F9
  \l 2B3FA
  \l 2B3FB
  \l 2B3FC
  \l 2B3FD
  \l 2B3FE
  \l 2B3FF
  \l 2B400
  \l 2B401
  \l 2B402
  \l 2B403
  \l 2B404
  \l 2B405
  \l 2B406
  \l 2B407
  \l 2B408
  \l 2B409
  \l 2B40A
  \l 2B40B
  \l 2B40C
  \l 2B40D
  \l 2B40E
  \l 2B40F
  \l 2B410
  \l 2B411
  \l 2B412
  \l 2B413
  \l 2B414
  \l 2B415
  \l 2B416
  \l 2B417
  \l 2B418
  \l 2B419
  \l 2B41A
  \l 2B41B
  \l 2B41C
  \l 2B41D
  \l 2B41E
  \l 2B41F
  \l 2B420
  \l 2B421
  \l 2B422
  \l 2B423
  \l 2B424
  \l 2B425
  \l 2B426
  \l 2B427
  \l 2B428
  \l 2B429
  \l 2B42A
  \l 2B42B
  \l 2B42C
  \l 2B42D
  \l 2B42E
  \l 2B42F
  \l 2B430
  \l 2B431
  \l 2B432
  \l 2B433
  \l 2B434
  \l 2B435
  \l 2B436
  \l 2B437
  \l 2B438
  \l 2B439
  \l 2B43A
  \l 2B43B
  \l 2B43C
  \l 2B43D
  \l 2B43E
  \l 2B43F
  \l 2B440
  \l 2B441
  \l 2B442
  \l 2B443
  \l 2B444
  \l 2B445
  \l 2B446
  \l 2B447
  \l 2B448
  \l 2B449
  \l 2B44A
  \l 2B44B
  \l 2B44C
  \l 2B44D
  \l 2B44E
  \l 2B44F
  \l 2B450
  \l 2B451
  \l 2B452
  \l 2B453
  \l 2B454
  \l 2B455
  \l 2B456
  \l 2B457
  \l 2B458
  \l 2B459
  \l 2B45A
  \l 2B45B
  \l 2B45C
  \l 2B45D
  \l 2B45E
  \l 2B45F
  \l 2B460
  \l 2B461
  \l 2B462
  \l 2B463
  \l 2B464
  \l 2B465
  \l 2B466
  \l 2B467
  \l 2B468
  \l 2B469
  \l 2B46A
  \l 2B46B
  \l 2B46C
  \l 2B46D
  \l 2B46E
  \l 2B46F
  \l 2B470
  \l 2B471
  \l 2B472
  \l 2B473
  \l 2B474
  \l 2B475
  \l 2B476
  \l 2B477
  \l 2B478
  \l 2B479
  \l 2B47A
  \l 2B47B
  \l 2B47C
  \l 2B47D
  \l 2B47E
  \l 2B47F
  \l 2B480
  \l 2B481
  \l 2B482
  \l 2B483
  \l 2B484
  \l 2B485
  \l 2B486
  \l 2B487
  \l 2B488
  \l 2B489
  \l 2B48A
  \l 2B48B
  \l 2B48C
  \l 2B48D
  \l 2B48E
  \l 2B48F
  \l 2B490
  \l 2B491
  \l 2B492
  \l 2B493
  \l 2B494
  \l 2B495
  \l 2B496
  \l 2B497
  \l 2B498
  \l 2B499
  \l 2B49A
  \l 2B49B
  \l 2B49C
  \l 2B49D
  \l 2B49E
  \l 2B49F
  \l 2B4A0
  \l 2B4A1
  \l 2B4A2
  \l 2B4A3
  \l 2B4A4
  \l 2B4A5
  \l 2B4A6
  \l 2B4A7
  \l 2B4A8
  \l 2B4A9
  \l 2B4AA
  \l 2B4AB
  \l 2B4AC
  \l 2B4AD
  \l 2B4AE
  \l 2B4AF
  \l 2B4B0
  \l 2B4B1
  \l 2B4B2
  \l 2B4B3
  \l 2B4B4
  \l 2B4B5
  \l 2B4B6
  \l 2B4B7
  \l 2B4B8
  \l 2B4B9
  \l 2B4BA
  \l 2B4BB
  \l 2B4BC
  \l 2B4BD
  \l 2B4BE
  \l 2B4BF
  \l 2B4C0
  \l 2B4C1
  \l 2B4C2
  \l 2B4C3
  \l 2B4C4
  \l 2B4C5
  \l 2B4C6
  \l 2B4C7
  \l 2B4C8
  \l 2B4C9
  \l 2B4CA
  \l 2B4CB
  \l 2B4CC
  \l 2B4CD
  \l 2B4CE
  \l 2B4CF
  \l 2B4D0
  \l 2B4D1
  \l 2B4D2
  \l 2B4D3
  \l 2B4D4
  \l 2B4D5
  \l 2B4D6
  \l 2B4D7
  \l 2B4D8
  \l 2B4D9
  \l 2B4DA
  \l 2B4DB
  \l 2B4DC
  \l 2B4DD
  \l 2B4DE
  \l 2B4DF
  \l 2B4E0
  \l 2B4E1
  \l 2B4E2
  \l 2B4E3
  \l 2B4E4
  \l 2B4E5
  \l 2B4E6
  \l 2B4E7
  \l 2B4E8
  \l 2B4E9
  \l 2B4EA
  \l 2B4EB
  \l 2B4EC
  \l 2B4ED
  \l 2B4EE
  \l 2B4EF
  \l 2B4F0
  \l 2B4F1
  \l 2B4F2
  \l 2B4F3
  \l 2B4F4
  \l 2B4F5
  \l 2B4F6
  \l 2B4F7
  \l 2B4F8
  \l 2B4F9
  \l 2B4FA
  \l 2B4FB
  \l 2B4FC
  \l 2B4FD
  \l 2B4FE
  \l 2B4FF
  \l 2B500
  \l 2B501
  \l 2B502
  \l 2B503
  \l 2B504
  \l 2B505
  \l 2B506
  \l 2B507
  \l 2B508
  \l 2B509
  \l 2B50A
  \l 2B50B
  \l 2B50C
  \l 2B50D
  \l 2B50E
  \l 2B50F
  \l 2B510
  \l 2B511
  \l 2B512
  \l 2B513
  \l 2B514
  \l 2B515
  \l 2B516
  \l 2B517
  \l 2B518
  \l 2B519
  \l 2B51A
  \l 2B51B
  \l 2B51C
  \l 2B51D
  \l 2B51E
  \l 2B51F
  \l 2B520
  \l 2B521
  \l 2B522
  \l 2B523
  \l 2B524
  \l 2B525
  \l 2B526
  \l 2B527
  \l 2B528
  \l 2B529
  \l 2B52A
  \l 2B52B
  \l 2B52C
  \l 2B52D
  \l 2B52E
  \l 2B52F
  \l 2B530
  \l 2B531
  \l 2B532
  \l 2B533
  \l 2B534
  \l 2B535
  \l 2B536
  \l 2B537
  \l 2B538
  \l 2B539
  \l 2B53A
  \l 2B53B
  \l 2B53C
  \l 2B53D
  \l 2B53E
  \l 2B53F
  \l 2B540
  \l 2B541
  \l 2B542
  \l 2B543
  \l 2B544
  \l 2B545
  \l 2B546
  \l 2B547
  \l 2B548
  \l 2B549
  \l 2B54A
  \l 2B54B
  \l 2B54C
  \l 2B54D
  \l 2B54E
  \l 2B54F
  \l 2B550
  \l 2B551
  \l 2B552
  \l 2B553
  \l 2B554
  \l 2B555
  \l 2B556
  \l 2B557
  \l 2B558
  \l 2B559
  \l 2B55A
  \l 2B55B
  \l 2B55C
  \l 2B55D
  \l 2B55E
  \l 2B55F
  \l 2B560
  \l 2B561
  \l 2B562
  \l 2B563
  \l 2B564
  \l 2B565
  \l 2B566
  \l 2B567
  \l 2B568
  \l 2B569
  \l 2B56A
  \l 2B56B
  \l 2B56C
  \l 2B56D
  \l 2B56E
  \l 2B56F
  \l 2B570
  \l 2B571
  \l 2B572
  \l 2B573
  \l 2B574
  \l 2B575
  \l 2B576
  \l 2B577
  \l 2B578
  \l 2B579
  \l 2B57A
  \l 2B57B
  \l 2B57C
  \l 2B57D
  \l 2B57E
  \l 2B57F
  \l 2B580
  \l 2B581
  \l 2B582
  \l 2B583
  \l 2B584
  \l 2B585
  \l 2B586
  \l 2B587
  \l 2B588
  \l 2B589
  \l 2B58A
  \l 2B58B
  \l 2B58C
  \l 2B58D
  \l 2B58E
  \l 2B58F
  \l 2B590
  \l 2B591
  \l 2B592
  \l 2B593
  \l 2B594
  \l 2B595
  \l 2B596
  \l 2B597
  \l 2B598
  \l 2B599
  \l 2B59A
  \l 2B59B
  \l 2B59C
  \l 2B59D
  \l 2B59E
  \l 2B59F
  \l 2B5A0
  \l 2B5A1
  \l 2B5A2
  \l 2B5A3
  \l 2B5A4
  \l 2B5A5
  \l 2B5A6
  \l 2B5A7
  \l 2B5A8
  \l 2B5A9
  \l 2B5AA
  \l 2B5AB
  \l 2B5AC
  \l 2B5AD
  \l 2B5AE
  \l 2B5AF
  \l 2B5B0
  \l 2B5B1
  \l 2B5B2
  \l 2B5B3
  \l 2B5B4
  \l 2B5B5
  \l 2B5B6
  \l 2B5B7
  \l 2B5B8
  \l 2B5B9
  \l 2B5BA
  \l 2B5BB
  \l 2B5BC
  \l 2B5BD
  \l 2B5BE
  \l 2B5BF
  \l 2B5C0
  \l 2B5C1
  \l 2B5C2
  \l 2B5C3
  \l 2B5C4
  \l 2B5C5
  \l 2B5C6
  \l 2B5C7
  \l 2B5C8
  \l 2B5C9
  \l 2B5CA
  \l 2B5CB
  \l 2B5CC
  \l 2B5CD
  \l 2B5CE
  \l 2B5CF
  \l 2B5D0
  \l 2B5D1
  \l 2B5D2
  \l 2B5D3
  \l 2B5D4
  \l 2B5D5
  \l 2B5D6
  \l 2B5D7
  \l 2B5D8
  \l 2B5D9
  \l 2B5DA
  \l 2B5DB
  \l 2B5DC
  \l 2B5DD
  \l 2B5DE
  \l 2B5DF
  \l 2B5E0
  \l 2B5E1
  \l 2B5E2
  \l 2B5E3
  \l 2B5E4
  \l 2B5E5
  \l 2B5E6
  \l 2B5E7
  \l 2B5E8
  \l 2B5E9
  \l 2B5EA
  \l 2B5EB
  \l 2B5EC
  \l 2B5ED
  \l 2B5EE
  \l 2B5EF
  \l 2B5F0
  \l 2B5F1
  \l 2B5F2
  \l 2B5F3
  \l 2B5F4
  \l 2B5F5
  \l 2B5F6
  \l 2B5F7
  \l 2B5F8
  \l 2B5F9
  \l 2B5FA
  \l 2B5FB
  \l 2B5FC
  \l 2B5FD
  \l 2B5FE
  \l 2B5FF
  \l 2B600
  \l 2B601
  \l 2B602
  \l 2B603
  \l 2B604
  \l 2B605
  \l 2B606
  \l 2B607
  \l 2B608
  \l 2B609
  \l 2B60A
  \l 2B60B
  \l 2B60C
  \l 2B60D
  \l 2B60E
  \l 2B60F
  \l 2B610
  \l 2B611
  \l 2B612
  \l 2B613
  \l 2B614
  \l 2B615
  \l 2B616
  \l 2B617
  \l 2B618
  \l 2B619
  \l 2B61A
  \l 2B61B
  \l 2B61C
  \l 2B61D
  \l 2B61E
  \l 2B61F
  \l 2B620
  \l 2B621
  \l 2B622
  \l 2B623
  \l 2B624
  \l 2B625
  \l 2B626
  \l 2B627
  \l 2B628
  \l 2B629
  \l 2B62A
  \l 2B62B
  \l 2B62C
  \l 2B62D
  \l 2B62E
  \l 2B62F
  \l 2B630
  \l 2B631
  \l 2B632
  \l 2B633
  \l 2B634
  \l 2B635
  \l 2B636
  \l 2B637
  \l 2B638
  \l 2B639
  \l 2B63A
  \l 2B63B
  \l 2B63C
  \l 2B63D
  \l 2B63E
  \l 2B63F
  \l 2B640
  \l 2B641
  \l 2B642
  \l 2B643
  \l 2B644
  \l 2B645
  \l 2B646
  \l 2B647
  \l 2B648
  \l 2B649
  \l 2B64A
  \l 2B64B
  \l 2B64C
  \l 2B64D
  \l 2B64E
  \l 2B64F
  \l 2B650
  \l 2B651
  \l 2B652
  \l 2B653
  \l 2B654
  \l 2B655
  \l 2B656
  \l 2B657
  \l 2B658
  \l 2B659
  \l 2B65A
  \l 2B65B
  \l 2B65C
  \l 2B65D
  \l 2B65E
  \l 2B65F
  \l 2B660
  \l 2B661
  \l 2B662
  \l 2B663
  \l 2B664
  \l 2B665
  \l 2B666
  \l 2B667
  \l 2B668
  \l 2B669
  \l 2B66A
  \l 2B66B
  \l 2B66C
  \l 2B66D
  \l 2B66E
  \l 2B66F
  \l 2B670
  \l 2B671
  \l 2B672
  \l 2B673
  \l 2B674
  \l 2B675
  \l 2B676
  \l 2B677
  \l 2B678
  \l 2B679
  \l 2B67A
  \l 2B67B
  \l 2B67C
  \l 2B67D
  \l 2B67E
  \l 2B67F
  \l 2B680
  \l 2B681
  \l 2B682
  \l 2B683
  \l 2B684
  \l 2B685
  \l 2B686
  \l 2B687
  \l 2B688
  \l 2B689
  \l 2B68A
  \l 2B68B
  \l 2B68C
  \l 2B68D
  \l 2B68E
  \l 2B68F
  \l 2B690
  \l 2B691
  \l 2B692
  \l 2B693
  \l 2B694
  \l 2B695
  \l 2B696
  \l 2B697
  \l 2B698
  \l 2B699
  \l 2B69A
  \l 2B69B
  \l 2B69C
  \l 2B69D
  \l 2B69E
  \l 2B69F
  \l 2B6A0
  \l 2B6A1
  \l 2B6A2
  \l 2B6A3
  \l 2B6A4
  \l 2B6A5
  \l 2B6A6
  \l 2B6A7
  \l 2B6A8
  \l 2B6A9
  \l 2B6AA
  \l 2B6AB
  \l 2B6AC
  \l 2B6AD
  \l 2B6AE
  \l 2B6AF
  \l 2B6B0
  \l 2B6B1
  \l 2B6B2
  \l 2B6B3
  \l 2B6B4
  \l 2B6B5
  \l 2B6B6
  \l 2B6B7
  \l 2B6B8
  \l 2B6B9
  \l 2B6BA
  \l 2B6BB
  \l 2B6BC
  \l 2B6BD
  \l 2B6BE
  \l 2B6BF
  \l 2B6C0
  \l 2B6C1
  \l 2B6C2
  \l 2B6C3
  \l 2B6C4
  \l 2B6C5
  \l 2B6C6
  \l 2B6C7
  \l 2B6C8
  \l 2B6C9
  \l 2B6CA
  \l 2B6CB
  \l 2B6CC
  \l 2B6CD
  \l 2B6CE
  \l 2B6CF
  \l 2B6D0
  \l 2B6D1
  \l 2B6D2
  \l 2B6D3
  \l 2B6D4
  \l 2B6D5
  \l 2B6D6
  \l 2B6D7
  \l 2B6D8
  \l 2B6D9
  \l 2B6DA
  \l 2B6DB
  \l 2B6DC
  \l 2B6DD
  \l 2B6DE
  \l 2B6DF
  \l 2B6E0
  \l 2B6E1
  \l 2B6E2
  \l 2B6E3
  \l 2B6E4
  \l 2B6E5
  \l 2B6E6
  \l 2B6E7
  \l 2B6E8
  \l 2B6E9
  \l 2B6EA
  \l 2B6EB
  \l 2B6EC
  \l 2B6ED
  \l 2B6EE
  \l 2B6EF
  \l 2B6F0
  \l 2B6F1
  \l 2B6F2
  \l 2B6F3
  \l 2B6F4
  \l 2B6F5
  \l 2B6F6
  \l 2B6F7
  \l 2B6F8
  \l 2B6F9
  \l 2B6FA
  \l 2B6FB
  \l 2B6FC
  \l 2B6FD
  \l 2B6FE
  \l 2B6FF
  \l 2B700
  \l 2B701
  \l 2B702
  \l 2B703
  \l 2B704
  \l 2B705
  \l 2B706
  \l 2B707
  \l 2B708
  \l 2B709
  \l 2B70A
  \l 2B70B
  \l 2B70C
  \l 2B70D
  \l 2B70E
  \l 2B70F
  \l 2B710
  \l 2B711
  \l 2B712
  \l 2B713
  \l 2B714
  \l 2B715
  \l 2B716
  \l 2B717
  \l 2B718
  \l 2B719
  \l 2B71A
  \l 2B71B
  \l 2B71C
  \l 2B71D
  \l 2B71E
  \l 2B71F
  \l 2B720
  \l 2B721
  \l 2B722
  \l 2B723
  \l 2B724
  \l 2B725
  \l 2B726
  \l 2B727
  \l 2B728
  \l 2B729
  \l 2B72A
  \l 2B72B
  \l 2B72C
  \l 2B72D
  \l 2B72E
  \l 2B72F
  \l 2B730
  \l 2B731
  \l 2B732
  \l 2B733
  \l 2B734
  \l 2B740
  \l 2B741
  \l 2B742
  \l 2B743
  \l 2B744
  \l 2B745
  \l 2B746
  \l 2B747
  \l 2B748
  \l 2B749
  \l 2B74A
  \l 2B74B
  \l 2B74C
  \l 2B74D
  \l 2B74E
  \l 2B74F
  \l 2B750
  \l 2B751
  \l 2B752
  \l 2B753
  \l 2B754
  \l 2B755
  \l 2B756
  \l 2B757
  \l 2B758
  \l 2B759
  \l 2B75A
  \l 2B75B
  \l 2B75C
  \l 2B75D
  \l 2B75E
  \l 2B75F
  \l 2B760
  \l 2B761
  \l 2B762
  \l 2B763
  \l 2B764
  \l 2B765
  \l 2B766
  \l 2B767
  \l 2B768
  \l 2B769
  \l 2B76A
  \l 2B76B
  \l 2B76C
  \l 2B76D
  \l 2B76E
  \l 2B76F
  \l 2B770
  \l 2B771
  \l 2B772
  \l 2B773
  \l 2B774
  \l 2B775
  \l 2B776
  \l 2B777
  \l 2B778
  \l 2B779
  \l 2B77A
  \l 2B77B
  \l 2B77C
  \l 2B77D
  \l 2B77E
  \l 2B77F
  \l 2B780
  \l 2B781
  \l 2B782
  \l 2B783
  \l 2B784
  \l 2B785
  \l 2B786
  \l 2B787
  \l 2B788
  \l 2B789
  \l 2B78A
  \l 2B78B
  \l 2B78C
  \l 2B78D
  \l 2B78E
  \l 2B78F
  \l 2B790
  \l 2B791
  \l 2B792
  \l 2B793
  \l 2B794
  \l 2B795
  \l 2B796
  \l 2B797
  \l 2B798
  \l 2B799
  \l 2B79A
  \l 2B79B
  \l 2B79C
  \l 2B79D
  \l 2B79E
  \l 2B79F
  \l 2B7A0
  \l 2B7A1
  \l 2B7A2
  \l 2B7A3
  \l 2B7A4
  \l 2B7A5
  \l 2B7A6
  \l 2B7A7
  \l 2B7A8
  \l 2B7A9
  \l 2B7AA
  \l 2B7AB
  \l 2B7AC
  \l 2B7AD
  \l 2B7AE
  \l 2B7AF
  \l 2B7B0
  \l 2B7B1
  \l 2B7B2
  \l 2B7B3
  \l 2B7B4
  \l 2B7B5
  \l 2B7B6
  \l 2B7B7
  \l 2B7B8
  \l 2B7B9
  \l 2B7BA
  \l 2B7BB
  \l 2B7BC
  \l 2B7BD
  \l 2B7BE
  \l 2B7BF
  \l 2B7C0
  \l 2B7C1
  \l 2B7C2
  \l 2B7C3
  \l 2B7C4
  \l 2B7C5
  \l 2B7C6
  \l 2B7C7
  \l 2B7C8
  \l 2B7C9
  \l 2B7CA
  \l 2B7CB
  \l 2B7CC
  \l 2B7CD
  \l 2B7CE
  \l 2B7CF
  \l 2B7D0
  \l 2B7D1
  \l 2B7D2
  \l 2B7D3
  \l 2B7D4
  \l 2B7D5
  \l 2B7D6
  \l 2B7D7
  \l 2B7D8
  \l 2B7D9
  \l 2B7DA
  \l 2B7DB
  \l 2B7DC
  \l 2B7DD
  \l 2B7DE
  \l 2B7DF
  \l 2B7E0
  \l 2B7E1
  \l 2B7E2
  \l 2B7E3
  \l 2B7E4
  \l 2B7E5
  \l 2B7E6
  \l 2B7E7
  \l 2B7E8
  \l 2B7E9
  \l 2B7EA
  \l 2B7EB
  \l 2B7EC
  \l 2B7ED
  \l 2B7EE
  \l 2B7EF
  \l 2B7F0
  \l 2B7F1
  \l 2B7F2
  \l 2B7F3
  \l 2B7F4
  \l 2B7F5
  \l 2B7F6
  \l 2B7F7
  \l 2B7F8
  \l 2B7F9
  \l 2B7FA
  \l 2B7FB
  \l 2B7FC
  \l 2B7FD
  \l 2B7FE
  \l 2B7FF
  \l 2B800
  \l 2B801
  \l 2B802
  \l 2B803
  \l 2B804
  \l 2B805
  \l 2B806
  \l 2B807
  \l 2B808
  \l 2B809
  \l 2B80A
  \l 2B80B
  \l 2B80C
  \l 2B80D
  \l 2B80E
  \l 2B80F
  \l 2B810
  \l 2B811
  \l 2B812
  \l 2B813
  \l 2B814
  \l 2B815
  \l 2B816
  \l 2B817
  \l 2B818
  \l 2B819
  \l 2B81A
  \l 2B81B
  \l 2B81C
  \l 2B81D
  \l 2F800
  \l 2F801
  \l 2F802
  \l 2F803
  \l 2F804
  \l 2F805
  \l 2F806
  \l 2F807
  \l 2F808
  \l 2F809
  \l 2F80A
  \l 2F80B
  \l 2F80C
  \l 2F80D
  \l 2F80E
  \l 2F80F
  \l 2F810
  \l 2F811
  \l 2F812
  \l 2F813
  \l 2F814
  \l 2F815
  \l 2F816
  \l 2F817
  \l 2F818
  \l 2F819
  \l 2F81A
  \l 2F81B
  \l 2F81C
  \l 2F81D
  \l 2F81E
  \l 2F81F
  \l 2F820
  \l 2F821
  \l 2F822
  \l 2F823
  \l 2F824
  \l 2F825
  \l 2F826
  \l 2F827
  \l 2F828
  \l 2F829
  \l 2F82A
  \l 2F82B
  \l 2F82C
  \l 2F82D
  \l 2F82E
  \l 2F82F
  \l 2F830
  \l 2F831
  \l 2F832
  \l 2F833
  \l 2F834
  \l 2F835
  \l 2F836
  \l 2F837
  \l 2F838
  \l 2F839
  \l 2F83A
  \l 2F83B
  \l 2F83C
  \l 2F83D
  \l 2F83E
  \l 2F83F
  \l 2F840
  \l 2F841
  \l 2F842
  \l 2F843
  \l 2F844
  \l 2F845
  \l 2F846
  \l 2F847
  \l 2F848
  \l 2F849
  \l 2F84A
  \l 2F84B
  \l 2F84C
  \l 2F84D
  \l 2F84E
  \l 2F84F
  \l 2F850
  \l 2F851
  \l 2F852
  \l 2F853
  \l 2F854
  \l 2F855
  \l 2F856
  \l 2F857
  \l 2F858
  \l 2F859
  \l 2F85A
  \l 2F85B
  \l 2F85C
  \l 2F85D
  \l 2F85E
  \l 2F85F
  \l 2F860
  \l 2F861
  \l 2F862
  \l 2F863
  \l 2F864
  \l 2F865
  \l 2F866
  \l 2F867
  \l 2F868
  \l 2F869
  \l 2F86A
  \l 2F86B
  \l 2F86C
  \l 2F86D
  \l 2F86E
  \l 2F86F
  \l 2F870
  \l 2F871
  \l 2F872
  \l 2F873
  \l 2F874
  \l 2F875
  \l 2F876
  \l 2F877
  \l 2F878
  \l 2F879
  \l 2F87A
  \l 2F87B
  \l 2F87C
  \l 2F87D
  \l 2F87E
  \l 2F87F
  \l 2F880
  \l 2F881
  \l 2F882
  \l 2F883
  \l 2F884
  \l 2F885
  \l 2F886
  \l 2F887
  \l 2F888
  \l 2F889
  \l 2F88A
  \l 2F88B
  \l 2F88C
  \l 2F88D
  \l 2F88E
  \l 2F88F
  \l 2F890
  \l 2F891
  \l 2F892
  \l 2F893
  \l 2F894
  \l 2F895
  \l 2F896
  \l 2F897
  \l 2F898
  \l 2F899
  \l 2F89A
  \l 2F89B
  \l 2F89C
  \l 2F89D
  \l 2F89E
  \l 2F89F
  \l 2F8A0
  \l 2F8A1
  \l 2F8A2
  \l 2F8A3
  \l 2F8A4
  \l 2F8A5
  \l 2F8A6
  \l 2F8A7
  \l 2F8A8
  \l 2F8A9
  \l 2F8AA
  \l 2F8AB
  \l 2F8AC
  \l 2F8AD
  \l 2F8AE
  \l 2F8AF
  \l 2F8B0
  \l 2F8B1
  \l 2F8B2
  \l 2F8B3
  \l 2F8B4
  \l 2F8B5
  \l 2F8B6
  \l 2F8B7
  \l 2F8B8
  \l 2F8B9
  \l 2F8BA
  \l 2F8BB
  \l 2F8BC
  \l 2F8BD
  \l 2F8BE
  \l 2F8BF
  \l 2F8C0
  \l 2F8C1
  \l 2F8C2
  \l 2F8C3
  \l 2F8C4
  \l 2F8C5
  \l 2F8C6
  \l 2F8C7
  \l 2F8C8
  \l 2F8C9
  \l 2F8CA
  \l 2F8CB
  \l 2F8CC
  \l 2F8CD
  \l 2F8CE
  \l 2F8CF
  \l 2F8D0
  \l 2F8D1
  \l 2F8D2
  \l 2F8D3
  \l 2F8D4
  \l 2F8D5
  \l 2F8D6
  \l 2F8D7
  \l 2F8D8
  \l 2F8D9
  \l 2F8DA
  \l 2F8DB
  \l 2F8DC
  \l 2F8DD
  \l 2F8DE
  \l 2F8DF
  \l 2F8E0
  \l 2F8E1
  \l 2F8E2
  \l 2F8E3
  \l 2F8E4
  \l 2F8E5
  \l 2F8E6
  \l 2F8E7
  \l 2F8E8
  \l 2F8E9
  \l 2F8EA
  \l 2F8EB
  \l 2F8EC
  \l 2F8ED
  \l 2F8EE
  \l 2F8EF
  \l 2F8F0
  \l 2F8F1
  \l 2F8F2
  \l 2F8F3
  \l 2F8F4
  \l 2F8F5
  \l 2F8F6
  \l 2F8F7
  \l 2F8F8
  \l 2F8F9
  \l 2F8FA
  \l 2F8FB
  \l 2F8FC
  \l 2F8FD
  \l 2F8FE
  \l 2F8FF
  \l 2F900
  \l 2F901
  \l 2F902
  \l 2F903
  \l 2F904
  \l 2F905
  \l 2F906
  \l 2F907
  \l 2F908
  \l 2F909
  \l 2F90A
  \l 2F90B
  \l 2F90C
  \l 2F90D
  \l 2F90E
  \l 2F90F
  \l 2F910
  \l 2F911
  \l 2F912
  \l 2F913
  \l 2F914
  \l 2F915
  \l 2F916
  \l 2F917
  \l 2F918
  \l 2F919
  \l 2F91A
  \l 2F91B
  \l 2F91C
  \l 2F91D
  \l 2F91E
  \l 2F91F
  \l 2F920
  \l 2F921
  \l 2F922
  \l 2F923
  \l 2F924
  \l 2F925
  \l 2F926
  \l 2F927
  \l 2F928
  \l 2F929
  \l 2F92A
  \l 2F92B
  \l 2F92C
  \l 2F92D
  \l 2F92E
  \l 2F92F
  \l 2F930
  \l 2F931
  \l 2F932
  \l 2F933
  \l 2F934
  \l 2F935
  \l 2F936
  \l 2F937
  \l 2F938
  \l 2F939
  \l 2F93A
  \l 2F93B
  \l 2F93C
  \l 2F93D
  \l 2F93E
  \l 2F93F
  \l 2F940
  \l 2F941
  \l 2F942
  \l 2F943
  \l 2F944
  \l 2F945
  \l 2F946
  \l 2F947
  \l 2F948
  \l 2F949
  \l 2F94A
  \l 2F94B
  \l 2F94C
  \l 2F94D
  \l 2F94E
  \l 2F94F
  \l 2F950
  \l 2F951
  \l 2F952
  \l 2F953
  \l 2F954
  \l 2F955
  \l 2F956
  \l 2F957
  \l 2F958
  \l 2F959
  \l 2F95A
  \l 2F95B
  \l 2F95C
  \l 2F95D
  \l 2F95E
  \l 2F95F
  \l 2F960
  \l 2F961
  \l 2F962
  \l 2F963
  \l 2F964
  \l 2F965
  \l 2F966
  \l 2F967
  \l 2F968
  \l 2F969
  \l 2F96A
  \l 2F96B
  \l 2F96C
  \l 2F96D
  \l 2F96E
  \l 2F96F
  \l 2F970
  \l 2F971
  \l 2F972
  \l 2F973
  \l 2F974
  \l 2F975
  \l 2F976
  \l 2F977
  \l 2F978
  \l 2F979
  \l 2F97A
  \l 2F97B
  \l 2F97C
  \l 2F97D
  \l 2F97E
  \l 2F97F
  \l 2F980
  \l 2F981
  \l 2F982
  \l 2F983
  \l 2F984
  \l 2F985
  \l 2F986
  \l 2F987
  \l 2F988
  \l 2F989
  \l 2F98A
  \l 2F98B
  \l 2F98C
  \l 2F98D
  \l 2F98E
  \l 2F98F
  \l 2F990
  \l 2F991
  \l 2F992
  \l 2F993
  \l 2F994
  \l 2F995
  \l 2F996
  \l 2F997
  \l 2F998
  \l 2F999
  \l 2F99A
  \l 2F99B
  \l 2F99C
  \l 2F99D
  \l 2F99E
  \l 2F99F
  \l 2F9A0
  \l 2F9A1
  \l 2F9A2
  \l 2F9A3
  \l 2F9A4
  \l 2F9A5
  \l 2F9A6
  \l 2F9A7
  \l 2F9A8
  \l 2F9A9
  \l 2F9AA
  \l 2F9AB
  \l 2F9AC
  \l 2F9AD
  \l 2F9AE
  \l 2F9AF
  \l 2F9B0
  \l 2F9B1
  \l 2F9B2
  \l 2F9B3
  \l 2F9B4
  \l 2F9B5
  \l 2F9B6
  \l 2F9B7
  \l 2F9B8
  \l 2F9B9
  \l 2F9BA
  \l 2F9BB
  \l 2F9BC
  \l 2F9BD
  \l 2F9BE
  \l 2F9BF
  \l 2F9C0
  \l 2F9C1
  \l 2F9C2
  \l 2F9C3
  \l 2F9C4
  \l 2F9C5
  \l 2F9C6
  \l 2F9C7
  \l 2F9C8
  \l 2F9C9
  \l 2F9CA
  \l 2F9CB
  \l 2F9CC
  \l 2F9CD
  \l 2F9CE
  \l 2F9CF
  \l 2F9D0
  \l 2F9D1
  \l 2F9D2
  \l 2F9D3
  \l 2F9D4
  \l 2F9D5
  \l 2F9D6
  \l 2F9D7
  \l 2F9D8
  \l 2F9D9
  \l 2F9DA
  \l 2F9DB
  \l 2F9DC
  \l 2F9DD
  \l 2F9DE
  \l 2F9DF
  \l 2F9E0
  \l 2F9E1
  \l 2F9E2
  \l 2F9E3
  \l 2F9E4
  \l 2F9E5
  \l 2F9E6
  \l 2F9E7
  \l 2F9E8
  \l 2F9E9
  \l 2F9EA
  \l 2F9EB
  \l 2F9EC
  \l 2F9ED
  \l 2F9EE
  \l 2F9EF
  \l 2F9F0
  \l 2F9F1
  \l 2F9F2
  \l 2F9F3
  \l 2F9F4
  \l 2F9F5
  \l 2F9F6
  \l 2F9F7
  \l 2F9F8
  \l 2F9F9
  \l 2F9FA
  \l 2F9FB
  \l 2F9FC
  \l 2F9FD
  \l 2F9FE
  \l 2F9FF
  \l 2FA00
  \l 2FA01
  \l 2FA02
  \l 2FA03
  \l 2FA04
  \l 2FA05
  \l 2FA06
  \l 2FA07
  \l 2FA08
  \l 2FA09
  \l 2FA0A
  \l 2FA0B
  \l 2FA0C
  \l 2FA0D
  \l 2FA0E
  \l 2FA0F
  \l 2FA10
  \l 2FA11
  \l 2FA12
  \l 2FA13
  \l 2FA14
  \l 2FA15
  \l 2FA16
  \l 2FA17
  \l 2FA18
  \l 2FA19
  \l 2FA1A
  \l 2FA1B
  \l 2FA1C
  \l 2FA1D
  \l E0100
  \l E0101
  \l E0102
  \l E0103
  \l E0104
  \l E0105
  \l E0106
  \l E0107
  \l E0108
  \l E0109
  \l E010A
  \l E010B
  \l E010C
  \l E010D
  \l E010E
  \l E010F
  \l E0110
  \l E0111
  \l E0112
  \l E0113
  \l E0114
  \l E0115
  \l E0116
  \l E0117
  \l E0118
  \l E0119
  \l E011A
  \l E011B
  \l E011C
  \l E011D
  \l E011E
  \l E011F
  \l E0120
  \l E0121
  \l E0122
  \l E0123
  \l E0124
  \l E0125
  \l E0126
  \l E0127
  \l E0128
  \l E0129
  \l E012A
  \l E012B
  \l E012C
  \l E012D
  \l E012E
  \l E012F
  \l E0130
  \l E0131
  \l E0132
  \l E0133
  \l E0134
  \l E0135
  \l E0136
  \l E0137
  \l E0138
  \l E0139
  \l E013A
  \l E013B
  \l E013C
  \l E013D
  \l E013E
  \l E013F
  \l E0140
  \l E0141
  \l E0142
  \l E0143
  \l E0144
  \l E0145
  \l E0146
  \l E0147
  \l E0148
  \l E0149
  \l E014A
  \l E014B
  \l E014C
  \l E014D
  \l E014E
  \l E014F
  \l E0150
  \l E0151
  \l E0152
  \l E0153
  \l E0154
  \l E0155
  \l E0156
  \l E0157
  \l E0158
  \l E0159
  \l E015A
  \l E015B
  \l E015C
  \l E015D
  \l E015E
  \l E015F
  \l E0160
  \l E0161
  \l E0162
  \l E0163
  \l E0164
  \l E0165
  \l E0166
  \l E0167
  \l E0168
  \l E0169
  \l E016A
  \l E016B
  \l E016C
  \l E016D
  \l E016E
  \l E016F
  \l E0170
  \l E0171
  \l E0172
  \l E0173
  \l E0174
  \l E0175
  \l E0176
  \l E0177
  \l E0178
  \l E0179
  \l E017A
  \l E017B
  \l E017C
  \l E017D
  \l E017E
  \l E017F
  \l E0180
  \l E0181
  \l E0182
  \l E0183
  \l E0184
  \l E0185
  \l E0186
  \l E0187
  \l E0188
  \l E0189
  \l E018A
  \l E018B
  \l E018C
  \l E018D
  \l E018E
  \l E018F
  \l E0190
  \l E0191
  \l E0192
  \l E0193
  \l E0194
  \l E0195
  \l E0196
  \l E0197
  \l E0198
  \l E0199
  \l E019A
  \l E019B
  \l E019C
  \l E019D
  \l E019E
  \l E019F
  \l E01A0
  \l E01A1
  \l E01A2
  \l E01A3
  \l E01A4
  \l E01A5
  \l E01A6
  \l E01A7
  \l E01A8
  \l E01A9
  \l E01AA
  \l E01AB
  \l E01AC
  \l E01AD
  \l E01AE
  \l E01AF
  \l E01B0
  \l E01B1
  \l E01B2
  \l E01B3
  \l E01B4
  \l E01B5
  \l E01B6
  \l E01B7
  \l E01B8
  \l E01B9
  \l E01BA
  \l E01BB
  \l E01BC
  \l E01BD
  \l E01BE
  \l E01BF
  \l E01C0
  \l E01C1
  \l E01C2
  \l E01C3
  \l E01C4
  \l E01C5
  \l E01C6
  \l E01C7
  \l E01C8
  \l E01C9
  \l E01CA
  \l E01CB
  \l E01CC
  \l E01CD
  \l E01CE
  \l E01CF
  \l E01D0
  \l E01D1
  \l E01D2
  \l E01D3
  \l E01D4
  \l E01D5
  \l E01D6
  \l E01D7
  \l E01D8
  \l E01D9
  \l E01DA
  \l E01DB
  \l E01DC
  \l E01DD
  \l E01DE
  \l E01DF
  \l E01E0
  \l E01E1
  \l E01E2
  \l E01E3
  \l E01E4
  \l E01E5
  \l E01E6
  \l E01E7
  \l E01E8
  \l E01E9
  \l E01EA
  \l E01EB
  \l E01EC
  \l E01ED
  \l E01EE
  \l E01EF
\endgroup
\global\sfcode"2019=0 %
\global\sfcode"201D=0 %
\begingroup
  \ifx\XeTeXchartoks\XeTeXcharclass
    \endgroup\expandafter\endinput
  \else
    \def\setclass#1#2#3{%
      \ifnum#1>#2 %
        \expandafter\gobble
      \else
        \expandafter\firstofone
      \fi
        {%
          \global\XeTeXcharclass#1=#3 %
          \expandafter\setclass\expandafter
            {\number\numexpr#1+1\relax}{#2}{#3}%
        }%
    }%
    \def\gobble#1{}
    \def\firstofone#1{#1}
    \def\ID#1 #2 {\setclass{"#1}{"#2}{1}}
    \def\OP#1 {\setclass{"#1}{"#1}{2}}
    \def\CL#1 {\setclass{"#1}{"#1}{3}}
    \def\EX#1 {\setclass{"#1}{"#1}{3}}
    \def\IS#1 {\setclass{"#1}{"#1}{3}}
    \def\NS#1 {\setclass{"#1}{"#1}{3}}
    \def\CM#1 {\setclass{"#1}{"#1}{256}}
  \fi
\ID 231A 231B
  \OP 2329
  \CL 232A
\ID 23F0 23F3
\ID 2600 2603
\ID 2614 2615
\ID 2618 2618
\ID 261A 261F
\ID 2639 263B
\ID 2668 2668
\ID 267F 267F
\ID 26BD 26C8
\ID 26CD 26CD
\ID 26CF 26D1
\ID 26D3 26D4
\ID 26D8 26D9
\ID 26DC 26DC
\ID 26DF 26E1
\ID 26EA 26EA
\ID 26F1 26F5
\ID 26F7 26FA
\ID 26FD 26FF
\ID 2700 2704
\ID 2708 270D
\ID 2E80 2E99
\ID 2E9B 2EF3
\ID 2F00 2FD5
\ID 2FF0 2FFB
  \CL 3001
  \CL 3002
\ID 3003 3003
\ID 3004 3004
  \NS 3005
\ID 3006 3006
\ID 3007 3007
  \OP 3008
  \CL 3009
  \OP 300A
  \CL 300B
  \OP 300C
  \CL 300D
  \OP 300E
  \CL 300F
  \OP 3010
  \CL 3011
\ID 3012 3013
  \OP 3014
  \CL 3015
  \OP 3016
  \CL 3017
  \OP 3018
  \CL 3019
  \OP 301A
  \CL 301B
  \NS 301C
  \OP 301D
  \CL 301E
  \CL 301F
\ID 3020 3020
\ID 3021 3029
  \CM 302A
  \CM 302B
  \CM 302C
  \CM 302D
  \CM 302E
  \CM 302F
\ID 3030 3030
\ID 3031 3034
  \CM 3035
\ID 3036 3037
\ID 3038 303A
  \NS 303B
  \NS 303C
\ID 303D 303D
\ID 303E 303F
\ID 3042 3042
\ID 3044 3044
\ID 3046 3046
\ID 3048 3048
\ID 304A 3062
\ID 3064 3082
\ID 3084 3084
\ID 3086 3086
\ID 3088 308D
\ID 308F 3094
  \CM 3099
  \CM 309A
  \NS 309B
  \NS 309C
  \NS 309D
  \NS 309E
\ID 309F 309F
  \NS 30A0
\ID 30A2 30A2
\ID 30A4 30A4
\ID 30A6 30A6
\ID 30A8 30A8
\ID 30AA 30C2
\ID 30C4 30E2
\ID 30E4 30E4
\ID 30E6 30E6
\ID 30E8 30ED
\ID 30EF 30F4
\ID 30F7 30FA
  \NS 30FB
  \NS 30FD
  \NS 30FE
\ID 30FF 30FF
\ID 3105 312D
\ID 3131 318E
\ID 3190 3191
\ID 3192 3195
\ID 3196 319F
\ID 31A0 31BA
\ID 31C0 31E3
\ID 3200 321E
\ID 3220 3229
\ID 322A 3247
\ID 3250 3250
\ID 3251 325F
\ID 3260 327F
\ID 3280 3289
\ID 328A 32B0
\ID 32B1 32BF
\ID 32C0 32FE
\ID 3300 33FF
\ID 3400 4DB5
\ID 4DB6 4DBF
\ID 4E00 9FCC
\ID 9FCD 9FFF
\ID A000 A014
  \NS A015
\ID A016 A48C
\ID A490 A4C6
\ID F900 FA6D
\ID FA6E FA6F
\ID FA70 FAD9
\ID FADA FAFF
  \IS FE10
  \CL FE11
  \CL FE12
  \IS FE13
  \IS FE14
  \EX FE15
  \EX FE16
  \OP FE17
  \CL FE18
\ID FE30 FE30
\ID FE31 FE32
\ID FE33 FE34
  \OP FE35
  \CL FE36
  \OP FE37
  \CL FE38
  \OP FE39
  \CL FE3A
  \OP FE3B
  \CL FE3C
  \OP FE3D
  \CL FE3E
  \OP FE3F
  \CL FE40
  \OP FE41
  \CL FE42
  \OP FE43
  \CL FE44
\ID FE45 FE46
  \OP FE47
  \CL FE48
\ID FE49 FE4C
\ID FE4D FE4F
  \CL FE50
\ID FE51 FE51
  \CL FE52
  \NS FE54
  \NS FE55
  \EX FE56
  \EX FE57
\ID FE58 FE58
  \OP FE59
  \CL FE5A
  \OP FE5B
  \CL FE5C
  \OP FE5D
  \CL FE5E
\ID FE5F FE61
\ID FE62 FE62
\ID FE63 FE63
\ID FE64 FE66
\ID FE68 FE68
\ID FE6B FE6B
  \EX FF01
\ID FF02 FF03
\ID FF06 FF07
  \OP FF08
  \CL FF09
\ID FF0A FF0A
\ID FF0B FF0B
  \CL FF0C
\ID FF0D FF0D
  \CL FF0E
\ID FF0F FF0F
\ID FF10 FF19
  \NS FF1A
  \NS FF1B
\ID FF1C FF1E
  \EX FF1F
\ID FF20 FF20
\ID FF21 FF3A
  \OP FF3B
\ID FF3C FF3C
  \CL FF3D
\ID FF3E FF3E
\ID FF3F FF3F
\ID FF40 FF40
\ID FF41 FF5A
  \OP FF5B
\ID FF5C FF5C
  \CL FF5D
\ID FF5E FF5E
  \OP FF5F
  \CL FF60
  \CL FF61
  \OP FF62
  \CL FF63
  \CL FF64
  \NS FF65
  \NS FF9E
  \NS FF9F
\ID FFE2 FFE2
\ID FFE3 FFE3
\ID FFE4 FFE4
\ID 1B000 1B001
\ID 1F000 1F02B
\ID 1F030 1F093
\ID 1F0A0 1F0AE
\ID 1F0B1 1F0BF
\ID 1F0C1 1F0CF
\ID 1F0D1 1F0F5
\ID 1F200 1F202
\ID 1F210 1F23A
\ID 1F240 1F248
\ID 1F250 1F251
\ID 1F300 1F32C
\ID 1F330 1F37D
\ID 1F380 1F39B
\ID 1F39E 1F3B4
\ID 1F3B7 1F3BB
\ID 1F3BD 1F3CE
\ID 1F3D4 1F3F7
\ID 1F400 1F49F
\ID 1F4A1 1F4A1
\ID 1F4A3 1F4A3
\ID 1F4A5 1F4AE
\ID 1F4B0 1F4B0
\ID 1F4B3 1F4FE
\ID 1F507 1F516
\ID 1F525 1F531
\ID 1F54A 1F54A
\ID 1F550 1F579
\ID 1F57B 1F5A3
\ID 1F5A5 1F5D3
\ID 1F5DC 1F5F3
\ID 1F5FA 1F5FF
\ID 1F600 1F642
\ID 1F645 1F64F
\ID 1F680 1F6CF
\ID 1F6E0 1F6EC
\ID 1F6F0 1F6F3
\ID 20000 2A6D6
\ID 2A6D7 2A6FF
\ID 2A700 2B734
\ID 2B735 2B73F
\ID 2B740 2B81D
\ID 2B81E 2F7FF
\ID 2F800 2FA1D
\ID 2FA1E 2FFFD
\ID 30000 3FFFD
\endgroup
\gdef\xtxHanGlue{\hskip0pt plus 0.1em\relax}
\gdef\xtxHanSpace{\hskip0.2em plus 0.2em minus 0.1em\relax}
\global\XeTeXinterchartoks 0 1 = {\xtxHanSpace}
\global\XeTeXinterchartoks 0 2 = {\xtxHanSpace}
\global\XeTeXinterchartoks 0 3 = {\nobreak\xtxHanSpace}
\global\XeTeXinterchartoks 1 0 = {\xtxHanSpace}
\global\XeTeXinterchartoks 2 0 = {\nobreak\xtxHanSpace}
\global\XeTeXinterchartoks 3 0 = {\xtxHanSpace}
\global\XeTeXinterchartoks 1 1 = {\xtxHanGlue}
\global\XeTeXinterchartoks 1 2 = {\xtxHanGlue}
\global\XeTeXinterchartoks 1 3 = {\nobreak\xtxHanGlue}
\global\XeTeXinterchartoks 2 1 = {\nobreak\xtxHanGlue}
\global\XeTeXinterchartoks 2 2 = {\nobreak\xtxHanGlue}
\global\XeTeXinterchartoks 2 3 = {\xtxHanGlue}
\global\XeTeXinterchartoks 3 1 = {\xtxHanGlue}
\global\XeTeXinterchartoks 3 2 = {\xtxHanGlue}
\global\XeTeXinterchartoks 3 3 = {\nobreak\xtxHanGlue}
%
  \let\endgroup\ENDGROUP
  \@firstofone{%
    \catcode64=12 %
    \savecatcodetable\catcodetable@latex
    \catcode64=11 %
    \savecatcodetable\catcodetable@atletter
   }
\endgroup
%    \end{macrocode}
% \end{macro}
% \end{macro}
% \end{macro}
% \end{macro}
%
% \subsection{Named Lua functions}
%
% \begin{macro}{\newluafunction}
% \changes{v1.0a}{0000/00/00}{Macro added}
%   Much the same story for allocating Lua\TeX{} functions except here they are
%   just numbers so are allocated in the same way as boxes. Lua index from~$1$
%   so once again slot~$0$ is skipped.
%    \begin{macrocode}
\ifx\e@alloc@luafunction@count\@undefined
  \countdef\e@alloc@luafunction@count=260
\fi
\def\newluafunction{%
  \e@alloc\luafunction\e@alloc@chardef
    \e@alloc@luafunction@count\m@ne\e@alloc@top
}
\e@alloc@luafunction@count=\z@
%    \end{macrocode}
% \end{macro}
%
% \subsection{Custom whatsits}
%
% \begin{macro}{\newwhatsit}
% \changes{v1.0a}{0000/00/00}{Macro added}
%   These are only settable from Lua but for consistency are definable
%   here.
%    \begin{macrocode}
\ifx\e@alloc@whatsit@count\@undefined
  \countdef\e@alloc@whatsit@count=261
\fi
\def\newwhatsit#1{%
  \e@alloc\whatsit\e@alloc@chardef
    \e@alloc@whatsit@count\m@ne\e@alloc@top#1%
}
\e@alloc@whatsit@count=\z@
%    \end{macrocode}
% \end{macro}
%
% \subsection{Lua loader}
%
% Load the Lua code at the start of every job.
% For the conversion of \TeX{} into numbers at the Lua side we need some
% known registers: for convenience we use a set of systematic names, which
% means using a group around the Lua loader.
%    \begin{macrocode}
%<2ekernel>\everyjob\expandafter{%
%<2ekernel>  \the\everyjob
  \begingroup
    \attributedef\attributezero=0 %
    \chardef     \charzero     =0 %
%    \end{macrocode}
% Note name change required on older luatex, for hash table access.
%    \begin{macrocode}
    \countdef    \CountZero    =0 % 
    \dimendef    \dimenzero    =0 %
    \mathchardef \mathcharzero =0 %
    \muskipdef   \muskipzero   =0 %
    \skipdef     \skipzero     =0 %
    \toksdef     \tokszero     =0 %
    \directlua{require("ltluatex")}
  \endgroup
%<2ekernel>}
%<latexrelease>\EndIncludeInRelease
%    \end{macrocode}
%
%    \begin{macrocode}
%<latexrelease>\IncludeInRelease{0000/00/00}
%<latexrelease>                 {\newluafunction}{LuaTeX}%
%<latexrelease>\let\e@alloc@attribute@count\@undefined
%<latexrelease>\let\newattribute\@undefined
%<latexrelease>\let\setattribute\@undefined
%<latexrelease>\let\unsetattribute\@undefined
%<latexrelease>\let\e@alloc@ccodetable@count\@undefined
%<latexrelease>\let\newcatcodetable\@undefined
%<latexrelease>\let\catcodetable@initex\@undefined
%<latexrelease>\let\catcodetable@string\@undefined
%<latexrelease>\let\catcodetable@latex\@undefined
%<latexrelease>\let\catcodetable@atletter\@undefined
%<latexrelease>\let\e@alloc@luafunction@count\@undefined
%<latexrelease>\let\newluafunction\@undefined
%<latexrelease>\let\e@alloc@luafunction@count\@undefined
%<latexrelease>\let\newwhatsit\@undefined
%<latexrelease>\let\e@alloc@whatsit@count\@undefined
%<latexrelease>\EndIncludeInRelease
%    \end{macrocode}
%
%    \begin{macrocode}
%<2ekernel|latexrelease>\fi
%</2ekernel|tex|latexrelease>
%    \end{macrocode}
%
% \subsection{Lua module preliminaries}
%
% \begingroup
%
%  \begingroup\lccode`~=`_
%  \lowercase{\endgroup\let~}_
%  \catcode`_=12
%
%    \begin{macrocode}
%<*lua>
%    \end{macrocode}
%
% Some set up for the Lua module which is needed for all of the Lua
% functionality added here.
%
% \begin{macro}{luatexbase}
% \changes{v1.0a}{0000/00/00}{Table added}
%   Set up the table for the returned functions. This is used to expose
%   all of the public functions.
%    \begin{macrocode}
luatexbase       = luatexbase or { }
local luatexbase = luatexbase
%    \end{macrocode}
% \end{macro}
%
% Some Lua best practice: use local versions of functions where possible.
%    \begin{macrocode}
local string_gsub      = string.gsub
local tex_count        = tex.count
local tex_setattribute = tex.setattribute
local tex_setcount     = tex.setcount
local texio_write_nl   = texio.write_nl
%    \end{macrocode}
%
% \subsection{Lua module utilities}
%
% \subsubsection{Module tracking}
%
% \begin{macro}{luatexbase.modules}
% \changes{v1.0a}{0000/00/00}{Function modified}
%   To allow tracking of module usage, a structure is provided to store
%   information and to return it.
%    \begin{macrocode}
local modules = modules or { }
%    \end{macrocode}
% \end{macro}
%
% \begin{macro}{luatexbase.provides_module}
% \changes{v1.0a}{0000/00/00}{Function added}
%   Modelled on |\ProvidesPackage|, we store much the same information but
%   with a little more structure.
%    \begin{macrocode}
local function provides_module(info)
  if not (info and info.name) then
    luatexbase_error("Missing module name for provides_modules")
    return
  end
  local function spaced(text)
    return text and (" " .. text) or ""
  end
  texio_write_nl(
    "log",
    "Lua module: " .. info.name
      .. spaced(info.date)
      .. spaced(info.version)
      .. spaced(info.description)
  )
  modules[info.name] = info
end
luatexbase.provides_module = provides_module
%    \end{macrocode}
% \end{macro}
%
% \subsubsection{Module messages}
%
% There are various warnings and errors that need to be given. For warnings
% we can get exactly the same formatting as from \TeX{}. For errors we have to
% make some changes. Here we give the text of the error in the \LaTeX{} format
% then force an error from Lua to halt the run. Splitting the message text is
% done using |\n| which takes the place of |\MessageBreak|.
%
% First an auxiliary for the formatting: this measures up the message
% leader so we always get the correct indent.
%    \begin{macrocode}
local function msg_format(mod, msg_type, text)
  local leader = ""
  if type == "Error" then
    leader = "! "
  end
  local cont
  if mod == "LaTeX" then
    cont = string_gsub(leader, ".", " ")
    leader = leader .. "LaTeX: "
  else
    first_head = leader .. "Module "  .. msg_type 
    cont = "(" .. mod .. ")"
      .. string_gsub(first_head, ".", " ")
    first_head =  leader .. "Module "  .. mod .. " " .. msg_type  .. ":"
  end
  return first_head .. " "
    .. string_gsub(
         text .. " on input line "
         .. tex.inputlineno, "\n", "\n" .. cont .. " "
      )
   .. "\n"
end
%    \end{macrocode}
%
% \begin{macro}{luatexbase.module_info}
% \changes{v1.0a}{0000/00/00}{Function added}
% \begin{macro}{luatexbase.module_warning}
% \changes{v1.0a}{0000/00/00}{Function added}
% \begin{macro}{luatexbase.module_error}
% \changes{v1.0a}{0000/00/00}{Function added}
%   Write messages.
%    \begin{macrocode}
local function module_info(mod, text)
  local i
  for _,i in ipairs(msg_format(mod, "Info", text):explode("\n")) do
    texio_write_nl("log", i)
  end
end
luatexbase.module_info = module_info
local function module_warning(mod, text)
  local i
  for _,i in ipairs(msg_format(mod, "Warning", text):explode("\n")) do
    texio_write_nl("term and log", i)
  end
end
luatexbase.module_warning = module_warning
local function module_error(mod, text)
  local i
  for _,i in ipairs(msg_format(mod, "Error", text):explode("\n")) do
    texio_write_nl("term and log", i)
  end
  texio_write_nl("term and log", "\n")
  error("See " .. mod .. " Error")
end
luatexbase.module_error = module_error
%    \end{macrocode}
% \end{macro}
% \end{macro}
% \end{macro}
%
% Dedicated versions for the rest of the code here.
%    \begin{macrocode}
local function luatexbase_warning(text)
  module_warning("luatexbase", text)
end
local function luatexbase_error(text)
  module_error("luatexbase", text)
end
%    \end{macrocode}
%
%
% \subsection{Accessing register numbers from Lua}
%
% Collect up the data from the \TeX{} level into a Lua table: from
% version~0.80, Lua\TeX{} makes that easy.
%    \begin{macrocode}
local luaregisterbasetable = { }
local registermap = {
  attributezero = "assign_attr"    ,
  charzero      = "char_given"     ,
  CountZero     = "assign_int"     ,
  dimenzero     = "assign_dimen"   ,
  mathcharzero  = "math_given"     ,
  muskipzero    = "assign_mu_skip" ,
  skipzero      = "assign_skip"    ,
  tokszero      = "assign_toks"    ,
}
local i, j
local createtoken
if tex.luatexversion >79 then
 createtoken   = newtoken.create
end
local hashtokens    = tex.hashtokens
local luatexversion = tex.luatexversion
for i,j in pairs (registermap) do
  if luatexversion < 80 then
    luaregisterbasetable[hashtokens()[i][1]] =
      hashtokens()[i][2]
  else
    luaregisterbasetable[j] = createtoken(i).mode
  end
end
%    \end{macrocode}
%
% \begin{macro}{luatexbase.registernumber}
%   Working out the correct return value can be done in two ways. For older
%   Lua\TeX{} releases it has to be extracted from the |hashtokens|. On the
%   other hand, newer Lua\TeX{}'s have |newtoken|, and whilst |.mode| isn't
%   currently documented, Hans Hagen pointed to this approach so we should be
%   OK.
%    \begin{macrocode}
local registernumber
if luatexversion < 80 then
  function registernumber(name)
    local nt = hashtokens()[name]
    if(nt and luaregisterbasetable[nt[1]]) then
      return nt[2] - luaregisterbasetable[nt[1]]
    else
      return false
    end
  end
else
  function registernumber(name)
    local nt = createtoken(name)
    if(luaregisterbasetable[nt.cmdname]) then
      return nt.mode - luaregisterbasetable[nt.cmdname]
    else
      return false
    end
  end
end
luatexbase.registernumber = registernumber
%    \end{macrocode}
% \end{macro}
%
% \subsection{Attribute allocation}
%
% \begin{macro}{luatexbase.new_attribute}
% \changes{v1.0a}{0000/00/00}{Function added}
%   As attributes are used for Lua manipulations its useful to be able
%   to assign from this end.
%    \begin{macrocode}
local attributes=setmetatable(
{},
{
__index = function(t,key)
return registernumber(key) or nil
end}
)
luatexbase.attributes=attributes
%    \end{macrocode}
%
%    \begin{macrocode}
local function new_attribute(name)
  tex_setcount("global", "e@alloc@attribute@count",
                          tex_count["e@alloc@attribute@count"] + 1)
  if tex_count["e@alloc@attribute@count"] > 65534 then
    luatexbase_error("No room for a new \\attribute")
    return -1
  end
  attributes[name]= tex_count["e@alloc@attribute@count"]
  texio_write_nl("Lua-only attribute " .. name .. " = " ..
                 tex_count["e@alloc@attribute@count"])
  return tex_count["e@alloc@attribute@count"]
end
luatexbase.new_attribute = new_attribute
%    \end{macrocode}
% \end{macro}
%
% \subsection{Custom whatsit allocation}
%
% \begin{macro}{luatexbase.new_whatsit}
% Much the same as for attribute allocation in Lua
%    \begin{macrocode}
local function new_whatsit(name)
  tex_setcount("global", "e@alloc@whatsit@count", 
                         tex_count["e@alloc@whatsit@count"] + 1)
  if tex_count["e@alloc@whatsit@count"] > 65534 then
    luatexbase_error("No room for a new custom whatsit")
    return -1
  end
  texio_write_nl("Custom whatsit " .. name .. " = " ..
                 tex_count["e@alloc@whatsit@count"])
  return tex_count["e@alloc@whatsit@count"]
end
luatexbase.new_whatsit = new_whatsit
%    \end{macrocode}
% \end{macro}
%
% \subsection{Bytecode register allocation}
%
% \begin{macro}{luatexbase.new_bytecode}
% Much the same as for attribute allocation in Lua.
% Currently we maintain the allocation count purely in lua, not using
% a \TeX\ counter.
% The optional \meta{name} argument is used in the log if given.
%    \begin{macrocode}
local bytecode_count = 0
local function new_bytecode(name)
  bytecode_count = bytecode_count + 1
  if bytecode_count > 65534 then
    luatexbase_error("No room for a new bytecode")
    return -1
  end
  texio_write_nl("Lua bytecode " .. (name or "") .. " = " .. 
                 bytecode_count .. "\n")
  return bytecode_count
end
luatexbase.new_bytecode = new_bytecode
%    \end{macrocode}
% \end{macro}
%
% \subsection{Lua chunk name allocation}
%
% \begin{macro}{luatexbase.new_chunkname}
% As for bytecode registers but here we also store the name in the
% |lua.name| table.
%    \begin{macrocode}
local chunkname_count = 0
local function new_chunkname(name)
  chunkname_count = chunkname_count + 1
  if chunkname_count > 65534 then
    luatexbase_error("No room for a new chunkname")
    return -1
  end
  lua.name[chunkname_count]=name
  texio_write_nl("Lua chunkname " .. name .. " = " .. 
                 chunkname_count .. "\n")
  return chunkname_count
end
luatexbase.new_chunkname = new_chunkname
%    \end{macrocode}
% \end{macro}
%
% \subsection{Lua callback management}
%
% The native mechanism for callbacks in Lua allows only one per function.
% That is extremely restrictive and so a mechanism is needed to add and
% remove callbacks from the appropriate hooks.
%
% \subsubsection{Housekeeping}
%
% The main table: keys are callback names, and values are the associated lists
% of functions. More precisely, the entries in the list are tables holding the
% actual function as |func| and the identifying description as |description|.
% Only callbacks with a non-empty list of functions have an entry in this
% list.
%    \begin{macrocode}
local callbacklist = callbacklist or { }
%    \end{macrocode}
%
% Numerical codes for callback types, and name-to-value association (the
% table keys are strings, the values are numbers).
%    \begin{macrocode}
local list, data, exclusive, simple = 1, 2, 3, 4
local types = {
  list      = list,
  data      = data,
  exclusive = exclusive,
  simple    = simple,
}
%    \end{macrocode}
%
% Now, list all predefined callbacks with their current type, based on the
% Lua\TeX{} manual version~0.80. A full list of the currently-available
% callbacks can be obtained using
%  \begin{verbatim}
%    \directlua{
%      for i,_ in pairs(callback.list()) do
%        texio.write_nl("- " .. i)
%      end
%    }
%    \bye
%  \end{verbatim}
% in plain Lua\TeX{}. (Some undocumented callbacks are omitted as they are
% to be removed.)
%    \begin{macrocode}
local callbacktypes = callbacktypes or {
%    \end{macrocode}
%   Section 4.1.1: file discovery callbacks.
%    \begin{macrocode}
  find_read_file     = exclusive,
  find_write_file    = exclusive,
  find_font_file     = data,
  find_output_file   = data,
  find_format_file   = data,
  find_vf_file       = data,
  find_map_file      = data,
  find_enc_file      = data,
  find_sfd_file      = data,
  find_pk_file       = data,
  find_data_file     = data,
  find_opentype_file = data,
  find_truetype_file = data,
  find_type1_file    = data,
  find_image_file    = data,
%    \end{macrocode}
% Section 4.1.2: file reading callbacks.
%    \begin{macrocode}
  open_read_file     = exclusive,
  read_font_file     = exclusive,
  read_vf_file       = exclusive,
  read_map_file      = exclusive,
  read_enc_file      = exclusive,
  read_sfd_file      = exclusive,
  read_pk_file       = exclusive,
  read_data_file     = exclusive,
  read_truetype_file = exclusive,
  read_type1_file    = exclusive,
  read_opentype_file = exclusive,
%    \end{macrocode}
% Section 4.1.3: data processing callbacks.
%    \begin{macrocode}
  process_input_buffer  = data,
  process_output_buffer = data,
  process_jobname       = data,
  token_filter          = exclusive,
%    \end{macrocode}
% Section 4.1.4: node list processing callbacks.
%    \begin{macrocode}
  buildpage_filter      = simple,
  pre_linebreak_filter  = list,
  linebreak_filter      = list,
  post_linebreak_filter = list,
  hpack_filter          = list,
  vpack_filter          = list,
  pre_output_filter     = list,
  hyphenate             = simple,
  ligaturing            = simple,
  kerning               = simple,
  mlist_to_hlist        = list,
%    \end{macrocode}
% Section 4.1.5: information reporting callbacks.
%    \begin{macrocode}
  pre_dump            = simple,
  start_run           = simple,
  stop_run            = simple,
  start_page_number   = simple,
  stop_page_number    = simple,
  show_error_hook     = simple,
  show_error_message  = simple,
  show_lua_error_hook = simple,
  start_file          = simple,
  stop_file           = simple,
%    \end{macrocode}
% Section 4.1.6: PDF-related callbacks.
%    \begin{macrocode}
  finish_pdffile = data,
  finish_pdfpage = data,
%    \end{macrocode}
% Section 4.1.7: font-related callbacks.
%    \begin{macrocode}
  define_font = exclusive,
%    \end{macrocode}
% Undocumented callbacks which are likely to get documented.
%    \begin{macrocode}
  find_cidmap_file           = data,
  pdf_stream_filter_callback = data,
}
luatexbase.callbacktypes=callbacktypes
%    \end{macrocode}
%
% \begin{macro}{callback.register}
% \changes{v1.0a}{0000/00/00}{Function modified}
%   Save the original function for registering callbacks and prevent the
%   original being used. The original is saved in a place that remains
%   available so other more sophisticated code can override the approach
%   taken by the kernel if desired.
%    \begin{macrocode}
local callback_register = callback_register or callback.register
function callback.register()
  luatexbase_error("Attempt to use callback.register() directly.")
end
%    \end{macrocode}
% \end{macro}
%
% \subsubsection{Handlers}
%
% The handler function is registered into the callback when the
% first function is added to this callback's list. Then, when the callback
% is called, then handler takes care of running all functions in the list.
% When the last function is removed from the callback's list, the handler
% is unregistered.
%
% More precisely, the functions below are used to generate a specialized
% function (closure) for a given callback, which is the actual handler.
%
% Handler for |data| callbacks.
%    \begin{macrocode}
local function data_handler(name)
  return function(data, ...)
    local i
    for _,i in ipairs(callbacklist[name]) do
      data = i.func(data,...)
    end
    return data
  end
end
%    \end{macrocode}
% Handler for |exclusive| callbacks. We can assume |callbacklist[name]| is not
% empty: otherwise, the function wouldn't be registered in the callback any
% more.
%    \begin{macrocode}
local function exclusive_handler(name)
  return function(...)
    return callbacklist[name][1].func(...)
  end
end
%    \end{macrocode}
% Handler for |list| callbacks.
%    \begin{macrocode}
local function list_handler(name)
  return function(head, ...)
    local ret
    local alltrue = true
    local i
    for _,i in ipairs(callbacklist[name]) do
      ret = i.func(head, ...)
      if ret == false then
        luatexbase_warning(
          "Function `i.description' returned false\n"
            .. "in callback `name'"
         )
         break
      end
      if ret ~= true then
        alltrue = false
        head = ret
      end
    end
    return alltrue and true or head
  end
end
%    \end{macrocode}
% Handler for |simple| callbacks.
%    \begin{macrocode}
local function simple_handler(name)
  return function(...)
    local i
    for _,i in ipairs(callbacklist[name]) do
      i.func(...)
    end
  end
end
%    \end{macrocode}
%
% Keep a handlers table for indexed access.
%    \begin{macrocode}
local handlers = {
  [data]      = data_handler,
  [exclusive] = exclusive_handler,
  [list]      = list_handler,
  [simple]    = simple_handler,
}
%    \end{macrocode}
%
% \subsubsection{Public functions for callback management}
%
% Defining user callbacks perhaps should be in package code,
% but impacts on |add_to_callback|.
% If a default function is not required, may may be declared as |false|.
% First we need a list of user callbacks.
%    \begin{macrocode}
local user_callbacks_defaults = { }
%    \end{macrocode}
%
% \begin{macro}{luatexbase.create_callback}
% \changes{v1.0a}{0000/00/00}{Function added}
%   The allocator itself.
%    \begin{macrocode}
local function create_callback(name, ctype, default)
  if not name or
    name == "" or
    callbacktypes[name] or
    not(default == false or  type(default) == "function")
    then
      luatexbase_error("Unable to create callback " .. name)
  end
  user_callbacks_defaults[name] = default
  callbacktypes[name] = types[ctype]
end
luatexbase.create_callback = create_callback
%    \end{macrocode}
% \end{macro}
%
% \begin{macro}{luatexbase.call_callback}
% \changes{v1.0a}{0000/00/00}{Function added}
%  Call a user defined callback. First check arguments.
%    \begin{macrocode}
local function call_callback(name,...)
  if not name or
    name == "" or
    user_callbacks_defaults[name] == nil
    then
        luatexbase_error("Unable to call callback " .. name)
  end
  local l = callbacklist[name]
  local f
  if not l then
    f = user_callbacks_defaults[name]
    if l == false then
	   return nil
	 end
  else
    f = handlers[callbacktypes[name]](name)
  end
  return f(...)
end
luatexbase.call_callback=call_callback
%    \end{macrocode}
% \end{macro}
%
% \begin{macro}{luatexbase.add_to_callback}
% \changes{v1.0a}{0000/00/00}{Function added}
%   Add a function to a callback. First check arguments.
%    \begin{macrocode}
local function add_to_callback(name, func, description)
  if
    not name or
    name == "" or
    not callbacktypes[name] or
    type(func) ~= "function" or
    not description or
    description == "" then
    luatexbase_error(
      "Unable to register callback.\n\n"
        .. "Correct usage:\n"
        .. "add_to_callback(<callback>, <function>, <description>)"
    )
    return
  end
%    \end{macrocode}
%   Then test if this callback is already in use. If not, initialise its list
%   and register the proper handler.
%    \begin{macrocode}
  local l = callbacklist[name]
  if l == nil then
    l = { }
    callbacklist[name] = l
%    \end{macrocode}
% If it is not a user defined callback use the primitive callback register.
%    \begin{macrocode}
    if user_callbacks_defaults[name] == nil then
      callback_register(name, handlers[callbacktypes[name]](name))
    end
  end
%    \end{macrocode}
%  Actually register the function and give an error if more than one
%  |exclusive| one is registered.
%    \begin{macrocode}
  local f = {
    func        = func,
    description = description,
  }
  local priority = #l + 1
  if callbacktypes[name] == exclusive then
    if #l == 1 then
      luatexbase_error(
        "Cannot add second callback to exclusive function `" ..
        name .. "'.")
    end
  end
  table.insert(l, priority, f)
%    \end{macrocode}
%  Keep user informed.
%    \begin{macrocode}
  texio_write_nl(
    "Inserting `" .. description .. "' at position "
      .. priority .. " in `" .. name .. "'."
  )
end
luatexbase.add_to_callback = add_to_callback
%    \end{macrocode}
% \end{macro}
%
% \begin{macro}{luatexbase.remove_from_callback}
% \changes{v1.0a}{0000/00/00}{Function added}
%   Remove a function from a callback. First check arguments.
%    \begin{macrocode}
local function remove_from_callback(name, description)
  if
    not name or
    name == "" or
    not callbacktypes[name] or
    not description or
    description == "" then
    luatexbase_error(
      "Unable to remove function from callback.\n\n"
        .. "Correct usage:\n"
        .. "remove_to_callback(<callback>, <description>)"
    )
    return
  end
  local l = callbacklist[name]
  if not l then
    luatexbase_error(
      "No callback list for `" .. name .. "'.")
  end
%    \end{macrocode}
%  Loop over the callback's function list until we find a matching entry.
%  Remove it and check if the list is empty: if so, unregister the
%   callback handler.
%    \begin{macrocode}
  local index = false
  local i,j
  local cb = {}
  for i,j in ipairs(l) do
    if j.description == description then
      index = i
      break
    end
  end
  if not index then
    luatexbase_error(
      "No callback `" .. description .. "' registered for `" ..
      name .. "'.")
    return
  end
  cb = l[index]
  table.remove(l, index)
  texio_write_nl(
    "Removing  `" .. description .. "' from `" .. name .. "'."
  )
  if #l == 0 then
    callbacklist[name] = nil
    callback_register(name, nil)
  end
  return cb.func,cb.description
end
luatexbase.remove_from_callback = remove_from_callback
%    \end{macrocode}
% \end{macro}
%
% \begin{macro}{luatexbase.in_callback}
% \changes{v1.0a}{0000/00/00}{Function added}
%   Look for a function description in a callback.
%    \begin{macrocode}
local function in_callback(name, description)
  if not name
    or name == ""
    or not callbacktypes[name]
    or not description then
      return false
  end
  local i
  for _, i in pairs(callbacklist[name]) do
    if i.description == description then
      return true
    end
  end
  return false
end
luatexbase.in_callback = in_callback
%    \end{macrocode}
% \end{macro}
%
% \begin{macro}{luatexbase.disable_callback}
% \changes{v1.0a}{0000/00/00}{Function added}
%   As we subvert the engine interface we need to provide a way to access
%   this functionality.
%    \begin{macrocode}
local function disable_callback(name)
  if(callbacklist[name] == nil) then
    callback_register(name, false)
  else
    luatexbase_error("Callback list for " .. name .. " not empty")
  end
end
luatexbase.disable_callback = disable_callback
%    \end{macrocode}
% \end{macro}
%
% \begin{macro}{luatexbase.in_callback}
% \changes{v1.0a}{0000/00/00}{Function added}
%   List the descriptions of functions registed for the given callback.
%    \begin{macrocode}
local function callback_descriptions (name)
  local d = {}
  if not name
    or name == ""
    or not callbacktypes[name]
    then
    return d
  else
  local i
  for k, i in pairs(callbacklist[name] or {}) do
    d[k]= i.description
    end
  end
  return d
end
luatexbase.callback_descriptions =callback_descriptions 
%    \end{macrocode}
% \end{macro}
% \endgroup
%
%    \begin{macrocode}
%</lua>
%    \end{macrocode}
%
% Reset the catcode of |@|.
%    \begin{macrocode}
%<tex>\catcode`\@=\etatcatcode\relax
%    \end{macrocode}
%
%
% \Finale
| this inputs |ltluatex.tex| which inputs
% |etex.src| (or |etex.sty| if used with \LaTeX)
% if it is not already input, and then defines some internal commands to
% allow the \textsf{ltluatex} interface to be defined.
%
% The \textsf{luatexbase} package interface may also be used in plain \TeX,
% as before, by inputting the package |\input luatexbase.sty|. The new
% version of \textsf{luatexbase} is based on this \textsf{ltluatex}
% code but implements a compatibility layer providing the interface
% of the original package.
%
% \section{Lua functionality}
%
% \begingroup
%
% \begingroup\lccode`~=`_
% \lowercase{\endgroup\let~}_
% \catcode`_=12
%
% \subsection{Allocators in Lua}
%
% \DescribeMacro{luatexbase.new_attribute}
% |luatexbase.new_attribute(|\meta{attribute}|)|\\
% Returns an allocation number for the \meta{attribute}, indexed from~$1$.
% The attribute will be initialised with the marker value |-"7FFFFFFF|
% (`unset'). The attribute allocation sequence is shared with the \TeX{}
% code but this function does \emph{not} define a token using
% |\attributedef|.
% The attribute name is recorded in the |attributes| table. A
% metatable is provided so that the table syntax can be used
% consistently for attributes declared in \TeX\ or lua.
%
% \noindent
% \DescribeMacro{luatexbase.new_whatsit}
% |luatexbase.new_whatsit(|\meta{whatsit}|)|\\
% Returns an allocation number for the custom \meta{whatsit}, indexed from~$1$.
%
% \noindent
% \DescribeMacro{luatexbase.new_bytecode}
% |luatexbase.new_bytecode(|\meta{bytecode}|)|\\
% Returns an allocation number for a bytecode register, indexed from~$1$.
% The optional \meta{name} argument is just used for logging.
%
% \noindent
% \DescribeMacro{luatexbase.new_chunkname}
% |luatexbase.new_chunkname(|\meta{chunkname}|)|\\
% Returns an allocation number for a lua chunk name for use with 
% |\directlua| and |\latelua|, indexed from~$1$.
% The number is returned and also \meta{name} argument is added to the
% |lua.name| array at that index.
%
%
% \subsection{Lua access to \TeX{} register numbers}
%
% \DescribeMacro{luatexbase.registernumber}
% |luatexbase.registernumer(|\meta{name}|)|\\
% Sometimes (notably in the case of Lua attributes) it is necessary to
% access a register \emph{by number} that has been allocated by \TeX{}.
% This package provides a function to look up the relevant number
% using Lua\TeX{}'s internal tables. After for example
% |\newattribute\myattrib|, |\myattrib| would be defined by (say)
% |\myattrib=\attribute15|.  |luatexbase.registernumer("myattrib")|
% would then return the register number, $15$ in this case. If the string passed
% as argument does not correspond to a token defined by |\attributedef|,
% |\countdef| or similar commands, the Lua value |false| is returned.
%
% As an example, consider the input:
%\begin{verbatim}
% \newcommand\test[1]{%
% \typeout{#1: \expandafter\meaning\csname#1\endcsname^^J
% \space\space\space\space
% \directlua{tex.write(luatexbase.registernumber("#1") or "bad input")}%
% }}
%
% \test{undefinedrubbish}
%
% \test{space}
%
% \test{hbox}
%
% \test{@MM}
%
% \test{@tempdima}
% \test{@tempdimb}
%
% \test{strutbox}
%
% \test{sixt@@n}
%
% \attrbutedef\myattr=12
% \myattr=200
% \test{myattr}
%
%\end{verbatim}
%
% If the demonstration code is processed with Lua\LaTeX{} then the following
% would be produced in the log and terminal output.
%\begin{verbatim}
% undefinedrubbish: \relax
%      bad input
% space: macro:->
%      bad input
% hbox: \hbox
%      bad input
% @MM: \mathchar"4E20
%      20000
% @tempdima: \dimen14
%      14
% @tempdimb: \dimen15
%      15
% strutbox: \char"B
%      11
% sixt@@n: \char"10
%      16
% myattr: \attribute12
%      12
%\end{verbatim}
%
% Notice how undefined commands, or commands unrelated to registers
% do not produce an error, just return |false| and so print
% |bad input| here. Note also that commands defined by |\newbox| work and
% return the number of the box register even though the actual command
% holding this number is a |\chardef| defined token (there is no
% |\boxdef|).
%
% \subsection{Module utilities}
%
% \DescribeMacro{luatexbase.provides_module}
% |luatexbase.provides_module(|\meta{info}|)|\\
% This function is used by modules to identify themselves; the |info| should be
% a table containing information about the module. The required field
% |name| must contain the name of the module. It is recommended to provide a
% field |date| in the usual \LaTeX{} format |yyyy/mm/dd|. Optional fields
% |version| (a string) and |description| may be used if present. This
% information will be recorded in the log. Other fields are ignored.
%
% \noindent
% \DescribeMacro{luatexbase.module_info}
% \DescribeMacro{luatexbase.module_warning}
% \DescribeMacro{luatexbase.module_error}
% |luatexbase.module_info(|\meta{module}, \meta{text}|)|\\
% |luatexbase.module_warning(|\meta{module}, \meta{text}|)|\\
% |luatexbase.module_error(|\meta{module}, \meta{text}|)|\\
% These functions are similar to \LaTeX{}'s |\PackageError|, |\PackageWarning|
% and |\PackageInfo| in the way they format the output.  No automatic line
% breaking is done, you may still use |\n| as usual for that, and the name of
% the package will be prepended to each output line.
%
% Note that |luatexbase.module_error| raises an actual Lua error with |error()|,
% which currently means a call stack will be dumped. While this may not
% look pretty, at least it provides useful information for tracking the
% error down.
%
% \subsection{Callback management}
%
% \noindent
% \DescribeMacro{luatexbase.add_to_callback}
% |luatexbase.add_to_callback(|^^A
% \meta{callback}, \meta{function}, \meta{description}|)|
% Registers the \meta{function} into the \meta{callback} with a textual
% \meta{description} of the function. Functions are inserted into the callback
% in the order loaded.
%
% \noindent
% \DescribeMacro{luatexbase.remove_from_callback}
% |luatexbase.remove_from_callback(|\meta{callback}, \meta{description}|)|
% Removes the function with \meta{description} from the \meta{callback}.
% The removed function and its description 
% are returned as the results of this function.
%
% \noindent
% \DescribeMacro{luatexbase.in_callback}
% |luatexbase.in_callback(|\meta{callback}, \meta{description}|)|
% Checks if the \meta{description} matches one of the functions added
% to the list for the \meta{callback}, returning a boolean value.
%
% \noindent
% \DescribeMacro{luatexbase.disable_callback}
% |luatexbase.disable_callback(|\meta{callback}|)|
% Sets the \meta{callback} to \texttt{false} as described in the Lua\TeX{}
% manual for the underlying \texttt{callback.register} built-in. Callbacks
% will only be set to false (and thus be skipped entirely) if there are
% no functions registered using the callback.
%
% \noindent
% \DescribeMacro{luatexbase.callback_descriptions}
% A list of the descriptions of functions registered to the specified
% callback is returned. |{}| is returned if there are no functions registered.
%
% \noindent
% \DescribeMacro{luatexbase.create_callback}
% |luatexbase.create_callback(|\meta{name},meta{type},\meta{default}|)|
% Defines a user defined callback. The last argument is a default
% functtion of |false|.
%
% \noindent
% \DescribeMacro{luatexbase.call_callback}
% |luatexbase.create_callback(|\meta{name},\ldots|)|
% Calls a user defined callback with the supplied arguments.
%
% \endgroup
%
% \CheckSum{425}
% \StopEventually{}
%
% \section{Implementation}
%
%    \begin{macrocode}
%<*2ekernel|tex|latexrelease>
%<2ekernel|latexrelease>\ifx\directlua\@undefined\else
%    \end{macrocode}
%
%
% \subsection{Minimum Lua\TeX{} version}
%
% Lua\TeX{} has changed a lot over time. In the kernel support for ancient
% versions is not provided: trying to build a format with a very old binary
% therefore gives some information in the log and loading stops. The cut-off
% selected here relates to the tree-searching behaviour of |require()|:
% from version~0.60, Lua\TeX{} will correctly find Lua files in the |texmf|
% tree without `help'.
%    \begin{macrocode}
%<latexrelease>\IncludeInRelease{2015/11/01}
%<latexrelease>                 {\newluafunction}{LuaTeX}%
\ifnum\luatexversion<60 %
  \wlog{***************************************************}
  \wlog{* LuaTeX version too old for ltluatex support *}
  \wlog{***************************************************}
  \expandafter\endinput
\fi
%    \end{macrocode}
%
% \subsection{Older \LaTeX{}/Plain \TeX\ setup}
% 
%    \begin{macrocode}
%<*tex>
%    \end{macrocode}
%
% Older \LaTeX{] formats don't have the primitives with `native' names:
% sort that out. I fthey already exist this will still be safe.
%    \begin{macrocode}
\directlua{tex.enableprimitives("",tex.extraprimitives("luatex"))}
%    \end{macrocode}
%
%    \begin{macrocode}
\ifx\e@alloc\@undefined
%    \end{macrocode}
%
% In pre-2014 \LaTeX{}, or plain \TeX{}, load |etex.{sty,src}|.
%    \begin{macrocode}
  \ifx\documentclass\@undefined
    \ifx\loccount\@undefined
      \input{etex.src}%
    \fi
    \catcode`\@=11 %
    \outer\expandafter\def\csname newfam\endcsname
                          {\alloc@8\fam\chardef\et@xmaxfam}
  \else
    \RequirePackage{etex}
    \expandafter\def\csname newfam\endcsname
                    {\alloc@8\fam\chardef\et@xmaxfam}
    \expandafter\let\expandafter\new@mathgroup\csname newfam\endcsname
  \fi
%    \end{macrocode}
%
% \subsubsection{Fixes to \texttt{etex.src}/\texttt{etex.sty}}
%
% These could and probably should be made directly in an
% update to etex.src which already has some luatex-specific
% code, but does not define the correct range for luatex.
%
%    \begin{macrocode}
% 2015-07-13 higher range in luatex
\edef \et@xmaxregs {\ifx\directlua\@undefined 32768\else 65536\fi}
% luatex/xetex also allow more math fam
\edef \et@xmaxfam {\ifx\Umathchar\@undefined\sixt@@n\else\@cclvi\fi}
%    \end{macrocode}
%
%    \begin{macrocode}
\count 270=\et@xmaxregs % locally allocates \count registers
\count 271=\et@xmaxregs % ditto for \dimen registers
\count 272=\et@xmaxregs % ditto for \skip registers
\count 273=\et@xmaxregs % ditto for \muskip registers
\count 274=\et@xmaxregs % ditto for \box registers
\count 275=\et@xmaxregs % ditto for \toks registers
\count 276=\et@xmaxregs % ditto for \marks classes
%    \end{macrocode}
%
% and 256 or 16 fam. (Done above due to plain/\LaTeX\ differences in
% \textsf{ltluatex}.)
%    \begin{macrocode}
% \outer\def\newfam{\alloc@8\fam\chardef\et@xmaxfam}
%    \end{macrocode}
%
% End of proposed changes to \texttt{etex.src}
%
% \subsubsection{luatex specific settings}
% 
% Switch to global cf |luatex.sty| to leave room for inserts
% not really needed for luatex but possibly most compatible
% with existing use.
%    \begin{macrocode}
\expandafter\let\csname newcount\expandafter\expandafter\endcsname
                \csname globcount\endcsname
\expandafter\let\csname newdimen\expandafter\expandafter\endcsname
                \csname globdimen\endcsname
\expandafter\let\csname newskip\expandafter\expandafter\endcsname
                \csname globskip\endcsname
\expandafter\let\csname newbox\expandafter\expandafter\endcsname
                \csname globbox\endcsname
%    \end{macrocode}
%
% Define|\e@alloc| as in latex (the existing macros in |etex.src|
% hard to extend to further register types as they assume specific
% 26x and 27x count range. For compatibility the existing register
% allocation is not changed.
%
%    \begin{macrocode}
\chardef\e@alloc@top=65535
\let\e@alloc@chardef\chardef
%    \end{macrocode}
%
%    \begin{macrocode}
\def\e@alloc#1#2#3#4#5#6{%
  \global\advance#3\@ne
  \e@ch@ck{#3}{#4}{#5}#1%
  \allocationnumber#3\relax
  \global#2#6\allocationnumber
  \wlog{\string#6=\string#1\the\allocationnumber}}%
%    \end{macrocode}
%
%    \begin{macrocode}
\gdef\e@ch@ck#1#2#3#4{%
  \ifnum#1<#2\else
    \ifnum#1=#2\relax
      #1\@cclvi
      \ifx\count#4\advance#1 10 \fi
    \fi
    \ifnum#1<#3\relax
    \else
      \errmessage{No room for a new \string#4}%
    \fi
  \fi}%
%    \end{macrocode}
%
% Two simple \LaTeX\ macros used in |ltlatex.sty|.
%    \begin{macrocode}
\long\def\@gobble#1{}
\long\def\@firstofone#1{#1}
%    \end{macrocode}
%
%    \begin{macrocode}
% Fix up allocations not to clash with |etex.src|.
%    \end{macrocode}
%
%    \begin{macrocode}
\expandafter\csname newcount\endcsname\e@alloc@attribute@count
\expandafter\csname newcount\endcsname\e@alloc@ccodetable@count
\expandafter\csname newcount\endcsname\e@alloc@luafunction@count
\expandafter\csname newcount\endcsname\e@alloc@whatsit@count
%    \end{macrocode}
%
% End of conditional setup for plain \TeX\ / old \LaTeX.
%    \begin{macrocode}
\fi
%</tex>
%    \end{macrocode}
%
%
% \subsection{Attributes}
%
% \begin{macro}{\newattribute}
% \changes{v1.0a}{0000/00/00}{Macro added}
%   As is generally the case for the Lua\TeX{} registers we start here
%   from~$1$. Notably, some code assumes that |\attribute0| is never used so
%   this is important in this case.
%    \begin{macrocode}
\ifx\e@alloc@attribute@count\@undefined
  \countdef\e@alloc@attribute@count=258
\fi
\def\newattribute#1{%
  \e@alloc\attribute\attributedef
    \e@alloc@attribute@count\m@ne\e@alloc@top#1%
}
\e@alloc@attribute@count=\z@
%    \end{macrocode}
% \end{macro}
%
% \begin{macro}{\setattribute}
% \begin{macro}{\unsetattribute}
%   Handy utilities.
%    \begin{macrocode}
\def\setattribute#1#2{#1=\numexpr#2\relax}
\def\unsetattribute#1{#1=-"7FFFFFFF\relax}
%    \end{macrocode}
% \end{macro}
% \end{macro}
%
% \subsection{Category code tables}
%
% \begin{macro}{\newcatcodetable}
% \changes{v1.0a}{0000/00/00}{Macro added}
%   Category code tables are allocated with a limit half of that used by Lua\TeX{}
%   for everything else. At the end of allocation there needs to be an
%   initialisation step. Table~$0$ is already taken (it's the global one for
%   current use) so the allocation starts at~$1$.
%    \begin{macrocode}
\ifx\e@alloc@ccodetable@count\@undefined
  \countdef\e@alloc@ccodetable@count=259
\fi
\def\newcatcodetable#1{%
  \e@alloc\catcodetable\chardef
    \e@alloc@ccodetable@count\m@ne{"8000}#1%
  \initcatcodetable\allocationnumber
}
\e@alloc@ccodetable@count=\z@
%    \end{macrocode}
% \end{macro}
%
% \begin{macro}{\catcodetable@initex}
% \changes{v1.0a}{0000/00/00}{Macro added}
% \begin{macro}{\catcodetable@string}
% \changes{v1.0a}{0000/00/00}{Macro added}
% \begin{macro}{\catcodetable@latex}
% \changes{v1.0a}{0000/00/00}{Macro added}
% \begin{macro}{\catcodetable@atletter}
% \changes{v1.0a}{0000/00/00}{Macro added}
%   Save a small set of standard tables. The Unicode data is read
%   here in a group avoiding any global definitions: that needs a bit
%   of effort so that in package/plain mode there is no effect on any
%   settings already in force.
%    \begin{macrocode}
\newcatcodetable\catcodetable@initex
\newcatcodetable\catcodetable@string
\begingroup
  \def\setrangecatcode#1#2#3{%
    \ifnum#1>#2 %
      \expandafter\@gobble
    \else
      \expandafter\@firstofone
    \fi
      {%
        \catcode#1=#3 %
        \expandafter\setrangecatcode\expandafter
          {\number\numexpr#1 + 1\relax}{#2}{#3}
      }%
  }
  \@firstofone{%
    \catcodetable\catcodetable@initex
      \catcode0=12 %
      \catcode12=12 %
      \catcode37=12 %
      \setrangecatcode{65}{90}{12}%
      \setrangecatcode{97}{122}{12}%
      \catcode92=12 %
      \catcode127=12 %
      \savecatcodetable\catcodetable@string
    \endgroup
  }%
\newcatcodetable\catcodetable@latex
\newcatcodetable\catcodetable@atletter
\begingroup
  \let\ENDGROUP\endgroup
  \let\begingroup\relax
  \let\endgroup\relax
  \let\global\relax
  \let\gdef\def
  %% This is the file `unicode-letters.def',
%% generated using the script ltunicode.dtx.
%%
%% The data here are derived from the files
%% - UnicodeData.txt 
%%   MD5 sum 01C77522E09E6A60D15C0E9983C06975
%% - EastAsianWidth.txt 
%%   Version 7.0.0 dated 2014-02-28, 23:15:00
%%   MD5 sum 1148EADA16B2123FA7D3DA310D52C4C6
%% - LineBreak.txt 
%%   Version 7.0.0 dated 2014-02-28, 23:15:00
%%   MD5 sum C95A5E60B2B527F4E77286F9C9CF09D1
%% which are maintained by the Unicode Consortium.
%%
%% Generated on 2015-08-06. 
%%
%% Copyright 2014-2015
%% The LaTeX3 Project and any individual authors listed elsewhere
%% in this file.
%%
%% This file is part of the LaTeX base system.
%% -------------------------------------------
%%
%% It may be distributed and/or modified under the
%% conditions of the LaTeX Project Public License, either version 1.3c
%% of this license or (at your option) any later version.
%% The latest version of this license is in
%%    http://www.latex-project.org/lppl.txt
%% and version 1.3c or later is part of all distributions of LaTeX
%% version 2005/12/01 or later.
%%
%% This file has the LPPL maintenance status "maintained".
%%
%% The list of all files belonging to the LaTeX base distribution is
%% given in the file `manifest.txt'. See also `legal.txt' for additional
%% information.
\begingroup
  \def\C#1 #2 #3 {%
    \XeTeXcheck{#1}%
    \global\uccode"#1="#2 %
    \global\lccode"#1="#3 %
  }
  \def\L#1 #2 #3 {%
    \C #1 #2 #3 %
    \global\catcode"#1=11 %
    \ifnum"#1="#3 %
    \else
      \global\sfcode"#1=999 %
    \fi
    \ifnum"#1<"10000 %
      \global\Umathcode"#1="7"01"#1 %
    \else
      \global\Umathcode"#1="0"01"#1 %
    \fi
  }
  \def\l#1 {\L#1 #1 #1 }
  \ifx\Umathcode\undefined
    \let\Umathcode\XeTeXmathcode
  \fi
  \def\XeTeXcheck#1{}
  \ifx\XeTeXversion\undefined
  \else
    \def\XeTeXcheck.#1.#2-#3\relax{#1}
     \ifnum\expandafter\XeTeXcheck\XeTeXrevision.-\relax>996 %
       \def\XeTeXcheck#1{}
     \else
       \def\XeTeXcheck#1{%
          \ifnum"#1>"FFFF %
            \long\def\XeTeXcheck##1\endgroup{\endgroup}
            \expandafter\XeTeXcheck
          \fi
       }
     \fi
  \fi
  \L 0041 0041 0061
  \L 0042 0042 0062
  \L 0043 0043 0063
  \L 0044 0044 0064
  \L 0045 0045 0065
  \L 0046 0046 0066
  \L 0047 0047 0067
  \L 0048 0048 0068
  \L 0049 0049 0069
  \L 004A 004A 006A
  \L 004B 004B 006B
  \L 004C 004C 006C
  \L 004D 004D 006D
  \L 004E 004E 006E
  \L 004F 004F 006F
  \L 0050 0050 0070
  \L 0051 0051 0071
  \L 0052 0052 0072
  \L 0053 0053 0073
  \L 0054 0054 0074
  \L 0055 0055 0075
  \L 0056 0056 0076
  \L 0057 0057 0077
  \L 0058 0058 0078
  \L 0059 0059 0079
  \L 005A 005A 007A
  \L 0061 0041 0061
  \L 0062 0042 0062
  \L 0063 0043 0063
  \L 0064 0044 0064
  \L 0065 0045 0065
  \L 0066 0046 0066
  \L 0067 0047 0067
  \L 0068 0048 0068
  \L 0069 0049 0069
  \L 006A 004A 006A
  \L 006B 004B 006B
  \L 006C 004C 006C
  \L 006D 004D 006D
  \L 006E 004E 006E
  \L 006F 004F 006F
  \L 0070 0050 0070
  \L 0071 0051 0071
  \L 0072 0052 0072
  \L 0073 0053 0073
  \L 0074 0054 0074
  \L 0075 0055 0075
  \L 0076 0056 0076
  \L 0077 0057 0077
  \L 0078 0058 0078
  \L 0079 0059 0079
  \L 007A 005A 007A
  \l 00AA
  \L 00B5 039C 00B5
  \l 00BA
  \L 00C0 00C0 00E0
  \L 00C1 00C1 00E1
  \L 00C2 00C2 00E2
  \L 00C3 00C3 00E3
  \L 00C4 00C4 00E4
  \L 00C5 00C5 00E5
  \L 00C6 00C6 00E6
  \L 00C7 00C7 00E7
  \L 00C8 00C8 00E8
  \L 00C9 00C9 00E9
  \L 00CA 00CA 00EA
  \L 00CB 00CB 00EB
  \L 00CC 00CC 00EC
  \L 00CD 00CD 00ED
  \L 00CE 00CE 00EE
  \L 00CF 00CF 00EF
  \L 00D0 00D0 00F0
  \L 00D1 00D1 00F1
  \L 00D2 00D2 00F2
  \L 00D3 00D3 00F3
  \L 00D4 00D4 00F4
  \L 00D5 00D5 00F5
  \L 00D6 00D6 00F6
  \L 00D8 00D8 00F8
  \L 00D9 00D9 00F9
  \L 00DA 00DA 00FA
  \L 00DB 00DB 00FB
  \L 00DC 00DC 00FC
  \L 00DD 00DD 00FD
  \L 00DE 00DE 00FE
  \l 00DF
  \L 00E0 00C0 00E0
  \L 00E1 00C1 00E1
  \L 00E2 00C2 00E2
  \L 00E3 00C3 00E3
  \L 00E4 00C4 00E4
  \L 00E5 00C5 00E5
  \L 00E6 00C6 00E6
  \L 00E7 00C7 00E7
  \L 00E8 00C8 00E8
  \L 00E9 00C9 00E9
  \L 00EA 00CA 00EA
  \L 00EB 00CB 00EB
  \L 00EC 00CC 00EC
  \L 00ED 00CD 00ED
  \L 00EE 00CE 00EE
  \L 00EF 00CF 00EF
  \L 00F0 00D0 00F0
  \L 00F1 00D1 00F1
  \L 00F2 00D2 00F2
  \L 00F3 00D3 00F3
  \L 00F4 00D4 00F4
  \L 00F5 00D5 00F5
  \L 00F6 00D6 00F6
  \L 00F8 00D8 00F8
  \L 00F9 00D9 00F9
  \L 00FA 00DA 00FA
  \L 00FB 00DB 00FB
  \L 00FC 00DC 00FC
  \L 00FD 00DD 00FD
  \L 00FE 00DE 00FE
  \L 00FF 0178 00FF
  \L 0100 0100 0101
  \L 0101 0100 0101
  \L 0102 0102 0103
  \L 0103 0102 0103
  \L 0104 0104 0105
  \L 0105 0104 0105
  \L 0106 0106 0107
  \L 0107 0106 0107
  \L 0108 0108 0109
  \L 0109 0108 0109
  \L 010A 010A 010B
  \L 010B 010A 010B
  \L 010C 010C 010D
  \L 010D 010C 010D
  \L 010E 010E 010F
  \L 010F 010E 010F
  \L 0110 0110 0111
  \L 0111 0110 0111
  \L 0112 0112 0113
  \L 0113 0112 0113
  \L 0114 0114 0115
  \L 0115 0114 0115
  \L 0116 0116 0117
  \L 0117 0116 0117
  \L 0118 0118 0119
  \L 0119 0118 0119
  \L 011A 011A 011B
  \L 011B 011A 011B
  \L 011C 011C 011D
  \L 011D 011C 011D
  \L 011E 011E 011F
  \L 011F 011E 011F
  \L 0120 0120 0121
  \L 0121 0120 0121
  \L 0122 0122 0123
  \L 0123 0122 0123
  \L 0124 0124 0125
  \L 0125 0124 0125
  \L 0126 0126 0127
  \L 0127 0126 0127
  \L 0128 0128 0129
  \L 0129 0128 0129
  \L 012A 012A 012B
  \L 012B 012A 012B
  \L 012C 012C 012D
  \L 012D 012C 012D
  \L 012E 012E 012F
  \L 012F 012E 012F
  \L 0130 0130 0069
  \L 0131 0049 0131
  \L 0132 0132 0133
  \L 0133 0132 0133
  \L 0134 0134 0135
  \L 0135 0134 0135
  \L 0136 0136 0137
  \L 0137 0136 0137
  \l 0138
  \L 0139 0139 013A
  \L 013A 0139 013A
  \L 013B 013B 013C
  \L 013C 013B 013C
  \L 013D 013D 013E
  \L 013E 013D 013E
  \L 013F 013F 0140
  \L 0140 013F 0140
  \L 0141 0141 0142
  \L 0142 0141 0142
  \L 0143 0143 0144
  \L 0144 0143 0144
  \L 0145 0145 0146
  \L 0146 0145 0146
  \L 0147 0147 0148
  \L 0148 0147 0148
  \l 0149
  \L 014A 014A 014B
  \L 014B 014A 014B
  \L 014C 014C 014D
  \L 014D 014C 014D
  \L 014E 014E 014F
  \L 014F 014E 014F
  \L 0150 0150 0151
  \L 0151 0150 0151
  \L 0152 0152 0153
  \L 0153 0152 0153
  \L 0154 0154 0155
  \L 0155 0154 0155
  \L 0156 0156 0157
  \L 0157 0156 0157
  \L 0158 0158 0159
  \L 0159 0158 0159
  \L 015A 015A 015B
  \L 015B 015A 015B
  \L 015C 015C 015D
  \L 015D 015C 015D
  \L 015E 015E 015F
  \L 015F 015E 015F
  \L 0160 0160 0161
  \L 0161 0160 0161
  \L 0162 0162 0163
  \L 0163 0162 0163
  \L 0164 0164 0165
  \L 0165 0164 0165
  \L 0166 0166 0167
  \L 0167 0166 0167
  \L 0168 0168 0169
  \L 0169 0168 0169
  \L 016A 016A 016B
  \L 016B 016A 016B
  \L 016C 016C 016D
  \L 016D 016C 016D
  \L 016E 016E 016F
  \L 016F 016E 016F
  \L 0170 0170 0171
  \L 0171 0170 0171
  \L 0172 0172 0173
  \L 0173 0172 0173
  \L 0174 0174 0175
  \L 0175 0174 0175
  \L 0176 0176 0177
  \L 0177 0176 0177
  \L 0178 0178 00FF
  \L 0179 0179 017A
  \L 017A 0179 017A
  \L 017B 017B 017C
  \L 017C 017B 017C
  \L 017D 017D 017E
  \L 017E 017D 017E
  \L 017F 0053 017F
  \L 0180 0243 0180
  \L 0181 0181 0253
  \L 0182 0182 0183
  \L 0183 0182 0183
  \L 0184 0184 0185
  \L 0185 0184 0185
  \L 0186 0186 0254
  \L 0187 0187 0188
  \L 0188 0187 0188
  \L 0189 0189 0256
  \L 018A 018A 0257
  \L 018B 018B 018C
  \L 018C 018B 018C
  \l 018D
  \L 018E 018E 01DD
  \L 018F 018F 0259
  \L 0190 0190 025B
  \L 0191 0191 0192
  \L 0192 0191 0192
  \L 0193 0193 0260
  \L 0194 0194 0263
  \L 0195 01F6 0195
  \L 0196 0196 0269
  \L 0197 0197 0268
  \L 0198 0198 0199
  \L 0199 0198 0199
  \L 019A 023D 019A
  \l 019B
  \L 019C 019C 026F
  \L 019D 019D 0272
  \L 019E 0220 019E
  \L 019F 019F 0275
  \L 01A0 01A0 01A1
  \L 01A1 01A0 01A1
  \L 01A2 01A2 01A3
  \L 01A3 01A2 01A3
  \L 01A4 01A4 01A5
  \L 01A5 01A4 01A5
  \L 01A6 01A6 0280
  \L 01A7 01A7 01A8
  \L 01A8 01A7 01A8
  \L 01A9 01A9 0283
  \l 01AA
  \l 01AB
  \L 01AC 01AC 01AD
  \L 01AD 01AC 01AD
  \L 01AE 01AE 0288
  \L 01AF 01AF 01B0
  \L 01B0 01AF 01B0
  \L 01B1 01B1 028A
  \L 01B2 01B2 028B
  \L 01B3 01B3 01B4
  \L 01B4 01B3 01B4
  \L 01B5 01B5 01B6
  \L 01B6 01B5 01B6
  \L 01B7 01B7 0292
  \L 01B8 01B8 01B9
  \L 01B9 01B8 01B9
  \l 01BA
  \l 01BB
  \L 01BC 01BC 01BD
  \L 01BD 01BC 01BD
  \l 01BE
  \L 01BF 01F7 01BF
  \l 01C0
  \l 01C1
  \l 01C2
  \l 01C3
  \L 01C4 01C4 01C6
  \L 01C5 01C4 01C6
  \L 01C6 01C4 01C6
  \L 01C7 01C7 01C9
  \L 01C8 01C7 01C9
  \L 01C9 01C7 01C9
  \L 01CA 01CA 01CC
  \L 01CB 01CA 01CC
  \L 01CC 01CA 01CC
  \L 01CD 01CD 01CE
  \L 01CE 01CD 01CE
  \L 01CF 01CF 01D0
  \L 01D0 01CF 01D0
  \L 01D1 01D1 01D2
  \L 01D2 01D1 01D2
  \L 01D3 01D3 01D4
  \L 01D4 01D3 01D4
  \L 01D5 01D5 01D6
  \L 01D6 01D5 01D6
  \L 01D7 01D7 01D8
  \L 01D8 01D7 01D8
  \L 01D9 01D9 01DA
  \L 01DA 01D9 01DA
  \L 01DB 01DB 01DC
  \L 01DC 01DB 01DC
  \L 01DD 018E 01DD
  \L 01DE 01DE 01DF
  \L 01DF 01DE 01DF
  \L 01E0 01E0 01E1
  \L 01E1 01E0 01E1
  \L 01E2 01E2 01E3
  \L 01E3 01E2 01E3
  \L 01E4 01E4 01E5
  \L 01E5 01E4 01E5
  \L 01E6 01E6 01E7
  \L 01E7 01E6 01E7
  \L 01E8 01E8 01E9
  \L 01E9 01E8 01E9
  \L 01EA 01EA 01EB
  \L 01EB 01EA 01EB
  \L 01EC 01EC 01ED
  \L 01ED 01EC 01ED
  \L 01EE 01EE 01EF
  \L 01EF 01EE 01EF
  \l 01F0
  \L 01F1 01F1 01F3
  \L 01F2 01F1 01F3
  \L 01F3 01F1 01F3
  \L 01F4 01F4 01F5
  \L 01F5 01F4 01F5
  \L 01F6 01F6 0195
  \L 01F7 01F7 01BF
  \L 01F8 01F8 01F9
  \L 01F9 01F8 01F9
  \L 01FA 01FA 01FB
  \L 01FB 01FA 01FB
  \L 01FC 01FC 01FD
  \L 01FD 01FC 01FD
  \L 01FE 01FE 01FF
  \L 01FF 01FE 01FF
  \L 0200 0200 0201
  \L 0201 0200 0201
  \L 0202 0202 0203
  \L 0203 0202 0203
  \L 0204 0204 0205
  \L 0205 0204 0205
  \L 0206 0206 0207
  \L 0207 0206 0207
  \L 0208 0208 0209
  \L 0209 0208 0209
  \L 020A 020A 020B
  \L 020B 020A 020B
  \L 020C 020C 020D
  \L 020D 020C 020D
  \L 020E 020E 020F
  \L 020F 020E 020F
  \L 0210 0210 0211
  \L 0211 0210 0211
  \L 0212 0212 0213
  \L 0213 0212 0213
  \L 0214 0214 0215
  \L 0215 0214 0215
  \L 0216 0216 0217
  \L 0217 0216 0217
  \L 0218 0218 0219
  \L 0219 0218 0219
  \L 021A 021A 021B
  \L 021B 021A 021B
  \L 021C 021C 021D
  \L 021D 021C 021D
  \L 021E 021E 021F
  \L 021F 021E 021F
  \L 0220 0220 019E
  \l 0221
  \L 0222 0222 0223
  \L 0223 0222 0223
  \L 0224 0224 0225
  \L 0225 0224 0225
  \L 0226 0226 0227
  \L 0227 0226 0227
  \L 0228 0228 0229
  \L 0229 0228 0229
  \L 022A 022A 022B
  \L 022B 022A 022B
  \L 022C 022C 022D
  \L 022D 022C 022D
  \L 022E 022E 022F
  \L 022F 022E 022F
  \L 0230 0230 0231
  \L 0231 0230 0231
  \L 0232 0232 0233
  \L 0233 0232 0233
  \l 0234
  \l 0235
  \l 0236
  \l 0237
  \l 0238
  \l 0239
  \L 023A 023A 2C65
  \L 023B 023B 023C
  \L 023C 023B 023C
  \L 023D 023D 019A
  \L 023E 023E 2C66
  \L 023F 2C7E 023F
  \L 0240 2C7F 0240
  \L 0241 0241 0242
  \L 0242 0241 0242
  \L 0243 0243 0180
  \L 0244 0244 0289
  \L 0245 0245 028C
  \L 0246 0246 0247
  \L 0247 0246 0247
  \L 0248 0248 0249
  \L 0249 0248 0249
  \L 024A 024A 024B
  \L 024B 024A 024B
  \L 024C 024C 024D
  \L 024D 024C 024D
  \L 024E 024E 024F
  \L 024F 024E 024F
  \L 0250 2C6F 0250
  \L 0251 2C6D 0251
  \L 0252 2C70 0252
  \L 0253 0181 0253
  \L 0254 0186 0254
  \l 0255
  \L 0256 0189 0256
  \L 0257 018A 0257
  \l 0258
  \L 0259 018F 0259
  \l 025A
  \L 025B 0190 025B
  \L 025C A7AB 025C
  \l 025D
  \l 025E
  \l 025F
  \L 0260 0193 0260
  \L 0261 A7AC 0261
  \l 0262
  \L 0263 0194 0263
  \l 0264
  \L 0265 A78D 0265
  \L 0266 A7AA 0266
  \l 0267
  \L 0268 0197 0268
  \L 0269 0196 0269
  \l 026A
  \L 026B 2C62 026B
  \L 026C A7AD 026C
  \l 026D
  \l 026E
  \L 026F 019C 026F
  \l 0270
  \L 0271 2C6E 0271
  \L 0272 019D 0272
  \l 0273
  \l 0274
  \L 0275 019F 0275
  \l 0276
  \l 0277
  \l 0278
  \l 0279
  \l 027A
  \l 027B
  \l 027C
  \L 027D 2C64 027D
  \l 027E
  \l 027F
  \L 0280 01A6 0280
  \l 0281
  \l 0282
  \L 0283 01A9 0283
  \l 0284
  \l 0285
  \l 0286
  \L 0287 A7B1 0287
  \L 0288 01AE 0288
  \L 0289 0244 0289
  \L 028A 01B1 028A
  \L 028B 01B2 028B
  \L 028C 0245 028C
  \l 028D
  \l 028E
  \l 028F
  \l 0290
  \l 0291
  \L 0292 01B7 0292
  \l 0293
  \l 0294
  \l 0295
  \l 0296
  \l 0297
  \l 0298
  \l 0299
  \l 029A
  \l 029B
  \l 029C
  \l 029D
  \L 029E A7B0 029E
  \l 029F
  \l 02A0
  \l 02A1
  \l 02A2
  \l 02A3
  \l 02A4
  \l 02A5
  \l 02A6
  \l 02A7
  \l 02A8
  \l 02A9
  \l 02AA
  \l 02AB
  \l 02AC
  \l 02AD
  \l 02AE
  \l 02AF
  \l 02B0
  \l 02B1
  \l 02B2
  \l 02B3
  \l 02B4
  \l 02B5
  \l 02B6
  \l 02B7
  \l 02B8
  \l 02B9
  \l 02BA
  \l 02BB
  \l 02BC
  \l 02BD
  \l 02BE
  \l 02BF
  \l 02C0
  \l 02C1
  \l 02C6
  \l 02C7
  \l 02C8
  \l 02C9
  \l 02CA
  \l 02CB
  \l 02CC
  \l 02CD
  \l 02CE
  \l 02CF
  \l 02D0
  \l 02D1
  \l 02E0
  \l 02E1
  \l 02E2
  \l 02E3
  \l 02E4
  \l 02EC
  \l 02EE
  \l 0300
  \l 0301
  \l 0302
  \l 0303
  \l 0304
  \l 0305
  \l 0306
  \l 0307
  \l 0308
  \l 0309
  \l 030A
  \l 030B
  \l 030C
  \l 030D
  \l 030E
  \l 030F
  \l 0310
  \l 0311
  \l 0312
  \l 0313
  \l 0314
  \l 0315
  \l 0316
  \l 0317
  \l 0318
  \l 0319
  \l 031A
  \l 031B
  \l 031C
  \l 031D
  \l 031E
  \l 031F
  \l 0320
  \l 0321
  \l 0322
  \l 0323
  \l 0324
  \l 0325
  \l 0326
  \l 0327
  \l 0328
  \l 0329
  \l 032A
  \l 032B
  \l 032C
  \l 032D
  \l 032E
  \l 032F
  \l 0330
  \l 0331
  \l 0332
  \l 0333
  \l 0334
  \l 0335
  \l 0336
  \l 0337
  \l 0338
  \l 0339
  \l 033A
  \l 033B
  \l 033C
  \l 033D
  \l 033E
  \l 033F
  \l 0340
  \l 0341
  \l 0342
  \l 0343
  \l 0344
  \L 0345 0399 0345
  \l 0346
  \l 0347
  \l 0348
  \l 0349
  \l 034A
  \l 034B
  \l 034C
  \l 034D
  \l 034E
  \l 034F
  \l 0350
  \l 0351
  \l 0352
  \l 0353
  \l 0354
  \l 0355
  \l 0356
  \l 0357
  \l 0358
  \l 0359
  \l 035A
  \l 035B
  \l 035C
  \l 035D
  \l 035E
  \l 035F
  \l 0360
  \l 0361
  \l 0362
  \l 0363
  \l 0364
  \l 0365
  \l 0366
  \l 0367
  \l 0368
  \l 0369
  \l 036A
  \l 036B
  \l 036C
  \l 036D
  \l 036E
  \l 036F
  \L 0370 0370 0371
  \L 0371 0370 0371
  \L 0372 0372 0373
  \L 0373 0372 0373
  \l 0374
  \L 0376 0376 0377
  \L 0377 0376 0377
  \l 037A
  \L 037B 03FD 037B
  \L 037C 03FE 037C
  \L 037D 03FF 037D
  \L 037F 037F 03F3
  \L 0386 0386 03AC
  \L 0388 0388 03AD
  \L 0389 0389 03AE
  \L 038A 038A 03AF
  \L 038C 038C 03CC
  \L 038E 038E 03CD
  \L 038F 038F 03CE
  \l 0390
  \L 0391 0391 03B1
  \L 0392 0392 03B2
  \L 0393 0393 03B3
  \L 0394 0394 03B4
  \L 0395 0395 03B5
  \L 0396 0396 03B6
  \L 0397 0397 03B7
  \L 0398 0398 03B8
  \L 0399 0399 03B9
  \L 039A 039A 03BA
  \L 039B 039B 03BB
  \L 039C 039C 03BC
  \L 039D 039D 03BD
  \L 039E 039E 03BE
  \L 039F 039F 03BF
  \L 03A0 03A0 03C0
  \L 03A1 03A1 03C1
  \L 03A3 03A3 03C3
  \L 03A4 03A4 03C4
  \L 03A5 03A5 03C5
  \L 03A6 03A6 03C6
  \L 03A7 03A7 03C7
  \L 03A8 03A8 03C8
  \L 03A9 03A9 03C9
  \L 03AA 03AA 03CA
  \L 03AB 03AB 03CB
  \L 03AC 0386 03AC
  \L 03AD 0388 03AD
  \L 03AE 0389 03AE
  \L 03AF 038A 03AF
  \l 03B0
  \L 03B1 0391 03B1
  \L 03B2 0392 03B2
  \L 03B3 0393 03B3
  \L 03B4 0394 03B4
  \L 03B5 0395 03B5
  \L 03B6 0396 03B6
  \L 03B7 0397 03B7
  \L 03B8 0398 03B8
  \L 03B9 0399 03B9
  \L 03BA 039A 03BA
  \L 03BB 039B 03BB
  \L 03BC 039C 03BC
  \L 03BD 039D 03BD
  \L 03BE 039E 03BE
  \L 03BF 039F 03BF
  \L 03C0 03A0 03C0
  \L 03C1 03A1 03C1
  \L 03C2 03A3 03C2
  \L 03C3 03A3 03C3
  \L 03C4 03A4 03C4
  \L 03C5 03A5 03C5
  \L 03C6 03A6 03C6
  \L 03C7 03A7 03C7
  \L 03C8 03A8 03C8
  \L 03C9 03A9 03C9
  \L 03CA 03AA 03CA
  \L 03CB 03AB 03CB
  \L 03CC 038C 03CC
  \L 03CD 038E 03CD
  \L 03CE 038F 03CE
  \L 03CF 03CF 03D7
  \L 03D0 0392 03D0
  \L 03D1 0398 03D1
  \l 03D2
  \l 03D3
  \l 03D4
  \L 03D5 03A6 03D5
  \L 03D6 03A0 03D6
  \L 03D7 03CF 03D7
  \L 03D8 03D8 03D9
  \L 03D9 03D8 03D9
  \L 03DA 03DA 03DB
  \L 03DB 03DA 03DB
  \L 03DC 03DC 03DD
  \L 03DD 03DC 03DD
  \L 03DE 03DE 03DF
  \L 03DF 03DE 03DF
  \L 03E0 03E0 03E1
  \L 03E1 03E0 03E1
  \L 03E2 03E2 03E3
  \L 03E3 03E2 03E3
  \L 03E4 03E4 03E5
  \L 03E5 03E4 03E5
  \L 03E6 03E6 03E7
  \L 03E7 03E6 03E7
  \L 03E8 03E8 03E9
  \L 03E9 03E8 03E9
  \L 03EA 03EA 03EB
  \L 03EB 03EA 03EB
  \L 03EC 03EC 03ED
  \L 03ED 03EC 03ED
  \L 03EE 03EE 03EF
  \L 03EF 03EE 03EF
  \L 03F0 039A 03F0
  \L 03F1 03A1 03F1
  \L 03F2 03F9 03F2
  \L 03F3 037F 03F3
  \L 03F4 03F4 03B8
  \L 03F5 0395 03F5
  \L 03F7 03F7 03F8
  \L 03F8 03F7 03F8
  \L 03F9 03F9 03F2
  \L 03FA 03FA 03FB
  \L 03FB 03FA 03FB
  \l 03FC
  \L 03FD 03FD 037B
  \L 03FE 03FE 037C
  \L 03FF 03FF 037D
  \L 0400 0400 0450
  \L 0401 0401 0451
  \L 0402 0402 0452
  \L 0403 0403 0453
  \L 0404 0404 0454
  \L 0405 0405 0455
  \L 0406 0406 0456
  \L 0407 0407 0457
  \L 0408 0408 0458
  \L 0409 0409 0459
  \L 040A 040A 045A
  \L 040B 040B 045B
  \L 040C 040C 045C
  \L 040D 040D 045D
  \L 040E 040E 045E
  \L 040F 040F 045F
  \L 0410 0410 0430
  \L 0411 0411 0431
  \L 0412 0412 0432
  \L 0413 0413 0433
  \L 0414 0414 0434
  \L 0415 0415 0435
  \L 0416 0416 0436
  \L 0417 0417 0437
  \L 0418 0418 0438
  \L 0419 0419 0439
  \L 041A 041A 043A
  \L 041B 041B 043B
  \L 041C 041C 043C
  \L 041D 041D 043D
  \L 041E 041E 043E
  \L 041F 041F 043F
  \L 0420 0420 0440
  \L 0421 0421 0441
  \L 0422 0422 0442
  \L 0423 0423 0443
  \L 0424 0424 0444
  \L 0425 0425 0445
  \L 0426 0426 0446
  \L 0427 0427 0447
  \L 0428 0428 0448
  \L 0429 0429 0449
  \L 042A 042A 044A
  \L 042B 042B 044B
  \L 042C 042C 044C
  \L 042D 042D 044D
  \L 042E 042E 044E
  \L 042F 042F 044F
  \L 0430 0410 0430
  \L 0431 0411 0431
  \L 0432 0412 0432
  \L 0433 0413 0433
  \L 0434 0414 0434
  \L 0435 0415 0435
  \L 0436 0416 0436
  \L 0437 0417 0437
  \L 0438 0418 0438
  \L 0439 0419 0439
  \L 043A 041A 043A
  \L 043B 041B 043B
  \L 043C 041C 043C
  \L 043D 041D 043D
  \L 043E 041E 043E
  \L 043F 041F 043F
  \L 0440 0420 0440
  \L 0441 0421 0441
  \L 0442 0422 0442
  \L 0443 0423 0443
  \L 0444 0424 0444
  \L 0445 0425 0445
  \L 0446 0426 0446
  \L 0447 0427 0447
  \L 0448 0428 0448
  \L 0449 0429 0449
  \L 044A 042A 044A
  \L 044B 042B 044B
  \L 044C 042C 044C
  \L 044D 042D 044D
  \L 044E 042E 044E
  \L 044F 042F 044F
  \L 0450 0400 0450
  \L 0451 0401 0451
  \L 0452 0402 0452
  \L 0453 0403 0453
  \L 0454 0404 0454
  \L 0455 0405 0455
  \L 0456 0406 0456
  \L 0457 0407 0457
  \L 0458 0408 0458
  \L 0459 0409 0459
  \L 045A 040A 045A
  \L 045B 040B 045B
  \L 045C 040C 045C
  \L 045D 040D 045D
  \L 045E 040E 045E
  \L 045F 040F 045F
  \L 0460 0460 0461
  \L 0461 0460 0461
  \L 0462 0462 0463
  \L 0463 0462 0463
  \L 0464 0464 0465
  \L 0465 0464 0465
  \L 0466 0466 0467
  \L 0467 0466 0467
  \L 0468 0468 0469
  \L 0469 0468 0469
  \L 046A 046A 046B
  \L 046B 046A 046B
  \L 046C 046C 046D
  \L 046D 046C 046D
  \L 046E 046E 046F
  \L 046F 046E 046F
  \L 0470 0470 0471
  \L 0471 0470 0471
  \L 0472 0472 0473
  \L 0473 0472 0473
  \L 0474 0474 0475
  \L 0475 0474 0475
  \L 0476 0476 0477
  \L 0477 0476 0477
  \L 0478 0478 0479
  \L 0479 0478 0479
  \L 047A 047A 047B
  \L 047B 047A 047B
  \L 047C 047C 047D
  \L 047D 047C 047D
  \L 047E 047E 047F
  \L 047F 047E 047F
  \L 0480 0480 0481
  \L 0481 0480 0481
  \l 0483
  \l 0484
  \l 0485
  \l 0486
  \l 0487
  \l 0488
  \l 0489
  \L 048A 048A 048B
  \L 048B 048A 048B
  \L 048C 048C 048D
  \L 048D 048C 048D
  \L 048E 048E 048F
  \L 048F 048E 048F
  \L 0490 0490 0491
  \L 0491 0490 0491
  \L 0492 0492 0493
  \L 0493 0492 0493
  \L 0494 0494 0495
  \L 0495 0494 0495
  \L 0496 0496 0497
  \L 0497 0496 0497
  \L 0498 0498 0499
  \L 0499 0498 0499
  \L 049A 049A 049B
  \L 049B 049A 049B
  \L 049C 049C 049D
  \L 049D 049C 049D
  \L 049E 049E 049F
  \L 049F 049E 049F
  \L 04A0 04A0 04A1
  \L 04A1 04A0 04A1
  \L 04A2 04A2 04A3
  \L 04A3 04A2 04A3
  \L 04A4 04A4 04A5
  \L 04A5 04A4 04A5
  \L 04A6 04A6 04A7
  \L 04A7 04A6 04A7
  \L 04A8 04A8 04A9
  \L 04A9 04A8 04A9
  \L 04AA 04AA 04AB
  \L 04AB 04AA 04AB
  \L 04AC 04AC 04AD
  \L 04AD 04AC 04AD
  \L 04AE 04AE 04AF
  \L 04AF 04AE 04AF
  \L 04B0 04B0 04B1
  \L 04B1 04B0 04B1
  \L 04B2 04B2 04B3
  \L 04B3 04B2 04B3
  \L 04B4 04B4 04B5
  \L 04B5 04B4 04B5
  \L 04B6 04B6 04B7
  \L 04B7 04B6 04B7
  \L 04B8 04B8 04B9
  \L 04B9 04B8 04B9
  \L 04BA 04BA 04BB
  \L 04BB 04BA 04BB
  \L 04BC 04BC 04BD
  \L 04BD 04BC 04BD
  \L 04BE 04BE 04BF
  \L 04BF 04BE 04BF
  \L 04C0 04C0 04CF
  \L 04C1 04C1 04C2
  \L 04C2 04C1 04C2
  \L 04C3 04C3 04C4
  \L 04C4 04C3 04C4
  \L 04C5 04C5 04C6
  \L 04C6 04C5 04C6
  \L 04C7 04C7 04C8
  \L 04C8 04C7 04C8
  \L 04C9 04C9 04CA
  \L 04CA 04C9 04CA
  \L 04CB 04CB 04CC
  \L 04CC 04CB 04CC
  \L 04CD 04CD 04CE
  \L 04CE 04CD 04CE
  \L 04CF 04C0 04CF
  \L 04D0 04D0 04D1
  \L 04D1 04D0 04D1
  \L 04D2 04D2 04D3
  \L 04D3 04D2 04D3
  \L 04D4 04D4 04D5
  \L 04D5 04D4 04D5
  \L 04D6 04D6 04D7
  \L 04D7 04D6 04D7
  \L 04D8 04D8 04D9
  \L 04D9 04D8 04D9
  \L 04DA 04DA 04DB
  \L 04DB 04DA 04DB
  \L 04DC 04DC 04DD
  \L 04DD 04DC 04DD
  \L 04DE 04DE 04DF
  \L 04DF 04DE 04DF
  \L 04E0 04E0 04E1
  \L 04E1 04E0 04E1
  \L 04E2 04E2 04E3
  \L 04E3 04E2 04E3
  \L 04E4 04E4 04E5
  \L 04E5 04E4 04E5
  \L 04E6 04E6 04E7
  \L 04E7 04E6 04E7
  \L 04E8 04E8 04E9
  \L 04E9 04E8 04E9
  \L 04EA 04EA 04EB
  \L 04EB 04EA 04EB
  \L 04EC 04EC 04ED
  \L 04ED 04EC 04ED
  \L 04EE 04EE 04EF
  \L 04EF 04EE 04EF
  \L 04F0 04F0 04F1
  \L 04F1 04F0 04F1
  \L 04F2 04F2 04F3
  \L 04F3 04F2 04F3
  \L 04F4 04F4 04F5
  \L 04F5 04F4 04F5
  \L 04F6 04F6 04F7
  \L 04F7 04F6 04F7
  \L 04F8 04F8 04F9
  \L 04F9 04F8 04F9
  \L 04FA 04FA 04FB
  \L 04FB 04FA 04FB
  \L 04FC 04FC 04FD
  \L 04FD 04FC 04FD
  \L 04FE 04FE 04FF
  \L 04FF 04FE 04FF
  \L 0500 0500 0501
  \L 0501 0500 0501
  \L 0502 0502 0503
  \L 0503 0502 0503
  \L 0504 0504 0505
  \L 0505 0504 0505
  \L 0506 0506 0507
  \L 0507 0506 0507
  \L 0508 0508 0509
  \L 0509 0508 0509
  \L 050A 050A 050B
  \L 050B 050A 050B
  \L 050C 050C 050D
  \L 050D 050C 050D
  \L 050E 050E 050F
  \L 050F 050E 050F
  \L 0510 0510 0511
  \L 0511 0510 0511
  \L 0512 0512 0513
  \L 0513 0512 0513
  \L 0514 0514 0515
  \L 0515 0514 0515
  \L 0516 0516 0517
  \L 0517 0516 0517
  \L 0518 0518 0519
  \L 0519 0518 0519
  \L 051A 051A 051B
  \L 051B 051A 051B
  \L 051C 051C 051D
  \L 051D 051C 051D
  \L 051E 051E 051F
  \L 051F 051E 051F
  \L 0520 0520 0521
  \L 0521 0520 0521
  \L 0522 0522 0523
  \L 0523 0522 0523
  \L 0524 0524 0525
  \L 0525 0524 0525
  \L 0526 0526 0527
  \L 0527 0526 0527
  \L 0528 0528 0529
  \L 0529 0528 0529
  \L 052A 052A 052B
  \L 052B 052A 052B
  \L 052C 052C 052D
  \L 052D 052C 052D
  \L 052E 052E 052F
  \L 052F 052E 052F
  \L 0531 0531 0561
  \L 0532 0532 0562
  \L 0533 0533 0563
  \L 0534 0534 0564
  \L 0535 0535 0565
  \L 0536 0536 0566
  \L 0537 0537 0567
  \L 0538 0538 0568
  \L 0539 0539 0569
  \L 053A 053A 056A
  \L 053B 053B 056B
  \L 053C 053C 056C
  \L 053D 053D 056D
  \L 053E 053E 056E
  \L 053F 053F 056F
  \L 0540 0540 0570
  \L 0541 0541 0571
  \L 0542 0542 0572
  \L 0543 0543 0573
  \L 0544 0544 0574
  \L 0545 0545 0575
  \L 0546 0546 0576
  \L 0547 0547 0577
  \L 0548 0548 0578
  \L 0549 0549 0579
  \L 054A 054A 057A
  \L 054B 054B 057B
  \L 054C 054C 057C
  \L 054D 054D 057D
  \L 054E 054E 057E
  \L 054F 054F 057F
  \L 0550 0550 0580
  \L 0551 0551 0581
  \L 0552 0552 0582
  \L 0553 0553 0583
  \L 0554 0554 0584
  \L 0555 0555 0585
  \L 0556 0556 0586
  \l 0559
  \L 0561 0531 0561
  \L 0562 0532 0562
  \L 0563 0533 0563
  \L 0564 0534 0564
  \L 0565 0535 0565
  \L 0566 0536 0566
  \L 0567 0537 0567
  \L 0568 0538 0568
  \L 0569 0539 0569
  \L 056A 053A 056A
  \L 056B 053B 056B
  \L 056C 053C 056C
  \L 056D 053D 056D
  \L 056E 053E 056E
  \L 056F 053F 056F
  \L 0570 0540 0570
  \L 0571 0541 0571
  \L 0572 0542 0572
  \L 0573 0543 0573
  \L 0574 0544 0574
  \L 0575 0545 0575
  \L 0576 0546 0576
  \L 0577 0547 0577
  \L 0578 0548 0578
  \L 0579 0549 0579
  \L 057A 054A 057A
  \L 057B 054B 057B
  \L 057C 054C 057C
  \L 057D 054D 057D
  \L 057E 054E 057E
  \L 057F 054F 057F
  \L 0580 0550 0580
  \L 0581 0551 0581
  \L 0582 0552 0582
  \L 0583 0553 0583
  \L 0584 0554 0584
  \L 0585 0555 0585
  \L 0586 0556 0586
  \l 0587
  \l 0591
  \l 0592
  \l 0593
  \l 0594
  \l 0595
  \l 0596
  \l 0597
  \l 0598
  \l 0599
  \l 059A
  \l 059B
  \l 059C
  \l 059D
  \l 059E
  \l 059F
  \l 05A0
  \l 05A1
  \l 05A2
  \l 05A3
  \l 05A4
  \l 05A5
  \l 05A6
  \l 05A7
  \l 05A8
  \l 05A9
  \l 05AA
  \l 05AB
  \l 05AC
  \l 05AD
  \l 05AE
  \l 05AF
  \l 05B0
  \l 05B1
  \l 05B2
  \l 05B3
  \l 05B4
  \l 05B5
  \l 05B6
  \l 05B7
  \l 05B8
  \l 05B9
  \l 05BA
  \l 05BB
  \l 05BC
  \l 05BD
  \l 05BF
  \l 05C1
  \l 05C2
  \l 05C4
  \l 05C5
  \l 05C7
  \l 05D0
  \l 05D1
  \l 05D2
  \l 05D3
  \l 05D4
  \l 05D5
  \l 05D6
  \l 05D7
  \l 05D8
  \l 05D9
  \l 05DA
  \l 05DB
  \l 05DC
  \l 05DD
  \l 05DE
  \l 05DF
  \l 05E0
  \l 05E1
  \l 05E2
  \l 05E3
  \l 05E4
  \l 05E5
  \l 05E6
  \l 05E7
  \l 05E8
  \l 05E9
  \l 05EA
  \l 05F0
  \l 05F1
  \l 05F2
  \l 0610
  \l 0611
  \l 0612
  \l 0613
  \l 0614
  \l 0615
  \l 0616
  \l 0617
  \l 0618
  \l 0619
  \l 061A
  \l 0620
  \l 0621
  \l 0622
  \l 0623
  \l 0624
  \l 0625
  \l 0626
  \l 0627
  \l 0628
  \l 0629
  \l 062A
  \l 062B
  \l 062C
  \l 062D
  \l 062E
  \l 062F
  \l 0630
  \l 0631
  \l 0632
  \l 0633
  \l 0634
  \l 0635
  \l 0636
  \l 0637
  \l 0638
  \l 0639
  \l 063A
  \l 063B
  \l 063C
  \l 063D
  \l 063E
  \l 063F
  \l 0640
  \l 0641
  \l 0642
  \l 0643
  \l 0644
  \l 0645
  \l 0646
  \l 0647
  \l 0648
  \l 0649
  \l 064A
  \l 064B
  \l 064C
  \l 064D
  \l 064E
  \l 064F
  \l 0650
  \l 0651
  \l 0652
  \l 0653
  \l 0654
  \l 0655
  \l 0656
  \l 0657
  \l 0658
  \l 0659
  \l 065A
  \l 065B
  \l 065C
  \l 065D
  \l 065E
  \l 065F
  \l 066E
  \l 066F
  \l 0670
  \l 0671
  \l 0672
  \l 0673
  \l 0674
  \l 0675
  \l 0676
  \l 0677
  \l 0678
  \l 0679
  \l 067A
  \l 067B
  \l 067C
  \l 067D
  \l 067E
  \l 067F
  \l 0680
  \l 0681
  \l 0682
  \l 0683
  \l 0684
  \l 0685
  \l 0686
  \l 0687
  \l 0688
  \l 0689
  \l 068A
  \l 068B
  \l 068C
  \l 068D
  \l 068E
  \l 068F
  \l 0690
  \l 0691
  \l 0692
  \l 0693
  \l 0694
  \l 0695
  \l 0696
  \l 0697
  \l 0698
  \l 0699
  \l 069A
  \l 069B
  \l 069C
  \l 069D
  \l 069E
  \l 069F
  \l 06A0
  \l 06A1
  \l 06A2
  \l 06A3
  \l 06A4
  \l 06A5
  \l 06A6
  \l 06A7
  \l 06A8
  \l 06A9
  \l 06AA
  \l 06AB
  \l 06AC
  \l 06AD
  \l 06AE
  \l 06AF
  \l 06B0
  \l 06B1
  \l 06B2
  \l 06B3
  \l 06B4
  \l 06B5
  \l 06B6
  \l 06B7
  \l 06B8
  \l 06B9
  \l 06BA
  \l 06BB
  \l 06BC
  \l 06BD
  \l 06BE
  \l 06BF
  \l 06C0
  \l 06C1
  \l 06C2
  \l 06C3
  \l 06C4
  \l 06C5
  \l 06C6
  \l 06C7
  \l 06C8
  \l 06C9
  \l 06CA
  \l 06CB
  \l 06CC
  \l 06CD
  \l 06CE
  \l 06CF
  \l 06D0
  \l 06D1
  \l 06D2
  \l 06D3
  \l 06D5
  \l 06D6
  \l 06D7
  \l 06D8
  \l 06D9
  \l 06DA
  \l 06DB
  \l 06DC
  \l 06DF
  \l 06E0
  \l 06E1
  \l 06E2
  \l 06E3
  \l 06E4
  \l 06E5
  \l 06E6
  \l 06E7
  \l 06E8
  \l 06EA
  \l 06EB
  \l 06EC
  \l 06ED
  \l 06EE
  \l 06EF
  \l 06FA
  \l 06FB
  \l 06FC
  \l 06FF
  \l 0710
  \l 0711
  \l 0712
  \l 0713
  \l 0714
  \l 0715
  \l 0716
  \l 0717
  \l 0718
  \l 0719
  \l 071A
  \l 071B
  \l 071C
  \l 071D
  \l 071E
  \l 071F
  \l 0720
  \l 0721
  \l 0722
  \l 0723
  \l 0724
  \l 0725
  \l 0726
  \l 0727
  \l 0728
  \l 0729
  \l 072A
  \l 072B
  \l 072C
  \l 072D
  \l 072E
  \l 072F
  \l 0730
  \l 0731
  \l 0732
  \l 0733
  \l 0734
  \l 0735
  \l 0736
  \l 0737
  \l 0738
  \l 0739
  \l 073A
  \l 073B
  \l 073C
  \l 073D
  \l 073E
  \l 073F
  \l 0740
  \l 0741
  \l 0742
  \l 0743
  \l 0744
  \l 0745
  \l 0746
  \l 0747
  \l 0748
  \l 0749
  \l 074A
  \l 074D
  \l 074E
  \l 074F
  \l 0750
  \l 0751
  \l 0752
  \l 0753
  \l 0754
  \l 0755
  \l 0756
  \l 0757
  \l 0758
  \l 0759
  \l 075A
  \l 075B
  \l 075C
  \l 075D
  \l 075E
  \l 075F
  \l 0760
  \l 0761
  \l 0762
  \l 0763
  \l 0764
  \l 0765
  \l 0766
  \l 0767
  \l 0768
  \l 0769
  \l 076A
  \l 076B
  \l 076C
  \l 076D
  \l 076E
  \l 076F
  \l 0770
  \l 0771
  \l 0772
  \l 0773
  \l 0774
  \l 0775
  \l 0776
  \l 0777
  \l 0778
  \l 0779
  \l 077A
  \l 077B
  \l 077C
  \l 077D
  \l 077E
  \l 077F
  \l 0780
  \l 0781
  \l 0782
  \l 0783
  \l 0784
  \l 0785
  \l 0786
  \l 0787
  \l 0788
  \l 0789
  \l 078A
  \l 078B
  \l 078C
  \l 078D
  \l 078E
  \l 078F
  \l 0790
  \l 0791
  \l 0792
  \l 0793
  \l 0794
  \l 0795
  \l 0796
  \l 0797
  \l 0798
  \l 0799
  \l 079A
  \l 079B
  \l 079C
  \l 079D
  \l 079E
  \l 079F
  \l 07A0
  \l 07A1
  \l 07A2
  \l 07A3
  \l 07A4
  \l 07A5
  \l 07A6
  \l 07A7
  \l 07A8
  \l 07A9
  \l 07AA
  \l 07AB
  \l 07AC
  \l 07AD
  \l 07AE
  \l 07AF
  \l 07B0
  \l 07B1
  \l 07CA
  \l 07CB
  \l 07CC
  \l 07CD
  \l 07CE
  \l 07CF
  \l 07D0
  \l 07D1
  \l 07D2
  \l 07D3
  \l 07D4
  \l 07D5
  \l 07D6
  \l 07D7
  \l 07D8
  \l 07D9
  \l 07DA
  \l 07DB
  \l 07DC
  \l 07DD
  \l 07DE
  \l 07DF
  \l 07E0
  \l 07E1
  \l 07E2
  \l 07E3
  \l 07E4
  \l 07E5
  \l 07E6
  \l 07E7
  \l 07E8
  \l 07E9
  \l 07EA
  \l 07EB
  \l 07EC
  \l 07ED
  \l 07EE
  \l 07EF
  \l 07F0
  \l 07F1
  \l 07F2
  \l 07F3
  \l 07F4
  \l 07F5
  \l 07FA
  \l 0800
  \l 0801
  \l 0802
  \l 0803
  \l 0804
  \l 0805
  \l 0806
  \l 0807
  \l 0808
  \l 0809
  \l 080A
  \l 080B
  \l 080C
  \l 080D
  \l 080E
  \l 080F
  \l 0810
  \l 0811
  \l 0812
  \l 0813
  \l 0814
  \l 0815
  \l 0816
  \l 0817
  \l 0818
  \l 0819
  \l 081A
  \l 081B
  \l 081C
  \l 081D
  \l 081E
  \l 081F
  \l 0820
  \l 0821
  \l 0822
  \l 0823
  \l 0824
  \l 0825
  \l 0826
  \l 0827
  \l 0828
  \l 0829
  \l 082A
  \l 082B
  \l 082C
  \l 082D
  \l 0840
  \l 0841
  \l 0842
  \l 0843
  \l 0844
  \l 0845
  \l 0846
  \l 0847
  \l 0848
  \l 0849
  \l 084A
  \l 084B
  \l 084C
  \l 084D
  \l 084E
  \l 084F
  \l 0850
  \l 0851
  \l 0852
  \l 0853
  \l 0854
  \l 0855
  \l 0856
  \l 0857
  \l 0858
  \l 0859
  \l 085A
  \l 085B
  \l 08A0
  \l 08A1
  \l 08A2
  \l 08A3
  \l 08A4
  \l 08A5
  \l 08A6
  \l 08A7
  \l 08A8
  \l 08A9
  \l 08AA
  \l 08AB
  \l 08AC
  \l 08AD
  \l 08AE
  \l 08AF
  \l 08B0
  \l 08B1
  \l 08B2
  \l 08E4
  \l 08E5
  \l 08E6
  \l 08E7
  \l 08E8
  \l 08E9
  \l 08EA
  \l 08EB
  \l 08EC
  \l 08ED
  \l 08EE
  \l 08EF
  \l 08F0
  \l 08F1
  \l 08F2
  \l 08F3
  \l 08F4
  \l 08F5
  \l 08F6
  \l 08F7
  \l 08F8
  \l 08F9
  \l 08FA
  \l 08FB
  \l 08FC
  \l 08FD
  \l 08FE
  \l 08FF
  \l 0900
  \l 0901
  \l 0902
  \l 0903
  \l 0904
  \l 0905
  \l 0906
  \l 0907
  \l 0908
  \l 0909
  \l 090A
  \l 090B
  \l 090C
  \l 090D
  \l 090E
  \l 090F
  \l 0910
  \l 0911
  \l 0912
  \l 0913
  \l 0914
  \l 0915
  \l 0916
  \l 0917
  \l 0918
  \l 0919
  \l 091A
  \l 091B
  \l 091C
  \l 091D
  \l 091E
  \l 091F
  \l 0920
  \l 0921
  \l 0922
  \l 0923
  \l 0924
  \l 0925
  \l 0926
  \l 0927
  \l 0928
  \l 0929
  \l 092A
  \l 092B
  \l 092C
  \l 092D
  \l 092E
  \l 092F
  \l 0930
  \l 0931
  \l 0932
  \l 0933
  \l 0934
  \l 0935
  \l 0936
  \l 0937
  \l 0938
  \l 0939
  \l 093A
  \l 093B
  \l 093C
  \l 093D
  \l 093E
  \l 093F
  \l 0940
  \l 0941
  \l 0942
  \l 0943
  \l 0944
  \l 0945
  \l 0946
  \l 0947
  \l 0948
  \l 0949
  \l 094A
  \l 094B
  \l 094C
  \l 094D
  \l 094E
  \l 094F
  \l 0950
  \l 0951
  \l 0952
  \l 0953
  \l 0954
  \l 0955
  \l 0956
  \l 0957
  \l 0958
  \l 0959
  \l 095A
  \l 095B
  \l 095C
  \l 095D
  \l 095E
  \l 095F
  \l 0960
  \l 0961
  \l 0962
  \l 0963
  \l 0971
  \l 0972
  \l 0973
  \l 0974
  \l 0975
  \l 0976
  \l 0977
  \l 0978
  \l 0979
  \l 097A
  \l 097B
  \l 097C
  \l 097D
  \l 097E
  \l 097F
  \l 0980
  \l 0981
  \l 0982
  \l 0983
  \l 0985
  \l 0986
  \l 0987
  \l 0988
  \l 0989
  \l 098A
  \l 098B
  \l 098C
  \l 098F
  \l 0990
  \l 0993
  \l 0994
  \l 0995
  \l 0996
  \l 0997
  \l 0998
  \l 0999
  \l 099A
  \l 099B
  \l 099C
  \l 099D
  \l 099E
  \l 099F
  \l 09A0
  \l 09A1
  \l 09A2
  \l 09A3
  \l 09A4
  \l 09A5
  \l 09A6
  \l 09A7
  \l 09A8
  \l 09AA
  \l 09AB
  \l 09AC
  \l 09AD
  \l 09AE
  \l 09AF
  \l 09B0
  \l 09B2
  \l 09B6
  \l 09B7
  \l 09B8
  \l 09B9
  \l 09BC
  \l 09BD
  \l 09BE
  \l 09BF
  \l 09C0
  \l 09C1
  \l 09C2
  \l 09C3
  \l 09C4
  \l 09C7
  \l 09C8
  \l 09CB
  \l 09CC
  \l 09CD
  \l 09CE
  \l 09D7
  \l 09DC
  \l 09DD
  \l 09DF
  \l 09E0
  \l 09E1
  \l 09E2
  \l 09E3
  \l 09F0
  \l 09F1
  \l 0A01
  \l 0A02
  \l 0A03
  \l 0A05
  \l 0A06
  \l 0A07
  \l 0A08
  \l 0A09
  \l 0A0A
  \l 0A0F
  \l 0A10
  \l 0A13
  \l 0A14
  \l 0A15
  \l 0A16
  \l 0A17
  \l 0A18
  \l 0A19
  \l 0A1A
  \l 0A1B
  \l 0A1C
  \l 0A1D
  \l 0A1E
  \l 0A1F
  \l 0A20
  \l 0A21
  \l 0A22
  \l 0A23
  \l 0A24
  \l 0A25
  \l 0A26
  \l 0A27
  \l 0A28
  \l 0A2A
  \l 0A2B
  \l 0A2C
  \l 0A2D
  \l 0A2E
  \l 0A2F
  \l 0A30
  \l 0A32
  \l 0A33
  \l 0A35
  \l 0A36
  \l 0A38
  \l 0A39
  \l 0A3C
  \l 0A3E
  \l 0A3F
  \l 0A40
  \l 0A41
  \l 0A42
  \l 0A47
  \l 0A48
  \l 0A4B
  \l 0A4C
  \l 0A4D
  \l 0A51
  \l 0A59
  \l 0A5A
  \l 0A5B
  \l 0A5C
  \l 0A5E
  \l 0A70
  \l 0A71
  \l 0A72
  \l 0A73
  \l 0A74
  \l 0A75
  \l 0A81
  \l 0A82
  \l 0A83
  \l 0A85
  \l 0A86
  \l 0A87
  \l 0A88
  \l 0A89
  \l 0A8A
  \l 0A8B
  \l 0A8C
  \l 0A8D
  \l 0A8F
  \l 0A90
  \l 0A91
  \l 0A93
  \l 0A94
  \l 0A95
  \l 0A96
  \l 0A97
  \l 0A98
  \l 0A99
  \l 0A9A
  \l 0A9B
  \l 0A9C
  \l 0A9D
  \l 0A9E
  \l 0A9F
  \l 0AA0
  \l 0AA1
  \l 0AA2
  \l 0AA3
  \l 0AA4
  \l 0AA5
  \l 0AA6
  \l 0AA7
  \l 0AA8
  \l 0AAA
  \l 0AAB
  \l 0AAC
  \l 0AAD
  \l 0AAE
  \l 0AAF
  \l 0AB0
  \l 0AB2
  \l 0AB3
  \l 0AB5
  \l 0AB6
  \l 0AB7
  \l 0AB8
  \l 0AB9
  \l 0ABC
  \l 0ABD
  \l 0ABE
  \l 0ABF
  \l 0AC0
  \l 0AC1
  \l 0AC2
  \l 0AC3
  \l 0AC4
  \l 0AC5
  \l 0AC7
  \l 0AC8
  \l 0AC9
  \l 0ACB
  \l 0ACC
  \l 0ACD
  \l 0AD0
  \l 0AE0
  \l 0AE1
  \l 0AE2
  \l 0AE3
  \l 0B01
  \l 0B02
  \l 0B03
  \l 0B05
  \l 0B06
  \l 0B07
  \l 0B08
  \l 0B09
  \l 0B0A
  \l 0B0B
  \l 0B0C
  \l 0B0F
  \l 0B10
  \l 0B13
  \l 0B14
  \l 0B15
  \l 0B16
  \l 0B17
  \l 0B18
  \l 0B19
  \l 0B1A
  \l 0B1B
  \l 0B1C
  \l 0B1D
  \l 0B1E
  \l 0B1F
  \l 0B20
  \l 0B21
  \l 0B22
  \l 0B23
  \l 0B24
  \l 0B25
  \l 0B26
  \l 0B27
  \l 0B28
  \l 0B2A
  \l 0B2B
  \l 0B2C
  \l 0B2D
  \l 0B2E
  \l 0B2F
  \l 0B30
  \l 0B32
  \l 0B33
  \l 0B35
  \l 0B36
  \l 0B37
  \l 0B38
  \l 0B39
  \l 0B3C
  \l 0B3D
  \l 0B3E
  \l 0B3F
  \l 0B40
  \l 0B41
  \l 0B42
  \l 0B43
  \l 0B44
  \l 0B47
  \l 0B48
  \l 0B4B
  \l 0B4C
  \l 0B4D
  \l 0B56
  \l 0B57
  \l 0B5C
  \l 0B5D
  \l 0B5F
  \l 0B60
  \l 0B61
  \l 0B62
  \l 0B63
  \l 0B71
  \l 0B82
  \l 0B83
  \l 0B85
  \l 0B86
  \l 0B87
  \l 0B88
  \l 0B89
  \l 0B8A
  \l 0B8E
  \l 0B8F
  \l 0B90
  \l 0B92
  \l 0B93
  \l 0B94
  \l 0B95
  \l 0B99
  \l 0B9A
  \l 0B9C
  \l 0B9E
  \l 0B9F
  \l 0BA3
  \l 0BA4
  \l 0BA8
  \l 0BA9
  \l 0BAA
  \l 0BAE
  \l 0BAF
  \l 0BB0
  \l 0BB1
  \l 0BB2
  \l 0BB3
  \l 0BB4
  \l 0BB5
  \l 0BB6
  \l 0BB7
  \l 0BB8
  \l 0BB9
  \l 0BBE
  \l 0BBF
  \l 0BC0
  \l 0BC1
  \l 0BC2
  \l 0BC6
  \l 0BC7
  \l 0BC8
  \l 0BCA
  \l 0BCB
  \l 0BCC
  \l 0BCD
  \l 0BD0
  \l 0BD7
  \l 0C00
  \l 0C01
  \l 0C02
  \l 0C03
  \l 0C05
  \l 0C06
  \l 0C07
  \l 0C08
  \l 0C09
  \l 0C0A
  \l 0C0B
  \l 0C0C
  \l 0C0E
  \l 0C0F
  \l 0C10
  \l 0C12
  \l 0C13
  \l 0C14
  \l 0C15
  \l 0C16
  \l 0C17
  \l 0C18
  \l 0C19
  \l 0C1A
  \l 0C1B
  \l 0C1C
  \l 0C1D
  \l 0C1E
  \l 0C1F
  \l 0C20
  \l 0C21
  \l 0C22
  \l 0C23
  \l 0C24
  \l 0C25
  \l 0C26
  \l 0C27
  \l 0C28
  \l 0C2A
  \l 0C2B
  \l 0C2C
  \l 0C2D
  \l 0C2E
  \l 0C2F
  \l 0C30
  \l 0C31
  \l 0C32
  \l 0C33
  \l 0C34
  \l 0C35
  \l 0C36
  \l 0C37
  \l 0C38
  \l 0C39
  \l 0C3D
  \l 0C3E
  \l 0C3F
  \l 0C40
  \l 0C41
  \l 0C42
  \l 0C43
  \l 0C44
  \l 0C46
  \l 0C47
  \l 0C48
  \l 0C4A
  \l 0C4B
  \l 0C4C
  \l 0C4D
  \l 0C55
  \l 0C56
  \l 0C58
  \l 0C59
  \l 0C60
  \l 0C61
  \l 0C62
  \l 0C63
  \l 0C81
  \l 0C82
  \l 0C83
  \l 0C85
  \l 0C86
  \l 0C87
  \l 0C88
  \l 0C89
  \l 0C8A
  \l 0C8B
  \l 0C8C
  \l 0C8E
  \l 0C8F
  \l 0C90
  \l 0C92
  \l 0C93
  \l 0C94
  \l 0C95
  \l 0C96
  \l 0C97
  \l 0C98
  \l 0C99
  \l 0C9A
  \l 0C9B
  \l 0C9C
  \l 0C9D
  \l 0C9E
  \l 0C9F
  \l 0CA0
  \l 0CA1
  \l 0CA2
  \l 0CA3
  \l 0CA4
  \l 0CA5
  \l 0CA6
  \l 0CA7
  \l 0CA8
  \l 0CAA
  \l 0CAB
  \l 0CAC
  \l 0CAD
  \l 0CAE
  \l 0CAF
  \l 0CB0
  \l 0CB1
  \l 0CB2
  \l 0CB3
  \l 0CB5
  \l 0CB6
  \l 0CB7
  \l 0CB8
  \l 0CB9
  \l 0CBC
  \l 0CBD
  \l 0CBE
  \l 0CBF
  \l 0CC0
  \l 0CC1
  \l 0CC2
  \l 0CC3
  \l 0CC4
  \l 0CC6
  \l 0CC7
  \l 0CC8
  \l 0CCA
  \l 0CCB
  \l 0CCC
  \l 0CCD
  \l 0CD5
  \l 0CD6
  \l 0CDE
  \l 0CE0
  \l 0CE1
  \l 0CE2
  \l 0CE3
  \l 0CF1
  \l 0CF2
  \l 0D01
  \l 0D02
  \l 0D03
  \l 0D05
  \l 0D06
  \l 0D07
  \l 0D08
  \l 0D09
  \l 0D0A
  \l 0D0B
  \l 0D0C
  \l 0D0E
  \l 0D0F
  \l 0D10
  \l 0D12
  \l 0D13
  \l 0D14
  \l 0D15
  \l 0D16
  \l 0D17
  \l 0D18
  \l 0D19
  \l 0D1A
  \l 0D1B
  \l 0D1C
  \l 0D1D
  \l 0D1E
  \l 0D1F
  \l 0D20
  \l 0D21
  \l 0D22
  \l 0D23
  \l 0D24
  \l 0D25
  \l 0D26
  \l 0D27
  \l 0D28
  \l 0D29
  \l 0D2A
  \l 0D2B
  \l 0D2C
  \l 0D2D
  \l 0D2E
  \l 0D2F
  \l 0D30
  \l 0D31
  \l 0D32
  \l 0D33
  \l 0D34
  \l 0D35
  \l 0D36
  \l 0D37
  \l 0D38
  \l 0D39
  \l 0D3A
  \l 0D3D
  \l 0D3E
  \l 0D3F
  \l 0D40
  \l 0D41
  \l 0D42
  \l 0D43
  \l 0D44
  \l 0D46
  \l 0D47
  \l 0D48
  \l 0D4A
  \l 0D4B
  \l 0D4C
  \l 0D4D
  \l 0D4E
  \l 0D57
  \l 0D60
  \l 0D61
  \l 0D62
  \l 0D63
  \l 0D7A
  \l 0D7B
  \l 0D7C
  \l 0D7D
  \l 0D7E
  \l 0D7F
  \l 0D82
  \l 0D83
  \l 0D85
  \l 0D86
  \l 0D87
  \l 0D88
  \l 0D89
  \l 0D8A
  \l 0D8B
  \l 0D8C
  \l 0D8D
  \l 0D8E
  \l 0D8F
  \l 0D90
  \l 0D91
  \l 0D92
  \l 0D93
  \l 0D94
  \l 0D95
  \l 0D96
  \l 0D9A
  \l 0D9B
  \l 0D9C
  \l 0D9D
  \l 0D9E
  \l 0D9F
  \l 0DA0
  \l 0DA1
  \l 0DA2
  \l 0DA3
  \l 0DA4
  \l 0DA5
  \l 0DA6
  \l 0DA7
  \l 0DA8
  \l 0DA9
  \l 0DAA
  \l 0DAB
  \l 0DAC
  \l 0DAD
  \l 0DAE
  \l 0DAF
  \l 0DB0
  \l 0DB1
  \l 0DB3
  \l 0DB4
  \l 0DB5
  \l 0DB6
  \l 0DB7
  \l 0DB8
  \l 0DB9
  \l 0DBA
  \l 0DBB
  \l 0DBD
  \l 0DC0
  \l 0DC1
  \l 0DC2
  \l 0DC3
  \l 0DC4
  \l 0DC5
  \l 0DC6
  \l 0DCA
  \l 0DCF
  \l 0DD0
  \l 0DD1
  \l 0DD2
  \l 0DD3
  \l 0DD4
  \l 0DD6
  \l 0DD8
  \l 0DD9
  \l 0DDA
  \l 0DDB
  \l 0DDC
  \l 0DDD
  \l 0DDE
  \l 0DDF
  \l 0DF2
  \l 0DF3
  \l 0E01
  \l 0E02
  \l 0E03
  \l 0E04
  \l 0E05
  \l 0E06
  \l 0E07
  \l 0E08
  \l 0E09
  \l 0E0A
  \l 0E0B
  \l 0E0C
  \l 0E0D
  \l 0E0E
  \l 0E0F
  \l 0E10
  \l 0E11
  \l 0E12
  \l 0E13
  \l 0E14
  \l 0E15
  \l 0E16
  \l 0E17
  \l 0E18
  \l 0E19
  \l 0E1A
  \l 0E1B
  \l 0E1C
  \l 0E1D
  \l 0E1E
  \l 0E1F
  \l 0E20
  \l 0E21
  \l 0E22
  \l 0E23
  \l 0E24
  \l 0E25
  \l 0E26
  \l 0E27
  \l 0E28
  \l 0E29
  \l 0E2A
  \l 0E2B
  \l 0E2C
  \l 0E2D
  \l 0E2E
  \l 0E2F
  \l 0E30
  \l 0E31
  \l 0E32
  \l 0E33
  \l 0E34
  \l 0E35
  \l 0E36
  \l 0E37
  \l 0E38
  \l 0E39
  \l 0E3A
  \l 0E40
  \l 0E41
  \l 0E42
  \l 0E43
  \l 0E44
  \l 0E45
  \l 0E46
  \l 0E47
  \l 0E48
  \l 0E49
  \l 0E4A
  \l 0E4B
  \l 0E4C
  \l 0E4D
  \l 0E4E
  \l 0E81
  \l 0E82
  \l 0E84
  \l 0E87
  \l 0E88
  \l 0E8A
  \l 0E8D
  \l 0E94
  \l 0E95
  \l 0E96
  \l 0E97
  \l 0E99
  \l 0E9A
  \l 0E9B
  \l 0E9C
  \l 0E9D
  \l 0E9E
  \l 0E9F
  \l 0EA1
  \l 0EA2
  \l 0EA3
  \l 0EA5
  \l 0EA7
  \l 0EAA
  \l 0EAB
  \l 0EAD
  \l 0EAE
  \l 0EAF
  \l 0EB0
  \l 0EB1
  \l 0EB2
  \l 0EB3
  \l 0EB4
  \l 0EB5
  \l 0EB6
  \l 0EB7
  \l 0EB8
  \l 0EB9
  \l 0EBB
  \l 0EBC
  \l 0EBD
  \l 0EC0
  \l 0EC1
  \l 0EC2
  \l 0EC3
  \l 0EC4
  \l 0EC6
  \l 0EC8
  \l 0EC9
  \l 0ECA
  \l 0ECB
  \l 0ECC
  \l 0ECD
  \l 0EDC
  \l 0EDD
  \l 0EDE
  \l 0EDF
  \l 0F00
  \l 0F18
  \l 0F19
  \l 0F35
  \l 0F37
  \l 0F39
  \l 0F3E
  \l 0F3F
  \l 0F40
  \l 0F41
  \l 0F42
  \l 0F43
  \l 0F44
  \l 0F45
  \l 0F46
  \l 0F47
  \l 0F49
  \l 0F4A
  \l 0F4B
  \l 0F4C
  \l 0F4D
  \l 0F4E
  \l 0F4F
  \l 0F50
  \l 0F51
  \l 0F52
  \l 0F53
  \l 0F54
  \l 0F55
  \l 0F56
  \l 0F57
  \l 0F58
  \l 0F59
  \l 0F5A
  \l 0F5B
  \l 0F5C
  \l 0F5D
  \l 0F5E
  \l 0F5F
  \l 0F60
  \l 0F61
  \l 0F62
  \l 0F63
  \l 0F64
  \l 0F65
  \l 0F66
  \l 0F67
  \l 0F68
  \l 0F69
  \l 0F6A
  \l 0F6B
  \l 0F6C
  \l 0F71
  \l 0F72
  \l 0F73
  \l 0F74
  \l 0F75
  \l 0F76
  \l 0F77
  \l 0F78
  \l 0F79
  \l 0F7A
  \l 0F7B
  \l 0F7C
  \l 0F7D
  \l 0F7E
  \l 0F7F
  \l 0F80
  \l 0F81
  \l 0F82
  \l 0F83
  \l 0F84
  \l 0F86
  \l 0F87
  \l 0F88
  \l 0F89
  \l 0F8A
  \l 0F8B
  \l 0F8C
  \l 0F8D
  \l 0F8E
  \l 0F8F
  \l 0F90
  \l 0F91
  \l 0F92
  \l 0F93
  \l 0F94
  \l 0F95
  \l 0F96
  \l 0F97
  \l 0F99
  \l 0F9A
  \l 0F9B
  \l 0F9C
  \l 0F9D
  \l 0F9E
  \l 0F9F
  \l 0FA0
  \l 0FA1
  \l 0FA2
  \l 0FA3
  \l 0FA4
  \l 0FA5
  \l 0FA6
  \l 0FA7
  \l 0FA8
  \l 0FA9
  \l 0FAA
  \l 0FAB
  \l 0FAC
  \l 0FAD
  \l 0FAE
  \l 0FAF
  \l 0FB0
  \l 0FB1
  \l 0FB2
  \l 0FB3
  \l 0FB4
  \l 0FB5
  \l 0FB6
  \l 0FB7
  \l 0FB8
  \l 0FB9
  \l 0FBA
  \l 0FBB
  \l 0FBC
  \l 0FC6
  \l 1000
  \l 1001
  \l 1002
  \l 1003
  \l 1004
  \l 1005
  \l 1006
  \l 1007
  \l 1008
  \l 1009
  \l 100A
  \l 100B
  \l 100C
  \l 100D
  \l 100E
  \l 100F
  \l 1010
  \l 1011
  \l 1012
  \l 1013
  \l 1014
  \l 1015
  \l 1016
  \l 1017
  \l 1018
  \l 1019
  \l 101A
  \l 101B
  \l 101C
  \l 101D
  \l 101E
  \l 101F
  \l 1020
  \l 1021
  \l 1022
  \l 1023
  \l 1024
  \l 1025
  \l 1026
  \l 1027
  \l 1028
  \l 1029
  \l 102A
  \l 102B
  \l 102C
  \l 102D
  \l 102E
  \l 102F
  \l 1030
  \l 1031
  \l 1032
  \l 1033
  \l 1034
  \l 1035
  \l 1036
  \l 1037
  \l 1038
  \l 1039
  \l 103A
  \l 103B
  \l 103C
  \l 103D
  \l 103E
  \l 103F
  \l 1050
  \l 1051
  \l 1052
  \l 1053
  \l 1054
  \l 1055
  \l 1056
  \l 1057
  \l 1058
  \l 1059
  \l 105A
  \l 105B
  \l 105C
  \l 105D
  \l 105E
  \l 105F
  \l 1060
  \l 1061
  \l 1062
  \l 1063
  \l 1064
  \l 1065
  \l 1066
  \l 1067
  \l 1068
  \l 1069
  \l 106A
  \l 106B
  \l 106C
  \l 106D
  \l 106E
  \l 106F
  \l 1070
  \l 1071
  \l 1072
  \l 1073
  \l 1074
  \l 1075
  \l 1076
  \l 1077
  \l 1078
  \l 1079
  \l 107A
  \l 107B
  \l 107C
  \l 107D
  \l 107E
  \l 107F
  \l 1080
  \l 1081
  \l 1082
  \l 1083
  \l 1084
  \l 1085
  \l 1086
  \l 1087
  \l 1088
  \l 1089
  \l 108A
  \l 108B
  \l 108C
  \l 108D
  \l 108E
  \l 108F
  \l 109A
  \l 109B
  \l 109C
  \l 109D
  \L 10A0 10A0 2D00
  \L 10A1 10A1 2D01
  \L 10A2 10A2 2D02
  \L 10A3 10A3 2D03
  \L 10A4 10A4 2D04
  \L 10A5 10A5 2D05
  \L 10A6 10A6 2D06
  \L 10A7 10A7 2D07
  \L 10A8 10A8 2D08
  \L 10A9 10A9 2D09
  \L 10AA 10AA 2D0A
  \L 10AB 10AB 2D0B
  \L 10AC 10AC 2D0C
  \L 10AD 10AD 2D0D
  \L 10AE 10AE 2D0E
  \L 10AF 10AF 2D0F
  \L 10B0 10B0 2D10
  \L 10B1 10B1 2D11
  \L 10B2 10B2 2D12
  \L 10B3 10B3 2D13
  \L 10B4 10B4 2D14
  \L 10B5 10B5 2D15
  \L 10B6 10B6 2D16
  \L 10B7 10B7 2D17
  \L 10B8 10B8 2D18
  \L 10B9 10B9 2D19
  \L 10BA 10BA 2D1A
  \L 10BB 10BB 2D1B
  \L 10BC 10BC 2D1C
  \L 10BD 10BD 2D1D
  \L 10BE 10BE 2D1E
  \L 10BF 10BF 2D1F
  \L 10C0 10C0 2D20
  \L 10C1 10C1 2D21
  \L 10C2 10C2 2D22
  \L 10C3 10C3 2D23
  \L 10C4 10C4 2D24
  \L 10C5 10C5 2D25
  \L 10C7 10C7 2D27
  \L 10CD 10CD 2D2D
  \l 10D0
  \l 10D1
  \l 10D2
  \l 10D3
  \l 10D4
  \l 10D5
  \l 10D6
  \l 10D7
  \l 10D8
  \l 10D9
  \l 10DA
  \l 10DB
  \l 10DC
  \l 10DD
  \l 10DE
  \l 10DF
  \l 10E0
  \l 10E1
  \l 10E2
  \l 10E3
  \l 10E4
  \l 10E5
  \l 10E6
  \l 10E7
  \l 10E8
  \l 10E9
  \l 10EA
  \l 10EB
  \l 10EC
  \l 10ED
  \l 10EE
  \l 10EF
  \l 10F0
  \l 10F1
  \l 10F2
  \l 10F3
  \l 10F4
  \l 10F5
  \l 10F6
  \l 10F7
  \l 10F8
  \l 10F9
  \l 10FA
  \l 10FC
  \l 10FD
  \l 10FE
  \l 10FF
  \l 1100
  \l 1101
  \l 1102
  \l 1103
  \l 1104
  \l 1105
  \l 1106
  \l 1107
  \l 1108
  \l 1109
  \l 110A
  \l 110B
  \l 110C
  \l 110D
  \l 110E
  \l 110F
  \l 1110
  \l 1111
  \l 1112
  \l 1113
  \l 1114
  \l 1115
  \l 1116
  \l 1117
  \l 1118
  \l 1119
  \l 111A
  \l 111B
  \l 111C
  \l 111D
  \l 111E
  \l 111F
  \l 1120
  \l 1121
  \l 1122
  \l 1123
  \l 1124
  \l 1125
  \l 1126
  \l 1127
  \l 1128
  \l 1129
  \l 112A
  \l 112B
  \l 112C
  \l 112D
  \l 112E
  \l 112F
  \l 1130
  \l 1131
  \l 1132
  \l 1133
  \l 1134
  \l 1135
  \l 1136
  \l 1137
  \l 1138
  \l 1139
  \l 113A
  \l 113B
  \l 113C
  \l 113D
  \l 113E
  \l 113F
  \l 1140
  \l 1141
  \l 1142
  \l 1143
  \l 1144
  \l 1145
  \l 1146
  \l 1147
  \l 1148
  \l 1149
  \l 114A
  \l 114B
  \l 114C
  \l 114D
  \l 114E
  \l 114F
  \l 1150
  \l 1151
  \l 1152
  \l 1153
  \l 1154
  \l 1155
  \l 1156
  \l 1157
  \l 1158
  \l 1159
  \l 115A
  \l 115B
  \l 115C
  \l 115D
  \l 115E
  \l 115F
  \l 1160
  \l 1161
  \l 1162
  \l 1163
  \l 1164
  \l 1165
  \l 1166
  \l 1167
  \l 1168
  \l 1169
  \l 116A
  \l 116B
  \l 116C
  \l 116D
  \l 116E
  \l 116F
  \l 1170
  \l 1171
  \l 1172
  \l 1173
  \l 1174
  \l 1175
  \l 1176
  \l 1177
  \l 1178
  \l 1179
  \l 117A
  \l 117B
  \l 117C
  \l 117D
  \l 117E
  \l 117F
  \l 1180
  \l 1181
  \l 1182
  \l 1183
  \l 1184
  \l 1185
  \l 1186
  \l 1187
  \l 1188
  \l 1189
  \l 118A
  \l 118B
  \l 118C
  \l 118D
  \l 118E
  \l 118F
  \l 1190
  \l 1191
  \l 1192
  \l 1193
  \l 1194
  \l 1195
  \l 1196
  \l 1197
  \l 1198
  \l 1199
  \l 119A
  \l 119B
  \l 119C
  \l 119D
  \l 119E
  \l 119F
  \l 11A0
  \l 11A1
  \l 11A2
  \l 11A3
  \l 11A4
  \l 11A5
  \l 11A6
  \l 11A7
  \l 11A8
  \l 11A9
  \l 11AA
  \l 11AB
  \l 11AC
  \l 11AD
  \l 11AE
  \l 11AF
  \l 11B0
  \l 11B1
  \l 11B2
  \l 11B3
  \l 11B4
  \l 11B5
  \l 11B6
  \l 11B7
  \l 11B8
  \l 11B9
  \l 11BA
  \l 11BB
  \l 11BC
  \l 11BD
  \l 11BE
  \l 11BF
  \l 11C0
  \l 11C1
  \l 11C2
  \l 11C3
  \l 11C4
  \l 11C5
  \l 11C6
  \l 11C7
  \l 11C8
  \l 11C9
  \l 11CA
  \l 11CB
  \l 11CC
  \l 11CD
  \l 11CE
  \l 11CF
  \l 11D0
  \l 11D1
  \l 11D2
  \l 11D3
  \l 11D4
  \l 11D5
  \l 11D6
  \l 11D7
  \l 11D8
  \l 11D9
  \l 11DA
  \l 11DB
  \l 11DC
  \l 11DD
  \l 11DE
  \l 11DF
  \l 11E0
  \l 11E1
  \l 11E2
  \l 11E3
  \l 11E4
  \l 11E5
  \l 11E6
  \l 11E7
  \l 11E8
  \l 11E9
  \l 11EA
  \l 11EB
  \l 11EC
  \l 11ED
  \l 11EE
  \l 11EF
  \l 11F0
  \l 11F1
  \l 11F2
  \l 11F3
  \l 11F4
  \l 11F5
  \l 11F6
  \l 11F7
  \l 11F8
  \l 11F9
  \l 11FA
  \l 11FB
  \l 11FC
  \l 11FD
  \l 11FE
  \l 11FF
  \l 1200
  \l 1201
  \l 1202
  \l 1203
  \l 1204
  \l 1205
  \l 1206
  \l 1207
  \l 1208
  \l 1209
  \l 120A
  \l 120B
  \l 120C
  \l 120D
  \l 120E
  \l 120F
  \l 1210
  \l 1211
  \l 1212
  \l 1213
  \l 1214
  \l 1215
  \l 1216
  \l 1217
  \l 1218
  \l 1219
  \l 121A
  \l 121B
  \l 121C
  \l 121D
  \l 121E
  \l 121F
  \l 1220
  \l 1221
  \l 1222
  \l 1223
  \l 1224
  \l 1225
  \l 1226
  \l 1227
  \l 1228
  \l 1229
  \l 122A
  \l 122B
  \l 122C
  \l 122D
  \l 122E
  \l 122F
  \l 1230
  \l 1231
  \l 1232
  \l 1233
  \l 1234
  \l 1235
  \l 1236
  \l 1237
  \l 1238
  \l 1239
  \l 123A
  \l 123B
  \l 123C
  \l 123D
  \l 123E
  \l 123F
  \l 1240
  \l 1241
  \l 1242
  \l 1243
  \l 1244
  \l 1245
  \l 1246
  \l 1247
  \l 1248
  \l 124A
  \l 124B
  \l 124C
  \l 124D
  \l 1250
  \l 1251
  \l 1252
  \l 1253
  \l 1254
  \l 1255
  \l 1256
  \l 1258
  \l 125A
  \l 125B
  \l 125C
  \l 125D
  \l 1260
  \l 1261
  \l 1262
  \l 1263
  \l 1264
  \l 1265
  \l 1266
  \l 1267
  \l 1268
  \l 1269
  \l 126A
  \l 126B
  \l 126C
  \l 126D
  \l 126E
  \l 126F
  \l 1270
  \l 1271
  \l 1272
  \l 1273
  \l 1274
  \l 1275
  \l 1276
  \l 1277
  \l 1278
  \l 1279
  \l 127A
  \l 127B
  \l 127C
  \l 127D
  \l 127E
  \l 127F
  \l 1280
  \l 1281
  \l 1282
  \l 1283
  \l 1284
  \l 1285
  \l 1286
  \l 1287
  \l 1288
  \l 128A
  \l 128B
  \l 128C
  \l 128D
  \l 1290
  \l 1291
  \l 1292
  \l 1293
  \l 1294
  \l 1295
  \l 1296
  \l 1297
  \l 1298
  \l 1299
  \l 129A
  \l 129B
  \l 129C
  \l 129D
  \l 129E
  \l 129F
  \l 12A0
  \l 12A1
  \l 12A2
  \l 12A3
  \l 12A4
  \l 12A5
  \l 12A6
  \l 12A7
  \l 12A8
  \l 12A9
  \l 12AA
  \l 12AB
  \l 12AC
  \l 12AD
  \l 12AE
  \l 12AF
  \l 12B0
  \l 12B2
  \l 12B3
  \l 12B4
  \l 12B5
  \l 12B8
  \l 12B9
  \l 12BA
  \l 12BB
  \l 12BC
  \l 12BD
  \l 12BE
  \l 12C0
  \l 12C2
  \l 12C3
  \l 12C4
  \l 12C5
  \l 12C8
  \l 12C9
  \l 12CA
  \l 12CB
  \l 12CC
  \l 12CD
  \l 12CE
  \l 12CF
  \l 12D0
  \l 12D1
  \l 12D2
  \l 12D3
  \l 12D4
  \l 12D5
  \l 12D6
  \l 12D8
  \l 12D9
  \l 12DA
  \l 12DB
  \l 12DC
  \l 12DD
  \l 12DE
  \l 12DF
  \l 12E0
  \l 12E1
  \l 12E2
  \l 12E3
  \l 12E4
  \l 12E5
  \l 12E6
  \l 12E7
  \l 12E8
  \l 12E9
  \l 12EA
  \l 12EB
  \l 12EC
  \l 12ED
  \l 12EE
  \l 12EF
  \l 12F0
  \l 12F1
  \l 12F2
  \l 12F3
  \l 12F4
  \l 12F5
  \l 12F6
  \l 12F7
  \l 12F8
  \l 12F9
  \l 12FA
  \l 12FB
  \l 12FC
  \l 12FD
  \l 12FE
  \l 12FF
  \l 1300
  \l 1301
  \l 1302
  \l 1303
  \l 1304
  \l 1305
  \l 1306
  \l 1307
  \l 1308
  \l 1309
  \l 130A
  \l 130B
  \l 130C
  \l 130D
  \l 130E
  \l 130F
  \l 1310
  \l 1312
  \l 1313
  \l 1314
  \l 1315
  \l 1318
  \l 1319
  \l 131A
  \l 131B
  \l 131C
  \l 131D
  \l 131E
  \l 131F
  \l 1320
  \l 1321
  \l 1322
  \l 1323
  \l 1324
  \l 1325
  \l 1326
  \l 1327
  \l 1328
  \l 1329
  \l 132A
  \l 132B
  \l 132C
  \l 132D
  \l 132E
  \l 132F
  \l 1330
  \l 1331
  \l 1332
  \l 1333
  \l 1334
  \l 1335
  \l 1336
  \l 1337
  \l 1338
  \l 1339
  \l 133A
  \l 133B
  \l 133C
  \l 133D
  \l 133E
  \l 133F
  \l 1340
  \l 1341
  \l 1342
  \l 1343
  \l 1344
  \l 1345
  \l 1346
  \l 1347
  \l 1348
  \l 1349
  \l 134A
  \l 134B
  \l 134C
  \l 134D
  \l 134E
  \l 134F
  \l 1350
  \l 1351
  \l 1352
  \l 1353
  \l 1354
  \l 1355
  \l 1356
  \l 1357
  \l 1358
  \l 1359
  \l 135A
  \l 135D
  \l 135E
  \l 135F
  \l 1380
  \l 1381
  \l 1382
  \l 1383
  \l 1384
  \l 1385
  \l 1386
  \l 1387
  \l 1388
  \l 1389
  \l 138A
  \l 138B
  \l 138C
  \l 138D
  \l 138E
  \l 138F
  \l 13A0
  \l 13A1
  \l 13A2
  \l 13A3
  \l 13A4
  \l 13A5
  \l 13A6
  \l 13A7
  \l 13A8
  \l 13A9
  \l 13AA
  \l 13AB
  \l 13AC
  \l 13AD
  \l 13AE
  \l 13AF
  \l 13B0
  \l 13B1
  \l 13B2
  \l 13B3
  \l 13B4
  \l 13B5
  \l 13B6
  \l 13B7
  \l 13B8
  \l 13B9
  \l 13BA
  \l 13BB
  \l 13BC
  \l 13BD
  \l 13BE
  \l 13BF
  \l 13C0
  \l 13C1
  \l 13C2
  \l 13C3
  \l 13C4
  \l 13C5
  \l 13C6
  \l 13C7
  \l 13C8
  \l 13C9
  \l 13CA
  \l 13CB
  \l 13CC
  \l 13CD
  \l 13CE
  \l 13CF
  \l 13D0
  \l 13D1
  \l 13D2
  \l 13D3
  \l 13D4
  \l 13D5
  \l 13D6
  \l 13D7
  \l 13D8
  \l 13D9
  \l 13DA
  \l 13DB
  \l 13DC
  \l 13DD
  \l 13DE
  \l 13DF
  \l 13E0
  \l 13E1
  \l 13E2
  \l 13E3
  \l 13E4
  \l 13E5
  \l 13E6
  \l 13E7
  \l 13E8
  \l 13E9
  \l 13EA
  \l 13EB
  \l 13EC
  \l 13ED
  \l 13EE
  \l 13EF
  \l 13F0
  \l 13F1
  \l 13F2
  \l 13F3
  \l 13F4
  \l 1401
  \l 1402
  \l 1403
  \l 1404
  \l 1405
  \l 1406
  \l 1407
  \l 1408
  \l 1409
  \l 140A
  \l 140B
  \l 140C
  \l 140D
  \l 140E
  \l 140F
  \l 1410
  \l 1411
  \l 1412
  \l 1413
  \l 1414
  \l 1415
  \l 1416
  \l 1417
  \l 1418
  \l 1419
  \l 141A
  \l 141B
  \l 141C
  \l 141D
  \l 141E
  \l 141F
  \l 1420
  \l 1421
  \l 1422
  \l 1423
  \l 1424
  \l 1425
  \l 1426
  \l 1427
  \l 1428
  \l 1429
  \l 142A
  \l 142B
  \l 142C
  \l 142D
  \l 142E
  \l 142F
  \l 1430
  \l 1431
  \l 1432
  \l 1433
  \l 1434
  \l 1435
  \l 1436
  \l 1437
  \l 1438
  \l 1439
  \l 143A
  \l 143B
  \l 143C
  \l 143D
  \l 143E
  \l 143F
  \l 1440
  \l 1441
  \l 1442
  \l 1443
  \l 1444
  \l 1445
  \l 1446
  \l 1447
  \l 1448
  \l 1449
  \l 144A
  \l 144B
  \l 144C
  \l 144D
  \l 144E
  \l 144F
  \l 1450
  \l 1451
  \l 1452
  \l 1453
  \l 1454
  \l 1455
  \l 1456
  \l 1457
  \l 1458
  \l 1459
  \l 145A
  \l 145B
  \l 145C
  \l 145D
  \l 145E
  \l 145F
  \l 1460
  \l 1461
  \l 1462
  \l 1463
  \l 1464
  \l 1465
  \l 1466
  \l 1467
  \l 1468
  \l 1469
  \l 146A
  \l 146B
  \l 146C
  \l 146D
  \l 146E
  \l 146F
  \l 1470
  \l 1471
  \l 1472
  \l 1473
  \l 1474
  \l 1475
  \l 1476
  \l 1477
  \l 1478
  \l 1479
  \l 147A
  \l 147B
  \l 147C
  \l 147D
  \l 147E
  \l 147F
  \l 1480
  \l 1481
  \l 1482
  \l 1483
  \l 1484
  \l 1485
  \l 1486
  \l 1487
  \l 1488
  \l 1489
  \l 148A
  \l 148B
  \l 148C
  \l 148D
  \l 148E
  \l 148F
  \l 1490
  \l 1491
  \l 1492
  \l 1493
  \l 1494
  \l 1495
  \l 1496
  \l 1497
  \l 1498
  \l 1499
  \l 149A
  \l 149B
  \l 149C
  \l 149D
  \l 149E
  \l 149F
  \l 14A0
  \l 14A1
  \l 14A2
  \l 14A3
  \l 14A4
  \l 14A5
  \l 14A6
  \l 14A7
  \l 14A8
  \l 14A9
  \l 14AA
  \l 14AB
  \l 14AC
  \l 14AD
  \l 14AE
  \l 14AF
  \l 14B0
  \l 14B1
  \l 14B2
  \l 14B3
  \l 14B4
  \l 14B5
  \l 14B6
  \l 14B7
  \l 14B8
  \l 14B9
  \l 14BA
  \l 14BB
  \l 14BC
  \l 14BD
  \l 14BE
  \l 14BF
  \l 14C0
  \l 14C1
  \l 14C2
  \l 14C3
  \l 14C4
  \l 14C5
  \l 14C6
  \l 14C7
  \l 14C8
  \l 14C9
  \l 14CA
  \l 14CB
  \l 14CC
  \l 14CD
  \l 14CE
  \l 14CF
  \l 14D0
  \l 14D1
  \l 14D2
  \l 14D3
  \l 14D4
  \l 14D5
  \l 14D6
  \l 14D7
  \l 14D8
  \l 14D9
  \l 14DA
  \l 14DB
  \l 14DC
  \l 14DD
  \l 14DE
  \l 14DF
  \l 14E0
  \l 14E1
  \l 14E2
  \l 14E3
  \l 14E4
  \l 14E5
  \l 14E6
  \l 14E7
  \l 14E8
  \l 14E9
  \l 14EA
  \l 14EB
  \l 14EC
  \l 14ED
  \l 14EE
  \l 14EF
  \l 14F0
  \l 14F1
  \l 14F2
  \l 14F3
  \l 14F4
  \l 14F5
  \l 14F6
  \l 14F7
  \l 14F8
  \l 14F9
  \l 14FA
  \l 14FB
  \l 14FC
  \l 14FD
  \l 14FE
  \l 14FF
  \l 1500
  \l 1501
  \l 1502
  \l 1503
  \l 1504
  \l 1505
  \l 1506
  \l 1507
  \l 1508
  \l 1509
  \l 150A
  \l 150B
  \l 150C
  \l 150D
  \l 150E
  \l 150F
  \l 1510
  \l 1511
  \l 1512
  \l 1513
  \l 1514
  \l 1515
  \l 1516
  \l 1517
  \l 1518
  \l 1519
  \l 151A
  \l 151B
  \l 151C
  \l 151D
  \l 151E
  \l 151F
  \l 1520
  \l 1521
  \l 1522
  \l 1523
  \l 1524
  \l 1525
  \l 1526
  \l 1527
  \l 1528
  \l 1529
  \l 152A
  \l 152B
  \l 152C
  \l 152D
  \l 152E
  \l 152F
  \l 1530
  \l 1531
  \l 1532
  \l 1533
  \l 1534
  \l 1535
  \l 1536
  \l 1537
  \l 1538
  \l 1539
  \l 153A
  \l 153B
  \l 153C
  \l 153D
  \l 153E
  \l 153F
  \l 1540
  \l 1541
  \l 1542
  \l 1543
  \l 1544
  \l 1545
  \l 1546
  \l 1547
  \l 1548
  \l 1549
  \l 154A
  \l 154B
  \l 154C
  \l 154D
  \l 154E
  \l 154F
  \l 1550
  \l 1551
  \l 1552
  \l 1553
  \l 1554
  \l 1555
  \l 1556
  \l 1557
  \l 1558
  \l 1559
  \l 155A
  \l 155B
  \l 155C
  \l 155D
  \l 155E
  \l 155F
  \l 1560
  \l 1561
  \l 1562
  \l 1563
  \l 1564
  \l 1565
  \l 1566
  \l 1567
  \l 1568
  \l 1569
  \l 156A
  \l 156B
  \l 156C
  \l 156D
  \l 156E
  \l 156F
  \l 1570
  \l 1571
  \l 1572
  \l 1573
  \l 1574
  \l 1575
  \l 1576
  \l 1577
  \l 1578
  \l 1579
  \l 157A
  \l 157B
  \l 157C
  \l 157D
  \l 157E
  \l 157F
  \l 1580
  \l 1581
  \l 1582
  \l 1583
  \l 1584
  \l 1585
  \l 1586
  \l 1587
  \l 1588
  \l 1589
  \l 158A
  \l 158B
  \l 158C
  \l 158D
  \l 158E
  \l 158F
  \l 1590
  \l 1591
  \l 1592
  \l 1593
  \l 1594
  \l 1595
  \l 1596
  \l 1597
  \l 1598
  \l 1599
  \l 159A
  \l 159B
  \l 159C
  \l 159D
  \l 159E
  \l 159F
  \l 15A0
  \l 15A1
  \l 15A2
  \l 15A3
  \l 15A4
  \l 15A5
  \l 15A6
  \l 15A7
  \l 15A8
  \l 15A9
  \l 15AA
  \l 15AB
  \l 15AC
  \l 15AD
  \l 15AE
  \l 15AF
  \l 15B0
  \l 15B1
  \l 15B2
  \l 15B3
  \l 15B4
  \l 15B5
  \l 15B6
  \l 15B7
  \l 15B8
  \l 15B9
  \l 15BA
  \l 15BB
  \l 15BC
  \l 15BD
  \l 15BE
  \l 15BF
  \l 15C0
  \l 15C1
  \l 15C2
  \l 15C3
  \l 15C4
  \l 15C5
  \l 15C6
  \l 15C7
  \l 15C8
  \l 15C9
  \l 15CA
  \l 15CB
  \l 15CC
  \l 15CD
  \l 15CE
  \l 15CF
  \l 15D0
  \l 15D1
  \l 15D2
  \l 15D3
  \l 15D4
  \l 15D5
  \l 15D6
  \l 15D7
  \l 15D8
  \l 15D9
  \l 15DA
  \l 15DB
  \l 15DC
  \l 15DD
  \l 15DE
  \l 15DF
  \l 15E0
  \l 15E1
  \l 15E2
  \l 15E3
  \l 15E4
  \l 15E5
  \l 15E6
  \l 15E7
  \l 15E8
  \l 15E9
  \l 15EA
  \l 15EB
  \l 15EC
  \l 15ED
  \l 15EE
  \l 15EF
  \l 15F0
  \l 15F1
  \l 15F2
  \l 15F3
  \l 15F4
  \l 15F5
  \l 15F6
  \l 15F7
  \l 15F8
  \l 15F9
  \l 15FA
  \l 15FB
  \l 15FC
  \l 15FD
  \l 15FE
  \l 15FF
  \l 1600
  \l 1601
  \l 1602
  \l 1603
  \l 1604
  \l 1605
  \l 1606
  \l 1607
  \l 1608
  \l 1609
  \l 160A
  \l 160B
  \l 160C
  \l 160D
  \l 160E
  \l 160F
  \l 1610
  \l 1611
  \l 1612
  \l 1613
  \l 1614
  \l 1615
  \l 1616
  \l 1617
  \l 1618
  \l 1619
  \l 161A
  \l 161B
  \l 161C
  \l 161D
  \l 161E
  \l 161F
  \l 1620
  \l 1621
  \l 1622
  \l 1623
  \l 1624
  \l 1625
  \l 1626
  \l 1627
  \l 1628
  \l 1629
  \l 162A
  \l 162B
  \l 162C
  \l 162D
  \l 162E
  \l 162F
  \l 1630
  \l 1631
  \l 1632
  \l 1633
  \l 1634
  \l 1635
  \l 1636
  \l 1637
  \l 1638
  \l 1639
  \l 163A
  \l 163B
  \l 163C
  \l 163D
  \l 163E
  \l 163F
  \l 1640
  \l 1641
  \l 1642
  \l 1643
  \l 1644
  \l 1645
  \l 1646
  \l 1647
  \l 1648
  \l 1649
  \l 164A
  \l 164B
  \l 164C
  \l 164D
  \l 164E
  \l 164F
  \l 1650
  \l 1651
  \l 1652
  \l 1653
  \l 1654
  \l 1655
  \l 1656
  \l 1657
  \l 1658
  \l 1659
  \l 165A
  \l 165B
  \l 165C
  \l 165D
  \l 165E
  \l 165F
  \l 1660
  \l 1661
  \l 1662
  \l 1663
  \l 1664
  \l 1665
  \l 1666
  \l 1667
  \l 1668
  \l 1669
  \l 166A
  \l 166B
  \l 166C
  \l 166F
  \l 1670
  \l 1671
  \l 1672
  \l 1673
  \l 1674
  \l 1675
  \l 1676
  \l 1677
  \l 1678
  \l 1679
  \l 167A
  \l 167B
  \l 167C
  \l 167D
  \l 167E
  \l 167F
  \l 1681
  \l 1682
  \l 1683
  \l 1684
  \l 1685
  \l 1686
  \l 1687
  \l 1688
  \l 1689
  \l 168A
  \l 168B
  \l 168C
  \l 168D
  \l 168E
  \l 168F
  \l 1690
  \l 1691
  \l 1692
  \l 1693
  \l 1694
  \l 1695
  \l 1696
  \l 1697
  \l 1698
  \l 1699
  \l 169A
  \l 16A0
  \l 16A1
  \l 16A2
  \l 16A3
  \l 16A4
  \l 16A5
  \l 16A6
  \l 16A7
  \l 16A8
  \l 16A9
  \l 16AA
  \l 16AB
  \l 16AC
  \l 16AD
  \l 16AE
  \l 16AF
  \l 16B0
  \l 16B1
  \l 16B2
  \l 16B3
  \l 16B4
  \l 16B5
  \l 16B6
  \l 16B7
  \l 16B8
  \l 16B9
  \l 16BA
  \l 16BB
  \l 16BC
  \l 16BD
  \l 16BE
  \l 16BF
  \l 16C0
  \l 16C1
  \l 16C2
  \l 16C3
  \l 16C4
  \l 16C5
  \l 16C6
  \l 16C7
  \l 16C8
  \l 16C9
  \l 16CA
  \l 16CB
  \l 16CC
  \l 16CD
  \l 16CE
  \l 16CF
  \l 16D0
  \l 16D1
  \l 16D2
  \l 16D3
  \l 16D4
  \l 16D5
  \l 16D6
  \l 16D7
  \l 16D8
  \l 16D9
  \l 16DA
  \l 16DB
  \l 16DC
  \l 16DD
  \l 16DE
  \l 16DF
  \l 16E0
  \l 16E1
  \l 16E2
  \l 16E3
  \l 16E4
  \l 16E5
  \l 16E6
  \l 16E7
  \l 16E8
  \l 16E9
  \l 16EA
  \l 16F1
  \l 16F2
  \l 16F3
  \l 16F4
  \l 16F5
  \l 16F6
  \l 16F7
  \l 16F8
  \l 1700
  \l 1701
  \l 1702
  \l 1703
  \l 1704
  \l 1705
  \l 1706
  \l 1707
  \l 1708
  \l 1709
  \l 170A
  \l 170B
  \l 170C
  \l 170E
  \l 170F
  \l 1710
  \l 1711
  \l 1712
  \l 1713
  \l 1714
  \l 1720
  \l 1721
  \l 1722
  \l 1723
  \l 1724
  \l 1725
  \l 1726
  \l 1727
  \l 1728
  \l 1729
  \l 172A
  \l 172B
  \l 172C
  \l 172D
  \l 172E
  \l 172F
  \l 1730
  \l 1731
  \l 1732
  \l 1733
  \l 1734
  \l 1740
  \l 1741
  \l 1742
  \l 1743
  \l 1744
  \l 1745
  \l 1746
  \l 1747
  \l 1748
  \l 1749
  \l 174A
  \l 174B
  \l 174C
  \l 174D
  \l 174E
  \l 174F
  \l 1750
  \l 1751
  \l 1752
  \l 1753
  \l 1760
  \l 1761
  \l 1762
  \l 1763
  \l 1764
  \l 1765
  \l 1766
  \l 1767
  \l 1768
  \l 1769
  \l 176A
  \l 176B
  \l 176C
  \l 176E
  \l 176F
  \l 1770
  \l 1772
  \l 1773
  \l 1780
  \l 1781
  \l 1782
  \l 1783
  \l 1784
  \l 1785
  \l 1786
  \l 1787
  \l 1788
  \l 1789
  \l 178A
  \l 178B
  \l 178C
  \l 178D
  \l 178E
  \l 178F
  \l 1790
  \l 1791
  \l 1792
  \l 1793
  \l 1794
  \l 1795
  \l 1796
  \l 1797
  \l 1798
  \l 1799
  \l 179A
  \l 179B
  \l 179C
  \l 179D
  \l 179E
  \l 179F
  \l 17A0
  \l 17A1
  \l 17A2
  \l 17A3
  \l 17A4
  \l 17A5
  \l 17A6
  \l 17A7
  \l 17A8
  \l 17A9
  \l 17AA
  \l 17AB
  \l 17AC
  \l 17AD
  \l 17AE
  \l 17AF
  \l 17B0
  \l 17B1
  \l 17B2
  \l 17B3
  \l 17B4
  \l 17B5
  \l 17B6
  \l 17B7
  \l 17B8
  \l 17B9
  \l 17BA
  \l 17BB
  \l 17BC
  \l 17BD
  \l 17BE
  \l 17BF
  \l 17C0
  \l 17C1
  \l 17C2
  \l 17C3
  \l 17C4
  \l 17C5
  \l 17C6
  \l 17C7
  \l 17C8
  \l 17C9
  \l 17CA
  \l 17CB
  \l 17CC
  \l 17CD
  \l 17CE
  \l 17CF
  \l 17D0
  \l 17D1
  \l 17D2
  \l 17D3
  \l 17D7
  \l 17DC
  \l 17DD
  \l 180B
  \l 180C
  \l 180D
  \l 1820
  \l 1821
  \l 1822
  \l 1823
  \l 1824
  \l 1825
  \l 1826
  \l 1827
  \l 1828
  \l 1829
  \l 182A
  \l 182B
  \l 182C
  \l 182D
  \l 182E
  \l 182F
  \l 1830
  \l 1831
  \l 1832
  \l 1833
  \l 1834
  \l 1835
  \l 1836
  \l 1837
  \l 1838
  \l 1839
  \l 183A
  \l 183B
  \l 183C
  \l 183D
  \l 183E
  \l 183F
  \l 1840
  \l 1841
  \l 1842
  \l 1843
  \l 1844
  \l 1845
  \l 1846
  \l 1847
  \l 1848
  \l 1849
  \l 184A
  \l 184B
  \l 184C
  \l 184D
  \l 184E
  \l 184F
  \l 1850
  \l 1851
  \l 1852
  \l 1853
  \l 1854
  \l 1855
  \l 1856
  \l 1857
  \l 1858
  \l 1859
  \l 185A
  \l 185B
  \l 185C
  \l 185D
  \l 185E
  \l 185F
  \l 1860
  \l 1861
  \l 1862
  \l 1863
  \l 1864
  \l 1865
  \l 1866
  \l 1867
  \l 1868
  \l 1869
  \l 186A
  \l 186B
  \l 186C
  \l 186D
  \l 186E
  \l 186F
  \l 1870
  \l 1871
  \l 1872
  \l 1873
  \l 1874
  \l 1875
  \l 1876
  \l 1877
  \l 1880
  \l 1881
  \l 1882
  \l 1883
  \l 1884
  \l 1885
  \l 1886
  \l 1887
  \l 1888
  \l 1889
  \l 188A
  \l 188B
  \l 188C
  \l 188D
  \l 188E
  \l 188F
  \l 1890
  \l 1891
  \l 1892
  \l 1893
  \l 1894
  \l 1895
  \l 1896
  \l 1897
  \l 1898
  \l 1899
  \l 189A
  \l 189B
  \l 189C
  \l 189D
  \l 189E
  \l 189F
  \l 18A0
  \l 18A1
  \l 18A2
  \l 18A3
  \l 18A4
  \l 18A5
  \l 18A6
  \l 18A7
  \l 18A8
  \l 18A9
  \l 18AA
  \l 18B0
  \l 18B1
  \l 18B2
  \l 18B3
  \l 18B4
  \l 18B5
  \l 18B6
  \l 18B7
  \l 18B8
  \l 18B9
  \l 18BA
  \l 18BB
  \l 18BC
  \l 18BD
  \l 18BE
  \l 18BF
  \l 18C0
  \l 18C1
  \l 18C2
  \l 18C3
  \l 18C4
  \l 18C5
  \l 18C6
  \l 18C7
  \l 18C8
  \l 18C9
  \l 18CA
  \l 18CB
  \l 18CC
  \l 18CD
  \l 18CE
  \l 18CF
  \l 18D0
  \l 18D1
  \l 18D2
  \l 18D3
  \l 18D4
  \l 18D5
  \l 18D6
  \l 18D7
  \l 18D8
  \l 18D9
  \l 18DA
  \l 18DB
  \l 18DC
  \l 18DD
  \l 18DE
  \l 18DF
  \l 18E0
  \l 18E1
  \l 18E2
  \l 18E3
  \l 18E4
  \l 18E5
  \l 18E6
  \l 18E7
  \l 18E8
  \l 18E9
  \l 18EA
  \l 18EB
  \l 18EC
  \l 18ED
  \l 18EE
  \l 18EF
  \l 18F0
  \l 18F1
  \l 18F2
  \l 18F3
  \l 18F4
  \l 18F5
  \l 1900
  \l 1901
  \l 1902
  \l 1903
  \l 1904
  \l 1905
  \l 1906
  \l 1907
  \l 1908
  \l 1909
  \l 190A
  \l 190B
  \l 190C
  \l 190D
  \l 190E
  \l 190F
  \l 1910
  \l 1911
  \l 1912
  \l 1913
  \l 1914
  \l 1915
  \l 1916
  \l 1917
  \l 1918
  \l 1919
  \l 191A
  \l 191B
  \l 191C
  \l 191D
  \l 191E
  \l 1920
  \l 1921
  \l 1922
  \l 1923
  \l 1924
  \l 1925
  \l 1926
  \l 1927
  \l 1928
  \l 1929
  \l 192A
  \l 192B
  \l 1930
  \l 1931
  \l 1932
  \l 1933
  \l 1934
  \l 1935
  \l 1936
  \l 1937
  \l 1938
  \l 1939
  \l 193A
  \l 193B
  \l 1950
  \l 1951
  \l 1952
  \l 1953
  \l 1954
  \l 1955
  \l 1956
  \l 1957
  \l 1958
  \l 1959
  \l 195A
  \l 195B
  \l 195C
  \l 195D
  \l 195E
  \l 195F
  \l 1960
  \l 1961
  \l 1962
  \l 1963
  \l 1964
  \l 1965
  \l 1966
  \l 1967
  \l 1968
  \l 1969
  \l 196A
  \l 196B
  \l 196C
  \l 196D
  \l 1970
  \l 1971
  \l 1972
  \l 1973
  \l 1974
  \l 1980
  \l 1981
  \l 1982
  \l 1983
  \l 1984
  \l 1985
  \l 1986
  \l 1987
  \l 1988
  \l 1989
  \l 198A
  \l 198B
  \l 198C
  \l 198D
  \l 198E
  \l 198F
  \l 1990
  \l 1991
  \l 1992
  \l 1993
  \l 1994
  \l 1995
  \l 1996
  \l 1997
  \l 1998
  \l 1999
  \l 199A
  \l 199B
  \l 199C
  \l 199D
  \l 199E
  \l 199F
  \l 19A0
  \l 19A1
  \l 19A2
  \l 19A3
  \l 19A4
  \l 19A5
  \l 19A6
  \l 19A7
  \l 19A8
  \l 19A9
  \l 19AA
  \l 19AB
  \l 19B0
  \l 19B1
  \l 19B2
  \l 19B3
  \l 19B4
  \l 19B5
  \l 19B6
  \l 19B7
  \l 19B8
  \l 19B9
  \l 19BA
  \l 19BB
  \l 19BC
  \l 19BD
  \l 19BE
  \l 19BF
  \l 19C0
  \l 19C1
  \l 19C2
  \l 19C3
  \l 19C4
  \l 19C5
  \l 19C6
  \l 19C7
  \l 19C8
  \l 19C9
  \l 1A00
  \l 1A01
  \l 1A02
  \l 1A03
  \l 1A04
  \l 1A05
  \l 1A06
  \l 1A07
  \l 1A08
  \l 1A09
  \l 1A0A
  \l 1A0B
  \l 1A0C
  \l 1A0D
  \l 1A0E
  \l 1A0F
  \l 1A10
  \l 1A11
  \l 1A12
  \l 1A13
  \l 1A14
  \l 1A15
  \l 1A16
  \l 1A17
  \l 1A18
  \l 1A19
  \l 1A1A
  \l 1A1B
  \l 1A20
  \l 1A21
  \l 1A22
  \l 1A23
  \l 1A24
  \l 1A25
  \l 1A26
  \l 1A27
  \l 1A28
  \l 1A29
  \l 1A2A
  \l 1A2B
  \l 1A2C
  \l 1A2D
  \l 1A2E
  \l 1A2F
  \l 1A30
  \l 1A31
  \l 1A32
  \l 1A33
  \l 1A34
  \l 1A35
  \l 1A36
  \l 1A37
  \l 1A38
  \l 1A39
  \l 1A3A
  \l 1A3B
  \l 1A3C
  \l 1A3D
  \l 1A3E
  \l 1A3F
  \l 1A40
  \l 1A41
  \l 1A42
  \l 1A43
  \l 1A44
  \l 1A45
  \l 1A46
  \l 1A47
  \l 1A48
  \l 1A49
  \l 1A4A
  \l 1A4B
  \l 1A4C
  \l 1A4D
  \l 1A4E
  \l 1A4F
  \l 1A50
  \l 1A51
  \l 1A52
  \l 1A53
  \l 1A54
  \l 1A55
  \l 1A56
  \l 1A57
  \l 1A58
  \l 1A59
  \l 1A5A
  \l 1A5B
  \l 1A5C
  \l 1A5D
  \l 1A5E
  \l 1A60
  \l 1A61
  \l 1A62
  \l 1A63
  \l 1A64
  \l 1A65
  \l 1A66
  \l 1A67
  \l 1A68
  \l 1A69
  \l 1A6A
  \l 1A6B
  \l 1A6C
  \l 1A6D
  \l 1A6E
  \l 1A6F
  \l 1A70
  \l 1A71
  \l 1A72
  \l 1A73
  \l 1A74
  \l 1A75
  \l 1A76
  \l 1A77
  \l 1A78
  \l 1A79
  \l 1A7A
  \l 1A7B
  \l 1A7C
  \l 1A7F
  \l 1AA7
  \l 1AB0
  \l 1AB1
  \l 1AB2
  \l 1AB3
  \l 1AB4
  \l 1AB5
  \l 1AB6
  \l 1AB7
  \l 1AB8
  \l 1AB9
  \l 1ABA
  \l 1ABB
  \l 1ABC
  \l 1ABD
  \l 1ABE
  \l 1B00
  \l 1B01
  \l 1B02
  \l 1B03
  \l 1B04
  \l 1B05
  \l 1B06
  \l 1B07
  \l 1B08
  \l 1B09
  \l 1B0A
  \l 1B0B
  \l 1B0C
  \l 1B0D
  \l 1B0E
  \l 1B0F
  \l 1B10
  \l 1B11
  \l 1B12
  \l 1B13
  \l 1B14
  \l 1B15
  \l 1B16
  \l 1B17
  \l 1B18
  \l 1B19
  \l 1B1A
  \l 1B1B
  \l 1B1C
  \l 1B1D
  \l 1B1E
  \l 1B1F
  \l 1B20
  \l 1B21
  \l 1B22
  \l 1B23
  \l 1B24
  \l 1B25
  \l 1B26
  \l 1B27
  \l 1B28
  \l 1B29
  \l 1B2A
  \l 1B2B
  \l 1B2C
  \l 1B2D
  \l 1B2E
  \l 1B2F
  \l 1B30
  \l 1B31
  \l 1B32
  \l 1B33
  \l 1B34
  \l 1B35
  \l 1B36
  \l 1B37
  \l 1B38
  \l 1B39
  \l 1B3A
  \l 1B3B
  \l 1B3C
  \l 1B3D
  \l 1B3E
  \l 1B3F
  \l 1B40
  \l 1B41
  \l 1B42
  \l 1B43
  \l 1B44
  \l 1B45
  \l 1B46
  \l 1B47
  \l 1B48
  \l 1B49
  \l 1B4A
  \l 1B4B
  \l 1B6B
  \l 1B6C
  \l 1B6D
  \l 1B6E
  \l 1B6F
  \l 1B70
  \l 1B71
  \l 1B72
  \l 1B73
  \l 1B80
  \l 1B81
  \l 1B82
  \l 1B83
  \l 1B84
  \l 1B85
  \l 1B86
  \l 1B87
  \l 1B88
  \l 1B89
  \l 1B8A
  \l 1B8B
  \l 1B8C
  \l 1B8D
  \l 1B8E
  \l 1B8F
  \l 1B90
  \l 1B91
  \l 1B92
  \l 1B93
  \l 1B94
  \l 1B95
  \l 1B96
  \l 1B97
  \l 1B98
  \l 1B99
  \l 1B9A
  \l 1B9B
  \l 1B9C
  \l 1B9D
  \l 1B9E
  \l 1B9F
  \l 1BA0
  \l 1BA1
  \l 1BA2
  \l 1BA3
  \l 1BA4
  \l 1BA5
  \l 1BA6
  \l 1BA7
  \l 1BA8
  \l 1BA9
  \l 1BAA
  \l 1BAB
  \l 1BAC
  \l 1BAD
  \l 1BAE
  \l 1BAF
  \l 1BBA
  \l 1BBB
  \l 1BBC
  \l 1BBD
  \l 1BBE
  \l 1BBF
  \l 1BC0
  \l 1BC1
  \l 1BC2
  \l 1BC3
  \l 1BC4
  \l 1BC5
  \l 1BC6
  \l 1BC7
  \l 1BC8
  \l 1BC9
  \l 1BCA
  \l 1BCB
  \l 1BCC
  \l 1BCD
  \l 1BCE
  \l 1BCF
  \l 1BD0
  \l 1BD1
  \l 1BD2
  \l 1BD3
  \l 1BD4
  \l 1BD5
  \l 1BD6
  \l 1BD7
  \l 1BD8
  \l 1BD9
  \l 1BDA
  \l 1BDB
  \l 1BDC
  \l 1BDD
  \l 1BDE
  \l 1BDF
  \l 1BE0
  \l 1BE1
  \l 1BE2
  \l 1BE3
  \l 1BE4
  \l 1BE5
  \l 1BE6
  \l 1BE7
  \l 1BE8
  \l 1BE9
  \l 1BEA
  \l 1BEB
  \l 1BEC
  \l 1BED
  \l 1BEE
  \l 1BEF
  \l 1BF0
  \l 1BF1
  \l 1BF2
  \l 1BF3
  \l 1C00
  \l 1C01
  \l 1C02
  \l 1C03
  \l 1C04
  \l 1C05
  \l 1C06
  \l 1C07
  \l 1C08
  \l 1C09
  \l 1C0A
  \l 1C0B
  \l 1C0C
  \l 1C0D
  \l 1C0E
  \l 1C0F
  \l 1C10
  \l 1C11
  \l 1C12
  \l 1C13
  \l 1C14
  \l 1C15
  \l 1C16
  \l 1C17
  \l 1C18
  \l 1C19
  \l 1C1A
  \l 1C1B
  \l 1C1C
  \l 1C1D
  \l 1C1E
  \l 1C1F
  \l 1C20
  \l 1C21
  \l 1C22
  \l 1C23
  \l 1C24
  \l 1C25
  \l 1C26
  \l 1C27
  \l 1C28
  \l 1C29
  \l 1C2A
  \l 1C2B
  \l 1C2C
  \l 1C2D
  \l 1C2E
  \l 1C2F
  \l 1C30
  \l 1C31
  \l 1C32
  \l 1C33
  \l 1C34
  \l 1C35
  \l 1C36
  \l 1C37
  \l 1C4D
  \l 1C4E
  \l 1C4F
  \l 1C5A
  \l 1C5B
  \l 1C5C
  \l 1C5D
  \l 1C5E
  \l 1C5F
  \l 1C60
  \l 1C61
  \l 1C62
  \l 1C63
  \l 1C64
  \l 1C65
  \l 1C66
  \l 1C67
  \l 1C68
  \l 1C69
  \l 1C6A
  \l 1C6B
  \l 1C6C
  \l 1C6D
  \l 1C6E
  \l 1C6F
  \l 1C70
  \l 1C71
  \l 1C72
  \l 1C73
  \l 1C74
  \l 1C75
  \l 1C76
  \l 1C77
  \l 1C78
  \l 1C79
  \l 1C7A
  \l 1C7B
  \l 1C7C
  \l 1C7D
  \l 1CD0
  \l 1CD1
  \l 1CD2
  \l 1CD4
  \l 1CD5
  \l 1CD6
  \l 1CD7
  \l 1CD8
  \l 1CD9
  \l 1CDA
  \l 1CDB
  \l 1CDC
  \l 1CDD
  \l 1CDE
  \l 1CDF
  \l 1CE0
  \l 1CE1
  \l 1CE2
  \l 1CE3
  \l 1CE4
  \l 1CE5
  \l 1CE6
  \l 1CE7
  \l 1CE8
  \l 1CE9
  \l 1CEA
  \l 1CEB
  \l 1CEC
  \l 1CED
  \l 1CEE
  \l 1CEF
  \l 1CF0
  \l 1CF1
  \l 1CF2
  \l 1CF3
  \l 1CF4
  \l 1CF5
  \l 1CF6
  \l 1CF8
  \l 1CF9
  \l 1D00
  \l 1D01
  \l 1D02
  \l 1D03
  \l 1D04
  \l 1D05
  \l 1D06
  \l 1D07
  \l 1D08
  \l 1D09
  \l 1D0A
  \l 1D0B
  \l 1D0C
  \l 1D0D
  \l 1D0E
  \l 1D0F
  \l 1D10
  \l 1D11
  \l 1D12
  \l 1D13
  \l 1D14
  \l 1D15
  \l 1D16
  \l 1D17
  \l 1D18
  \l 1D19
  \l 1D1A
  \l 1D1B
  \l 1D1C
  \l 1D1D
  \l 1D1E
  \l 1D1F
  \l 1D20
  \l 1D21
  \l 1D22
  \l 1D23
  \l 1D24
  \l 1D25
  \l 1D26
  \l 1D27
  \l 1D28
  \l 1D29
  \l 1D2A
  \l 1D2B
  \l 1D2C
  \l 1D2D
  \l 1D2E
  \l 1D2F
  \l 1D30
  \l 1D31
  \l 1D32
  \l 1D33
  \l 1D34
  \l 1D35
  \l 1D36
  \l 1D37
  \l 1D38
  \l 1D39
  \l 1D3A
  \l 1D3B
  \l 1D3C
  \l 1D3D
  \l 1D3E
  \l 1D3F
  \l 1D40
  \l 1D41
  \l 1D42
  \l 1D43
  \l 1D44
  \l 1D45
  \l 1D46
  \l 1D47
  \l 1D48
  \l 1D49
  \l 1D4A
  \l 1D4B
  \l 1D4C
  \l 1D4D
  \l 1D4E
  \l 1D4F
  \l 1D50
  \l 1D51
  \l 1D52
  \l 1D53
  \l 1D54
  \l 1D55
  \l 1D56
  \l 1D57
  \l 1D58
  \l 1D59
  \l 1D5A
  \l 1D5B
  \l 1D5C
  \l 1D5D
  \l 1D5E
  \l 1D5F
  \l 1D60
  \l 1D61
  \l 1D62
  \l 1D63
  \l 1D64
  \l 1D65
  \l 1D66
  \l 1D67
  \l 1D68
  \l 1D69
  \l 1D6A
  \l 1D6B
  \l 1D6C
  \l 1D6D
  \l 1D6E
  \l 1D6F
  \l 1D70
  \l 1D71
  \l 1D72
  \l 1D73
  \l 1D74
  \l 1D75
  \l 1D76
  \l 1D77
  \l 1D78
  \L 1D79 A77D 1D79
  \l 1D7A
  \l 1D7B
  \l 1D7C
  \L 1D7D 2C63 1D7D
  \l 1D7E
  \l 1D7F
  \l 1D80
  \l 1D81
  \l 1D82
  \l 1D83
  \l 1D84
  \l 1D85
  \l 1D86
  \l 1D87
  \l 1D88
  \l 1D89
  \l 1D8A
  \l 1D8B
  \l 1D8C
  \l 1D8D
  \l 1D8E
  \l 1D8F
  \l 1D90
  \l 1D91
  \l 1D92
  \l 1D93
  \l 1D94
  \l 1D95
  \l 1D96
  \l 1D97
  \l 1D98
  \l 1D99
  \l 1D9A
  \l 1D9B
  \l 1D9C
  \l 1D9D
  \l 1D9E
  \l 1D9F
  \l 1DA0
  \l 1DA1
  \l 1DA2
  \l 1DA3
  \l 1DA4
  \l 1DA5
  \l 1DA6
  \l 1DA7
  \l 1DA8
  \l 1DA9
  \l 1DAA
  \l 1DAB
  \l 1DAC
  \l 1DAD
  \l 1DAE
  \l 1DAF
  \l 1DB0
  \l 1DB1
  \l 1DB2
  \l 1DB3
  \l 1DB4
  \l 1DB5
  \l 1DB6
  \l 1DB7
  \l 1DB8
  \l 1DB9
  \l 1DBA
  \l 1DBB
  \l 1DBC
  \l 1DBD
  \l 1DBE
  \l 1DBF
  \l 1DC0
  \l 1DC1
  \l 1DC2
  \l 1DC3
  \l 1DC4
  \l 1DC5
  \l 1DC6
  \l 1DC7
  \l 1DC8
  \l 1DC9
  \l 1DCA
  \l 1DCB
  \l 1DCC
  \l 1DCD
  \l 1DCE
  \l 1DCF
  \l 1DD0
  \l 1DD1
  \l 1DD2
  \l 1DD3
  \l 1DD4
  \l 1DD5
  \l 1DD6
  \l 1DD7
  \l 1DD8
  \l 1DD9
  \l 1DDA
  \l 1DDB
  \l 1DDC
  \l 1DDD
  \l 1DDE
  \l 1DDF
  \l 1DE0
  \l 1DE1
  \l 1DE2
  \l 1DE3
  \l 1DE4
  \l 1DE5
  \l 1DE6
  \l 1DE7
  \l 1DE8
  \l 1DE9
  \l 1DEA
  \l 1DEB
  \l 1DEC
  \l 1DED
  \l 1DEE
  \l 1DEF
  \l 1DF0
  \l 1DF1
  \l 1DF2
  \l 1DF3
  \l 1DF4
  \l 1DF5
  \l 1DFC
  \l 1DFD
  \l 1DFE
  \l 1DFF
  \L 1E00 1E00 1E01
  \L 1E01 1E00 1E01
  \L 1E02 1E02 1E03
  \L 1E03 1E02 1E03
  \L 1E04 1E04 1E05
  \L 1E05 1E04 1E05
  \L 1E06 1E06 1E07
  \L 1E07 1E06 1E07
  \L 1E08 1E08 1E09
  \L 1E09 1E08 1E09
  \L 1E0A 1E0A 1E0B
  \L 1E0B 1E0A 1E0B
  \L 1E0C 1E0C 1E0D
  \L 1E0D 1E0C 1E0D
  \L 1E0E 1E0E 1E0F
  \L 1E0F 1E0E 1E0F
  \L 1E10 1E10 1E11
  \L 1E11 1E10 1E11
  \L 1E12 1E12 1E13
  \L 1E13 1E12 1E13
  \L 1E14 1E14 1E15
  \L 1E15 1E14 1E15
  \L 1E16 1E16 1E17
  \L 1E17 1E16 1E17
  \L 1E18 1E18 1E19
  \L 1E19 1E18 1E19
  \L 1E1A 1E1A 1E1B
  \L 1E1B 1E1A 1E1B
  \L 1E1C 1E1C 1E1D
  \L 1E1D 1E1C 1E1D
  \L 1E1E 1E1E 1E1F
  \L 1E1F 1E1E 1E1F
  \L 1E20 1E20 1E21
  \L 1E21 1E20 1E21
  \L 1E22 1E22 1E23
  \L 1E23 1E22 1E23
  \L 1E24 1E24 1E25
  \L 1E25 1E24 1E25
  \L 1E26 1E26 1E27
  \L 1E27 1E26 1E27
  \L 1E28 1E28 1E29
  \L 1E29 1E28 1E29
  \L 1E2A 1E2A 1E2B
  \L 1E2B 1E2A 1E2B
  \L 1E2C 1E2C 1E2D
  \L 1E2D 1E2C 1E2D
  \L 1E2E 1E2E 1E2F
  \L 1E2F 1E2E 1E2F
  \L 1E30 1E30 1E31
  \L 1E31 1E30 1E31
  \L 1E32 1E32 1E33
  \L 1E33 1E32 1E33
  \L 1E34 1E34 1E35
  \L 1E35 1E34 1E35
  \L 1E36 1E36 1E37
  \L 1E37 1E36 1E37
  \L 1E38 1E38 1E39
  \L 1E39 1E38 1E39
  \L 1E3A 1E3A 1E3B
  \L 1E3B 1E3A 1E3B
  \L 1E3C 1E3C 1E3D
  \L 1E3D 1E3C 1E3D
  \L 1E3E 1E3E 1E3F
  \L 1E3F 1E3E 1E3F
  \L 1E40 1E40 1E41
  \L 1E41 1E40 1E41
  \L 1E42 1E42 1E43
  \L 1E43 1E42 1E43
  \L 1E44 1E44 1E45
  \L 1E45 1E44 1E45
  \L 1E46 1E46 1E47
  \L 1E47 1E46 1E47
  \L 1E48 1E48 1E49
  \L 1E49 1E48 1E49
  \L 1E4A 1E4A 1E4B
  \L 1E4B 1E4A 1E4B
  \L 1E4C 1E4C 1E4D
  \L 1E4D 1E4C 1E4D
  \L 1E4E 1E4E 1E4F
  \L 1E4F 1E4E 1E4F
  \L 1E50 1E50 1E51
  \L 1E51 1E50 1E51
  \L 1E52 1E52 1E53
  \L 1E53 1E52 1E53
  \L 1E54 1E54 1E55
  \L 1E55 1E54 1E55
  \L 1E56 1E56 1E57
  \L 1E57 1E56 1E57
  \L 1E58 1E58 1E59
  \L 1E59 1E58 1E59
  \L 1E5A 1E5A 1E5B
  \L 1E5B 1E5A 1E5B
  \L 1E5C 1E5C 1E5D
  \L 1E5D 1E5C 1E5D
  \L 1E5E 1E5E 1E5F
  \L 1E5F 1E5E 1E5F
  \L 1E60 1E60 1E61
  \L 1E61 1E60 1E61
  \L 1E62 1E62 1E63
  \L 1E63 1E62 1E63
  \L 1E64 1E64 1E65
  \L 1E65 1E64 1E65
  \L 1E66 1E66 1E67
  \L 1E67 1E66 1E67
  \L 1E68 1E68 1E69
  \L 1E69 1E68 1E69
  \L 1E6A 1E6A 1E6B
  \L 1E6B 1E6A 1E6B
  \L 1E6C 1E6C 1E6D
  \L 1E6D 1E6C 1E6D
  \L 1E6E 1E6E 1E6F
  \L 1E6F 1E6E 1E6F
  \L 1E70 1E70 1E71
  \L 1E71 1E70 1E71
  \L 1E72 1E72 1E73
  \L 1E73 1E72 1E73
  \L 1E74 1E74 1E75
  \L 1E75 1E74 1E75
  \L 1E76 1E76 1E77
  \L 1E77 1E76 1E77
  \L 1E78 1E78 1E79
  \L 1E79 1E78 1E79
  \L 1E7A 1E7A 1E7B
  \L 1E7B 1E7A 1E7B
  \L 1E7C 1E7C 1E7D
  \L 1E7D 1E7C 1E7D
  \L 1E7E 1E7E 1E7F
  \L 1E7F 1E7E 1E7F
  \L 1E80 1E80 1E81
  \L 1E81 1E80 1E81
  \L 1E82 1E82 1E83
  \L 1E83 1E82 1E83
  \L 1E84 1E84 1E85
  \L 1E85 1E84 1E85
  \L 1E86 1E86 1E87
  \L 1E87 1E86 1E87
  \L 1E88 1E88 1E89
  \L 1E89 1E88 1E89
  \L 1E8A 1E8A 1E8B
  \L 1E8B 1E8A 1E8B
  \L 1E8C 1E8C 1E8D
  \L 1E8D 1E8C 1E8D
  \L 1E8E 1E8E 1E8F
  \L 1E8F 1E8E 1E8F
  \L 1E90 1E90 1E91
  \L 1E91 1E90 1E91
  \L 1E92 1E92 1E93
  \L 1E93 1E92 1E93
  \L 1E94 1E94 1E95
  \L 1E95 1E94 1E95
  \l 1E96
  \l 1E97
  \l 1E98
  \l 1E99
  \l 1E9A
  \L 1E9B 1E60 1E9B
  \l 1E9C
  \l 1E9D
  \L 1E9E 1E9E 00DF
  \l 1E9F
  \L 1EA0 1EA0 1EA1
  \L 1EA1 1EA0 1EA1
  \L 1EA2 1EA2 1EA3
  \L 1EA3 1EA2 1EA3
  \L 1EA4 1EA4 1EA5
  \L 1EA5 1EA4 1EA5
  \L 1EA6 1EA6 1EA7
  \L 1EA7 1EA6 1EA7
  \L 1EA8 1EA8 1EA9
  \L 1EA9 1EA8 1EA9
  \L 1EAA 1EAA 1EAB
  \L 1EAB 1EAA 1EAB
  \L 1EAC 1EAC 1EAD
  \L 1EAD 1EAC 1EAD
  \L 1EAE 1EAE 1EAF
  \L 1EAF 1EAE 1EAF
  \L 1EB0 1EB0 1EB1
  \L 1EB1 1EB0 1EB1
  \L 1EB2 1EB2 1EB3
  \L 1EB3 1EB2 1EB3
  \L 1EB4 1EB4 1EB5
  \L 1EB5 1EB4 1EB5
  \L 1EB6 1EB6 1EB7
  \L 1EB7 1EB6 1EB7
  \L 1EB8 1EB8 1EB9
  \L 1EB9 1EB8 1EB9
  \L 1EBA 1EBA 1EBB
  \L 1EBB 1EBA 1EBB
  \L 1EBC 1EBC 1EBD
  \L 1EBD 1EBC 1EBD
  \L 1EBE 1EBE 1EBF
  \L 1EBF 1EBE 1EBF
  \L 1EC0 1EC0 1EC1
  \L 1EC1 1EC0 1EC1
  \L 1EC2 1EC2 1EC3
  \L 1EC3 1EC2 1EC3
  \L 1EC4 1EC4 1EC5
  \L 1EC5 1EC4 1EC5
  \L 1EC6 1EC6 1EC7
  \L 1EC7 1EC6 1EC7
  \L 1EC8 1EC8 1EC9
  \L 1EC9 1EC8 1EC9
  \L 1ECA 1ECA 1ECB
  \L 1ECB 1ECA 1ECB
  \L 1ECC 1ECC 1ECD
  \L 1ECD 1ECC 1ECD
  \L 1ECE 1ECE 1ECF
  \L 1ECF 1ECE 1ECF
  \L 1ED0 1ED0 1ED1
  \L 1ED1 1ED0 1ED1
  \L 1ED2 1ED2 1ED3
  \L 1ED3 1ED2 1ED3
  \L 1ED4 1ED4 1ED5
  \L 1ED5 1ED4 1ED5
  \L 1ED6 1ED6 1ED7
  \L 1ED7 1ED6 1ED7
  \L 1ED8 1ED8 1ED9
  \L 1ED9 1ED8 1ED9
  \L 1EDA 1EDA 1EDB
  \L 1EDB 1EDA 1EDB
  \L 1EDC 1EDC 1EDD
  \L 1EDD 1EDC 1EDD
  \L 1EDE 1EDE 1EDF
  \L 1EDF 1EDE 1EDF
  \L 1EE0 1EE0 1EE1
  \L 1EE1 1EE0 1EE1
  \L 1EE2 1EE2 1EE3
  \L 1EE3 1EE2 1EE3
  \L 1EE4 1EE4 1EE5
  \L 1EE5 1EE4 1EE5
  \L 1EE6 1EE6 1EE7
  \L 1EE7 1EE6 1EE7
  \L 1EE8 1EE8 1EE9
  \L 1EE9 1EE8 1EE9
  \L 1EEA 1EEA 1EEB
  \L 1EEB 1EEA 1EEB
  \L 1EEC 1EEC 1EED
  \L 1EED 1EEC 1EED
  \L 1EEE 1EEE 1EEF
  \L 1EEF 1EEE 1EEF
  \L 1EF0 1EF0 1EF1
  \L 1EF1 1EF0 1EF1
  \L 1EF2 1EF2 1EF3
  \L 1EF3 1EF2 1EF3
  \L 1EF4 1EF4 1EF5
  \L 1EF5 1EF4 1EF5
  \L 1EF6 1EF6 1EF7
  \L 1EF7 1EF6 1EF7
  \L 1EF8 1EF8 1EF9
  \L 1EF9 1EF8 1EF9
  \L 1EFA 1EFA 1EFB
  \L 1EFB 1EFA 1EFB
  \L 1EFC 1EFC 1EFD
  \L 1EFD 1EFC 1EFD
  \L 1EFE 1EFE 1EFF
  \L 1EFF 1EFE 1EFF
  \L 1F00 1F08 1F00
  \L 1F01 1F09 1F01
  \L 1F02 1F0A 1F02
  \L 1F03 1F0B 1F03
  \L 1F04 1F0C 1F04
  \L 1F05 1F0D 1F05
  \L 1F06 1F0E 1F06
  \L 1F07 1F0F 1F07
  \L 1F08 1F08 1F00
  \L 1F09 1F09 1F01
  \L 1F0A 1F0A 1F02
  \L 1F0B 1F0B 1F03
  \L 1F0C 1F0C 1F04
  \L 1F0D 1F0D 1F05
  \L 1F0E 1F0E 1F06
  \L 1F0F 1F0F 1F07
  \L 1F10 1F18 1F10
  \L 1F11 1F19 1F11
  \L 1F12 1F1A 1F12
  \L 1F13 1F1B 1F13
  \L 1F14 1F1C 1F14
  \L 1F15 1F1D 1F15
  \L 1F18 1F18 1F10
  \L 1F19 1F19 1F11
  \L 1F1A 1F1A 1F12
  \L 1F1B 1F1B 1F13
  \L 1F1C 1F1C 1F14
  \L 1F1D 1F1D 1F15
  \L 1F20 1F28 1F20
  \L 1F21 1F29 1F21
  \L 1F22 1F2A 1F22
  \L 1F23 1F2B 1F23
  \L 1F24 1F2C 1F24
  \L 1F25 1F2D 1F25
  \L 1F26 1F2E 1F26
  \L 1F27 1F2F 1F27
  \L 1F28 1F28 1F20
  \L 1F29 1F29 1F21
  \L 1F2A 1F2A 1F22
  \L 1F2B 1F2B 1F23
  \L 1F2C 1F2C 1F24
  \L 1F2D 1F2D 1F25
  \L 1F2E 1F2E 1F26
  \L 1F2F 1F2F 1F27
  \L 1F30 1F38 1F30
  \L 1F31 1F39 1F31
  \L 1F32 1F3A 1F32
  \L 1F33 1F3B 1F33
  \L 1F34 1F3C 1F34
  \L 1F35 1F3D 1F35
  \L 1F36 1F3E 1F36
  \L 1F37 1F3F 1F37
  \L 1F38 1F38 1F30
  \L 1F39 1F39 1F31
  \L 1F3A 1F3A 1F32
  \L 1F3B 1F3B 1F33
  \L 1F3C 1F3C 1F34
  \L 1F3D 1F3D 1F35
  \L 1F3E 1F3E 1F36
  \L 1F3F 1F3F 1F37
  \L 1F40 1F48 1F40
  \L 1F41 1F49 1F41
  \L 1F42 1F4A 1F42
  \L 1F43 1F4B 1F43
  \L 1F44 1F4C 1F44
  \L 1F45 1F4D 1F45
  \L 1F48 1F48 1F40
  \L 1F49 1F49 1F41
  \L 1F4A 1F4A 1F42
  \L 1F4B 1F4B 1F43
  \L 1F4C 1F4C 1F44
  \L 1F4D 1F4D 1F45
  \l 1F50
  \L 1F51 1F59 1F51
  \l 1F52
  \L 1F53 1F5B 1F53
  \l 1F54
  \L 1F55 1F5D 1F55
  \l 1F56
  \L 1F57 1F5F 1F57
  \L 1F59 1F59 1F51
  \L 1F5B 1F5B 1F53
  \L 1F5D 1F5D 1F55
  \L 1F5F 1F5F 1F57
  \L 1F60 1F68 1F60
  \L 1F61 1F69 1F61
  \L 1F62 1F6A 1F62
  \L 1F63 1F6B 1F63
  \L 1F64 1F6C 1F64
  \L 1F65 1F6D 1F65
  \L 1F66 1F6E 1F66
  \L 1F67 1F6F 1F67
  \L 1F68 1F68 1F60
  \L 1F69 1F69 1F61
  \L 1F6A 1F6A 1F62
  \L 1F6B 1F6B 1F63
  \L 1F6C 1F6C 1F64
  \L 1F6D 1F6D 1F65
  \L 1F6E 1F6E 1F66
  \L 1F6F 1F6F 1F67
  \L 1F70 1FBA 1F70
  \L 1F71 1FBB 1F71
  \L 1F72 1FC8 1F72
  \L 1F73 1FC9 1F73
  \L 1F74 1FCA 1F74
  \L 1F75 1FCB 1F75
  \L 1F76 1FDA 1F76
  \L 1F77 1FDB 1F77
  \L 1F78 1FF8 1F78
  \L 1F79 1FF9 1F79
  \L 1F7A 1FEA 1F7A
  \L 1F7B 1FEB 1F7B
  \L 1F7C 1FFA 1F7C
  \L 1F7D 1FFB 1F7D
  \L 1F80 1F88 1F80
  \L 1F81 1F89 1F81
  \L 1F82 1F8A 1F82
  \L 1F83 1F8B 1F83
  \L 1F84 1F8C 1F84
  \L 1F85 1F8D 1F85
  \L 1F86 1F8E 1F86
  \L 1F87 1F8F 1F87
  \L 1F88 1F88 1F80
  \L 1F89 1F89 1F81
  \L 1F8A 1F8A 1F82
  \L 1F8B 1F8B 1F83
  \L 1F8C 1F8C 1F84
  \L 1F8D 1F8D 1F85
  \L 1F8E 1F8E 1F86
  \L 1F8F 1F8F 1F87
  \L 1F90 1F98 1F90
  \L 1F91 1F99 1F91
  \L 1F92 1F9A 1F92
  \L 1F93 1F9B 1F93
  \L 1F94 1F9C 1F94
  \L 1F95 1F9D 1F95
  \L 1F96 1F9E 1F96
  \L 1F97 1F9F 1F97
  \L 1F98 1F98 1F90
  \L 1F99 1F99 1F91
  \L 1F9A 1F9A 1F92
  \L 1F9B 1F9B 1F93
  \L 1F9C 1F9C 1F94
  \L 1F9D 1F9D 1F95
  \L 1F9E 1F9E 1F96
  \L 1F9F 1F9F 1F97
  \L 1FA0 1FA8 1FA0
  \L 1FA1 1FA9 1FA1
  \L 1FA2 1FAA 1FA2
  \L 1FA3 1FAB 1FA3
  \L 1FA4 1FAC 1FA4
  \L 1FA5 1FAD 1FA5
  \L 1FA6 1FAE 1FA6
  \L 1FA7 1FAF 1FA7
  \L 1FA8 1FA8 1FA0
  \L 1FA9 1FA9 1FA1
  \L 1FAA 1FAA 1FA2
  \L 1FAB 1FAB 1FA3
  \L 1FAC 1FAC 1FA4
  \L 1FAD 1FAD 1FA5
  \L 1FAE 1FAE 1FA6
  \L 1FAF 1FAF 1FA7
  \L 1FB0 1FB8 1FB0
  \L 1FB1 1FB9 1FB1
  \l 1FB2
  \L 1FB3 1FBC 1FB3
  \l 1FB4
  \l 1FB6
  \l 1FB7
  \L 1FB8 1FB8 1FB0
  \L 1FB9 1FB9 1FB1
  \L 1FBA 1FBA 1F70
  \L 1FBB 1FBB 1F71
  \L 1FBC 1FBC 1FB3
  \L 1FBE 0399 1FBE
  \l 1FC2
  \L 1FC3 1FCC 1FC3
  \l 1FC4
  \l 1FC6
  \l 1FC7
  \L 1FC8 1FC8 1F72
  \L 1FC9 1FC9 1F73
  \L 1FCA 1FCA 1F74
  \L 1FCB 1FCB 1F75
  \L 1FCC 1FCC 1FC3
  \L 1FD0 1FD8 1FD0
  \L 1FD1 1FD9 1FD1
  \l 1FD2
  \l 1FD3
  \l 1FD6
  \l 1FD7
  \L 1FD8 1FD8 1FD0
  \L 1FD9 1FD9 1FD1
  \L 1FDA 1FDA 1F76
  \L 1FDB 1FDB 1F77
  \L 1FE0 1FE8 1FE0
  \L 1FE1 1FE9 1FE1
  \l 1FE2
  \l 1FE3
  \l 1FE4
  \L 1FE5 1FEC 1FE5
  \l 1FE6
  \l 1FE7
  \L 1FE8 1FE8 1FE0
  \L 1FE9 1FE9 1FE1
  \L 1FEA 1FEA 1F7A
  \L 1FEB 1FEB 1F7B
  \L 1FEC 1FEC 1FE5
  \l 1FF2
  \L 1FF3 1FFC 1FF3
  \l 1FF4
  \l 1FF6
  \l 1FF7
  \L 1FF8 1FF8 1F78
  \L 1FF9 1FF9 1F79
  \L 1FFA 1FFA 1F7C
  \L 1FFB 1FFB 1F7D
  \L 1FFC 1FFC 1FF3
  \l 2071
  \l 207F
  \l 2090
  \l 2091
  \l 2092
  \l 2093
  \l 2094
  \l 2095
  \l 2096
  \l 2097
  \l 2098
  \l 2099
  \l 209A
  \l 209B
  \l 209C
  \l 20D0
  \l 20D1
  \l 20D2
  \l 20D3
  \l 20D4
  \l 20D5
  \l 20D6
  \l 20D7
  \l 20D8
  \l 20D9
  \l 20DA
  \l 20DB
  \l 20DC
  \l 20DD
  \l 20DE
  \l 20DF
  \l 20E0
  \l 20E1
  \l 20E2
  \l 20E3
  \l 20E4
  \l 20E5
  \l 20E6
  \l 20E7
  \l 20E8
  \l 20E9
  \l 20EA
  \l 20EB
  \l 20EC
  \l 20ED
  \l 20EE
  \l 20EF
  \l 20F0
  \l 2102
  \l 2107
  \l 210A
  \l 210B
  \l 210C
  \l 210D
  \l 210E
  \l 210F
  \l 2110
  \l 2111
  \l 2112
  \l 2113
  \l 2115
  \l 2119
  \l 211A
  \l 211B
  \l 211C
  \l 211D
  \l 2124
  \L 2126 2126 03C9
  \l 2128
  \L 212A 212A 006B
  \L 212B 212B 00E5
  \l 212C
  \l 212D
  \l 212F
  \l 2130
  \l 2131
  \L 2132 2132 214E
  \l 2133
  \l 2134
  \l 2135
  \l 2136
  \l 2137
  \l 2138
  \l 2139
  \l 213C
  \l 213D
  \l 213E
  \l 213F
  \l 2145
  \l 2146
  \l 2147
  \l 2148
  \l 2149
  \L 214E 2132 214E
  \C 2160 2160 2170
  \C 2161 2161 2171
  \C 2162 2162 2172
  \C 2163 2163 2173
  \C 2164 2164 2174
  \C 2165 2165 2175
  \C 2166 2166 2176
  \C 2167 2167 2177
  \C 2168 2168 2178
  \C 2169 2169 2179
  \C 216A 216A 217A
  \C 216B 216B 217B
  \C 216C 216C 217C
  \C 216D 216D 217D
  \C 216E 216E 217E
  \C 216F 216F 217F
  \C 2170 2160 2170
  \C 2171 2161 2171
  \C 2172 2162 2172
  \C 2173 2163 2173
  \C 2174 2164 2174
  \C 2175 2165 2175
  \C 2176 2166 2176
  \C 2177 2167 2177
  \C 2178 2168 2178
  \C 2179 2169 2179
  \C 217A 216A 217A
  \C 217B 216B 217B
  \C 217C 216C 217C
  \C 217D 216D 217D
  \C 217E 216E 217E
  \C 217F 216F 217F
  \L 2183 2183 2184
  \L 2184 2183 2184
  \C 24B6 24B6 24D0
  \C 24B7 24B7 24D1
  \C 24B8 24B8 24D2
  \C 24B9 24B9 24D3
  \C 24BA 24BA 24D4
  \C 24BB 24BB 24D5
  \C 24BC 24BC 24D6
  \C 24BD 24BD 24D7
  \C 24BE 24BE 24D8
  \C 24BF 24BF 24D9
  \C 24C0 24C0 24DA
  \C 24C1 24C1 24DB
  \C 24C2 24C2 24DC
  \C 24C3 24C3 24DD
  \C 24C4 24C4 24DE
  \C 24C5 24C5 24DF
  \C 24C6 24C6 24E0
  \C 24C7 24C7 24E1
  \C 24C8 24C8 24E2
  \C 24C9 24C9 24E3
  \C 24CA 24CA 24E4
  \C 24CB 24CB 24E5
  \C 24CC 24CC 24E6
  \C 24CD 24CD 24E7
  \C 24CE 24CE 24E8
  \C 24CF 24CF 24E9
  \C 24D0 24B6 24D0
  \C 24D1 24B7 24D1
  \C 24D2 24B8 24D2
  \C 24D3 24B9 24D3
  \C 24D4 24BA 24D4
  \C 24D5 24BB 24D5
  \C 24D6 24BC 24D6
  \C 24D7 24BD 24D7
  \C 24D8 24BE 24D8
  \C 24D9 24BF 24D9
  \C 24DA 24C0 24DA
  \C 24DB 24C1 24DB
  \C 24DC 24C2 24DC
  \C 24DD 24C3 24DD
  \C 24DE 24C4 24DE
  \C 24DF 24C5 24DF
  \C 24E0 24C6 24E0
  \C 24E1 24C7 24E1
  \C 24E2 24C8 24E2
  \C 24E3 24C9 24E3
  \C 24E4 24CA 24E4
  \C 24E5 24CB 24E5
  \C 24E6 24CC 24E6
  \C 24E7 24CD 24E7
  \C 24E8 24CE 24E8
  \C 24E9 24CF 24E9
  \L 2C00 2C00 2C30
  \L 2C01 2C01 2C31
  \L 2C02 2C02 2C32
  \L 2C03 2C03 2C33
  \L 2C04 2C04 2C34
  \L 2C05 2C05 2C35
  \L 2C06 2C06 2C36
  \L 2C07 2C07 2C37
  \L 2C08 2C08 2C38
  \L 2C09 2C09 2C39
  \L 2C0A 2C0A 2C3A
  \L 2C0B 2C0B 2C3B
  \L 2C0C 2C0C 2C3C
  \L 2C0D 2C0D 2C3D
  \L 2C0E 2C0E 2C3E
  \L 2C0F 2C0F 2C3F
  \L 2C10 2C10 2C40
  \L 2C11 2C11 2C41
  \L 2C12 2C12 2C42
  \L 2C13 2C13 2C43
  \L 2C14 2C14 2C44
  \L 2C15 2C15 2C45
  \L 2C16 2C16 2C46
  \L 2C17 2C17 2C47
  \L 2C18 2C18 2C48
  \L 2C19 2C19 2C49
  \L 2C1A 2C1A 2C4A
  \L 2C1B 2C1B 2C4B
  \L 2C1C 2C1C 2C4C
  \L 2C1D 2C1D 2C4D
  \L 2C1E 2C1E 2C4E
  \L 2C1F 2C1F 2C4F
  \L 2C20 2C20 2C50
  \L 2C21 2C21 2C51
  \L 2C22 2C22 2C52
  \L 2C23 2C23 2C53
  \L 2C24 2C24 2C54
  \L 2C25 2C25 2C55
  \L 2C26 2C26 2C56
  \L 2C27 2C27 2C57
  \L 2C28 2C28 2C58
  \L 2C29 2C29 2C59
  \L 2C2A 2C2A 2C5A
  \L 2C2B 2C2B 2C5B
  \L 2C2C 2C2C 2C5C
  \L 2C2D 2C2D 2C5D
  \L 2C2E 2C2E 2C5E
  \L 2C30 2C00 2C30
  \L 2C31 2C01 2C31
  \L 2C32 2C02 2C32
  \L 2C33 2C03 2C33
  \L 2C34 2C04 2C34
  \L 2C35 2C05 2C35
  \L 2C36 2C06 2C36
  \L 2C37 2C07 2C37
  \L 2C38 2C08 2C38
  \L 2C39 2C09 2C39
  \L 2C3A 2C0A 2C3A
  \L 2C3B 2C0B 2C3B
  \L 2C3C 2C0C 2C3C
  \L 2C3D 2C0D 2C3D
  \L 2C3E 2C0E 2C3E
  \L 2C3F 2C0F 2C3F
  \L 2C40 2C10 2C40
  \L 2C41 2C11 2C41
  \L 2C42 2C12 2C42
  \L 2C43 2C13 2C43
  \L 2C44 2C14 2C44
  \L 2C45 2C15 2C45
  \L 2C46 2C16 2C46
  \L 2C47 2C17 2C47
  \L 2C48 2C18 2C48
  \L 2C49 2C19 2C49
  \L 2C4A 2C1A 2C4A
  \L 2C4B 2C1B 2C4B
  \L 2C4C 2C1C 2C4C
  \L 2C4D 2C1D 2C4D
  \L 2C4E 2C1E 2C4E
  \L 2C4F 2C1F 2C4F
  \L 2C50 2C20 2C50
  \L 2C51 2C21 2C51
  \L 2C52 2C22 2C52
  \L 2C53 2C23 2C53
  \L 2C54 2C24 2C54
  \L 2C55 2C25 2C55
  \L 2C56 2C26 2C56
  \L 2C57 2C27 2C57
  \L 2C58 2C28 2C58
  \L 2C59 2C29 2C59
  \L 2C5A 2C2A 2C5A
  \L 2C5B 2C2B 2C5B
  \L 2C5C 2C2C 2C5C
  \L 2C5D 2C2D 2C5D
  \L 2C5E 2C2E 2C5E
  \L 2C60 2C60 2C61
  \L 2C61 2C60 2C61
  \L 2C62 2C62 026B
  \L 2C63 2C63 1D7D
  \L 2C64 2C64 027D
  \L 2C65 023A 2C65
  \L 2C66 023E 2C66
  \L 2C67 2C67 2C68
  \L 2C68 2C67 2C68
  \L 2C69 2C69 2C6A
  \L 2C6A 2C69 2C6A
  \L 2C6B 2C6B 2C6C
  \L 2C6C 2C6B 2C6C
  \L 2C6D 2C6D 0251
  \L 2C6E 2C6E 0271
  \L 2C6F 2C6F 0250
  \L 2C70 2C70 0252
  \l 2C71
  \L 2C72 2C72 2C73
  \L 2C73 2C72 2C73
  \l 2C74
  \L 2C75 2C75 2C76
  \L 2C76 2C75 2C76
  \l 2C77
  \l 2C78
  \l 2C79
  \l 2C7A
  \l 2C7B
  \l 2C7C
  \l 2C7D
  \L 2C7E 2C7E 023F
  \L 2C7F 2C7F 0240
  \L 2C80 2C80 2C81
  \L 2C81 2C80 2C81
  \L 2C82 2C82 2C83
  \L 2C83 2C82 2C83
  \L 2C84 2C84 2C85
  \L 2C85 2C84 2C85
  \L 2C86 2C86 2C87
  \L 2C87 2C86 2C87
  \L 2C88 2C88 2C89
  \L 2C89 2C88 2C89
  \L 2C8A 2C8A 2C8B
  \L 2C8B 2C8A 2C8B
  \L 2C8C 2C8C 2C8D
  \L 2C8D 2C8C 2C8D
  \L 2C8E 2C8E 2C8F
  \L 2C8F 2C8E 2C8F
  \L 2C90 2C90 2C91
  \L 2C91 2C90 2C91
  \L 2C92 2C92 2C93
  \L 2C93 2C92 2C93
  \L 2C94 2C94 2C95
  \L 2C95 2C94 2C95
  \L 2C96 2C96 2C97
  \L 2C97 2C96 2C97
  \L 2C98 2C98 2C99
  \L 2C99 2C98 2C99
  \L 2C9A 2C9A 2C9B
  \L 2C9B 2C9A 2C9B
  \L 2C9C 2C9C 2C9D
  \L 2C9D 2C9C 2C9D
  \L 2C9E 2C9E 2C9F
  \L 2C9F 2C9E 2C9F
  \L 2CA0 2CA0 2CA1
  \L 2CA1 2CA0 2CA1
  \L 2CA2 2CA2 2CA3
  \L 2CA3 2CA2 2CA3
  \L 2CA4 2CA4 2CA5
  \L 2CA5 2CA4 2CA5
  \L 2CA6 2CA6 2CA7
  \L 2CA7 2CA6 2CA7
  \L 2CA8 2CA8 2CA9
  \L 2CA9 2CA8 2CA9
  \L 2CAA 2CAA 2CAB
  \L 2CAB 2CAA 2CAB
  \L 2CAC 2CAC 2CAD
  \L 2CAD 2CAC 2CAD
  \L 2CAE 2CAE 2CAF
  \L 2CAF 2CAE 2CAF
  \L 2CB0 2CB0 2CB1
  \L 2CB1 2CB0 2CB1
  \L 2CB2 2CB2 2CB3
  \L 2CB3 2CB2 2CB3
  \L 2CB4 2CB4 2CB5
  \L 2CB5 2CB4 2CB5
  \L 2CB6 2CB6 2CB7
  \L 2CB7 2CB6 2CB7
  \L 2CB8 2CB8 2CB9
  \L 2CB9 2CB8 2CB9
  \L 2CBA 2CBA 2CBB
  \L 2CBB 2CBA 2CBB
  \L 2CBC 2CBC 2CBD
  \L 2CBD 2CBC 2CBD
  \L 2CBE 2CBE 2CBF
  \L 2CBF 2CBE 2CBF
  \L 2CC0 2CC0 2CC1
  \L 2CC1 2CC0 2CC1
  \L 2CC2 2CC2 2CC3
  \L 2CC3 2CC2 2CC3
  \L 2CC4 2CC4 2CC5
  \L 2CC5 2CC4 2CC5
  \L 2CC6 2CC6 2CC7
  \L 2CC7 2CC6 2CC7
  \L 2CC8 2CC8 2CC9
  \L 2CC9 2CC8 2CC9
  \L 2CCA 2CCA 2CCB
  \L 2CCB 2CCA 2CCB
  \L 2CCC 2CCC 2CCD
  \L 2CCD 2CCC 2CCD
  \L 2CCE 2CCE 2CCF
  \L 2CCF 2CCE 2CCF
  \L 2CD0 2CD0 2CD1
  \L 2CD1 2CD0 2CD1
  \L 2CD2 2CD2 2CD3
  \L 2CD3 2CD2 2CD3
  \L 2CD4 2CD4 2CD5
  \L 2CD5 2CD4 2CD5
  \L 2CD6 2CD6 2CD7
  \L 2CD7 2CD6 2CD7
  \L 2CD8 2CD8 2CD9
  \L 2CD9 2CD8 2CD9
  \L 2CDA 2CDA 2CDB
  \L 2CDB 2CDA 2CDB
  \L 2CDC 2CDC 2CDD
  \L 2CDD 2CDC 2CDD
  \L 2CDE 2CDE 2CDF
  \L 2CDF 2CDE 2CDF
  \L 2CE0 2CE0 2CE1
  \L 2CE1 2CE0 2CE1
  \L 2CE2 2CE2 2CE3
  \L 2CE3 2CE2 2CE3
  \l 2CE4
  \L 2CEB 2CEB 2CEC
  \L 2CEC 2CEB 2CEC
  \L 2CED 2CED 2CEE
  \L 2CEE 2CED 2CEE
  \l 2CEF
  \l 2CF0
  \l 2CF1
  \L 2CF2 2CF2 2CF3
  \L 2CF3 2CF2 2CF3
  \L 2D00 10A0 2D00
  \L 2D01 10A1 2D01
  \L 2D02 10A2 2D02
  \L 2D03 10A3 2D03
  \L 2D04 10A4 2D04
  \L 2D05 10A5 2D05
  \L 2D06 10A6 2D06
  \L 2D07 10A7 2D07
  \L 2D08 10A8 2D08
  \L 2D09 10A9 2D09
  \L 2D0A 10AA 2D0A
  \L 2D0B 10AB 2D0B
  \L 2D0C 10AC 2D0C
  \L 2D0D 10AD 2D0D
  \L 2D0E 10AE 2D0E
  \L 2D0F 10AF 2D0F
  \L 2D10 10B0 2D10
  \L 2D11 10B1 2D11
  \L 2D12 10B2 2D12
  \L 2D13 10B3 2D13
  \L 2D14 10B4 2D14
  \L 2D15 10B5 2D15
  \L 2D16 10B6 2D16
  \L 2D17 10B7 2D17
  \L 2D18 10B8 2D18
  \L 2D19 10B9 2D19
  \L 2D1A 10BA 2D1A
  \L 2D1B 10BB 2D1B
  \L 2D1C 10BC 2D1C
  \L 2D1D 10BD 2D1D
  \L 2D1E 10BE 2D1E
  \L 2D1F 10BF 2D1F
  \L 2D20 10C0 2D20
  \L 2D21 10C1 2D21
  \L 2D22 10C2 2D22
  \L 2D23 10C3 2D23
  \L 2D24 10C4 2D24
  \L 2D25 10C5 2D25
  \L 2D27 10C7 2D27
  \L 2D2D 10CD 2D2D
  \l 2D30
  \l 2D31
  \l 2D32
  \l 2D33
  \l 2D34
  \l 2D35
  \l 2D36
  \l 2D37
  \l 2D38
  \l 2D39
  \l 2D3A
  \l 2D3B
  \l 2D3C
  \l 2D3D
  \l 2D3E
  \l 2D3F
  \l 2D40
  \l 2D41
  \l 2D42
  \l 2D43
  \l 2D44
  \l 2D45
  \l 2D46
  \l 2D47
  \l 2D48
  \l 2D49
  \l 2D4A
  \l 2D4B
  \l 2D4C
  \l 2D4D
  \l 2D4E
  \l 2D4F
  \l 2D50
  \l 2D51
  \l 2D52
  \l 2D53
  \l 2D54
  \l 2D55
  \l 2D56
  \l 2D57
  \l 2D58
  \l 2D59
  \l 2D5A
  \l 2D5B
  \l 2D5C
  \l 2D5D
  \l 2D5E
  \l 2D5F
  \l 2D60
  \l 2D61
  \l 2D62
  \l 2D63
  \l 2D64
  \l 2D65
  \l 2D66
  \l 2D67
  \l 2D6F
  \l 2D7F
  \l 2D80
  \l 2D81
  \l 2D82
  \l 2D83
  \l 2D84
  \l 2D85
  \l 2D86
  \l 2D87
  \l 2D88
  \l 2D89
  \l 2D8A
  \l 2D8B
  \l 2D8C
  \l 2D8D
  \l 2D8E
  \l 2D8F
  \l 2D90
  \l 2D91
  \l 2D92
  \l 2D93
  \l 2D94
  \l 2D95
  \l 2D96
  \l 2DA0
  \l 2DA1
  \l 2DA2
  \l 2DA3
  \l 2DA4
  \l 2DA5
  \l 2DA6
  \l 2DA8
  \l 2DA9
  \l 2DAA
  \l 2DAB
  \l 2DAC
  \l 2DAD
  \l 2DAE
  \l 2DB0
  \l 2DB1
  \l 2DB2
  \l 2DB3
  \l 2DB4
  \l 2DB5
  \l 2DB6
  \l 2DB8
  \l 2DB9
  \l 2DBA
  \l 2DBB
  \l 2DBC
  \l 2DBD
  \l 2DBE
  \l 2DC0
  \l 2DC1
  \l 2DC2
  \l 2DC3
  \l 2DC4
  \l 2DC5
  \l 2DC6
  \l 2DC8
  \l 2DC9
  \l 2DCA
  \l 2DCB
  \l 2DCC
  \l 2DCD
  \l 2DCE
  \l 2DD0
  \l 2DD1
  \l 2DD2
  \l 2DD3
  \l 2DD4
  \l 2DD5
  \l 2DD6
  \l 2DD8
  \l 2DD9
  \l 2DDA
  \l 2DDB
  \l 2DDC
  \l 2DDD
  \l 2DDE
  \l 2DE0
  \l 2DE1
  \l 2DE2
  \l 2DE3
  \l 2DE4
  \l 2DE5
  \l 2DE6
  \l 2DE7
  \l 2DE8
  \l 2DE9
  \l 2DEA
  \l 2DEB
  \l 2DEC
  \l 2DED
  \l 2DEE
  \l 2DEF
  \l 2DF0
  \l 2DF1
  \l 2DF2
  \l 2DF3
  \l 2DF4
  \l 2DF5
  \l 2DF6
  \l 2DF7
  \l 2DF8
  \l 2DF9
  \l 2DFA
  \l 2DFB
  \l 2DFC
  \l 2DFD
  \l 2DFE
  \l 2DFF
  \l 2E2F
  \l 3005
  \l 3006
  \l 302A
  \l 302B
  \l 302C
  \l 302D
  \l 302E
  \l 302F
  \l 3031
  \l 3032
  \l 3033
  \l 3034
  \l 3035
  \l 303B
  \l 303C
  \l 3041
  \l 3042
  \l 3043
  \l 3044
  \l 3045
  \l 3046
  \l 3047
  \l 3048
  \l 3049
  \l 304A
  \l 304B
  \l 304C
  \l 304D
  \l 304E
  \l 304F
  \l 3050
  \l 3051
  \l 3052
  \l 3053
  \l 3054
  \l 3055
  \l 3056
  \l 3057
  \l 3058
  \l 3059
  \l 305A
  \l 305B
  \l 305C
  \l 305D
  \l 305E
  \l 305F
  \l 3060
  \l 3061
  \l 3062
  \l 3063
  \l 3064
  \l 3065
  \l 3066
  \l 3067
  \l 3068
  \l 3069
  \l 306A
  \l 306B
  \l 306C
  \l 306D
  \l 306E
  \l 306F
  \l 3070
  \l 3071
  \l 3072
  \l 3073
  \l 3074
  \l 3075
  \l 3076
  \l 3077
  \l 3078
  \l 3079
  \l 307A
  \l 307B
  \l 307C
  \l 307D
  \l 307E
  \l 307F
  \l 3080
  \l 3081
  \l 3082
  \l 3083
  \l 3084
  \l 3085
  \l 3086
  \l 3087
  \l 3088
  \l 3089
  \l 308A
  \l 308B
  \l 308C
  \l 308D
  \l 308E
  \l 308F
  \l 3090
  \l 3091
  \l 3092
  \l 3093
  \l 3094
  \l 3095
  \l 3096
  \l 3099
  \l 309A
  \l 309D
  \l 309E
  \l 309F
  \l 30A1
  \l 30A2
  \l 30A3
  \l 30A4
  \l 30A5
  \l 30A6
  \l 30A7
  \l 30A8
  \l 30A9
  \l 30AA
  \l 30AB
  \l 30AC
  \l 30AD
  \l 30AE
  \l 30AF
  \l 30B0
  \l 30B1
  \l 30B2
  \l 30B3
  \l 30B4
  \l 30B5
  \l 30B6
  \l 30B7
  \l 30B8
  \l 30B9
  \l 30BA
  \l 30BB
  \l 30BC
  \l 30BD
  \l 30BE
  \l 30BF
  \l 30C0
  \l 30C1
  \l 30C2
  \l 30C3
  \l 30C4
  \l 30C5
  \l 30C6
  \l 30C7
  \l 30C8
  \l 30C9
  \l 30CA
  \l 30CB
  \l 30CC
  \l 30CD
  \l 30CE
  \l 30CF
  \l 30D0
  \l 30D1
  \l 30D2
  \l 30D3
  \l 30D4
  \l 30D5
  \l 30D6
  \l 30D7
  \l 30D8
  \l 30D9
  \l 30DA
  \l 30DB
  \l 30DC
  \l 30DD
  \l 30DE
  \l 30DF
  \l 30E0
  \l 30E1
  \l 30E2
  \l 30E3
  \l 30E4
  \l 30E5
  \l 30E6
  \l 30E7
  \l 30E8
  \l 30E9
  \l 30EA
  \l 30EB
  \l 30EC
  \l 30ED
  \l 30EE
  \l 30EF
  \l 30F0
  \l 30F1
  \l 30F2
  \l 30F3
  \l 30F4
  \l 30F5
  \l 30F6
  \l 30F7
  \l 30F8
  \l 30F9
  \l 30FA
  \l 30FC
  \l 30FD
  \l 30FE
  \l 30FF
  \l 3105
  \l 3106
  \l 3107
  \l 3108
  \l 3109
  \l 310A
  \l 310B
  \l 310C
  \l 310D
  \l 310E
  \l 310F
  \l 3110
  \l 3111
  \l 3112
  \l 3113
  \l 3114
  \l 3115
  \l 3116
  \l 3117
  \l 3118
  \l 3119
  \l 311A
  \l 311B
  \l 311C
  \l 311D
  \l 311E
  \l 311F
  \l 3120
  \l 3121
  \l 3122
  \l 3123
  \l 3124
  \l 3125
  \l 3126
  \l 3127
  \l 3128
  \l 3129
  \l 312A
  \l 312B
  \l 312C
  \l 312D
  \l 3131
  \l 3132
  \l 3133
  \l 3134
  \l 3135
  \l 3136
  \l 3137
  \l 3138
  \l 3139
  \l 313A
  \l 313B
  \l 313C
  \l 313D
  \l 313E
  \l 313F
  \l 3140
  \l 3141
  \l 3142
  \l 3143
  \l 3144
  \l 3145
  \l 3146
  \l 3147
  \l 3148
  \l 3149
  \l 314A
  \l 314B
  \l 314C
  \l 314D
  \l 314E
  \l 314F
  \l 3150
  \l 3151
  \l 3152
  \l 3153
  \l 3154
  \l 3155
  \l 3156
  \l 3157
  \l 3158
  \l 3159
  \l 315A
  \l 315B
  \l 315C
  \l 315D
  \l 315E
  \l 315F
  \l 3160
  \l 3161
  \l 3162
  \l 3163
  \l 3164
  \l 3165
  \l 3166
  \l 3167
  \l 3168
  \l 3169
  \l 316A
  \l 316B
  \l 316C
  \l 316D
  \l 316E
  \l 316F
  \l 3170
  \l 3171
  \l 3172
  \l 3173
  \l 3174
  \l 3175
  \l 3176
  \l 3177
  \l 3178
  \l 3179
  \l 317A
  \l 317B
  \l 317C
  \l 317D
  \l 317E
  \l 317F
  \l 3180
  \l 3181
  \l 3182
  \l 3183
  \l 3184
  \l 3185
  \l 3186
  \l 3187
  \l 3188
  \l 3189
  \l 318A
  \l 318B
  \l 318C
  \l 318D
  \l 318E
  \l 31A0
  \l 31A1
  \l 31A2
  \l 31A3
  \l 31A4
  \l 31A5
  \l 31A6
  \l 31A7
  \l 31A8
  \l 31A9
  \l 31AA
  \l 31AB
  \l 31AC
  \l 31AD
  \l 31AE
  \l 31AF
  \l 31B0
  \l 31B1
  \l 31B2
  \l 31B3
  \l 31B4
  \l 31B5
  \l 31B6
  \l 31B7
  \l 31B8
  \l 31B9
  \l 31BA
  \l 31F0
  \l 31F1
  \l 31F2
  \l 31F3
  \l 31F4
  \l 31F5
  \l 31F6
  \l 31F7
  \l 31F8
  \l 31F9
  \l 31FA
  \l 31FB
  \l 31FC
  \l 31FD
  \l 31FE
  \l 31FF
  \l 3400
  \l 3401
  \l 3402
  \l 3403
  \l 3404
  \l 3405
  \l 3406
  \l 3407
  \l 3408
  \l 3409
  \l 340A
  \l 340B
  \l 340C
  \l 340D
  \l 340E
  \l 340F
  \l 3410
  \l 3411
  \l 3412
  \l 3413
  \l 3414
  \l 3415
  \l 3416
  \l 3417
  \l 3418
  \l 3419
  \l 341A
  \l 341B
  \l 341C
  \l 341D
  \l 341E
  \l 341F
  \l 3420
  \l 3421
  \l 3422
  \l 3423
  \l 3424
  \l 3425
  \l 3426
  \l 3427
  \l 3428
  \l 3429
  \l 342A
  \l 342B
  \l 342C
  \l 342D
  \l 342E
  \l 342F
  \l 3430
  \l 3431
  \l 3432
  \l 3433
  \l 3434
  \l 3435
  \l 3436
  \l 3437
  \l 3438
  \l 3439
  \l 343A
  \l 343B
  \l 343C
  \l 343D
  \l 343E
  \l 343F
  \l 3440
  \l 3441
  \l 3442
  \l 3443
  \l 3444
  \l 3445
  \l 3446
  \l 3447
  \l 3448
  \l 3449
  \l 344A
  \l 344B
  \l 344C
  \l 344D
  \l 344E
  \l 344F
  \l 3450
  \l 3451
  \l 3452
  \l 3453
  \l 3454
  \l 3455
  \l 3456
  \l 3457
  \l 3458
  \l 3459
  \l 345A
  \l 345B
  \l 345C
  \l 345D
  \l 345E
  \l 345F
  \l 3460
  \l 3461
  \l 3462
  \l 3463
  \l 3464
  \l 3465
  \l 3466
  \l 3467
  \l 3468
  \l 3469
  \l 346A
  \l 346B
  \l 346C
  \l 346D
  \l 346E
  \l 346F
  \l 3470
  \l 3471
  \l 3472
  \l 3473
  \l 3474
  \l 3475
  \l 3476
  \l 3477
  \l 3478
  \l 3479
  \l 347A
  \l 347B
  \l 347C
  \l 347D
  \l 347E
  \l 347F
  \l 3480
  \l 3481
  \l 3482
  \l 3483
  \l 3484
  \l 3485
  \l 3486
  \l 3487
  \l 3488
  \l 3489
  \l 348A
  \l 348B
  \l 348C
  \l 348D
  \l 348E
  \l 348F
  \l 3490
  \l 3491
  \l 3492
  \l 3493
  \l 3494
  \l 3495
  \l 3496
  \l 3497
  \l 3498
  \l 3499
  \l 349A
  \l 349B
  \l 349C
  \l 349D
  \l 349E
  \l 349F
  \l 34A0
  \l 34A1
  \l 34A2
  \l 34A3
  \l 34A4
  \l 34A5
  \l 34A6
  \l 34A7
  \l 34A8
  \l 34A9
  \l 34AA
  \l 34AB
  \l 34AC
  \l 34AD
  \l 34AE
  \l 34AF
  \l 34B0
  \l 34B1
  \l 34B2
  \l 34B3
  \l 34B4
  \l 34B5
  \l 34B6
  \l 34B7
  \l 34B8
  \l 34B9
  \l 34BA
  \l 34BB
  \l 34BC
  \l 34BD
  \l 34BE
  \l 34BF
  \l 34C0
  \l 34C1
  \l 34C2
  \l 34C3
  \l 34C4
  \l 34C5
  \l 34C6
  \l 34C7
  \l 34C8
  \l 34C9
  \l 34CA
  \l 34CB
  \l 34CC
  \l 34CD
  \l 34CE
  \l 34CF
  \l 34D0
  \l 34D1
  \l 34D2
  \l 34D3
  \l 34D4
  \l 34D5
  \l 34D6
  \l 34D7
  \l 34D8
  \l 34D9
  \l 34DA
  \l 34DB
  \l 34DC
  \l 34DD
  \l 34DE
  \l 34DF
  \l 34E0
  \l 34E1
  \l 34E2
  \l 34E3
  \l 34E4
  \l 34E5
  \l 34E6
  \l 34E7
  \l 34E8
  \l 34E9
  \l 34EA
  \l 34EB
  \l 34EC
  \l 34ED
  \l 34EE
  \l 34EF
  \l 34F0
  \l 34F1
  \l 34F2
  \l 34F3
  \l 34F4
  \l 34F5
  \l 34F6
  \l 34F7
  \l 34F8
  \l 34F9
  \l 34FA
  \l 34FB
  \l 34FC
  \l 34FD
  \l 34FE
  \l 34FF
  \l 3500
  \l 3501
  \l 3502
  \l 3503
  \l 3504
  \l 3505
  \l 3506
  \l 3507
  \l 3508
  \l 3509
  \l 350A
  \l 350B
  \l 350C
  \l 350D
  \l 350E
  \l 350F
  \l 3510
  \l 3511
  \l 3512
  \l 3513
  \l 3514
  \l 3515
  \l 3516
  \l 3517
  \l 3518
  \l 3519
  \l 351A
  \l 351B
  \l 351C
  \l 351D
  \l 351E
  \l 351F
  \l 3520
  \l 3521
  \l 3522
  \l 3523
  \l 3524
  \l 3525
  \l 3526
  \l 3527
  \l 3528
  \l 3529
  \l 352A
  \l 352B
  \l 352C
  \l 352D
  \l 352E
  \l 352F
  \l 3530
  \l 3531
  \l 3532
  \l 3533
  \l 3534
  \l 3535
  \l 3536
  \l 3537
  \l 3538
  \l 3539
  \l 353A
  \l 353B
  \l 353C
  \l 353D
  \l 353E
  \l 353F
  \l 3540
  \l 3541
  \l 3542
  \l 3543
  \l 3544
  \l 3545
  \l 3546
  \l 3547
  \l 3548
  \l 3549
  \l 354A
  \l 354B
  \l 354C
  \l 354D
  \l 354E
  \l 354F
  \l 3550
  \l 3551
  \l 3552
  \l 3553
  \l 3554
  \l 3555
  \l 3556
  \l 3557
  \l 3558
  \l 3559
  \l 355A
  \l 355B
  \l 355C
  \l 355D
  \l 355E
  \l 355F
  \l 3560
  \l 3561
  \l 3562
  \l 3563
  \l 3564
  \l 3565
  \l 3566
  \l 3567
  \l 3568
  \l 3569
  \l 356A
  \l 356B
  \l 356C
  \l 356D
  \l 356E
  \l 356F
  \l 3570
  \l 3571
  \l 3572
  \l 3573
  \l 3574
  \l 3575
  \l 3576
  \l 3577
  \l 3578
  \l 3579
  \l 357A
  \l 357B
  \l 357C
  \l 357D
  \l 357E
  \l 357F
  \l 3580
  \l 3581
  \l 3582
  \l 3583
  \l 3584
  \l 3585
  \l 3586
  \l 3587
  \l 3588
  \l 3589
  \l 358A
  \l 358B
  \l 358C
  \l 358D
  \l 358E
  \l 358F
  \l 3590
  \l 3591
  \l 3592
  \l 3593
  \l 3594
  \l 3595
  \l 3596
  \l 3597
  \l 3598
  \l 3599
  \l 359A
  \l 359B
  \l 359C
  \l 359D
  \l 359E
  \l 359F
  \l 35A0
  \l 35A1
  \l 35A2
  \l 35A3
  \l 35A4
  \l 35A5
  \l 35A6
  \l 35A7
  \l 35A8
  \l 35A9
  \l 35AA
  \l 35AB
  \l 35AC
  \l 35AD
  \l 35AE
  \l 35AF
  \l 35B0
  \l 35B1
  \l 35B2
  \l 35B3
  \l 35B4
  \l 35B5
  \l 35B6
  \l 35B7
  \l 35B8
  \l 35B9
  \l 35BA
  \l 35BB
  \l 35BC
  \l 35BD
  \l 35BE
  \l 35BF
  \l 35C0
  \l 35C1
  \l 35C2
  \l 35C3
  \l 35C4
  \l 35C5
  \l 35C6
  \l 35C7
  \l 35C8
  \l 35C9
  \l 35CA
  \l 35CB
  \l 35CC
  \l 35CD
  \l 35CE
  \l 35CF
  \l 35D0
  \l 35D1
  \l 35D2
  \l 35D3
  \l 35D4
  \l 35D5
  \l 35D6
  \l 35D7
  \l 35D8
  \l 35D9
  \l 35DA
  \l 35DB
  \l 35DC
  \l 35DD
  \l 35DE
  \l 35DF
  \l 35E0
  \l 35E1
  \l 35E2
  \l 35E3
  \l 35E4
  \l 35E5
  \l 35E6
  \l 35E7
  \l 35E8
  \l 35E9
  \l 35EA
  \l 35EB
  \l 35EC
  \l 35ED
  \l 35EE
  \l 35EF
  \l 35F0
  \l 35F1
  \l 35F2
  \l 35F3
  \l 35F4
  \l 35F5
  \l 35F6
  \l 35F7
  \l 35F8
  \l 35F9
  \l 35FA
  \l 35FB
  \l 35FC
  \l 35FD
  \l 35FE
  \l 35FF
  \l 3600
  \l 3601
  \l 3602
  \l 3603
  \l 3604
  \l 3605
  \l 3606
  \l 3607
  \l 3608
  \l 3609
  \l 360A
  \l 360B
  \l 360C
  \l 360D
  \l 360E
  \l 360F
  \l 3610
  \l 3611
  \l 3612
  \l 3613
  \l 3614
  \l 3615
  \l 3616
  \l 3617
  \l 3618
  \l 3619
  \l 361A
  \l 361B
  \l 361C
  \l 361D
  \l 361E
  \l 361F
  \l 3620
  \l 3621
  \l 3622
  \l 3623
  \l 3624
  \l 3625
  \l 3626
  \l 3627
  \l 3628
  \l 3629
  \l 362A
  \l 362B
  \l 362C
  \l 362D
  \l 362E
  \l 362F
  \l 3630
  \l 3631
  \l 3632
  \l 3633
  \l 3634
  \l 3635
  \l 3636
  \l 3637
  \l 3638
  \l 3639
  \l 363A
  \l 363B
  \l 363C
  \l 363D
  \l 363E
  \l 363F
  \l 3640
  \l 3641
  \l 3642
  \l 3643
  \l 3644
  \l 3645
  \l 3646
  \l 3647
  \l 3648
  \l 3649
  \l 364A
  \l 364B
  \l 364C
  \l 364D
  \l 364E
  \l 364F
  \l 3650
  \l 3651
  \l 3652
  \l 3653
  \l 3654
  \l 3655
  \l 3656
  \l 3657
  \l 3658
  \l 3659
  \l 365A
  \l 365B
  \l 365C
  \l 365D
  \l 365E
  \l 365F
  \l 3660
  \l 3661
  \l 3662
  \l 3663
  \l 3664
  \l 3665
  \l 3666
  \l 3667
  \l 3668
  \l 3669
  \l 366A
  \l 366B
  \l 366C
  \l 366D
  \l 366E
  \l 366F
  \l 3670
  \l 3671
  \l 3672
  \l 3673
  \l 3674
  \l 3675
  \l 3676
  \l 3677
  \l 3678
  \l 3679
  \l 367A
  \l 367B
  \l 367C
  \l 367D
  \l 367E
  \l 367F
  \l 3680
  \l 3681
  \l 3682
  \l 3683
  \l 3684
  \l 3685
  \l 3686
  \l 3687
  \l 3688
  \l 3689
  \l 368A
  \l 368B
  \l 368C
  \l 368D
  \l 368E
  \l 368F
  \l 3690
  \l 3691
  \l 3692
  \l 3693
  \l 3694
  \l 3695
  \l 3696
  \l 3697
  \l 3698
  \l 3699
  \l 369A
  \l 369B
  \l 369C
  \l 369D
  \l 369E
  \l 369F
  \l 36A0
  \l 36A1
  \l 36A2
  \l 36A3
  \l 36A4
  \l 36A5
  \l 36A6
  \l 36A7
  \l 36A8
  \l 36A9
  \l 36AA
  \l 36AB
  \l 36AC
  \l 36AD
  \l 36AE
  \l 36AF
  \l 36B0
  \l 36B1
  \l 36B2
  \l 36B3
  \l 36B4
  \l 36B5
  \l 36B6
  \l 36B7
  \l 36B8
  \l 36B9
  \l 36BA
  \l 36BB
  \l 36BC
  \l 36BD
  \l 36BE
  \l 36BF
  \l 36C0
  \l 36C1
  \l 36C2
  \l 36C3
  \l 36C4
  \l 36C5
  \l 36C6
  \l 36C7
  \l 36C8
  \l 36C9
  \l 36CA
  \l 36CB
  \l 36CC
  \l 36CD
  \l 36CE
  \l 36CF
  \l 36D0
  \l 36D1
  \l 36D2
  \l 36D3
  \l 36D4
  \l 36D5
  \l 36D6
  \l 36D7
  \l 36D8
  \l 36D9
  \l 36DA
  \l 36DB
  \l 36DC
  \l 36DD
  \l 36DE
  \l 36DF
  \l 36E0
  \l 36E1
  \l 36E2
  \l 36E3
  \l 36E4
  \l 36E5
  \l 36E6
  \l 36E7
  \l 36E8
  \l 36E9
  \l 36EA
  \l 36EB
  \l 36EC
  \l 36ED
  \l 36EE
  \l 36EF
  \l 36F0
  \l 36F1
  \l 36F2
  \l 36F3
  \l 36F4
  \l 36F5
  \l 36F6
  \l 36F7
  \l 36F8
  \l 36F9
  \l 36FA
  \l 36FB
  \l 36FC
  \l 36FD
  \l 36FE
  \l 36FF
  \l 3700
  \l 3701
  \l 3702
  \l 3703
  \l 3704
  \l 3705
  \l 3706
  \l 3707
  \l 3708
  \l 3709
  \l 370A
  \l 370B
  \l 370C
  \l 370D
  \l 370E
  \l 370F
  \l 3710
  \l 3711
  \l 3712
  \l 3713
  \l 3714
  \l 3715
  \l 3716
  \l 3717
  \l 3718
  \l 3719
  \l 371A
  \l 371B
  \l 371C
  \l 371D
  \l 371E
  \l 371F
  \l 3720
  \l 3721
  \l 3722
  \l 3723
  \l 3724
  \l 3725
  \l 3726
  \l 3727
  \l 3728
  \l 3729
  \l 372A
  \l 372B
  \l 372C
  \l 372D
  \l 372E
  \l 372F
  \l 3730
  \l 3731
  \l 3732
  \l 3733
  \l 3734
  \l 3735
  \l 3736
  \l 3737
  \l 3738
  \l 3739
  \l 373A
  \l 373B
  \l 373C
  \l 373D
  \l 373E
  \l 373F
  \l 3740
  \l 3741
  \l 3742
  \l 3743
  \l 3744
  \l 3745
  \l 3746
  \l 3747
  \l 3748
  \l 3749
  \l 374A
  \l 374B
  \l 374C
  \l 374D
  \l 374E
  \l 374F
  \l 3750
  \l 3751
  \l 3752
  \l 3753
  \l 3754
  \l 3755
  \l 3756
  \l 3757
  \l 3758
  \l 3759
  \l 375A
  \l 375B
  \l 375C
  \l 375D
  \l 375E
  \l 375F
  \l 3760
  \l 3761
  \l 3762
  \l 3763
  \l 3764
  \l 3765
  \l 3766
  \l 3767
  \l 3768
  \l 3769
  \l 376A
  \l 376B
  \l 376C
  \l 376D
  \l 376E
  \l 376F
  \l 3770
  \l 3771
  \l 3772
  \l 3773
  \l 3774
  \l 3775
  \l 3776
  \l 3777
  \l 3778
  \l 3779
  \l 377A
  \l 377B
  \l 377C
  \l 377D
  \l 377E
  \l 377F
  \l 3780
  \l 3781
  \l 3782
  \l 3783
  \l 3784
  \l 3785
  \l 3786
  \l 3787
  \l 3788
  \l 3789
  \l 378A
  \l 378B
  \l 378C
  \l 378D
  \l 378E
  \l 378F
  \l 3790
  \l 3791
  \l 3792
  \l 3793
  \l 3794
  \l 3795
  \l 3796
  \l 3797
  \l 3798
  \l 3799
  \l 379A
  \l 379B
  \l 379C
  \l 379D
  \l 379E
  \l 379F
  \l 37A0
  \l 37A1
  \l 37A2
  \l 37A3
  \l 37A4
  \l 37A5
  \l 37A6
  \l 37A7
  \l 37A8
  \l 37A9
  \l 37AA
  \l 37AB
  \l 37AC
  \l 37AD
  \l 37AE
  \l 37AF
  \l 37B0
  \l 37B1
  \l 37B2
  \l 37B3
  \l 37B4
  \l 37B5
  \l 37B6
  \l 37B7
  \l 37B8
  \l 37B9
  \l 37BA
  \l 37BB
  \l 37BC
  \l 37BD
  \l 37BE
  \l 37BF
  \l 37C0
  \l 37C1
  \l 37C2
  \l 37C3
  \l 37C4
  \l 37C5
  \l 37C6
  \l 37C7
  \l 37C8
  \l 37C9
  \l 37CA
  \l 37CB
  \l 37CC
  \l 37CD
  \l 37CE
  \l 37CF
  \l 37D0
  \l 37D1
  \l 37D2
  \l 37D3
  \l 37D4
  \l 37D5
  \l 37D6
  \l 37D7
  \l 37D8
  \l 37D9
  \l 37DA
  \l 37DB
  \l 37DC
  \l 37DD
  \l 37DE
  \l 37DF
  \l 37E0
  \l 37E1
  \l 37E2
  \l 37E3
  \l 37E4
  \l 37E5
  \l 37E6
  \l 37E7
  \l 37E8
  \l 37E9
  \l 37EA
  \l 37EB
  \l 37EC
  \l 37ED
  \l 37EE
  \l 37EF
  \l 37F0
  \l 37F1
  \l 37F2
  \l 37F3
  \l 37F4
  \l 37F5
  \l 37F6
  \l 37F7
  \l 37F8
  \l 37F9
  \l 37FA
  \l 37FB
  \l 37FC
  \l 37FD
  \l 37FE
  \l 37FF
  \l 3800
  \l 3801
  \l 3802
  \l 3803
  \l 3804
  \l 3805
  \l 3806
  \l 3807
  \l 3808
  \l 3809
  \l 380A
  \l 380B
  \l 380C
  \l 380D
  \l 380E
  \l 380F
  \l 3810
  \l 3811
  \l 3812
  \l 3813
  \l 3814
  \l 3815
  \l 3816
  \l 3817
  \l 3818
  \l 3819
  \l 381A
  \l 381B
  \l 381C
  \l 381D
  \l 381E
  \l 381F
  \l 3820
  \l 3821
  \l 3822
  \l 3823
  \l 3824
  \l 3825
  \l 3826
  \l 3827
  \l 3828
  \l 3829
  \l 382A
  \l 382B
  \l 382C
  \l 382D
  \l 382E
  \l 382F
  \l 3830
  \l 3831
  \l 3832
  \l 3833
  \l 3834
  \l 3835
  \l 3836
  \l 3837
  \l 3838
  \l 3839
  \l 383A
  \l 383B
  \l 383C
  \l 383D
  \l 383E
  \l 383F
  \l 3840
  \l 3841
  \l 3842
  \l 3843
  \l 3844
  \l 3845
  \l 3846
  \l 3847
  \l 3848
  \l 3849
  \l 384A
  \l 384B
  \l 384C
  \l 384D
  \l 384E
  \l 384F
  \l 3850
  \l 3851
  \l 3852
  \l 3853
  \l 3854
  \l 3855
  \l 3856
  \l 3857
  \l 3858
  \l 3859
  \l 385A
  \l 385B
  \l 385C
  \l 385D
  \l 385E
  \l 385F
  \l 3860
  \l 3861
  \l 3862
  \l 3863
  \l 3864
  \l 3865
  \l 3866
  \l 3867
  \l 3868
  \l 3869
  \l 386A
  \l 386B
  \l 386C
  \l 386D
  \l 386E
  \l 386F
  \l 3870
  \l 3871
  \l 3872
  \l 3873
  \l 3874
  \l 3875
  \l 3876
  \l 3877
  \l 3878
  \l 3879
  \l 387A
  \l 387B
  \l 387C
  \l 387D
  \l 387E
  \l 387F
  \l 3880
  \l 3881
  \l 3882
  \l 3883
  \l 3884
  \l 3885
  \l 3886
  \l 3887
  \l 3888
  \l 3889
  \l 388A
  \l 388B
  \l 388C
  \l 388D
  \l 388E
  \l 388F
  \l 3890
  \l 3891
  \l 3892
  \l 3893
  \l 3894
  \l 3895
  \l 3896
  \l 3897
  \l 3898
  \l 3899
  \l 389A
  \l 389B
  \l 389C
  \l 389D
  \l 389E
  \l 389F
  \l 38A0
  \l 38A1
  \l 38A2
  \l 38A3
  \l 38A4
  \l 38A5
  \l 38A6
  \l 38A7
  \l 38A8
  \l 38A9
  \l 38AA
  \l 38AB
  \l 38AC
  \l 38AD
  \l 38AE
  \l 38AF
  \l 38B0
  \l 38B1
  \l 38B2
  \l 38B3
  \l 38B4
  \l 38B5
  \l 38B6
  \l 38B7
  \l 38B8
  \l 38B9
  \l 38BA
  \l 38BB
  \l 38BC
  \l 38BD
  \l 38BE
  \l 38BF
  \l 38C0
  \l 38C1
  \l 38C2
  \l 38C3
  \l 38C4
  \l 38C5
  \l 38C6
  \l 38C7
  \l 38C8
  \l 38C9
  \l 38CA
  \l 38CB
  \l 38CC
  \l 38CD
  \l 38CE
  \l 38CF
  \l 38D0
  \l 38D1
  \l 38D2
  \l 38D3
  \l 38D4
  \l 38D5
  \l 38D6
  \l 38D7
  \l 38D8
  \l 38D9
  \l 38DA
  \l 38DB
  \l 38DC
  \l 38DD
  \l 38DE
  \l 38DF
  \l 38E0
  \l 38E1
  \l 38E2
  \l 38E3
  \l 38E4
  \l 38E5
  \l 38E6
  \l 38E7
  \l 38E8
  \l 38E9
  \l 38EA
  \l 38EB
  \l 38EC
  \l 38ED
  \l 38EE
  \l 38EF
  \l 38F0
  \l 38F1
  \l 38F2
  \l 38F3
  \l 38F4
  \l 38F5
  \l 38F6
  \l 38F7
  \l 38F8
  \l 38F9
  \l 38FA
  \l 38FB
  \l 38FC
  \l 38FD
  \l 38FE
  \l 38FF
  \l 3900
  \l 3901
  \l 3902
  \l 3903
  \l 3904
  \l 3905
  \l 3906
  \l 3907
  \l 3908
  \l 3909
  \l 390A
  \l 390B
  \l 390C
  \l 390D
  \l 390E
  \l 390F
  \l 3910
  \l 3911
  \l 3912
  \l 3913
  \l 3914
  \l 3915
  \l 3916
  \l 3917
  \l 3918
  \l 3919
  \l 391A
  \l 391B
  \l 391C
  \l 391D
  \l 391E
  \l 391F
  \l 3920
  \l 3921
  \l 3922
  \l 3923
  \l 3924
  \l 3925
  \l 3926
  \l 3927
  \l 3928
  \l 3929
  \l 392A
  \l 392B
  \l 392C
  \l 392D
  \l 392E
  \l 392F
  \l 3930
  \l 3931
  \l 3932
  \l 3933
  \l 3934
  \l 3935
  \l 3936
  \l 3937
  \l 3938
  \l 3939
  \l 393A
  \l 393B
  \l 393C
  \l 393D
  \l 393E
  \l 393F
  \l 3940
  \l 3941
  \l 3942
  \l 3943
  \l 3944
  \l 3945
  \l 3946
  \l 3947
  \l 3948
  \l 3949
  \l 394A
  \l 394B
  \l 394C
  \l 394D
  \l 394E
  \l 394F
  \l 3950
  \l 3951
  \l 3952
  \l 3953
  \l 3954
  \l 3955
  \l 3956
  \l 3957
  \l 3958
  \l 3959
  \l 395A
  \l 395B
  \l 395C
  \l 395D
  \l 395E
  \l 395F
  \l 3960
  \l 3961
  \l 3962
  \l 3963
  \l 3964
  \l 3965
  \l 3966
  \l 3967
  \l 3968
  \l 3969
  \l 396A
  \l 396B
  \l 396C
  \l 396D
  \l 396E
  \l 396F
  \l 3970
  \l 3971
  \l 3972
  \l 3973
  \l 3974
  \l 3975
  \l 3976
  \l 3977
  \l 3978
  \l 3979
  \l 397A
  \l 397B
  \l 397C
  \l 397D
  \l 397E
  \l 397F
  \l 3980
  \l 3981
  \l 3982
  \l 3983
  \l 3984
  \l 3985
  \l 3986
  \l 3987
  \l 3988
  \l 3989
  \l 398A
  \l 398B
  \l 398C
  \l 398D
  \l 398E
  \l 398F
  \l 3990
  \l 3991
  \l 3992
  \l 3993
  \l 3994
  \l 3995
  \l 3996
  \l 3997
  \l 3998
  \l 3999
  \l 399A
  \l 399B
  \l 399C
  \l 399D
  \l 399E
  \l 399F
  \l 39A0
  \l 39A1
  \l 39A2
  \l 39A3
  \l 39A4
  \l 39A5
  \l 39A6
  \l 39A7
  \l 39A8
  \l 39A9
  \l 39AA
  \l 39AB
  \l 39AC
  \l 39AD
  \l 39AE
  \l 39AF
  \l 39B0
  \l 39B1
  \l 39B2
  \l 39B3
  \l 39B4
  \l 39B5
  \l 39B6
  \l 39B7
  \l 39B8
  \l 39B9
  \l 39BA
  \l 39BB
  \l 39BC
  \l 39BD
  \l 39BE
  \l 39BF
  \l 39C0
  \l 39C1
  \l 39C2
  \l 39C3
  \l 39C4
  \l 39C5
  \l 39C6
  \l 39C7
  \l 39C8
  \l 39C9
  \l 39CA
  \l 39CB
  \l 39CC
  \l 39CD
  \l 39CE
  \l 39CF
  \l 39D0
  \l 39D1
  \l 39D2
  \l 39D3
  \l 39D4
  \l 39D5
  \l 39D6
  \l 39D7
  \l 39D8
  \l 39D9
  \l 39DA
  \l 39DB
  \l 39DC
  \l 39DD
  \l 39DE
  \l 39DF
  \l 39E0
  \l 39E1
  \l 39E2
  \l 39E3
  \l 39E4
  \l 39E5
  \l 39E6
  \l 39E7
  \l 39E8
  \l 39E9
  \l 39EA
  \l 39EB
  \l 39EC
  \l 39ED
  \l 39EE
  \l 39EF
  \l 39F0
  \l 39F1
  \l 39F2
  \l 39F3
  \l 39F4
  \l 39F5
  \l 39F6
  \l 39F7
  \l 39F8
  \l 39F9
  \l 39FA
  \l 39FB
  \l 39FC
  \l 39FD
  \l 39FE
  \l 39FF
  \l 3A00
  \l 3A01
  \l 3A02
  \l 3A03
  \l 3A04
  \l 3A05
  \l 3A06
  \l 3A07
  \l 3A08
  \l 3A09
  \l 3A0A
  \l 3A0B
  \l 3A0C
  \l 3A0D
  \l 3A0E
  \l 3A0F
  \l 3A10
  \l 3A11
  \l 3A12
  \l 3A13
  \l 3A14
  \l 3A15
  \l 3A16
  \l 3A17
  \l 3A18
  \l 3A19
  \l 3A1A
  \l 3A1B
  \l 3A1C
  \l 3A1D
  \l 3A1E
  \l 3A1F
  \l 3A20
  \l 3A21
  \l 3A22
  \l 3A23
  \l 3A24
  \l 3A25
  \l 3A26
  \l 3A27
  \l 3A28
  \l 3A29
  \l 3A2A
  \l 3A2B
  \l 3A2C
  \l 3A2D
  \l 3A2E
  \l 3A2F
  \l 3A30
  \l 3A31
  \l 3A32
  \l 3A33
  \l 3A34
  \l 3A35
  \l 3A36
  \l 3A37
  \l 3A38
  \l 3A39
  \l 3A3A
  \l 3A3B
  \l 3A3C
  \l 3A3D
  \l 3A3E
  \l 3A3F
  \l 3A40
  \l 3A41
  \l 3A42
  \l 3A43
  \l 3A44
  \l 3A45
  \l 3A46
  \l 3A47
  \l 3A48
  \l 3A49
  \l 3A4A
  \l 3A4B
  \l 3A4C
  \l 3A4D
  \l 3A4E
  \l 3A4F
  \l 3A50
  \l 3A51
  \l 3A52
  \l 3A53
  \l 3A54
  \l 3A55
  \l 3A56
  \l 3A57
  \l 3A58
  \l 3A59
  \l 3A5A
  \l 3A5B
  \l 3A5C
  \l 3A5D
  \l 3A5E
  \l 3A5F
  \l 3A60
  \l 3A61
  \l 3A62
  \l 3A63
  \l 3A64
  \l 3A65
  \l 3A66
  \l 3A67
  \l 3A68
  \l 3A69
  \l 3A6A
  \l 3A6B
  \l 3A6C
  \l 3A6D
  \l 3A6E
  \l 3A6F
  \l 3A70
  \l 3A71
  \l 3A72
  \l 3A73
  \l 3A74
  \l 3A75
  \l 3A76
  \l 3A77
  \l 3A78
  \l 3A79
  \l 3A7A
  \l 3A7B
  \l 3A7C
  \l 3A7D
  \l 3A7E
  \l 3A7F
  \l 3A80
  \l 3A81
  \l 3A82
  \l 3A83
  \l 3A84
  \l 3A85
  \l 3A86
  \l 3A87
  \l 3A88
  \l 3A89
  \l 3A8A
  \l 3A8B
  \l 3A8C
  \l 3A8D
  \l 3A8E
  \l 3A8F
  \l 3A90
  \l 3A91
  \l 3A92
  \l 3A93
  \l 3A94
  \l 3A95
  \l 3A96
  \l 3A97
  \l 3A98
  \l 3A99
  \l 3A9A
  \l 3A9B
  \l 3A9C
  \l 3A9D
  \l 3A9E
  \l 3A9F
  \l 3AA0
  \l 3AA1
  \l 3AA2
  \l 3AA3
  \l 3AA4
  \l 3AA5
  \l 3AA6
  \l 3AA7
  \l 3AA8
  \l 3AA9
  \l 3AAA
  \l 3AAB
  \l 3AAC
  \l 3AAD
  \l 3AAE
  \l 3AAF
  \l 3AB0
  \l 3AB1
  \l 3AB2
  \l 3AB3
  \l 3AB4
  \l 3AB5
  \l 3AB6
  \l 3AB7
  \l 3AB8
  \l 3AB9
  \l 3ABA
  \l 3ABB
  \l 3ABC
  \l 3ABD
  \l 3ABE
  \l 3ABF
  \l 3AC0
  \l 3AC1
  \l 3AC2
  \l 3AC3
  \l 3AC4
  \l 3AC5
  \l 3AC6
  \l 3AC7
  \l 3AC8
  \l 3AC9
  \l 3ACA
  \l 3ACB
  \l 3ACC
  \l 3ACD
  \l 3ACE
  \l 3ACF
  \l 3AD0
  \l 3AD1
  \l 3AD2
  \l 3AD3
  \l 3AD4
  \l 3AD5
  \l 3AD6
  \l 3AD7
  \l 3AD8
  \l 3AD9
  \l 3ADA
  \l 3ADB
  \l 3ADC
  \l 3ADD
  \l 3ADE
  \l 3ADF
  \l 3AE0
  \l 3AE1
  \l 3AE2
  \l 3AE3
  \l 3AE4
  \l 3AE5
  \l 3AE6
  \l 3AE7
  \l 3AE8
  \l 3AE9
  \l 3AEA
  \l 3AEB
  \l 3AEC
  \l 3AED
  \l 3AEE
  \l 3AEF
  \l 3AF0
  \l 3AF1
  \l 3AF2
  \l 3AF3
  \l 3AF4
  \l 3AF5
  \l 3AF6
  \l 3AF7
  \l 3AF8
  \l 3AF9
  \l 3AFA
  \l 3AFB
  \l 3AFC
  \l 3AFD
  \l 3AFE
  \l 3AFF
  \l 3B00
  \l 3B01
  \l 3B02
  \l 3B03
  \l 3B04
  \l 3B05
  \l 3B06
  \l 3B07
  \l 3B08
  \l 3B09
  \l 3B0A
  \l 3B0B
  \l 3B0C
  \l 3B0D
  \l 3B0E
  \l 3B0F
  \l 3B10
  \l 3B11
  \l 3B12
  \l 3B13
  \l 3B14
  \l 3B15
  \l 3B16
  \l 3B17
  \l 3B18
  \l 3B19
  \l 3B1A
  \l 3B1B
  \l 3B1C
  \l 3B1D
  \l 3B1E
  \l 3B1F
  \l 3B20
  \l 3B21
  \l 3B22
  \l 3B23
  \l 3B24
  \l 3B25
  \l 3B26
  \l 3B27
  \l 3B28
  \l 3B29
  \l 3B2A
  \l 3B2B
  \l 3B2C
  \l 3B2D
  \l 3B2E
  \l 3B2F
  \l 3B30
  \l 3B31
  \l 3B32
  \l 3B33
  \l 3B34
  \l 3B35
  \l 3B36
  \l 3B37
  \l 3B38
  \l 3B39
  \l 3B3A
  \l 3B3B
  \l 3B3C
  \l 3B3D
  \l 3B3E
  \l 3B3F
  \l 3B40
  \l 3B41
  \l 3B42
  \l 3B43
  \l 3B44
  \l 3B45
  \l 3B46
  \l 3B47
  \l 3B48
  \l 3B49
  \l 3B4A
  \l 3B4B
  \l 3B4C
  \l 3B4D
  \l 3B4E
  \l 3B4F
  \l 3B50
  \l 3B51
  \l 3B52
  \l 3B53
  \l 3B54
  \l 3B55
  \l 3B56
  \l 3B57
  \l 3B58
  \l 3B59
  \l 3B5A
  \l 3B5B
  \l 3B5C
  \l 3B5D
  \l 3B5E
  \l 3B5F
  \l 3B60
  \l 3B61
  \l 3B62
  \l 3B63
  \l 3B64
  \l 3B65
  \l 3B66
  \l 3B67
  \l 3B68
  \l 3B69
  \l 3B6A
  \l 3B6B
  \l 3B6C
  \l 3B6D
  \l 3B6E
  \l 3B6F
  \l 3B70
  \l 3B71
  \l 3B72
  \l 3B73
  \l 3B74
  \l 3B75
  \l 3B76
  \l 3B77
  \l 3B78
  \l 3B79
  \l 3B7A
  \l 3B7B
  \l 3B7C
  \l 3B7D
  \l 3B7E
  \l 3B7F
  \l 3B80
  \l 3B81
  \l 3B82
  \l 3B83
  \l 3B84
  \l 3B85
  \l 3B86
  \l 3B87
  \l 3B88
  \l 3B89
  \l 3B8A
  \l 3B8B
  \l 3B8C
  \l 3B8D
  \l 3B8E
  \l 3B8F
  \l 3B90
  \l 3B91
  \l 3B92
  \l 3B93
  \l 3B94
  \l 3B95
  \l 3B96
  \l 3B97
  \l 3B98
  \l 3B99
  \l 3B9A
  \l 3B9B
  \l 3B9C
  \l 3B9D
  \l 3B9E
  \l 3B9F
  \l 3BA0
  \l 3BA1
  \l 3BA2
  \l 3BA3
  \l 3BA4
  \l 3BA5
  \l 3BA6
  \l 3BA7
  \l 3BA8
  \l 3BA9
  \l 3BAA
  \l 3BAB
  \l 3BAC
  \l 3BAD
  \l 3BAE
  \l 3BAF
  \l 3BB0
  \l 3BB1
  \l 3BB2
  \l 3BB3
  \l 3BB4
  \l 3BB5
  \l 3BB6
  \l 3BB7
  \l 3BB8
  \l 3BB9
  \l 3BBA
  \l 3BBB
  \l 3BBC
  \l 3BBD
  \l 3BBE
  \l 3BBF
  \l 3BC0
  \l 3BC1
  \l 3BC2
  \l 3BC3
  \l 3BC4
  \l 3BC5
  \l 3BC6
  \l 3BC7
  \l 3BC8
  \l 3BC9
  \l 3BCA
  \l 3BCB
  \l 3BCC
  \l 3BCD
  \l 3BCE
  \l 3BCF
  \l 3BD0
  \l 3BD1
  \l 3BD2
  \l 3BD3
  \l 3BD4
  \l 3BD5
  \l 3BD6
  \l 3BD7
  \l 3BD8
  \l 3BD9
  \l 3BDA
  \l 3BDB
  \l 3BDC
  \l 3BDD
  \l 3BDE
  \l 3BDF
  \l 3BE0
  \l 3BE1
  \l 3BE2
  \l 3BE3
  \l 3BE4
  \l 3BE5
  \l 3BE6
  \l 3BE7
  \l 3BE8
  \l 3BE9
  \l 3BEA
  \l 3BEB
  \l 3BEC
  \l 3BED
  \l 3BEE
  \l 3BEF
  \l 3BF0
  \l 3BF1
  \l 3BF2
  \l 3BF3
  \l 3BF4
  \l 3BF5
  \l 3BF6
  \l 3BF7
  \l 3BF8
  \l 3BF9
  \l 3BFA
  \l 3BFB
  \l 3BFC
  \l 3BFD
  \l 3BFE
  \l 3BFF
  \l 3C00
  \l 3C01
  \l 3C02
  \l 3C03
  \l 3C04
  \l 3C05
  \l 3C06
  \l 3C07
  \l 3C08
  \l 3C09
  \l 3C0A
  \l 3C0B
  \l 3C0C
  \l 3C0D
  \l 3C0E
  \l 3C0F
  \l 3C10
  \l 3C11
  \l 3C12
  \l 3C13
  \l 3C14
  \l 3C15
  \l 3C16
  \l 3C17
  \l 3C18
  \l 3C19
  \l 3C1A
  \l 3C1B
  \l 3C1C
  \l 3C1D
  \l 3C1E
  \l 3C1F
  \l 3C20
  \l 3C21
  \l 3C22
  \l 3C23
  \l 3C24
  \l 3C25
  \l 3C26
  \l 3C27
  \l 3C28
  \l 3C29
  \l 3C2A
  \l 3C2B
  \l 3C2C
  \l 3C2D
  \l 3C2E
  \l 3C2F
  \l 3C30
  \l 3C31
  \l 3C32
  \l 3C33
  \l 3C34
  \l 3C35
  \l 3C36
  \l 3C37
  \l 3C38
  \l 3C39
  \l 3C3A
  \l 3C3B
  \l 3C3C
  \l 3C3D
  \l 3C3E
  \l 3C3F
  \l 3C40
  \l 3C41
  \l 3C42
  \l 3C43
  \l 3C44
  \l 3C45
  \l 3C46
  \l 3C47
  \l 3C48
  \l 3C49
  \l 3C4A
  \l 3C4B
  \l 3C4C
  \l 3C4D
  \l 3C4E
  \l 3C4F
  \l 3C50
  \l 3C51
  \l 3C52
  \l 3C53
  \l 3C54
  \l 3C55
  \l 3C56
  \l 3C57
  \l 3C58
  \l 3C59
  \l 3C5A
  \l 3C5B
  \l 3C5C
  \l 3C5D
  \l 3C5E
  \l 3C5F
  \l 3C60
  \l 3C61
  \l 3C62
  \l 3C63
  \l 3C64
  \l 3C65
  \l 3C66
  \l 3C67
  \l 3C68
  \l 3C69
  \l 3C6A
  \l 3C6B
  \l 3C6C
  \l 3C6D
  \l 3C6E
  \l 3C6F
  \l 3C70
  \l 3C71
  \l 3C72
  \l 3C73
  \l 3C74
  \l 3C75
  \l 3C76
  \l 3C77
  \l 3C78
  \l 3C79
  \l 3C7A
  \l 3C7B
  \l 3C7C
  \l 3C7D
  \l 3C7E
  \l 3C7F
  \l 3C80
  \l 3C81
  \l 3C82
  \l 3C83
  \l 3C84
  \l 3C85
  \l 3C86
  \l 3C87
  \l 3C88
  \l 3C89
  \l 3C8A
  \l 3C8B
  \l 3C8C
  \l 3C8D
  \l 3C8E
  \l 3C8F
  \l 3C90
  \l 3C91
  \l 3C92
  \l 3C93
  \l 3C94
  \l 3C95
  \l 3C96
  \l 3C97
  \l 3C98
  \l 3C99
  \l 3C9A
  \l 3C9B
  \l 3C9C
  \l 3C9D
  \l 3C9E
  \l 3C9F
  \l 3CA0
  \l 3CA1
  \l 3CA2
  \l 3CA3
  \l 3CA4
  \l 3CA5
  \l 3CA6
  \l 3CA7
  \l 3CA8
  \l 3CA9
  \l 3CAA
  \l 3CAB
  \l 3CAC
  \l 3CAD
  \l 3CAE
  \l 3CAF
  \l 3CB0
  \l 3CB1
  \l 3CB2
  \l 3CB3
  \l 3CB4
  \l 3CB5
  \l 3CB6
  \l 3CB7
  \l 3CB8
  \l 3CB9
  \l 3CBA
  \l 3CBB
  \l 3CBC
  \l 3CBD
  \l 3CBE
  \l 3CBF
  \l 3CC0
  \l 3CC1
  \l 3CC2
  \l 3CC3
  \l 3CC4
  \l 3CC5
  \l 3CC6
  \l 3CC7
  \l 3CC8
  \l 3CC9
  \l 3CCA
  \l 3CCB
  \l 3CCC
  \l 3CCD
  \l 3CCE
  \l 3CCF
  \l 3CD0
  \l 3CD1
  \l 3CD2
  \l 3CD3
  \l 3CD4
  \l 3CD5
  \l 3CD6
  \l 3CD7
  \l 3CD8
  \l 3CD9
  \l 3CDA
  \l 3CDB
  \l 3CDC
  \l 3CDD
  \l 3CDE
  \l 3CDF
  \l 3CE0
  \l 3CE1
  \l 3CE2
  \l 3CE3
  \l 3CE4
  \l 3CE5
  \l 3CE6
  \l 3CE7
  \l 3CE8
  \l 3CE9
  \l 3CEA
  \l 3CEB
  \l 3CEC
  \l 3CED
  \l 3CEE
  \l 3CEF
  \l 3CF0
  \l 3CF1
  \l 3CF2
  \l 3CF3
  \l 3CF4
  \l 3CF5
  \l 3CF6
  \l 3CF7
  \l 3CF8
  \l 3CF9
  \l 3CFA
  \l 3CFB
  \l 3CFC
  \l 3CFD
  \l 3CFE
  \l 3CFF
  \l 3D00
  \l 3D01
  \l 3D02
  \l 3D03
  \l 3D04
  \l 3D05
  \l 3D06
  \l 3D07
  \l 3D08
  \l 3D09
  \l 3D0A
  \l 3D0B
  \l 3D0C
  \l 3D0D
  \l 3D0E
  \l 3D0F
  \l 3D10
  \l 3D11
  \l 3D12
  \l 3D13
  \l 3D14
  \l 3D15
  \l 3D16
  \l 3D17
  \l 3D18
  \l 3D19
  \l 3D1A
  \l 3D1B
  \l 3D1C
  \l 3D1D
  \l 3D1E
  \l 3D1F
  \l 3D20
  \l 3D21
  \l 3D22
  \l 3D23
  \l 3D24
  \l 3D25
  \l 3D26
  \l 3D27
  \l 3D28
  \l 3D29
  \l 3D2A
  \l 3D2B
  \l 3D2C
  \l 3D2D
  \l 3D2E
  \l 3D2F
  \l 3D30
  \l 3D31
  \l 3D32
  \l 3D33
  \l 3D34
  \l 3D35
  \l 3D36
  \l 3D37
  \l 3D38
  \l 3D39
  \l 3D3A
  \l 3D3B
  \l 3D3C
  \l 3D3D
  \l 3D3E
  \l 3D3F
  \l 3D40
  \l 3D41
  \l 3D42
  \l 3D43
  \l 3D44
  \l 3D45
  \l 3D46
  \l 3D47
  \l 3D48
  \l 3D49
  \l 3D4A
  \l 3D4B
  \l 3D4C
  \l 3D4D
  \l 3D4E
  \l 3D4F
  \l 3D50
  \l 3D51
  \l 3D52
  \l 3D53
  \l 3D54
  \l 3D55
  \l 3D56
  \l 3D57
  \l 3D58
  \l 3D59
  \l 3D5A
  \l 3D5B
  \l 3D5C
  \l 3D5D
  \l 3D5E
  \l 3D5F
  \l 3D60
  \l 3D61
  \l 3D62
  \l 3D63
  \l 3D64
  \l 3D65
  \l 3D66
  \l 3D67
  \l 3D68
  \l 3D69
  \l 3D6A
  \l 3D6B
  \l 3D6C
  \l 3D6D
  \l 3D6E
  \l 3D6F
  \l 3D70
  \l 3D71
  \l 3D72
  \l 3D73
  \l 3D74
  \l 3D75
  \l 3D76
  \l 3D77
  \l 3D78
  \l 3D79
  \l 3D7A
  \l 3D7B
  \l 3D7C
  \l 3D7D
  \l 3D7E
  \l 3D7F
  \l 3D80
  \l 3D81
  \l 3D82
  \l 3D83
  \l 3D84
  \l 3D85
  \l 3D86
  \l 3D87
  \l 3D88
  \l 3D89
  \l 3D8A
  \l 3D8B
  \l 3D8C
  \l 3D8D
  \l 3D8E
  \l 3D8F
  \l 3D90
  \l 3D91
  \l 3D92
  \l 3D93
  \l 3D94
  \l 3D95
  \l 3D96
  \l 3D97
  \l 3D98
  \l 3D99
  \l 3D9A
  \l 3D9B
  \l 3D9C
  \l 3D9D
  \l 3D9E
  \l 3D9F
  \l 3DA0
  \l 3DA1
  \l 3DA2
  \l 3DA3
  \l 3DA4
  \l 3DA5
  \l 3DA6
  \l 3DA7
  \l 3DA8
  \l 3DA9
  \l 3DAA
  \l 3DAB
  \l 3DAC
  \l 3DAD
  \l 3DAE
  \l 3DAF
  \l 3DB0
  \l 3DB1
  \l 3DB2
  \l 3DB3
  \l 3DB4
  \l 3DB5
  \l 3DB6
  \l 3DB7
  \l 3DB8
  \l 3DB9
  \l 3DBA
  \l 3DBB
  \l 3DBC
  \l 3DBD
  \l 3DBE
  \l 3DBF
  \l 3DC0
  \l 3DC1
  \l 3DC2
  \l 3DC3
  \l 3DC4
  \l 3DC5
  \l 3DC6
  \l 3DC7
  \l 3DC8
  \l 3DC9
  \l 3DCA
  \l 3DCB
  \l 3DCC
  \l 3DCD
  \l 3DCE
  \l 3DCF
  \l 3DD0
  \l 3DD1
  \l 3DD2
  \l 3DD3
  \l 3DD4
  \l 3DD5
  \l 3DD6
  \l 3DD7
  \l 3DD8
  \l 3DD9
  \l 3DDA
  \l 3DDB
  \l 3DDC
  \l 3DDD
  \l 3DDE
  \l 3DDF
  \l 3DE0
  \l 3DE1
  \l 3DE2
  \l 3DE3
  \l 3DE4
  \l 3DE5
  \l 3DE6
  \l 3DE7
  \l 3DE8
  \l 3DE9
  \l 3DEA
  \l 3DEB
  \l 3DEC
  \l 3DED
  \l 3DEE
  \l 3DEF
  \l 3DF0
  \l 3DF1
  \l 3DF2
  \l 3DF3
  \l 3DF4
  \l 3DF5
  \l 3DF6
  \l 3DF7
  \l 3DF8
  \l 3DF9
  \l 3DFA
  \l 3DFB
  \l 3DFC
  \l 3DFD
  \l 3DFE
  \l 3DFF
  \l 3E00
  \l 3E01
  \l 3E02
  \l 3E03
  \l 3E04
  \l 3E05
  \l 3E06
  \l 3E07
  \l 3E08
  \l 3E09
  \l 3E0A
  \l 3E0B
  \l 3E0C
  \l 3E0D
  \l 3E0E
  \l 3E0F
  \l 3E10
  \l 3E11
  \l 3E12
  \l 3E13
  \l 3E14
  \l 3E15
  \l 3E16
  \l 3E17
  \l 3E18
  \l 3E19
  \l 3E1A
  \l 3E1B
  \l 3E1C
  \l 3E1D
  \l 3E1E
  \l 3E1F
  \l 3E20
  \l 3E21
  \l 3E22
  \l 3E23
  \l 3E24
  \l 3E25
  \l 3E26
  \l 3E27
  \l 3E28
  \l 3E29
  \l 3E2A
  \l 3E2B
  \l 3E2C
  \l 3E2D
  \l 3E2E
  \l 3E2F
  \l 3E30
  \l 3E31
  \l 3E32
  \l 3E33
  \l 3E34
  \l 3E35
  \l 3E36
  \l 3E37
  \l 3E38
  \l 3E39
  \l 3E3A
  \l 3E3B
  \l 3E3C
  \l 3E3D
  \l 3E3E
  \l 3E3F
  \l 3E40
  \l 3E41
  \l 3E42
  \l 3E43
  \l 3E44
  \l 3E45
  \l 3E46
  \l 3E47
  \l 3E48
  \l 3E49
  \l 3E4A
  \l 3E4B
  \l 3E4C
  \l 3E4D
  \l 3E4E
  \l 3E4F
  \l 3E50
  \l 3E51
  \l 3E52
  \l 3E53
  \l 3E54
  \l 3E55
  \l 3E56
  \l 3E57
  \l 3E58
  \l 3E59
  \l 3E5A
  \l 3E5B
  \l 3E5C
  \l 3E5D
  \l 3E5E
  \l 3E5F
  \l 3E60
  \l 3E61
  \l 3E62
  \l 3E63
  \l 3E64
  \l 3E65
  \l 3E66
  \l 3E67
  \l 3E68
  \l 3E69
  \l 3E6A
  \l 3E6B
  \l 3E6C
  \l 3E6D
  \l 3E6E
  \l 3E6F
  \l 3E70
  \l 3E71
  \l 3E72
  \l 3E73
  \l 3E74
  \l 3E75
  \l 3E76
  \l 3E77
  \l 3E78
  \l 3E79
  \l 3E7A
  \l 3E7B
  \l 3E7C
  \l 3E7D
  \l 3E7E
  \l 3E7F
  \l 3E80
  \l 3E81
  \l 3E82
  \l 3E83
  \l 3E84
  \l 3E85
  \l 3E86
  \l 3E87
  \l 3E88
  \l 3E89
  \l 3E8A
  \l 3E8B
  \l 3E8C
  \l 3E8D
  \l 3E8E
  \l 3E8F
  \l 3E90
  \l 3E91
  \l 3E92
  \l 3E93
  \l 3E94
  \l 3E95
  \l 3E96
  \l 3E97
  \l 3E98
  \l 3E99
  \l 3E9A
  \l 3E9B
  \l 3E9C
  \l 3E9D
  \l 3E9E
  \l 3E9F
  \l 3EA0
  \l 3EA1
  \l 3EA2
  \l 3EA3
  \l 3EA4
  \l 3EA5
  \l 3EA6
  \l 3EA7
  \l 3EA8
  \l 3EA9
  \l 3EAA
  \l 3EAB
  \l 3EAC
  \l 3EAD
  \l 3EAE
  \l 3EAF
  \l 3EB0
  \l 3EB1
  \l 3EB2
  \l 3EB3
  \l 3EB4
  \l 3EB5
  \l 3EB6
  \l 3EB7
  \l 3EB8
  \l 3EB9
  \l 3EBA
  \l 3EBB
  \l 3EBC
  \l 3EBD
  \l 3EBE
  \l 3EBF
  \l 3EC0
  \l 3EC1
  \l 3EC2
  \l 3EC3
  \l 3EC4
  \l 3EC5
  \l 3EC6
  \l 3EC7
  \l 3EC8
  \l 3EC9
  \l 3ECA
  \l 3ECB
  \l 3ECC
  \l 3ECD
  \l 3ECE
  \l 3ECF
  \l 3ED0
  \l 3ED1
  \l 3ED2
  \l 3ED3
  \l 3ED4
  \l 3ED5
  \l 3ED6
  \l 3ED7
  \l 3ED8
  \l 3ED9
  \l 3EDA
  \l 3EDB
  \l 3EDC
  \l 3EDD
  \l 3EDE
  \l 3EDF
  \l 3EE0
  \l 3EE1
  \l 3EE2
  \l 3EE3
  \l 3EE4
  \l 3EE5
  \l 3EE6
  \l 3EE7
  \l 3EE8
  \l 3EE9
  \l 3EEA
  \l 3EEB
  \l 3EEC
  \l 3EED
  \l 3EEE
  \l 3EEF
  \l 3EF0
  \l 3EF1
  \l 3EF2
  \l 3EF3
  \l 3EF4
  \l 3EF5
  \l 3EF6
  \l 3EF7
  \l 3EF8
  \l 3EF9
  \l 3EFA
  \l 3EFB
  \l 3EFC
  \l 3EFD
  \l 3EFE
  \l 3EFF
  \l 3F00
  \l 3F01
  \l 3F02
  \l 3F03
  \l 3F04
  \l 3F05
  \l 3F06
  \l 3F07
  \l 3F08
  \l 3F09
  \l 3F0A
  \l 3F0B
  \l 3F0C
  \l 3F0D
  \l 3F0E
  \l 3F0F
  \l 3F10
  \l 3F11
  \l 3F12
  \l 3F13
  \l 3F14
  \l 3F15
  \l 3F16
  \l 3F17
  \l 3F18
  \l 3F19
  \l 3F1A
  \l 3F1B
  \l 3F1C
  \l 3F1D
  \l 3F1E
  \l 3F1F
  \l 3F20
  \l 3F21
  \l 3F22
  \l 3F23
  \l 3F24
  \l 3F25
  \l 3F26
  \l 3F27
  \l 3F28
  \l 3F29
  \l 3F2A
  \l 3F2B
  \l 3F2C
  \l 3F2D
  \l 3F2E
  \l 3F2F
  \l 3F30
  \l 3F31
  \l 3F32
  \l 3F33
  \l 3F34
  \l 3F35
  \l 3F36
  \l 3F37
  \l 3F38
  \l 3F39
  \l 3F3A
  \l 3F3B
  \l 3F3C
  \l 3F3D
  \l 3F3E
  \l 3F3F
  \l 3F40
  \l 3F41
  \l 3F42
  \l 3F43
  \l 3F44
  \l 3F45
  \l 3F46
  \l 3F47
  \l 3F48
  \l 3F49
  \l 3F4A
  \l 3F4B
  \l 3F4C
  \l 3F4D
  \l 3F4E
  \l 3F4F
  \l 3F50
  \l 3F51
  \l 3F52
  \l 3F53
  \l 3F54
  \l 3F55
  \l 3F56
  \l 3F57
  \l 3F58
  \l 3F59
  \l 3F5A
  \l 3F5B
  \l 3F5C
  \l 3F5D
  \l 3F5E
  \l 3F5F
  \l 3F60
  \l 3F61
  \l 3F62
  \l 3F63
  \l 3F64
  \l 3F65
  \l 3F66
  \l 3F67
  \l 3F68
  \l 3F69
  \l 3F6A
  \l 3F6B
  \l 3F6C
  \l 3F6D
  \l 3F6E
  \l 3F6F
  \l 3F70
  \l 3F71
  \l 3F72
  \l 3F73
  \l 3F74
  \l 3F75
  \l 3F76
  \l 3F77
  \l 3F78
  \l 3F79
  \l 3F7A
  \l 3F7B
  \l 3F7C
  \l 3F7D
  \l 3F7E
  \l 3F7F
  \l 3F80
  \l 3F81
  \l 3F82
  \l 3F83
  \l 3F84
  \l 3F85
  \l 3F86
  \l 3F87
  \l 3F88
  \l 3F89
  \l 3F8A
  \l 3F8B
  \l 3F8C
  \l 3F8D
  \l 3F8E
  \l 3F8F
  \l 3F90
  \l 3F91
  \l 3F92
  \l 3F93
  \l 3F94
  \l 3F95
  \l 3F96
  \l 3F97
  \l 3F98
  \l 3F99
  \l 3F9A
  \l 3F9B
  \l 3F9C
  \l 3F9D
  \l 3F9E
  \l 3F9F
  \l 3FA0
  \l 3FA1
  \l 3FA2
  \l 3FA3
  \l 3FA4
  \l 3FA5
  \l 3FA6
  \l 3FA7
  \l 3FA8
  \l 3FA9
  \l 3FAA
  \l 3FAB
  \l 3FAC
  \l 3FAD
  \l 3FAE
  \l 3FAF
  \l 3FB0
  \l 3FB1
  \l 3FB2
  \l 3FB3
  \l 3FB4
  \l 3FB5
  \l 3FB6
  \l 3FB7
  \l 3FB8
  \l 3FB9
  \l 3FBA
  \l 3FBB
  \l 3FBC
  \l 3FBD
  \l 3FBE
  \l 3FBF
  \l 3FC0
  \l 3FC1
  \l 3FC2
  \l 3FC3
  \l 3FC4
  \l 3FC5
  \l 3FC6
  \l 3FC7
  \l 3FC8
  \l 3FC9
  \l 3FCA
  \l 3FCB
  \l 3FCC
  \l 3FCD
  \l 3FCE
  \l 3FCF
  \l 3FD0
  \l 3FD1
  \l 3FD2
  \l 3FD3
  \l 3FD4
  \l 3FD5
  \l 3FD6
  \l 3FD7
  \l 3FD8
  \l 3FD9
  \l 3FDA
  \l 3FDB
  \l 3FDC
  \l 3FDD
  \l 3FDE
  \l 3FDF
  \l 3FE0
  \l 3FE1
  \l 3FE2
  \l 3FE3
  \l 3FE4
  \l 3FE5
  \l 3FE6
  \l 3FE7
  \l 3FE8
  \l 3FE9
  \l 3FEA
  \l 3FEB
  \l 3FEC
  \l 3FED
  \l 3FEE
  \l 3FEF
  \l 3FF0
  \l 3FF1
  \l 3FF2
  \l 3FF3
  \l 3FF4
  \l 3FF5
  \l 3FF6
  \l 3FF7
  \l 3FF8
  \l 3FF9
  \l 3FFA
  \l 3FFB
  \l 3FFC
  \l 3FFD
  \l 3FFE
  \l 3FFF
  \l 4000
  \l 4001
  \l 4002
  \l 4003
  \l 4004
  \l 4005
  \l 4006
  \l 4007
  \l 4008
  \l 4009
  \l 400A
  \l 400B
  \l 400C
  \l 400D
  \l 400E
  \l 400F
  \l 4010
  \l 4011
  \l 4012
  \l 4013
  \l 4014
  \l 4015
  \l 4016
  \l 4017
  \l 4018
  \l 4019
  \l 401A
  \l 401B
  \l 401C
  \l 401D
  \l 401E
  \l 401F
  \l 4020
  \l 4021
  \l 4022
  \l 4023
  \l 4024
  \l 4025
  \l 4026
  \l 4027
  \l 4028
  \l 4029
  \l 402A
  \l 402B
  \l 402C
  \l 402D
  \l 402E
  \l 402F
  \l 4030
  \l 4031
  \l 4032
  \l 4033
  \l 4034
  \l 4035
  \l 4036
  \l 4037
  \l 4038
  \l 4039
  \l 403A
  \l 403B
  \l 403C
  \l 403D
  \l 403E
  \l 403F
  \l 4040
  \l 4041
  \l 4042
  \l 4043
  \l 4044
  \l 4045
  \l 4046
  \l 4047
  \l 4048
  \l 4049
  \l 404A
  \l 404B
  \l 404C
  \l 404D
  \l 404E
  \l 404F
  \l 4050
  \l 4051
  \l 4052
  \l 4053
  \l 4054
  \l 4055
  \l 4056
  \l 4057
  \l 4058
  \l 4059
  \l 405A
  \l 405B
  \l 405C
  \l 405D
  \l 405E
  \l 405F
  \l 4060
  \l 4061
  \l 4062
  \l 4063
  \l 4064
  \l 4065
  \l 4066
  \l 4067
  \l 4068
  \l 4069
  \l 406A
  \l 406B
  \l 406C
  \l 406D
  \l 406E
  \l 406F
  \l 4070
  \l 4071
  \l 4072
  \l 4073
  \l 4074
  \l 4075
  \l 4076
  \l 4077
  \l 4078
  \l 4079
  \l 407A
  \l 407B
  \l 407C
  \l 407D
  \l 407E
  \l 407F
  \l 4080
  \l 4081
  \l 4082
  \l 4083
  \l 4084
  \l 4085
  \l 4086
  \l 4087
  \l 4088
  \l 4089
  \l 408A
  \l 408B
  \l 408C
  \l 408D
  \l 408E
  \l 408F
  \l 4090
  \l 4091
  \l 4092
  \l 4093
  \l 4094
  \l 4095
  \l 4096
  \l 4097
  \l 4098
  \l 4099
  \l 409A
  \l 409B
  \l 409C
  \l 409D
  \l 409E
  \l 409F
  \l 40A0
  \l 40A1
  \l 40A2
  \l 40A3
  \l 40A4
  \l 40A5
  \l 40A6
  \l 40A7
  \l 40A8
  \l 40A9
  \l 40AA
  \l 40AB
  \l 40AC
  \l 40AD
  \l 40AE
  \l 40AF
  \l 40B0
  \l 40B1
  \l 40B2
  \l 40B3
  \l 40B4
  \l 40B5
  \l 40B6
  \l 40B7
  \l 40B8
  \l 40B9
  \l 40BA
  \l 40BB
  \l 40BC
  \l 40BD
  \l 40BE
  \l 40BF
  \l 40C0
  \l 40C1
  \l 40C2
  \l 40C3
  \l 40C4
  \l 40C5
  \l 40C6
  \l 40C7
  \l 40C8
  \l 40C9
  \l 40CA
  \l 40CB
  \l 40CC
  \l 40CD
  \l 40CE
  \l 40CF
  \l 40D0
  \l 40D1
  \l 40D2
  \l 40D3
  \l 40D4
  \l 40D5
  \l 40D6
  \l 40D7
  \l 40D8
  \l 40D9
  \l 40DA
  \l 40DB
  \l 40DC
  \l 40DD
  \l 40DE
  \l 40DF
  \l 40E0
  \l 40E1
  \l 40E2
  \l 40E3
  \l 40E4
  \l 40E5
  \l 40E6
  \l 40E7
  \l 40E8
  \l 40E9
  \l 40EA
  \l 40EB
  \l 40EC
  \l 40ED
  \l 40EE
  \l 40EF
  \l 40F0
  \l 40F1
  \l 40F2
  \l 40F3
  \l 40F4
  \l 40F5
  \l 40F6
  \l 40F7
  \l 40F8
  \l 40F9
  \l 40FA
  \l 40FB
  \l 40FC
  \l 40FD
  \l 40FE
  \l 40FF
  \l 4100
  \l 4101
  \l 4102
  \l 4103
  \l 4104
  \l 4105
  \l 4106
  \l 4107
  \l 4108
  \l 4109
  \l 410A
  \l 410B
  \l 410C
  \l 410D
  \l 410E
  \l 410F
  \l 4110
  \l 4111
  \l 4112
  \l 4113
  \l 4114
  \l 4115
  \l 4116
  \l 4117
  \l 4118
  \l 4119
  \l 411A
  \l 411B
  \l 411C
  \l 411D
  \l 411E
  \l 411F
  \l 4120
  \l 4121
  \l 4122
  \l 4123
  \l 4124
  \l 4125
  \l 4126
  \l 4127
  \l 4128
  \l 4129
  \l 412A
  \l 412B
  \l 412C
  \l 412D
  \l 412E
  \l 412F
  \l 4130
  \l 4131
  \l 4132
  \l 4133
  \l 4134
  \l 4135
  \l 4136
  \l 4137
  \l 4138
  \l 4139
  \l 413A
  \l 413B
  \l 413C
  \l 413D
  \l 413E
  \l 413F
  \l 4140
  \l 4141
  \l 4142
  \l 4143
  \l 4144
  \l 4145
  \l 4146
  \l 4147
  \l 4148
  \l 4149
  \l 414A
  \l 414B
  \l 414C
  \l 414D
  \l 414E
  \l 414F
  \l 4150
  \l 4151
  \l 4152
  \l 4153
  \l 4154
  \l 4155
  \l 4156
  \l 4157
  \l 4158
  \l 4159
  \l 415A
  \l 415B
  \l 415C
  \l 415D
  \l 415E
  \l 415F
  \l 4160
  \l 4161
  \l 4162
  \l 4163
  \l 4164
  \l 4165
  \l 4166
  \l 4167
  \l 4168
  \l 4169
  \l 416A
  \l 416B
  \l 416C
  \l 416D
  \l 416E
  \l 416F
  \l 4170
  \l 4171
  \l 4172
  \l 4173
  \l 4174
  \l 4175
  \l 4176
  \l 4177
  \l 4178
  \l 4179
  \l 417A
  \l 417B
  \l 417C
  \l 417D
  \l 417E
  \l 417F
  \l 4180
  \l 4181
  \l 4182
  \l 4183
  \l 4184
  \l 4185
  \l 4186
  \l 4187
  \l 4188
  \l 4189
  \l 418A
  \l 418B
  \l 418C
  \l 418D
  \l 418E
  \l 418F
  \l 4190
  \l 4191
  \l 4192
  \l 4193
  \l 4194
  \l 4195
  \l 4196
  \l 4197
  \l 4198
  \l 4199
  \l 419A
  \l 419B
  \l 419C
  \l 419D
  \l 419E
  \l 419F
  \l 41A0
  \l 41A1
  \l 41A2
  \l 41A3
  \l 41A4
  \l 41A5
  \l 41A6
  \l 41A7
  \l 41A8
  \l 41A9
  \l 41AA
  \l 41AB
  \l 41AC
  \l 41AD
  \l 41AE
  \l 41AF
  \l 41B0
  \l 41B1
  \l 41B2
  \l 41B3
  \l 41B4
  \l 41B5
  \l 41B6
  \l 41B7
  \l 41B8
  \l 41B9
  \l 41BA
  \l 41BB
  \l 41BC
  \l 41BD
  \l 41BE
  \l 41BF
  \l 41C0
  \l 41C1
  \l 41C2
  \l 41C3
  \l 41C4
  \l 41C5
  \l 41C6
  \l 41C7
  \l 41C8
  \l 41C9
  \l 41CA
  \l 41CB
  \l 41CC
  \l 41CD
  \l 41CE
  \l 41CF
  \l 41D0
  \l 41D1
  \l 41D2
  \l 41D3
  \l 41D4
  \l 41D5
  \l 41D6
  \l 41D7
  \l 41D8
  \l 41D9
  \l 41DA
  \l 41DB
  \l 41DC
  \l 41DD
  \l 41DE
  \l 41DF
  \l 41E0
  \l 41E1
  \l 41E2
  \l 41E3
  \l 41E4
  \l 41E5
  \l 41E6
  \l 41E7
  \l 41E8
  \l 41E9
  \l 41EA
  \l 41EB
  \l 41EC
  \l 41ED
  \l 41EE
  \l 41EF
  \l 41F0
  \l 41F1
  \l 41F2
  \l 41F3
  \l 41F4
  \l 41F5
  \l 41F6
  \l 41F7
  \l 41F8
  \l 41F9
  \l 41FA
  \l 41FB
  \l 41FC
  \l 41FD
  \l 41FE
  \l 41FF
  \l 4200
  \l 4201
  \l 4202
  \l 4203
  \l 4204
  \l 4205
  \l 4206
  \l 4207
  \l 4208
  \l 4209
  \l 420A
  \l 420B
  \l 420C
  \l 420D
  \l 420E
  \l 420F
  \l 4210
  \l 4211
  \l 4212
  \l 4213
  \l 4214
  \l 4215
  \l 4216
  \l 4217
  \l 4218
  \l 4219
  \l 421A
  \l 421B
  \l 421C
  \l 421D
  \l 421E
  \l 421F
  \l 4220
  \l 4221
  \l 4222
  \l 4223
  \l 4224
  \l 4225
  \l 4226
  \l 4227
  \l 4228
  \l 4229
  \l 422A
  \l 422B
  \l 422C
  \l 422D
  \l 422E
  \l 422F
  \l 4230
  \l 4231
  \l 4232
  \l 4233
  \l 4234
  \l 4235
  \l 4236
  \l 4237
  \l 4238
  \l 4239
  \l 423A
  \l 423B
  \l 423C
  \l 423D
  \l 423E
  \l 423F
  \l 4240
  \l 4241
  \l 4242
  \l 4243
  \l 4244
  \l 4245
  \l 4246
  \l 4247
  \l 4248
  \l 4249
  \l 424A
  \l 424B
  \l 424C
  \l 424D
  \l 424E
  \l 424F
  \l 4250
  \l 4251
  \l 4252
  \l 4253
  \l 4254
  \l 4255
  \l 4256
  \l 4257
  \l 4258
  \l 4259
  \l 425A
  \l 425B
  \l 425C
  \l 425D
  \l 425E
  \l 425F
  \l 4260
  \l 4261
  \l 4262
  \l 4263
  \l 4264
  \l 4265
  \l 4266
  \l 4267
  \l 4268
  \l 4269
  \l 426A
  \l 426B
  \l 426C
  \l 426D
  \l 426E
  \l 426F
  \l 4270
  \l 4271
  \l 4272
  \l 4273
  \l 4274
  \l 4275
  \l 4276
  \l 4277
  \l 4278
  \l 4279
  \l 427A
  \l 427B
  \l 427C
  \l 427D
  \l 427E
  \l 427F
  \l 4280
  \l 4281
  \l 4282
  \l 4283
  \l 4284
  \l 4285
  \l 4286
  \l 4287
  \l 4288
  \l 4289
  \l 428A
  \l 428B
  \l 428C
  \l 428D
  \l 428E
  \l 428F
  \l 4290
  \l 4291
  \l 4292
  \l 4293
  \l 4294
  \l 4295
  \l 4296
  \l 4297
  \l 4298
  \l 4299
  \l 429A
  \l 429B
  \l 429C
  \l 429D
  \l 429E
  \l 429F
  \l 42A0
  \l 42A1
  \l 42A2
  \l 42A3
  \l 42A4
  \l 42A5
  \l 42A6
  \l 42A7
  \l 42A8
  \l 42A9
  \l 42AA
  \l 42AB
  \l 42AC
  \l 42AD
  \l 42AE
  \l 42AF
  \l 42B0
  \l 42B1
  \l 42B2
  \l 42B3
  \l 42B4
  \l 42B5
  \l 42B6
  \l 42B7
  \l 42B8
  \l 42B9
  \l 42BA
  \l 42BB
  \l 42BC
  \l 42BD
  \l 42BE
  \l 42BF
  \l 42C0
  \l 42C1
  \l 42C2
  \l 42C3
  \l 42C4
  \l 42C5
  \l 42C6
  \l 42C7
  \l 42C8
  \l 42C9
  \l 42CA
  \l 42CB
  \l 42CC
  \l 42CD
  \l 42CE
  \l 42CF
  \l 42D0
  \l 42D1
  \l 42D2
  \l 42D3
  \l 42D4
  \l 42D5
  \l 42D6
  \l 42D7
  \l 42D8
  \l 42D9
  \l 42DA
  \l 42DB
  \l 42DC
  \l 42DD
  \l 42DE
  \l 42DF
  \l 42E0
  \l 42E1
  \l 42E2
  \l 42E3
  \l 42E4
  \l 42E5
  \l 42E6
  \l 42E7
  \l 42E8
  \l 42E9
  \l 42EA
  \l 42EB
  \l 42EC
  \l 42ED
  \l 42EE
  \l 42EF
  \l 42F0
  \l 42F1
  \l 42F2
  \l 42F3
  \l 42F4
  \l 42F5
  \l 42F6
  \l 42F7
  \l 42F8
  \l 42F9
  \l 42FA
  \l 42FB
  \l 42FC
  \l 42FD
  \l 42FE
  \l 42FF
  \l 4300
  \l 4301
  \l 4302
  \l 4303
  \l 4304
  \l 4305
  \l 4306
  \l 4307
  \l 4308
  \l 4309
  \l 430A
  \l 430B
  \l 430C
  \l 430D
  \l 430E
  \l 430F
  \l 4310
  \l 4311
  \l 4312
  \l 4313
  \l 4314
  \l 4315
  \l 4316
  \l 4317
  \l 4318
  \l 4319
  \l 431A
  \l 431B
  \l 431C
  \l 431D
  \l 431E
  \l 431F
  \l 4320
  \l 4321
  \l 4322
  \l 4323
  \l 4324
  \l 4325
  \l 4326
  \l 4327
  \l 4328
  \l 4329
  \l 432A
  \l 432B
  \l 432C
  \l 432D
  \l 432E
  \l 432F
  \l 4330
  \l 4331
  \l 4332
  \l 4333
  \l 4334
  \l 4335
  \l 4336
  \l 4337
  \l 4338
  \l 4339
  \l 433A
  \l 433B
  \l 433C
  \l 433D
  \l 433E
  \l 433F
  \l 4340
  \l 4341
  \l 4342
  \l 4343
  \l 4344
  \l 4345
  \l 4346
  \l 4347
  \l 4348
  \l 4349
  \l 434A
  \l 434B
  \l 434C
  \l 434D
  \l 434E
  \l 434F
  \l 4350
  \l 4351
  \l 4352
  \l 4353
  \l 4354
  \l 4355
  \l 4356
  \l 4357
  \l 4358
  \l 4359
  \l 435A
  \l 435B
  \l 435C
  \l 435D
  \l 435E
  \l 435F
  \l 4360
  \l 4361
  \l 4362
  \l 4363
  \l 4364
  \l 4365
  \l 4366
  \l 4367
  \l 4368
  \l 4369
  \l 436A
  \l 436B
  \l 436C
  \l 436D
  \l 436E
  \l 436F
  \l 4370
  \l 4371
  \l 4372
  \l 4373
  \l 4374
  \l 4375
  \l 4376
  \l 4377
  \l 4378
  \l 4379
  \l 437A
  \l 437B
  \l 437C
  \l 437D
  \l 437E
  \l 437F
  \l 4380
  \l 4381
  \l 4382
  \l 4383
  \l 4384
  \l 4385
  \l 4386
  \l 4387
  \l 4388
  \l 4389
  \l 438A
  \l 438B
  \l 438C
  \l 438D
  \l 438E
  \l 438F
  \l 4390
  \l 4391
  \l 4392
  \l 4393
  \l 4394
  \l 4395
  \l 4396
  \l 4397
  \l 4398
  \l 4399
  \l 439A
  \l 439B
  \l 439C
  \l 439D
  \l 439E
  \l 439F
  \l 43A0
  \l 43A1
  \l 43A2
  \l 43A3
  \l 43A4
  \l 43A5
  \l 43A6
  \l 43A7
  \l 43A8
  \l 43A9
  \l 43AA
  \l 43AB
  \l 43AC
  \l 43AD
  \l 43AE
  \l 43AF
  \l 43B0
  \l 43B1
  \l 43B2
  \l 43B3
  \l 43B4
  \l 43B5
  \l 43B6
  \l 43B7
  \l 43B8
  \l 43B9
  \l 43BA
  \l 43BB
  \l 43BC
  \l 43BD
  \l 43BE
  \l 43BF
  \l 43C0
  \l 43C1
  \l 43C2
  \l 43C3
  \l 43C4
  \l 43C5
  \l 43C6
  \l 43C7
  \l 43C8
  \l 43C9
  \l 43CA
  \l 43CB
  \l 43CC
  \l 43CD
  \l 43CE
  \l 43CF
  \l 43D0
  \l 43D1
  \l 43D2
  \l 43D3
  \l 43D4
  \l 43D5
  \l 43D6
  \l 43D7
  \l 43D8
  \l 43D9
  \l 43DA
  \l 43DB
  \l 43DC
  \l 43DD
  \l 43DE
  \l 43DF
  \l 43E0
  \l 43E1
  \l 43E2
  \l 43E3
  \l 43E4
  \l 43E5
  \l 43E6
  \l 43E7
  \l 43E8
  \l 43E9
  \l 43EA
  \l 43EB
  \l 43EC
  \l 43ED
  \l 43EE
  \l 43EF
  \l 43F0
  \l 43F1
  \l 43F2
  \l 43F3
  \l 43F4
  \l 43F5
  \l 43F6
  \l 43F7
  \l 43F8
  \l 43F9
  \l 43FA
  \l 43FB
  \l 43FC
  \l 43FD
  \l 43FE
  \l 43FF
  \l 4400
  \l 4401
  \l 4402
  \l 4403
  \l 4404
  \l 4405
  \l 4406
  \l 4407
  \l 4408
  \l 4409
  \l 440A
  \l 440B
  \l 440C
  \l 440D
  \l 440E
  \l 440F
  \l 4410
  \l 4411
  \l 4412
  \l 4413
  \l 4414
  \l 4415
  \l 4416
  \l 4417
  \l 4418
  \l 4419
  \l 441A
  \l 441B
  \l 441C
  \l 441D
  \l 441E
  \l 441F
  \l 4420
  \l 4421
  \l 4422
  \l 4423
  \l 4424
  \l 4425
  \l 4426
  \l 4427
  \l 4428
  \l 4429
  \l 442A
  \l 442B
  \l 442C
  \l 442D
  \l 442E
  \l 442F
  \l 4430
  \l 4431
  \l 4432
  \l 4433
  \l 4434
  \l 4435
  \l 4436
  \l 4437
  \l 4438
  \l 4439
  \l 443A
  \l 443B
  \l 443C
  \l 443D
  \l 443E
  \l 443F
  \l 4440
  \l 4441
  \l 4442
  \l 4443
  \l 4444
  \l 4445
  \l 4446
  \l 4447
  \l 4448
  \l 4449
  \l 444A
  \l 444B
  \l 444C
  \l 444D
  \l 444E
  \l 444F
  \l 4450
  \l 4451
  \l 4452
  \l 4453
  \l 4454
  \l 4455
  \l 4456
  \l 4457
  \l 4458
  \l 4459
  \l 445A
  \l 445B
  \l 445C
  \l 445D
  \l 445E
  \l 445F
  \l 4460
  \l 4461
  \l 4462
  \l 4463
  \l 4464
  \l 4465
  \l 4466
  \l 4467
  \l 4468
  \l 4469
  \l 446A
  \l 446B
  \l 446C
  \l 446D
  \l 446E
  \l 446F
  \l 4470
  \l 4471
  \l 4472
  \l 4473
  \l 4474
  \l 4475
  \l 4476
  \l 4477
  \l 4478
  \l 4479
  \l 447A
  \l 447B
  \l 447C
  \l 447D
  \l 447E
  \l 447F
  \l 4480
  \l 4481
  \l 4482
  \l 4483
  \l 4484
  \l 4485
  \l 4486
  \l 4487
  \l 4488
  \l 4489
  \l 448A
  \l 448B
  \l 448C
  \l 448D
  \l 448E
  \l 448F
  \l 4490
  \l 4491
  \l 4492
  \l 4493
  \l 4494
  \l 4495
  \l 4496
  \l 4497
  \l 4498
  \l 4499
  \l 449A
  \l 449B
  \l 449C
  \l 449D
  \l 449E
  \l 449F
  \l 44A0
  \l 44A1
  \l 44A2
  \l 44A3
  \l 44A4
  \l 44A5
  \l 44A6
  \l 44A7
  \l 44A8
  \l 44A9
  \l 44AA
  \l 44AB
  \l 44AC
  \l 44AD
  \l 44AE
  \l 44AF
  \l 44B0
  \l 44B1
  \l 44B2
  \l 44B3
  \l 44B4
  \l 44B5
  \l 44B6
  \l 44B7
  \l 44B8
  \l 44B9
  \l 44BA
  \l 44BB
  \l 44BC
  \l 44BD
  \l 44BE
  \l 44BF
  \l 44C0
  \l 44C1
  \l 44C2
  \l 44C3
  \l 44C4
  \l 44C5
  \l 44C6
  \l 44C7
  \l 44C8
  \l 44C9
  \l 44CA
  \l 44CB
  \l 44CC
  \l 44CD
  \l 44CE
  \l 44CF
  \l 44D0
  \l 44D1
  \l 44D2
  \l 44D3
  \l 44D4
  \l 44D5
  \l 44D6
  \l 44D7
  \l 44D8
  \l 44D9
  \l 44DA
  \l 44DB
  \l 44DC
  \l 44DD
  \l 44DE
  \l 44DF
  \l 44E0
  \l 44E1
  \l 44E2
  \l 44E3
  \l 44E4
  \l 44E5
  \l 44E6
  \l 44E7
  \l 44E8
  \l 44E9
  \l 44EA
  \l 44EB
  \l 44EC
  \l 44ED
  \l 44EE
  \l 44EF
  \l 44F0
  \l 44F1
  \l 44F2
  \l 44F3
  \l 44F4
  \l 44F5
  \l 44F6
  \l 44F7
  \l 44F8
  \l 44F9
  \l 44FA
  \l 44FB
  \l 44FC
  \l 44FD
  \l 44FE
  \l 44FF
  \l 4500
  \l 4501
  \l 4502
  \l 4503
  \l 4504
  \l 4505
  \l 4506
  \l 4507
  \l 4508
  \l 4509
  \l 450A
  \l 450B
  \l 450C
  \l 450D
  \l 450E
  \l 450F
  \l 4510
  \l 4511
  \l 4512
  \l 4513
  \l 4514
  \l 4515
  \l 4516
  \l 4517
  \l 4518
  \l 4519
  \l 451A
  \l 451B
  \l 451C
  \l 451D
  \l 451E
  \l 451F
  \l 4520
  \l 4521
  \l 4522
  \l 4523
  \l 4524
  \l 4525
  \l 4526
  \l 4527
  \l 4528
  \l 4529
  \l 452A
  \l 452B
  \l 452C
  \l 452D
  \l 452E
  \l 452F
  \l 4530
  \l 4531
  \l 4532
  \l 4533
  \l 4534
  \l 4535
  \l 4536
  \l 4537
  \l 4538
  \l 4539
  \l 453A
  \l 453B
  \l 453C
  \l 453D
  \l 453E
  \l 453F
  \l 4540
  \l 4541
  \l 4542
  \l 4543
  \l 4544
  \l 4545
  \l 4546
  \l 4547
  \l 4548
  \l 4549
  \l 454A
  \l 454B
  \l 454C
  \l 454D
  \l 454E
  \l 454F
  \l 4550
  \l 4551
  \l 4552
  \l 4553
  \l 4554
  \l 4555
  \l 4556
  \l 4557
  \l 4558
  \l 4559
  \l 455A
  \l 455B
  \l 455C
  \l 455D
  \l 455E
  \l 455F
  \l 4560
  \l 4561
  \l 4562
  \l 4563
  \l 4564
  \l 4565
  \l 4566
  \l 4567
  \l 4568
  \l 4569
  \l 456A
  \l 456B
  \l 456C
  \l 456D
  \l 456E
  \l 456F
  \l 4570
  \l 4571
  \l 4572
  \l 4573
  \l 4574
  \l 4575
  \l 4576
  \l 4577
  \l 4578
  \l 4579
  \l 457A
  \l 457B
  \l 457C
  \l 457D
  \l 457E
  \l 457F
  \l 4580
  \l 4581
  \l 4582
  \l 4583
  \l 4584
  \l 4585
  \l 4586
  \l 4587
  \l 4588
  \l 4589
  \l 458A
  \l 458B
  \l 458C
  \l 458D
  \l 458E
  \l 458F
  \l 4590
  \l 4591
  \l 4592
  \l 4593
  \l 4594
  \l 4595
  \l 4596
  \l 4597
  \l 4598
  \l 4599
  \l 459A
  \l 459B
  \l 459C
  \l 459D
  \l 459E
  \l 459F
  \l 45A0
  \l 45A1
  \l 45A2
  \l 45A3
  \l 45A4
  \l 45A5
  \l 45A6
  \l 45A7
  \l 45A8
  \l 45A9
  \l 45AA
  \l 45AB
  \l 45AC
  \l 45AD
  \l 45AE
  \l 45AF
  \l 45B0
  \l 45B1
  \l 45B2
  \l 45B3
  \l 45B4
  \l 45B5
  \l 45B6
  \l 45B7
  \l 45B8
  \l 45B9
  \l 45BA
  \l 45BB
  \l 45BC
  \l 45BD
  \l 45BE
  \l 45BF
  \l 45C0
  \l 45C1
  \l 45C2
  \l 45C3
  \l 45C4
  \l 45C5
  \l 45C6
  \l 45C7
  \l 45C8
  \l 45C9
  \l 45CA
  \l 45CB
  \l 45CC
  \l 45CD
  \l 45CE
  \l 45CF
  \l 45D0
  \l 45D1
  \l 45D2
  \l 45D3
  \l 45D4
  \l 45D5
  \l 45D6
  \l 45D7
  \l 45D8
  \l 45D9
  \l 45DA
  \l 45DB
  \l 45DC
  \l 45DD
  \l 45DE
  \l 45DF
  \l 45E0
  \l 45E1
  \l 45E2
  \l 45E3
  \l 45E4
  \l 45E5
  \l 45E6
  \l 45E7
  \l 45E8
  \l 45E9
  \l 45EA
  \l 45EB
  \l 45EC
  \l 45ED
  \l 45EE
  \l 45EF
  \l 45F0
  \l 45F1
  \l 45F2
  \l 45F3
  \l 45F4
  \l 45F5
  \l 45F6
  \l 45F7
  \l 45F8
  \l 45F9
  \l 45FA
  \l 45FB
  \l 45FC
  \l 45FD
  \l 45FE
  \l 45FF
  \l 4600
  \l 4601
  \l 4602
  \l 4603
  \l 4604
  \l 4605
  \l 4606
  \l 4607
  \l 4608
  \l 4609
  \l 460A
  \l 460B
  \l 460C
  \l 460D
  \l 460E
  \l 460F
  \l 4610
  \l 4611
  \l 4612
  \l 4613
  \l 4614
  \l 4615
  \l 4616
  \l 4617
  \l 4618
  \l 4619
  \l 461A
  \l 461B
  \l 461C
  \l 461D
  \l 461E
  \l 461F
  \l 4620
  \l 4621
  \l 4622
  \l 4623
  \l 4624
  \l 4625
  \l 4626
  \l 4627
  \l 4628
  \l 4629
  \l 462A
  \l 462B
  \l 462C
  \l 462D
  \l 462E
  \l 462F
  \l 4630
  \l 4631
  \l 4632
  \l 4633
  \l 4634
  \l 4635
  \l 4636
  \l 4637
  \l 4638
  \l 4639
  \l 463A
  \l 463B
  \l 463C
  \l 463D
  \l 463E
  \l 463F
  \l 4640
  \l 4641
  \l 4642
  \l 4643
  \l 4644
  \l 4645
  \l 4646
  \l 4647
  \l 4648
  \l 4649
  \l 464A
  \l 464B
  \l 464C
  \l 464D
  \l 464E
  \l 464F
  \l 4650
  \l 4651
  \l 4652
  \l 4653
  \l 4654
  \l 4655
  \l 4656
  \l 4657
  \l 4658
  \l 4659
  \l 465A
  \l 465B
  \l 465C
  \l 465D
  \l 465E
  \l 465F
  \l 4660
  \l 4661
  \l 4662
  \l 4663
  \l 4664
  \l 4665
  \l 4666
  \l 4667
  \l 4668
  \l 4669
  \l 466A
  \l 466B
  \l 466C
  \l 466D
  \l 466E
  \l 466F
  \l 4670
  \l 4671
  \l 4672
  \l 4673
  \l 4674
  \l 4675
  \l 4676
  \l 4677
  \l 4678
  \l 4679
  \l 467A
  \l 467B
  \l 467C
  \l 467D
  \l 467E
  \l 467F
  \l 4680
  \l 4681
  \l 4682
  \l 4683
  \l 4684
  \l 4685
  \l 4686
  \l 4687
  \l 4688
  \l 4689
  \l 468A
  \l 468B
  \l 468C
  \l 468D
  \l 468E
  \l 468F
  \l 4690
  \l 4691
  \l 4692
  \l 4693
  \l 4694
  \l 4695
  \l 4696
  \l 4697
  \l 4698
  \l 4699
  \l 469A
  \l 469B
  \l 469C
  \l 469D
  \l 469E
  \l 469F
  \l 46A0
  \l 46A1
  \l 46A2
  \l 46A3
  \l 46A4
  \l 46A5
  \l 46A6
  \l 46A7
  \l 46A8
  \l 46A9
  \l 46AA
  \l 46AB
  \l 46AC
  \l 46AD
  \l 46AE
  \l 46AF
  \l 46B0
  \l 46B1
  \l 46B2
  \l 46B3
  \l 46B4
  \l 46B5
  \l 46B6
  \l 46B7
  \l 46B8
  \l 46B9
  \l 46BA
  \l 46BB
  \l 46BC
  \l 46BD
  \l 46BE
  \l 46BF
  \l 46C0
  \l 46C1
  \l 46C2
  \l 46C3
  \l 46C4
  \l 46C5
  \l 46C6
  \l 46C7
  \l 46C8
  \l 46C9
  \l 46CA
  \l 46CB
  \l 46CC
  \l 46CD
  \l 46CE
  \l 46CF
  \l 46D0
  \l 46D1
  \l 46D2
  \l 46D3
  \l 46D4
  \l 46D5
  \l 46D6
  \l 46D7
  \l 46D8
  \l 46D9
  \l 46DA
  \l 46DB
  \l 46DC
  \l 46DD
  \l 46DE
  \l 46DF
  \l 46E0
  \l 46E1
  \l 46E2
  \l 46E3
  \l 46E4
  \l 46E5
  \l 46E6
  \l 46E7
  \l 46E8
  \l 46E9
  \l 46EA
  \l 46EB
  \l 46EC
  \l 46ED
  \l 46EE
  \l 46EF
  \l 46F0
  \l 46F1
  \l 46F2
  \l 46F3
  \l 46F4
  \l 46F5
  \l 46F6
  \l 46F7
  \l 46F8
  \l 46F9
  \l 46FA
  \l 46FB
  \l 46FC
  \l 46FD
  \l 46FE
  \l 46FF
  \l 4700
  \l 4701
  \l 4702
  \l 4703
  \l 4704
  \l 4705
  \l 4706
  \l 4707
  \l 4708
  \l 4709
  \l 470A
  \l 470B
  \l 470C
  \l 470D
  \l 470E
  \l 470F
  \l 4710
  \l 4711
  \l 4712
  \l 4713
  \l 4714
  \l 4715
  \l 4716
  \l 4717
  \l 4718
  \l 4719
  \l 471A
  \l 471B
  \l 471C
  \l 471D
  \l 471E
  \l 471F
  \l 4720
  \l 4721
  \l 4722
  \l 4723
  \l 4724
  \l 4725
  \l 4726
  \l 4727
  \l 4728
  \l 4729
  \l 472A
  \l 472B
  \l 472C
  \l 472D
  \l 472E
  \l 472F
  \l 4730
  \l 4731
  \l 4732
  \l 4733
  \l 4734
  \l 4735
  \l 4736
  \l 4737
  \l 4738
  \l 4739
  \l 473A
  \l 473B
  \l 473C
  \l 473D
  \l 473E
  \l 473F
  \l 4740
  \l 4741
  \l 4742
  \l 4743
  \l 4744
  \l 4745
  \l 4746
  \l 4747
  \l 4748
  \l 4749
  \l 474A
  \l 474B
  \l 474C
  \l 474D
  \l 474E
  \l 474F
  \l 4750
  \l 4751
  \l 4752
  \l 4753
  \l 4754
  \l 4755
  \l 4756
  \l 4757
  \l 4758
  \l 4759
  \l 475A
  \l 475B
  \l 475C
  \l 475D
  \l 475E
  \l 475F
  \l 4760
  \l 4761
  \l 4762
  \l 4763
  \l 4764
  \l 4765
  \l 4766
  \l 4767
  \l 4768
  \l 4769
  \l 476A
  \l 476B
  \l 476C
  \l 476D
  \l 476E
  \l 476F
  \l 4770
  \l 4771
  \l 4772
  \l 4773
  \l 4774
  \l 4775
  \l 4776
  \l 4777
  \l 4778
  \l 4779
  \l 477A
  \l 477B
  \l 477C
  \l 477D
  \l 477E
  \l 477F
  \l 4780
  \l 4781
  \l 4782
  \l 4783
  \l 4784
  \l 4785
  \l 4786
  \l 4787
  \l 4788
  \l 4789
  \l 478A
  \l 478B
  \l 478C
  \l 478D
  \l 478E
  \l 478F
  \l 4790
  \l 4791
  \l 4792
  \l 4793
  \l 4794
  \l 4795
  \l 4796
  \l 4797
  \l 4798
  \l 4799
  \l 479A
  \l 479B
  \l 479C
  \l 479D
  \l 479E
  \l 479F
  \l 47A0
  \l 47A1
  \l 47A2
  \l 47A3
  \l 47A4
  \l 47A5
  \l 47A6
  \l 47A7
  \l 47A8
  \l 47A9
  \l 47AA
  \l 47AB
  \l 47AC
  \l 47AD
  \l 47AE
  \l 47AF
  \l 47B0
  \l 47B1
  \l 47B2
  \l 47B3
  \l 47B4
  \l 47B5
  \l 47B6
  \l 47B7
  \l 47B8
  \l 47B9
  \l 47BA
  \l 47BB
  \l 47BC
  \l 47BD
  \l 47BE
  \l 47BF
  \l 47C0
  \l 47C1
  \l 47C2
  \l 47C3
  \l 47C4
  \l 47C5
  \l 47C6
  \l 47C7
  \l 47C8
  \l 47C9
  \l 47CA
  \l 47CB
  \l 47CC
  \l 47CD
  \l 47CE
  \l 47CF
  \l 47D0
  \l 47D1
  \l 47D2
  \l 47D3
  \l 47D4
  \l 47D5
  \l 47D6
  \l 47D7
  \l 47D8
  \l 47D9
  \l 47DA
  \l 47DB
  \l 47DC
  \l 47DD
  \l 47DE
  \l 47DF
  \l 47E0
  \l 47E1
  \l 47E2
  \l 47E3
  \l 47E4
  \l 47E5
  \l 47E6
  \l 47E7
  \l 47E8
  \l 47E9
  \l 47EA
  \l 47EB
  \l 47EC
  \l 47ED
  \l 47EE
  \l 47EF
  \l 47F0
  \l 47F1
  \l 47F2
  \l 47F3
  \l 47F4
  \l 47F5
  \l 47F6
  \l 47F7
  \l 47F8
  \l 47F9
  \l 47FA
  \l 47FB
  \l 47FC
  \l 47FD
  \l 47FE
  \l 47FF
  \l 4800
  \l 4801
  \l 4802
  \l 4803
  \l 4804
  \l 4805
  \l 4806
  \l 4807
  \l 4808
  \l 4809
  \l 480A
  \l 480B
  \l 480C
  \l 480D
  \l 480E
  \l 480F
  \l 4810
  \l 4811
  \l 4812
  \l 4813
  \l 4814
  \l 4815
  \l 4816
  \l 4817
  \l 4818
  \l 4819
  \l 481A
  \l 481B
  \l 481C
  \l 481D
  \l 481E
  \l 481F
  \l 4820
  \l 4821
  \l 4822
  \l 4823
  \l 4824
  \l 4825
  \l 4826
  \l 4827
  \l 4828
  \l 4829
  \l 482A
  \l 482B
  \l 482C
  \l 482D
  \l 482E
  \l 482F
  \l 4830
  \l 4831
  \l 4832
  \l 4833
  \l 4834
  \l 4835
  \l 4836
  \l 4837
  \l 4838
  \l 4839
  \l 483A
  \l 483B
  \l 483C
  \l 483D
  \l 483E
  \l 483F
  \l 4840
  \l 4841
  \l 4842
  \l 4843
  \l 4844
  \l 4845
  \l 4846
  \l 4847
  \l 4848
  \l 4849
  \l 484A
  \l 484B
  \l 484C
  \l 484D
  \l 484E
  \l 484F
  \l 4850
  \l 4851
  \l 4852
  \l 4853
  \l 4854
  \l 4855
  \l 4856
  \l 4857
  \l 4858
  \l 4859
  \l 485A
  \l 485B
  \l 485C
  \l 485D
  \l 485E
  \l 485F
  \l 4860
  \l 4861
  \l 4862
  \l 4863
  \l 4864
  \l 4865
  \l 4866
  \l 4867
  \l 4868
  \l 4869
  \l 486A
  \l 486B
  \l 486C
  \l 486D
  \l 486E
  \l 486F
  \l 4870
  \l 4871
  \l 4872
  \l 4873
  \l 4874
  \l 4875
  \l 4876
  \l 4877
  \l 4878
  \l 4879
  \l 487A
  \l 487B
  \l 487C
  \l 487D
  \l 487E
  \l 487F
  \l 4880
  \l 4881
  \l 4882
  \l 4883
  \l 4884
  \l 4885
  \l 4886
  \l 4887
  \l 4888
  \l 4889
  \l 488A
  \l 488B
  \l 488C
  \l 488D
  \l 488E
  \l 488F
  \l 4890
  \l 4891
  \l 4892
  \l 4893
  \l 4894
  \l 4895
  \l 4896
  \l 4897
  \l 4898
  \l 4899
  \l 489A
  \l 489B
  \l 489C
  \l 489D
  \l 489E
  \l 489F
  \l 48A0
  \l 48A1
  \l 48A2
  \l 48A3
  \l 48A4
  \l 48A5
  \l 48A6
  \l 48A7
  \l 48A8
  \l 48A9
  \l 48AA
  \l 48AB
  \l 48AC
  \l 48AD
  \l 48AE
  \l 48AF
  \l 48B0
  \l 48B1
  \l 48B2
  \l 48B3
  \l 48B4
  \l 48B5
  \l 48B6
  \l 48B7
  \l 48B8
  \l 48B9
  \l 48BA
  \l 48BB
  \l 48BC
  \l 48BD
  \l 48BE
  \l 48BF
  \l 48C0
  \l 48C1
  \l 48C2
  \l 48C3
  \l 48C4
  \l 48C5
  \l 48C6
  \l 48C7
  \l 48C8
  \l 48C9
  \l 48CA
  \l 48CB
  \l 48CC
  \l 48CD
  \l 48CE
  \l 48CF
  \l 48D0
  \l 48D1
  \l 48D2
  \l 48D3
  \l 48D4
  \l 48D5
  \l 48D6
  \l 48D7
  \l 48D8
  \l 48D9
  \l 48DA
  \l 48DB
  \l 48DC
  \l 48DD
  \l 48DE
  \l 48DF
  \l 48E0
  \l 48E1
  \l 48E2
  \l 48E3
  \l 48E4
  \l 48E5
  \l 48E6
  \l 48E7
  \l 48E8
  \l 48E9
  \l 48EA
  \l 48EB
  \l 48EC
  \l 48ED
  \l 48EE
  \l 48EF
  \l 48F0
  \l 48F1
  \l 48F2
  \l 48F3
  \l 48F4
  \l 48F5
  \l 48F6
  \l 48F7
  \l 48F8
  \l 48F9
  \l 48FA
  \l 48FB
  \l 48FC
  \l 48FD
  \l 48FE
  \l 48FF
  \l 4900
  \l 4901
  \l 4902
  \l 4903
  \l 4904
  \l 4905
  \l 4906
  \l 4907
  \l 4908
  \l 4909
  \l 490A
  \l 490B
  \l 490C
  \l 490D
  \l 490E
  \l 490F
  \l 4910
  \l 4911
  \l 4912
  \l 4913
  \l 4914
  \l 4915
  \l 4916
  \l 4917
  \l 4918
  \l 4919
  \l 491A
  \l 491B
  \l 491C
  \l 491D
  \l 491E
  \l 491F
  \l 4920
  \l 4921
  \l 4922
  \l 4923
  \l 4924
  \l 4925
  \l 4926
  \l 4927
  \l 4928
  \l 4929
  \l 492A
  \l 492B
  \l 492C
  \l 492D
  \l 492E
  \l 492F
  \l 4930
  \l 4931
  \l 4932
  \l 4933
  \l 4934
  \l 4935
  \l 4936
  \l 4937
  \l 4938
  \l 4939
  \l 493A
  \l 493B
  \l 493C
  \l 493D
  \l 493E
  \l 493F
  \l 4940
  \l 4941
  \l 4942
  \l 4943
  \l 4944
  \l 4945
  \l 4946
  \l 4947
  \l 4948
  \l 4949
  \l 494A
  \l 494B
  \l 494C
  \l 494D
  \l 494E
  \l 494F
  \l 4950
  \l 4951
  \l 4952
  \l 4953
  \l 4954
  \l 4955
  \l 4956
  \l 4957
  \l 4958
  \l 4959
  \l 495A
  \l 495B
  \l 495C
  \l 495D
  \l 495E
  \l 495F
  \l 4960
  \l 4961
  \l 4962
  \l 4963
  \l 4964
  \l 4965
  \l 4966
  \l 4967
  \l 4968
  \l 4969
  \l 496A
  \l 496B
  \l 496C
  \l 496D
  \l 496E
  \l 496F
  \l 4970
  \l 4971
  \l 4972
  \l 4973
  \l 4974
  \l 4975
  \l 4976
  \l 4977
  \l 4978
  \l 4979
  \l 497A
  \l 497B
  \l 497C
  \l 497D
  \l 497E
  \l 497F
  \l 4980
  \l 4981
  \l 4982
  \l 4983
  \l 4984
  \l 4985
  \l 4986
  \l 4987
  \l 4988
  \l 4989
  \l 498A
  \l 498B
  \l 498C
  \l 498D
  \l 498E
  \l 498F
  \l 4990
  \l 4991
  \l 4992
  \l 4993
  \l 4994
  \l 4995
  \l 4996
  \l 4997
  \l 4998
  \l 4999
  \l 499A
  \l 499B
  \l 499C
  \l 499D
  \l 499E
  \l 499F
  \l 49A0
  \l 49A1
  \l 49A2
  \l 49A3
  \l 49A4
  \l 49A5
  \l 49A6
  \l 49A7
  \l 49A8
  \l 49A9
  \l 49AA
  \l 49AB
  \l 49AC
  \l 49AD
  \l 49AE
  \l 49AF
  \l 49B0
  \l 49B1
  \l 49B2
  \l 49B3
  \l 49B4
  \l 49B5
  \l 49B6
  \l 49B7
  \l 49B8
  \l 49B9
  \l 49BA
  \l 49BB
  \l 49BC
  \l 49BD
  \l 49BE
  \l 49BF
  \l 49C0
  \l 49C1
  \l 49C2
  \l 49C3
  \l 49C4
  \l 49C5
  \l 49C6
  \l 49C7
  \l 49C8
  \l 49C9
  \l 49CA
  \l 49CB
  \l 49CC
  \l 49CD
  \l 49CE
  \l 49CF
  \l 49D0
  \l 49D1
  \l 49D2
  \l 49D3
  \l 49D4
  \l 49D5
  \l 49D6
  \l 49D7
  \l 49D8
  \l 49D9
  \l 49DA
  \l 49DB
  \l 49DC
  \l 49DD
  \l 49DE
  \l 49DF
  \l 49E0
  \l 49E1
  \l 49E2
  \l 49E3
  \l 49E4
  \l 49E5
  \l 49E6
  \l 49E7
  \l 49E8
  \l 49E9
  \l 49EA
  \l 49EB
  \l 49EC
  \l 49ED
  \l 49EE
  \l 49EF
  \l 49F0
  \l 49F1
  \l 49F2
  \l 49F3
  \l 49F4
  \l 49F5
  \l 49F6
  \l 49F7
  \l 49F8
  \l 49F9
  \l 49FA
  \l 49FB
  \l 49FC
  \l 49FD
  \l 49FE
  \l 49FF
  \l 4A00
  \l 4A01
  \l 4A02
  \l 4A03
  \l 4A04
  \l 4A05
  \l 4A06
  \l 4A07
  \l 4A08
  \l 4A09
  \l 4A0A
  \l 4A0B
  \l 4A0C
  \l 4A0D
  \l 4A0E
  \l 4A0F
  \l 4A10
  \l 4A11
  \l 4A12
  \l 4A13
  \l 4A14
  \l 4A15
  \l 4A16
  \l 4A17
  \l 4A18
  \l 4A19
  \l 4A1A
  \l 4A1B
  \l 4A1C
  \l 4A1D
  \l 4A1E
  \l 4A1F
  \l 4A20
  \l 4A21
  \l 4A22
  \l 4A23
  \l 4A24
  \l 4A25
  \l 4A26
  \l 4A27
  \l 4A28
  \l 4A29
  \l 4A2A
  \l 4A2B
  \l 4A2C
  \l 4A2D
  \l 4A2E
  \l 4A2F
  \l 4A30
  \l 4A31
  \l 4A32
  \l 4A33
  \l 4A34
  \l 4A35
  \l 4A36
  \l 4A37
  \l 4A38
  \l 4A39
  \l 4A3A
  \l 4A3B
  \l 4A3C
  \l 4A3D
  \l 4A3E
  \l 4A3F
  \l 4A40
  \l 4A41
  \l 4A42
  \l 4A43
  \l 4A44
  \l 4A45
  \l 4A46
  \l 4A47
  \l 4A48
  \l 4A49
  \l 4A4A
  \l 4A4B
  \l 4A4C
  \l 4A4D
  \l 4A4E
  \l 4A4F
  \l 4A50
  \l 4A51
  \l 4A52
  \l 4A53
  \l 4A54
  \l 4A55
  \l 4A56
  \l 4A57
  \l 4A58
  \l 4A59
  \l 4A5A
  \l 4A5B
  \l 4A5C
  \l 4A5D
  \l 4A5E
  \l 4A5F
  \l 4A60
  \l 4A61
  \l 4A62
  \l 4A63
  \l 4A64
  \l 4A65
  \l 4A66
  \l 4A67
  \l 4A68
  \l 4A69
  \l 4A6A
  \l 4A6B
  \l 4A6C
  \l 4A6D
  \l 4A6E
  \l 4A6F
  \l 4A70
  \l 4A71
  \l 4A72
  \l 4A73
  \l 4A74
  \l 4A75
  \l 4A76
  \l 4A77
  \l 4A78
  \l 4A79
  \l 4A7A
  \l 4A7B
  \l 4A7C
  \l 4A7D
  \l 4A7E
  \l 4A7F
  \l 4A80
  \l 4A81
  \l 4A82
  \l 4A83
  \l 4A84
  \l 4A85
  \l 4A86
  \l 4A87
  \l 4A88
  \l 4A89
  \l 4A8A
  \l 4A8B
  \l 4A8C
  \l 4A8D
  \l 4A8E
  \l 4A8F
  \l 4A90
  \l 4A91
  \l 4A92
  \l 4A93
  \l 4A94
  \l 4A95
  \l 4A96
  \l 4A97
  \l 4A98
  \l 4A99
  \l 4A9A
  \l 4A9B
  \l 4A9C
  \l 4A9D
  \l 4A9E
  \l 4A9F
  \l 4AA0
  \l 4AA1
  \l 4AA2
  \l 4AA3
  \l 4AA4
  \l 4AA5
  \l 4AA6
  \l 4AA7
  \l 4AA8
  \l 4AA9
  \l 4AAA
  \l 4AAB
  \l 4AAC
  \l 4AAD
  \l 4AAE
  \l 4AAF
  \l 4AB0
  \l 4AB1
  \l 4AB2
  \l 4AB3
  \l 4AB4
  \l 4AB5
  \l 4AB6
  \l 4AB7
  \l 4AB8
  \l 4AB9
  \l 4ABA
  \l 4ABB
  \l 4ABC
  \l 4ABD
  \l 4ABE
  \l 4ABF
  \l 4AC0
  \l 4AC1
  \l 4AC2
  \l 4AC3
  \l 4AC4
  \l 4AC5
  \l 4AC6
  \l 4AC7
  \l 4AC8
  \l 4AC9
  \l 4ACA
  \l 4ACB
  \l 4ACC
  \l 4ACD
  \l 4ACE
  \l 4ACF
  \l 4AD0
  \l 4AD1
  \l 4AD2
  \l 4AD3
  \l 4AD4
  \l 4AD5
  \l 4AD6
  \l 4AD7
  \l 4AD8
  \l 4AD9
  \l 4ADA
  \l 4ADB
  \l 4ADC
  \l 4ADD
  \l 4ADE
  \l 4ADF
  \l 4AE0
  \l 4AE1
  \l 4AE2
  \l 4AE3
  \l 4AE4
  \l 4AE5
  \l 4AE6
  \l 4AE7
  \l 4AE8
  \l 4AE9
  \l 4AEA
  \l 4AEB
  \l 4AEC
  \l 4AED
  \l 4AEE
  \l 4AEF
  \l 4AF0
  \l 4AF1
  \l 4AF2
  \l 4AF3
  \l 4AF4
  \l 4AF5
  \l 4AF6
  \l 4AF7
  \l 4AF8
  \l 4AF9
  \l 4AFA
  \l 4AFB
  \l 4AFC
  \l 4AFD
  \l 4AFE
  \l 4AFF
  \l 4B00
  \l 4B01
  \l 4B02
  \l 4B03
  \l 4B04
  \l 4B05
  \l 4B06
  \l 4B07
  \l 4B08
  \l 4B09
  \l 4B0A
  \l 4B0B
  \l 4B0C
  \l 4B0D
  \l 4B0E
  \l 4B0F
  \l 4B10
  \l 4B11
  \l 4B12
  \l 4B13
  \l 4B14
  \l 4B15
  \l 4B16
  \l 4B17
  \l 4B18
  \l 4B19
  \l 4B1A
  \l 4B1B
  \l 4B1C
  \l 4B1D
  \l 4B1E
  \l 4B1F
  \l 4B20
  \l 4B21
  \l 4B22
  \l 4B23
  \l 4B24
  \l 4B25
  \l 4B26
  \l 4B27
  \l 4B28
  \l 4B29
  \l 4B2A
  \l 4B2B
  \l 4B2C
  \l 4B2D
  \l 4B2E
  \l 4B2F
  \l 4B30
  \l 4B31
  \l 4B32
  \l 4B33
  \l 4B34
  \l 4B35
  \l 4B36
  \l 4B37
  \l 4B38
  \l 4B39
  \l 4B3A
  \l 4B3B
  \l 4B3C
  \l 4B3D
  \l 4B3E
  \l 4B3F
  \l 4B40
  \l 4B41
  \l 4B42
  \l 4B43
  \l 4B44
  \l 4B45
  \l 4B46
  \l 4B47
  \l 4B48
  \l 4B49
  \l 4B4A
  \l 4B4B
  \l 4B4C
  \l 4B4D
  \l 4B4E
  \l 4B4F
  \l 4B50
  \l 4B51
  \l 4B52
  \l 4B53
  \l 4B54
  \l 4B55
  \l 4B56
  \l 4B57
  \l 4B58
  \l 4B59
  \l 4B5A
  \l 4B5B
  \l 4B5C
  \l 4B5D
  \l 4B5E
  \l 4B5F
  \l 4B60
  \l 4B61
  \l 4B62
  \l 4B63
  \l 4B64
  \l 4B65
  \l 4B66
  \l 4B67
  \l 4B68
  \l 4B69
  \l 4B6A
  \l 4B6B
  \l 4B6C
  \l 4B6D
  \l 4B6E
  \l 4B6F
  \l 4B70
  \l 4B71
  \l 4B72
  \l 4B73
  \l 4B74
  \l 4B75
  \l 4B76
  \l 4B77
  \l 4B78
  \l 4B79
  \l 4B7A
  \l 4B7B
  \l 4B7C
  \l 4B7D
  \l 4B7E
  \l 4B7F
  \l 4B80
  \l 4B81
  \l 4B82
  \l 4B83
  \l 4B84
  \l 4B85
  \l 4B86
  \l 4B87
  \l 4B88
  \l 4B89
  \l 4B8A
  \l 4B8B
  \l 4B8C
  \l 4B8D
  \l 4B8E
  \l 4B8F
  \l 4B90
  \l 4B91
  \l 4B92
  \l 4B93
  \l 4B94
  \l 4B95
  \l 4B96
  \l 4B97
  \l 4B98
  \l 4B99
  \l 4B9A
  \l 4B9B
  \l 4B9C
  \l 4B9D
  \l 4B9E
  \l 4B9F
  \l 4BA0
  \l 4BA1
  \l 4BA2
  \l 4BA3
  \l 4BA4
  \l 4BA5
  \l 4BA6
  \l 4BA7
  \l 4BA8
  \l 4BA9
  \l 4BAA
  \l 4BAB
  \l 4BAC
  \l 4BAD
  \l 4BAE
  \l 4BAF
  \l 4BB0
  \l 4BB1
  \l 4BB2
  \l 4BB3
  \l 4BB4
  \l 4BB5
  \l 4BB6
  \l 4BB7
  \l 4BB8
  \l 4BB9
  \l 4BBA
  \l 4BBB
  \l 4BBC
  \l 4BBD
  \l 4BBE
  \l 4BBF
  \l 4BC0
  \l 4BC1
  \l 4BC2
  \l 4BC3
  \l 4BC4
  \l 4BC5
  \l 4BC6
  \l 4BC7
  \l 4BC8
  \l 4BC9
  \l 4BCA
  \l 4BCB
  \l 4BCC
  \l 4BCD
  \l 4BCE
  \l 4BCF
  \l 4BD0
  \l 4BD1
  \l 4BD2
  \l 4BD3
  \l 4BD4
  \l 4BD5
  \l 4BD6
  \l 4BD7
  \l 4BD8
  \l 4BD9
  \l 4BDA
  \l 4BDB
  \l 4BDC
  \l 4BDD
  \l 4BDE
  \l 4BDF
  \l 4BE0
  \l 4BE1
  \l 4BE2
  \l 4BE3
  \l 4BE4
  \l 4BE5
  \l 4BE6
  \l 4BE7
  \l 4BE8
  \l 4BE9
  \l 4BEA
  \l 4BEB
  \l 4BEC
  \l 4BED
  \l 4BEE
  \l 4BEF
  \l 4BF0
  \l 4BF1
  \l 4BF2
  \l 4BF3
  \l 4BF4
  \l 4BF5
  \l 4BF6
  \l 4BF7
  \l 4BF8
  \l 4BF9
  \l 4BFA
  \l 4BFB
  \l 4BFC
  \l 4BFD
  \l 4BFE
  \l 4BFF
  \l 4C00
  \l 4C01
  \l 4C02
  \l 4C03
  \l 4C04
  \l 4C05
  \l 4C06
  \l 4C07
  \l 4C08
  \l 4C09
  \l 4C0A
  \l 4C0B
  \l 4C0C
  \l 4C0D
  \l 4C0E
  \l 4C0F
  \l 4C10
  \l 4C11
  \l 4C12
  \l 4C13
  \l 4C14
  \l 4C15
  \l 4C16
  \l 4C17
  \l 4C18
  \l 4C19
  \l 4C1A
  \l 4C1B
  \l 4C1C
  \l 4C1D
  \l 4C1E
  \l 4C1F
  \l 4C20
  \l 4C21
  \l 4C22
  \l 4C23
  \l 4C24
  \l 4C25
  \l 4C26
  \l 4C27
  \l 4C28
  \l 4C29
  \l 4C2A
  \l 4C2B
  \l 4C2C
  \l 4C2D
  \l 4C2E
  \l 4C2F
  \l 4C30
  \l 4C31
  \l 4C32
  \l 4C33
  \l 4C34
  \l 4C35
  \l 4C36
  \l 4C37
  \l 4C38
  \l 4C39
  \l 4C3A
  \l 4C3B
  \l 4C3C
  \l 4C3D
  \l 4C3E
  \l 4C3F
  \l 4C40
  \l 4C41
  \l 4C42
  \l 4C43
  \l 4C44
  \l 4C45
  \l 4C46
  \l 4C47
  \l 4C48
  \l 4C49
  \l 4C4A
  \l 4C4B
  \l 4C4C
  \l 4C4D
  \l 4C4E
  \l 4C4F
  \l 4C50
  \l 4C51
  \l 4C52
  \l 4C53
  \l 4C54
  \l 4C55
  \l 4C56
  \l 4C57
  \l 4C58
  \l 4C59
  \l 4C5A
  \l 4C5B
  \l 4C5C
  \l 4C5D
  \l 4C5E
  \l 4C5F
  \l 4C60
  \l 4C61
  \l 4C62
  \l 4C63
  \l 4C64
  \l 4C65
  \l 4C66
  \l 4C67
  \l 4C68
  \l 4C69
  \l 4C6A
  \l 4C6B
  \l 4C6C
  \l 4C6D
  \l 4C6E
  \l 4C6F
  \l 4C70
  \l 4C71
  \l 4C72
  \l 4C73
  \l 4C74
  \l 4C75
  \l 4C76
  \l 4C77
  \l 4C78
  \l 4C79
  \l 4C7A
  \l 4C7B
  \l 4C7C
  \l 4C7D
  \l 4C7E
  \l 4C7F
  \l 4C80
  \l 4C81
  \l 4C82
  \l 4C83
  \l 4C84
  \l 4C85
  \l 4C86
  \l 4C87
  \l 4C88
  \l 4C89
  \l 4C8A
  \l 4C8B
  \l 4C8C
  \l 4C8D
  \l 4C8E
  \l 4C8F
  \l 4C90
  \l 4C91
  \l 4C92
  \l 4C93
  \l 4C94
  \l 4C95
  \l 4C96
  \l 4C97
  \l 4C98
  \l 4C99
  \l 4C9A
  \l 4C9B
  \l 4C9C
  \l 4C9D
  \l 4C9E
  \l 4C9F
  \l 4CA0
  \l 4CA1
  \l 4CA2
  \l 4CA3
  \l 4CA4
  \l 4CA5
  \l 4CA6
  \l 4CA7
  \l 4CA8
  \l 4CA9
  \l 4CAA
  \l 4CAB
  \l 4CAC
  \l 4CAD
  \l 4CAE
  \l 4CAF
  \l 4CB0
  \l 4CB1
  \l 4CB2
  \l 4CB3
  \l 4CB4
  \l 4CB5
  \l 4CB6
  \l 4CB7
  \l 4CB8
  \l 4CB9
  \l 4CBA
  \l 4CBB
  \l 4CBC
  \l 4CBD
  \l 4CBE
  \l 4CBF
  \l 4CC0
  \l 4CC1
  \l 4CC2
  \l 4CC3
  \l 4CC4
  \l 4CC5
  \l 4CC6
  \l 4CC7
  \l 4CC8
  \l 4CC9
  \l 4CCA
  \l 4CCB
  \l 4CCC
  \l 4CCD
  \l 4CCE
  \l 4CCF
  \l 4CD0
  \l 4CD1
  \l 4CD2
  \l 4CD3
  \l 4CD4
  \l 4CD5
  \l 4CD6
  \l 4CD7
  \l 4CD8
  \l 4CD9
  \l 4CDA
  \l 4CDB
  \l 4CDC
  \l 4CDD
  \l 4CDE
  \l 4CDF
  \l 4CE0
  \l 4CE1
  \l 4CE2
  \l 4CE3
  \l 4CE4
  \l 4CE5
  \l 4CE6
  \l 4CE7
  \l 4CE8
  \l 4CE9
  \l 4CEA
  \l 4CEB
  \l 4CEC
  \l 4CED
  \l 4CEE
  \l 4CEF
  \l 4CF0
  \l 4CF1
  \l 4CF2
  \l 4CF3
  \l 4CF4
  \l 4CF5
  \l 4CF6
  \l 4CF7
  \l 4CF8
  \l 4CF9
  \l 4CFA
  \l 4CFB
  \l 4CFC
  \l 4CFD
  \l 4CFE
  \l 4CFF
  \l 4D00
  \l 4D01
  \l 4D02
  \l 4D03
  \l 4D04
  \l 4D05
  \l 4D06
  \l 4D07
  \l 4D08
  \l 4D09
  \l 4D0A
  \l 4D0B
  \l 4D0C
  \l 4D0D
  \l 4D0E
  \l 4D0F
  \l 4D10
  \l 4D11
  \l 4D12
  \l 4D13
  \l 4D14
  \l 4D15
  \l 4D16
  \l 4D17
  \l 4D18
  \l 4D19
  \l 4D1A
  \l 4D1B
  \l 4D1C
  \l 4D1D
  \l 4D1E
  \l 4D1F
  \l 4D20
  \l 4D21
  \l 4D22
  \l 4D23
  \l 4D24
  \l 4D25
  \l 4D26
  \l 4D27
  \l 4D28
  \l 4D29
  \l 4D2A
  \l 4D2B
  \l 4D2C
  \l 4D2D
  \l 4D2E
  \l 4D2F
  \l 4D30
  \l 4D31
  \l 4D32
  \l 4D33
  \l 4D34
  \l 4D35
  \l 4D36
  \l 4D37
  \l 4D38
  \l 4D39
  \l 4D3A
  \l 4D3B
  \l 4D3C
  \l 4D3D
  \l 4D3E
  \l 4D3F
  \l 4D40
  \l 4D41
  \l 4D42
  \l 4D43
  \l 4D44
  \l 4D45
  \l 4D46
  \l 4D47
  \l 4D48
  \l 4D49
  \l 4D4A
  \l 4D4B
  \l 4D4C
  \l 4D4D
  \l 4D4E
  \l 4D4F
  \l 4D50
  \l 4D51
  \l 4D52
  \l 4D53
  \l 4D54
  \l 4D55
  \l 4D56
  \l 4D57
  \l 4D58
  \l 4D59
  \l 4D5A
  \l 4D5B
  \l 4D5C
  \l 4D5D
  \l 4D5E
  \l 4D5F
  \l 4D60
  \l 4D61
  \l 4D62
  \l 4D63
  \l 4D64
  \l 4D65
  \l 4D66
  \l 4D67
  \l 4D68
  \l 4D69
  \l 4D6A
  \l 4D6B
  \l 4D6C
  \l 4D6D
  \l 4D6E
  \l 4D6F
  \l 4D70
  \l 4D71
  \l 4D72
  \l 4D73
  \l 4D74
  \l 4D75
  \l 4D76
  \l 4D77
  \l 4D78
  \l 4D79
  \l 4D7A
  \l 4D7B
  \l 4D7C
  \l 4D7D
  \l 4D7E
  \l 4D7F
  \l 4D80
  \l 4D81
  \l 4D82
  \l 4D83
  \l 4D84
  \l 4D85
  \l 4D86
  \l 4D87
  \l 4D88
  \l 4D89
  \l 4D8A
  \l 4D8B
  \l 4D8C
  \l 4D8D
  \l 4D8E
  \l 4D8F
  \l 4D90
  \l 4D91
  \l 4D92
  \l 4D93
  \l 4D94
  \l 4D95
  \l 4D96
  \l 4D97
  \l 4D98
  \l 4D99
  \l 4D9A
  \l 4D9B
  \l 4D9C
  \l 4D9D
  \l 4D9E
  \l 4D9F
  \l 4DA0
  \l 4DA1
  \l 4DA2
  \l 4DA3
  \l 4DA4
  \l 4DA5
  \l 4DA6
  \l 4DA7
  \l 4DA8
  \l 4DA9
  \l 4DAA
  \l 4DAB
  \l 4DAC
  \l 4DAD
  \l 4DAE
  \l 4DAF
  \l 4DB0
  \l 4DB1
  \l 4DB2
  \l 4DB3
  \l 4DB4
  \l 4DB5
  \l 4E00
  \l 4E01
  \l 4E02
  \l 4E03
  \l 4E04
  \l 4E05
  \l 4E06
  \l 4E07
  \l 4E08
  \l 4E09
  \l 4E0A
  \l 4E0B
  \l 4E0C
  \l 4E0D
  \l 4E0E
  \l 4E0F
  \l 4E10
  \l 4E11
  \l 4E12
  \l 4E13
  \l 4E14
  \l 4E15
  \l 4E16
  \l 4E17
  \l 4E18
  \l 4E19
  \l 4E1A
  \l 4E1B
  \l 4E1C
  \l 4E1D
  \l 4E1E
  \l 4E1F
  \l 4E20
  \l 4E21
  \l 4E22
  \l 4E23
  \l 4E24
  \l 4E25
  \l 4E26
  \l 4E27
  \l 4E28
  \l 4E29
  \l 4E2A
  \l 4E2B
  \l 4E2C
  \l 4E2D
  \l 4E2E
  \l 4E2F
  \l 4E30
  \l 4E31
  \l 4E32
  \l 4E33
  \l 4E34
  \l 4E35
  \l 4E36
  \l 4E37
  \l 4E38
  \l 4E39
  \l 4E3A
  \l 4E3B
  \l 4E3C
  \l 4E3D
  \l 4E3E
  \l 4E3F
  \l 4E40
  \l 4E41
  \l 4E42
  \l 4E43
  \l 4E44
  \l 4E45
  \l 4E46
  \l 4E47
  \l 4E48
  \l 4E49
  \l 4E4A
  \l 4E4B
  \l 4E4C
  \l 4E4D
  \l 4E4E
  \l 4E4F
  \l 4E50
  \l 4E51
  \l 4E52
  \l 4E53
  \l 4E54
  \l 4E55
  \l 4E56
  \l 4E57
  \l 4E58
  \l 4E59
  \l 4E5A
  \l 4E5B
  \l 4E5C
  \l 4E5D
  \l 4E5E
  \l 4E5F
  \l 4E60
  \l 4E61
  \l 4E62
  \l 4E63
  \l 4E64
  \l 4E65
  \l 4E66
  \l 4E67
  \l 4E68
  \l 4E69
  \l 4E6A
  \l 4E6B
  \l 4E6C
  \l 4E6D
  \l 4E6E
  \l 4E6F
  \l 4E70
  \l 4E71
  \l 4E72
  \l 4E73
  \l 4E74
  \l 4E75
  \l 4E76
  \l 4E77
  \l 4E78
  \l 4E79
  \l 4E7A
  \l 4E7B
  \l 4E7C
  \l 4E7D
  \l 4E7E
  \l 4E7F
  \l 4E80
  \l 4E81
  \l 4E82
  \l 4E83
  \l 4E84
  \l 4E85
  \l 4E86
  \l 4E87
  \l 4E88
  \l 4E89
  \l 4E8A
  \l 4E8B
  \l 4E8C
  \l 4E8D
  \l 4E8E
  \l 4E8F
  \l 4E90
  \l 4E91
  \l 4E92
  \l 4E93
  \l 4E94
  \l 4E95
  \l 4E96
  \l 4E97
  \l 4E98
  \l 4E99
  \l 4E9A
  \l 4E9B
  \l 4E9C
  \l 4E9D
  \l 4E9E
  \l 4E9F
  \l 4EA0
  \l 4EA1
  \l 4EA2
  \l 4EA3
  \l 4EA4
  \l 4EA5
  \l 4EA6
  \l 4EA7
  \l 4EA8
  \l 4EA9
  \l 4EAA
  \l 4EAB
  \l 4EAC
  \l 4EAD
  \l 4EAE
  \l 4EAF
  \l 4EB0
  \l 4EB1
  \l 4EB2
  \l 4EB3
  \l 4EB4
  \l 4EB5
  \l 4EB6
  \l 4EB7
  \l 4EB8
  \l 4EB9
  \l 4EBA
  \l 4EBB
  \l 4EBC
  \l 4EBD
  \l 4EBE
  \l 4EBF
  \l 4EC0
  \l 4EC1
  \l 4EC2
  \l 4EC3
  \l 4EC4
  \l 4EC5
  \l 4EC6
  \l 4EC7
  \l 4EC8
  \l 4EC9
  \l 4ECA
  \l 4ECB
  \l 4ECC
  \l 4ECD
  \l 4ECE
  \l 4ECF
  \l 4ED0
  \l 4ED1
  \l 4ED2
  \l 4ED3
  \l 4ED4
  \l 4ED5
  \l 4ED6
  \l 4ED7
  \l 4ED8
  \l 4ED9
  \l 4EDA
  \l 4EDB
  \l 4EDC
  \l 4EDD
  \l 4EDE
  \l 4EDF
  \l 4EE0
  \l 4EE1
  \l 4EE2
  \l 4EE3
  \l 4EE4
  \l 4EE5
  \l 4EE6
  \l 4EE7
  \l 4EE8
  \l 4EE9
  \l 4EEA
  \l 4EEB
  \l 4EEC
  \l 4EED
  \l 4EEE
  \l 4EEF
  \l 4EF0
  \l 4EF1
  \l 4EF2
  \l 4EF3
  \l 4EF4
  \l 4EF5
  \l 4EF6
  \l 4EF7
  \l 4EF8
  \l 4EF9
  \l 4EFA
  \l 4EFB
  \l 4EFC
  \l 4EFD
  \l 4EFE
  \l 4EFF
  \l 4F00
  \l 4F01
  \l 4F02
  \l 4F03
  \l 4F04
  \l 4F05
  \l 4F06
  \l 4F07
  \l 4F08
  \l 4F09
  \l 4F0A
  \l 4F0B
  \l 4F0C
  \l 4F0D
  \l 4F0E
  \l 4F0F
  \l 4F10
  \l 4F11
  \l 4F12
  \l 4F13
  \l 4F14
  \l 4F15
  \l 4F16
  \l 4F17
  \l 4F18
  \l 4F19
  \l 4F1A
  \l 4F1B
  \l 4F1C
  \l 4F1D
  \l 4F1E
  \l 4F1F
  \l 4F20
  \l 4F21
  \l 4F22
  \l 4F23
  \l 4F24
  \l 4F25
  \l 4F26
  \l 4F27
  \l 4F28
  \l 4F29
  \l 4F2A
  \l 4F2B
  \l 4F2C
  \l 4F2D
  \l 4F2E
  \l 4F2F
  \l 4F30
  \l 4F31
  \l 4F32
  \l 4F33
  \l 4F34
  \l 4F35
  \l 4F36
  \l 4F37
  \l 4F38
  \l 4F39
  \l 4F3A
  \l 4F3B
  \l 4F3C
  \l 4F3D
  \l 4F3E
  \l 4F3F
  \l 4F40
  \l 4F41
  \l 4F42
  \l 4F43
  \l 4F44
  \l 4F45
  \l 4F46
  \l 4F47
  \l 4F48
  \l 4F49
  \l 4F4A
  \l 4F4B
  \l 4F4C
  \l 4F4D
  \l 4F4E
  \l 4F4F
  \l 4F50
  \l 4F51
  \l 4F52
  \l 4F53
  \l 4F54
  \l 4F55
  \l 4F56
  \l 4F57
  \l 4F58
  \l 4F59
  \l 4F5A
  \l 4F5B
  \l 4F5C
  \l 4F5D
  \l 4F5E
  \l 4F5F
  \l 4F60
  \l 4F61
  \l 4F62
  \l 4F63
  \l 4F64
  \l 4F65
  \l 4F66
  \l 4F67
  \l 4F68
  \l 4F69
  \l 4F6A
  \l 4F6B
  \l 4F6C
  \l 4F6D
  \l 4F6E
  \l 4F6F
  \l 4F70
  \l 4F71
  \l 4F72
  \l 4F73
  \l 4F74
  \l 4F75
  \l 4F76
  \l 4F77
  \l 4F78
  \l 4F79
  \l 4F7A
  \l 4F7B
  \l 4F7C
  \l 4F7D
  \l 4F7E
  \l 4F7F
  \l 4F80
  \l 4F81
  \l 4F82
  \l 4F83
  \l 4F84
  \l 4F85
  \l 4F86
  \l 4F87
  \l 4F88
  \l 4F89
  \l 4F8A
  \l 4F8B
  \l 4F8C
  \l 4F8D
  \l 4F8E
  \l 4F8F
  \l 4F90
  \l 4F91
  \l 4F92
  \l 4F93
  \l 4F94
  \l 4F95
  \l 4F96
  \l 4F97
  \l 4F98
  \l 4F99
  \l 4F9A
  \l 4F9B
  \l 4F9C
  \l 4F9D
  \l 4F9E
  \l 4F9F
  \l 4FA0
  \l 4FA1
  \l 4FA2
  \l 4FA3
  \l 4FA4
  \l 4FA5
  \l 4FA6
  \l 4FA7
  \l 4FA8
  \l 4FA9
  \l 4FAA
  \l 4FAB
  \l 4FAC
  \l 4FAD
  \l 4FAE
  \l 4FAF
  \l 4FB0
  \l 4FB1
  \l 4FB2
  \l 4FB3
  \l 4FB4
  \l 4FB5
  \l 4FB6
  \l 4FB7
  \l 4FB8
  \l 4FB9
  \l 4FBA
  \l 4FBB
  \l 4FBC
  \l 4FBD
  \l 4FBE
  \l 4FBF
  \l 4FC0
  \l 4FC1
  \l 4FC2
  \l 4FC3
  \l 4FC4
  \l 4FC5
  \l 4FC6
  \l 4FC7
  \l 4FC8
  \l 4FC9
  \l 4FCA
  \l 4FCB
  \l 4FCC
  \l 4FCD
  \l 4FCE
  \l 4FCF
  \l 4FD0
  \l 4FD1
  \l 4FD2
  \l 4FD3
  \l 4FD4
  \l 4FD5
  \l 4FD6
  \l 4FD7
  \l 4FD8
  \l 4FD9
  \l 4FDA
  \l 4FDB
  \l 4FDC
  \l 4FDD
  \l 4FDE
  \l 4FDF
  \l 4FE0
  \l 4FE1
  \l 4FE2
  \l 4FE3
  \l 4FE4
  \l 4FE5
  \l 4FE6
  \l 4FE7
  \l 4FE8
  \l 4FE9
  \l 4FEA
  \l 4FEB
  \l 4FEC
  \l 4FED
  \l 4FEE
  \l 4FEF
  \l 4FF0
  \l 4FF1
  \l 4FF2
  \l 4FF3
  \l 4FF4
  \l 4FF5
  \l 4FF6
  \l 4FF7
  \l 4FF8
  \l 4FF9
  \l 4FFA
  \l 4FFB
  \l 4FFC
  \l 4FFD
  \l 4FFE
  \l 4FFF
  \l 5000
  \l 5001
  \l 5002
  \l 5003
  \l 5004
  \l 5005
  \l 5006
  \l 5007
  \l 5008
  \l 5009
  \l 500A
  \l 500B
  \l 500C
  \l 500D
  \l 500E
  \l 500F
  \l 5010
  \l 5011
  \l 5012
  \l 5013
  \l 5014
  \l 5015
  \l 5016
  \l 5017
  \l 5018
  \l 5019
  \l 501A
  \l 501B
  \l 501C
  \l 501D
  \l 501E
  \l 501F
  \l 5020
  \l 5021
  \l 5022
  \l 5023
  \l 5024
  \l 5025
  \l 5026
  \l 5027
  \l 5028
  \l 5029
  \l 502A
  \l 502B
  \l 502C
  \l 502D
  \l 502E
  \l 502F
  \l 5030
  \l 5031
  \l 5032
  \l 5033
  \l 5034
  \l 5035
  \l 5036
  \l 5037
  \l 5038
  \l 5039
  \l 503A
  \l 503B
  \l 503C
  \l 503D
  \l 503E
  \l 503F
  \l 5040
  \l 5041
  \l 5042
  \l 5043
  \l 5044
  \l 5045
  \l 5046
  \l 5047
  \l 5048
  \l 5049
  \l 504A
  \l 504B
  \l 504C
  \l 504D
  \l 504E
  \l 504F
  \l 5050
  \l 5051
  \l 5052
  \l 5053
  \l 5054
  \l 5055
  \l 5056
  \l 5057
  \l 5058
  \l 5059
  \l 505A
  \l 505B
  \l 505C
  \l 505D
  \l 505E
  \l 505F
  \l 5060
  \l 5061
  \l 5062
  \l 5063
  \l 5064
  \l 5065
  \l 5066
  \l 5067
  \l 5068
  \l 5069
  \l 506A
  \l 506B
  \l 506C
  \l 506D
  \l 506E
  \l 506F
  \l 5070
  \l 5071
  \l 5072
  \l 5073
  \l 5074
  \l 5075
  \l 5076
  \l 5077
  \l 5078
  \l 5079
  \l 507A
  \l 507B
  \l 507C
  \l 507D
  \l 507E
  \l 507F
  \l 5080
  \l 5081
  \l 5082
  \l 5083
  \l 5084
  \l 5085
  \l 5086
  \l 5087
  \l 5088
  \l 5089
  \l 508A
  \l 508B
  \l 508C
  \l 508D
  \l 508E
  \l 508F
  \l 5090
  \l 5091
  \l 5092
  \l 5093
  \l 5094
  \l 5095
  \l 5096
  \l 5097
  \l 5098
  \l 5099
  \l 509A
  \l 509B
  \l 509C
  \l 509D
  \l 509E
  \l 509F
  \l 50A0
  \l 50A1
  \l 50A2
  \l 50A3
  \l 50A4
  \l 50A5
  \l 50A6
  \l 50A7
  \l 50A8
  \l 50A9
  \l 50AA
  \l 50AB
  \l 50AC
  \l 50AD
  \l 50AE
  \l 50AF
  \l 50B0
  \l 50B1
  \l 50B2
  \l 50B3
  \l 50B4
  \l 50B5
  \l 50B6
  \l 50B7
  \l 50B8
  \l 50B9
  \l 50BA
  \l 50BB
  \l 50BC
  \l 50BD
  \l 50BE
  \l 50BF
  \l 50C0
  \l 50C1
  \l 50C2
  \l 50C3
  \l 50C4
  \l 50C5
  \l 50C6
  \l 50C7
  \l 50C8
  \l 50C9
  \l 50CA
  \l 50CB
  \l 50CC
  \l 50CD
  \l 50CE
  \l 50CF
  \l 50D0
  \l 50D1
  \l 50D2
  \l 50D3
  \l 50D4
  \l 50D5
  \l 50D6
  \l 50D7
  \l 50D8
  \l 50D9
  \l 50DA
  \l 50DB
  \l 50DC
  \l 50DD
  \l 50DE
  \l 50DF
  \l 50E0
  \l 50E1
  \l 50E2
  \l 50E3
  \l 50E4
  \l 50E5
  \l 50E6
  \l 50E7
  \l 50E8
  \l 50E9
  \l 50EA
  \l 50EB
  \l 50EC
  \l 50ED
  \l 50EE
  \l 50EF
  \l 50F0
  \l 50F1
  \l 50F2
  \l 50F3
  \l 50F4
  \l 50F5
  \l 50F6
  \l 50F7
  \l 50F8
  \l 50F9
  \l 50FA
  \l 50FB
  \l 50FC
  \l 50FD
  \l 50FE
  \l 50FF
  \l 5100
  \l 5101
  \l 5102
  \l 5103
  \l 5104
  \l 5105
  \l 5106
  \l 5107
  \l 5108
  \l 5109
  \l 510A
  \l 510B
  \l 510C
  \l 510D
  \l 510E
  \l 510F
  \l 5110
  \l 5111
  \l 5112
  \l 5113
  \l 5114
  \l 5115
  \l 5116
  \l 5117
  \l 5118
  \l 5119
  \l 511A
  \l 511B
  \l 511C
  \l 511D
  \l 511E
  \l 511F
  \l 5120
  \l 5121
  \l 5122
  \l 5123
  \l 5124
  \l 5125
  \l 5126
  \l 5127
  \l 5128
  \l 5129
  \l 512A
  \l 512B
  \l 512C
  \l 512D
  \l 512E
  \l 512F
  \l 5130
  \l 5131
  \l 5132
  \l 5133
  \l 5134
  \l 5135
  \l 5136
  \l 5137
  \l 5138
  \l 5139
  \l 513A
  \l 513B
  \l 513C
  \l 513D
  \l 513E
  \l 513F
  \l 5140
  \l 5141
  \l 5142
  \l 5143
  \l 5144
  \l 5145
  \l 5146
  \l 5147
  \l 5148
  \l 5149
  \l 514A
  \l 514B
  \l 514C
  \l 514D
  \l 514E
  \l 514F
  \l 5150
  \l 5151
  \l 5152
  \l 5153
  \l 5154
  \l 5155
  \l 5156
  \l 5157
  \l 5158
  \l 5159
  \l 515A
  \l 515B
  \l 515C
  \l 515D
  \l 515E
  \l 515F
  \l 5160
  \l 5161
  \l 5162
  \l 5163
  \l 5164
  \l 5165
  \l 5166
  \l 5167
  \l 5168
  \l 5169
  \l 516A
  \l 516B
  \l 516C
  \l 516D
  \l 516E
  \l 516F
  \l 5170
  \l 5171
  \l 5172
  \l 5173
  \l 5174
  \l 5175
  \l 5176
  \l 5177
  \l 5178
  \l 5179
  \l 517A
  \l 517B
  \l 517C
  \l 517D
  \l 517E
  \l 517F
  \l 5180
  \l 5181
  \l 5182
  \l 5183
  \l 5184
  \l 5185
  \l 5186
  \l 5187
  \l 5188
  \l 5189
  \l 518A
  \l 518B
  \l 518C
  \l 518D
  \l 518E
  \l 518F
  \l 5190
  \l 5191
  \l 5192
  \l 5193
  \l 5194
  \l 5195
  \l 5196
  \l 5197
  \l 5198
  \l 5199
  \l 519A
  \l 519B
  \l 519C
  \l 519D
  \l 519E
  \l 519F
  \l 51A0
  \l 51A1
  \l 51A2
  \l 51A3
  \l 51A4
  \l 51A5
  \l 51A6
  \l 51A7
  \l 51A8
  \l 51A9
  \l 51AA
  \l 51AB
  \l 51AC
  \l 51AD
  \l 51AE
  \l 51AF
  \l 51B0
  \l 51B1
  \l 51B2
  \l 51B3
  \l 51B4
  \l 51B5
  \l 51B6
  \l 51B7
  \l 51B8
  \l 51B9
  \l 51BA
  \l 51BB
  \l 51BC
  \l 51BD
  \l 51BE
  \l 51BF
  \l 51C0
  \l 51C1
  \l 51C2
  \l 51C3
  \l 51C4
  \l 51C5
  \l 51C6
  \l 51C7
  \l 51C8
  \l 51C9
  \l 51CA
  \l 51CB
  \l 51CC
  \l 51CD
  \l 51CE
  \l 51CF
  \l 51D0
  \l 51D1
  \l 51D2
  \l 51D3
  \l 51D4
  \l 51D5
  \l 51D6
  \l 51D7
  \l 51D8
  \l 51D9
  \l 51DA
  \l 51DB
  \l 51DC
  \l 51DD
  \l 51DE
  \l 51DF
  \l 51E0
  \l 51E1
  \l 51E2
  \l 51E3
  \l 51E4
  \l 51E5
  \l 51E6
  \l 51E7
  \l 51E8
  \l 51E9
  \l 51EA
  \l 51EB
  \l 51EC
  \l 51ED
  \l 51EE
  \l 51EF
  \l 51F0
  \l 51F1
  \l 51F2
  \l 51F3
  \l 51F4
  \l 51F5
  \l 51F6
  \l 51F7
  \l 51F8
  \l 51F9
  \l 51FA
  \l 51FB
  \l 51FC
  \l 51FD
  \l 51FE
  \l 51FF
  \l 5200
  \l 5201
  \l 5202
  \l 5203
  \l 5204
  \l 5205
  \l 5206
  \l 5207
  \l 5208
  \l 5209
  \l 520A
  \l 520B
  \l 520C
  \l 520D
  \l 520E
  \l 520F
  \l 5210
  \l 5211
  \l 5212
  \l 5213
  \l 5214
  \l 5215
  \l 5216
  \l 5217
  \l 5218
  \l 5219
  \l 521A
  \l 521B
  \l 521C
  \l 521D
  \l 521E
  \l 521F
  \l 5220
  \l 5221
  \l 5222
  \l 5223
  \l 5224
  \l 5225
  \l 5226
  \l 5227
  \l 5228
  \l 5229
  \l 522A
  \l 522B
  \l 522C
  \l 522D
  \l 522E
  \l 522F
  \l 5230
  \l 5231
  \l 5232
  \l 5233
  \l 5234
  \l 5235
  \l 5236
  \l 5237
  \l 5238
  \l 5239
  \l 523A
  \l 523B
  \l 523C
  \l 523D
  \l 523E
  \l 523F
  \l 5240
  \l 5241
  \l 5242
  \l 5243
  \l 5244
  \l 5245
  \l 5246
  \l 5247
  \l 5248
  \l 5249
  \l 524A
  \l 524B
  \l 524C
  \l 524D
  \l 524E
  \l 524F
  \l 5250
  \l 5251
  \l 5252
  \l 5253
  \l 5254
  \l 5255
  \l 5256
  \l 5257
  \l 5258
  \l 5259
  \l 525A
  \l 525B
  \l 525C
  \l 525D
  \l 525E
  \l 525F
  \l 5260
  \l 5261
  \l 5262
  \l 5263
  \l 5264
  \l 5265
  \l 5266
  \l 5267
  \l 5268
  \l 5269
  \l 526A
  \l 526B
  \l 526C
  \l 526D
  \l 526E
  \l 526F
  \l 5270
  \l 5271
  \l 5272
  \l 5273
  \l 5274
  \l 5275
  \l 5276
  \l 5277
  \l 5278
  \l 5279
  \l 527A
  \l 527B
  \l 527C
  \l 527D
  \l 527E
  \l 527F
  \l 5280
  \l 5281
  \l 5282
  \l 5283
  \l 5284
  \l 5285
  \l 5286
  \l 5287
  \l 5288
  \l 5289
  \l 528A
  \l 528B
  \l 528C
  \l 528D
  \l 528E
  \l 528F
  \l 5290
  \l 5291
  \l 5292
  \l 5293
  \l 5294
  \l 5295
  \l 5296
  \l 5297
  \l 5298
  \l 5299
  \l 529A
  \l 529B
  \l 529C
  \l 529D
  \l 529E
  \l 529F
  \l 52A0
  \l 52A1
  \l 52A2
  \l 52A3
  \l 52A4
  \l 52A5
  \l 52A6
  \l 52A7
  \l 52A8
  \l 52A9
  \l 52AA
  \l 52AB
  \l 52AC
  \l 52AD
  \l 52AE
  \l 52AF
  \l 52B0
  \l 52B1
  \l 52B2
  \l 52B3
  \l 52B4
  \l 52B5
  \l 52B6
  \l 52B7
  \l 52B8
  \l 52B9
  \l 52BA
  \l 52BB
  \l 52BC
  \l 52BD
  \l 52BE
  \l 52BF
  \l 52C0
  \l 52C1
  \l 52C2
  \l 52C3
  \l 52C4
  \l 52C5
  \l 52C6
  \l 52C7
  \l 52C8
  \l 52C9
  \l 52CA
  \l 52CB
  \l 52CC
  \l 52CD
  \l 52CE
  \l 52CF
  \l 52D0
  \l 52D1
  \l 52D2
  \l 52D3
  \l 52D4
  \l 52D5
  \l 52D6
  \l 52D7
  \l 52D8
  \l 52D9
  \l 52DA
  \l 52DB
  \l 52DC
  \l 52DD
  \l 52DE
  \l 52DF
  \l 52E0
  \l 52E1
  \l 52E2
  \l 52E3
  \l 52E4
  \l 52E5
  \l 52E6
  \l 52E7
  \l 52E8
  \l 52E9
  \l 52EA
  \l 52EB
  \l 52EC
  \l 52ED
  \l 52EE
  \l 52EF
  \l 52F0
  \l 52F1
  \l 52F2
  \l 52F3
  \l 52F4
  \l 52F5
  \l 52F6
  \l 52F7
  \l 52F8
  \l 52F9
  \l 52FA
  \l 52FB
  \l 52FC
  \l 52FD
  \l 52FE
  \l 52FF
  \l 5300
  \l 5301
  \l 5302
  \l 5303
  \l 5304
  \l 5305
  \l 5306
  \l 5307
  \l 5308
  \l 5309
  \l 530A
  \l 530B
  \l 530C
  \l 530D
  \l 530E
  \l 530F
  \l 5310
  \l 5311
  \l 5312
  \l 5313
  \l 5314
  \l 5315
  \l 5316
  \l 5317
  \l 5318
  \l 5319
  \l 531A
  \l 531B
  \l 531C
  \l 531D
  \l 531E
  \l 531F
  \l 5320
  \l 5321
  \l 5322
  \l 5323
  \l 5324
  \l 5325
  \l 5326
  \l 5327
  \l 5328
  \l 5329
  \l 532A
  \l 532B
  \l 532C
  \l 532D
  \l 532E
  \l 532F
  \l 5330
  \l 5331
  \l 5332
  \l 5333
  \l 5334
  \l 5335
  \l 5336
  \l 5337
  \l 5338
  \l 5339
  \l 533A
  \l 533B
  \l 533C
  \l 533D
  \l 533E
  \l 533F
  \l 5340
  \l 5341
  \l 5342
  \l 5343
  \l 5344
  \l 5345
  \l 5346
  \l 5347
  \l 5348
  \l 5349
  \l 534A
  \l 534B
  \l 534C
  \l 534D
  \l 534E
  \l 534F
  \l 5350
  \l 5351
  \l 5352
  \l 5353
  \l 5354
  \l 5355
  \l 5356
  \l 5357
  \l 5358
  \l 5359
  \l 535A
  \l 535B
  \l 535C
  \l 535D
  \l 535E
  \l 535F
  \l 5360
  \l 5361
  \l 5362
  \l 5363
  \l 5364
  \l 5365
  \l 5366
  \l 5367
  \l 5368
  \l 5369
  \l 536A
  \l 536B
  \l 536C
  \l 536D
  \l 536E
  \l 536F
  \l 5370
  \l 5371
  \l 5372
  \l 5373
  \l 5374
  \l 5375
  \l 5376
  \l 5377
  \l 5378
  \l 5379
  \l 537A
  \l 537B
  \l 537C
  \l 537D
  \l 537E
  \l 537F
  \l 5380
  \l 5381
  \l 5382
  \l 5383
  \l 5384
  \l 5385
  \l 5386
  \l 5387
  \l 5388
  \l 5389
  \l 538A
  \l 538B
  \l 538C
  \l 538D
  \l 538E
  \l 538F
  \l 5390
  \l 5391
  \l 5392
  \l 5393
  \l 5394
  \l 5395
  \l 5396
  \l 5397
  \l 5398
  \l 5399
  \l 539A
  \l 539B
  \l 539C
  \l 539D
  \l 539E
  \l 539F
  \l 53A0
  \l 53A1
  \l 53A2
  \l 53A3
  \l 53A4
  \l 53A5
  \l 53A6
  \l 53A7
  \l 53A8
  \l 53A9
  \l 53AA
  \l 53AB
  \l 53AC
  \l 53AD
  \l 53AE
  \l 53AF
  \l 53B0
  \l 53B1
  \l 53B2
  \l 53B3
  \l 53B4
  \l 53B5
  \l 53B6
  \l 53B7
  \l 53B8
  \l 53B9
  \l 53BA
  \l 53BB
  \l 53BC
  \l 53BD
  \l 53BE
  \l 53BF
  \l 53C0
  \l 53C1
  \l 53C2
  \l 53C3
  \l 53C4
  \l 53C5
  \l 53C6
  \l 53C7
  \l 53C8
  \l 53C9
  \l 53CA
  \l 53CB
  \l 53CC
  \l 53CD
  \l 53CE
  \l 53CF
  \l 53D0
  \l 53D1
  \l 53D2
  \l 53D3
  \l 53D4
  \l 53D5
  \l 53D6
  \l 53D7
  \l 53D8
  \l 53D9
  \l 53DA
  \l 53DB
  \l 53DC
  \l 53DD
  \l 53DE
  \l 53DF
  \l 53E0
  \l 53E1
  \l 53E2
  \l 53E3
  \l 53E4
  \l 53E5
  \l 53E6
  \l 53E7
  \l 53E8
  \l 53E9
  \l 53EA
  \l 53EB
  \l 53EC
  \l 53ED
  \l 53EE
  \l 53EF
  \l 53F0
  \l 53F1
  \l 53F2
  \l 53F3
  \l 53F4
  \l 53F5
  \l 53F6
  \l 53F7
  \l 53F8
  \l 53F9
  \l 53FA
  \l 53FB
  \l 53FC
  \l 53FD
  \l 53FE
  \l 53FF
  \l 5400
  \l 5401
  \l 5402
  \l 5403
  \l 5404
  \l 5405
  \l 5406
  \l 5407
  \l 5408
  \l 5409
  \l 540A
  \l 540B
  \l 540C
  \l 540D
  \l 540E
  \l 540F
  \l 5410
  \l 5411
  \l 5412
  \l 5413
  \l 5414
  \l 5415
  \l 5416
  \l 5417
  \l 5418
  \l 5419
  \l 541A
  \l 541B
  \l 541C
  \l 541D
  \l 541E
  \l 541F
  \l 5420
  \l 5421
  \l 5422
  \l 5423
  \l 5424
  \l 5425
  \l 5426
  \l 5427
  \l 5428
  \l 5429
  \l 542A
  \l 542B
  \l 542C
  \l 542D
  \l 542E
  \l 542F
  \l 5430
  \l 5431
  \l 5432
  \l 5433
  \l 5434
  \l 5435
  \l 5436
  \l 5437
  \l 5438
  \l 5439
  \l 543A
  \l 543B
  \l 543C
  \l 543D
  \l 543E
  \l 543F
  \l 5440
  \l 5441
  \l 5442
  \l 5443
  \l 5444
  \l 5445
  \l 5446
  \l 5447
  \l 5448
  \l 5449
  \l 544A
  \l 544B
  \l 544C
  \l 544D
  \l 544E
  \l 544F
  \l 5450
  \l 5451
  \l 5452
  \l 5453
  \l 5454
  \l 5455
  \l 5456
  \l 5457
  \l 5458
  \l 5459
  \l 545A
  \l 545B
  \l 545C
  \l 545D
  \l 545E
  \l 545F
  \l 5460
  \l 5461
  \l 5462
  \l 5463
  \l 5464
  \l 5465
  \l 5466
  \l 5467
  \l 5468
  \l 5469
  \l 546A
  \l 546B
  \l 546C
  \l 546D
  \l 546E
  \l 546F
  \l 5470
  \l 5471
  \l 5472
  \l 5473
  \l 5474
  \l 5475
  \l 5476
  \l 5477
  \l 5478
  \l 5479
  \l 547A
  \l 547B
  \l 547C
  \l 547D
  \l 547E
  \l 547F
  \l 5480
  \l 5481
  \l 5482
  \l 5483
  \l 5484
  \l 5485
  \l 5486
  \l 5487
  \l 5488
  \l 5489
  \l 548A
  \l 548B
  \l 548C
  \l 548D
  \l 548E
  \l 548F
  \l 5490
  \l 5491
  \l 5492
  \l 5493
  \l 5494
  \l 5495
  \l 5496
  \l 5497
  \l 5498
  \l 5499
  \l 549A
  \l 549B
  \l 549C
  \l 549D
  \l 549E
  \l 549F
  \l 54A0
  \l 54A1
  \l 54A2
  \l 54A3
  \l 54A4
  \l 54A5
  \l 54A6
  \l 54A7
  \l 54A8
  \l 54A9
  \l 54AA
  \l 54AB
  \l 54AC
  \l 54AD
  \l 54AE
  \l 54AF
  \l 54B0
  \l 54B1
  \l 54B2
  \l 54B3
  \l 54B4
  \l 54B5
  \l 54B6
  \l 54B7
  \l 54B8
  \l 54B9
  \l 54BA
  \l 54BB
  \l 54BC
  \l 54BD
  \l 54BE
  \l 54BF
  \l 54C0
  \l 54C1
  \l 54C2
  \l 54C3
  \l 54C4
  \l 54C5
  \l 54C6
  \l 54C7
  \l 54C8
  \l 54C9
  \l 54CA
  \l 54CB
  \l 54CC
  \l 54CD
  \l 54CE
  \l 54CF
  \l 54D0
  \l 54D1
  \l 54D2
  \l 54D3
  \l 54D4
  \l 54D5
  \l 54D6
  \l 54D7
  \l 54D8
  \l 54D9
  \l 54DA
  \l 54DB
  \l 54DC
  \l 54DD
  \l 54DE
  \l 54DF
  \l 54E0
  \l 54E1
  \l 54E2
  \l 54E3
  \l 54E4
  \l 54E5
  \l 54E6
  \l 54E7
  \l 54E8
  \l 54E9
  \l 54EA
  \l 54EB
  \l 54EC
  \l 54ED
  \l 54EE
  \l 54EF
  \l 54F0
  \l 54F1
  \l 54F2
  \l 54F3
  \l 54F4
  \l 54F5
  \l 54F6
  \l 54F7
  \l 54F8
  \l 54F9
  \l 54FA
  \l 54FB
  \l 54FC
  \l 54FD
  \l 54FE
  \l 54FF
  \l 5500
  \l 5501
  \l 5502
  \l 5503
  \l 5504
  \l 5505
  \l 5506
  \l 5507
  \l 5508
  \l 5509
  \l 550A
  \l 550B
  \l 550C
  \l 550D
  \l 550E
  \l 550F
  \l 5510
  \l 5511
  \l 5512
  \l 5513
  \l 5514
  \l 5515
  \l 5516
  \l 5517
  \l 5518
  \l 5519
  \l 551A
  \l 551B
  \l 551C
  \l 551D
  \l 551E
  \l 551F
  \l 5520
  \l 5521
  \l 5522
  \l 5523
  \l 5524
  \l 5525
  \l 5526
  \l 5527
  \l 5528
  \l 5529
  \l 552A
  \l 552B
  \l 552C
  \l 552D
  \l 552E
  \l 552F
  \l 5530
  \l 5531
  \l 5532
  \l 5533
  \l 5534
  \l 5535
  \l 5536
  \l 5537
  \l 5538
  \l 5539
  \l 553A
  \l 553B
  \l 553C
  \l 553D
  \l 553E
  \l 553F
  \l 5540
  \l 5541
  \l 5542
  \l 5543
  \l 5544
  \l 5545
  \l 5546
  \l 5547
  \l 5548
  \l 5549
  \l 554A
  \l 554B
  \l 554C
  \l 554D
  \l 554E
  \l 554F
  \l 5550
  \l 5551
  \l 5552
  \l 5553
  \l 5554
  \l 5555
  \l 5556
  \l 5557
  \l 5558
  \l 5559
  \l 555A
  \l 555B
  \l 555C
  \l 555D
  \l 555E
  \l 555F
  \l 5560
  \l 5561
  \l 5562
  \l 5563
  \l 5564
  \l 5565
  \l 5566
  \l 5567
  \l 5568
  \l 5569
  \l 556A
  \l 556B
  \l 556C
  \l 556D
  \l 556E
  \l 556F
  \l 5570
  \l 5571
  \l 5572
  \l 5573
  \l 5574
  \l 5575
  \l 5576
  \l 5577
  \l 5578
  \l 5579
  \l 557A
  \l 557B
  \l 557C
  \l 557D
  \l 557E
  \l 557F
  \l 5580
  \l 5581
  \l 5582
  \l 5583
  \l 5584
  \l 5585
  \l 5586
  \l 5587
  \l 5588
  \l 5589
  \l 558A
  \l 558B
  \l 558C
  \l 558D
  \l 558E
  \l 558F
  \l 5590
  \l 5591
  \l 5592
  \l 5593
  \l 5594
  \l 5595
  \l 5596
  \l 5597
  \l 5598
  \l 5599
  \l 559A
  \l 559B
  \l 559C
  \l 559D
  \l 559E
  \l 559F
  \l 55A0
  \l 55A1
  \l 55A2
  \l 55A3
  \l 55A4
  \l 55A5
  \l 55A6
  \l 55A7
  \l 55A8
  \l 55A9
  \l 55AA
  \l 55AB
  \l 55AC
  \l 55AD
  \l 55AE
  \l 55AF
  \l 55B0
  \l 55B1
  \l 55B2
  \l 55B3
  \l 55B4
  \l 55B5
  \l 55B6
  \l 55B7
  \l 55B8
  \l 55B9
  \l 55BA
  \l 55BB
  \l 55BC
  \l 55BD
  \l 55BE
  \l 55BF
  \l 55C0
  \l 55C1
  \l 55C2
  \l 55C3
  \l 55C4
  \l 55C5
  \l 55C6
  \l 55C7
  \l 55C8
  \l 55C9
  \l 55CA
  \l 55CB
  \l 55CC
  \l 55CD
  \l 55CE
  \l 55CF
  \l 55D0
  \l 55D1
  \l 55D2
  \l 55D3
  \l 55D4
  \l 55D5
  \l 55D6
  \l 55D7
  \l 55D8
  \l 55D9
  \l 55DA
  \l 55DB
  \l 55DC
  \l 55DD
  \l 55DE
  \l 55DF
  \l 55E0
  \l 55E1
  \l 55E2
  \l 55E3
  \l 55E4
  \l 55E5
  \l 55E6
  \l 55E7
  \l 55E8
  \l 55E9
  \l 55EA
  \l 55EB
  \l 55EC
  \l 55ED
  \l 55EE
  \l 55EF
  \l 55F0
  \l 55F1
  \l 55F2
  \l 55F3
  \l 55F4
  \l 55F5
  \l 55F6
  \l 55F7
  \l 55F8
  \l 55F9
  \l 55FA
  \l 55FB
  \l 55FC
  \l 55FD
  \l 55FE
  \l 55FF
  \l 5600
  \l 5601
  \l 5602
  \l 5603
  \l 5604
  \l 5605
  \l 5606
  \l 5607
  \l 5608
  \l 5609
  \l 560A
  \l 560B
  \l 560C
  \l 560D
  \l 560E
  \l 560F
  \l 5610
  \l 5611
  \l 5612
  \l 5613
  \l 5614
  \l 5615
  \l 5616
  \l 5617
  \l 5618
  \l 5619
  \l 561A
  \l 561B
  \l 561C
  \l 561D
  \l 561E
  \l 561F
  \l 5620
  \l 5621
  \l 5622
  \l 5623
  \l 5624
  \l 5625
  \l 5626
  \l 5627
  \l 5628
  \l 5629
  \l 562A
  \l 562B
  \l 562C
  \l 562D
  \l 562E
  \l 562F
  \l 5630
  \l 5631
  \l 5632
  \l 5633
  \l 5634
  \l 5635
  \l 5636
  \l 5637
  \l 5638
  \l 5639
  \l 563A
  \l 563B
  \l 563C
  \l 563D
  \l 563E
  \l 563F
  \l 5640
  \l 5641
  \l 5642
  \l 5643
  \l 5644
  \l 5645
  \l 5646
  \l 5647
  \l 5648
  \l 5649
  \l 564A
  \l 564B
  \l 564C
  \l 564D
  \l 564E
  \l 564F
  \l 5650
  \l 5651
  \l 5652
  \l 5653
  \l 5654
  \l 5655
  \l 5656
  \l 5657
  \l 5658
  \l 5659
  \l 565A
  \l 565B
  \l 565C
  \l 565D
  \l 565E
  \l 565F
  \l 5660
  \l 5661
  \l 5662
  \l 5663
  \l 5664
  \l 5665
  \l 5666
  \l 5667
  \l 5668
  \l 5669
  \l 566A
  \l 566B
  \l 566C
  \l 566D
  \l 566E
  \l 566F
  \l 5670
  \l 5671
  \l 5672
  \l 5673
  \l 5674
  \l 5675
  \l 5676
  \l 5677
  \l 5678
  \l 5679
  \l 567A
  \l 567B
  \l 567C
  \l 567D
  \l 567E
  \l 567F
  \l 5680
  \l 5681
  \l 5682
  \l 5683
  \l 5684
  \l 5685
  \l 5686
  \l 5687
  \l 5688
  \l 5689
  \l 568A
  \l 568B
  \l 568C
  \l 568D
  \l 568E
  \l 568F
  \l 5690
  \l 5691
  \l 5692
  \l 5693
  \l 5694
  \l 5695
  \l 5696
  \l 5697
  \l 5698
  \l 5699
  \l 569A
  \l 569B
  \l 569C
  \l 569D
  \l 569E
  \l 569F
  \l 56A0
  \l 56A1
  \l 56A2
  \l 56A3
  \l 56A4
  \l 56A5
  \l 56A6
  \l 56A7
  \l 56A8
  \l 56A9
  \l 56AA
  \l 56AB
  \l 56AC
  \l 56AD
  \l 56AE
  \l 56AF
  \l 56B0
  \l 56B1
  \l 56B2
  \l 56B3
  \l 56B4
  \l 56B5
  \l 56B6
  \l 56B7
  \l 56B8
  \l 56B9
  \l 56BA
  \l 56BB
  \l 56BC
  \l 56BD
  \l 56BE
  \l 56BF
  \l 56C0
  \l 56C1
  \l 56C2
  \l 56C3
  \l 56C4
  \l 56C5
  \l 56C6
  \l 56C7
  \l 56C8
  \l 56C9
  \l 56CA
  \l 56CB
  \l 56CC
  \l 56CD
  \l 56CE
  \l 56CF
  \l 56D0
  \l 56D1
  \l 56D2
  \l 56D3
  \l 56D4
  \l 56D5
  \l 56D6
  \l 56D7
  \l 56D8
  \l 56D9
  \l 56DA
  \l 56DB
  \l 56DC
  \l 56DD
  \l 56DE
  \l 56DF
  \l 56E0
  \l 56E1
  \l 56E2
  \l 56E3
  \l 56E4
  \l 56E5
  \l 56E6
  \l 56E7
  \l 56E8
  \l 56E9
  \l 56EA
  \l 56EB
  \l 56EC
  \l 56ED
  \l 56EE
  \l 56EF
  \l 56F0
  \l 56F1
  \l 56F2
  \l 56F3
  \l 56F4
  \l 56F5
  \l 56F6
  \l 56F7
  \l 56F8
  \l 56F9
  \l 56FA
  \l 56FB
  \l 56FC
  \l 56FD
  \l 56FE
  \l 56FF
  \l 5700
  \l 5701
  \l 5702
  \l 5703
  \l 5704
  \l 5705
  \l 5706
  \l 5707
  \l 5708
  \l 5709
  \l 570A
  \l 570B
  \l 570C
  \l 570D
  \l 570E
  \l 570F
  \l 5710
  \l 5711
  \l 5712
  \l 5713
  \l 5714
  \l 5715
  \l 5716
  \l 5717
  \l 5718
  \l 5719
  \l 571A
  \l 571B
  \l 571C
  \l 571D
  \l 571E
  \l 571F
  \l 5720
  \l 5721
  \l 5722
  \l 5723
  \l 5724
  \l 5725
  \l 5726
  \l 5727
  \l 5728
  \l 5729
  \l 572A
  \l 572B
  \l 572C
  \l 572D
  \l 572E
  \l 572F
  \l 5730
  \l 5731
  \l 5732
  \l 5733
  \l 5734
  \l 5735
  \l 5736
  \l 5737
  \l 5738
  \l 5739
  \l 573A
  \l 573B
  \l 573C
  \l 573D
  \l 573E
  \l 573F
  \l 5740
  \l 5741
  \l 5742
  \l 5743
  \l 5744
  \l 5745
  \l 5746
  \l 5747
  \l 5748
  \l 5749
  \l 574A
  \l 574B
  \l 574C
  \l 574D
  \l 574E
  \l 574F
  \l 5750
  \l 5751
  \l 5752
  \l 5753
  \l 5754
  \l 5755
  \l 5756
  \l 5757
  \l 5758
  \l 5759
  \l 575A
  \l 575B
  \l 575C
  \l 575D
  \l 575E
  \l 575F
  \l 5760
  \l 5761
  \l 5762
  \l 5763
  \l 5764
  \l 5765
  \l 5766
  \l 5767
  \l 5768
  \l 5769
  \l 576A
  \l 576B
  \l 576C
  \l 576D
  \l 576E
  \l 576F
  \l 5770
  \l 5771
  \l 5772
  \l 5773
  \l 5774
  \l 5775
  \l 5776
  \l 5777
  \l 5778
  \l 5779
  \l 577A
  \l 577B
  \l 577C
  \l 577D
  \l 577E
  \l 577F
  \l 5780
  \l 5781
  \l 5782
  \l 5783
  \l 5784
  \l 5785
  \l 5786
  \l 5787
  \l 5788
  \l 5789
  \l 578A
  \l 578B
  \l 578C
  \l 578D
  \l 578E
  \l 578F
  \l 5790
  \l 5791
  \l 5792
  \l 5793
  \l 5794
  \l 5795
  \l 5796
  \l 5797
  \l 5798
  \l 5799
  \l 579A
  \l 579B
  \l 579C
  \l 579D
  \l 579E
  \l 579F
  \l 57A0
  \l 57A1
  \l 57A2
  \l 57A3
  \l 57A4
  \l 57A5
  \l 57A6
  \l 57A7
  \l 57A8
  \l 57A9
  \l 57AA
  \l 57AB
  \l 57AC
  \l 57AD
  \l 57AE
  \l 57AF
  \l 57B0
  \l 57B1
  \l 57B2
  \l 57B3
  \l 57B4
  \l 57B5
  \l 57B6
  \l 57B7
  \l 57B8
  \l 57B9
  \l 57BA
  \l 57BB
  \l 57BC
  \l 57BD
  \l 57BE
  \l 57BF
  \l 57C0
  \l 57C1
  \l 57C2
  \l 57C3
  \l 57C4
  \l 57C5
  \l 57C6
  \l 57C7
  \l 57C8
  \l 57C9
  \l 57CA
  \l 57CB
  \l 57CC
  \l 57CD
  \l 57CE
  \l 57CF
  \l 57D0
  \l 57D1
  \l 57D2
  \l 57D3
  \l 57D4
  \l 57D5
  \l 57D6
  \l 57D7
  \l 57D8
  \l 57D9
  \l 57DA
  \l 57DB
  \l 57DC
  \l 57DD
  \l 57DE
  \l 57DF
  \l 57E0
  \l 57E1
  \l 57E2
  \l 57E3
  \l 57E4
  \l 57E5
  \l 57E6
  \l 57E7
  \l 57E8
  \l 57E9
  \l 57EA
  \l 57EB
  \l 57EC
  \l 57ED
  \l 57EE
  \l 57EF
  \l 57F0
  \l 57F1
  \l 57F2
  \l 57F3
  \l 57F4
  \l 57F5
  \l 57F6
  \l 57F7
  \l 57F8
  \l 57F9
  \l 57FA
  \l 57FB
  \l 57FC
  \l 57FD
  \l 57FE
  \l 57FF
  \l 5800
  \l 5801
  \l 5802
  \l 5803
  \l 5804
  \l 5805
  \l 5806
  \l 5807
  \l 5808
  \l 5809
  \l 580A
  \l 580B
  \l 580C
  \l 580D
  \l 580E
  \l 580F
  \l 5810
  \l 5811
  \l 5812
  \l 5813
  \l 5814
  \l 5815
  \l 5816
  \l 5817
  \l 5818
  \l 5819
  \l 581A
  \l 581B
  \l 581C
  \l 581D
  \l 581E
  \l 581F
  \l 5820
  \l 5821
  \l 5822
  \l 5823
  \l 5824
  \l 5825
  \l 5826
  \l 5827
  \l 5828
  \l 5829
  \l 582A
  \l 582B
  \l 582C
  \l 582D
  \l 582E
  \l 582F
  \l 5830
  \l 5831
  \l 5832
  \l 5833
  \l 5834
  \l 5835
  \l 5836
  \l 5837
  \l 5838
  \l 5839
  \l 583A
  \l 583B
  \l 583C
  \l 583D
  \l 583E
  \l 583F
  \l 5840
  \l 5841
  \l 5842
  \l 5843
  \l 5844
  \l 5845
  \l 5846
  \l 5847
  \l 5848
  \l 5849
  \l 584A
  \l 584B
  \l 584C
  \l 584D
  \l 584E
  \l 584F
  \l 5850
  \l 5851
  \l 5852
  \l 5853
  \l 5854
  \l 5855
  \l 5856
  \l 5857
  \l 5858
  \l 5859
  \l 585A
  \l 585B
  \l 585C
  \l 585D
  \l 585E
  \l 585F
  \l 5860
  \l 5861
  \l 5862
  \l 5863
  \l 5864
  \l 5865
  \l 5866
  \l 5867
  \l 5868
  \l 5869
  \l 586A
  \l 586B
  \l 586C
  \l 586D
  \l 586E
  \l 586F
  \l 5870
  \l 5871
  \l 5872
  \l 5873
  \l 5874
  \l 5875
  \l 5876
  \l 5877
  \l 5878
  \l 5879
  \l 587A
  \l 587B
  \l 587C
  \l 587D
  \l 587E
  \l 587F
  \l 5880
  \l 5881
  \l 5882
  \l 5883
  \l 5884
  \l 5885
  \l 5886
  \l 5887
  \l 5888
  \l 5889
  \l 588A
  \l 588B
  \l 588C
  \l 588D
  \l 588E
  \l 588F
  \l 5890
  \l 5891
  \l 5892
  \l 5893
  \l 5894
  \l 5895
  \l 5896
  \l 5897
  \l 5898
  \l 5899
  \l 589A
  \l 589B
  \l 589C
  \l 589D
  \l 589E
  \l 589F
  \l 58A0
  \l 58A1
  \l 58A2
  \l 58A3
  \l 58A4
  \l 58A5
  \l 58A6
  \l 58A7
  \l 58A8
  \l 58A9
  \l 58AA
  \l 58AB
  \l 58AC
  \l 58AD
  \l 58AE
  \l 58AF
  \l 58B0
  \l 58B1
  \l 58B2
  \l 58B3
  \l 58B4
  \l 58B5
  \l 58B6
  \l 58B7
  \l 58B8
  \l 58B9
  \l 58BA
  \l 58BB
  \l 58BC
  \l 58BD
  \l 58BE
  \l 58BF
  \l 58C0
  \l 58C1
  \l 58C2
  \l 58C3
  \l 58C4
  \l 58C5
  \l 58C6
  \l 58C7
  \l 58C8
  \l 58C9
  \l 58CA
  \l 58CB
  \l 58CC
  \l 58CD
  \l 58CE
  \l 58CF
  \l 58D0
  \l 58D1
  \l 58D2
  \l 58D3
  \l 58D4
  \l 58D5
  \l 58D6
  \l 58D7
  \l 58D8
  \l 58D9
  \l 58DA
  \l 58DB
  \l 58DC
  \l 58DD
  \l 58DE
  \l 58DF
  \l 58E0
  \l 58E1
  \l 58E2
  \l 58E3
  \l 58E4
  \l 58E5
  \l 58E6
  \l 58E7
  \l 58E8
  \l 58E9
  \l 58EA
  \l 58EB
  \l 58EC
  \l 58ED
  \l 58EE
  \l 58EF
  \l 58F0
  \l 58F1
  \l 58F2
  \l 58F3
  \l 58F4
  \l 58F5
  \l 58F6
  \l 58F7
  \l 58F8
  \l 58F9
  \l 58FA
  \l 58FB
  \l 58FC
  \l 58FD
  \l 58FE
  \l 58FF
  \l 5900
  \l 5901
  \l 5902
  \l 5903
  \l 5904
  \l 5905
  \l 5906
  \l 5907
  \l 5908
  \l 5909
  \l 590A
  \l 590B
  \l 590C
  \l 590D
  \l 590E
  \l 590F
  \l 5910
  \l 5911
  \l 5912
  \l 5913
  \l 5914
  \l 5915
  \l 5916
  \l 5917
  \l 5918
  \l 5919
  \l 591A
  \l 591B
  \l 591C
  \l 591D
  \l 591E
  \l 591F
  \l 5920
  \l 5921
  \l 5922
  \l 5923
  \l 5924
  \l 5925
  \l 5926
  \l 5927
  \l 5928
  \l 5929
  \l 592A
  \l 592B
  \l 592C
  \l 592D
  \l 592E
  \l 592F
  \l 5930
  \l 5931
  \l 5932
  \l 5933
  \l 5934
  \l 5935
  \l 5936
  \l 5937
  \l 5938
  \l 5939
  \l 593A
  \l 593B
  \l 593C
  \l 593D
  \l 593E
  \l 593F
  \l 5940
  \l 5941
  \l 5942
  \l 5943
  \l 5944
  \l 5945
  \l 5946
  \l 5947
  \l 5948
  \l 5949
  \l 594A
  \l 594B
  \l 594C
  \l 594D
  \l 594E
  \l 594F
  \l 5950
  \l 5951
  \l 5952
  \l 5953
  \l 5954
  \l 5955
  \l 5956
  \l 5957
  \l 5958
  \l 5959
  \l 595A
  \l 595B
  \l 595C
  \l 595D
  \l 595E
  \l 595F
  \l 5960
  \l 5961
  \l 5962
  \l 5963
  \l 5964
  \l 5965
  \l 5966
  \l 5967
  \l 5968
  \l 5969
  \l 596A
  \l 596B
  \l 596C
  \l 596D
  \l 596E
  \l 596F
  \l 5970
  \l 5971
  \l 5972
  \l 5973
  \l 5974
  \l 5975
  \l 5976
  \l 5977
  \l 5978
  \l 5979
  \l 597A
  \l 597B
  \l 597C
  \l 597D
  \l 597E
  \l 597F
  \l 5980
  \l 5981
  \l 5982
  \l 5983
  \l 5984
  \l 5985
  \l 5986
  \l 5987
  \l 5988
  \l 5989
  \l 598A
  \l 598B
  \l 598C
  \l 598D
  \l 598E
  \l 598F
  \l 5990
  \l 5991
  \l 5992
  \l 5993
  \l 5994
  \l 5995
  \l 5996
  \l 5997
  \l 5998
  \l 5999
  \l 599A
  \l 599B
  \l 599C
  \l 599D
  \l 599E
  \l 599F
  \l 59A0
  \l 59A1
  \l 59A2
  \l 59A3
  \l 59A4
  \l 59A5
  \l 59A6
  \l 59A7
  \l 59A8
  \l 59A9
  \l 59AA
  \l 59AB
  \l 59AC
  \l 59AD
  \l 59AE
  \l 59AF
  \l 59B0
  \l 59B1
  \l 59B2
  \l 59B3
  \l 59B4
  \l 59B5
  \l 59B6
  \l 59B7
  \l 59B8
  \l 59B9
  \l 59BA
  \l 59BB
  \l 59BC
  \l 59BD
  \l 59BE
  \l 59BF
  \l 59C0
  \l 59C1
  \l 59C2
  \l 59C3
  \l 59C4
  \l 59C5
  \l 59C6
  \l 59C7
  \l 59C8
  \l 59C9
  \l 59CA
  \l 59CB
  \l 59CC
  \l 59CD
  \l 59CE
  \l 59CF
  \l 59D0
  \l 59D1
  \l 59D2
  \l 59D3
  \l 59D4
  \l 59D5
  \l 59D6
  \l 59D7
  \l 59D8
  \l 59D9
  \l 59DA
  \l 59DB
  \l 59DC
  \l 59DD
  \l 59DE
  \l 59DF
  \l 59E0
  \l 59E1
  \l 59E2
  \l 59E3
  \l 59E4
  \l 59E5
  \l 59E6
  \l 59E7
  \l 59E8
  \l 59E9
  \l 59EA
  \l 59EB
  \l 59EC
  \l 59ED
  \l 59EE
  \l 59EF
  \l 59F0
  \l 59F1
  \l 59F2
  \l 59F3
  \l 59F4
  \l 59F5
  \l 59F6
  \l 59F7
  \l 59F8
  \l 59F9
  \l 59FA
  \l 59FB
  \l 59FC
  \l 59FD
  \l 59FE
  \l 59FF
  \l 5A00
  \l 5A01
  \l 5A02
  \l 5A03
  \l 5A04
  \l 5A05
  \l 5A06
  \l 5A07
  \l 5A08
  \l 5A09
  \l 5A0A
  \l 5A0B
  \l 5A0C
  \l 5A0D
  \l 5A0E
  \l 5A0F
  \l 5A10
  \l 5A11
  \l 5A12
  \l 5A13
  \l 5A14
  \l 5A15
  \l 5A16
  \l 5A17
  \l 5A18
  \l 5A19
  \l 5A1A
  \l 5A1B
  \l 5A1C
  \l 5A1D
  \l 5A1E
  \l 5A1F
  \l 5A20
  \l 5A21
  \l 5A22
  \l 5A23
  \l 5A24
  \l 5A25
  \l 5A26
  \l 5A27
  \l 5A28
  \l 5A29
  \l 5A2A
  \l 5A2B
  \l 5A2C
  \l 5A2D
  \l 5A2E
  \l 5A2F
  \l 5A30
  \l 5A31
  \l 5A32
  \l 5A33
  \l 5A34
  \l 5A35
  \l 5A36
  \l 5A37
  \l 5A38
  \l 5A39
  \l 5A3A
  \l 5A3B
  \l 5A3C
  \l 5A3D
  \l 5A3E
  \l 5A3F
  \l 5A40
  \l 5A41
  \l 5A42
  \l 5A43
  \l 5A44
  \l 5A45
  \l 5A46
  \l 5A47
  \l 5A48
  \l 5A49
  \l 5A4A
  \l 5A4B
  \l 5A4C
  \l 5A4D
  \l 5A4E
  \l 5A4F
  \l 5A50
  \l 5A51
  \l 5A52
  \l 5A53
  \l 5A54
  \l 5A55
  \l 5A56
  \l 5A57
  \l 5A58
  \l 5A59
  \l 5A5A
  \l 5A5B
  \l 5A5C
  \l 5A5D
  \l 5A5E
  \l 5A5F
  \l 5A60
  \l 5A61
  \l 5A62
  \l 5A63
  \l 5A64
  \l 5A65
  \l 5A66
  \l 5A67
  \l 5A68
  \l 5A69
  \l 5A6A
  \l 5A6B
  \l 5A6C
  \l 5A6D
  \l 5A6E
  \l 5A6F
  \l 5A70
  \l 5A71
  \l 5A72
  \l 5A73
  \l 5A74
  \l 5A75
  \l 5A76
  \l 5A77
  \l 5A78
  \l 5A79
  \l 5A7A
  \l 5A7B
  \l 5A7C
  \l 5A7D
  \l 5A7E
  \l 5A7F
  \l 5A80
  \l 5A81
  \l 5A82
  \l 5A83
  \l 5A84
  \l 5A85
  \l 5A86
  \l 5A87
  \l 5A88
  \l 5A89
  \l 5A8A
  \l 5A8B
  \l 5A8C
  \l 5A8D
  \l 5A8E
  \l 5A8F
  \l 5A90
  \l 5A91
  \l 5A92
  \l 5A93
  \l 5A94
  \l 5A95
  \l 5A96
  \l 5A97
  \l 5A98
  \l 5A99
  \l 5A9A
  \l 5A9B
  \l 5A9C
  \l 5A9D
  \l 5A9E
  \l 5A9F
  \l 5AA0
  \l 5AA1
  \l 5AA2
  \l 5AA3
  \l 5AA4
  \l 5AA5
  \l 5AA6
  \l 5AA7
  \l 5AA8
  \l 5AA9
  \l 5AAA
  \l 5AAB
  \l 5AAC
  \l 5AAD
  \l 5AAE
  \l 5AAF
  \l 5AB0
  \l 5AB1
  \l 5AB2
  \l 5AB3
  \l 5AB4
  \l 5AB5
  \l 5AB6
  \l 5AB7
  \l 5AB8
  \l 5AB9
  \l 5ABA
  \l 5ABB
  \l 5ABC
  \l 5ABD
  \l 5ABE
  \l 5ABF
  \l 5AC0
  \l 5AC1
  \l 5AC2
  \l 5AC3
  \l 5AC4
  \l 5AC5
  \l 5AC6
  \l 5AC7
  \l 5AC8
  \l 5AC9
  \l 5ACA
  \l 5ACB
  \l 5ACC
  \l 5ACD
  \l 5ACE
  \l 5ACF
  \l 5AD0
  \l 5AD1
  \l 5AD2
  \l 5AD3
  \l 5AD4
  \l 5AD5
  \l 5AD6
  \l 5AD7
  \l 5AD8
  \l 5AD9
  \l 5ADA
  \l 5ADB
  \l 5ADC
  \l 5ADD
  \l 5ADE
  \l 5ADF
  \l 5AE0
  \l 5AE1
  \l 5AE2
  \l 5AE3
  \l 5AE4
  \l 5AE5
  \l 5AE6
  \l 5AE7
  \l 5AE8
  \l 5AE9
  \l 5AEA
  \l 5AEB
  \l 5AEC
  \l 5AED
  \l 5AEE
  \l 5AEF
  \l 5AF0
  \l 5AF1
  \l 5AF2
  \l 5AF3
  \l 5AF4
  \l 5AF5
  \l 5AF6
  \l 5AF7
  \l 5AF8
  \l 5AF9
  \l 5AFA
  \l 5AFB
  \l 5AFC
  \l 5AFD
  \l 5AFE
  \l 5AFF
  \l 5B00
  \l 5B01
  \l 5B02
  \l 5B03
  \l 5B04
  \l 5B05
  \l 5B06
  \l 5B07
  \l 5B08
  \l 5B09
  \l 5B0A
  \l 5B0B
  \l 5B0C
  \l 5B0D
  \l 5B0E
  \l 5B0F
  \l 5B10
  \l 5B11
  \l 5B12
  \l 5B13
  \l 5B14
  \l 5B15
  \l 5B16
  \l 5B17
  \l 5B18
  \l 5B19
  \l 5B1A
  \l 5B1B
  \l 5B1C
  \l 5B1D
  \l 5B1E
  \l 5B1F
  \l 5B20
  \l 5B21
  \l 5B22
  \l 5B23
  \l 5B24
  \l 5B25
  \l 5B26
  \l 5B27
  \l 5B28
  \l 5B29
  \l 5B2A
  \l 5B2B
  \l 5B2C
  \l 5B2D
  \l 5B2E
  \l 5B2F
  \l 5B30
  \l 5B31
  \l 5B32
  \l 5B33
  \l 5B34
  \l 5B35
  \l 5B36
  \l 5B37
  \l 5B38
  \l 5B39
  \l 5B3A
  \l 5B3B
  \l 5B3C
  \l 5B3D
  \l 5B3E
  \l 5B3F
  \l 5B40
  \l 5B41
  \l 5B42
  \l 5B43
  \l 5B44
  \l 5B45
  \l 5B46
  \l 5B47
  \l 5B48
  \l 5B49
  \l 5B4A
  \l 5B4B
  \l 5B4C
  \l 5B4D
  \l 5B4E
  \l 5B4F
  \l 5B50
  \l 5B51
  \l 5B52
  \l 5B53
  \l 5B54
  \l 5B55
  \l 5B56
  \l 5B57
  \l 5B58
  \l 5B59
  \l 5B5A
  \l 5B5B
  \l 5B5C
  \l 5B5D
  \l 5B5E
  \l 5B5F
  \l 5B60
  \l 5B61
  \l 5B62
  \l 5B63
  \l 5B64
  \l 5B65
  \l 5B66
  \l 5B67
  \l 5B68
  \l 5B69
  \l 5B6A
  \l 5B6B
  \l 5B6C
  \l 5B6D
  \l 5B6E
  \l 5B6F
  \l 5B70
  \l 5B71
  \l 5B72
  \l 5B73
  \l 5B74
  \l 5B75
  \l 5B76
  \l 5B77
  \l 5B78
  \l 5B79
  \l 5B7A
  \l 5B7B
  \l 5B7C
  \l 5B7D
  \l 5B7E
  \l 5B7F
  \l 5B80
  \l 5B81
  \l 5B82
  \l 5B83
  \l 5B84
  \l 5B85
  \l 5B86
  \l 5B87
  \l 5B88
  \l 5B89
  \l 5B8A
  \l 5B8B
  \l 5B8C
  \l 5B8D
  \l 5B8E
  \l 5B8F
  \l 5B90
  \l 5B91
  \l 5B92
  \l 5B93
  \l 5B94
  \l 5B95
  \l 5B96
  \l 5B97
  \l 5B98
  \l 5B99
  \l 5B9A
  \l 5B9B
  \l 5B9C
  \l 5B9D
  \l 5B9E
  \l 5B9F
  \l 5BA0
  \l 5BA1
  \l 5BA2
  \l 5BA3
  \l 5BA4
  \l 5BA5
  \l 5BA6
  \l 5BA7
  \l 5BA8
  \l 5BA9
  \l 5BAA
  \l 5BAB
  \l 5BAC
  \l 5BAD
  \l 5BAE
  \l 5BAF
  \l 5BB0
  \l 5BB1
  \l 5BB2
  \l 5BB3
  \l 5BB4
  \l 5BB5
  \l 5BB6
  \l 5BB7
  \l 5BB8
  \l 5BB9
  \l 5BBA
  \l 5BBB
  \l 5BBC
  \l 5BBD
  \l 5BBE
  \l 5BBF
  \l 5BC0
  \l 5BC1
  \l 5BC2
  \l 5BC3
  \l 5BC4
  \l 5BC5
  \l 5BC6
  \l 5BC7
  \l 5BC8
  \l 5BC9
  \l 5BCA
  \l 5BCB
  \l 5BCC
  \l 5BCD
  \l 5BCE
  \l 5BCF
  \l 5BD0
  \l 5BD1
  \l 5BD2
  \l 5BD3
  \l 5BD4
  \l 5BD5
  \l 5BD6
  \l 5BD7
  \l 5BD8
  \l 5BD9
  \l 5BDA
  \l 5BDB
  \l 5BDC
  \l 5BDD
  \l 5BDE
  \l 5BDF
  \l 5BE0
  \l 5BE1
  \l 5BE2
  \l 5BE3
  \l 5BE4
  \l 5BE5
  \l 5BE6
  \l 5BE7
  \l 5BE8
  \l 5BE9
  \l 5BEA
  \l 5BEB
  \l 5BEC
  \l 5BED
  \l 5BEE
  \l 5BEF
  \l 5BF0
  \l 5BF1
  \l 5BF2
  \l 5BF3
  \l 5BF4
  \l 5BF5
  \l 5BF6
  \l 5BF7
  \l 5BF8
  \l 5BF9
  \l 5BFA
  \l 5BFB
  \l 5BFC
  \l 5BFD
  \l 5BFE
  \l 5BFF
  \l 5C00
  \l 5C01
  \l 5C02
  \l 5C03
  \l 5C04
  \l 5C05
  \l 5C06
  \l 5C07
  \l 5C08
  \l 5C09
  \l 5C0A
  \l 5C0B
  \l 5C0C
  \l 5C0D
  \l 5C0E
  \l 5C0F
  \l 5C10
  \l 5C11
  \l 5C12
  \l 5C13
  \l 5C14
  \l 5C15
  \l 5C16
  \l 5C17
  \l 5C18
  \l 5C19
  \l 5C1A
  \l 5C1B
  \l 5C1C
  \l 5C1D
  \l 5C1E
  \l 5C1F
  \l 5C20
  \l 5C21
  \l 5C22
  \l 5C23
  \l 5C24
  \l 5C25
  \l 5C26
  \l 5C27
  \l 5C28
  \l 5C29
  \l 5C2A
  \l 5C2B
  \l 5C2C
  \l 5C2D
  \l 5C2E
  \l 5C2F
  \l 5C30
  \l 5C31
  \l 5C32
  \l 5C33
  \l 5C34
  \l 5C35
  \l 5C36
  \l 5C37
  \l 5C38
  \l 5C39
  \l 5C3A
  \l 5C3B
  \l 5C3C
  \l 5C3D
  \l 5C3E
  \l 5C3F
  \l 5C40
  \l 5C41
  \l 5C42
  \l 5C43
  \l 5C44
  \l 5C45
  \l 5C46
  \l 5C47
  \l 5C48
  \l 5C49
  \l 5C4A
  \l 5C4B
  \l 5C4C
  \l 5C4D
  \l 5C4E
  \l 5C4F
  \l 5C50
  \l 5C51
  \l 5C52
  \l 5C53
  \l 5C54
  \l 5C55
  \l 5C56
  \l 5C57
  \l 5C58
  \l 5C59
  \l 5C5A
  \l 5C5B
  \l 5C5C
  \l 5C5D
  \l 5C5E
  \l 5C5F
  \l 5C60
  \l 5C61
  \l 5C62
  \l 5C63
  \l 5C64
  \l 5C65
  \l 5C66
  \l 5C67
  \l 5C68
  \l 5C69
  \l 5C6A
  \l 5C6B
  \l 5C6C
  \l 5C6D
  \l 5C6E
  \l 5C6F
  \l 5C70
  \l 5C71
  \l 5C72
  \l 5C73
  \l 5C74
  \l 5C75
  \l 5C76
  \l 5C77
  \l 5C78
  \l 5C79
  \l 5C7A
  \l 5C7B
  \l 5C7C
  \l 5C7D
  \l 5C7E
  \l 5C7F
  \l 5C80
  \l 5C81
  \l 5C82
  \l 5C83
  \l 5C84
  \l 5C85
  \l 5C86
  \l 5C87
  \l 5C88
  \l 5C89
  \l 5C8A
  \l 5C8B
  \l 5C8C
  \l 5C8D
  \l 5C8E
  \l 5C8F
  \l 5C90
  \l 5C91
  \l 5C92
  \l 5C93
  \l 5C94
  \l 5C95
  \l 5C96
  \l 5C97
  \l 5C98
  \l 5C99
  \l 5C9A
  \l 5C9B
  \l 5C9C
  \l 5C9D
  \l 5C9E
  \l 5C9F
  \l 5CA0
  \l 5CA1
  \l 5CA2
  \l 5CA3
  \l 5CA4
  \l 5CA5
  \l 5CA6
  \l 5CA7
  \l 5CA8
  \l 5CA9
  \l 5CAA
  \l 5CAB
  \l 5CAC
  \l 5CAD
  \l 5CAE
  \l 5CAF
  \l 5CB0
  \l 5CB1
  \l 5CB2
  \l 5CB3
  \l 5CB4
  \l 5CB5
  \l 5CB6
  \l 5CB7
  \l 5CB8
  \l 5CB9
  \l 5CBA
  \l 5CBB
  \l 5CBC
  \l 5CBD
  \l 5CBE
  \l 5CBF
  \l 5CC0
  \l 5CC1
  \l 5CC2
  \l 5CC3
  \l 5CC4
  \l 5CC5
  \l 5CC6
  \l 5CC7
  \l 5CC8
  \l 5CC9
  \l 5CCA
  \l 5CCB
  \l 5CCC
  \l 5CCD
  \l 5CCE
  \l 5CCF
  \l 5CD0
  \l 5CD1
  \l 5CD2
  \l 5CD3
  \l 5CD4
  \l 5CD5
  \l 5CD6
  \l 5CD7
  \l 5CD8
  \l 5CD9
  \l 5CDA
  \l 5CDB
  \l 5CDC
  \l 5CDD
  \l 5CDE
  \l 5CDF
  \l 5CE0
  \l 5CE1
  \l 5CE2
  \l 5CE3
  \l 5CE4
  \l 5CE5
  \l 5CE6
  \l 5CE7
  \l 5CE8
  \l 5CE9
  \l 5CEA
  \l 5CEB
  \l 5CEC
  \l 5CED
  \l 5CEE
  \l 5CEF
  \l 5CF0
  \l 5CF1
  \l 5CF2
  \l 5CF3
  \l 5CF4
  \l 5CF5
  \l 5CF6
  \l 5CF7
  \l 5CF8
  \l 5CF9
  \l 5CFA
  \l 5CFB
  \l 5CFC
  \l 5CFD
  \l 5CFE
  \l 5CFF
  \l 5D00
  \l 5D01
  \l 5D02
  \l 5D03
  \l 5D04
  \l 5D05
  \l 5D06
  \l 5D07
  \l 5D08
  \l 5D09
  \l 5D0A
  \l 5D0B
  \l 5D0C
  \l 5D0D
  \l 5D0E
  \l 5D0F
  \l 5D10
  \l 5D11
  \l 5D12
  \l 5D13
  \l 5D14
  \l 5D15
  \l 5D16
  \l 5D17
  \l 5D18
  \l 5D19
  \l 5D1A
  \l 5D1B
  \l 5D1C
  \l 5D1D
  \l 5D1E
  \l 5D1F
  \l 5D20
  \l 5D21
  \l 5D22
  \l 5D23
  \l 5D24
  \l 5D25
  \l 5D26
  \l 5D27
  \l 5D28
  \l 5D29
  \l 5D2A
  \l 5D2B
  \l 5D2C
  \l 5D2D
  \l 5D2E
  \l 5D2F
  \l 5D30
  \l 5D31
  \l 5D32
  \l 5D33
  \l 5D34
  \l 5D35
  \l 5D36
  \l 5D37
  \l 5D38
  \l 5D39
  \l 5D3A
  \l 5D3B
  \l 5D3C
  \l 5D3D
  \l 5D3E
  \l 5D3F
  \l 5D40
  \l 5D41
  \l 5D42
  \l 5D43
  \l 5D44
  \l 5D45
  \l 5D46
  \l 5D47
  \l 5D48
  \l 5D49
  \l 5D4A
  \l 5D4B
  \l 5D4C
  \l 5D4D
  \l 5D4E
  \l 5D4F
  \l 5D50
  \l 5D51
  \l 5D52
  \l 5D53
  \l 5D54
  \l 5D55
  \l 5D56
  \l 5D57
  \l 5D58
  \l 5D59
  \l 5D5A
  \l 5D5B
  \l 5D5C
  \l 5D5D
  \l 5D5E
  \l 5D5F
  \l 5D60
  \l 5D61
  \l 5D62
  \l 5D63
  \l 5D64
  \l 5D65
  \l 5D66
  \l 5D67
  \l 5D68
  \l 5D69
  \l 5D6A
  \l 5D6B
  \l 5D6C
  \l 5D6D
  \l 5D6E
  \l 5D6F
  \l 5D70
  \l 5D71
  \l 5D72
  \l 5D73
  \l 5D74
  \l 5D75
  \l 5D76
  \l 5D77
  \l 5D78
  \l 5D79
  \l 5D7A
  \l 5D7B
  \l 5D7C
  \l 5D7D
  \l 5D7E
  \l 5D7F
  \l 5D80
  \l 5D81
  \l 5D82
  \l 5D83
  \l 5D84
  \l 5D85
  \l 5D86
  \l 5D87
  \l 5D88
  \l 5D89
  \l 5D8A
  \l 5D8B
  \l 5D8C
  \l 5D8D
  \l 5D8E
  \l 5D8F
  \l 5D90
  \l 5D91
  \l 5D92
  \l 5D93
  \l 5D94
  \l 5D95
  \l 5D96
  \l 5D97
  \l 5D98
  \l 5D99
  \l 5D9A
  \l 5D9B
  \l 5D9C
  \l 5D9D
  \l 5D9E
  \l 5D9F
  \l 5DA0
  \l 5DA1
  \l 5DA2
  \l 5DA3
  \l 5DA4
  \l 5DA5
  \l 5DA6
  \l 5DA7
  \l 5DA8
  \l 5DA9
  \l 5DAA
  \l 5DAB
  \l 5DAC
  \l 5DAD
  \l 5DAE
  \l 5DAF
  \l 5DB0
  \l 5DB1
  \l 5DB2
  \l 5DB3
  \l 5DB4
  \l 5DB5
  \l 5DB6
  \l 5DB7
  \l 5DB8
  \l 5DB9
  \l 5DBA
  \l 5DBB
  \l 5DBC
  \l 5DBD
  \l 5DBE
  \l 5DBF
  \l 5DC0
  \l 5DC1
  \l 5DC2
  \l 5DC3
  \l 5DC4
  \l 5DC5
  \l 5DC6
  \l 5DC7
  \l 5DC8
  \l 5DC9
  \l 5DCA
  \l 5DCB
  \l 5DCC
  \l 5DCD
  \l 5DCE
  \l 5DCF
  \l 5DD0
  \l 5DD1
  \l 5DD2
  \l 5DD3
  \l 5DD4
  \l 5DD5
  \l 5DD6
  \l 5DD7
  \l 5DD8
  \l 5DD9
  \l 5DDA
  \l 5DDB
  \l 5DDC
  \l 5DDD
  \l 5DDE
  \l 5DDF
  \l 5DE0
  \l 5DE1
  \l 5DE2
  \l 5DE3
  \l 5DE4
  \l 5DE5
  \l 5DE6
  \l 5DE7
  \l 5DE8
  \l 5DE9
  \l 5DEA
  \l 5DEB
  \l 5DEC
  \l 5DED
  \l 5DEE
  \l 5DEF
  \l 5DF0
  \l 5DF1
  \l 5DF2
  \l 5DF3
  \l 5DF4
  \l 5DF5
  \l 5DF6
  \l 5DF7
  \l 5DF8
  \l 5DF9
  \l 5DFA
  \l 5DFB
  \l 5DFC
  \l 5DFD
  \l 5DFE
  \l 5DFF
  \l 5E00
  \l 5E01
  \l 5E02
  \l 5E03
  \l 5E04
  \l 5E05
  \l 5E06
  \l 5E07
  \l 5E08
  \l 5E09
  \l 5E0A
  \l 5E0B
  \l 5E0C
  \l 5E0D
  \l 5E0E
  \l 5E0F
  \l 5E10
  \l 5E11
  \l 5E12
  \l 5E13
  \l 5E14
  \l 5E15
  \l 5E16
  \l 5E17
  \l 5E18
  \l 5E19
  \l 5E1A
  \l 5E1B
  \l 5E1C
  \l 5E1D
  \l 5E1E
  \l 5E1F
  \l 5E20
  \l 5E21
  \l 5E22
  \l 5E23
  \l 5E24
  \l 5E25
  \l 5E26
  \l 5E27
  \l 5E28
  \l 5E29
  \l 5E2A
  \l 5E2B
  \l 5E2C
  \l 5E2D
  \l 5E2E
  \l 5E2F
  \l 5E30
  \l 5E31
  \l 5E32
  \l 5E33
  \l 5E34
  \l 5E35
  \l 5E36
  \l 5E37
  \l 5E38
  \l 5E39
  \l 5E3A
  \l 5E3B
  \l 5E3C
  \l 5E3D
  \l 5E3E
  \l 5E3F
  \l 5E40
  \l 5E41
  \l 5E42
  \l 5E43
  \l 5E44
  \l 5E45
  \l 5E46
  \l 5E47
  \l 5E48
  \l 5E49
  \l 5E4A
  \l 5E4B
  \l 5E4C
  \l 5E4D
  \l 5E4E
  \l 5E4F
  \l 5E50
  \l 5E51
  \l 5E52
  \l 5E53
  \l 5E54
  \l 5E55
  \l 5E56
  \l 5E57
  \l 5E58
  \l 5E59
  \l 5E5A
  \l 5E5B
  \l 5E5C
  \l 5E5D
  \l 5E5E
  \l 5E5F
  \l 5E60
  \l 5E61
  \l 5E62
  \l 5E63
  \l 5E64
  \l 5E65
  \l 5E66
  \l 5E67
  \l 5E68
  \l 5E69
  \l 5E6A
  \l 5E6B
  \l 5E6C
  \l 5E6D
  \l 5E6E
  \l 5E6F
  \l 5E70
  \l 5E71
  \l 5E72
  \l 5E73
  \l 5E74
  \l 5E75
  \l 5E76
  \l 5E77
  \l 5E78
  \l 5E79
  \l 5E7A
  \l 5E7B
  \l 5E7C
  \l 5E7D
  \l 5E7E
  \l 5E7F
  \l 5E80
  \l 5E81
  \l 5E82
  \l 5E83
  \l 5E84
  \l 5E85
  \l 5E86
  \l 5E87
  \l 5E88
  \l 5E89
  \l 5E8A
  \l 5E8B
  \l 5E8C
  \l 5E8D
  \l 5E8E
  \l 5E8F
  \l 5E90
  \l 5E91
  \l 5E92
  \l 5E93
  \l 5E94
  \l 5E95
  \l 5E96
  \l 5E97
  \l 5E98
  \l 5E99
  \l 5E9A
  \l 5E9B
  \l 5E9C
  \l 5E9D
  \l 5E9E
  \l 5E9F
  \l 5EA0
  \l 5EA1
  \l 5EA2
  \l 5EA3
  \l 5EA4
  \l 5EA5
  \l 5EA6
  \l 5EA7
  \l 5EA8
  \l 5EA9
  \l 5EAA
  \l 5EAB
  \l 5EAC
  \l 5EAD
  \l 5EAE
  \l 5EAF
  \l 5EB0
  \l 5EB1
  \l 5EB2
  \l 5EB3
  \l 5EB4
  \l 5EB5
  \l 5EB6
  \l 5EB7
  \l 5EB8
  \l 5EB9
  \l 5EBA
  \l 5EBB
  \l 5EBC
  \l 5EBD
  \l 5EBE
  \l 5EBF
  \l 5EC0
  \l 5EC1
  \l 5EC2
  \l 5EC3
  \l 5EC4
  \l 5EC5
  \l 5EC6
  \l 5EC7
  \l 5EC8
  \l 5EC9
  \l 5ECA
  \l 5ECB
  \l 5ECC
  \l 5ECD
  \l 5ECE
  \l 5ECF
  \l 5ED0
  \l 5ED1
  \l 5ED2
  \l 5ED3
  \l 5ED4
  \l 5ED5
  \l 5ED6
  \l 5ED7
  \l 5ED8
  \l 5ED9
  \l 5EDA
  \l 5EDB
  \l 5EDC
  \l 5EDD
  \l 5EDE
  \l 5EDF
  \l 5EE0
  \l 5EE1
  \l 5EE2
  \l 5EE3
  \l 5EE4
  \l 5EE5
  \l 5EE6
  \l 5EE7
  \l 5EE8
  \l 5EE9
  \l 5EEA
  \l 5EEB
  \l 5EEC
  \l 5EED
  \l 5EEE
  \l 5EEF
  \l 5EF0
  \l 5EF1
  \l 5EF2
  \l 5EF3
  \l 5EF4
  \l 5EF5
  \l 5EF6
  \l 5EF7
  \l 5EF8
  \l 5EF9
  \l 5EFA
  \l 5EFB
  \l 5EFC
  \l 5EFD
  \l 5EFE
  \l 5EFF
  \l 5F00
  \l 5F01
  \l 5F02
  \l 5F03
  \l 5F04
  \l 5F05
  \l 5F06
  \l 5F07
  \l 5F08
  \l 5F09
  \l 5F0A
  \l 5F0B
  \l 5F0C
  \l 5F0D
  \l 5F0E
  \l 5F0F
  \l 5F10
  \l 5F11
  \l 5F12
  \l 5F13
  \l 5F14
  \l 5F15
  \l 5F16
  \l 5F17
  \l 5F18
  \l 5F19
  \l 5F1A
  \l 5F1B
  \l 5F1C
  \l 5F1D
  \l 5F1E
  \l 5F1F
  \l 5F20
  \l 5F21
  \l 5F22
  \l 5F23
  \l 5F24
  \l 5F25
  \l 5F26
  \l 5F27
  \l 5F28
  \l 5F29
  \l 5F2A
  \l 5F2B
  \l 5F2C
  \l 5F2D
  \l 5F2E
  \l 5F2F
  \l 5F30
  \l 5F31
  \l 5F32
  \l 5F33
  \l 5F34
  \l 5F35
  \l 5F36
  \l 5F37
  \l 5F38
  \l 5F39
  \l 5F3A
  \l 5F3B
  \l 5F3C
  \l 5F3D
  \l 5F3E
  \l 5F3F
  \l 5F40
  \l 5F41
  \l 5F42
  \l 5F43
  \l 5F44
  \l 5F45
  \l 5F46
  \l 5F47
  \l 5F48
  \l 5F49
  \l 5F4A
  \l 5F4B
  \l 5F4C
  \l 5F4D
  \l 5F4E
  \l 5F4F
  \l 5F50
  \l 5F51
  \l 5F52
  \l 5F53
  \l 5F54
  \l 5F55
  \l 5F56
  \l 5F57
  \l 5F58
  \l 5F59
  \l 5F5A
  \l 5F5B
  \l 5F5C
  \l 5F5D
  \l 5F5E
  \l 5F5F
  \l 5F60
  \l 5F61
  \l 5F62
  \l 5F63
  \l 5F64
  \l 5F65
  \l 5F66
  \l 5F67
  \l 5F68
  \l 5F69
  \l 5F6A
  \l 5F6B
  \l 5F6C
  \l 5F6D
  \l 5F6E
  \l 5F6F
  \l 5F70
  \l 5F71
  \l 5F72
  \l 5F73
  \l 5F74
  \l 5F75
  \l 5F76
  \l 5F77
  \l 5F78
  \l 5F79
  \l 5F7A
  \l 5F7B
  \l 5F7C
  \l 5F7D
  \l 5F7E
  \l 5F7F
  \l 5F80
  \l 5F81
  \l 5F82
  \l 5F83
  \l 5F84
  \l 5F85
  \l 5F86
  \l 5F87
  \l 5F88
  \l 5F89
  \l 5F8A
  \l 5F8B
  \l 5F8C
  \l 5F8D
  \l 5F8E
  \l 5F8F
  \l 5F90
  \l 5F91
  \l 5F92
  \l 5F93
  \l 5F94
  \l 5F95
  \l 5F96
  \l 5F97
  \l 5F98
  \l 5F99
  \l 5F9A
  \l 5F9B
  \l 5F9C
  \l 5F9D
  \l 5F9E
  \l 5F9F
  \l 5FA0
  \l 5FA1
  \l 5FA2
  \l 5FA3
  \l 5FA4
  \l 5FA5
  \l 5FA6
  \l 5FA7
  \l 5FA8
  \l 5FA9
  \l 5FAA
  \l 5FAB
  \l 5FAC
  \l 5FAD
  \l 5FAE
  \l 5FAF
  \l 5FB0
  \l 5FB1
  \l 5FB2
  \l 5FB3
  \l 5FB4
  \l 5FB5
  \l 5FB6
  \l 5FB7
  \l 5FB8
  \l 5FB9
  \l 5FBA
  \l 5FBB
  \l 5FBC
  \l 5FBD
  \l 5FBE
  \l 5FBF
  \l 5FC0
  \l 5FC1
  \l 5FC2
  \l 5FC3
  \l 5FC4
  \l 5FC5
  \l 5FC6
  \l 5FC7
  \l 5FC8
  \l 5FC9
  \l 5FCA
  \l 5FCB
  \l 5FCC
  \l 5FCD
  \l 5FCE
  \l 5FCF
  \l 5FD0
  \l 5FD1
  \l 5FD2
  \l 5FD3
  \l 5FD4
  \l 5FD5
  \l 5FD6
  \l 5FD7
  \l 5FD8
  \l 5FD9
  \l 5FDA
  \l 5FDB
  \l 5FDC
  \l 5FDD
  \l 5FDE
  \l 5FDF
  \l 5FE0
  \l 5FE1
  \l 5FE2
  \l 5FE3
  \l 5FE4
  \l 5FE5
  \l 5FE6
  \l 5FE7
  \l 5FE8
  \l 5FE9
  \l 5FEA
  \l 5FEB
  \l 5FEC
  \l 5FED
  \l 5FEE
  \l 5FEF
  \l 5FF0
  \l 5FF1
  \l 5FF2
  \l 5FF3
  \l 5FF4
  \l 5FF5
  \l 5FF6
  \l 5FF7
  \l 5FF8
  \l 5FF9
  \l 5FFA
  \l 5FFB
  \l 5FFC
  \l 5FFD
  \l 5FFE
  \l 5FFF
  \l 6000
  \l 6001
  \l 6002
  \l 6003
  \l 6004
  \l 6005
  \l 6006
  \l 6007
  \l 6008
  \l 6009
  \l 600A
  \l 600B
  \l 600C
  \l 600D
  \l 600E
  \l 600F
  \l 6010
  \l 6011
  \l 6012
  \l 6013
  \l 6014
  \l 6015
  \l 6016
  \l 6017
  \l 6018
  \l 6019
  \l 601A
  \l 601B
  \l 601C
  \l 601D
  \l 601E
  \l 601F
  \l 6020
  \l 6021
  \l 6022
  \l 6023
  \l 6024
  \l 6025
  \l 6026
  \l 6027
  \l 6028
  \l 6029
  \l 602A
  \l 602B
  \l 602C
  \l 602D
  \l 602E
  \l 602F
  \l 6030
  \l 6031
  \l 6032
  \l 6033
  \l 6034
  \l 6035
  \l 6036
  \l 6037
  \l 6038
  \l 6039
  \l 603A
  \l 603B
  \l 603C
  \l 603D
  \l 603E
  \l 603F
  \l 6040
  \l 6041
  \l 6042
  \l 6043
  \l 6044
  \l 6045
  \l 6046
  \l 6047
  \l 6048
  \l 6049
  \l 604A
  \l 604B
  \l 604C
  \l 604D
  \l 604E
  \l 604F
  \l 6050
  \l 6051
  \l 6052
  \l 6053
  \l 6054
  \l 6055
  \l 6056
  \l 6057
  \l 6058
  \l 6059
  \l 605A
  \l 605B
  \l 605C
  \l 605D
  \l 605E
  \l 605F
  \l 6060
  \l 6061
  \l 6062
  \l 6063
  \l 6064
  \l 6065
  \l 6066
  \l 6067
  \l 6068
  \l 6069
  \l 606A
  \l 606B
  \l 606C
  \l 606D
  \l 606E
  \l 606F
  \l 6070
  \l 6071
  \l 6072
  \l 6073
  \l 6074
  \l 6075
  \l 6076
  \l 6077
  \l 6078
  \l 6079
  \l 607A
  \l 607B
  \l 607C
  \l 607D
  \l 607E
  \l 607F
  \l 6080
  \l 6081
  \l 6082
  \l 6083
  \l 6084
  \l 6085
  \l 6086
  \l 6087
  \l 6088
  \l 6089
  \l 608A
  \l 608B
  \l 608C
  \l 608D
  \l 608E
  \l 608F
  \l 6090
  \l 6091
  \l 6092
  \l 6093
  \l 6094
  \l 6095
  \l 6096
  \l 6097
  \l 6098
  \l 6099
  \l 609A
  \l 609B
  \l 609C
  \l 609D
  \l 609E
  \l 609F
  \l 60A0
  \l 60A1
  \l 60A2
  \l 60A3
  \l 60A4
  \l 60A5
  \l 60A6
  \l 60A7
  \l 60A8
  \l 60A9
  \l 60AA
  \l 60AB
  \l 60AC
  \l 60AD
  \l 60AE
  \l 60AF
  \l 60B0
  \l 60B1
  \l 60B2
  \l 60B3
  \l 60B4
  \l 60B5
  \l 60B6
  \l 60B7
  \l 60B8
  \l 60B9
  \l 60BA
  \l 60BB
  \l 60BC
  \l 60BD
  \l 60BE
  \l 60BF
  \l 60C0
  \l 60C1
  \l 60C2
  \l 60C3
  \l 60C4
  \l 60C5
  \l 60C6
  \l 60C7
  \l 60C8
  \l 60C9
  \l 60CA
  \l 60CB
  \l 60CC
  \l 60CD
  \l 60CE
  \l 60CF
  \l 60D0
  \l 60D1
  \l 60D2
  \l 60D3
  \l 60D4
  \l 60D5
  \l 60D6
  \l 60D7
  \l 60D8
  \l 60D9
  \l 60DA
  \l 60DB
  \l 60DC
  \l 60DD
  \l 60DE
  \l 60DF
  \l 60E0
  \l 60E1
  \l 60E2
  \l 60E3
  \l 60E4
  \l 60E5
  \l 60E6
  \l 60E7
  \l 60E8
  \l 60E9
  \l 60EA
  \l 60EB
  \l 60EC
  \l 60ED
  \l 60EE
  \l 60EF
  \l 60F0
  \l 60F1
  \l 60F2
  \l 60F3
  \l 60F4
  \l 60F5
  \l 60F6
  \l 60F7
  \l 60F8
  \l 60F9
  \l 60FA
  \l 60FB
  \l 60FC
  \l 60FD
  \l 60FE
  \l 60FF
  \l 6100
  \l 6101
  \l 6102
  \l 6103
  \l 6104
  \l 6105
  \l 6106
  \l 6107
  \l 6108
  \l 6109
  \l 610A
  \l 610B
  \l 610C
  \l 610D
  \l 610E
  \l 610F
  \l 6110
  \l 6111
  \l 6112
  \l 6113
  \l 6114
  \l 6115
  \l 6116
  \l 6117
  \l 6118
  \l 6119
  \l 611A
  \l 611B
  \l 611C
  \l 611D
  \l 611E
  \l 611F
  \l 6120
  \l 6121
  \l 6122
  \l 6123
  \l 6124
  \l 6125
  \l 6126
  \l 6127
  \l 6128
  \l 6129
  \l 612A
  \l 612B
  \l 612C
  \l 612D
  \l 612E
  \l 612F
  \l 6130
  \l 6131
  \l 6132
  \l 6133
  \l 6134
  \l 6135
  \l 6136
  \l 6137
  \l 6138
  \l 6139
  \l 613A
  \l 613B
  \l 613C
  \l 613D
  \l 613E
  \l 613F
  \l 6140
  \l 6141
  \l 6142
  \l 6143
  \l 6144
  \l 6145
  \l 6146
  \l 6147
  \l 6148
  \l 6149
  \l 614A
  \l 614B
  \l 614C
  \l 614D
  \l 614E
  \l 614F
  \l 6150
  \l 6151
  \l 6152
  \l 6153
  \l 6154
  \l 6155
  \l 6156
  \l 6157
  \l 6158
  \l 6159
  \l 615A
  \l 615B
  \l 615C
  \l 615D
  \l 615E
  \l 615F
  \l 6160
  \l 6161
  \l 6162
  \l 6163
  \l 6164
  \l 6165
  \l 6166
  \l 6167
  \l 6168
  \l 6169
  \l 616A
  \l 616B
  \l 616C
  \l 616D
  \l 616E
  \l 616F
  \l 6170
  \l 6171
  \l 6172
  \l 6173
  \l 6174
  \l 6175
  \l 6176
  \l 6177
  \l 6178
  \l 6179
  \l 617A
  \l 617B
  \l 617C
  \l 617D
  \l 617E
  \l 617F
  \l 6180
  \l 6181
  \l 6182
  \l 6183
  \l 6184
  \l 6185
  \l 6186
  \l 6187
  \l 6188
  \l 6189
  \l 618A
  \l 618B
  \l 618C
  \l 618D
  \l 618E
  \l 618F
  \l 6190
  \l 6191
  \l 6192
  \l 6193
  \l 6194
  \l 6195
  \l 6196
  \l 6197
  \l 6198
  \l 6199
  \l 619A
  \l 619B
  \l 619C
  \l 619D
  \l 619E
  \l 619F
  \l 61A0
  \l 61A1
  \l 61A2
  \l 61A3
  \l 61A4
  \l 61A5
  \l 61A6
  \l 61A7
  \l 61A8
  \l 61A9
  \l 61AA
  \l 61AB
  \l 61AC
  \l 61AD
  \l 61AE
  \l 61AF
  \l 61B0
  \l 61B1
  \l 61B2
  \l 61B3
  \l 61B4
  \l 61B5
  \l 61B6
  \l 61B7
  \l 61B8
  \l 61B9
  \l 61BA
  \l 61BB
  \l 61BC
  \l 61BD
  \l 61BE
  \l 61BF
  \l 61C0
  \l 61C1
  \l 61C2
  \l 61C3
  \l 61C4
  \l 61C5
  \l 61C6
  \l 61C7
  \l 61C8
  \l 61C9
  \l 61CA
  \l 61CB
  \l 61CC
  \l 61CD
  \l 61CE
  \l 61CF
  \l 61D0
  \l 61D1
  \l 61D2
  \l 61D3
  \l 61D4
  \l 61D5
  \l 61D6
  \l 61D7
  \l 61D8
  \l 61D9
  \l 61DA
  \l 61DB
  \l 61DC
  \l 61DD
  \l 61DE
  \l 61DF
  \l 61E0
  \l 61E1
  \l 61E2
  \l 61E3
  \l 61E4
  \l 61E5
  \l 61E6
  \l 61E7
  \l 61E8
  \l 61E9
  \l 61EA
  \l 61EB
  \l 61EC
  \l 61ED
  \l 61EE
  \l 61EF
  \l 61F0
  \l 61F1
  \l 61F2
  \l 61F3
  \l 61F4
  \l 61F5
  \l 61F6
  \l 61F7
  \l 61F8
  \l 61F9
  \l 61FA
  \l 61FB
  \l 61FC
  \l 61FD
  \l 61FE
  \l 61FF
  \l 6200
  \l 6201
  \l 6202
  \l 6203
  \l 6204
  \l 6205
  \l 6206
  \l 6207
  \l 6208
  \l 6209
  \l 620A
  \l 620B
  \l 620C
  \l 620D
  \l 620E
  \l 620F
  \l 6210
  \l 6211
  \l 6212
  \l 6213
  \l 6214
  \l 6215
  \l 6216
  \l 6217
  \l 6218
  \l 6219
  \l 621A
  \l 621B
  \l 621C
  \l 621D
  \l 621E
  \l 621F
  \l 6220
  \l 6221
  \l 6222
  \l 6223
  \l 6224
  \l 6225
  \l 6226
  \l 6227
  \l 6228
  \l 6229
  \l 622A
  \l 622B
  \l 622C
  \l 622D
  \l 622E
  \l 622F
  \l 6230
  \l 6231
  \l 6232
  \l 6233
  \l 6234
  \l 6235
  \l 6236
  \l 6237
  \l 6238
  \l 6239
  \l 623A
  \l 623B
  \l 623C
  \l 623D
  \l 623E
  \l 623F
  \l 6240
  \l 6241
  \l 6242
  \l 6243
  \l 6244
  \l 6245
  \l 6246
  \l 6247
  \l 6248
  \l 6249
  \l 624A
  \l 624B
  \l 624C
  \l 624D
  \l 624E
  \l 624F
  \l 6250
  \l 6251
  \l 6252
  \l 6253
  \l 6254
  \l 6255
  \l 6256
  \l 6257
  \l 6258
  \l 6259
  \l 625A
  \l 625B
  \l 625C
  \l 625D
  \l 625E
  \l 625F
  \l 6260
  \l 6261
  \l 6262
  \l 6263
  \l 6264
  \l 6265
  \l 6266
  \l 6267
  \l 6268
  \l 6269
  \l 626A
  \l 626B
  \l 626C
  \l 626D
  \l 626E
  \l 626F
  \l 6270
  \l 6271
  \l 6272
  \l 6273
  \l 6274
  \l 6275
  \l 6276
  \l 6277
  \l 6278
  \l 6279
  \l 627A
  \l 627B
  \l 627C
  \l 627D
  \l 627E
  \l 627F
  \l 6280
  \l 6281
  \l 6282
  \l 6283
  \l 6284
  \l 6285
  \l 6286
  \l 6287
  \l 6288
  \l 6289
  \l 628A
  \l 628B
  \l 628C
  \l 628D
  \l 628E
  \l 628F
  \l 6290
  \l 6291
  \l 6292
  \l 6293
  \l 6294
  \l 6295
  \l 6296
  \l 6297
  \l 6298
  \l 6299
  \l 629A
  \l 629B
  \l 629C
  \l 629D
  \l 629E
  \l 629F
  \l 62A0
  \l 62A1
  \l 62A2
  \l 62A3
  \l 62A4
  \l 62A5
  \l 62A6
  \l 62A7
  \l 62A8
  \l 62A9
  \l 62AA
  \l 62AB
  \l 62AC
  \l 62AD
  \l 62AE
  \l 62AF
  \l 62B0
  \l 62B1
  \l 62B2
  \l 62B3
  \l 62B4
  \l 62B5
  \l 62B6
  \l 62B7
  \l 62B8
  \l 62B9
  \l 62BA
  \l 62BB
  \l 62BC
  \l 62BD
  \l 62BE
  \l 62BF
  \l 62C0
  \l 62C1
  \l 62C2
  \l 62C3
  \l 62C4
  \l 62C5
  \l 62C6
  \l 62C7
  \l 62C8
  \l 62C9
  \l 62CA
  \l 62CB
  \l 62CC
  \l 62CD
  \l 62CE
  \l 62CF
  \l 62D0
  \l 62D1
  \l 62D2
  \l 62D3
  \l 62D4
  \l 62D5
  \l 62D6
  \l 62D7
  \l 62D8
  \l 62D9
  \l 62DA
  \l 62DB
  \l 62DC
  \l 62DD
  \l 62DE
  \l 62DF
  \l 62E0
  \l 62E1
  \l 62E2
  \l 62E3
  \l 62E4
  \l 62E5
  \l 62E6
  \l 62E7
  \l 62E8
  \l 62E9
  \l 62EA
  \l 62EB
  \l 62EC
  \l 62ED
  \l 62EE
  \l 62EF
  \l 62F0
  \l 62F1
  \l 62F2
  \l 62F3
  \l 62F4
  \l 62F5
  \l 62F6
  \l 62F7
  \l 62F8
  \l 62F9
  \l 62FA
  \l 62FB
  \l 62FC
  \l 62FD
  \l 62FE
  \l 62FF
  \l 6300
  \l 6301
  \l 6302
  \l 6303
  \l 6304
  \l 6305
  \l 6306
  \l 6307
  \l 6308
  \l 6309
  \l 630A
  \l 630B
  \l 630C
  \l 630D
  \l 630E
  \l 630F
  \l 6310
  \l 6311
  \l 6312
  \l 6313
  \l 6314
  \l 6315
  \l 6316
  \l 6317
  \l 6318
  \l 6319
  \l 631A
  \l 631B
  \l 631C
  \l 631D
  \l 631E
  \l 631F
  \l 6320
  \l 6321
  \l 6322
  \l 6323
  \l 6324
  \l 6325
  \l 6326
  \l 6327
  \l 6328
  \l 6329
  \l 632A
  \l 632B
  \l 632C
  \l 632D
  \l 632E
  \l 632F
  \l 6330
  \l 6331
  \l 6332
  \l 6333
  \l 6334
  \l 6335
  \l 6336
  \l 6337
  \l 6338
  \l 6339
  \l 633A
  \l 633B
  \l 633C
  \l 633D
  \l 633E
  \l 633F
  \l 6340
  \l 6341
  \l 6342
  \l 6343
  \l 6344
  \l 6345
  \l 6346
  \l 6347
  \l 6348
  \l 6349
  \l 634A
  \l 634B
  \l 634C
  \l 634D
  \l 634E
  \l 634F
  \l 6350
  \l 6351
  \l 6352
  \l 6353
  \l 6354
  \l 6355
  \l 6356
  \l 6357
  \l 6358
  \l 6359
  \l 635A
  \l 635B
  \l 635C
  \l 635D
  \l 635E
  \l 635F
  \l 6360
  \l 6361
  \l 6362
  \l 6363
  \l 6364
  \l 6365
  \l 6366
  \l 6367
  \l 6368
  \l 6369
  \l 636A
  \l 636B
  \l 636C
  \l 636D
  \l 636E
  \l 636F
  \l 6370
  \l 6371
  \l 6372
  \l 6373
  \l 6374
  \l 6375
  \l 6376
  \l 6377
  \l 6378
  \l 6379
  \l 637A
  \l 637B
  \l 637C
  \l 637D
  \l 637E
  \l 637F
  \l 6380
  \l 6381
  \l 6382
  \l 6383
  \l 6384
  \l 6385
  \l 6386
  \l 6387
  \l 6388
  \l 6389
  \l 638A
  \l 638B
  \l 638C
  \l 638D
  \l 638E
  \l 638F
  \l 6390
  \l 6391
  \l 6392
  \l 6393
  \l 6394
  \l 6395
  \l 6396
  \l 6397
  \l 6398
  \l 6399
  \l 639A
  \l 639B
  \l 639C
  \l 639D
  \l 639E
  \l 639F
  \l 63A0
  \l 63A1
  \l 63A2
  \l 63A3
  \l 63A4
  \l 63A5
  \l 63A6
  \l 63A7
  \l 63A8
  \l 63A9
  \l 63AA
  \l 63AB
  \l 63AC
  \l 63AD
  \l 63AE
  \l 63AF
  \l 63B0
  \l 63B1
  \l 63B2
  \l 63B3
  \l 63B4
  \l 63B5
  \l 63B6
  \l 63B7
  \l 63B8
  \l 63B9
  \l 63BA
  \l 63BB
  \l 63BC
  \l 63BD
  \l 63BE
  \l 63BF
  \l 63C0
  \l 63C1
  \l 63C2
  \l 63C3
  \l 63C4
  \l 63C5
  \l 63C6
  \l 63C7
  \l 63C8
  \l 63C9
  \l 63CA
  \l 63CB
  \l 63CC
  \l 63CD
  \l 63CE
  \l 63CF
  \l 63D0
  \l 63D1
  \l 63D2
  \l 63D3
  \l 63D4
  \l 63D5
  \l 63D6
  \l 63D7
  \l 63D8
  \l 63D9
  \l 63DA
  \l 63DB
  \l 63DC
  \l 63DD
  \l 63DE
  \l 63DF
  \l 63E0
  \l 63E1
  \l 63E2
  \l 63E3
  \l 63E4
  \l 63E5
  \l 63E6
  \l 63E7
  \l 63E8
  \l 63E9
  \l 63EA
  \l 63EB
  \l 63EC
  \l 63ED
  \l 63EE
  \l 63EF
  \l 63F0
  \l 63F1
  \l 63F2
  \l 63F3
  \l 63F4
  \l 63F5
  \l 63F6
  \l 63F7
  \l 63F8
  \l 63F9
  \l 63FA
  \l 63FB
  \l 63FC
  \l 63FD
  \l 63FE
  \l 63FF
  \l 6400
  \l 6401
  \l 6402
  \l 6403
  \l 6404
  \l 6405
  \l 6406
  \l 6407
  \l 6408
  \l 6409
  \l 640A
  \l 640B
  \l 640C
  \l 640D
  \l 640E
  \l 640F
  \l 6410
  \l 6411
  \l 6412
  \l 6413
  \l 6414
  \l 6415
  \l 6416
  \l 6417
  \l 6418
  \l 6419
  \l 641A
  \l 641B
  \l 641C
  \l 641D
  \l 641E
  \l 641F
  \l 6420
  \l 6421
  \l 6422
  \l 6423
  \l 6424
  \l 6425
  \l 6426
  \l 6427
  \l 6428
  \l 6429
  \l 642A
  \l 642B
  \l 642C
  \l 642D
  \l 642E
  \l 642F
  \l 6430
  \l 6431
  \l 6432
  \l 6433
  \l 6434
  \l 6435
  \l 6436
  \l 6437
  \l 6438
  \l 6439
  \l 643A
  \l 643B
  \l 643C
  \l 643D
  \l 643E
  \l 643F
  \l 6440
  \l 6441
  \l 6442
  \l 6443
  \l 6444
  \l 6445
  \l 6446
  \l 6447
  \l 6448
  \l 6449
  \l 644A
  \l 644B
  \l 644C
  \l 644D
  \l 644E
  \l 644F
  \l 6450
  \l 6451
  \l 6452
  \l 6453
  \l 6454
  \l 6455
  \l 6456
  \l 6457
  \l 6458
  \l 6459
  \l 645A
  \l 645B
  \l 645C
  \l 645D
  \l 645E
  \l 645F
  \l 6460
  \l 6461
  \l 6462
  \l 6463
  \l 6464
  \l 6465
  \l 6466
  \l 6467
  \l 6468
  \l 6469
  \l 646A
  \l 646B
  \l 646C
  \l 646D
  \l 646E
  \l 646F
  \l 6470
  \l 6471
  \l 6472
  \l 6473
  \l 6474
  \l 6475
  \l 6476
  \l 6477
  \l 6478
  \l 6479
  \l 647A
  \l 647B
  \l 647C
  \l 647D
  \l 647E
  \l 647F
  \l 6480
  \l 6481
  \l 6482
  \l 6483
  \l 6484
  \l 6485
  \l 6486
  \l 6487
  \l 6488
  \l 6489
  \l 648A
  \l 648B
  \l 648C
  \l 648D
  \l 648E
  \l 648F
  \l 6490
  \l 6491
  \l 6492
  \l 6493
  \l 6494
  \l 6495
  \l 6496
  \l 6497
  \l 6498
  \l 6499
  \l 649A
  \l 649B
  \l 649C
  \l 649D
  \l 649E
  \l 649F
  \l 64A0
  \l 64A1
  \l 64A2
  \l 64A3
  \l 64A4
  \l 64A5
  \l 64A6
  \l 64A7
  \l 64A8
  \l 64A9
  \l 64AA
  \l 64AB
  \l 64AC
  \l 64AD
  \l 64AE
  \l 64AF
  \l 64B0
  \l 64B1
  \l 64B2
  \l 64B3
  \l 64B4
  \l 64B5
  \l 64B6
  \l 64B7
  \l 64B8
  \l 64B9
  \l 64BA
  \l 64BB
  \l 64BC
  \l 64BD
  \l 64BE
  \l 64BF
  \l 64C0
  \l 64C1
  \l 64C2
  \l 64C3
  \l 64C4
  \l 64C5
  \l 64C6
  \l 64C7
  \l 64C8
  \l 64C9
  \l 64CA
  \l 64CB
  \l 64CC
  \l 64CD
  \l 64CE
  \l 64CF
  \l 64D0
  \l 64D1
  \l 64D2
  \l 64D3
  \l 64D4
  \l 64D5
  \l 64D6
  \l 64D7
  \l 64D8
  \l 64D9
  \l 64DA
  \l 64DB
  \l 64DC
  \l 64DD
  \l 64DE
  \l 64DF
  \l 64E0
  \l 64E1
  \l 64E2
  \l 64E3
  \l 64E4
  \l 64E5
  \l 64E6
  \l 64E7
  \l 64E8
  \l 64E9
  \l 64EA
  \l 64EB
  \l 64EC
  \l 64ED
  \l 64EE
  \l 64EF
  \l 64F0
  \l 64F1
  \l 64F2
  \l 64F3
  \l 64F4
  \l 64F5
  \l 64F6
  \l 64F7
  \l 64F8
  \l 64F9
  \l 64FA
  \l 64FB
  \l 64FC
  \l 64FD
  \l 64FE
  \l 64FF
  \l 6500
  \l 6501
  \l 6502
  \l 6503
  \l 6504
  \l 6505
  \l 6506
  \l 6507
  \l 6508
  \l 6509
  \l 650A
  \l 650B
  \l 650C
  \l 650D
  \l 650E
  \l 650F
  \l 6510
  \l 6511
  \l 6512
  \l 6513
  \l 6514
  \l 6515
  \l 6516
  \l 6517
  \l 6518
  \l 6519
  \l 651A
  \l 651B
  \l 651C
  \l 651D
  \l 651E
  \l 651F
  \l 6520
  \l 6521
  \l 6522
  \l 6523
  \l 6524
  \l 6525
  \l 6526
  \l 6527
  \l 6528
  \l 6529
  \l 652A
  \l 652B
  \l 652C
  \l 652D
  \l 652E
  \l 652F
  \l 6530
  \l 6531
  \l 6532
  \l 6533
  \l 6534
  \l 6535
  \l 6536
  \l 6537
  \l 6538
  \l 6539
  \l 653A
  \l 653B
  \l 653C
  \l 653D
  \l 653E
  \l 653F
  \l 6540
  \l 6541
  \l 6542
  \l 6543
  \l 6544
  \l 6545
  \l 6546
  \l 6547
  \l 6548
  \l 6549
  \l 654A
  \l 654B
  \l 654C
  \l 654D
  \l 654E
  \l 654F
  \l 6550
  \l 6551
  \l 6552
  \l 6553
  \l 6554
  \l 6555
  \l 6556
  \l 6557
  \l 6558
  \l 6559
  \l 655A
  \l 655B
  \l 655C
  \l 655D
  \l 655E
  \l 655F
  \l 6560
  \l 6561
  \l 6562
  \l 6563
  \l 6564
  \l 6565
  \l 6566
  \l 6567
  \l 6568
  \l 6569
  \l 656A
  \l 656B
  \l 656C
  \l 656D
  \l 656E
  \l 656F
  \l 6570
  \l 6571
  \l 6572
  \l 6573
  \l 6574
  \l 6575
  \l 6576
  \l 6577
  \l 6578
  \l 6579
  \l 657A
  \l 657B
  \l 657C
  \l 657D
  \l 657E
  \l 657F
  \l 6580
  \l 6581
  \l 6582
  \l 6583
  \l 6584
  \l 6585
  \l 6586
  \l 6587
  \l 6588
  \l 6589
  \l 658A
  \l 658B
  \l 658C
  \l 658D
  \l 658E
  \l 658F
  \l 6590
  \l 6591
  \l 6592
  \l 6593
  \l 6594
  \l 6595
  \l 6596
  \l 6597
  \l 6598
  \l 6599
  \l 659A
  \l 659B
  \l 659C
  \l 659D
  \l 659E
  \l 659F
  \l 65A0
  \l 65A1
  \l 65A2
  \l 65A3
  \l 65A4
  \l 65A5
  \l 65A6
  \l 65A7
  \l 65A8
  \l 65A9
  \l 65AA
  \l 65AB
  \l 65AC
  \l 65AD
  \l 65AE
  \l 65AF
  \l 65B0
  \l 65B1
  \l 65B2
  \l 65B3
  \l 65B4
  \l 65B5
  \l 65B6
  \l 65B7
  \l 65B8
  \l 65B9
  \l 65BA
  \l 65BB
  \l 65BC
  \l 65BD
  \l 65BE
  \l 65BF
  \l 65C0
  \l 65C1
  \l 65C2
  \l 65C3
  \l 65C4
  \l 65C5
  \l 65C6
  \l 65C7
  \l 65C8
  \l 65C9
  \l 65CA
  \l 65CB
  \l 65CC
  \l 65CD
  \l 65CE
  \l 65CF
  \l 65D0
  \l 65D1
  \l 65D2
  \l 65D3
  \l 65D4
  \l 65D5
  \l 65D6
  \l 65D7
  \l 65D8
  \l 65D9
  \l 65DA
  \l 65DB
  \l 65DC
  \l 65DD
  \l 65DE
  \l 65DF
  \l 65E0
  \l 65E1
  \l 65E2
  \l 65E3
  \l 65E4
  \l 65E5
  \l 65E6
  \l 65E7
  \l 65E8
  \l 65E9
  \l 65EA
  \l 65EB
  \l 65EC
  \l 65ED
  \l 65EE
  \l 65EF
  \l 65F0
  \l 65F1
  \l 65F2
  \l 65F3
  \l 65F4
  \l 65F5
  \l 65F6
  \l 65F7
  \l 65F8
  \l 65F9
  \l 65FA
  \l 65FB
  \l 65FC
  \l 65FD
  \l 65FE
  \l 65FF
  \l 6600
  \l 6601
  \l 6602
  \l 6603
  \l 6604
  \l 6605
  \l 6606
  \l 6607
  \l 6608
  \l 6609
  \l 660A
  \l 660B
  \l 660C
  \l 660D
  \l 660E
  \l 660F
  \l 6610
  \l 6611
  \l 6612
  \l 6613
  \l 6614
  \l 6615
  \l 6616
  \l 6617
  \l 6618
  \l 6619
  \l 661A
  \l 661B
  \l 661C
  \l 661D
  \l 661E
  \l 661F
  \l 6620
  \l 6621
  \l 6622
  \l 6623
  \l 6624
  \l 6625
  \l 6626
  \l 6627
  \l 6628
  \l 6629
  \l 662A
  \l 662B
  \l 662C
  \l 662D
  \l 662E
  \l 662F
  \l 6630
  \l 6631
  \l 6632
  \l 6633
  \l 6634
  \l 6635
  \l 6636
  \l 6637
  \l 6638
  \l 6639
  \l 663A
  \l 663B
  \l 663C
  \l 663D
  \l 663E
  \l 663F
  \l 6640
  \l 6641
  \l 6642
  \l 6643
  \l 6644
  \l 6645
  \l 6646
  \l 6647
  \l 6648
  \l 6649
  \l 664A
  \l 664B
  \l 664C
  \l 664D
  \l 664E
  \l 664F
  \l 6650
  \l 6651
  \l 6652
  \l 6653
  \l 6654
  \l 6655
  \l 6656
  \l 6657
  \l 6658
  \l 6659
  \l 665A
  \l 665B
  \l 665C
  \l 665D
  \l 665E
  \l 665F
  \l 6660
  \l 6661
  \l 6662
  \l 6663
  \l 6664
  \l 6665
  \l 6666
  \l 6667
  \l 6668
  \l 6669
  \l 666A
  \l 666B
  \l 666C
  \l 666D
  \l 666E
  \l 666F
  \l 6670
  \l 6671
  \l 6672
  \l 6673
  \l 6674
  \l 6675
  \l 6676
  \l 6677
  \l 6678
  \l 6679
  \l 667A
  \l 667B
  \l 667C
  \l 667D
  \l 667E
  \l 667F
  \l 6680
  \l 6681
  \l 6682
  \l 6683
  \l 6684
  \l 6685
  \l 6686
  \l 6687
  \l 6688
  \l 6689
  \l 668A
  \l 668B
  \l 668C
  \l 668D
  \l 668E
  \l 668F
  \l 6690
  \l 6691
  \l 6692
  \l 6693
  \l 6694
  \l 6695
  \l 6696
  \l 6697
  \l 6698
  \l 6699
  \l 669A
  \l 669B
  \l 669C
  \l 669D
  \l 669E
  \l 669F
  \l 66A0
  \l 66A1
  \l 66A2
  \l 66A3
  \l 66A4
  \l 66A5
  \l 66A6
  \l 66A7
  \l 66A8
  \l 66A9
  \l 66AA
  \l 66AB
  \l 66AC
  \l 66AD
  \l 66AE
  \l 66AF
  \l 66B0
  \l 66B1
  \l 66B2
  \l 66B3
  \l 66B4
  \l 66B5
  \l 66B6
  \l 66B7
  \l 66B8
  \l 66B9
  \l 66BA
  \l 66BB
  \l 66BC
  \l 66BD
  \l 66BE
  \l 66BF
  \l 66C0
  \l 66C1
  \l 66C2
  \l 66C3
  \l 66C4
  \l 66C5
  \l 66C6
  \l 66C7
  \l 66C8
  \l 66C9
  \l 66CA
  \l 66CB
  \l 66CC
  \l 66CD
  \l 66CE
  \l 66CF
  \l 66D0
  \l 66D1
  \l 66D2
  \l 66D3
  \l 66D4
  \l 66D5
  \l 66D6
  \l 66D7
  \l 66D8
  \l 66D9
  \l 66DA
  \l 66DB
  \l 66DC
  \l 66DD
  \l 66DE
  \l 66DF
  \l 66E0
  \l 66E1
  \l 66E2
  \l 66E3
  \l 66E4
  \l 66E5
  \l 66E6
  \l 66E7
  \l 66E8
  \l 66E9
  \l 66EA
  \l 66EB
  \l 66EC
  \l 66ED
  \l 66EE
  \l 66EF
  \l 66F0
  \l 66F1
  \l 66F2
  \l 66F3
  \l 66F4
  \l 66F5
  \l 66F6
  \l 66F7
  \l 66F8
  \l 66F9
  \l 66FA
  \l 66FB
  \l 66FC
  \l 66FD
  \l 66FE
  \l 66FF
  \l 6700
  \l 6701
  \l 6702
  \l 6703
  \l 6704
  \l 6705
  \l 6706
  \l 6707
  \l 6708
  \l 6709
  \l 670A
  \l 670B
  \l 670C
  \l 670D
  \l 670E
  \l 670F
  \l 6710
  \l 6711
  \l 6712
  \l 6713
  \l 6714
  \l 6715
  \l 6716
  \l 6717
  \l 6718
  \l 6719
  \l 671A
  \l 671B
  \l 671C
  \l 671D
  \l 671E
  \l 671F
  \l 6720
  \l 6721
  \l 6722
  \l 6723
  \l 6724
  \l 6725
  \l 6726
  \l 6727
  \l 6728
  \l 6729
  \l 672A
  \l 672B
  \l 672C
  \l 672D
  \l 672E
  \l 672F
  \l 6730
  \l 6731
  \l 6732
  \l 6733
  \l 6734
  \l 6735
  \l 6736
  \l 6737
  \l 6738
  \l 6739
  \l 673A
  \l 673B
  \l 673C
  \l 673D
  \l 673E
  \l 673F
  \l 6740
  \l 6741
  \l 6742
  \l 6743
  \l 6744
  \l 6745
  \l 6746
  \l 6747
  \l 6748
  \l 6749
  \l 674A
  \l 674B
  \l 674C
  \l 674D
  \l 674E
  \l 674F
  \l 6750
  \l 6751
  \l 6752
  \l 6753
  \l 6754
  \l 6755
  \l 6756
  \l 6757
  \l 6758
  \l 6759
  \l 675A
  \l 675B
  \l 675C
  \l 675D
  \l 675E
  \l 675F
  \l 6760
  \l 6761
  \l 6762
  \l 6763
  \l 6764
  \l 6765
  \l 6766
  \l 6767
  \l 6768
  \l 6769
  \l 676A
  \l 676B
  \l 676C
  \l 676D
  \l 676E
  \l 676F
  \l 6770
  \l 6771
  \l 6772
  \l 6773
  \l 6774
  \l 6775
  \l 6776
  \l 6777
  \l 6778
  \l 6779
  \l 677A
  \l 677B
  \l 677C
  \l 677D
  \l 677E
  \l 677F
  \l 6780
  \l 6781
  \l 6782
  \l 6783
  \l 6784
  \l 6785
  \l 6786
  \l 6787
  \l 6788
  \l 6789
  \l 678A
  \l 678B
  \l 678C
  \l 678D
  \l 678E
  \l 678F
  \l 6790
  \l 6791
  \l 6792
  \l 6793
  \l 6794
  \l 6795
  \l 6796
  \l 6797
  \l 6798
  \l 6799
  \l 679A
  \l 679B
  \l 679C
  \l 679D
  \l 679E
  \l 679F
  \l 67A0
  \l 67A1
  \l 67A2
  \l 67A3
  \l 67A4
  \l 67A5
  \l 67A6
  \l 67A7
  \l 67A8
  \l 67A9
  \l 67AA
  \l 67AB
  \l 67AC
  \l 67AD
  \l 67AE
  \l 67AF
  \l 67B0
  \l 67B1
  \l 67B2
  \l 67B3
  \l 67B4
  \l 67B5
  \l 67B6
  \l 67B7
  \l 67B8
  \l 67B9
  \l 67BA
  \l 67BB
  \l 67BC
  \l 67BD
  \l 67BE
  \l 67BF
  \l 67C0
  \l 67C1
  \l 67C2
  \l 67C3
  \l 67C4
  \l 67C5
  \l 67C6
  \l 67C7
  \l 67C8
  \l 67C9
  \l 67CA
  \l 67CB
  \l 67CC
  \l 67CD
  \l 67CE
  \l 67CF
  \l 67D0
  \l 67D1
  \l 67D2
  \l 67D3
  \l 67D4
  \l 67D5
  \l 67D6
  \l 67D7
  \l 67D8
  \l 67D9
  \l 67DA
  \l 67DB
  \l 67DC
  \l 67DD
  \l 67DE
  \l 67DF
  \l 67E0
  \l 67E1
  \l 67E2
  \l 67E3
  \l 67E4
  \l 67E5
  \l 67E6
  \l 67E7
  \l 67E8
  \l 67E9
  \l 67EA
  \l 67EB
  \l 67EC
  \l 67ED
  \l 67EE
  \l 67EF
  \l 67F0
  \l 67F1
  \l 67F2
  \l 67F3
  \l 67F4
  \l 67F5
  \l 67F6
  \l 67F7
  \l 67F8
  \l 67F9
  \l 67FA
  \l 67FB
  \l 67FC
  \l 67FD
  \l 67FE
  \l 67FF
  \l 6800
  \l 6801
  \l 6802
  \l 6803
  \l 6804
  \l 6805
  \l 6806
  \l 6807
  \l 6808
  \l 6809
  \l 680A
  \l 680B
  \l 680C
  \l 680D
  \l 680E
  \l 680F
  \l 6810
  \l 6811
  \l 6812
  \l 6813
  \l 6814
  \l 6815
  \l 6816
  \l 6817
  \l 6818
  \l 6819
  \l 681A
  \l 681B
  \l 681C
  \l 681D
  \l 681E
  \l 681F
  \l 6820
  \l 6821
  \l 6822
  \l 6823
  \l 6824
  \l 6825
  \l 6826
  \l 6827
  \l 6828
  \l 6829
  \l 682A
  \l 682B
  \l 682C
  \l 682D
  \l 682E
  \l 682F
  \l 6830
  \l 6831
  \l 6832
  \l 6833
  \l 6834
  \l 6835
  \l 6836
  \l 6837
  \l 6838
  \l 6839
  \l 683A
  \l 683B
  \l 683C
  \l 683D
  \l 683E
  \l 683F
  \l 6840
  \l 6841
  \l 6842
  \l 6843
  \l 6844
  \l 6845
  \l 6846
  \l 6847
  \l 6848
  \l 6849
  \l 684A
  \l 684B
  \l 684C
  \l 684D
  \l 684E
  \l 684F
  \l 6850
  \l 6851
  \l 6852
  \l 6853
  \l 6854
  \l 6855
  \l 6856
  \l 6857
  \l 6858
  \l 6859
  \l 685A
  \l 685B
  \l 685C
  \l 685D
  \l 685E
  \l 685F
  \l 6860
  \l 6861
  \l 6862
  \l 6863
  \l 6864
  \l 6865
  \l 6866
  \l 6867
  \l 6868
  \l 6869
  \l 686A
  \l 686B
  \l 686C
  \l 686D
  \l 686E
  \l 686F
  \l 6870
  \l 6871
  \l 6872
  \l 6873
  \l 6874
  \l 6875
  \l 6876
  \l 6877
  \l 6878
  \l 6879
  \l 687A
  \l 687B
  \l 687C
  \l 687D
  \l 687E
  \l 687F
  \l 6880
  \l 6881
  \l 6882
  \l 6883
  \l 6884
  \l 6885
  \l 6886
  \l 6887
  \l 6888
  \l 6889
  \l 688A
  \l 688B
  \l 688C
  \l 688D
  \l 688E
  \l 688F
  \l 6890
  \l 6891
  \l 6892
  \l 6893
  \l 6894
  \l 6895
  \l 6896
  \l 6897
  \l 6898
  \l 6899
  \l 689A
  \l 689B
  \l 689C
  \l 689D
  \l 689E
  \l 689F
  \l 68A0
  \l 68A1
  \l 68A2
  \l 68A3
  \l 68A4
  \l 68A5
  \l 68A6
  \l 68A7
  \l 68A8
  \l 68A9
  \l 68AA
  \l 68AB
  \l 68AC
  \l 68AD
  \l 68AE
  \l 68AF
  \l 68B0
  \l 68B1
  \l 68B2
  \l 68B3
  \l 68B4
  \l 68B5
  \l 68B6
  \l 68B7
  \l 68B8
  \l 68B9
  \l 68BA
  \l 68BB
  \l 68BC
  \l 68BD
  \l 68BE
  \l 68BF
  \l 68C0
  \l 68C1
  \l 68C2
  \l 68C3
  \l 68C4
  \l 68C5
  \l 68C6
  \l 68C7
  \l 68C8
  \l 68C9
  \l 68CA
  \l 68CB
  \l 68CC
  \l 68CD
  \l 68CE
  \l 68CF
  \l 68D0
  \l 68D1
  \l 68D2
  \l 68D3
  \l 68D4
  \l 68D5
  \l 68D6
  \l 68D7
  \l 68D8
  \l 68D9
  \l 68DA
  \l 68DB
  \l 68DC
  \l 68DD
  \l 68DE
  \l 68DF
  \l 68E0
  \l 68E1
  \l 68E2
  \l 68E3
  \l 68E4
  \l 68E5
  \l 68E6
  \l 68E7
  \l 68E8
  \l 68E9
  \l 68EA
  \l 68EB
  \l 68EC
  \l 68ED
  \l 68EE
  \l 68EF
  \l 68F0
  \l 68F1
  \l 68F2
  \l 68F3
  \l 68F4
  \l 68F5
  \l 68F6
  \l 68F7
  \l 68F8
  \l 68F9
  \l 68FA
  \l 68FB
  \l 68FC
  \l 68FD
  \l 68FE
  \l 68FF
  \l 6900
  \l 6901
  \l 6902
  \l 6903
  \l 6904
  \l 6905
  \l 6906
  \l 6907
  \l 6908
  \l 6909
  \l 690A
  \l 690B
  \l 690C
  \l 690D
  \l 690E
  \l 690F
  \l 6910
  \l 6911
  \l 6912
  \l 6913
  \l 6914
  \l 6915
  \l 6916
  \l 6917
  \l 6918
  \l 6919
  \l 691A
  \l 691B
  \l 691C
  \l 691D
  \l 691E
  \l 691F
  \l 6920
  \l 6921
  \l 6922
  \l 6923
  \l 6924
  \l 6925
  \l 6926
  \l 6927
  \l 6928
  \l 6929
  \l 692A
  \l 692B
  \l 692C
  \l 692D
  \l 692E
  \l 692F
  \l 6930
  \l 6931
  \l 6932
  \l 6933
  \l 6934
  \l 6935
  \l 6936
  \l 6937
  \l 6938
  \l 6939
  \l 693A
  \l 693B
  \l 693C
  \l 693D
  \l 693E
  \l 693F
  \l 6940
  \l 6941
  \l 6942
  \l 6943
  \l 6944
  \l 6945
  \l 6946
  \l 6947
  \l 6948
  \l 6949
  \l 694A
  \l 694B
  \l 694C
  \l 694D
  \l 694E
  \l 694F
  \l 6950
  \l 6951
  \l 6952
  \l 6953
  \l 6954
  \l 6955
  \l 6956
  \l 6957
  \l 6958
  \l 6959
  \l 695A
  \l 695B
  \l 695C
  \l 695D
  \l 695E
  \l 695F
  \l 6960
  \l 6961
  \l 6962
  \l 6963
  \l 6964
  \l 6965
  \l 6966
  \l 6967
  \l 6968
  \l 6969
  \l 696A
  \l 696B
  \l 696C
  \l 696D
  \l 696E
  \l 696F
  \l 6970
  \l 6971
  \l 6972
  \l 6973
  \l 6974
  \l 6975
  \l 6976
  \l 6977
  \l 6978
  \l 6979
  \l 697A
  \l 697B
  \l 697C
  \l 697D
  \l 697E
  \l 697F
  \l 6980
  \l 6981
  \l 6982
  \l 6983
  \l 6984
  \l 6985
  \l 6986
  \l 6987
  \l 6988
  \l 6989
  \l 698A
  \l 698B
  \l 698C
  \l 698D
  \l 698E
  \l 698F
  \l 6990
  \l 6991
  \l 6992
  \l 6993
  \l 6994
  \l 6995
  \l 6996
  \l 6997
  \l 6998
  \l 6999
  \l 699A
  \l 699B
  \l 699C
  \l 699D
  \l 699E
  \l 699F
  \l 69A0
  \l 69A1
  \l 69A2
  \l 69A3
  \l 69A4
  \l 69A5
  \l 69A6
  \l 69A7
  \l 69A8
  \l 69A9
  \l 69AA
  \l 69AB
  \l 69AC
  \l 69AD
  \l 69AE
  \l 69AF
  \l 69B0
  \l 69B1
  \l 69B2
  \l 69B3
  \l 69B4
  \l 69B5
  \l 69B6
  \l 69B7
  \l 69B8
  \l 69B9
  \l 69BA
  \l 69BB
  \l 69BC
  \l 69BD
  \l 69BE
  \l 69BF
  \l 69C0
  \l 69C1
  \l 69C2
  \l 69C3
  \l 69C4
  \l 69C5
  \l 69C6
  \l 69C7
  \l 69C8
  \l 69C9
  \l 69CA
  \l 69CB
  \l 69CC
  \l 69CD
  \l 69CE
  \l 69CF
  \l 69D0
  \l 69D1
  \l 69D2
  \l 69D3
  \l 69D4
  \l 69D5
  \l 69D6
  \l 69D7
  \l 69D8
  \l 69D9
  \l 69DA
  \l 69DB
  \l 69DC
  \l 69DD
  \l 69DE
  \l 69DF
  \l 69E0
  \l 69E1
  \l 69E2
  \l 69E3
  \l 69E4
  \l 69E5
  \l 69E6
  \l 69E7
  \l 69E8
  \l 69E9
  \l 69EA
  \l 69EB
  \l 69EC
  \l 69ED
  \l 69EE
  \l 69EF
  \l 69F0
  \l 69F1
  \l 69F2
  \l 69F3
  \l 69F4
  \l 69F5
  \l 69F6
  \l 69F7
  \l 69F8
  \l 69F9
  \l 69FA
  \l 69FB
  \l 69FC
  \l 69FD
  \l 69FE
  \l 69FF
  \l 6A00
  \l 6A01
  \l 6A02
  \l 6A03
  \l 6A04
  \l 6A05
  \l 6A06
  \l 6A07
  \l 6A08
  \l 6A09
  \l 6A0A
  \l 6A0B
  \l 6A0C
  \l 6A0D
  \l 6A0E
  \l 6A0F
  \l 6A10
  \l 6A11
  \l 6A12
  \l 6A13
  \l 6A14
  \l 6A15
  \l 6A16
  \l 6A17
  \l 6A18
  \l 6A19
  \l 6A1A
  \l 6A1B
  \l 6A1C
  \l 6A1D
  \l 6A1E
  \l 6A1F
  \l 6A20
  \l 6A21
  \l 6A22
  \l 6A23
  \l 6A24
  \l 6A25
  \l 6A26
  \l 6A27
  \l 6A28
  \l 6A29
  \l 6A2A
  \l 6A2B
  \l 6A2C
  \l 6A2D
  \l 6A2E
  \l 6A2F
  \l 6A30
  \l 6A31
  \l 6A32
  \l 6A33
  \l 6A34
  \l 6A35
  \l 6A36
  \l 6A37
  \l 6A38
  \l 6A39
  \l 6A3A
  \l 6A3B
  \l 6A3C
  \l 6A3D
  \l 6A3E
  \l 6A3F
  \l 6A40
  \l 6A41
  \l 6A42
  \l 6A43
  \l 6A44
  \l 6A45
  \l 6A46
  \l 6A47
  \l 6A48
  \l 6A49
  \l 6A4A
  \l 6A4B
  \l 6A4C
  \l 6A4D
  \l 6A4E
  \l 6A4F
  \l 6A50
  \l 6A51
  \l 6A52
  \l 6A53
  \l 6A54
  \l 6A55
  \l 6A56
  \l 6A57
  \l 6A58
  \l 6A59
  \l 6A5A
  \l 6A5B
  \l 6A5C
  \l 6A5D
  \l 6A5E
  \l 6A5F
  \l 6A60
  \l 6A61
  \l 6A62
  \l 6A63
  \l 6A64
  \l 6A65
  \l 6A66
  \l 6A67
  \l 6A68
  \l 6A69
  \l 6A6A
  \l 6A6B
  \l 6A6C
  \l 6A6D
  \l 6A6E
  \l 6A6F
  \l 6A70
  \l 6A71
  \l 6A72
  \l 6A73
  \l 6A74
  \l 6A75
  \l 6A76
  \l 6A77
  \l 6A78
  \l 6A79
  \l 6A7A
  \l 6A7B
  \l 6A7C
  \l 6A7D
  \l 6A7E
  \l 6A7F
  \l 6A80
  \l 6A81
  \l 6A82
  \l 6A83
  \l 6A84
  \l 6A85
  \l 6A86
  \l 6A87
  \l 6A88
  \l 6A89
  \l 6A8A
  \l 6A8B
  \l 6A8C
  \l 6A8D
  \l 6A8E
  \l 6A8F
  \l 6A90
  \l 6A91
  \l 6A92
  \l 6A93
  \l 6A94
  \l 6A95
  \l 6A96
  \l 6A97
  \l 6A98
  \l 6A99
  \l 6A9A
  \l 6A9B
  \l 6A9C
  \l 6A9D
  \l 6A9E
  \l 6A9F
  \l 6AA0
  \l 6AA1
  \l 6AA2
  \l 6AA3
  \l 6AA4
  \l 6AA5
  \l 6AA6
  \l 6AA7
  \l 6AA8
  \l 6AA9
  \l 6AAA
  \l 6AAB
  \l 6AAC
  \l 6AAD
  \l 6AAE
  \l 6AAF
  \l 6AB0
  \l 6AB1
  \l 6AB2
  \l 6AB3
  \l 6AB4
  \l 6AB5
  \l 6AB6
  \l 6AB7
  \l 6AB8
  \l 6AB9
  \l 6ABA
  \l 6ABB
  \l 6ABC
  \l 6ABD
  \l 6ABE
  \l 6ABF
  \l 6AC0
  \l 6AC1
  \l 6AC2
  \l 6AC3
  \l 6AC4
  \l 6AC5
  \l 6AC6
  \l 6AC7
  \l 6AC8
  \l 6AC9
  \l 6ACA
  \l 6ACB
  \l 6ACC
  \l 6ACD
  \l 6ACE
  \l 6ACF
  \l 6AD0
  \l 6AD1
  \l 6AD2
  \l 6AD3
  \l 6AD4
  \l 6AD5
  \l 6AD6
  \l 6AD7
  \l 6AD8
  \l 6AD9
  \l 6ADA
  \l 6ADB
  \l 6ADC
  \l 6ADD
  \l 6ADE
  \l 6ADF
  \l 6AE0
  \l 6AE1
  \l 6AE2
  \l 6AE3
  \l 6AE4
  \l 6AE5
  \l 6AE6
  \l 6AE7
  \l 6AE8
  \l 6AE9
  \l 6AEA
  \l 6AEB
  \l 6AEC
  \l 6AED
  \l 6AEE
  \l 6AEF
  \l 6AF0
  \l 6AF1
  \l 6AF2
  \l 6AF3
  \l 6AF4
  \l 6AF5
  \l 6AF6
  \l 6AF7
  \l 6AF8
  \l 6AF9
  \l 6AFA
  \l 6AFB
  \l 6AFC
  \l 6AFD
  \l 6AFE
  \l 6AFF
  \l 6B00
  \l 6B01
  \l 6B02
  \l 6B03
  \l 6B04
  \l 6B05
  \l 6B06
  \l 6B07
  \l 6B08
  \l 6B09
  \l 6B0A
  \l 6B0B
  \l 6B0C
  \l 6B0D
  \l 6B0E
  \l 6B0F
  \l 6B10
  \l 6B11
  \l 6B12
  \l 6B13
  \l 6B14
  \l 6B15
  \l 6B16
  \l 6B17
  \l 6B18
  \l 6B19
  \l 6B1A
  \l 6B1B
  \l 6B1C
  \l 6B1D
  \l 6B1E
  \l 6B1F
  \l 6B20
  \l 6B21
  \l 6B22
  \l 6B23
  \l 6B24
  \l 6B25
  \l 6B26
  \l 6B27
  \l 6B28
  \l 6B29
  \l 6B2A
  \l 6B2B
  \l 6B2C
  \l 6B2D
  \l 6B2E
  \l 6B2F
  \l 6B30
  \l 6B31
  \l 6B32
  \l 6B33
  \l 6B34
  \l 6B35
  \l 6B36
  \l 6B37
  \l 6B38
  \l 6B39
  \l 6B3A
  \l 6B3B
  \l 6B3C
  \l 6B3D
  \l 6B3E
  \l 6B3F
  \l 6B40
  \l 6B41
  \l 6B42
  \l 6B43
  \l 6B44
  \l 6B45
  \l 6B46
  \l 6B47
  \l 6B48
  \l 6B49
  \l 6B4A
  \l 6B4B
  \l 6B4C
  \l 6B4D
  \l 6B4E
  \l 6B4F
  \l 6B50
  \l 6B51
  \l 6B52
  \l 6B53
  \l 6B54
  \l 6B55
  \l 6B56
  \l 6B57
  \l 6B58
  \l 6B59
  \l 6B5A
  \l 6B5B
  \l 6B5C
  \l 6B5D
  \l 6B5E
  \l 6B5F
  \l 6B60
  \l 6B61
  \l 6B62
  \l 6B63
  \l 6B64
  \l 6B65
  \l 6B66
  \l 6B67
  \l 6B68
  \l 6B69
  \l 6B6A
  \l 6B6B
  \l 6B6C
  \l 6B6D
  \l 6B6E
  \l 6B6F
  \l 6B70
  \l 6B71
  \l 6B72
  \l 6B73
  \l 6B74
  \l 6B75
  \l 6B76
  \l 6B77
  \l 6B78
  \l 6B79
  \l 6B7A
  \l 6B7B
  \l 6B7C
  \l 6B7D
  \l 6B7E
  \l 6B7F
  \l 6B80
  \l 6B81
  \l 6B82
  \l 6B83
  \l 6B84
  \l 6B85
  \l 6B86
  \l 6B87
  \l 6B88
  \l 6B89
  \l 6B8A
  \l 6B8B
  \l 6B8C
  \l 6B8D
  \l 6B8E
  \l 6B8F
  \l 6B90
  \l 6B91
  \l 6B92
  \l 6B93
  \l 6B94
  \l 6B95
  \l 6B96
  \l 6B97
  \l 6B98
  \l 6B99
  \l 6B9A
  \l 6B9B
  \l 6B9C
  \l 6B9D
  \l 6B9E
  \l 6B9F
  \l 6BA0
  \l 6BA1
  \l 6BA2
  \l 6BA3
  \l 6BA4
  \l 6BA5
  \l 6BA6
  \l 6BA7
  \l 6BA8
  \l 6BA9
  \l 6BAA
  \l 6BAB
  \l 6BAC
  \l 6BAD
  \l 6BAE
  \l 6BAF
  \l 6BB0
  \l 6BB1
  \l 6BB2
  \l 6BB3
  \l 6BB4
  \l 6BB5
  \l 6BB6
  \l 6BB7
  \l 6BB8
  \l 6BB9
  \l 6BBA
  \l 6BBB
  \l 6BBC
  \l 6BBD
  \l 6BBE
  \l 6BBF
  \l 6BC0
  \l 6BC1
  \l 6BC2
  \l 6BC3
  \l 6BC4
  \l 6BC5
  \l 6BC6
  \l 6BC7
  \l 6BC8
  \l 6BC9
  \l 6BCA
  \l 6BCB
  \l 6BCC
  \l 6BCD
  \l 6BCE
  \l 6BCF
  \l 6BD0
  \l 6BD1
  \l 6BD2
  \l 6BD3
  \l 6BD4
  \l 6BD5
  \l 6BD6
  \l 6BD7
  \l 6BD8
  \l 6BD9
  \l 6BDA
  \l 6BDB
  \l 6BDC
  \l 6BDD
  \l 6BDE
  \l 6BDF
  \l 6BE0
  \l 6BE1
  \l 6BE2
  \l 6BE3
  \l 6BE4
  \l 6BE5
  \l 6BE6
  \l 6BE7
  \l 6BE8
  \l 6BE9
  \l 6BEA
  \l 6BEB
  \l 6BEC
  \l 6BED
  \l 6BEE
  \l 6BEF
  \l 6BF0
  \l 6BF1
  \l 6BF2
  \l 6BF3
  \l 6BF4
  \l 6BF5
  \l 6BF6
  \l 6BF7
  \l 6BF8
  \l 6BF9
  \l 6BFA
  \l 6BFB
  \l 6BFC
  \l 6BFD
  \l 6BFE
  \l 6BFF
  \l 6C00
  \l 6C01
  \l 6C02
  \l 6C03
  \l 6C04
  \l 6C05
  \l 6C06
  \l 6C07
  \l 6C08
  \l 6C09
  \l 6C0A
  \l 6C0B
  \l 6C0C
  \l 6C0D
  \l 6C0E
  \l 6C0F
  \l 6C10
  \l 6C11
  \l 6C12
  \l 6C13
  \l 6C14
  \l 6C15
  \l 6C16
  \l 6C17
  \l 6C18
  \l 6C19
  \l 6C1A
  \l 6C1B
  \l 6C1C
  \l 6C1D
  \l 6C1E
  \l 6C1F
  \l 6C20
  \l 6C21
  \l 6C22
  \l 6C23
  \l 6C24
  \l 6C25
  \l 6C26
  \l 6C27
  \l 6C28
  \l 6C29
  \l 6C2A
  \l 6C2B
  \l 6C2C
  \l 6C2D
  \l 6C2E
  \l 6C2F
  \l 6C30
  \l 6C31
  \l 6C32
  \l 6C33
  \l 6C34
  \l 6C35
  \l 6C36
  \l 6C37
  \l 6C38
  \l 6C39
  \l 6C3A
  \l 6C3B
  \l 6C3C
  \l 6C3D
  \l 6C3E
  \l 6C3F
  \l 6C40
  \l 6C41
  \l 6C42
  \l 6C43
  \l 6C44
  \l 6C45
  \l 6C46
  \l 6C47
  \l 6C48
  \l 6C49
  \l 6C4A
  \l 6C4B
  \l 6C4C
  \l 6C4D
  \l 6C4E
  \l 6C4F
  \l 6C50
  \l 6C51
  \l 6C52
  \l 6C53
  \l 6C54
  \l 6C55
  \l 6C56
  \l 6C57
  \l 6C58
  \l 6C59
  \l 6C5A
  \l 6C5B
  \l 6C5C
  \l 6C5D
  \l 6C5E
  \l 6C5F
  \l 6C60
  \l 6C61
  \l 6C62
  \l 6C63
  \l 6C64
  \l 6C65
  \l 6C66
  \l 6C67
  \l 6C68
  \l 6C69
  \l 6C6A
  \l 6C6B
  \l 6C6C
  \l 6C6D
  \l 6C6E
  \l 6C6F
  \l 6C70
  \l 6C71
  \l 6C72
  \l 6C73
  \l 6C74
  \l 6C75
  \l 6C76
  \l 6C77
  \l 6C78
  \l 6C79
  \l 6C7A
  \l 6C7B
  \l 6C7C
  \l 6C7D
  \l 6C7E
  \l 6C7F
  \l 6C80
  \l 6C81
  \l 6C82
  \l 6C83
  \l 6C84
  \l 6C85
  \l 6C86
  \l 6C87
  \l 6C88
  \l 6C89
  \l 6C8A
  \l 6C8B
  \l 6C8C
  \l 6C8D
  \l 6C8E
  \l 6C8F
  \l 6C90
  \l 6C91
  \l 6C92
  \l 6C93
  \l 6C94
  \l 6C95
  \l 6C96
  \l 6C97
  \l 6C98
  \l 6C99
  \l 6C9A
  \l 6C9B
  \l 6C9C
  \l 6C9D
  \l 6C9E
  \l 6C9F
  \l 6CA0
  \l 6CA1
  \l 6CA2
  \l 6CA3
  \l 6CA4
  \l 6CA5
  \l 6CA6
  \l 6CA7
  \l 6CA8
  \l 6CA9
  \l 6CAA
  \l 6CAB
  \l 6CAC
  \l 6CAD
  \l 6CAE
  \l 6CAF
  \l 6CB0
  \l 6CB1
  \l 6CB2
  \l 6CB3
  \l 6CB4
  \l 6CB5
  \l 6CB6
  \l 6CB7
  \l 6CB8
  \l 6CB9
  \l 6CBA
  \l 6CBB
  \l 6CBC
  \l 6CBD
  \l 6CBE
  \l 6CBF
  \l 6CC0
  \l 6CC1
  \l 6CC2
  \l 6CC3
  \l 6CC4
  \l 6CC5
  \l 6CC6
  \l 6CC7
  \l 6CC8
  \l 6CC9
  \l 6CCA
  \l 6CCB
  \l 6CCC
  \l 6CCD
  \l 6CCE
  \l 6CCF
  \l 6CD0
  \l 6CD1
  \l 6CD2
  \l 6CD3
  \l 6CD4
  \l 6CD5
  \l 6CD6
  \l 6CD7
  \l 6CD8
  \l 6CD9
  \l 6CDA
  \l 6CDB
  \l 6CDC
  \l 6CDD
  \l 6CDE
  \l 6CDF
  \l 6CE0
  \l 6CE1
  \l 6CE2
  \l 6CE3
  \l 6CE4
  \l 6CE5
  \l 6CE6
  \l 6CE7
  \l 6CE8
  \l 6CE9
  \l 6CEA
  \l 6CEB
  \l 6CEC
  \l 6CED
  \l 6CEE
  \l 6CEF
  \l 6CF0
  \l 6CF1
  \l 6CF2
  \l 6CF3
  \l 6CF4
  \l 6CF5
  \l 6CF6
  \l 6CF7
  \l 6CF8
  \l 6CF9
  \l 6CFA
  \l 6CFB
  \l 6CFC
  \l 6CFD
  \l 6CFE
  \l 6CFF
  \l 6D00
  \l 6D01
  \l 6D02
  \l 6D03
  \l 6D04
  \l 6D05
  \l 6D06
  \l 6D07
  \l 6D08
  \l 6D09
  \l 6D0A
  \l 6D0B
  \l 6D0C
  \l 6D0D
  \l 6D0E
  \l 6D0F
  \l 6D10
  \l 6D11
  \l 6D12
  \l 6D13
  \l 6D14
  \l 6D15
  \l 6D16
  \l 6D17
  \l 6D18
  \l 6D19
  \l 6D1A
  \l 6D1B
  \l 6D1C
  \l 6D1D
  \l 6D1E
  \l 6D1F
  \l 6D20
  \l 6D21
  \l 6D22
  \l 6D23
  \l 6D24
  \l 6D25
  \l 6D26
  \l 6D27
  \l 6D28
  \l 6D29
  \l 6D2A
  \l 6D2B
  \l 6D2C
  \l 6D2D
  \l 6D2E
  \l 6D2F
  \l 6D30
  \l 6D31
  \l 6D32
  \l 6D33
  \l 6D34
  \l 6D35
  \l 6D36
  \l 6D37
  \l 6D38
  \l 6D39
  \l 6D3A
  \l 6D3B
  \l 6D3C
  \l 6D3D
  \l 6D3E
  \l 6D3F
  \l 6D40
  \l 6D41
  \l 6D42
  \l 6D43
  \l 6D44
  \l 6D45
  \l 6D46
  \l 6D47
  \l 6D48
  \l 6D49
  \l 6D4A
  \l 6D4B
  \l 6D4C
  \l 6D4D
  \l 6D4E
  \l 6D4F
  \l 6D50
  \l 6D51
  \l 6D52
  \l 6D53
  \l 6D54
  \l 6D55
  \l 6D56
  \l 6D57
  \l 6D58
  \l 6D59
  \l 6D5A
  \l 6D5B
  \l 6D5C
  \l 6D5D
  \l 6D5E
  \l 6D5F
  \l 6D60
  \l 6D61
  \l 6D62
  \l 6D63
  \l 6D64
  \l 6D65
  \l 6D66
  \l 6D67
  \l 6D68
  \l 6D69
  \l 6D6A
  \l 6D6B
  \l 6D6C
  \l 6D6D
  \l 6D6E
  \l 6D6F
  \l 6D70
  \l 6D71
  \l 6D72
  \l 6D73
  \l 6D74
  \l 6D75
  \l 6D76
  \l 6D77
  \l 6D78
  \l 6D79
  \l 6D7A
  \l 6D7B
  \l 6D7C
  \l 6D7D
  \l 6D7E
  \l 6D7F
  \l 6D80
  \l 6D81
  \l 6D82
  \l 6D83
  \l 6D84
  \l 6D85
  \l 6D86
  \l 6D87
  \l 6D88
  \l 6D89
  \l 6D8A
  \l 6D8B
  \l 6D8C
  \l 6D8D
  \l 6D8E
  \l 6D8F
  \l 6D90
  \l 6D91
  \l 6D92
  \l 6D93
  \l 6D94
  \l 6D95
  \l 6D96
  \l 6D97
  \l 6D98
  \l 6D99
  \l 6D9A
  \l 6D9B
  \l 6D9C
  \l 6D9D
  \l 6D9E
  \l 6D9F
  \l 6DA0
  \l 6DA1
  \l 6DA2
  \l 6DA3
  \l 6DA4
  \l 6DA5
  \l 6DA6
  \l 6DA7
  \l 6DA8
  \l 6DA9
  \l 6DAA
  \l 6DAB
  \l 6DAC
  \l 6DAD
  \l 6DAE
  \l 6DAF
  \l 6DB0
  \l 6DB1
  \l 6DB2
  \l 6DB3
  \l 6DB4
  \l 6DB5
  \l 6DB6
  \l 6DB7
  \l 6DB8
  \l 6DB9
  \l 6DBA
  \l 6DBB
  \l 6DBC
  \l 6DBD
  \l 6DBE
  \l 6DBF
  \l 6DC0
  \l 6DC1
  \l 6DC2
  \l 6DC3
  \l 6DC4
  \l 6DC5
  \l 6DC6
  \l 6DC7
  \l 6DC8
  \l 6DC9
  \l 6DCA
  \l 6DCB
  \l 6DCC
  \l 6DCD
  \l 6DCE
  \l 6DCF
  \l 6DD0
  \l 6DD1
  \l 6DD2
  \l 6DD3
  \l 6DD4
  \l 6DD5
  \l 6DD6
  \l 6DD7
  \l 6DD8
  \l 6DD9
  \l 6DDA
  \l 6DDB
  \l 6DDC
  \l 6DDD
  \l 6DDE
  \l 6DDF
  \l 6DE0
  \l 6DE1
  \l 6DE2
  \l 6DE3
  \l 6DE4
  \l 6DE5
  \l 6DE6
  \l 6DE7
  \l 6DE8
  \l 6DE9
  \l 6DEA
  \l 6DEB
  \l 6DEC
  \l 6DED
  \l 6DEE
  \l 6DEF
  \l 6DF0
  \l 6DF1
  \l 6DF2
  \l 6DF3
  \l 6DF4
  \l 6DF5
  \l 6DF6
  \l 6DF7
  \l 6DF8
  \l 6DF9
  \l 6DFA
  \l 6DFB
  \l 6DFC
  \l 6DFD
  \l 6DFE
  \l 6DFF
  \l 6E00
  \l 6E01
  \l 6E02
  \l 6E03
  \l 6E04
  \l 6E05
  \l 6E06
  \l 6E07
  \l 6E08
  \l 6E09
  \l 6E0A
  \l 6E0B
  \l 6E0C
  \l 6E0D
  \l 6E0E
  \l 6E0F
  \l 6E10
  \l 6E11
  \l 6E12
  \l 6E13
  \l 6E14
  \l 6E15
  \l 6E16
  \l 6E17
  \l 6E18
  \l 6E19
  \l 6E1A
  \l 6E1B
  \l 6E1C
  \l 6E1D
  \l 6E1E
  \l 6E1F
  \l 6E20
  \l 6E21
  \l 6E22
  \l 6E23
  \l 6E24
  \l 6E25
  \l 6E26
  \l 6E27
  \l 6E28
  \l 6E29
  \l 6E2A
  \l 6E2B
  \l 6E2C
  \l 6E2D
  \l 6E2E
  \l 6E2F
  \l 6E30
  \l 6E31
  \l 6E32
  \l 6E33
  \l 6E34
  \l 6E35
  \l 6E36
  \l 6E37
  \l 6E38
  \l 6E39
  \l 6E3A
  \l 6E3B
  \l 6E3C
  \l 6E3D
  \l 6E3E
  \l 6E3F
  \l 6E40
  \l 6E41
  \l 6E42
  \l 6E43
  \l 6E44
  \l 6E45
  \l 6E46
  \l 6E47
  \l 6E48
  \l 6E49
  \l 6E4A
  \l 6E4B
  \l 6E4C
  \l 6E4D
  \l 6E4E
  \l 6E4F
  \l 6E50
  \l 6E51
  \l 6E52
  \l 6E53
  \l 6E54
  \l 6E55
  \l 6E56
  \l 6E57
  \l 6E58
  \l 6E59
  \l 6E5A
  \l 6E5B
  \l 6E5C
  \l 6E5D
  \l 6E5E
  \l 6E5F
  \l 6E60
  \l 6E61
  \l 6E62
  \l 6E63
  \l 6E64
  \l 6E65
  \l 6E66
  \l 6E67
  \l 6E68
  \l 6E69
  \l 6E6A
  \l 6E6B
  \l 6E6C
  \l 6E6D
  \l 6E6E
  \l 6E6F
  \l 6E70
  \l 6E71
  \l 6E72
  \l 6E73
  \l 6E74
  \l 6E75
  \l 6E76
  \l 6E77
  \l 6E78
  \l 6E79
  \l 6E7A
  \l 6E7B
  \l 6E7C
  \l 6E7D
  \l 6E7E
  \l 6E7F
  \l 6E80
  \l 6E81
  \l 6E82
  \l 6E83
  \l 6E84
  \l 6E85
  \l 6E86
  \l 6E87
  \l 6E88
  \l 6E89
  \l 6E8A
  \l 6E8B
  \l 6E8C
  \l 6E8D
  \l 6E8E
  \l 6E8F
  \l 6E90
  \l 6E91
  \l 6E92
  \l 6E93
  \l 6E94
  \l 6E95
  \l 6E96
  \l 6E97
  \l 6E98
  \l 6E99
  \l 6E9A
  \l 6E9B
  \l 6E9C
  \l 6E9D
  \l 6E9E
  \l 6E9F
  \l 6EA0
  \l 6EA1
  \l 6EA2
  \l 6EA3
  \l 6EA4
  \l 6EA5
  \l 6EA6
  \l 6EA7
  \l 6EA8
  \l 6EA9
  \l 6EAA
  \l 6EAB
  \l 6EAC
  \l 6EAD
  \l 6EAE
  \l 6EAF
  \l 6EB0
  \l 6EB1
  \l 6EB2
  \l 6EB3
  \l 6EB4
  \l 6EB5
  \l 6EB6
  \l 6EB7
  \l 6EB8
  \l 6EB9
  \l 6EBA
  \l 6EBB
  \l 6EBC
  \l 6EBD
  \l 6EBE
  \l 6EBF
  \l 6EC0
  \l 6EC1
  \l 6EC2
  \l 6EC3
  \l 6EC4
  \l 6EC5
  \l 6EC6
  \l 6EC7
  \l 6EC8
  \l 6EC9
  \l 6ECA
  \l 6ECB
  \l 6ECC
  \l 6ECD
  \l 6ECE
  \l 6ECF
  \l 6ED0
  \l 6ED1
  \l 6ED2
  \l 6ED3
  \l 6ED4
  \l 6ED5
  \l 6ED6
  \l 6ED7
  \l 6ED8
  \l 6ED9
  \l 6EDA
  \l 6EDB
  \l 6EDC
  \l 6EDD
  \l 6EDE
  \l 6EDF
  \l 6EE0
  \l 6EE1
  \l 6EE2
  \l 6EE3
  \l 6EE4
  \l 6EE5
  \l 6EE6
  \l 6EE7
  \l 6EE8
  \l 6EE9
  \l 6EEA
  \l 6EEB
  \l 6EEC
  \l 6EED
  \l 6EEE
  \l 6EEF
  \l 6EF0
  \l 6EF1
  \l 6EF2
  \l 6EF3
  \l 6EF4
  \l 6EF5
  \l 6EF6
  \l 6EF7
  \l 6EF8
  \l 6EF9
  \l 6EFA
  \l 6EFB
  \l 6EFC
  \l 6EFD
  \l 6EFE
  \l 6EFF
  \l 6F00
  \l 6F01
  \l 6F02
  \l 6F03
  \l 6F04
  \l 6F05
  \l 6F06
  \l 6F07
  \l 6F08
  \l 6F09
  \l 6F0A
  \l 6F0B
  \l 6F0C
  \l 6F0D
  \l 6F0E
  \l 6F0F
  \l 6F10
  \l 6F11
  \l 6F12
  \l 6F13
  \l 6F14
  \l 6F15
  \l 6F16
  \l 6F17
  \l 6F18
  \l 6F19
  \l 6F1A
  \l 6F1B
  \l 6F1C
  \l 6F1D
  \l 6F1E
  \l 6F1F
  \l 6F20
  \l 6F21
  \l 6F22
  \l 6F23
  \l 6F24
  \l 6F25
  \l 6F26
  \l 6F27
  \l 6F28
  \l 6F29
  \l 6F2A
  \l 6F2B
  \l 6F2C
  \l 6F2D
  \l 6F2E
  \l 6F2F
  \l 6F30
  \l 6F31
  \l 6F32
  \l 6F33
  \l 6F34
  \l 6F35
  \l 6F36
  \l 6F37
  \l 6F38
  \l 6F39
  \l 6F3A
  \l 6F3B
  \l 6F3C
  \l 6F3D
  \l 6F3E
  \l 6F3F
  \l 6F40
  \l 6F41
  \l 6F42
  \l 6F43
  \l 6F44
  \l 6F45
  \l 6F46
  \l 6F47
  \l 6F48
  \l 6F49
  \l 6F4A
  \l 6F4B
  \l 6F4C
  \l 6F4D
  \l 6F4E
  \l 6F4F
  \l 6F50
  \l 6F51
  \l 6F52
  \l 6F53
  \l 6F54
  \l 6F55
  \l 6F56
  \l 6F57
  \l 6F58
  \l 6F59
  \l 6F5A
  \l 6F5B
  \l 6F5C
  \l 6F5D
  \l 6F5E
  \l 6F5F
  \l 6F60
  \l 6F61
  \l 6F62
  \l 6F63
  \l 6F64
  \l 6F65
  \l 6F66
  \l 6F67
  \l 6F68
  \l 6F69
  \l 6F6A
  \l 6F6B
  \l 6F6C
  \l 6F6D
  \l 6F6E
  \l 6F6F
  \l 6F70
  \l 6F71
  \l 6F72
  \l 6F73
  \l 6F74
  \l 6F75
  \l 6F76
  \l 6F77
  \l 6F78
  \l 6F79
  \l 6F7A
  \l 6F7B
  \l 6F7C
  \l 6F7D
  \l 6F7E
  \l 6F7F
  \l 6F80
  \l 6F81
  \l 6F82
  \l 6F83
  \l 6F84
  \l 6F85
  \l 6F86
  \l 6F87
  \l 6F88
  \l 6F89
  \l 6F8A
  \l 6F8B
  \l 6F8C
  \l 6F8D
  \l 6F8E
  \l 6F8F
  \l 6F90
  \l 6F91
  \l 6F92
  \l 6F93
  \l 6F94
  \l 6F95
  \l 6F96
  \l 6F97
  \l 6F98
  \l 6F99
  \l 6F9A
  \l 6F9B
  \l 6F9C
  \l 6F9D
  \l 6F9E
  \l 6F9F
  \l 6FA0
  \l 6FA1
  \l 6FA2
  \l 6FA3
  \l 6FA4
  \l 6FA5
  \l 6FA6
  \l 6FA7
  \l 6FA8
  \l 6FA9
  \l 6FAA
  \l 6FAB
  \l 6FAC
  \l 6FAD
  \l 6FAE
  \l 6FAF
  \l 6FB0
  \l 6FB1
  \l 6FB2
  \l 6FB3
  \l 6FB4
  \l 6FB5
  \l 6FB6
  \l 6FB7
  \l 6FB8
  \l 6FB9
  \l 6FBA
  \l 6FBB
  \l 6FBC
  \l 6FBD
  \l 6FBE
  \l 6FBF
  \l 6FC0
  \l 6FC1
  \l 6FC2
  \l 6FC3
  \l 6FC4
  \l 6FC5
  \l 6FC6
  \l 6FC7
  \l 6FC8
  \l 6FC9
  \l 6FCA
  \l 6FCB
  \l 6FCC
  \l 6FCD
  \l 6FCE
  \l 6FCF
  \l 6FD0
  \l 6FD1
  \l 6FD2
  \l 6FD3
  \l 6FD4
  \l 6FD5
  \l 6FD6
  \l 6FD7
  \l 6FD8
  \l 6FD9
  \l 6FDA
  \l 6FDB
  \l 6FDC
  \l 6FDD
  \l 6FDE
  \l 6FDF
  \l 6FE0
  \l 6FE1
  \l 6FE2
  \l 6FE3
  \l 6FE4
  \l 6FE5
  \l 6FE6
  \l 6FE7
  \l 6FE8
  \l 6FE9
  \l 6FEA
  \l 6FEB
  \l 6FEC
  \l 6FED
  \l 6FEE
  \l 6FEF
  \l 6FF0
  \l 6FF1
  \l 6FF2
  \l 6FF3
  \l 6FF4
  \l 6FF5
  \l 6FF6
  \l 6FF7
  \l 6FF8
  \l 6FF9
  \l 6FFA
  \l 6FFB
  \l 6FFC
  \l 6FFD
  \l 6FFE
  \l 6FFF
  \l 7000
  \l 7001
  \l 7002
  \l 7003
  \l 7004
  \l 7005
  \l 7006
  \l 7007
  \l 7008
  \l 7009
  \l 700A
  \l 700B
  \l 700C
  \l 700D
  \l 700E
  \l 700F
  \l 7010
  \l 7011
  \l 7012
  \l 7013
  \l 7014
  \l 7015
  \l 7016
  \l 7017
  \l 7018
  \l 7019
  \l 701A
  \l 701B
  \l 701C
  \l 701D
  \l 701E
  \l 701F
  \l 7020
  \l 7021
  \l 7022
  \l 7023
  \l 7024
  \l 7025
  \l 7026
  \l 7027
  \l 7028
  \l 7029
  \l 702A
  \l 702B
  \l 702C
  \l 702D
  \l 702E
  \l 702F
  \l 7030
  \l 7031
  \l 7032
  \l 7033
  \l 7034
  \l 7035
  \l 7036
  \l 7037
  \l 7038
  \l 7039
  \l 703A
  \l 703B
  \l 703C
  \l 703D
  \l 703E
  \l 703F
  \l 7040
  \l 7041
  \l 7042
  \l 7043
  \l 7044
  \l 7045
  \l 7046
  \l 7047
  \l 7048
  \l 7049
  \l 704A
  \l 704B
  \l 704C
  \l 704D
  \l 704E
  \l 704F
  \l 7050
  \l 7051
  \l 7052
  \l 7053
  \l 7054
  \l 7055
  \l 7056
  \l 7057
  \l 7058
  \l 7059
  \l 705A
  \l 705B
  \l 705C
  \l 705D
  \l 705E
  \l 705F
  \l 7060
  \l 7061
  \l 7062
  \l 7063
  \l 7064
  \l 7065
  \l 7066
  \l 7067
  \l 7068
  \l 7069
  \l 706A
  \l 706B
  \l 706C
  \l 706D
  \l 706E
  \l 706F
  \l 7070
  \l 7071
  \l 7072
  \l 7073
  \l 7074
  \l 7075
  \l 7076
  \l 7077
  \l 7078
  \l 7079
  \l 707A
  \l 707B
  \l 707C
  \l 707D
  \l 707E
  \l 707F
  \l 7080
  \l 7081
  \l 7082
  \l 7083
  \l 7084
  \l 7085
  \l 7086
  \l 7087
  \l 7088
  \l 7089
  \l 708A
  \l 708B
  \l 708C
  \l 708D
  \l 708E
  \l 708F
  \l 7090
  \l 7091
  \l 7092
  \l 7093
  \l 7094
  \l 7095
  \l 7096
  \l 7097
  \l 7098
  \l 7099
  \l 709A
  \l 709B
  \l 709C
  \l 709D
  \l 709E
  \l 709F
  \l 70A0
  \l 70A1
  \l 70A2
  \l 70A3
  \l 70A4
  \l 70A5
  \l 70A6
  \l 70A7
  \l 70A8
  \l 70A9
  \l 70AA
  \l 70AB
  \l 70AC
  \l 70AD
  \l 70AE
  \l 70AF
  \l 70B0
  \l 70B1
  \l 70B2
  \l 70B3
  \l 70B4
  \l 70B5
  \l 70B6
  \l 70B7
  \l 70B8
  \l 70B9
  \l 70BA
  \l 70BB
  \l 70BC
  \l 70BD
  \l 70BE
  \l 70BF
  \l 70C0
  \l 70C1
  \l 70C2
  \l 70C3
  \l 70C4
  \l 70C5
  \l 70C6
  \l 70C7
  \l 70C8
  \l 70C9
  \l 70CA
  \l 70CB
  \l 70CC
  \l 70CD
  \l 70CE
  \l 70CF
  \l 70D0
  \l 70D1
  \l 70D2
  \l 70D3
  \l 70D4
  \l 70D5
  \l 70D6
  \l 70D7
  \l 70D8
  \l 70D9
  \l 70DA
  \l 70DB
  \l 70DC
  \l 70DD
  \l 70DE
  \l 70DF
  \l 70E0
  \l 70E1
  \l 70E2
  \l 70E3
  \l 70E4
  \l 70E5
  \l 70E6
  \l 70E7
  \l 70E8
  \l 70E9
  \l 70EA
  \l 70EB
  \l 70EC
  \l 70ED
  \l 70EE
  \l 70EF
  \l 70F0
  \l 70F1
  \l 70F2
  \l 70F3
  \l 70F4
  \l 70F5
  \l 70F6
  \l 70F7
  \l 70F8
  \l 70F9
  \l 70FA
  \l 70FB
  \l 70FC
  \l 70FD
  \l 70FE
  \l 70FF
  \l 7100
  \l 7101
  \l 7102
  \l 7103
  \l 7104
  \l 7105
  \l 7106
  \l 7107
  \l 7108
  \l 7109
  \l 710A
  \l 710B
  \l 710C
  \l 710D
  \l 710E
  \l 710F
  \l 7110
  \l 7111
  \l 7112
  \l 7113
  \l 7114
  \l 7115
  \l 7116
  \l 7117
  \l 7118
  \l 7119
  \l 711A
  \l 711B
  \l 711C
  \l 711D
  \l 711E
  \l 711F
  \l 7120
  \l 7121
  \l 7122
  \l 7123
  \l 7124
  \l 7125
  \l 7126
  \l 7127
  \l 7128
  \l 7129
  \l 712A
  \l 712B
  \l 712C
  \l 712D
  \l 712E
  \l 712F
  \l 7130
  \l 7131
  \l 7132
  \l 7133
  \l 7134
  \l 7135
  \l 7136
  \l 7137
  \l 7138
  \l 7139
  \l 713A
  \l 713B
  \l 713C
  \l 713D
  \l 713E
  \l 713F
  \l 7140
  \l 7141
  \l 7142
  \l 7143
  \l 7144
  \l 7145
  \l 7146
  \l 7147
  \l 7148
  \l 7149
  \l 714A
  \l 714B
  \l 714C
  \l 714D
  \l 714E
  \l 714F
  \l 7150
  \l 7151
  \l 7152
  \l 7153
  \l 7154
  \l 7155
  \l 7156
  \l 7157
  \l 7158
  \l 7159
  \l 715A
  \l 715B
  \l 715C
  \l 715D
  \l 715E
  \l 715F
  \l 7160
  \l 7161
  \l 7162
  \l 7163
  \l 7164
  \l 7165
  \l 7166
  \l 7167
  \l 7168
  \l 7169
  \l 716A
  \l 716B
  \l 716C
  \l 716D
  \l 716E
  \l 716F
  \l 7170
  \l 7171
  \l 7172
  \l 7173
  \l 7174
  \l 7175
  \l 7176
  \l 7177
  \l 7178
  \l 7179
  \l 717A
  \l 717B
  \l 717C
  \l 717D
  \l 717E
  \l 717F
  \l 7180
  \l 7181
  \l 7182
  \l 7183
  \l 7184
  \l 7185
  \l 7186
  \l 7187
  \l 7188
  \l 7189
  \l 718A
  \l 718B
  \l 718C
  \l 718D
  \l 718E
  \l 718F
  \l 7190
  \l 7191
  \l 7192
  \l 7193
  \l 7194
  \l 7195
  \l 7196
  \l 7197
  \l 7198
  \l 7199
  \l 719A
  \l 719B
  \l 719C
  \l 719D
  \l 719E
  \l 719F
  \l 71A0
  \l 71A1
  \l 71A2
  \l 71A3
  \l 71A4
  \l 71A5
  \l 71A6
  \l 71A7
  \l 71A8
  \l 71A9
  \l 71AA
  \l 71AB
  \l 71AC
  \l 71AD
  \l 71AE
  \l 71AF
  \l 71B0
  \l 71B1
  \l 71B2
  \l 71B3
  \l 71B4
  \l 71B5
  \l 71B6
  \l 71B7
  \l 71B8
  \l 71B9
  \l 71BA
  \l 71BB
  \l 71BC
  \l 71BD
  \l 71BE
  \l 71BF
  \l 71C0
  \l 71C1
  \l 71C2
  \l 71C3
  \l 71C4
  \l 71C5
  \l 71C6
  \l 71C7
  \l 71C8
  \l 71C9
  \l 71CA
  \l 71CB
  \l 71CC
  \l 71CD
  \l 71CE
  \l 71CF
  \l 71D0
  \l 71D1
  \l 71D2
  \l 71D3
  \l 71D4
  \l 71D5
  \l 71D6
  \l 71D7
  \l 71D8
  \l 71D9
  \l 71DA
  \l 71DB
  \l 71DC
  \l 71DD
  \l 71DE
  \l 71DF
  \l 71E0
  \l 71E1
  \l 71E2
  \l 71E3
  \l 71E4
  \l 71E5
  \l 71E6
  \l 71E7
  \l 71E8
  \l 71E9
  \l 71EA
  \l 71EB
  \l 71EC
  \l 71ED
  \l 71EE
  \l 71EF
  \l 71F0
  \l 71F1
  \l 71F2
  \l 71F3
  \l 71F4
  \l 71F5
  \l 71F6
  \l 71F7
  \l 71F8
  \l 71F9
  \l 71FA
  \l 71FB
  \l 71FC
  \l 71FD
  \l 71FE
  \l 71FF
  \l 7200
  \l 7201
  \l 7202
  \l 7203
  \l 7204
  \l 7205
  \l 7206
  \l 7207
  \l 7208
  \l 7209
  \l 720A
  \l 720B
  \l 720C
  \l 720D
  \l 720E
  \l 720F
  \l 7210
  \l 7211
  \l 7212
  \l 7213
  \l 7214
  \l 7215
  \l 7216
  \l 7217
  \l 7218
  \l 7219
  \l 721A
  \l 721B
  \l 721C
  \l 721D
  \l 721E
  \l 721F
  \l 7220
  \l 7221
  \l 7222
  \l 7223
  \l 7224
  \l 7225
  \l 7226
  \l 7227
  \l 7228
  \l 7229
  \l 722A
  \l 722B
  \l 722C
  \l 722D
  \l 722E
  \l 722F
  \l 7230
  \l 7231
  \l 7232
  \l 7233
  \l 7234
  \l 7235
  \l 7236
  \l 7237
  \l 7238
  \l 7239
  \l 723A
  \l 723B
  \l 723C
  \l 723D
  \l 723E
  \l 723F
  \l 7240
  \l 7241
  \l 7242
  \l 7243
  \l 7244
  \l 7245
  \l 7246
  \l 7247
  \l 7248
  \l 7249
  \l 724A
  \l 724B
  \l 724C
  \l 724D
  \l 724E
  \l 724F
  \l 7250
  \l 7251
  \l 7252
  \l 7253
  \l 7254
  \l 7255
  \l 7256
  \l 7257
  \l 7258
  \l 7259
  \l 725A
  \l 725B
  \l 725C
  \l 725D
  \l 725E
  \l 725F
  \l 7260
  \l 7261
  \l 7262
  \l 7263
  \l 7264
  \l 7265
  \l 7266
  \l 7267
  \l 7268
  \l 7269
  \l 726A
  \l 726B
  \l 726C
  \l 726D
  \l 726E
  \l 726F
  \l 7270
  \l 7271
  \l 7272
  \l 7273
  \l 7274
  \l 7275
  \l 7276
  \l 7277
  \l 7278
  \l 7279
  \l 727A
  \l 727B
  \l 727C
  \l 727D
  \l 727E
  \l 727F
  \l 7280
  \l 7281
  \l 7282
  \l 7283
  \l 7284
  \l 7285
  \l 7286
  \l 7287
  \l 7288
  \l 7289
  \l 728A
  \l 728B
  \l 728C
  \l 728D
  \l 728E
  \l 728F
  \l 7290
  \l 7291
  \l 7292
  \l 7293
  \l 7294
  \l 7295
  \l 7296
  \l 7297
  \l 7298
  \l 7299
  \l 729A
  \l 729B
  \l 729C
  \l 729D
  \l 729E
  \l 729F
  \l 72A0
  \l 72A1
  \l 72A2
  \l 72A3
  \l 72A4
  \l 72A5
  \l 72A6
  \l 72A7
  \l 72A8
  \l 72A9
  \l 72AA
  \l 72AB
  \l 72AC
  \l 72AD
  \l 72AE
  \l 72AF
  \l 72B0
  \l 72B1
  \l 72B2
  \l 72B3
  \l 72B4
  \l 72B5
  \l 72B6
  \l 72B7
  \l 72B8
  \l 72B9
  \l 72BA
  \l 72BB
  \l 72BC
  \l 72BD
  \l 72BE
  \l 72BF
  \l 72C0
  \l 72C1
  \l 72C2
  \l 72C3
  \l 72C4
  \l 72C5
  \l 72C6
  \l 72C7
  \l 72C8
  \l 72C9
  \l 72CA
  \l 72CB
  \l 72CC
  \l 72CD
  \l 72CE
  \l 72CF
  \l 72D0
  \l 72D1
  \l 72D2
  \l 72D3
  \l 72D4
  \l 72D5
  \l 72D6
  \l 72D7
  \l 72D8
  \l 72D9
  \l 72DA
  \l 72DB
  \l 72DC
  \l 72DD
  \l 72DE
  \l 72DF
  \l 72E0
  \l 72E1
  \l 72E2
  \l 72E3
  \l 72E4
  \l 72E5
  \l 72E6
  \l 72E7
  \l 72E8
  \l 72E9
  \l 72EA
  \l 72EB
  \l 72EC
  \l 72ED
  \l 72EE
  \l 72EF
  \l 72F0
  \l 72F1
  \l 72F2
  \l 72F3
  \l 72F4
  \l 72F5
  \l 72F6
  \l 72F7
  \l 72F8
  \l 72F9
  \l 72FA
  \l 72FB
  \l 72FC
  \l 72FD
  \l 72FE
  \l 72FF
  \l 7300
  \l 7301
  \l 7302
  \l 7303
  \l 7304
  \l 7305
  \l 7306
  \l 7307
  \l 7308
  \l 7309
  \l 730A
  \l 730B
  \l 730C
  \l 730D
  \l 730E
  \l 730F
  \l 7310
  \l 7311
  \l 7312
  \l 7313
  \l 7314
  \l 7315
  \l 7316
  \l 7317
  \l 7318
  \l 7319
  \l 731A
  \l 731B
  \l 731C
  \l 731D
  \l 731E
  \l 731F
  \l 7320
  \l 7321
  \l 7322
  \l 7323
  \l 7324
  \l 7325
  \l 7326
  \l 7327
  \l 7328
  \l 7329
  \l 732A
  \l 732B
  \l 732C
  \l 732D
  \l 732E
  \l 732F
  \l 7330
  \l 7331
  \l 7332
  \l 7333
  \l 7334
  \l 7335
  \l 7336
  \l 7337
  \l 7338
  \l 7339
  \l 733A
  \l 733B
  \l 733C
  \l 733D
  \l 733E
  \l 733F
  \l 7340
  \l 7341
  \l 7342
  \l 7343
  \l 7344
  \l 7345
  \l 7346
  \l 7347
  \l 7348
  \l 7349
  \l 734A
  \l 734B
  \l 734C
  \l 734D
  \l 734E
  \l 734F
  \l 7350
  \l 7351
  \l 7352
  \l 7353
  \l 7354
  \l 7355
  \l 7356
  \l 7357
  \l 7358
  \l 7359
  \l 735A
  \l 735B
  \l 735C
  \l 735D
  \l 735E
  \l 735F
  \l 7360
  \l 7361
  \l 7362
  \l 7363
  \l 7364
  \l 7365
  \l 7366
  \l 7367
  \l 7368
  \l 7369
  \l 736A
  \l 736B
  \l 736C
  \l 736D
  \l 736E
  \l 736F
  \l 7370
  \l 7371
  \l 7372
  \l 7373
  \l 7374
  \l 7375
  \l 7376
  \l 7377
  \l 7378
  \l 7379
  \l 737A
  \l 737B
  \l 737C
  \l 737D
  \l 737E
  \l 737F
  \l 7380
  \l 7381
  \l 7382
  \l 7383
  \l 7384
  \l 7385
  \l 7386
  \l 7387
  \l 7388
  \l 7389
  \l 738A
  \l 738B
  \l 738C
  \l 738D
  \l 738E
  \l 738F
  \l 7390
  \l 7391
  \l 7392
  \l 7393
  \l 7394
  \l 7395
  \l 7396
  \l 7397
  \l 7398
  \l 7399
  \l 739A
  \l 739B
  \l 739C
  \l 739D
  \l 739E
  \l 739F
  \l 73A0
  \l 73A1
  \l 73A2
  \l 73A3
  \l 73A4
  \l 73A5
  \l 73A6
  \l 73A7
  \l 73A8
  \l 73A9
  \l 73AA
  \l 73AB
  \l 73AC
  \l 73AD
  \l 73AE
  \l 73AF
  \l 73B0
  \l 73B1
  \l 73B2
  \l 73B3
  \l 73B4
  \l 73B5
  \l 73B6
  \l 73B7
  \l 73B8
  \l 73B9
  \l 73BA
  \l 73BB
  \l 73BC
  \l 73BD
  \l 73BE
  \l 73BF
  \l 73C0
  \l 73C1
  \l 73C2
  \l 73C3
  \l 73C4
  \l 73C5
  \l 73C6
  \l 73C7
  \l 73C8
  \l 73C9
  \l 73CA
  \l 73CB
  \l 73CC
  \l 73CD
  \l 73CE
  \l 73CF
  \l 73D0
  \l 73D1
  \l 73D2
  \l 73D3
  \l 73D4
  \l 73D5
  \l 73D6
  \l 73D7
  \l 73D8
  \l 73D9
  \l 73DA
  \l 73DB
  \l 73DC
  \l 73DD
  \l 73DE
  \l 73DF
  \l 73E0
  \l 73E1
  \l 73E2
  \l 73E3
  \l 73E4
  \l 73E5
  \l 73E6
  \l 73E7
  \l 73E8
  \l 73E9
  \l 73EA
  \l 73EB
  \l 73EC
  \l 73ED
  \l 73EE
  \l 73EF
  \l 73F0
  \l 73F1
  \l 73F2
  \l 73F3
  \l 73F4
  \l 73F5
  \l 73F6
  \l 73F7
  \l 73F8
  \l 73F9
  \l 73FA
  \l 73FB
  \l 73FC
  \l 73FD
  \l 73FE
  \l 73FF
  \l 7400
  \l 7401
  \l 7402
  \l 7403
  \l 7404
  \l 7405
  \l 7406
  \l 7407
  \l 7408
  \l 7409
  \l 740A
  \l 740B
  \l 740C
  \l 740D
  \l 740E
  \l 740F
  \l 7410
  \l 7411
  \l 7412
  \l 7413
  \l 7414
  \l 7415
  \l 7416
  \l 7417
  \l 7418
  \l 7419
  \l 741A
  \l 741B
  \l 741C
  \l 741D
  \l 741E
  \l 741F
  \l 7420
  \l 7421
  \l 7422
  \l 7423
  \l 7424
  \l 7425
  \l 7426
  \l 7427
  \l 7428
  \l 7429
  \l 742A
  \l 742B
  \l 742C
  \l 742D
  \l 742E
  \l 742F
  \l 7430
  \l 7431
  \l 7432
  \l 7433
  \l 7434
  \l 7435
  \l 7436
  \l 7437
  \l 7438
  \l 7439
  \l 743A
  \l 743B
  \l 743C
  \l 743D
  \l 743E
  \l 743F
  \l 7440
  \l 7441
  \l 7442
  \l 7443
  \l 7444
  \l 7445
  \l 7446
  \l 7447
  \l 7448
  \l 7449
  \l 744A
  \l 744B
  \l 744C
  \l 744D
  \l 744E
  \l 744F
  \l 7450
  \l 7451
  \l 7452
  \l 7453
  \l 7454
  \l 7455
  \l 7456
  \l 7457
  \l 7458
  \l 7459
  \l 745A
  \l 745B
  \l 745C
  \l 745D
  \l 745E
  \l 745F
  \l 7460
  \l 7461
  \l 7462
  \l 7463
  \l 7464
  \l 7465
  \l 7466
  \l 7467
  \l 7468
  \l 7469
  \l 746A
  \l 746B
  \l 746C
  \l 746D
  \l 746E
  \l 746F
  \l 7470
  \l 7471
  \l 7472
  \l 7473
  \l 7474
  \l 7475
  \l 7476
  \l 7477
  \l 7478
  \l 7479
  \l 747A
  \l 747B
  \l 747C
  \l 747D
  \l 747E
  \l 747F
  \l 7480
  \l 7481
  \l 7482
  \l 7483
  \l 7484
  \l 7485
  \l 7486
  \l 7487
  \l 7488
  \l 7489
  \l 748A
  \l 748B
  \l 748C
  \l 748D
  \l 748E
  \l 748F
  \l 7490
  \l 7491
  \l 7492
  \l 7493
  \l 7494
  \l 7495
  \l 7496
  \l 7497
  \l 7498
  \l 7499
  \l 749A
  \l 749B
  \l 749C
  \l 749D
  \l 749E
  \l 749F
  \l 74A0
  \l 74A1
  \l 74A2
  \l 74A3
  \l 74A4
  \l 74A5
  \l 74A6
  \l 74A7
  \l 74A8
  \l 74A9
  \l 74AA
  \l 74AB
  \l 74AC
  \l 74AD
  \l 74AE
  \l 74AF
  \l 74B0
  \l 74B1
  \l 74B2
  \l 74B3
  \l 74B4
  \l 74B5
  \l 74B6
  \l 74B7
  \l 74B8
  \l 74B9
  \l 74BA
  \l 74BB
  \l 74BC
  \l 74BD
  \l 74BE
  \l 74BF
  \l 74C0
  \l 74C1
  \l 74C2
  \l 74C3
  \l 74C4
  \l 74C5
  \l 74C6
  \l 74C7
  \l 74C8
  \l 74C9
  \l 74CA
  \l 74CB
  \l 74CC
  \l 74CD
  \l 74CE
  \l 74CF
  \l 74D0
  \l 74D1
  \l 74D2
  \l 74D3
  \l 74D4
  \l 74D5
  \l 74D6
  \l 74D7
  \l 74D8
  \l 74D9
  \l 74DA
  \l 74DB
  \l 74DC
  \l 74DD
  \l 74DE
  \l 74DF
  \l 74E0
  \l 74E1
  \l 74E2
  \l 74E3
  \l 74E4
  \l 74E5
  \l 74E6
  \l 74E7
  \l 74E8
  \l 74E9
  \l 74EA
  \l 74EB
  \l 74EC
  \l 74ED
  \l 74EE
  \l 74EF
  \l 74F0
  \l 74F1
  \l 74F2
  \l 74F3
  \l 74F4
  \l 74F5
  \l 74F6
  \l 74F7
  \l 74F8
  \l 74F9
  \l 74FA
  \l 74FB
  \l 74FC
  \l 74FD
  \l 74FE
  \l 74FF
  \l 7500
  \l 7501
  \l 7502
  \l 7503
  \l 7504
  \l 7505
  \l 7506
  \l 7507
  \l 7508
  \l 7509
  \l 750A
  \l 750B
  \l 750C
  \l 750D
  \l 750E
  \l 750F
  \l 7510
  \l 7511
  \l 7512
  \l 7513
  \l 7514
  \l 7515
  \l 7516
  \l 7517
  \l 7518
  \l 7519
  \l 751A
  \l 751B
  \l 751C
  \l 751D
  \l 751E
  \l 751F
  \l 7520
  \l 7521
  \l 7522
  \l 7523
  \l 7524
  \l 7525
  \l 7526
  \l 7527
  \l 7528
  \l 7529
  \l 752A
  \l 752B
  \l 752C
  \l 752D
  \l 752E
  \l 752F
  \l 7530
  \l 7531
  \l 7532
  \l 7533
  \l 7534
  \l 7535
  \l 7536
  \l 7537
  \l 7538
  \l 7539
  \l 753A
  \l 753B
  \l 753C
  \l 753D
  \l 753E
  \l 753F
  \l 7540
  \l 7541
  \l 7542
  \l 7543
  \l 7544
  \l 7545
  \l 7546
  \l 7547
  \l 7548
  \l 7549
  \l 754A
  \l 754B
  \l 754C
  \l 754D
  \l 754E
  \l 754F
  \l 7550
  \l 7551
  \l 7552
  \l 7553
  \l 7554
  \l 7555
  \l 7556
  \l 7557
  \l 7558
  \l 7559
  \l 755A
  \l 755B
  \l 755C
  \l 755D
  \l 755E
  \l 755F
  \l 7560
  \l 7561
  \l 7562
  \l 7563
  \l 7564
  \l 7565
  \l 7566
  \l 7567
  \l 7568
  \l 7569
  \l 756A
  \l 756B
  \l 756C
  \l 756D
  \l 756E
  \l 756F
  \l 7570
  \l 7571
  \l 7572
  \l 7573
  \l 7574
  \l 7575
  \l 7576
  \l 7577
  \l 7578
  \l 7579
  \l 757A
  \l 757B
  \l 757C
  \l 757D
  \l 757E
  \l 757F
  \l 7580
  \l 7581
  \l 7582
  \l 7583
  \l 7584
  \l 7585
  \l 7586
  \l 7587
  \l 7588
  \l 7589
  \l 758A
  \l 758B
  \l 758C
  \l 758D
  \l 758E
  \l 758F
  \l 7590
  \l 7591
  \l 7592
  \l 7593
  \l 7594
  \l 7595
  \l 7596
  \l 7597
  \l 7598
  \l 7599
  \l 759A
  \l 759B
  \l 759C
  \l 759D
  \l 759E
  \l 759F
  \l 75A0
  \l 75A1
  \l 75A2
  \l 75A3
  \l 75A4
  \l 75A5
  \l 75A6
  \l 75A7
  \l 75A8
  \l 75A9
  \l 75AA
  \l 75AB
  \l 75AC
  \l 75AD
  \l 75AE
  \l 75AF
  \l 75B0
  \l 75B1
  \l 75B2
  \l 75B3
  \l 75B4
  \l 75B5
  \l 75B6
  \l 75B7
  \l 75B8
  \l 75B9
  \l 75BA
  \l 75BB
  \l 75BC
  \l 75BD
  \l 75BE
  \l 75BF
  \l 75C0
  \l 75C1
  \l 75C2
  \l 75C3
  \l 75C4
  \l 75C5
  \l 75C6
  \l 75C7
  \l 75C8
  \l 75C9
  \l 75CA
  \l 75CB
  \l 75CC
  \l 75CD
  \l 75CE
  \l 75CF
  \l 75D0
  \l 75D1
  \l 75D2
  \l 75D3
  \l 75D4
  \l 75D5
  \l 75D6
  \l 75D7
  \l 75D8
  \l 75D9
  \l 75DA
  \l 75DB
  \l 75DC
  \l 75DD
  \l 75DE
  \l 75DF
  \l 75E0
  \l 75E1
  \l 75E2
  \l 75E3
  \l 75E4
  \l 75E5
  \l 75E6
  \l 75E7
  \l 75E8
  \l 75E9
  \l 75EA
  \l 75EB
  \l 75EC
  \l 75ED
  \l 75EE
  \l 75EF
  \l 75F0
  \l 75F1
  \l 75F2
  \l 75F3
  \l 75F4
  \l 75F5
  \l 75F6
  \l 75F7
  \l 75F8
  \l 75F9
  \l 75FA
  \l 75FB
  \l 75FC
  \l 75FD
  \l 75FE
  \l 75FF
  \l 7600
  \l 7601
  \l 7602
  \l 7603
  \l 7604
  \l 7605
  \l 7606
  \l 7607
  \l 7608
  \l 7609
  \l 760A
  \l 760B
  \l 760C
  \l 760D
  \l 760E
  \l 760F
  \l 7610
  \l 7611
  \l 7612
  \l 7613
  \l 7614
  \l 7615
  \l 7616
  \l 7617
  \l 7618
  \l 7619
  \l 761A
  \l 761B
  \l 761C
  \l 761D
  \l 761E
  \l 761F
  \l 7620
  \l 7621
  \l 7622
  \l 7623
  \l 7624
  \l 7625
  \l 7626
  \l 7627
  \l 7628
  \l 7629
  \l 762A
  \l 762B
  \l 762C
  \l 762D
  \l 762E
  \l 762F
  \l 7630
  \l 7631
  \l 7632
  \l 7633
  \l 7634
  \l 7635
  \l 7636
  \l 7637
  \l 7638
  \l 7639
  \l 763A
  \l 763B
  \l 763C
  \l 763D
  \l 763E
  \l 763F
  \l 7640
  \l 7641
  \l 7642
  \l 7643
  \l 7644
  \l 7645
  \l 7646
  \l 7647
  \l 7648
  \l 7649
  \l 764A
  \l 764B
  \l 764C
  \l 764D
  \l 764E
  \l 764F
  \l 7650
  \l 7651
  \l 7652
  \l 7653
  \l 7654
  \l 7655
  \l 7656
  \l 7657
  \l 7658
  \l 7659
  \l 765A
  \l 765B
  \l 765C
  \l 765D
  \l 765E
  \l 765F
  \l 7660
  \l 7661
  \l 7662
  \l 7663
  \l 7664
  \l 7665
  \l 7666
  \l 7667
  \l 7668
  \l 7669
  \l 766A
  \l 766B
  \l 766C
  \l 766D
  \l 766E
  \l 766F
  \l 7670
  \l 7671
  \l 7672
  \l 7673
  \l 7674
  \l 7675
  \l 7676
  \l 7677
  \l 7678
  \l 7679
  \l 767A
  \l 767B
  \l 767C
  \l 767D
  \l 767E
  \l 767F
  \l 7680
  \l 7681
  \l 7682
  \l 7683
  \l 7684
  \l 7685
  \l 7686
  \l 7687
  \l 7688
  \l 7689
  \l 768A
  \l 768B
  \l 768C
  \l 768D
  \l 768E
  \l 768F
  \l 7690
  \l 7691
  \l 7692
  \l 7693
  \l 7694
  \l 7695
  \l 7696
  \l 7697
  \l 7698
  \l 7699
  \l 769A
  \l 769B
  \l 769C
  \l 769D
  \l 769E
  \l 769F
  \l 76A0
  \l 76A1
  \l 76A2
  \l 76A3
  \l 76A4
  \l 76A5
  \l 76A6
  \l 76A7
  \l 76A8
  \l 76A9
  \l 76AA
  \l 76AB
  \l 76AC
  \l 76AD
  \l 76AE
  \l 76AF
  \l 76B0
  \l 76B1
  \l 76B2
  \l 76B3
  \l 76B4
  \l 76B5
  \l 76B6
  \l 76B7
  \l 76B8
  \l 76B9
  \l 76BA
  \l 76BB
  \l 76BC
  \l 76BD
  \l 76BE
  \l 76BF
  \l 76C0
  \l 76C1
  \l 76C2
  \l 76C3
  \l 76C4
  \l 76C5
  \l 76C6
  \l 76C7
  \l 76C8
  \l 76C9
  \l 76CA
  \l 76CB
  \l 76CC
  \l 76CD
  \l 76CE
  \l 76CF
  \l 76D0
  \l 76D1
  \l 76D2
  \l 76D3
  \l 76D4
  \l 76D5
  \l 76D6
  \l 76D7
  \l 76D8
  \l 76D9
  \l 76DA
  \l 76DB
  \l 76DC
  \l 76DD
  \l 76DE
  \l 76DF
  \l 76E0
  \l 76E1
  \l 76E2
  \l 76E3
  \l 76E4
  \l 76E5
  \l 76E6
  \l 76E7
  \l 76E8
  \l 76E9
  \l 76EA
  \l 76EB
  \l 76EC
  \l 76ED
  \l 76EE
  \l 76EF
  \l 76F0
  \l 76F1
  \l 76F2
  \l 76F3
  \l 76F4
  \l 76F5
  \l 76F6
  \l 76F7
  \l 76F8
  \l 76F9
  \l 76FA
  \l 76FB
  \l 76FC
  \l 76FD
  \l 76FE
  \l 76FF
  \l 7700
  \l 7701
  \l 7702
  \l 7703
  \l 7704
  \l 7705
  \l 7706
  \l 7707
  \l 7708
  \l 7709
  \l 770A
  \l 770B
  \l 770C
  \l 770D
  \l 770E
  \l 770F
  \l 7710
  \l 7711
  \l 7712
  \l 7713
  \l 7714
  \l 7715
  \l 7716
  \l 7717
  \l 7718
  \l 7719
  \l 771A
  \l 771B
  \l 771C
  \l 771D
  \l 771E
  \l 771F
  \l 7720
  \l 7721
  \l 7722
  \l 7723
  \l 7724
  \l 7725
  \l 7726
  \l 7727
  \l 7728
  \l 7729
  \l 772A
  \l 772B
  \l 772C
  \l 772D
  \l 772E
  \l 772F
  \l 7730
  \l 7731
  \l 7732
  \l 7733
  \l 7734
  \l 7735
  \l 7736
  \l 7737
  \l 7738
  \l 7739
  \l 773A
  \l 773B
  \l 773C
  \l 773D
  \l 773E
  \l 773F
  \l 7740
  \l 7741
  \l 7742
  \l 7743
  \l 7744
  \l 7745
  \l 7746
  \l 7747
  \l 7748
  \l 7749
  \l 774A
  \l 774B
  \l 774C
  \l 774D
  \l 774E
  \l 774F
  \l 7750
  \l 7751
  \l 7752
  \l 7753
  \l 7754
  \l 7755
  \l 7756
  \l 7757
  \l 7758
  \l 7759
  \l 775A
  \l 775B
  \l 775C
  \l 775D
  \l 775E
  \l 775F
  \l 7760
  \l 7761
  \l 7762
  \l 7763
  \l 7764
  \l 7765
  \l 7766
  \l 7767
  \l 7768
  \l 7769
  \l 776A
  \l 776B
  \l 776C
  \l 776D
  \l 776E
  \l 776F
  \l 7770
  \l 7771
  \l 7772
  \l 7773
  \l 7774
  \l 7775
  \l 7776
  \l 7777
  \l 7778
  \l 7779
  \l 777A
  \l 777B
  \l 777C
  \l 777D
  \l 777E
  \l 777F
  \l 7780
  \l 7781
  \l 7782
  \l 7783
  \l 7784
  \l 7785
  \l 7786
  \l 7787
  \l 7788
  \l 7789
  \l 778A
  \l 778B
  \l 778C
  \l 778D
  \l 778E
  \l 778F
  \l 7790
  \l 7791
  \l 7792
  \l 7793
  \l 7794
  \l 7795
  \l 7796
  \l 7797
  \l 7798
  \l 7799
  \l 779A
  \l 779B
  \l 779C
  \l 779D
  \l 779E
  \l 779F
  \l 77A0
  \l 77A1
  \l 77A2
  \l 77A3
  \l 77A4
  \l 77A5
  \l 77A6
  \l 77A7
  \l 77A8
  \l 77A9
  \l 77AA
  \l 77AB
  \l 77AC
  \l 77AD
  \l 77AE
  \l 77AF
  \l 77B0
  \l 77B1
  \l 77B2
  \l 77B3
  \l 77B4
  \l 77B5
  \l 77B6
  \l 77B7
  \l 77B8
  \l 77B9
  \l 77BA
  \l 77BB
  \l 77BC
  \l 77BD
  \l 77BE
  \l 77BF
  \l 77C0
  \l 77C1
  \l 77C2
  \l 77C3
  \l 77C4
  \l 77C5
  \l 77C6
  \l 77C7
  \l 77C8
  \l 77C9
  \l 77CA
  \l 77CB
  \l 77CC
  \l 77CD
  \l 77CE
  \l 77CF
  \l 77D0
  \l 77D1
  \l 77D2
  \l 77D3
  \l 77D4
  \l 77D5
  \l 77D6
  \l 77D7
  \l 77D8
  \l 77D9
  \l 77DA
  \l 77DB
  \l 77DC
  \l 77DD
  \l 77DE
  \l 77DF
  \l 77E0
  \l 77E1
  \l 77E2
  \l 77E3
  \l 77E4
  \l 77E5
  \l 77E6
  \l 77E7
  \l 77E8
  \l 77E9
  \l 77EA
  \l 77EB
  \l 77EC
  \l 77ED
  \l 77EE
  \l 77EF
  \l 77F0
  \l 77F1
  \l 77F2
  \l 77F3
  \l 77F4
  \l 77F5
  \l 77F6
  \l 77F7
  \l 77F8
  \l 77F9
  \l 77FA
  \l 77FB
  \l 77FC
  \l 77FD
  \l 77FE
  \l 77FF
  \l 7800
  \l 7801
  \l 7802
  \l 7803
  \l 7804
  \l 7805
  \l 7806
  \l 7807
  \l 7808
  \l 7809
  \l 780A
  \l 780B
  \l 780C
  \l 780D
  \l 780E
  \l 780F
  \l 7810
  \l 7811
  \l 7812
  \l 7813
  \l 7814
  \l 7815
  \l 7816
  \l 7817
  \l 7818
  \l 7819
  \l 781A
  \l 781B
  \l 781C
  \l 781D
  \l 781E
  \l 781F
  \l 7820
  \l 7821
  \l 7822
  \l 7823
  \l 7824
  \l 7825
  \l 7826
  \l 7827
  \l 7828
  \l 7829
  \l 782A
  \l 782B
  \l 782C
  \l 782D
  \l 782E
  \l 782F
  \l 7830
  \l 7831
  \l 7832
  \l 7833
  \l 7834
  \l 7835
  \l 7836
  \l 7837
  \l 7838
  \l 7839
  \l 783A
  \l 783B
  \l 783C
  \l 783D
  \l 783E
  \l 783F
  \l 7840
  \l 7841
  \l 7842
  \l 7843
  \l 7844
  \l 7845
  \l 7846
  \l 7847
  \l 7848
  \l 7849
  \l 784A
  \l 784B
  \l 784C
  \l 784D
  \l 784E
  \l 784F
  \l 7850
  \l 7851
  \l 7852
  \l 7853
  \l 7854
  \l 7855
  \l 7856
  \l 7857
  \l 7858
  \l 7859
  \l 785A
  \l 785B
  \l 785C
  \l 785D
  \l 785E
  \l 785F
  \l 7860
  \l 7861
  \l 7862
  \l 7863
  \l 7864
  \l 7865
  \l 7866
  \l 7867
  \l 7868
  \l 7869
  \l 786A
  \l 786B
  \l 786C
  \l 786D
  \l 786E
  \l 786F
  \l 7870
  \l 7871
  \l 7872
  \l 7873
  \l 7874
  \l 7875
  \l 7876
  \l 7877
  \l 7878
  \l 7879
  \l 787A
  \l 787B
  \l 787C
  \l 787D
  \l 787E
  \l 787F
  \l 7880
  \l 7881
  \l 7882
  \l 7883
  \l 7884
  \l 7885
  \l 7886
  \l 7887
  \l 7888
  \l 7889
  \l 788A
  \l 788B
  \l 788C
  \l 788D
  \l 788E
  \l 788F
  \l 7890
  \l 7891
  \l 7892
  \l 7893
  \l 7894
  \l 7895
  \l 7896
  \l 7897
  \l 7898
  \l 7899
  \l 789A
  \l 789B
  \l 789C
  \l 789D
  \l 789E
  \l 789F
  \l 78A0
  \l 78A1
  \l 78A2
  \l 78A3
  \l 78A4
  \l 78A5
  \l 78A6
  \l 78A7
  \l 78A8
  \l 78A9
  \l 78AA
  \l 78AB
  \l 78AC
  \l 78AD
  \l 78AE
  \l 78AF
  \l 78B0
  \l 78B1
  \l 78B2
  \l 78B3
  \l 78B4
  \l 78B5
  \l 78B6
  \l 78B7
  \l 78B8
  \l 78B9
  \l 78BA
  \l 78BB
  \l 78BC
  \l 78BD
  \l 78BE
  \l 78BF
  \l 78C0
  \l 78C1
  \l 78C2
  \l 78C3
  \l 78C4
  \l 78C5
  \l 78C6
  \l 78C7
  \l 78C8
  \l 78C9
  \l 78CA
  \l 78CB
  \l 78CC
  \l 78CD
  \l 78CE
  \l 78CF
  \l 78D0
  \l 78D1
  \l 78D2
  \l 78D3
  \l 78D4
  \l 78D5
  \l 78D6
  \l 78D7
  \l 78D8
  \l 78D9
  \l 78DA
  \l 78DB
  \l 78DC
  \l 78DD
  \l 78DE
  \l 78DF
  \l 78E0
  \l 78E1
  \l 78E2
  \l 78E3
  \l 78E4
  \l 78E5
  \l 78E6
  \l 78E7
  \l 78E8
  \l 78E9
  \l 78EA
  \l 78EB
  \l 78EC
  \l 78ED
  \l 78EE
  \l 78EF
  \l 78F0
  \l 78F1
  \l 78F2
  \l 78F3
  \l 78F4
  \l 78F5
  \l 78F6
  \l 78F7
  \l 78F8
  \l 78F9
  \l 78FA
  \l 78FB
  \l 78FC
  \l 78FD
  \l 78FE
  \l 78FF
  \l 7900
  \l 7901
  \l 7902
  \l 7903
  \l 7904
  \l 7905
  \l 7906
  \l 7907
  \l 7908
  \l 7909
  \l 790A
  \l 790B
  \l 790C
  \l 790D
  \l 790E
  \l 790F
  \l 7910
  \l 7911
  \l 7912
  \l 7913
  \l 7914
  \l 7915
  \l 7916
  \l 7917
  \l 7918
  \l 7919
  \l 791A
  \l 791B
  \l 791C
  \l 791D
  \l 791E
  \l 791F
  \l 7920
  \l 7921
  \l 7922
  \l 7923
  \l 7924
  \l 7925
  \l 7926
  \l 7927
  \l 7928
  \l 7929
  \l 792A
  \l 792B
  \l 792C
  \l 792D
  \l 792E
  \l 792F
  \l 7930
  \l 7931
  \l 7932
  \l 7933
  \l 7934
  \l 7935
  \l 7936
  \l 7937
  \l 7938
  \l 7939
  \l 793A
  \l 793B
  \l 793C
  \l 793D
  \l 793E
  \l 793F
  \l 7940
  \l 7941
  \l 7942
  \l 7943
  \l 7944
  \l 7945
  \l 7946
  \l 7947
  \l 7948
  \l 7949
  \l 794A
  \l 794B
  \l 794C
  \l 794D
  \l 794E
  \l 794F
  \l 7950
  \l 7951
  \l 7952
  \l 7953
  \l 7954
  \l 7955
  \l 7956
  \l 7957
  \l 7958
  \l 7959
  \l 795A
  \l 795B
  \l 795C
  \l 795D
  \l 795E
  \l 795F
  \l 7960
  \l 7961
  \l 7962
  \l 7963
  \l 7964
  \l 7965
  \l 7966
  \l 7967
  \l 7968
  \l 7969
  \l 796A
  \l 796B
  \l 796C
  \l 796D
  \l 796E
  \l 796F
  \l 7970
  \l 7971
  \l 7972
  \l 7973
  \l 7974
  \l 7975
  \l 7976
  \l 7977
  \l 7978
  \l 7979
  \l 797A
  \l 797B
  \l 797C
  \l 797D
  \l 797E
  \l 797F
  \l 7980
  \l 7981
  \l 7982
  \l 7983
  \l 7984
  \l 7985
  \l 7986
  \l 7987
  \l 7988
  \l 7989
  \l 798A
  \l 798B
  \l 798C
  \l 798D
  \l 798E
  \l 798F
  \l 7990
  \l 7991
  \l 7992
  \l 7993
  \l 7994
  \l 7995
  \l 7996
  \l 7997
  \l 7998
  \l 7999
  \l 799A
  \l 799B
  \l 799C
  \l 799D
  \l 799E
  \l 799F
  \l 79A0
  \l 79A1
  \l 79A2
  \l 79A3
  \l 79A4
  \l 79A5
  \l 79A6
  \l 79A7
  \l 79A8
  \l 79A9
  \l 79AA
  \l 79AB
  \l 79AC
  \l 79AD
  \l 79AE
  \l 79AF
  \l 79B0
  \l 79B1
  \l 79B2
  \l 79B3
  \l 79B4
  \l 79B5
  \l 79B6
  \l 79B7
  \l 79B8
  \l 79B9
  \l 79BA
  \l 79BB
  \l 79BC
  \l 79BD
  \l 79BE
  \l 79BF
  \l 79C0
  \l 79C1
  \l 79C2
  \l 79C3
  \l 79C4
  \l 79C5
  \l 79C6
  \l 79C7
  \l 79C8
  \l 79C9
  \l 79CA
  \l 79CB
  \l 79CC
  \l 79CD
  \l 79CE
  \l 79CF
  \l 79D0
  \l 79D1
  \l 79D2
  \l 79D3
  \l 79D4
  \l 79D5
  \l 79D6
  \l 79D7
  \l 79D8
  \l 79D9
  \l 79DA
  \l 79DB
  \l 79DC
  \l 79DD
  \l 79DE
  \l 79DF
  \l 79E0
  \l 79E1
  \l 79E2
  \l 79E3
  \l 79E4
  \l 79E5
  \l 79E6
  \l 79E7
  \l 79E8
  \l 79E9
  \l 79EA
  \l 79EB
  \l 79EC
  \l 79ED
  \l 79EE
  \l 79EF
  \l 79F0
  \l 79F1
  \l 79F2
  \l 79F3
  \l 79F4
  \l 79F5
  \l 79F6
  \l 79F7
  \l 79F8
  \l 79F9
  \l 79FA
  \l 79FB
  \l 79FC
  \l 79FD
  \l 79FE
  \l 79FF
  \l 7A00
  \l 7A01
  \l 7A02
  \l 7A03
  \l 7A04
  \l 7A05
  \l 7A06
  \l 7A07
  \l 7A08
  \l 7A09
  \l 7A0A
  \l 7A0B
  \l 7A0C
  \l 7A0D
  \l 7A0E
  \l 7A0F
  \l 7A10
  \l 7A11
  \l 7A12
  \l 7A13
  \l 7A14
  \l 7A15
  \l 7A16
  \l 7A17
  \l 7A18
  \l 7A19
  \l 7A1A
  \l 7A1B
  \l 7A1C
  \l 7A1D
  \l 7A1E
  \l 7A1F
  \l 7A20
  \l 7A21
  \l 7A22
  \l 7A23
  \l 7A24
  \l 7A25
  \l 7A26
  \l 7A27
  \l 7A28
  \l 7A29
  \l 7A2A
  \l 7A2B
  \l 7A2C
  \l 7A2D
  \l 7A2E
  \l 7A2F
  \l 7A30
  \l 7A31
  \l 7A32
  \l 7A33
  \l 7A34
  \l 7A35
  \l 7A36
  \l 7A37
  \l 7A38
  \l 7A39
  \l 7A3A
  \l 7A3B
  \l 7A3C
  \l 7A3D
  \l 7A3E
  \l 7A3F
  \l 7A40
  \l 7A41
  \l 7A42
  \l 7A43
  \l 7A44
  \l 7A45
  \l 7A46
  \l 7A47
  \l 7A48
  \l 7A49
  \l 7A4A
  \l 7A4B
  \l 7A4C
  \l 7A4D
  \l 7A4E
  \l 7A4F
  \l 7A50
  \l 7A51
  \l 7A52
  \l 7A53
  \l 7A54
  \l 7A55
  \l 7A56
  \l 7A57
  \l 7A58
  \l 7A59
  \l 7A5A
  \l 7A5B
  \l 7A5C
  \l 7A5D
  \l 7A5E
  \l 7A5F
  \l 7A60
  \l 7A61
  \l 7A62
  \l 7A63
  \l 7A64
  \l 7A65
  \l 7A66
  \l 7A67
  \l 7A68
  \l 7A69
  \l 7A6A
  \l 7A6B
  \l 7A6C
  \l 7A6D
  \l 7A6E
  \l 7A6F
  \l 7A70
  \l 7A71
  \l 7A72
  \l 7A73
  \l 7A74
  \l 7A75
  \l 7A76
  \l 7A77
  \l 7A78
  \l 7A79
  \l 7A7A
  \l 7A7B
  \l 7A7C
  \l 7A7D
  \l 7A7E
  \l 7A7F
  \l 7A80
  \l 7A81
  \l 7A82
  \l 7A83
  \l 7A84
  \l 7A85
  \l 7A86
  \l 7A87
  \l 7A88
  \l 7A89
  \l 7A8A
  \l 7A8B
  \l 7A8C
  \l 7A8D
  \l 7A8E
  \l 7A8F
  \l 7A90
  \l 7A91
  \l 7A92
  \l 7A93
  \l 7A94
  \l 7A95
  \l 7A96
  \l 7A97
  \l 7A98
  \l 7A99
  \l 7A9A
  \l 7A9B
  \l 7A9C
  \l 7A9D
  \l 7A9E
  \l 7A9F
  \l 7AA0
  \l 7AA1
  \l 7AA2
  \l 7AA3
  \l 7AA4
  \l 7AA5
  \l 7AA6
  \l 7AA7
  \l 7AA8
  \l 7AA9
  \l 7AAA
  \l 7AAB
  \l 7AAC
  \l 7AAD
  \l 7AAE
  \l 7AAF
  \l 7AB0
  \l 7AB1
  \l 7AB2
  \l 7AB3
  \l 7AB4
  \l 7AB5
  \l 7AB6
  \l 7AB7
  \l 7AB8
  \l 7AB9
  \l 7ABA
  \l 7ABB
  \l 7ABC
  \l 7ABD
  \l 7ABE
  \l 7ABF
  \l 7AC0
  \l 7AC1
  \l 7AC2
  \l 7AC3
  \l 7AC4
  \l 7AC5
  \l 7AC6
  \l 7AC7
  \l 7AC8
  \l 7AC9
  \l 7ACA
  \l 7ACB
  \l 7ACC
  \l 7ACD
  \l 7ACE
  \l 7ACF
  \l 7AD0
  \l 7AD1
  \l 7AD2
  \l 7AD3
  \l 7AD4
  \l 7AD5
  \l 7AD6
  \l 7AD7
  \l 7AD8
  \l 7AD9
  \l 7ADA
  \l 7ADB
  \l 7ADC
  \l 7ADD
  \l 7ADE
  \l 7ADF
  \l 7AE0
  \l 7AE1
  \l 7AE2
  \l 7AE3
  \l 7AE4
  \l 7AE5
  \l 7AE6
  \l 7AE7
  \l 7AE8
  \l 7AE9
  \l 7AEA
  \l 7AEB
  \l 7AEC
  \l 7AED
  \l 7AEE
  \l 7AEF
  \l 7AF0
  \l 7AF1
  \l 7AF2
  \l 7AF3
  \l 7AF4
  \l 7AF5
  \l 7AF6
  \l 7AF7
  \l 7AF8
  \l 7AF9
  \l 7AFA
  \l 7AFB
  \l 7AFC
  \l 7AFD
  \l 7AFE
  \l 7AFF
  \l 7B00
  \l 7B01
  \l 7B02
  \l 7B03
  \l 7B04
  \l 7B05
  \l 7B06
  \l 7B07
  \l 7B08
  \l 7B09
  \l 7B0A
  \l 7B0B
  \l 7B0C
  \l 7B0D
  \l 7B0E
  \l 7B0F
  \l 7B10
  \l 7B11
  \l 7B12
  \l 7B13
  \l 7B14
  \l 7B15
  \l 7B16
  \l 7B17
  \l 7B18
  \l 7B19
  \l 7B1A
  \l 7B1B
  \l 7B1C
  \l 7B1D
  \l 7B1E
  \l 7B1F
  \l 7B20
  \l 7B21
  \l 7B22
  \l 7B23
  \l 7B24
  \l 7B25
  \l 7B26
  \l 7B27
  \l 7B28
  \l 7B29
  \l 7B2A
  \l 7B2B
  \l 7B2C
  \l 7B2D
  \l 7B2E
  \l 7B2F
  \l 7B30
  \l 7B31
  \l 7B32
  \l 7B33
  \l 7B34
  \l 7B35
  \l 7B36
  \l 7B37
  \l 7B38
  \l 7B39
  \l 7B3A
  \l 7B3B
  \l 7B3C
  \l 7B3D
  \l 7B3E
  \l 7B3F
  \l 7B40
  \l 7B41
  \l 7B42
  \l 7B43
  \l 7B44
  \l 7B45
  \l 7B46
  \l 7B47
  \l 7B48
  \l 7B49
  \l 7B4A
  \l 7B4B
  \l 7B4C
  \l 7B4D
  \l 7B4E
  \l 7B4F
  \l 7B50
  \l 7B51
  \l 7B52
  \l 7B53
  \l 7B54
  \l 7B55
  \l 7B56
  \l 7B57
  \l 7B58
  \l 7B59
  \l 7B5A
  \l 7B5B
  \l 7B5C
  \l 7B5D
  \l 7B5E
  \l 7B5F
  \l 7B60
  \l 7B61
  \l 7B62
  \l 7B63
  \l 7B64
  \l 7B65
  \l 7B66
  \l 7B67
  \l 7B68
  \l 7B69
  \l 7B6A
  \l 7B6B
  \l 7B6C
  \l 7B6D
  \l 7B6E
  \l 7B6F
  \l 7B70
  \l 7B71
  \l 7B72
  \l 7B73
  \l 7B74
  \l 7B75
  \l 7B76
  \l 7B77
  \l 7B78
  \l 7B79
  \l 7B7A
  \l 7B7B
  \l 7B7C
  \l 7B7D
  \l 7B7E
  \l 7B7F
  \l 7B80
  \l 7B81
  \l 7B82
  \l 7B83
  \l 7B84
  \l 7B85
  \l 7B86
  \l 7B87
  \l 7B88
  \l 7B89
  \l 7B8A
  \l 7B8B
  \l 7B8C
  \l 7B8D
  \l 7B8E
  \l 7B8F
  \l 7B90
  \l 7B91
  \l 7B92
  \l 7B93
  \l 7B94
  \l 7B95
  \l 7B96
  \l 7B97
  \l 7B98
  \l 7B99
  \l 7B9A
  \l 7B9B
  \l 7B9C
  \l 7B9D
  \l 7B9E
  \l 7B9F
  \l 7BA0
  \l 7BA1
  \l 7BA2
  \l 7BA3
  \l 7BA4
  \l 7BA5
  \l 7BA6
  \l 7BA7
  \l 7BA8
  \l 7BA9
  \l 7BAA
  \l 7BAB
  \l 7BAC
  \l 7BAD
  \l 7BAE
  \l 7BAF
  \l 7BB0
  \l 7BB1
  \l 7BB2
  \l 7BB3
  \l 7BB4
  \l 7BB5
  \l 7BB6
  \l 7BB7
  \l 7BB8
  \l 7BB9
  \l 7BBA
  \l 7BBB
  \l 7BBC
  \l 7BBD
  \l 7BBE
  \l 7BBF
  \l 7BC0
  \l 7BC1
  \l 7BC2
  \l 7BC3
  \l 7BC4
  \l 7BC5
  \l 7BC6
  \l 7BC7
  \l 7BC8
  \l 7BC9
  \l 7BCA
  \l 7BCB
  \l 7BCC
  \l 7BCD
  \l 7BCE
  \l 7BCF
  \l 7BD0
  \l 7BD1
  \l 7BD2
  \l 7BD3
  \l 7BD4
  \l 7BD5
  \l 7BD6
  \l 7BD7
  \l 7BD8
  \l 7BD9
  \l 7BDA
  \l 7BDB
  \l 7BDC
  \l 7BDD
  \l 7BDE
  \l 7BDF
  \l 7BE0
  \l 7BE1
  \l 7BE2
  \l 7BE3
  \l 7BE4
  \l 7BE5
  \l 7BE6
  \l 7BE7
  \l 7BE8
  \l 7BE9
  \l 7BEA
  \l 7BEB
  \l 7BEC
  \l 7BED
  \l 7BEE
  \l 7BEF
  \l 7BF0
  \l 7BF1
  \l 7BF2
  \l 7BF3
  \l 7BF4
  \l 7BF5
  \l 7BF6
  \l 7BF7
  \l 7BF8
  \l 7BF9
  \l 7BFA
  \l 7BFB
  \l 7BFC
  \l 7BFD
  \l 7BFE
  \l 7BFF
  \l 7C00
  \l 7C01
  \l 7C02
  \l 7C03
  \l 7C04
  \l 7C05
  \l 7C06
  \l 7C07
  \l 7C08
  \l 7C09
  \l 7C0A
  \l 7C0B
  \l 7C0C
  \l 7C0D
  \l 7C0E
  \l 7C0F
  \l 7C10
  \l 7C11
  \l 7C12
  \l 7C13
  \l 7C14
  \l 7C15
  \l 7C16
  \l 7C17
  \l 7C18
  \l 7C19
  \l 7C1A
  \l 7C1B
  \l 7C1C
  \l 7C1D
  \l 7C1E
  \l 7C1F
  \l 7C20
  \l 7C21
  \l 7C22
  \l 7C23
  \l 7C24
  \l 7C25
  \l 7C26
  \l 7C27
  \l 7C28
  \l 7C29
  \l 7C2A
  \l 7C2B
  \l 7C2C
  \l 7C2D
  \l 7C2E
  \l 7C2F
  \l 7C30
  \l 7C31
  \l 7C32
  \l 7C33
  \l 7C34
  \l 7C35
  \l 7C36
  \l 7C37
  \l 7C38
  \l 7C39
  \l 7C3A
  \l 7C3B
  \l 7C3C
  \l 7C3D
  \l 7C3E
  \l 7C3F
  \l 7C40
  \l 7C41
  \l 7C42
  \l 7C43
  \l 7C44
  \l 7C45
  \l 7C46
  \l 7C47
  \l 7C48
  \l 7C49
  \l 7C4A
  \l 7C4B
  \l 7C4C
  \l 7C4D
  \l 7C4E
  \l 7C4F
  \l 7C50
  \l 7C51
  \l 7C52
  \l 7C53
  \l 7C54
  \l 7C55
  \l 7C56
  \l 7C57
  \l 7C58
  \l 7C59
  \l 7C5A
  \l 7C5B
  \l 7C5C
  \l 7C5D
  \l 7C5E
  \l 7C5F
  \l 7C60
  \l 7C61
  \l 7C62
  \l 7C63
  \l 7C64
  \l 7C65
  \l 7C66
  \l 7C67
  \l 7C68
  \l 7C69
  \l 7C6A
  \l 7C6B
  \l 7C6C
  \l 7C6D
  \l 7C6E
  \l 7C6F
  \l 7C70
  \l 7C71
  \l 7C72
  \l 7C73
  \l 7C74
  \l 7C75
  \l 7C76
  \l 7C77
  \l 7C78
  \l 7C79
  \l 7C7A
  \l 7C7B
  \l 7C7C
  \l 7C7D
  \l 7C7E
  \l 7C7F
  \l 7C80
  \l 7C81
  \l 7C82
  \l 7C83
  \l 7C84
  \l 7C85
  \l 7C86
  \l 7C87
  \l 7C88
  \l 7C89
  \l 7C8A
  \l 7C8B
  \l 7C8C
  \l 7C8D
  \l 7C8E
  \l 7C8F
  \l 7C90
  \l 7C91
  \l 7C92
  \l 7C93
  \l 7C94
  \l 7C95
  \l 7C96
  \l 7C97
  \l 7C98
  \l 7C99
  \l 7C9A
  \l 7C9B
  \l 7C9C
  \l 7C9D
  \l 7C9E
  \l 7C9F
  \l 7CA0
  \l 7CA1
  \l 7CA2
  \l 7CA3
  \l 7CA4
  \l 7CA5
  \l 7CA6
  \l 7CA7
  \l 7CA8
  \l 7CA9
  \l 7CAA
  \l 7CAB
  \l 7CAC
  \l 7CAD
  \l 7CAE
  \l 7CAF
  \l 7CB0
  \l 7CB1
  \l 7CB2
  \l 7CB3
  \l 7CB4
  \l 7CB5
  \l 7CB6
  \l 7CB7
  \l 7CB8
  \l 7CB9
  \l 7CBA
  \l 7CBB
  \l 7CBC
  \l 7CBD
  \l 7CBE
  \l 7CBF
  \l 7CC0
  \l 7CC1
  \l 7CC2
  \l 7CC3
  \l 7CC4
  \l 7CC5
  \l 7CC6
  \l 7CC7
  \l 7CC8
  \l 7CC9
  \l 7CCA
  \l 7CCB
  \l 7CCC
  \l 7CCD
  \l 7CCE
  \l 7CCF
  \l 7CD0
  \l 7CD1
  \l 7CD2
  \l 7CD3
  \l 7CD4
  \l 7CD5
  \l 7CD6
  \l 7CD7
  \l 7CD8
  \l 7CD9
  \l 7CDA
  \l 7CDB
  \l 7CDC
  \l 7CDD
  \l 7CDE
  \l 7CDF
  \l 7CE0
  \l 7CE1
  \l 7CE2
  \l 7CE3
  \l 7CE4
  \l 7CE5
  \l 7CE6
  \l 7CE7
  \l 7CE8
  \l 7CE9
  \l 7CEA
  \l 7CEB
  \l 7CEC
  \l 7CED
  \l 7CEE
  \l 7CEF
  \l 7CF0
  \l 7CF1
  \l 7CF2
  \l 7CF3
  \l 7CF4
  \l 7CF5
  \l 7CF6
  \l 7CF7
  \l 7CF8
  \l 7CF9
  \l 7CFA
  \l 7CFB
  \l 7CFC
  \l 7CFD
  \l 7CFE
  \l 7CFF
  \l 7D00
  \l 7D01
  \l 7D02
  \l 7D03
  \l 7D04
  \l 7D05
  \l 7D06
  \l 7D07
  \l 7D08
  \l 7D09
  \l 7D0A
  \l 7D0B
  \l 7D0C
  \l 7D0D
  \l 7D0E
  \l 7D0F
  \l 7D10
  \l 7D11
  \l 7D12
  \l 7D13
  \l 7D14
  \l 7D15
  \l 7D16
  \l 7D17
  \l 7D18
  \l 7D19
  \l 7D1A
  \l 7D1B
  \l 7D1C
  \l 7D1D
  \l 7D1E
  \l 7D1F
  \l 7D20
  \l 7D21
  \l 7D22
  \l 7D23
  \l 7D24
  \l 7D25
  \l 7D26
  \l 7D27
  \l 7D28
  \l 7D29
  \l 7D2A
  \l 7D2B
  \l 7D2C
  \l 7D2D
  \l 7D2E
  \l 7D2F
  \l 7D30
  \l 7D31
  \l 7D32
  \l 7D33
  \l 7D34
  \l 7D35
  \l 7D36
  \l 7D37
  \l 7D38
  \l 7D39
  \l 7D3A
  \l 7D3B
  \l 7D3C
  \l 7D3D
  \l 7D3E
  \l 7D3F
  \l 7D40
  \l 7D41
  \l 7D42
  \l 7D43
  \l 7D44
  \l 7D45
  \l 7D46
  \l 7D47
  \l 7D48
  \l 7D49
  \l 7D4A
  \l 7D4B
  \l 7D4C
  \l 7D4D
  \l 7D4E
  \l 7D4F
  \l 7D50
  \l 7D51
  \l 7D52
  \l 7D53
  \l 7D54
  \l 7D55
  \l 7D56
  \l 7D57
  \l 7D58
  \l 7D59
  \l 7D5A
  \l 7D5B
  \l 7D5C
  \l 7D5D
  \l 7D5E
  \l 7D5F
  \l 7D60
  \l 7D61
  \l 7D62
  \l 7D63
  \l 7D64
  \l 7D65
  \l 7D66
  \l 7D67
  \l 7D68
  \l 7D69
  \l 7D6A
  \l 7D6B
  \l 7D6C
  \l 7D6D
  \l 7D6E
  \l 7D6F
  \l 7D70
  \l 7D71
  \l 7D72
  \l 7D73
  \l 7D74
  \l 7D75
  \l 7D76
  \l 7D77
  \l 7D78
  \l 7D79
  \l 7D7A
  \l 7D7B
  \l 7D7C
  \l 7D7D
  \l 7D7E
  \l 7D7F
  \l 7D80
  \l 7D81
  \l 7D82
  \l 7D83
  \l 7D84
  \l 7D85
  \l 7D86
  \l 7D87
  \l 7D88
  \l 7D89
  \l 7D8A
  \l 7D8B
  \l 7D8C
  \l 7D8D
  \l 7D8E
  \l 7D8F
  \l 7D90
  \l 7D91
  \l 7D92
  \l 7D93
  \l 7D94
  \l 7D95
  \l 7D96
  \l 7D97
  \l 7D98
  \l 7D99
  \l 7D9A
  \l 7D9B
  \l 7D9C
  \l 7D9D
  \l 7D9E
  \l 7D9F
  \l 7DA0
  \l 7DA1
  \l 7DA2
  \l 7DA3
  \l 7DA4
  \l 7DA5
  \l 7DA6
  \l 7DA7
  \l 7DA8
  \l 7DA9
  \l 7DAA
  \l 7DAB
  \l 7DAC
  \l 7DAD
  \l 7DAE
  \l 7DAF
  \l 7DB0
  \l 7DB1
  \l 7DB2
  \l 7DB3
  \l 7DB4
  \l 7DB5
  \l 7DB6
  \l 7DB7
  \l 7DB8
  \l 7DB9
  \l 7DBA
  \l 7DBB
  \l 7DBC
  \l 7DBD
  \l 7DBE
  \l 7DBF
  \l 7DC0
  \l 7DC1
  \l 7DC2
  \l 7DC3
  \l 7DC4
  \l 7DC5
  \l 7DC6
  \l 7DC7
  \l 7DC8
  \l 7DC9
  \l 7DCA
  \l 7DCB
  \l 7DCC
  \l 7DCD
  \l 7DCE
  \l 7DCF
  \l 7DD0
  \l 7DD1
  \l 7DD2
  \l 7DD3
  \l 7DD4
  \l 7DD5
  \l 7DD6
  \l 7DD7
  \l 7DD8
  \l 7DD9
  \l 7DDA
  \l 7DDB
  \l 7DDC
  \l 7DDD
  \l 7DDE
  \l 7DDF
  \l 7DE0
  \l 7DE1
  \l 7DE2
  \l 7DE3
  \l 7DE4
  \l 7DE5
  \l 7DE6
  \l 7DE7
  \l 7DE8
  \l 7DE9
  \l 7DEA
  \l 7DEB
  \l 7DEC
  \l 7DED
  \l 7DEE
  \l 7DEF
  \l 7DF0
  \l 7DF1
  \l 7DF2
  \l 7DF3
  \l 7DF4
  \l 7DF5
  \l 7DF6
  \l 7DF7
  \l 7DF8
  \l 7DF9
  \l 7DFA
  \l 7DFB
  \l 7DFC
  \l 7DFD
  \l 7DFE
  \l 7DFF
  \l 7E00
  \l 7E01
  \l 7E02
  \l 7E03
  \l 7E04
  \l 7E05
  \l 7E06
  \l 7E07
  \l 7E08
  \l 7E09
  \l 7E0A
  \l 7E0B
  \l 7E0C
  \l 7E0D
  \l 7E0E
  \l 7E0F
  \l 7E10
  \l 7E11
  \l 7E12
  \l 7E13
  \l 7E14
  \l 7E15
  \l 7E16
  \l 7E17
  \l 7E18
  \l 7E19
  \l 7E1A
  \l 7E1B
  \l 7E1C
  \l 7E1D
  \l 7E1E
  \l 7E1F
  \l 7E20
  \l 7E21
  \l 7E22
  \l 7E23
  \l 7E24
  \l 7E25
  \l 7E26
  \l 7E27
  \l 7E28
  \l 7E29
  \l 7E2A
  \l 7E2B
  \l 7E2C
  \l 7E2D
  \l 7E2E
  \l 7E2F
  \l 7E30
  \l 7E31
  \l 7E32
  \l 7E33
  \l 7E34
  \l 7E35
  \l 7E36
  \l 7E37
  \l 7E38
  \l 7E39
  \l 7E3A
  \l 7E3B
  \l 7E3C
  \l 7E3D
  \l 7E3E
  \l 7E3F
  \l 7E40
  \l 7E41
  \l 7E42
  \l 7E43
  \l 7E44
  \l 7E45
  \l 7E46
  \l 7E47
  \l 7E48
  \l 7E49
  \l 7E4A
  \l 7E4B
  \l 7E4C
  \l 7E4D
  \l 7E4E
  \l 7E4F
  \l 7E50
  \l 7E51
  \l 7E52
  \l 7E53
  \l 7E54
  \l 7E55
  \l 7E56
  \l 7E57
  \l 7E58
  \l 7E59
  \l 7E5A
  \l 7E5B
  \l 7E5C
  \l 7E5D
  \l 7E5E
  \l 7E5F
  \l 7E60
  \l 7E61
  \l 7E62
  \l 7E63
  \l 7E64
  \l 7E65
  \l 7E66
  \l 7E67
  \l 7E68
  \l 7E69
  \l 7E6A
  \l 7E6B
  \l 7E6C
  \l 7E6D
  \l 7E6E
  \l 7E6F
  \l 7E70
  \l 7E71
  \l 7E72
  \l 7E73
  \l 7E74
  \l 7E75
  \l 7E76
  \l 7E77
  \l 7E78
  \l 7E79
  \l 7E7A
  \l 7E7B
  \l 7E7C
  \l 7E7D
  \l 7E7E
  \l 7E7F
  \l 7E80
  \l 7E81
  \l 7E82
  \l 7E83
  \l 7E84
  \l 7E85
  \l 7E86
  \l 7E87
  \l 7E88
  \l 7E89
  \l 7E8A
  \l 7E8B
  \l 7E8C
  \l 7E8D
  \l 7E8E
  \l 7E8F
  \l 7E90
  \l 7E91
  \l 7E92
  \l 7E93
  \l 7E94
  \l 7E95
  \l 7E96
  \l 7E97
  \l 7E98
  \l 7E99
  \l 7E9A
  \l 7E9B
  \l 7E9C
  \l 7E9D
  \l 7E9E
  \l 7E9F
  \l 7EA0
  \l 7EA1
  \l 7EA2
  \l 7EA3
  \l 7EA4
  \l 7EA5
  \l 7EA6
  \l 7EA7
  \l 7EA8
  \l 7EA9
  \l 7EAA
  \l 7EAB
  \l 7EAC
  \l 7EAD
  \l 7EAE
  \l 7EAF
  \l 7EB0
  \l 7EB1
  \l 7EB2
  \l 7EB3
  \l 7EB4
  \l 7EB5
  \l 7EB6
  \l 7EB7
  \l 7EB8
  \l 7EB9
  \l 7EBA
  \l 7EBB
  \l 7EBC
  \l 7EBD
  \l 7EBE
  \l 7EBF
  \l 7EC0
  \l 7EC1
  \l 7EC2
  \l 7EC3
  \l 7EC4
  \l 7EC5
  \l 7EC6
  \l 7EC7
  \l 7EC8
  \l 7EC9
  \l 7ECA
  \l 7ECB
  \l 7ECC
  \l 7ECD
  \l 7ECE
  \l 7ECF
  \l 7ED0
  \l 7ED1
  \l 7ED2
  \l 7ED3
  \l 7ED4
  \l 7ED5
  \l 7ED6
  \l 7ED7
  \l 7ED8
  \l 7ED9
  \l 7EDA
  \l 7EDB
  \l 7EDC
  \l 7EDD
  \l 7EDE
  \l 7EDF
  \l 7EE0
  \l 7EE1
  \l 7EE2
  \l 7EE3
  \l 7EE4
  \l 7EE5
  \l 7EE6
  \l 7EE7
  \l 7EE8
  \l 7EE9
  \l 7EEA
  \l 7EEB
  \l 7EEC
  \l 7EED
  \l 7EEE
  \l 7EEF
  \l 7EF0
  \l 7EF1
  \l 7EF2
  \l 7EF3
  \l 7EF4
  \l 7EF5
  \l 7EF6
  \l 7EF7
  \l 7EF8
  \l 7EF9
  \l 7EFA
  \l 7EFB
  \l 7EFC
  \l 7EFD
  \l 7EFE
  \l 7EFF
  \l 7F00
  \l 7F01
  \l 7F02
  \l 7F03
  \l 7F04
  \l 7F05
  \l 7F06
  \l 7F07
  \l 7F08
  \l 7F09
  \l 7F0A
  \l 7F0B
  \l 7F0C
  \l 7F0D
  \l 7F0E
  \l 7F0F
  \l 7F10
  \l 7F11
  \l 7F12
  \l 7F13
  \l 7F14
  \l 7F15
  \l 7F16
  \l 7F17
  \l 7F18
  \l 7F19
  \l 7F1A
  \l 7F1B
  \l 7F1C
  \l 7F1D
  \l 7F1E
  \l 7F1F
  \l 7F20
  \l 7F21
  \l 7F22
  \l 7F23
  \l 7F24
  \l 7F25
  \l 7F26
  \l 7F27
  \l 7F28
  \l 7F29
  \l 7F2A
  \l 7F2B
  \l 7F2C
  \l 7F2D
  \l 7F2E
  \l 7F2F
  \l 7F30
  \l 7F31
  \l 7F32
  \l 7F33
  \l 7F34
  \l 7F35
  \l 7F36
  \l 7F37
  \l 7F38
  \l 7F39
  \l 7F3A
  \l 7F3B
  \l 7F3C
  \l 7F3D
  \l 7F3E
  \l 7F3F
  \l 7F40
  \l 7F41
  \l 7F42
  \l 7F43
  \l 7F44
  \l 7F45
  \l 7F46
  \l 7F47
  \l 7F48
  \l 7F49
  \l 7F4A
  \l 7F4B
  \l 7F4C
  \l 7F4D
  \l 7F4E
  \l 7F4F
  \l 7F50
  \l 7F51
  \l 7F52
  \l 7F53
  \l 7F54
  \l 7F55
  \l 7F56
  \l 7F57
  \l 7F58
  \l 7F59
  \l 7F5A
  \l 7F5B
  \l 7F5C
  \l 7F5D
  \l 7F5E
  \l 7F5F
  \l 7F60
  \l 7F61
  \l 7F62
  \l 7F63
  \l 7F64
  \l 7F65
  \l 7F66
  \l 7F67
  \l 7F68
  \l 7F69
  \l 7F6A
  \l 7F6B
  \l 7F6C
  \l 7F6D
  \l 7F6E
  \l 7F6F
  \l 7F70
  \l 7F71
  \l 7F72
  \l 7F73
  \l 7F74
  \l 7F75
  \l 7F76
  \l 7F77
  \l 7F78
  \l 7F79
  \l 7F7A
  \l 7F7B
  \l 7F7C
  \l 7F7D
  \l 7F7E
  \l 7F7F
  \l 7F80
  \l 7F81
  \l 7F82
  \l 7F83
  \l 7F84
  \l 7F85
  \l 7F86
  \l 7F87
  \l 7F88
  \l 7F89
  \l 7F8A
  \l 7F8B
  \l 7F8C
  \l 7F8D
  \l 7F8E
  \l 7F8F
  \l 7F90
  \l 7F91
  \l 7F92
  \l 7F93
  \l 7F94
  \l 7F95
  \l 7F96
  \l 7F97
  \l 7F98
  \l 7F99
  \l 7F9A
  \l 7F9B
  \l 7F9C
  \l 7F9D
  \l 7F9E
  \l 7F9F
  \l 7FA0
  \l 7FA1
  \l 7FA2
  \l 7FA3
  \l 7FA4
  \l 7FA5
  \l 7FA6
  \l 7FA7
  \l 7FA8
  \l 7FA9
  \l 7FAA
  \l 7FAB
  \l 7FAC
  \l 7FAD
  \l 7FAE
  \l 7FAF
  \l 7FB0
  \l 7FB1
  \l 7FB2
  \l 7FB3
  \l 7FB4
  \l 7FB5
  \l 7FB6
  \l 7FB7
  \l 7FB8
  \l 7FB9
  \l 7FBA
  \l 7FBB
  \l 7FBC
  \l 7FBD
  \l 7FBE
  \l 7FBF
  \l 7FC0
  \l 7FC1
  \l 7FC2
  \l 7FC3
  \l 7FC4
  \l 7FC5
  \l 7FC6
  \l 7FC7
  \l 7FC8
  \l 7FC9
  \l 7FCA
  \l 7FCB
  \l 7FCC
  \l 7FCD
  \l 7FCE
  \l 7FCF
  \l 7FD0
  \l 7FD1
  \l 7FD2
  \l 7FD3
  \l 7FD4
  \l 7FD5
  \l 7FD6
  \l 7FD7
  \l 7FD8
  \l 7FD9
  \l 7FDA
  \l 7FDB
  \l 7FDC
  \l 7FDD
  \l 7FDE
  \l 7FDF
  \l 7FE0
  \l 7FE1
  \l 7FE2
  \l 7FE3
  \l 7FE4
  \l 7FE5
  \l 7FE6
  \l 7FE7
  \l 7FE8
  \l 7FE9
  \l 7FEA
  \l 7FEB
  \l 7FEC
  \l 7FED
  \l 7FEE
  \l 7FEF
  \l 7FF0
  \l 7FF1
  \l 7FF2
  \l 7FF3
  \l 7FF4
  \l 7FF5
  \l 7FF6
  \l 7FF7
  \l 7FF8
  \l 7FF9
  \l 7FFA
  \l 7FFB
  \l 7FFC
  \l 7FFD
  \l 7FFE
  \l 7FFF
  \l 8000
  \l 8001
  \l 8002
  \l 8003
  \l 8004
  \l 8005
  \l 8006
  \l 8007
  \l 8008
  \l 8009
  \l 800A
  \l 800B
  \l 800C
  \l 800D
  \l 800E
  \l 800F
  \l 8010
  \l 8011
  \l 8012
  \l 8013
  \l 8014
  \l 8015
  \l 8016
  \l 8017
  \l 8018
  \l 8019
  \l 801A
  \l 801B
  \l 801C
  \l 801D
  \l 801E
  \l 801F
  \l 8020
  \l 8021
  \l 8022
  \l 8023
  \l 8024
  \l 8025
  \l 8026
  \l 8027
  \l 8028
  \l 8029
  \l 802A
  \l 802B
  \l 802C
  \l 802D
  \l 802E
  \l 802F
  \l 8030
  \l 8031
  \l 8032
  \l 8033
  \l 8034
  \l 8035
  \l 8036
  \l 8037
  \l 8038
  \l 8039
  \l 803A
  \l 803B
  \l 803C
  \l 803D
  \l 803E
  \l 803F
  \l 8040
  \l 8041
  \l 8042
  \l 8043
  \l 8044
  \l 8045
  \l 8046
  \l 8047
  \l 8048
  \l 8049
  \l 804A
  \l 804B
  \l 804C
  \l 804D
  \l 804E
  \l 804F
  \l 8050
  \l 8051
  \l 8052
  \l 8053
  \l 8054
  \l 8055
  \l 8056
  \l 8057
  \l 8058
  \l 8059
  \l 805A
  \l 805B
  \l 805C
  \l 805D
  \l 805E
  \l 805F
  \l 8060
  \l 8061
  \l 8062
  \l 8063
  \l 8064
  \l 8065
  \l 8066
  \l 8067
  \l 8068
  \l 8069
  \l 806A
  \l 806B
  \l 806C
  \l 806D
  \l 806E
  \l 806F
  \l 8070
  \l 8071
  \l 8072
  \l 8073
  \l 8074
  \l 8075
  \l 8076
  \l 8077
  \l 8078
  \l 8079
  \l 807A
  \l 807B
  \l 807C
  \l 807D
  \l 807E
  \l 807F
  \l 8080
  \l 8081
  \l 8082
  \l 8083
  \l 8084
  \l 8085
  \l 8086
  \l 8087
  \l 8088
  \l 8089
  \l 808A
  \l 808B
  \l 808C
  \l 808D
  \l 808E
  \l 808F
  \l 8090
  \l 8091
  \l 8092
  \l 8093
  \l 8094
  \l 8095
  \l 8096
  \l 8097
  \l 8098
  \l 8099
  \l 809A
  \l 809B
  \l 809C
  \l 809D
  \l 809E
  \l 809F
  \l 80A0
  \l 80A1
  \l 80A2
  \l 80A3
  \l 80A4
  \l 80A5
  \l 80A6
  \l 80A7
  \l 80A8
  \l 80A9
  \l 80AA
  \l 80AB
  \l 80AC
  \l 80AD
  \l 80AE
  \l 80AF
  \l 80B0
  \l 80B1
  \l 80B2
  \l 80B3
  \l 80B4
  \l 80B5
  \l 80B6
  \l 80B7
  \l 80B8
  \l 80B9
  \l 80BA
  \l 80BB
  \l 80BC
  \l 80BD
  \l 80BE
  \l 80BF
  \l 80C0
  \l 80C1
  \l 80C2
  \l 80C3
  \l 80C4
  \l 80C5
  \l 80C6
  \l 80C7
  \l 80C8
  \l 80C9
  \l 80CA
  \l 80CB
  \l 80CC
  \l 80CD
  \l 80CE
  \l 80CF
  \l 80D0
  \l 80D1
  \l 80D2
  \l 80D3
  \l 80D4
  \l 80D5
  \l 80D6
  \l 80D7
  \l 80D8
  \l 80D9
  \l 80DA
  \l 80DB
  \l 80DC
  \l 80DD
  \l 80DE
  \l 80DF
  \l 80E0
  \l 80E1
  \l 80E2
  \l 80E3
  \l 80E4
  \l 80E5
  \l 80E6
  \l 80E7
  \l 80E8
  \l 80E9
  \l 80EA
  \l 80EB
  \l 80EC
  \l 80ED
  \l 80EE
  \l 80EF
  \l 80F0
  \l 80F1
  \l 80F2
  \l 80F3
  \l 80F4
  \l 80F5
  \l 80F6
  \l 80F7
  \l 80F8
  \l 80F9
  \l 80FA
  \l 80FB
  \l 80FC
  \l 80FD
  \l 80FE
  \l 80FF
  \l 8100
  \l 8101
  \l 8102
  \l 8103
  \l 8104
  \l 8105
  \l 8106
  \l 8107
  \l 8108
  \l 8109
  \l 810A
  \l 810B
  \l 810C
  \l 810D
  \l 810E
  \l 810F
  \l 8110
  \l 8111
  \l 8112
  \l 8113
  \l 8114
  \l 8115
  \l 8116
  \l 8117
  \l 8118
  \l 8119
  \l 811A
  \l 811B
  \l 811C
  \l 811D
  \l 811E
  \l 811F
  \l 8120
  \l 8121
  \l 8122
  \l 8123
  \l 8124
  \l 8125
  \l 8126
  \l 8127
  \l 8128
  \l 8129
  \l 812A
  \l 812B
  \l 812C
  \l 812D
  \l 812E
  \l 812F
  \l 8130
  \l 8131
  \l 8132
  \l 8133
  \l 8134
  \l 8135
  \l 8136
  \l 8137
  \l 8138
  \l 8139
  \l 813A
  \l 813B
  \l 813C
  \l 813D
  \l 813E
  \l 813F
  \l 8140
  \l 8141
  \l 8142
  \l 8143
  \l 8144
  \l 8145
  \l 8146
  \l 8147
  \l 8148
  \l 8149
  \l 814A
  \l 814B
  \l 814C
  \l 814D
  \l 814E
  \l 814F
  \l 8150
  \l 8151
  \l 8152
  \l 8153
  \l 8154
  \l 8155
  \l 8156
  \l 8157
  \l 8158
  \l 8159
  \l 815A
  \l 815B
  \l 815C
  \l 815D
  \l 815E
  \l 815F
  \l 8160
  \l 8161
  \l 8162
  \l 8163
  \l 8164
  \l 8165
  \l 8166
  \l 8167
  \l 8168
  \l 8169
  \l 816A
  \l 816B
  \l 816C
  \l 816D
  \l 816E
  \l 816F
  \l 8170
  \l 8171
  \l 8172
  \l 8173
  \l 8174
  \l 8175
  \l 8176
  \l 8177
  \l 8178
  \l 8179
  \l 817A
  \l 817B
  \l 817C
  \l 817D
  \l 817E
  \l 817F
  \l 8180
  \l 8181
  \l 8182
  \l 8183
  \l 8184
  \l 8185
  \l 8186
  \l 8187
  \l 8188
  \l 8189
  \l 818A
  \l 818B
  \l 818C
  \l 818D
  \l 818E
  \l 818F
  \l 8190
  \l 8191
  \l 8192
  \l 8193
  \l 8194
  \l 8195
  \l 8196
  \l 8197
  \l 8198
  \l 8199
  \l 819A
  \l 819B
  \l 819C
  \l 819D
  \l 819E
  \l 819F
  \l 81A0
  \l 81A1
  \l 81A2
  \l 81A3
  \l 81A4
  \l 81A5
  \l 81A6
  \l 81A7
  \l 81A8
  \l 81A9
  \l 81AA
  \l 81AB
  \l 81AC
  \l 81AD
  \l 81AE
  \l 81AF
  \l 81B0
  \l 81B1
  \l 81B2
  \l 81B3
  \l 81B4
  \l 81B5
  \l 81B6
  \l 81B7
  \l 81B8
  \l 81B9
  \l 81BA
  \l 81BB
  \l 81BC
  \l 81BD
  \l 81BE
  \l 81BF
  \l 81C0
  \l 81C1
  \l 81C2
  \l 81C3
  \l 81C4
  \l 81C5
  \l 81C6
  \l 81C7
  \l 81C8
  \l 81C9
  \l 81CA
  \l 81CB
  \l 81CC
  \l 81CD
  \l 81CE
  \l 81CF
  \l 81D0
  \l 81D1
  \l 81D2
  \l 81D3
  \l 81D4
  \l 81D5
  \l 81D6
  \l 81D7
  \l 81D8
  \l 81D9
  \l 81DA
  \l 81DB
  \l 81DC
  \l 81DD
  \l 81DE
  \l 81DF
  \l 81E0
  \l 81E1
  \l 81E2
  \l 81E3
  \l 81E4
  \l 81E5
  \l 81E6
  \l 81E7
  \l 81E8
  \l 81E9
  \l 81EA
  \l 81EB
  \l 81EC
  \l 81ED
  \l 81EE
  \l 81EF
  \l 81F0
  \l 81F1
  \l 81F2
  \l 81F3
  \l 81F4
  \l 81F5
  \l 81F6
  \l 81F7
  \l 81F8
  \l 81F9
  \l 81FA
  \l 81FB
  \l 81FC
  \l 81FD
  \l 81FE
  \l 81FF
  \l 8200
  \l 8201
  \l 8202
  \l 8203
  \l 8204
  \l 8205
  \l 8206
  \l 8207
  \l 8208
  \l 8209
  \l 820A
  \l 820B
  \l 820C
  \l 820D
  \l 820E
  \l 820F
  \l 8210
  \l 8211
  \l 8212
  \l 8213
  \l 8214
  \l 8215
  \l 8216
  \l 8217
  \l 8218
  \l 8219
  \l 821A
  \l 821B
  \l 821C
  \l 821D
  \l 821E
  \l 821F
  \l 8220
  \l 8221
  \l 8222
  \l 8223
  \l 8224
  \l 8225
  \l 8226
  \l 8227
  \l 8228
  \l 8229
  \l 822A
  \l 822B
  \l 822C
  \l 822D
  \l 822E
  \l 822F
  \l 8230
  \l 8231
  \l 8232
  \l 8233
  \l 8234
  \l 8235
  \l 8236
  \l 8237
  \l 8238
  \l 8239
  \l 823A
  \l 823B
  \l 823C
  \l 823D
  \l 823E
  \l 823F
  \l 8240
  \l 8241
  \l 8242
  \l 8243
  \l 8244
  \l 8245
  \l 8246
  \l 8247
  \l 8248
  \l 8249
  \l 824A
  \l 824B
  \l 824C
  \l 824D
  \l 824E
  \l 824F
  \l 8250
  \l 8251
  \l 8252
  \l 8253
  \l 8254
  \l 8255
  \l 8256
  \l 8257
  \l 8258
  \l 8259
  \l 825A
  \l 825B
  \l 825C
  \l 825D
  \l 825E
  \l 825F
  \l 8260
  \l 8261
  \l 8262
  \l 8263
  \l 8264
  \l 8265
  \l 8266
  \l 8267
  \l 8268
  \l 8269
  \l 826A
  \l 826B
  \l 826C
  \l 826D
  \l 826E
  \l 826F
  \l 8270
  \l 8271
  \l 8272
  \l 8273
  \l 8274
  \l 8275
  \l 8276
  \l 8277
  \l 8278
  \l 8279
  \l 827A
  \l 827B
  \l 827C
  \l 827D
  \l 827E
  \l 827F
  \l 8280
  \l 8281
  \l 8282
  \l 8283
  \l 8284
  \l 8285
  \l 8286
  \l 8287
  \l 8288
  \l 8289
  \l 828A
  \l 828B
  \l 828C
  \l 828D
  \l 828E
  \l 828F
  \l 8290
  \l 8291
  \l 8292
  \l 8293
  \l 8294
  \l 8295
  \l 8296
  \l 8297
  \l 8298
  \l 8299
  \l 829A
  \l 829B
  \l 829C
  \l 829D
  \l 829E
  \l 829F
  \l 82A0
  \l 82A1
  \l 82A2
  \l 82A3
  \l 82A4
  \l 82A5
  \l 82A6
  \l 82A7
  \l 82A8
  \l 82A9
  \l 82AA
  \l 82AB
  \l 82AC
  \l 82AD
  \l 82AE
  \l 82AF
  \l 82B0
  \l 82B1
  \l 82B2
  \l 82B3
  \l 82B4
  \l 82B5
  \l 82B6
  \l 82B7
  \l 82B8
  \l 82B9
  \l 82BA
  \l 82BB
  \l 82BC
  \l 82BD
  \l 82BE
  \l 82BF
  \l 82C0
  \l 82C1
  \l 82C2
  \l 82C3
  \l 82C4
  \l 82C5
  \l 82C6
  \l 82C7
  \l 82C8
  \l 82C9
  \l 82CA
  \l 82CB
  \l 82CC
  \l 82CD
  \l 82CE
  \l 82CF
  \l 82D0
  \l 82D1
  \l 82D2
  \l 82D3
  \l 82D4
  \l 82D5
  \l 82D6
  \l 82D7
  \l 82D8
  \l 82D9
  \l 82DA
  \l 82DB
  \l 82DC
  \l 82DD
  \l 82DE
  \l 82DF
  \l 82E0
  \l 82E1
  \l 82E2
  \l 82E3
  \l 82E4
  \l 82E5
  \l 82E6
  \l 82E7
  \l 82E8
  \l 82E9
  \l 82EA
  \l 82EB
  \l 82EC
  \l 82ED
  \l 82EE
  \l 82EF
  \l 82F0
  \l 82F1
  \l 82F2
  \l 82F3
  \l 82F4
  \l 82F5
  \l 82F6
  \l 82F7
  \l 82F8
  \l 82F9
  \l 82FA
  \l 82FB
  \l 82FC
  \l 82FD
  \l 82FE
  \l 82FF
  \l 8300
  \l 8301
  \l 8302
  \l 8303
  \l 8304
  \l 8305
  \l 8306
  \l 8307
  \l 8308
  \l 8309
  \l 830A
  \l 830B
  \l 830C
  \l 830D
  \l 830E
  \l 830F
  \l 8310
  \l 8311
  \l 8312
  \l 8313
  \l 8314
  \l 8315
  \l 8316
  \l 8317
  \l 8318
  \l 8319
  \l 831A
  \l 831B
  \l 831C
  \l 831D
  \l 831E
  \l 831F
  \l 8320
  \l 8321
  \l 8322
  \l 8323
  \l 8324
  \l 8325
  \l 8326
  \l 8327
  \l 8328
  \l 8329
  \l 832A
  \l 832B
  \l 832C
  \l 832D
  \l 832E
  \l 832F
  \l 8330
  \l 8331
  \l 8332
  \l 8333
  \l 8334
  \l 8335
  \l 8336
  \l 8337
  \l 8338
  \l 8339
  \l 833A
  \l 833B
  \l 833C
  \l 833D
  \l 833E
  \l 833F
  \l 8340
  \l 8341
  \l 8342
  \l 8343
  \l 8344
  \l 8345
  \l 8346
  \l 8347
  \l 8348
  \l 8349
  \l 834A
  \l 834B
  \l 834C
  \l 834D
  \l 834E
  \l 834F
  \l 8350
  \l 8351
  \l 8352
  \l 8353
  \l 8354
  \l 8355
  \l 8356
  \l 8357
  \l 8358
  \l 8359
  \l 835A
  \l 835B
  \l 835C
  \l 835D
  \l 835E
  \l 835F
  \l 8360
  \l 8361
  \l 8362
  \l 8363
  \l 8364
  \l 8365
  \l 8366
  \l 8367
  \l 8368
  \l 8369
  \l 836A
  \l 836B
  \l 836C
  \l 836D
  \l 836E
  \l 836F
  \l 8370
  \l 8371
  \l 8372
  \l 8373
  \l 8374
  \l 8375
  \l 8376
  \l 8377
  \l 8378
  \l 8379
  \l 837A
  \l 837B
  \l 837C
  \l 837D
  \l 837E
  \l 837F
  \l 8380
  \l 8381
  \l 8382
  \l 8383
  \l 8384
  \l 8385
  \l 8386
  \l 8387
  \l 8388
  \l 8389
  \l 838A
  \l 838B
  \l 838C
  \l 838D
  \l 838E
  \l 838F
  \l 8390
  \l 8391
  \l 8392
  \l 8393
  \l 8394
  \l 8395
  \l 8396
  \l 8397
  \l 8398
  \l 8399
  \l 839A
  \l 839B
  \l 839C
  \l 839D
  \l 839E
  \l 839F
  \l 83A0
  \l 83A1
  \l 83A2
  \l 83A3
  \l 83A4
  \l 83A5
  \l 83A6
  \l 83A7
  \l 83A8
  \l 83A9
  \l 83AA
  \l 83AB
  \l 83AC
  \l 83AD
  \l 83AE
  \l 83AF
  \l 83B0
  \l 83B1
  \l 83B2
  \l 83B3
  \l 83B4
  \l 83B5
  \l 83B6
  \l 83B7
  \l 83B8
  \l 83B9
  \l 83BA
  \l 83BB
  \l 83BC
  \l 83BD
  \l 83BE
  \l 83BF
  \l 83C0
  \l 83C1
  \l 83C2
  \l 83C3
  \l 83C4
  \l 83C5
  \l 83C6
  \l 83C7
  \l 83C8
  \l 83C9
  \l 83CA
  \l 83CB
  \l 83CC
  \l 83CD
  \l 83CE
  \l 83CF
  \l 83D0
  \l 83D1
  \l 83D2
  \l 83D3
  \l 83D4
  \l 83D5
  \l 83D6
  \l 83D7
  \l 83D8
  \l 83D9
  \l 83DA
  \l 83DB
  \l 83DC
  \l 83DD
  \l 83DE
  \l 83DF
  \l 83E0
  \l 83E1
  \l 83E2
  \l 83E3
  \l 83E4
  \l 83E5
  \l 83E6
  \l 83E7
  \l 83E8
  \l 83E9
  \l 83EA
  \l 83EB
  \l 83EC
  \l 83ED
  \l 83EE
  \l 83EF
  \l 83F0
  \l 83F1
  \l 83F2
  \l 83F3
  \l 83F4
  \l 83F5
  \l 83F6
  \l 83F7
  \l 83F8
  \l 83F9
  \l 83FA
  \l 83FB
  \l 83FC
  \l 83FD
  \l 83FE
  \l 83FF
  \l 8400
  \l 8401
  \l 8402
  \l 8403
  \l 8404
  \l 8405
  \l 8406
  \l 8407
  \l 8408
  \l 8409
  \l 840A
  \l 840B
  \l 840C
  \l 840D
  \l 840E
  \l 840F
  \l 8410
  \l 8411
  \l 8412
  \l 8413
  \l 8414
  \l 8415
  \l 8416
  \l 8417
  \l 8418
  \l 8419
  \l 841A
  \l 841B
  \l 841C
  \l 841D
  \l 841E
  \l 841F
  \l 8420
  \l 8421
  \l 8422
  \l 8423
  \l 8424
  \l 8425
  \l 8426
  \l 8427
  \l 8428
  \l 8429
  \l 842A
  \l 842B
  \l 842C
  \l 842D
  \l 842E
  \l 842F
  \l 8430
  \l 8431
  \l 8432
  \l 8433
  \l 8434
  \l 8435
  \l 8436
  \l 8437
  \l 8438
  \l 8439
  \l 843A
  \l 843B
  \l 843C
  \l 843D
  \l 843E
  \l 843F
  \l 8440
  \l 8441
  \l 8442
  \l 8443
  \l 8444
  \l 8445
  \l 8446
  \l 8447
  \l 8448
  \l 8449
  \l 844A
  \l 844B
  \l 844C
  \l 844D
  \l 844E
  \l 844F
  \l 8450
  \l 8451
  \l 8452
  \l 8453
  \l 8454
  \l 8455
  \l 8456
  \l 8457
  \l 8458
  \l 8459
  \l 845A
  \l 845B
  \l 845C
  \l 845D
  \l 845E
  \l 845F
  \l 8460
  \l 8461
  \l 8462
  \l 8463
  \l 8464
  \l 8465
  \l 8466
  \l 8467
  \l 8468
  \l 8469
  \l 846A
  \l 846B
  \l 846C
  \l 846D
  \l 846E
  \l 846F
  \l 8470
  \l 8471
  \l 8472
  \l 8473
  \l 8474
  \l 8475
  \l 8476
  \l 8477
  \l 8478
  \l 8479
  \l 847A
  \l 847B
  \l 847C
  \l 847D
  \l 847E
  \l 847F
  \l 8480
  \l 8481
  \l 8482
  \l 8483
  \l 8484
  \l 8485
  \l 8486
  \l 8487
  \l 8488
  \l 8489
  \l 848A
  \l 848B
  \l 848C
  \l 848D
  \l 848E
  \l 848F
  \l 8490
  \l 8491
  \l 8492
  \l 8493
  \l 8494
  \l 8495
  \l 8496
  \l 8497
  \l 8498
  \l 8499
  \l 849A
  \l 849B
  \l 849C
  \l 849D
  \l 849E
  \l 849F
  \l 84A0
  \l 84A1
  \l 84A2
  \l 84A3
  \l 84A4
  \l 84A5
  \l 84A6
  \l 84A7
  \l 84A8
  \l 84A9
  \l 84AA
  \l 84AB
  \l 84AC
  \l 84AD
  \l 84AE
  \l 84AF
  \l 84B0
  \l 84B1
  \l 84B2
  \l 84B3
  \l 84B4
  \l 84B5
  \l 84B6
  \l 84B7
  \l 84B8
  \l 84B9
  \l 84BA
  \l 84BB
  \l 84BC
  \l 84BD
  \l 84BE
  \l 84BF
  \l 84C0
  \l 84C1
  \l 84C2
  \l 84C3
  \l 84C4
  \l 84C5
  \l 84C6
  \l 84C7
  \l 84C8
  \l 84C9
  \l 84CA
  \l 84CB
  \l 84CC
  \l 84CD
  \l 84CE
  \l 84CF
  \l 84D0
  \l 84D1
  \l 84D2
  \l 84D3
  \l 84D4
  \l 84D5
  \l 84D6
  \l 84D7
  \l 84D8
  \l 84D9
  \l 84DA
  \l 84DB
  \l 84DC
  \l 84DD
  \l 84DE
  \l 84DF
  \l 84E0
  \l 84E1
  \l 84E2
  \l 84E3
  \l 84E4
  \l 84E5
  \l 84E6
  \l 84E7
  \l 84E8
  \l 84E9
  \l 84EA
  \l 84EB
  \l 84EC
  \l 84ED
  \l 84EE
  \l 84EF
  \l 84F0
  \l 84F1
  \l 84F2
  \l 84F3
  \l 84F4
  \l 84F5
  \l 84F6
  \l 84F7
  \l 84F8
  \l 84F9
  \l 84FA
  \l 84FB
  \l 84FC
  \l 84FD
  \l 84FE
  \l 84FF
  \l 8500
  \l 8501
  \l 8502
  \l 8503
  \l 8504
  \l 8505
  \l 8506
  \l 8507
  \l 8508
  \l 8509
  \l 850A
  \l 850B
  \l 850C
  \l 850D
  \l 850E
  \l 850F
  \l 8510
  \l 8511
  \l 8512
  \l 8513
  \l 8514
  \l 8515
  \l 8516
  \l 8517
  \l 8518
  \l 8519
  \l 851A
  \l 851B
  \l 851C
  \l 851D
  \l 851E
  \l 851F
  \l 8520
  \l 8521
  \l 8522
  \l 8523
  \l 8524
  \l 8525
  \l 8526
  \l 8527
  \l 8528
  \l 8529
  \l 852A
  \l 852B
  \l 852C
  \l 852D
  \l 852E
  \l 852F
  \l 8530
  \l 8531
  \l 8532
  \l 8533
  \l 8534
  \l 8535
  \l 8536
  \l 8537
  \l 8538
  \l 8539
  \l 853A
  \l 853B
  \l 853C
  \l 853D
  \l 853E
  \l 853F
  \l 8540
  \l 8541
  \l 8542
  \l 8543
  \l 8544
  \l 8545
  \l 8546
  \l 8547
  \l 8548
  \l 8549
  \l 854A
  \l 854B
  \l 854C
  \l 854D
  \l 854E
  \l 854F
  \l 8550
  \l 8551
  \l 8552
  \l 8553
  \l 8554
  \l 8555
  \l 8556
  \l 8557
  \l 8558
  \l 8559
  \l 855A
  \l 855B
  \l 855C
  \l 855D
  \l 855E
  \l 855F
  \l 8560
  \l 8561
  \l 8562
  \l 8563
  \l 8564
  \l 8565
  \l 8566
  \l 8567
  \l 8568
  \l 8569
  \l 856A
  \l 856B
  \l 856C
  \l 856D
  \l 856E
  \l 856F
  \l 8570
  \l 8571
  \l 8572
  \l 8573
  \l 8574
  \l 8575
  \l 8576
  \l 8577
  \l 8578
  \l 8579
  \l 857A
  \l 857B
  \l 857C
  \l 857D
  \l 857E
  \l 857F
  \l 8580
  \l 8581
  \l 8582
  \l 8583
  \l 8584
  \l 8585
  \l 8586
  \l 8587
  \l 8588
  \l 8589
  \l 858A
  \l 858B
  \l 858C
  \l 858D
  \l 858E
  \l 858F
  \l 8590
  \l 8591
  \l 8592
  \l 8593
  \l 8594
  \l 8595
  \l 8596
  \l 8597
  \l 8598
  \l 8599
  \l 859A
  \l 859B
  \l 859C
  \l 859D
  \l 859E
  \l 859F
  \l 85A0
  \l 85A1
  \l 85A2
  \l 85A3
  \l 85A4
  \l 85A5
  \l 85A6
  \l 85A7
  \l 85A8
  \l 85A9
  \l 85AA
  \l 85AB
  \l 85AC
  \l 85AD
  \l 85AE
  \l 85AF
  \l 85B0
  \l 85B1
  \l 85B2
  \l 85B3
  \l 85B4
  \l 85B5
  \l 85B6
  \l 85B7
  \l 85B8
  \l 85B9
  \l 85BA
  \l 85BB
  \l 85BC
  \l 85BD
  \l 85BE
  \l 85BF
  \l 85C0
  \l 85C1
  \l 85C2
  \l 85C3
  \l 85C4
  \l 85C5
  \l 85C6
  \l 85C7
  \l 85C8
  \l 85C9
  \l 85CA
  \l 85CB
  \l 85CC
  \l 85CD
  \l 85CE
  \l 85CF
  \l 85D0
  \l 85D1
  \l 85D2
  \l 85D3
  \l 85D4
  \l 85D5
  \l 85D6
  \l 85D7
  \l 85D8
  \l 85D9
  \l 85DA
  \l 85DB
  \l 85DC
  \l 85DD
  \l 85DE
  \l 85DF
  \l 85E0
  \l 85E1
  \l 85E2
  \l 85E3
  \l 85E4
  \l 85E5
  \l 85E6
  \l 85E7
  \l 85E8
  \l 85E9
  \l 85EA
  \l 85EB
  \l 85EC
  \l 85ED
  \l 85EE
  \l 85EF
  \l 85F0
  \l 85F1
  \l 85F2
  \l 85F3
  \l 85F4
  \l 85F5
  \l 85F6
  \l 85F7
  \l 85F8
  \l 85F9
  \l 85FA
  \l 85FB
  \l 85FC
  \l 85FD
  \l 85FE
  \l 85FF
  \l 8600
  \l 8601
  \l 8602
  \l 8603
  \l 8604
  \l 8605
  \l 8606
  \l 8607
  \l 8608
  \l 8609
  \l 860A
  \l 860B
  \l 860C
  \l 860D
  \l 860E
  \l 860F
  \l 8610
  \l 8611
  \l 8612
  \l 8613
  \l 8614
  \l 8615
  \l 8616
  \l 8617
  \l 8618
  \l 8619
  \l 861A
  \l 861B
  \l 861C
  \l 861D
  \l 861E
  \l 861F
  \l 8620
  \l 8621
  \l 8622
  \l 8623
  \l 8624
  \l 8625
  \l 8626
  \l 8627
  \l 8628
  \l 8629
  \l 862A
  \l 862B
  \l 862C
  \l 862D
  \l 862E
  \l 862F
  \l 8630
  \l 8631
  \l 8632
  \l 8633
  \l 8634
  \l 8635
  \l 8636
  \l 8637
  \l 8638
  \l 8639
  \l 863A
  \l 863B
  \l 863C
  \l 863D
  \l 863E
  \l 863F
  \l 8640
  \l 8641
  \l 8642
  \l 8643
  \l 8644
  \l 8645
  \l 8646
  \l 8647
  \l 8648
  \l 8649
  \l 864A
  \l 864B
  \l 864C
  \l 864D
  \l 864E
  \l 864F
  \l 8650
  \l 8651
  \l 8652
  \l 8653
  \l 8654
  \l 8655
  \l 8656
  \l 8657
  \l 8658
  \l 8659
  \l 865A
  \l 865B
  \l 865C
  \l 865D
  \l 865E
  \l 865F
  \l 8660
  \l 8661
  \l 8662
  \l 8663
  \l 8664
  \l 8665
  \l 8666
  \l 8667
  \l 8668
  \l 8669
  \l 866A
  \l 866B
  \l 866C
  \l 866D
  \l 866E
  \l 866F
  \l 8670
  \l 8671
  \l 8672
  \l 8673
  \l 8674
  \l 8675
  \l 8676
  \l 8677
  \l 8678
  \l 8679
  \l 867A
  \l 867B
  \l 867C
  \l 867D
  \l 867E
  \l 867F
  \l 8680
  \l 8681
  \l 8682
  \l 8683
  \l 8684
  \l 8685
  \l 8686
  \l 8687
  \l 8688
  \l 8689
  \l 868A
  \l 868B
  \l 868C
  \l 868D
  \l 868E
  \l 868F
  \l 8690
  \l 8691
  \l 8692
  \l 8693
  \l 8694
  \l 8695
  \l 8696
  \l 8697
  \l 8698
  \l 8699
  \l 869A
  \l 869B
  \l 869C
  \l 869D
  \l 869E
  \l 869F
  \l 86A0
  \l 86A1
  \l 86A2
  \l 86A3
  \l 86A4
  \l 86A5
  \l 86A6
  \l 86A7
  \l 86A8
  \l 86A9
  \l 86AA
  \l 86AB
  \l 86AC
  \l 86AD
  \l 86AE
  \l 86AF
  \l 86B0
  \l 86B1
  \l 86B2
  \l 86B3
  \l 86B4
  \l 86B5
  \l 86B6
  \l 86B7
  \l 86B8
  \l 86B9
  \l 86BA
  \l 86BB
  \l 86BC
  \l 86BD
  \l 86BE
  \l 86BF
  \l 86C0
  \l 86C1
  \l 86C2
  \l 86C3
  \l 86C4
  \l 86C5
  \l 86C6
  \l 86C7
  \l 86C8
  \l 86C9
  \l 86CA
  \l 86CB
  \l 86CC
  \l 86CD
  \l 86CE
  \l 86CF
  \l 86D0
  \l 86D1
  \l 86D2
  \l 86D3
  \l 86D4
  \l 86D5
  \l 86D6
  \l 86D7
  \l 86D8
  \l 86D9
  \l 86DA
  \l 86DB
  \l 86DC
  \l 86DD
  \l 86DE
  \l 86DF
  \l 86E0
  \l 86E1
  \l 86E2
  \l 86E3
  \l 86E4
  \l 86E5
  \l 86E6
  \l 86E7
  \l 86E8
  \l 86E9
  \l 86EA
  \l 86EB
  \l 86EC
  \l 86ED
  \l 86EE
  \l 86EF
  \l 86F0
  \l 86F1
  \l 86F2
  \l 86F3
  \l 86F4
  \l 86F5
  \l 86F6
  \l 86F7
  \l 86F8
  \l 86F9
  \l 86FA
  \l 86FB
  \l 86FC
  \l 86FD
  \l 86FE
  \l 86FF
  \l 8700
  \l 8701
  \l 8702
  \l 8703
  \l 8704
  \l 8705
  \l 8706
  \l 8707
  \l 8708
  \l 8709
  \l 870A
  \l 870B
  \l 870C
  \l 870D
  \l 870E
  \l 870F
  \l 8710
  \l 8711
  \l 8712
  \l 8713
  \l 8714
  \l 8715
  \l 8716
  \l 8717
  \l 8718
  \l 8719
  \l 871A
  \l 871B
  \l 871C
  \l 871D
  \l 871E
  \l 871F
  \l 8720
  \l 8721
  \l 8722
  \l 8723
  \l 8724
  \l 8725
  \l 8726
  \l 8727
  \l 8728
  \l 8729
  \l 872A
  \l 872B
  \l 872C
  \l 872D
  \l 872E
  \l 872F
  \l 8730
  \l 8731
  \l 8732
  \l 8733
  \l 8734
  \l 8735
  \l 8736
  \l 8737
  \l 8738
  \l 8739
  \l 873A
  \l 873B
  \l 873C
  \l 873D
  \l 873E
  \l 873F
  \l 8740
  \l 8741
  \l 8742
  \l 8743
  \l 8744
  \l 8745
  \l 8746
  \l 8747
  \l 8748
  \l 8749
  \l 874A
  \l 874B
  \l 874C
  \l 874D
  \l 874E
  \l 874F
  \l 8750
  \l 8751
  \l 8752
  \l 8753
  \l 8754
  \l 8755
  \l 8756
  \l 8757
  \l 8758
  \l 8759
  \l 875A
  \l 875B
  \l 875C
  \l 875D
  \l 875E
  \l 875F
  \l 8760
  \l 8761
  \l 8762
  \l 8763
  \l 8764
  \l 8765
  \l 8766
  \l 8767
  \l 8768
  \l 8769
  \l 876A
  \l 876B
  \l 876C
  \l 876D
  \l 876E
  \l 876F
  \l 8770
  \l 8771
  \l 8772
  \l 8773
  \l 8774
  \l 8775
  \l 8776
  \l 8777
  \l 8778
  \l 8779
  \l 877A
  \l 877B
  \l 877C
  \l 877D
  \l 877E
  \l 877F
  \l 8780
  \l 8781
  \l 8782
  \l 8783
  \l 8784
  \l 8785
  \l 8786
  \l 8787
  \l 8788
  \l 8789
  \l 878A
  \l 878B
  \l 878C
  \l 878D
  \l 878E
  \l 878F
  \l 8790
  \l 8791
  \l 8792
  \l 8793
  \l 8794
  \l 8795
  \l 8796
  \l 8797
  \l 8798
  \l 8799
  \l 879A
  \l 879B
  \l 879C
  \l 879D
  \l 879E
  \l 879F
  \l 87A0
  \l 87A1
  \l 87A2
  \l 87A3
  \l 87A4
  \l 87A5
  \l 87A6
  \l 87A7
  \l 87A8
  \l 87A9
  \l 87AA
  \l 87AB
  \l 87AC
  \l 87AD
  \l 87AE
  \l 87AF
  \l 87B0
  \l 87B1
  \l 87B2
  \l 87B3
  \l 87B4
  \l 87B5
  \l 87B6
  \l 87B7
  \l 87B8
  \l 87B9
  \l 87BA
  \l 87BB
  \l 87BC
  \l 87BD
  \l 87BE
  \l 87BF
  \l 87C0
  \l 87C1
  \l 87C2
  \l 87C3
  \l 87C4
  \l 87C5
  \l 87C6
  \l 87C7
  \l 87C8
  \l 87C9
  \l 87CA
  \l 87CB
  \l 87CC
  \l 87CD
  \l 87CE
  \l 87CF
  \l 87D0
  \l 87D1
  \l 87D2
  \l 87D3
  \l 87D4
  \l 87D5
  \l 87D6
  \l 87D7
  \l 87D8
  \l 87D9
  \l 87DA
  \l 87DB
  \l 87DC
  \l 87DD
  \l 87DE
  \l 87DF
  \l 87E0
  \l 87E1
  \l 87E2
  \l 87E3
  \l 87E4
  \l 87E5
  \l 87E6
  \l 87E7
  \l 87E8
  \l 87E9
  \l 87EA
  \l 87EB
  \l 87EC
  \l 87ED
  \l 87EE
  \l 87EF
  \l 87F0
  \l 87F1
  \l 87F2
  \l 87F3
  \l 87F4
  \l 87F5
  \l 87F6
  \l 87F7
  \l 87F8
  \l 87F9
  \l 87FA
  \l 87FB
  \l 87FC
  \l 87FD
  \l 87FE
  \l 87FF
  \l 8800
  \l 8801
  \l 8802
  \l 8803
  \l 8804
  \l 8805
  \l 8806
  \l 8807
  \l 8808
  \l 8809
  \l 880A
  \l 880B
  \l 880C
  \l 880D
  \l 880E
  \l 880F
  \l 8810
  \l 8811
  \l 8812
  \l 8813
  \l 8814
  \l 8815
  \l 8816
  \l 8817
  \l 8818
  \l 8819
  \l 881A
  \l 881B
  \l 881C
  \l 881D
  \l 881E
  \l 881F
  \l 8820
  \l 8821
  \l 8822
  \l 8823
  \l 8824
  \l 8825
  \l 8826
  \l 8827
  \l 8828
  \l 8829
  \l 882A
  \l 882B
  \l 882C
  \l 882D
  \l 882E
  \l 882F
  \l 8830
  \l 8831
  \l 8832
  \l 8833
  \l 8834
  \l 8835
  \l 8836
  \l 8837
  \l 8838
  \l 8839
  \l 883A
  \l 883B
  \l 883C
  \l 883D
  \l 883E
  \l 883F
  \l 8840
  \l 8841
  \l 8842
  \l 8843
  \l 8844
  \l 8845
  \l 8846
  \l 8847
  \l 8848
  \l 8849
  \l 884A
  \l 884B
  \l 884C
  \l 884D
  \l 884E
  \l 884F
  \l 8850
  \l 8851
  \l 8852
  \l 8853
  \l 8854
  \l 8855
  \l 8856
  \l 8857
  \l 8858
  \l 8859
  \l 885A
  \l 885B
  \l 885C
  \l 885D
  \l 885E
  \l 885F
  \l 8860
  \l 8861
  \l 8862
  \l 8863
  \l 8864
  \l 8865
  \l 8866
  \l 8867
  \l 8868
  \l 8869
  \l 886A
  \l 886B
  \l 886C
  \l 886D
  \l 886E
  \l 886F
  \l 8870
  \l 8871
  \l 8872
  \l 8873
  \l 8874
  \l 8875
  \l 8876
  \l 8877
  \l 8878
  \l 8879
  \l 887A
  \l 887B
  \l 887C
  \l 887D
  \l 887E
  \l 887F
  \l 8880
  \l 8881
  \l 8882
  \l 8883
  \l 8884
  \l 8885
  \l 8886
  \l 8887
  \l 8888
  \l 8889
  \l 888A
  \l 888B
  \l 888C
  \l 888D
  \l 888E
  \l 888F
  \l 8890
  \l 8891
  \l 8892
  \l 8893
  \l 8894
  \l 8895
  \l 8896
  \l 8897
  \l 8898
  \l 8899
  \l 889A
  \l 889B
  \l 889C
  \l 889D
  \l 889E
  \l 889F
  \l 88A0
  \l 88A1
  \l 88A2
  \l 88A3
  \l 88A4
  \l 88A5
  \l 88A6
  \l 88A7
  \l 88A8
  \l 88A9
  \l 88AA
  \l 88AB
  \l 88AC
  \l 88AD
  \l 88AE
  \l 88AF
  \l 88B0
  \l 88B1
  \l 88B2
  \l 88B3
  \l 88B4
  \l 88B5
  \l 88B6
  \l 88B7
  \l 88B8
  \l 88B9
  \l 88BA
  \l 88BB
  \l 88BC
  \l 88BD
  \l 88BE
  \l 88BF
  \l 88C0
  \l 88C1
  \l 88C2
  \l 88C3
  \l 88C4
  \l 88C5
  \l 88C6
  \l 88C7
  \l 88C8
  \l 88C9
  \l 88CA
  \l 88CB
  \l 88CC
  \l 88CD
  \l 88CE
  \l 88CF
  \l 88D0
  \l 88D1
  \l 88D2
  \l 88D3
  \l 88D4
  \l 88D5
  \l 88D6
  \l 88D7
  \l 88D8
  \l 88D9
  \l 88DA
  \l 88DB
  \l 88DC
  \l 88DD
  \l 88DE
  \l 88DF
  \l 88E0
  \l 88E1
  \l 88E2
  \l 88E3
  \l 88E4
  \l 88E5
  \l 88E6
  \l 88E7
  \l 88E8
  \l 88E9
  \l 88EA
  \l 88EB
  \l 88EC
  \l 88ED
  \l 88EE
  \l 88EF
  \l 88F0
  \l 88F1
  \l 88F2
  \l 88F3
  \l 88F4
  \l 88F5
  \l 88F6
  \l 88F7
  \l 88F8
  \l 88F9
  \l 88FA
  \l 88FB
  \l 88FC
  \l 88FD
  \l 88FE
  \l 88FF
  \l 8900
  \l 8901
  \l 8902
  \l 8903
  \l 8904
  \l 8905
  \l 8906
  \l 8907
  \l 8908
  \l 8909
  \l 890A
  \l 890B
  \l 890C
  \l 890D
  \l 890E
  \l 890F
  \l 8910
  \l 8911
  \l 8912
  \l 8913
  \l 8914
  \l 8915
  \l 8916
  \l 8917
  \l 8918
  \l 8919
  \l 891A
  \l 891B
  \l 891C
  \l 891D
  \l 891E
  \l 891F
  \l 8920
  \l 8921
  \l 8922
  \l 8923
  \l 8924
  \l 8925
  \l 8926
  \l 8927
  \l 8928
  \l 8929
  \l 892A
  \l 892B
  \l 892C
  \l 892D
  \l 892E
  \l 892F
  \l 8930
  \l 8931
  \l 8932
  \l 8933
  \l 8934
  \l 8935
  \l 8936
  \l 8937
  \l 8938
  \l 8939
  \l 893A
  \l 893B
  \l 893C
  \l 893D
  \l 893E
  \l 893F
  \l 8940
  \l 8941
  \l 8942
  \l 8943
  \l 8944
  \l 8945
  \l 8946
  \l 8947
  \l 8948
  \l 8949
  \l 894A
  \l 894B
  \l 894C
  \l 894D
  \l 894E
  \l 894F
  \l 8950
  \l 8951
  \l 8952
  \l 8953
  \l 8954
  \l 8955
  \l 8956
  \l 8957
  \l 8958
  \l 8959
  \l 895A
  \l 895B
  \l 895C
  \l 895D
  \l 895E
  \l 895F
  \l 8960
  \l 8961
  \l 8962
  \l 8963
  \l 8964
  \l 8965
  \l 8966
  \l 8967
  \l 8968
  \l 8969
  \l 896A
  \l 896B
  \l 896C
  \l 896D
  \l 896E
  \l 896F
  \l 8970
  \l 8971
  \l 8972
  \l 8973
  \l 8974
  \l 8975
  \l 8976
  \l 8977
  \l 8978
  \l 8979
  \l 897A
  \l 897B
  \l 897C
  \l 897D
  \l 897E
  \l 897F
  \l 8980
  \l 8981
  \l 8982
  \l 8983
  \l 8984
  \l 8985
  \l 8986
  \l 8987
  \l 8988
  \l 8989
  \l 898A
  \l 898B
  \l 898C
  \l 898D
  \l 898E
  \l 898F
  \l 8990
  \l 8991
  \l 8992
  \l 8993
  \l 8994
  \l 8995
  \l 8996
  \l 8997
  \l 8998
  \l 8999
  \l 899A
  \l 899B
  \l 899C
  \l 899D
  \l 899E
  \l 899F
  \l 89A0
  \l 89A1
  \l 89A2
  \l 89A3
  \l 89A4
  \l 89A5
  \l 89A6
  \l 89A7
  \l 89A8
  \l 89A9
  \l 89AA
  \l 89AB
  \l 89AC
  \l 89AD
  \l 89AE
  \l 89AF
  \l 89B0
  \l 89B1
  \l 89B2
  \l 89B3
  \l 89B4
  \l 89B5
  \l 89B6
  \l 89B7
  \l 89B8
  \l 89B9
  \l 89BA
  \l 89BB
  \l 89BC
  \l 89BD
  \l 89BE
  \l 89BF
  \l 89C0
  \l 89C1
  \l 89C2
  \l 89C3
  \l 89C4
  \l 89C5
  \l 89C6
  \l 89C7
  \l 89C8
  \l 89C9
  \l 89CA
  \l 89CB
  \l 89CC
  \l 89CD
  \l 89CE
  \l 89CF
  \l 89D0
  \l 89D1
  \l 89D2
  \l 89D3
  \l 89D4
  \l 89D5
  \l 89D6
  \l 89D7
  \l 89D8
  \l 89D9
  \l 89DA
  \l 89DB
  \l 89DC
  \l 89DD
  \l 89DE
  \l 89DF
  \l 89E0
  \l 89E1
  \l 89E2
  \l 89E3
  \l 89E4
  \l 89E5
  \l 89E6
  \l 89E7
  \l 89E8
  \l 89E9
  \l 89EA
  \l 89EB
  \l 89EC
  \l 89ED
  \l 89EE
  \l 89EF
  \l 89F0
  \l 89F1
  \l 89F2
  \l 89F3
  \l 89F4
  \l 89F5
  \l 89F6
  \l 89F7
  \l 89F8
  \l 89F9
  \l 89FA
  \l 89FB
  \l 89FC
  \l 89FD
  \l 89FE
  \l 89FF
  \l 8A00
  \l 8A01
  \l 8A02
  \l 8A03
  \l 8A04
  \l 8A05
  \l 8A06
  \l 8A07
  \l 8A08
  \l 8A09
  \l 8A0A
  \l 8A0B
  \l 8A0C
  \l 8A0D
  \l 8A0E
  \l 8A0F
  \l 8A10
  \l 8A11
  \l 8A12
  \l 8A13
  \l 8A14
  \l 8A15
  \l 8A16
  \l 8A17
  \l 8A18
  \l 8A19
  \l 8A1A
  \l 8A1B
  \l 8A1C
  \l 8A1D
  \l 8A1E
  \l 8A1F
  \l 8A20
  \l 8A21
  \l 8A22
  \l 8A23
  \l 8A24
  \l 8A25
  \l 8A26
  \l 8A27
  \l 8A28
  \l 8A29
  \l 8A2A
  \l 8A2B
  \l 8A2C
  \l 8A2D
  \l 8A2E
  \l 8A2F
  \l 8A30
  \l 8A31
  \l 8A32
  \l 8A33
  \l 8A34
  \l 8A35
  \l 8A36
  \l 8A37
  \l 8A38
  \l 8A39
  \l 8A3A
  \l 8A3B
  \l 8A3C
  \l 8A3D
  \l 8A3E
  \l 8A3F
  \l 8A40
  \l 8A41
  \l 8A42
  \l 8A43
  \l 8A44
  \l 8A45
  \l 8A46
  \l 8A47
  \l 8A48
  \l 8A49
  \l 8A4A
  \l 8A4B
  \l 8A4C
  \l 8A4D
  \l 8A4E
  \l 8A4F
  \l 8A50
  \l 8A51
  \l 8A52
  \l 8A53
  \l 8A54
  \l 8A55
  \l 8A56
  \l 8A57
  \l 8A58
  \l 8A59
  \l 8A5A
  \l 8A5B
  \l 8A5C
  \l 8A5D
  \l 8A5E
  \l 8A5F
  \l 8A60
  \l 8A61
  \l 8A62
  \l 8A63
  \l 8A64
  \l 8A65
  \l 8A66
  \l 8A67
  \l 8A68
  \l 8A69
  \l 8A6A
  \l 8A6B
  \l 8A6C
  \l 8A6D
  \l 8A6E
  \l 8A6F
  \l 8A70
  \l 8A71
  \l 8A72
  \l 8A73
  \l 8A74
  \l 8A75
  \l 8A76
  \l 8A77
  \l 8A78
  \l 8A79
  \l 8A7A
  \l 8A7B
  \l 8A7C
  \l 8A7D
  \l 8A7E
  \l 8A7F
  \l 8A80
  \l 8A81
  \l 8A82
  \l 8A83
  \l 8A84
  \l 8A85
  \l 8A86
  \l 8A87
  \l 8A88
  \l 8A89
  \l 8A8A
  \l 8A8B
  \l 8A8C
  \l 8A8D
  \l 8A8E
  \l 8A8F
  \l 8A90
  \l 8A91
  \l 8A92
  \l 8A93
  \l 8A94
  \l 8A95
  \l 8A96
  \l 8A97
  \l 8A98
  \l 8A99
  \l 8A9A
  \l 8A9B
  \l 8A9C
  \l 8A9D
  \l 8A9E
  \l 8A9F
  \l 8AA0
  \l 8AA1
  \l 8AA2
  \l 8AA3
  \l 8AA4
  \l 8AA5
  \l 8AA6
  \l 8AA7
  \l 8AA8
  \l 8AA9
  \l 8AAA
  \l 8AAB
  \l 8AAC
  \l 8AAD
  \l 8AAE
  \l 8AAF
  \l 8AB0
  \l 8AB1
  \l 8AB2
  \l 8AB3
  \l 8AB4
  \l 8AB5
  \l 8AB6
  \l 8AB7
  \l 8AB8
  \l 8AB9
  \l 8ABA
  \l 8ABB
  \l 8ABC
  \l 8ABD
  \l 8ABE
  \l 8ABF
  \l 8AC0
  \l 8AC1
  \l 8AC2
  \l 8AC3
  \l 8AC4
  \l 8AC5
  \l 8AC6
  \l 8AC7
  \l 8AC8
  \l 8AC9
  \l 8ACA
  \l 8ACB
  \l 8ACC
  \l 8ACD
  \l 8ACE
  \l 8ACF
  \l 8AD0
  \l 8AD1
  \l 8AD2
  \l 8AD3
  \l 8AD4
  \l 8AD5
  \l 8AD6
  \l 8AD7
  \l 8AD8
  \l 8AD9
  \l 8ADA
  \l 8ADB
  \l 8ADC
  \l 8ADD
  \l 8ADE
  \l 8ADF
  \l 8AE0
  \l 8AE1
  \l 8AE2
  \l 8AE3
  \l 8AE4
  \l 8AE5
  \l 8AE6
  \l 8AE7
  \l 8AE8
  \l 8AE9
  \l 8AEA
  \l 8AEB
  \l 8AEC
  \l 8AED
  \l 8AEE
  \l 8AEF
  \l 8AF0
  \l 8AF1
  \l 8AF2
  \l 8AF3
  \l 8AF4
  \l 8AF5
  \l 8AF6
  \l 8AF7
  \l 8AF8
  \l 8AF9
  \l 8AFA
  \l 8AFB
  \l 8AFC
  \l 8AFD
  \l 8AFE
  \l 8AFF
  \l 8B00
  \l 8B01
  \l 8B02
  \l 8B03
  \l 8B04
  \l 8B05
  \l 8B06
  \l 8B07
  \l 8B08
  \l 8B09
  \l 8B0A
  \l 8B0B
  \l 8B0C
  \l 8B0D
  \l 8B0E
  \l 8B0F
  \l 8B10
  \l 8B11
  \l 8B12
  \l 8B13
  \l 8B14
  \l 8B15
  \l 8B16
  \l 8B17
  \l 8B18
  \l 8B19
  \l 8B1A
  \l 8B1B
  \l 8B1C
  \l 8B1D
  \l 8B1E
  \l 8B1F
  \l 8B20
  \l 8B21
  \l 8B22
  \l 8B23
  \l 8B24
  \l 8B25
  \l 8B26
  \l 8B27
  \l 8B28
  \l 8B29
  \l 8B2A
  \l 8B2B
  \l 8B2C
  \l 8B2D
  \l 8B2E
  \l 8B2F
  \l 8B30
  \l 8B31
  \l 8B32
  \l 8B33
  \l 8B34
  \l 8B35
  \l 8B36
  \l 8B37
  \l 8B38
  \l 8B39
  \l 8B3A
  \l 8B3B
  \l 8B3C
  \l 8B3D
  \l 8B3E
  \l 8B3F
  \l 8B40
  \l 8B41
  \l 8B42
  \l 8B43
  \l 8B44
  \l 8B45
  \l 8B46
  \l 8B47
  \l 8B48
  \l 8B49
  \l 8B4A
  \l 8B4B
  \l 8B4C
  \l 8B4D
  \l 8B4E
  \l 8B4F
  \l 8B50
  \l 8B51
  \l 8B52
  \l 8B53
  \l 8B54
  \l 8B55
  \l 8B56
  \l 8B57
  \l 8B58
  \l 8B59
  \l 8B5A
  \l 8B5B
  \l 8B5C
  \l 8B5D
  \l 8B5E
  \l 8B5F
  \l 8B60
  \l 8B61
  \l 8B62
  \l 8B63
  \l 8B64
  \l 8B65
  \l 8B66
  \l 8B67
  \l 8B68
  \l 8B69
  \l 8B6A
  \l 8B6B
  \l 8B6C
  \l 8B6D
  \l 8B6E
  \l 8B6F
  \l 8B70
  \l 8B71
  \l 8B72
  \l 8B73
  \l 8B74
  \l 8B75
  \l 8B76
  \l 8B77
  \l 8B78
  \l 8B79
  \l 8B7A
  \l 8B7B
  \l 8B7C
  \l 8B7D
  \l 8B7E
  \l 8B7F
  \l 8B80
  \l 8B81
  \l 8B82
  \l 8B83
  \l 8B84
  \l 8B85
  \l 8B86
  \l 8B87
  \l 8B88
  \l 8B89
  \l 8B8A
  \l 8B8B
  \l 8B8C
  \l 8B8D
  \l 8B8E
  \l 8B8F
  \l 8B90
  \l 8B91
  \l 8B92
  \l 8B93
  \l 8B94
  \l 8B95
  \l 8B96
  \l 8B97
  \l 8B98
  \l 8B99
  \l 8B9A
  \l 8B9B
  \l 8B9C
  \l 8B9D
  \l 8B9E
  \l 8B9F
  \l 8BA0
  \l 8BA1
  \l 8BA2
  \l 8BA3
  \l 8BA4
  \l 8BA5
  \l 8BA6
  \l 8BA7
  \l 8BA8
  \l 8BA9
  \l 8BAA
  \l 8BAB
  \l 8BAC
  \l 8BAD
  \l 8BAE
  \l 8BAF
  \l 8BB0
  \l 8BB1
  \l 8BB2
  \l 8BB3
  \l 8BB4
  \l 8BB5
  \l 8BB6
  \l 8BB7
  \l 8BB8
  \l 8BB9
  \l 8BBA
  \l 8BBB
  \l 8BBC
  \l 8BBD
  \l 8BBE
  \l 8BBF
  \l 8BC0
  \l 8BC1
  \l 8BC2
  \l 8BC3
  \l 8BC4
  \l 8BC5
  \l 8BC6
  \l 8BC7
  \l 8BC8
  \l 8BC9
  \l 8BCA
  \l 8BCB
  \l 8BCC
  \l 8BCD
  \l 8BCE
  \l 8BCF
  \l 8BD0
  \l 8BD1
  \l 8BD2
  \l 8BD3
  \l 8BD4
  \l 8BD5
  \l 8BD6
  \l 8BD7
  \l 8BD8
  \l 8BD9
  \l 8BDA
  \l 8BDB
  \l 8BDC
  \l 8BDD
  \l 8BDE
  \l 8BDF
  \l 8BE0
  \l 8BE1
  \l 8BE2
  \l 8BE3
  \l 8BE4
  \l 8BE5
  \l 8BE6
  \l 8BE7
  \l 8BE8
  \l 8BE9
  \l 8BEA
  \l 8BEB
  \l 8BEC
  \l 8BED
  \l 8BEE
  \l 8BEF
  \l 8BF0
  \l 8BF1
  \l 8BF2
  \l 8BF3
  \l 8BF4
  \l 8BF5
  \l 8BF6
  \l 8BF7
  \l 8BF8
  \l 8BF9
  \l 8BFA
  \l 8BFB
  \l 8BFC
  \l 8BFD
  \l 8BFE
  \l 8BFF
  \l 8C00
  \l 8C01
  \l 8C02
  \l 8C03
  \l 8C04
  \l 8C05
  \l 8C06
  \l 8C07
  \l 8C08
  \l 8C09
  \l 8C0A
  \l 8C0B
  \l 8C0C
  \l 8C0D
  \l 8C0E
  \l 8C0F
  \l 8C10
  \l 8C11
  \l 8C12
  \l 8C13
  \l 8C14
  \l 8C15
  \l 8C16
  \l 8C17
  \l 8C18
  \l 8C19
  \l 8C1A
  \l 8C1B
  \l 8C1C
  \l 8C1D
  \l 8C1E
  \l 8C1F
  \l 8C20
  \l 8C21
  \l 8C22
  \l 8C23
  \l 8C24
  \l 8C25
  \l 8C26
  \l 8C27
  \l 8C28
  \l 8C29
  \l 8C2A
  \l 8C2B
  \l 8C2C
  \l 8C2D
  \l 8C2E
  \l 8C2F
  \l 8C30
  \l 8C31
  \l 8C32
  \l 8C33
  \l 8C34
  \l 8C35
  \l 8C36
  \l 8C37
  \l 8C38
  \l 8C39
  \l 8C3A
  \l 8C3B
  \l 8C3C
  \l 8C3D
  \l 8C3E
  \l 8C3F
  \l 8C40
  \l 8C41
  \l 8C42
  \l 8C43
  \l 8C44
  \l 8C45
  \l 8C46
  \l 8C47
  \l 8C48
  \l 8C49
  \l 8C4A
  \l 8C4B
  \l 8C4C
  \l 8C4D
  \l 8C4E
  \l 8C4F
  \l 8C50
  \l 8C51
  \l 8C52
  \l 8C53
  \l 8C54
  \l 8C55
  \l 8C56
  \l 8C57
  \l 8C58
  \l 8C59
  \l 8C5A
  \l 8C5B
  \l 8C5C
  \l 8C5D
  \l 8C5E
  \l 8C5F
  \l 8C60
  \l 8C61
  \l 8C62
  \l 8C63
  \l 8C64
  \l 8C65
  \l 8C66
  \l 8C67
  \l 8C68
  \l 8C69
  \l 8C6A
  \l 8C6B
  \l 8C6C
  \l 8C6D
  \l 8C6E
  \l 8C6F
  \l 8C70
  \l 8C71
  \l 8C72
  \l 8C73
  \l 8C74
  \l 8C75
  \l 8C76
  \l 8C77
  \l 8C78
  \l 8C79
  \l 8C7A
  \l 8C7B
  \l 8C7C
  \l 8C7D
  \l 8C7E
  \l 8C7F
  \l 8C80
  \l 8C81
  \l 8C82
  \l 8C83
  \l 8C84
  \l 8C85
  \l 8C86
  \l 8C87
  \l 8C88
  \l 8C89
  \l 8C8A
  \l 8C8B
  \l 8C8C
  \l 8C8D
  \l 8C8E
  \l 8C8F
  \l 8C90
  \l 8C91
  \l 8C92
  \l 8C93
  \l 8C94
  \l 8C95
  \l 8C96
  \l 8C97
  \l 8C98
  \l 8C99
  \l 8C9A
  \l 8C9B
  \l 8C9C
  \l 8C9D
  \l 8C9E
  \l 8C9F
  \l 8CA0
  \l 8CA1
  \l 8CA2
  \l 8CA3
  \l 8CA4
  \l 8CA5
  \l 8CA6
  \l 8CA7
  \l 8CA8
  \l 8CA9
  \l 8CAA
  \l 8CAB
  \l 8CAC
  \l 8CAD
  \l 8CAE
  \l 8CAF
  \l 8CB0
  \l 8CB1
  \l 8CB2
  \l 8CB3
  \l 8CB4
  \l 8CB5
  \l 8CB6
  \l 8CB7
  \l 8CB8
  \l 8CB9
  \l 8CBA
  \l 8CBB
  \l 8CBC
  \l 8CBD
  \l 8CBE
  \l 8CBF
  \l 8CC0
  \l 8CC1
  \l 8CC2
  \l 8CC3
  \l 8CC4
  \l 8CC5
  \l 8CC6
  \l 8CC7
  \l 8CC8
  \l 8CC9
  \l 8CCA
  \l 8CCB
  \l 8CCC
  \l 8CCD
  \l 8CCE
  \l 8CCF
  \l 8CD0
  \l 8CD1
  \l 8CD2
  \l 8CD3
  \l 8CD4
  \l 8CD5
  \l 8CD6
  \l 8CD7
  \l 8CD8
  \l 8CD9
  \l 8CDA
  \l 8CDB
  \l 8CDC
  \l 8CDD
  \l 8CDE
  \l 8CDF
  \l 8CE0
  \l 8CE1
  \l 8CE2
  \l 8CE3
  \l 8CE4
  \l 8CE5
  \l 8CE6
  \l 8CE7
  \l 8CE8
  \l 8CE9
  \l 8CEA
  \l 8CEB
  \l 8CEC
  \l 8CED
  \l 8CEE
  \l 8CEF
  \l 8CF0
  \l 8CF1
  \l 8CF2
  \l 8CF3
  \l 8CF4
  \l 8CF5
  \l 8CF6
  \l 8CF7
  \l 8CF8
  \l 8CF9
  \l 8CFA
  \l 8CFB
  \l 8CFC
  \l 8CFD
  \l 8CFE
  \l 8CFF
  \l 8D00
  \l 8D01
  \l 8D02
  \l 8D03
  \l 8D04
  \l 8D05
  \l 8D06
  \l 8D07
  \l 8D08
  \l 8D09
  \l 8D0A
  \l 8D0B
  \l 8D0C
  \l 8D0D
  \l 8D0E
  \l 8D0F
  \l 8D10
  \l 8D11
  \l 8D12
  \l 8D13
  \l 8D14
  \l 8D15
  \l 8D16
  \l 8D17
  \l 8D18
  \l 8D19
  \l 8D1A
  \l 8D1B
  \l 8D1C
  \l 8D1D
  \l 8D1E
  \l 8D1F
  \l 8D20
  \l 8D21
  \l 8D22
  \l 8D23
  \l 8D24
  \l 8D25
  \l 8D26
  \l 8D27
  \l 8D28
  \l 8D29
  \l 8D2A
  \l 8D2B
  \l 8D2C
  \l 8D2D
  \l 8D2E
  \l 8D2F
  \l 8D30
  \l 8D31
  \l 8D32
  \l 8D33
  \l 8D34
  \l 8D35
  \l 8D36
  \l 8D37
  \l 8D38
  \l 8D39
  \l 8D3A
  \l 8D3B
  \l 8D3C
  \l 8D3D
  \l 8D3E
  \l 8D3F
  \l 8D40
  \l 8D41
  \l 8D42
  \l 8D43
  \l 8D44
  \l 8D45
  \l 8D46
  \l 8D47
  \l 8D48
  \l 8D49
  \l 8D4A
  \l 8D4B
  \l 8D4C
  \l 8D4D
  \l 8D4E
  \l 8D4F
  \l 8D50
  \l 8D51
  \l 8D52
  \l 8D53
  \l 8D54
  \l 8D55
  \l 8D56
  \l 8D57
  \l 8D58
  \l 8D59
  \l 8D5A
  \l 8D5B
  \l 8D5C
  \l 8D5D
  \l 8D5E
  \l 8D5F
  \l 8D60
  \l 8D61
  \l 8D62
  \l 8D63
  \l 8D64
  \l 8D65
  \l 8D66
  \l 8D67
  \l 8D68
  \l 8D69
  \l 8D6A
  \l 8D6B
  \l 8D6C
  \l 8D6D
  \l 8D6E
  \l 8D6F
  \l 8D70
  \l 8D71
  \l 8D72
  \l 8D73
  \l 8D74
  \l 8D75
  \l 8D76
  \l 8D77
  \l 8D78
  \l 8D79
  \l 8D7A
  \l 8D7B
  \l 8D7C
  \l 8D7D
  \l 8D7E
  \l 8D7F
  \l 8D80
  \l 8D81
  \l 8D82
  \l 8D83
  \l 8D84
  \l 8D85
  \l 8D86
  \l 8D87
  \l 8D88
  \l 8D89
  \l 8D8A
  \l 8D8B
  \l 8D8C
  \l 8D8D
  \l 8D8E
  \l 8D8F
  \l 8D90
  \l 8D91
  \l 8D92
  \l 8D93
  \l 8D94
  \l 8D95
  \l 8D96
  \l 8D97
  \l 8D98
  \l 8D99
  \l 8D9A
  \l 8D9B
  \l 8D9C
  \l 8D9D
  \l 8D9E
  \l 8D9F
  \l 8DA0
  \l 8DA1
  \l 8DA2
  \l 8DA3
  \l 8DA4
  \l 8DA5
  \l 8DA6
  \l 8DA7
  \l 8DA8
  \l 8DA9
  \l 8DAA
  \l 8DAB
  \l 8DAC
  \l 8DAD
  \l 8DAE
  \l 8DAF
  \l 8DB0
  \l 8DB1
  \l 8DB2
  \l 8DB3
  \l 8DB4
  \l 8DB5
  \l 8DB6
  \l 8DB7
  \l 8DB8
  \l 8DB9
  \l 8DBA
  \l 8DBB
  \l 8DBC
  \l 8DBD
  \l 8DBE
  \l 8DBF
  \l 8DC0
  \l 8DC1
  \l 8DC2
  \l 8DC3
  \l 8DC4
  \l 8DC5
  \l 8DC6
  \l 8DC7
  \l 8DC8
  \l 8DC9
  \l 8DCA
  \l 8DCB
  \l 8DCC
  \l 8DCD
  \l 8DCE
  \l 8DCF
  \l 8DD0
  \l 8DD1
  \l 8DD2
  \l 8DD3
  \l 8DD4
  \l 8DD5
  \l 8DD6
  \l 8DD7
  \l 8DD8
  \l 8DD9
  \l 8DDA
  \l 8DDB
  \l 8DDC
  \l 8DDD
  \l 8DDE
  \l 8DDF
  \l 8DE0
  \l 8DE1
  \l 8DE2
  \l 8DE3
  \l 8DE4
  \l 8DE5
  \l 8DE6
  \l 8DE7
  \l 8DE8
  \l 8DE9
  \l 8DEA
  \l 8DEB
  \l 8DEC
  \l 8DED
  \l 8DEE
  \l 8DEF
  \l 8DF0
  \l 8DF1
  \l 8DF2
  \l 8DF3
  \l 8DF4
  \l 8DF5
  \l 8DF6
  \l 8DF7
  \l 8DF8
  \l 8DF9
  \l 8DFA
  \l 8DFB
  \l 8DFC
  \l 8DFD
  \l 8DFE
  \l 8DFF
  \l 8E00
  \l 8E01
  \l 8E02
  \l 8E03
  \l 8E04
  \l 8E05
  \l 8E06
  \l 8E07
  \l 8E08
  \l 8E09
  \l 8E0A
  \l 8E0B
  \l 8E0C
  \l 8E0D
  \l 8E0E
  \l 8E0F
  \l 8E10
  \l 8E11
  \l 8E12
  \l 8E13
  \l 8E14
  \l 8E15
  \l 8E16
  \l 8E17
  \l 8E18
  \l 8E19
  \l 8E1A
  \l 8E1B
  \l 8E1C
  \l 8E1D
  \l 8E1E
  \l 8E1F
  \l 8E20
  \l 8E21
  \l 8E22
  \l 8E23
  \l 8E24
  \l 8E25
  \l 8E26
  \l 8E27
  \l 8E28
  \l 8E29
  \l 8E2A
  \l 8E2B
  \l 8E2C
  \l 8E2D
  \l 8E2E
  \l 8E2F
  \l 8E30
  \l 8E31
  \l 8E32
  \l 8E33
  \l 8E34
  \l 8E35
  \l 8E36
  \l 8E37
  \l 8E38
  \l 8E39
  \l 8E3A
  \l 8E3B
  \l 8E3C
  \l 8E3D
  \l 8E3E
  \l 8E3F
  \l 8E40
  \l 8E41
  \l 8E42
  \l 8E43
  \l 8E44
  \l 8E45
  \l 8E46
  \l 8E47
  \l 8E48
  \l 8E49
  \l 8E4A
  \l 8E4B
  \l 8E4C
  \l 8E4D
  \l 8E4E
  \l 8E4F
  \l 8E50
  \l 8E51
  \l 8E52
  \l 8E53
  \l 8E54
  \l 8E55
  \l 8E56
  \l 8E57
  \l 8E58
  \l 8E59
  \l 8E5A
  \l 8E5B
  \l 8E5C
  \l 8E5D
  \l 8E5E
  \l 8E5F
  \l 8E60
  \l 8E61
  \l 8E62
  \l 8E63
  \l 8E64
  \l 8E65
  \l 8E66
  \l 8E67
  \l 8E68
  \l 8E69
  \l 8E6A
  \l 8E6B
  \l 8E6C
  \l 8E6D
  \l 8E6E
  \l 8E6F
  \l 8E70
  \l 8E71
  \l 8E72
  \l 8E73
  \l 8E74
  \l 8E75
  \l 8E76
  \l 8E77
  \l 8E78
  \l 8E79
  \l 8E7A
  \l 8E7B
  \l 8E7C
  \l 8E7D
  \l 8E7E
  \l 8E7F
  \l 8E80
  \l 8E81
  \l 8E82
  \l 8E83
  \l 8E84
  \l 8E85
  \l 8E86
  \l 8E87
  \l 8E88
  \l 8E89
  \l 8E8A
  \l 8E8B
  \l 8E8C
  \l 8E8D
  \l 8E8E
  \l 8E8F
  \l 8E90
  \l 8E91
  \l 8E92
  \l 8E93
  \l 8E94
  \l 8E95
  \l 8E96
  \l 8E97
  \l 8E98
  \l 8E99
  \l 8E9A
  \l 8E9B
  \l 8E9C
  \l 8E9D
  \l 8E9E
  \l 8E9F
  \l 8EA0
  \l 8EA1
  \l 8EA2
  \l 8EA3
  \l 8EA4
  \l 8EA5
  \l 8EA6
  \l 8EA7
  \l 8EA8
  \l 8EA9
  \l 8EAA
  \l 8EAB
  \l 8EAC
  \l 8EAD
  \l 8EAE
  \l 8EAF
  \l 8EB0
  \l 8EB1
  \l 8EB2
  \l 8EB3
  \l 8EB4
  \l 8EB5
  \l 8EB6
  \l 8EB7
  \l 8EB8
  \l 8EB9
  \l 8EBA
  \l 8EBB
  \l 8EBC
  \l 8EBD
  \l 8EBE
  \l 8EBF
  \l 8EC0
  \l 8EC1
  \l 8EC2
  \l 8EC3
  \l 8EC4
  \l 8EC5
  \l 8EC6
  \l 8EC7
  \l 8EC8
  \l 8EC9
  \l 8ECA
  \l 8ECB
  \l 8ECC
  \l 8ECD
  \l 8ECE
  \l 8ECF
  \l 8ED0
  \l 8ED1
  \l 8ED2
  \l 8ED3
  \l 8ED4
  \l 8ED5
  \l 8ED6
  \l 8ED7
  \l 8ED8
  \l 8ED9
  \l 8EDA
  \l 8EDB
  \l 8EDC
  \l 8EDD
  \l 8EDE
  \l 8EDF
  \l 8EE0
  \l 8EE1
  \l 8EE2
  \l 8EE3
  \l 8EE4
  \l 8EE5
  \l 8EE6
  \l 8EE7
  \l 8EE8
  \l 8EE9
  \l 8EEA
  \l 8EEB
  \l 8EEC
  \l 8EED
  \l 8EEE
  \l 8EEF
  \l 8EF0
  \l 8EF1
  \l 8EF2
  \l 8EF3
  \l 8EF4
  \l 8EF5
  \l 8EF6
  \l 8EF7
  \l 8EF8
  \l 8EF9
  \l 8EFA
  \l 8EFB
  \l 8EFC
  \l 8EFD
  \l 8EFE
  \l 8EFF
  \l 8F00
  \l 8F01
  \l 8F02
  \l 8F03
  \l 8F04
  \l 8F05
  \l 8F06
  \l 8F07
  \l 8F08
  \l 8F09
  \l 8F0A
  \l 8F0B
  \l 8F0C
  \l 8F0D
  \l 8F0E
  \l 8F0F
  \l 8F10
  \l 8F11
  \l 8F12
  \l 8F13
  \l 8F14
  \l 8F15
  \l 8F16
  \l 8F17
  \l 8F18
  \l 8F19
  \l 8F1A
  \l 8F1B
  \l 8F1C
  \l 8F1D
  \l 8F1E
  \l 8F1F
  \l 8F20
  \l 8F21
  \l 8F22
  \l 8F23
  \l 8F24
  \l 8F25
  \l 8F26
  \l 8F27
  \l 8F28
  \l 8F29
  \l 8F2A
  \l 8F2B
  \l 8F2C
  \l 8F2D
  \l 8F2E
  \l 8F2F
  \l 8F30
  \l 8F31
  \l 8F32
  \l 8F33
  \l 8F34
  \l 8F35
  \l 8F36
  \l 8F37
  \l 8F38
  \l 8F39
  \l 8F3A
  \l 8F3B
  \l 8F3C
  \l 8F3D
  \l 8F3E
  \l 8F3F
  \l 8F40
  \l 8F41
  \l 8F42
  \l 8F43
  \l 8F44
  \l 8F45
  \l 8F46
  \l 8F47
  \l 8F48
  \l 8F49
  \l 8F4A
  \l 8F4B
  \l 8F4C
  \l 8F4D
  \l 8F4E
  \l 8F4F
  \l 8F50
  \l 8F51
  \l 8F52
  \l 8F53
  \l 8F54
  \l 8F55
  \l 8F56
  \l 8F57
  \l 8F58
  \l 8F59
  \l 8F5A
  \l 8F5B
  \l 8F5C
  \l 8F5D
  \l 8F5E
  \l 8F5F
  \l 8F60
  \l 8F61
  \l 8F62
  \l 8F63
  \l 8F64
  \l 8F65
  \l 8F66
  \l 8F67
  \l 8F68
  \l 8F69
  \l 8F6A
  \l 8F6B
  \l 8F6C
  \l 8F6D
  \l 8F6E
  \l 8F6F
  \l 8F70
  \l 8F71
  \l 8F72
  \l 8F73
  \l 8F74
  \l 8F75
  \l 8F76
  \l 8F77
  \l 8F78
  \l 8F79
  \l 8F7A
  \l 8F7B
  \l 8F7C
  \l 8F7D
  \l 8F7E
  \l 8F7F
  \l 8F80
  \l 8F81
  \l 8F82
  \l 8F83
  \l 8F84
  \l 8F85
  \l 8F86
  \l 8F87
  \l 8F88
  \l 8F89
  \l 8F8A
  \l 8F8B
  \l 8F8C
  \l 8F8D
  \l 8F8E
  \l 8F8F
  \l 8F90
  \l 8F91
  \l 8F92
  \l 8F93
  \l 8F94
  \l 8F95
  \l 8F96
  \l 8F97
  \l 8F98
  \l 8F99
  \l 8F9A
  \l 8F9B
  \l 8F9C
  \l 8F9D
  \l 8F9E
  \l 8F9F
  \l 8FA0
  \l 8FA1
  \l 8FA2
  \l 8FA3
  \l 8FA4
  \l 8FA5
  \l 8FA6
  \l 8FA7
  \l 8FA8
  \l 8FA9
  \l 8FAA
  \l 8FAB
  \l 8FAC
  \l 8FAD
  \l 8FAE
  \l 8FAF
  \l 8FB0
  \l 8FB1
  \l 8FB2
  \l 8FB3
  \l 8FB4
  \l 8FB5
  \l 8FB6
  \l 8FB7
  \l 8FB8
  \l 8FB9
  \l 8FBA
  \l 8FBB
  \l 8FBC
  \l 8FBD
  \l 8FBE
  \l 8FBF
  \l 8FC0
  \l 8FC1
  \l 8FC2
  \l 8FC3
  \l 8FC4
  \l 8FC5
  \l 8FC6
  \l 8FC7
  \l 8FC8
  \l 8FC9
  \l 8FCA
  \l 8FCB
  \l 8FCC
  \l 8FCD
  \l 8FCE
  \l 8FCF
  \l 8FD0
  \l 8FD1
  \l 8FD2
  \l 8FD3
  \l 8FD4
  \l 8FD5
  \l 8FD6
  \l 8FD7
  \l 8FD8
  \l 8FD9
  \l 8FDA
  \l 8FDB
  \l 8FDC
  \l 8FDD
  \l 8FDE
  \l 8FDF
  \l 8FE0
  \l 8FE1
  \l 8FE2
  \l 8FE3
  \l 8FE4
  \l 8FE5
  \l 8FE6
  \l 8FE7
  \l 8FE8
  \l 8FE9
  \l 8FEA
  \l 8FEB
  \l 8FEC
  \l 8FED
  \l 8FEE
  \l 8FEF
  \l 8FF0
  \l 8FF1
  \l 8FF2
  \l 8FF3
  \l 8FF4
  \l 8FF5
  \l 8FF6
  \l 8FF7
  \l 8FF8
  \l 8FF9
  \l 8FFA
  \l 8FFB
  \l 8FFC
  \l 8FFD
  \l 8FFE
  \l 8FFF
  \l 9000
  \l 9001
  \l 9002
  \l 9003
  \l 9004
  \l 9005
  \l 9006
  \l 9007
  \l 9008
  \l 9009
  \l 900A
  \l 900B
  \l 900C
  \l 900D
  \l 900E
  \l 900F
  \l 9010
  \l 9011
  \l 9012
  \l 9013
  \l 9014
  \l 9015
  \l 9016
  \l 9017
  \l 9018
  \l 9019
  \l 901A
  \l 901B
  \l 901C
  \l 901D
  \l 901E
  \l 901F
  \l 9020
  \l 9021
  \l 9022
  \l 9023
  \l 9024
  \l 9025
  \l 9026
  \l 9027
  \l 9028
  \l 9029
  \l 902A
  \l 902B
  \l 902C
  \l 902D
  \l 902E
  \l 902F
  \l 9030
  \l 9031
  \l 9032
  \l 9033
  \l 9034
  \l 9035
  \l 9036
  \l 9037
  \l 9038
  \l 9039
  \l 903A
  \l 903B
  \l 903C
  \l 903D
  \l 903E
  \l 903F
  \l 9040
  \l 9041
  \l 9042
  \l 9043
  \l 9044
  \l 9045
  \l 9046
  \l 9047
  \l 9048
  \l 9049
  \l 904A
  \l 904B
  \l 904C
  \l 904D
  \l 904E
  \l 904F
  \l 9050
  \l 9051
  \l 9052
  \l 9053
  \l 9054
  \l 9055
  \l 9056
  \l 9057
  \l 9058
  \l 9059
  \l 905A
  \l 905B
  \l 905C
  \l 905D
  \l 905E
  \l 905F
  \l 9060
  \l 9061
  \l 9062
  \l 9063
  \l 9064
  \l 9065
  \l 9066
  \l 9067
  \l 9068
  \l 9069
  \l 906A
  \l 906B
  \l 906C
  \l 906D
  \l 906E
  \l 906F
  \l 9070
  \l 9071
  \l 9072
  \l 9073
  \l 9074
  \l 9075
  \l 9076
  \l 9077
  \l 9078
  \l 9079
  \l 907A
  \l 907B
  \l 907C
  \l 907D
  \l 907E
  \l 907F
  \l 9080
  \l 9081
  \l 9082
  \l 9083
  \l 9084
  \l 9085
  \l 9086
  \l 9087
  \l 9088
  \l 9089
  \l 908A
  \l 908B
  \l 908C
  \l 908D
  \l 908E
  \l 908F
  \l 9090
  \l 9091
  \l 9092
  \l 9093
  \l 9094
  \l 9095
  \l 9096
  \l 9097
  \l 9098
  \l 9099
  \l 909A
  \l 909B
  \l 909C
  \l 909D
  \l 909E
  \l 909F
  \l 90A0
  \l 90A1
  \l 90A2
  \l 90A3
  \l 90A4
  \l 90A5
  \l 90A6
  \l 90A7
  \l 90A8
  \l 90A9
  \l 90AA
  \l 90AB
  \l 90AC
  \l 90AD
  \l 90AE
  \l 90AF
  \l 90B0
  \l 90B1
  \l 90B2
  \l 90B3
  \l 90B4
  \l 90B5
  \l 90B6
  \l 90B7
  \l 90B8
  \l 90B9
  \l 90BA
  \l 90BB
  \l 90BC
  \l 90BD
  \l 90BE
  \l 90BF
  \l 90C0
  \l 90C1
  \l 90C2
  \l 90C3
  \l 90C4
  \l 90C5
  \l 90C6
  \l 90C7
  \l 90C8
  \l 90C9
  \l 90CA
  \l 90CB
  \l 90CC
  \l 90CD
  \l 90CE
  \l 90CF
  \l 90D0
  \l 90D1
  \l 90D2
  \l 90D3
  \l 90D4
  \l 90D5
  \l 90D6
  \l 90D7
  \l 90D8
  \l 90D9
  \l 90DA
  \l 90DB
  \l 90DC
  \l 90DD
  \l 90DE
  \l 90DF
  \l 90E0
  \l 90E1
  \l 90E2
  \l 90E3
  \l 90E4
  \l 90E5
  \l 90E6
  \l 90E7
  \l 90E8
  \l 90E9
  \l 90EA
  \l 90EB
  \l 90EC
  \l 90ED
  \l 90EE
  \l 90EF
  \l 90F0
  \l 90F1
  \l 90F2
  \l 90F3
  \l 90F4
  \l 90F5
  \l 90F6
  \l 90F7
  \l 90F8
  \l 90F9
  \l 90FA
  \l 90FB
  \l 90FC
  \l 90FD
  \l 90FE
  \l 90FF
  \l 9100
  \l 9101
  \l 9102
  \l 9103
  \l 9104
  \l 9105
  \l 9106
  \l 9107
  \l 9108
  \l 9109
  \l 910A
  \l 910B
  \l 910C
  \l 910D
  \l 910E
  \l 910F
  \l 9110
  \l 9111
  \l 9112
  \l 9113
  \l 9114
  \l 9115
  \l 9116
  \l 9117
  \l 9118
  \l 9119
  \l 911A
  \l 911B
  \l 911C
  \l 911D
  \l 911E
  \l 911F
  \l 9120
  \l 9121
  \l 9122
  \l 9123
  \l 9124
  \l 9125
  \l 9126
  \l 9127
  \l 9128
  \l 9129
  \l 912A
  \l 912B
  \l 912C
  \l 912D
  \l 912E
  \l 912F
  \l 9130
  \l 9131
  \l 9132
  \l 9133
  \l 9134
  \l 9135
  \l 9136
  \l 9137
  \l 9138
  \l 9139
  \l 913A
  \l 913B
  \l 913C
  \l 913D
  \l 913E
  \l 913F
  \l 9140
  \l 9141
  \l 9142
  \l 9143
  \l 9144
  \l 9145
  \l 9146
  \l 9147
  \l 9148
  \l 9149
  \l 914A
  \l 914B
  \l 914C
  \l 914D
  \l 914E
  \l 914F
  \l 9150
  \l 9151
  \l 9152
  \l 9153
  \l 9154
  \l 9155
  \l 9156
  \l 9157
  \l 9158
  \l 9159
  \l 915A
  \l 915B
  \l 915C
  \l 915D
  \l 915E
  \l 915F
  \l 9160
  \l 9161
  \l 9162
  \l 9163
  \l 9164
  \l 9165
  \l 9166
  \l 9167
  \l 9168
  \l 9169
  \l 916A
  \l 916B
  \l 916C
  \l 916D
  \l 916E
  \l 916F
  \l 9170
  \l 9171
  \l 9172
  \l 9173
  \l 9174
  \l 9175
  \l 9176
  \l 9177
  \l 9178
  \l 9179
  \l 917A
  \l 917B
  \l 917C
  \l 917D
  \l 917E
  \l 917F
  \l 9180
  \l 9181
  \l 9182
  \l 9183
  \l 9184
  \l 9185
  \l 9186
  \l 9187
  \l 9188
  \l 9189
  \l 918A
  \l 918B
  \l 918C
  \l 918D
  \l 918E
  \l 918F
  \l 9190
  \l 9191
  \l 9192
  \l 9193
  \l 9194
  \l 9195
  \l 9196
  \l 9197
  \l 9198
  \l 9199
  \l 919A
  \l 919B
  \l 919C
  \l 919D
  \l 919E
  \l 919F
  \l 91A0
  \l 91A1
  \l 91A2
  \l 91A3
  \l 91A4
  \l 91A5
  \l 91A6
  \l 91A7
  \l 91A8
  \l 91A9
  \l 91AA
  \l 91AB
  \l 91AC
  \l 91AD
  \l 91AE
  \l 91AF
  \l 91B0
  \l 91B1
  \l 91B2
  \l 91B3
  \l 91B4
  \l 91B5
  \l 91B6
  \l 91B7
  \l 91B8
  \l 91B9
  \l 91BA
  \l 91BB
  \l 91BC
  \l 91BD
  \l 91BE
  \l 91BF
  \l 91C0
  \l 91C1
  \l 91C2
  \l 91C3
  \l 91C4
  \l 91C5
  \l 91C6
  \l 91C7
  \l 91C8
  \l 91C9
  \l 91CA
  \l 91CB
  \l 91CC
  \l 91CD
  \l 91CE
  \l 91CF
  \l 91D0
  \l 91D1
  \l 91D2
  \l 91D3
  \l 91D4
  \l 91D5
  \l 91D6
  \l 91D7
  \l 91D8
  \l 91D9
  \l 91DA
  \l 91DB
  \l 91DC
  \l 91DD
  \l 91DE
  \l 91DF
  \l 91E0
  \l 91E1
  \l 91E2
  \l 91E3
  \l 91E4
  \l 91E5
  \l 91E6
  \l 91E7
  \l 91E8
  \l 91E9
  \l 91EA
  \l 91EB
  \l 91EC
  \l 91ED
  \l 91EE
  \l 91EF
  \l 91F0
  \l 91F1
  \l 91F2
  \l 91F3
  \l 91F4
  \l 91F5
  \l 91F6
  \l 91F7
  \l 91F8
  \l 91F9
  \l 91FA
  \l 91FB
  \l 91FC
  \l 91FD
  \l 91FE
  \l 91FF
  \l 9200
  \l 9201
  \l 9202
  \l 9203
  \l 9204
  \l 9205
  \l 9206
  \l 9207
  \l 9208
  \l 9209
  \l 920A
  \l 920B
  \l 920C
  \l 920D
  \l 920E
  \l 920F
  \l 9210
  \l 9211
  \l 9212
  \l 9213
  \l 9214
  \l 9215
  \l 9216
  \l 9217
  \l 9218
  \l 9219
  \l 921A
  \l 921B
  \l 921C
  \l 921D
  \l 921E
  \l 921F
  \l 9220
  \l 9221
  \l 9222
  \l 9223
  \l 9224
  \l 9225
  \l 9226
  \l 9227
  \l 9228
  \l 9229
  \l 922A
  \l 922B
  \l 922C
  \l 922D
  \l 922E
  \l 922F
  \l 9230
  \l 9231
  \l 9232
  \l 9233
  \l 9234
  \l 9235
  \l 9236
  \l 9237
  \l 9238
  \l 9239
  \l 923A
  \l 923B
  \l 923C
  \l 923D
  \l 923E
  \l 923F
  \l 9240
  \l 9241
  \l 9242
  \l 9243
  \l 9244
  \l 9245
  \l 9246
  \l 9247
  \l 9248
  \l 9249
  \l 924A
  \l 924B
  \l 924C
  \l 924D
  \l 924E
  \l 924F
  \l 9250
  \l 9251
  \l 9252
  \l 9253
  \l 9254
  \l 9255
  \l 9256
  \l 9257
  \l 9258
  \l 9259
  \l 925A
  \l 925B
  \l 925C
  \l 925D
  \l 925E
  \l 925F
  \l 9260
  \l 9261
  \l 9262
  \l 9263
  \l 9264
  \l 9265
  \l 9266
  \l 9267
  \l 9268
  \l 9269
  \l 926A
  \l 926B
  \l 926C
  \l 926D
  \l 926E
  \l 926F
  \l 9270
  \l 9271
  \l 9272
  \l 9273
  \l 9274
  \l 9275
  \l 9276
  \l 9277
  \l 9278
  \l 9279
  \l 927A
  \l 927B
  \l 927C
  \l 927D
  \l 927E
  \l 927F
  \l 9280
  \l 9281
  \l 9282
  \l 9283
  \l 9284
  \l 9285
  \l 9286
  \l 9287
  \l 9288
  \l 9289
  \l 928A
  \l 928B
  \l 928C
  \l 928D
  \l 928E
  \l 928F
  \l 9290
  \l 9291
  \l 9292
  \l 9293
  \l 9294
  \l 9295
  \l 9296
  \l 9297
  \l 9298
  \l 9299
  \l 929A
  \l 929B
  \l 929C
  \l 929D
  \l 929E
  \l 929F
  \l 92A0
  \l 92A1
  \l 92A2
  \l 92A3
  \l 92A4
  \l 92A5
  \l 92A6
  \l 92A7
  \l 92A8
  \l 92A9
  \l 92AA
  \l 92AB
  \l 92AC
  \l 92AD
  \l 92AE
  \l 92AF
  \l 92B0
  \l 92B1
  \l 92B2
  \l 92B3
  \l 92B4
  \l 92B5
  \l 92B6
  \l 92B7
  \l 92B8
  \l 92B9
  \l 92BA
  \l 92BB
  \l 92BC
  \l 92BD
  \l 92BE
  \l 92BF
  \l 92C0
  \l 92C1
  \l 92C2
  \l 92C3
  \l 92C4
  \l 92C5
  \l 92C6
  \l 92C7
  \l 92C8
  \l 92C9
  \l 92CA
  \l 92CB
  \l 92CC
  \l 92CD
  \l 92CE
  \l 92CF
  \l 92D0
  \l 92D1
  \l 92D2
  \l 92D3
  \l 92D4
  \l 92D5
  \l 92D6
  \l 92D7
  \l 92D8
  \l 92D9
  \l 92DA
  \l 92DB
  \l 92DC
  \l 92DD
  \l 92DE
  \l 92DF
  \l 92E0
  \l 92E1
  \l 92E2
  \l 92E3
  \l 92E4
  \l 92E5
  \l 92E6
  \l 92E7
  \l 92E8
  \l 92E9
  \l 92EA
  \l 92EB
  \l 92EC
  \l 92ED
  \l 92EE
  \l 92EF
  \l 92F0
  \l 92F1
  \l 92F2
  \l 92F3
  \l 92F4
  \l 92F5
  \l 92F6
  \l 92F7
  \l 92F8
  \l 92F9
  \l 92FA
  \l 92FB
  \l 92FC
  \l 92FD
  \l 92FE
  \l 92FF
  \l 9300
  \l 9301
  \l 9302
  \l 9303
  \l 9304
  \l 9305
  \l 9306
  \l 9307
  \l 9308
  \l 9309
  \l 930A
  \l 930B
  \l 930C
  \l 930D
  \l 930E
  \l 930F
  \l 9310
  \l 9311
  \l 9312
  \l 9313
  \l 9314
  \l 9315
  \l 9316
  \l 9317
  \l 9318
  \l 9319
  \l 931A
  \l 931B
  \l 931C
  \l 931D
  \l 931E
  \l 931F
  \l 9320
  \l 9321
  \l 9322
  \l 9323
  \l 9324
  \l 9325
  \l 9326
  \l 9327
  \l 9328
  \l 9329
  \l 932A
  \l 932B
  \l 932C
  \l 932D
  \l 932E
  \l 932F
  \l 9330
  \l 9331
  \l 9332
  \l 9333
  \l 9334
  \l 9335
  \l 9336
  \l 9337
  \l 9338
  \l 9339
  \l 933A
  \l 933B
  \l 933C
  \l 933D
  \l 933E
  \l 933F
  \l 9340
  \l 9341
  \l 9342
  \l 9343
  \l 9344
  \l 9345
  \l 9346
  \l 9347
  \l 9348
  \l 9349
  \l 934A
  \l 934B
  \l 934C
  \l 934D
  \l 934E
  \l 934F
  \l 9350
  \l 9351
  \l 9352
  \l 9353
  \l 9354
  \l 9355
  \l 9356
  \l 9357
  \l 9358
  \l 9359
  \l 935A
  \l 935B
  \l 935C
  \l 935D
  \l 935E
  \l 935F
  \l 9360
  \l 9361
  \l 9362
  \l 9363
  \l 9364
  \l 9365
  \l 9366
  \l 9367
  \l 9368
  \l 9369
  \l 936A
  \l 936B
  \l 936C
  \l 936D
  \l 936E
  \l 936F
  \l 9370
  \l 9371
  \l 9372
  \l 9373
  \l 9374
  \l 9375
  \l 9376
  \l 9377
  \l 9378
  \l 9379
  \l 937A
  \l 937B
  \l 937C
  \l 937D
  \l 937E
  \l 937F
  \l 9380
  \l 9381
  \l 9382
  \l 9383
  \l 9384
  \l 9385
  \l 9386
  \l 9387
  \l 9388
  \l 9389
  \l 938A
  \l 938B
  \l 938C
  \l 938D
  \l 938E
  \l 938F
  \l 9390
  \l 9391
  \l 9392
  \l 9393
  \l 9394
  \l 9395
  \l 9396
  \l 9397
  \l 9398
  \l 9399
  \l 939A
  \l 939B
  \l 939C
  \l 939D
  \l 939E
  \l 939F
  \l 93A0
  \l 93A1
  \l 93A2
  \l 93A3
  \l 93A4
  \l 93A5
  \l 93A6
  \l 93A7
  \l 93A8
  \l 93A9
  \l 93AA
  \l 93AB
  \l 93AC
  \l 93AD
  \l 93AE
  \l 93AF
  \l 93B0
  \l 93B1
  \l 93B2
  \l 93B3
  \l 93B4
  \l 93B5
  \l 93B6
  \l 93B7
  \l 93B8
  \l 93B9
  \l 93BA
  \l 93BB
  \l 93BC
  \l 93BD
  \l 93BE
  \l 93BF
  \l 93C0
  \l 93C1
  \l 93C2
  \l 93C3
  \l 93C4
  \l 93C5
  \l 93C6
  \l 93C7
  \l 93C8
  \l 93C9
  \l 93CA
  \l 93CB
  \l 93CC
  \l 93CD
  \l 93CE
  \l 93CF
  \l 93D0
  \l 93D1
  \l 93D2
  \l 93D3
  \l 93D4
  \l 93D5
  \l 93D6
  \l 93D7
  \l 93D8
  \l 93D9
  \l 93DA
  \l 93DB
  \l 93DC
  \l 93DD
  \l 93DE
  \l 93DF
  \l 93E0
  \l 93E1
  \l 93E2
  \l 93E3
  \l 93E4
  \l 93E5
  \l 93E6
  \l 93E7
  \l 93E8
  \l 93E9
  \l 93EA
  \l 93EB
  \l 93EC
  \l 93ED
  \l 93EE
  \l 93EF
  \l 93F0
  \l 93F1
  \l 93F2
  \l 93F3
  \l 93F4
  \l 93F5
  \l 93F6
  \l 93F7
  \l 93F8
  \l 93F9
  \l 93FA
  \l 93FB
  \l 93FC
  \l 93FD
  \l 93FE
  \l 93FF
  \l 9400
  \l 9401
  \l 9402
  \l 9403
  \l 9404
  \l 9405
  \l 9406
  \l 9407
  \l 9408
  \l 9409
  \l 940A
  \l 940B
  \l 940C
  \l 940D
  \l 940E
  \l 940F
  \l 9410
  \l 9411
  \l 9412
  \l 9413
  \l 9414
  \l 9415
  \l 9416
  \l 9417
  \l 9418
  \l 9419
  \l 941A
  \l 941B
  \l 941C
  \l 941D
  \l 941E
  \l 941F
  \l 9420
  \l 9421
  \l 9422
  \l 9423
  \l 9424
  \l 9425
  \l 9426
  \l 9427
  \l 9428
  \l 9429
  \l 942A
  \l 942B
  \l 942C
  \l 942D
  \l 942E
  \l 942F
  \l 9430
  \l 9431
  \l 9432
  \l 9433
  \l 9434
  \l 9435
  \l 9436
  \l 9437
  \l 9438
  \l 9439
  \l 943A
  \l 943B
  \l 943C
  \l 943D
  \l 943E
  \l 943F
  \l 9440
  \l 9441
  \l 9442
  \l 9443
  \l 9444
  \l 9445
  \l 9446
  \l 9447
  \l 9448
  \l 9449
  \l 944A
  \l 944B
  \l 944C
  \l 944D
  \l 944E
  \l 944F
  \l 9450
  \l 9451
  \l 9452
  \l 9453
  \l 9454
  \l 9455
  \l 9456
  \l 9457
  \l 9458
  \l 9459
  \l 945A
  \l 945B
  \l 945C
  \l 945D
  \l 945E
  \l 945F
  \l 9460
  \l 9461
  \l 9462
  \l 9463
  \l 9464
  \l 9465
  \l 9466
  \l 9467
  \l 9468
  \l 9469
  \l 946A
  \l 946B
  \l 946C
  \l 946D
  \l 946E
  \l 946F
  \l 9470
  \l 9471
  \l 9472
  \l 9473
  \l 9474
  \l 9475
  \l 9476
  \l 9477
  \l 9478
  \l 9479
  \l 947A
  \l 947B
  \l 947C
  \l 947D
  \l 947E
  \l 947F
  \l 9480
  \l 9481
  \l 9482
  \l 9483
  \l 9484
  \l 9485
  \l 9486
  \l 9487
  \l 9488
  \l 9489
  \l 948A
  \l 948B
  \l 948C
  \l 948D
  \l 948E
  \l 948F
  \l 9490
  \l 9491
  \l 9492
  \l 9493
  \l 9494
  \l 9495
  \l 9496
  \l 9497
  \l 9498
  \l 9499
  \l 949A
  \l 949B
  \l 949C
  \l 949D
  \l 949E
  \l 949F
  \l 94A0
  \l 94A1
  \l 94A2
  \l 94A3
  \l 94A4
  \l 94A5
  \l 94A6
  \l 94A7
  \l 94A8
  \l 94A9
  \l 94AA
  \l 94AB
  \l 94AC
  \l 94AD
  \l 94AE
  \l 94AF
  \l 94B0
  \l 94B1
  \l 94B2
  \l 94B3
  \l 94B4
  \l 94B5
  \l 94B6
  \l 94B7
  \l 94B8
  \l 94B9
  \l 94BA
  \l 94BB
  \l 94BC
  \l 94BD
  \l 94BE
  \l 94BF
  \l 94C0
  \l 94C1
  \l 94C2
  \l 94C3
  \l 94C4
  \l 94C5
  \l 94C6
  \l 94C7
  \l 94C8
  \l 94C9
  \l 94CA
  \l 94CB
  \l 94CC
  \l 94CD
  \l 94CE
  \l 94CF
  \l 94D0
  \l 94D1
  \l 94D2
  \l 94D3
  \l 94D4
  \l 94D5
  \l 94D6
  \l 94D7
  \l 94D8
  \l 94D9
  \l 94DA
  \l 94DB
  \l 94DC
  \l 94DD
  \l 94DE
  \l 94DF
  \l 94E0
  \l 94E1
  \l 94E2
  \l 94E3
  \l 94E4
  \l 94E5
  \l 94E6
  \l 94E7
  \l 94E8
  \l 94E9
  \l 94EA
  \l 94EB
  \l 94EC
  \l 94ED
  \l 94EE
  \l 94EF
  \l 94F0
  \l 94F1
  \l 94F2
  \l 94F3
  \l 94F4
  \l 94F5
  \l 94F6
  \l 94F7
  \l 94F8
  \l 94F9
  \l 94FA
  \l 94FB
  \l 94FC
  \l 94FD
  \l 94FE
  \l 94FF
  \l 9500
  \l 9501
  \l 9502
  \l 9503
  \l 9504
  \l 9505
  \l 9506
  \l 9507
  \l 9508
  \l 9509
  \l 950A
  \l 950B
  \l 950C
  \l 950D
  \l 950E
  \l 950F
  \l 9510
  \l 9511
  \l 9512
  \l 9513
  \l 9514
  \l 9515
  \l 9516
  \l 9517
  \l 9518
  \l 9519
  \l 951A
  \l 951B
  \l 951C
  \l 951D
  \l 951E
  \l 951F
  \l 9520
  \l 9521
  \l 9522
  \l 9523
  \l 9524
  \l 9525
  \l 9526
  \l 9527
  \l 9528
  \l 9529
  \l 952A
  \l 952B
  \l 952C
  \l 952D
  \l 952E
  \l 952F
  \l 9530
  \l 9531
  \l 9532
  \l 9533
  \l 9534
  \l 9535
  \l 9536
  \l 9537
  \l 9538
  \l 9539
  \l 953A
  \l 953B
  \l 953C
  \l 953D
  \l 953E
  \l 953F
  \l 9540
  \l 9541
  \l 9542
  \l 9543
  \l 9544
  \l 9545
  \l 9546
  \l 9547
  \l 9548
  \l 9549
  \l 954A
  \l 954B
  \l 954C
  \l 954D
  \l 954E
  \l 954F
  \l 9550
  \l 9551
  \l 9552
  \l 9553
  \l 9554
  \l 9555
  \l 9556
  \l 9557
  \l 9558
  \l 9559
  \l 955A
  \l 955B
  \l 955C
  \l 955D
  \l 955E
  \l 955F
  \l 9560
  \l 9561
  \l 9562
  \l 9563
  \l 9564
  \l 9565
  \l 9566
  \l 9567
  \l 9568
  \l 9569
  \l 956A
  \l 956B
  \l 956C
  \l 956D
  \l 956E
  \l 956F
  \l 9570
  \l 9571
  \l 9572
  \l 9573
  \l 9574
  \l 9575
  \l 9576
  \l 9577
  \l 9578
  \l 9579
  \l 957A
  \l 957B
  \l 957C
  \l 957D
  \l 957E
  \l 957F
  \l 9580
  \l 9581
  \l 9582
  \l 9583
  \l 9584
  \l 9585
  \l 9586
  \l 9587
  \l 9588
  \l 9589
  \l 958A
  \l 958B
  \l 958C
  \l 958D
  \l 958E
  \l 958F
  \l 9590
  \l 9591
  \l 9592
  \l 9593
  \l 9594
  \l 9595
  \l 9596
  \l 9597
  \l 9598
  \l 9599
  \l 959A
  \l 959B
  \l 959C
  \l 959D
  \l 959E
  \l 959F
  \l 95A0
  \l 95A1
  \l 95A2
  \l 95A3
  \l 95A4
  \l 95A5
  \l 95A6
  \l 95A7
  \l 95A8
  \l 95A9
  \l 95AA
  \l 95AB
  \l 95AC
  \l 95AD
  \l 95AE
  \l 95AF
  \l 95B0
  \l 95B1
  \l 95B2
  \l 95B3
  \l 95B4
  \l 95B5
  \l 95B6
  \l 95B7
  \l 95B8
  \l 95B9
  \l 95BA
  \l 95BB
  \l 95BC
  \l 95BD
  \l 95BE
  \l 95BF
  \l 95C0
  \l 95C1
  \l 95C2
  \l 95C3
  \l 95C4
  \l 95C5
  \l 95C6
  \l 95C7
  \l 95C8
  \l 95C9
  \l 95CA
  \l 95CB
  \l 95CC
  \l 95CD
  \l 95CE
  \l 95CF
  \l 95D0
  \l 95D1
  \l 95D2
  \l 95D3
  \l 95D4
  \l 95D5
  \l 95D6
  \l 95D7
  \l 95D8
  \l 95D9
  \l 95DA
  \l 95DB
  \l 95DC
  \l 95DD
  \l 95DE
  \l 95DF
  \l 95E0
  \l 95E1
  \l 95E2
  \l 95E3
  \l 95E4
  \l 95E5
  \l 95E6
  \l 95E7
  \l 95E8
  \l 95E9
  \l 95EA
  \l 95EB
  \l 95EC
  \l 95ED
  \l 95EE
  \l 95EF
  \l 95F0
  \l 95F1
  \l 95F2
  \l 95F3
  \l 95F4
  \l 95F5
  \l 95F6
  \l 95F7
  \l 95F8
  \l 95F9
  \l 95FA
  \l 95FB
  \l 95FC
  \l 95FD
  \l 95FE
  \l 95FF
  \l 9600
  \l 9601
  \l 9602
  \l 9603
  \l 9604
  \l 9605
  \l 9606
  \l 9607
  \l 9608
  \l 9609
  \l 960A
  \l 960B
  \l 960C
  \l 960D
  \l 960E
  \l 960F
  \l 9610
  \l 9611
  \l 9612
  \l 9613
  \l 9614
  \l 9615
  \l 9616
  \l 9617
  \l 9618
  \l 9619
  \l 961A
  \l 961B
  \l 961C
  \l 961D
  \l 961E
  \l 961F
  \l 9620
  \l 9621
  \l 9622
  \l 9623
  \l 9624
  \l 9625
  \l 9626
  \l 9627
  \l 9628
  \l 9629
  \l 962A
  \l 962B
  \l 962C
  \l 962D
  \l 962E
  \l 962F
  \l 9630
  \l 9631
  \l 9632
  \l 9633
  \l 9634
  \l 9635
  \l 9636
  \l 9637
  \l 9638
  \l 9639
  \l 963A
  \l 963B
  \l 963C
  \l 963D
  \l 963E
  \l 963F
  \l 9640
  \l 9641
  \l 9642
  \l 9643
  \l 9644
  \l 9645
  \l 9646
  \l 9647
  \l 9648
  \l 9649
  \l 964A
  \l 964B
  \l 964C
  \l 964D
  \l 964E
  \l 964F
  \l 9650
  \l 9651
  \l 9652
  \l 9653
  \l 9654
  \l 9655
  \l 9656
  \l 9657
  \l 9658
  \l 9659
  \l 965A
  \l 965B
  \l 965C
  \l 965D
  \l 965E
  \l 965F
  \l 9660
  \l 9661
  \l 9662
  \l 9663
  \l 9664
  \l 9665
  \l 9666
  \l 9667
  \l 9668
  \l 9669
  \l 966A
  \l 966B
  \l 966C
  \l 966D
  \l 966E
  \l 966F
  \l 9670
  \l 9671
  \l 9672
  \l 9673
  \l 9674
  \l 9675
  \l 9676
  \l 9677
  \l 9678
  \l 9679
  \l 967A
  \l 967B
  \l 967C
  \l 967D
  \l 967E
  \l 967F
  \l 9680
  \l 9681
  \l 9682
  \l 9683
  \l 9684
  \l 9685
  \l 9686
  \l 9687
  \l 9688
  \l 9689
  \l 968A
  \l 968B
  \l 968C
  \l 968D
  \l 968E
  \l 968F
  \l 9690
  \l 9691
  \l 9692
  \l 9693
  \l 9694
  \l 9695
  \l 9696
  \l 9697
  \l 9698
  \l 9699
  \l 969A
  \l 969B
  \l 969C
  \l 969D
  \l 969E
  \l 969F
  \l 96A0
  \l 96A1
  \l 96A2
  \l 96A3
  \l 96A4
  \l 96A5
  \l 96A6
  \l 96A7
  \l 96A8
  \l 96A9
  \l 96AA
  \l 96AB
  \l 96AC
  \l 96AD
  \l 96AE
  \l 96AF
  \l 96B0
  \l 96B1
  \l 96B2
  \l 96B3
  \l 96B4
  \l 96B5
  \l 96B6
  \l 96B7
  \l 96B8
  \l 96B9
  \l 96BA
  \l 96BB
  \l 96BC
  \l 96BD
  \l 96BE
  \l 96BF
  \l 96C0
  \l 96C1
  \l 96C2
  \l 96C3
  \l 96C4
  \l 96C5
  \l 96C6
  \l 96C7
  \l 96C8
  \l 96C9
  \l 96CA
  \l 96CB
  \l 96CC
  \l 96CD
  \l 96CE
  \l 96CF
  \l 96D0
  \l 96D1
  \l 96D2
  \l 96D3
  \l 96D4
  \l 96D5
  \l 96D6
  \l 96D7
  \l 96D8
  \l 96D9
  \l 96DA
  \l 96DB
  \l 96DC
  \l 96DD
  \l 96DE
  \l 96DF
  \l 96E0
  \l 96E1
  \l 96E2
  \l 96E3
  \l 96E4
  \l 96E5
  \l 96E6
  \l 96E7
  \l 96E8
  \l 96E9
  \l 96EA
  \l 96EB
  \l 96EC
  \l 96ED
  \l 96EE
  \l 96EF
  \l 96F0
  \l 96F1
  \l 96F2
  \l 96F3
  \l 96F4
  \l 96F5
  \l 96F6
  \l 96F7
  \l 96F8
  \l 96F9
  \l 96FA
  \l 96FB
  \l 96FC
  \l 96FD
  \l 96FE
  \l 96FF
  \l 9700
  \l 9701
  \l 9702
  \l 9703
  \l 9704
  \l 9705
  \l 9706
  \l 9707
  \l 9708
  \l 9709
  \l 970A
  \l 970B
  \l 970C
  \l 970D
  \l 970E
  \l 970F
  \l 9710
  \l 9711
  \l 9712
  \l 9713
  \l 9714
  \l 9715
  \l 9716
  \l 9717
  \l 9718
  \l 9719
  \l 971A
  \l 971B
  \l 971C
  \l 971D
  \l 971E
  \l 971F
  \l 9720
  \l 9721
  \l 9722
  \l 9723
  \l 9724
  \l 9725
  \l 9726
  \l 9727
  \l 9728
  \l 9729
  \l 972A
  \l 972B
  \l 972C
  \l 972D
  \l 972E
  \l 972F
  \l 9730
  \l 9731
  \l 9732
  \l 9733
  \l 9734
  \l 9735
  \l 9736
  \l 9737
  \l 9738
  \l 9739
  \l 973A
  \l 973B
  \l 973C
  \l 973D
  \l 973E
  \l 973F
  \l 9740
  \l 9741
  \l 9742
  \l 9743
  \l 9744
  \l 9745
  \l 9746
  \l 9747
  \l 9748
  \l 9749
  \l 974A
  \l 974B
  \l 974C
  \l 974D
  \l 974E
  \l 974F
  \l 9750
  \l 9751
  \l 9752
  \l 9753
  \l 9754
  \l 9755
  \l 9756
  \l 9757
  \l 9758
  \l 9759
  \l 975A
  \l 975B
  \l 975C
  \l 975D
  \l 975E
  \l 975F
  \l 9760
  \l 9761
  \l 9762
  \l 9763
  \l 9764
  \l 9765
  \l 9766
  \l 9767
  \l 9768
  \l 9769
  \l 976A
  \l 976B
  \l 976C
  \l 976D
  \l 976E
  \l 976F
  \l 9770
  \l 9771
  \l 9772
  \l 9773
  \l 9774
  \l 9775
  \l 9776
  \l 9777
  \l 9778
  \l 9779
  \l 977A
  \l 977B
  \l 977C
  \l 977D
  \l 977E
  \l 977F
  \l 9780
  \l 9781
  \l 9782
  \l 9783
  \l 9784
  \l 9785
  \l 9786
  \l 9787
  \l 9788
  \l 9789
  \l 978A
  \l 978B
  \l 978C
  \l 978D
  \l 978E
  \l 978F
  \l 9790
  \l 9791
  \l 9792
  \l 9793
  \l 9794
  \l 9795
  \l 9796
  \l 9797
  \l 9798
  \l 9799
  \l 979A
  \l 979B
  \l 979C
  \l 979D
  \l 979E
  \l 979F
  \l 97A0
  \l 97A1
  \l 97A2
  \l 97A3
  \l 97A4
  \l 97A5
  \l 97A6
  \l 97A7
  \l 97A8
  \l 97A9
  \l 97AA
  \l 97AB
  \l 97AC
  \l 97AD
  \l 97AE
  \l 97AF
  \l 97B0
  \l 97B1
  \l 97B2
  \l 97B3
  \l 97B4
  \l 97B5
  \l 97B6
  \l 97B7
  \l 97B8
  \l 97B9
  \l 97BA
  \l 97BB
  \l 97BC
  \l 97BD
  \l 97BE
  \l 97BF
  \l 97C0
  \l 97C1
  \l 97C2
  \l 97C3
  \l 97C4
  \l 97C5
  \l 97C6
  \l 97C7
  \l 97C8
  \l 97C9
  \l 97CA
  \l 97CB
  \l 97CC
  \l 97CD
  \l 97CE
  \l 97CF
  \l 97D0
  \l 97D1
  \l 97D2
  \l 97D3
  \l 97D4
  \l 97D5
  \l 97D6
  \l 97D7
  \l 97D8
  \l 97D9
  \l 97DA
  \l 97DB
  \l 97DC
  \l 97DD
  \l 97DE
  \l 97DF
  \l 97E0
  \l 97E1
  \l 97E2
  \l 97E3
  \l 97E4
  \l 97E5
  \l 97E6
  \l 97E7
  \l 97E8
  \l 97E9
  \l 97EA
  \l 97EB
  \l 97EC
  \l 97ED
  \l 97EE
  \l 97EF
  \l 97F0
  \l 97F1
  \l 97F2
  \l 97F3
  \l 97F4
  \l 97F5
  \l 97F6
  \l 97F7
  \l 97F8
  \l 97F9
  \l 97FA
  \l 97FB
  \l 97FC
  \l 97FD
  \l 97FE
  \l 97FF
  \l 9800
  \l 9801
  \l 9802
  \l 9803
  \l 9804
  \l 9805
  \l 9806
  \l 9807
  \l 9808
  \l 9809
  \l 980A
  \l 980B
  \l 980C
  \l 980D
  \l 980E
  \l 980F
  \l 9810
  \l 9811
  \l 9812
  \l 9813
  \l 9814
  \l 9815
  \l 9816
  \l 9817
  \l 9818
  \l 9819
  \l 981A
  \l 981B
  \l 981C
  \l 981D
  \l 981E
  \l 981F
  \l 9820
  \l 9821
  \l 9822
  \l 9823
  \l 9824
  \l 9825
  \l 9826
  \l 9827
  \l 9828
  \l 9829
  \l 982A
  \l 982B
  \l 982C
  \l 982D
  \l 982E
  \l 982F
  \l 9830
  \l 9831
  \l 9832
  \l 9833
  \l 9834
  \l 9835
  \l 9836
  \l 9837
  \l 9838
  \l 9839
  \l 983A
  \l 983B
  \l 983C
  \l 983D
  \l 983E
  \l 983F
  \l 9840
  \l 9841
  \l 9842
  \l 9843
  \l 9844
  \l 9845
  \l 9846
  \l 9847
  \l 9848
  \l 9849
  \l 984A
  \l 984B
  \l 984C
  \l 984D
  \l 984E
  \l 984F
  \l 9850
  \l 9851
  \l 9852
  \l 9853
  \l 9854
  \l 9855
  \l 9856
  \l 9857
  \l 9858
  \l 9859
  \l 985A
  \l 985B
  \l 985C
  \l 985D
  \l 985E
  \l 985F
  \l 9860
  \l 9861
  \l 9862
  \l 9863
  \l 9864
  \l 9865
  \l 9866
  \l 9867
  \l 9868
  \l 9869
  \l 986A
  \l 986B
  \l 986C
  \l 986D
  \l 986E
  \l 986F
  \l 9870
  \l 9871
  \l 9872
  \l 9873
  \l 9874
  \l 9875
  \l 9876
  \l 9877
  \l 9878
  \l 9879
  \l 987A
  \l 987B
  \l 987C
  \l 987D
  \l 987E
  \l 987F
  \l 9880
  \l 9881
  \l 9882
  \l 9883
  \l 9884
  \l 9885
  \l 9886
  \l 9887
  \l 9888
  \l 9889
  \l 988A
  \l 988B
  \l 988C
  \l 988D
  \l 988E
  \l 988F
  \l 9890
  \l 9891
  \l 9892
  \l 9893
  \l 9894
  \l 9895
  \l 9896
  \l 9897
  \l 9898
  \l 9899
  \l 989A
  \l 989B
  \l 989C
  \l 989D
  \l 989E
  \l 989F
  \l 98A0
  \l 98A1
  \l 98A2
  \l 98A3
  \l 98A4
  \l 98A5
  \l 98A6
  \l 98A7
  \l 98A8
  \l 98A9
  \l 98AA
  \l 98AB
  \l 98AC
  \l 98AD
  \l 98AE
  \l 98AF
  \l 98B0
  \l 98B1
  \l 98B2
  \l 98B3
  \l 98B4
  \l 98B5
  \l 98B6
  \l 98B7
  \l 98B8
  \l 98B9
  \l 98BA
  \l 98BB
  \l 98BC
  \l 98BD
  \l 98BE
  \l 98BF
  \l 98C0
  \l 98C1
  \l 98C2
  \l 98C3
  \l 98C4
  \l 98C5
  \l 98C6
  \l 98C7
  \l 98C8
  \l 98C9
  \l 98CA
  \l 98CB
  \l 98CC
  \l 98CD
  \l 98CE
  \l 98CF
  \l 98D0
  \l 98D1
  \l 98D2
  \l 98D3
  \l 98D4
  \l 98D5
  \l 98D6
  \l 98D7
  \l 98D8
  \l 98D9
  \l 98DA
  \l 98DB
  \l 98DC
  \l 98DD
  \l 98DE
  \l 98DF
  \l 98E0
  \l 98E1
  \l 98E2
  \l 98E3
  \l 98E4
  \l 98E5
  \l 98E6
  \l 98E7
  \l 98E8
  \l 98E9
  \l 98EA
  \l 98EB
  \l 98EC
  \l 98ED
  \l 98EE
  \l 98EF
  \l 98F0
  \l 98F1
  \l 98F2
  \l 98F3
  \l 98F4
  \l 98F5
  \l 98F6
  \l 98F7
  \l 98F8
  \l 98F9
  \l 98FA
  \l 98FB
  \l 98FC
  \l 98FD
  \l 98FE
  \l 98FF
  \l 9900
  \l 9901
  \l 9902
  \l 9903
  \l 9904
  \l 9905
  \l 9906
  \l 9907
  \l 9908
  \l 9909
  \l 990A
  \l 990B
  \l 990C
  \l 990D
  \l 990E
  \l 990F
  \l 9910
  \l 9911
  \l 9912
  \l 9913
  \l 9914
  \l 9915
  \l 9916
  \l 9917
  \l 9918
  \l 9919
  \l 991A
  \l 991B
  \l 991C
  \l 991D
  \l 991E
  \l 991F
  \l 9920
  \l 9921
  \l 9922
  \l 9923
  \l 9924
  \l 9925
  \l 9926
  \l 9927
  \l 9928
  \l 9929
  \l 992A
  \l 992B
  \l 992C
  \l 992D
  \l 992E
  \l 992F
  \l 9930
  \l 9931
  \l 9932
  \l 9933
  \l 9934
  \l 9935
  \l 9936
  \l 9937
  \l 9938
  \l 9939
  \l 993A
  \l 993B
  \l 993C
  \l 993D
  \l 993E
  \l 993F
  \l 9940
  \l 9941
  \l 9942
  \l 9943
  \l 9944
  \l 9945
  \l 9946
  \l 9947
  \l 9948
  \l 9949
  \l 994A
  \l 994B
  \l 994C
  \l 994D
  \l 994E
  \l 994F
  \l 9950
  \l 9951
  \l 9952
  \l 9953
  \l 9954
  \l 9955
  \l 9956
  \l 9957
  \l 9958
  \l 9959
  \l 995A
  \l 995B
  \l 995C
  \l 995D
  \l 995E
  \l 995F
  \l 9960
  \l 9961
  \l 9962
  \l 9963
  \l 9964
  \l 9965
  \l 9966
  \l 9967
  \l 9968
  \l 9969
  \l 996A
  \l 996B
  \l 996C
  \l 996D
  \l 996E
  \l 996F
  \l 9970
  \l 9971
  \l 9972
  \l 9973
  \l 9974
  \l 9975
  \l 9976
  \l 9977
  \l 9978
  \l 9979
  \l 997A
  \l 997B
  \l 997C
  \l 997D
  \l 997E
  \l 997F
  \l 9980
  \l 9981
  \l 9982
  \l 9983
  \l 9984
  \l 9985
  \l 9986
  \l 9987
  \l 9988
  \l 9989
  \l 998A
  \l 998B
  \l 998C
  \l 998D
  \l 998E
  \l 998F
  \l 9990
  \l 9991
  \l 9992
  \l 9993
  \l 9994
  \l 9995
  \l 9996
  \l 9997
  \l 9998
  \l 9999
  \l 999A
  \l 999B
  \l 999C
  \l 999D
  \l 999E
  \l 999F
  \l 99A0
  \l 99A1
  \l 99A2
  \l 99A3
  \l 99A4
  \l 99A5
  \l 99A6
  \l 99A7
  \l 99A8
  \l 99A9
  \l 99AA
  \l 99AB
  \l 99AC
  \l 99AD
  \l 99AE
  \l 99AF
  \l 99B0
  \l 99B1
  \l 99B2
  \l 99B3
  \l 99B4
  \l 99B5
  \l 99B6
  \l 99B7
  \l 99B8
  \l 99B9
  \l 99BA
  \l 99BB
  \l 99BC
  \l 99BD
  \l 99BE
  \l 99BF
  \l 99C0
  \l 99C1
  \l 99C2
  \l 99C3
  \l 99C4
  \l 99C5
  \l 99C6
  \l 99C7
  \l 99C8
  \l 99C9
  \l 99CA
  \l 99CB
  \l 99CC
  \l 99CD
  \l 99CE
  \l 99CF
  \l 99D0
  \l 99D1
  \l 99D2
  \l 99D3
  \l 99D4
  \l 99D5
  \l 99D6
  \l 99D7
  \l 99D8
  \l 99D9
  \l 99DA
  \l 99DB
  \l 99DC
  \l 99DD
  \l 99DE
  \l 99DF
  \l 99E0
  \l 99E1
  \l 99E2
  \l 99E3
  \l 99E4
  \l 99E5
  \l 99E6
  \l 99E7
  \l 99E8
  \l 99E9
  \l 99EA
  \l 99EB
  \l 99EC
  \l 99ED
  \l 99EE
  \l 99EF
  \l 99F0
  \l 99F1
  \l 99F2
  \l 99F3
  \l 99F4
  \l 99F5
  \l 99F6
  \l 99F7
  \l 99F8
  \l 99F9
  \l 99FA
  \l 99FB
  \l 99FC
  \l 99FD
  \l 99FE
  \l 99FF
  \l 9A00
  \l 9A01
  \l 9A02
  \l 9A03
  \l 9A04
  \l 9A05
  \l 9A06
  \l 9A07
  \l 9A08
  \l 9A09
  \l 9A0A
  \l 9A0B
  \l 9A0C
  \l 9A0D
  \l 9A0E
  \l 9A0F
  \l 9A10
  \l 9A11
  \l 9A12
  \l 9A13
  \l 9A14
  \l 9A15
  \l 9A16
  \l 9A17
  \l 9A18
  \l 9A19
  \l 9A1A
  \l 9A1B
  \l 9A1C
  \l 9A1D
  \l 9A1E
  \l 9A1F
  \l 9A20
  \l 9A21
  \l 9A22
  \l 9A23
  \l 9A24
  \l 9A25
  \l 9A26
  \l 9A27
  \l 9A28
  \l 9A29
  \l 9A2A
  \l 9A2B
  \l 9A2C
  \l 9A2D
  \l 9A2E
  \l 9A2F
  \l 9A30
  \l 9A31
  \l 9A32
  \l 9A33
  \l 9A34
  \l 9A35
  \l 9A36
  \l 9A37
  \l 9A38
  \l 9A39
  \l 9A3A
  \l 9A3B
  \l 9A3C
  \l 9A3D
  \l 9A3E
  \l 9A3F
  \l 9A40
  \l 9A41
  \l 9A42
  \l 9A43
  \l 9A44
  \l 9A45
  \l 9A46
  \l 9A47
  \l 9A48
  \l 9A49
  \l 9A4A
  \l 9A4B
  \l 9A4C
  \l 9A4D
  \l 9A4E
  \l 9A4F
  \l 9A50
  \l 9A51
  \l 9A52
  \l 9A53
  \l 9A54
  \l 9A55
  \l 9A56
  \l 9A57
  \l 9A58
  \l 9A59
  \l 9A5A
  \l 9A5B
  \l 9A5C
  \l 9A5D
  \l 9A5E
  \l 9A5F
  \l 9A60
  \l 9A61
  \l 9A62
  \l 9A63
  \l 9A64
  \l 9A65
  \l 9A66
  \l 9A67
  \l 9A68
  \l 9A69
  \l 9A6A
  \l 9A6B
  \l 9A6C
  \l 9A6D
  \l 9A6E
  \l 9A6F
  \l 9A70
  \l 9A71
  \l 9A72
  \l 9A73
  \l 9A74
  \l 9A75
  \l 9A76
  \l 9A77
  \l 9A78
  \l 9A79
  \l 9A7A
  \l 9A7B
  \l 9A7C
  \l 9A7D
  \l 9A7E
  \l 9A7F
  \l 9A80
  \l 9A81
  \l 9A82
  \l 9A83
  \l 9A84
  \l 9A85
  \l 9A86
  \l 9A87
  \l 9A88
  \l 9A89
  \l 9A8A
  \l 9A8B
  \l 9A8C
  \l 9A8D
  \l 9A8E
  \l 9A8F
  \l 9A90
  \l 9A91
  \l 9A92
  \l 9A93
  \l 9A94
  \l 9A95
  \l 9A96
  \l 9A97
  \l 9A98
  \l 9A99
  \l 9A9A
  \l 9A9B
  \l 9A9C
  \l 9A9D
  \l 9A9E
  \l 9A9F
  \l 9AA0
  \l 9AA1
  \l 9AA2
  \l 9AA3
  \l 9AA4
  \l 9AA5
  \l 9AA6
  \l 9AA7
  \l 9AA8
  \l 9AA9
  \l 9AAA
  \l 9AAB
  \l 9AAC
  \l 9AAD
  \l 9AAE
  \l 9AAF
  \l 9AB0
  \l 9AB1
  \l 9AB2
  \l 9AB3
  \l 9AB4
  \l 9AB5
  \l 9AB6
  \l 9AB7
  \l 9AB8
  \l 9AB9
  \l 9ABA
  \l 9ABB
  \l 9ABC
  \l 9ABD
  \l 9ABE
  \l 9ABF
  \l 9AC0
  \l 9AC1
  \l 9AC2
  \l 9AC3
  \l 9AC4
  \l 9AC5
  \l 9AC6
  \l 9AC7
  \l 9AC8
  \l 9AC9
  \l 9ACA
  \l 9ACB
  \l 9ACC
  \l 9ACD
  \l 9ACE
  \l 9ACF
  \l 9AD0
  \l 9AD1
  \l 9AD2
  \l 9AD3
  \l 9AD4
  \l 9AD5
  \l 9AD6
  \l 9AD7
  \l 9AD8
  \l 9AD9
  \l 9ADA
  \l 9ADB
  \l 9ADC
  \l 9ADD
  \l 9ADE
  \l 9ADF
  \l 9AE0
  \l 9AE1
  \l 9AE2
  \l 9AE3
  \l 9AE4
  \l 9AE5
  \l 9AE6
  \l 9AE7
  \l 9AE8
  \l 9AE9
  \l 9AEA
  \l 9AEB
  \l 9AEC
  \l 9AED
  \l 9AEE
  \l 9AEF
  \l 9AF0
  \l 9AF1
  \l 9AF2
  \l 9AF3
  \l 9AF4
  \l 9AF5
  \l 9AF6
  \l 9AF7
  \l 9AF8
  \l 9AF9
  \l 9AFA
  \l 9AFB
  \l 9AFC
  \l 9AFD
  \l 9AFE
  \l 9AFF
  \l 9B00
  \l 9B01
  \l 9B02
  \l 9B03
  \l 9B04
  \l 9B05
  \l 9B06
  \l 9B07
  \l 9B08
  \l 9B09
  \l 9B0A
  \l 9B0B
  \l 9B0C
  \l 9B0D
  \l 9B0E
  \l 9B0F
  \l 9B10
  \l 9B11
  \l 9B12
  \l 9B13
  \l 9B14
  \l 9B15
  \l 9B16
  \l 9B17
  \l 9B18
  \l 9B19
  \l 9B1A
  \l 9B1B
  \l 9B1C
  \l 9B1D
  \l 9B1E
  \l 9B1F
  \l 9B20
  \l 9B21
  \l 9B22
  \l 9B23
  \l 9B24
  \l 9B25
  \l 9B26
  \l 9B27
  \l 9B28
  \l 9B29
  \l 9B2A
  \l 9B2B
  \l 9B2C
  \l 9B2D
  \l 9B2E
  \l 9B2F
  \l 9B30
  \l 9B31
  \l 9B32
  \l 9B33
  \l 9B34
  \l 9B35
  \l 9B36
  \l 9B37
  \l 9B38
  \l 9B39
  \l 9B3A
  \l 9B3B
  \l 9B3C
  \l 9B3D
  \l 9B3E
  \l 9B3F
  \l 9B40
  \l 9B41
  \l 9B42
  \l 9B43
  \l 9B44
  \l 9B45
  \l 9B46
  \l 9B47
  \l 9B48
  \l 9B49
  \l 9B4A
  \l 9B4B
  \l 9B4C
  \l 9B4D
  \l 9B4E
  \l 9B4F
  \l 9B50
  \l 9B51
  \l 9B52
  \l 9B53
  \l 9B54
  \l 9B55
  \l 9B56
  \l 9B57
  \l 9B58
  \l 9B59
  \l 9B5A
  \l 9B5B
  \l 9B5C
  \l 9B5D
  \l 9B5E
  \l 9B5F
  \l 9B60
  \l 9B61
  \l 9B62
  \l 9B63
  \l 9B64
  \l 9B65
  \l 9B66
  \l 9B67
  \l 9B68
  \l 9B69
  \l 9B6A
  \l 9B6B
  \l 9B6C
  \l 9B6D
  \l 9B6E
  \l 9B6F
  \l 9B70
  \l 9B71
  \l 9B72
  \l 9B73
  \l 9B74
  \l 9B75
  \l 9B76
  \l 9B77
  \l 9B78
  \l 9B79
  \l 9B7A
  \l 9B7B
  \l 9B7C
  \l 9B7D
  \l 9B7E
  \l 9B7F
  \l 9B80
  \l 9B81
  \l 9B82
  \l 9B83
  \l 9B84
  \l 9B85
  \l 9B86
  \l 9B87
  \l 9B88
  \l 9B89
  \l 9B8A
  \l 9B8B
  \l 9B8C
  \l 9B8D
  \l 9B8E
  \l 9B8F
  \l 9B90
  \l 9B91
  \l 9B92
  \l 9B93
  \l 9B94
  \l 9B95
  \l 9B96
  \l 9B97
  \l 9B98
  \l 9B99
  \l 9B9A
  \l 9B9B
  \l 9B9C
  \l 9B9D
  \l 9B9E
  \l 9B9F
  \l 9BA0
  \l 9BA1
  \l 9BA2
  \l 9BA3
  \l 9BA4
  \l 9BA5
  \l 9BA6
  \l 9BA7
  \l 9BA8
  \l 9BA9
  \l 9BAA
  \l 9BAB
  \l 9BAC
  \l 9BAD
  \l 9BAE
  \l 9BAF
  \l 9BB0
  \l 9BB1
  \l 9BB2
  \l 9BB3
  \l 9BB4
  \l 9BB5
  \l 9BB6
  \l 9BB7
  \l 9BB8
  \l 9BB9
  \l 9BBA
  \l 9BBB
  \l 9BBC
  \l 9BBD
  \l 9BBE
  \l 9BBF
  \l 9BC0
  \l 9BC1
  \l 9BC2
  \l 9BC3
  \l 9BC4
  \l 9BC5
  \l 9BC6
  \l 9BC7
  \l 9BC8
  \l 9BC9
  \l 9BCA
  \l 9BCB
  \l 9BCC
  \l 9BCD
  \l 9BCE
  \l 9BCF
  \l 9BD0
  \l 9BD1
  \l 9BD2
  \l 9BD3
  \l 9BD4
  \l 9BD5
  \l 9BD6
  \l 9BD7
  \l 9BD8
  \l 9BD9
  \l 9BDA
  \l 9BDB
  \l 9BDC
  \l 9BDD
  \l 9BDE
  \l 9BDF
  \l 9BE0
  \l 9BE1
  \l 9BE2
  \l 9BE3
  \l 9BE4
  \l 9BE5
  \l 9BE6
  \l 9BE7
  \l 9BE8
  \l 9BE9
  \l 9BEA
  \l 9BEB
  \l 9BEC
  \l 9BED
  \l 9BEE
  \l 9BEF
  \l 9BF0
  \l 9BF1
  \l 9BF2
  \l 9BF3
  \l 9BF4
  \l 9BF5
  \l 9BF6
  \l 9BF7
  \l 9BF8
  \l 9BF9
  \l 9BFA
  \l 9BFB
  \l 9BFC
  \l 9BFD
  \l 9BFE
  \l 9BFF
  \l 9C00
  \l 9C01
  \l 9C02
  \l 9C03
  \l 9C04
  \l 9C05
  \l 9C06
  \l 9C07
  \l 9C08
  \l 9C09
  \l 9C0A
  \l 9C0B
  \l 9C0C
  \l 9C0D
  \l 9C0E
  \l 9C0F
  \l 9C10
  \l 9C11
  \l 9C12
  \l 9C13
  \l 9C14
  \l 9C15
  \l 9C16
  \l 9C17
  \l 9C18
  \l 9C19
  \l 9C1A
  \l 9C1B
  \l 9C1C
  \l 9C1D
  \l 9C1E
  \l 9C1F
  \l 9C20
  \l 9C21
  \l 9C22
  \l 9C23
  \l 9C24
  \l 9C25
  \l 9C26
  \l 9C27
  \l 9C28
  \l 9C29
  \l 9C2A
  \l 9C2B
  \l 9C2C
  \l 9C2D
  \l 9C2E
  \l 9C2F
  \l 9C30
  \l 9C31
  \l 9C32
  \l 9C33
  \l 9C34
  \l 9C35
  \l 9C36
  \l 9C37
  \l 9C38
  \l 9C39
  \l 9C3A
  \l 9C3B
  \l 9C3C
  \l 9C3D
  \l 9C3E
  \l 9C3F
  \l 9C40
  \l 9C41
  \l 9C42
  \l 9C43
  \l 9C44
  \l 9C45
  \l 9C46
  \l 9C47
  \l 9C48
  \l 9C49
  \l 9C4A
  \l 9C4B
  \l 9C4C
  \l 9C4D
  \l 9C4E
  \l 9C4F
  \l 9C50
  \l 9C51
  \l 9C52
  \l 9C53
  \l 9C54
  \l 9C55
  \l 9C56
  \l 9C57
  \l 9C58
  \l 9C59
  \l 9C5A
  \l 9C5B
  \l 9C5C
  \l 9C5D
  \l 9C5E
  \l 9C5F
  \l 9C60
  \l 9C61
  \l 9C62
  \l 9C63
  \l 9C64
  \l 9C65
  \l 9C66
  \l 9C67
  \l 9C68
  \l 9C69
  \l 9C6A
  \l 9C6B
  \l 9C6C
  \l 9C6D
  \l 9C6E
  \l 9C6F
  \l 9C70
  \l 9C71
  \l 9C72
  \l 9C73
  \l 9C74
  \l 9C75
  \l 9C76
  \l 9C77
  \l 9C78
  \l 9C79
  \l 9C7A
  \l 9C7B
  \l 9C7C
  \l 9C7D
  \l 9C7E
  \l 9C7F
  \l 9C80
  \l 9C81
  \l 9C82
  \l 9C83
  \l 9C84
  \l 9C85
  \l 9C86
  \l 9C87
  \l 9C88
  \l 9C89
  \l 9C8A
  \l 9C8B
  \l 9C8C
  \l 9C8D
  \l 9C8E
  \l 9C8F
  \l 9C90
  \l 9C91
  \l 9C92
  \l 9C93
  \l 9C94
  \l 9C95
  \l 9C96
  \l 9C97
  \l 9C98
  \l 9C99
  \l 9C9A
  \l 9C9B
  \l 9C9C
  \l 9C9D
  \l 9C9E
  \l 9C9F
  \l 9CA0
  \l 9CA1
  \l 9CA2
  \l 9CA3
  \l 9CA4
  \l 9CA5
  \l 9CA6
  \l 9CA7
  \l 9CA8
  \l 9CA9
  \l 9CAA
  \l 9CAB
  \l 9CAC
  \l 9CAD
  \l 9CAE
  \l 9CAF
  \l 9CB0
  \l 9CB1
  \l 9CB2
  \l 9CB3
  \l 9CB4
  \l 9CB5
  \l 9CB6
  \l 9CB7
  \l 9CB8
  \l 9CB9
  \l 9CBA
  \l 9CBB
  \l 9CBC
  \l 9CBD
  \l 9CBE
  \l 9CBF
  \l 9CC0
  \l 9CC1
  \l 9CC2
  \l 9CC3
  \l 9CC4
  \l 9CC5
  \l 9CC6
  \l 9CC7
  \l 9CC8
  \l 9CC9
  \l 9CCA
  \l 9CCB
  \l 9CCC
  \l 9CCD
  \l 9CCE
  \l 9CCF
  \l 9CD0
  \l 9CD1
  \l 9CD2
  \l 9CD3
  \l 9CD4
  \l 9CD5
  \l 9CD6
  \l 9CD7
  \l 9CD8
  \l 9CD9
  \l 9CDA
  \l 9CDB
  \l 9CDC
  \l 9CDD
  \l 9CDE
  \l 9CDF
  \l 9CE0
  \l 9CE1
  \l 9CE2
  \l 9CE3
  \l 9CE4
  \l 9CE5
  \l 9CE6
  \l 9CE7
  \l 9CE8
  \l 9CE9
  \l 9CEA
  \l 9CEB
  \l 9CEC
  \l 9CED
  \l 9CEE
  \l 9CEF
  \l 9CF0
  \l 9CF1
  \l 9CF2
  \l 9CF3
  \l 9CF4
  \l 9CF5
  \l 9CF6
  \l 9CF7
  \l 9CF8
  \l 9CF9
  \l 9CFA
  \l 9CFB
  \l 9CFC
  \l 9CFD
  \l 9CFE
  \l 9CFF
  \l 9D00
  \l 9D01
  \l 9D02
  \l 9D03
  \l 9D04
  \l 9D05
  \l 9D06
  \l 9D07
  \l 9D08
  \l 9D09
  \l 9D0A
  \l 9D0B
  \l 9D0C
  \l 9D0D
  \l 9D0E
  \l 9D0F
  \l 9D10
  \l 9D11
  \l 9D12
  \l 9D13
  \l 9D14
  \l 9D15
  \l 9D16
  \l 9D17
  \l 9D18
  \l 9D19
  \l 9D1A
  \l 9D1B
  \l 9D1C
  \l 9D1D
  \l 9D1E
  \l 9D1F
  \l 9D20
  \l 9D21
  \l 9D22
  \l 9D23
  \l 9D24
  \l 9D25
  \l 9D26
  \l 9D27
  \l 9D28
  \l 9D29
  \l 9D2A
  \l 9D2B
  \l 9D2C
  \l 9D2D
  \l 9D2E
  \l 9D2F
  \l 9D30
  \l 9D31
  \l 9D32
  \l 9D33
  \l 9D34
  \l 9D35
  \l 9D36
  \l 9D37
  \l 9D38
  \l 9D39
  \l 9D3A
  \l 9D3B
  \l 9D3C
  \l 9D3D
  \l 9D3E
  \l 9D3F
  \l 9D40
  \l 9D41
  \l 9D42
  \l 9D43
  \l 9D44
  \l 9D45
  \l 9D46
  \l 9D47
  \l 9D48
  \l 9D49
  \l 9D4A
  \l 9D4B
  \l 9D4C
  \l 9D4D
  \l 9D4E
  \l 9D4F
  \l 9D50
  \l 9D51
  \l 9D52
  \l 9D53
  \l 9D54
  \l 9D55
  \l 9D56
  \l 9D57
  \l 9D58
  \l 9D59
  \l 9D5A
  \l 9D5B
  \l 9D5C
  \l 9D5D
  \l 9D5E
  \l 9D5F
  \l 9D60
  \l 9D61
  \l 9D62
  \l 9D63
  \l 9D64
  \l 9D65
  \l 9D66
  \l 9D67
  \l 9D68
  \l 9D69
  \l 9D6A
  \l 9D6B
  \l 9D6C
  \l 9D6D
  \l 9D6E
  \l 9D6F
  \l 9D70
  \l 9D71
  \l 9D72
  \l 9D73
  \l 9D74
  \l 9D75
  \l 9D76
  \l 9D77
  \l 9D78
  \l 9D79
  \l 9D7A
  \l 9D7B
  \l 9D7C
  \l 9D7D
  \l 9D7E
  \l 9D7F
  \l 9D80
  \l 9D81
  \l 9D82
  \l 9D83
  \l 9D84
  \l 9D85
  \l 9D86
  \l 9D87
  \l 9D88
  \l 9D89
  \l 9D8A
  \l 9D8B
  \l 9D8C
  \l 9D8D
  \l 9D8E
  \l 9D8F
  \l 9D90
  \l 9D91
  \l 9D92
  \l 9D93
  \l 9D94
  \l 9D95
  \l 9D96
  \l 9D97
  \l 9D98
  \l 9D99
  \l 9D9A
  \l 9D9B
  \l 9D9C
  \l 9D9D
  \l 9D9E
  \l 9D9F
  \l 9DA0
  \l 9DA1
  \l 9DA2
  \l 9DA3
  \l 9DA4
  \l 9DA5
  \l 9DA6
  \l 9DA7
  \l 9DA8
  \l 9DA9
  \l 9DAA
  \l 9DAB
  \l 9DAC
  \l 9DAD
  \l 9DAE
  \l 9DAF
  \l 9DB0
  \l 9DB1
  \l 9DB2
  \l 9DB3
  \l 9DB4
  \l 9DB5
  \l 9DB6
  \l 9DB7
  \l 9DB8
  \l 9DB9
  \l 9DBA
  \l 9DBB
  \l 9DBC
  \l 9DBD
  \l 9DBE
  \l 9DBF
  \l 9DC0
  \l 9DC1
  \l 9DC2
  \l 9DC3
  \l 9DC4
  \l 9DC5
  \l 9DC6
  \l 9DC7
  \l 9DC8
  \l 9DC9
  \l 9DCA
  \l 9DCB
  \l 9DCC
  \l 9DCD
  \l 9DCE
  \l 9DCF
  \l 9DD0
  \l 9DD1
  \l 9DD2
  \l 9DD3
  \l 9DD4
  \l 9DD5
  \l 9DD6
  \l 9DD7
  \l 9DD8
  \l 9DD9
  \l 9DDA
  \l 9DDB
  \l 9DDC
  \l 9DDD
  \l 9DDE
  \l 9DDF
  \l 9DE0
  \l 9DE1
  \l 9DE2
  \l 9DE3
  \l 9DE4
  \l 9DE5
  \l 9DE6
  \l 9DE7
  \l 9DE8
  \l 9DE9
  \l 9DEA
  \l 9DEB
  \l 9DEC
  \l 9DED
  \l 9DEE
  \l 9DEF
  \l 9DF0
  \l 9DF1
  \l 9DF2
  \l 9DF3
  \l 9DF4
  \l 9DF5
  \l 9DF6
  \l 9DF7
  \l 9DF8
  \l 9DF9
  \l 9DFA
  \l 9DFB
  \l 9DFC
  \l 9DFD
  \l 9DFE
  \l 9DFF
  \l 9E00
  \l 9E01
  \l 9E02
  \l 9E03
  \l 9E04
  \l 9E05
  \l 9E06
  \l 9E07
  \l 9E08
  \l 9E09
  \l 9E0A
  \l 9E0B
  \l 9E0C
  \l 9E0D
  \l 9E0E
  \l 9E0F
  \l 9E10
  \l 9E11
  \l 9E12
  \l 9E13
  \l 9E14
  \l 9E15
  \l 9E16
  \l 9E17
  \l 9E18
  \l 9E19
  \l 9E1A
  \l 9E1B
  \l 9E1C
  \l 9E1D
  \l 9E1E
  \l 9E1F
  \l 9E20
  \l 9E21
  \l 9E22
  \l 9E23
  \l 9E24
  \l 9E25
  \l 9E26
  \l 9E27
  \l 9E28
  \l 9E29
  \l 9E2A
  \l 9E2B
  \l 9E2C
  \l 9E2D
  \l 9E2E
  \l 9E2F
  \l 9E30
  \l 9E31
  \l 9E32
  \l 9E33
  \l 9E34
  \l 9E35
  \l 9E36
  \l 9E37
  \l 9E38
  \l 9E39
  \l 9E3A
  \l 9E3B
  \l 9E3C
  \l 9E3D
  \l 9E3E
  \l 9E3F
  \l 9E40
  \l 9E41
  \l 9E42
  \l 9E43
  \l 9E44
  \l 9E45
  \l 9E46
  \l 9E47
  \l 9E48
  \l 9E49
  \l 9E4A
  \l 9E4B
  \l 9E4C
  \l 9E4D
  \l 9E4E
  \l 9E4F
  \l 9E50
  \l 9E51
  \l 9E52
  \l 9E53
  \l 9E54
  \l 9E55
  \l 9E56
  \l 9E57
  \l 9E58
  \l 9E59
  \l 9E5A
  \l 9E5B
  \l 9E5C
  \l 9E5D
  \l 9E5E
  \l 9E5F
  \l 9E60
  \l 9E61
  \l 9E62
  \l 9E63
  \l 9E64
  \l 9E65
  \l 9E66
  \l 9E67
  \l 9E68
  \l 9E69
  \l 9E6A
  \l 9E6B
  \l 9E6C
  \l 9E6D
  \l 9E6E
  \l 9E6F
  \l 9E70
  \l 9E71
  \l 9E72
  \l 9E73
  \l 9E74
  \l 9E75
  \l 9E76
  \l 9E77
  \l 9E78
  \l 9E79
  \l 9E7A
  \l 9E7B
  \l 9E7C
  \l 9E7D
  \l 9E7E
  \l 9E7F
  \l 9E80
  \l 9E81
  \l 9E82
  \l 9E83
  \l 9E84
  \l 9E85
  \l 9E86
  \l 9E87
  \l 9E88
  \l 9E89
  \l 9E8A
  \l 9E8B
  \l 9E8C
  \l 9E8D
  \l 9E8E
  \l 9E8F
  \l 9E90
  \l 9E91
  \l 9E92
  \l 9E93
  \l 9E94
  \l 9E95
  \l 9E96
  \l 9E97
  \l 9E98
  \l 9E99
  \l 9E9A
  \l 9E9B
  \l 9E9C
  \l 9E9D
  \l 9E9E
  \l 9E9F
  \l 9EA0
  \l 9EA1
  \l 9EA2
  \l 9EA3
  \l 9EA4
  \l 9EA5
  \l 9EA6
  \l 9EA7
  \l 9EA8
  \l 9EA9
  \l 9EAA
  \l 9EAB
  \l 9EAC
  \l 9EAD
  \l 9EAE
  \l 9EAF
  \l 9EB0
  \l 9EB1
  \l 9EB2
  \l 9EB3
  \l 9EB4
  \l 9EB5
  \l 9EB6
  \l 9EB7
  \l 9EB8
  \l 9EB9
  \l 9EBA
  \l 9EBB
  \l 9EBC
  \l 9EBD
  \l 9EBE
  \l 9EBF
  \l 9EC0
  \l 9EC1
  \l 9EC2
  \l 9EC3
  \l 9EC4
  \l 9EC5
  \l 9EC6
  \l 9EC7
  \l 9EC8
  \l 9EC9
  \l 9ECA
  \l 9ECB
  \l 9ECC
  \l 9ECD
  \l 9ECE
  \l 9ECF
  \l 9ED0
  \l 9ED1
  \l 9ED2
  \l 9ED3
  \l 9ED4
  \l 9ED5
  \l 9ED6
  \l 9ED7
  \l 9ED8
  \l 9ED9
  \l 9EDA
  \l 9EDB
  \l 9EDC
  \l 9EDD
  \l 9EDE
  \l 9EDF
  \l 9EE0
  \l 9EE1
  \l 9EE2
  \l 9EE3
  \l 9EE4
  \l 9EE5
  \l 9EE6
  \l 9EE7
  \l 9EE8
  \l 9EE9
  \l 9EEA
  \l 9EEB
  \l 9EEC
  \l 9EED
  \l 9EEE
  \l 9EEF
  \l 9EF0
  \l 9EF1
  \l 9EF2
  \l 9EF3
  \l 9EF4
  \l 9EF5
  \l 9EF6
  \l 9EF7
  \l 9EF8
  \l 9EF9
  \l 9EFA
  \l 9EFB
  \l 9EFC
  \l 9EFD
  \l 9EFE
  \l 9EFF
  \l 9F00
  \l 9F01
  \l 9F02
  \l 9F03
  \l 9F04
  \l 9F05
  \l 9F06
  \l 9F07
  \l 9F08
  \l 9F09
  \l 9F0A
  \l 9F0B
  \l 9F0C
  \l 9F0D
  \l 9F0E
  \l 9F0F
  \l 9F10
  \l 9F11
  \l 9F12
  \l 9F13
  \l 9F14
  \l 9F15
  \l 9F16
  \l 9F17
  \l 9F18
  \l 9F19
  \l 9F1A
  \l 9F1B
  \l 9F1C
  \l 9F1D
  \l 9F1E
  \l 9F1F
  \l 9F20
  \l 9F21
  \l 9F22
  \l 9F23
  \l 9F24
  \l 9F25
  \l 9F26
  \l 9F27
  \l 9F28
  \l 9F29
  \l 9F2A
  \l 9F2B
  \l 9F2C
  \l 9F2D
  \l 9F2E
  \l 9F2F
  \l 9F30
  \l 9F31
  \l 9F32
  \l 9F33
  \l 9F34
  \l 9F35
  \l 9F36
  \l 9F37
  \l 9F38
  \l 9F39
  \l 9F3A
  \l 9F3B
  \l 9F3C
  \l 9F3D
  \l 9F3E
  \l 9F3F
  \l 9F40
  \l 9F41
  \l 9F42
  \l 9F43
  \l 9F44
  \l 9F45
  \l 9F46
  \l 9F47
  \l 9F48
  \l 9F49
  \l 9F4A
  \l 9F4B
  \l 9F4C
  \l 9F4D
  \l 9F4E
  \l 9F4F
  \l 9F50
  \l 9F51
  \l 9F52
  \l 9F53
  \l 9F54
  \l 9F55
  \l 9F56
  \l 9F57
  \l 9F58
  \l 9F59
  \l 9F5A
  \l 9F5B
  \l 9F5C
  \l 9F5D
  \l 9F5E
  \l 9F5F
  \l 9F60
  \l 9F61
  \l 9F62
  \l 9F63
  \l 9F64
  \l 9F65
  \l 9F66
  \l 9F67
  \l 9F68
  \l 9F69
  \l 9F6A
  \l 9F6B
  \l 9F6C
  \l 9F6D
  \l 9F6E
  \l 9F6F
  \l 9F70
  \l 9F71
  \l 9F72
  \l 9F73
  \l 9F74
  \l 9F75
  \l 9F76
  \l 9F77
  \l 9F78
  \l 9F79
  \l 9F7A
  \l 9F7B
  \l 9F7C
  \l 9F7D
  \l 9F7E
  \l 9F7F
  \l 9F80
  \l 9F81
  \l 9F82
  \l 9F83
  \l 9F84
  \l 9F85
  \l 9F86
  \l 9F87
  \l 9F88
  \l 9F89
  \l 9F8A
  \l 9F8B
  \l 9F8C
  \l 9F8D
  \l 9F8E
  \l 9F8F
  \l 9F90
  \l 9F91
  \l 9F92
  \l 9F93
  \l 9F94
  \l 9F95
  \l 9F96
  \l 9F97
  \l 9F98
  \l 9F99
  \l 9F9A
  \l 9F9B
  \l 9F9C
  \l 9F9D
  \l 9F9E
  \l 9F9F
  \l 9FA0
  \l 9FA1
  \l 9FA2
  \l 9FA3
  \l 9FA4
  \l 9FA5
  \l 9FA6
  \l 9FA7
  \l 9FA8
  \l 9FA9
  \l 9FAA
  \l 9FAB
  \l 9FAC
  \l 9FAD
  \l 9FAE
  \l 9FAF
  \l 9FB0
  \l 9FB1
  \l 9FB2
  \l 9FB3
  \l 9FB4
  \l 9FB5
  \l 9FB6
  \l 9FB7
  \l 9FB8
  \l 9FB9
  \l 9FBA
  \l 9FBB
  \l 9FBC
  \l 9FBD
  \l 9FBE
  \l 9FBF
  \l 9FC0
  \l 9FC1
  \l 9FC2
  \l 9FC3
  \l 9FC4
  \l 9FC5
  \l 9FC6
  \l 9FC7
  \l 9FC8
  \l 9FC9
  \l 9FCA
  \l 9FCB
  \l 9FCC
  \l A000
  \l A001
  \l A002
  \l A003
  \l A004
  \l A005
  \l A006
  \l A007
  \l A008
  \l A009
  \l A00A
  \l A00B
  \l A00C
  \l A00D
  \l A00E
  \l A00F
  \l A010
  \l A011
  \l A012
  \l A013
  \l A014
  \l A015
  \l A016
  \l A017
  \l A018
  \l A019
  \l A01A
  \l A01B
  \l A01C
  \l A01D
  \l A01E
  \l A01F
  \l A020
  \l A021
  \l A022
  \l A023
  \l A024
  \l A025
  \l A026
  \l A027
  \l A028
  \l A029
  \l A02A
  \l A02B
  \l A02C
  \l A02D
  \l A02E
  \l A02F
  \l A030
  \l A031
  \l A032
  \l A033
  \l A034
  \l A035
  \l A036
  \l A037
  \l A038
  \l A039
  \l A03A
  \l A03B
  \l A03C
  \l A03D
  \l A03E
  \l A03F
  \l A040
  \l A041
  \l A042
  \l A043
  \l A044
  \l A045
  \l A046
  \l A047
  \l A048
  \l A049
  \l A04A
  \l A04B
  \l A04C
  \l A04D
  \l A04E
  \l A04F
  \l A050
  \l A051
  \l A052
  \l A053
  \l A054
  \l A055
  \l A056
  \l A057
  \l A058
  \l A059
  \l A05A
  \l A05B
  \l A05C
  \l A05D
  \l A05E
  \l A05F
  \l A060
  \l A061
  \l A062
  \l A063
  \l A064
  \l A065
  \l A066
  \l A067
  \l A068
  \l A069
  \l A06A
  \l A06B
  \l A06C
  \l A06D
  \l A06E
  \l A06F
  \l A070
  \l A071
  \l A072
  \l A073
  \l A074
  \l A075
  \l A076
  \l A077
  \l A078
  \l A079
  \l A07A
  \l A07B
  \l A07C
  \l A07D
  \l A07E
  \l A07F
  \l A080
  \l A081
  \l A082
  \l A083
  \l A084
  \l A085
  \l A086
  \l A087
  \l A088
  \l A089
  \l A08A
  \l A08B
  \l A08C
  \l A08D
  \l A08E
  \l A08F
  \l A090
  \l A091
  \l A092
  \l A093
  \l A094
  \l A095
  \l A096
  \l A097
  \l A098
  \l A099
  \l A09A
  \l A09B
  \l A09C
  \l A09D
  \l A09E
  \l A09F
  \l A0A0
  \l A0A1
  \l A0A2
  \l A0A3
  \l A0A4
  \l A0A5
  \l A0A6
  \l A0A7
  \l A0A8
  \l A0A9
  \l A0AA
  \l A0AB
  \l A0AC
  \l A0AD
  \l A0AE
  \l A0AF
  \l A0B0
  \l A0B1
  \l A0B2
  \l A0B3
  \l A0B4
  \l A0B5
  \l A0B6
  \l A0B7
  \l A0B8
  \l A0B9
  \l A0BA
  \l A0BB
  \l A0BC
  \l A0BD
  \l A0BE
  \l A0BF
  \l A0C0
  \l A0C1
  \l A0C2
  \l A0C3
  \l A0C4
  \l A0C5
  \l A0C6
  \l A0C7
  \l A0C8
  \l A0C9
  \l A0CA
  \l A0CB
  \l A0CC
  \l A0CD
  \l A0CE
  \l A0CF
  \l A0D0
  \l A0D1
  \l A0D2
  \l A0D3
  \l A0D4
  \l A0D5
  \l A0D6
  \l A0D7
  \l A0D8
  \l A0D9
  \l A0DA
  \l A0DB
  \l A0DC
  \l A0DD
  \l A0DE
  \l A0DF
  \l A0E0
  \l A0E1
  \l A0E2
  \l A0E3
  \l A0E4
  \l A0E5
  \l A0E6
  \l A0E7
  \l A0E8
  \l A0E9
  \l A0EA
  \l A0EB
  \l A0EC
  \l A0ED
  \l A0EE
  \l A0EF
  \l A0F0
  \l A0F1
  \l A0F2
  \l A0F3
  \l A0F4
  \l A0F5
  \l A0F6
  \l A0F7
  \l A0F8
  \l A0F9
  \l A0FA
  \l A0FB
  \l A0FC
  \l A0FD
  \l A0FE
  \l A0FF
  \l A100
  \l A101
  \l A102
  \l A103
  \l A104
  \l A105
  \l A106
  \l A107
  \l A108
  \l A109
  \l A10A
  \l A10B
  \l A10C
  \l A10D
  \l A10E
  \l A10F
  \l A110
  \l A111
  \l A112
  \l A113
  \l A114
  \l A115
  \l A116
  \l A117
  \l A118
  \l A119
  \l A11A
  \l A11B
  \l A11C
  \l A11D
  \l A11E
  \l A11F
  \l A120
  \l A121
  \l A122
  \l A123
  \l A124
  \l A125
  \l A126
  \l A127
  \l A128
  \l A129
  \l A12A
  \l A12B
  \l A12C
  \l A12D
  \l A12E
  \l A12F
  \l A130
  \l A131
  \l A132
  \l A133
  \l A134
  \l A135
  \l A136
  \l A137
  \l A138
  \l A139
  \l A13A
  \l A13B
  \l A13C
  \l A13D
  \l A13E
  \l A13F
  \l A140
  \l A141
  \l A142
  \l A143
  \l A144
  \l A145
  \l A146
  \l A147
  \l A148
  \l A149
  \l A14A
  \l A14B
  \l A14C
  \l A14D
  \l A14E
  \l A14F
  \l A150
  \l A151
  \l A152
  \l A153
  \l A154
  \l A155
  \l A156
  \l A157
  \l A158
  \l A159
  \l A15A
  \l A15B
  \l A15C
  \l A15D
  \l A15E
  \l A15F
  \l A160
  \l A161
  \l A162
  \l A163
  \l A164
  \l A165
  \l A166
  \l A167
  \l A168
  \l A169
  \l A16A
  \l A16B
  \l A16C
  \l A16D
  \l A16E
  \l A16F
  \l A170
  \l A171
  \l A172
  \l A173
  \l A174
  \l A175
  \l A176
  \l A177
  \l A178
  \l A179
  \l A17A
  \l A17B
  \l A17C
  \l A17D
  \l A17E
  \l A17F
  \l A180
  \l A181
  \l A182
  \l A183
  \l A184
  \l A185
  \l A186
  \l A187
  \l A188
  \l A189
  \l A18A
  \l A18B
  \l A18C
  \l A18D
  \l A18E
  \l A18F
  \l A190
  \l A191
  \l A192
  \l A193
  \l A194
  \l A195
  \l A196
  \l A197
  \l A198
  \l A199
  \l A19A
  \l A19B
  \l A19C
  \l A19D
  \l A19E
  \l A19F
  \l A1A0
  \l A1A1
  \l A1A2
  \l A1A3
  \l A1A4
  \l A1A5
  \l A1A6
  \l A1A7
  \l A1A8
  \l A1A9
  \l A1AA
  \l A1AB
  \l A1AC
  \l A1AD
  \l A1AE
  \l A1AF
  \l A1B0
  \l A1B1
  \l A1B2
  \l A1B3
  \l A1B4
  \l A1B5
  \l A1B6
  \l A1B7
  \l A1B8
  \l A1B9
  \l A1BA
  \l A1BB
  \l A1BC
  \l A1BD
  \l A1BE
  \l A1BF
  \l A1C0
  \l A1C1
  \l A1C2
  \l A1C3
  \l A1C4
  \l A1C5
  \l A1C6
  \l A1C7
  \l A1C8
  \l A1C9
  \l A1CA
  \l A1CB
  \l A1CC
  \l A1CD
  \l A1CE
  \l A1CF
  \l A1D0
  \l A1D1
  \l A1D2
  \l A1D3
  \l A1D4
  \l A1D5
  \l A1D6
  \l A1D7
  \l A1D8
  \l A1D9
  \l A1DA
  \l A1DB
  \l A1DC
  \l A1DD
  \l A1DE
  \l A1DF
  \l A1E0
  \l A1E1
  \l A1E2
  \l A1E3
  \l A1E4
  \l A1E5
  \l A1E6
  \l A1E7
  \l A1E8
  \l A1E9
  \l A1EA
  \l A1EB
  \l A1EC
  \l A1ED
  \l A1EE
  \l A1EF
  \l A1F0
  \l A1F1
  \l A1F2
  \l A1F3
  \l A1F4
  \l A1F5
  \l A1F6
  \l A1F7
  \l A1F8
  \l A1F9
  \l A1FA
  \l A1FB
  \l A1FC
  \l A1FD
  \l A1FE
  \l A1FF
  \l A200
  \l A201
  \l A202
  \l A203
  \l A204
  \l A205
  \l A206
  \l A207
  \l A208
  \l A209
  \l A20A
  \l A20B
  \l A20C
  \l A20D
  \l A20E
  \l A20F
  \l A210
  \l A211
  \l A212
  \l A213
  \l A214
  \l A215
  \l A216
  \l A217
  \l A218
  \l A219
  \l A21A
  \l A21B
  \l A21C
  \l A21D
  \l A21E
  \l A21F
  \l A220
  \l A221
  \l A222
  \l A223
  \l A224
  \l A225
  \l A226
  \l A227
  \l A228
  \l A229
  \l A22A
  \l A22B
  \l A22C
  \l A22D
  \l A22E
  \l A22F
  \l A230
  \l A231
  \l A232
  \l A233
  \l A234
  \l A235
  \l A236
  \l A237
  \l A238
  \l A239
  \l A23A
  \l A23B
  \l A23C
  \l A23D
  \l A23E
  \l A23F
  \l A240
  \l A241
  \l A242
  \l A243
  \l A244
  \l A245
  \l A246
  \l A247
  \l A248
  \l A249
  \l A24A
  \l A24B
  \l A24C
  \l A24D
  \l A24E
  \l A24F
  \l A250
  \l A251
  \l A252
  \l A253
  \l A254
  \l A255
  \l A256
  \l A257
  \l A258
  \l A259
  \l A25A
  \l A25B
  \l A25C
  \l A25D
  \l A25E
  \l A25F
  \l A260
  \l A261
  \l A262
  \l A263
  \l A264
  \l A265
  \l A266
  \l A267
  \l A268
  \l A269
  \l A26A
  \l A26B
  \l A26C
  \l A26D
  \l A26E
  \l A26F
  \l A270
  \l A271
  \l A272
  \l A273
  \l A274
  \l A275
  \l A276
  \l A277
  \l A278
  \l A279
  \l A27A
  \l A27B
  \l A27C
  \l A27D
  \l A27E
  \l A27F
  \l A280
  \l A281
  \l A282
  \l A283
  \l A284
  \l A285
  \l A286
  \l A287
  \l A288
  \l A289
  \l A28A
  \l A28B
  \l A28C
  \l A28D
  \l A28E
  \l A28F
  \l A290
  \l A291
  \l A292
  \l A293
  \l A294
  \l A295
  \l A296
  \l A297
  \l A298
  \l A299
  \l A29A
  \l A29B
  \l A29C
  \l A29D
  \l A29E
  \l A29F
  \l A2A0
  \l A2A1
  \l A2A2
  \l A2A3
  \l A2A4
  \l A2A5
  \l A2A6
  \l A2A7
  \l A2A8
  \l A2A9
  \l A2AA
  \l A2AB
  \l A2AC
  \l A2AD
  \l A2AE
  \l A2AF
  \l A2B0
  \l A2B1
  \l A2B2
  \l A2B3
  \l A2B4
  \l A2B5
  \l A2B6
  \l A2B7
  \l A2B8
  \l A2B9
  \l A2BA
  \l A2BB
  \l A2BC
  \l A2BD
  \l A2BE
  \l A2BF
  \l A2C0
  \l A2C1
  \l A2C2
  \l A2C3
  \l A2C4
  \l A2C5
  \l A2C6
  \l A2C7
  \l A2C8
  \l A2C9
  \l A2CA
  \l A2CB
  \l A2CC
  \l A2CD
  \l A2CE
  \l A2CF
  \l A2D0
  \l A2D1
  \l A2D2
  \l A2D3
  \l A2D4
  \l A2D5
  \l A2D6
  \l A2D7
  \l A2D8
  \l A2D9
  \l A2DA
  \l A2DB
  \l A2DC
  \l A2DD
  \l A2DE
  \l A2DF
  \l A2E0
  \l A2E1
  \l A2E2
  \l A2E3
  \l A2E4
  \l A2E5
  \l A2E6
  \l A2E7
  \l A2E8
  \l A2E9
  \l A2EA
  \l A2EB
  \l A2EC
  \l A2ED
  \l A2EE
  \l A2EF
  \l A2F0
  \l A2F1
  \l A2F2
  \l A2F3
  \l A2F4
  \l A2F5
  \l A2F6
  \l A2F7
  \l A2F8
  \l A2F9
  \l A2FA
  \l A2FB
  \l A2FC
  \l A2FD
  \l A2FE
  \l A2FF
  \l A300
  \l A301
  \l A302
  \l A303
  \l A304
  \l A305
  \l A306
  \l A307
  \l A308
  \l A309
  \l A30A
  \l A30B
  \l A30C
  \l A30D
  \l A30E
  \l A30F
  \l A310
  \l A311
  \l A312
  \l A313
  \l A314
  \l A315
  \l A316
  \l A317
  \l A318
  \l A319
  \l A31A
  \l A31B
  \l A31C
  \l A31D
  \l A31E
  \l A31F
  \l A320
  \l A321
  \l A322
  \l A323
  \l A324
  \l A325
  \l A326
  \l A327
  \l A328
  \l A329
  \l A32A
  \l A32B
  \l A32C
  \l A32D
  \l A32E
  \l A32F
  \l A330
  \l A331
  \l A332
  \l A333
  \l A334
  \l A335
  \l A336
  \l A337
  \l A338
  \l A339
  \l A33A
  \l A33B
  \l A33C
  \l A33D
  \l A33E
  \l A33F
  \l A340
  \l A341
  \l A342
  \l A343
  \l A344
  \l A345
  \l A346
  \l A347
  \l A348
  \l A349
  \l A34A
  \l A34B
  \l A34C
  \l A34D
  \l A34E
  \l A34F
  \l A350
  \l A351
  \l A352
  \l A353
  \l A354
  \l A355
  \l A356
  \l A357
  \l A358
  \l A359
  \l A35A
  \l A35B
  \l A35C
  \l A35D
  \l A35E
  \l A35F
  \l A360
  \l A361
  \l A362
  \l A363
  \l A364
  \l A365
  \l A366
  \l A367
  \l A368
  \l A369
  \l A36A
  \l A36B
  \l A36C
  \l A36D
  \l A36E
  \l A36F
  \l A370
  \l A371
  \l A372
  \l A373
  \l A374
  \l A375
  \l A376
  \l A377
  \l A378
  \l A379
  \l A37A
  \l A37B
  \l A37C
  \l A37D
  \l A37E
  \l A37F
  \l A380
  \l A381
  \l A382
  \l A383
  \l A384
  \l A385
  \l A386
  \l A387
  \l A388
  \l A389
  \l A38A
  \l A38B
  \l A38C
  \l A38D
  \l A38E
  \l A38F
  \l A390
  \l A391
  \l A392
  \l A393
  \l A394
  \l A395
  \l A396
  \l A397
  \l A398
  \l A399
  \l A39A
  \l A39B
  \l A39C
  \l A39D
  \l A39E
  \l A39F
  \l A3A0
  \l A3A1
  \l A3A2
  \l A3A3
  \l A3A4
  \l A3A5
  \l A3A6
  \l A3A7
  \l A3A8
  \l A3A9
  \l A3AA
  \l A3AB
  \l A3AC
  \l A3AD
  \l A3AE
  \l A3AF
  \l A3B0
  \l A3B1
  \l A3B2
  \l A3B3
  \l A3B4
  \l A3B5
  \l A3B6
  \l A3B7
  \l A3B8
  \l A3B9
  \l A3BA
  \l A3BB
  \l A3BC
  \l A3BD
  \l A3BE
  \l A3BF
  \l A3C0
  \l A3C1
  \l A3C2
  \l A3C3
  \l A3C4
  \l A3C5
  \l A3C6
  \l A3C7
  \l A3C8
  \l A3C9
  \l A3CA
  \l A3CB
  \l A3CC
  \l A3CD
  \l A3CE
  \l A3CF
  \l A3D0
  \l A3D1
  \l A3D2
  \l A3D3
  \l A3D4
  \l A3D5
  \l A3D6
  \l A3D7
  \l A3D8
  \l A3D9
  \l A3DA
  \l A3DB
  \l A3DC
  \l A3DD
  \l A3DE
  \l A3DF
  \l A3E0
  \l A3E1
  \l A3E2
  \l A3E3
  \l A3E4
  \l A3E5
  \l A3E6
  \l A3E7
  \l A3E8
  \l A3E9
  \l A3EA
  \l A3EB
  \l A3EC
  \l A3ED
  \l A3EE
  \l A3EF
  \l A3F0
  \l A3F1
  \l A3F2
  \l A3F3
  \l A3F4
  \l A3F5
  \l A3F6
  \l A3F7
  \l A3F8
  \l A3F9
  \l A3FA
  \l A3FB
  \l A3FC
  \l A3FD
  \l A3FE
  \l A3FF
  \l A400
  \l A401
  \l A402
  \l A403
  \l A404
  \l A405
  \l A406
  \l A407
  \l A408
  \l A409
  \l A40A
  \l A40B
  \l A40C
  \l A40D
  \l A40E
  \l A40F
  \l A410
  \l A411
  \l A412
  \l A413
  \l A414
  \l A415
  \l A416
  \l A417
  \l A418
  \l A419
  \l A41A
  \l A41B
  \l A41C
  \l A41D
  \l A41E
  \l A41F
  \l A420
  \l A421
  \l A422
  \l A423
  \l A424
  \l A425
  \l A426
  \l A427
  \l A428
  \l A429
  \l A42A
  \l A42B
  \l A42C
  \l A42D
  \l A42E
  \l A42F
  \l A430
  \l A431
  \l A432
  \l A433
  \l A434
  \l A435
  \l A436
  \l A437
  \l A438
  \l A439
  \l A43A
  \l A43B
  \l A43C
  \l A43D
  \l A43E
  \l A43F
  \l A440
  \l A441
  \l A442
  \l A443
  \l A444
  \l A445
  \l A446
  \l A447
  \l A448
  \l A449
  \l A44A
  \l A44B
  \l A44C
  \l A44D
  \l A44E
  \l A44F
  \l A450
  \l A451
  \l A452
  \l A453
  \l A454
  \l A455
  \l A456
  \l A457
  \l A458
  \l A459
  \l A45A
  \l A45B
  \l A45C
  \l A45D
  \l A45E
  \l A45F
  \l A460
  \l A461
  \l A462
  \l A463
  \l A464
  \l A465
  \l A466
  \l A467
  \l A468
  \l A469
  \l A46A
  \l A46B
  \l A46C
  \l A46D
  \l A46E
  \l A46F
  \l A470
  \l A471
  \l A472
  \l A473
  \l A474
  \l A475
  \l A476
  \l A477
  \l A478
  \l A479
  \l A47A
  \l A47B
  \l A47C
  \l A47D
  \l A47E
  \l A47F
  \l A480
  \l A481
  \l A482
  \l A483
  \l A484
  \l A485
  \l A486
  \l A487
  \l A488
  \l A489
  \l A48A
  \l A48B
  \l A48C
  \l A4D0
  \l A4D1
  \l A4D2
  \l A4D3
  \l A4D4
  \l A4D5
  \l A4D6
  \l A4D7
  \l A4D8
  \l A4D9
  \l A4DA
  \l A4DB
  \l A4DC
  \l A4DD
  \l A4DE
  \l A4DF
  \l A4E0
  \l A4E1
  \l A4E2
  \l A4E3
  \l A4E4
  \l A4E5
  \l A4E6
  \l A4E7
  \l A4E8
  \l A4E9
  \l A4EA
  \l A4EB
  \l A4EC
  \l A4ED
  \l A4EE
  \l A4EF
  \l A4F0
  \l A4F1
  \l A4F2
  \l A4F3
  \l A4F4
  \l A4F5
  \l A4F6
  \l A4F7
  \l A4F8
  \l A4F9
  \l A4FA
  \l A4FB
  \l A4FC
  \l A4FD
  \l A500
  \l A501
  \l A502
  \l A503
  \l A504
  \l A505
  \l A506
  \l A507
  \l A508
  \l A509
  \l A50A
  \l A50B
  \l A50C
  \l A50D
  \l A50E
  \l A50F
  \l A510
  \l A511
  \l A512
  \l A513
  \l A514
  \l A515
  \l A516
  \l A517
  \l A518
  \l A519
  \l A51A
  \l A51B
  \l A51C
  \l A51D
  \l A51E
  \l A51F
  \l A520
  \l A521
  \l A522
  \l A523
  \l A524
  \l A525
  \l A526
  \l A527
  \l A528
  \l A529
  \l A52A
  \l A52B
  \l A52C
  \l A52D
  \l A52E
  \l A52F
  \l A530
  \l A531
  \l A532
  \l A533
  \l A534
  \l A535
  \l A536
  \l A537
  \l A538
  \l A539
  \l A53A
  \l A53B
  \l A53C
  \l A53D
  \l A53E
  \l A53F
  \l A540
  \l A541
  \l A542
  \l A543
  \l A544
  \l A545
  \l A546
  \l A547
  \l A548
  \l A549
  \l A54A
  \l A54B
  \l A54C
  \l A54D
  \l A54E
  \l A54F
  \l A550
  \l A551
  \l A552
  \l A553
  \l A554
  \l A555
  \l A556
  \l A557
  \l A558
  \l A559
  \l A55A
  \l A55B
  \l A55C
  \l A55D
  \l A55E
  \l A55F
  \l A560
  \l A561
  \l A562
  \l A563
  \l A564
  \l A565
  \l A566
  \l A567
  \l A568
  \l A569
  \l A56A
  \l A56B
  \l A56C
  \l A56D
  \l A56E
  \l A56F
  \l A570
  \l A571
  \l A572
  \l A573
  \l A574
  \l A575
  \l A576
  \l A577
  \l A578
  \l A579
  \l A57A
  \l A57B
  \l A57C
  \l A57D
  \l A57E
  \l A57F
  \l A580
  \l A581
  \l A582
  \l A583
  \l A584
  \l A585
  \l A586
  \l A587
  \l A588
  \l A589
  \l A58A
  \l A58B
  \l A58C
  \l A58D
  \l A58E
  \l A58F
  \l A590
  \l A591
  \l A592
  \l A593
  \l A594
  \l A595
  \l A596
  \l A597
  \l A598
  \l A599
  \l A59A
  \l A59B
  \l A59C
  \l A59D
  \l A59E
  \l A59F
  \l A5A0
  \l A5A1
  \l A5A2
  \l A5A3
  \l A5A4
  \l A5A5
  \l A5A6
  \l A5A7
  \l A5A8
  \l A5A9
  \l A5AA
  \l A5AB
  \l A5AC
  \l A5AD
  \l A5AE
  \l A5AF
  \l A5B0
  \l A5B1
  \l A5B2
  \l A5B3
  \l A5B4
  \l A5B5
  \l A5B6
  \l A5B7
  \l A5B8
  \l A5B9
  \l A5BA
  \l A5BB
  \l A5BC
  \l A5BD
  \l A5BE
  \l A5BF
  \l A5C0
  \l A5C1
  \l A5C2
  \l A5C3
  \l A5C4
  \l A5C5
  \l A5C6
  \l A5C7
  \l A5C8
  \l A5C9
  \l A5CA
  \l A5CB
  \l A5CC
  \l A5CD
  \l A5CE
  \l A5CF
  \l A5D0
  \l A5D1
  \l A5D2
  \l A5D3
  \l A5D4
  \l A5D5
  \l A5D6
  \l A5D7
  \l A5D8
  \l A5D9
  \l A5DA
  \l A5DB
  \l A5DC
  \l A5DD
  \l A5DE
  \l A5DF
  \l A5E0
  \l A5E1
  \l A5E2
  \l A5E3
  \l A5E4
  \l A5E5
  \l A5E6
  \l A5E7
  \l A5E8
  \l A5E9
  \l A5EA
  \l A5EB
  \l A5EC
  \l A5ED
  \l A5EE
  \l A5EF
  \l A5F0
  \l A5F1
  \l A5F2
  \l A5F3
  \l A5F4
  \l A5F5
  \l A5F6
  \l A5F7
  \l A5F8
  \l A5F9
  \l A5FA
  \l A5FB
  \l A5FC
  \l A5FD
  \l A5FE
  \l A5FF
  \l A600
  \l A601
  \l A602
  \l A603
  \l A604
  \l A605
  \l A606
  \l A607
  \l A608
  \l A609
  \l A60A
  \l A60B
  \l A60C
  \l A610
  \l A611
  \l A612
  \l A613
  \l A614
  \l A615
  \l A616
  \l A617
  \l A618
  \l A619
  \l A61A
  \l A61B
  \l A61C
  \l A61D
  \l A61E
  \l A61F
  \l A62A
  \l A62B
  \L A640 A640 A641
  \L A641 A640 A641
  \L A642 A642 A643
  \L A643 A642 A643
  \L A644 A644 A645
  \L A645 A644 A645
  \L A646 A646 A647
  \L A647 A646 A647
  \L A648 A648 A649
  \L A649 A648 A649
  \L A64A A64A A64B
  \L A64B A64A A64B
  \L A64C A64C A64D
  \L A64D A64C A64D
  \L A64E A64E A64F
  \L A64F A64E A64F
  \L A650 A650 A651
  \L A651 A650 A651
  \L A652 A652 A653
  \L A653 A652 A653
  \L A654 A654 A655
  \L A655 A654 A655
  \L A656 A656 A657
  \L A657 A656 A657
  \L A658 A658 A659
  \L A659 A658 A659
  \L A65A A65A A65B
  \L A65B A65A A65B
  \L A65C A65C A65D
  \L A65D A65C A65D
  \L A65E A65E A65F
  \L A65F A65E A65F
  \L A660 A660 A661
  \L A661 A660 A661
  \L A662 A662 A663
  \L A663 A662 A663
  \L A664 A664 A665
  \L A665 A664 A665
  \L A666 A666 A667
  \L A667 A666 A667
  \L A668 A668 A669
  \L A669 A668 A669
  \L A66A A66A A66B
  \L A66B A66A A66B
  \L A66C A66C A66D
  \L A66D A66C A66D
  \l A66E
  \l A66F
  \l A670
  \l A671
  \l A672
  \l A674
  \l A675
  \l A676
  \l A677
  \l A678
  \l A679
  \l A67A
  \l A67B
  \l A67C
  \l A67D
  \l A67F
  \L A680 A680 A681
  \L A681 A680 A681
  \L A682 A682 A683
  \L A683 A682 A683
  \L A684 A684 A685
  \L A685 A684 A685
  \L A686 A686 A687
  \L A687 A686 A687
  \L A688 A688 A689
  \L A689 A688 A689
  \L A68A A68A A68B
  \L A68B A68A A68B
  \L A68C A68C A68D
  \L A68D A68C A68D
  \L A68E A68E A68F
  \L A68F A68E A68F
  \L A690 A690 A691
  \L A691 A690 A691
  \L A692 A692 A693
  \L A693 A692 A693
  \L A694 A694 A695
  \L A695 A694 A695
  \L A696 A696 A697
  \L A697 A696 A697
  \L A698 A698 A699
  \L A699 A698 A699
  \L A69A A69A A69B
  \L A69B A69A A69B
  \l A69C
  \l A69D
  \l A69F
  \l A6A0
  \l A6A1
  \l A6A2
  \l A6A3
  \l A6A4
  \l A6A5
  \l A6A6
  \l A6A7
  \l A6A8
  \l A6A9
  \l A6AA
  \l A6AB
  \l A6AC
  \l A6AD
  \l A6AE
  \l A6AF
  \l A6B0
  \l A6B1
  \l A6B2
  \l A6B3
  \l A6B4
  \l A6B5
  \l A6B6
  \l A6B7
  \l A6B8
  \l A6B9
  \l A6BA
  \l A6BB
  \l A6BC
  \l A6BD
  \l A6BE
  \l A6BF
  \l A6C0
  \l A6C1
  \l A6C2
  \l A6C3
  \l A6C4
  \l A6C5
  \l A6C6
  \l A6C7
  \l A6C8
  \l A6C9
  \l A6CA
  \l A6CB
  \l A6CC
  \l A6CD
  \l A6CE
  \l A6CF
  \l A6D0
  \l A6D1
  \l A6D2
  \l A6D3
  \l A6D4
  \l A6D5
  \l A6D6
  \l A6D7
  \l A6D8
  \l A6D9
  \l A6DA
  \l A6DB
  \l A6DC
  \l A6DD
  \l A6DE
  \l A6DF
  \l A6E0
  \l A6E1
  \l A6E2
  \l A6E3
  \l A6E4
  \l A6E5
  \l A6F0
  \l A6F1
  \l A717
  \l A718
  \l A719
  \l A71A
  \l A71B
  \l A71C
  \l A71D
  \l A71E
  \l A71F
  \L A722 A722 A723
  \L A723 A722 A723
  \L A724 A724 A725
  \L A725 A724 A725
  \L A726 A726 A727
  \L A727 A726 A727
  \L A728 A728 A729
  \L A729 A728 A729
  \L A72A A72A A72B
  \L A72B A72A A72B
  \L A72C A72C A72D
  \L A72D A72C A72D
  \L A72E A72E A72F
  \L A72F A72E A72F
  \l A730
  \l A731
  \L A732 A732 A733
  \L A733 A732 A733
  \L A734 A734 A735
  \L A735 A734 A735
  \L A736 A736 A737
  \L A737 A736 A737
  \L A738 A738 A739
  \L A739 A738 A739
  \L A73A A73A A73B
  \L A73B A73A A73B
  \L A73C A73C A73D
  \L A73D A73C A73D
  \L A73E A73E A73F
  \L A73F A73E A73F
  \L A740 A740 A741
  \L A741 A740 A741
  \L A742 A742 A743
  \L A743 A742 A743
  \L A744 A744 A745
  \L A745 A744 A745
  \L A746 A746 A747
  \L A747 A746 A747
  \L A748 A748 A749
  \L A749 A748 A749
  \L A74A A74A A74B
  \L A74B A74A A74B
  \L A74C A74C A74D
  \L A74D A74C A74D
  \L A74E A74E A74F
  \L A74F A74E A74F
  \L A750 A750 A751
  \L A751 A750 A751
  \L A752 A752 A753
  \L A753 A752 A753
  \L A754 A754 A755
  \L A755 A754 A755
  \L A756 A756 A757
  \L A757 A756 A757
  \L A758 A758 A759
  \L A759 A758 A759
  \L A75A A75A A75B
  \L A75B A75A A75B
  \L A75C A75C A75D
  \L A75D A75C A75D
  \L A75E A75E A75F
  \L A75F A75E A75F
  \L A760 A760 A761
  \L A761 A760 A761
  \L A762 A762 A763
  \L A763 A762 A763
  \L A764 A764 A765
  \L A765 A764 A765
  \L A766 A766 A767
  \L A767 A766 A767
  \L A768 A768 A769
  \L A769 A768 A769
  \L A76A A76A A76B
  \L A76B A76A A76B
  \L A76C A76C A76D
  \L A76D A76C A76D
  \L A76E A76E A76F
  \L A76F A76E A76F
  \l A770
  \l A771
  \l A772
  \l A773
  \l A774
  \l A775
  \l A776
  \l A777
  \l A778
  \L A779 A779 A77A
  \L A77A A779 A77A
  \L A77B A77B A77C
  \L A77C A77B A77C
  \L A77D A77D 1D79
  \L A77E A77E A77F
  \L A77F A77E A77F
  \L A780 A780 A781
  \L A781 A780 A781
  \L A782 A782 A783
  \L A783 A782 A783
  \L A784 A784 A785
  \L A785 A784 A785
  \L A786 A786 A787
  \L A787 A786 A787
  \l A788
  \L A78B A78B A78C
  \L A78C A78B A78C
  \L A78D A78D 0265
  \l A78E
  \L A790 A790 A791
  \L A791 A790 A791
  \L A792 A792 A793
  \L A793 A792 A793
  \l A794
  \l A795
  \L A796 A796 A797
  \L A797 A796 A797
  \L A798 A798 A799
  \L A799 A798 A799
  \L A79A A79A A79B
  \L A79B A79A A79B
  \L A79C A79C A79D
  \L A79D A79C A79D
  \L A79E A79E A79F
  \L A79F A79E A79F
  \L A7A0 A7A0 A7A1
  \L A7A1 A7A0 A7A1
  \L A7A2 A7A2 A7A3
  \L A7A3 A7A2 A7A3
  \L A7A4 A7A4 A7A5
  \L A7A5 A7A4 A7A5
  \L A7A6 A7A6 A7A7
  \L A7A7 A7A6 A7A7
  \L A7A8 A7A8 A7A9
  \L A7A9 A7A8 A7A9
  \L A7AA A7AA 0266
  \L A7AB A7AB 025C
  \L A7AC A7AC 0261
  \L A7AD A7AD 026C
  \L A7B0 A7B0 029E
  \L A7B1 A7B1 0287
  \l A7F7
  \l A7F8
  \l A7F9
  \l A7FA
  \l A7FB
  \l A7FC
  \l A7FD
  \l A7FE
  \l A7FF
  \l A800
  \l A801
  \l A802
  \l A803
  \l A804
  \l A805
  \l A806
  \l A807
  \l A808
  \l A809
  \l A80A
  \l A80B
  \l A80C
  \l A80D
  \l A80E
  \l A80F
  \l A810
  \l A811
  \l A812
  \l A813
  \l A814
  \l A815
  \l A816
  \l A817
  \l A818
  \l A819
  \l A81A
  \l A81B
  \l A81C
  \l A81D
  \l A81E
  \l A81F
  \l A820
  \l A821
  \l A822
  \l A823
  \l A824
  \l A825
  \l A826
  \l A827
  \l A840
  \l A841
  \l A842
  \l A843
  \l A844
  \l A845
  \l A846
  \l A847
  \l A848
  \l A849
  \l A84A
  \l A84B
  \l A84C
  \l A84D
  \l A84E
  \l A84F
  \l A850
  \l A851
  \l A852
  \l A853
  \l A854
  \l A855
  \l A856
  \l A857
  \l A858
  \l A859
  \l A85A
  \l A85B
  \l A85C
  \l A85D
  \l A85E
  \l A85F
  \l A860
  \l A861
  \l A862
  \l A863
  \l A864
  \l A865
  \l A866
  \l A867
  \l A868
  \l A869
  \l A86A
  \l A86B
  \l A86C
  \l A86D
  \l A86E
  \l A86F
  \l A870
  \l A871
  \l A872
  \l A873
  \l A880
  \l A881
  \l A882
  \l A883
  \l A884
  \l A885
  \l A886
  \l A887
  \l A888
  \l A889
  \l A88A
  \l A88B
  \l A88C
  \l A88D
  \l A88E
  \l A88F
  \l A890
  \l A891
  \l A892
  \l A893
  \l A894
  \l A895
  \l A896
  \l A897
  \l A898
  \l A899
  \l A89A
  \l A89B
  \l A89C
  \l A89D
  \l A89E
  \l A89F
  \l A8A0
  \l A8A1
  \l A8A2
  \l A8A3
  \l A8A4
  \l A8A5
  \l A8A6
  \l A8A7
  \l A8A8
  \l A8A9
  \l A8AA
  \l A8AB
  \l A8AC
  \l A8AD
  \l A8AE
  \l A8AF
  \l A8B0
  \l A8B1
  \l A8B2
  \l A8B3
  \l A8B4
  \l A8B5
  \l A8B6
  \l A8B7
  \l A8B8
  \l A8B9
  \l A8BA
  \l A8BB
  \l A8BC
  \l A8BD
  \l A8BE
  \l A8BF
  \l A8C0
  \l A8C1
  \l A8C2
  \l A8C3
  \l A8C4
  \l A8E0
  \l A8E1
  \l A8E2
  \l A8E3
  \l A8E4
  \l A8E5
  \l A8E6
  \l A8E7
  \l A8E8
  \l A8E9
  \l A8EA
  \l A8EB
  \l A8EC
  \l A8ED
  \l A8EE
  \l A8EF
  \l A8F0
  \l A8F1
  \l A8F2
  \l A8F3
  \l A8F4
  \l A8F5
  \l A8F6
  \l A8F7
  \l A8FB
  \l A90A
  \l A90B
  \l A90C
  \l A90D
  \l A90E
  \l A90F
  \l A910
  \l A911
  \l A912
  \l A913
  \l A914
  \l A915
  \l A916
  \l A917
  \l A918
  \l A919
  \l A91A
  \l A91B
  \l A91C
  \l A91D
  \l A91E
  \l A91F
  \l A920
  \l A921
  \l A922
  \l A923
  \l A924
  \l A925
  \l A926
  \l A927
  \l A928
  \l A929
  \l A92A
  \l A92B
  \l A92C
  \l A92D
  \l A930
  \l A931
  \l A932
  \l A933
  \l A934
  \l A935
  \l A936
  \l A937
  \l A938
  \l A939
  \l A93A
  \l A93B
  \l A93C
  \l A93D
  \l A93E
  \l A93F
  \l A940
  \l A941
  \l A942
  \l A943
  \l A944
  \l A945
  \l A946
  \l A947
  \l A948
  \l A949
  \l A94A
  \l A94B
  \l A94C
  \l A94D
  \l A94E
  \l A94F
  \l A950
  \l A951
  \l A952
  \l A953
  \l A960
  \l A961
  \l A962
  \l A963
  \l A964
  \l A965
  \l A966
  \l A967
  \l A968
  \l A969
  \l A96A
  \l A96B
  \l A96C
  \l A96D
  \l A96E
  \l A96F
  \l A970
  \l A971
  \l A972
  \l A973
  \l A974
  \l A975
  \l A976
  \l A977
  \l A978
  \l A979
  \l A97A
  \l A97B
  \l A97C
  \l A980
  \l A981
  \l A982
  \l A983
  \l A984
  \l A985
  \l A986
  \l A987
  \l A988
  \l A989
  \l A98A
  \l A98B
  \l A98C
  \l A98D
  \l A98E
  \l A98F
  \l A990
  \l A991
  \l A992
  \l A993
  \l A994
  \l A995
  \l A996
  \l A997
  \l A998
  \l A999
  \l A99A
  \l A99B
  \l A99C
  \l A99D
  \l A99E
  \l A99F
  \l A9A0
  \l A9A1
  \l A9A2
  \l A9A3
  \l A9A4
  \l A9A5
  \l A9A6
  \l A9A7
  \l A9A8
  \l A9A9
  \l A9AA
  \l A9AB
  \l A9AC
  \l A9AD
  \l A9AE
  \l A9AF
  \l A9B0
  \l A9B1
  \l A9B2
  \l A9B3
  \l A9B4
  \l A9B5
  \l A9B6
  \l A9B7
  \l A9B8
  \l A9B9
  \l A9BA
  \l A9BB
  \l A9BC
  \l A9BD
  \l A9BE
  \l A9BF
  \l A9C0
  \l A9CF
  \l A9E0
  \l A9E1
  \l A9E2
  \l A9E3
  \l A9E4
  \l A9E5
  \l A9E6
  \l A9E7
  \l A9E8
  \l A9E9
  \l A9EA
  \l A9EB
  \l A9EC
  \l A9ED
  \l A9EE
  \l A9EF
  \l A9FA
  \l A9FB
  \l A9FC
  \l A9FD
  \l A9FE
  \l AA00
  \l AA01
  \l AA02
  \l AA03
  \l AA04
  \l AA05
  \l AA06
  \l AA07
  \l AA08
  \l AA09
  \l AA0A
  \l AA0B
  \l AA0C
  \l AA0D
  \l AA0E
  \l AA0F
  \l AA10
  \l AA11
  \l AA12
  \l AA13
  \l AA14
  \l AA15
  \l AA16
  \l AA17
  \l AA18
  \l AA19
  \l AA1A
  \l AA1B
  \l AA1C
  \l AA1D
  \l AA1E
  \l AA1F
  \l AA20
  \l AA21
  \l AA22
  \l AA23
  \l AA24
  \l AA25
  \l AA26
  \l AA27
  \l AA28
  \l AA29
  \l AA2A
  \l AA2B
  \l AA2C
  \l AA2D
  \l AA2E
  \l AA2F
  \l AA30
  \l AA31
  \l AA32
  \l AA33
  \l AA34
  \l AA35
  \l AA36
  \l AA40
  \l AA41
  \l AA42
  \l AA43
  \l AA44
  \l AA45
  \l AA46
  \l AA47
  \l AA48
  \l AA49
  \l AA4A
  \l AA4B
  \l AA4C
  \l AA4D
  \l AA60
  \l AA61
  \l AA62
  \l AA63
  \l AA64
  \l AA65
  \l AA66
  \l AA67
  \l AA68
  \l AA69
  \l AA6A
  \l AA6B
  \l AA6C
  \l AA6D
  \l AA6E
  \l AA6F
  \l AA70
  \l AA71
  \l AA72
  \l AA73
  \l AA74
  \l AA75
  \l AA76
  \l AA7A
  \l AA7B
  \l AA7C
  \l AA7D
  \l AA7E
  \l AA7F
  \l AA80
  \l AA81
  \l AA82
  \l AA83
  \l AA84
  \l AA85
  \l AA86
  \l AA87
  \l AA88
  \l AA89
  \l AA8A
  \l AA8B
  \l AA8C
  \l AA8D
  \l AA8E
  \l AA8F
  \l AA90
  \l AA91
  \l AA92
  \l AA93
  \l AA94
  \l AA95
  \l AA96
  \l AA97
  \l AA98
  \l AA99
  \l AA9A
  \l AA9B
  \l AA9C
  \l AA9D
  \l AA9E
  \l AA9F
  \l AAA0
  \l AAA1
  \l AAA2
  \l AAA3
  \l AAA4
  \l AAA5
  \l AAA6
  \l AAA7
  \l AAA8
  \l AAA9
  \l AAAA
  \l AAAB
  \l AAAC
  \l AAAD
  \l AAAE
  \l AAAF
  \l AAB0
  \l AAB1
  \l AAB2
  \l AAB3
  \l AAB4
  \l AAB5
  \l AAB6
  \l AAB7
  \l AAB8
  \l AAB9
  \l AABA
  \l AABB
  \l AABC
  \l AABD
  \l AABE
  \l AABF
  \l AAC0
  \l AAC1
  \l AAC2
  \l AADB
  \l AADC
  \l AADD
  \l AAE0
  \l AAE1
  \l AAE2
  \l AAE3
  \l AAE4
  \l AAE5
  \l AAE6
  \l AAE7
  \l AAE8
  \l AAE9
  \l AAEA
  \l AAEB
  \l AAEC
  \l AAED
  \l AAEE
  \l AAEF
  \l AAF2
  \l AAF3
  \l AAF4
  \l AAF5
  \l AAF6
  \l AB01
  \l AB02
  \l AB03
  \l AB04
  \l AB05
  \l AB06
  \l AB09
  \l AB0A
  \l AB0B
  \l AB0C
  \l AB0D
  \l AB0E
  \l AB11
  \l AB12
  \l AB13
  \l AB14
  \l AB15
  \l AB16
  \l AB20
  \l AB21
  \l AB22
  \l AB23
  \l AB24
  \l AB25
  \l AB26
  \l AB28
  \l AB29
  \l AB2A
  \l AB2B
  \l AB2C
  \l AB2D
  \l AB2E
  \l AB30
  \l AB31
  \l AB32
  \l AB33
  \l AB34
  \l AB35
  \l AB36
  \l AB37
  \l AB38
  \l AB39
  \l AB3A
  \l AB3B
  \l AB3C
  \l AB3D
  \l AB3E
  \l AB3F
  \l AB40
  \l AB41
  \l AB42
  \l AB43
  \l AB44
  \l AB45
  \l AB46
  \l AB47
  \l AB48
  \l AB49
  \l AB4A
  \l AB4B
  \l AB4C
  \l AB4D
  \l AB4E
  \l AB4F
  \l AB50
  \l AB51
  \l AB52
  \l AB53
  \l AB54
  \l AB55
  \l AB56
  \l AB57
  \l AB58
  \l AB59
  \l AB5A
  \l AB5C
  \l AB5D
  \l AB5E
  \l AB5F
  \l AB64
  \l AB65
  \l ABC0
  \l ABC1
  \l ABC2
  \l ABC3
  \l ABC4
  \l ABC5
  \l ABC6
  \l ABC7
  \l ABC8
  \l ABC9
  \l ABCA
  \l ABCB
  \l ABCC
  \l ABCD
  \l ABCE
  \l ABCF
  \l ABD0
  \l ABD1
  \l ABD2
  \l ABD3
  \l ABD4
  \l ABD5
  \l ABD6
  \l ABD7
  \l ABD8
  \l ABD9
  \l ABDA
  \l ABDB
  \l ABDC
  \l ABDD
  \l ABDE
  \l ABDF
  \l ABE0
  \l ABE1
  \l ABE2
  \l ABE3
  \l ABE4
  \l ABE5
  \l ABE6
  \l ABE7
  \l ABE8
  \l ABE9
  \l ABEA
  \l ABEC
  \l ABED
  \l AC00
  \l AC01
  \l AC02
  \l AC03
  \l AC04
  \l AC05
  \l AC06
  \l AC07
  \l AC08
  \l AC09
  \l AC0A
  \l AC0B
  \l AC0C
  \l AC0D
  \l AC0E
  \l AC0F
  \l AC10
  \l AC11
  \l AC12
  \l AC13
  \l AC14
  \l AC15
  \l AC16
  \l AC17
  \l AC18
  \l AC19
  \l AC1A
  \l AC1B
  \l AC1C
  \l AC1D
  \l AC1E
  \l AC1F
  \l AC20
  \l AC21
  \l AC22
  \l AC23
  \l AC24
  \l AC25
  \l AC26
  \l AC27
  \l AC28
  \l AC29
  \l AC2A
  \l AC2B
  \l AC2C
  \l AC2D
  \l AC2E
  \l AC2F
  \l AC30
  \l AC31
  \l AC32
  \l AC33
  \l AC34
  \l AC35
  \l AC36
  \l AC37
  \l AC38
  \l AC39
  \l AC3A
  \l AC3B
  \l AC3C
  \l AC3D
  \l AC3E
  \l AC3F
  \l AC40
  \l AC41
  \l AC42
  \l AC43
  \l AC44
  \l AC45
  \l AC46
  \l AC47
  \l AC48
  \l AC49
  \l AC4A
  \l AC4B
  \l AC4C
  \l AC4D
  \l AC4E
  \l AC4F
  \l AC50
  \l AC51
  \l AC52
  \l AC53
  \l AC54
  \l AC55
  \l AC56
  \l AC57
  \l AC58
  \l AC59
  \l AC5A
  \l AC5B
  \l AC5C
  \l AC5D
  \l AC5E
  \l AC5F
  \l AC60
  \l AC61
  \l AC62
  \l AC63
  \l AC64
  \l AC65
  \l AC66
  \l AC67
  \l AC68
  \l AC69
  \l AC6A
  \l AC6B
  \l AC6C
  \l AC6D
  \l AC6E
  \l AC6F
  \l AC70
  \l AC71
  \l AC72
  \l AC73
  \l AC74
  \l AC75
  \l AC76
  \l AC77
  \l AC78
  \l AC79
  \l AC7A
  \l AC7B
  \l AC7C
  \l AC7D
  \l AC7E
  \l AC7F
  \l AC80
  \l AC81
  \l AC82
  \l AC83
  \l AC84
  \l AC85
  \l AC86
  \l AC87
  \l AC88
  \l AC89
  \l AC8A
  \l AC8B
  \l AC8C
  \l AC8D
  \l AC8E
  \l AC8F
  \l AC90
  \l AC91
  \l AC92
  \l AC93
  \l AC94
  \l AC95
  \l AC96
  \l AC97
  \l AC98
  \l AC99
  \l AC9A
  \l AC9B
  \l AC9C
  \l AC9D
  \l AC9E
  \l AC9F
  \l ACA0
  \l ACA1
  \l ACA2
  \l ACA3
  \l ACA4
  \l ACA5
  \l ACA6
  \l ACA7
  \l ACA8
  \l ACA9
  \l ACAA
  \l ACAB
  \l ACAC
  \l ACAD
  \l ACAE
  \l ACAF
  \l ACB0
  \l ACB1
  \l ACB2
  \l ACB3
  \l ACB4
  \l ACB5
  \l ACB6
  \l ACB7
  \l ACB8
  \l ACB9
  \l ACBA
  \l ACBB
  \l ACBC
  \l ACBD
  \l ACBE
  \l ACBF
  \l ACC0
  \l ACC1
  \l ACC2
  \l ACC3
  \l ACC4
  \l ACC5
  \l ACC6
  \l ACC7
  \l ACC8
  \l ACC9
  \l ACCA
  \l ACCB
  \l ACCC
  \l ACCD
  \l ACCE
  \l ACCF
  \l ACD0
  \l ACD1
  \l ACD2
  \l ACD3
  \l ACD4
  \l ACD5
  \l ACD6
  \l ACD7
  \l ACD8
  \l ACD9
  \l ACDA
  \l ACDB
  \l ACDC
  \l ACDD
  \l ACDE
  \l ACDF
  \l ACE0
  \l ACE1
  \l ACE2
  \l ACE3
  \l ACE4
  \l ACE5
  \l ACE6
  \l ACE7
  \l ACE8
  \l ACE9
  \l ACEA
  \l ACEB
  \l ACEC
  \l ACED
  \l ACEE
  \l ACEF
  \l ACF0
  \l ACF1
  \l ACF2
  \l ACF3
  \l ACF4
  \l ACF5
  \l ACF6
  \l ACF7
  \l ACF8
  \l ACF9
  \l ACFA
  \l ACFB
  \l ACFC
  \l ACFD
  \l ACFE
  \l ACFF
  \l AD00
  \l AD01
  \l AD02
  \l AD03
  \l AD04
  \l AD05
  \l AD06
  \l AD07
  \l AD08
  \l AD09
  \l AD0A
  \l AD0B
  \l AD0C
  \l AD0D
  \l AD0E
  \l AD0F
  \l AD10
  \l AD11
  \l AD12
  \l AD13
  \l AD14
  \l AD15
  \l AD16
  \l AD17
  \l AD18
  \l AD19
  \l AD1A
  \l AD1B
  \l AD1C
  \l AD1D
  \l AD1E
  \l AD1F
  \l AD20
  \l AD21
  \l AD22
  \l AD23
  \l AD24
  \l AD25
  \l AD26
  \l AD27
  \l AD28
  \l AD29
  \l AD2A
  \l AD2B
  \l AD2C
  \l AD2D
  \l AD2E
  \l AD2F
  \l AD30
  \l AD31
  \l AD32
  \l AD33
  \l AD34
  \l AD35
  \l AD36
  \l AD37
  \l AD38
  \l AD39
  \l AD3A
  \l AD3B
  \l AD3C
  \l AD3D
  \l AD3E
  \l AD3F
  \l AD40
  \l AD41
  \l AD42
  \l AD43
  \l AD44
  \l AD45
  \l AD46
  \l AD47
  \l AD48
  \l AD49
  \l AD4A
  \l AD4B
  \l AD4C
  \l AD4D
  \l AD4E
  \l AD4F
  \l AD50
  \l AD51
  \l AD52
  \l AD53
  \l AD54
  \l AD55
  \l AD56
  \l AD57
  \l AD58
  \l AD59
  \l AD5A
  \l AD5B
  \l AD5C
  \l AD5D
  \l AD5E
  \l AD5F
  \l AD60
  \l AD61
  \l AD62
  \l AD63
  \l AD64
  \l AD65
  \l AD66
  \l AD67
  \l AD68
  \l AD69
  \l AD6A
  \l AD6B
  \l AD6C
  \l AD6D
  \l AD6E
  \l AD6F
  \l AD70
  \l AD71
  \l AD72
  \l AD73
  \l AD74
  \l AD75
  \l AD76
  \l AD77
  \l AD78
  \l AD79
  \l AD7A
  \l AD7B
  \l AD7C
  \l AD7D
  \l AD7E
  \l AD7F
  \l AD80
  \l AD81
  \l AD82
  \l AD83
  \l AD84
  \l AD85
  \l AD86
  \l AD87
  \l AD88
  \l AD89
  \l AD8A
  \l AD8B
  \l AD8C
  \l AD8D
  \l AD8E
  \l AD8F
  \l AD90
  \l AD91
  \l AD92
  \l AD93
  \l AD94
  \l AD95
  \l AD96
  \l AD97
  \l AD98
  \l AD99
  \l AD9A
  \l AD9B
  \l AD9C
  \l AD9D
  \l AD9E
  \l AD9F
  \l ADA0
  \l ADA1
  \l ADA2
  \l ADA3
  \l ADA4
  \l ADA5
  \l ADA6
  \l ADA7
  \l ADA8
  \l ADA9
  \l ADAA
  \l ADAB
  \l ADAC
  \l ADAD
  \l ADAE
  \l ADAF
  \l ADB0
  \l ADB1
  \l ADB2
  \l ADB3
  \l ADB4
  \l ADB5
  \l ADB6
  \l ADB7
  \l ADB8
  \l ADB9
  \l ADBA
  \l ADBB
  \l ADBC
  \l ADBD
  \l ADBE
  \l ADBF
  \l ADC0
  \l ADC1
  \l ADC2
  \l ADC3
  \l ADC4
  \l ADC5
  \l ADC6
  \l ADC7
  \l ADC8
  \l ADC9
  \l ADCA
  \l ADCB
  \l ADCC
  \l ADCD
  \l ADCE
  \l ADCF
  \l ADD0
  \l ADD1
  \l ADD2
  \l ADD3
  \l ADD4
  \l ADD5
  \l ADD6
  \l ADD7
  \l ADD8
  \l ADD9
  \l ADDA
  \l ADDB
  \l ADDC
  \l ADDD
  \l ADDE
  \l ADDF
  \l ADE0
  \l ADE1
  \l ADE2
  \l ADE3
  \l ADE4
  \l ADE5
  \l ADE6
  \l ADE7
  \l ADE8
  \l ADE9
  \l ADEA
  \l ADEB
  \l ADEC
  \l ADED
  \l ADEE
  \l ADEF
  \l ADF0
  \l ADF1
  \l ADF2
  \l ADF3
  \l ADF4
  \l ADF5
  \l ADF6
  \l ADF7
  \l ADF8
  \l ADF9
  \l ADFA
  \l ADFB
  \l ADFC
  \l ADFD
  \l ADFE
  \l ADFF
  \l AE00
  \l AE01
  \l AE02
  \l AE03
  \l AE04
  \l AE05
  \l AE06
  \l AE07
  \l AE08
  \l AE09
  \l AE0A
  \l AE0B
  \l AE0C
  \l AE0D
  \l AE0E
  \l AE0F
  \l AE10
  \l AE11
  \l AE12
  \l AE13
  \l AE14
  \l AE15
  \l AE16
  \l AE17
  \l AE18
  \l AE19
  \l AE1A
  \l AE1B
  \l AE1C
  \l AE1D
  \l AE1E
  \l AE1F
  \l AE20
  \l AE21
  \l AE22
  \l AE23
  \l AE24
  \l AE25
  \l AE26
  \l AE27
  \l AE28
  \l AE29
  \l AE2A
  \l AE2B
  \l AE2C
  \l AE2D
  \l AE2E
  \l AE2F
  \l AE30
  \l AE31
  \l AE32
  \l AE33
  \l AE34
  \l AE35
  \l AE36
  \l AE37
  \l AE38
  \l AE39
  \l AE3A
  \l AE3B
  \l AE3C
  \l AE3D
  \l AE3E
  \l AE3F
  \l AE40
  \l AE41
  \l AE42
  \l AE43
  \l AE44
  \l AE45
  \l AE46
  \l AE47
  \l AE48
  \l AE49
  \l AE4A
  \l AE4B
  \l AE4C
  \l AE4D
  \l AE4E
  \l AE4F
  \l AE50
  \l AE51
  \l AE52
  \l AE53
  \l AE54
  \l AE55
  \l AE56
  \l AE57
  \l AE58
  \l AE59
  \l AE5A
  \l AE5B
  \l AE5C
  \l AE5D
  \l AE5E
  \l AE5F
  \l AE60
  \l AE61
  \l AE62
  \l AE63
  \l AE64
  \l AE65
  \l AE66
  \l AE67
  \l AE68
  \l AE69
  \l AE6A
  \l AE6B
  \l AE6C
  \l AE6D
  \l AE6E
  \l AE6F
  \l AE70
  \l AE71
  \l AE72
  \l AE73
  \l AE74
  \l AE75
  \l AE76
  \l AE77
  \l AE78
  \l AE79
  \l AE7A
  \l AE7B
  \l AE7C
  \l AE7D
  \l AE7E
  \l AE7F
  \l AE80
  \l AE81
  \l AE82
  \l AE83
  \l AE84
  \l AE85
  \l AE86
  \l AE87
  \l AE88
  \l AE89
  \l AE8A
  \l AE8B
  \l AE8C
  \l AE8D
  \l AE8E
  \l AE8F
  \l AE90
  \l AE91
  \l AE92
  \l AE93
  \l AE94
  \l AE95
  \l AE96
  \l AE97
  \l AE98
  \l AE99
  \l AE9A
  \l AE9B
  \l AE9C
  \l AE9D
  \l AE9E
  \l AE9F
  \l AEA0
  \l AEA1
  \l AEA2
  \l AEA3
  \l AEA4
  \l AEA5
  \l AEA6
  \l AEA7
  \l AEA8
  \l AEA9
  \l AEAA
  \l AEAB
  \l AEAC
  \l AEAD
  \l AEAE
  \l AEAF
  \l AEB0
  \l AEB1
  \l AEB2
  \l AEB3
  \l AEB4
  \l AEB5
  \l AEB6
  \l AEB7
  \l AEB8
  \l AEB9
  \l AEBA
  \l AEBB
  \l AEBC
  \l AEBD
  \l AEBE
  \l AEBF
  \l AEC0
  \l AEC1
  \l AEC2
  \l AEC3
  \l AEC4
  \l AEC5
  \l AEC6
  \l AEC7
  \l AEC8
  \l AEC9
  \l AECA
  \l AECB
  \l AECC
  \l AECD
  \l AECE
  \l AECF
  \l AED0
  \l AED1
  \l AED2
  \l AED3
  \l AED4
  \l AED5
  \l AED6
  \l AED7
  \l AED8
  \l AED9
  \l AEDA
  \l AEDB
  \l AEDC
  \l AEDD
  \l AEDE
  \l AEDF
  \l AEE0
  \l AEE1
  \l AEE2
  \l AEE3
  \l AEE4
  \l AEE5
  \l AEE6
  \l AEE7
  \l AEE8
  \l AEE9
  \l AEEA
  \l AEEB
  \l AEEC
  \l AEED
  \l AEEE
  \l AEEF
  \l AEF0
  \l AEF1
  \l AEF2
  \l AEF3
  \l AEF4
  \l AEF5
  \l AEF6
  \l AEF7
  \l AEF8
  \l AEF9
  \l AEFA
  \l AEFB
  \l AEFC
  \l AEFD
  \l AEFE
  \l AEFF
  \l AF00
  \l AF01
  \l AF02
  \l AF03
  \l AF04
  \l AF05
  \l AF06
  \l AF07
  \l AF08
  \l AF09
  \l AF0A
  \l AF0B
  \l AF0C
  \l AF0D
  \l AF0E
  \l AF0F
  \l AF10
  \l AF11
  \l AF12
  \l AF13
  \l AF14
  \l AF15
  \l AF16
  \l AF17
  \l AF18
  \l AF19
  \l AF1A
  \l AF1B
  \l AF1C
  \l AF1D
  \l AF1E
  \l AF1F
  \l AF20
  \l AF21
  \l AF22
  \l AF23
  \l AF24
  \l AF25
  \l AF26
  \l AF27
  \l AF28
  \l AF29
  \l AF2A
  \l AF2B
  \l AF2C
  \l AF2D
  \l AF2E
  \l AF2F
  \l AF30
  \l AF31
  \l AF32
  \l AF33
  \l AF34
  \l AF35
  \l AF36
  \l AF37
  \l AF38
  \l AF39
  \l AF3A
  \l AF3B
  \l AF3C
  \l AF3D
  \l AF3E
  \l AF3F
  \l AF40
  \l AF41
  \l AF42
  \l AF43
  \l AF44
  \l AF45
  \l AF46
  \l AF47
  \l AF48
  \l AF49
  \l AF4A
  \l AF4B
  \l AF4C
  \l AF4D
  \l AF4E
  \l AF4F
  \l AF50
  \l AF51
  \l AF52
  \l AF53
  \l AF54
  \l AF55
  \l AF56
  \l AF57
  \l AF58
  \l AF59
  \l AF5A
  \l AF5B
  \l AF5C
  \l AF5D
  \l AF5E
  \l AF5F
  \l AF60
  \l AF61
  \l AF62
  \l AF63
  \l AF64
  \l AF65
  \l AF66
  \l AF67
  \l AF68
  \l AF69
  \l AF6A
  \l AF6B
  \l AF6C
  \l AF6D
  \l AF6E
  \l AF6F
  \l AF70
  \l AF71
  \l AF72
  \l AF73
  \l AF74
  \l AF75
  \l AF76
  \l AF77
  \l AF78
  \l AF79
  \l AF7A
  \l AF7B
  \l AF7C
  \l AF7D
  \l AF7E
  \l AF7F
  \l AF80
  \l AF81
  \l AF82
  \l AF83
  \l AF84
  \l AF85
  \l AF86
  \l AF87
  \l AF88
  \l AF89
  \l AF8A
  \l AF8B
  \l AF8C
  \l AF8D
  \l AF8E
  \l AF8F
  \l AF90
  \l AF91
  \l AF92
  \l AF93
  \l AF94
  \l AF95
  \l AF96
  \l AF97
  \l AF98
  \l AF99
  \l AF9A
  \l AF9B
  \l AF9C
  \l AF9D
  \l AF9E
  \l AF9F
  \l AFA0
  \l AFA1
  \l AFA2
  \l AFA3
  \l AFA4
  \l AFA5
  \l AFA6
  \l AFA7
  \l AFA8
  \l AFA9
  \l AFAA
  \l AFAB
  \l AFAC
  \l AFAD
  \l AFAE
  \l AFAF
  \l AFB0
  \l AFB1
  \l AFB2
  \l AFB3
  \l AFB4
  \l AFB5
  \l AFB6
  \l AFB7
  \l AFB8
  \l AFB9
  \l AFBA
  \l AFBB
  \l AFBC
  \l AFBD
  \l AFBE
  \l AFBF
  \l AFC0
  \l AFC1
  \l AFC2
  \l AFC3
  \l AFC4
  \l AFC5
  \l AFC6
  \l AFC7
  \l AFC8
  \l AFC9
  \l AFCA
  \l AFCB
  \l AFCC
  \l AFCD
  \l AFCE
  \l AFCF
  \l AFD0
  \l AFD1
  \l AFD2
  \l AFD3
  \l AFD4
  \l AFD5
  \l AFD6
  \l AFD7
  \l AFD8
  \l AFD9
  \l AFDA
  \l AFDB
  \l AFDC
  \l AFDD
  \l AFDE
  \l AFDF
  \l AFE0
  \l AFE1
  \l AFE2
  \l AFE3
  \l AFE4
  \l AFE5
  \l AFE6
  \l AFE7
  \l AFE8
  \l AFE9
  \l AFEA
  \l AFEB
  \l AFEC
  \l AFED
  \l AFEE
  \l AFEF
  \l AFF0
  \l AFF1
  \l AFF2
  \l AFF3
  \l AFF4
  \l AFF5
  \l AFF6
  \l AFF7
  \l AFF8
  \l AFF9
  \l AFFA
  \l AFFB
  \l AFFC
  \l AFFD
  \l AFFE
  \l AFFF
  \l B000
  \l B001
  \l B002
  \l B003
  \l B004
  \l B005
  \l B006
  \l B007
  \l B008
  \l B009
  \l B00A
  \l B00B
  \l B00C
  \l B00D
  \l B00E
  \l B00F
  \l B010
  \l B011
  \l B012
  \l B013
  \l B014
  \l B015
  \l B016
  \l B017
  \l B018
  \l B019
  \l B01A
  \l B01B
  \l B01C
  \l B01D
  \l B01E
  \l B01F
  \l B020
  \l B021
  \l B022
  \l B023
  \l B024
  \l B025
  \l B026
  \l B027
  \l B028
  \l B029
  \l B02A
  \l B02B
  \l B02C
  \l B02D
  \l B02E
  \l B02F
  \l B030
  \l B031
  \l B032
  \l B033
  \l B034
  \l B035
  \l B036
  \l B037
  \l B038
  \l B039
  \l B03A
  \l B03B
  \l B03C
  \l B03D
  \l B03E
  \l B03F
  \l B040
  \l B041
  \l B042
  \l B043
  \l B044
  \l B045
  \l B046
  \l B047
  \l B048
  \l B049
  \l B04A
  \l B04B
  \l B04C
  \l B04D
  \l B04E
  \l B04F
  \l B050
  \l B051
  \l B052
  \l B053
  \l B054
  \l B055
  \l B056
  \l B057
  \l B058
  \l B059
  \l B05A
  \l B05B
  \l B05C
  \l B05D
  \l B05E
  \l B05F
  \l B060
  \l B061
  \l B062
  \l B063
  \l B064
  \l B065
  \l B066
  \l B067
  \l B068
  \l B069
  \l B06A
  \l B06B
  \l B06C
  \l B06D
  \l B06E
  \l B06F
  \l B070
  \l B071
  \l B072
  \l B073
  \l B074
  \l B075
  \l B076
  \l B077
  \l B078
  \l B079
  \l B07A
  \l B07B
  \l B07C
  \l B07D
  \l B07E
  \l B07F
  \l B080
  \l B081
  \l B082
  \l B083
  \l B084
  \l B085
  \l B086
  \l B087
  \l B088
  \l B089
  \l B08A
  \l B08B
  \l B08C
  \l B08D
  \l B08E
  \l B08F
  \l B090
  \l B091
  \l B092
  \l B093
  \l B094
  \l B095
  \l B096
  \l B097
  \l B098
  \l B099
  \l B09A
  \l B09B
  \l B09C
  \l B09D
  \l B09E
  \l B09F
  \l B0A0
  \l B0A1
  \l B0A2
  \l B0A3
  \l B0A4
  \l B0A5
  \l B0A6
  \l B0A7
  \l B0A8
  \l B0A9
  \l B0AA
  \l B0AB
  \l B0AC
  \l B0AD
  \l B0AE
  \l B0AF
  \l B0B0
  \l B0B1
  \l B0B2
  \l B0B3
  \l B0B4
  \l B0B5
  \l B0B6
  \l B0B7
  \l B0B8
  \l B0B9
  \l B0BA
  \l B0BB
  \l B0BC
  \l B0BD
  \l B0BE
  \l B0BF
  \l B0C0
  \l B0C1
  \l B0C2
  \l B0C3
  \l B0C4
  \l B0C5
  \l B0C6
  \l B0C7
  \l B0C8
  \l B0C9
  \l B0CA
  \l B0CB
  \l B0CC
  \l B0CD
  \l B0CE
  \l B0CF
  \l B0D0
  \l B0D1
  \l B0D2
  \l B0D3
  \l B0D4
  \l B0D5
  \l B0D6
  \l B0D7
  \l B0D8
  \l B0D9
  \l B0DA
  \l B0DB
  \l B0DC
  \l B0DD
  \l B0DE
  \l B0DF
  \l B0E0
  \l B0E1
  \l B0E2
  \l B0E3
  \l B0E4
  \l B0E5
  \l B0E6
  \l B0E7
  \l B0E8
  \l B0E9
  \l B0EA
  \l B0EB
  \l B0EC
  \l B0ED
  \l B0EE
  \l B0EF
  \l B0F0
  \l B0F1
  \l B0F2
  \l B0F3
  \l B0F4
  \l B0F5
  \l B0F6
  \l B0F7
  \l B0F8
  \l B0F9
  \l B0FA
  \l B0FB
  \l B0FC
  \l B0FD
  \l B0FE
  \l B0FF
  \l B100
  \l B101
  \l B102
  \l B103
  \l B104
  \l B105
  \l B106
  \l B107
  \l B108
  \l B109
  \l B10A
  \l B10B
  \l B10C
  \l B10D
  \l B10E
  \l B10F
  \l B110
  \l B111
  \l B112
  \l B113
  \l B114
  \l B115
  \l B116
  \l B117
  \l B118
  \l B119
  \l B11A
  \l B11B
  \l B11C
  \l B11D
  \l B11E
  \l B11F
  \l B120
  \l B121
  \l B122
  \l B123
  \l B124
  \l B125
  \l B126
  \l B127
  \l B128
  \l B129
  \l B12A
  \l B12B
  \l B12C
  \l B12D
  \l B12E
  \l B12F
  \l B130
  \l B131
  \l B132
  \l B133
  \l B134
  \l B135
  \l B136
  \l B137
  \l B138
  \l B139
  \l B13A
  \l B13B
  \l B13C
  \l B13D
  \l B13E
  \l B13F
  \l B140
  \l B141
  \l B142
  \l B143
  \l B144
  \l B145
  \l B146
  \l B147
  \l B148
  \l B149
  \l B14A
  \l B14B
  \l B14C
  \l B14D
  \l B14E
  \l B14F
  \l B150
  \l B151
  \l B152
  \l B153
  \l B154
  \l B155
  \l B156
  \l B157
  \l B158
  \l B159
  \l B15A
  \l B15B
  \l B15C
  \l B15D
  \l B15E
  \l B15F
  \l B160
  \l B161
  \l B162
  \l B163
  \l B164
  \l B165
  \l B166
  \l B167
  \l B168
  \l B169
  \l B16A
  \l B16B
  \l B16C
  \l B16D
  \l B16E
  \l B16F
  \l B170
  \l B171
  \l B172
  \l B173
  \l B174
  \l B175
  \l B176
  \l B177
  \l B178
  \l B179
  \l B17A
  \l B17B
  \l B17C
  \l B17D
  \l B17E
  \l B17F
  \l B180
  \l B181
  \l B182
  \l B183
  \l B184
  \l B185
  \l B186
  \l B187
  \l B188
  \l B189
  \l B18A
  \l B18B
  \l B18C
  \l B18D
  \l B18E
  \l B18F
  \l B190
  \l B191
  \l B192
  \l B193
  \l B194
  \l B195
  \l B196
  \l B197
  \l B198
  \l B199
  \l B19A
  \l B19B
  \l B19C
  \l B19D
  \l B19E
  \l B19F
  \l B1A0
  \l B1A1
  \l B1A2
  \l B1A3
  \l B1A4
  \l B1A5
  \l B1A6
  \l B1A7
  \l B1A8
  \l B1A9
  \l B1AA
  \l B1AB
  \l B1AC
  \l B1AD
  \l B1AE
  \l B1AF
  \l B1B0
  \l B1B1
  \l B1B2
  \l B1B3
  \l B1B4
  \l B1B5
  \l B1B6
  \l B1B7
  \l B1B8
  \l B1B9
  \l B1BA
  \l B1BB
  \l B1BC
  \l B1BD
  \l B1BE
  \l B1BF
  \l B1C0
  \l B1C1
  \l B1C2
  \l B1C3
  \l B1C4
  \l B1C5
  \l B1C6
  \l B1C7
  \l B1C8
  \l B1C9
  \l B1CA
  \l B1CB
  \l B1CC
  \l B1CD
  \l B1CE
  \l B1CF
  \l B1D0
  \l B1D1
  \l B1D2
  \l B1D3
  \l B1D4
  \l B1D5
  \l B1D6
  \l B1D7
  \l B1D8
  \l B1D9
  \l B1DA
  \l B1DB
  \l B1DC
  \l B1DD
  \l B1DE
  \l B1DF
  \l B1E0
  \l B1E1
  \l B1E2
  \l B1E3
  \l B1E4
  \l B1E5
  \l B1E6
  \l B1E7
  \l B1E8
  \l B1E9
  \l B1EA
  \l B1EB
  \l B1EC
  \l B1ED
  \l B1EE
  \l B1EF
  \l B1F0
  \l B1F1
  \l B1F2
  \l B1F3
  \l B1F4
  \l B1F5
  \l B1F6
  \l B1F7
  \l B1F8
  \l B1F9
  \l B1FA
  \l B1FB
  \l B1FC
  \l B1FD
  \l B1FE
  \l B1FF
  \l B200
  \l B201
  \l B202
  \l B203
  \l B204
  \l B205
  \l B206
  \l B207
  \l B208
  \l B209
  \l B20A
  \l B20B
  \l B20C
  \l B20D
  \l B20E
  \l B20F
  \l B210
  \l B211
  \l B212
  \l B213
  \l B214
  \l B215
  \l B216
  \l B217
  \l B218
  \l B219
  \l B21A
  \l B21B
  \l B21C
  \l B21D
  \l B21E
  \l B21F
  \l B220
  \l B221
  \l B222
  \l B223
  \l B224
  \l B225
  \l B226
  \l B227
  \l B228
  \l B229
  \l B22A
  \l B22B
  \l B22C
  \l B22D
  \l B22E
  \l B22F
  \l B230
  \l B231
  \l B232
  \l B233
  \l B234
  \l B235
  \l B236
  \l B237
  \l B238
  \l B239
  \l B23A
  \l B23B
  \l B23C
  \l B23D
  \l B23E
  \l B23F
  \l B240
  \l B241
  \l B242
  \l B243
  \l B244
  \l B245
  \l B246
  \l B247
  \l B248
  \l B249
  \l B24A
  \l B24B
  \l B24C
  \l B24D
  \l B24E
  \l B24F
  \l B250
  \l B251
  \l B252
  \l B253
  \l B254
  \l B255
  \l B256
  \l B257
  \l B258
  \l B259
  \l B25A
  \l B25B
  \l B25C
  \l B25D
  \l B25E
  \l B25F
  \l B260
  \l B261
  \l B262
  \l B263
  \l B264
  \l B265
  \l B266
  \l B267
  \l B268
  \l B269
  \l B26A
  \l B26B
  \l B26C
  \l B26D
  \l B26E
  \l B26F
  \l B270
  \l B271
  \l B272
  \l B273
  \l B274
  \l B275
  \l B276
  \l B277
  \l B278
  \l B279
  \l B27A
  \l B27B
  \l B27C
  \l B27D
  \l B27E
  \l B27F
  \l B280
  \l B281
  \l B282
  \l B283
  \l B284
  \l B285
  \l B286
  \l B287
  \l B288
  \l B289
  \l B28A
  \l B28B
  \l B28C
  \l B28D
  \l B28E
  \l B28F
  \l B290
  \l B291
  \l B292
  \l B293
  \l B294
  \l B295
  \l B296
  \l B297
  \l B298
  \l B299
  \l B29A
  \l B29B
  \l B29C
  \l B29D
  \l B29E
  \l B29F
  \l B2A0
  \l B2A1
  \l B2A2
  \l B2A3
  \l B2A4
  \l B2A5
  \l B2A6
  \l B2A7
  \l B2A8
  \l B2A9
  \l B2AA
  \l B2AB
  \l B2AC
  \l B2AD
  \l B2AE
  \l B2AF
  \l B2B0
  \l B2B1
  \l B2B2
  \l B2B3
  \l B2B4
  \l B2B5
  \l B2B6
  \l B2B7
  \l B2B8
  \l B2B9
  \l B2BA
  \l B2BB
  \l B2BC
  \l B2BD
  \l B2BE
  \l B2BF
  \l B2C0
  \l B2C1
  \l B2C2
  \l B2C3
  \l B2C4
  \l B2C5
  \l B2C6
  \l B2C7
  \l B2C8
  \l B2C9
  \l B2CA
  \l B2CB
  \l B2CC
  \l B2CD
  \l B2CE
  \l B2CF
  \l B2D0
  \l B2D1
  \l B2D2
  \l B2D3
  \l B2D4
  \l B2D5
  \l B2D6
  \l B2D7
  \l B2D8
  \l B2D9
  \l B2DA
  \l B2DB
  \l B2DC
  \l B2DD
  \l B2DE
  \l B2DF
  \l B2E0
  \l B2E1
  \l B2E2
  \l B2E3
  \l B2E4
  \l B2E5
  \l B2E6
  \l B2E7
  \l B2E8
  \l B2E9
  \l B2EA
  \l B2EB
  \l B2EC
  \l B2ED
  \l B2EE
  \l B2EF
  \l B2F0
  \l B2F1
  \l B2F2
  \l B2F3
  \l B2F4
  \l B2F5
  \l B2F6
  \l B2F7
  \l B2F8
  \l B2F9
  \l B2FA
  \l B2FB
  \l B2FC
  \l B2FD
  \l B2FE
  \l B2FF
  \l B300
  \l B301
  \l B302
  \l B303
  \l B304
  \l B305
  \l B306
  \l B307
  \l B308
  \l B309
  \l B30A
  \l B30B
  \l B30C
  \l B30D
  \l B30E
  \l B30F
  \l B310
  \l B311
  \l B312
  \l B313
  \l B314
  \l B315
  \l B316
  \l B317
  \l B318
  \l B319
  \l B31A
  \l B31B
  \l B31C
  \l B31D
  \l B31E
  \l B31F
  \l B320
  \l B321
  \l B322
  \l B323
  \l B324
  \l B325
  \l B326
  \l B327
  \l B328
  \l B329
  \l B32A
  \l B32B
  \l B32C
  \l B32D
  \l B32E
  \l B32F
  \l B330
  \l B331
  \l B332
  \l B333
  \l B334
  \l B335
  \l B336
  \l B337
  \l B338
  \l B339
  \l B33A
  \l B33B
  \l B33C
  \l B33D
  \l B33E
  \l B33F
  \l B340
  \l B341
  \l B342
  \l B343
  \l B344
  \l B345
  \l B346
  \l B347
  \l B348
  \l B349
  \l B34A
  \l B34B
  \l B34C
  \l B34D
  \l B34E
  \l B34F
  \l B350
  \l B351
  \l B352
  \l B353
  \l B354
  \l B355
  \l B356
  \l B357
  \l B358
  \l B359
  \l B35A
  \l B35B
  \l B35C
  \l B35D
  \l B35E
  \l B35F
  \l B360
  \l B361
  \l B362
  \l B363
  \l B364
  \l B365
  \l B366
  \l B367
  \l B368
  \l B369
  \l B36A
  \l B36B
  \l B36C
  \l B36D
  \l B36E
  \l B36F
  \l B370
  \l B371
  \l B372
  \l B373
  \l B374
  \l B375
  \l B376
  \l B377
  \l B378
  \l B379
  \l B37A
  \l B37B
  \l B37C
  \l B37D
  \l B37E
  \l B37F
  \l B380
  \l B381
  \l B382
  \l B383
  \l B384
  \l B385
  \l B386
  \l B387
  \l B388
  \l B389
  \l B38A
  \l B38B
  \l B38C
  \l B38D
  \l B38E
  \l B38F
  \l B390
  \l B391
  \l B392
  \l B393
  \l B394
  \l B395
  \l B396
  \l B397
  \l B398
  \l B399
  \l B39A
  \l B39B
  \l B39C
  \l B39D
  \l B39E
  \l B39F
  \l B3A0
  \l B3A1
  \l B3A2
  \l B3A3
  \l B3A4
  \l B3A5
  \l B3A6
  \l B3A7
  \l B3A8
  \l B3A9
  \l B3AA
  \l B3AB
  \l B3AC
  \l B3AD
  \l B3AE
  \l B3AF
  \l B3B0
  \l B3B1
  \l B3B2
  \l B3B3
  \l B3B4
  \l B3B5
  \l B3B6
  \l B3B7
  \l B3B8
  \l B3B9
  \l B3BA
  \l B3BB
  \l B3BC
  \l B3BD
  \l B3BE
  \l B3BF
  \l B3C0
  \l B3C1
  \l B3C2
  \l B3C3
  \l B3C4
  \l B3C5
  \l B3C6
  \l B3C7
  \l B3C8
  \l B3C9
  \l B3CA
  \l B3CB
  \l B3CC
  \l B3CD
  \l B3CE
  \l B3CF
  \l B3D0
  \l B3D1
  \l B3D2
  \l B3D3
  \l B3D4
  \l B3D5
  \l B3D6
  \l B3D7
  \l B3D8
  \l B3D9
  \l B3DA
  \l B3DB
  \l B3DC
  \l B3DD
  \l B3DE
  \l B3DF
  \l B3E0
  \l B3E1
  \l B3E2
  \l B3E3
  \l B3E4
  \l B3E5
  \l B3E6
  \l B3E7
  \l B3E8
  \l B3E9
  \l B3EA
  \l B3EB
  \l B3EC
  \l B3ED
  \l B3EE
  \l B3EF
  \l B3F0
  \l B3F1
  \l B3F2
  \l B3F3
  \l B3F4
  \l B3F5
  \l B3F6
  \l B3F7
  \l B3F8
  \l B3F9
  \l B3FA
  \l B3FB
  \l B3FC
  \l B3FD
  \l B3FE
  \l B3FF
  \l B400
  \l B401
  \l B402
  \l B403
  \l B404
  \l B405
  \l B406
  \l B407
  \l B408
  \l B409
  \l B40A
  \l B40B
  \l B40C
  \l B40D
  \l B40E
  \l B40F
  \l B410
  \l B411
  \l B412
  \l B413
  \l B414
  \l B415
  \l B416
  \l B417
  \l B418
  \l B419
  \l B41A
  \l B41B
  \l B41C
  \l B41D
  \l B41E
  \l B41F
  \l B420
  \l B421
  \l B422
  \l B423
  \l B424
  \l B425
  \l B426
  \l B427
  \l B428
  \l B429
  \l B42A
  \l B42B
  \l B42C
  \l B42D
  \l B42E
  \l B42F
  \l B430
  \l B431
  \l B432
  \l B433
  \l B434
  \l B435
  \l B436
  \l B437
  \l B438
  \l B439
  \l B43A
  \l B43B
  \l B43C
  \l B43D
  \l B43E
  \l B43F
  \l B440
  \l B441
  \l B442
  \l B443
  \l B444
  \l B445
  \l B446
  \l B447
  \l B448
  \l B449
  \l B44A
  \l B44B
  \l B44C
  \l B44D
  \l B44E
  \l B44F
  \l B450
  \l B451
  \l B452
  \l B453
  \l B454
  \l B455
  \l B456
  \l B457
  \l B458
  \l B459
  \l B45A
  \l B45B
  \l B45C
  \l B45D
  \l B45E
  \l B45F
  \l B460
  \l B461
  \l B462
  \l B463
  \l B464
  \l B465
  \l B466
  \l B467
  \l B468
  \l B469
  \l B46A
  \l B46B
  \l B46C
  \l B46D
  \l B46E
  \l B46F
  \l B470
  \l B471
  \l B472
  \l B473
  \l B474
  \l B475
  \l B476
  \l B477
  \l B478
  \l B479
  \l B47A
  \l B47B
  \l B47C
  \l B47D
  \l B47E
  \l B47F
  \l B480
  \l B481
  \l B482
  \l B483
  \l B484
  \l B485
  \l B486
  \l B487
  \l B488
  \l B489
  \l B48A
  \l B48B
  \l B48C
  \l B48D
  \l B48E
  \l B48F
  \l B490
  \l B491
  \l B492
  \l B493
  \l B494
  \l B495
  \l B496
  \l B497
  \l B498
  \l B499
  \l B49A
  \l B49B
  \l B49C
  \l B49D
  \l B49E
  \l B49F
  \l B4A0
  \l B4A1
  \l B4A2
  \l B4A3
  \l B4A4
  \l B4A5
  \l B4A6
  \l B4A7
  \l B4A8
  \l B4A9
  \l B4AA
  \l B4AB
  \l B4AC
  \l B4AD
  \l B4AE
  \l B4AF
  \l B4B0
  \l B4B1
  \l B4B2
  \l B4B3
  \l B4B4
  \l B4B5
  \l B4B6
  \l B4B7
  \l B4B8
  \l B4B9
  \l B4BA
  \l B4BB
  \l B4BC
  \l B4BD
  \l B4BE
  \l B4BF
  \l B4C0
  \l B4C1
  \l B4C2
  \l B4C3
  \l B4C4
  \l B4C5
  \l B4C6
  \l B4C7
  \l B4C8
  \l B4C9
  \l B4CA
  \l B4CB
  \l B4CC
  \l B4CD
  \l B4CE
  \l B4CF
  \l B4D0
  \l B4D1
  \l B4D2
  \l B4D3
  \l B4D4
  \l B4D5
  \l B4D6
  \l B4D7
  \l B4D8
  \l B4D9
  \l B4DA
  \l B4DB
  \l B4DC
  \l B4DD
  \l B4DE
  \l B4DF
  \l B4E0
  \l B4E1
  \l B4E2
  \l B4E3
  \l B4E4
  \l B4E5
  \l B4E6
  \l B4E7
  \l B4E8
  \l B4E9
  \l B4EA
  \l B4EB
  \l B4EC
  \l B4ED
  \l B4EE
  \l B4EF
  \l B4F0
  \l B4F1
  \l B4F2
  \l B4F3
  \l B4F4
  \l B4F5
  \l B4F6
  \l B4F7
  \l B4F8
  \l B4F9
  \l B4FA
  \l B4FB
  \l B4FC
  \l B4FD
  \l B4FE
  \l B4FF
  \l B500
  \l B501
  \l B502
  \l B503
  \l B504
  \l B505
  \l B506
  \l B507
  \l B508
  \l B509
  \l B50A
  \l B50B
  \l B50C
  \l B50D
  \l B50E
  \l B50F
  \l B510
  \l B511
  \l B512
  \l B513
  \l B514
  \l B515
  \l B516
  \l B517
  \l B518
  \l B519
  \l B51A
  \l B51B
  \l B51C
  \l B51D
  \l B51E
  \l B51F
  \l B520
  \l B521
  \l B522
  \l B523
  \l B524
  \l B525
  \l B526
  \l B527
  \l B528
  \l B529
  \l B52A
  \l B52B
  \l B52C
  \l B52D
  \l B52E
  \l B52F
  \l B530
  \l B531
  \l B532
  \l B533
  \l B534
  \l B535
  \l B536
  \l B537
  \l B538
  \l B539
  \l B53A
  \l B53B
  \l B53C
  \l B53D
  \l B53E
  \l B53F
  \l B540
  \l B541
  \l B542
  \l B543
  \l B544
  \l B545
  \l B546
  \l B547
  \l B548
  \l B549
  \l B54A
  \l B54B
  \l B54C
  \l B54D
  \l B54E
  \l B54F
  \l B550
  \l B551
  \l B552
  \l B553
  \l B554
  \l B555
  \l B556
  \l B557
  \l B558
  \l B559
  \l B55A
  \l B55B
  \l B55C
  \l B55D
  \l B55E
  \l B55F
  \l B560
  \l B561
  \l B562
  \l B563
  \l B564
  \l B565
  \l B566
  \l B567
  \l B568
  \l B569
  \l B56A
  \l B56B
  \l B56C
  \l B56D
  \l B56E
  \l B56F
  \l B570
  \l B571
  \l B572
  \l B573
  \l B574
  \l B575
  \l B576
  \l B577
  \l B578
  \l B579
  \l B57A
  \l B57B
  \l B57C
  \l B57D
  \l B57E
  \l B57F
  \l B580
  \l B581
  \l B582
  \l B583
  \l B584
  \l B585
  \l B586
  \l B587
  \l B588
  \l B589
  \l B58A
  \l B58B
  \l B58C
  \l B58D
  \l B58E
  \l B58F
  \l B590
  \l B591
  \l B592
  \l B593
  \l B594
  \l B595
  \l B596
  \l B597
  \l B598
  \l B599
  \l B59A
  \l B59B
  \l B59C
  \l B59D
  \l B59E
  \l B59F
  \l B5A0
  \l B5A1
  \l B5A2
  \l B5A3
  \l B5A4
  \l B5A5
  \l B5A6
  \l B5A7
  \l B5A8
  \l B5A9
  \l B5AA
  \l B5AB
  \l B5AC
  \l B5AD
  \l B5AE
  \l B5AF
  \l B5B0
  \l B5B1
  \l B5B2
  \l B5B3
  \l B5B4
  \l B5B5
  \l B5B6
  \l B5B7
  \l B5B8
  \l B5B9
  \l B5BA
  \l B5BB
  \l B5BC
  \l B5BD
  \l B5BE
  \l B5BF
  \l B5C0
  \l B5C1
  \l B5C2
  \l B5C3
  \l B5C4
  \l B5C5
  \l B5C6
  \l B5C7
  \l B5C8
  \l B5C9
  \l B5CA
  \l B5CB
  \l B5CC
  \l B5CD
  \l B5CE
  \l B5CF
  \l B5D0
  \l B5D1
  \l B5D2
  \l B5D3
  \l B5D4
  \l B5D5
  \l B5D6
  \l B5D7
  \l B5D8
  \l B5D9
  \l B5DA
  \l B5DB
  \l B5DC
  \l B5DD
  \l B5DE
  \l B5DF
  \l B5E0
  \l B5E1
  \l B5E2
  \l B5E3
  \l B5E4
  \l B5E5
  \l B5E6
  \l B5E7
  \l B5E8
  \l B5E9
  \l B5EA
  \l B5EB
  \l B5EC
  \l B5ED
  \l B5EE
  \l B5EF
  \l B5F0
  \l B5F1
  \l B5F2
  \l B5F3
  \l B5F4
  \l B5F5
  \l B5F6
  \l B5F7
  \l B5F8
  \l B5F9
  \l B5FA
  \l B5FB
  \l B5FC
  \l B5FD
  \l B5FE
  \l B5FF
  \l B600
  \l B601
  \l B602
  \l B603
  \l B604
  \l B605
  \l B606
  \l B607
  \l B608
  \l B609
  \l B60A
  \l B60B
  \l B60C
  \l B60D
  \l B60E
  \l B60F
  \l B610
  \l B611
  \l B612
  \l B613
  \l B614
  \l B615
  \l B616
  \l B617
  \l B618
  \l B619
  \l B61A
  \l B61B
  \l B61C
  \l B61D
  \l B61E
  \l B61F
  \l B620
  \l B621
  \l B622
  \l B623
  \l B624
  \l B625
  \l B626
  \l B627
  \l B628
  \l B629
  \l B62A
  \l B62B
  \l B62C
  \l B62D
  \l B62E
  \l B62F
  \l B630
  \l B631
  \l B632
  \l B633
  \l B634
  \l B635
  \l B636
  \l B637
  \l B638
  \l B639
  \l B63A
  \l B63B
  \l B63C
  \l B63D
  \l B63E
  \l B63F
  \l B640
  \l B641
  \l B642
  \l B643
  \l B644
  \l B645
  \l B646
  \l B647
  \l B648
  \l B649
  \l B64A
  \l B64B
  \l B64C
  \l B64D
  \l B64E
  \l B64F
  \l B650
  \l B651
  \l B652
  \l B653
  \l B654
  \l B655
  \l B656
  \l B657
  \l B658
  \l B659
  \l B65A
  \l B65B
  \l B65C
  \l B65D
  \l B65E
  \l B65F
  \l B660
  \l B661
  \l B662
  \l B663
  \l B664
  \l B665
  \l B666
  \l B667
  \l B668
  \l B669
  \l B66A
  \l B66B
  \l B66C
  \l B66D
  \l B66E
  \l B66F
  \l B670
  \l B671
  \l B672
  \l B673
  \l B674
  \l B675
  \l B676
  \l B677
  \l B678
  \l B679
  \l B67A
  \l B67B
  \l B67C
  \l B67D
  \l B67E
  \l B67F
  \l B680
  \l B681
  \l B682
  \l B683
  \l B684
  \l B685
  \l B686
  \l B687
  \l B688
  \l B689
  \l B68A
  \l B68B
  \l B68C
  \l B68D
  \l B68E
  \l B68F
  \l B690
  \l B691
  \l B692
  \l B693
  \l B694
  \l B695
  \l B696
  \l B697
  \l B698
  \l B699
  \l B69A
  \l B69B
  \l B69C
  \l B69D
  \l B69E
  \l B69F
  \l B6A0
  \l B6A1
  \l B6A2
  \l B6A3
  \l B6A4
  \l B6A5
  \l B6A6
  \l B6A7
  \l B6A8
  \l B6A9
  \l B6AA
  \l B6AB
  \l B6AC
  \l B6AD
  \l B6AE
  \l B6AF
  \l B6B0
  \l B6B1
  \l B6B2
  \l B6B3
  \l B6B4
  \l B6B5
  \l B6B6
  \l B6B7
  \l B6B8
  \l B6B9
  \l B6BA
  \l B6BB
  \l B6BC
  \l B6BD
  \l B6BE
  \l B6BF
  \l B6C0
  \l B6C1
  \l B6C2
  \l B6C3
  \l B6C4
  \l B6C5
  \l B6C6
  \l B6C7
  \l B6C8
  \l B6C9
  \l B6CA
  \l B6CB
  \l B6CC
  \l B6CD
  \l B6CE
  \l B6CF
  \l B6D0
  \l B6D1
  \l B6D2
  \l B6D3
  \l B6D4
  \l B6D5
  \l B6D6
  \l B6D7
  \l B6D8
  \l B6D9
  \l B6DA
  \l B6DB
  \l B6DC
  \l B6DD
  \l B6DE
  \l B6DF
  \l B6E0
  \l B6E1
  \l B6E2
  \l B6E3
  \l B6E4
  \l B6E5
  \l B6E6
  \l B6E7
  \l B6E8
  \l B6E9
  \l B6EA
  \l B6EB
  \l B6EC
  \l B6ED
  \l B6EE
  \l B6EF
  \l B6F0
  \l B6F1
  \l B6F2
  \l B6F3
  \l B6F4
  \l B6F5
  \l B6F6
  \l B6F7
  \l B6F8
  \l B6F9
  \l B6FA
  \l B6FB
  \l B6FC
  \l B6FD
  \l B6FE
  \l B6FF
  \l B700
  \l B701
  \l B702
  \l B703
  \l B704
  \l B705
  \l B706
  \l B707
  \l B708
  \l B709
  \l B70A
  \l B70B
  \l B70C
  \l B70D
  \l B70E
  \l B70F
  \l B710
  \l B711
  \l B712
  \l B713
  \l B714
  \l B715
  \l B716
  \l B717
  \l B718
  \l B719
  \l B71A
  \l B71B
  \l B71C
  \l B71D
  \l B71E
  \l B71F
  \l B720
  \l B721
  \l B722
  \l B723
  \l B724
  \l B725
  \l B726
  \l B727
  \l B728
  \l B729
  \l B72A
  \l B72B
  \l B72C
  \l B72D
  \l B72E
  \l B72F
  \l B730
  \l B731
  \l B732
  \l B733
  \l B734
  \l B735
  \l B736
  \l B737
  \l B738
  \l B739
  \l B73A
  \l B73B
  \l B73C
  \l B73D
  \l B73E
  \l B73F
  \l B740
  \l B741
  \l B742
  \l B743
  \l B744
  \l B745
  \l B746
  \l B747
  \l B748
  \l B749
  \l B74A
  \l B74B
  \l B74C
  \l B74D
  \l B74E
  \l B74F
  \l B750
  \l B751
  \l B752
  \l B753
  \l B754
  \l B755
  \l B756
  \l B757
  \l B758
  \l B759
  \l B75A
  \l B75B
  \l B75C
  \l B75D
  \l B75E
  \l B75F
  \l B760
  \l B761
  \l B762
  \l B763
  \l B764
  \l B765
  \l B766
  \l B767
  \l B768
  \l B769
  \l B76A
  \l B76B
  \l B76C
  \l B76D
  \l B76E
  \l B76F
  \l B770
  \l B771
  \l B772
  \l B773
  \l B774
  \l B775
  \l B776
  \l B777
  \l B778
  \l B779
  \l B77A
  \l B77B
  \l B77C
  \l B77D
  \l B77E
  \l B77F
  \l B780
  \l B781
  \l B782
  \l B783
  \l B784
  \l B785
  \l B786
  \l B787
  \l B788
  \l B789
  \l B78A
  \l B78B
  \l B78C
  \l B78D
  \l B78E
  \l B78F
  \l B790
  \l B791
  \l B792
  \l B793
  \l B794
  \l B795
  \l B796
  \l B797
  \l B798
  \l B799
  \l B79A
  \l B79B
  \l B79C
  \l B79D
  \l B79E
  \l B79F
  \l B7A0
  \l B7A1
  \l B7A2
  \l B7A3
  \l B7A4
  \l B7A5
  \l B7A6
  \l B7A7
  \l B7A8
  \l B7A9
  \l B7AA
  \l B7AB
  \l B7AC
  \l B7AD
  \l B7AE
  \l B7AF
  \l B7B0
  \l B7B1
  \l B7B2
  \l B7B3
  \l B7B4
  \l B7B5
  \l B7B6
  \l B7B7
  \l B7B8
  \l B7B9
  \l B7BA
  \l B7BB
  \l B7BC
  \l B7BD
  \l B7BE
  \l B7BF
  \l B7C0
  \l B7C1
  \l B7C2
  \l B7C3
  \l B7C4
  \l B7C5
  \l B7C6
  \l B7C7
  \l B7C8
  \l B7C9
  \l B7CA
  \l B7CB
  \l B7CC
  \l B7CD
  \l B7CE
  \l B7CF
  \l B7D0
  \l B7D1
  \l B7D2
  \l B7D3
  \l B7D4
  \l B7D5
  \l B7D6
  \l B7D7
  \l B7D8
  \l B7D9
  \l B7DA
  \l B7DB
  \l B7DC
  \l B7DD
  \l B7DE
  \l B7DF
  \l B7E0
  \l B7E1
  \l B7E2
  \l B7E3
  \l B7E4
  \l B7E5
  \l B7E6
  \l B7E7
  \l B7E8
  \l B7E9
  \l B7EA
  \l B7EB
  \l B7EC
  \l B7ED
  \l B7EE
  \l B7EF
  \l B7F0
  \l B7F1
  \l B7F2
  \l B7F3
  \l B7F4
  \l B7F5
  \l B7F6
  \l B7F7
  \l B7F8
  \l B7F9
  \l B7FA
  \l B7FB
  \l B7FC
  \l B7FD
  \l B7FE
  \l B7FF
  \l B800
  \l B801
  \l B802
  \l B803
  \l B804
  \l B805
  \l B806
  \l B807
  \l B808
  \l B809
  \l B80A
  \l B80B
  \l B80C
  \l B80D
  \l B80E
  \l B80F
  \l B810
  \l B811
  \l B812
  \l B813
  \l B814
  \l B815
  \l B816
  \l B817
  \l B818
  \l B819
  \l B81A
  \l B81B
  \l B81C
  \l B81D
  \l B81E
  \l B81F
  \l B820
  \l B821
  \l B822
  \l B823
  \l B824
  \l B825
  \l B826
  \l B827
  \l B828
  \l B829
  \l B82A
  \l B82B
  \l B82C
  \l B82D
  \l B82E
  \l B82F
  \l B830
  \l B831
  \l B832
  \l B833
  \l B834
  \l B835
  \l B836
  \l B837
  \l B838
  \l B839
  \l B83A
  \l B83B
  \l B83C
  \l B83D
  \l B83E
  \l B83F
  \l B840
  \l B841
  \l B842
  \l B843
  \l B844
  \l B845
  \l B846
  \l B847
  \l B848
  \l B849
  \l B84A
  \l B84B
  \l B84C
  \l B84D
  \l B84E
  \l B84F
  \l B850
  \l B851
  \l B852
  \l B853
  \l B854
  \l B855
  \l B856
  \l B857
  \l B858
  \l B859
  \l B85A
  \l B85B
  \l B85C
  \l B85D
  \l B85E
  \l B85F
  \l B860
  \l B861
  \l B862
  \l B863
  \l B864
  \l B865
  \l B866
  \l B867
  \l B868
  \l B869
  \l B86A
  \l B86B
  \l B86C
  \l B86D
  \l B86E
  \l B86F
  \l B870
  \l B871
  \l B872
  \l B873
  \l B874
  \l B875
  \l B876
  \l B877
  \l B878
  \l B879
  \l B87A
  \l B87B
  \l B87C
  \l B87D
  \l B87E
  \l B87F
  \l B880
  \l B881
  \l B882
  \l B883
  \l B884
  \l B885
  \l B886
  \l B887
  \l B888
  \l B889
  \l B88A
  \l B88B
  \l B88C
  \l B88D
  \l B88E
  \l B88F
  \l B890
  \l B891
  \l B892
  \l B893
  \l B894
  \l B895
  \l B896
  \l B897
  \l B898
  \l B899
  \l B89A
  \l B89B
  \l B89C
  \l B89D
  \l B89E
  \l B89F
  \l B8A0
  \l B8A1
  \l B8A2
  \l B8A3
  \l B8A4
  \l B8A5
  \l B8A6
  \l B8A7
  \l B8A8
  \l B8A9
  \l B8AA
  \l B8AB
  \l B8AC
  \l B8AD
  \l B8AE
  \l B8AF
  \l B8B0
  \l B8B1
  \l B8B2
  \l B8B3
  \l B8B4
  \l B8B5
  \l B8B6
  \l B8B7
  \l B8B8
  \l B8B9
  \l B8BA
  \l B8BB
  \l B8BC
  \l B8BD
  \l B8BE
  \l B8BF
  \l B8C0
  \l B8C1
  \l B8C2
  \l B8C3
  \l B8C4
  \l B8C5
  \l B8C6
  \l B8C7
  \l B8C8
  \l B8C9
  \l B8CA
  \l B8CB
  \l B8CC
  \l B8CD
  \l B8CE
  \l B8CF
  \l B8D0
  \l B8D1
  \l B8D2
  \l B8D3
  \l B8D4
  \l B8D5
  \l B8D6
  \l B8D7
  \l B8D8
  \l B8D9
  \l B8DA
  \l B8DB
  \l B8DC
  \l B8DD
  \l B8DE
  \l B8DF
  \l B8E0
  \l B8E1
  \l B8E2
  \l B8E3
  \l B8E4
  \l B8E5
  \l B8E6
  \l B8E7
  \l B8E8
  \l B8E9
  \l B8EA
  \l B8EB
  \l B8EC
  \l B8ED
  \l B8EE
  \l B8EF
  \l B8F0
  \l B8F1
  \l B8F2
  \l B8F3
  \l B8F4
  \l B8F5
  \l B8F6
  \l B8F7
  \l B8F8
  \l B8F9
  \l B8FA
  \l B8FB
  \l B8FC
  \l B8FD
  \l B8FE
  \l B8FF
  \l B900
  \l B901
  \l B902
  \l B903
  \l B904
  \l B905
  \l B906
  \l B907
  \l B908
  \l B909
  \l B90A
  \l B90B
  \l B90C
  \l B90D
  \l B90E
  \l B90F
  \l B910
  \l B911
  \l B912
  \l B913
  \l B914
  \l B915
  \l B916
  \l B917
  \l B918
  \l B919
  \l B91A
  \l B91B
  \l B91C
  \l B91D
  \l B91E
  \l B91F
  \l B920
  \l B921
  \l B922
  \l B923
  \l B924
  \l B925
  \l B926
  \l B927
  \l B928
  \l B929
  \l B92A
  \l B92B
  \l B92C
  \l B92D
  \l B92E
  \l B92F
  \l B930
  \l B931
  \l B932
  \l B933
  \l B934
  \l B935
  \l B936
  \l B937
  \l B938
  \l B939
  \l B93A
  \l B93B
  \l B93C
  \l B93D
  \l B93E
  \l B93F
  \l B940
  \l B941
  \l B942
  \l B943
  \l B944
  \l B945
  \l B946
  \l B947
  \l B948
  \l B949
  \l B94A
  \l B94B
  \l B94C
  \l B94D
  \l B94E
  \l B94F
  \l B950
  \l B951
  \l B952
  \l B953
  \l B954
  \l B955
  \l B956
  \l B957
  \l B958
  \l B959
  \l B95A
  \l B95B
  \l B95C
  \l B95D
  \l B95E
  \l B95F
  \l B960
  \l B961
  \l B962
  \l B963
  \l B964
  \l B965
  \l B966
  \l B967
  \l B968
  \l B969
  \l B96A
  \l B96B
  \l B96C
  \l B96D
  \l B96E
  \l B96F
  \l B970
  \l B971
  \l B972
  \l B973
  \l B974
  \l B975
  \l B976
  \l B977
  \l B978
  \l B979
  \l B97A
  \l B97B
  \l B97C
  \l B97D
  \l B97E
  \l B97F
  \l B980
  \l B981
  \l B982
  \l B983
  \l B984
  \l B985
  \l B986
  \l B987
  \l B988
  \l B989
  \l B98A
  \l B98B
  \l B98C
  \l B98D
  \l B98E
  \l B98F
  \l B990
  \l B991
  \l B992
  \l B993
  \l B994
  \l B995
  \l B996
  \l B997
  \l B998
  \l B999
  \l B99A
  \l B99B
  \l B99C
  \l B99D
  \l B99E
  \l B99F
  \l B9A0
  \l B9A1
  \l B9A2
  \l B9A3
  \l B9A4
  \l B9A5
  \l B9A6
  \l B9A7
  \l B9A8
  \l B9A9
  \l B9AA
  \l B9AB
  \l B9AC
  \l B9AD
  \l B9AE
  \l B9AF
  \l B9B0
  \l B9B1
  \l B9B2
  \l B9B3
  \l B9B4
  \l B9B5
  \l B9B6
  \l B9B7
  \l B9B8
  \l B9B9
  \l B9BA
  \l B9BB
  \l B9BC
  \l B9BD
  \l B9BE
  \l B9BF
  \l B9C0
  \l B9C1
  \l B9C2
  \l B9C3
  \l B9C4
  \l B9C5
  \l B9C6
  \l B9C7
  \l B9C8
  \l B9C9
  \l B9CA
  \l B9CB
  \l B9CC
  \l B9CD
  \l B9CE
  \l B9CF
  \l B9D0
  \l B9D1
  \l B9D2
  \l B9D3
  \l B9D4
  \l B9D5
  \l B9D6
  \l B9D7
  \l B9D8
  \l B9D9
  \l B9DA
  \l B9DB
  \l B9DC
  \l B9DD
  \l B9DE
  \l B9DF
  \l B9E0
  \l B9E1
  \l B9E2
  \l B9E3
  \l B9E4
  \l B9E5
  \l B9E6
  \l B9E7
  \l B9E8
  \l B9E9
  \l B9EA
  \l B9EB
  \l B9EC
  \l B9ED
  \l B9EE
  \l B9EF
  \l B9F0
  \l B9F1
  \l B9F2
  \l B9F3
  \l B9F4
  \l B9F5
  \l B9F6
  \l B9F7
  \l B9F8
  \l B9F9
  \l B9FA
  \l B9FB
  \l B9FC
  \l B9FD
  \l B9FE
  \l B9FF
  \l BA00
  \l BA01
  \l BA02
  \l BA03
  \l BA04
  \l BA05
  \l BA06
  \l BA07
  \l BA08
  \l BA09
  \l BA0A
  \l BA0B
  \l BA0C
  \l BA0D
  \l BA0E
  \l BA0F
  \l BA10
  \l BA11
  \l BA12
  \l BA13
  \l BA14
  \l BA15
  \l BA16
  \l BA17
  \l BA18
  \l BA19
  \l BA1A
  \l BA1B
  \l BA1C
  \l BA1D
  \l BA1E
  \l BA1F
  \l BA20
  \l BA21
  \l BA22
  \l BA23
  \l BA24
  \l BA25
  \l BA26
  \l BA27
  \l BA28
  \l BA29
  \l BA2A
  \l BA2B
  \l BA2C
  \l BA2D
  \l BA2E
  \l BA2F
  \l BA30
  \l BA31
  \l BA32
  \l BA33
  \l BA34
  \l BA35
  \l BA36
  \l BA37
  \l BA38
  \l BA39
  \l BA3A
  \l BA3B
  \l BA3C
  \l BA3D
  \l BA3E
  \l BA3F
  \l BA40
  \l BA41
  \l BA42
  \l BA43
  \l BA44
  \l BA45
  \l BA46
  \l BA47
  \l BA48
  \l BA49
  \l BA4A
  \l BA4B
  \l BA4C
  \l BA4D
  \l BA4E
  \l BA4F
  \l BA50
  \l BA51
  \l BA52
  \l BA53
  \l BA54
  \l BA55
  \l BA56
  \l BA57
  \l BA58
  \l BA59
  \l BA5A
  \l BA5B
  \l BA5C
  \l BA5D
  \l BA5E
  \l BA5F
  \l BA60
  \l BA61
  \l BA62
  \l BA63
  \l BA64
  \l BA65
  \l BA66
  \l BA67
  \l BA68
  \l BA69
  \l BA6A
  \l BA6B
  \l BA6C
  \l BA6D
  \l BA6E
  \l BA6F
  \l BA70
  \l BA71
  \l BA72
  \l BA73
  \l BA74
  \l BA75
  \l BA76
  \l BA77
  \l BA78
  \l BA79
  \l BA7A
  \l BA7B
  \l BA7C
  \l BA7D
  \l BA7E
  \l BA7F
  \l BA80
  \l BA81
  \l BA82
  \l BA83
  \l BA84
  \l BA85
  \l BA86
  \l BA87
  \l BA88
  \l BA89
  \l BA8A
  \l BA8B
  \l BA8C
  \l BA8D
  \l BA8E
  \l BA8F
  \l BA90
  \l BA91
  \l BA92
  \l BA93
  \l BA94
  \l BA95
  \l BA96
  \l BA97
  \l BA98
  \l BA99
  \l BA9A
  \l BA9B
  \l BA9C
  \l BA9D
  \l BA9E
  \l BA9F
  \l BAA0
  \l BAA1
  \l BAA2
  \l BAA3
  \l BAA4
  \l BAA5
  \l BAA6
  \l BAA7
  \l BAA8
  \l BAA9
  \l BAAA
  \l BAAB
  \l BAAC
  \l BAAD
  \l BAAE
  \l BAAF
  \l BAB0
  \l BAB1
  \l BAB2
  \l BAB3
  \l BAB4
  \l BAB5
  \l BAB6
  \l BAB7
  \l BAB8
  \l BAB9
  \l BABA
  \l BABB
  \l BABC
  \l BABD
  \l BABE
  \l BABF
  \l BAC0
  \l BAC1
  \l BAC2
  \l BAC3
  \l BAC4
  \l BAC5
  \l BAC6
  \l BAC7
  \l BAC8
  \l BAC9
  \l BACA
  \l BACB
  \l BACC
  \l BACD
  \l BACE
  \l BACF
  \l BAD0
  \l BAD1
  \l BAD2
  \l BAD3
  \l BAD4
  \l BAD5
  \l BAD6
  \l BAD7
  \l BAD8
  \l BAD9
  \l BADA
  \l BADB
  \l BADC
  \l BADD
  \l BADE
  \l BADF
  \l BAE0
  \l BAE1
  \l BAE2
  \l BAE3
  \l BAE4
  \l BAE5
  \l BAE6
  \l BAE7
  \l BAE8
  \l BAE9
  \l BAEA
  \l BAEB
  \l BAEC
  \l BAED
  \l BAEE
  \l BAEF
  \l BAF0
  \l BAF1
  \l BAF2
  \l BAF3
  \l BAF4
  \l BAF5
  \l BAF6
  \l BAF7
  \l BAF8
  \l BAF9
  \l BAFA
  \l BAFB
  \l BAFC
  \l BAFD
  \l BAFE
  \l BAFF
  \l BB00
  \l BB01
  \l BB02
  \l BB03
  \l BB04
  \l BB05
  \l BB06
  \l BB07
  \l BB08
  \l BB09
  \l BB0A
  \l BB0B
  \l BB0C
  \l BB0D
  \l BB0E
  \l BB0F
  \l BB10
  \l BB11
  \l BB12
  \l BB13
  \l BB14
  \l BB15
  \l BB16
  \l BB17
  \l BB18
  \l BB19
  \l BB1A
  \l BB1B
  \l BB1C
  \l BB1D
  \l BB1E
  \l BB1F
  \l BB20
  \l BB21
  \l BB22
  \l BB23
  \l BB24
  \l BB25
  \l BB26
  \l BB27
  \l BB28
  \l BB29
  \l BB2A
  \l BB2B
  \l BB2C
  \l BB2D
  \l BB2E
  \l BB2F
  \l BB30
  \l BB31
  \l BB32
  \l BB33
  \l BB34
  \l BB35
  \l BB36
  \l BB37
  \l BB38
  \l BB39
  \l BB3A
  \l BB3B
  \l BB3C
  \l BB3D
  \l BB3E
  \l BB3F
  \l BB40
  \l BB41
  \l BB42
  \l BB43
  \l BB44
  \l BB45
  \l BB46
  \l BB47
  \l BB48
  \l BB49
  \l BB4A
  \l BB4B
  \l BB4C
  \l BB4D
  \l BB4E
  \l BB4F
  \l BB50
  \l BB51
  \l BB52
  \l BB53
  \l BB54
  \l BB55
  \l BB56
  \l BB57
  \l BB58
  \l BB59
  \l BB5A
  \l BB5B
  \l BB5C
  \l BB5D
  \l BB5E
  \l BB5F
  \l BB60
  \l BB61
  \l BB62
  \l BB63
  \l BB64
  \l BB65
  \l BB66
  \l BB67
  \l BB68
  \l BB69
  \l BB6A
  \l BB6B
  \l BB6C
  \l BB6D
  \l BB6E
  \l BB6F
  \l BB70
  \l BB71
  \l BB72
  \l BB73
  \l BB74
  \l BB75
  \l BB76
  \l BB77
  \l BB78
  \l BB79
  \l BB7A
  \l BB7B
  \l BB7C
  \l BB7D
  \l BB7E
  \l BB7F
  \l BB80
  \l BB81
  \l BB82
  \l BB83
  \l BB84
  \l BB85
  \l BB86
  \l BB87
  \l BB88
  \l BB89
  \l BB8A
  \l BB8B
  \l BB8C
  \l BB8D
  \l BB8E
  \l BB8F
  \l BB90
  \l BB91
  \l BB92
  \l BB93
  \l BB94
  \l BB95
  \l BB96
  \l BB97
  \l BB98
  \l BB99
  \l BB9A
  \l BB9B
  \l BB9C
  \l BB9D
  \l BB9E
  \l BB9F
  \l BBA0
  \l BBA1
  \l BBA2
  \l BBA3
  \l BBA4
  \l BBA5
  \l BBA6
  \l BBA7
  \l BBA8
  \l BBA9
  \l BBAA
  \l BBAB
  \l BBAC
  \l BBAD
  \l BBAE
  \l BBAF
  \l BBB0
  \l BBB1
  \l BBB2
  \l BBB3
  \l BBB4
  \l BBB5
  \l BBB6
  \l BBB7
  \l BBB8
  \l BBB9
  \l BBBA
  \l BBBB
  \l BBBC
  \l BBBD
  \l BBBE
  \l BBBF
  \l BBC0
  \l BBC1
  \l BBC2
  \l BBC3
  \l BBC4
  \l BBC5
  \l BBC6
  \l BBC7
  \l BBC8
  \l BBC9
  \l BBCA
  \l BBCB
  \l BBCC
  \l BBCD
  \l BBCE
  \l BBCF
  \l BBD0
  \l BBD1
  \l BBD2
  \l BBD3
  \l BBD4
  \l BBD5
  \l BBD6
  \l BBD7
  \l BBD8
  \l BBD9
  \l BBDA
  \l BBDB
  \l BBDC
  \l BBDD
  \l BBDE
  \l BBDF
  \l BBE0
  \l BBE1
  \l BBE2
  \l BBE3
  \l BBE4
  \l BBE5
  \l BBE6
  \l BBE7
  \l BBE8
  \l BBE9
  \l BBEA
  \l BBEB
  \l BBEC
  \l BBED
  \l BBEE
  \l BBEF
  \l BBF0
  \l BBF1
  \l BBF2
  \l BBF3
  \l BBF4
  \l BBF5
  \l BBF6
  \l BBF7
  \l BBF8
  \l BBF9
  \l BBFA
  \l BBFB
  \l BBFC
  \l BBFD
  \l BBFE
  \l BBFF
  \l BC00
  \l BC01
  \l BC02
  \l BC03
  \l BC04
  \l BC05
  \l BC06
  \l BC07
  \l BC08
  \l BC09
  \l BC0A
  \l BC0B
  \l BC0C
  \l BC0D
  \l BC0E
  \l BC0F
  \l BC10
  \l BC11
  \l BC12
  \l BC13
  \l BC14
  \l BC15
  \l BC16
  \l BC17
  \l BC18
  \l BC19
  \l BC1A
  \l BC1B
  \l BC1C
  \l BC1D
  \l BC1E
  \l BC1F
  \l BC20
  \l BC21
  \l BC22
  \l BC23
  \l BC24
  \l BC25
  \l BC26
  \l BC27
  \l BC28
  \l BC29
  \l BC2A
  \l BC2B
  \l BC2C
  \l BC2D
  \l BC2E
  \l BC2F
  \l BC30
  \l BC31
  \l BC32
  \l BC33
  \l BC34
  \l BC35
  \l BC36
  \l BC37
  \l BC38
  \l BC39
  \l BC3A
  \l BC3B
  \l BC3C
  \l BC3D
  \l BC3E
  \l BC3F
  \l BC40
  \l BC41
  \l BC42
  \l BC43
  \l BC44
  \l BC45
  \l BC46
  \l BC47
  \l BC48
  \l BC49
  \l BC4A
  \l BC4B
  \l BC4C
  \l BC4D
  \l BC4E
  \l BC4F
  \l BC50
  \l BC51
  \l BC52
  \l BC53
  \l BC54
  \l BC55
  \l BC56
  \l BC57
  \l BC58
  \l BC59
  \l BC5A
  \l BC5B
  \l BC5C
  \l BC5D
  \l BC5E
  \l BC5F
  \l BC60
  \l BC61
  \l BC62
  \l BC63
  \l BC64
  \l BC65
  \l BC66
  \l BC67
  \l BC68
  \l BC69
  \l BC6A
  \l BC6B
  \l BC6C
  \l BC6D
  \l BC6E
  \l BC6F
  \l BC70
  \l BC71
  \l BC72
  \l BC73
  \l BC74
  \l BC75
  \l BC76
  \l BC77
  \l BC78
  \l BC79
  \l BC7A
  \l BC7B
  \l BC7C
  \l BC7D
  \l BC7E
  \l BC7F
  \l BC80
  \l BC81
  \l BC82
  \l BC83
  \l BC84
  \l BC85
  \l BC86
  \l BC87
  \l BC88
  \l BC89
  \l BC8A
  \l BC8B
  \l BC8C
  \l BC8D
  \l BC8E
  \l BC8F
  \l BC90
  \l BC91
  \l BC92
  \l BC93
  \l BC94
  \l BC95
  \l BC96
  \l BC97
  \l BC98
  \l BC99
  \l BC9A
  \l BC9B
  \l BC9C
  \l BC9D
  \l BC9E
  \l BC9F
  \l BCA0
  \l BCA1
  \l BCA2
  \l BCA3
  \l BCA4
  \l BCA5
  \l BCA6
  \l BCA7
  \l BCA8
  \l BCA9
  \l BCAA
  \l BCAB
  \l BCAC
  \l BCAD
  \l BCAE
  \l BCAF
  \l BCB0
  \l BCB1
  \l BCB2
  \l BCB3
  \l BCB4
  \l BCB5
  \l BCB6
  \l BCB7
  \l BCB8
  \l BCB9
  \l BCBA
  \l BCBB
  \l BCBC
  \l BCBD
  \l BCBE
  \l BCBF
  \l BCC0
  \l BCC1
  \l BCC2
  \l BCC3
  \l BCC4
  \l BCC5
  \l BCC6
  \l BCC7
  \l BCC8
  \l BCC9
  \l BCCA
  \l BCCB
  \l BCCC
  \l BCCD
  \l BCCE
  \l BCCF
  \l BCD0
  \l BCD1
  \l BCD2
  \l BCD3
  \l BCD4
  \l BCD5
  \l BCD6
  \l BCD7
  \l BCD8
  \l BCD9
  \l BCDA
  \l BCDB
  \l BCDC
  \l BCDD
  \l BCDE
  \l BCDF
  \l BCE0
  \l BCE1
  \l BCE2
  \l BCE3
  \l BCE4
  \l BCE5
  \l BCE6
  \l BCE7
  \l BCE8
  \l BCE9
  \l BCEA
  \l BCEB
  \l BCEC
  \l BCED
  \l BCEE
  \l BCEF
  \l BCF0
  \l BCF1
  \l BCF2
  \l BCF3
  \l BCF4
  \l BCF5
  \l BCF6
  \l BCF7
  \l BCF8
  \l BCF9
  \l BCFA
  \l BCFB
  \l BCFC
  \l BCFD
  \l BCFE
  \l BCFF
  \l BD00
  \l BD01
  \l BD02
  \l BD03
  \l BD04
  \l BD05
  \l BD06
  \l BD07
  \l BD08
  \l BD09
  \l BD0A
  \l BD0B
  \l BD0C
  \l BD0D
  \l BD0E
  \l BD0F
  \l BD10
  \l BD11
  \l BD12
  \l BD13
  \l BD14
  \l BD15
  \l BD16
  \l BD17
  \l BD18
  \l BD19
  \l BD1A
  \l BD1B
  \l BD1C
  \l BD1D
  \l BD1E
  \l BD1F
  \l BD20
  \l BD21
  \l BD22
  \l BD23
  \l BD24
  \l BD25
  \l BD26
  \l BD27
  \l BD28
  \l BD29
  \l BD2A
  \l BD2B
  \l BD2C
  \l BD2D
  \l BD2E
  \l BD2F
  \l BD30
  \l BD31
  \l BD32
  \l BD33
  \l BD34
  \l BD35
  \l BD36
  \l BD37
  \l BD38
  \l BD39
  \l BD3A
  \l BD3B
  \l BD3C
  \l BD3D
  \l BD3E
  \l BD3F
  \l BD40
  \l BD41
  \l BD42
  \l BD43
  \l BD44
  \l BD45
  \l BD46
  \l BD47
  \l BD48
  \l BD49
  \l BD4A
  \l BD4B
  \l BD4C
  \l BD4D
  \l BD4E
  \l BD4F
  \l BD50
  \l BD51
  \l BD52
  \l BD53
  \l BD54
  \l BD55
  \l BD56
  \l BD57
  \l BD58
  \l BD59
  \l BD5A
  \l BD5B
  \l BD5C
  \l BD5D
  \l BD5E
  \l BD5F
  \l BD60
  \l BD61
  \l BD62
  \l BD63
  \l BD64
  \l BD65
  \l BD66
  \l BD67
  \l BD68
  \l BD69
  \l BD6A
  \l BD6B
  \l BD6C
  \l BD6D
  \l BD6E
  \l BD6F
  \l BD70
  \l BD71
  \l BD72
  \l BD73
  \l BD74
  \l BD75
  \l BD76
  \l BD77
  \l BD78
  \l BD79
  \l BD7A
  \l BD7B
  \l BD7C
  \l BD7D
  \l BD7E
  \l BD7F
  \l BD80
  \l BD81
  \l BD82
  \l BD83
  \l BD84
  \l BD85
  \l BD86
  \l BD87
  \l BD88
  \l BD89
  \l BD8A
  \l BD8B
  \l BD8C
  \l BD8D
  \l BD8E
  \l BD8F
  \l BD90
  \l BD91
  \l BD92
  \l BD93
  \l BD94
  \l BD95
  \l BD96
  \l BD97
  \l BD98
  \l BD99
  \l BD9A
  \l BD9B
  \l BD9C
  \l BD9D
  \l BD9E
  \l BD9F
  \l BDA0
  \l BDA1
  \l BDA2
  \l BDA3
  \l BDA4
  \l BDA5
  \l BDA6
  \l BDA7
  \l BDA8
  \l BDA9
  \l BDAA
  \l BDAB
  \l BDAC
  \l BDAD
  \l BDAE
  \l BDAF
  \l BDB0
  \l BDB1
  \l BDB2
  \l BDB3
  \l BDB4
  \l BDB5
  \l BDB6
  \l BDB7
  \l BDB8
  \l BDB9
  \l BDBA
  \l BDBB
  \l BDBC
  \l BDBD
  \l BDBE
  \l BDBF
  \l BDC0
  \l BDC1
  \l BDC2
  \l BDC3
  \l BDC4
  \l BDC5
  \l BDC6
  \l BDC7
  \l BDC8
  \l BDC9
  \l BDCA
  \l BDCB
  \l BDCC
  \l BDCD
  \l BDCE
  \l BDCF
  \l BDD0
  \l BDD1
  \l BDD2
  \l BDD3
  \l BDD4
  \l BDD5
  \l BDD6
  \l BDD7
  \l BDD8
  \l BDD9
  \l BDDA
  \l BDDB
  \l BDDC
  \l BDDD
  \l BDDE
  \l BDDF
  \l BDE0
  \l BDE1
  \l BDE2
  \l BDE3
  \l BDE4
  \l BDE5
  \l BDE6
  \l BDE7
  \l BDE8
  \l BDE9
  \l BDEA
  \l BDEB
  \l BDEC
  \l BDED
  \l BDEE
  \l BDEF
  \l BDF0
  \l BDF1
  \l BDF2
  \l BDF3
  \l BDF4
  \l BDF5
  \l BDF6
  \l BDF7
  \l BDF8
  \l BDF9
  \l BDFA
  \l BDFB
  \l BDFC
  \l BDFD
  \l BDFE
  \l BDFF
  \l BE00
  \l BE01
  \l BE02
  \l BE03
  \l BE04
  \l BE05
  \l BE06
  \l BE07
  \l BE08
  \l BE09
  \l BE0A
  \l BE0B
  \l BE0C
  \l BE0D
  \l BE0E
  \l BE0F
  \l BE10
  \l BE11
  \l BE12
  \l BE13
  \l BE14
  \l BE15
  \l BE16
  \l BE17
  \l BE18
  \l BE19
  \l BE1A
  \l BE1B
  \l BE1C
  \l BE1D
  \l BE1E
  \l BE1F
  \l BE20
  \l BE21
  \l BE22
  \l BE23
  \l BE24
  \l BE25
  \l BE26
  \l BE27
  \l BE28
  \l BE29
  \l BE2A
  \l BE2B
  \l BE2C
  \l BE2D
  \l BE2E
  \l BE2F
  \l BE30
  \l BE31
  \l BE32
  \l BE33
  \l BE34
  \l BE35
  \l BE36
  \l BE37
  \l BE38
  \l BE39
  \l BE3A
  \l BE3B
  \l BE3C
  \l BE3D
  \l BE3E
  \l BE3F
  \l BE40
  \l BE41
  \l BE42
  \l BE43
  \l BE44
  \l BE45
  \l BE46
  \l BE47
  \l BE48
  \l BE49
  \l BE4A
  \l BE4B
  \l BE4C
  \l BE4D
  \l BE4E
  \l BE4F
  \l BE50
  \l BE51
  \l BE52
  \l BE53
  \l BE54
  \l BE55
  \l BE56
  \l BE57
  \l BE58
  \l BE59
  \l BE5A
  \l BE5B
  \l BE5C
  \l BE5D
  \l BE5E
  \l BE5F
  \l BE60
  \l BE61
  \l BE62
  \l BE63
  \l BE64
  \l BE65
  \l BE66
  \l BE67
  \l BE68
  \l BE69
  \l BE6A
  \l BE6B
  \l BE6C
  \l BE6D
  \l BE6E
  \l BE6F
  \l BE70
  \l BE71
  \l BE72
  \l BE73
  \l BE74
  \l BE75
  \l BE76
  \l BE77
  \l BE78
  \l BE79
  \l BE7A
  \l BE7B
  \l BE7C
  \l BE7D
  \l BE7E
  \l BE7F
  \l BE80
  \l BE81
  \l BE82
  \l BE83
  \l BE84
  \l BE85
  \l BE86
  \l BE87
  \l BE88
  \l BE89
  \l BE8A
  \l BE8B
  \l BE8C
  \l BE8D
  \l BE8E
  \l BE8F
  \l BE90
  \l BE91
  \l BE92
  \l BE93
  \l BE94
  \l BE95
  \l BE96
  \l BE97
  \l BE98
  \l BE99
  \l BE9A
  \l BE9B
  \l BE9C
  \l BE9D
  \l BE9E
  \l BE9F
  \l BEA0
  \l BEA1
  \l BEA2
  \l BEA3
  \l BEA4
  \l BEA5
  \l BEA6
  \l BEA7
  \l BEA8
  \l BEA9
  \l BEAA
  \l BEAB
  \l BEAC
  \l BEAD
  \l BEAE
  \l BEAF
  \l BEB0
  \l BEB1
  \l BEB2
  \l BEB3
  \l BEB4
  \l BEB5
  \l BEB6
  \l BEB7
  \l BEB8
  \l BEB9
  \l BEBA
  \l BEBB
  \l BEBC
  \l BEBD
  \l BEBE
  \l BEBF
  \l BEC0
  \l BEC1
  \l BEC2
  \l BEC3
  \l BEC4
  \l BEC5
  \l BEC6
  \l BEC7
  \l BEC8
  \l BEC9
  \l BECA
  \l BECB
  \l BECC
  \l BECD
  \l BECE
  \l BECF
  \l BED0
  \l BED1
  \l BED2
  \l BED3
  \l BED4
  \l BED5
  \l BED6
  \l BED7
  \l BED8
  \l BED9
  \l BEDA
  \l BEDB
  \l BEDC
  \l BEDD
  \l BEDE
  \l BEDF
  \l BEE0
  \l BEE1
  \l BEE2
  \l BEE3
  \l BEE4
  \l BEE5
  \l BEE6
  \l BEE7
  \l BEE8
  \l BEE9
  \l BEEA
  \l BEEB
  \l BEEC
  \l BEED
  \l BEEE
  \l BEEF
  \l BEF0
  \l BEF1
  \l BEF2
  \l BEF3
  \l BEF4
  \l BEF5
  \l BEF6
  \l BEF7
  \l BEF8
  \l BEF9
  \l BEFA
  \l BEFB
  \l BEFC
  \l BEFD
  \l BEFE
  \l BEFF
  \l BF00
  \l BF01
  \l BF02
  \l BF03
  \l BF04
  \l BF05
  \l BF06
  \l BF07
  \l BF08
  \l BF09
  \l BF0A
  \l BF0B
  \l BF0C
  \l BF0D
  \l BF0E
  \l BF0F
  \l BF10
  \l BF11
  \l BF12
  \l BF13
  \l BF14
  \l BF15
  \l BF16
  \l BF17
  \l BF18
  \l BF19
  \l BF1A
  \l BF1B
  \l BF1C
  \l BF1D
  \l BF1E
  \l BF1F
  \l BF20
  \l BF21
  \l BF22
  \l BF23
  \l BF24
  \l BF25
  \l BF26
  \l BF27
  \l BF28
  \l BF29
  \l BF2A
  \l BF2B
  \l BF2C
  \l BF2D
  \l BF2E
  \l BF2F
  \l BF30
  \l BF31
  \l BF32
  \l BF33
  \l BF34
  \l BF35
  \l BF36
  \l BF37
  \l BF38
  \l BF39
  \l BF3A
  \l BF3B
  \l BF3C
  \l BF3D
  \l BF3E
  \l BF3F
  \l BF40
  \l BF41
  \l BF42
  \l BF43
  \l BF44
  \l BF45
  \l BF46
  \l BF47
  \l BF48
  \l BF49
  \l BF4A
  \l BF4B
  \l BF4C
  \l BF4D
  \l BF4E
  \l BF4F
  \l BF50
  \l BF51
  \l BF52
  \l BF53
  \l BF54
  \l BF55
  \l BF56
  \l BF57
  \l BF58
  \l BF59
  \l BF5A
  \l BF5B
  \l BF5C
  \l BF5D
  \l BF5E
  \l BF5F
  \l BF60
  \l BF61
  \l BF62
  \l BF63
  \l BF64
  \l BF65
  \l BF66
  \l BF67
  \l BF68
  \l BF69
  \l BF6A
  \l BF6B
  \l BF6C
  \l BF6D
  \l BF6E
  \l BF6F
  \l BF70
  \l BF71
  \l BF72
  \l BF73
  \l BF74
  \l BF75
  \l BF76
  \l BF77
  \l BF78
  \l BF79
  \l BF7A
  \l BF7B
  \l BF7C
  \l BF7D
  \l BF7E
  \l BF7F
  \l BF80
  \l BF81
  \l BF82
  \l BF83
  \l BF84
  \l BF85
  \l BF86
  \l BF87
  \l BF88
  \l BF89
  \l BF8A
  \l BF8B
  \l BF8C
  \l BF8D
  \l BF8E
  \l BF8F
  \l BF90
  \l BF91
  \l BF92
  \l BF93
  \l BF94
  \l BF95
  \l BF96
  \l BF97
  \l BF98
  \l BF99
  \l BF9A
  \l BF9B
  \l BF9C
  \l BF9D
  \l BF9E
  \l BF9F
  \l BFA0
  \l BFA1
  \l BFA2
  \l BFA3
  \l BFA4
  \l BFA5
  \l BFA6
  \l BFA7
  \l BFA8
  \l BFA9
  \l BFAA
  \l BFAB
  \l BFAC
  \l BFAD
  \l BFAE
  \l BFAF
  \l BFB0
  \l BFB1
  \l BFB2
  \l BFB3
  \l BFB4
  \l BFB5
  \l BFB6
  \l BFB7
  \l BFB8
  \l BFB9
  \l BFBA
  \l BFBB
  \l BFBC
  \l BFBD
  \l BFBE
  \l BFBF
  \l BFC0
  \l BFC1
  \l BFC2
  \l BFC3
  \l BFC4
  \l BFC5
  \l BFC6
  \l BFC7
  \l BFC8
  \l BFC9
  \l BFCA
  \l BFCB
  \l BFCC
  \l BFCD
  \l BFCE
  \l BFCF
  \l BFD0
  \l BFD1
  \l BFD2
  \l BFD3
  \l BFD4
  \l BFD5
  \l BFD6
  \l BFD7
  \l BFD8
  \l BFD9
  \l BFDA
  \l BFDB
  \l BFDC
  \l BFDD
  \l BFDE
  \l BFDF
  \l BFE0
  \l BFE1
  \l BFE2
  \l BFE3
  \l BFE4
  \l BFE5
  \l BFE6
  \l BFE7
  \l BFE8
  \l BFE9
  \l BFEA
  \l BFEB
  \l BFEC
  \l BFED
  \l BFEE
  \l BFEF
  \l BFF0
  \l BFF1
  \l BFF2
  \l BFF3
  \l BFF4
  \l BFF5
  \l BFF6
  \l BFF7
  \l BFF8
  \l BFF9
  \l BFFA
  \l BFFB
  \l BFFC
  \l BFFD
  \l BFFE
  \l BFFF
  \l C000
  \l C001
  \l C002
  \l C003
  \l C004
  \l C005
  \l C006
  \l C007
  \l C008
  \l C009
  \l C00A
  \l C00B
  \l C00C
  \l C00D
  \l C00E
  \l C00F
  \l C010
  \l C011
  \l C012
  \l C013
  \l C014
  \l C015
  \l C016
  \l C017
  \l C018
  \l C019
  \l C01A
  \l C01B
  \l C01C
  \l C01D
  \l C01E
  \l C01F
  \l C020
  \l C021
  \l C022
  \l C023
  \l C024
  \l C025
  \l C026
  \l C027
  \l C028
  \l C029
  \l C02A
  \l C02B
  \l C02C
  \l C02D
  \l C02E
  \l C02F
  \l C030
  \l C031
  \l C032
  \l C033
  \l C034
  \l C035
  \l C036
  \l C037
  \l C038
  \l C039
  \l C03A
  \l C03B
  \l C03C
  \l C03D
  \l C03E
  \l C03F
  \l C040
  \l C041
  \l C042
  \l C043
  \l C044
  \l C045
  \l C046
  \l C047
  \l C048
  \l C049
  \l C04A
  \l C04B
  \l C04C
  \l C04D
  \l C04E
  \l C04F
  \l C050
  \l C051
  \l C052
  \l C053
  \l C054
  \l C055
  \l C056
  \l C057
  \l C058
  \l C059
  \l C05A
  \l C05B
  \l C05C
  \l C05D
  \l C05E
  \l C05F
  \l C060
  \l C061
  \l C062
  \l C063
  \l C064
  \l C065
  \l C066
  \l C067
  \l C068
  \l C069
  \l C06A
  \l C06B
  \l C06C
  \l C06D
  \l C06E
  \l C06F
  \l C070
  \l C071
  \l C072
  \l C073
  \l C074
  \l C075
  \l C076
  \l C077
  \l C078
  \l C079
  \l C07A
  \l C07B
  \l C07C
  \l C07D
  \l C07E
  \l C07F
  \l C080
  \l C081
  \l C082
  \l C083
  \l C084
  \l C085
  \l C086
  \l C087
  \l C088
  \l C089
  \l C08A
  \l C08B
  \l C08C
  \l C08D
  \l C08E
  \l C08F
  \l C090
  \l C091
  \l C092
  \l C093
  \l C094
  \l C095
  \l C096
  \l C097
  \l C098
  \l C099
  \l C09A
  \l C09B
  \l C09C
  \l C09D
  \l C09E
  \l C09F
  \l C0A0
  \l C0A1
  \l C0A2
  \l C0A3
  \l C0A4
  \l C0A5
  \l C0A6
  \l C0A7
  \l C0A8
  \l C0A9
  \l C0AA
  \l C0AB
  \l C0AC
  \l C0AD
  \l C0AE
  \l C0AF
  \l C0B0
  \l C0B1
  \l C0B2
  \l C0B3
  \l C0B4
  \l C0B5
  \l C0B6
  \l C0B7
  \l C0B8
  \l C0B9
  \l C0BA
  \l C0BB
  \l C0BC
  \l C0BD
  \l C0BE
  \l C0BF
  \l C0C0
  \l C0C1
  \l C0C2
  \l C0C3
  \l C0C4
  \l C0C5
  \l C0C6
  \l C0C7
  \l C0C8
  \l C0C9
  \l C0CA
  \l C0CB
  \l C0CC
  \l C0CD
  \l C0CE
  \l C0CF
  \l C0D0
  \l C0D1
  \l C0D2
  \l C0D3
  \l C0D4
  \l C0D5
  \l C0D6
  \l C0D7
  \l C0D8
  \l C0D9
  \l C0DA
  \l C0DB
  \l C0DC
  \l C0DD
  \l C0DE
  \l C0DF
  \l C0E0
  \l C0E1
  \l C0E2
  \l C0E3
  \l C0E4
  \l C0E5
  \l C0E6
  \l C0E7
  \l C0E8
  \l C0E9
  \l C0EA
  \l C0EB
  \l C0EC
  \l C0ED
  \l C0EE
  \l C0EF
  \l C0F0
  \l C0F1
  \l C0F2
  \l C0F3
  \l C0F4
  \l C0F5
  \l C0F6
  \l C0F7
  \l C0F8
  \l C0F9
  \l C0FA
  \l C0FB
  \l C0FC
  \l C0FD
  \l C0FE
  \l C0FF
  \l C100
  \l C101
  \l C102
  \l C103
  \l C104
  \l C105
  \l C106
  \l C107
  \l C108
  \l C109
  \l C10A
  \l C10B
  \l C10C
  \l C10D
  \l C10E
  \l C10F
  \l C110
  \l C111
  \l C112
  \l C113
  \l C114
  \l C115
  \l C116
  \l C117
  \l C118
  \l C119
  \l C11A
  \l C11B
  \l C11C
  \l C11D
  \l C11E
  \l C11F
  \l C120
  \l C121
  \l C122
  \l C123
  \l C124
  \l C125
  \l C126
  \l C127
  \l C128
  \l C129
  \l C12A
  \l C12B
  \l C12C
  \l C12D
  \l C12E
  \l C12F
  \l C130
  \l C131
  \l C132
  \l C133
  \l C134
  \l C135
  \l C136
  \l C137
  \l C138
  \l C139
  \l C13A
  \l C13B
  \l C13C
  \l C13D
  \l C13E
  \l C13F
  \l C140
  \l C141
  \l C142
  \l C143
  \l C144
  \l C145
  \l C146
  \l C147
  \l C148
  \l C149
  \l C14A
  \l C14B
  \l C14C
  \l C14D
  \l C14E
  \l C14F
  \l C150
  \l C151
  \l C152
  \l C153
  \l C154
  \l C155
  \l C156
  \l C157
  \l C158
  \l C159
  \l C15A
  \l C15B
  \l C15C
  \l C15D
  \l C15E
  \l C15F
  \l C160
  \l C161
  \l C162
  \l C163
  \l C164
  \l C165
  \l C166
  \l C167
  \l C168
  \l C169
  \l C16A
  \l C16B
  \l C16C
  \l C16D
  \l C16E
  \l C16F
  \l C170
  \l C171
  \l C172
  \l C173
  \l C174
  \l C175
  \l C176
  \l C177
  \l C178
  \l C179
  \l C17A
  \l C17B
  \l C17C
  \l C17D
  \l C17E
  \l C17F
  \l C180
  \l C181
  \l C182
  \l C183
  \l C184
  \l C185
  \l C186
  \l C187
  \l C188
  \l C189
  \l C18A
  \l C18B
  \l C18C
  \l C18D
  \l C18E
  \l C18F
  \l C190
  \l C191
  \l C192
  \l C193
  \l C194
  \l C195
  \l C196
  \l C197
  \l C198
  \l C199
  \l C19A
  \l C19B
  \l C19C
  \l C19D
  \l C19E
  \l C19F
  \l C1A0
  \l C1A1
  \l C1A2
  \l C1A3
  \l C1A4
  \l C1A5
  \l C1A6
  \l C1A7
  \l C1A8
  \l C1A9
  \l C1AA
  \l C1AB
  \l C1AC
  \l C1AD
  \l C1AE
  \l C1AF
  \l C1B0
  \l C1B1
  \l C1B2
  \l C1B3
  \l C1B4
  \l C1B5
  \l C1B6
  \l C1B7
  \l C1B8
  \l C1B9
  \l C1BA
  \l C1BB
  \l C1BC
  \l C1BD
  \l C1BE
  \l C1BF
  \l C1C0
  \l C1C1
  \l C1C2
  \l C1C3
  \l C1C4
  \l C1C5
  \l C1C6
  \l C1C7
  \l C1C8
  \l C1C9
  \l C1CA
  \l C1CB
  \l C1CC
  \l C1CD
  \l C1CE
  \l C1CF
  \l C1D0
  \l C1D1
  \l C1D2
  \l C1D3
  \l C1D4
  \l C1D5
  \l C1D6
  \l C1D7
  \l C1D8
  \l C1D9
  \l C1DA
  \l C1DB
  \l C1DC
  \l C1DD
  \l C1DE
  \l C1DF
  \l C1E0
  \l C1E1
  \l C1E2
  \l C1E3
  \l C1E4
  \l C1E5
  \l C1E6
  \l C1E7
  \l C1E8
  \l C1E9
  \l C1EA
  \l C1EB
  \l C1EC
  \l C1ED
  \l C1EE
  \l C1EF
  \l C1F0
  \l C1F1
  \l C1F2
  \l C1F3
  \l C1F4
  \l C1F5
  \l C1F6
  \l C1F7
  \l C1F8
  \l C1F9
  \l C1FA
  \l C1FB
  \l C1FC
  \l C1FD
  \l C1FE
  \l C1FF
  \l C200
  \l C201
  \l C202
  \l C203
  \l C204
  \l C205
  \l C206
  \l C207
  \l C208
  \l C209
  \l C20A
  \l C20B
  \l C20C
  \l C20D
  \l C20E
  \l C20F
  \l C210
  \l C211
  \l C212
  \l C213
  \l C214
  \l C215
  \l C216
  \l C217
  \l C218
  \l C219
  \l C21A
  \l C21B
  \l C21C
  \l C21D
  \l C21E
  \l C21F
  \l C220
  \l C221
  \l C222
  \l C223
  \l C224
  \l C225
  \l C226
  \l C227
  \l C228
  \l C229
  \l C22A
  \l C22B
  \l C22C
  \l C22D
  \l C22E
  \l C22F
  \l C230
  \l C231
  \l C232
  \l C233
  \l C234
  \l C235
  \l C236
  \l C237
  \l C238
  \l C239
  \l C23A
  \l C23B
  \l C23C
  \l C23D
  \l C23E
  \l C23F
  \l C240
  \l C241
  \l C242
  \l C243
  \l C244
  \l C245
  \l C246
  \l C247
  \l C248
  \l C249
  \l C24A
  \l C24B
  \l C24C
  \l C24D
  \l C24E
  \l C24F
  \l C250
  \l C251
  \l C252
  \l C253
  \l C254
  \l C255
  \l C256
  \l C257
  \l C258
  \l C259
  \l C25A
  \l C25B
  \l C25C
  \l C25D
  \l C25E
  \l C25F
  \l C260
  \l C261
  \l C262
  \l C263
  \l C264
  \l C265
  \l C266
  \l C267
  \l C268
  \l C269
  \l C26A
  \l C26B
  \l C26C
  \l C26D
  \l C26E
  \l C26F
  \l C270
  \l C271
  \l C272
  \l C273
  \l C274
  \l C275
  \l C276
  \l C277
  \l C278
  \l C279
  \l C27A
  \l C27B
  \l C27C
  \l C27D
  \l C27E
  \l C27F
  \l C280
  \l C281
  \l C282
  \l C283
  \l C284
  \l C285
  \l C286
  \l C287
  \l C288
  \l C289
  \l C28A
  \l C28B
  \l C28C
  \l C28D
  \l C28E
  \l C28F
  \l C290
  \l C291
  \l C292
  \l C293
  \l C294
  \l C295
  \l C296
  \l C297
  \l C298
  \l C299
  \l C29A
  \l C29B
  \l C29C
  \l C29D
  \l C29E
  \l C29F
  \l C2A0
  \l C2A1
  \l C2A2
  \l C2A3
  \l C2A4
  \l C2A5
  \l C2A6
  \l C2A7
  \l C2A8
  \l C2A9
  \l C2AA
  \l C2AB
  \l C2AC
  \l C2AD
  \l C2AE
  \l C2AF
  \l C2B0
  \l C2B1
  \l C2B2
  \l C2B3
  \l C2B4
  \l C2B5
  \l C2B6
  \l C2B7
  \l C2B8
  \l C2B9
  \l C2BA
  \l C2BB
  \l C2BC
  \l C2BD
  \l C2BE
  \l C2BF
  \l C2C0
  \l C2C1
  \l C2C2
  \l C2C3
  \l C2C4
  \l C2C5
  \l C2C6
  \l C2C7
  \l C2C8
  \l C2C9
  \l C2CA
  \l C2CB
  \l C2CC
  \l C2CD
  \l C2CE
  \l C2CF
  \l C2D0
  \l C2D1
  \l C2D2
  \l C2D3
  \l C2D4
  \l C2D5
  \l C2D6
  \l C2D7
  \l C2D8
  \l C2D9
  \l C2DA
  \l C2DB
  \l C2DC
  \l C2DD
  \l C2DE
  \l C2DF
  \l C2E0
  \l C2E1
  \l C2E2
  \l C2E3
  \l C2E4
  \l C2E5
  \l C2E6
  \l C2E7
  \l C2E8
  \l C2E9
  \l C2EA
  \l C2EB
  \l C2EC
  \l C2ED
  \l C2EE
  \l C2EF
  \l C2F0
  \l C2F1
  \l C2F2
  \l C2F3
  \l C2F4
  \l C2F5
  \l C2F6
  \l C2F7
  \l C2F8
  \l C2F9
  \l C2FA
  \l C2FB
  \l C2FC
  \l C2FD
  \l C2FE
  \l C2FF
  \l C300
  \l C301
  \l C302
  \l C303
  \l C304
  \l C305
  \l C306
  \l C307
  \l C308
  \l C309
  \l C30A
  \l C30B
  \l C30C
  \l C30D
  \l C30E
  \l C30F
  \l C310
  \l C311
  \l C312
  \l C313
  \l C314
  \l C315
  \l C316
  \l C317
  \l C318
  \l C319
  \l C31A
  \l C31B
  \l C31C
  \l C31D
  \l C31E
  \l C31F
  \l C320
  \l C321
  \l C322
  \l C323
  \l C324
  \l C325
  \l C326
  \l C327
  \l C328
  \l C329
  \l C32A
  \l C32B
  \l C32C
  \l C32D
  \l C32E
  \l C32F
  \l C330
  \l C331
  \l C332
  \l C333
  \l C334
  \l C335
  \l C336
  \l C337
  \l C338
  \l C339
  \l C33A
  \l C33B
  \l C33C
  \l C33D
  \l C33E
  \l C33F
  \l C340
  \l C341
  \l C342
  \l C343
  \l C344
  \l C345
  \l C346
  \l C347
  \l C348
  \l C349
  \l C34A
  \l C34B
  \l C34C
  \l C34D
  \l C34E
  \l C34F
  \l C350
  \l C351
  \l C352
  \l C353
  \l C354
  \l C355
  \l C356
  \l C357
  \l C358
  \l C359
  \l C35A
  \l C35B
  \l C35C
  \l C35D
  \l C35E
  \l C35F
  \l C360
  \l C361
  \l C362
  \l C363
  \l C364
  \l C365
  \l C366
  \l C367
  \l C368
  \l C369
  \l C36A
  \l C36B
  \l C36C
  \l C36D
  \l C36E
  \l C36F
  \l C370
  \l C371
  \l C372
  \l C373
  \l C374
  \l C375
  \l C376
  \l C377
  \l C378
  \l C379
  \l C37A
  \l C37B
  \l C37C
  \l C37D
  \l C37E
  \l C37F
  \l C380
  \l C381
  \l C382
  \l C383
  \l C384
  \l C385
  \l C386
  \l C387
  \l C388
  \l C389
  \l C38A
  \l C38B
  \l C38C
  \l C38D
  \l C38E
  \l C38F
  \l C390
  \l C391
  \l C392
  \l C393
  \l C394
  \l C395
  \l C396
  \l C397
  \l C398
  \l C399
  \l C39A
  \l C39B
  \l C39C
  \l C39D
  \l C39E
  \l C39F
  \l C3A0
  \l C3A1
  \l C3A2
  \l C3A3
  \l C3A4
  \l C3A5
  \l C3A6
  \l C3A7
  \l C3A8
  \l C3A9
  \l C3AA
  \l C3AB
  \l C3AC
  \l C3AD
  \l C3AE
  \l C3AF
  \l C3B0
  \l C3B1
  \l C3B2
  \l C3B3
  \l C3B4
  \l C3B5
  \l C3B6
  \l C3B7
  \l C3B8
  \l C3B9
  \l C3BA
  \l C3BB
  \l C3BC
  \l C3BD
  \l C3BE
  \l C3BF
  \l C3C0
  \l C3C1
  \l C3C2
  \l C3C3
  \l C3C4
  \l C3C5
  \l C3C6
  \l C3C7
  \l C3C8
  \l C3C9
  \l C3CA
  \l C3CB
  \l C3CC
  \l C3CD
  \l C3CE
  \l C3CF
  \l C3D0
  \l C3D1
  \l C3D2
  \l C3D3
  \l C3D4
  \l C3D5
  \l C3D6
  \l C3D7
  \l C3D8
  \l C3D9
  \l C3DA
  \l C3DB
  \l C3DC
  \l C3DD
  \l C3DE
  \l C3DF
  \l C3E0
  \l C3E1
  \l C3E2
  \l C3E3
  \l C3E4
  \l C3E5
  \l C3E6
  \l C3E7
  \l C3E8
  \l C3E9
  \l C3EA
  \l C3EB
  \l C3EC
  \l C3ED
  \l C3EE
  \l C3EF
  \l C3F0
  \l C3F1
  \l C3F2
  \l C3F3
  \l C3F4
  \l C3F5
  \l C3F6
  \l C3F7
  \l C3F8
  \l C3F9
  \l C3FA
  \l C3FB
  \l C3FC
  \l C3FD
  \l C3FE
  \l C3FF
  \l C400
  \l C401
  \l C402
  \l C403
  \l C404
  \l C405
  \l C406
  \l C407
  \l C408
  \l C409
  \l C40A
  \l C40B
  \l C40C
  \l C40D
  \l C40E
  \l C40F
  \l C410
  \l C411
  \l C412
  \l C413
  \l C414
  \l C415
  \l C416
  \l C417
  \l C418
  \l C419
  \l C41A
  \l C41B
  \l C41C
  \l C41D
  \l C41E
  \l C41F
  \l C420
  \l C421
  \l C422
  \l C423
  \l C424
  \l C425
  \l C426
  \l C427
  \l C428
  \l C429
  \l C42A
  \l C42B
  \l C42C
  \l C42D
  \l C42E
  \l C42F
  \l C430
  \l C431
  \l C432
  \l C433
  \l C434
  \l C435
  \l C436
  \l C437
  \l C438
  \l C439
  \l C43A
  \l C43B
  \l C43C
  \l C43D
  \l C43E
  \l C43F
  \l C440
  \l C441
  \l C442
  \l C443
  \l C444
  \l C445
  \l C446
  \l C447
  \l C448
  \l C449
  \l C44A
  \l C44B
  \l C44C
  \l C44D
  \l C44E
  \l C44F
  \l C450
  \l C451
  \l C452
  \l C453
  \l C454
  \l C455
  \l C456
  \l C457
  \l C458
  \l C459
  \l C45A
  \l C45B
  \l C45C
  \l C45D
  \l C45E
  \l C45F
  \l C460
  \l C461
  \l C462
  \l C463
  \l C464
  \l C465
  \l C466
  \l C467
  \l C468
  \l C469
  \l C46A
  \l C46B
  \l C46C
  \l C46D
  \l C46E
  \l C46F
  \l C470
  \l C471
  \l C472
  \l C473
  \l C474
  \l C475
  \l C476
  \l C477
  \l C478
  \l C479
  \l C47A
  \l C47B
  \l C47C
  \l C47D
  \l C47E
  \l C47F
  \l C480
  \l C481
  \l C482
  \l C483
  \l C484
  \l C485
  \l C486
  \l C487
  \l C488
  \l C489
  \l C48A
  \l C48B
  \l C48C
  \l C48D
  \l C48E
  \l C48F
  \l C490
  \l C491
  \l C492
  \l C493
  \l C494
  \l C495
  \l C496
  \l C497
  \l C498
  \l C499
  \l C49A
  \l C49B
  \l C49C
  \l C49D
  \l C49E
  \l C49F
  \l C4A0
  \l C4A1
  \l C4A2
  \l C4A3
  \l C4A4
  \l C4A5
  \l C4A6
  \l C4A7
  \l C4A8
  \l C4A9
  \l C4AA
  \l C4AB
  \l C4AC
  \l C4AD
  \l C4AE
  \l C4AF
  \l C4B0
  \l C4B1
  \l C4B2
  \l C4B3
  \l C4B4
  \l C4B5
  \l C4B6
  \l C4B7
  \l C4B8
  \l C4B9
  \l C4BA
  \l C4BB
  \l C4BC
  \l C4BD
  \l C4BE
  \l C4BF
  \l C4C0
  \l C4C1
  \l C4C2
  \l C4C3
  \l C4C4
  \l C4C5
  \l C4C6
  \l C4C7
  \l C4C8
  \l C4C9
  \l C4CA
  \l C4CB
  \l C4CC
  \l C4CD
  \l C4CE
  \l C4CF
  \l C4D0
  \l C4D1
  \l C4D2
  \l C4D3
  \l C4D4
  \l C4D5
  \l C4D6
  \l C4D7
  \l C4D8
  \l C4D9
  \l C4DA
  \l C4DB
  \l C4DC
  \l C4DD
  \l C4DE
  \l C4DF
  \l C4E0
  \l C4E1
  \l C4E2
  \l C4E3
  \l C4E4
  \l C4E5
  \l C4E6
  \l C4E7
  \l C4E8
  \l C4E9
  \l C4EA
  \l C4EB
  \l C4EC
  \l C4ED
  \l C4EE
  \l C4EF
  \l C4F0
  \l C4F1
  \l C4F2
  \l C4F3
  \l C4F4
  \l C4F5
  \l C4F6
  \l C4F7
  \l C4F8
  \l C4F9
  \l C4FA
  \l C4FB
  \l C4FC
  \l C4FD
  \l C4FE
  \l C4FF
  \l C500
  \l C501
  \l C502
  \l C503
  \l C504
  \l C505
  \l C506
  \l C507
  \l C508
  \l C509
  \l C50A
  \l C50B
  \l C50C
  \l C50D
  \l C50E
  \l C50F
  \l C510
  \l C511
  \l C512
  \l C513
  \l C514
  \l C515
  \l C516
  \l C517
  \l C518
  \l C519
  \l C51A
  \l C51B
  \l C51C
  \l C51D
  \l C51E
  \l C51F
  \l C520
  \l C521
  \l C522
  \l C523
  \l C524
  \l C525
  \l C526
  \l C527
  \l C528
  \l C529
  \l C52A
  \l C52B
  \l C52C
  \l C52D
  \l C52E
  \l C52F
  \l C530
  \l C531
  \l C532
  \l C533
  \l C534
  \l C535
  \l C536
  \l C537
  \l C538
  \l C539
  \l C53A
  \l C53B
  \l C53C
  \l C53D
  \l C53E
  \l C53F
  \l C540
  \l C541
  \l C542
  \l C543
  \l C544
  \l C545
  \l C546
  \l C547
  \l C548
  \l C549
  \l C54A
  \l C54B
  \l C54C
  \l C54D
  \l C54E
  \l C54F
  \l C550
  \l C551
  \l C552
  \l C553
  \l C554
  \l C555
  \l C556
  \l C557
  \l C558
  \l C559
  \l C55A
  \l C55B
  \l C55C
  \l C55D
  \l C55E
  \l C55F
  \l C560
  \l C561
  \l C562
  \l C563
  \l C564
  \l C565
  \l C566
  \l C567
  \l C568
  \l C569
  \l C56A
  \l C56B
  \l C56C
  \l C56D
  \l C56E
  \l C56F
  \l C570
  \l C571
  \l C572
  \l C573
  \l C574
  \l C575
  \l C576
  \l C577
  \l C578
  \l C579
  \l C57A
  \l C57B
  \l C57C
  \l C57D
  \l C57E
  \l C57F
  \l C580
  \l C581
  \l C582
  \l C583
  \l C584
  \l C585
  \l C586
  \l C587
  \l C588
  \l C589
  \l C58A
  \l C58B
  \l C58C
  \l C58D
  \l C58E
  \l C58F
  \l C590
  \l C591
  \l C592
  \l C593
  \l C594
  \l C595
  \l C596
  \l C597
  \l C598
  \l C599
  \l C59A
  \l C59B
  \l C59C
  \l C59D
  \l C59E
  \l C59F
  \l C5A0
  \l C5A1
  \l C5A2
  \l C5A3
  \l C5A4
  \l C5A5
  \l C5A6
  \l C5A7
  \l C5A8
  \l C5A9
  \l C5AA
  \l C5AB
  \l C5AC
  \l C5AD
  \l C5AE
  \l C5AF
  \l C5B0
  \l C5B1
  \l C5B2
  \l C5B3
  \l C5B4
  \l C5B5
  \l C5B6
  \l C5B7
  \l C5B8
  \l C5B9
  \l C5BA
  \l C5BB
  \l C5BC
  \l C5BD
  \l C5BE
  \l C5BF
  \l C5C0
  \l C5C1
  \l C5C2
  \l C5C3
  \l C5C4
  \l C5C5
  \l C5C6
  \l C5C7
  \l C5C8
  \l C5C9
  \l C5CA
  \l C5CB
  \l C5CC
  \l C5CD
  \l C5CE
  \l C5CF
  \l C5D0
  \l C5D1
  \l C5D2
  \l C5D3
  \l C5D4
  \l C5D5
  \l C5D6
  \l C5D7
  \l C5D8
  \l C5D9
  \l C5DA
  \l C5DB
  \l C5DC
  \l C5DD
  \l C5DE
  \l C5DF
  \l C5E0
  \l C5E1
  \l C5E2
  \l C5E3
  \l C5E4
  \l C5E5
  \l C5E6
  \l C5E7
  \l C5E8
  \l C5E9
  \l C5EA
  \l C5EB
  \l C5EC
  \l C5ED
  \l C5EE
  \l C5EF
  \l C5F0
  \l C5F1
  \l C5F2
  \l C5F3
  \l C5F4
  \l C5F5
  \l C5F6
  \l C5F7
  \l C5F8
  \l C5F9
  \l C5FA
  \l C5FB
  \l C5FC
  \l C5FD
  \l C5FE
  \l C5FF
  \l C600
  \l C601
  \l C602
  \l C603
  \l C604
  \l C605
  \l C606
  \l C607
  \l C608
  \l C609
  \l C60A
  \l C60B
  \l C60C
  \l C60D
  \l C60E
  \l C60F
  \l C610
  \l C611
  \l C612
  \l C613
  \l C614
  \l C615
  \l C616
  \l C617
  \l C618
  \l C619
  \l C61A
  \l C61B
  \l C61C
  \l C61D
  \l C61E
  \l C61F
  \l C620
  \l C621
  \l C622
  \l C623
  \l C624
  \l C625
  \l C626
  \l C627
  \l C628
  \l C629
  \l C62A
  \l C62B
  \l C62C
  \l C62D
  \l C62E
  \l C62F
  \l C630
  \l C631
  \l C632
  \l C633
  \l C634
  \l C635
  \l C636
  \l C637
  \l C638
  \l C639
  \l C63A
  \l C63B
  \l C63C
  \l C63D
  \l C63E
  \l C63F
  \l C640
  \l C641
  \l C642
  \l C643
  \l C644
  \l C645
  \l C646
  \l C647
  \l C648
  \l C649
  \l C64A
  \l C64B
  \l C64C
  \l C64D
  \l C64E
  \l C64F
  \l C650
  \l C651
  \l C652
  \l C653
  \l C654
  \l C655
  \l C656
  \l C657
  \l C658
  \l C659
  \l C65A
  \l C65B
  \l C65C
  \l C65D
  \l C65E
  \l C65F
  \l C660
  \l C661
  \l C662
  \l C663
  \l C664
  \l C665
  \l C666
  \l C667
  \l C668
  \l C669
  \l C66A
  \l C66B
  \l C66C
  \l C66D
  \l C66E
  \l C66F
  \l C670
  \l C671
  \l C672
  \l C673
  \l C674
  \l C675
  \l C676
  \l C677
  \l C678
  \l C679
  \l C67A
  \l C67B
  \l C67C
  \l C67D
  \l C67E
  \l C67F
  \l C680
  \l C681
  \l C682
  \l C683
  \l C684
  \l C685
  \l C686
  \l C687
  \l C688
  \l C689
  \l C68A
  \l C68B
  \l C68C
  \l C68D
  \l C68E
  \l C68F
  \l C690
  \l C691
  \l C692
  \l C693
  \l C694
  \l C695
  \l C696
  \l C697
  \l C698
  \l C699
  \l C69A
  \l C69B
  \l C69C
  \l C69D
  \l C69E
  \l C69F
  \l C6A0
  \l C6A1
  \l C6A2
  \l C6A3
  \l C6A4
  \l C6A5
  \l C6A6
  \l C6A7
  \l C6A8
  \l C6A9
  \l C6AA
  \l C6AB
  \l C6AC
  \l C6AD
  \l C6AE
  \l C6AF
  \l C6B0
  \l C6B1
  \l C6B2
  \l C6B3
  \l C6B4
  \l C6B5
  \l C6B6
  \l C6B7
  \l C6B8
  \l C6B9
  \l C6BA
  \l C6BB
  \l C6BC
  \l C6BD
  \l C6BE
  \l C6BF
  \l C6C0
  \l C6C1
  \l C6C2
  \l C6C3
  \l C6C4
  \l C6C5
  \l C6C6
  \l C6C7
  \l C6C8
  \l C6C9
  \l C6CA
  \l C6CB
  \l C6CC
  \l C6CD
  \l C6CE
  \l C6CF
  \l C6D0
  \l C6D1
  \l C6D2
  \l C6D3
  \l C6D4
  \l C6D5
  \l C6D6
  \l C6D7
  \l C6D8
  \l C6D9
  \l C6DA
  \l C6DB
  \l C6DC
  \l C6DD
  \l C6DE
  \l C6DF
  \l C6E0
  \l C6E1
  \l C6E2
  \l C6E3
  \l C6E4
  \l C6E5
  \l C6E6
  \l C6E7
  \l C6E8
  \l C6E9
  \l C6EA
  \l C6EB
  \l C6EC
  \l C6ED
  \l C6EE
  \l C6EF
  \l C6F0
  \l C6F1
  \l C6F2
  \l C6F3
  \l C6F4
  \l C6F5
  \l C6F6
  \l C6F7
  \l C6F8
  \l C6F9
  \l C6FA
  \l C6FB
  \l C6FC
  \l C6FD
  \l C6FE
  \l C6FF
  \l C700
  \l C701
  \l C702
  \l C703
  \l C704
  \l C705
  \l C706
  \l C707
  \l C708
  \l C709
  \l C70A
  \l C70B
  \l C70C
  \l C70D
  \l C70E
  \l C70F
  \l C710
  \l C711
  \l C712
  \l C713
  \l C714
  \l C715
  \l C716
  \l C717
  \l C718
  \l C719
  \l C71A
  \l C71B
  \l C71C
  \l C71D
  \l C71E
  \l C71F
  \l C720
  \l C721
  \l C722
  \l C723
  \l C724
  \l C725
  \l C726
  \l C727
  \l C728
  \l C729
  \l C72A
  \l C72B
  \l C72C
  \l C72D
  \l C72E
  \l C72F
  \l C730
  \l C731
  \l C732
  \l C733
  \l C734
  \l C735
  \l C736
  \l C737
  \l C738
  \l C739
  \l C73A
  \l C73B
  \l C73C
  \l C73D
  \l C73E
  \l C73F
  \l C740
  \l C741
  \l C742
  \l C743
  \l C744
  \l C745
  \l C746
  \l C747
  \l C748
  \l C749
  \l C74A
  \l C74B
  \l C74C
  \l C74D
  \l C74E
  \l C74F
  \l C750
  \l C751
  \l C752
  \l C753
  \l C754
  \l C755
  \l C756
  \l C757
  \l C758
  \l C759
  \l C75A
  \l C75B
  \l C75C
  \l C75D
  \l C75E
  \l C75F
  \l C760
  \l C761
  \l C762
  \l C763
  \l C764
  \l C765
  \l C766
  \l C767
  \l C768
  \l C769
  \l C76A
  \l C76B
  \l C76C
  \l C76D
  \l C76E
  \l C76F
  \l C770
  \l C771
  \l C772
  \l C773
  \l C774
  \l C775
  \l C776
  \l C777
  \l C778
  \l C779
  \l C77A
  \l C77B
  \l C77C
  \l C77D
  \l C77E
  \l C77F
  \l C780
  \l C781
  \l C782
  \l C783
  \l C784
  \l C785
  \l C786
  \l C787
  \l C788
  \l C789
  \l C78A
  \l C78B
  \l C78C
  \l C78D
  \l C78E
  \l C78F
  \l C790
  \l C791
  \l C792
  \l C793
  \l C794
  \l C795
  \l C796
  \l C797
  \l C798
  \l C799
  \l C79A
  \l C79B
  \l C79C
  \l C79D
  \l C79E
  \l C79F
  \l C7A0
  \l C7A1
  \l C7A2
  \l C7A3
  \l C7A4
  \l C7A5
  \l C7A6
  \l C7A7
  \l C7A8
  \l C7A9
  \l C7AA
  \l C7AB
  \l C7AC
  \l C7AD
  \l C7AE
  \l C7AF
  \l C7B0
  \l C7B1
  \l C7B2
  \l C7B3
  \l C7B4
  \l C7B5
  \l C7B6
  \l C7B7
  \l C7B8
  \l C7B9
  \l C7BA
  \l C7BB
  \l C7BC
  \l C7BD
  \l C7BE
  \l C7BF
  \l C7C0
  \l C7C1
  \l C7C2
  \l C7C3
  \l C7C4
  \l C7C5
  \l C7C6
  \l C7C7
  \l C7C8
  \l C7C9
  \l C7CA
  \l C7CB
  \l C7CC
  \l C7CD
  \l C7CE
  \l C7CF
  \l C7D0
  \l C7D1
  \l C7D2
  \l C7D3
  \l C7D4
  \l C7D5
  \l C7D6
  \l C7D7
  \l C7D8
  \l C7D9
  \l C7DA
  \l C7DB
  \l C7DC
  \l C7DD
  \l C7DE
  \l C7DF
  \l C7E0
  \l C7E1
  \l C7E2
  \l C7E3
  \l C7E4
  \l C7E5
  \l C7E6
  \l C7E7
  \l C7E8
  \l C7E9
  \l C7EA
  \l C7EB
  \l C7EC
  \l C7ED
  \l C7EE
  \l C7EF
  \l C7F0
  \l C7F1
  \l C7F2
  \l C7F3
  \l C7F4
  \l C7F5
  \l C7F6
  \l C7F7
  \l C7F8
  \l C7F9
  \l C7FA
  \l C7FB
  \l C7FC
  \l C7FD
  \l C7FE
  \l C7FF
  \l C800
  \l C801
  \l C802
  \l C803
  \l C804
  \l C805
  \l C806
  \l C807
  \l C808
  \l C809
  \l C80A
  \l C80B
  \l C80C
  \l C80D
  \l C80E
  \l C80F
  \l C810
  \l C811
  \l C812
  \l C813
  \l C814
  \l C815
  \l C816
  \l C817
  \l C818
  \l C819
  \l C81A
  \l C81B
  \l C81C
  \l C81D
  \l C81E
  \l C81F
  \l C820
  \l C821
  \l C822
  \l C823
  \l C824
  \l C825
  \l C826
  \l C827
  \l C828
  \l C829
  \l C82A
  \l C82B
  \l C82C
  \l C82D
  \l C82E
  \l C82F
  \l C830
  \l C831
  \l C832
  \l C833
  \l C834
  \l C835
  \l C836
  \l C837
  \l C838
  \l C839
  \l C83A
  \l C83B
  \l C83C
  \l C83D
  \l C83E
  \l C83F
  \l C840
  \l C841
  \l C842
  \l C843
  \l C844
  \l C845
  \l C846
  \l C847
  \l C848
  \l C849
  \l C84A
  \l C84B
  \l C84C
  \l C84D
  \l C84E
  \l C84F
  \l C850
  \l C851
  \l C852
  \l C853
  \l C854
  \l C855
  \l C856
  \l C857
  \l C858
  \l C859
  \l C85A
  \l C85B
  \l C85C
  \l C85D
  \l C85E
  \l C85F
  \l C860
  \l C861
  \l C862
  \l C863
  \l C864
  \l C865
  \l C866
  \l C867
  \l C868
  \l C869
  \l C86A
  \l C86B
  \l C86C
  \l C86D
  \l C86E
  \l C86F
  \l C870
  \l C871
  \l C872
  \l C873
  \l C874
  \l C875
  \l C876
  \l C877
  \l C878
  \l C879
  \l C87A
  \l C87B
  \l C87C
  \l C87D
  \l C87E
  \l C87F
  \l C880
  \l C881
  \l C882
  \l C883
  \l C884
  \l C885
  \l C886
  \l C887
  \l C888
  \l C889
  \l C88A
  \l C88B
  \l C88C
  \l C88D
  \l C88E
  \l C88F
  \l C890
  \l C891
  \l C892
  \l C893
  \l C894
  \l C895
  \l C896
  \l C897
  \l C898
  \l C899
  \l C89A
  \l C89B
  \l C89C
  \l C89D
  \l C89E
  \l C89F
  \l C8A0
  \l C8A1
  \l C8A2
  \l C8A3
  \l C8A4
  \l C8A5
  \l C8A6
  \l C8A7
  \l C8A8
  \l C8A9
  \l C8AA
  \l C8AB
  \l C8AC
  \l C8AD
  \l C8AE
  \l C8AF
  \l C8B0
  \l C8B1
  \l C8B2
  \l C8B3
  \l C8B4
  \l C8B5
  \l C8B6
  \l C8B7
  \l C8B8
  \l C8B9
  \l C8BA
  \l C8BB
  \l C8BC
  \l C8BD
  \l C8BE
  \l C8BF
  \l C8C0
  \l C8C1
  \l C8C2
  \l C8C3
  \l C8C4
  \l C8C5
  \l C8C6
  \l C8C7
  \l C8C8
  \l C8C9
  \l C8CA
  \l C8CB
  \l C8CC
  \l C8CD
  \l C8CE
  \l C8CF
  \l C8D0
  \l C8D1
  \l C8D2
  \l C8D3
  \l C8D4
  \l C8D5
  \l C8D6
  \l C8D7
  \l C8D8
  \l C8D9
  \l C8DA
  \l C8DB
  \l C8DC
  \l C8DD
  \l C8DE
  \l C8DF
  \l C8E0
  \l C8E1
  \l C8E2
  \l C8E3
  \l C8E4
  \l C8E5
  \l C8E6
  \l C8E7
  \l C8E8
  \l C8E9
  \l C8EA
  \l C8EB
  \l C8EC
  \l C8ED
  \l C8EE
  \l C8EF
  \l C8F0
  \l C8F1
  \l C8F2
  \l C8F3
  \l C8F4
  \l C8F5
  \l C8F6
  \l C8F7
  \l C8F8
  \l C8F9
  \l C8FA
  \l C8FB
  \l C8FC
  \l C8FD
  \l C8FE
  \l C8FF
  \l C900
  \l C901
  \l C902
  \l C903
  \l C904
  \l C905
  \l C906
  \l C907
  \l C908
  \l C909
  \l C90A
  \l C90B
  \l C90C
  \l C90D
  \l C90E
  \l C90F
  \l C910
  \l C911
  \l C912
  \l C913
  \l C914
  \l C915
  \l C916
  \l C917
  \l C918
  \l C919
  \l C91A
  \l C91B
  \l C91C
  \l C91D
  \l C91E
  \l C91F
  \l C920
  \l C921
  \l C922
  \l C923
  \l C924
  \l C925
  \l C926
  \l C927
  \l C928
  \l C929
  \l C92A
  \l C92B
  \l C92C
  \l C92D
  \l C92E
  \l C92F
  \l C930
  \l C931
  \l C932
  \l C933
  \l C934
  \l C935
  \l C936
  \l C937
  \l C938
  \l C939
  \l C93A
  \l C93B
  \l C93C
  \l C93D
  \l C93E
  \l C93F
  \l C940
  \l C941
  \l C942
  \l C943
  \l C944
  \l C945
  \l C946
  \l C947
  \l C948
  \l C949
  \l C94A
  \l C94B
  \l C94C
  \l C94D
  \l C94E
  \l C94F
  \l C950
  \l C951
  \l C952
  \l C953
  \l C954
  \l C955
  \l C956
  \l C957
  \l C958
  \l C959
  \l C95A
  \l C95B
  \l C95C
  \l C95D
  \l C95E
  \l C95F
  \l C960
  \l C961
  \l C962
  \l C963
  \l C964
  \l C965
  \l C966
  \l C967
  \l C968
  \l C969
  \l C96A
  \l C96B
  \l C96C
  \l C96D
  \l C96E
  \l C96F
  \l C970
  \l C971
  \l C972
  \l C973
  \l C974
  \l C975
  \l C976
  \l C977
  \l C978
  \l C979
  \l C97A
  \l C97B
  \l C97C
  \l C97D
  \l C97E
  \l C97F
  \l C980
  \l C981
  \l C982
  \l C983
  \l C984
  \l C985
  \l C986
  \l C987
  \l C988
  \l C989
  \l C98A
  \l C98B
  \l C98C
  \l C98D
  \l C98E
  \l C98F
  \l C990
  \l C991
  \l C992
  \l C993
  \l C994
  \l C995
  \l C996
  \l C997
  \l C998
  \l C999
  \l C99A
  \l C99B
  \l C99C
  \l C99D
  \l C99E
  \l C99F
  \l C9A0
  \l C9A1
  \l C9A2
  \l C9A3
  \l C9A4
  \l C9A5
  \l C9A6
  \l C9A7
  \l C9A8
  \l C9A9
  \l C9AA
  \l C9AB
  \l C9AC
  \l C9AD
  \l C9AE
  \l C9AF
  \l C9B0
  \l C9B1
  \l C9B2
  \l C9B3
  \l C9B4
  \l C9B5
  \l C9B6
  \l C9B7
  \l C9B8
  \l C9B9
  \l C9BA
  \l C9BB
  \l C9BC
  \l C9BD
  \l C9BE
  \l C9BF
  \l C9C0
  \l C9C1
  \l C9C2
  \l C9C3
  \l C9C4
  \l C9C5
  \l C9C6
  \l C9C7
  \l C9C8
  \l C9C9
  \l C9CA
  \l C9CB
  \l C9CC
  \l C9CD
  \l C9CE
  \l C9CF
  \l C9D0
  \l C9D1
  \l C9D2
  \l C9D3
  \l C9D4
  \l C9D5
  \l C9D6
  \l C9D7
  \l C9D8
  \l C9D9
  \l C9DA
  \l C9DB
  \l C9DC
  \l C9DD
  \l C9DE
  \l C9DF
  \l C9E0
  \l C9E1
  \l C9E2
  \l C9E3
  \l C9E4
  \l C9E5
  \l C9E6
  \l C9E7
  \l C9E8
  \l C9E9
  \l C9EA
  \l C9EB
  \l C9EC
  \l C9ED
  \l C9EE
  \l C9EF
  \l C9F0
  \l C9F1
  \l C9F2
  \l C9F3
  \l C9F4
  \l C9F5
  \l C9F6
  \l C9F7
  \l C9F8
  \l C9F9
  \l C9FA
  \l C9FB
  \l C9FC
  \l C9FD
  \l C9FE
  \l C9FF
  \l CA00
  \l CA01
  \l CA02
  \l CA03
  \l CA04
  \l CA05
  \l CA06
  \l CA07
  \l CA08
  \l CA09
  \l CA0A
  \l CA0B
  \l CA0C
  \l CA0D
  \l CA0E
  \l CA0F
  \l CA10
  \l CA11
  \l CA12
  \l CA13
  \l CA14
  \l CA15
  \l CA16
  \l CA17
  \l CA18
  \l CA19
  \l CA1A
  \l CA1B
  \l CA1C
  \l CA1D
  \l CA1E
  \l CA1F
  \l CA20
  \l CA21
  \l CA22
  \l CA23
  \l CA24
  \l CA25
  \l CA26
  \l CA27
  \l CA28
  \l CA29
  \l CA2A
  \l CA2B
  \l CA2C
  \l CA2D
  \l CA2E
  \l CA2F
  \l CA30
  \l CA31
  \l CA32
  \l CA33
  \l CA34
  \l CA35
  \l CA36
  \l CA37
  \l CA38
  \l CA39
  \l CA3A
  \l CA3B
  \l CA3C
  \l CA3D
  \l CA3E
  \l CA3F
  \l CA40
  \l CA41
  \l CA42
  \l CA43
  \l CA44
  \l CA45
  \l CA46
  \l CA47
  \l CA48
  \l CA49
  \l CA4A
  \l CA4B
  \l CA4C
  \l CA4D
  \l CA4E
  \l CA4F
  \l CA50
  \l CA51
  \l CA52
  \l CA53
  \l CA54
  \l CA55
  \l CA56
  \l CA57
  \l CA58
  \l CA59
  \l CA5A
  \l CA5B
  \l CA5C
  \l CA5D
  \l CA5E
  \l CA5F
  \l CA60
  \l CA61
  \l CA62
  \l CA63
  \l CA64
  \l CA65
  \l CA66
  \l CA67
  \l CA68
  \l CA69
  \l CA6A
  \l CA6B
  \l CA6C
  \l CA6D
  \l CA6E
  \l CA6F
  \l CA70
  \l CA71
  \l CA72
  \l CA73
  \l CA74
  \l CA75
  \l CA76
  \l CA77
  \l CA78
  \l CA79
  \l CA7A
  \l CA7B
  \l CA7C
  \l CA7D
  \l CA7E
  \l CA7F
  \l CA80
  \l CA81
  \l CA82
  \l CA83
  \l CA84
  \l CA85
  \l CA86
  \l CA87
  \l CA88
  \l CA89
  \l CA8A
  \l CA8B
  \l CA8C
  \l CA8D
  \l CA8E
  \l CA8F
  \l CA90
  \l CA91
  \l CA92
  \l CA93
  \l CA94
  \l CA95
  \l CA96
  \l CA97
  \l CA98
  \l CA99
  \l CA9A
  \l CA9B
  \l CA9C
  \l CA9D
  \l CA9E
  \l CA9F
  \l CAA0
  \l CAA1
  \l CAA2
  \l CAA3
  \l CAA4
  \l CAA5
  \l CAA6
  \l CAA7
  \l CAA8
  \l CAA9
  \l CAAA
  \l CAAB
  \l CAAC
  \l CAAD
  \l CAAE
  \l CAAF
  \l CAB0
  \l CAB1
  \l CAB2
  \l CAB3
  \l CAB4
  \l CAB5
  \l CAB6
  \l CAB7
  \l CAB8
  \l CAB9
  \l CABA
  \l CABB
  \l CABC
  \l CABD
  \l CABE
  \l CABF
  \l CAC0
  \l CAC1
  \l CAC2
  \l CAC3
  \l CAC4
  \l CAC5
  \l CAC6
  \l CAC7
  \l CAC8
  \l CAC9
  \l CACA
  \l CACB
  \l CACC
  \l CACD
  \l CACE
  \l CACF
  \l CAD0
  \l CAD1
  \l CAD2
  \l CAD3
  \l CAD4
  \l CAD5
  \l CAD6
  \l CAD7
  \l CAD8
  \l CAD9
  \l CADA
  \l CADB
  \l CADC
  \l CADD
  \l CADE
  \l CADF
  \l CAE0
  \l CAE1
  \l CAE2
  \l CAE3
  \l CAE4
  \l CAE5
  \l CAE6
  \l CAE7
  \l CAE8
  \l CAE9
  \l CAEA
  \l CAEB
  \l CAEC
  \l CAED
  \l CAEE
  \l CAEF
  \l CAF0
  \l CAF1
  \l CAF2
  \l CAF3
  \l CAF4
  \l CAF5
  \l CAF6
  \l CAF7
  \l CAF8
  \l CAF9
  \l CAFA
  \l CAFB
  \l CAFC
  \l CAFD
  \l CAFE
  \l CAFF
  \l CB00
  \l CB01
  \l CB02
  \l CB03
  \l CB04
  \l CB05
  \l CB06
  \l CB07
  \l CB08
  \l CB09
  \l CB0A
  \l CB0B
  \l CB0C
  \l CB0D
  \l CB0E
  \l CB0F
  \l CB10
  \l CB11
  \l CB12
  \l CB13
  \l CB14
  \l CB15
  \l CB16
  \l CB17
  \l CB18
  \l CB19
  \l CB1A
  \l CB1B
  \l CB1C
  \l CB1D
  \l CB1E
  \l CB1F
  \l CB20
  \l CB21
  \l CB22
  \l CB23
  \l CB24
  \l CB25
  \l CB26
  \l CB27
  \l CB28
  \l CB29
  \l CB2A
  \l CB2B
  \l CB2C
  \l CB2D
  \l CB2E
  \l CB2F
  \l CB30
  \l CB31
  \l CB32
  \l CB33
  \l CB34
  \l CB35
  \l CB36
  \l CB37
  \l CB38
  \l CB39
  \l CB3A
  \l CB3B
  \l CB3C
  \l CB3D
  \l CB3E
  \l CB3F
  \l CB40
  \l CB41
  \l CB42
  \l CB43
  \l CB44
  \l CB45
  \l CB46
  \l CB47
  \l CB48
  \l CB49
  \l CB4A
  \l CB4B
  \l CB4C
  \l CB4D
  \l CB4E
  \l CB4F
  \l CB50
  \l CB51
  \l CB52
  \l CB53
  \l CB54
  \l CB55
  \l CB56
  \l CB57
  \l CB58
  \l CB59
  \l CB5A
  \l CB5B
  \l CB5C
  \l CB5D
  \l CB5E
  \l CB5F
  \l CB60
  \l CB61
  \l CB62
  \l CB63
  \l CB64
  \l CB65
  \l CB66
  \l CB67
  \l CB68
  \l CB69
  \l CB6A
  \l CB6B
  \l CB6C
  \l CB6D
  \l CB6E
  \l CB6F
  \l CB70
  \l CB71
  \l CB72
  \l CB73
  \l CB74
  \l CB75
  \l CB76
  \l CB77
  \l CB78
  \l CB79
  \l CB7A
  \l CB7B
  \l CB7C
  \l CB7D
  \l CB7E
  \l CB7F
  \l CB80
  \l CB81
  \l CB82
  \l CB83
  \l CB84
  \l CB85
  \l CB86
  \l CB87
  \l CB88
  \l CB89
  \l CB8A
  \l CB8B
  \l CB8C
  \l CB8D
  \l CB8E
  \l CB8F
  \l CB90
  \l CB91
  \l CB92
  \l CB93
  \l CB94
  \l CB95
  \l CB96
  \l CB97
  \l CB98
  \l CB99
  \l CB9A
  \l CB9B
  \l CB9C
  \l CB9D
  \l CB9E
  \l CB9F
  \l CBA0
  \l CBA1
  \l CBA2
  \l CBA3
  \l CBA4
  \l CBA5
  \l CBA6
  \l CBA7
  \l CBA8
  \l CBA9
  \l CBAA
  \l CBAB
  \l CBAC
  \l CBAD
  \l CBAE
  \l CBAF
  \l CBB0
  \l CBB1
  \l CBB2
  \l CBB3
  \l CBB4
  \l CBB5
  \l CBB6
  \l CBB7
  \l CBB8
  \l CBB9
  \l CBBA
  \l CBBB
  \l CBBC
  \l CBBD
  \l CBBE
  \l CBBF
  \l CBC0
  \l CBC1
  \l CBC2
  \l CBC3
  \l CBC4
  \l CBC5
  \l CBC6
  \l CBC7
  \l CBC8
  \l CBC9
  \l CBCA
  \l CBCB
  \l CBCC
  \l CBCD
  \l CBCE
  \l CBCF
  \l CBD0
  \l CBD1
  \l CBD2
  \l CBD3
  \l CBD4
  \l CBD5
  \l CBD6
  \l CBD7
  \l CBD8
  \l CBD9
  \l CBDA
  \l CBDB
  \l CBDC
  \l CBDD
  \l CBDE
  \l CBDF
  \l CBE0
  \l CBE1
  \l CBE2
  \l CBE3
  \l CBE4
  \l CBE5
  \l CBE6
  \l CBE7
  \l CBE8
  \l CBE9
  \l CBEA
  \l CBEB
  \l CBEC
  \l CBED
  \l CBEE
  \l CBEF
  \l CBF0
  \l CBF1
  \l CBF2
  \l CBF3
  \l CBF4
  \l CBF5
  \l CBF6
  \l CBF7
  \l CBF8
  \l CBF9
  \l CBFA
  \l CBFB
  \l CBFC
  \l CBFD
  \l CBFE
  \l CBFF
  \l CC00
  \l CC01
  \l CC02
  \l CC03
  \l CC04
  \l CC05
  \l CC06
  \l CC07
  \l CC08
  \l CC09
  \l CC0A
  \l CC0B
  \l CC0C
  \l CC0D
  \l CC0E
  \l CC0F
  \l CC10
  \l CC11
  \l CC12
  \l CC13
  \l CC14
  \l CC15
  \l CC16
  \l CC17
  \l CC18
  \l CC19
  \l CC1A
  \l CC1B
  \l CC1C
  \l CC1D
  \l CC1E
  \l CC1F
  \l CC20
  \l CC21
  \l CC22
  \l CC23
  \l CC24
  \l CC25
  \l CC26
  \l CC27
  \l CC28
  \l CC29
  \l CC2A
  \l CC2B
  \l CC2C
  \l CC2D
  \l CC2E
  \l CC2F
  \l CC30
  \l CC31
  \l CC32
  \l CC33
  \l CC34
  \l CC35
  \l CC36
  \l CC37
  \l CC38
  \l CC39
  \l CC3A
  \l CC3B
  \l CC3C
  \l CC3D
  \l CC3E
  \l CC3F
  \l CC40
  \l CC41
  \l CC42
  \l CC43
  \l CC44
  \l CC45
  \l CC46
  \l CC47
  \l CC48
  \l CC49
  \l CC4A
  \l CC4B
  \l CC4C
  \l CC4D
  \l CC4E
  \l CC4F
  \l CC50
  \l CC51
  \l CC52
  \l CC53
  \l CC54
  \l CC55
  \l CC56
  \l CC57
  \l CC58
  \l CC59
  \l CC5A
  \l CC5B
  \l CC5C
  \l CC5D
  \l CC5E
  \l CC5F
  \l CC60
  \l CC61
  \l CC62
  \l CC63
  \l CC64
  \l CC65
  \l CC66
  \l CC67
  \l CC68
  \l CC69
  \l CC6A
  \l CC6B
  \l CC6C
  \l CC6D
  \l CC6E
  \l CC6F
  \l CC70
  \l CC71
  \l CC72
  \l CC73
  \l CC74
  \l CC75
  \l CC76
  \l CC77
  \l CC78
  \l CC79
  \l CC7A
  \l CC7B
  \l CC7C
  \l CC7D
  \l CC7E
  \l CC7F
  \l CC80
  \l CC81
  \l CC82
  \l CC83
  \l CC84
  \l CC85
  \l CC86
  \l CC87
  \l CC88
  \l CC89
  \l CC8A
  \l CC8B
  \l CC8C
  \l CC8D
  \l CC8E
  \l CC8F
  \l CC90
  \l CC91
  \l CC92
  \l CC93
  \l CC94
  \l CC95
  \l CC96
  \l CC97
  \l CC98
  \l CC99
  \l CC9A
  \l CC9B
  \l CC9C
  \l CC9D
  \l CC9E
  \l CC9F
  \l CCA0
  \l CCA1
  \l CCA2
  \l CCA3
  \l CCA4
  \l CCA5
  \l CCA6
  \l CCA7
  \l CCA8
  \l CCA9
  \l CCAA
  \l CCAB
  \l CCAC
  \l CCAD
  \l CCAE
  \l CCAF
  \l CCB0
  \l CCB1
  \l CCB2
  \l CCB3
  \l CCB4
  \l CCB5
  \l CCB6
  \l CCB7
  \l CCB8
  \l CCB9
  \l CCBA
  \l CCBB
  \l CCBC
  \l CCBD
  \l CCBE
  \l CCBF
  \l CCC0
  \l CCC1
  \l CCC2
  \l CCC3
  \l CCC4
  \l CCC5
  \l CCC6
  \l CCC7
  \l CCC8
  \l CCC9
  \l CCCA
  \l CCCB
  \l CCCC
  \l CCCD
  \l CCCE
  \l CCCF
  \l CCD0
  \l CCD1
  \l CCD2
  \l CCD3
  \l CCD4
  \l CCD5
  \l CCD6
  \l CCD7
  \l CCD8
  \l CCD9
  \l CCDA
  \l CCDB
  \l CCDC
  \l CCDD
  \l CCDE
  \l CCDF
  \l CCE0
  \l CCE1
  \l CCE2
  \l CCE3
  \l CCE4
  \l CCE5
  \l CCE6
  \l CCE7
  \l CCE8
  \l CCE9
  \l CCEA
  \l CCEB
  \l CCEC
  \l CCED
  \l CCEE
  \l CCEF
  \l CCF0
  \l CCF1
  \l CCF2
  \l CCF3
  \l CCF4
  \l CCF5
  \l CCF6
  \l CCF7
  \l CCF8
  \l CCF9
  \l CCFA
  \l CCFB
  \l CCFC
  \l CCFD
  \l CCFE
  \l CCFF
  \l CD00
  \l CD01
  \l CD02
  \l CD03
  \l CD04
  \l CD05
  \l CD06
  \l CD07
  \l CD08
  \l CD09
  \l CD0A
  \l CD0B
  \l CD0C
  \l CD0D
  \l CD0E
  \l CD0F
  \l CD10
  \l CD11
  \l CD12
  \l CD13
  \l CD14
  \l CD15
  \l CD16
  \l CD17
  \l CD18
  \l CD19
  \l CD1A
  \l CD1B
  \l CD1C
  \l CD1D
  \l CD1E
  \l CD1F
  \l CD20
  \l CD21
  \l CD22
  \l CD23
  \l CD24
  \l CD25
  \l CD26
  \l CD27
  \l CD28
  \l CD29
  \l CD2A
  \l CD2B
  \l CD2C
  \l CD2D
  \l CD2E
  \l CD2F
  \l CD30
  \l CD31
  \l CD32
  \l CD33
  \l CD34
  \l CD35
  \l CD36
  \l CD37
  \l CD38
  \l CD39
  \l CD3A
  \l CD3B
  \l CD3C
  \l CD3D
  \l CD3E
  \l CD3F
  \l CD40
  \l CD41
  \l CD42
  \l CD43
  \l CD44
  \l CD45
  \l CD46
  \l CD47
  \l CD48
  \l CD49
  \l CD4A
  \l CD4B
  \l CD4C
  \l CD4D
  \l CD4E
  \l CD4F
  \l CD50
  \l CD51
  \l CD52
  \l CD53
  \l CD54
  \l CD55
  \l CD56
  \l CD57
  \l CD58
  \l CD59
  \l CD5A
  \l CD5B
  \l CD5C
  \l CD5D
  \l CD5E
  \l CD5F
  \l CD60
  \l CD61
  \l CD62
  \l CD63
  \l CD64
  \l CD65
  \l CD66
  \l CD67
  \l CD68
  \l CD69
  \l CD6A
  \l CD6B
  \l CD6C
  \l CD6D
  \l CD6E
  \l CD6F
  \l CD70
  \l CD71
  \l CD72
  \l CD73
  \l CD74
  \l CD75
  \l CD76
  \l CD77
  \l CD78
  \l CD79
  \l CD7A
  \l CD7B
  \l CD7C
  \l CD7D
  \l CD7E
  \l CD7F
  \l CD80
  \l CD81
  \l CD82
  \l CD83
  \l CD84
  \l CD85
  \l CD86
  \l CD87
  \l CD88
  \l CD89
  \l CD8A
  \l CD8B
  \l CD8C
  \l CD8D
  \l CD8E
  \l CD8F
  \l CD90
  \l CD91
  \l CD92
  \l CD93
  \l CD94
  \l CD95
  \l CD96
  \l CD97
  \l CD98
  \l CD99
  \l CD9A
  \l CD9B
  \l CD9C
  \l CD9D
  \l CD9E
  \l CD9F
  \l CDA0
  \l CDA1
  \l CDA2
  \l CDA3
  \l CDA4
  \l CDA5
  \l CDA6
  \l CDA7
  \l CDA8
  \l CDA9
  \l CDAA
  \l CDAB
  \l CDAC
  \l CDAD
  \l CDAE
  \l CDAF
  \l CDB0
  \l CDB1
  \l CDB2
  \l CDB3
  \l CDB4
  \l CDB5
  \l CDB6
  \l CDB7
  \l CDB8
  \l CDB9
  \l CDBA
  \l CDBB
  \l CDBC
  \l CDBD
  \l CDBE
  \l CDBF
  \l CDC0
  \l CDC1
  \l CDC2
  \l CDC3
  \l CDC4
  \l CDC5
  \l CDC6
  \l CDC7
  \l CDC8
  \l CDC9
  \l CDCA
  \l CDCB
  \l CDCC
  \l CDCD
  \l CDCE
  \l CDCF
  \l CDD0
  \l CDD1
  \l CDD2
  \l CDD3
  \l CDD4
  \l CDD5
  \l CDD6
  \l CDD7
  \l CDD8
  \l CDD9
  \l CDDA
  \l CDDB
  \l CDDC
  \l CDDD
  \l CDDE
  \l CDDF
  \l CDE0
  \l CDE1
  \l CDE2
  \l CDE3
  \l CDE4
  \l CDE5
  \l CDE6
  \l CDE7
  \l CDE8
  \l CDE9
  \l CDEA
  \l CDEB
  \l CDEC
  \l CDED
  \l CDEE
  \l CDEF
  \l CDF0
  \l CDF1
  \l CDF2
  \l CDF3
  \l CDF4
  \l CDF5
  \l CDF6
  \l CDF7
  \l CDF8
  \l CDF9
  \l CDFA
  \l CDFB
  \l CDFC
  \l CDFD
  \l CDFE
  \l CDFF
  \l CE00
  \l CE01
  \l CE02
  \l CE03
  \l CE04
  \l CE05
  \l CE06
  \l CE07
  \l CE08
  \l CE09
  \l CE0A
  \l CE0B
  \l CE0C
  \l CE0D
  \l CE0E
  \l CE0F
  \l CE10
  \l CE11
  \l CE12
  \l CE13
  \l CE14
  \l CE15
  \l CE16
  \l CE17
  \l CE18
  \l CE19
  \l CE1A
  \l CE1B
  \l CE1C
  \l CE1D
  \l CE1E
  \l CE1F
  \l CE20
  \l CE21
  \l CE22
  \l CE23
  \l CE24
  \l CE25
  \l CE26
  \l CE27
  \l CE28
  \l CE29
  \l CE2A
  \l CE2B
  \l CE2C
  \l CE2D
  \l CE2E
  \l CE2F
  \l CE30
  \l CE31
  \l CE32
  \l CE33
  \l CE34
  \l CE35
  \l CE36
  \l CE37
  \l CE38
  \l CE39
  \l CE3A
  \l CE3B
  \l CE3C
  \l CE3D
  \l CE3E
  \l CE3F
  \l CE40
  \l CE41
  \l CE42
  \l CE43
  \l CE44
  \l CE45
  \l CE46
  \l CE47
  \l CE48
  \l CE49
  \l CE4A
  \l CE4B
  \l CE4C
  \l CE4D
  \l CE4E
  \l CE4F
  \l CE50
  \l CE51
  \l CE52
  \l CE53
  \l CE54
  \l CE55
  \l CE56
  \l CE57
  \l CE58
  \l CE59
  \l CE5A
  \l CE5B
  \l CE5C
  \l CE5D
  \l CE5E
  \l CE5F
  \l CE60
  \l CE61
  \l CE62
  \l CE63
  \l CE64
  \l CE65
  \l CE66
  \l CE67
  \l CE68
  \l CE69
  \l CE6A
  \l CE6B
  \l CE6C
  \l CE6D
  \l CE6E
  \l CE6F
  \l CE70
  \l CE71
  \l CE72
  \l CE73
  \l CE74
  \l CE75
  \l CE76
  \l CE77
  \l CE78
  \l CE79
  \l CE7A
  \l CE7B
  \l CE7C
  \l CE7D
  \l CE7E
  \l CE7F
  \l CE80
  \l CE81
  \l CE82
  \l CE83
  \l CE84
  \l CE85
  \l CE86
  \l CE87
  \l CE88
  \l CE89
  \l CE8A
  \l CE8B
  \l CE8C
  \l CE8D
  \l CE8E
  \l CE8F
  \l CE90
  \l CE91
  \l CE92
  \l CE93
  \l CE94
  \l CE95
  \l CE96
  \l CE97
  \l CE98
  \l CE99
  \l CE9A
  \l CE9B
  \l CE9C
  \l CE9D
  \l CE9E
  \l CE9F
  \l CEA0
  \l CEA1
  \l CEA2
  \l CEA3
  \l CEA4
  \l CEA5
  \l CEA6
  \l CEA7
  \l CEA8
  \l CEA9
  \l CEAA
  \l CEAB
  \l CEAC
  \l CEAD
  \l CEAE
  \l CEAF
  \l CEB0
  \l CEB1
  \l CEB2
  \l CEB3
  \l CEB4
  \l CEB5
  \l CEB6
  \l CEB7
  \l CEB8
  \l CEB9
  \l CEBA
  \l CEBB
  \l CEBC
  \l CEBD
  \l CEBE
  \l CEBF
  \l CEC0
  \l CEC1
  \l CEC2
  \l CEC3
  \l CEC4
  \l CEC5
  \l CEC6
  \l CEC7
  \l CEC8
  \l CEC9
  \l CECA
  \l CECB
  \l CECC
  \l CECD
  \l CECE
  \l CECF
  \l CED0
  \l CED1
  \l CED2
  \l CED3
  \l CED4
  \l CED5
  \l CED6
  \l CED7
  \l CED8
  \l CED9
  \l CEDA
  \l CEDB
  \l CEDC
  \l CEDD
  \l CEDE
  \l CEDF
  \l CEE0
  \l CEE1
  \l CEE2
  \l CEE3
  \l CEE4
  \l CEE5
  \l CEE6
  \l CEE7
  \l CEE8
  \l CEE9
  \l CEEA
  \l CEEB
  \l CEEC
  \l CEED
  \l CEEE
  \l CEEF
  \l CEF0
  \l CEF1
  \l CEF2
  \l CEF3
  \l CEF4
  \l CEF5
  \l CEF6
  \l CEF7
  \l CEF8
  \l CEF9
  \l CEFA
  \l CEFB
  \l CEFC
  \l CEFD
  \l CEFE
  \l CEFF
  \l CF00
  \l CF01
  \l CF02
  \l CF03
  \l CF04
  \l CF05
  \l CF06
  \l CF07
  \l CF08
  \l CF09
  \l CF0A
  \l CF0B
  \l CF0C
  \l CF0D
  \l CF0E
  \l CF0F
  \l CF10
  \l CF11
  \l CF12
  \l CF13
  \l CF14
  \l CF15
  \l CF16
  \l CF17
  \l CF18
  \l CF19
  \l CF1A
  \l CF1B
  \l CF1C
  \l CF1D
  \l CF1E
  \l CF1F
  \l CF20
  \l CF21
  \l CF22
  \l CF23
  \l CF24
  \l CF25
  \l CF26
  \l CF27
  \l CF28
  \l CF29
  \l CF2A
  \l CF2B
  \l CF2C
  \l CF2D
  \l CF2E
  \l CF2F
  \l CF30
  \l CF31
  \l CF32
  \l CF33
  \l CF34
  \l CF35
  \l CF36
  \l CF37
  \l CF38
  \l CF39
  \l CF3A
  \l CF3B
  \l CF3C
  \l CF3D
  \l CF3E
  \l CF3F
  \l CF40
  \l CF41
  \l CF42
  \l CF43
  \l CF44
  \l CF45
  \l CF46
  \l CF47
  \l CF48
  \l CF49
  \l CF4A
  \l CF4B
  \l CF4C
  \l CF4D
  \l CF4E
  \l CF4F
  \l CF50
  \l CF51
  \l CF52
  \l CF53
  \l CF54
  \l CF55
  \l CF56
  \l CF57
  \l CF58
  \l CF59
  \l CF5A
  \l CF5B
  \l CF5C
  \l CF5D
  \l CF5E
  \l CF5F
  \l CF60
  \l CF61
  \l CF62
  \l CF63
  \l CF64
  \l CF65
  \l CF66
  \l CF67
  \l CF68
  \l CF69
  \l CF6A
  \l CF6B
  \l CF6C
  \l CF6D
  \l CF6E
  \l CF6F
  \l CF70
  \l CF71
  \l CF72
  \l CF73
  \l CF74
  \l CF75
  \l CF76
  \l CF77
  \l CF78
  \l CF79
  \l CF7A
  \l CF7B
  \l CF7C
  \l CF7D
  \l CF7E
  \l CF7F
  \l CF80
  \l CF81
  \l CF82
  \l CF83
  \l CF84
  \l CF85
  \l CF86
  \l CF87
  \l CF88
  \l CF89
  \l CF8A
  \l CF8B
  \l CF8C
  \l CF8D
  \l CF8E
  \l CF8F
  \l CF90
  \l CF91
  \l CF92
  \l CF93
  \l CF94
  \l CF95
  \l CF96
  \l CF97
  \l CF98
  \l CF99
  \l CF9A
  \l CF9B
  \l CF9C
  \l CF9D
  \l CF9E
  \l CF9F
  \l CFA0
  \l CFA1
  \l CFA2
  \l CFA3
  \l CFA4
  \l CFA5
  \l CFA6
  \l CFA7
  \l CFA8
  \l CFA9
  \l CFAA
  \l CFAB
  \l CFAC
  \l CFAD
  \l CFAE
  \l CFAF
  \l CFB0
  \l CFB1
  \l CFB2
  \l CFB3
  \l CFB4
  \l CFB5
  \l CFB6
  \l CFB7
  \l CFB8
  \l CFB9
  \l CFBA
  \l CFBB
  \l CFBC
  \l CFBD
  \l CFBE
  \l CFBF
  \l CFC0
  \l CFC1
  \l CFC2
  \l CFC3
  \l CFC4
  \l CFC5
  \l CFC6
  \l CFC7
  \l CFC8
  \l CFC9
  \l CFCA
  \l CFCB
  \l CFCC
  \l CFCD
  \l CFCE
  \l CFCF
  \l CFD0
  \l CFD1
  \l CFD2
  \l CFD3
  \l CFD4
  \l CFD5
  \l CFD6
  \l CFD7
  \l CFD8
  \l CFD9
  \l CFDA
  \l CFDB
  \l CFDC
  \l CFDD
  \l CFDE
  \l CFDF
  \l CFE0
  \l CFE1
  \l CFE2
  \l CFE3
  \l CFE4
  \l CFE5
  \l CFE6
  \l CFE7
  \l CFE8
  \l CFE9
  \l CFEA
  \l CFEB
  \l CFEC
  \l CFED
  \l CFEE
  \l CFEF
  \l CFF0
  \l CFF1
  \l CFF2
  \l CFF3
  \l CFF4
  \l CFF5
  \l CFF6
  \l CFF7
  \l CFF8
  \l CFF9
  \l CFFA
  \l CFFB
  \l CFFC
  \l CFFD
  \l CFFE
  \l CFFF
  \l D000
  \l D001
  \l D002
  \l D003
  \l D004
  \l D005
  \l D006
  \l D007
  \l D008
  \l D009
  \l D00A
  \l D00B
  \l D00C
  \l D00D
  \l D00E
  \l D00F
  \l D010
  \l D011
  \l D012
  \l D013
  \l D014
  \l D015
  \l D016
  \l D017
  \l D018
  \l D019
  \l D01A
  \l D01B
  \l D01C
  \l D01D
  \l D01E
  \l D01F
  \l D020
  \l D021
  \l D022
  \l D023
  \l D024
  \l D025
  \l D026
  \l D027
  \l D028
  \l D029
  \l D02A
  \l D02B
  \l D02C
  \l D02D
  \l D02E
  \l D02F
  \l D030
  \l D031
  \l D032
  \l D033
  \l D034
  \l D035
  \l D036
  \l D037
  \l D038
  \l D039
  \l D03A
  \l D03B
  \l D03C
  \l D03D
  \l D03E
  \l D03F
  \l D040
  \l D041
  \l D042
  \l D043
  \l D044
  \l D045
  \l D046
  \l D047
  \l D048
  \l D049
  \l D04A
  \l D04B
  \l D04C
  \l D04D
  \l D04E
  \l D04F
  \l D050
  \l D051
  \l D052
  \l D053
  \l D054
  \l D055
  \l D056
  \l D057
  \l D058
  \l D059
  \l D05A
  \l D05B
  \l D05C
  \l D05D
  \l D05E
  \l D05F
  \l D060
  \l D061
  \l D062
  \l D063
  \l D064
  \l D065
  \l D066
  \l D067
  \l D068
  \l D069
  \l D06A
  \l D06B
  \l D06C
  \l D06D
  \l D06E
  \l D06F
  \l D070
  \l D071
  \l D072
  \l D073
  \l D074
  \l D075
  \l D076
  \l D077
  \l D078
  \l D079
  \l D07A
  \l D07B
  \l D07C
  \l D07D
  \l D07E
  \l D07F
  \l D080
  \l D081
  \l D082
  \l D083
  \l D084
  \l D085
  \l D086
  \l D087
  \l D088
  \l D089
  \l D08A
  \l D08B
  \l D08C
  \l D08D
  \l D08E
  \l D08F
  \l D090
  \l D091
  \l D092
  \l D093
  \l D094
  \l D095
  \l D096
  \l D097
  \l D098
  \l D099
  \l D09A
  \l D09B
  \l D09C
  \l D09D
  \l D09E
  \l D09F
  \l D0A0
  \l D0A1
  \l D0A2
  \l D0A3
  \l D0A4
  \l D0A5
  \l D0A6
  \l D0A7
  \l D0A8
  \l D0A9
  \l D0AA
  \l D0AB
  \l D0AC
  \l D0AD
  \l D0AE
  \l D0AF
  \l D0B0
  \l D0B1
  \l D0B2
  \l D0B3
  \l D0B4
  \l D0B5
  \l D0B6
  \l D0B7
  \l D0B8
  \l D0B9
  \l D0BA
  \l D0BB
  \l D0BC
  \l D0BD
  \l D0BE
  \l D0BF
  \l D0C0
  \l D0C1
  \l D0C2
  \l D0C3
  \l D0C4
  \l D0C5
  \l D0C6
  \l D0C7
  \l D0C8
  \l D0C9
  \l D0CA
  \l D0CB
  \l D0CC
  \l D0CD
  \l D0CE
  \l D0CF
  \l D0D0
  \l D0D1
  \l D0D2
  \l D0D3
  \l D0D4
  \l D0D5
  \l D0D6
  \l D0D7
  \l D0D8
  \l D0D9
  \l D0DA
  \l D0DB
  \l D0DC
  \l D0DD
  \l D0DE
  \l D0DF
  \l D0E0
  \l D0E1
  \l D0E2
  \l D0E3
  \l D0E4
  \l D0E5
  \l D0E6
  \l D0E7
  \l D0E8
  \l D0E9
  \l D0EA
  \l D0EB
  \l D0EC
  \l D0ED
  \l D0EE
  \l D0EF
  \l D0F0
  \l D0F1
  \l D0F2
  \l D0F3
  \l D0F4
  \l D0F5
  \l D0F6
  \l D0F7
  \l D0F8
  \l D0F9
  \l D0FA
  \l D0FB
  \l D0FC
  \l D0FD
  \l D0FE
  \l D0FF
  \l D100
  \l D101
  \l D102
  \l D103
  \l D104
  \l D105
  \l D106
  \l D107
  \l D108
  \l D109
  \l D10A
  \l D10B
  \l D10C
  \l D10D
  \l D10E
  \l D10F
  \l D110
  \l D111
  \l D112
  \l D113
  \l D114
  \l D115
  \l D116
  \l D117
  \l D118
  \l D119
  \l D11A
  \l D11B
  \l D11C
  \l D11D
  \l D11E
  \l D11F
  \l D120
  \l D121
  \l D122
  \l D123
  \l D124
  \l D125
  \l D126
  \l D127
  \l D128
  \l D129
  \l D12A
  \l D12B
  \l D12C
  \l D12D
  \l D12E
  \l D12F
  \l D130
  \l D131
  \l D132
  \l D133
  \l D134
  \l D135
  \l D136
  \l D137
  \l D138
  \l D139
  \l D13A
  \l D13B
  \l D13C
  \l D13D
  \l D13E
  \l D13F
  \l D140
  \l D141
  \l D142
  \l D143
  \l D144
  \l D145
  \l D146
  \l D147
  \l D148
  \l D149
  \l D14A
  \l D14B
  \l D14C
  \l D14D
  \l D14E
  \l D14F
  \l D150
  \l D151
  \l D152
  \l D153
  \l D154
  \l D155
  \l D156
  \l D157
  \l D158
  \l D159
  \l D15A
  \l D15B
  \l D15C
  \l D15D
  \l D15E
  \l D15F
  \l D160
  \l D161
  \l D162
  \l D163
  \l D164
  \l D165
  \l D166
  \l D167
  \l D168
  \l D169
  \l D16A
  \l D16B
  \l D16C
  \l D16D
  \l D16E
  \l D16F
  \l D170
  \l D171
  \l D172
  \l D173
  \l D174
  \l D175
  \l D176
  \l D177
  \l D178
  \l D179
  \l D17A
  \l D17B
  \l D17C
  \l D17D
  \l D17E
  \l D17F
  \l D180
  \l D181
  \l D182
  \l D183
  \l D184
  \l D185
  \l D186
  \l D187
  \l D188
  \l D189
  \l D18A
  \l D18B
  \l D18C
  \l D18D
  \l D18E
  \l D18F
  \l D190
  \l D191
  \l D192
  \l D193
  \l D194
  \l D195
  \l D196
  \l D197
  \l D198
  \l D199
  \l D19A
  \l D19B
  \l D19C
  \l D19D
  \l D19E
  \l D19F
  \l D1A0
  \l D1A1
  \l D1A2
  \l D1A3
  \l D1A4
  \l D1A5
  \l D1A6
  \l D1A7
  \l D1A8
  \l D1A9
  \l D1AA
  \l D1AB
  \l D1AC
  \l D1AD
  \l D1AE
  \l D1AF
  \l D1B0
  \l D1B1
  \l D1B2
  \l D1B3
  \l D1B4
  \l D1B5
  \l D1B6
  \l D1B7
  \l D1B8
  \l D1B9
  \l D1BA
  \l D1BB
  \l D1BC
  \l D1BD
  \l D1BE
  \l D1BF
  \l D1C0
  \l D1C1
  \l D1C2
  \l D1C3
  \l D1C4
  \l D1C5
  \l D1C6
  \l D1C7
  \l D1C8
  \l D1C9
  \l D1CA
  \l D1CB
  \l D1CC
  \l D1CD
  \l D1CE
  \l D1CF
  \l D1D0
  \l D1D1
  \l D1D2
  \l D1D3
  \l D1D4
  \l D1D5
  \l D1D6
  \l D1D7
  \l D1D8
  \l D1D9
  \l D1DA
  \l D1DB
  \l D1DC
  \l D1DD
  \l D1DE
  \l D1DF
  \l D1E0
  \l D1E1
  \l D1E2
  \l D1E3
  \l D1E4
  \l D1E5
  \l D1E6
  \l D1E7
  \l D1E8
  \l D1E9
  \l D1EA
  \l D1EB
  \l D1EC
  \l D1ED
  \l D1EE
  \l D1EF
  \l D1F0
  \l D1F1
  \l D1F2
  \l D1F3
  \l D1F4
  \l D1F5
  \l D1F6
  \l D1F7
  \l D1F8
  \l D1F9
  \l D1FA
  \l D1FB
  \l D1FC
  \l D1FD
  \l D1FE
  \l D1FF
  \l D200
  \l D201
  \l D202
  \l D203
  \l D204
  \l D205
  \l D206
  \l D207
  \l D208
  \l D209
  \l D20A
  \l D20B
  \l D20C
  \l D20D
  \l D20E
  \l D20F
  \l D210
  \l D211
  \l D212
  \l D213
  \l D214
  \l D215
  \l D216
  \l D217
  \l D218
  \l D219
  \l D21A
  \l D21B
  \l D21C
  \l D21D
  \l D21E
  \l D21F
  \l D220
  \l D221
  \l D222
  \l D223
  \l D224
  \l D225
  \l D226
  \l D227
  \l D228
  \l D229
  \l D22A
  \l D22B
  \l D22C
  \l D22D
  \l D22E
  \l D22F
  \l D230
  \l D231
  \l D232
  \l D233
  \l D234
  \l D235
  \l D236
  \l D237
  \l D238
  \l D239
  \l D23A
  \l D23B
  \l D23C
  \l D23D
  \l D23E
  \l D23F
  \l D240
  \l D241
  \l D242
  \l D243
  \l D244
  \l D245
  \l D246
  \l D247
  \l D248
  \l D249
  \l D24A
  \l D24B
  \l D24C
  \l D24D
  \l D24E
  \l D24F
  \l D250
  \l D251
  \l D252
  \l D253
  \l D254
  \l D255
  \l D256
  \l D257
  \l D258
  \l D259
  \l D25A
  \l D25B
  \l D25C
  \l D25D
  \l D25E
  \l D25F
  \l D260
  \l D261
  \l D262
  \l D263
  \l D264
  \l D265
  \l D266
  \l D267
  \l D268
  \l D269
  \l D26A
  \l D26B
  \l D26C
  \l D26D
  \l D26E
  \l D26F
  \l D270
  \l D271
  \l D272
  \l D273
  \l D274
  \l D275
  \l D276
  \l D277
  \l D278
  \l D279
  \l D27A
  \l D27B
  \l D27C
  \l D27D
  \l D27E
  \l D27F
  \l D280
  \l D281
  \l D282
  \l D283
  \l D284
  \l D285
  \l D286
  \l D287
  \l D288
  \l D289
  \l D28A
  \l D28B
  \l D28C
  \l D28D
  \l D28E
  \l D28F
  \l D290
  \l D291
  \l D292
  \l D293
  \l D294
  \l D295
  \l D296
  \l D297
  \l D298
  \l D299
  \l D29A
  \l D29B
  \l D29C
  \l D29D
  \l D29E
  \l D29F
  \l D2A0
  \l D2A1
  \l D2A2
  \l D2A3
  \l D2A4
  \l D2A5
  \l D2A6
  \l D2A7
  \l D2A8
  \l D2A9
  \l D2AA
  \l D2AB
  \l D2AC
  \l D2AD
  \l D2AE
  \l D2AF
  \l D2B0
  \l D2B1
  \l D2B2
  \l D2B3
  \l D2B4
  \l D2B5
  \l D2B6
  \l D2B7
  \l D2B8
  \l D2B9
  \l D2BA
  \l D2BB
  \l D2BC
  \l D2BD
  \l D2BE
  \l D2BF
  \l D2C0
  \l D2C1
  \l D2C2
  \l D2C3
  \l D2C4
  \l D2C5
  \l D2C6
  \l D2C7
  \l D2C8
  \l D2C9
  \l D2CA
  \l D2CB
  \l D2CC
  \l D2CD
  \l D2CE
  \l D2CF
  \l D2D0
  \l D2D1
  \l D2D2
  \l D2D3
  \l D2D4
  \l D2D5
  \l D2D6
  \l D2D7
  \l D2D8
  \l D2D9
  \l D2DA
  \l D2DB
  \l D2DC
  \l D2DD
  \l D2DE
  \l D2DF
  \l D2E0
  \l D2E1
  \l D2E2
  \l D2E3
  \l D2E4
  \l D2E5
  \l D2E6
  \l D2E7
  \l D2E8
  \l D2E9
  \l D2EA
  \l D2EB
  \l D2EC
  \l D2ED
  \l D2EE
  \l D2EF
  \l D2F0
  \l D2F1
  \l D2F2
  \l D2F3
  \l D2F4
  \l D2F5
  \l D2F6
  \l D2F7
  \l D2F8
  \l D2F9
  \l D2FA
  \l D2FB
  \l D2FC
  \l D2FD
  \l D2FE
  \l D2FF
  \l D300
  \l D301
  \l D302
  \l D303
  \l D304
  \l D305
  \l D306
  \l D307
  \l D308
  \l D309
  \l D30A
  \l D30B
  \l D30C
  \l D30D
  \l D30E
  \l D30F
  \l D310
  \l D311
  \l D312
  \l D313
  \l D314
  \l D315
  \l D316
  \l D317
  \l D318
  \l D319
  \l D31A
  \l D31B
  \l D31C
  \l D31D
  \l D31E
  \l D31F
  \l D320
  \l D321
  \l D322
  \l D323
  \l D324
  \l D325
  \l D326
  \l D327
  \l D328
  \l D329
  \l D32A
  \l D32B
  \l D32C
  \l D32D
  \l D32E
  \l D32F
  \l D330
  \l D331
  \l D332
  \l D333
  \l D334
  \l D335
  \l D336
  \l D337
  \l D338
  \l D339
  \l D33A
  \l D33B
  \l D33C
  \l D33D
  \l D33E
  \l D33F
  \l D340
  \l D341
  \l D342
  \l D343
  \l D344
  \l D345
  \l D346
  \l D347
  \l D348
  \l D349
  \l D34A
  \l D34B
  \l D34C
  \l D34D
  \l D34E
  \l D34F
  \l D350
  \l D351
  \l D352
  \l D353
  \l D354
  \l D355
  \l D356
  \l D357
  \l D358
  \l D359
  \l D35A
  \l D35B
  \l D35C
  \l D35D
  \l D35E
  \l D35F
  \l D360
  \l D361
  \l D362
  \l D363
  \l D364
  \l D365
  \l D366
  \l D367
  \l D368
  \l D369
  \l D36A
  \l D36B
  \l D36C
  \l D36D
  \l D36E
  \l D36F
  \l D370
  \l D371
  \l D372
  \l D373
  \l D374
  \l D375
  \l D376
  \l D377
  \l D378
  \l D379
  \l D37A
  \l D37B
  \l D37C
  \l D37D
  \l D37E
  \l D37F
  \l D380
  \l D381
  \l D382
  \l D383
  \l D384
  \l D385
  \l D386
  \l D387
  \l D388
  \l D389
  \l D38A
  \l D38B
  \l D38C
  \l D38D
  \l D38E
  \l D38F
  \l D390
  \l D391
  \l D392
  \l D393
  \l D394
  \l D395
  \l D396
  \l D397
  \l D398
  \l D399
  \l D39A
  \l D39B
  \l D39C
  \l D39D
  \l D39E
  \l D39F
  \l D3A0
  \l D3A1
  \l D3A2
  \l D3A3
  \l D3A4
  \l D3A5
  \l D3A6
  \l D3A7
  \l D3A8
  \l D3A9
  \l D3AA
  \l D3AB
  \l D3AC
  \l D3AD
  \l D3AE
  \l D3AF
  \l D3B0
  \l D3B1
  \l D3B2
  \l D3B3
  \l D3B4
  \l D3B5
  \l D3B6
  \l D3B7
  \l D3B8
  \l D3B9
  \l D3BA
  \l D3BB
  \l D3BC
  \l D3BD
  \l D3BE
  \l D3BF
  \l D3C0
  \l D3C1
  \l D3C2
  \l D3C3
  \l D3C4
  \l D3C5
  \l D3C6
  \l D3C7
  \l D3C8
  \l D3C9
  \l D3CA
  \l D3CB
  \l D3CC
  \l D3CD
  \l D3CE
  \l D3CF
  \l D3D0
  \l D3D1
  \l D3D2
  \l D3D3
  \l D3D4
  \l D3D5
  \l D3D6
  \l D3D7
  \l D3D8
  \l D3D9
  \l D3DA
  \l D3DB
  \l D3DC
  \l D3DD
  \l D3DE
  \l D3DF
  \l D3E0
  \l D3E1
  \l D3E2
  \l D3E3
  \l D3E4
  \l D3E5
  \l D3E6
  \l D3E7
  \l D3E8
  \l D3E9
  \l D3EA
  \l D3EB
  \l D3EC
  \l D3ED
  \l D3EE
  \l D3EF
  \l D3F0
  \l D3F1
  \l D3F2
  \l D3F3
  \l D3F4
  \l D3F5
  \l D3F6
  \l D3F7
  \l D3F8
  \l D3F9
  \l D3FA
  \l D3FB
  \l D3FC
  \l D3FD
  \l D3FE
  \l D3FF
  \l D400
  \l D401
  \l D402
  \l D403
  \l D404
  \l D405
  \l D406
  \l D407
  \l D408
  \l D409
  \l D40A
  \l D40B
  \l D40C
  \l D40D
  \l D40E
  \l D40F
  \l D410
  \l D411
  \l D412
  \l D413
  \l D414
  \l D415
  \l D416
  \l D417
  \l D418
  \l D419
  \l D41A
  \l D41B
  \l D41C
  \l D41D
  \l D41E
  \l D41F
  \l D420
  \l D421
  \l D422
  \l D423
  \l D424
  \l D425
  \l D426
  \l D427
  \l D428
  \l D429
  \l D42A
  \l D42B
  \l D42C
  \l D42D
  \l D42E
  \l D42F
  \l D430
  \l D431
  \l D432
  \l D433
  \l D434
  \l D435
  \l D436
  \l D437
  \l D438
  \l D439
  \l D43A
  \l D43B
  \l D43C
  \l D43D
  \l D43E
  \l D43F
  \l D440
  \l D441
  \l D442
  \l D443
  \l D444
  \l D445
  \l D446
  \l D447
  \l D448
  \l D449
  \l D44A
  \l D44B
  \l D44C
  \l D44D
  \l D44E
  \l D44F
  \l D450
  \l D451
  \l D452
  \l D453
  \l D454
  \l D455
  \l D456
  \l D457
  \l D458
  \l D459
  \l D45A
  \l D45B
  \l D45C
  \l D45D
  \l D45E
  \l D45F
  \l D460
  \l D461
  \l D462
  \l D463
  \l D464
  \l D465
  \l D466
  \l D467
  \l D468
  \l D469
  \l D46A
  \l D46B
  \l D46C
  \l D46D
  \l D46E
  \l D46F
  \l D470
  \l D471
  \l D472
  \l D473
  \l D474
  \l D475
  \l D476
  \l D477
  \l D478
  \l D479
  \l D47A
  \l D47B
  \l D47C
  \l D47D
  \l D47E
  \l D47F
  \l D480
  \l D481
  \l D482
  \l D483
  \l D484
  \l D485
  \l D486
  \l D487
  \l D488
  \l D489
  \l D48A
  \l D48B
  \l D48C
  \l D48D
  \l D48E
  \l D48F
  \l D490
  \l D491
  \l D492
  \l D493
  \l D494
  \l D495
  \l D496
  \l D497
  \l D498
  \l D499
  \l D49A
  \l D49B
  \l D49C
  \l D49D
  \l D49E
  \l D49F
  \l D4A0
  \l D4A1
  \l D4A2
  \l D4A3
  \l D4A4
  \l D4A5
  \l D4A6
  \l D4A7
  \l D4A8
  \l D4A9
  \l D4AA
  \l D4AB
  \l D4AC
  \l D4AD
  \l D4AE
  \l D4AF
  \l D4B0
  \l D4B1
  \l D4B2
  \l D4B3
  \l D4B4
  \l D4B5
  \l D4B6
  \l D4B7
  \l D4B8
  \l D4B9
  \l D4BA
  \l D4BB
  \l D4BC
  \l D4BD
  \l D4BE
  \l D4BF
  \l D4C0
  \l D4C1
  \l D4C2
  \l D4C3
  \l D4C4
  \l D4C5
  \l D4C6
  \l D4C7
  \l D4C8
  \l D4C9
  \l D4CA
  \l D4CB
  \l D4CC
  \l D4CD
  \l D4CE
  \l D4CF
  \l D4D0
  \l D4D1
  \l D4D2
  \l D4D3
  \l D4D4
  \l D4D5
  \l D4D6
  \l D4D7
  \l D4D8
  \l D4D9
  \l D4DA
  \l D4DB
  \l D4DC
  \l D4DD
  \l D4DE
  \l D4DF
  \l D4E0
  \l D4E1
  \l D4E2
  \l D4E3
  \l D4E4
  \l D4E5
  \l D4E6
  \l D4E7
  \l D4E8
  \l D4E9
  \l D4EA
  \l D4EB
  \l D4EC
  \l D4ED
  \l D4EE
  \l D4EF
  \l D4F0
  \l D4F1
  \l D4F2
  \l D4F3
  \l D4F4
  \l D4F5
  \l D4F6
  \l D4F7
  \l D4F8
  \l D4F9
  \l D4FA
  \l D4FB
  \l D4FC
  \l D4FD
  \l D4FE
  \l D4FF
  \l D500
  \l D501
  \l D502
  \l D503
  \l D504
  \l D505
  \l D506
  \l D507
  \l D508
  \l D509
  \l D50A
  \l D50B
  \l D50C
  \l D50D
  \l D50E
  \l D50F
  \l D510
  \l D511
  \l D512
  \l D513
  \l D514
  \l D515
  \l D516
  \l D517
  \l D518
  \l D519
  \l D51A
  \l D51B
  \l D51C
  \l D51D
  \l D51E
  \l D51F
  \l D520
  \l D521
  \l D522
  \l D523
  \l D524
  \l D525
  \l D526
  \l D527
  \l D528
  \l D529
  \l D52A
  \l D52B
  \l D52C
  \l D52D
  \l D52E
  \l D52F
  \l D530
  \l D531
  \l D532
  \l D533
  \l D534
  \l D535
  \l D536
  \l D537
  \l D538
  \l D539
  \l D53A
  \l D53B
  \l D53C
  \l D53D
  \l D53E
  \l D53F
  \l D540
  \l D541
  \l D542
  \l D543
  \l D544
  \l D545
  \l D546
  \l D547
  \l D548
  \l D549
  \l D54A
  \l D54B
  \l D54C
  \l D54D
  \l D54E
  \l D54F
  \l D550
  \l D551
  \l D552
  \l D553
  \l D554
  \l D555
  \l D556
  \l D557
  \l D558
  \l D559
  \l D55A
  \l D55B
  \l D55C
  \l D55D
  \l D55E
  \l D55F
  \l D560
  \l D561
  \l D562
  \l D563
  \l D564
  \l D565
  \l D566
  \l D567
  \l D568
  \l D569
  \l D56A
  \l D56B
  \l D56C
  \l D56D
  \l D56E
  \l D56F
  \l D570
  \l D571
  \l D572
  \l D573
  \l D574
  \l D575
  \l D576
  \l D577
  \l D578
  \l D579
  \l D57A
  \l D57B
  \l D57C
  \l D57D
  \l D57E
  \l D57F
  \l D580
  \l D581
  \l D582
  \l D583
  \l D584
  \l D585
  \l D586
  \l D587
  \l D588
  \l D589
  \l D58A
  \l D58B
  \l D58C
  \l D58D
  \l D58E
  \l D58F
  \l D590
  \l D591
  \l D592
  \l D593
  \l D594
  \l D595
  \l D596
  \l D597
  \l D598
  \l D599
  \l D59A
  \l D59B
  \l D59C
  \l D59D
  \l D59E
  \l D59F
  \l D5A0
  \l D5A1
  \l D5A2
  \l D5A3
  \l D5A4
  \l D5A5
  \l D5A6
  \l D5A7
  \l D5A8
  \l D5A9
  \l D5AA
  \l D5AB
  \l D5AC
  \l D5AD
  \l D5AE
  \l D5AF
  \l D5B0
  \l D5B1
  \l D5B2
  \l D5B3
  \l D5B4
  \l D5B5
  \l D5B6
  \l D5B7
  \l D5B8
  \l D5B9
  \l D5BA
  \l D5BB
  \l D5BC
  \l D5BD
  \l D5BE
  \l D5BF
  \l D5C0
  \l D5C1
  \l D5C2
  \l D5C3
  \l D5C4
  \l D5C5
  \l D5C6
  \l D5C7
  \l D5C8
  \l D5C9
  \l D5CA
  \l D5CB
  \l D5CC
  \l D5CD
  \l D5CE
  \l D5CF
  \l D5D0
  \l D5D1
  \l D5D2
  \l D5D3
  \l D5D4
  \l D5D5
  \l D5D6
  \l D5D7
  \l D5D8
  \l D5D9
  \l D5DA
  \l D5DB
  \l D5DC
  \l D5DD
  \l D5DE
  \l D5DF
  \l D5E0
  \l D5E1
  \l D5E2
  \l D5E3
  \l D5E4
  \l D5E5
  \l D5E6
  \l D5E7
  \l D5E8
  \l D5E9
  \l D5EA
  \l D5EB
  \l D5EC
  \l D5ED
  \l D5EE
  \l D5EF
  \l D5F0
  \l D5F1
  \l D5F2
  \l D5F3
  \l D5F4
  \l D5F5
  \l D5F6
  \l D5F7
  \l D5F8
  \l D5F9
  \l D5FA
  \l D5FB
  \l D5FC
  \l D5FD
  \l D5FE
  \l D5FF
  \l D600
  \l D601
  \l D602
  \l D603
  \l D604
  \l D605
  \l D606
  \l D607
  \l D608
  \l D609
  \l D60A
  \l D60B
  \l D60C
  \l D60D
  \l D60E
  \l D60F
  \l D610
  \l D611
  \l D612
  \l D613
  \l D614
  \l D615
  \l D616
  \l D617
  \l D618
  \l D619
  \l D61A
  \l D61B
  \l D61C
  \l D61D
  \l D61E
  \l D61F
  \l D620
  \l D621
  \l D622
  \l D623
  \l D624
  \l D625
  \l D626
  \l D627
  \l D628
  \l D629
  \l D62A
  \l D62B
  \l D62C
  \l D62D
  \l D62E
  \l D62F
  \l D630
  \l D631
  \l D632
  \l D633
  \l D634
  \l D635
  \l D636
  \l D637
  \l D638
  \l D639
  \l D63A
  \l D63B
  \l D63C
  \l D63D
  \l D63E
  \l D63F
  \l D640
  \l D641
  \l D642
  \l D643
  \l D644
  \l D645
  \l D646
  \l D647
  \l D648
  \l D649
  \l D64A
  \l D64B
  \l D64C
  \l D64D
  \l D64E
  \l D64F
  \l D650
  \l D651
  \l D652
  \l D653
  \l D654
  \l D655
  \l D656
  \l D657
  \l D658
  \l D659
  \l D65A
  \l D65B
  \l D65C
  \l D65D
  \l D65E
  \l D65F
  \l D660
  \l D661
  \l D662
  \l D663
  \l D664
  \l D665
  \l D666
  \l D667
  \l D668
  \l D669
  \l D66A
  \l D66B
  \l D66C
  \l D66D
  \l D66E
  \l D66F
  \l D670
  \l D671
  \l D672
  \l D673
  \l D674
  \l D675
  \l D676
  \l D677
  \l D678
  \l D679
  \l D67A
  \l D67B
  \l D67C
  \l D67D
  \l D67E
  \l D67F
  \l D680
  \l D681
  \l D682
  \l D683
  \l D684
  \l D685
  \l D686
  \l D687
  \l D688
  \l D689
  \l D68A
  \l D68B
  \l D68C
  \l D68D
  \l D68E
  \l D68F
  \l D690
  \l D691
  \l D692
  \l D693
  \l D694
  \l D695
  \l D696
  \l D697
  \l D698
  \l D699
  \l D69A
  \l D69B
  \l D69C
  \l D69D
  \l D69E
  \l D69F
  \l D6A0
  \l D6A1
  \l D6A2
  \l D6A3
  \l D6A4
  \l D6A5
  \l D6A6
  \l D6A7
  \l D6A8
  \l D6A9
  \l D6AA
  \l D6AB
  \l D6AC
  \l D6AD
  \l D6AE
  \l D6AF
  \l D6B0
  \l D6B1
  \l D6B2
  \l D6B3
  \l D6B4
  \l D6B5
  \l D6B6
  \l D6B7
  \l D6B8
  \l D6B9
  \l D6BA
  \l D6BB
  \l D6BC
  \l D6BD
  \l D6BE
  \l D6BF
  \l D6C0
  \l D6C1
  \l D6C2
  \l D6C3
  \l D6C4
  \l D6C5
  \l D6C6
  \l D6C7
  \l D6C8
  \l D6C9
  \l D6CA
  \l D6CB
  \l D6CC
  \l D6CD
  \l D6CE
  \l D6CF
  \l D6D0
  \l D6D1
  \l D6D2
  \l D6D3
  \l D6D4
  \l D6D5
  \l D6D6
  \l D6D7
  \l D6D8
  \l D6D9
  \l D6DA
  \l D6DB
  \l D6DC
  \l D6DD
  \l D6DE
  \l D6DF
  \l D6E0
  \l D6E1
  \l D6E2
  \l D6E3
  \l D6E4
  \l D6E5
  \l D6E6
  \l D6E7
  \l D6E8
  \l D6E9
  \l D6EA
  \l D6EB
  \l D6EC
  \l D6ED
  \l D6EE
  \l D6EF
  \l D6F0
  \l D6F1
  \l D6F2
  \l D6F3
  \l D6F4
  \l D6F5
  \l D6F6
  \l D6F7
  \l D6F8
  \l D6F9
  \l D6FA
  \l D6FB
  \l D6FC
  \l D6FD
  \l D6FE
  \l D6FF
  \l D700
  \l D701
  \l D702
  \l D703
  \l D704
  \l D705
  \l D706
  \l D707
  \l D708
  \l D709
  \l D70A
  \l D70B
  \l D70C
  \l D70D
  \l D70E
  \l D70F
  \l D710
  \l D711
  \l D712
  \l D713
  \l D714
  \l D715
  \l D716
  \l D717
  \l D718
  \l D719
  \l D71A
  \l D71B
  \l D71C
  \l D71D
  \l D71E
  \l D71F
  \l D720
  \l D721
  \l D722
  \l D723
  \l D724
  \l D725
  \l D726
  \l D727
  \l D728
  \l D729
  \l D72A
  \l D72B
  \l D72C
  \l D72D
  \l D72E
  \l D72F
  \l D730
  \l D731
  \l D732
  \l D733
  \l D734
  \l D735
  \l D736
  \l D737
  \l D738
  \l D739
  \l D73A
  \l D73B
  \l D73C
  \l D73D
  \l D73E
  \l D73F
  \l D740
  \l D741
  \l D742
  \l D743
  \l D744
  \l D745
  \l D746
  \l D747
  \l D748
  \l D749
  \l D74A
  \l D74B
  \l D74C
  \l D74D
  \l D74E
  \l D74F
  \l D750
  \l D751
  \l D752
  \l D753
  \l D754
  \l D755
  \l D756
  \l D757
  \l D758
  \l D759
  \l D75A
  \l D75B
  \l D75C
  \l D75D
  \l D75E
  \l D75F
  \l D760
  \l D761
  \l D762
  \l D763
  \l D764
  \l D765
  \l D766
  \l D767
  \l D768
  \l D769
  \l D76A
  \l D76B
  \l D76C
  \l D76D
  \l D76E
  \l D76F
  \l D770
  \l D771
  \l D772
  \l D773
  \l D774
  \l D775
  \l D776
  \l D777
  \l D778
  \l D779
  \l D77A
  \l D77B
  \l D77C
  \l D77D
  \l D77E
  \l D77F
  \l D780
  \l D781
  \l D782
  \l D783
  \l D784
  \l D785
  \l D786
  \l D787
  \l D788
  \l D789
  \l D78A
  \l D78B
  \l D78C
  \l D78D
  \l D78E
  \l D78F
  \l D790
  \l D791
  \l D792
  \l D793
  \l D794
  \l D795
  \l D796
  \l D797
  \l D798
  \l D799
  \l D79A
  \l D79B
  \l D79C
  \l D79D
  \l D79E
  \l D79F
  \l D7A0
  \l D7A1
  \l D7A2
  \l D7A3
  \l D7B0
  \l D7B1
  \l D7B2
  \l D7B3
  \l D7B4
  \l D7B5
  \l D7B6
  \l D7B7
  \l D7B8
  \l D7B9
  \l D7BA
  \l D7BB
  \l D7BC
  \l D7BD
  \l D7BE
  \l D7BF
  \l D7C0
  \l D7C1
  \l D7C2
  \l D7C3
  \l D7C4
  \l D7C5
  \l D7C6
  \l D7CB
  \l D7CC
  \l D7CD
  \l D7CE
  \l D7CF
  \l D7D0
  \l D7D1
  \l D7D2
  \l D7D3
  \l D7D4
  \l D7D5
  \l D7D6
  \l D7D7
  \l D7D8
  \l D7D9
  \l D7DA
  \l D7DB
  \l D7DC
  \l D7DD
  \l D7DE
  \l D7DF
  \l D7E0
  \l D7E1
  \l D7E2
  \l D7E3
  \l D7E4
  \l D7E5
  \l D7E6
  \l D7E7
  \l D7E8
  \l D7E9
  \l D7EA
  \l D7EB
  \l D7EC
  \l D7ED
  \l D7EE
  \l D7EF
  \l D7F0
  \l D7F1
  \l D7F2
  \l D7F3
  \l D7F4
  \l D7F5
  \l D7F6
  \l D7F7
  \l D7F8
  \l D7F9
  \l D7FA
  \l D7FB
  \l F900
  \l F901
  \l F902
  \l F903
  \l F904
  \l F905
  \l F906
  \l F907
  \l F908
  \l F909
  \l F90A
  \l F90B
  \l F90C
  \l F90D
  \l F90E
  \l F90F
  \l F910
  \l F911
  \l F912
  \l F913
  \l F914
  \l F915
  \l F916
  \l F917
  \l F918
  \l F919
  \l F91A
  \l F91B
  \l F91C
  \l F91D
  \l F91E
  \l F91F
  \l F920
  \l F921
  \l F922
  \l F923
  \l F924
  \l F925
  \l F926
  \l F927
  \l F928
  \l F929
  \l F92A
  \l F92B
  \l F92C
  \l F92D
  \l F92E
  \l F92F
  \l F930
  \l F931
  \l F932
  \l F933
  \l F934
  \l F935
  \l F936
  \l F937
  \l F938
  \l F939
  \l F93A
  \l F93B
  \l F93C
  \l F93D
  \l F93E
  \l F93F
  \l F940
  \l F941
  \l F942
  \l F943
  \l F944
  \l F945
  \l F946
  \l F947
  \l F948
  \l F949
  \l F94A
  \l F94B
  \l F94C
  \l F94D
  \l F94E
  \l F94F
  \l F950
  \l F951
  \l F952
  \l F953
  \l F954
  \l F955
  \l F956
  \l F957
  \l F958
  \l F959
  \l F95A
  \l F95B
  \l F95C
  \l F95D
  \l F95E
  \l F95F
  \l F960
  \l F961
  \l F962
  \l F963
  \l F964
  \l F965
  \l F966
  \l F967
  \l F968
  \l F969
  \l F96A
  \l F96B
  \l F96C
  \l F96D
  \l F96E
  \l F96F
  \l F970
  \l F971
  \l F972
  \l F973
  \l F974
  \l F975
  \l F976
  \l F977
  \l F978
  \l F979
  \l F97A
  \l F97B
  \l F97C
  \l F97D
  \l F97E
  \l F97F
  \l F980
  \l F981
  \l F982
  \l F983
  \l F984
  \l F985
  \l F986
  \l F987
  \l F988
  \l F989
  \l F98A
  \l F98B
  \l F98C
  \l F98D
  \l F98E
  \l F98F
  \l F990
  \l F991
  \l F992
  \l F993
  \l F994
  \l F995
  \l F996
  \l F997
  \l F998
  \l F999
  \l F99A
  \l F99B
  \l F99C
  \l F99D
  \l F99E
  \l F99F
  \l F9A0
  \l F9A1
  \l F9A2
  \l F9A3
  \l F9A4
  \l F9A5
  \l F9A6
  \l F9A7
  \l F9A8
  \l F9A9
  \l F9AA
  \l F9AB
  \l F9AC
  \l F9AD
  \l F9AE
  \l F9AF
  \l F9B0
  \l F9B1
  \l F9B2
  \l F9B3
  \l F9B4
  \l F9B5
  \l F9B6
  \l F9B7
  \l F9B8
  \l F9B9
  \l F9BA
  \l F9BB
  \l F9BC
  \l F9BD
  \l F9BE
  \l F9BF
  \l F9C0
  \l F9C1
  \l F9C2
  \l F9C3
  \l F9C4
  \l F9C5
  \l F9C6
  \l F9C7
  \l F9C8
  \l F9C9
  \l F9CA
  \l F9CB
  \l F9CC
  \l F9CD
  \l F9CE
  \l F9CF
  \l F9D0
  \l F9D1
  \l F9D2
  \l F9D3
  \l F9D4
  \l F9D5
  \l F9D6
  \l F9D7
  \l F9D8
  \l F9D9
  \l F9DA
  \l F9DB
  \l F9DC
  \l F9DD
  \l F9DE
  \l F9DF
  \l F9E0
  \l F9E1
  \l F9E2
  \l F9E3
  \l F9E4
  \l F9E5
  \l F9E6
  \l F9E7
  \l F9E8
  \l F9E9
  \l F9EA
  \l F9EB
  \l F9EC
  \l F9ED
  \l F9EE
  \l F9EF
  \l F9F0
  \l F9F1
  \l F9F2
  \l F9F3
  \l F9F4
  \l F9F5
  \l F9F6
  \l F9F7
  \l F9F8
  \l F9F9
  \l F9FA
  \l F9FB
  \l F9FC
  \l F9FD
  \l F9FE
  \l F9FF
  \l FA00
  \l FA01
  \l FA02
  \l FA03
  \l FA04
  \l FA05
  \l FA06
  \l FA07
  \l FA08
  \l FA09
  \l FA0A
  \l FA0B
  \l FA0C
  \l FA0D
  \l FA0E
  \l FA0F
  \l FA10
  \l FA11
  \l FA12
  \l FA13
  \l FA14
  \l FA15
  \l FA16
  \l FA17
  \l FA18
  \l FA19
  \l FA1A
  \l FA1B
  \l FA1C
  \l FA1D
  \l FA1E
  \l FA1F
  \l FA20
  \l FA21
  \l FA22
  \l FA23
  \l FA24
  \l FA25
  \l FA26
  \l FA27
  \l FA28
  \l FA29
  \l FA2A
  \l FA2B
  \l FA2C
  \l FA2D
  \l FA2E
  \l FA2F
  \l FA30
  \l FA31
  \l FA32
  \l FA33
  \l FA34
  \l FA35
  \l FA36
  \l FA37
  \l FA38
  \l FA39
  \l FA3A
  \l FA3B
  \l FA3C
  \l FA3D
  \l FA3E
  \l FA3F
  \l FA40
  \l FA41
  \l FA42
  \l FA43
  \l FA44
  \l FA45
  \l FA46
  \l FA47
  \l FA48
  \l FA49
  \l FA4A
  \l FA4B
  \l FA4C
  \l FA4D
  \l FA4E
  \l FA4F
  \l FA50
  \l FA51
  \l FA52
  \l FA53
  \l FA54
  \l FA55
  \l FA56
  \l FA57
  \l FA58
  \l FA59
  \l FA5A
  \l FA5B
  \l FA5C
  \l FA5D
  \l FA5E
  \l FA5F
  \l FA60
  \l FA61
  \l FA62
  \l FA63
  \l FA64
  \l FA65
  \l FA66
  \l FA67
  \l FA68
  \l FA69
  \l FA6A
  \l FA6B
  \l FA6C
  \l FA6D
  \l FA70
  \l FA71
  \l FA72
  \l FA73
  \l FA74
  \l FA75
  \l FA76
  \l FA77
  \l FA78
  \l FA79
  \l FA7A
  \l FA7B
  \l FA7C
  \l FA7D
  \l FA7E
  \l FA7F
  \l FA80
  \l FA81
  \l FA82
  \l FA83
  \l FA84
  \l FA85
  \l FA86
  \l FA87
  \l FA88
  \l FA89
  \l FA8A
  \l FA8B
  \l FA8C
  \l FA8D
  \l FA8E
  \l FA8F
  \l FA90
  \l FA91
  \l FA92
  \l FA93
  \l FA94
  \l FA95
  \l FA96
  \l FA97
  \l FA98
  \l FA99
  \l FA9A
  \l FA9B
  \l FA9C
  \l FA9D
  \l FA9E
  \l FA9F
  \l FAA0
  \l FAA1
  \l FAA2
  \l FAA3
  \l FAA4
  \l FAA5
  \l FAA6
  \l FAA7
  \l FAA8
  \l FAA9
  \l FAAA
  \l FAAB
  \l FAAC
  \l FAAD
  \l FAAE
  \l FAAF
  \l FAB0
  \l FAB1
  \l FAB2
  \l FAB3
  \l FAB4
  \l FAB5
  \l FAB6
  \l FAB7
  \l FAB8
  \l FAB9
  \l FABA
  \l FABB
  \l FABC
  \l FABD
  \l FABE
  \l FABF
  \l FAC0
  \l FAC1
  \l FAC2
  \l FAC3
  \l FAC4
  \l FAC5
  \l FAC6
  \l FAC7
  \l FAC8
  \l FAC9
  \l FACA
  \l FACB
  \l FACC
  \l FACD
  \l FACE
  \l FACF
  \l FAD0
  \l FAD1
  \l FAD2
  \l FAD3
  \l FAD4
  \l FAD5
  \l FAD6
  \l FAD7
  \l FAD8
  \l FAD9
  \l FB00
  \l FB01
  \l FB02
  \l FB03
  \l FB04
  \l FB05
  \l FB06
  \l FB13
  \l FB14
  \l FB15
  \l FB16
  \l FB17
  \l FB1D
  \l FB1E
  \l FB1F
  \l FB20
  \l FB21
  \l FB22
  \l FB23
  \l FB24
  \l FB25
  \l FB26
  \l FB27
  \l FB28
  \l FB2A
  \l FB2B
  \l FB2C
  \l FB2D
  \l FB2E
  \l FB2F
  \l FB30
  \l FB31
  \l FB32
  \l FB33
  \l FB34
  \l FB35
  \l FB36
  \l FB38
  \l FB39
  \l FB3A
  \l FB3B
  \l FB3C
  \l FB3E
  \l FB40
  \l FB41
  \l FB43
  \l FB44
  \l FB46
  \l FB47
  \l FB48
  \l FB49
  \l FB4A
  \l FB4B
  \l FB4C
  \l FB4D
  \l FB4E
  \l FB4F
  \l FB50
  \l FB51
  \l FB52
  \l FB53
  \l FB54
  \l FB55
  \l FB56
  \l FB57
  \l FB58
  \l FB59
  \l FB5A
  \l FB5B
  \l FB5C
  \l FB5D
  \l FB5E
  \l FB5F
  \l FB60
  \l FB61
  \l FB62
  \l FB63
  \l FB64
  \l FB65
  \l FB66
  \l FB67
  \l FB68
  \l FB69
  \l FB6A
  \l FB6B
  \l FB6C
  \l FB6D
  \l FB6E
  \l FB6F
  \l FB70
  \l FB71
  \l FB72
  \l FB73
  \l FB74
  \l FB75
  \l FB76
  \l FB77
  \l FB78
  \l FB79
  \l FB7A
  \l FB7B
  \l FB7C
  \l FB7D
  \l FB7E
  \l FB7F
  \l FB80
  \l FB81
  \l FB82
  \l FB83
  \l FB84
  \l FB85
  \l FB86
  \l FB87
  \l FB88
  \l FB89
  \l FB8A
  \l FB8B
  \l FB8C
  \l FB8D
  \l FB8E
  \l FB8F
  \l FB90
  \l FB91
  \l FB92
  \l FB93
  \l FB94
  \l FB95
  \l FB96
  \l FB97
  \l FB98
  \l FB99
  \l FB9A
  \l FB9B
  \l FB9C
  \l FB9D
  \l FB9E
  \l FB9F
  \l FBA0
  \l FBA1
  \l FBA2
  \l FBA3
  \l FBA4
  \l FBA5
  \l FBA6
  \l FBA7
  \l FBA8
  \l FBA9
  \l FBAA
  \l FBAB
  \l FBAC
  \l FBAD
  \l FBAE
  \l FBAF
  \l FBB0
  \l FBB1
  \l FBD3
  \l FBD4
  \l FBD5
  \l FBD6
  \l FBD7
  \l FBD8
  \l FBD9
  \l FBDA
  \l FBDB
  \l FBDC
  \l FBDD
  \l FBDE
  \l FBDF
  \l FBE0
  \l FBE1
  \l FBE2
  \l FBE3
  \l FBE4
  \l FBE5
  \l FBE6
  \l FBE7
  \l FBE8
  \l FBE9
  \l FBEA
  \l FBEB
  \l FBEC
  \l FBED
  \l FBEE
  \l FBEF
  \l FBF0
  \l FBF1
  \l FBF2
  \l FBF3
  \l FBF4
  \l FBF5
  \l FBF6
  \l FBF7
  \l FBF8
  \l FBF9
  \l FBFA
  \l FBFB
  \l FBFC
  \l FBFD
  \l FBFE
  \l FBFF
  \l FC00
  \l FC01
  \l FC02
  \l FC03
  \l FC04
  \l FC05
  \l FC06
  \l FC07
  \l FC08
  \l FC09
  \l FC0A
  \l FC0B
  \l FC0C
  \l FC0D
  \l FC0E
  \l FC0F
  \l FC10
  \l FC11
  \l FC12
  \l FC13
  \l FC14
  \l FC15
  \l FC16
  \l FC17
  \l FC18
  \l FC19
  \l FC1A
  \l FC1B
  \l FC1C
  \l FC1D
  \l FC1E
  \l FC1F
  \l FC20
  \l FC21
  \l FC22
  \l FC23
  \l FC24
  \l FC25
  \l FC26
  \l FC27
  \l FC28
  \l FC29
  \l FC2A
  \l FC2B
  \l FC2C
  \l FC2D
  \l FC2E
  \l FC2F
  \l FC30
  \l FC31
  \l FC32
  \l FC33
  \l FC34
  \l FC35
  \l FC36
  \l FC37
  \l FC38
  \l FC39
  \l FC3A
  \l FC3B
  \l FC3C
  \l FC3D
  \l FC3E
  \l FC3F
  \l FC40
  \l FC41
  \l FC42
  \l FC43
  \l FC44
  \l FC45
  \l FC46
  \l FC47
  \l FC48
  \l FC49
  \l FC4A
  \l FC4B
  \l FC4C
  \l FC4D
  \l FC4E
  \l FC4F
  \l FC50
  \l FC51
  \l FC52
  \l FC53
  \l FC54
  \l FC55
  \l FC56
  \l FC57
  \l FC58
  \l FC59
  \l FC5A
  \l FC5B
  \l FC5C
  \l FC5D
  \l FC5E
  \l FC5F
  \l FC60
  \l FC61
  \l FC62
  \l FC63
  \l FC64
  \l FC65
  \l FC66
  \l FC67
  \l FC68
  \l FC69
  \l FC6A
  \l FC6B
  \l FC6C
  \l FC6D
  \l FC6E
  \l FC6F
  \l FC70
  \l FC71
  \l FC72
  \l FC73
  \l FC74
  \l FC75
  \l FC76
  \l FC77
  \l FC78
  \l FC79
  \l FC7A
  \l FC7B
  \l FC7C
  \l FC7D
  \l FC7E
  \l FC7F
  \l FC80
  \l FC81
  \l FC82
  \l FC83
  \l FC84
  \l FC85
  \l FC86
  \l FC87
  \l FC88
  \l FC89
  \l FC8A
  \l FC8B
  \l FC8C
  \l FC8D
  \l FC8E
  \l FC8F
  \l FC90
  \l FC91
  \l FC92
  \l FC93
  \l FC94
  \l FC95
  \l FC96
  \l FC97
  \l FC98
  \l FC99
  \l FC9A
  \l FC9B
  \l FC9C
  \l FC9D
  \l FC9E
  \l FC9F
  \l FCA0
  \l FCA1
  \l FCA2
  \l FCA3
  \l FCA4
  \l FCA5
  \l FCA6
  \l FCA7
  \l FCA8
  \l FCA9
  \l FCAA
  \l FCAB
  \l FCAC
  \l FCAD
  \l FCAE
  \l FCAF
  \l FCB0
  \l FCB1
  \l FCB2
  \l FCB3
  \l FCB4
  \l FCB5
  \l FCB6
  \l FCB7
  \l FCB8
  \l FCB9
  \l FCBA
  \l FCBB
  \l FCBC
  \l FCBD
  \l FCBE
  \l FCBF
  \l FCC0
  \l FCC1
  \l FCC2
  \l FCC3
  \l FCC4
  \l FCC5
  \l FCC6
  \l FCC7
  \l FCC8
  \l FCC9
  \l FCCA
  \l FCCB
  \l FCCC
  \l FCCD
  \l FCCE
  \l FCCF
  \l FCD0
  \l FCD1
  \l FCD2
  \l FCD3
  \l FCD4
  \l FCD5
  \l FCD6
  \l FCD7
  \l FCD8
  \l FCD9
  \l FCDA
  \l FCDB
  \l FCDC
  \l FCDD
  \l FCDE
  \l FCDF
  \l FCE0
  \l FCE1
  \l FCE2
  \l FCE3
  \l FCE4
  \l FCE5
  \l FCE6
  \l FCE7
  \l FCE8
  \l FCE9
  \l FCEA
  \l FCEB
  \l FCEC
  \l FCED
  \l FCEE
  \l FCEF
  \l FCF0
  \l FCF1
  \l FCF2
  \l FCF3
  \l FCF4
  \l FCF5
  \l FCF6
  \l FCF7
  \l FCF8
  \l FCF9
  \l FCFA
  \l FCFB
  \l FCFC
  \l FCFD
  \l FCFE
  \l FCFF
  \l FD00
  \l FD01
  \l FD02
  \l FD03
  \l FD04
  \l FD05
  \l FD06
  \l FD07
  \l FD08
  \l FD09
  \l FD0A
  \l FD0B
  \l FD0C
  \l FD0D
  \l FD0E
  \l FD0F
  \l FD10
  \l FD11
  \l FD12
  \l FD13
  \l FD14
  \l FD15
  \l FD16
  \l FD17
  \l FD18
  \l FD19
  \l FD1A
  \l FD1B
  \l FD1C
  \l FD1D
  \l FD1E
  \l FD1F
  \l FD20
  \l FD21
  \l FD22
  \l FD23
  \l FD24
  \l FD25
  \l FD26
  \l FD27
  \l FD28
  \l FD29
  \l FD2A
  \l FD2B
  \l FD2C
  \l FD2D
  \l FD2E
  \l FD2F
  \l FD30
  \l FD31
  \l FD32
  \l FD33
  \l FD34
  \l FD35
  \l FD36
  \l FD37
  \l FD38
  \l FD39
  \l FD3A
  \l FD3B
  \l FD3C
  \l FD3D
  \l FD50
  \l FD51
  \l FD52
  \l FD53
  \l FD54
  \l FD55
  \l FD56
  \l FD57
  \l FD58
  \l FD59
  \l FD5A
  \l FD5B
  \l FD5C
  \l FD5D
  \l FD5E
  \l FD5F
  \l FD60
  \l FD61
  \l FD62
  \l FD63
  \l FD64
  \l FD65
  \l FD66
  \l FD67
  \l FD68
  \l FD69
  \l FD6A
  \l FD6B
  \l FD6C
  \l FD6D
  \l FD6E
  \l FD6F
  \l FD70
  \l FD71
  \l FD72
  \l FD73
  \l FD74
  \l FD75
  \l FD76
  \l FD77
  \l FD78
  \l FD79
  \l FD7A
  \l FD7B
  \l FD7C
  \l FD7D
  \l FD7E
  \l FD7F
  \l FD80
  \l FD81
  \l FD82
  \l FD83
  \l FD84
  \l FD85
  \l FD86
  \l FD87
  \l FD88
  \l FD89
  \l FD8A
  \l FD8B
  \l FD8C
  \l FD8D
  \l FD8E
  \l FD8F
  \l FD92
  \l FD93
  \l FD94
  \l FD95
  \l FD96
  \l FD97
  \l FD98
  \l FD99
  \l FD9A
  \l FD9B
  \l FD9C
  \l FD9D
  \l FD9E
  \l FD9F
  \l FDA0
  \l FDA1
  \l FDA2
  \l FDA3
  \l FDA4
  \l FDA5
  \l FDA6
  \l FDA7
  \l FDA8
  \l FDA9
  \l FDAA
  \l FDAB
  \l FDAC
  \l FDAD
  \l FDAE
  \l FDAF
  \l FDB0
  \l FDB1
  \l FDB2
  \l FDB3
  \l FDB4
  \l FDB5
  \l FDB6
  \l FDB7
  \l FDB8
  \l FDB9
  \l FDBA
  \l FDBB
  \l FDBC
  \l FDBD
  \l FDBE
  \l FDBF
  \l FDC0
  \l FDC1
  \l FDC2
  \l FDC3
  \l FDC4
  \l FDC5
  \l FDC6
  \l FDC7
  \l FDF0
  \l FDF1
  \l FDF2
  \l FDF3
  \l FDF4
  \l FDF5
  \l FDF6
  \l FDF7
  \l FDF8
  \l FDF9
  \l FDFA
  \l FDFB
  \l FE00
  \l FE01
  \l FE02
  \l FE03
  \l FE04
  \l FE05
  \l FE06
  \l FE07
  \l FE08
  \l FE09
  \l FE0A
  \l FE0B
  \l FE0C
  \l FE0D
  \l FE0E
  \l FE0F
  \l FE20
  \l FE21
  \l FE22
  \l FE23
  \l FE24
  \l FE25
  \l FE26
  \l FE27
  \l FE28
  \l FE29
  \l FE2A
  \l FE2B
  \l FE2C
  \l FE2D
  \l FE70
  \l FE71
  \l FE72
  \l FE73
  \l FE74
  \l FE76
  \l FE77
  \l FE78
  \l FE79
  \l FE7A
  \l FE7B
  \l FE7C
  \l FE7D
  \l FE7E
  \l FE7F
  \l FE80
  \l FE81
  \l FE82
  \l FE83
  \l FE84
  \l FE85
  \l FE86
  \l FE87
  \l FE88
  \l FE89
  \l FE8A
  \l FE8B
  \l FE8C
  \l FE8D
  \l FE8E
  \l FE8F
  \l FE90
  \l FE91
  \l FE92
  \l FE93
  \l FE94
  \l FE95
  \l FE96
  \l FE97
  \l FE98
  \l FE99
  \l FE9A
  \l FE9B
  \l FE9C
  \l FE9D
  \l FE9E
  \l FE9F
  \l FEA0
  \l FEA1
  \l FEA2
  \l FEA3
  \l FEA4
  \l FEA5
  \l FEA6
  \l FEA7
  \l FEA8
  \l FEA9
  \l FEAA
  \l FEAB
  \l FEAC
  \l FEAD
  \l FEAE
  \l FEAF
  \l FEB0
  \l FEB1
  \l FEB2
  \l FEB3
  \l FEB4
  \l FEB5
  \l FEB6
  \l FEB7
  \l FEB8
  \l FEB9
  \l FEBA
  \l FEBB
  \l FEBC
  \l FEBD
  \l FEBE
  \l FEBF
  \l FEC0
  \l FEC1
  \l FEC2
  \l FEC3
  \l FEC4
  \l FEC5
  \l FEC6
  \l FEC7
  \l FEC8
  \l FEC9
  \l FECA
  \l FECB
  \l FECC
  \l FECD
  \l FECE
  \l FECF
  \l FED0
  \l FED1
  \l FED2
  \l FED3
  \l FED4
  \l FED5
  \l FED6
  \l FED7
  \l FED8
  \l FED9
  \l FEDA
  \l FEDB
  \l FEDC
  \l FEDD
  \l FEDE
  \l FEDF
  \l FEE0
  \l FEE1
  \l FEE2
  \l FEE3
  \l FEE4
  \l FEE5
  \l FEE6
  \l FEE7
  \l FEE8
  \l FEE9
  \l FEEA
  \l FEEB
  \l FEEC
  \l FEED
  \l FEEE
  \l FEEF
  \l FEF0
  \l FEF1
  \l FEF2
  \l FEF3
  \l FEF4
  \l FEF5
  \l FEF6
  \l FEF7
  \l FEF8
  \l FEF9
  \l FEFA
  \l FEFB
  \l FEFC
  \L FF21 FF21 FF41
  \L FF22 FF22 FF42
  \L FF23 FF23 FF43
  \L FF24 FF24 FF44
  \L FF25 FF25 FF45
  \L FF26 FF26 FF46
  \L FF27 FF27 FF47
  \L FF28 FF28 FF48
  \L FF29 FF29 FF49
  \L FF2A FF2A FF4A
  \L FF2B FF2B FF4B
  \L FF2C FF2C FF4C
  \L FF2D FF2D FF4D
  \L FF2E FF2E FF4E
  \L FF2F FF2F FF4F
  \L FF30 FF30 FF50
  \L FF31 FF31 FF51
  \L FF32 FF32 FF52
  \L FF33 FF33 FF53
  \L FF34 FF34 FF54
  \L FF35 FF35 FF55
  \L FF36 FF36 FF56
  \L FF37 FF37 FF57
  \L FF38 FF38 FF58
  \L FF39 FF39 FF59
  \L FF3A FF3A FF5A
  \L FF41 FF21 FF41
  \L FF42 FF22 FF42
  \L FF43 FF23 FF43
  \L FF44 FF24 FF44
  \L FF45 FF25 FF45
  \L FF46 FF26 FF46
  \L FF47 FF27 FF47
  \L FF48 FF28 FF48
  \L FF49 FF29 FF49
  \L FF4A FF2A FF4A
  \L FF4B FF2B FF4B
  \L FF4C FF2C FF4C
  \L FF4D FF2D FF4D
  \L FF4E FF2E FF4E
  \L FF4F FF2F FF4F
  \L FF50 FF30 FF50
  \L FF51 FF31 FF51
  \L FF52 FF32 FF52
  \L FF53 FF33 FF53
  \L FF54 FF34 FF54
  \L FF55 FF35 FF55
  \L FF56 FF36 FF56
  \L FF57 FF37 FF57
  \L FF58 FF38 FF58
  \L FF59 FF39 FF59
  \L FF5A FF3A FF5A
  \l FF66
  \l FF67
  \l FF68
  \l FF69
  \l FF6A
  \l FF6B
  \l FF6C
  \l FF6D
  \l FF6E
  \l FF6F
  \l FF70
  \l FF71
  \l FF72
  \l FF73
  \l FF74
  \l FF75
  \l FF76
  \l FF77
  \l FF78
  \l FF79
  \l FF7A
  \l FF7B
  \l FF7C
  \l FF7D
  \l FF7E
  \l FF7F
  \l FF80
  \l FF81
  \l FF82
  \l FF83
  \l FF84
  \l FF85
  \l FF86
  \l FF87
  \l FF88
  \l FF89
  \l FF8A
  \l FF8B
  \l FF8C
  \l FF8D
  \l FF8E
  \l FF8F
  \l FF90
  \l FF91
  \l FF92
  \l FF93
  \l FF94
  \l FF95
  \l FF96
  \l FF97
  \l FF98
  \l FF99
  \l FF9A
  \l FF9B
  \l FF9C
  \l FF9D
  \l FF9E
  \l FF9F
  \l FFA0
  \l FFA1
  \l FFA2
  \l FFA3
  \l FFA4
  \l FFA5
  \l FFA6
  \l FFA7
  \l FFA8
  \l FFA9
  \l FFAA
  \l FFAB
  \l FFAC
  \l FFAD
  \l FFAE
  \l FFAF
  \l FFB0
  \l FFB1
  \l FFB2
  \l FFB3
  \l FFB4
  \l FFB5
  \l FFB6
  \l FFB7
  \l FFB8
  \l FFB9
  \l FFBA
  \l FFBB
  \l FFBC
  \l FFBD
  \l FFBE
  \l FFC2
  \l FFC3
  \l FFC4
  \l FFC5
  \l FFC6
  \l FFC7
  \l FFCA
  \l FFCB
  \l FFCC
  \l FFCD
  \l FFCE
  \l FFCF
  \l FFD2
  \l FFD3
  \l FFD4
  \l FFD5
  \l FFD6
  \l FFD7
  \l FFDA
  \l FFDB
  \l FFDC
  \l 10000
  \l 10001
  \l 10002
  \l 10003
  \l 10004
  \l 10005
  \l 10006
  \l 10007
  \l 10008
  \l 10009
  \l 1000A
  \l 1000B
  \l 1000D
  \l 1000E
  \l 1000F
  \l 10010
  \l 10011
  \l 10012
  \l 10013
  \l 10014
  \l 10015
  \l 10016
  \l 10017
  \l 10018
  \l 10019
  \l 1001A
  \l 1001B
  \l 1001C
  \l 1001D
  \l 1001E
  \l 1001F
  \l 10020
  \l 10021
  \l 10022
  \l 10023
  \l 10024
  \l 10025
  \l 10026
  \l 10028
  \l 10029
  \l 1002A
  \l 1002B
  \l 1002C
  \l 1002D
  \l 1002E
  \l 1002F
  \l 10030
  \l 10031
  \l 10032
  \l 10033
  \l 10034
  \l 10035
  \l 10036
  \l 10037
  \l 10038
  \l 10039
  \l 1003A
  \l 1003C
  \l 1003D
  \l 1003F
  \l 10040
  \l 10041
  \l 10042
  \l 10043
  \l 10044
  \l 10045
  \l 10046
  \l 10047
  \l 10048
  \l 10049
  \l 1004A
  \l 1004B
  \l 1004C
  \l 1004D
  \l 10050
  \l 10051
  \l 10052
  \l 10053
  \l 10054
  \l 10055
  \l 10056
  \l 10057
  \l 10058
  \l 10059
  \l 1005A
  \l 1005B
  \l 1005C
  \l 1005D
  \l 10080
  \l 10081
  \l 10082
  \l 10083
  \l 10084
  \l 10085
  \l 10086
  \l 10087
  \l 10088
  \l 10089
  \l 1008A
  \l 1008B
  \l 1008C
  \l 1008D
  \l 1008E
  \l 1008F
  \l 10090
  \l 10091
  \l 10092
  \l 10093
  \l 10094
  \l 10095
  \l 10096
  \l 10097
  \l 10098
  \l 10099
  \l 1009A
  \l 1009B
  \l 1009C
  \l 1009D
  \l 1009E
  \l 1009F
  \l 100A0
  \l 100A1
  \l 100A2
  \l 100A3
  \l 100A4
  \l 100A5
  \l 100A6
  \l 100A7
  \l 100A8
  \l 100A9
  \l 100AA
  \l 100AB
  \l 100AC
  \l 100AD
  \l 100AE
  \l 100AF
  \l 100B0
  \l 100B1
  \l 100B2
  \l 100B3
  \l 100B4
  \l 100B5
  \l 100B6
  \l 100B7
  \l 100B8
  \l 100B9
  \l 100BA
  \l 100BB
  \l 100BC
  \l 100BD
  \l 100BE
  \l 100BF
  \l 100C0
  \l 100C1
  \l 100C2
  \l 100C3
  \l 100C4
  \l 100C5
  \l 100C6
  \l 100C7
  \l 100C8
  \l 100C9
  \l 100CA
  \l 100CB
  \l 100CC
  \l 100CD
  \l 100CE
  \l 100CF
  \l 100D0
  \l 100D1
  \l 100D2
  \l 100D3
  \l 100D4
  \l 100D5
  \l 100D6
  \l 100D7
  \l 100D8
  \l 100D9
  \l 100DA
  \l 100DB
  \l 100DC
  \l 100DD
  \l 100DE
  \l 100DF
  \l 100E0
  \l 100E1
  \l 100E2
  \l 100E3
  \l 100E4
  \l 100E5
  \l 100E6
  \l 100E7
  \l 100E8
  \l 100E9
  \l 100EA
  \l 100EB
  \l 100EC
  \l 100ED
  \l 100EE
  \l 100EF
  \l 100F0
  \l 100F1
  \l 100F2
  \l 100F3
  \l 100F4
  \l 100F5
  \l 100F6
  \l 100F7
  \l 100F8
  \l 100F9
  \l 100FA
  \l 101FD
  \l 10280
  \l 10281
  \l 10282
  \l 10283
  \l 10284
  \l 10285
  \l 10286
  \l 10287
  \l 10288
  \l 10289
  \l 1028A
  \l 1028B
  \l 1028C
  \l 1028D
  \l 1028E
  \l 1028F
  \l 10290
  \l 10291
  \l 10292
  \l 10293
  \l 10294
  \l 10295
  \l 10296
  \l 10297
  \l 10298
  \l 10299
  \l 1029A
  \l 1029B
  \l 1029C
  \l 102A0
  \l 102A1
  \l 102A2
  \l 102A3
  \l 102A4
  \l 102A5
  \l 102A6
  \l 102A7
  \l 102A8
  \l 102A9
  \l 102AA
  \l 102AB
  \l 102AC
  \l 102AD
  \l 102AE
  \l 102AF
  \l 102B0
  \l 102B1
  \l 102B2
  \l 102B3
  \l 102B4
  \l 102B5
  \l 102B6
  \l 102B7
  \l 102B8
  \l 102B9
  \l 102BA
  \l 102BB
  \l 102BC
  \l 102BD
  \l 102BE
  \l 102BF
  \l 102C0
  \l 102C1
  \l 102C2
  \l 102C3
  \l 102C4
  \l 102C5
  \l 102C6
  \l 102C7
  \l 102C8
  \l 102C9
  \l 102CA
  \l 102CB
  \l 102CC
  \l 102CD
  \l 102CE
  \l 102CF
  \l 102D0
  \l 102E0
  \l 10300
  \l 10301
  \l 10302
  \l 10303
  \l 10304
  \l 10305
  \l 10306
  \l 10307
  \l 10308
  \l 10309
  \l 1030A
  \l 1030B
  \l 1030C
  \l 1030D
  \l 1030E
  \l 1030F
  \l 10310
  \l 10311
  \l 10312
  \l 10313
  \l 10314
  \l 10315
  \l 10316
  \l 10317
  \l 10318
  \l 10319
  \l 1031A
  \l 1031B
  \l 1031C
  \l 1031D
  \l 1031E
  \l 1031F
  \l 10330
  \l 10331
  \l 10332
  \l 10333
  \l 10334
  \l 10335
  \l 10336
  \l 10337
  \l 10338
  \l 10339
  \l 1033A
  \l 1033B
  \l 1033C
  \l 1033D
  \l 1033E
  \l 1033F
  \l 10340
  \l 10342
  \l 10343
  \l 10344
  \l 10345
  \l 10346
  \l 10347
  \l 10348
  \l 10349
  \l 10350
  \l 10351
  \l 10352
  \l 10353
  \l 10354
  \l 10355
  \l 10356
  \l 10357
  \l 10358
  \l 10359
  \l 1035A
  \l 1035B
  \l 1035C
  \l 1035D
  \l 1035E
  \l 1035F
  \l 10360
  \l 10361
  \l 10362
  \l 10363
  \l 10364
  \l 10365
  \l 10366
  \l 10367
  \l 10368
  \l 10369
  \l 1036A
  \l 1036B
  \l 1036C
  \l 1036D
  \l 1036E
  \l 1036F
  \l 10370
  \l 10371
  \l 10372
  \l 10373
  \l 10374
  \l 10375
  \l 10376
  \l 10377
  \l 10378
  \l 10379
  \l 1037A
  \l 10380
  \l 10381
  \l 10382
  \l 10383
  \l 10384
  \l 10385
  \l 10386
  \l 10387
  \l 10388
  \l 10389
  \l 1038A
  \l 1038B
  \l 1038C
  \l 1038D
  \l 1038E
  \l 1038F
  \l 10390
  \l 10391
  \l 10392
  \l 10393
  \l 10394
  \l 10395
  \l 10396
  \l 10397
  \l 10398
  \l 10399
  \l 1039A
  \l 1039B
  \l 1039C
  \l 1039D
  \l 103A0
  \l 103A1
  \l 103A2
  \l 103A3
  \l 103A4
  \l 103A5
  \l 103A6
  \l 103A7
  \l 103A8
  \l 103A9
  \l 103AA
  \l 103AB
  \l 103AC
  \l 103AD
  \l 103AE
  \l 103AF
  \l 103B0
  \l 103B1
  \l 103B2
  \l 103B3
  \l 103B4
  \l 103B5
  \l 103B6
  \l 103B7
  \l 103B8
  \l 103B9
  \l 103BA
  \l 103BB
  \l 103BC
  \l 103BD
  \l 103BE
  \l 103BF
  \l 103C0
  \l 103C1
  \l 103C2
  \l 103C3
  \l 103C8
  \l 103C9
  \l 103CA
  \l 103CB
  \l 103CC
  \l 103CD
  \l 103CE
  \l 103CF
  \L 10400 10400 10428
  \L 10401 10401 10429
  \L 10402 10402 1042A
  \L 10403 10403 1042B
  \L 10404 10404 1042C
  \L 10405 10405 1042D
  \L 10406 10406 1042E
  \L 10407 10407 1042F
  \L 10408 10408 10430
  \L 10409 10409 10431
  \L 1040A 1040A 10432
  \L 1040B 1040B 10433
  \L 1040C 1040C 10434
  \L 1040D 1040D 10435
  \L 1040E 1040E 10436
  \L 1040F 1040F 10437
  \L 10410 10410 10438
  \L 10411 10411 10439
  \L 10412 10412 1043A
  \L 10413 10413 1043B
  \L 10414 10414 1043C
  \L 10415 10415 1043D
  \L 10416 10416 1043E
  \L 10417 10417 1043F
  \L 10418 10418 10440
  \L 10419 10419 10441
  \L 1041A 1041A 10442
  \L 1041B 1041B 10443
  \L 1041C 1041C 10444
  \L 1041D 1041D 10445
  \L 1041E 1041E 10446
  \L 1041F 1041F 10447
  \L 10420 10420 10448
  \L 10421 10421 10449
  \L 10422 10422 1044A
  \L 10423 10423 1044B
  \L 10424 10424 1044C
  \L 10425 10425 1044D
  \L 10426 10426 1044E
  \L 10427 10427 1044F
  \L 10428 10400 10428
  \L 10429 10401 10429
  \L 1042A 10402 1042A
  \L 1042B 10403 1042B
  \L 1042C 10404 1042C
  \L 1042D 10405 1042D
  \L 1042E 10406 1042E
  \L 1042F 10407 1042F
  \L 10430 10408 10430
  \L 10431 10409 10431
  \L 10432 1040A 10432
  \L 10433 1040B 10433
  \L 10434 1040C 10434
  \L 10435 1040D 10435
  \L 10436 1040E 10436
  \L 10437 1040F 10437
  \L 10438 10410 10438
  \L 10439 10411 10439
  \L 1043A 10412 1043A
  \L 1043B 10413 1043B
  \L 1043C 10414 1043C
  \L 1043D 10415 1043D
  \L 1043E 10416 1043E
  \L 1043F 10417 1043F
  \L 10440 10418 10440
  \L 10441 10419 10441
  \L 10442 1041A 10442
  \L 10443 1041B 10443
  \L 10444 1041C 10444
  \L 10445 1041D 10445
  \L 10446 1041E 10446
  \L 10447 1041F 10447
  \L 10448 10420 10448
  \L 10449 10421 10449
  \L 1044A 10422 1044A
  \L 1044B 10423 1044B
  \L 1044C 10424 1044C
  \L 1044D 10425 1044D
  \L 1044E 10426 1044E
  \L 1044F 10427 1044F
  \l 10450
  \l 10451
  \l 10452
  \l 10453
  \l 10454
  \l 10455
  \l 10456
  \l 10457
  \l 10458
  \l 10459
  \l 1045A
  \l 1045B
  \l 1045C
  \l 1045D
  \l 1045E
  \l 1045F
  \l 10460
  \l 10461
  \l 10462
  \l 10463
  \l 10464
  \l 10465
  \l 10466
  \l 10467
  \l 10468
  \l 10469
  \l 1046A
  \l 1046B
  \l 1046C
  \l 1046D
  \l 1046E
  \l 1046F
  \l 10470
  \l 10471
  \l 10472
  \l 10473
  \l 10474
  \l 10475
  \l 10476
  \l 10477
  \l 10478
  \l 10479
  \l 1047A
  \l 1047B
  \l 1047C
  \l 1047D
  \l 1047E
  \l 1047F
  \l 10480
  \l 10481
  \l 10482
  \l 10483
  \l 10484
  \l 10485
  \l 10486
  \l 10487
  \l 10488
  \l 10489
  \l 1048A
  \l 1048B
  \l 1048C
  \l 1048D
  \l 1048E
  \l 1048F
  \l 10490
  \l 10491
  \l 10492
  \l 10493
  \l 10494
  \l 10495
  \l 10496
  \l 10497
  \l 10498
  \l 10499
  \l 1049A
  \l 1049B
  \l 1049C
  \l 1049D
  \l 10500
  \l 10501
  \l 10502
  \l 10503
  \l 10504
  \l 10505
  \l 10506
  \l 10507
  \l 10508
  \l 10509
  \l 1050A
  \l 1050B
  \l 1050C
  \l 1050D
  \l 1050E
  \l 1050F
  \l 10510
  \l 10511
  \l 10512
  \l 10513
  \l 10514
  \l 10515
  \l 10516
  \l 10517
  \l 10518
  \l 10519
  \l 1051A
  \l 1051B
  \l 1051C
  \l 1051D
  \l 1051E
  \l 1051F
  \l 10520
  \l 10521
  \l 10522
  \l 10523
  \l 10524
  \l 10525
  \l 10526
  \l 10527
  \l 10530
  \l 10531
  \l 10532
  \l 10533
  \l 10534
  \l 10535
  \l 10536
  \l 10537
  \l 10538
  \l 10539
  \l 1053A
  \l 1053B
  \l 1053C
  \l 1053D
  \l 1053E
  \l 1053F
  \l 10540
  \l 10541
  \l 10542
  \l 10543
  \l 10544
  \l 10545
  \l 10546
  \l 10547
  \l 10548
  \l 10549
  \l 1054A
  \l 1054B
  \l 1054C
  \l 1054D
  \l 1054E
  \l 1054F
  \l 10550
  \l 10551
  \l 10552
  \l 10553
  \l 10554
  \l 10555
  \l 10556
  \l 10557
  \l 10558
  \l 10559
  \l 1055A
  \l 1055B
  \l 1055C
  \l 1055D
  \l 1055E
  \l 1055F
  \l 10560
  \l 10561
  \l 10562
  \l 10563
  \l 10600
  \l 10601
  \l 10602
  \l 10603
  \l 10604
  \l 10605
  \l 10606
  \l 10607
  \l 10608
  \l 10609
  \l 1060A
  \l 1060B
  \l 1060C
  \l 1060D
  \l 1060E
  \l 1060F
  \l 10610
  \l 10611
  \l 10612
  \l 10613
  \l 10614
  \l 10615
  \l 10616
  \l 10617
  \l 10618
  \l 10619
  \l 1061A
  \l 1061B
  \l 1061C
  \l 1061D
  \l 1061E
  \l 1061F
  \l 10620
  \l 10621
  \l 10622
  \l 10623
  \l 10624
  \l 10625
  \l 10626
  \l 10627
  \l 10628
  \l 10629
  \l 1062A
  \l 1062B
  \l 1062C
  \l 1062D
  \l 1062E
  \l 1062F
  \l 10630
  \l 10631
  \l 10632
  \l 10633
  \l 10634
  \l 10635
  \l 10636
  \l 10637
  \l 10638
  \l 10639
  \l 1063A
  \l 1063B
  \l 1063C
  \l 1063D
  \l 1063E
  \l 1063F
  \l 10640
  \l 10641
  \l 10642
  \l 10643
  \l 10644
  \l 10645
  \l 10646
  \l 10647
  \l 10648
  \l 10649
  \l 1064A
  \l 1064B
  \l 1064C
  \l 1064D
  \l 1064E
  \l 1064F
  \l 10650
  \l 10651
  \l 10652
  \l 10653
  \l 10654
  \l 10655
  \l 10656
  \l 10657
  \l 10658
  \l 10659
  \l 1065A
  \l 1065B
  \l 1065C
  \l 1065D
  \l 1065E
  \l 1065F
  \l 10660
  \l 10661
  \l 10662
  \l 10663
  \l 10664
  \l 10665
  \l 10666
  \l 10667
  \l 10668
  \l 10669
  \l 1066A
  \l 1066B
  \l 1066C
  \l 1066D
  \l 1066E
  \l 1066F
  \l 10670
  \l 10671
  \l 10672
  \l 10673
  \l 10674
  \l 10675
  \l 10676
  \l 10677
  \l 10678
  \l 10679
  \l 1067A
  \l 1067B
  \l 1067C
  \l 1067D
  \l 1067E
  \l 1067F
  \l 10680
  \l 10681
  \l 10682
  \l 10683
  \l 10684
  \l 10685
  \l 10686
  \l 10687
  \l 10688
  \l 10689
  \l 1068A
  \l 1068B
  \l 1068C
  \l 1068D
  \l 1068E
  \l 1068F
  \l 10690
  \l 10691
  \l 10692
  \l 10693
  \l 10694
  \l 10695
  \l 10696
  \l 10697
  \l 10698
  \l 10699
  \l 1069A
  \l 1069B
  \l 1069C
  \l 1069D
  \l 1069E
  \l 1069F
  \l 106A0
  \l 106A1
  \l 106A2
  \l 106A3
  \l 106A4
  \l 106A5
  \l 106A6
  \l 106A7
  \l 106A8
  \l 106A9
  \l 106AA
  \l 106AB
  \l 106AC
  \l 106AD
  \l 106AE
  \l 106AF
  \l 106B0
  \l 106B1
  \l 106B2
  \l 106B3
  \l 106B4
  \l 106B5
  \l 106B6
  \l 106B7
  \l 106B8
  \l 106B9
  \l 106BA
  \l 106BB
  \l 106BC
  \l 106BD
  \l 106BE
  \l 106BF
  \l 106C0
  \l 106C1
  \l 106C2
  \l 106C3
  \l 106C4
  \l 106C5
  \l 106C6
  \l 106C7
  \l 106C8
  \l 106C9
  \l 106CA
  \l 106CB
  \l 106CC
  \l 106CD
  \l 106CE
  \l 106CF
  \l 106D0
  \l 106D1
  \l 106D2
  \l 106D3
  \l 106D4
  \l 106D5
  \l 106D6
  \l 106D7
  \l 106D8
  \l 106D9
  \l 106DA
  \l 106DB
  \l 106DC
  \l 106DD
  \l 106DE
  \l 106DF
  \l 106E0
  \l 106E1
  \l 106E2
  \l 106E3
  \l 106E4
  \l 106E5
  \l 106E6
  \l 106E7
  \l 106E8
  \l 106E9
  \l 106EA
  \l 106EB
  \l 106EC
  \l 106ED
  \l 106EE
  \l 106EF
  \l 106F0
  \l 106F1
  \l 106F2
  \l 106F3
  \l 106F4
  \l 106F5
  \l 106F6
  \l 106F7
  \l 106F8
  \l 106F9
  \l 106FA
  \l 106FB
  \l 106FC
  \l 106FD
  \l 106FE
  \l 106FF
  \l 10700
  \l 10701
  \l 10702
  \l 10703
  \l 10704
  \l 10705
  \l 10706
  \l 10707
  \l 10708
  \l 10709
  \l 1070A
  \l 1070B
  \l 1070C
  \l 1070D
  \l 1070E
  \l 1070F
  \l 10710
  \l 10711
  \l 10712
  \l 10713
  \l 10714
  \l 10715
  \l 10716
  \l 10717
  \l 10718
  \l 10719
  \l 1071A
  \l 1071B
  \l 1071C
  \l 1071D
  \l 1071E
  \l 1071F
  \l 10720
  \l 10721
  \l 10722
  \l 10723
  \l 10724
  \l 10725
  \l 10726
  \l 10727
  \l 10728
  \l 10729
  \l 1072A
  \l 1072B
  \l 1072C
  \l 1072D
  \l 1072E
  \l 1072F
  \l 10730
  \l 10731
  \l 10732
  \l 10733
  \l 10734
  \l 10735
  \l 10736
  \l 10740
  \l 10741
  \l 10742
  \l 10743
  \l 10744
  \l 10745
  \l 10746
  \l 10747
  \l 10748
  \l 10749
  \l 1074A
  \l 1074B
  \l 1074C
  \l 1074D
  \l 1074E
  \l 1074F
  \l 10750
  \l 10751
  \l 10752
  \l 10753
  \l 10754
  \l 10755
  \l 10760
  \l 10761
  \l 10762
  \l 10763
  \l 10764
  \l 10765
  \l 10766
  \l 10767
  \l 10800
  \l 10801
  \l 10802
  \l 10803
  \l 10804
  \l 10805
  \l 10808
  \l 1080A
  \l 1080B
  \l 1080C
  \l 1080D
  \l 1080E
  \l 1080F
  \l 10810
  \l 10811
  \l 10812
  \l 10813
  \l 10814
  \l 10815
  \l 10816
  \l 10817
  \l 10818
  \l 10819
  \l 1081A
  \l 1081B
  \l 1081C
  \l 1081D
  \l 1081E
  \l 1081F
  \l 10820
  \l 10821
  \l 10822
  \l 10823
  \l 10824
  \l 10825
  \l 10826
  \l 10827
  \l 10828
  \l 10829
  \l 1082A
  \l 1082B
  \l 1082C
  \l 1082D
  \l 1082E
  \l 1082F
  \l 10830
  \l 10831
  \l 10832
  \l 10833
  \l 10834
  \l 10835
  \l 10837
  \l 10838
  \l 1083C
  \l 1083F
  \l 10840
  \l 10841
  \l 10842
  \l 10843
  \l 10844
  \l 10845
  \l 10846
  \l 10847
  \l 10848
  \l 10849
  \l 1084A
  \l 1084B
  \l 1084C
  \l 1084D
  \l 1084E
  \l 1084F
  \l 10850
  \l 10851
  \l 10852
  \l 10853
  \l 10854
  \l 10855
  \l 10860
  \l 10861
  \l 10862
  \l 10863
  \l 10864
  \l 10865
  \l 10866
  \l 10867
  \l 10868
  \l 10869
  \l 1086A
  \l 1086B
  \l 1086C
  \l 1086D
  \l 1086E
  \l 1086F
  \l 10870
  \l 10871
  \l 10872
  \l 10873
  \l 10874
  \l 10875
  \l 10876
  \l 10880
  \l 10881
  \l 10882
  \l 10883
  \l 10884
  \l 10885
  \l 10886
  \l 10887
  \l 10888
  \l 10889
  \l 1088A
  \l 1088B
  \l 1088C
  \l 1088D
  \l 1088E
  \l 1088F
  \l 10890
  \l 10891
  \l 10892
  \l 10893
  \l 10894
  \l 10895
  \l 10896
  \l 10897
  \l 10898
  \l 10899
  \l 1089A
  \l 1089B
  \l 1089C
  \l 1089D
  \l 1089E
  \l 10900
  \l 10901
  \l 10902
  \l 10903
  \l 10904
  \l 10905
  \l 10906
  \l 10907
  \l 10908
  \l 10909
  \l 1090A
  \l 1090B
  \l 1090C
  \l 1090D
  \l 1090E
  \l 1090F
  \l 10910
  \l 10911
  \l 10912
  \l 10913
  \l 10914
  \l 10915
  \l 10920
  \l 10921
  \l 10922
  \l 10923
  \l 10924
  \l 10925
  \l 10926
  \l 10927
  \l 10928
  \l 10929
  \l 1092A
  \l 1092B
  \l 1092C
  \l 1092D
  \l 1092E
  \l 1092F
  \l 10930
  \l 10931
  \l 10932
  \l 10933
  \l 10934
  \l 10935
  \l 10936
  \l 10937
  \l 10938
  \l 10939
  \l 10980
  \l 10981
  \l 10982
  \l 10983
  \l 10984
  \l 10985
  \l 10986
  \l 10987
  \l 10988
  \l 10989
  \l 1098A
  \l 1098B
  \l 1098C
  \l 1098D
  \l 1098E
  \l 1098F
  \l 10990
  \l 10991
  \l 10992
  \l 10993
  \l 10994
  \l 10995
  \l 10996
  \l 10997
  \l 10998
  \l 10999
  \l 1099A
  \l 1099B
  \l 1099C
  \l 1099D
  \l 1099E
  \l 1099F
  \l 109A0
  \l 109A1
  \l 109A2
  \l 109A3
  \l 109A4
  \l 109A5
  \l 109A6
  \l 109A7
  \l 109A8
  \l 109A9
  \l 109AA
  \l 109AB
  \l 109AC
  \l 109AD
  \l 109AE
  \l 109AF
  \l 109B0
  \l 109B1
  \l 109B2
  \l 109B3
  \l 109B4
  \l 109B5
  \l 109B6
  \l 109B7
  \l 109BE
  \l 109BF
  \l 10A00
  \l 10A01
  \l 10A02
  \l 10A03
  \l 10A05
  \l 10A06
  \l 10A0C
  \l 10A0D
  \l 10A0E
  \l 10A0F
  \l 10A10
  \l 10A11
  \l 10A12
  \l 10A13
  \l 10A15
  \l 10A16
  \l 10A17
  \l 10A19
  \l 10A1A
  \l 10A1B
  \l 10A1C
  \l 10A1D
  \l 10A1E
  \l 10A1F
  \l 10A20
  \l 10A21
  \l 10A22
  \l 10A23
  \l 10A24
  \l 10A25
  \l 10A26
  \l 10A27
  \l 10A28
  \l 10A29
  \l 10A2A
  \l 10A2B
  \l 10A2C
  \l 10A2D
  \l 10A2E
  \l 10A2F
  \l 10A30
  \l 10A31
  \l 10A32
  \l 10A33
  \l 10A38
  \l 10A39
  \l 10A3A
  \l 10A3F
  \l 10A60
  \l 10A61
  \l 10A62
  \l 10A63
  \l 10A64
  \l 10A65
  \l 10A66
  \l 10A67
  \l 10A68
  \l 10A69
  \l 10A6A
  \l 10A6B
  \l 10A6C
  \l 10A6D
  \l 10A6E
  \l 10A6F
  \l 10A70
  \l 10A71
  \l 10A72
  \l 10A73
  \l 10A74
  \l 10A75
  \l 10A76
  \l 10A77
  \l 10A78
  \l 10A79
  \l 10A7A
  \l 10A7B
  \l 10A7C
  \l 10A80
  \l 10A81
  \l 10A82
  \l 10A83
  \l 10A84
  \l 10A85
  \l 10A86
  \l 10A87
  \l 10A88
  \l 10A89
  \l 10A8A
  \l 10A8B
  \l 10A8C
  \l 10A8D
  \l 10A8E
  \l 10A8F
  \l 10A90
  \l 10A91
  \l 10A92
  \l 10A93
  \l 10A94
  \l 10A95
  \l 10A96
  \l 10A97
  \l 10A98
  \l 10A99
  \l 10A9A
  \l 10A9B
  \l 10A9C
  \l 10AC0
  \l 10AC1
  \l 10AC2
  \l 10AC3
  \l 10AC4
  \l 10AC5
  \l 10AC6
  \l 10AC7
  \l 10AC9
  \l 10ACA
  \l 10ACB
  \l 10ACC
  \l 10ACD
  \l 10ACE
  \l 10ACF
  \l 10AD0
  \l 10AD1
  \l 10AD2
  \l 10AD3
  \l 10AD4
  \l 10AD5
  \l 10AD6
  \l 10AD7
  \l 10AD8
  \l 10AD9
  \l 10ADA
  \l 10ADB
  \l 10ADC
  \l 10ADD
  \l 10ADE
  \l 10ADF
  \l 10AE0
  \l 10AE1
  \l 10AE2
  \l 10AE3
  \l 10AE4
  \l 10AE5
  \l 10AE6
  \l 10B00
  \l 10B01
  \l 10B02
  \l 10B03
  \l 10B04
  \l 10B05
  \l 10B06
  \l 10B07
  \l 10B08
  \l 10B09
  \l 10B0A
  \l 10B0B
  \l 10B0C
  \l 10B0D
  \l 10B0E
  \l 10B0F
  \l 10B10
  \l 10B11
  \l 10B12
  \l 10B13
  \l 10B14
  \l 10B15
  \l 10B16
  \l 10B17
  \l 10B18
  \l 10B19
  \l 10B1A
  \l 10B1B
  \l 10B1C
  \l 10B1D
  \l 10B1E
  \l 10B1F
  \l 10B20
  \l 10B21
  \l 10B22
  \l 10B23
  \l 10B24
  \l 10B25
  \l 10B26
  \l 10B27
  \l 10B28
  \l 10B29
  \l 10B2A
  \l 10B2B
  \l 10B2C
  \l 10B2D
  \l 10B2E
  \l 10B2F
  \l 10B30
  \l 10B31
  \l 10B32
  \l 10B33
  \l 10B34
  \l 10B35
  \l 10B40
  \l 10B41
  \l 10B42
  \l 10B43
  \l 10B44
  \l 10B45
  \l 10B46
  \l 10B47
  \l 10B48
  \l 10B49
  \l 10B4A
  \l 10B4B
  \l 10B4C
  \l 10B4D
  \l 10B4E
  \l 10B4F
  \l 10B50
  \l 10B51
  \l 10B52
  \l 10B53
  \l 10B54
  \l 10B55
  \l 10B60
  \l 10B61
  \l 10B62
  \l 10B63
  \l 10B64
  \l 10B65
  \l 10B66
  \l 10B67
  \l 10B68
  \l 10B69
  \l 10B6A
  \l 10B6B
  \l 10B6C
  \l 10B6D
  \l 10B6E
  \l 10B6F
  \l 10B70
  \l 10B71
  \l 10B72
  \l 10B80
  \l 10B81
  \l 10B82
  \l 10B83
  \l 10B84
  \l 10B85
  \l 10B86
  \l 10B87
  \l 10B88
  \l 10B89
  \l 10B8A
  \l 10B8B
  \l 10B8C
  \l 10B8D
  \l 10B8E
  \l 10B8F
  \l 10B90
  \l 10B91
  \l 10C00
  \l 10C01
  \l 10C02
  \l 10C03
  \l 10C04
  \l 10C05
  \l 10C06
  \l 10C07
  \l 10C08
  \l 10C09
  \l 10C0A
  \l 10C0B
  \l 10C0C
  \l 10C0D
  \l 10C0E
  \l 10C0F
  \l 10C10
  \l 10C11
  \l 10C12
  \l 10C13
  \l 10C14
  \l 10C15
  \l 10C16
  \l 10C17
  \l 10C18
  \l 10C19
  \l 10C1A
  \l 10C1B
  \l 10C1C
  \l 10C1D
  \l 10C1E
  \l 10C1F
  \l 10C20
  \l 10C21
  \l 10C22
  \l 10C23
  \l 10C24
  \l 10C25
  \l 10C26
  \l 10C27
  \l 10C28
  \l 10C29
  \l 10C2A
  \l 10C2B
  \l 10C2C
  \l 10C2D
  \l 10C2E
  \l 10C2F
  \l 10C30
  \l 10C31
  \l 10C32
  \l 10C33
  \l 10C34
  \l 10C35
  \l 10C36
  \l 10C37
  \l 10C38
  \l 10C39
  \l 10C3A
  \l 10C3B
  \l 10C3C
  \l 10C3D
  \l 10C3E
  \l 10C3F
  \l 10C40
  \l 10C41
  \l 10C42
  \l 10C43
  \l 10C44
  \l 10C45
  \l 10C46
  \l 10C47
  \l 10C48
  \l 11000
  \l 11001
  \l 11002
  \l 11003
  \l 11004
  \l 11005
  \l 11006
  \l 11007
  \l 11008
  \l 11009
  \l 1100A
  \l 1100B
  \l 1100C
  \l 1100D
  \l 1100E
  \l 1100F
  \l 11010
  \l 11011
  \l 11012
  \l 11013
  \l 11014
  \l 11015
  \l 11016
  \l 11017
  \l 11018
  \l 11019
  \l 1101A
  \l 1101B
  \l 1101C
  \l 1101D
  \l 1101E
  \l 1101F
  \l 11020
  \l 11021
  \l 11022
  \l 11023
  \l 11024
  \l 11025
  \l 11026
  \l 11027
  \l 11028
  \l 11029
  \l 1102A
  \l 1102B
  \l 1102C
  \l 1102D
  \l 1102E
  \l 1102F
  \l 11030
  \l 11031
  \l 11032
  \l 11033
  \l 11034
  \l 11035
  \l 11036
  \l 11037
  \l 11038
  \l 11039
  \l 1103A
  \l 1103B
  \l 1103C
  \l 1103D
  \l 1103E
  \l 1103F
  \l 11040
  \l 11041
  \l 11042
  \l 11043
  \l 11044
  \l 11045
  \l 11046
  \l 1107F
  \l 11080
  \l 11081
  \l 11082
  \l 11083
  \l 11084
  \l 11085
  \l 11086
  \l 11087
  \l 11088
  \l 11089
  \l 1108A
  \l 1108B
  \l 1108C
  \l 1108D
  \l 1108E
  \l 1108F
  \l 11090
  \l 11091
  \l 11092
  \l 11093
  \l 11094
  \l 11095
  \l 11096
  \l 11097
  \l 11098
  \l 11099
  \l 1109A
  \l 1109B
  \l 1109C
  \l 1109D
  \l 1109E
  \l 1109F
  \l 110A0
  \l 110A1
  \l 110A2
  \l 110A3
  \l 110A4
  \l 110A5
  \l 110A6
  \l 110A7
  \l 110A8
  \l 110A9
  \l 110AA
  \l 110AB
  \l 110AC
  \l 110AD
  \l 110AE
  \l 110AF
  \l 110B0
  \l 110B1
  \l 110B2
  \l 110B3
  \l 110B4
  \l 110B5
  \l 110B6
  \l 110B7
  \l 110B8
  \l 110B9
  \l 110BA
  \l 110D0
  \l 110D1
  \l 110D2
  \l 110D3
  \l 110D4
  \l 110D5
  \l 110D6
  \l 110D7
  \l 110D8
  \l 110D9
  \l 110DA
  \l 110DB
  \l 110DC
  \l 110DD
  \l 110DE
  \l 110DF
  \l 110E0
  \l 110E1
  \l 110E2
  \l 110E3
  \l 110E4
  \l 110E5
  \l 110E6
  \l 110E7
  \l 110E8
  \l 11100
  \l 11101
  \l 11102
  \l 11103
  \l 11104
  \l 11105
  \l 11106
  \l 11107
  \l 11108
  \l 11109
  \l 1110A
  \l 1110B
  \l 1110C
  \l 1110D
  \l 1110E
  \l 1110F
  \l 11110
  \l 11111
  \l 11112
  \l 11113
  \l 11114
  \l 11115
  \l 11116
  \l 11117
  \l 11118
  \l 11119
  \l 1111A
  \l 1111B
  \l 1111C
  \l 1111D
  \l 1111E
  \l 1111F
  \l 11120
  \l 11121
  \l 11122
  \l 11123
  \l 11124
  \l 11125
  \l 11126
  \l 11127
  \l 11128
  \l 11129
  \l 1112A
  \l 1112B
  \l 1112C
  \l 1112D
  \l 1112E
  \l 1112F
  \l 11130
  \l 11131
  \l 11132
  \l 11133
  \l 11134
  \l 11150
  \l 11151
  \l 11152
  \l 11153
  \l 11154
  \l 11155
  \l 11156
  \l 11157
  \l 11158
  \l 11159
  \l 1115A
  \l 1115B
  \l 1115C
  \l 1115D
  \l 1115E
  \l 1115F
  \l 11160
  \l 11161
  \l 11162
  \l 11163
  \l 11164
  \l 11165
  \l 11166
  \l 11167
  \l 11168
  \l 11169
  \l 1116A
  \l 1116B
  \l 1116C
  \l 1116D
  \l 1116E
  \l 1116F
  \l 11170
  \l 11171
  \l 11172
  \l 11173
  \l 11176
  \l 11180
  \l 11181
  \l 11182
  \l 11183
  \l 11184
  \l 11185
  \l 11186
  \l 11187
  \l 11188
  \l 11189
  \l 1118A
  \l 1118B
  \l 1118C
  \l 1118D
  \l 1118E
  \l 1118F
  \l 11190
  \l 11191
  \l 11192
  \l 11193
  \l 11194
  \l 11195
  \l 11196
  \l 11197
  \l 11198
  \l 11199
  \l 1119A
  \l 1119B
  \l 1119C
  \l 1119D
  \l 1119E
  \l 1119F
  \l 111A0
  \l 111A1
  \l 111A2
  \l 111A3
  \l 111A4
  \l 111A5
  \l 111A6
  \l 111A7
  \l 111A8
  \l 111A9
  \l 111AA
  \l 111AB
  \l 111AC
  \l 111AD
  \l 111AE
  \l 111AF
  \l 111B0
  \l 111B1
  \l 111B2
  \l 111B3
  \l 111B4
  \l 111B5
  \l 111B6
  \l 111B7
  \l 111B8
  \l 111B9
  \l 111BA
  \l 111BB
  \l 111BC
  \l 111BD
  \l 111BE
  \l 111BF
  \l 111C0
  \l 111C1
  \l 111C2
  \l 111C3
  \l 111C4
  \l 111DA
  \l 11200
  \l 11201
  \l 11202
  \l 11203
  \l 11204
  \l 11205
  \l 11206
  \l 11207
  \l 11208
  \l 11209
  \l 1120A
  \l 1120B
  \l 1120C
  \l 1120D
  \l 1120E
  \l 1120F
  \l 11210
  \l 11211
  \l 11213
  \l 11214
  \l 11215
  \l 11216
  \l 11217
  \l 11218
  \l 11219
  \l 1121A
  \l 1121B
  \l 1121C
  \l 1121D
  \l 1121E
  \l 1121F
  \l 11220
  \l 11221
  \l 11222
  \l 11223
  \l 11224
  \l 11225
  \l 11226
  \l 11227
  \l 11228
  \l 11229
  \l 1122A
  \l 1122B
  \l 1122C
  \l 1122D
  \l 1122E
  \l 1122F
  \l 11230
  \l 11231
  \l 11232
  \l 11233
  \l 11234
  \l 11235
  \l 11236
  \l 11237
  \l 112B0
  \l 112B1
  \l 112B2
  \l 112B3
  \l 112B4
  \l 112B5
  \l 112B6
  \l 112B7
  \l 112B8
  \l 112B9
  \l 112BA
  \l 112BB
  \l 112BC
  \l 112BD
  \l 112BE
  \l 112BF
  \l 112C0
  \l 112C1
  \l 112C2
  \l 112C3
  \l 112C4
  \l 112C5
  \l 112C6
  \l 112C7
  \l 112C8
  \l 112C9
  \l 112CA
  \l 112CB
  \l 112CC
  \l 112CD
  \l 112CE
  \l 112CF
  \l 112D0
  \l 112D1
  \l 112D2
  \l 112D3
  \l 112D4
  \l 112D5
  \l 112D6
  \l 112D7
  \l 112D8
  \l 112D9
  \l 112DA
  \l 112DB
  \l 112DC
  \l 112DD
  \l 112DE
  \l 112DF
  \l 112E0
  \l 112E1
  \l 112E2
  \l 112E3
  \l 112E4
  \l 112E5
  \l 112E6
  \l 112E7
  \l 112E8
  \l 112E9
  \l 112EA
  \l 11301
  \l 11302
  \l 11303
  \l 11305
  \l 11306
  \l 11307
  \l 11308
  \l 11309
  \l 1130A
  \l 1130B
  \l 1130C
  \l 1130F
  \l 11310
  \l 11313
  \l 11314
  \l 11315
  \l 11316
  \l 11317
  \l 11318
  \l 11319
  \l 1131A
  \l 1131B
  \l 1131C
  \l 1131D
  \l 1131E
  \l 1131F
  \l 11320
  \l 11321
  \l 11322
  \l 11323
  \l 11324
  \l 11325
  \l 11326
  \l 11327
  \l 11328
  \l 1132A
  \l 1132B
  \l 1132C
  \l 1132D
  \l 1132E
  \l 1132F
  \l 11330
  \l 11332
  \l 11333
  \l 11335
  \l 11336
  \l 11337
  \l 11338
  \l 11339
  \l 1133C
  \l 1133D
  \l 1133E
  \l 1133F
  \l 11340
  \l 11341
  \l 11342
  \l 11343
  \l 11344
  \l 11347
  \l 11348
  \l 1134B
  \l 1134C
  \l 1134D
  \l 11357
  \l 1135D
  \l 1135E
  \l 1135F
  \l 11360
  \l 11361
  \l 11362
  \l 11363
  \l 11366
  \l 11367
  \l 11368
  \l 11369
  \l 1136A
  \l 1136B
  \l 1136C
  \l 11370
  \l 11371
  \l 11372
  \l 11373
  \l 11374
  \l 11480
  \l 11481
  \l 11482
  \l 11483
  \l 11484
  \l 11485
  \l 11486
  \l 11487
  \l 11488
  \l 11489
  \l 1148A
  \l 1148B
  \l 1148C
  \l 1148D
  \l 1148E
  \l 1148F
  \l 11490
  \l 11491
  \l 11492
  \l 11493
  \l 11494
  \l 11495
  \l 11496
  \l 11497
  \l 11498
  \l 11499
  \l 1149A
  \l 1149B
  \l 1149C
  \l 1149D
  \l 1149E
  \l 1149F
  \l 114A0
  \l 114A1
  \l 114A2
  \l 114A3
  \l 114A4
  \l 114A5
  \l 114A6
  \l 114A7
  \l 114A8
  \l 114A9
  \l 114AA
  \l 114AB
  \l 114AC
  \l 114AD
  \l 114AE
  \l 114AF
  \l 114B0
  \l 114B1
  \l 114B2
  \l 114B3
  \l 114B4
  \l 114B5
  \l 114B6
  \l 114B7
  \l 114B8
  \l 114B9
  \l 114BA
  \l 114BB
  \l 114BC
  \l 114BD
  \l 114BE
  \l 114BF
  \l 114C0
  \l 114C1
  \l 114C2
  \l 114C3
  \l 114C4
  \l 114C5
  \l 114C7
  \l 11580
  \l 11581
  \l 11582
  \l 11583
  \l 11584
  \l 11585
  \l 11586
  \l 11587
  \l 11588
  \l 11589
  \l 1158A
  \l 1158B
  \l 1158C
  \l 1158D
  \l 1158E
  \l 1158F
  \l 11590
  \l 11591
  \l 11592
  \l 11593
  \l 11594
  \l 11595
  \l 11596
  \l 11597
  \l 11598
  \l 11599
  \l 1159A
  \l 1159B
  \l 1159C
  \l 1159D
  \l 1159E
  \l 1159F
  \l 115A0
  \l 115A1
  \l 115A2
  \l 115A3
  \l 115A4
  \l 115A5
  \l 115A6
  \l 115A7
  \l 115A8
  \l 115A9
  \l 115AA
  \l 115AB
  \l 115AC
  \l 115AD
  \l 115AE
  \l 115AF
  \l 115B0
  \l 115B1
  \l 115B2
  \l 115B3
  \l 115B4
  \l 115B5
  \l 115B8
  \l 115B9
  \l 115BA
  \l 115BB
  \l 115BC
  \l 115BD
  \l 115BE
  \l 115BF
  \l 115C0
  \l 11600
  \l 11601
  \l 11602
  \l 11603
  \l 11604
  \l 11605
  \l 11606
  \l 11607
  \l 11608
  \l 11609
  \l 1160A
  \l 1160B
  \l 1160C
  \l 1160D
  \l 1160E
  \l 1160F
  \l 11610
  \l 11611
  \l 11612
  \l 11613
  \l 11614
  \l 11615
  \l 11616
  \l 11617
  \l 11618
  \l 11619
  \l 1161A
  \l 1161B
  \l 1161C
  \l 1161D
  \l 1161E
  \l 1161F
  \l 11620
  \l 11621
  \l 11622
  \l 11623
  \l 11624
  \l 11625
  \l 11626
  \l 11627
  \l 11628
  \l 11629
  \l 1162A
  \l 1162B
  \l 1162C
  \l 1162D
  \l 1162E
  \l 1162F
  \l 11630
  \l 11631
  \l 11632
  \l 11633
  \l 11634
  \l 11635
  \l 11636
  \l 11637
  \l 11638
  \l 11639
  \l 1163A
  \l 1163B
  \l 1163C
  \l 1163D
  \l 1163E
  \l 1163F
  \l 11640
  \l 11644
  \l 11680
  \l 11681
  \l 11682
  \l 11683
  \l 11684
  \l 11685
  \l 11686
  \l 11687
  \l 11688
  \l 11689
  \l 1168A
  \l 1168B
  \l 1168C
  \l 1168D
  \l 1168E
  \l 1168F
  \l 11690
  \l 11691
  \l 11692
  \l 11693
  \l 11694
  \l 11695
  \l 11696
  \l 11697
  \l 11698
  \l 11699
  \l 1169A
  \l 1169B
  \l 1169C
  \l 1169D
  \l 1169E
  \l 1169F
  \l 116A0
  \l 116A1
  \l 116A2
  \l 116A3
  \l 116A4
  \l 116A5
  \l 116A6
  \l 116A7
  \l 116A8
  \l 116A9
  \l 116AA
  \l 116AB
  \l 116AC
  \l 116AD
  \l 116AE
  \l 116AF
  \l 116B0
  \l 116B1
  \l 116B2
  \l 116B3
  \l 116B4
  \l 116B5
  \l 116B6
  \l 116B7
  \L 118A0 118A0 118C0
  \L 118A1 118A1 118C1
  \L 118A2 118A2 118C2
  \L 118A3 118A3 118C3
  \L 118A4 118A4 118C4
  \L 118A5 118A5 118C5
  \L 118A6 118A6 118C6
  \L 118A7 118A7 118C7
  \L 118A8 118A8 118C8
  \L 118A9 118A9 118C9
  \L 118AA 118AA 118CA
  \L 118AB 118AB 118CB
  \L 118AC 118AC 118CC
  \L 118AD 118AD 118CD
  \L 118AE 118AE 118CE
  \L 118AF 118AF 118CF
  \L 118B0 118B0 118D0
  \L 118B1 118B1 118D1
  \L 118B2 118B2 118D2
  \L 118B3 118B3 118D3
  \L 118B4 118B4 118D4
  \L 118B5 118B5 118D5
  \L 118B6 118B6 118D6
  \L 118B7 118B7 118D7
  \L 118B8 118B8 118D8
  \L 118B9 118B9 118D9
  \L 118BA 118BA 118DA
  \L 118BB 118BB 118DB
  \L 118BC 118BC 118DC
  \L 118BD 118BD 118DD
  \L 118BE 118BE 118DE
  \L 118BF 118BF 118DF
  \L 118C0 118A0 118C0
  \L 118C1 118A1 118C1
  \L 118C2 118A2 118C2
  \L 118C3 118A3 118C3
  \L 118C4 118A4 118C4
  \L 118C5 118A5 118C5
  \L 118C6 118A6 118C6
  \L 118C7 118A7 118C7
  \L 118C8 118A8 118C8
  \L 118C9 118A9 118C9
  \L 118CA 118AA 118CA
  \L 118CB 118AB 118CB
  \L 118CC 118AC 118CC
  \L 118CD 118AD 118CD
  \L 118CE 118AE 118CE
  \L 118CF 118AF 118CF
  \L 118D0 118B0 118D0
  \L 118D1 118B1 118D1
  \L 118D2 118B2 118D2
  \L 118D3 118B3 118D3
  \L 118D4 118B4 118D4
  \L 118D5 118B5 118D5
  \L 118D6 118B6 118D6
  \L 118D7 118B7 118D7
  \L 118D8 118B8 118D8
  \L 118D9 118B9 118D9
  \L 118DA 118BA 118DA
  \L 118DB 118BB 118DB
  \L 118DC 118BC 118DC
  \L 118DD 118BD 118DD
  \L 118DE 118BE 118DE
  \L 118DF 118BF 118DF
  \l 118FF
  \l 11AC0
  \l 11AC1
  \l 11AC2
  \l 11AC3
  \l 11AC4
  \l 11AC5
  \l 11AC6
  \l 11AC7
  \l 11AC8
  \l 11AC9
  \l 11ACA
  \l 11ACB
  \l 11ACC
  \l 11ACD
  \l 11ACE
  \l 11ACF
  \l 11AD0
  \l 11AD1
  \l 11AD2
  \l 11AD3
  \l 11AD4
  \l 11AD5
  \l 11AD6
  \l 11AD7
  \l 11AD8
  \l 11AD9
  \l 11ADA
  \l 11ADB
  \l 11ADC
  \l 11ADD
  \l 11ADE
  \l 11ADF
  \l 11AE0
  \l 11AE1
  \l 11AE2
  \l 11AE3
  \l 11AE4
  \l 11AE5
  \l 11AE6
  \l 11AE7
  \l 11AE8
  \l 11AE9
  \l 11AEA
  \l 11AEB
  \l 11AEC
  \l 11AED
  \l 11AEE
  \l 11AEF
  \l 11AF0
  \l 11AF1
  \l 11AF2
  \l 11AF3
  \l 11AF4
  \l 11AF5
  \l 11AF6
  \l 11AF7
  \l 11AF8
  \l 12000
  \l 12001
  \l 12002
  \l 12003
  \l 12004
  \l 12005
  \l 12006
  \l 12007
  \l 12008
  \l 12009
  \l 1200A
  \l 1200B
  \l 1200C
  \l 1200D
  \l 1200E
  \l 1200F
  \l 12010
  \l 12011
  \l 12012
  \l 12013
  \l 12014
  \l 12015
  \l 12016
  \l 12017
  \l 12018
  \l 12019
  \l 1201A
  \l 1201B
  \l 1201C
  \l 1201D
  \l 1201E
  \l 1201F
  \l 12020
  \l 12021
  \l 12022
  \l 12023
  \l 12024
  \l 12025
  \l 12026
  \l 12027
  \l 12028
  \l 12029
  \l 1202A
  \l 1202B
  \l 1202C
  \l 1202D
  \l 1202E
  \l 1202F
  \l 12030
  \l 12031
  \l 12032
  \l 12033
  \l 12034
  \l 12035
  \l 12036
  \l 12037
  \l 12038
  \l 12039
  \l 1203A
  \l 1203B
  \l 1203C
  \l 1203D
  \l 1203E
  \l 1203F
  \l 12040
  \l 12041
  \l 12042
  \l 12043
  \l 12044
  \l 12045
  \l 12046
  \l 12047
  \l 12048
  \l 12049
  \l 1204A
  \l 1204B
  \l 1204C
  \l 1204D
  \l 1204E
  \l 1204F
  \l 12050
  \l 12051
  \l 12052
  \l 12053
  \l 12054
  \l 12055
  \l 12056
  \l 12057
  \l 12058
  \l 12059
  \l 1205A
  \l 1205B
  \l 1205C
  \l 1205D
  \l 1205E
  \l 1205F
  \l 12060
  \l 12061
  \l 12062
  \l 12063
  \l 12064
  \l 12065
  \l 12066
  \l 12067
  \l 12068
  \l 12069
  \l 1206A
  \l 1206B
  \l 1206C
  \l 1206D
  \l 1206E
  \l 1206F
  \l 12070
  \l 12071
  \l 12072
  \l 12073
  \l 12074
  \l 12075
  \l 12076
  \l 12077
  \l 12078
  \l 12079
  \l 1207A
  \l 1207B
  \l 1207C
  \l 1207D
  \l 1207E
  \l 1207F
  \l 12080
  \l 12081
  \l 12082
  \l 12083
  \l 12084
  \l 12085
  \l 12086
  \l 12087
  \l 12088
  \l 12089
  \l 1208A
  \l 1208B
  \l 1208C
  \l 1208D
  \l 1208E
  \l 1208F
  \l 12090
  \l 12091
  \l 12092
  \l 12093
  \l 12094
  \l 12095
  \l 12096
  \l 12097
  \l 12098
  \l 12099
  \l 1209A
  \l 1209B
  \l 1209C
  \l 1209D
  \l 1209E
  \l 1209F
  \l 120A0
  \l 120A1
  \l 120A2
  \l 120A3
  \l 120A4
  \l 120A5
  \l 120A6
  \l 120A7
  \l 120A8
  \l 120A9
  \l 120AA
  \l 120AB
  \l 120AC
  \l 120AD
  \l 120AE
  \l 120AF
  \l 120B0
  \l 120B1
  \l 120B2
  \l 120B3
  \l 120B4
  \l 120B5
  \l 120B6
  \l 120B7
  \l 120B8
  \l 120B9
  \l 120BA
  \l 120BB
  \l 120BC
  \l 120BD
  \l 120BE
  \l 120BF
  \l 120C0
  \l 120C1
  \l 120C2
  \l 120C3
  \l 120C4
  \l 120C5
  \l 120C6
  \l 120C7
  \l 120C8
  \l 120C9
  \l 120CA
  \l 120CB
  \l 120CC
  \l 120CD
  \l 120CE
  \l 120CF
  \l 120D0
  \l 120D1
  \l 120D2
  \l 120D3
  \l 120D4
  \l 120D5
  \l 120D6
  \l 120D7
  \l 120D8
  \l 120D9
  \l 120DA
  \l 120DB
  \l 120DC
  \l 120DD
  \l 120DE
  \l 120DF
  \l 120E0
  \l 120E1
  \l 120E2
  \l 120E3
  \l 120E4
  \l 120E5
  \l 120E6
  \l 120E7
  \l 120E8
  \l 120E9
  \l 120EA
  \l 120EB
  \l 120EC
  \l 120ED
  \l 120EE
  \l 120EF
  \l 120F0
  \l 120F1
  \l 120F2
  \l 120F3
  \l 120F4
  \l 120F5
  \l 120F6
  \l 120F7
  \l 120F8
  \l 120F9
  \l 120FA
  \l 120FB
  \l 120FC
  \l 120FD
  \l 120FE
  \l 120FF
  \l 12100
  \l 12101
  \l 12102
  \l 12103
  \l 12104
  \l 12105
  \l 12106
  \l 12107
  \l 12108
  \l 12109
  \l 1210A
  \l 1210B
  \l 1210C
  \l 1210D
  \l 1210E
  \l 1210F
  \l 12110
  \l 12111
  \l 12112
  \l 12113
  \l 12114
  \l 12115
  \l 12116
  \l 12117
  \l 12118
  \l 12119
  \l 1211A
  \l 1211B
  \l 1211C
  \l 1211D
  \l 1211E
  \l 1211F
  \l 12120
  \l 12121
  \l 12122
  \l 12123
  \l 12124
  \l 12125
  \l 12126
  \l 12127
  \l 12128
  \l 12129
  \l 1212A
  \l 1212B
  \l 1212C
  \l 1212D
  \l 1212E
  \l 1212F
  \l 12130
  \l 12131
  \l 12132
  \l 12133
  \l 12134
  \l 12135
  \l 12136
  \l 12137
  \l 12138
  \l 12139
  \l 1213A
  \l 1213B
  \l 1213C
  \l 1213D
  \l 1213E
  \l 1213F
  \l 12140
  \l 12141
  \l 12142
  \l 12143
  \l 12144
  \l 12145
  \l 12146
  \l 12147
  \l 12148
  \l 12149
  \l 1214A
  \l 1214B
  \l 1214C
  \l 1214D
  \l 1214E
  \l 1214F
  \l 12150
  \l 12151
  \l 12152
  \l 12153
  \l 12154
  \l 12155
  \l 12156
  \l 12157
  \l 12158
  \l 12159
  \l 1215A
  \l 1215B
  \l 1215C
  \l 1215D
  \l 1215E
  \l 1215F
  \l 12160
  \l 12161
  \l 12162
  \l 12163
  \l 12164
  \l 12165
  \l 12166
  \l 12167
  \l 12168
  \l 12169
  \l 1216A
  \l 1216B
  \l 1216C
  \l 1216D
  \l 1216E
  \l 1216F
  \l 12170
  \l 12171
  \l 12172
  \l 12173
  \l 12174
  \l 12175
  \l 12176
  \l 12177
  \l 12178
  \l 12179
  \l 1217A
  \l 1217B
  \l 1217C
  \l 1217D
  \l 1217E
  \l 1217F
  \l 12180
  \l 12181
  \l 12182
  \l 12183
  \l 12184
  \l 12185
  \l 12186
  \l 12187
  \l 12188
  \l 12189
  \l 1218A
  \l 1218B
  \l 1218C
  \l 1218D
  \l 1218E
  \l 1218F
  \l 12190
  \l 12191
  \l 12192
  \l 12193
  \l 12194
  \l 12195
  \l 12196
  \l 12197
  \l 12198
  \l 12199
  \l 1219A
  \l 1219B
  \l 1219C
  \l 1219D
  \l 1219E
  \l 1219F
  \l 121A0
  \l 121A1
  \l 121A2
  \l 121A3
  \l 121A4
  \l 121A5
  \l 121A6
  \l 121A7
  \l 121A8
  \l 121A9
  \l 121AA
  \l 121AB
  \l 121AC
  \l 121AD
  \l 121AE
  \l 121AF
  \l 121B0
  \l 121B1
  \l 121B2
  \l 121B3
  \l 121B4
  \l 121B5
  \l 121B6
  \l 121B7
  \l 121B8
  \l 121B9
  \l 121BA
  \l 121BB
  \l 121BC
  \l 121BD
  \l 121BE
  \l 121BF
  \l 121C0
  \l 121C1
  \l 121C2
  \l 121C3
  \l 121C4
  \l 121C5
  \l 121C6
  \l 121C7
  \l 121C8
  \l 121C9
  \l 121CA
  \l 121CB
  \l 121CC
  \l 121CD
  \l 121CE
  \l 121CF
  \l 121D0
  \l 121D1
  \l 121D2
  \l 121D3
  \l 121D4
  \l 121D5
  \l 121D6
  \l 121D7
  \l 121D8
  \l 121D9
  \l 121DA
  \l 121DB
  \l 121DC
  \l 121DD
  \l 121DE
  \l 121DF
  \l 121E0
  \l 121E1
  \l 121E2
  \l 121E3
  \l 121E4
  \l 121E5
  \l 121E6
  \l 121E7
  \l 121E8
  \l 121E9
  \l 121EA
  \l 121EB
  \l 121EC
  \l 121ED
  \l 121EE
  \l 121EF
  \l 121F0
  \l 121F1
  \l 121F2
  \l 121F3
  \l 121F4
  \l 121F5
  \l 121F6
  \l 121F7
  \l 121F8
  \l 121F9
  \l 121FA
  \l 121FB
  \l 121FC
  \l 121FD
  \l 121FE
  \l 121FF
  \l 12200
  \l 12201
  \l 12202
  \l 12203
  \l 12204
  \l 12205
  \l 12206
  \l 12207
  \l 12208
  \l 12209
  \l 1220A
  \l 1220B
  \l 1220C
  \l 1220D
  \l 1220E
  \l 1220F
  \l 12210
  \l 12211
  \l 12212
  \l 12213
  \l 12214
  \l 12215
  \l 12216
  \l 12217
  \l 12218
  \l 12219
  \l 1221A
  \l 1221B
  \l 1221C
  \l 1221D
  \l 1221E
  \l 1221F
  \l 12220
  \l 12221
  \l 12222
  \l 12223
  \l 12224
  \l 12225
  \l 12226
  \l 12227
  \l 12228
  \l 12229
  \l 1222A
  \l 1222B
  \l 1222C
  \l 1222D
  \l 1222E
  \l 1222F
  \l 12230
  \l 12231
  \l 12232
  \l 12233
  \l 12234
  \l 12235
  \l 12236
  \l 12237
  \l 12238
  \l 12239
  \l 1223A
  \l 1223B
  \l 1223C
  \l 1223D
  \l 1223E
  \l 1223F
  \l 12240
  \l 12241
  \l 12242
  \l 12243
  \l 12244
  \l 12245
  \l 12246
  \l 12247
  \l 12248
  \l 12249
  \l 1224A
  \l 1224B
  \l 1224C
  \l 1224D
  \l 1224E
  \l 1224F
  \l 12250
  \l 12251
  \l 12252
  \l 12253
  \l 12254
  \l 12255
  \l 12256
  \l 12257
  \l 12258
  \l 12259
  \l 1225A
  \l 1225B
  \l 1225C
  \l 1225D
  \l 1225E
  \l 1225F
  \l 12260
  \l 12261
  \l 12262
  \l 12263
  \l 12264
  \l 12265
  \l 12266
  \l 12267
  \l 12268
  \l 12269
  \l 1226A
  \l 1226B
  \l 1226C
  \l 1226D
  \l 1226E
  \l 1226F
  \l 12270
  \l 12271
  \l 12272
  \l 12273
  \l 12274
  \l 12275
  \l 12276
  \l 12277
  \l 12278
  \l 12279
  \l 1227A
  \l 1227B
  \l 1227C
  \l 1227D
  \l 1227E
  \l 1227F
  \l 12280
  \l 12281
  \l 12282
  \l 12283
  \l 12284
  \l 12285
  \l 12286
  \l 12287
  \l 12288
  \l 12289
  \l 1228A
  \l 1228B
  \l 1228C
  \l 1228D
  \l 1228E
  \l 1228F
  \l 12290
  \l 12291
  \l 12292
  \l 12293
  \l 12294
  \l 12295
  \l 12296
  \l 12297
  \l 12298
  \l 12299
  \l 1229A
  \l 1229B
  \l 1229C
  \l 1229D
  \l 1229E
  \l 1229F
  \l 122A0
  \l 122A1
  \l 122A2
  \l 122A3
  \l 122A4
  \l 122A5
  \l 122A6
  \l 122A7
  \l 122A8
  \l 122A9
  \l 122AA
  \l 122AB
  \l 122AC
  \l 122AD
  \l 122AE
  \l 122AF
  \l 122B0
  \l 122B1
  \l 122B2
  \l 122B3
  \l 122B4
  \l 122B5
  \l 122B6
  \l 122B7
  \l 122B8
  \l 122B9
  \l 122BA
  \l 122BB
  \l 122BC
  \l 122BD
  \l 122BE
  \l 122BF
  \l 122C0
  \l 122C1
  \l 122C2
  \l 122C3
  \l 122C4
  \l 122C5
  \l 122C6
  \l 122C7
  \l 122C8
  \l 122C9
  \l 122CA
  \l 122CB
  \l 122CC
  \l 122CD
  \l 122CE
  \l 122CF
  \l 122D0
  \l 122D1
  \l 122D2
  \l 122D3
  \l 122D4
  \l 122D5
  \l 122D6
  \l 122D7
  \l 122D8
  \l 122D9
  \l 122DA
  \l 122DB
  \l 122DC
  \l 122DD
  \l 122DE
  \l 122DF
  \l 122E0
  \l 122E1
  \l 122E2
  \l 122E3
  \l 122E4
  \l 122E5
  \l 122E6
  \l 122E7
  \l 122E8
  \l 122E9
  \l 122EA
  \l 122EB
  \l 122EC
  \l 122ED
  \l 122EE
  \l 122EF
  \l 122F0
  \l 122F1
  \l 122F2
  \l 122F3
  \l 122F4
  \l 122F5
  \l 122F6
  \l 122F7
  \l 122F8
  \l 122F9
  \l 122FA
  \l 122FB
  \l 122FC
  \l 122FD
  \l 122FE
  \l 122FF
  \l 12300
  \l 12301
  \l 12302
  \l 12303
  \l 12304
  \l 12305
  \l 12306
  \l 12307
  \l 12308
  \l 12309
  \l 1230A
  \l 1230B
  \l 1230C
  \l 1230D
  \l 1230E
  \l 1230F
  \l 12310
  \l 12311
  \l 12312
  \l 12313
  \l 12314
  \l 12315
  \l 12316
  \l 12317
  \l 12318
  \l 12319
  \l 1231A
  \l 1231B
  \l 1231C
  \l 1231D
  \l 1231E
  \l 1231F
  \l 12320
  \l 12321
  \l 12322
  \l 12323
  \l 12324
  \l 12325
  \l 12326
  \l 12327
  \l 12328
  \l 12329
  \l 1232A
  \l 1232B
  \l 1232C
  \l 1232D
  \l 1232E
  \l 1232F
  \l 12330
  \l 12331
  \l 12332
  \l 12333
  \l 12334
  \l 12335
  \l 12336
  \l 12337
  \l 12338
  \l 12339
  \l 1233A
  \l 1233B
  \l 1233C
  \l 1233D
  \l 1233E
  \l 1233F
  \l 12340
  \l 12341
  \l 12342
  \l 12343
  \l 12344
  \l 12345
  \l 12346
  \l 12347
  \l 12348
  \l 12349
  \l 1234A
  \l 1234B
  \l 1234C
  \l 1234D
  \l 1234E
  \l 1234F
  \l 12350
  \l 12351
  \l 12352
  \l 12353
  \l 12354
  \l 12355
  \l 12356
  \l 12357
  \l 12358
  \l 12359
  \l 1235A
  \l 1235B
  \l 1235C
  \l 1235D
  \l 1235E
  \l 1235F
  \l 12360
  \l 12361
  \l 12362
  \l 12363
  \l 12364
  \l 12365
  \l 12366
  \l 12367
  \l 12368
  \l 12369
  \l 1236A
  \l 1236B
  \l 1236C
  \l 1236D
  \l 1236E
  \l 1236F
  \l 12370
  \l 12371
  \l 12372
  \l 12373
  \l 12374
  \l 12375
  \l 12376
  \l 12377
  \l 12378
  \l 12379
  \l 1237A
  \l 1237B
  \l 1237C
  \l 1237D
  \l 1237E
  \l 1237F
  \l 12380
  \l 12381
  \l 12382
  \l 12383
  \l 12384
  \l 12385
  \l 12386
  \l 12387
  \l 12388
  \l 12389
  \l 1238A
  \l 1238B
  \l 1238C
  \l 1238D
  \l 1238E
  \l 1238F
  \l 12390
  \l 12391
  \l 12392
  \l 12393
  \l 12394
  \l 12395
  \l 12396
  \l 12397
  \l 12398
  \l 13000
  \l 13001
  \l 13002
  \l 13003
  \l 13004
  \l 13005
  \l 13006
  \l 13007
  \l 13008
  \l 13009
  \l 1300A
  \l 1300B
  \l 1300C
  \l 1300D
  \l 1300E
  \l 1300F
  \l 13010
  \l 13011
  \l 13012
  \l 13013
  \l 13014
  \l 13015
  \l 13016
  \l 13017
  \l 13018
  \l 13019
  \l 1301A
  \l 1301B
  \l 1301C
  \l 1301D
  \l 1301E
  \l 1301F
  \l 13020
  \l 13021
  \l 13022
  \l 13023
  \l 13024
  \l 13025
  \l 13026
  \l 13027
  \l 13028
  \l 13029
  \l 1302A
  \l 1302B
  \l 1302C
  \l 1302D
  \l 1302E
  \l 1302F
  \l 13030
  \l 13031
  \l 13032
  \l 13033
  \l 13034
  \l 13035
  \l 13036
  \l 13037
  \l 13038
  \l 13039
  \l 1303A
  \l 1303B
  \l 1303C
  \l 1303D
  \l 1303E
  \l 1303F
  \l 13040
  \l 13041
  \l 13042
  \l 13043
  \l 13044
  \l 13045
  \l 13046
  \l 13047
  \l 13048
  \l 13049
  \l 1304A
  \l 1304B
  \l 1304C
  \l 1304D
  \l 1304E
  \l 1304F
  \l 13050
  \l 13051
  \l 13052
  \l 13053
  \l 13054
  \l 13055
  \l 13056
  \l 13057
  \l 13058
  \l 13059
  \l 1305A
  \l 1305B
  \l 1305C
  \l 1305D
  \l 1305E
  \l 1305F
  \l 13060
  \l 13061
  \l 13062
  \l 13063
  \l 13064
  \l 13065
  \l 13066
  \l 13067
  \l 13068
  \l 13069
  \l 1306A
  \l 1306B
  \l 1306C
  \l 1306D
  \l 1306E
  \l 1306F
  \l 13070
  \l 13071
  \l 13072
  \l 13073
  \l 13074
  \l 13075
  \l 13076
  \l 13077
  \l 13078
  \l 13079
  \l 1307A
  \l 1307B
  \l 1307C
  \l 1307D
  \l 1307E
  \l 1307F
  \l 13080
  \l 13081
  \l 13082
  \l 13083
  \l 13084
  \l 13085
  \l 13086
  \l 13087
  \l 13088
  \l 13089
  \l 1308A
  \l 1308B
  \l 1308C
  \l 1308D
  \l 1308E
  \l 1308F
  \l 13090
  \l 13091
  \l 13092
  \l 13093
  \l 13094
  \l 13095
  \l 13096
  \l 13097
  \l 13098
  \l 13099
  \l 1309A
  \l 1309B
  \l 1309C
  \l 1309D
  \l 1309E
  \l 1309F
  \l 130A0
  \l 130A1
  \l 130A2
  \l 130A3
  \l 130A4
  \l 130A5
  \l 130A6
  \l 130A7
  \l 130A8
  \l 130A9
  \l 130AA
  \l 130AB
  \l 130AC
  \l 130AD
  \l 130AE
  \l 130AF
  \l 130B0
  \l 130B1
  \l 130B2
  \l 130B3
  \l 130B4
  \l 130B5
  \l 130B6
  \l 130B7
  \l 130B8
  \l 130B9
  \l 130BA
  \l 130BB
  \l 130BC
  \l 130BD
  \l 130BE
  \l 130BF
  \l 130C0
  \l 130C1
  \l 130C2
  \l 130C3
  \l 130C4
  \l 130C5
  \l 130C6
  \l 130C7
  \l 130C8
  \l 130C9
  \l 130CA
  \l 130CB
  \l 130CC
  \l 130CD
  \l 130CE
  \l 130CF
  \l 130D0
  \l 130D1
  \l 130D2
  \l 130D3
  \l 130D4
  \l 130D5
  \l 130D6
  \l 130D7
  \l 130D8
  \l 130D9
  \l 130DA
  \l 130DB
  \l 130DC
  \l 130DD
  \l 130DE
  \l 130DF
  \l 130E0
  \l 130E1
  \l 130E2
  \l 130E3
  \l 130E4
  \l 130E5
  \l 130E6
  \l 130E7
  \l 130E8
  \l 130E9
  \l 130EA
  \l 130EB
  \l 130EC
  \l 130ED
  \l 130EE
  \l 130EF
  \l 130F0
  \l 130F1
  \l 130F2
  \l 130F3
  \l 130F4
  \l 130F5
  \l 130F6
  \l 130F7
  \l 130F8
  \l 130F9
  \l 130FA
  \l 130FB
  \l 130FC
  \l 130FD
  \l 130FE
  \l 130FF
  \l 13100
  \l 13101
  \l 13102
  \l 13103
  \l 13104
  \l 13105
  \l 13106
  \l 13107
  \l 13108
  \l 13109
  \l 1310A
  \l 1310B
  \l 1310C
  \l 1310D
  \l 1310E
  \l 1310F
  \l 13110
  \l 13111
  \l 13112
  \l 13113
  \l 13114
  \l 13115
  \l 13116
  \l 13117
  \l 13118
  \l 13119
  \l 1311A
  \l 1311B
  \l 1311C
  \l 1311D
  \l 1311E
  \l 1311F
  \l 13120
  \l 13121
  \l 13122
  \l 13123
  \l 13124
  \l 13125
  \l 13126
  \l 13127
  \l 13128
  \l 13129
  \l 1312A
  \l 1312B
  \l 1312C
  \l 1312D
  \l 1312E
  \l 1312F
  \l 13130
  \l 13131
  \l 13132
  \l 13133
  \l 13134
  \l 13135
  \l 13136
  \l 13137
  \l 13138
  \l 13139
  \l 1313A
  \l 1313B
  \l 1313C
  \l 1313D
  \l 1313E
  \l 1313F
  \l 13140
  \l 13141
  \l 13142
  \l 13143
  \l 13144
  \l 13145
  \l 13146
  \l 13147
  \l 13148
  \l 13149
  \l 1314A
  \l 1314B
  \l 1314C
  \l 1314D
  \l 1314E
  \l 1314F
  \l 13150
  \l 13151
  \l 13152
  \l 13153
  \l 13154
  \l 13155
  \l 13156
  \l 13157
  \l 13158
  \l 13159
  \l 1315A
  \l 1315B
  \l 1315C
  \l 1315D
  \l 1315E
  \l 1315F
  \l 13160
  \l 13161
  \l 13162
  \l 13163
  \l 13164
  \l 13165
  \l 13166
  \l 13167
  \l 13168
  \l 13169
  \l 1316A
  \l 1316B
  \l 1316C
  \l 1316D
  \l 1316E
  \l 1316F
  \l 13170
  \l 13171
  \l 13172
  \l 13173
  \l 13174
  \l 13175
  \l 13176
  \l 13177
  \l 13178
  \l 13179
  \l 1317A
  \l 1317B
  \l 1317C
  \l 1317D
  \l 1317E
  \l 1317F
  \l 13180
  \l 13181
  \l 13182
  \l 13183
  \l 13184
  \l 13185
  \l 13186
  \l 13187
  \l 13188
  \l 13189
  \l 1318A
  \l 1318B
  \l 1318C
  \l 1318D
  \l 1318E
  \l 1318F
  \l 13190
  \l 13191
  \l 13192
  \l 13193
  \l 13194
  \l 13195
  \l 13196
  \l 13197
  \l 13198
  \l 13199
  \l 1319A
  \l 1319B
  \l 1319C
  \l 1319D
  \l 1319E
  \l 1319F
  \l 131A0
  \l 131A1
  \l 131A2
  \l 131A3
  \l 131A4
  \l 131A5
  \l 131A6
  \l 131A7
  \l 131A8
  \l 131A9
  \l 131AA
  \l 131AB
  \l 131AC
  \l 131AD
  \l 131AE
  \l 131AF
  \l 131B0
  \l 131B1
  \l 131B2
  \l 131B3
  \l 131B4
  \l 131B5
  \l 131B6
  \l 131B7
  \l 131B8
  \l 131B9
  \l 131BA
  \l 131BB
  \l 131BC
  \l 131BD
  \l 131BE
  \l 131BF
  \l 131C0
  \l 131C1
  \l 131C2
  \l 131C3
  \l 131C4
  \l 131C5
  \l 131C6
  \l 131C7
  \l 131C8
  \l 131C9
  \l 131CA
  \l 131CB
  \l 131CC
  \l 131CD
  \l 131CE
  \l 131CF
  \l 131D0
  \l 131D1
  \l 131D2
  \l 131D3
  \l 131D4
  \l 131D5
  \l 131D6
  \l 131D7
  \l 131D8
  \l 131D9
  \l 131DA
  \l 131DB
  \l 131DC
  \l 131DD
  \l 131DE
  \l 131DF
  \l 131E0
  \l 131E1
  \l 131E2
  \l 131E3
  \l 131E4
  \l 131E5
  \l 131E6
  \l 131E7
  \l 131E8
  \l 131E9
  \l 131EA
  \l 131EB
  \l 131EC
  \l 131ED
  \l 131EE
  \l 131EF
  \l 131F0
  \l 131F1
  \l 131F2
  \l 131F3
  \l 131F4
  \l 131F5
  \l 131F6
  \l 131F7
  \l 131F8
  \l 131F9
  \l 131FA
  \l 131FB
  \l 131FC
  \l 131FD
  \l 131FE
  \l 131FF
  \l 13200
  \l 13201
  \l 13202
  \l 13203
  \l 13204
  \l 13205
  \l 13206
  \l 13207
  \l 13208
  \l 13209
  \l 1320A
  \l 1320B
  \l 1320C
  \l 1320D
  \l 1320E
  \l 1320F
  \l 13210
  \l 13211
  \l 13212
  \l 13213
  \l 13214
  \l 13215
  \l 13216
  \l 13217
  \l 13218
  \l 13219
  \l 1321A
  \l 1321B
  \l 1321C
  \l 1321D
  \l 1321E
  \l 1321F
  \l 13220
  \l 13221
  \l 13222
  \l 13223
  \l 13224
  \l 13225
  \l 13226
  \l 13227
  \l 13228
  \l 13229
  \l 1322A
  \l 1322B
  \l 1322C
  \l 1322D
  \l 1322E
  \l 1322F
  \l 13230
  \l 13231
  \l 13232
  \l 13233
  \l 13234
  \l 13235
  \l 13236
  \l 13237
  \l 13238
  \l 13239
  \l 1323A
  \l 1323B
  \l 1323C
  \l 1323D
  \l 1323E
  \l 1323F
  \l 13240
  \l 13241
  \l 13242
  \l 13243
  \l 13244
  \l 13245
  \l 13246
  \l 13247
  \l 13248
  \l 13249
  \l 1324A
  \l 1324B
  \l 1324C
  \l 1324D
  \l 1324E
  \l 1324F
  \l 13250
  \l 13251
  \l 13252
  \l 13253
  \l 13254
  \l 13255
  \l 13256
  \l 13257
  \l 13258
  \l 13259
  \l 1325A
  \l 1325B
  \l 1325C
  \l 1325D
  \l 1325E
  \l 1325F
  \l 13260
  \l 13261
  \l 13262
  \l 13263
  \l 13264
  \l 13265
  \l 13266
  \l 13267
  \l 13268
  \l 13269
  \l 1326A
  \l 1326B
  \l 1326C
  \l 1326D
  \l 1326E
  \l 1326F
  \l 13270
  \l 13271
  \l 13272
  \l 13273
  \l 13274
  \l 13275
  \l 13276
  \l 13277
  \l 13278
  \l 13279
  \l 1327A
  \l 1327B
  \l 1327C
  \l 1327D
  \l 1327E
  \l 1327F
  \l 13280
  \l 13281
  \l 13282
  \l 13283
  \l 13284
  \l 13285
  \l 13286
  \l 13287
  \l 13288
  \l 13289
  \l 1328A
  \l 1328B
  \l 1328C
  \l 1328D
  \l 1328E
  \l 1328F
  \l 13290
  \l 13291
  \l 13292
  \l 13293
  \l 13294
  \l 13295
  \l 13296
  \l 13297
  \l 13298
  \l 13299
  \l 1329A
  \l 1329B
  \l 1329C
  \l 1329D
  \l 1329E
  \l 1329F
  \l 132A0
  \l 132A1
  \l 132A2
  \l 132A3
  \l 132A4
  \l 132A5
  \l 132A6
  \l 132A7
  \l 132A8
  \l 132A9
  \l 132AA
  \l 132AB
  \l 132AC
  \l 132AD
  \l 132AE
  \l 132AF
  \l 132B0
  \l 132B1
  \l 132B2
  \l 132B3
  \l 132B4
  \l 132B5
  \l 132B6
  \l 132B7
  \l 132B8
  \l 132B9
  \l 132BA
  \l 132BB
  \l 132BC
  \l 132BD
  \l 132BE
  \l 132BF
  \l 132C0
  \l 132C1
  \l 132C2
  \l 132C3
  \l 132C4
  \l 132C5
  \l 132C6
  \l 132C7
  \l 132C8
  \l 132C9
  \l 132CA
  \l 132CB
  \l 132CC
  \l 132CD
  \l 132CE
  \l 132CF
  \l 132D0
  \l 132D1
  \l 132D2
  \l 132D3
  \l 132D4
  \l 132D5
  \l 132D6
  \l 132D7
  \l 132D8
  \l 132D9
  \l 132DA
  \l 132DB
  \l 132DC
  \l 132DD
  \l 132DE
  \l 132DF
  \l 132E0
  \l 132E1
  \l 132E2
  \l 132E3
  \l 132E4
  \l 132E5
  \l 132E6
  \l 132E7
  \l 132E8
  \l 132E9
  \l 132EA
  \l 132EB
  \l 132EC
  \l 132ED
  \l 132EE
  \l 132EF
  \l 132F0
  \l 132F1
  \l 132F2
  \l 132F3
  \l 132F4
  \l 132F5
  \l 132F6
  \l 132F7
  \l 132F8
  \l 132F9
  \l 132FA
  \l 132FB
  \l 132FC
  \l 132FD
  \l 132FE
  \l 132FF
  \l 13300
  \l 13301
  \l 13302
  \l 13303
  \l 13304
  \l 13305
  \l 13306
  \l 13307
  \l 13308
  \l 13309
  \l 1330A
  \l 1330B
  \l 1330C
  \l 1330D
  \l 1330E
  \l 1330F
  \l 13310
  \l 13311
  \l 13312
  \l 13313
  \l 13314
  \l 13315
  \l 13316
  \l 13317
  \l 13318
  \l 13319
  \l 1331A
  \l 1331B
  \l 1331C
  \l 1331D
  \l 1331E
  \l 1331F
  \l 13320
  \l 13321
  \l 13322
  \l 13323
  \l 13324
  \l 13325
  \l 13326
  \l 13327
  \l 13328
  \l 13329
  \l 1332A
  \l 1332B
  \l 1332C
  \l 1332D
  \l 1332E
  \l 1332F
  \l 13330
  \l 13331
  \l 13332
  \l 13333
  \l 13334
  \l 13335
  \l 13336
  \l 13337
  \l 13338
  \l 13339
  \l 1333A
  \l 1333B
  \l 1333C
  \l 1333D
  \l 1333E
  \l 1333F
  \l 13340
  \l 13341
  \l 13342
  \l 13343
  \l 13344
  \l 13345
  \l 13346
  \l 13347
  \l 13348
  \l 13349
  \l 1334A
  \l 1334B
  \l 1334C
  \l 1334D
  \l 1334E
  \l 1334F
  \l 13350
  \l 13351
  \l 13352
  \l 13353
  \l 13354
  \l 13355
  \l 13356
  \l 13357
  \l 13358
  \l 13359
  \l 1335A
  \l 1335B
  \l 1335C
  \l 1335D
  \l 1335E
  \l 1335F
  \l 13360
  \l 13361
  \l 13362
  \l 13363
  \l 13364
  \l 13365
  \l 13366
  \l 13367
  \l 13368
  \l 13369
  \l 1336A
  \l 1336B
  \l 1336C
  \l 1336D
  \l 1336E
  \l 1336F
  \l 13370
  \l 13371
  \l 13372
  \l 13373
  \l 13374
  \l 13375
  \l 13376
  \l 13377
  \l 13378
  \l 13379
  \l 1337A
  \l 1337B
  \l 1337C
  \l 1337D
  \l 1337E
  \l 1337F
  \l 13380
  \l 13381
  \l 13382
  \l 13383
  \l 13384
  \l 13385
  \l 13386
  \l 13387
  \l 13388
  \l 13389
  \l 1338A
  \l 1338B
  \l 1338C
  \l 1338D
  \l 1338E
  \l 1338F
  \l 13390
  \l 13391
  \l 13392
  \l 13393
  \l 13394
  \l 13395
  \l 13396
  \l 13397
  \l 13398
  \l 13399
  \l 1339A
  \l 1339B
  \l 1339C
  \l 1339D
  \l 1339E
  \l 1339F
  \l 133A0
  \l 133A1
  \l 133A2
  \l 133A3
  \l 133A4
  \l 133A5
  \l 133A6
  \l 133A7
  \l 133A8
  \l 133A9
  \l 133AA
  \l 133AB
  \l 133AC
  \l 133AD
  \l 133AE
  \l 133AF
  \l 133B0
  \l 133B1
  \l 133B2
  \l 133B3
  \l 133B4
  \l 133B5
  \l 133B6
  \l 133B7
  \l 133B8
  \l 133B9
  \l 133BA
  \l 133BB
  \l 133BC
  \l 133BD
  \l 133BE
  \l 133BF
  \l 133C0
  \l 133C1
  \l 133C2
  \l 133C3
  \l 133C4
  \l 133C5
  \l 133C6
  \l 133C7
  \l 133C8
  \l 133C9
  \l 133CA
  \l 133CB
  \l 133CC
  \l 133CD
  \l 133CE
  \l 133CF
  \l 133D0
  \l 133D1
  \l 133D2
  \l 133D3
  \l 133D4
  \l 133D5
  \l 133D6
  \l 133D7
  \l 133D8
  \l 133D9
  \l 133DA
  \l 133DB
  \l 133DC
  \l 133DD
  \l 133DE
  \l 133DF
  \l 133E0
  \l 133E1
  \l 133E2
  \l 133E3
  \l 133E4
  \l 133E5
  \l 133E6
  \l 133E7
  \l 133E8
  \l 133E9
  \l 133EA
  \l 133EB
  \l 133EC
  \l 133ED
  \l 133EE
  \l 133EF
  \l 133F0
  \l 133F1
  \l 133F2
  \l 133F3
  \l 133F4
  \l 133F5
  \l 133F6
  \l 133F7
  \l 133F8
  \l 133F9
  \l 133FA
  \l 133FB
  \l 133FC
  \l 133FD
  \l 133FE
  \l 133FF
  \l 13400
  \l 13401
  \l 13402
  \l 13403
  \l 13404
  \l 13405
  \l 13406
  \l 13407
  \l 13408
  \l 13409
  \l 1340A
  \l 1340B
  \l 1340C
  \l 1340D
  \l 1340E
  \l 1340F
  \l 13410
  \l 13411
  \l 13412
  \l 13413
  \l 13414
  \l 13415
  \l 13416
  \l 13417
  \l 13418
  \l 13419
  \l 1341A
  \l 1341B
  \l 1341C
  \l 1341D
  \l 1341E
  \l 1341F
  \l 13420
  \l 13421
  \l 13422
  \l 13423
  \l 13424
  \l 13425
  \l 13426
  \l 13427
  \l 13428
  \l 13429
  \l 1342A
  \l 1342B
  \l 1342C
  \l 1342D
  \l 1342E
  \l 16800
  \l 16801
  \l 16802
  \l 16803
  \l 16804
  \l 16805
  \l 16806
  \l 16807
  \l 16808
  \l 16809
  \l 1680A
  \l 1680B
  \l 1680C
  \l 1680D
  \l 1680E
  \l 1680F
  \l 16810
  \l 16811
  \l 16812
  \l 16813
  \l 16814
  \l 16815
  \l 16816
  \l 16817
  \l 16818
  \l 16819
  \l 1681A
  \l 1681B
  \l 1681C
  \l 1681D
  \l 1681E
  \l 1681F
  \l 16820
  \l 16821
  \l 16822
  \l 16823
  \l 16824
  \l 16825
  \l 16826
  \l 16827
  \l 16828
  \l 16829
  \l 1682A
  \l 1682B
  \l 1682C
  \l 1682D
  \l 1682E
  \l 1682F
  \l 16830
  \l 16831
  \l 16832
  \l 16833
  \l 16834
  \l 16835
  \l 16836
  \l 16837
  \l 16838
  \l 16839
  \l 1683A
  \l 1683B
  \l 1683C
  \l 1683D
  \l 1683E
  \l 1683F
  \l 16840
  \l 16841
  \l 16842
  \l 16843
  \l 16844
  \l 16845
  \l 16846
  \l 16847
  \l 16848
  \l 16849
  \l 1684A
  \l 1684B
  \l 1684C
  \l 1684D
  \l 1684E
  \l 1684F
  \l 16850
  \l 16851
  \l 16852
  \l 16853
  \l 16854
  \l 16855
  \l 16856
  \l 16857
  \l 16858
  \l 16859
  \l 1685A
  \l 1685B
  \l 1685C
  \l 1685D
  \l 1685E
  \l 1685F
  \l 16860
  \l 16861
  \l 16862
  \l 16863
  \l 16864
  \l 16865
  \l 16866
  \l 16867
  \l 16868
  \l 16869
  \l 1686A
  \l 1686B
  \l 1686C
  \l 1686D
  \l 1686E
  \l 1686F
  \l 16870
  \l 16871
  \l 16872
  \l 16873
  \l 16874
  \l 16875
  \l 16876
  \l 16877
  \l 16878
  \l 16879
  \l 1687A
  \l 1687B
  \l 1687C
  \l 1687D
  \l 1687E
  \l 1687F
  \l 16880
  \l 16881
  \l 16882
  \l 16883
  \l 16884
  \l 16885
  \l 16886
  \l 16887
  \l 16888
  \l 16889
  \l 1688A
  \l 1688B
  \l 1688C
  \l 1688D
  \l 1688E
  \l 1688F
  \l 16890
  \l 16891
  \l 16892
  \l 16893
  \l 16894
  \l 16895
  \l 16896
  \l 16897
  \l 16898
  \l 16899
  \l 1689A
  \l 1689B
  \l 1689C
  \l 1689D
  \l 1689E
  \l 1689F
  \l 168A0
  \l 168A1
  \l 168A2
  \l 168A3
  \l 168A4
  \l 168A5
  \l 168A6
  \l 168A7
  \l 168A8
  \l 168A9
  \l 168AA
  \l 168AB
  \l 168AC
  \l 168AD
  \l 168AE
  \l 168AF
  \l 168B0
  \l 168B1
  \l 168B2
  \l 168B3
  \l 168B4
  \l 168B5
  \l 168B6
  \l 168B7
  \l 168B8
  \l 168B9
  \l 168BA
  \l 168BB
  \l 168BC
  \l 168BD
  \l 168BE
  \l 168BF
  \l 168C0
  \l 168C1
  \l 168C2
  \l 168C3
  \l 168C4
  \l 168C5
  \l 168C6
  \l 168C7
  \l 168C8
  \l 168C9
  \l 168CA
  \l 168CB
  \l 168CC
  \l 168CD
  \l 168CE
  \l 168CF
  \l 168D0
  \l 168D1
  \l 168D2
  \l 168D3
  \l 168D4
  \l 168D5
  \l 168D6
  \l 168D7
  \l 168D8
  \l 168D9
  \l 168DA
  \l 168DB
  \l 168DC
  \l 168DD
  \l 168DE
  \l 168DF
  \l 168E0
  \l 168E1
  \l 168E2
  \l 168E3
  \l 168E4
  \l 168E5
  \l 168E6
  \l 168E7
  \l 168E8
  \l 168E9
  \l 168EA
  \l 168EB
  \l 168EC
  \l 168ED
  \l 168EE
  \l 168EF
  \l 168F0
  \l 168F1
  \l 168F2
  \l 168F3
  \l 168F4
  \l 168F5
  \l 168F6
  \l 168F7
  \l 168F8
  \l 168F9
  \l 168FA
  \l 168FB
  \l 168FC
  \l 168FD
  \l 168FE
  \l 168FF
  \l 16900
  \l 16901
  \l 16902
  \l 16903
  \l 16904
  \l 16905
  \l 16906
  \l 16907
  \l 16908
  \l 16909
  \l 1690A
  \l 1690B
  \l 1690C
  \l 1690D
  \l 1690E
  \l 1690F
  \l 16910
  \l 16911
  \l 16912
  \l 16913
  \l 16914
  \l 16915
  \l 16916
  \l 16917
  \l 16918
  \l 16919
  \l 1691A
  \l 1691B
  \l 1691C
  \l 1691D
  \l 1691E
  \l 1691F
  \l 16920
  \l 16921
  \l 16922
  \l 16923
  \l 16924
  \l 16925
  \l 16926
  \l 16927
  \l 16928
  \l 16929
  \l 1692A
  \l 1692B
  \l 1692C
  \l 1692D
  \l 1692E
  \l 1692F
  \l 16930
  \l 16931
  \l 16932
  \l 16933
  \l 16934
  \l 16935
  \l 16936
  \l 16937
  \l 16938
  \l 16939
  \l 1693A
  \l 1693B
  \l 1693C
  \l 1693D
  \l 1693E
  \l 1693F
  \l 16940
  \l 16941
  \l 16942
  \l 16943
  \l 16944
  \l 16945
  \l 16946
  \l 16947
  \l 16948
  \l 16949
  \l 1694A
  \l 1694B
  \l 1694C
  \l 1694D
  \l 1694E
  \l 1694F
  \l 16950
  \l 16951
  \l 16952
  \l 16953
  \l 16954
  \l 16955
  \l 16956
  \l 16957
  \l 16958
  \l 16959
  \l 1695A
  \l 1695B
  \l 1695C
  \l 1695D
  \l 1695E
  \l 1695F
  \l 16960
  \l 16961
  \l 16962
  \l 16963
  \l 16964
  \l 16965
  \l 16966
  \l 16967
  \l 16968
  \l 16969
  \l 1696A
  \l 1696B
  \l 1696C
  \l 1696D
  \l 1696E
  \l 1696F
  \l 16970
  \l 16971
  \l 16972
  \l 16973
  \l 16974
  \l 16975
  \l 16976
  \l 16977
  \l 16978
  \l 16979
  \l 1697A
  \l 1697B
  \l 1697C
  \l 1697D
  \l 1697E
  \l 1697F
  \l 16980
  \l 16981
  \l 16982
  \l 16983
  \l 16984
  \l 16985
  \l 16986
  \l 16987
  \l 16988
  \l 16989
  \l 1698A
  \l 1698B
  \l 1698C
  \l 1698D
  \l 1698E
  \l 1698F
  \l 16990
  \l 16991
  \l 16992
  \l 16993
  \l 16994
  \l 16995
  \l 16996
  \l 16997
  \l 16998
  \l 16999
  \l 1699A
  \l 1699B
  \l 1699C
  \l 1699D
  \l 1699E
  \l 1699F
  \l 169A0
  \l 169A1
  \l 169A2
  \l 169A3
  \l 169A4
  \l 169A5
  \l 169A6
  \l 169A7
  \l 169A8
  \l 169A9
  \l 169AA
  \l 169AB
  \l 169AC
  \l 169AD
  \l 169AE
  \l 169AF
  \l 169B0
  \l 169B1
  \l 169B2
  \l 169B3
  \l 169B4
  \l 169B5
  \l 169B6
  \l 169B7
  \l 169B8
  \l 169B9
  \l 169BA
  \l 169BB
  \l 169BC
  \l 169BD
  \l 169BE
  \l 169BF
  \l 169C0
  \l 169C1
  \l 169C2
  \l 169C3
  \l 169C4
  \l 169C5
  \l 169C6
  \l 169C7
  \l 169C8
  \l 169C9
  \l 169CA
  \l 169CB
  \l 169CC
  \l 169CD
  \l 169CE
  \l 169CF
  \l 169D0
  \l 169D1
  \l 169D2
  \l 169D3
  \l 169D4
  \l 169D5
  \l 169D6
  \l 169D7
  \l 169D8
  \l 169D9
  \l 169DA
  \l 169DB
  \l 169DC
  \l 169DD
  \l 169DE
  \l 169DF
  \l 169E0
  \l 169E1
  \l 169E2
  \l 169E3
  \l 169E4
  \l 169E5
  \l 169E6
  \l 169E7
  \l 169E8
  \l 169E9
  \l 169EA
  \l 169EB
  \l 169EC
  \l 169ED
  \l 169EE
  \l 169EF
  \l 169F0
  \l 169F1
  \l 169F2
  \l 169F3
  \l 169F4
  \l 169F5
  \l 169F6
  \l 169F7
  \l 169F8
  \l 169F9
  \l 169FA
  \l 169FB
  \l 169FC
  \l 169FD
  \l 169FE
  \l 169FF
  \l 16A00
  \l 16A01
  \l 16A02
  \l 16A03
  \l 16A04
  \l 16A05
  \l 16A06
  \l 16A07
  \l 16A08
  \l 16A09
  \l 16A0A
  \l 16A0B
  \l 16A0C
  \l 16A0D
  \l 16A0E
  \l 16A0F
  \l 16A10
  \l 16A11
  \l 16A12
  \l 16A13
  \l 16A14
  \l 16A15
  \l 16A16
  \l 16A17
  \l 16A18
  \l 16A19
  \l 16A1A
  \l 16A1B
  \l 16A1C
  \l 16A1D
  \l 16A1E
  \l 16A1F
  \l 16A20
  \l 16A21
  \l 16A22
  \l 16A23
  \l 16A24
  \l 16A25
  \l 16A26
  \l 16A27
  \l 16A28
  \l 16A29
  \l 16A2A
  \l 16A2B
  \l 16A2C
  \l 16A2D
  \l 16A2E
  \l 16A2F
  \l 16A30
  \l 16A31
  \l 16A32
  \l 16A33
  \l 16A34
  \l 16A35
  \l 16A36
  \l 16A37
  \l 16A38
  \l 16A40
  \l 16A41
  \l 16A42
  \l 16A43
  \l 16A44
  \l 16A45
  \l 16A46
  \l 16A47
  \l 16A48
  \l 16A49
  \l 16A4A
  \l 16A4B
  \l 16A4C
  \l 16A4D
  \l 16A4E
  \l 16A4F
  \l 16A50
  \l 16A51
  \l 16A52
  \l 16A53
  \l 16A54
  \l 16A55
  \l 16A56
  \l 16A57
  \l 16A58
  \l 16A59
  \l 16A5A
  \l 16A5B
  \l 16A5C
  \l 16A5D
  \l 16A5E
  \l 16AD0
  \l 16AD1
  \l 16AD2
  \l 16AD3
  \l 16AD4
  \l 16AD5
  \l 16AD6
  \l 16AD7
  \l 16AD8
  \l 16AD9
  \l 16ADA
  \l 16ADB
  \l 16ADC
  \l 16ADD
  \l 16ADE
  \l 16ADF
  \l 16AE0
  \l 16AE1
  \l 16AE2
  \l 16AE3
  \l 16AE4
  \l 16AE5
  \l 16AE6
  \l 16AE7
  \l 16AE8
  \l 16AE9
  \l 16AEA
  \l 16AEB
  \l 16AEC
  \l 16AED
  \l 16AF0
  \l 16AF1
  \l 16AF2
  \l 16AF3
  \l 16AF4
  \l 16B00
  \l 16B01
  \l 16B02
  \l 16B03
  \l 16B04
  \l 16B05
  \l 16B06
  \l 16B07
  \l 16B08
  \l 16B09
  \l 16B0A
  \l 16B0B
  \l 16B0C
  \l 16B0D
  \l 16B0E
  \l 16B0F
  \l 16B10
  \l 16B11
  \l 16B12
  \l 16B13
  \l 16B14
  \l 16B15
  \l 16B16
  \l 16B17
  \l 16B18
  \l 16B19
  \l 16B1A
  \l 16B1B
  \l 16B1C
  \l 16B1D
  \l 16B1E
  \l 16B1F
  \l 16B20
  \l 16B21
  \l 16B22
  \l 16B23
  \l 16B24
  \l 16B25
  \l 16B26
  \l 16B27
  \l 16B28
  \l 16B29
  \l 16B2A
  \l 16B2B
  \l 16B2C
  \l 16B2D
  \l 16B2E
  \l 16B2F
  \l 16B30
  \l 16B31
  \l 16B32
  \l 16B33
  \l 16B34
  \l 16B35
  \l 16B36
  \l 16B40
  \l 16B41
  \l 16B42
  \l 16B43
  \l 16B63
  \l 16B64
  \l 16B65
  \l 16B66
  \l 16B67
  \l 16B68
  \l 16B69
  \l 16B6A
  \l 16B6B
  \l 16B6C
  \l 16B6D
  \l 16B6E
  \l 16B6F
  \l 16B70
  \l 16B71
  \l 16B72
  \l 16B73
  \l 16B74
  \l 16B75
  \l 16B76
  \l 16B77
  \l 16B7D
  \l 16B7E
  \l 16B7F
  \l 16B80
  \l 16B81
  \l 16B82
  \l 16B83
  \l 16B84
  \l 16B85
  \l 16B86
  \l 16B87
  \l 16B88
  \l 16B89
  \l 16B8A
  \l 16B8B
  \l 16B8C
  \l 16B8D
  \l 16B8E
  \l 16B8F
  \l 16F00
  \l 16F01
  \l 16F02
  \l 16F03
  \l 16F04
  \l 16F05
  \l 16F06
  \l 16F07
  \l 16F08
  \l 16F09
  \l 16F0A
  \l 16F0B
  \l 16F0C
  \l 16F0D
  \l 16F0E
  \l 16F0F
  \l 16F10
  \l 16F11
  \l 16F12
  \l 16F13
  \l 16F14
  \l 16F15
  \l 16F16
  \l 16F17
  \l 16F18
  \l 16F19
  \l 16F1A
  \l 16F1B
  \l 16F1C
  \l 16F1D
  \l 16F1E
  \l 16F1F
  \l 16F20
  \l 16F21
  \l 16F22
  \l 16F23
  \l 16F24
  \l 16F25
  \l 16F26
  \l 16F27
  \l 16F28
  \l 16F29
  \l 16F2A
  \l 16F2B
  \l 16F2C
  \l 16F2D
  \l 16F2E
  \l 16F2F
  \l 16F30
  \l 16F31
  \l 16F32
  \l 16F33
  \l 16F34
  \l 16F35
  \l 16F36
  \l 16F37
  \l 16F38
  \l 16F39
  \l 16F3A
  \l 16F3B
  \l 16F3C
  \l 16F3D
  \l 16F3E
  \l 16F3F
  \l 16F40
  \l 16F41
  \l 16F42
  \l 16F43
  \l 16F44
  \l 16F50
  \l 16F51
  \l 16F52
  \l 16F53
  \l 16F54
  \l 16F55
  \l 16F56
  \l 16F57
  \l 16F58
  \l 16F59
  \l 16F5A
  \l 16F5B
  \l 16F5C
  \l 16F5D
  \l 16F5E
  \l 16F5F
  \l 16F60
  \l 16F61
  \l 16F62
  \l 16F63
  \l 16F64
  \l 16F65
  \l 16F66
  \l 16F67
  \l 16F68
  \l 16F69
  \l 16F6A
  \l 16F6B
  \l 16F6C
  \l 16F6D
  \l 16F6E
  \l 16F6F
  \l 16F70
  \l 16F71
  \l 16F72
  \l 16F73
  \l 16F74
  \l 16F75
  \l 16F76
  \l 16F77
  \l 16F78
  \l 16F79
  \l 16F7A
  \l 16F7B
  \l 16F7C
  \l 16F7D
  \l 16F7E
  \l 16F8F
  \l 16F90
  \l 16F91
  \l 16F92
  \l 16F93
  \l 16F94
  \l 16F95
  \l 16F96
  \l 16F97
  \l 16F98
  \l 16F99
  \l 16F9A
  \l 16F9B
  \l 16F9C
  \l 16F9D
  \l 16F9E
  \l 16F9F
  \l 1B000
  \l 1B001
  \l 1BC00
  \l 1BC01
  \l 1BC02
  \l 1BC03
  \l 1BC04
  \l 1BC05
  \l 1BC06
  \l 1BC07
  \l 1BC08
  \l 1BC09
  \l 1BC0A
  \l 1BC0B
  \l 1BC0C
  \l 1BC0D
  \l 1BC0E
  \l 1BC0F
  \l 1BC10
  \l 1BC11
  \l 1BC12
  \l 1BC13
  \l 1BC14
  \l 1BC15
  \l 1BC16
  \l 1BC17
  \l 1BC18
  \l 1BC19
  \l 1BC1A
  \l 1BC1B
  \l 1BC1C
  \l 1BC1D
  \l 1BC1E
  \l 1BC1F
  \l 1BC20
  \l 1BC21
  \l 1BC22
  \l 1BC23
  \l 1BC24
  \l 1BC25
  \l 1BC26
  \l 1BC27
  \l 1BC28
  \l 1BC29
  \l 1BC2A
  \l 1BC2B
  \l 1BC2C
  \l 1BC2D
  \l 1BC2E
  \l 1BC2F
  \l 1BC30
  \l 1BC31
  \l 1BC32
  \l 1BC33
  \l 1BC34
  \l 1BC35
  \l 1BC36
  \l 1BC37
  \l 1BC38
  \l 1BC39
  \l 1BC3A
  \l 1BC3B
  \l 1BC3C
  \l 1BC3D
  \l 1BC3E
  \l 1BC3F
  \l 1BC40
  \l 1BC41
  \l 1BC42
  \l 1BC43
  \l 1BC44
  \l 1BC45
  \l 1BC46
  \l 1BC47
  \l 1BC48
  \l 1BC49
  \l 1BC4A
  \l 1BC4B
  \l 1BC4C
  \l 1BC4D
  \l 1BC4E
  \l 1BC4F
  \l 1BC50
  \l 1BC51
  \l 1BC52
  \l 1BC53
  \l 1BC54
  \l 1BC55
  \l 1BC56
  \l 1BC57
  \l 1BC58
  \l 1BC59
  \l 1BC5A
  \l 1BC5B
  \l 1BC5C
  \l 1BC5D
  \l 1BC5E
  \l 1BC5F
  \l 1BC60
  \l 1BC61
  \l 1BC62
  \l 1BC63
  \l 1BC64
  \l 1BC65
  \l 1BC66
  \l 1BC67
  \l 1BC68
  \l 1BC69
  \l 1BC6A
  \l 1BC70
  \l 1BC71
  \l 1BC72
  \l 1BC73
  \l 1BC74
  \l 1BC75
  \l 1BC76
  \l 1BC77
  \l 1BC78
  \l 1BC79
  \l 1BC7A
  \l 1BC7B
  \l 1BC7C
  \l 1BC80
  \l 1BC81
  \l 1BC82
  \l 1BC83
  \l 1BC84
  \l 1BC85
  \l 1BC86
  \l 1BC87
  \l 1BC88
  \l 1BC90
  \l 1BC91
  \l 1BC92
  \l 1BC93
  \l 1BC94
  \l 1BC95
  \l 1BC96
  \l 1BC97
  \l 1BC98
  \l 1BC99
  \l 1BC9D
  \l 1BC9E
  \l 1D165
  \l 1D166
  \l 1D167
  \l 1D168
  \l 1D169
  \l 1D16D
  \l 1D16E
  \l 1D16F
  \l 1D170
  \l 1D171
  \l 1D172
  \l 1D17B
  \l 1D17C
  \l 1D17D
  \l 1D17E
  \l 1D17F
  \l 1D180
  \l 1D181
  \l 1D182
  \l 1D185
  \l 1D186
  \l 1D187
  \l 1D188
  \l 1D189
  \l 1D18A
  \l 1D18B
  \l 1D1AA
  \l 1D1AB
  \l 1D1AC
  \l 1D1AD
  \l 1D242
  \l 1D243
  \l 1D244
  \l 1D400
  \l 1D401
  \l 1D402
  \l 1D403
  \l 1D404
  \l 1D405
  \l 1D406
  \l 1D407
  \l 1D408
  \l 1D409
  \l 1D40A
  \l 1D40B
  \l 1D40C
  \l 1D40D
  \l 1D40E
  \l 1D40F
  \l 1D410
  \l 1D411
  \l 1D412
  \l 1D413
  \l 1D414
  \l 1D415
  \l 1D416
  \l 1D417
  \l 1D418
  \l 1D419
  \l 1D41A
  \l 1D41B
  \l 1D41C
  \l 1D41D
  \l 1D41E
  \l 1D41F
  \l 1D420
  \l 1D421
  \l 1D422
  \l 1D423
  \l 1D424
  \l 1D425
  \l 1D426
  \l 1D427
  \l 1D428
  \l 1D429
  \l 1D42A
  \l 1D42B
  \l 1D42C
  \l 1D42D
  \l 1D42E
  \l 1D42F
  \l 1D430
  \l 1D431
  \l 1D432
  \l 1D433
  \l 1D434
  \l 1D435
  \l 1D436
  \l 1D437
  \l 1D438
  \l 1D439
  \l 1D43A
  \l 1D43B
  \l 1D43C
  \l 1D43D
  \l 1D43E
  \l 1D43F
  \l 1D440
  \l 1D441
  \l 1D442
  \l 1D443
  \l 1D444
  \l 1D445
  \l 1D446
  \l 1D447
  \l 1D448
  \l 1D449
  \l 1D44A
  \l 1D44B
  \l 1D44C
  \l 1D44D
  \l 1D44E
  \l 1D44F
  \l 1D450
  \l 1D451
  \l 1D452
  \l 1D453
  \l 1D454
  \l 1D456
  \l 1D457
  \l 1D458
  \l 1D459
  \l 1D45A
  \l 1D45B
  \l 1D45C
  \l 1D45D
  \l 1D45E
  \l 1D45F
  \l 1D460
  \l 1D461
  \l 1D462
  \l 1D463
  \l 1D464
  \l 1D465
  \l 1D466
  \l 1D467
  \l 1D468
  \l 1D469
  \l 1D46A
  \l 1D46B
  \l 1D46C
  \l 1D46D
  \l 1D46E
  \l 1D46F
  \l 1D470
  \l 1D471
  \l 1D472
  \l 1D473
  \l 1D474
  \l 1D475
  \l 1D476
  \l 1D477
  \l 1D478
  \l 1D479
  \l 1D47A
  \l 1D47B
  \l 1D47C
  \l 1D47D
  \l 1D47E
  \l 1D47F
  \l 1D480
  \l 1D481
  \l 1D482
  \l 1D483
  \l 1D484
  \l 1D485
  \l 1D486
  \l 1D487
  \l 1D488
  \l 1D489
  \l 1D48A
  \l 1D48B
  \l 1D48C
  \l 1D48D
  \l 1D48E
  \l 1D48F
  \l 1D490
  \l 1D491
  \l 1D492
  \l 1D493
  \l 1D494
  \l 1D495
  \l 1D496
  \l 1D497
  \l 1D498
  \l 1D499
  \l 1D49A
  \l 1D49B
  \l 1D49C
  \l 1D49E
  \l 1D49F
  \l 1D4A2
  \l 1D4A5
  \l 1D4A6
  \l 1D4A9
  \l 1D4AA
  \l 1D4AB
  \l 1D4AC
  \l 1D4AE
  \l 1D4AF
  \l 1D4B0
  \l 1D4B1
  \l 1D4B2
  \l 1D4B3
  \l 1D4B4
  \l 1D4B5
  \l 1D4B6
  \l 1D4B7
  \l 1D4B8
  \l 1D4B9
  \l 1D4BB
  \l 1D4BD
  \l 1D4BE
  \l 1D4BF
  \l 1D4C0
  \l 1D4C1
  \l 1D4C2
  \l 1D4C3
  \l 1D4C5
  \l 1D4C6
  \l 1D4C7
  \l 1D4C8
  \l 1D4C9
  \l 1D4CA
  \l 1D4CB
  \l 1D4CC
  \l 1D4CD
  \l 1D4CE
  \l 1D4CF
  \l 1D4D0
  \l 1D4D1
  \l 1D4D2
  \l 1D4D3
  \l 1D4D4
  \l 1D4D5
  \l 1D4D6
  \l 1D4D7
  \l 1D4D8
  \l 1D4D9
  \l 1D4DA
  \l 1D4DB
  \l 1D4DC
  \l 1D4DD
  \l 1D4DE
  \l 1D4DF
  \l 1D4E0
  \l 1D4E1
  \l 1D4E2
  \l 1D4E3
  \l 1D4E4
  \l 1D4E5
  \l 1D4E6
  \l 1D4E7
  \l 1D4E8
  \l 1D4E9
  \l 1D4EA
  \l 1D4EB
  \l 1D4EC
  \l 1D4ED
  \l 1D4EE
  \l 1D4EF
  \l 1D4F0
  \l 1D4F1
  \l 1D4F2
  \l 1D4F3
  \l 1D4F4
  \l 1D4F5
  \l 1D4F6
  \l 1D4F7
  \l 1D4F8
  \l 1D4F9
  \l 1D4FA
  \l 1D4FB
  \l 1D4FC
  \l 1D4FD
  \l 1D4FE
  \l 1D4FF
  \l 1D500
  \l 1D501
  \l 1D502
  \l 1D503
  \l 1D504
  \l 1D505
  \l 1D507
  \l 1D508
  \l 1D509
  \l 1D50A
  \l 1D50D
  \l 1D50E
  \l 1D50F
  \l 1D510
  \l 1D511
  \l 1D512
  \l 1D513
  \l 1D514
  \l 1D516
  \l 1D517
  \l 1D518
  \l 1D519
  \l 1D51A
  \l 1D51B
  \l 1D51C
  \l 1D51E
  \l 1D51F
  \l 1D520
  \l 1D521
  \l 1D522
  \l 1D523
  \l 1D524
  \l 1D525
  \l 1D526
  \l 1D527
  \l 1D528
  \l 1D529
  \l 1D52A
  \l 1D52B
  \l 1D52C
  \l 1D52D
  \l 1D52E
  \l 1D52F
  \l 1D530
  \l 1D531
  \l 1D532
  \l 1D533
  \l 1D534
  \l 1D535
  \l 1D536
  \l 1D537
  \l 1D538
  \l 1D539
  \l 1D53B
  \l 1D53C
  \l 1D53D
  \l 1D53E
  \l 1D540
  \l 1D541
  \l 1D542
  \l 1D543
  \l 1D544
  \l 1D546
  \l 1D54A
  \l 1D54B
  \l 1D54C
  \l 1D54D
  \l 1D54E
  \l 1D54F
  \l 1D550
  \l 1D552
  \l 1D553
  \l 1D554
  \l 1D555
  \l 1D556
  \l 1D557
  \l 1D558
  \l 1D559
  \l 1D55A
  \l 1D55B
  \l 1D55C
  \l 1D55D
  \l 1D55E
  \l 1D55F
  \l 1D560
  \l 1D561
  \l 1D562
  \l 1D563
  \l 1D564
  \l 1D565
  \l 1D566
  \l 1D567
  \l 1D568
  \l 1D569
  \l 1D56A
  \l 1D56B
  \l 1D56C
  \l 1D56D
  \l 1D56E
  \l 1D56F
  \l 1D570
  \l 1D571
  \l 1D572
  \l 1D573
  \l 1D574
  \l 1D575
  \l 1D576
  \l 1D577
  \l 1D578
  \l 1D579
  \l 1D57A
  \l 1D57B
  \l 1D57C
  \l 1D57D
  \l 1D57E
  \l 1D57F
  \l 1D580
  \l 1D581
  \l 1D582
  \l 1D583
  \l 1D584
  \l 1D585
  \l 1D586
  \l 1D587
  \l 1D588
  \l 1D589
  \l 1D58A
  \l 1D58B
  \l 1D58C
  \l 1D58D
  \l 1D58E
  \l 1D58F
  \l 1D590
  \l 1D591
  \l 1D592
  \l 1D593
  \l 1D594
  \l 1D595
  \l 1D596
  \l 1D597
  \l 1D598
  \l 1D599
  \l 1D59A
  \l 1D59B
  \l 1D59C
  \l 1D59D
  \l 1D59E
  \l 1D59F
  \l 1D5A0
  \l 1D5A1
  \l 1D5A2
  \l 1D5A3
  \l 1D5A4
  \l 1D5A5
  \l 1D5A6
  \l 1D5A7
  \l 1D5A8
  \l 1D5A9
  \l 1D5AA
  \l 1D5AB
  \l 1D5AC
  \l 1D5AD
  \l 1D5AE
  \l 1D5AF
  \l 1D5B0
  \l 1D5B1
  \l 1D5B2
  \l 1D5B3
  \l 1D5B4
  \l 1D5B5
  \l 1D5B6
  \l 1D5B7
  \l 1D5B8
  \l 1D5B9
  \l 1D5BA
  \l 1D5BB
  \l 1D5BC
  \l 1D5BD
  \l 1D5BE
  \l 1D5BF
  \l 1D5C0
  \l 1D5C1
  \l 1D5C2
  \l 1D5C3
  \l 1D5C4
  \l 1D5C5
  \l 1D5C6
  \l 1D5C7
  \l 1D5C8
  \l 1D5C9
  \l 1D5CA
  \l 1D5CB
  \l 1D5CC
  \l 1D5CD
  \l 1D5CE
  \l 1D5CF
  \l 1D5D0
  \l 1D5D1
  \l 1D5D2
  \l 1D5D3
  \l 1D5D4
  \l 1D5D5
  \l 1D5D6
  \l 1D5D7
  \l 1D5D8
  \l 1D5D9
  \l 1D5DA
  \l 1D5DB
  \l 1D5DC
  \l 1D5DD
  \l 1D5DE
  \l 1D5DF
  \l 1D5E0
  \l 1D5E1
  \l 1D5E2
  \l 1D5E3
  \l 1D5E4
  \l 1D5E5
  \l 1D5E6
  \l 1D5E7
  \l 1D5E8
  \l 1D5E9
  \l 1D5EA
  \l 1D5EB
  \l 1D5EC
  \l 1D5ED
  \l 1D5EE
  \l 1D5EF
  \l 1D5F0
  \l 1D5F1
  \l 1D5F2
  \l 1D5F3
  \l 1D5F4
  \l 1D5F5
  \l 1D5F6
  \l 1D5F7
  \l 1D5F8
  \l 1D5F9
  \l 1D5FA
  \l 1D5FB
  \l 1D5FC
  \l 1D5FD
  \l 1D5FE
  \l 1D5FF
  \l 1D600
  \l 1D601
  \l 1D602
  \l 1D603
  \l 1D604
  \l 1D605
  \l 1D606
  \l 1D607
  \l 1D608
  \l 1D609
  \l 1D60A
  \l 1D60B
  \l 1D60C
  \l 1D60D
  \l 1D60E
  \l 1D60F
  \l 1D610
  \l 1D611
  \l 1D612
  \l 1D613
  \l 1D614
  \l 1D615
  \l 1D616
  \l 1D617
  \l 1D618
  \l 1D619
  \l 1D61A
  \l 1D61B
  \l 1D61C
  \l 1D61D
  \l 1D61E
  \l 1D61F
  \l 1D620
  \l 1D621
  \l 1D622
  \l 1D623
  \l 1D624
  \l 1D625
  \l 1D626
  \l 1D627
  \l 1D628
  \l 1D629
  \l 1D62A
  \l 1D62B
  \l 1D62C
  \l 1D62D
  \l 1D62E
  \l 1D62F
  \l 1D630
  \l 1D631
  \l 1D632
  \l 1D633
  \l 1D634
  \l 1D635
  \l 1D636
  \l 1D637
  \l 1D638
  \l 1D639
  \l 1D63A
  \l 1D63B
  \l 1D63C
  \l 1D63D
  \l 1D63E
  \l 1D63F
  \l 1D640
  \l 1D641
  \l 1D642
  \l 1D643
  \l 1D644
  \l 1D645
  \l 1D646
  \l 1D647
  \l 1D648
  \l 1D649
  \l 1D64A
  \l 1D64B
  \l 1D64C
  \l 1D64D
  \l 1D64E
  \l 1D64F
  \l 1D650
  \l 1D651
  \l 1D652
  \l 1D653
  \l 1D654
  \l 1D655
  \l 1D656
  \l 1D657
  \l 1D658
  \l 1D659
  \l 1D65A
  \l 1D65B
  \l 1D65C
  \l 1D65D
  \l 1D65E
  \l 1D65F
  \l 1D660
  \l 1D661
  \l 1D662
  \l 1D663
  \l 1D664
  \l 1D665
  \l 1D666
  \l 1D667
  \l 1D668
  \l 1D669
  \l 1D66A
  \l 1D66B
  \l 1D66C
  \l 1D66D
  \l 1D66E
  \l 1D66F
  \l 1D670
  \l 1D671
  \l 1D672
  \l 1D673
  \l 1D674
  \l 1D675
  \l 1D676
  \l 1D677
  \l 1D678
  \l 1D679
  \l 1D67A
  \l 1D67B
  \l 1D67C
  \l 1D67D
  \l 1D67E
  \l 1D67F
  \l 1D680
  \l 1D681
  \l 1D682
  \l 1D683
  \l 1D684
  \l 1D685
  \l 1D686
  \l 1D687
  \l 1D688
  \l 1D689
  \l 1D68A
  \l 1D68B
  \l 1D68C
  \l 1D68D
  \l 1D68E
  \l 1D68F
  \l 1D690
  \l 1D691
  \l 1D692
  \l 1D693
  \l 1D694
  \l 1D695
  \l 1D696
  \l 1D697
  \l 1D698
  \l 1D699
  \l 1D69A
  \l 1D69B
  \l 1D69C
  \l 1D69D
  \l 1D69E
  \l 1D69F
  \l 1D6A0
  \l 1D6A1
  \l 1D6A2
  \l 1D6A3
  \l 1D6A4
  \l 1D6A5
  \l 1D6A8
  \l 1D6A9
  \l 1D6AA
  \l 1D6AB
  \l 1D6AC
  \l 1D6AD
  \l 1D6AE
  \l 1D6AF
  \l 1D6B0
  \l 1D6B1
  \l 1D6B2
  \l 1D6B3
  \l 1D6B4
  \l 1D6B5
  \l 1D6B6
  \l 1D6B7
  \l 1D6B8
  \l 1D6B9
  \l 1D6BA
  \l 1D6BB
  \l 1D6BC
  \l 1D6BD
  \l 1D6BE
  \l 1D6BF
  \l 1D6C0
  \l 1D6C2
  \l 1D6C3
  \l 1D6C4
  \l 1D6C5
  \l 1D6C6
  \l 1D6C7
  \l 1D6C8
  \l 1D6C9
  \l 1D6CA
  \l 1D6CB
  \l 1D6CC
  \l 1D6CD
  \l 1D6CE
  \l 1D6CF
  \l 1D6D0
  \l 1D6D1
  \l 1D6D2
  \l 1D6D3
  \l 1D6D4
  \l 1D6D5
  \l 1D6D6
  \l 1D6D7
  \l 1D6D8
  \l 1D6D9
  \l 1D6DA
  \l 1D6DC
  \l 1D6DD
  \l 1D6DE
  \l 1D6DF
  \l 1D6E0
  \l 1D6E1
  \l 1D6E2
  \l 1D6E3
  \l 1D6E4
  \l 1D6E5
  \l 1D6E6
  \l 1D6E7
  \l 1D6E8
  \l 1D6E9
  \l 1D6EA
  \l 1D6EB
  \l 1D6EC
  \l 1D6ED
  \l 1D6EE
  \l 1D6EF
  \l 1D6F0
  \l 1D6F1
  \l 1D6F2
  \l 1D6F3
  \l 1D6F4
  \l 1D6F5
  \l 1D6F6
  \l 1D6F7
  \l 1D6F8
  \l 1D6F9
  \l 1D6FA
  \l 1D6FC
  \l 1D6FD
  \l 1D6FE
  \l 1D6FF
  \l 1D700
  \l 1D701
  \l 1D702
  \l 1D703
  \l 1D704
  \l 1D705
  \l 1D706
  \l 1D707
  \l 1D708
  \l 1D709
  \l 1D70A
  \l 1D70B
  \l 1D70C
  \l 1D70D
  \l 1D70E
  \l 1D70F
  \l 1D710
  \l 1D711
  \l 1D712
  \l 1D713
  \l 1D714
  \l 1D716
  \l 1D717
  \l 1D718
  \l 1D719
  \l 1D71A
  \l 1D71B
  \l 1D71C
  \l 1D71D
  \l 1D71E
  \l 1D71F
  \l 1D720
  \l 1D721
  \l 1D722
  \l 1D723
  \l 1D724
  \l 1D725
  \l 1D726
  \l 1D727
  \l 1D728
  \l 1D729
  \l 1D72A
  \l 1D72B
  \l 1D72C
  \l 1D72D
  \l 1D72E
  \l 1D72F
  \l 1D730
  \l 1D731
  \l 1D732
  \l 1D733
  \l 1D734
  \l 1D736
  \l 1D737
  \l 1D738
  \l 1D739
  \l 1D73A
  \l 1D73B
  \l 1D73C
  \l 1D73D
  \l 1D73E
  \l 1D73F
  \l 1D740
  \l 1D741
  \l 1D742
  \l 1D743
  \l 1D744
  \l 1D745
  \l 1D746
  \l 1D747
  \l 1D748
  \l 1D749
  \l 1D74A
  \l 1D74B
  \l 1D74C
  \l 1D74D
  \l 1D74E
  \l 1D750
  \l 1D751
  \l 1D752
  \l 1D753
  \l 1D754
  \l 1D755
  \l 1D756
  \l 1D757
  \l 1D758
  \l 1D759
  \l 1D75A
  \l 1D75B
  \l 1D75C
  \l 1D75D
  \l 1D75E
  \l 1D75F
  \l 1D760
  \l 1D761
  \l 1D762
  \l 1D763
  \l 1D764
  \l 1D765
  \l 1D766
  \l 1D767
  \l 1D768
  \l 1D769
  \l 1D76A
  \l 1D76B
  \l 1D76C
  \l 1D76D
  \l 1D76E
  \l 1D770
  \l 1D771
  \l 1D772
  \l 1D773
  \l 1D774
  \l 1D775
  \l 1D776
  \l 1D777
  \l 1D778
  \l 1D779
  \l 1D77A
  \l 1D77B
  \l 1D77C
  \l 1D77D
  \l 1D77E
  \l 1D77F
  \l 1D780
  \l 1D781
  \l 1D782
  \l 1D783
  \l 1D784
  \l 1D785
  \l 1D786
  \l 1D787
  \l 1D788
  \l 1D78A
  \l 1D78B
  \l 1D78C
  \l 1D78D
  \l 1D78E
  \l 1D78F
  \l 1D790
  \l 1D791
  \l 1D792
  \l 1D793
  \l 1D794
  \l 1D795
  \l 1D796
  \l 1D797
  \l 1D798
  \l 1D799
  \l 1D79A
  \l 1D79B
  \l 1D79C
  \l 1D79D
  \l 1D79E
  \l 1D79F
  \l 1D7A0
  \l 1D7A1
  \l 1D7A2
  \l 1D7A3
  \l 1D7A4
  \l 1D7A5
  \l 1D7A6
  \l 1D7A7
  \l 1D7A8
  \l 1D7AA
  \l 1D7AB
  \l 1D7AC
  \l 1D7AD
  \l 1D7AE
  \l 1D7AF
  \l 1D7B0
  \l 1D7B1
  \l 1D7B2
  \l 1D7B3
  \l 1D7B4
  \l 1D7B5
  \l 1D7B6
  \l 1D7B7
  \l 1D7B8
  \l 1D7B9
  \l 1D7BA
  \l 1D7BB
  \l 1D7BC
  \l 1D7BD
  \l 1D7BE
  \l 1D7BF
  \l 1D7C0
  \l 1D7C1
  \l 1D7C2
  \l 1D7C4
  \l 1D7C5
  \l 1D7C6
  \l 1D7C7
  \l 1D7C8
  \l 1D7C9
  \l 1D7CA
  \l 1D7CB
  \l 1E800
  \l 1E801
  \l 1E802
  \l 1E803
  \l 1E804
  \l 1E805
  \l 1E806
  \l 1E807
  \l 1E808
  \l 1E809
  \l 1E80A
  \l 1E80B
  \l 1E80C
  \l 1E80D
  \l 1E80E
  \l 1E80F
  \l 1E810
  \l 1E811
  \l 1E812
  \l 1E813
  \l 1E814
  \l 1E815
  \l 1E816
  \l 1E817
  \l 1E818
  \l 1E819
  \l 1E81A
  \l 1E81B
  \l 1E81C
  \l 1E81D
  \l 1E81E
  \l 1E81F
  \l 1E820
  \l 1E821
  \l 1E822
  \l 1E823
  \l 1E824
  \l 1E825
  \l 1E826
  \l 1E827
  \l 1E828
  \l 1E829
  \l 1E82A
  \l 1E82B
  \l 1E82C
  \l 1E82D
  \l 1E82E
  \l 1E82F
  \l 1E830
  \l 1E831
  \l 1E832
  \l 1E833
  \l 1E834
  \l 1E835
  \l 1E836
  \l 1E837
  \l 1E838
  \l 1E839
  \l 1E83A
  \l 1E83B
  \l 1E83C
  \l 1E83D
  \l 1E83E
  \l 1E83F
  \l 1E840
  \l 1E841
  \l 1E842
  \l 1E843
  \l 1E844
  \l 1E845
  \l 1E846
  \l 1E847
  \l 1E848
  \l 1E849
  \l 1E84A
  \l 1E84B
  \l 1E84C
  \l 1E84D
  \l 1E84E
  \l 1E84F
  \l 1E850
  \l 1E851
  \l 1E852
  \l 1E853
  \l 1E854
  \l 1E855
  \l 1E856
  \l 1E857
  \l 1E858
  \l 1E859
  \l 1E85A
  \l 1E85B
  \l 1E85C
  \l 1E85D
  \l 1E85E
  \l 1E85F
  \l 1E860
  \l 1E861
  \l 1E862
  \l 1E863
  \l 1E864
  \l 1E865
  \l 1E866
  \l 1E867
  \l 1E868
  \l 1E869
  \l 1E86A
  \l 1E86B
  \l 1E86C
  \l 1E86D
  \l 1E86E
  \l 1E86F
  \l 1E870
  \l 1E871
  \l 1E872
  \l 1E873
  \l 1E874
  \l 1E875
  \l 1E876
  \l 1E877
  \l 1E878
  \l 1E879
  \l 1E87A
  \l 1E87B
  \l 1E87C
  \l 1E87D
  \l 1E87E
  \l 1E87F
  \l 1E880
  \l 1E881
  \l 1E882
  \l 1E883
  \l 1E884
  \l 1E885
  \l 1E886
  \l 1E887
  \l 1E888
  \l 1E889
  \l 1E88A
  \l 1E88B
  \l 1E88C
  \l 1E88D
  \l 1E88E
  \l 1E88F
  \l 1E890
  \l 1E891
  \l 1E892
  \l 1E893
  \l 1E894
  \l 1E895
  \l 1E896
  \l 1E897
  \l 1E898
  \l 1E899
  \l 1E89A
  \l 1E89B
  \l 1E89C
  \l 1E89D
  \l 1E89E
  \l 1E89F
  \l 1E8A0
  \l 1E8A1
  \l 1E8A2
  \l 1E8A3
  \l 1E8A4
  \l 1E8A5
  \l 1E8A6
  \l 1E8A7
  \l 1E8A8
  \l 1E8A9
  \l 1E8AA
  \l 1E8AB
  \l 1E8AC
  \l 1E8AD
  \l 1E8AE
  \l 1E8AF
  \l 1E8B0
  \l 1E8B1
  \l 1E8B2
  \l 1E8B3
  \l 1E8B4
  \l 1E8B5
  \l 1E8B6
  \l 1E8B7
  \l 1E8B8
  \l 1E8B9
  \l 1E8BA
  \l 1E8BB
  \l 1E8BC
  \l 1E8BD
  \l 1E8BE
  \l 1E8BF
  \l 1E8C0
  \l 1E8C1
  \l 1E8C2
  \l 1E8C3
  \l 1E8C4
  \l 1E8D0
  \l 1E8D1
  \l 1E8D2
  \l 1E8D3
  \l 1E8D4
  \l 1E8D5
  \l 1E8D6
  \l 1EE00
  \l 1EE01
  \l 1EE02
  \l 1EE03
  \l 1EE05
  \l 1EE06
  \l 1EE07
  \l 1EE08
  \l 1EE09
  \l 1EE0A
  \l 1EE0B
  \l 1EE0C
  \l 1EE0D
  \l 1EE0E
  \l 1EE0F
  \l 1EE10
  \l 1EE11
  \l 1EE12
  \l 1EE13
  \l 1EE14
  \l 1EE15
  \l 1EE16
  \l 1EE17
  \l 1EE18
  \l 1EE19
  \l 1EE1A
  \l 1EE1B
  \l 1EE1C
  \l 1EE1D
  \l 1EE1E
  \l 1EE1F
  \l 1EE21
  \l 1EE22
  \l 1EE24
  \l 1EE27
  \l 1EE29
  \l 1EE2A
  \l 1EE2B
  \l 1EE2C
  \l 1EE2D
  \l 1EE2E
  \l 1EE2F
  \l 1EE30
  \l 1EE31
  \l 1EE32
  \l 1EE34
  \l 1EE35
  \l 1EE36
  \l 1EE37
  \l 1EE39
  \l 1EE3B
  \l 1EE42
  \l 1EE47
  \l 1EE49
  \l 1EE4B
  \l 1EE4D
  \l 1EE4E
  \l 1EE4F
  \l 1EE51
  \l 1EE52
  \l 1EE54
  \l 1EE57
  \l 1EE59
  \l 1EE5B
  \l 1EE5D
  \l 1EE5F
  \l 1EE61
  \l 1EE62
  \l 1EE64
  \l 1EE67
  \l 1EE68
  \l 1EE69
  \l 1EE6A
  \l 1EE6C
  \l 1EE6D
  \l 1EE6E
  \l 1EE6F
  \l 1EE70
  \l 1EE71
  \l 1EE72
  \l 1EE74
  \l 1EE75
  \l 1EE76
  \l 1EE77
  \l 1EE79
  \l 1EE7A
  \l 1EE7B
  \l 1EE7C
  \l 1EE7E
  \l 1EE80
  \l 1EE81
  \l 1EE82
  \l 1EE83
  \l 1EE84
  \l 1EE85
  \l 1EE86
  \l 1EE87
  \l 1EE88
  \l 1EE89
  \l 1EE8B
  \l 1EE8C
  \l 1EE8D
  \l 1EE8E
  \l 1EE8F
  \l 1EE90
  \l 1EE91
  \l 1EE92
  \l 1EE93
  \l 1EE94
  \l 1EE95
  \l 1EE96
  \l 1EE97
  \l 1EE98
  \l 1EE99
  \l 1EE9A
  \l 1EE9B
  \l 1EEA1
  \l 1EEA2
  \l 1EEA3
  \l 1EEA5
  \l 1EEA6
  \l 1EEA7
  \l 1EEA8
  \l 1EEA9
  \l 1EEAB
  \l 1EEAC
  \l 1EEAD
  \l 1EEAE
  \l 1EEAF
  \l 1EEB0
  \l 1EEB1
  \l 1EEB2
  \l 1EEB3
  \l 1EEB4
  \l 1EEB5
  \l 1EEB6
  \l 1EEB7
  \l 1EEB8
  \l 1EEB9
  \l 1EEBA
  \l 1EEBB
  \l 20000
  \l 20001
  \l 20002
  \l 20003
  \l 20004
  \l 20005
  \l 20006
  \l 20007
  \l 20008
  \l 20009
  \l 2000A
  \l 2000B
  \l 2000C
  \l 2000D
  \l 2000E
  \l 2000F
  \l 20010
  \l 20011
  \l 20012
  \l 20013
  \l 20014
  \l 20015
  \l 20016
  \l 20017
  \l 20018
  \l 20019
  \l 2001A
  \l 2001B
  \l 2001C
  \l 2001D
  \l 2001E
  \l 2001F
  \l 20020
  \l 20021
  \l 20022
  \l 20023
  \l 20024
  \l 20025
  \l 20026
  \l 20027
  \l 20028
  \l 20029
  \l 2002A
  \l 2002B
  \l 2002C
  \l 2002D
  \l 2002E
  \l 2002F
  \l 20030
  \l 20031
  \l 20032
  \l 20033
  \l 20034
  \l 20035
  \l 20036
  \l 20037
  \l 20038
  \l 20039
  \l 2003A
  \l 2003B
  \l 2003C
  \l 2003D
  \l 2003E
  \l 2003F
  \l 20040
  \l 20041
  \l 20042
  \l 20043
  \l 20044
  \l 20045
  \l 20046
  \l 20047
  \l 20048
  \l 20049
  \l 2004A
  \l 2004B
  \l 2004C
  \l 2004D
  \l 2004E
  \l 2004F
  \l 20050
  \l 20051
  \l 20052
  \l 20053
  \l 20054
  \l 20055
  \l 20056
  \l 20057
  \l 20058
  \l 20059
  \l 2005A
  \l 2005B
  \l 2005C
  \l 2005D
  \l 2005E
  \l 2005F
  \l 20060
  \l 20061
  \l 20062
  \l 20063
  \l 20064
  \l 20065
  \l 20066
  \l 20067
  \l 20068
  \l 20069
  \l 2006A
  \l 2006B
  \l 2006C
  \l 2006D
  \l 2006E
  \l 2006F
  \l 20070
  \l 20071
  \l 20072
  \l 20073
  \l 20074
  \l 20075
  \l 20076
  \l 20077
  \l 20078
  \l 20079
  \l 2007A
  \l 2007B
  \l 2007C
  \l 2007D
  \l 2007E
  \l 2007F
  \l 20080
  \l 20081
  \l 20082
  \l 20083
  \l 20084
  \l 20085
  \l 20086
  \l 20087
  \l 20088
  \l 20089
  \l 2008A
  \l 2008B
  \l 2008C
  \l 2008D
  \l 2008E
  \l 2008F
  \l 20090
  \l 20091
  \l 20092
  \l 20093
  \l 20094
  \l 20095
  \l 20096
  \l 20097
  \l 20098
  \l 20099
  \l 2009A
  \l 2009B
  \l 2009C
  \l 2009D
  \l 2009E
  \l 2009F
  \l 200A0
  \l 200A1
  \l 200A2
  \l 200A3
  \l 200A4
  \l 200A5
  \l 200A6
  \l 200A7
  \l 200A8
  \l 200A9
  \l 200AA
  \l 200AB
  \l 200AC
  \l 200AD
  \l 200AE
  \l 200AF
  \l 200B0
  \l 200B1
  \l 200B2
  \l 200B3
  \l 200B4
  \l 200B5
  \l 200B6
  \l 200B7
  \l 200B8
  \l 200B9
  \l 200BA
  \l 200BB
  \l 200BC
  \l 200BD
  \l 200BE
  \l 200BF
  \l 200C0
  \l 200C1
  \l 200C2
  \l 200C3
  \l 200C4
  \l 200C5
  \l 200C6
  \l 200C7
  \l 200C8
  \l 200C9
  \l 200CA
  \l 200CB
  \l 200CC
  \l 200CD
  \l 200CE
  \l 200CF
  \l 200D0
  \l 200D1
  \l 200D2
  \l 200D3
  \l 200D4
  \l 200D5
  \l 200D6
  \l 200D7
  \l 200D8
  \l 200D9
  \l 200DA
  \l 200DB
  \l 200DC
  \l 200DD
  \l 200DE
  \l 200DF
  \l 200E0
  \l 200E1
  \l 200E2
  \l 200E3
  \l 200E4
  \l 200E5
  \l 200E6
  \l 200E7
  \l 200E8
  \l 200E9
  \l 200EA
  \l 200EB
  \l 200EC
  \l 200ED
  \l 200EE
  \l 200EF
  \l 200F0
  \l 200F1
  \l 200F2
  \l 200F3
  \l 200F4
  \l 200F5
  \l 200F6
  \l 200F7
  \l 200F8
  \l 200F9
  \l 200FA
  \l 200FB
  \l 200FC
  \l 200FD
  \l 200FE
  \l 200FF
  \l 20100
  \l 20101
  \l 20102
  \l 20103
  \l 20104
  \l 20105
  \l 20106
  \l 20107
  \l 20108
  \l 20109
  \l 2010A
  \l 2010B
  \l 2010C
  \l 2010D
  \l 2010E
  \l 2010F
  \l 20110
  \l 20111
  \l 20112
  \l 20113
  \l 20114
  \l 20115
  \l 20116
  \l 20117
  \l 20118
  \l 20119
  \l 2011A
  \l 2011B
  \l 2011C
  \l 2011D
  \l 2011E
  \l 2011F
  \l 20120
  \l 20121
  \l 20122
  \l 20123
  \l 20124
  \l 20125
  \l 20126
  \l 20127
  \l 20128
  \l 20129
  \l 2012A
  \l 2012B
  \l 2012C
  \l 2012D
  \l 2012E
  \l 2012F
  \l 20130
  \l 20131
  \l 20132
  \l 20133
  \l 20134
  \l 20135
  \l 20136
  \l 20137
  \l 20138
  \l 20139
  \l 2013A
  \l 2013B
  \l 2013C
  \l 2013D
  \l 2013E
  \l 2013F
  \l 20140
  \l 20141
  \l 20142
  \l 20143
  \l 20144
  \l 20145
  \l 20146
  \l 20147
  \l 20148
  \l 20149
  \l 2014A
  \l 2014B
  \l 2014C
  \l 2014D
  \l 2014E
  \l 2014F
  \l 20150
  \l 20151
  \l 20152
  \l 20153
  \l 20154
  \l 20155
  \l 20156
  \l 20157
  \l 20158
  \l 20159
  \l 2015A
  \l 2015B
  \l 2015C
  \l 2015D
  \l 2015E
  \l 2015F
  \l 20160
  \l 20161
  \l 20162
  \l 20163
  \l 20164
  \l 20165
  \l 20166
  \l 20167
  \l 20168
  \l 20169
  \l 2016A
  \l 2016B
  \l 2016C
  \l 2016D
  \l 2016E
  \l 2016F
  \l 20170
  \l 20171
  \l 20172
  \l 20173
  \l 20174
  \l 20175
  \l 20176
  \l 20177
  \l 20178
  \l 20179
  \l 2017A
  \l 2017B
  \l 2017C
  \l 2017D
  \l 2017E
  \l 2017F
  \l 20180
  \l 20181
  \l 20182
  \l 20183
  \l 20184
  \l 20185
  \l 20186
  \l 20187
  \l 20188
  \l 20189
  \l 2018A
  \l 2018B
  \l 2018C
  \l 2018D
  \l 2018E
  \l 2018F
  \l 20190
  \l 20191
  \l 20192
  \l 20193
  \l 20194
  \l 20195
  \l 20196
  \l 20197
  \l 20198
  \l 20199
  \l 2019A
  \l 2019B
  \l 2019C
  \l 2019D
  \l 2019E
  \l 2019F
  \l 201A0
  \l 201A1
  \l 201A2
  \l 201A3
  \l 201A4
  \l 201A5
  \l 201A6
  \l 201A7
  \l 201A8
  \l 201A9
  \l 201AA
  \l 201AB
  \l 201AC
  \l 201AD
  \l 201AE
  \l 201AF
  \l 201B0
  \l 201B1
  \l 201B2
  \l 201B3
  \l 201B4
  \l 201B5
  \l 201B6
  \l 201B7
  \l 201B8
  \l 201B9
  \l 201BA
  \l 201BB
  \l 201BC
  \l 201BD
  \l 201BE
  \l 201BF
  \l 201C0
  \l 201C1
  \l 201C2
  \l 201C3
  \l 201C4
  \l 201C5
  \l 201C6
  \l 201C7
  \l 201C8
  \l 201C9
  \l 201CA
  \l 201CB
  \l 201CC
  \l 201CD
  \l 201CE
  \l 201CF
  \l 201D0
  \l 201D1
  \l 201D2
  \l 201D3
  \l 201D4
  \l 201D5
  \l 201D6
  \l 201D7
  \l 201D8
  \l 201D9
  \l 201DA
  \l 201DB
  \l 201DC
  \l 201DD
  \l 201DE
  \l 201DF
  \l 201E0
  \l 201E1
  \l 201E2
  \l 201E3
  \l 201E4
  \l 201E5
  \l 201E6
  \l 201E7
  \l 201E8
  \l 201E9
  \l 201EA
  \l 201EB
  \l 201EC
  \l 201ED
  \l 201EE
  \l 201EF
  \l 201F0
  \l 201F1
  \l 201F2
  \l 201F3
  \l 201F4
  \l 201F5
  \l 201F6
  \l 201F7
  \l 201F8
  \l 201F9
  \l 201FA
  \l 201FB
  \l 201FC
  \l 201FD
  \l 201FE
  \l 201FF
  \l 20200
  \l 20201
  \l 20202
  \l 20203
  \l 20204
  \l 20205
  \l 20206
  \l 20207
  \l 20208
  \l 20209
  \l 2020A
  \l 2020B
  \l 2020C
  \l 2020D
  \l 2020E
  \l 2020F
  \l 20210
  \l 20211
  \l 20212
  \l 20213
  \l 20214
  \l 20215
  \l 20216
  \l 20217
  \l 20218
  \l 20219
  \l 2021A
  \l 2021B
  \l 2021C
  \l 2021D
  \l 2021E
  \l 2021F
  \l 20220
  \l 20221
  \l 20222
  \l 20223
  \l 20224
  \l 20225
  \l 20226
  \l 20227
  \l 20228
  \l 20229
  \l 2022A
  \l 2022B
  \l 2022C
  \l 2022D
  \l 2022E
  \l 2022F
  \l 20230
  \l 20231
  \l 20232
  \l 20233
  \l 20234
  \l 20235
  \l 20236
  \l 20237
  \l 20238
  \l 20239
  \l 2023A
  \l 2023B
  \l 2023C
  \l 2023D
  \l 2023E
  \l 2023F
  \l 20240
  \l 20241
  \l 20242
  \l 20243
  \l 20244
  \l 20245
  \l 20246
  \l 20247
  \l 20248
  \l 20249
  \l 2024A
  \l 2024B
  \l 2024C
  \l 2024D
  \l 2024E
  \l 2024F
  \l 20250
  \l 20251
  \l 20252
  \l 20253
  \l 20254
  \l 20255
  \l 20256
  \l 20257
  \l 20258
  \l 20259
  \l 2025A
  \l 2025B
  \l 2025C
  \l 2025D
  \l 2025E
  \l 2025F
  \l 20260
  \l 20261
  \l 20262
  \l 20263
  \l 20264
  \l 20265
  \l 20266
  \l 20267
  \l 20268
  \l 20269
  \l 2026A
  \l 2026B
  \l 2026C
  \l 2026D
  \l 2026E
  \l 2026F
  \l 20270
  \l 20271
  \l 20272
  \l 20273
  \l 20274
  \l 20275
  \l 20276
  \l 20277
  \l 20278
  \l 20279
  \l 2027A
  \l 2027B
  \l 2027C
  \l 2027D
  \l 2027E
  \l 2027F
  \l 20280
  \l 20281
  \l 20282
  \l 20283
  \l 20284
  \l 20285
  \l 20286
  \l 20287
  \l 20288
  \l 20289
  \l 2028A
  \l 2028B
  \l 2028C
  \l 2028D
  \l 2028E
  \l 2028F
  \l 20290
  \l 20291
  \l 20292
  \l 20293
  \l 20294
  \l 20295
  \l 20296
  \l 20297
  \l 20298
  \l 20299
  \l 2029A
  \l 2029B
  \l 2029C
  \l 2029D
  \l 2029E
  \l 2029F
  \l 202A0
  \l 202A1
  \l 202A2
  \l 202A3
  \l 202A4
  \l 202A5
  \l 202A6
  \l 202A7
  \l 202A8
  \l 202A9
  \l 202AA
  \l 202AB
  \l 202AC
  \l 202AD
  \l 202AE
  \l 202AF
  \l 202B0
  \l 202B1
  \l 202B2
  \l 202B3
  \l 202B4
  \l 202B5
  \l 202B6
  \l 202B7
  \l 202B8
  \l 202B9
  \l 202BA
  \l 202BB
  \l 202BC
  \l 202BD
  \l 202BE
  \l 202BF
  \l 202C0
  \l 202C1
  \l 202C2
  \l 202C3
  \l 202C4
  \l 202C5
  \l 202C6
  \l 202C7
  \l 202C8
  \l 202C9
  \l 202CA
  \l 202CB
  \l 202CC
  \l 202CD
  \l 202CE
  \l 202CF
  \l 202D0
  \l 202D1
  \l 202D2
  \l 202D3
  \l 202D4
  \l 202D5
  \l 202D6
  \l 202D7
  \l 202D8
  \l 202D9
  \l 202DA
  \l 202DB
  \l 202DC
  \l 202DD
  \l 202DE
  \l 202DF
  \l 202E0
  \l 202E1
  \l 202E2
  \l 202E3
  \l 202E4
  \l 202E5
  \l 202E6
  \l 202E7
  \l 202E8
  \l 202E9
  \l 202EA
  \l 202EB
  \l 202EC
  \l 202ED
  \l 202EE
  \l 202EF
  \l 202F0
  \l 202F1
  \l 202F2
  \l 202F3
  \l 202F4
  \l 202F5
  \l 202F6
  \l 202F7
  \l 202F8
  \l 202F9
  \l 202FA
  \l 202FB
  \l 202FC
  \l 202FD
  \l 202FE
  \l 202FF
  \l 20300
  \l 20301
  \l 20302
  \l 20303
  \l 20304
  \l 20305
  \l 20306
  \l 20307
  \l 20308
  \l 20309
  \l 2030A
  \l 2030B
  \l 2030C
  \l 2030D
  \l 2030E
  \l 2030F
  \l 20310
  \l 20311
  \l 20312
  \l 20313
  \l 20314
  \l 20315
  \l 20316
  \l 20317
  \l 20318
  \l 20319
  \l 2031A
  \l 2031B
  \l 2031C
  \l 2031D
  \l 2031E
  \l 2031F
  \l 20320
  \l 20321
  \l 20322
  \l 20323
  \l 20324
  \l 20325
  \l 20326
  \l 20327
  \l 20328
  \l 20329
  \l 2032A
  \l 2032B
  \l 2032C
  \l 2032D
  \l 2032E
  \l 2032F
  \l 20330
  \l 20331
  \l 20332
  \l 20333
  \l 20334
  \l 20335
  \l 20336
  \l 20337
  \l 20338
  \l 20339
  \l 2033A
  \l 2033B
  \l 2033C
  \l 2033D
  \l 2033E
  \l 2033F
  \l 20340
  \l 20341
  \l 20342
  \l 20343
  \l 20344
  \l 20345
  \l 20346
  \l 20347
  \l 20348
  \l 20349
  \l 2034A
  \l 2034B
  \l 2034C
  \l 2034D
  \l 2034E
  \l 2034F
  \l 20350
  \l 20351
  \l 20352
  \l 20353
  \l 20354
  \l 20355
  \l 20356
  \l 20357
  \l 20358
  \l 20359
  \l 2035A
  \l 2035B
  \l 2035C
  \l 2035D
  \l 2035E
  \l 2035F
  \l 20360
  \l 20361
  \l 20362
  \l 20363
  \l 20364
  \l 20365
  \l 20366
  \l 20367
  \l 20368
  \l 20369
  \l 2036A
  \l 2036B
  \l 2036C
  \l 2036D
  \l 2036E
  \l 2036F
  \l 20370
  \l 20371
  \l 20372
  \l 20373
  \l 20374
  \l 20375
  \l 20376
  \l 20377
  \l 20378
  \l 20379
  \l 2037A
  \l 2037B
  \l 2037C
  \l 2037D
  \l 2037E
  \l 2037F
  \l 20380
  \l 20381
  \l 20382
  \l 20383
  \l 20384
  \l 20385
  \l 20386
  \l 20387
  \l 20388
  \l 20389
  \l 2038A
  \l 2038B
  \l 2038C
  \l 2038D
  \l 2038E
  \l 2038F
  \l 20390
  \l 20391
  \l 20392
  \l 20393
  \l 20394
  \l 20395
  \l 20396
  \l 20397
  \l 20398
  \l 20399
  \l 2039A
  \l 2039B
  \l 2039C
  \l 2039D
  \l 2039E
  \l 2039F
  \l 203A0
  \l 203A1
  \l 203A2
  \l 203A3
  \l 203A4
  \l 203A5
  \l 203A6
  \l 203A7
  \l 203A8
  \l 203A9
  \l 203AA
  \l 203AB
  \l 203AC
  \l 203AD
  \l 203AE
  \l 203AF
  \l 203B0
  \l 203B1
  \l 203B2
  \l 203B3
  \l 203B4
  \l 203B5
  \l 203B6
  \l 203B7
  \l 203B8
  \l 203B9
  \l 203BA
  \l 203BB
  \l 203BC
  \l 203BD
  \l 203BE
  \l 203BF
  \l 203C0
  \l 203C1
  \l 203C2
  \l 203C3
  \l 203C4
  \l 203C5
  \l 203C6
  \l 203C7
  \l 203C8
  \l 203C9
  \l 203CA
  \l 203CB
  \l 203CC
  \l 203CD
  \l 203CE
  \l 203CF
  \l 203D0
  \l 203D1
  \l 203D2
  \l 203D3
  \l 203D4
  \l 203D5
  \l 203D6
  \l 203D7
  \l 203D8
  \l 203D9
  \l 203DA
  \l 203DB
  \l 203DC
  \l 203DD
  \l 203DE
  \l 203DF
  \l 203E0
  \l 203E1
  \l 203E2
  \l 203E3
  \l 203E4
  \l 203E5
  \l 203E6
  \l 203E7
  \l 203E8
  \l 203E9
  \l 203EA
  \l 203EB
  \l 203EC
  \l 203ED
  \l 203EE
  \l 203EF
  \l 203F0
  \l 203F1
  \l 203F2
  \l 203F3
  \l 203F4
  \l 203F5
  \l 203F6
  \l 203F7
  \l 203F8
  \l 203F9
  \l 203FA
  \l 203FB
  \l 203FC
  \l 203FD
  \l 203FE
  \l 203FF
  \l 20400
  \l 20401
  \l 20402
  \l 20403
  \l 20404
  \l 20405
  \l 20406
  \l 20407
  \l 20408
  \l 20409
  \l 2040A
  \l 2040B
  \l 2040C
  \l 2040D
  \l 2040E
  \l 2040F
  \l 20410
  \l 20411
  \l 20412
  \l 20413
  \l 20414
  \l 20415
  \l 20416
  \l 20417
  \l 20418
  \l 20419
  \l 2041A
  \l 2041B
  \l 2041C
  \l 2041D
  \l 2041E
  \l 2041F
  \l 20420
  \l 20421
  \l 20422
  \l 20423
  \l 20424
  \l 20425
  \l 20426
  \l 20427
  \l 20428
  \l 20429
  \l 2042A
  \l 2042B
  \l 2042C
  \l 2042D
  \l 2042E
  \l 2042F
  \l 20430
  \l 20431
  \l 20432
  \l 20433
  \l 20434
  \l 20435
  \l 20436
  \l 20437
  \l 20438
  \l 20439
  \l 2043A
  \l 2043B
  \l 2043C
  \l 2043D
  \l 2043E
  \l 2043F
  \l 20440
  \l 20441
  \l 20442
  \l 20443
  \l 20444
  \l 20445
  \l 20446
  \l 20447
  \l 20448
  \l 20449
  \l 2044A
  \l 2044B
  \l 2044C
  \l 2044D
  \l 2044E
  \l 2044F
  \l 20450
  \l 20451
  \l 20452
  \l 20453
  \l 20454
  \l 20455
  \l 20456
  \l 20457
  \l 20458
  \l 20459
  \l 2045A
  \l 2045B
  \l 2045C
  \l 2045D
  \l 2045E
  \l 2045F
  \l 20460
  \l 20461
  \l 20462
  \l 20463
  \l 20464
  \l 20465
  \l 20466
  \l 20467
  \l 20468
  \l 20469
  \l 2046A
  \l 2046B
  \l 2046C
  \l 2046D
  \l 2046E
  \l 2046F
  \l 20470
  \l 20471
  \l 20472
  \l 20473
  \l 20474
  \l 20475
  \l 20476
  \l 20477
  \l 20478
  \l 20479
  \l 2047A
  \l 2047B
  \l 2047C
  \l 2047D
  \l 2047E
  \l 2047F
  \l 20480
  \l 20481
  \l 20482
  \l 20483
  \l 20484
  \l 20485
  \l 20486
  \l 20487
  \l 20488
  \l 20489
  \l 2048A
  \l 2048B
  \l 2048C
  \l 2048D
  \l 2048E
  \l 2048F
  \l 20490
  \l 20491
  \l 20492
  \l 20493
  \l 20494
  \l 20495
  \l 20496
  \l 20497
  \l 20498
  \l 20499
  \l 2049A
  \l 2049B
  \l 2049C
  \l 2049D
  \l 2049E
  \l 2049F
  \l 204A0
  \l 204A1
  \l 204A2
  \l 204A3
  \l 204A4
  \l 204A5
  \l 204A6
  \l 204A7
  \l 204A8
  \l 204A9
  \l 204AA
  \l 204AB
  \l 204AC
  \l 204AD
  \l 204AE
  \l 204AF
  \l 204B0
  \l 204B1
  \l 204B2
  \l 204B3
  \l 204B4
  \l 204B5
  \l 204B6
  \l 204B7
  \l 204B8
  \l 204B9
  \l 204BA
  \l 204BB
  \l 204BC
  \l 204BD
  \l 204BE
  \l 204BF
  \l 204C0
  \l 204C1
  \l 204C2
  \l 204C3
  \l 204C4
  \l 204C5
  \l 204C6
  \l 204C7
  \l 204C8
  \l 204C9
  \l 204CA
  \l 204CB
  \l 204CC
  \l 204CD
  \l 204CE
  \l 204CF
  \l 204D0
  \l 204D1
  \l 204D2
  \l 204D3
  \l 204D4
  \l 204D5
  \l 204D6
  \l 204D7
  \l 204D8
  \l 204D9
  \l 204DA
  \l 204DB
  \l 204DC
  \l 204DD
  \l 204DE
  \l 204DF
  \l 204E0
  \l 204E1
  \l 204E2
  \l 204E3
  \l 204E4
  \l 204E5
  \l 204E6
  \l 204E7
  \l 204E8
  \l 204E9
  \l 204EA
  \l 204EB
  \l 204EC
  \l 204ED
  \l 204EE
  \l 204EF
  \l 204F0
  \l 204F1
  \l 204F2
  \l 204F3
  \l 204F4
  \l 204F5
  \l 204F6
  \l 204F7
  \l 204F8
  \l 204F9
  \l 204FA
  \l 204FB
  \l 204FC
  \l 204FD
  \l 204FE
  \l 204FF
  \l 20500
  \l 20501
  \l 20502
  \l 20503
  \l 20504
  \l 20505
  \l 20506
  \l 20507
  \l 20508
  \l 20509
  \l 2050A
  \l 2050B
  \l 2050C
  \l 2050D
  \l 2050E
  \l 2050F
  \l 20510
  \l 20511
  \l 20512
  \l 20513
  \l 20514
  \l 20515
  \l 20516
  \l 20517
  \l 20518
  \l 20519
  \l 2051A
  \l 2051B
  \l 2051C
  \l 2051D
  \l 2051E
  \l 2051F
  \l 20520
  \l 20521
  \l 20522
  \l 20523
  \l 20524
  \l 20525
  \l 20526
  \l 20527
  \l 20528
  \l 20529
  \l 2052A
  \l 2052B
  \l 2052C
  \l 2052D
  \l 2052E
  \l 2052F
  \l 20530
  \l 20531
  \l 20532
  \l 20533
  \l 20534
  \l 20535
  \l 20536
  \l 20537
  \l 20538
  \l 20539
  \l 2053A
  \l 2053B
  \l 2053C
  \l 2053D
  \l 2053E
  \l 2053F
  \l 20540
  \l 20541
  \l 20542
  \l 20543
  \l 20544
  \l 20545
  \l 20546
  \l 20547
  \l 20548
  \l 20549
  \l 2054A
  \l 2054B
  \l 2054C
  \l 2054D
  \l 2054E
  \l 2054F
  \l 20550
  \l 20551
  \l 20552
  \l 20553
  \l 20554
  \l 20555
  \l 20556
  \l 20557
  \l 20558
  \l 20559
  \l 2055A
  \l 2055B
  \l 2055C
  \l 2055D
  \l 2055E
  \l 2055F
  \l 20560
  \l 20561
  \l 20562
  \l 20563
  \l 20564
  \l 20565
  \l 20566
  \l 20567
  \l 20568
  \l 20569
  \l 2056A
  \l 2056B
  \l 2056C
  \l 2056D
  \l 2056E
  \l 2056F
  \l 20570
  \l 20571
  \l 20572
  \l 20573
  \l 20574
  \l 20575
  \l 20576
  \l 20577
  \l 20578
  \l 20579
  \l 2057A
  \l 2057B
  \l 2057C
  \l 2057D
  \l 2057E
  \l 2057F
  \l 20580
  \l 20581
  \l 20582
  \l 20583
  \l 20584
  \l 20585
  \l 20586
  \l 20587
  \l 20588
  \l 20589
  \l 2058A
  \l 2058B
  \l 2058C
  \l 2058D
  \l 2058E
  \l 2058F
  \l 20590
  \l 20591
  \l 20592
  \l 20593
  \l 20594
  \l 20595
  \l 20596
  \l 20597
  \l 20598
  \l 20599
  \l 2059A
  \l 2059B
  \l 2059C
  \l 2059D
  \l 2059E
  \l 2059F
  \l 205A0
  \l 205A1
  \l 205A2
  \l 205A3
  \l 205A4
  \l 205A5
  \l 205A6
  \l 205A7
  \l 205A8
  \l 205A9
  \l 205AA
  \l 205AB
  \l 205AC
  \l 205AD
  \l 205AE
  \l 205AF
  \l 205B0
  \l 205B1
  \l 205B2
  \l 205B3
  \l 205B4
  \l 205B5
  \l 205B6
  \l 205B7
  \l 205B8
  \l 205B9
  \l 205BA
  \l 205BB
  \l 205BC
  \l 205BD
  \l 205BE
  \l 205BF
  \l 205C0
  \l 205C1
  \l 205C2
  \l 205C3
  \l 205C4
  \l 205C5
  \l 205C6
  \l 205C7
  \l 205C8
  \l 205C9
  \l 205CA
  \l 205CB
  \l 205CC
  \l 205CD
  \l 205CE
  \l 205CF
  \l 205D0
  \l 205D1
  \l 205D2
  \l 205D3
  \l 205D4
  \l 205D5
  \l 205D6
  \l 205D7
  \l 205D8
  \l 205D9
  \l 205DA
  \l 205DB
  \l 205DC
  \l 205DD
  \l 205DE
  \l 205DF
  \l 205E0
  \l 205E1
  \l 205E2
  \l 205E3
  \l 205E4
  \l 205E5
  \l 205E6
  \l 205E7
  \l 205E8
  \l 205E9
  \l 205EA
  \l 205EB
  \l 205EC
  \l 205ED
  \l 205EE
  \l 205EF
  \l 205F0
  \l 205F1
  \l 205F2
  \l 205F3
  \l 205F4
  \l 205F5
  \l 205F6
  \l 205F7
  \l 205F8
  \l 205F9
  \l 205FA
  \l 205FB
  \l 205FC
  \l 205FD
  \l 205FE
  \l 205FF
  \l 20600
  \l 20601
  \l 20602
  \l 20603
  \l 20604
  \l 20605
  \l 20606
  \l 20607
  \l 20608
  \l 20609
  \l 2060A
  \l 2060B
  \l 2060C
  \l 2060D
  \l 2060E
  \l 2060F
  \l 20610
  \l 20611
  \l 20612
  \l 20613
  \l 20614
  \l 20615
  \l 20616
  \l 20617
  \l 20618
  \l 20619
  \l 2061A
  \l 2061B
  \l 2061C
  \l 2061D
  \l 2061E
  \l 2061F
  \l 20620
  \l 20621
  \l 20622
  \l 20623
  \l 20624
  \l 20625
  \l 20626
  \l 20627
  \l 20628
  \l 20629
  \l 2062A
  \l 2062B
  \l 2062C
  \l 2062D
  \l 2062E
  \l 2062F
  \l 20630
  \l 20631
  \l 20632
  \l 20633
  \l 20634
  \l 20635
  \l 20636
  \l 20637
  \l 20638
  \l 20639
  \l 2063A
  \l 2063B
  \l 2063C
  \l 2063D
  \l 2063E
  \l 2063F
  \l 20640
  \l 20641
  \l 20642
  \l 20643
  \l 20644
  \l 20645
  \l 20646
  \l 20647
  \l 20648
  \l 20649
  \l 2064A
  \l 2064B
  \l 2064C
  \l 2064D
  \l 2064E
  \l 2064F
  \l 20650
  \l 20651
  \l 20652
  \l 20653
  \l 20654
  \l 20655
  \l 20656
  \l 20657
  \l 20658
  \l 20659
  \l 2065A
  \l 2065B
  \l 2065C
  \l 2065D
  \l 2065E
  \l 2065F
  \l 20660
  \l 20661
  \l 20662
  \l 20663
  \l 20664
  \l 20665
  \l 20666
  \l 20667
  \l 20668
  \l 20669
  \l 2066A
  \l 2066B
  \l 2066C
  \l 2066D
  \l 2066E
  \l 2066F
  \l 20670
  \l 20671
  \l 20672
  \l 20673
  \l 20674
  \l 20675
  \l 20676
  \l 20677
  \l 20678
  \l 20679
  \l 2067A
  \l 2067B
  \l 2067C
  \l 2067D
  \l 2067E
  \l 2067F
  \l 20680
  \l 20681
  \l 20682
  \l 20683
  \l 20684
  \l 20685
  \l 20686
  \l 20687
  \l 20688
  \l 20689
  \l 2068A
  \l 2068B
  \l 2068C
  \l 2068D
  \l 2068E
  \l 2068F
  \l 20690
  \l 20691
  \l 20692
  \l 20693
  \l 20694
  \l 20695
  \l 20696
  \l 20697
  \l 20698
  \l 20699
  \l 2069A
  \l 2069B
  \l 2069C
  \l 2069D
  \l 2069E
  \l 2069F
  \l 206A0
  \l 206A1
  \l 206A2
  \l 206A3
  \l 206A4
  \l 206A5
  \l 206A6
  \l 206A7
  \l 206A8
  \l 206A9
  \l 206AA
  \l 206AB
  \l 206AC
  \l 206AD
  \l 206AE
  \l 206AF
  \l 206B0
  \l 206B1
  \l 206B2
  \l 206B3
  \l 206B4
  \l 206B5
  \l 206B6
  \l 206B7
  \l 206B8
  \l 206B9
  \l 206BA
  \l 206BB
  \l 206BC
  \l 206BD
  \l 206BE
  \l 206BF
  \l 206C0
  \l 206C1
  \l 206C2
  \l 206C3
  \l 206C4
  \l 206C5
  \l 206C6
  \l 206C7
  \l 206C8
  \l 206C9
  \l 206CA
  \l 206CB
  \l 206CC
  \l 206CD
  \l 206CE
  \l 206CF
  \l 206D0
  \l 206D1
  \l 206D2
  \l 206D3
  \l 206D4
  \l 206D5
  \l 206D6
  \l 206D7
  \l 206D8
  \l 206D9
  \l 206DA
  \l 206DB
  \l 206DC
  \l 206DD
  \l 206DE
  \l 206DF
  \l 206E0
  \l 206E1
  \l 206E2
  \l 206E3
  \l 206E4
  \l 206E5
  \l 206E6
  \l 206E7
  \l 206E8
  \l 206E9
  \l 206EA
  \l 206EB
  \l 206EC
  \l 206ED
  \l 206EE
  \l 206EF
  \l 206F0
  \l 206F1
  \l 206F2
  \l 206F3
  \l 206F4
  \l 206F5
  \l 206F6
  \l 206F7
  \l 206F8
  \l 206F9
  \l 206FA
  \l 206FB
  \l 206FC
  \l 206FD
  \l 206FE
  \l 206FF
  \l 20700
  \l 20701
  \l 20702
  \l 20703
  \l 20704
  \l 20705
  \l 20706
  \l 20707
  \l 20708
  \l 20709
  \l 2070A
  \l 2070B
  \l 2070C
  \l 2070D
  \l 2070E
  \l 2070F
  \l 20710
  \l 20711
  \l 20712
  \l 20713
  \l 20714
  \l 20715
  \l 20716
  \l 20717
  \l 20718
  \l 20719
  \l 2071A
  \l 2071B
  \l 2071C
  \l 2071D
  \l 2071E
  \l 2071F
  \l 20720
  \l 20721
  \l 20722
  \l 20723
  \l 20724
  \l 20725
  \l 20726
  \l 20727
  \l 20728
  \l 20729
  \l 2072A
  \l 2072B
  \l 2072C
  \l 2072D
  \l 2072E
  \l 2072F
  \l 20730
  \l 20731
  \l 20732
  \l 20733
  \l 20734
  \l 20735
  \l 20736
  \l 20737
  \l 20738
  \l 20739
  \l 2073A
  \l 2073B
  \l 2073C
  \l 2073D
  \l 2073E
  \l 2073F
  \l 20740
  \l 20741
  \l 20742
  \l 20743
  \l 20744
  \l 20745
  \l 20746
  \l 20747
  \l 20748
  \l 20749
  \l 2074A
  \l 2074B
  \l 2074C
  \l 2074D
  \l 2074E
  \l 2074F
  \l 20750
  \l 20751
  \l 20752
  \l 20753
  \l 20754
  \l 20755
  \l 20756
  \l 20757
  \l 20758
  \l 20759
  \l 2075A
  \l 2075B
  \l 2075C
  \l 2075D
  \l 2075E
  \l 2075F
  \l 20760
  \l 20761
  \l 20762
  \l 20763
  \l 20764
  \l 20765
  \l 20766
  \l 20767
  \l 20768
  \l 20769
  \l 2076A
  \l 2076B
  \l 2076C
  \l 2076D
  \l 2076E
  \l 2076F
  \l 20770
  \l 20771
  \l 20772
  \l 20773
  \l 20774
  \l 20775
  \l 20776
  \l 20777
  \l 20778
  \l 20779
  \l 2077A
  \l 2077B
  \l 2077C
  \l 2077D
  \l 2077E
  \l 2077F
  \l 20780
  \l 20781
  \l 20782
  \l 20783
  \l 20784
  \l 20785
  \l 20786
  \l 20787
  \l 20788
  \l 20789
  \l 2078A
  \l 2078B
  \l 2078C
  \l 2078D
  \l 2078E
  \l 2078F
  \l 20790
  \l 20791
  \l 20792
  \l 20793
  \l 20794
  \l 20795
  \l 20796
  \l 20797
  \l 20798
  \l 20799
  \l 2079A
  \l 2079B
  \l 2079C
  \l 2079D
  \l 2079E
  \l 2079F
  \l 207A0
  \l 207A1
  \l 207A2
  \l 207A3
  \l 207A4
  \l 207A5
  \l 207A6
  \l 207A7
  \l 207A8
  \l 207A9
  \l 207AA
  \l 207AB
  \l 207AC
  \l 207AD
  \l 207AE
  \l 207AF
  \l 207B0
  \l 207B1
  \l 207B2
  \l 207B3
  \l 207B4
  \l 207B5
  \l 207B6
  \l 207B7
  \l 207B8
  \l 207B9
  \l 207BA
  \l 207BB
  \l 207BC
  \l 207BD
  \l 207BE
  \l 207BF
  \l 207C0
  \l 207C1
  \l 207C2
  \l 207C3
  \l 207C4
  \l 207C5
  \l 207C6
  \l 207C7
  \l 207C8
  \l 207C9
  \l 207CA
  \l 207CB
  \l 207CC
  \l 207CD
  \l 207CE
  \l 207CF
  \l 207D0
  \l 207D1
  \l 207D2
  \l 207D3
  \l 207D4
  \l 207D5
  \l 207D6
  \l 207D7
  \l 207D8
  \l 207D9
  \l 207DA
  \l 207DB
  \l 207DC
  \l 207DD
  \l 207DE
  \l 207DF
  \l 207E0
  \l 207E1
  \l 207E2
  \l 207E3
  \l 207E4
  \l 207E5
  \l 207E6
  \l 207E7
  \l 207E8
  \l 207E9
  \l 207EA
  \l 207EB
  \l 207EC
  \l 207ED
  \l 207EE
  \l 207EF
  \l 207F0
  \l 207F1
  \l 207F2
  \l 207F3
  \l 207F4
  \l 207F5
  \l 207F6
  \l 207F7
  \l 207F8
  \l 207F9
  \l 207FA
  \l 207FB
  \l 207FC
  \l 207FD
  \l 207FE
  \l 207FF
  \l 20800
  \l 20801
  \l 20802
  \l 20803
  \l 20804
  \l 20805
  \l 20806
  \l 20807
  \l 20808
  \l 20809
  \l 2080A
  \l 2080B
  \l 2080C
  \l 2080D
  \l 2080E
  \l 2080F
  \l 20810
  \l 20811
  \l 20812
  \l 20813
  \l 20814
  \l 20815
  \l 20816
  \l 20817
  \l 20818
  \l 20819
  \l 2081A
  \l 2081B
  \l 2081C
  \l 2081D
  \l 2081E
  \l 2081F
  \l 20820
  \l 20821
  \l 20822
  \l 20823
  \l 20824
  \l 20825
  \l 20826
  \l 20827
  \l 20828
  \l 20829
  \l 2082A
  \l 2082B
  \l 2082C
  \l 2082D
  \l 2082E
  \l 2082F
  \l 20830
  \l 20831
  \l 20832
  \l 20833
  \l 20834
  \l 20835
  \l 20836
  \l 20837
  \l 20838
  \l 20839
  \l 2083A
  \l 2083B
  \l 2083C
  \l 2083D
  \l 2083E
  \l 2083F
  \l 20840
  \l 20841
  \l 20842
  \l 20843
  \l 20844
  \l 20845
  \l 20846
  \l 20847
  \l 20848
  \l 20849
  \l 2084A
  \l 2084B
  \l 2084C
  \l 2084D
  \l 2084E
  \l 2084F
  \l 20850
  \l 20851
  \l 20852
  \l 20853
  \l 20854
  \l 20855
  \l 20856
  \l 20857
  \l 20858
  \l 20859
  \l 2085A
  \l 2085B
  \l 2085C
  \l 2085D
  \l 2085E
  \l 2085F
  \l 20860
  \l 20861
  \l 20862
  \l 20863
  \l 20864
  \l 20865
  \l 20866
  \l 20867
  \l 20868
  \l 20869
  \l 2086A
  \l 2086B
  \l 2086C
  \l 2086D
  \l 2086E
  \l 2086F
  \l 20870
  \l 20871
  \l 20872
  \l 20873
  \l 20874
  \l 20875
  \l 20876
  \l 20877
  \l 20878
  \l 20879
  \l 2087A
  \l 2087B
  \l 2087C
  \l 2087D
  \l 2087E
  \l 2087F
  \l 20880
  \l 20881
  \l 20882
  \l 20883
  \l 20884
  \l 20885
  \l 20886
  \l 20887
  \l 20888
  \l 20889
  \l 2088A
  \l 2088B
  \l 2088C
  \l 2088D
  \l 2088E
  \l 2088F
  \l 20890
  \l 20891
  \l 20892
  \l 20893
  \l 20894
  \l 20895
  \l 20896
  \l 20897
  \l 20898
  \l 20899
  \l 2089A
  \l 2089B
  \l 2089C
  \l 2089D
  \l 2089E
  \l 2089F
  \l 208A0
  \l 208A1
  \l 208A2
  \l 208A3
  \l 208A4
  \l 208A5
  \l 208A6
  \l 208A7
  \l 208A8
  \l 208A9
  \l 208AA
  \l 208AB
  \l 208AC
  \l 208AD
  \l 208AE
  \l 208AF
  \l 208B0
  \l 208B1
  \l 208B2
  \l 208B3
  \l 208B4
  \l 208B5
  \l 208B6
  \l 208B7
  \l 208B8
  \l 208B9
  \l 208BA
  \l 208BB
  \l 208BC
  \l 208BD
  \l 208BE
  \l 208BF
  \l 208C0
  \l 208C1
  \l 208C2
  \l 208C3
  \l 208C4
  \l 208C5
  \l 208C6
  \l 208C7
  \l 208C8
  \l 208C9
  \l 208CA
  \l 208CB
  \l 208CC
  \l 208CD
  \l 208CE
  \l 208CF
  \l 208D0
  \l 208D1
  \l 208D2
  \l 208D3
  \l 208D4
  \l 208D5
  \l 208D6
  \l 208D7
  \l 208D8
  \l 208D9
  \l 208DA
  \l 208DB
  \l 208DC
  \l 208DD
  \l 208DE
  \l 208DF
  \l 208E0
  \l 208E1
  \l 208E2
  \l 208E3
  \l 208E4
  \l 208E5
  \l 208E6
  \l 208E7
  \l 208E8
  \l 208E9
  \l 208EA
  \l 208EB
  \l 208EC
  \l 208ED
  \l 208EE
  \l 208EF
  \l 208F0
  \l 208F1
  \l 208F2
  \l 208F3
  \l 208F4
  \l 208F5
  \l 208F6
  \l 208F7
  \l 208F8
  \l 208F9
  \l 208FA
  \l 208FB
  \l 208FC
  \l 208FD
  \l 208FE
  \l 208FF
  \l 20900
  \l 20901
  \l 20902
  \l 20903
  \l 20904
  \l 20905
  \l 20906
  \l 20907
  \l 20908
  \l 20909
  \l 2090A
  \l 2090B
  \l 2090C
  \l 2090D
  \l 2090E
  \l 2090F
  \l 20910
  \l 20911
  \l 20912
  \l 20913
  \l 20914
  \l 20915
  \l 20916
  \l 20917
  \l 20918
  \l 20919
  \l 2091A
  \l 2091B
  \l 2091C
  \l 2091D
  \l 2091E
  \l 2091F
  \l 20920
  \l 20921
  \l 20922
  \l 20923
  \l 20924
  \l 20925
  \l 20926
  \l 20927
  \l 20928
  \l 20929
  \l 2092A
  \l 2092B
  \l 2092C
  \l 2092D
  \l 2092E
  \l 2092F
  \l 20930
  \l 20931
  \l 20932
  \l 20933
  \l 20934
  \l 20935
  \l 20936
  \l 20937
  \l 20938
  \l 20939
  \l 2093A
  \l 2093B
  \l 2093C
  \l 2093D
  \l 2093E
  \l 2093F
  \l 20940
  \l 20941
  \l 20942
  \l 20943
  \l 20944
  \l 20945
  \l 20946
  \l 20947
  \l 20948
  \l 20949
  \l 2094A
  \l 2094B
  \l 2094C
  \l 2094D
  \l 2094E
  \l 2094F
  \l 20950
  \l 20951
  \l 20952
  \l 20953
  \l 20954
  \l 20955
  \l 20956
  \l 20957
  \l 20958
  \l 20959
  \l 2095A
  \l 2095B
  \l 2095C
  \l 2095D
  \l 2095E
  \l 2095F
  \l 20960
  \l 20961
  \l 20962
  \l 20963
  \l 20964
  \l 20965
  \l 20966
  \l 20967
  \l 20968
  \l 20969
  \l 2096A
  \l 2096B
  \l 2096C
  \l 2096D
  \l 2096E
  \l 2096F
  \l 20970
  \l 20971
  \l 20972
  \l 20973
  \l 20974
  \l 20975
  \l 20976
  \l 20977
  \l 20978
  \l 20979
  \l 2097A
  \l 2097B
  \l 2097C
  \l 2097D
  \l 2097E
  \l 2097F
  \l 20980
  \l 20981
  \l 20982
  \l 20983
  \l 20984
  \l 20985
  \l 20986
  \l 20987
  \l 20988
  \l 20989
  \l 2098A
  \l 2098B
  \l 2098C
  \l 2098D
  \l 2098E
  \l 2098F
  \l 20990
  \l 20991
  \l 20992
  \l 20993
  \l 20994
  \l 20995
  \l 20996
  \l 20997
  \l 20998
  \l 20999
  \l 2099A
  \l 2099B
  \l 2099C
  \l 2099D
  \l 2099E
  \l 2099F
  \l 209A0
  \l 209A1
  \l 209A2
  \l 209A3
  \l 209A4
  \l 209A5
  \l 209A6
  \l 209A7
  \l 209A8
  \l 209A9
  \l 209AA
  \l 209AB
  \l 209AC
  \l 209AD
  \l 209AE
  \l 209AF
  \l 209B0
  \l 209B1
  \l 209B2
  \l 209B3
  \l 209B4
  \l 209B5
  \l 209B6
  \l 209B7
  \l 209B8
  \l 209B9
  \l 209BA
  \l 209BB
  \l 209BC
  \l 209BD
  \l 209BE
  \l 209BF
  \l 209C0
  \l 209C1
  \l 209C2
  \l 209C3
  \l 209C4
  \l 209C5
  \l 209C6
  \l 209C7
  \l 209C8
  \l 209C9
  \l 209CA
  \l 209CB
  \l 209CC
  \l 209CD
  \l 209CE
  \l 209CF
  \l 209D0
  \l 209D1
  \l 209D2
  \l 209D3
  \l 209D4
  \l 209D5
  \l 209D6
  \l 209D7
  \l 209D8
  \l 209D9
  \l 209DA
  \l 209DB
  \l 209DC
  \l 209DD
  \l 209DE
  \l 209DF
  \l 209E0
  \l 209E1
  \l 209E2
  \l 209E3
  \l 209E4
  \l 209E5
  \l 209E6
  \l 209E7
  \l 209E8
  \l 209E9
  \l 209EA
  \l 209EB
  \l 209EC
  \l 209ED
  \l 209EE
  \l 209EF
  \l 209F0
  \l 209F1
  \l 209F2
  \l 209F3
  \l 209F4
  \l 209F5
  \l 209F6
  \l 209F7
  \l 209F8
  \l 209F9
  \l 209FA
  \l 209FB
  \l 209FC
  \l 209FD
  \l 209FE
  \l 209FF
  \l 20A00
  \l 20A01
  \l 20A02
  \l 20A03
  \l 20A04
  \l 20A05
  \l 20A06
  \l 20A07
  \l 20A08
  \l 20A09
  \l 20A0A
  \l 20A0B
  \l 20A0C
  \l 20A0D
  \l 20A0E
  \l 20A0F
  \l 20A10
  \l 20A11
  \l 20A12
  \l 20A13
  \l 20A14
  \l 20A15
  \l 20A16
  \l 20A17
  \l 20A18
  \l 20A19
  \l 20A1A
  \l 20A1B
  \l 20A1C
  \l 20A1D
  \l 20A1E
  \l 20A1F
  \l 20A20
  \l 20A21
  \l 20A22
  \l 20A23
  \l 20A24
  \l 20A25
  \l 20A26
  \l 20A27
  \l 20A28
  \l 20A29
  \l 20A2A
  \l 20A2B
  \l 20A2C
  \l 20A2D
  \l 20A2E
  \l 20A2F
  \l 20A30
  \l 20A31
  \l 20A32
  \l 20A33
  \l 20A34
  \l 20A35
  \l 20A36
  \l 20A37
  \l 20A38
  \l 20A39
  \l 20A3A
  \l 20A3B
  \l 20A3C
  \l 20A3D
  \l 20A3E
  \l 20A3F
  \l 20A40
  \l 20A41
  \l 20A42
  \l 20A43
  \l 20A44
  \l 20A45
  \l 20A46
  \l 20A47
  \l 20A48
  \l 20A49
  \l 20A4A
  \l 20A4B
  \l 20A4C
  \l 20A4D
  \l 20A4E
  \l 20A4F
  \l 20A50
  \l 20A51
  \l 20A52
  \l 20A53
  \l 20A54
  \l 20A55
  \l 20A56
  \l 20A57
  \l 20A58
  \l 20A59
  \l 20A5A
  \l 20A5B
  \l 20A5C
  \l 20A5D
  \l 20A5E
  \l 20A5F
  \l 20A60
  \l 20A61
  \l 20A62
  \l 20A63
  \l 20A64
  \l 20A65
  \l 20A66
  \l 20A67
  \l 20A68
  \l 20A69
  \l 20A6A
  \l 20A6B
  \l 20A6C
  \l 20A6D
  \l 20A6E
  \l 20A6F
  \l 20A70
  \l 20A71
  \l 20A72
  \l 20A73
  \l 20A74
  \l 20A75
  \l 20A76
  \l 20A77
  \l 20A78
  \l 20A79
  \l 20A7A
  \l 20A7B
  \l 20A7C
  \l 20A7D
  \l 20A7E
  \l 20A7F
  \l 20A80
  \l 20A81
  \l 20A82
  \l 20A83
  \l 20A84
  \l 20A85
  \l 20A86
  \l 20A87
  \l 20A88
  \l 20A89
  \l 20A8A
  \l 20A8B
  \l 20A8C
  \l 20A8D
  \l 20A8E
  \l 20A8F
  \l 20A90
  \l 20A91
  \l 20A92
  \l 20A93
  \l 20A94
  \l 20A95
  \l 20A96
  \l 20A97
  \l 20A98
  \l 20A99
  \l 20A9A
  \l 20A9B
  \l 20A9C
  \l 20A9D
  \l 20A9E
  \l 20A9F
  \l 20AA0
  \l 20AA1
  \l 20AA2
  \l 20AA3
  \l 20AA4
  \l 20AA5
  \l 20AA6
  \l 20AA7
  \l 20AA8
  \l 20AA9
  \l 20AAA
  \l 20AAB
  \l 20AAC
  \l 20AAD
  \l 20AAE
  \l 20AAF
  \l 20AB0
  \l 20AB1
  \l 20AB2
  \l 20AB3
  \l 20AB4
  \l 20AB5
  \l 20AB6
  \l 20AB7
  \l 20AB8
  \l 20AB9
  \l 20ABA
  \l 20ABB
  \l 20ABC
  \l 20ABD
  \l 20ABE
  \l 20ABF
  \l 20AC0
  \l 20AC1
  \l 20AC2
  \l 20AC3
  \l 20AC4
  \l 20AC5
  \l 20AC6
  \l 20AC7
  \l 20AC8
  \l 20AC9
  \l 20ACA
  \l 20ACB
  \l 20ACC
  \l 20ACD
  \l 20ACE
  \l 20ACF
  \l 20AD0
  \l 20AD1
  \l 20AD2
  \l 20AD3
  \l 20AD4
  \l 20AD5
  \l 20AD6
  \l 20AD7
  \l 20AD8
  \l 20AD9
  \l 20ADA
  \l 20ADB
  \l 20ADC
  \l 20ADD
  \l 20ADE
  \l 20ADF
  \l 20AE0
  \l 20AE1
  \l 20AE2
  \l 20AE3
  \l 20AE4
  \l 20AE5
  \l 20AE6
  \l 20AE7
  \l 20AE8
  \l 20AE9
  \l 20AEA
  \l 20AEB
  \l 20AEC
  \l 20AED
  \l 20AEE
  \l 20AEF
  \l 20AF0
  \l 20AF1
  \l 20AF2
  \l 20AF3
  \l 20AF4
  \l 20AF5
  \l 20AF6
  \l 20AF7
  \l 20AF8
  \l 20AF9
  \l 20AFA
  \l 20AFB
  \l 20AFC
  \l 20AFD
  \l 20AFE
  \l 20AFF
  \l 20B00
  \l 20B01
  \l 20B02
  \l 20B03
  \l 20B04
  \l 20B05
  \l 20B06
  \l 20B07
  \l 20B08
  \l 20B09
  \l 20B0A
  \l 20B0B
  \l 20B0C
  \l 20B0D
  \l 20B0E
  \l 20B0F
  \l 20B10
  \l 20B11
  \l 20B12
  \l 20B13
  \l 20B14
  \l 20B15
  \l 20B16
  \l 20B17
  \l 20B18
  \l 20B19
  \l 20B1A
  \l 20B1B
  \l 20B1C
  \l 20B1D
  \l 20B1E
  \l 20B1F
  \l 20B20
  \l 20B21
  \l 20B22
  \l 20B23
  \l 20B24
  \l 20B25
  \l 20B26
  \l 20B27
  \l 20B28
  \l 20B29
  \l 20B2A
  \l 20B2B
  \l 20B2C
  \l 20B2D
  \l 20B2E
  \l 20B2F
  \l 20B30
  \l 20B31
  \l 20B32
  \l 20B33
  \l 20B34
  \l 20B35
  \l 20B36
  \l 20B37
  \l 20B38
  \l 20B39
  \l 20B3A
  \l 20B3B
  \l 20B3C
  \l 20B3D
  \l 20B3E
  \l 20B3F
  \l 20B40
  \l 20B41
  \l 20B42
  \l 20B43
  \l 20B44
  \l 20B45
  \l 20B46
  \l 20B47
  \l 20B48
  \l 20B49
  \l 20B4A
  \l 20B4B
  \l 20B4C
  \l 20B4D
  \l 20B4E
  \l 20B4F
  \l 20B50
  \l 20B51
  \l 20B52
  \l 20B53
  \l 20B54
  \l 20B55
  \l 20B56
  \l 20B57
  \l 20B58
  \l 20B59
  \l 20B5A
  \l 20B5B
  \l 20B5C
  \l 20B5D
  \l 20B5E
  \l 20B5F
  \l 20B60
  \l 20B61
  \l 20B62
  \l 20B63
  \l 20B64
  \l 20B65
  \l 20B66
  \l 20B67
  \l 20B68
  \l 20B69
  \l 20B6A
  \l 20B6B
  \l 20B6C
  \l 20B6D
  \l 20B6E
  \l 20B6F
  \l 20B70
  \l 20B71
  \l 20B72
  \l 20B73
  \l 20B74
  \l 20B75
  \l 20B76
  \l 20B77
  \l 20B78
  \l 20B79
  \l 20B7A
  \l 20B7B
  \l 20B7C
  \l 20B7D
  \l 20B7E
  \l 20B7F
  \l 20B80
  \l 20B81
  \l 20B82
  \l 20B83
  \l 20B84
  \l 20B85
  \l 20B86
  \l 20B87
  \l 20B88
  \l 20B89
  \l 20B8A
  \l 20B8B
  \l 20B8C
  \l 20B8D
  \l 20B8E
  \l 20B8F
  \l 20B90
  \l 20B91
  \l 20B92
  \l 20B93
  \l 20B94
  \l 20B95
  \l 20B96
  \l 20B97
  \l 20B98
  \l 20B99
  \l 20B9A
  \l 20B9B
  \l 20B9C
  \l 20B9D
  \l 20B9E
  \l 20B9F
  \l 20BA0
  \l 20BA1
  \l 20BA2
  \l 20BA3
  \l 20BA4
  \l 20BA5
  \l 20BA6
  \l 20BA7
  \l 20BA8
  \l 20BA9
  \l 20BAA
  \l 20BAB
  \l 20BAC
  \l 20BAD
  \l 20BAE
  \l 20BAF
  \l 20BB0
  \l 20BB1
  \l 20BB2
  \l 20BB3
  \l 20BB4
  \l 20BB5
  \l 20BB6
  \l 20BB7
  \l 20BB8
  \l 20BB9
  \l 20BBA
  \l 20BBB
  \l 20BBC
  \l 20BBD
  \l 20BBE
  \l 20BBF
  \l 20BC0
  \l 20BC1
  \l 20BC2
  \l 20BC3
  \l 20BC4
  \l 20BC5
  \l 20BC6
  \l 20BC7
  \l 20BC8
  \l 20BC9
  \l 20BCA
  \l 20BCB
  \l 20BCC
  \l 20BCD
  \l 20BCE
  \l 20BCF
  \l 20BD0
  \l 20BD1
  \l 20BD2
  \l 20BD3
  \l 20BD4
  \l 20BD5
  \l 20BD6
  \l 20BD7
  \l 20BD8
  \l 20BD9
  \l 20BDA
  \l 20BDB
  \l 20BDC
  \l 20BDD
  \l 20BDE
  \l 20BDF
  \l 20BE0
  \l 20BE1
  \l 20BE2
  \l 20BE3
  \l 20BE4
  \l 20BE5
  \l 20BE6
  \l 20BE7
  \l 20BE8
  \l 20BE9
  \l 20BEA
  \l 20BEB
  \l 20BEC
  \l 20BED
  \l 20BEE
  \l 20BEF
  \l 20BF0
  \l 20BF1
  \l 20BF2
  \l 20BF3
  \l 20BF4
  \l 20BF5
  \l 20BF6
  \l 20BF7
  \l 20BF8
  \l 20BF9
  \l 20BFA
  \l 20BFB
  \l 20BFC
  \l 20BFD
  \l 20BFE
  \l 20BFF
  \l 20C00
  \l 20C01
  \l 20C02
  \l 20C03
  \l 20C04
  \l 20C05
  \l 20C06
  \l 20C07
  \l 20C08
  \l 20C09
  \l 20C0A
  \l 20C0B
  \l 20C0C
  \l 20C0D
  \l 20C0E
  \l 20C0F
  \l 20C10
  \l 20C11
  \l 20C12
  \l 20C13
  \l 20C14
  \l 20C15
  \l 20C16
  \l 20C17
  \l 20C18
  \l 20C19
  \l 20C1A
  \l 20C1B
  \l 20C1C
  \l 20C1D
  \l 20C1E
  \l 20C1F
  \l 20C20
  \l 20C21
  \l 20C22
  \l 20C23
  \l 20C24
  \l 20C25
  \l 20C26
  \l 20C27
  \l 20C28
  \l 20C29
  \l 20C2A
  \l 20C2B
  \l 20C2C
  \l 20C2D
  \l 20C2E
  \l 20C2F
  \l 20C30
  \l 20C31
  \l 20C32
  \l 20C33
  \l 20C34
  \l 20C35
  \l 20C36
  \l 20C37
  \l 20C38
  \l 20C39
  \l 20C3A
  \l 20C3B
  \l 20C3C
  \l 20C3D
  \l 20C3E
  \l 20C3F
  \l 20C40
  \l 20C41
  \l 20C42
  \l 20C43
  \l 20C44
  \l 20C45
  \l 20C46
  \l 20C47
  \l 20C48
  \l 20C49
  \l 20C4A
  \l 20C4B
  \l 20C4C
  \l 20C4D
  \l 20C4E
  \l 20C4F
  \l 20C50
  \l 20C51
  \l 20C52
  \l 20C53
  \l 20C54
  \l 20C55
  \l 20C56
  \l 20C57
  \l 20C58
  \l 20C59
  \l 20C5A
  \l 20C5B
  \l 20C5C
  \l 20C5D
  \l 20C5E
  \l 20C5F
  \l 20C60
  \l 20C61
  \l 20C62
  \l 20C63
  \l 20C64
  \l 20C65
  \l 20C66
  \l 20C67
  \l 20C68
  \l 20C69
  \l 20C6A
  \l 20C6B
  \l 20C6C
  \l 20C6D
  \l 20C6E
  \l 20C6F
  \l 20C70
  \l 20C71
  \l 20C72
  \l 20C73
  \l 20C74
  \l 20C75
  \l 20C76
  \l 20C77
  \l 20C78
  \l 20C79
  \l 20C7A
  \l 20C7B
  \l 20C7C
  \l 20C7D
  \l 20C7E
  \l 20C7F
  \l 20C80
  \l 20C81
  \l 20C82
  \l 20C83
  \l 20C84
  \l 20C85
  \l 20C86
  \l 20C87
  \l 20C88
  \l 20C89
  \l 20C8A
  \l 20C8B
  \l 20C8C
  \l 20C8D
  \l 20C8E
  \l 20C8F
  \l 20C90
  \l 20C91
  \l 20C92
  \l 20C93
  \l 20C94
  \l 20C95
  \l 20C96
  \l 20C97
  \l 20C98
  \l 20C99
  \l 20C9A
  \l 20C9B
  \l 20C9C
  \l 20C9D
  \l 20C9E
  \l 20C9F
  \l 20CA0
  \l 20CA1
  \l 20CA2
  \l 20CA3
  \l 20CA4
  \l 20CA5
  \l 20CA6
  \l 20CA7
  \l 20CA8
  \l 20CA9
  \l 20CAA
  \l 20CAB
  \l 20CAC
  \l 20CAD
  \l 20CAE
  \l 20CAF
  \l 20CB0
  \l 20CB1
  \l 20CB2
  \l 20CB3
  \l 20CB4
  \l 20CB5
  \l 20CB6
  \l 20CB7
  \l 20CB8
  \l 20CB9
  \l 20CBA
  \l 20CBB
  \l 20CBC
  \l 20CBD
  \l 20CBE
  \l 20CBF
  \l 20CC0
  \l 20CC1
  \l 20CC2
  \l 20CC3
  \l 20CC4
  \l 20CC5
  \l 20CC6
  \l 20CC7
  \l 20CC8
  \l 20CC9
  \l 20CCA
  \l 20CCB
  \l 20CCC
  \l 20CCD
  \l 20CCE
  \l 20CCF
  \l 20CD0
  \l 20CD1
  \l 20CD2
  \l 20CD3
  \l 20CD4
  \l 20CD5
  \l 20CD6
  \l 20CD7
  \l 20CD8
  \l 20CD9
  \l 20CDA
  \l 20CDB
  \l 20CDC
  \l 20CDD
  \l 20CDE
  \l 20CDF
  \l 20CE0
  \l 20CE1
  \l 20CE2
  \l 20CE3
  \l 20CE4
  \l 20CE5
  \l 20CE6
  \l 20CE7
  \l 20CE8
  \l 20CE9
  \l 20CEA
  \l 20CEB
  \l 20CEC
  \l 20CED
  \l 20CEE
  \l 20CEF
  \l 20CF0
  \l 20CF1
  \l 20CF2
  \l 20CF3
  \l 20CF4
  \l 20CF5
  \l 20CF6
  \l 20CF7
  \l 20CF8
  \l 20CF9
  \l 20CFA
  \l 20CFB
  \l 20CFC
  \l 20CFD
  \l 20CFE
  \l 20CFF
  \l 20D00
  \l 20D01
  \l 20D02
  \l 20D03
  \l 20D04
  \l 20D05
  \l 20D06
  \l 20D07
  \l 20D08
  \l 20D09
  \l 20D0A
  \l 20D0B
  \l 20D0C
  \l 20D0D
  \l 20D0E
  \l 20D0F
  \l 20D10
  \l 20D11
  \l 20D12
  \l 20D13
  \l 20D14
  \l 20D15
  \l 20D16
  \l 20D17
  \l 20D18
  \l 20D19
  \l 20D1A
  \l 20D1B
  \l 20D1C
  \l 20D1D
  \l 20D1E
  \l 20D1F
  \l 20D20
  \l 20D21
  \l 20D22
  \l 20D23
  \l 20D24
  \l 20D25
  \l 20D26
  \l 20D27
  \l 20D28
  \l 20D29
  \l 20D2A
  \l 20D2B
  \l 20D2C
  \l 20D2D
  \l 20D2E
  \l 20D2F
  \l 20D30
  \l 20D31
  \l 20D32
  \l 20D33
  \l 20D34
  \l 20D35
  \l 20D36
  \l 20D37
  \l 20D38
  \l 20D39
  \l 20D3A
  \l 20D3B
  \l 20D3C
  \l 20D3D
  \l 20D3E
  \l 20D3F
  \l 20D40
  \l 20D41
  \l 20D42
  \l 20D43
  \l 20D44
  \l 20D45
  \l 20D46
  \l 20D47
  \l 20D48
  \l 20D49
  \l 20D4A
  \l 20D4B
  \l 20D4C
  \l 20D4D
  \l 20D4E
  \l 20D4F
  \l 20D50
  \l 20D51
  \l 20D52
  \l 20D53
  \l 20D54
  \l 20D55
  \l 20D56
  \l 20D57
  \l 20D58
  \l 20D59
  \l 20D5A
  \l 20D5B
  \l 20D5C
  \l 20D5D
  \l 20D5E
  \l 20D5F
  \l 20D60
  \l 20D61
  \l 20D62
  \l 20D63
  \l 20D64
  \l 20D65
  \l 20D66
  \l 20D67
  \l 20D68
  \l 20D69
  \l 20D6A
  \l 20D6B
  \l 20D6C
  \l 20D6D
  \l 20D6E
  \l 20D6F
  \l 20D70
  \l 20D71
  \l 20D72
  \l 20D73
  \l 20D74
  \l 20D75
  \l 20D76
  \l 20D77
  \l 20D78
  \l 20D79
  \l 20D7A
  \l 20D7B
  \l 20D7C
  \l 20D7D
  \l 20D7E
  \l 20D7F
  \l 20D80
  \l 20D81
  \l 20D82
  \l 20D83
  \l 20D84
  \l 20D85
  \l 20D86
  \l 20D87
  \l 20D88
  \l 20D89
  \l 20D8A
  \l 20D8B
  \l 20D8C
  \l 20D8D
  \l 20D8E
  \l 20D8F
  \l 20D90
  \l 20D91
  \l 20D92
  \l 20D93
  \l 20D94
  \l 20D95
  \l 20D96
  \l 20D97
  \l 20D98
  \l 20D99
  \l 20D9A
  \l 20D9B
  \l 20D9C
  \l 20D9D
  \l 20D9E
  \l 20D9F
  \l 20DA0
  \l 20DA1
  \l 20DA2
  \l 20DA3
  \l 20DA4
  \l 20DA5
  \l 20DA6
  \l 20DA7
  \l 20DA8
  \l 20DA9
  \l 20DAA
  \l 20DAB
  \l 20DAC
  \l 20DAD
  \l 20DAE
  \l 20DAF
  \l 20DB0
  \l 20DB1
  \l 20DB2
  \l 20DB3
  \l 20DB4
  \l 20DB5
  \l 20DB6
  \l 20DB7
  \l 20DB8
  \l 20DB9
  \l 20DBA
  \l 20DBB
  \l 20DBC
  \l 20DBD
  \l 20DBE
  \l 20DBF
  \l 20DC0
  \l 20DC1
  \l 20DC2
  \l 20DC3
  \l 20DC4
  \l 20DC5
  \l 20DC6
  \l 20DC7
  \l 20DC8
  \l 20DC9
  \l 20DCA
  \l 20DCB
  \l 20DCC
  \l 20DCD
  \l 20DCE
  \l 20DCF
  \l 20DD0
  \l 20DD1
  \l 20DD2
  \l 20DD3
  \l 20DD4
  \l 20DD5
  \l 20DD6
  \l 20DD7
  \l 20DD8
  \l 20DD9
  \l 20DDA
  \l 20DDB
  \l 20DDC
  \l 20DDD
  \l 20DDE
  \l 20DDF
  \l 20DE0
  \l 20DE1
  \l 20DE2
  \l 20DE3
  \l 20DE4
  \l 20DE5
  \l 20DE6
  \l 20DE7
  \l 20DE8
  \l 20DE9
  \l 20DEA
  \l 20DEB
  \l 20DEC
  \l 20DED
  \l 20DEE
  \l 20DEF
  \l 20DF0
  \l 20DF1
  \l 20DF2
  \l 20DF3
  \l 20DF4
  \l 20DF5
  \l 20DF6
  \l 20DF7
  \l 20DF8
  \l 20DF9
  \l 20DFA
  \l 20DFB
  \l 20DFC
  \l 20DFD
  \l 20DFE
  \l 20DFF
  \l 20E00
  \l 20E01
  \l 20E02
  \l 20E03
  \l 20E04
  \l 20E05
  \l 20E06
  \l 20E07
  \l 20E08
  \l 20E09
  \l 20E0A
  \l 20E0B
  \l 20E0C
  \l 20E0D
  \l 20E0E
  \l 20E0F
  \l 20E10
  \l 20E11
  \l 20E12
  \l 20E13
  \l 20E14
  \l 20E15
  \l 20E16
  \l 20E17
  \l 20E18
  \l 20E19
  \l 20E1A
  \l 20E1B
  \l 20E1C
  \l 20E1D
  \l 20E1E
  \l 20E1F
  \l 20E20
  \l 20E21
  \l 20E22
  \l 20E23
  \l 20E24
  \l 20E25
  \l 20E26
  \l 20E27
  \l 20E28
  \l 20E29
  \l 20E2A
  \l 20E2B
  \l 20E2C
  \l 20E2D
  \l 20E2E
  \l 20E2F
  \l 20E30
  \l 20E31
  \l 20E32
  \l 20E33
  \l 20E34
  \l 20E35
  \l 20E36
  \l 20E37
  \l 20E38
  \l 20E39
  \l 20E3A
  \l 20E3B
  \l 20E3C
  \l 20E3D
  \l 20E3E
  \l 20E3F
  \l 20E40
  \l 20E41
  \l 20E42
  \l 20E43
  \l 20E44
  \l 20E45
  \l 20E46
  \l 20E47
  \l 20E48
  \l 20E49
  \l 20E4A
  \l 20E4B
  \l 20E4C
  \l 20E4D
  \l 20E4E
  \l 20E4F
  \l 20E50
  \l 20E51
  \l 20E52
  \l 20E53
  \l 20E54
  \l 20E55
  \l 20E56
  \l 20E57
  \l 20E58
  \l 20E59
  \l 20E5A
  \l 20E5B
  \l 20E5C
  \l 20E5D
  \l 20E5E
  \l 20E5F
  \l 20E60
  \l 20E61
  \l 20E62
  \l 20E63
  \l 20E64
  \l 20E65
  \l 20E66
  \l 20E67
  \l 20E68
  \l 20E69
  \l 20E6A
  \l 20E6B
  \l 20E6C
  \l 20E6D
  \l 20E6E
  \l 20E6F
  \l 20E70
  \l 20E71
  \l 20E72
  \l 20E73
  \l 20E74
  \l 20E75
  \l 20E76
  \l 20E77
  \l 20E78
  \l 20E79
  \l 20E7A
  \l 20E7B
  \l 20E7C
  \l 20E7D
  \l 20E7E
  \l 20E7F
  \l 20E80
  \l 20E81
  \l 20E82
  \l 20E83
  \l 20E84
  \l 20E85
  \l 20E86
  \l 20E87
  \l 20E88
  \l 20E89
  \l 20E8A
  \l 20E8B
  \l 20E8C
  \l 20E8D
  \l 20E8E
  \l 20E8F
  \l 20E90
  \l 20E91
  \l 20E92
  \l 20E93
  \l 20E94
  \l 20E95
  \l 20E96
  \l 20E97
  \l 20E98
  \l 20E99
  \l 20E9A
  \l 20E9B
  \l 20E9C
  \l 20E9D
  \l 20E9E
  \l 20E9F
  \l 20EA0
  \l 20EA1
  \l 20EA2
  \l 20EA3
  \l 20EA4
  \l 20EA5
  \l 20EA6
  \l 20EA7
  \l 20EA8
  \l 20EA9
  \l 20EAA
  \l 20EAB
  \l 20EAC
  \l 20EAD
  \l 20EAE
  \l 20EAF
  \l 20EB0
  \l 20EB1
  \l 20EB2
  \l 20EB3
  \l 20EB4
  \l 20EB5
  \l 20EB6
  \l 20EB7
  \l 20EB8
  \l 20EB9
  \l 20EBA
  \l 20EBB
  \l 20EBC
  \l 20EBD
  \l 20EBE
  \l 20EBF
  \l 20EC0
  \l 20EC1
  \l 20EC2
  \l 20EC3
  \l 20EC4
  \l 20EC5
  \l 20EC6
  \l 20EC7
  \l 20EC8
  \l 20EC9
  \l 20ECA
  \l 20ECB
  \l 20ECC
  \l 20ECD
  \l 20ECE
  \l 20ECF
  \l 20ED0
  \l 20ED1
  \l 20ED2
  \l 20ED3
  \l 20ED4
  \l 20ED5
  \l 20ED6
  \l 20ED7
  \l 20ED8
  \l 20ED9
  \l 20EDA
  \l 20EDB
  \l 20EDC
  \l 20EDD
  \l 20EDE
  \l 20EDF
  \l 20EE0
  \l 20EE1
  \l 20EE2
  \l 20EE3
  \l 20EE4
  \l 20EE5
  \l 20EE6
  \l 20EE7
  \l 20EE8
  \l 20EE9
  \l 20EEA
  \l 20EEB
  \l 20EEC
  \l 20EED
  \l 20EEE
  \l 20EEF
  \l 20EF0
  \l 20EF1
  \l 20EF2
  \l 20EF3
  \l 20EF4
  \l 20EF5
  \l 20EF6
  \l 20EF7
  \l 20EF8
  \l 20EF9
  \l 20EFA
  \l 20EFB
  \l 20EFC
  \l 20EFD
  \l 20EFE
  \l 20EFF
  \l 20F00
  \l 20F01
  \l 20F02
  \l 20F03
  \l 20F04
  \l 20F05
  \l 20F06
  \l 20F07
  \l 20F08
  \l 20F09
  \l 20F0A
  \l 20F0B
  \l 20F0C
  \l 20F0D
  \l 20F0E
  \l 20F0F
  \l 20F10
  \l 20F11
  \l 20F12
  \l 20F13
  \l 20F14
  \l 20F15
  \l 20F16
  \l 20F17
  \l 20F18
  \l 20F19
  \l 20F1A
  \l 20F1B
  \l 20F1C
  \l 20F1D
  \l 20F1E
  \l 20F1F
  \l 20F20
  \l 20F21
  \l 20F22
  \l 20F23
  \l 20F24
  \l 20F25
  \l 20F26
  \l 20F27
  \l 20F28
  \l 20F29
  \l 20F2A
  \l 20F2B
  \l 20F2C
  \l 20F2D
  \l 20F2E
  \l 20F2F
  \l 20F30
  \l 20F31
  \l 20F32
  \l 20F33
  \l 20F34
  \l 20F35
  \l 20F36
  \l 20F37
  \l 20F38
  \l 20F39
  \l 20F3A
  \l 20F3B
  \l 20F3C
  \l 20F3D
  \l 20F3E
  \l 20F3F
  \l 20F40
  \l 20F41
  \l 20F42
  \l 20F43
  \l 20F44
  \l 20F45
  \l 20F46
  \l 20F47
  \l 20F48
  \l 20F49
  \l 20F4A
  \l 20F4B
  \l 20F4C
  \l 20F4D
  \l 20F4E
  \l 20F4F
  \l 20F50
  \l 20F51
  \l 20F52
  \l 20F53
  \l 20F54
  \l 20F55
  \l 20F56
  \l 20F57
  \l 20F58
  \l 20F59
  \l 20F5A
  \l 20F5B
  \l 20F5C
  \l 20F5D
  \l 20F5E
  \l 20F5F
  \l 20F60
  \l 20F61
  \l 20F62
  \l 20F63
  \l 20F64
  \l 20F65
  \l 20F66
  \l 20F67
  \l 20F68
  \l 20F69
  \l 20F6A
  \l 20F6B
  \l 20F6C
  \l 20F6D
  \l 20F6E
  \l 20F6F
  \l 20F70
  \l 20F71
  \l 20F72
  \l 20F73
  \l 20F74
  \l 20F75
  \l 20F76
  \l 20F77
  \l 20F78
  \l 20F79
  \l 20F7A
  \l 20F7B
  \l 20F7C
  \l 20F7D
  \l 20F7E
  \l 20F7F
  \l 20F80
  \l 20F81
  \l 20F82
  \l 20F83
  \l 20F84
  \l 20F85
  \l 20F86
  \l 20F87
  \l 20F88
  \l 20F89
  \l 20F8A
  \l 20F8B
  \l 20F8C
  \l 20F8D
  \l 20F8E
  \l 20F8F
  \l 20F90
  \l 20F91
  \l 20F92
  \l 20F93
  \l 20F94
  \l 20F95
  \l 20F96
  \l 20F97
  \l 20F98
  \l 20F99
  \l 20F9A
  \l 20F9B
  \l 20F9C
  \l 20F9D
  \l 20F9E
  \l 20F9F
  \l 20FA0
  \l 20FA1
  \l 20FA2
  \l 20FA3
  \l 20FA4
  \l 20FA5
  \l 20FA6
  \l 20FA7
  \l 20FA8
  \l 20FA9
  \l 20FAA
  \l 20FAB
  \l 20FAC
  \l 20FAD
  \l 20FAE
  \l 20FAF
  \l 20FB0
  \l 20FB1
  \l 20FB2
  \l 20FB3
  \l 20FB4
  \l 20FB5
  \l 20FB6
  \l 20FB7
  \l 20FB8
  \l 20FB9
  \l 20FBA
  \l 20FBB
  \l 20FBC
  \l 20FBD
  \l 20FBE
  \l 20FBF
  \l 20FC0
  \l 20FC1
  \l 20FC2
  \l 20FC3
  \l 20FC4
  \l 20FC5
  \l 20FC6
  \l 20FC7
  \l 20FC8
  \l 20FC9
  \l 20FCA
  \l 20FCB
  \l 20FCC
  \l 20FCD
  \l 20FCE
  \l 20FCF
  \l 20FD0
  \l 20FD1
  \l 20FD2
  \l 20FD3
  \l 20FD4
  \l 20FD5
  \l 20FD6
  \l 20FD7
  \l 20FD8
  \l 20FD9
  \l 20FDA
  \l 20FDB
  \l 20FDC
  \l 20FDD
  \l 20FDE
  \l 20FDF
  \l 20FE0
  \l 20FE1
  \l 20FE2
  \l 20FE3
  \l 20FE4
  \l 20FE5
  \l 20FE6
  \l 20FE7
  \l 20FE8
  \l 20FE9
  \l 20FEA
  \l 20FEB
  \l 20FEC
  \l 20FED
  \l 20FEE
  \l 20FEF
  \l 20FF0
  \l 20FF1
  \l 20FF2
  \l 20FF3
  \l 20FF4
  \l 20FF5
  \l 20FF6
  \l 20FF7
  \l 20FF8
  \l 20FF9
  \l 20FFA
  \l 20FFB
  \l 20FFC
  \l 20FFD
  \l 20FFE
  \l 20FFF
  \l 21000
  \l 21001
  \l 21002
  \l 21003
  \l 21004
  \l 21005
  \l 21006
  \l 21007
  \l 21008
  \l 21009
  \l 2100A
  \l 2100B
  \l 2100C
  \l 2100D
  \l 2100E
  \l 2100F
  \l 21010
  \l 21011
  \l 21012
  \l 21013
  \l 21014
  \l 21015
  \l 21016
  \l 21017
  \l 21018
  \l 21019
  \l 2101A
  \l 2101B
  \l 2101C
  \l 2101D
  \l 2101E
  \l 2101F
  \l 21020
  \l 21021
  \l 21022
  \l 21023
  \l 21024
  \l 21025
  \l 21026
  \l 21027
  \l 21028
  \l 21029
  \l 2102A
  \l 2102B
  \l 2102C
  \l 2102D
  \l 2102E
  \l 2102F
  \l 21030
  \l 21031
  \l 21032
  \l 21033
  \l 21034
  \l 21035
  \l 21036
  \l 21037
  \l 21038
  \l 21039
  \l 2103A
  \l 2103B
  \l 2103C
  \l 2103D
  \l 2103E
  \l 2103F
  \l 21040
  \l 21041
  \l 21042
  \l 21043
  \l 21044
  \l 21045
  \l 21046
  \l 21047
  \l 21048
  \l 21049
  \l 2104A
  \l 2104B
  \l 2104C
  \l 2104D
  \l 2104E
  \l 2104F
  \l 21050
  \l 21051
  \l 21052
  \l 21053
  \l 21054
  \l 21055
  \l 21056
  \l 21057
  \l 21058
  \l 21059
  \l 2105A
  \l 2105B
  \l 2105C
  \l 2105D
  \l 2105E
  \l 2105F
  \l 21060
  \l 21061
  \l 21062
  \l 21063
  \l 21064
  \l 21065
  \l 21066
  \l 21067
  \l 21068
  \l 21069
  \l 2106A
  \l 2106B
  \l 2106C
  \l 2106D
  \l 2106E
  \l 2106F
  \l 21070
  \l 21071
  \l 21072
  \l 21073
  \l 21074
  \l 21075
  \l 21076
  \l 21077
  \l 21078
  \l 21079
  \l 2107A
  \l 2107B
  \l 2107C
  \l 2107D
  \l 2107E
  \l 2107F
  \l 21080
  \l 21081
  \l 21082
  \l 21083
  \l 21084
  \l 21085
  \l 21086
  \l 21087
  \l 21088
  \l 21089
  \l 2108A
  \l 2108B
  \l 2108C
  \l 2108D
  \l 2108E
  \l 2108F
  \l 21090
  \l 21091
  \l 21092
  \l 21093
  \l 21094
  \l 21095
  \l 21096
  \l 21097
  \l 21098
  \l 21099
  \l 2109A
  \l 2109B
  \l 2109C
  \l 2109D
  \l 2109E
  \l 2109F
  \l 210A0
  \l 210A1
  \l 210A2
  \l 210A3
  \l 210A4
  \l 210A5
  \l 210A6
  \l 210A7
  \l 210A8
  \l 210A9
  \l 210AA
  \l 210AB
  \l 210AC
  \l 210AD
  \l 210AE
  \l 210AF
  \l 210B0
  \l 210B1
  \l 210B2
  \l 210B3
  \l 210B4
  \l 210B5
  \l 210B6
  \l 210B7
  \l 210B8
  \l 210B9
  \l 210BA
  \l 210BB
  \l 210BC
  \l 210BD
  \l 210BE
  \l 210BF
  \l 210C0
  \l 210C1
  \l 210C2
  \l 210C3
  \l 210C4
  \l 210C5
  \l 210C6
  \l 210C7
  \l 210C8
  \l 210C9
  \l 210CA
  \l 210CB
  \l 210CC
  \l 210CD
  \l 210CE
  \l 210CF
  \l 210D0
  \l 210D1
  \l 210D2
  \l 210D3
  \l 210D4
  \l 210D5
  \l 210D6
  \l 210D7
  \l 210D8
  \l 210D9
  \l 210DA
  \l 210DB
  \l 210DC
  \l 210DD
  \l 210DE
  \l 210DF
  \l 210E0
  \l 210E1
  \l 210E2
  \l 210E3
  \l 210E4
  \l 210E5
  \l 210E6
  \l 210E7
  \l 210E8
  \l 210E9
  \l 210EA
  \l 210EB
  \l 210EC
  \l 210ED
  \l 210EE
  \l 210EF
  \l 210F0
  \l 210F1
  \l 210F2
  \l 210F3
  \l 210F4
  \l 210F5
  \l 210F6
  \l 210F7
  \l 210F8
  \l 210F9
  \l 210FA
  \l 210FB
  \l 210FC
  \l 210FD
  \l 210FE
  \l 210FF
  \l 21100
  \l 21101
  \l 21102
  \l 21103
  \l 21104
  \l 21105
  \l 21106
  \l 21107
  \l 21108
  \l 21109
  \l 2110A
  \l 2110B
  \l 2110C
  \l 2110D
  \l 2110E
  \l 2110F
  \l 21110
  \l 21111
  \l 21112
  \l 21113
  \l 21114
  \l 21115
  \l 21116
  \l 21117
  \l 21118
  \l 21119
  \l 2111A
  \l 2111B
  \l 2111C
  \l 2111D
  \l 2111E
  \l 2111F
  \l 21120
  \l 21121
  \l 21122
  \l 21123
  \l 21124
  \l 21125
  \l 21126
  \l 21127
  \l 21128
  \l 21129
  \l 2112A
  \l 2112B
  \l 2112C
  \l 2112D
  \l 2112E
  \l 2112F
  \l 21130
  \l 21131
  \l 21132
  \l 21133
  \l 21134
  \l 21135
  \l 21136
  \l 21137
  \l 21138
  \l 21139
  \l 2113A
  \l 2113B
  \l 2113C
  \l 2113D
  \l 2113E
  \l 2113F
  \l 21140
  \l 21141
  \l 21142
  \l 21143
  \l 21144
  \l 21145
  \l 21146
  \l 21147
  \l 21148
  \l 21149
  \l 2114A
  \l 2114B
  \l 2114C
  \l 2114D
  \l 2114E
  \l 2114F
  \l 21150
  \l 21151
  \l 21152
  \l 21153
  \l 21154
  \l 21155
  \l 21156
  \l 21157
  \l 21158
  \l 21159
  \l 2115A
  \l 2115B
  \l 2115C
  \l 2115D
  \l 2115E
  \l 2115F
  \l 21160
  \l 21161
  \l 21162
  \l 21163
  \l 21164
  \l 21165
  \l 21166
  \l 21167
  \l 21168
  \l 21169
  \l 2116A
  \l 2116B
  \l 2116C
  \l 2116D
  \l 2116E
  \l 2116F
  \l 21170
  \l 21171
  \l 21172
  \l 21173
  \l 21174
  \l 21175
  \l 21176
  \l 21177
  \l 21178
  \l 21179
  \l 2117A
  \l 2117B
  \l 2117C
  \l 2117D
  \l 2117E
  \l 2117F
  \l 21180
  \l 21181
  \l 21182
  \l 21183
  \l 21184
  \l 21185
  \l 21186
  \l 21187
  \l 21188
  \l 21189
  \l 2118A
  \l 2118B
  \l 2118C
  \l 2118D
  \l 2118E
  \l 2118F
  \l 21190
  \l 21191
  \l 21192
  \l 21193
  \l 21194
  \l 21195
  \l 21196
  \l 21197
  \l 21198
  \l 21199
  \l 2119A
  \l 2119B
  \l 2119C
  \l 2119D
  \l 2119E
  \l 2119F
  \l 211A0
  \l 211A1
  \l 211A2
  \l 211A3
  \l 211A4
  \l 211A5
  \l 211A6
  \l 211A7
  \l 211A8
  \l 211A9
  \l 211AA
  \l 211AB
  \l 211AC
  \l 211AD
  \l 211AE
  \l 211AF
  \l 211B0
  \l 211B1
  \l 211B2
  \l 211B3
  \l 211B4
  \l 211B5
  \l 211B6
  \l 211B7
  \l 211B8
  \l 211B9
  \l 211BA
  \l 211BB
  \l 211BC
  \l 211BD
  \l 211BE
  \l 211BF
  \l 211C0
  \l 211C1
  \l 211C2
  \l 211C3
  \l 211C4
  \l 211C5
  \l 211C6
  \l 211C7
  \l 211C8
  \l 211C9
  \l 211CA
  \l 211CB
  \l 211CC
  \l 211CD
  \l 211CE
  \l 211CF
  \l 211D0
  \l 211D1
  \l 211D2
  \l 211D3
  \l 211D4
  \l 211D5
  \l 211D6
  \l 211D7
  \l 211D8
  \l 211D9
  \l 211DA
  \l 211DB
  \l 211DC
  \l 211DD
  \l 211DE
  \l 211DF
  \l 211E0
  \l 211E1
  \l 211E2
  \l 211E3
  \l 211E4
  \l 211E5
  \l 211E6
  \l 211E7
  \l 211E8
  \l 211E9
  \l 211EA
  \l 211EB
  \l 211EC
  \l 211ED
  \l 211EE
  \l 211EF
  \l 211F0
  \l 211F1
  \l 211F2
  \l 211F3
  \l 211F4
  \l 211F5
  \l 211F6
  \l 211F7
  \l 211F8
  \l 211F9
  \l 211FA
  \l 211FB
  \l 211FC
  \l 211FD
  \l 211FE
  \l 211FF
  \l 21200
  \l 21201
  \l 21202
  \l 21203
  \l 21204
  \l 21205
  \l 21206
  \l 21207
  \l 21208
  \l 21209
  \l 2120A
  \l 2120B
  \l 2120C
  \l 2120D
  \l 2120E
  \l 2120F
  \l 21210
  \l 21211
  \l 21212
  \l 21213
  \l 21214
  \l 21215
  \l 21216
  \l 21217
  \l 21218
  \l 21219
  \l 2121A
  \l 2121B
  \l 2121C
  \l 2121D
  \l 2121E
  \l 2121F
  \l 21220
  \l 21221
  \l 21222
  \l 21223
  \l 21224
  \l 21225
  \l 21226
  \l 21227
  \l 21228
  \l 21229
  \l 2122A
  \l 2122B
  \l 2122C
  \l 2122D
  \l 2122E
  \l 2122F
  \l 21230
  \l 21231
  \l 21232
  \l 21233
  \l 21234
  \l 21235
  \l 21236
  \l 21237
  \l 21238
  \l 21239
  \l 2123A
  \l 2123B
  \l 2123C
  \l 2123D
  \l 2123E
  \l 2123F
  \l 21240
  \l 21241
  \l 21242
  \l 21243
  \l 21244
  \l 21245
  \l 21246
  \l 21247
  \l 21248
  \l 21249
  \l 2124A
  \l 2124B
  \l 2124C
  \l 2124D
  \l 2124E
  \l 2124F
  \l 21250
  \l 21251
  \l 21252
  \l 21253
  \l 21254
  \l 21255
  \l 21256
  \l 21257
  \l 21258
  \l 21259
  \l 2125A
  \l 2125B
  \l 2125C
  \l 2125D
  \l 2125E
  \l 2125F
  \l 21260
  \l 21261
  \l 21262
  \l 21263
  \l 21264
  \l 21265
  \l 21266
  \l 21267
  \l 21268
  \l 21269
  \l 2126A
  \l 2126B
  \l 2126C
  \l 2126D
  \l 2126E
  \l 2126F
  \l 21270
  \l 21271
  \l 21272
  \l 21273
  \l 21274
  \l 21275
  \l 21276
  \l 21277
  \l 21278
  \l 21279
  \l 2127A
  \l 2127B
  \l 2127C
  \l 2127D
  \l 2127E
  \l 2127F
  \l 21280
  \l 21281
  \l 21282
  \l 21283
  \l 21284
  \l 21285
  \l 21286
  \l 21287
  \l 21288
  \l 21289
  \l 2128A
  \l 2128B
  \l 2128C
  \l 2128D
  \l 2128E
  \l 2128F
  \l 21290
  \l 21291
  \l 21292
  \l 21293
  \l 21294
  \l 21295
  \l 21296
  \l 21297
  \l 21298
  \l 21299
  \l 2129A
  \l 2129B
  \l 2129C
  \l 2129D
  \l 2129E
  \l 2129F
  \l 212A0
  \l 212A1
  \l 212A2
  \l 212A3
  \l 212A4
  \l 212A5
  \l 212A6
  \l 212A7
  \l 212A8
  \l 212A9
  \l 212AA
  \l 212AB
  \l 212AC
  \l 212AD
  \l 212AE
  \l 212AF
  \l 212B0
  \l 212B1
  \l 212B2
  \l 212B3
  \l 212B4
  \l 212B5
  \l 212B6
  \l 212B7
  \l 212B8
  \l 212B9
  \l 212BA
  \l 212BB
  \l 212BC
  \l 212BD
  \l 212BE
  \l 212BF
  \l 212C0
  \l 212C1
  \l 212C2
  \l 212C3
  \l 212C4
  \l 212C5
  \l 212C6
  \l 212C7
  \l 212C8
  \l 212C9
  \l 212CA
  \l 212CB
  \l 212CC
  \l 212CD
  \l 212CE
  \l 212CF
  \l 212D0
  \l 212D1
  \l 212D2
  \l 212D3
  \l 212D4
  \l 212D5
  \l 212D6
  \l 212D7
  \l 212D8
  \l 212D9
  \l 212DA
  \l 212DB
  \l 212DC
  \l 212DD
  \l 212DE
  \l 212DF
  \l 212E0
  \l 212E1
  \l 212E2
  \l 212E3
  \l 212E4
  \l 212E5
  \l 212E6
  \l 212E7
  \l 212E8
  \l 212E9
  \l 212EA
  \l 212EB
  \l 212EC
  \l 212ED
  \l 212EE
  \l 212EF
  \l 212F0
  \l 212F1
  \l 212F2
  \l 212F3
  \l 212F4
  \l 212F5
  \l 212F6
  \l 212F7
  \l 212F8
  \l 212F9
  \l 212FA
  \l 212FB
  \l 212FC
  \l 212FD
  \l 212FE
  \l 212FF
  \l 21300
  \l 21301
  \l 21302
  \l 21303
  \l 21304
  \l 21305
  \l 21306
  \l 21307
  \l 21308
  \l 21309
  \l 2130A
  \l 2130B
  \l 2130C
  \l 2130D
  \l 2130E
  \l 2130F
  \l 21310
  \l 21311
  \l 21312
  \l 21313
  \l 21314
  \l 21315
  \l 21316
  \l 21317
  \l 21318
  \l 21319
  \l 2131A
  \l 2131B
  \l 2131C
  \l 2131D
  \l 2131E
  \l 2131F
  \l 21320
  \l 21321
  \l 21322
  \l 21323
  \l 21324
  \l 21325
  \l 21326
  \l 21327
  \l 21328
  \l 21329
  \l 2132A
  \l 2132B
  \l 2132C
  \l 2132D
  \l 2132E
  \l 2132F
  \l 21330
  \l 21331
  \l 21332
  \l 21333
  \l 21334
  \l 21335
  \l 21336
  \l 21337
  \l 21338
  \l 21339
  \l 2133A
  \l 2133B
  \l 2133C
  \l 2133D
  \l 2133E
  \l 2133F
  \l 21340
  \l 21341
  \l 21342
  \l 21343
  \l 21344
  \l 21345
  \l 21346
  \l 21347
  \l 21348
  \l 21349
  \l 2134A
  \l 2134B
  \l 2134C
  \l 2134D
  \l 2134E
  \l 2134F
  \l 21350
  \l 21351
  \l 21352
  \l 21353
  \l 21354
  \l 21355
  \l 21356
  \l 21357
  \l 21358
  \l 21359
  \l 2135A
  \l 2135B
  \l 2135C
  \l 2135D
  \l 2135E
  \l 2135F
  \l 21360
  \l 21361
  \l 21362
  \l 21363
  \l 21364
  \l 21365
  \l 21366
  \l 21367
  \l 21368
  \l 21369
  \l 2136A
  \l 2136B
  \l 2136C
  \l 2136D
  \l 2136E
  \l 2136F
  \l 21370
  \l 21371
  \l 21372
  \l 21373
  \l 21374
  \l 21375
  \l 21376
  \l 21377
  \l 21378
  \l 21379
  \l 2137A
  \l 2137B
  \l 2137C
  \l 2137D
  \l 2137E
  \l 2137F
  \l 21380
  \l 21381
  \l 21382
  \l 21383
  \l 21384
  \l 21385
  \l 21386
  \l 21387
  \l 21388
  \l 21389
  \l 2138A
  \l 2138B
  \l 2138C
  \l 2138D
  \l 2138E
  \l 2138F
  \l 21390
  \l 21391
  \l 21392
  \l 21393
  \l 21394
  \l 21395
  \l 21396
  \l 21397
  \l 21398
  \l 21399
  \l 2139A
  \l 2139B
  \l 2139C
  \l 2139D
  \l 2139E
  \l 2139F
  \l 213A0
  \l 213A1
  \l 213A2
  \l 213A3
  \l 213A4
  \l 213A5
  \l 213A6
  \l 213A7
  \l 213A8
  \l 213A9
  \l 213AA
  \l 213AB
  \l 213AC
  \l 213AD
  \l 213AE
  \l 213AF
  \l 213B0
  \l 213B1
  \l 213B2
  \l 213B3
  \l 213B4
  \l 213B5
  \l 213B6
  \l 213B7
  \l 213B8
  \l 213B9
  \l 213BA
  \l 213BB
  \l 213BC
  \l 213BD
  \l 213BE
  \l 213BF
  \l 213C0
  \l 213C1
  \l 213C2
  \l 213C3
  \l 213C4
  \l 213C5
  \l 213C6
  \l 213C7
  \l 213C8
  \l 213C9
  \l 213CA
  \l 213CB
  \l 213CC
  \l 213CD
  \l 213CE
  \l 213CF
  \l 213D0
  \l 213D1
  \l 213D2
  \l 213D3
  \l 213D4
  \l 213D5
  \l 213D6
  \l 213D7
  \l 213D8
  \l 213D9
  \l 213DA
  \l 213DB
  \l 213DC
  \l 213DD
  \l 213DE
  \l 213DF
  \l 213E0
  \l 213E1
  \l 213E2
  \l 213E3
  \l 213E4
  \l 213E5
  \l 213E6
  \l 213E7
  \l 213E8
  \l 213E9
  \l 213EA
  \l 213EB
  \l 213EC
  \l 213ED
  \l 213EE
  \l 213EF
  \l 213F0
  \l 213F1
  \l 213F2
  \l 213F3
  \l 213F4
  \l 213F5
  \l 213F6
  \l 213F7
  \l 213F8
  \l 213F9
  \l 213FA
  \l 213FB
  \l 213FC
  \l 213FD
  \l 213FE
  \l 213FF
  \l 21400
  \l 21401
  \l 21402
  \l 21403
  \l 21404
  \l 21405
  \l 21406
  \l 21407
  \l 21408
  \l 21409
  \l 2140A
  \l 2140B
  \l 2140C
  \l 2140D
  \l 2140E
  \l 2140F
  \l 21410
  \l 21411
  \l 21412
  \l 21413
  \l 21414
  \l 21415
  \l 21416
  \l 21417
  \l 21418
  \l 21419
  \l 2141A
  \l 2141B
  \l 2141C
  \l 2141D
  \l 2141E
  \l 2141F
  \l 21420
  \l 21421
  \l 21422
  \l 21423
  \l 21424
  \l 21425
  \l 21426
  \l 21427
  \l 21428
  \l 21429
  \l 2142A
  \l 2142B
  \l 2142C
  \l 2142D
  \l 2142E
  \l 2142F
  \l 21430
  \l 21431
  \l 21432
  \l 21433
  \l 21434
  \l 21435
  \l 21436
  \l 21437
  \l 21438
  \l 21439
  \l 2143A
  \l 2143B
  \l 2143C
  \l 2143D
  \l 2143E
  \l 2143F
  \l 21440
  \l 21441
  \l 21442
  \l 21443
  \l 21444
  \l 21445
  \l 21446
  \l 21447
  \l 21448
  \l 21449
  \l 2144A
  \l 2144B
  \l 2144C
  \l 2144D
  \l 2144E
  \l 2144F
  \l 21450
  \l 21451
  \l 21452
  \l 21453
  \l 21454
  \l 21455
  \l 21456
  \l 21457
  \l 21458
  \l 21459
  \l 2145A
  \l 2145B
  \l 2145C
  \l 2145D
  \l 2145E
  \l 2145F
  \l 21460
  \l 21461
  \l 21462
  \l 21463
  \l 21464
  \l 21465
  \l 21466
  \l 21467
  \l 21468
  \l 21469
  \l 2146A
  \l 2146B
  \l 2146C
  \l 2146D
  \l 2146E
  \l 2146F
  \l 21470
  \l 21471
  \l 21472
  \l 21473
  \l 21474
  \l 21475
  \l 21476
  \l 21477
  \l 21478
  \l 21479
  \l 2147A
  \l 2147B
  \l 2147C
  \l 2147D
  \l 2147E
  \l 2147F
  \l 21480
  \l 21481
  \l 21482
  \l 21483
  \l 21484
  \l 21485
  \l 21486
  \l 21487
  \l 21488
  \l 21489
  \l 2148A
  \l 2148B
  \l 2148C
  \l 2148D
  \l 2148E
  \l 2148F
  \l 21490
  \l 21491
  \l 21492
  \l 21493
  \l 21494
  \l 21495
  \l 21496
  \l 21497
  \l 21498
  \l 21499
  \l 2149A
  \l 2149B
  \l 2149C
  \l 2149D
  \l 2149E
  \l 2149F
  \l 214A0
  \l 214A1
  \l 214A2
  \l 214A3
  \l 214A4
  \l 214A5
  \l 214A6
  \l 214A7
  \l 214A8
  \l 214A9
  \l 214AA
  \l 214AB
  \l 214AC
  \l 214AD
  \l 214AE
  \l 214AF
  \l 214B0
  \l 214B1
  \l 214B2
  \l 214B3
  \l 214B4
  \l 214B5
  \l 214B6
  \l 214B7
  \l 214B8
  \l 214B9
  \l 214BA
  \l 214BB
  \l 214BC
  \l 214BD
  \l 214BE
  \l 214BF
  \l 214C0
  \l 214C1
  \l 214C2
  \l 214C3
  \l 214C4
  \l 214C5
  \l 214C6
  \l 214C7
  \l 214C8
  \l 214C9
  \l 214CA
  \l 214CB
  \l 214CC
  \l 214CD
  \l 214CE
  \l 214CF
  \l 214D0
  \l 214D1
  \l 214D2
  \l 214D3
  \l 214D4
  \l 214D5
  \l 214D6
  \l 214D7
  \l 214D8
  \l 214D9
  \l 214DA
  \l 214DB
  \l 214DC
  \l 214DD
  \l 214DE
  \l 214DF
  \l 214E0
  \l 214E1
  \l 214E2
  \l 214E3
  \l 214E4
  \l 214E5
  \l 214E6
  \l 214E7
  \l 214E8
  \l 214E9
  \l 214EA
  \l 214EB
  \l 214EC
  \l 214ED
  \l 214EE
  \l 214EF
  \l 214F0
  \l 214F1
  \l 214F2
  \l 214F3
  \l 214F4
  \l 214F5
  \l 214F6
  \l 214F7
  \l 214F8
  \l 214F9
  \l 214FA
  \l 214FB
  \l 214FC
  \l 214FD
  \l 214FE
  \l 214FF
  \l 21500
  \l 21501
  \l 21502
  \l 21503
  \l 21504
  \l 21505
  \l 21506
  \l 21507
  \l 21508
  \l 21509
  \l 2150A
  \l 2150B
  \l 2150C
  \l 2150D
  \l 2150E
  \l 2150F
  \l 21510
  \l 21511
  \l 21512
  \l 21513
  \l 21514
  \l 21515
  \l 21516
  \l 21517
  \l 21518
  \l 21519
  \l 2151A
  \l 2151B
  \l 2151C
  \l 2151D
  \l 2151E
  \l 2151F
  \l 21520
  \l 21521
  \l 21522
  \l 21523
  \l 21524
  \l 21525
  \l 21526
  \l 21527
  \l 21528
  \l 21529
  \l 2152A
  \l 2152B
  \l 2152C
  \l 2152D
  \l 2152E
  \l 2152F
  \l 21530
  \l 21531
  \l 21532
  \l 21533
  \l 21534
  \l 21535
  \l 21536
  \l 21537
  \l 21538
  \l 21539
  \l 2153A
  \l 2153B
  \l 2153C
  \l 2153D
  \l 2153E
  \l 2153F
  \l 21540
  \l 21541
  \l 21542
  \l 21543
  \l 21544
  \l 21545
  \l 21546
  \l 21547
  \l 21548
  \l 21549
  \l 2154A
  \l 2154B
  \l 2154C
  \l 2154D
  \l 2154E
  \l 2154F
  \l 21550
  \l 21551
  \l 21552
  \l 21553
  \l 21554
  \l 21555
  \l 21556
  \l 21557
  \l 21558
  \l 21559
  \l 2155A
  \l 2155B
  \l 2155C
  \l 2155D
  \l 2155E
  \l 2155F
  \l 21560
  \l 21561
  \l 21562
  \l 21563
  \l 21564
  \l 21565
  \l 21566
  \l 21567
  \l 21568
  \l 21569
  \l 2156A
  \l 2156B
  \l 2156C
  \l 2156D
  \l 2156E
  \l 2156F
  \l 21570
  \l 21571
  \l 21572
  \l 21573
  \l 21574
  \l 21575
  \l 21576
  \l 21577
  \l 21578
  \l 21579
  \l 2157A
  \l 2157B
  \l 2157C
  \l 2157D
  \l 2157E
  \l 2157F
  \l 21580
  \l 21581
  \l 21582
  \l 21583
  \l 21584
  \l 21585
  \l 21586
  \l 21587
  \l 21588
  \l 21589
  \l 2158A
  \l 2158B
  \l 2158C
  \l 2158D
  \l 2158E
  \l 2158F
  \l 21590
  \l 21591
  \l 21592
  \l 21593
  \l 21594
  \l 21595
  \l 21596
  \l 21597
  \l 21598
  \l 21599
  \l 2159A
  \l 2159B
  \l 2159C
  \l 2159D
  \l 2159E
  \l 2159F
  \l 215A0
  \l 215A1
  \l 215A2
  \l 215A3
  \l 215A4
  \l 215A5
  \l 215A6
  \l 215A7
  \l 215A8
  \l 215A9
  \l 215AA
  \l 215AB
  \l 215AC
  \l 215AD
  \l 215AE
  \l 215AF
  \l 215B0
  \l 215B1
  \l 215B2
  \l 215B3
  \l 215B4
  \l 215B5
  \l 215B6
  \l 215B7
  \l 215B8
  \l 215B9
  \l 215BA
  \l 215BB
  \l 215BC
  \l 215BD
  \l 215BE
  \l 215BF
  \l 215C0
  \l 215C1
  \l 215C2
  \l 215C3
  \l 215C4
  \l 215C5
  \l 215C6
  \l 215C7
  \l 215C8
  \l 215C9
  \l 215CA
  \l 215CB
  \l 215CC
  \l 215CD
  \l 215CE
  \l 215CF
  \l 215D0
  \l 215D1
  \l 215D2
  \l 215D3
  \l 215D4
  \l 215D5
  \l 215D6
  \l 215D7
  \l 215D8
  \l 215D9
  \l 215DA
  \l 215DB
  \l 215DC
  \l 215DD
  \l 215DE
  \l 215DF
  \l 215E0
  \l 215E1
  \l 215E2
  \l 215E3
  \l 215E4
  \l 215E5
  \l 215E6
  \l 215E7
  \l 215E8
  \l 215E9
  \l 215EA
  \l 215EB
  \l 215EC
  \l 215ED
  \l 215EE
  \l 215EF
  \l 215F0
  \l 215F1
  \l 215F2
  \l 215F3
  \l 215F4
  \l 215F5
  \l 215F6
  \l 215F7
  \l 215F8
  \l 215F9
  \l 215FA
  \l 215FB
  \l 215FC
  \l 215FD
  \l 215FE
  \l 215FF
  \l 21600
  \l 21601
  \l 21602
  \l 21603
  \l 21604
  \l 21605
  \l 21606
  \l 21607
  \l 21608
  \l 21609
  \l 2160A
  \l 2160B
  \l 2160C
  \l 2160D
  \l 2160E
  \l 2160F
  \l 21610
  \l 21611
  \l 21612
  \l 21613
  \l 21614
  \l 21615
  \l 21616
  \l 21617
  \l 21618
  \l 21619
  \l 2161A
  \l 2161B
  \l 2161C
  \l 2161D
  \l 2161E
  \l 2161F
  \l 21620
  \l 21621
  \l 21622
  \l 21623
  \l 21624
  \l 21625
  \l 21626
  \l 21627
  \l 21628
  \l 21629
  \l 2162A
  \l 2162B
  \l 2162C
  \l 2162D
  \l 2162E
  \l 2162F
  \l 21630
  \l 21631
  \l 21632
  \l 21633
  \l 21634
  \l 21635
  \l 21636
  \l 21637
  \l 21638
  \l 21639
  \l 2163A
  \l 2163B
  \l 2163C
  \l 2163D
  \l 2163E
  \l 2163F
  \l 21640
  \l 21641
  \l 21642
  \l 21643
  \l 21644
  \l 21645
  \l 21646
  \l 21647
  \l 21648
  \l 21649
  \l 2164A
  \l 2164B
  \l 2164C
  \l 2164D
  \l 2164E
  \l 2164F
  \l 21650
  \l 21651
  \l 21652
  \l 21653
  \l 21654
  \l 21655
  \l 21656
  \l 21657
  \l 21658
  \l 21659
  \l 2165A
  \l 2165B
  \l 2165C
  \l 2165D
  \l 2165E
  \l 2165F
  \l 21660
  \l 21661
  \l 21662
  \l 21663
  \l 21664
  \l 21665
  \l 21666
  \l 21667
  \l 21668
  \l 21669
  \l 2166A
  \l 2166B
  \l 2166C
  \l 2166D
  \l 2166E
  \l 2166F
  \l 21670
  \l 21671
  \l 21672
  \l 21673
  \l 21674
  \l 21675
  \l 21676
  \l 21677
  \l 21678
  \l 21679
  \l 2167A
  \l 2167B
  \l 2167C
  \l 2167D
  \l 2167E
  \l 2167F
  \l 21680
  \l 21681
  \l 21682
  \l 21683
  \l 21684
  \l 21685
  \l 21686
  \l 21687
  \l 21688
  \l 21689
  \l 2168A
  \l 2168B
  \l 2168C
  \l 2168D
  \l 2168E
  \l 2168F
  \l 21690
  \l 21691
  \l 21692
  \l 21693
  \l 21694
  \l 21695
  \l 21696
  \l 21697
  \l 21698
  \l 21699
  \l 2169A
  \l 2169B
  \l 2169C
  \l 2169D
  \l 2169E
  \l 2169F
  \l 216A0
  \l 216A1
  \l 216A2
  \l 216A3
  \l 216A4
  \l 216A5
  \l 216A6
  \l 216A7
  \l 216A8
  \l 216A9
  \l 216AA
  \l 216AB
  \l 216AC
  \l 216AD
  \l 216AE
  \l 216AF
  \l 216B0
  \l 216B1
  \l 216B2
  \l 216B3
  \l 216B4
  \l 216B5
  \l 216B6
  \l 216B7
  \l 216B8
  \l 216B9
  \l 216BA
  \l 216BB
  \l 216BC
  \l 216BD
  \l 216BE
  \l 216BF
  \l 216C0
  \l 216C1
  \l 216C2
  \l 216C3
  \l 216C4
  \l 216C5
  \l 216C6
  \l 216C7
  \l 216C8
  \l 216C9
  \l 216CA
  \l 216CB
  \l 216CC
  \l 216CD
  \l 216CE
  \l 216CF
  \l 216D0
  \l 216D1
  \l 216D2
  \l 216D3
  \l 216D4
  \l 216D5
  \l 216D6
  \l 216D7
  \l 216D8
  \l 216D9
  \l 216DA
  \l 216DB
  \l 216DC
  \l 216DD
  \l 216DE
  \l 216DF
  \l 216E0
  \l 216E1
  \l 216E2
  \l 216E3
  \l 216E4
  \l 216E5
  \l 216E6
  \l 216E7
  \l 216E8
  \l 216E9
  \l 216EA
  \l 216EB
  \l 216EC
  \l 216ED
  \l 216EE
  \l 216EF
  \l 216F0
  \l 216F1
  \l 216F2
  \l 216F3
  \l 216F4
  \l 216F5
  \l 216F6
  \l 216F7
  \l 216F8
  \l 216F9
  \l 216FA
  \l 216FB
  \l 216FC
  \l 216FD
  \l 216FE
  \l 216FF
  \l 21700
  \l 21701
  \l 21702
  \l 21703
  \l 21704
  \l 21705
  \l 21706
  \l 21707
  \l 21708
  \l 21709
  \l 2170A
  \l 2170B
  \l 2170C
  \l 2170D
  \l 2170E
  \l 2170F
  \l 21710
  \l 21711
  \l 21712
  \l 21713
  \l 21714
  \l 21715
  \l 21716
  \l 21717
  \l 21718
  \l 21719
  \l 2171A
  \l 2171B
  \l 2171C
  \l 2171D
  \l 2171E
  \l 2171F
  \l 21720
  \l 21721
  \l 21722
  \l 21723
  \l 21724
  \l 21725
  \l 21726
  \l 21727
  \l 21728
  \l 21729
  \l 2172A
  \l 2172B
  \l 2172C
  \l 2172D
  \l 2172E
  \l 2172F
  \l 21730
  \l 21731
  \l 21732
  \l 21733
  \l 21734
  \l 21735
  \l 21736
  \l 21737
  \l 21738
  \l 21739
  \l 2173A
  \l 2173B
  \l 2173C
  \l 2173D
  \l 2173E
  \l 2173F
  \l 21740
  \l 21741
  \l 21742
  \l 21743
  \l 21744
  \l 21745
  \l 21746
  \l 21747
  \l 21748
  \l 21749
  \l 2174A
  \l 2174B
  \l 2174C
  \l 2174D
  \l 2174E
  \l 2174F
  \l 21750
  \l 21751
  \l 21752
  \l 21753
  \l 21754
  \l 21755
  \l 21756
  \l 21757
  \l 21758
  \l 21759
  \l 2175A
  \l 2175B
  \l 2175C
  \l 2175D
  \l 2175E
  \l 2175F
  \l 21760
  \l 21761
  \l 21762
  \l 21763
  \l 21764
  \l 21765
  \l 21766
  \l 21767
  \l 21768
  \l 21769
  \l 2176A
  \l 2176B
  \l 2176C
  \l 2176D
  \l 2176E
  \l 2176F
  \l 21770
  \l 21771
  \l 21772
  \l 21773
  \l 21774
  \l 21775
  \l 21776
  \l 21777
  \l 21778
  \l 21779
  \l 2177A
  \l 2177B
  \l 2177C
  \l 2177D
  \l 2177E
  \l 2177F
  \l 21780
  \l 21781
  \l 21782
  \l 21783
  \l 21784
  \l 21785
  \l 21786
  \l 21787
  \l 21788
  \l 21789
  \l 2178A
  \l 2178B
  \l 2178C
  \l 2178D
  \l 2178E
  \l 2178F
  \l 21790
  \l 21791
  \l 21792
  \l 21793
  \l 21794
  \l 21795
  \l 21796
  \l 21797
  \l 21798
  \l 21799
  \l 2179A
  \l 2179B
  \l 2179C
  \l 2179D
  \l 2179E
  \l 2179F
  \l 217A0
  \l 217A1
  \l 217A2
  \l 217A3
  \l 217A4
  \l 217A5
  \l 217A6
  \l 217A7
  \l 217A8
  \l 217A9
  \l 217AA
  \l 217AB
  \l 217AC
  \l 217AD
  \l 217AE
  \l 217AF
  \l 217B0
  \l 217B1
  \l 217B2
  \l 217B3
  \l 217B4
  \l 217B5
  \l 217B6
  \l 217B7
  \l 217B8
  \l 217B9
  \l 217BA
  \l 217BB
  \l 217BC
  \l 217BD
  \l 217BE
  \l 217BF
  \l 217C0
  \l 217C1
  \l 217C2
  \l 217C3
  \l 217C4
  \l 217C5
  \l 217C6
  \l 217C7
  \l 217C8
  \l 217C9
  \l 217CA
  \l 217CB
  \l 217CC
  \l 217CD
  \l 217CE
  \l 217CF
  \l 217D0
  \l 217D1
  \l 217D2
  \l 217D3
  \l 217D4
  \l 217D5
  \l 217D6
  \l 217D7
  \l 217D8
  \l 217D9
  \l 217DA
  \l 217DB
  \l 217DC
  \l 217DD
  \l 217DE
  \l 217DF
  \l 217E0
  \l 217E1
  \l 217E2
  \l 217E3
  \l 217E4
  \l 217E5
  \l 217E6
  \l 217E7
  \l 217E8
  \l 217E9
  \l 217EA
  \l 217EB
  \l 217EC
  \l 217ED
  \l 217EE
  \l 217EF
  \l 217F0
  \l 217F1
  \l 217F2
  \l 217F3
  \l 217F4
  \l 217F5
  \l 217F6
  \l 217F7
  \l 217F8
  \l 217F9
  \l 217FA
  \l 217FB
  \l 217FC
  \l 217FD
  \l 217FE
  \l 217FF
  \l 21800
  \l 21801
  \l 21802
  \l 21803
  \l 21804
  \l 21805
  \l 21806
  \l 21807
  \l 21808
  \l 21809
  \l 2180A
  \l 2180B
  \l 2180C
  \l 2180D
  \l 2180E
  \l 2180F
  \l 21810
  \l 21811
  \l 21812
  \l 21813
  \l 21814
  \l 21815
  \l 21816
  \l 21817
  \l 21818
  \l 21819
  \l 2181A
  \l 2181B
  \l 2181C
  \l 2181D
  \l 2181E
  \l 2181F
  \l 21820
  \l 21821
  \l 21822
  \l 21823
  \l 21824
  \l 21825
  \l 21826
  \l 21827
  \l 21828
  \l 21829
  \l 2182A
  \l 2182B
  \l 2182C
  \l 2182D
  \l 2182E
  \l 2182F
  \l 21830
  \l 21831
  \l 21832
  \l 21833
  \l 21834
  \l 21835
  \l 21836
  \l 21837
  \l 21838
  \l 21839
  \l 2183A
  \l 2183B
  \l 2183C
  \l 2183D
  \l 2183E
  \l 2183F
  \l 21840
  \l 21841
  \l 21842
  \l 21843
  \l 21844
  \l 21845
  \l 21846
  \l 21847
  \l 21848
  \l 21849
  \l 2184A
  \l 2184B
  \l 2184C
  \l 2184D
  \l 2184E
  \l 2184F
  \l 21850
  \l 21851
  \l 21852
  \l 21853
  \l 21854
  \l 21855
  \l 21856
  \l 21857
  \l 21858
  \l 21859
  \l 2185A
  \l 2185B
  \l 2185C
  \l 2185D
  \l 2185E
  \l 2185F
  \l 21860
  \l 21861
  \l 21862
  \l 21863
  \l 21864
  \l 21865
  \l 21866
  \l 21867
  \l 21868
  \l 21869
  \l 2186A
  \l 2186B
  \l 2186C
  \l 2186D
  \l 2186E
  \l 2186F
  \l 21870
  \l 21871
  \l 21872
  \l 21873
  \l 21874
  \l 21875
  \l 21876
  \l 21877
  \l 21878
  \l 21879
  \l 2187A
  \l 2187B
  \l 2187C
  \l 2187D
  \l 2187E
  \l 2187F
  \l 21880
  \l 21881
  \l 21882
  \l 21883
  \l 21884
  \l 21885
  \l 21886
  \l 21887
  \l 21888
  \l 21889
  \l 2188A
  \l 2188B
  \l 2188C
  \l 2188D
  \l 2188E
  \l 2188F
  \l 21890
  \l 21891
  \l 21892
  \l 21893
  \l 21894
  \l 21895
  \l 21896
  \l 21897
  \l 21898
  \l 21899
  \l 2189A
  \l 2189B
  \l 2189C
  \l 2189D
  \l 2189E
  \l 2189F
  \l 218A0
  \l 218A1
  \l 218A2
  \l 218A3
  \l 218A4
  \l 218A5
  \l 218A6
  \l 218A7
  \l 218A8
  \l 218A9
  \l 218AA
  \l 218AB
  \l 218AC
  \l 218AD
  \l 218AE
  \l 218AF
  \l 218B0
  \l 218B1
  \l 218B2
  \l 218B3
  \l 218B4
  \l 218B5
  \l 218B6
  \l 218B7
  \l 218B8
  \l 218B9
  \l 218BA
  \l 218BB
  \l 218BC
  \l 218BD
  \l 218BE
  \l 218BF
  \l 218C0
  \l 218C1
  \l 218C2
  \l 218C3
  \l 218C4
  \l 218C5
  \l 218C6
  \l 218C7
  \l 218C8
  \l 218C9
  \l 218CA
  \l 218CB
  \l 218CC
  \l 218CD
  \l 218CE
  \l 218CF
  \l 218D0
  \l 218D1
  \l 218D2
  \l 218D3
  \l 218D4
  \l 218D5
  \l 218D6
  \l 218D7
  \l 218D8
  \l 218D9
  \l 218DA
  \l 218DB
  \l 218DC
  \l 218DD
  \l 218DE
  \l 218DF
  \l 218E0
  \l 218E1
  \l 218E2
  \l 218E3
  \l 218E4
  \l 218E5
  \l 218E6
  \l 218E7
  \l 218E8
  \l 218E9
  \l 218EA
  \l 218EB
  \l 218EC
  \l 218ED
  \l 218EE
  \l 218EF
  \l 218F0
  \l 218F1
  \l 218F2
  \l 218F3
  \l 218F4
  \l 218F5
  \l 218F6
  \l 218F7
  \l 218F8
  \l 218F9
  \l 218FA
  \l 218FB
  \l 218FC
  \l 218FD
  \l 218FE
  \l 218FF
  \l 21900
  \l 21901
  \l 21902
  \l 21903
  \l 21904
  \l 21905
  \l 21906
  \l 21907
  \l 21908
  \l 21909
  \l 2190A
  \l 2190B
  \l 2190C
  \l 2190D
  \l 2190E
  \l 2190F
  \l 21910
  \l 21911
  \l 21912
  \l 21913
  \l 21914
  \l 21915
  \l 21916
  \l 21917
  \l 21918
  \l 21919
  \l 2191A
  \l 2191B
  \l 2191C
  \l 2191D
  \l 2191E
  \l 2191F
  \l 21920
  \l 21921
  \l 21922
  \l 21923
  \l 21924
  \l 21925
  \l 21926
  \l 21927
  \l 21928
  \l 21929
  \l 2192A
  \l 2192B
  \l 2192C
  \l 2192D
  \l 2192E
  \l 2192F
  \l 21930
  \l 21931
  \l 21932
  \l 21933
  \l 21934
  \l 21935
  \l 21936
  \l 21937
  \l 21938
  \l 21939
  \l 2193A
  \l 2193B
  \l 2193C
  \l 2193D
  \l 2193E
  \l 2193F
  \l 21940
  \l 21941
  \l 21942
  \l 21943
  \l 21944
  \l 21945
  \l 21946
  \l 21947
  \l 21948
  \l 21949
  \l 2194A
  \l 2194B
  \l 2194C
  \l 2194D
  \l 2194E
  \l 2194F
  \l 21950
  \l 21951
  \l 21952
  \l 21953
  \l 21954
  \l 21955
  \l 21956
  \l 21957
  \l 21958
  \l 21959
  \l 2195A
  \l 2195B
  \l 2195C
  \l 2195D
  \l 2195E
  \l 2195F
  \l 21960
  \l 21961
  \l 21962
  \l 21963
  \l 21964
  \l 21965
  \l 21966
  \l 21967
  \l 21968
  \l 21969
  \l 2196A
  \l 2196B
  \l 2196C
  \l 2196D
  \l 2196E
  \l 2196F
  \l 21970
  \l 21971
  \l 21972
  \l 21973
  \l 21974
  \l 21975
  \l 21976
  \l 21977
  \l 21978
  \l 21979
  \l 2197A
  \l 2197B
  \l 2197C
  \l 2197D
  \l 2197E
  \l 2197F
  \l 21980
  \l 21981
  \l 21982
  \l 21983
  \l 21984
  \l 21985
  \l 21986
  \l 21987
  \l 21988
  \l 21989
  \l 2198A
  \l 2198B
  \l 2198C
  \l 2198D
  \l 2198E
  \l 2198F
  \l 21990
  \l 21991
  \l 21992
  \l 21993
  \l 21994
  \l 21995
  \l 21996
  \l 21997
  \l 21998
  \l 21999
  \l 2199A
  \l 2199B
  \l 2199C
  \l 2199D
  \l 2199E
  \l 2199F
  \l 219A0
  \l 219A1
  \l 219A2
  \l 219A3
  \l 219A4
  \l 219A5
  \l 219A6
  \l 219A7
  \l 219A8
  \l 219A9
  \l 219AA
  \l 219AB
  \l 219AC
  \l 219AD
  \l 219AE
  \l 219AF
  \l 219B0
  \l 219B1
  \l 219B2
  \l 219B3
  \l 219B4
  \l 219B5
  \l 219B6
  \l 219B7
  \l 219B8
  \l 219B9
  \l 219BA
  \l 219BB
  \l 219BC
  \l 219BD
  \l 219BE
  \l 219BF
  \l 219C0
  \l 219C1
  \l 219C2
  \l 219C3
  \l 219C4
  \l 219C5
  \l 219C6
  \l 219C7
  \l 219C8
  \l 219C9
  \l 219CA
  \l 219CB
  \l 219CC
  \l 219CD
  \l 219CE
  \l 219CF
  \l 219D0
  \l 219D1
  \l 219D2
  \l 219D3
  \l 219D4
  \l 219D5
  \l 219D6
  \l 219D7
  \l 219D8
  \l 219D9
  \l 219DA
  \l 219DB
  \l 219DC
  \l 219DD
  \l 219DE
  \l 219DF
  \l 219E0
  \l 219E1
  \l 219E2
  \l 219E3
  \l 219E4
  \l 219E5
  \l 219E6
  \l 219E7
  \l 219E8
  \l 219E9
  \l 219EA
  \l 219EB
  \l 219EC
  \l 219ED
  \l 219EE
  \l 219EF
  \l 219F0
  \l 219F1
  \l 219F2
  \l 219F3
  \l 219F4
  \l 219F5
  \l 219F6
  \l 219F7
  \l 219F8
  \l 219F9
  \l 219FA
  \l 219FB
  \l 219FC
  \l 219FD
  \l 219FE
  \l 219FF
  \l 21A00
  \l 21A01
  \l 21A02
  \l 21A03
  \l 21A04
  \l 21A05
  \l 21A06
  \l 21A07
  \l 21A08
  \l 21A09
  \l 21A0A
  \l 21A0B
  \l 21A0C
  \l 21A0D
  \l 21A0E
  \l 21A0F
  \l 21A10
  \l 21A11
  \l 21A12
  \l 21A13
  \l 21A14
  \l 21A15
  \l 21A16
  \l 21A17
  \l 21A18
  \l 21A19
  \l 21A1A
  \l 21A1B
  \l 21A1C
  \l 21A1D
  \l 21A1E
  \l 21A1F
  \l 21A20
  \l 21A21
  \l 21A22
  \l 21A23
  \l 21A24
  \l 21A25
  \l 21A26
  \l 21A27
  \l 21A28
  \l 21A29
  \l 21A2A
  \l 21A2B
  \l 21A2C
  \l 21A2D
  \l 21A2E
  \l 21A2F
  \l 21A30
  \l 21A31
  \l 21A32
  \l 21A33
  \l 21A34
  \l 21A35
  \l 21A36
  \l 21A37
  \l 21A38
  \l 21A39
  \l 21A3A
  \l 21A3B
  \l 21A3C
  \l 21A3D
  \l 21A3E
  \l 21A3F
  \l 21A40
  \l 21A41
  \l 21A42
  \l 21A43
  \l 21A44
  \l 21A45
  \l 21A46
  \l 21A47
  \l 21A48
  \l 21A49
  \l 21A4A
  \l 21A4B
  \l 21A4C
  \l 21A4D
  \l 21A4E
  \l 21A4F
  \l 21A50
  \l 21A51
  \l 21A52
  \l 21A53
  \l 21A54
  \l 21A55
  \l 21A56
  \l 21A57
  \l 21A58
  \l 21A59
  \l 21A5A
  \l 21A5B
  \l 21A5C
  \l 21A5D
  \l 21A5E
  \l 21A5F
  \l 21A60
  \l 21A61
  \l 21A62
  \l 21A63
  \l 21A64
  \l 21A65
  \l 21A66
  \l 21A67
  \l 21A68
  \l 21A69
  \l 21A6A
  \l 21A6B
  \l 21A6C
  \l 21A6D
  \l 21A6E
  \l 21A6F
  \l 21A70
  \l 21A71
  \l 21A72
  \l 21A73
  \l 21A74
  \l 21A75
  \l 21A76
  \l 21A77
  \l 21A78
  \l 21A79
  \l 21A7A
  \l 21A7B
  \l 21A7C
  \l 21A7D
  \l 21A7E
  \l 21A7F
  \l 21A80
  \l 21A81
  \l 21A82
  \l 21A83
  \l 21A84
  \l 21A85
  \l 21A86
  \l 21A87
  \l 21A88
  \l 21A89
  \l 21A8A
  \l 21A8B
  \l 21A8C
  \l 21A8D
  \l 21A8E
  \l 21A8F
  \l 21A90
  \l 21A91
  \l 21A92
  \l 21A93
  \l 21A94
  \l 21A95
  \l 21A96
  \l 21A97
  \l 21A98
  \l 21A99
  \l 21A9A
  \l 21A9B
  \l 21A9C
  \l 21A9D
  \l 21A9E
  \l 21A9F
  \l 21AA0
  \l 21AA1
  \l 21AA2
  \l 21AA3
  \l 21AA4
  \l 21AA5
  \l 21AA6
  \l 21AA7
  \l 21AA8
  \l 21AA9
  \l 21AAA
  \l 21AAB
  \l 21AAC
  \l 21AAD
  \l 21AAE
  \l 21AAF
  \l 21AB0
  \l 21AB1
  \l 21AB2
  \l 21AB3
  \l 21AB4
  \l 21AB5
  \l 21AB6
  \l 21AB7
  \l 21AB8
  \l 21AB9
  \l 21ABA
  \l 21ABB
  \l 21ABC
  \l 21ABD
  \l 21ABE
  \l 21ABF
  \l 21AC0
  \l 21AC1
  \l 21AC2
  \l 21AC3
  \l 21AC4
  \l 21AC5
  \l 21AC6
  \l 21AC7
  \l 21AC8
  \l 21AC9
  \l 21ACA
  \l 21ACB
  \l 21ACC
  \l 21ACD
  \l 21ACE
  \l 21ACF
  \l 21AD0
  \l 21AD1
  \l 21AD2
  \l 21AD3
  \l 21AD4
  \l 21AD5
  \l 21AD6
  \l 21AD7
  \l 21AD8
  \l 21AD9
  \l 21ADA
  \l 21ADB
  \l 21ADC
  \l 21ADD
  \l 21ADE
  \l 21ADF
  \l 21AE0
  \l 21AE1
  \l 21AE2
  \l 21AE3
  \l 21AE4
  \l 21AE5
  \l 21AE6
  \l 21AE7
  \l 21AE8
  \l 21AE9
  \l 21AEA
  \l 21AEB
  \l 21AEC
  \l 21AED
  \l 21AEE
  \l 21AEF
  \l 21AF0
  \l 21AF1
  \l 21AF2
  \l 21AF3
  \l 21AF4
  \l 21AF5
  \l 21AF6
  \l 21AF7
  \l 21AF8
  \l 21AF9
  \l 21AFA
  \l 21AFB
  \l 21AFC
  \l 21AFD
  \l 21AFE
  \l 21AFF
  \l 21B00
  \l 21B01
  \l 21B02
  \l 21B03
  \l 21B04
  \l 21B05
  \l 21B06
  \l 21B07
  \l 21B08
  \l 21B09
  \l 21B0A
  \l 21B0B
  \l 21B0C
  \l 21B0D
  \l 21B0E
  \l 21B0F
  \l 21B10
  \l 21B11
  \l 21B12
  \l 21B13
  \l 21B14
  \l 21B15
  \l 21B16
  \l 21B17
  \l 21B18
  \l 21B19
  \l 21B1A
  \l 21B1B
  \l 21B1C
  \l 21B1D
  \l 21B1E
  \l 21B1F
  \l 21B20
  \l 21B21
  \l 21B22
  \l 21B23
  \l 21B24
  \l 21B25
  \l 21B26
  \l 21B27
  \l 21B28
  \l 21B29
  \l 21B2A
  \l 21B2B
  \l 21B2C
  \l 21B2D
  \l 21B2E
  \l 21B2F
  \l 21B30
  \l 21B31
  \l 21B32
  \l 21B33
  \l 21B34
  \l 21B35
  \l 21B36
  \l 21B37
  \l 21B38
  \l 21B39
  \l 21B3A
  \l 21B3B
  \l 21B3C
  \l 21B3D
  \l 21B3E
  \l 21B3F
  \l 21B40
  \l 21B41
  \l 21B42
  \l 21B43
  \l 21B44
  \l 21B45
  \l 21B46
  \l 21B47
  \l 21B48
  \l 21B49
  \l 21B4A
  \l 21B4B
  \l 21B4C
  \l 21B4D
  \l 21B4E
  \l 21B4F
  \l 21B50
  \l 21B51
  \l 21B52
  \l 21B53
  \l 21B54
  \l 21B55
  \l 21B56
  \l 21B57
  \l 21B58
  \l 21B59
  \l 21B5A
  \l 21B5B
  \l 21B5C
  \l 21B5D
  \l 21B5E
  \l 21B5F
  \l 21B60
  \l 21B61
  \l 21B62
  \l 21B63
  \l 21B64
  \l 21B65
  \l 21B66
  \l 21B67
  \l 21B68
  \l 21B69
  \l 21B6A
  \l 21B6B
  \l 21B6C
  \l 21B6D
  \l 21B6E
  \l 21B6F
  \l 21B70
  \l 21B71
  \l 21B72
  \l 21B73
  \l 21B74
  \l 21B75
  \l 21B76
  \l 21B77
  \l 21B78
  \l 21B79
  \l 21B7A
  \l 21B7B
  \l 21B7C
  \l 21B7D
  \l 21B7E
  \l 21B7F
  \l 21B80
  \l 21B81
  \l 21B82
  \l 21B83
  \l 21B84
  \l 21B85
  \l 21B86
  \l 21B87
  \l 21B88
  \l 21B89
  \l 21B8A
  \l 21B8B
  \l 21B8C
  \l 21B8D
  \l 21B8E
  \l 21B8F
  \l 21B90
  \l 21B91
  \l 21B92
  \l 21B93
  \l 21B94
  \l 21B95
  \l 21B96
  \l 21B97
  \l 21B98
  \l 21B99
  \l 21B9A
  \l 21B9B
  \l 21B9C
  \l 21B9D
  \l 21B9E
  \l 21B9F
  \l 21BA0
  \l 21BA1
  \l 21BA2
  \l 21BA3
  \l 21BA4
  \l 21BA5
  \l 21BA6
  \l 21BA7
  \l 21BA8
  \l 21BA9
  \l 21BAA
  \l 21BAB
  \l 21BAC
  \l 21BAD
  \l 21BAE
  \l 21BAF
  \l 21BB0
  \l 21BB1
  \l 21BB2
  \l 21BB3
  \l 21BB4
  \l 21BB5
  \l 21BB6
  \l 21BB7
  \l 21BB8
  \l 21BB9
  \l 21BBA
  \l 21BBB
  \l 21BBC
  \l 21BBD
  \l 21BBE
  \l 21BBF
  \l 21BC0
  \l 21BC1
  \l 21BC2
  \l 21BC3
  \l 21BC4
  \l 21BC5
  \l 21BC6
  \l 21BC7
  \l 21BC8
  \l 21BC9
  \l 21BCA
  \l 21BCB
  \l 21BCC
  \l 21BCD
  \l 21BCE
  \l 21BCF
  \l 21BD0
  \l 21BD1
  \l 21BD2
  \l 21BD3
  \l 21BD4
  \l 21BD5
  \l 21BD6
  \l 21BD7
  \l 21BD8
  \l 21BD9
  \l 21BDA
  \l 21BDB
  \l 21BDC
  \l 21BDD
  \l 21BDE
  \l 21BDF
  \l 21BE0
  \l 21BE1
  \l 21BE2
  \l 21BE3
  \l 21BE4
  \l 21BE5
  \l 21BE6
  \l 21BE7
  \l 21BE8
  \l 21BE9
  \l 21BEA
  \l 21BEB
  \l 21BEC
  \l 21BED
  \l 21BEE
  \l 21BEF
  \l 21BF0
  \l 21BF1
  \l 21BF2
  \l 21BF3
  \l 21BF4
  \l 21BF5
  \l 21BF6
  \l 21BF7
  \l 21BF8
  \l 21BF9
  \l 21BFA
  \l 21BFB
  \l 21BFC
  \l 21BFD
  \l 21BFE
  \l 21BFF
  \l 21C00
  \l 21C01
  \l 21C02
  \l 21C03
  \l 21C04
  \l 21C05
  \l 21C06
  \l 21C07
  \l 21C08
  \l 21C09
  \l 21C0A
  \l 21C0B
  \l 21C0C
  \l 21C0D
  \l 21C0E
  \l 21C0F
  \l 21C10
  \l 21C11
  \l 21C12
  \l 21C13
  \l 21C14
  \l 21C15
  \l 21C16
  \l 21C17
  \l 21C18
  \l 21C19
  \l 21C1A
  \l 21C1B
  \l 21C1C
  \l 21C1D
  \l 21C1E
  \l 21C1F
  \l 21C20
  \l 21C21
  \l 21C22
  \l 21C23
  \l 21C24
  \l 21C25
  \l 21C26
  \l 21C27
  \l 21C28
  \l 21C29
  \l 21C2A
  \l 21C2B
  \l 21C2C
  \l 21C2D
  \l 21C2E
  \l 21C2F
  \l 21C30
  \l 21C31
  \l 21C32
  \l 21C33
  \l 21C34
  \l 21C35
  \l 21C36
  \l 21C37
  \l 21C38
  \l 21C39
  \l 21C3A
  \l 21C3B
  \l 21C3C
  \l 21C3D
  \l 21C3E
  \l 21C3F
  \l 21C40
  \l 21C41
  \l 21C42
  \l 21C43
  \l 21C44
  \l 21C45
  \l 21C46
  \l 21C47
  \l 21C48
  \l 21C49
  \l 21C4A
  \l 21C4B
  \l 21C4C
  \l 21C4D
  \l 21C4E
  \l 21C4F
  \l 21C50
  \l 21C51
  \l 21C52
  \l 21C53
  \l 21C54
  \l 21C55
  \l 21C56
  \l 21C57
  \l 21C58
  \l 21C59
  \l 21C5A
  \l 21C5B
  \l 21C5C
  \l 21C5D
  \l 21C5E
  \l 21C5F
  \l 21C60
  \l 21C61
  \l 21C62
  \l 21C63
  \l 21C64
  \l 21C65
  \l 21C66
  \l 21C67
  \l 21C68
  \l 21C69
  \l 21C6A
  \l 21C6B
  \l 21C6C
  \l 21C6D
  \l 21C6E
  \l 21C6F
  \l 21C70
  \l 21C71
  \l 21C72
  \l 21C73
  \l 21C74
  \l 21C75
  \l 21C76
  \l 21C77
  \l 21C78
  \l 21C79
  \l 21C7A
  \l 21C7B
  \l 21C7C
  \l 21C7D
  \l 21C7E
  \l 21C7F
  \l 21C80
  \l 21C81
  \l 21C82
  \l 21C83
  \l 21C84
  \l 21C85
  \l 21C86
  \l 21C87
  \l 21C88
  \l 21C89
  \l 21C8A
  \l 21C8B
  \l 21C8C
  \l 21C8D
  \l 21C8E
  \l 21C8F
  \l 21C90
  \l 21C91
  \l 21C92
  \l 21C93
  \l 21C94
  \l 21C95
  \l 21C96
  \l 21C97
  \l 21C98
  \l 21C99
  \l 21C9A
  \l 21C9B
  \l 21C9C
  \l 21C9D
  \l 21C9E
  \l 21C9F
  \l 21CA0
  \l 21CA1
  \l 21CA2
  \l 21CA3
  \l 21CA4
  \l 21CA5
  \l 21CA6
  \l 21CA7
  \l 21CA8
  \l 21CA9
  \l 21CAA
  \l 21CAB
  \l 21CAC
  \l 21CAD
  \l 21CAE
  \l 21CAF
  \l 21CB0
  \l 21CB1
  \l 21CB2
  \l 21CB3
  \l 21CB4
  \l 21CB5
  \l 21CB6
  \l 21CB7
  \l 21CB8
  \l 21CB9
  \l 21CBA
  \l 21CBB
  \l 21CBC
  \l 21CBD
  \l 21CBE
  \l 21CBF
  \l 21CC0
  \l 21CC1
  \l 21CC2
  \l 21CC3
  \l 21CC4
  \l 21CC5
  \l 21CC6
  \l 21CC7
  \l 21CC8
  \l 21CC9
  \l 21CCA
  \l 21CCB
  \l 21CCC
  \l 21CCD
  \l 21CCE
  \l 21CCF
  \l 21CD0
  \l 21CD1
  \l 21CD2
  \l 21CD3
  \l 21CD4
  \l 21CD5
  \l 21CD6
  \l 21CD7
  \l 21CD8
  \l 21CD9
  \l 21CDA
  \l 21CDB
  \l 21CDC
  \l 21CDD
  \l 21CDE
  \l 21CDF
  \l 21CE0
  \l 21CE1
  \l 21CE2
  \l 21CE3
  \l 21CE4
  \l 21CE5
  \l 21CE6
  \l 21CE7
  \l 21CE8
  \l 21CE9
  \l 21CEA
  \l 21CEB
  \l 21CEC
  \l 21CED
  \l 21CEE
  \l 21CEF
  \l 21CF0
  \l 21CF1
  \l 21CF2
  \l 21CF3
  \l 21CF4
  \l 21CF5
  \l 21CF6
  \l 21CF7
  \l 21CF8
  \l 21CF9
  \l 21CFA
  \l 21CFB
  \l 21CFC
  \l 21CFD
  \l 21CFE
  \l 21CFF
  \l 21D00
  \l 21D01
  \l 21D02
  \l 21D03
  \l 21D04
  \l 21D05
  \l 21D06
  \l 21D07
  \l 21D08
  \l 21D09
  \l 21D0A
  \l 21D0B
  \l 21D0C
  \l 21D0D
  \l 21D0E
  \l 21D0F
  \l 21D10
  \l 21D11
  \l 21D12
  \l 21D13
  \l 21D14
  \l 21D15
  \l 21D16
  \l 21D17
  \l 21D18
  \l 21D19
  \l 21D1A
  \l 21D1B
  \l 21D1C
  \l 21D1D
  \l 21D1E
  \l 21D1F
  \l 21D20
  \l 21D21
  \l 21D22
  \l 21D23
  \l 21D24
  \l 21D25
  \l 21D26
  \l 21D27
  \l 21D28
  \l 21D29
  \l 21D2A
  \l 21D2B
  \l 21D2C
  \l 21D2D
  \l 21D2E
  \l 21D2F
  \l 21D30
  \l 21D31
  \l 21D32
  \l 21D33
  \l 21D34
  \l 21D35
  \l 21D36
  \l 21D37
  \l 21D38
  \l 21D39
  \l 21D3A
  \l 21D3B
  \l 21D3C
  \l 21D3D
  \l 21D3E
  \l 21D3F
  \l 21D40
  \l 21D41
  \l 21D42
  \l 21D43
  \l 21D44
  \l 21D45
  \l 21D46
  \l 21D47
  \l 21D48
  \l 21D49
  \l 21D4A
  \l 21D4B
  \l 21D4C
  \l 21D4D
  \l 21D4E
  \l 21D4F
  \l 21D50
  \l 21D51
  \l 21D52
  \l 21D53
  \l 21D54
  \l 21D55
  \l 21D56
  \l 21D57
  \l 21D58
  \l 21D59
  \l 21D5A
  \l 21D5B
  \l 21D5C
  \l 21D5D
  \l 21D5E
  \l 21D5F
  \l 21D60
  \l 21D61
  \l 21D62
  \l 21D63
  \l 21D64
  \l 21D65
  \l 21D66
  \l 21D67
  \l 21D68
  \l 21D69
  \l 21D6A
  \l 21D6B
  \l 21D6C
  \l 21D6D
  \l 21D6E
  \l 21D6F
  \l 21D70
  \l 21D71
  \l 21D72
  \l 21D73
  \l 21D74
  \l 21D75
  \l 21D76
  \l 21D77
  \l 21D78
  \l 21D79
  \l 21D7A
  \l 21D7B
  \l 21D7C
  \l 21D7D
  \l 21D7E
  \l 21D7F
  \l 21D80
  \l 21D81
  \l 21D82
  \l 21D83
  \l 21D84
  \l 21D85
  \l 21D86
  \l 21D87
  \l 21D88
  \l 21D89
  \l 21D8A
  \l 21D8B
  \l 21D8C
  \l 21D8D
  \l 21D8E
  \l 21D8F
  \l 21D90
  \l 21D91
  \l 21D92
  \l 21D93
  \l 21D94
  \l 21D95
  \l 21D96
  \l 21D97
  \l 21D98
  \l 21D99
  \l 21D9A
  \l 21D9B
  \l 21D9C
  \l 21D9D
  \l 21D9E
  \l 21D9F
  \l 21DA0
  \l 21DA1
  \l 21DA2
  \l 21DA3
  \l 21DA4
  \l 21DA5
  \l 21DA6
  \l 21DA7
  \l 21DA8
  \l 21DA9
  \l 21DAA
  \l 21DAB
  \l 21DAC
  \l 21DAD
  \l 21DAE
  \l 21DAF
  \l 21DB0
  \l 21DB1
  \l 21DB2
  \l 21DB3
  \l 21DB4
  \l 21DB5
  \l 21DB6
  \l 21DB7
  \l 21DB8
  \l 21DB9
  \l 21DBA
  \l 21DBB
  \l 21DBC
  \l 21DBD
  \l 21DBE
  \l 21DBF
  \l 21DC0
  \l 21DC1
  \l 21DC2
  \l 21DC3
  \l 21DC4
  \l 21DC5
  \l 21DC6
  \l 21DC7
  \l 21DC8
  \l 21DC9
  \l 21DCA
  \l 21DCB
  \l 21DCC
  \l 21DCD
  \l 21DCE
  \l 21DCF
  \l 21DD0
  \l 21DD1
  \l 21DD2
  \l 21DD3
  \l 21DD4
  \l 21DD5
  \l 21DD6
  \l 21DD7
  \l 21DD8
  \l 21DD9
  \l 21DDA
  \l 21DDB
  \l 21DDC
  \l 21DDD
  \l 21DDE
  \l 21DDF
  \l 21DE0
  \l 21DE1
  \l 21DE2
  \l 21DE3
  \l 21DE4
  \l 21DE5
  \l 21DE6
  \l 21DE7
  \l 21DE8
  \l 21DE9
  \l 21DEA
  \l 21DEB
  \l 21DEC
  \l 21DED
  \l 21DEE
  \l 21DEF
  \l 21DF0
  \l 21DF1
  \l 21DF2
  \l 21DF3
  \l 21DF4
  \l 21DF5
  \l 21DF6
  \l 21DF7
  \l 21DF8
  \l 21DF9
  \l 21DFA
  \l 21DFB
  \l 21DFC
  \l 21DFD
  \l 21DFE
  \l 21DFF
  \l 21E00
  \l 21E01
  \l 21E02
  \l 21E03
  \l 21E04
  \l 21E05
  \l 21E06
  \l 21E07
  \l 21E08
  \l 21E09
  \l 21E0A
  \l 21E0B
  \l 21E0C
  \l 21E0D
  \l 21E0E
  \l 21E0F
  \l 21E10
  \l 21E11
  \l 21E12
  \l 21E13
  \l 21E14
  \l 21E15
  \l 21E16
  \l 21E17
  \l 21E18
  \l 21E19
  \l 21E1A
  \l 21E1B
  \l 21E1C
  \l 21E1D
  \l 21E1E
  \l 21E1F
  \l 21E20
  \l 21E21
  \l 21E22
  \l 21E23
  \l 21E24
  \l 21E25
  \l 21E26
  \l 21E27
  \l 21E28
  \l 21E29
  \l 21E2A
  \l 21E2B
  \l 21E2C
  \l 21E2D
  \l 21E2E
  \l 21E2F
  \l 21E30
  \l 21E31
  \l 21E32
  \l 21E33
  \l 21E34
  \l 21E35
  \l 21E36
  \l 21E37
  \l 21E38
  \l 21E39
  \l 21E3A
  \l 21E3B
  \l 21E3C
  \l 21E3D
  \l 21E3E
  \l 21E3F
  \l 21E40
  \l 21E41
  \l 21E42
  \l 21E43
  \l 21E44
  \l 21E45
  \l 21E46
  \l 21E47
  \l 21E48
  \l 21E49
  \l 21E4A
  \l 21E4B
  \l 21E4C
  \l 21E4D
  \l 21E4E
  \l 21E4F
  \l 21E50
  \l 21E51
  \l 21E52
  \l 21E53
  \l 21E54
  \l 21E55
  \l 21E56
  \l 21E57
  \l 21E58
  \l 21E59
  \l 21E5A
  \l 21E5B
  \l 21E5C
  \l 21E5D
  \l 21E5E
  \l 21E5F
  \l 21E60
  \l 21E61
  \l 21E62
  \l 21E63
  \l 21E64
  \l 21E65
  \l 21E66
  \l 21E67
  \l 21E68
  \l 21E69
  \l 21E6A
  \l 21E6B
  \l 21E6C
  \l 21E6D
  \l 21E6E
  \l 21E6F
  \l 21E70
  \l 21E71
  \l 21E72
  \l 21E73
  \l 21E74
  \l 21E75
  \l 21E76
  \l 21E77
  \l 21E78
  \l 21E79
  \l 21E7A
  \l 21E7B
  \l 21E7C
  \l 21E7D
  \l 21E7E
  \l 21E7F
  \l 21E80
  \l 21E81
  \l 21E82
  \l 21E83
  \l 21E84
  \l 21E85
  \l 21E86
  \l 21E87
  \l 21E88
  \l 21E89
  \l 21E8A
  \l 21E8B
  \l 21E8C
  \l 21E8D
  \l 21E8E
  \l 21E8F
  \l 21E90
  \l 21E91
  \l 21E92
  \l 21E93
  \l 21E94
  \l 21E95
  \l 21E96
  \l 21E97
  \l 21E98
  \l 21E99
  \l 21E9A
  \l 21E9B
  \l 21E9C
  \l 21E9D
  \l 21E9E
  \l 21E9F
  \l 21EA0
  \l 21EA1
  \l 21EA2
  \l 21EA3
  \l 21EA4
  \l 21EA5
  \l 21EA6
  \l 21EA7
  \l 21EA8
  \l 21EA9
  \l 21EAA
  \l 21EAB
  \l 21EAC
  \l 21EAD
  \l 21EAE
  \l 21EAF
  \l 21EB0
  \l 21EB1
  \l 21EB2
  \l 21EB3
  \l 21EB4
  \l 21EB5
  \l 21EB6
  \l 21EB7
  \l 21EB8
  \l 21EB9
  \l 21EBA
  \l 21EBB
  \l 21EBC
  \l 21EBD
  \l 21EBE
  \l 21EBF
  \l 21EC0
  \l 21EC1
  \l 21EC2
  \l 21EC3
  \l 21EC4
  \l 21EC5
  \l 21EC6
  \l 21EC7
  \l 21EC8
  \l 21EC9
  \l 21ECA
  \l 21ECB
  \l 21ECC
  \l 21ECD
  \l 21ECE
  \l 21ECF
  \l 21ED0
  \l 21ED1
  \l 21ED2
  \l 21ED3
  \l 21ED4
  \l 21ED5
  \l 21ED6
  \l 21ED7
  \l 21ED8
  \l 21ED9
  \l 21EDA
  \l 21EDB
  \l 21EDC
  \l 21EDD
  \l 21EDE
  \l 21EDF
  \l 21EE0
  \l 21EE1
  \l 21EE2
  \l 21EE3
  \l 21EE4
  \l 21EE5
  \l 21EE6
  \l 21EE7
  \l 21EE8
  \l 21EE9
  \l 21EEA
  \l 21EEB
  \l 21EEC
  \l 21EED
  \l 21EEE
  \l 21EEF
  \l 21EF0
  \l 21EF1
  \l 21EF2
  \l 21EF3
  \l 21EF4
  \l 21EF5
  \l 21EF6
  \l 21EF7
  \l 21EF8
  \l 21EF9
  \l 21EFA
  \l 21EFB
  \l 21EFC
  \l 21EFD
  \l 21EFE
  \l 21EFF
  \l 21F00
  \l 21F01
  \l 21F02
  \l 21F03
  \l 21F04
  \l 21F05
  \l 21F06
  \l 21F07
  \l 21F08
  \l 21F09
  \l 21F0A
  \l 21F0B
  \l 21F0C
  \l 21F0D
  \l 21F0E
  \l 21F0F
  \l 21F10
  \l 21F11
  \l 21F12
  \l 21F13
  \l 21F14
  \l 21F15
  \l 21F16
  \l 21F17
  \l 21F18
  \l 21F19
  \l 21F1A
  \l 21F1B
  \l 21F1C
  \l 21F1D
  \l 21F1E
  \l 21F1F
  \l 21F20
  \l 21F21
  \l 21F22
  \l 21F23
  \l 21F24
  \l 21F25
  \l 21F26
  \l 21F27
  \l 21F28
  \l 21F29
  \l 21F2A
  \l 21F2B
  \l 21F2C
  \l 21F2D
  \l 21F2E
  \l 21F2F
  \l 21F30
  \l 21F31
  \l 21F32
  \l 21F33
  \l 21F34
  \l 21F35
  \l 21F36
  \l 21F37
  \l 21F38
  \l 21F39
  \l 21F3A
  \l 21F3B
  \l 21F3C
  \l 21F3D
  \l 21F3E
  \l 21F3F
  \l 21F40
  \l 21F41
  \l 21F42
  \l 21F43
  \l 21F44
  \l 21F45
  \l 21F46
  \l 21F47
  \l 21F48
  \l 21F49
  \l 21F4A
  \l 21F4B
  \l 21F4C
  \l 21F4D
  \l 21F4E
  \l 21F4F
  \l 21F50
  \l 21F51
  \l 21F52
  \l 21F53
  \l 21F54
  \l 21F55
  \l 21F56
  \l 21F57
  \l 21F58
  \l 21F59
  \l 21F5A
  \l 21F5B
  \l 21F5C
  \l 21F5D
  \l 21F5E
  \l 21F5F
  \l 21F60
  \l 21F61
  \l 21F62
  \l 21F63
  \l 21F64
  \l 21F65
  \l 21F66
  \l 21F67
  \l 21F68
  \l 21F69
  \l 21F6A
  \l 21F6B
  \l 21F6C
  \l 21F6D
  \l 21F6E
  \l 21F6F
  \l 21F70
  \l 21F71
  \l 21F72
  \l 21F73
  \l 21F74
  \l 21F75
  \l 21F76
  \l 21F77
  \l 21F78
  \l 21F79
  \l 21F7A
  \l 21F7B
  \l 21F7C
  \l 21F7D
  \l 21F7E
  \l 21F7F
  \l 21F80
  \l 21F81
  \l 21F82
  \l 21F83
  \l 21F84
  \l 21F85
  \l 21F86
  \l 21F87
  \l 21F88
  \l 21F89
  \l 21F8A
  \l 21F8B
  \l 21F8C
  \l 21F8D
  \l 21F8E
  \l 21F8F
  \l 21F90
  \l 21F91
  \l 21F92
  \l 21F93
  \l 21F94
  \l 21F95
  \l 21F96
  \l 21F97
  \l 21F98
  \l 21F99
  \l 21F9A
  \l 21F9B
  \l 21F9C
  \l 21F9D
  \l 21F9E
  \l 21F9F
  \l 21FA0
  \l 21FA1
  \l 21FA2
  \l 21FA3
  \l 21FA4
  \l 21FA5
  \l 21FA6
  \l 21FA7
  \l 21FA8
  \l 21FA9
  \l 21FAA
  \l 21FAB
  \l 21FAC
  \l 21FAD
  \l 21FAE
  \l 21FAF
  \l 21FB0
  \l 21FB1
  \l 21FB2
  \l 21FB3
  \l 21FB4
  \l 21FB5
  \l 21FB6
  \l 21FB7
  \l 21FB8
  \l 21FB9
  \l 21FBA
  \l 21FBB
  \l 21FBC
  \l 21FBD
  \l 21FBE
  \l 21FBF
  \l 21FC0
  \l 21FC1
  \l 21FC2
  \l 21FC3
  \l 21FC4
  \l 21FC5
  \l 21FC6
  \l 21FC7
  \l 21FC8
  \l 21FC9
  \l 21FCA
  \l 21FCB
  \l 21FCC
  \l 21FCD
  \l 21FCE
  \l 21FCF
  \l 21FD0
  \l 21FD1
  \l 21FD2
  \l 21FD3
  \l 21FD4
  \l 21FD5
  \l 21FD6
  \l 21FD7
  \l 21FD8
  \l 21FD9
  \l 21FDA
  \l 21FDB
  \l 21FDC
  \l 21FDD
  \l 21FDE
  \l 21FDF
  \l 21FE0
  \l 21FE1
  \l 21FE2
  \l 21FE3
  \l 21FE4
  \l 21FE5
  \l 21FE6
  \l 21FE7
  \l 21FE8
  \l 21FE9
  \l 21FEA
  \l 21FEB
  \l 21FEC
  \l 21FED
  \l 21FEE
  \l 21FEF
  \l 21FF0
  \l 21FF1
  \l 21FF2
  \l 21FF3
  \l 21FF4
  \l 21FF5
  \l 21FF6
  \l 21FF7
  \l 21FF8
  \l 21FF9
  \l 21FFA
  \l 21FFB
  \l 21FFC
  \l 21FFD
  \l 21FFE
  \l 21FFF
  \l 22000
  \l 22001
  \l 22002
  \l 22003
  \l 22004
  \l 22005
  \l 22006
  \l 22007
  \l 22008
  \l 22009
  \l 2200A
  \l 2200B
  \l 2200C
  \l 2200D
  \l 2200E
  \l 2200F
  \l 22010
  \l 22011
  \l 22012
  \l 22013
  \l 22014
  \l 22015
  \l 22016
  \l 22017
  \l 22018
  \l 22019
  \l 2201A
  \l 2201B
  \l 2201C
  \l 2201D
  \l 2201E
  \l 2201F
  \l 22020
  \l 22021
  \l 22022
  \l 22023
  \l 22024
  \l 22025
  \l 22026
  \l 22027
  \l 22028
  \l 22029
  \l 2202A
  \l 2202B
  \l 2202C
  \l 2202D
  \l 2202E
  \l 2202F
  \l 22030
  \l 22031
  \l 22032
  \l 22033
  \l 22034
  \l 22035
  \l 22036
  \l 22037
  \l 22038
  \l 22039
  \l 2203A
  \l 2203B
  \l 2203C
  \l 2203D
  \l 2203E
  \l 2203F
  \l 22040
  \l 22041
  \l 22042
  \l 22043
  \l 22044
  \l 22045
  \l 22046
  \l 22047
  \l 22048
  \l 22049
  \l 2204A
  \l 2204B
  \l 2204C
  \l 2204D
  \l 2204E
  \l 2204F
  \l 22050
  \l 22051
  \l 22052
  \l 22053
  \l 22054
  \l 22055
  \l 22056
  \l 22057
  \l 22058
  \l 22059
  \l 2205A
  \l 2205B
  \l 2205C
  \l 2205D
  \l 2205E
  \l 2205F
  \l 22060
  \l 22061
  \l 22062
  \l 22063
  \l 22064
  \l 22065
  \l 22066
  \l 22067
  \l 22068
  \l 22069
  \l 2206A
  \l 2206B
  \l 2206C
  \l 2206D
  \l 2206E
  \l 2206F
  \l 22070
  \l 22071
  \l 22072
  \l 22073
  \l 22074
  \l 22075
  \l 22076
  \l 22077
  \l 22078
  \l 22079
  \l 2207A
  \l 2207B
  \l 2207C
  \l 2207D
  \l 2207E
  \l 2207F
  \l 22080
  \l 22081
  \l 22082
  \l 22083
  \l 22084
  \l 22085
  \l 22086
  \l 22087
  \l 22088
  \l 22089
  \l 2208A
  \l 2208B
  \l 2208C
  \l 2208D
  \l 2208E
  \l 2208F
  \l 22090
  \l 22091
  \l 22092
  \l 22093
  \l 22094
  \l 22095
  \l 22096
  \l 22097
  \l 22098
  \l 22099
  \l 2209A
  \l 2209B
  \l 2209C
  \l 2209D
  \l 2209E
  \l 2209F
  \l 220A0
  \l 220A1
  \l 220A2
  \l 220A3
  \l 220A4
  \l 220A5
  \l 220A6
  \l 220A7
  \l 220A8
  \l 220A9
  \l 220AA
  \l 220AB
  \l 220AC
  \l 220AD
  \l 220AE
  \l 220AF
  \l 220B0
  \l 220B1
  \l 220B2
  \l 220B3
  \l 220B4
  \l 220B5
  \l 220B6
  \l 220B7
  \l 220B8
  \l 220B9
  \l 220BA
  \l 220BB
  \l 220BC
  \l 220BD
  \l 220BE
  \l 220BF
  \l 220C0
  \l 220C1
  \l 220C2
  \l 220C3
  \l 220C4
  \l 220C5
  \l 220C6
  \l 220C7
  \l 220C8
  \l 220C9
  \l 220CA
  \l 220CB
  \l 220CC
  \l 220CD
  \l 220CE
  \l 220CF
  \l 220D0
  \l 220D1
  \l 220D2
  \l 220D3
  \l 220D4
  \l 220D5
  \l 220D6
  \l 220D7
  \l 220D8
  \l 220D9
  \l 220DA
  \l 220DB
  \l 220DC
  \l 220DD
  \l 220DE
  \l 220DF
  \l 220E0
  \l 220E1
  \l 220E2
  \l 220E3
  \l 220E4
  \l 220E5
  \l 220E6
  \l 220E7
  \l 220E8
  \l 220E9
  \l 220EA
  \l 220EB
  \l 220EC
  \l 220ED
  \l 220EE
  \l 220EF
  \l 220F0
  \l 220F1
  \l 220F2
  \l 220F3
  \l 220F4
  \l 220F5
  \l 220F6
  \l 220F7
  \l 220F8
  \l 220F9
  \l 220FA
  \l 220FB
  \l 220FC
  \l 220FD
  \l 220FE
  \l 220FF
  \l 22100
  \l 22101
  \l 22102
  \l 22103
  \l 22104
  \l 22105
  \l 22106
  \l 22107
  \l 22108
  \l 22109
  \l 2210A
  \l 2210B
  \l 2210C
  \l 2210D
  \l 2210E
  \l 2210F
  \l 22110
  \l 22111
  \l 22112
  \l 22113
  \l 22114
  \l 22115
  \l 22116
  \l 22117
  \l 22118
  \l 22119
  \l 2211A
  \l 2211B
  \l 2211C
  \l 2211D
  \l 2211E
  \l 2211F
  \l 22120
  \l 22121
  \l 22122
  \l 22123
  \l 22124
  \l 22125
  \l 22126
  \l 22127
  \l 22128
  \l 22129
  \l 2212A
  \l 2212B
  \l 2212C
  \l 2212D
  \l 2212E
  \l 2212F
  \l 22130
  \l 22131
  \l 22132
  \l 22133
  \l 22134
  \l 22135
  \l 22136
  \l 22137
  \l 22138
  \l 22139
  \l 2213A
  \l 2213B
  \l 2213C
  \l 2213D
  \l 2213E
  \l 2213F
  \l 22140
  \l 22141
  \l 22142
  \l 22143
  \l 22144
  \l 22145
  \l 22146
  \l 22147
  \l 22148
  \l 22149
  \l 2214A
  \l 2214B
  \l 2214C
  \l 2214D
  \l 2214E
  \l 2214F
  \l 22150
  \l 22151
  \l 22152
  \l 22153
  \l 22154
  \l 22155
  \l 22156
  \l 22157
  \l 22158
  \l 22159
  \l 2215A
  \l 2215B
  \l 2215C
  \l 2215D
  \l 2215E
  \l 2215F
  \l 22160
  \l 22161
  \l 22162
  \l 22163
  \l 22164
  \l 22165
  \l 22166
  \l 22167
  \l 22168
  \l 22169
  \l 2216A
  \l 2216B
  \l 2216C
  \l 2216D
  \l 2216E
  \l 2216F
  \l 22170
  \l 22171
  \l 22172
  \l 22173
  \l 22174
  \l 22175
  \l 22176
  \l 22177
  \l 22178
  \l 22179
  \l 2217A
  \l 2217B
  \l 2217C
  \l 2217D
  \l 2217E
  \l 2217F
  \l 22180
  \l 22181
  \l 22182
  \l 22183
  \l 22184
  \l 22185
  \l 22186
  \l 22187
  \l 22188
  \l 22189
  \l 2218A
  \l 2218B
  \l 2218C
  \l 2218D
  \l 2218E
  \l 2218F
  \l 22190
  \l 22191
  \l 22192
  \l 22193
  \l 22194
  \l 22195
  \l 22196
  \l 22197
  \l 22198
  \l 22199
  \l 2219A
  \l 2219B
  \l 2219C
  \l 2219D
  \l 2219E
  \l 2219F
  \l 221A0
  \l 221A1
  \l 221A2
  \l 221A3
  \l 221A4
  \l 221A5
  \l 221A6
  \l 221A7
  \l 221A8
  \l 221A9
  \l 221AA
  \l 221AB
  \l 221AC
  \l 221AD
  \l 221AE
  \l 221AF
  \l 221B0
  \l 221B1
  \l 221B2
  \l 221B3
  \l 221B4
  \l 221B5
  \l 221B6
  \l 221B7
  \l 221B8
  \l 221B9
  \l 221BA
  \l 221BB
  \l 221BC
  \l 221BD
  \l 221BE
  \l 221BF
  \l 221C0
  \l 221C1
  \l 221C2
  \l 221C3
  \l 221C4
  \l 221C5
  \l 221C6
  \l 221C7
  \l 221C8
  \l 221C9
  \l 221CA
  \l 221CB
  \l 221CC
  \l 221CD
  \l 221CE
  \l 221CF
  \l 221D0
  \l 221D1
  \l 221D2
  \l 221D3
  \l 221D4
  \l 221D5
  \l 221D6
  \l 221D7
  \l 221D8
  \l 221D9
  \l 221DA
  \l 221DB
  \l 221DC
  \l 221DD
  \l 221DE
  \l 221DF
  \l 221E0
  \l 221E1
  \l 221E2
  \l 221E3
  \l 221E4
  \l 221E5
  \l 221E6
  \l 221E7
  \l 221E8
  \l 221E9
  \l 221EA
  \l 221EB
  \l 221EC
  \l 221ED
  \l 221EE
  \l 221EF
  \l 221F0
  \l 221F1
  \l 221F2
  \l 221F3
  \l 221F4
  \l 221F5
  \l 221F6
  \l 221F7
  \l 221F8
  \l 221F9
  \l 221FA
  \l 221FB
  \l 221FC
  \l 221FD
  \l 221FE
  \l 221FF
  \l 22200
  \l 22201
  \l 22202
  \l 22203
  \l 22204
  \l 22205
  \l 22206
  \l 22207
  \l 22208
  \l 22209
  \l 2220A
  \l 2220B
  \l 2220C
  \l 2220D
  \l 2220E
  \l 2220F
  \l 22210
  \l 22211
  \l 22212
  \l 22213
  \l 22214
  \l 22215
  \l 22216
  \l 22217
  \l 22218
  \l 22219
  \l 2221A
  \l 2221B
  \l 2221C
  \l 2221D
  \l 2221E
  \l 2221F
  \l 22220
  \l 22221
  \l 22222
  \l 22223
  \l 22224
  \l 22225
  \l 22226
  \l 22227
  \l 22228
  \l 22229
  \l 2222A
  \l 2222B
  \l 2222C
  \l 2222D
  \l 2222E
  \l 2222F
  \l 22230
  \l 22231
  \l 22232
  \l 22233
  \l 22234
  \l 22235
  \l 22236
  \l 22237
  \l 22238
  \l 22239
  \l 2223A
  \l 2223B
  \l 2223C
  \l 2223D
  \l 2223E
  \l 2223F
  \l 22240
  \l 22241
  \l 22242
  \l 22243
  \l 22244
  \l 22245
  \l 22246
  \l 22247
  \l 22248
  \l 22249
  \l 2224A
  \l 2224B
  \l 2224C
  \l 2224D
  \l 2224E
  \l 2224F
  \l 22250
  \l 22251
  \l 22252
  \l 22253
  \l 22254
  \l 22255
  \l 22256
  \l 22257
  \l 22258
  \l 22259
  \l 2225A
  \l 2225B
  \l 2225C
  \l 2225D
  \l 2225E
  \l 2225F
  \l 22260
  \l 22261
  \l 22262
  \l 22263
  \l 22264
  \l 22265
  \l 22266
  \l 22267
  \l 22268
  \l 22269
  \l 2226A
  \l 2226B
  \l 2226C
  \l 2226D
  \l 2226E
  \l 2226F
  \l 22270
  \l 22271
  \l 22272
  \l 22273
  \l 22274
  \l 22275
  \l 22276
  \l 22277
  \l 22278
  \l 22279
  \l 2227A
  \l 2227B
  \l 2227C
  \l 2227D
  \l 2227E
  \l 2227F
  \l 22280
  \l 22281
  \l 22282
  \l 22283
  \l 22284
  \l 22285
  \l 22286
  \l 22287
  \l 22288
  \l 22289
  \l 2228A
  \l 2228B
  \l 2228C
  \l 2228D
  \l 2228E
  \l 2228F
  \l 22290
  \l 22291
  \l 22292
  \l 22293
  \l 22294
  \l 22295
  \l 22296
  \l 22297
  \l 22298
  \l 22299
  \l 2229A
  \l 2229B
  \l 2229C
  \l 2229D
  \l 2229E
  \l 2229F
  \l 222A0
  \l 222A1
  \l 222A2
  \l 222A3
  \l 222A4
  \l 222A5
  \l 222A6
  \l 222A7
  \l 222A8
  \l 222A9
  \l 222AA
  \l 222AB
  \l 222AC
  \l 222AD
  \l 222AE
  \l 222AF
  \l 222B0
  \l 222B1
  \l 222B2
  \l 222B3
  \l 222B4
  \l 222B5
  \l 222B6
  \l 222B7
  \l 222B8
  \l 222B9
  \l 222BA
  \l 222BB
  \l 222BC
  \l 222BD
  \l 222BE
  \l 222BF
  \l 222C0
  \l 222C1
  \l 222C2
  \l 222C3
  \l 222C4
  \l 222C5
  \l 222C6
  \l 222C7
  \l 222C8
  \l 222C9
  \l 222CA
  \l 222CB
  \l 222CC
  \l 222CD
  \l 222CE
  \l 222CF
  \l 222D0
  \l 222D1
  \l 222D2
  \l 222D3
  \l 222D4
  \l 222D5
  \l 222D6
  \l 222D7
  \l 222D8
  \l 222D9
  \l 222DA
  \l 222DB
  \l 222DC
  \l 222DD
  \l 222DE
  \l 222DF
  \l 222E0
  \l 222E1
  \l 222E2
  \l 222E3
  \l 222E4
  \l 222E5
  \l 222E6
  \l 222E7
  \l 222E8
  \l 222E9
  \l 222EA
  \l 222EB
  \l 222EC
  \l 222ED
  \l 222EE
  \l 222EF
  \l 222F0
  \l 222F1
  \l 222F2
  \l 222F3
  \l 222F4
  \l 222F5
  \l 222F6
  \l 222F7
  \l 222F8
  \l 222F9
  \l 222FA
  \l 222FB
  \l 222FC
  \l 222FD
  \l 222FE
  \l 222FF
  \l 22300
  \l 22301
  \l 22302
  \l 22303
  \l 22304
  \l 22305
  \l 22306
  \l 22307
  \l 22308
  \l 22309
  \l 2230A
  \l 2230B
  \l 2230C
  \l 2230D
  \l 2230E
  \l 2230F
  \l 22310
  \l 22311
  \l 22312
  \l 22313
  \l 22314
  \l 22315
  \l 22316
  \l 22317
  \l 22318
  \l 22319
  \l 2231A
  \l 2231B
  \l 2231C
  \l 2231D
  \l 2231E
  \l 2231F
  \l 22320
  \l 22321
  \l 22322
  \l 22323
  \l 22324
  \l 22325
  \l 22326
  \l 22327
  \l 22328
  \l 22329
  \l 2232A
  \l 2232B
  \l 2232C
  \l 2232D
  \l 2232E
  \l 2232F
  \l 22330
  \l 22331
  \l 22332
  \l 22333
  \l 22334
  \l 22335
  \l 22336
  \l 22337
  \l 22338
  \l 22339
  \l 2233A
  \l 2233B
  \l 2233C
  \l 2233D
  \l 2233E
  \l 2233F
  \l 22340
  \l 22341
  \l 22342
  \l 22343
  \l 22344
  \l 22345
  \l 22346
  \l 22347
  \l 22348
  \l 22349
  \l 2234A
  \l 2234B
  \l 2234C
  \l 2234D
  \l 2234E
  \l 2234F
  \l 22350
  \l 22351
  \l 22352
  \l 22353
  \l 22354
  \l 22355
  \l 22356
  \l 22357
  \l 22358
  \l 22359
  \l 2235A
  \l 2235B
  \l 2235C
  \l 2235D
  \l 2235E
  \l 2235F
  \l 22360
  \l 22361
  \l 22362
  \l 22363
  \l 22364
  \l 22365
  \l 22366
  \l 22367
  \l 22368
  \l 22369
  \l 2236A
  \l 2236B
  \l 2236C
  \l 2236D
  \l 2236E
  \l 2236F
  \l 22370
  \l 22371
  \l 22372
  \l 22373
  \l 22374
  \l 22375
  \l 22376
  \l 22377
  \l 22378
  \l 22379
  \l 2237A
  \l 2237B
  \l 2237C
  \l 2237D
  \l 2237E
  \l 2237F
  \l 22380
  \l 22381
  \l 22382
  \l 22383
  \l 22384
  \l 22385
  \l 22386
  \l 22387
  \l 22388
  \l 22389
  \l 2238A
  \l 2238B
  \l 2238C
  \l 2238D
  \l 2238E
  \l 2238F
  \l 22390
  \l 22391
  \l 22392
  \l 22393
  \l 22394
  \l 22395
  \l 22396
  \l 22397
  \l 22398
  \l 22399
  \l 2239A
  \l 2239B
  \l 2239C
  \l 2239D
  \l 2239E
  \l 2239F
  \l 223A0
  \l 223A1
  \l 223A2
  \l 223A3
  \l 223A4
  \l 223A5
  \l 223A6
  \l 223A7
  \l 223A8
  \l 223A9
  \l 223AA
  \l 223AB
  \l 223AC
  \l 223AD
  \l 223AE
  \l 223AF
  \l 223B0
  \l 223B1
  \l 223B2
  \l 223B3
  \l 223B4
  \l 223B5
  \l 223B6
  \l 223B7
  \l 223B8
  \l 223B9
  \l 223BA
  \l 223BB
  \l 223BC
  \l 223BD
  \l 223BE
  \l 223BF
  \l 223C0
  \l 223C1
  \l 223C2
  \l 223C3
  \l 223C4
  \l 223C5
  \l 223C6
  \l 223C7
  \l 223C8
  \l 223C9
  \l 223CA
  \l 223CB
  \l 223CC
  \l 223CD
  \l 223CE
  \l 223CF
  \l 223D0
  \l 223D1
  \l 223D2
  \l 223D3
  \l 223D4
  \l 223D5
  \l 223D6
  \l 223D7
  \l 223D8
  \l 223D9
  \l 223DA
  \l 223DB
  \l 223DC
  \l 223DD
  \l 223DE
  \l 223DF
  \l 223E0
  \l 223E1
  \l 223E2
  \l 223E3
  \l 223E4
  \l 223E5
  \l 223E6
  \l 223E7
  \l 223E8
  \l 223E9
  \l 223EA
  \l 223EB
  \l 223EC
  \l 223ED
  \l 223EE
  \l 223EF
  \l 223F0
  \l 223F1
  \l 223F2
  \l 223F3
  \l 223F4
  \l 223F5
  \l 223F6
  \l 223F7
  \l 223F8
  \l 223F9
  \l 223FA
  \l 223FB
  \l 223FC
  \l 223FD
  \l 223FE
  \l 223FF
  \l 22400
  \l 22401
  \l 22402
  \l 22403
  \l 22404
  \l 22405
  \l 22406
  \l 22407
  \l 22408
  \l 22409
  \l 2240A
  \l 2240B
  \l 2240C
  \l 2240D
  \l 2240E
  \l 2240F
  \l 22410
  \l 22411
  \l 22412
  \l 22413
  \l 22414
  \l 22415
  \l 22416
  \l 22417
  \l 22418
  \l 22419
  \l 2241A
  \l 2241B
  \l 2241C
  \l 2241D
  \l 2241E
  \l 2241F
  \l 22420
  \l 22421
  \l 22422
  \l 22423
  \l 22424
  \l 22425
  \l 22426
  \l 22427
  \l 22428
  \l 22429
  \l 2242A
  \l 2242B
  \l 2242C
  \l 2242D
  \l 2242E
  \l 2242F
  \l 22430
  \l 22431
  \l 22432
  \l 22433
  \l 22434
  \l 22435
  \l 22436
  \l 22437
  \l 22438
  \l 22439
  \l 2243A
  \l 2243B
  \l 2243C
  \l 2243D
  \l 2243E
  \l 2243F
  \l 22440
  \l 22441
  \l 22442
  \l 22443
  \l 22444
  \l 22445
  \l 22446
  \l 22447
  \l 22448
  \l 22449
  \l 2244A
  \l 2244B
  \l 2244C
  \l 2244D
  \l 2244E
  \l 2244F
  \l 22450
  \l 22451
  \l 22452
  \l 22453
  \l 22454
  \l 22455
  \l 22456
  \l 22457
  \l 22458
  \l 22459
  \l 2245A
  \l 2245B
  \l 2245C
  \l 2245D
  \l 2245E
  \l 2245F
  \l 22460
  \l 22461
  \l 22462
  \l 22463
  \l 22464
  \l 22465
  \l 22466
  \l 22467
  \l 22468
  \l 22469
  \l 2246A
  \l 2246B
  \l 2246C
  \l 2246D
  \l 2246E
  \l 2246F
  \l 22470
  \l 22471
  \l 22472
  \l 22473
  \l 22474
  \l 22475
  \l 22476
  \l 22477
  \l 22478
  \l 22479
  \l 2247A
  \l 2247B
  \l 2247C
  \l 2247D
  \l 2247E
  \l 2247F
  \l 22480
  \l 22481
  \l 22482
  \l 22483
  \l 22484
  \l 22485
  \l 22486
  \l 22487
  \l 22488
  \l 22489
  \l 2248A
  \l 2248B
  \l 2248C
  \l 2248D
  \l 2248E
  \l 2248F
  \l 22490
  \l 22491
  \l 22492
  \l 22493
  \l 22494
  \l 22495
  \l 22496
  \l 22497
  \l 22498
  \l 22499
  \l 2249A
  \l 2249B
  \l 2249C
  \l 2249D
  \l 2249E
  \l 2249F
  \l 224A0
  \l 224A1
  \l 224A2
  \l 224A3
  \l 224A4
  \l 224A5
  \l 224A6
  \l 224A7
  \l 224A8
  \l 224A9
  \l 224AA
  \l 224AB
  \l 224AC
  \l 224AD
  \l 224AE
  \l 224AF
  \l 224B0
  \l 224B1
  \l 224B2
  \l 224B3
  \l 224B4
  \l 224B5
  \l 224B6
  \l 224B7
  \l 224B8
  \l 224B9
  \l 224BA
  \l 224BB
  \l 224BC
  \l 224BD
  \l 224BE
  \l 224BF
  \l 224C0
  \l 224C1
  \l 224C2
  \l 224C3
  \l 224C4
  \l 224C5
  \l 224C6
  \l 224C7
  \l 224C8
  \l 224C9
  \l 224CA
  \l 224CB
  \l 224CC
  \l 224CD
  \l 224CE
  \l 224CF
  \l 224D0
  \l 224D1
  \l 224D2
  \l 224D3
  \l 224D4
  \l 224D5
  \l 224D6
  \l 224D7
  \l 224D8
  \l 224D9
  \l 224DA
  \l 224DB
  \l 224DC
  \l 224DD
  \l 224DE
  \l 224DF
  \l 224E0
  \l 224E1
  \l 224E2
  \l 224E3
  \l 224E4
  \l 224E5
  \l 224E6
  \l 224E7
  \l 224E8
  \l 224E9
  \l 224EA
  \l 224EB
  \l 224EC
  \l 224ED
  \l 224EE
  \l 224EF
  \l 224F0
  \l 224F1
  \l 224F2
  \l 224F3
  \l 224F4
  \l 224F5
  \l 224F6
  \l 224F7
  \l 224F8
  \l 224F9
  \l 224FA
  \l 224FB
  \l 224FC
  \l 224FD
  \l 224FE
  \l 224FF
  \l 22500
  \l 22501
  \l 22502
  \l 22503
  \l 22504
  \l 22505
  \l 22506
  \l 22507
  \l 22508
  \l 22509
  \l 2250A
  \l 2250B
  \l 2250C
  \l 2250D
  \l 2250E
  \l 2250F
  \l 22510
  \l 22511
  \l 22512
  \l 22513
  \l 22514
  \l 22515
  \l 22516
  \l 22517
  \l 22518
  \l 22519
  \l 2251A
  \l 2251B
  \l 2251C
  \l 2251D
  \l 2251E
  \l 2251F
  \l 22520
  \l 22521
  \l 22522
  \l 22523
  \l 22524
  \l 22525
  \l 22526
  \l 22527
  \l 22528
  \l 22529
  \l 2252A
  \l 2252B
  \l 2252C
  \l 2252D
  \l 2252E
  \l 2252F
  \l 22530
  \l 22531
  \l 22532
  \l 22533
  \l 22534
  \l 22535
  \l 22536
  \l 22537
  \l 22538
  \l 22539
  \l 2253A
  \l 2253B
  \l 2253C
  \l 2253D
  \l 2253E
  \l 2253F
  \l 22540
  \l 22541
  \l 22542
  \l 22543
  \l 22544
  \l 22545
  \l 22546
  \l 22547
  \l 22548
  \l 22549
  \l 2254A
  \l 2254B
  \l 2254C
  \l 2254D
  \l 2254E
  \l 2254F
  \l 22550
  \l 22551
  \l 22552
  \l 22553
  \l 22554
  \l 22555
  \l 22556
  \l 22557
  \l 22558
  \l 22559
  \l 2255A
  \l 2255B
  \l 2255C
  \l 2255D
  \l 2255E
  \l 2255F
  \l 22560
  \l 22561
  \l 22562
  \l 22563
  \l 22564
  \l 22565
  \l 22566
  \l 22567
  \l 22568
  \l 22569
  \l 2256A
  \l 2256B
  \l 2256C
  \l 2256D
  \l 2256E
  \l 2256F
  \l 22570
  \l 22571
  \l 22572
  \l 22573
  \l 22574
  \l 22575
  \l 22576
  \l 22577
  \l 22578
  \l 22579
  \l 2257A
  \l 2257B
  \l 2257C
  \l 2257D
  \l 2257E
  \l 2257F
  \l 22580
  \l 22581
  \l 22582
  \l 22583
  \l 22584
  \l 22585
  \l 22586
  \l 22587
  \l 22588
  \l 22589
  \l 2258A
  \l 2258B
  \l 2258C
  \l 2258D
  \l 2258E
  \l 2258F
  \l 22590
  \l 22591
  \l 22592
  \l 22593
  \l 22594
  \l 22595
  \l 22596
  \l 22597
  \l 22598
  \l 22599
  \l 2259A
  \l 2259B
  \l 2259C
  \l 2259D
  \l 2259E
  \l 2259F
  \l 225A0
  \l 225A1
  \l 225A2
  \l 225A3
  \l 225A4
  \l 225A5
  \l 225A6
  \l 225A7
  \l 225A8
  \l 225A9
  \l 225AA
  \l 225AB
  \l 225AC
  \l 225AD
  \l 225AE
  \l 225AF
  \l 225B0
  \l 225B1
  \l 225B2
  \l 225B3
  \l 225B4
  \l 225B5
  \l 225B6
  \l 225B7
  \l 225B8
  \l 225B9
  \l 225BA
  \l 225BB
  \l 225BC
  \l 225BD
  \l 225BE
  \l 225BF
  \l 225C0
  \l 225C1
  \l 225C2
  \l 225C3
  \l 225C4
  \l 225C5
  \l 225C6
  \l 225C7
  \l 225C8
  \l 225C9
  \l 225CA
  \l 225CB
  \l 225CC
  \l 225CD
  \l 225CE
  \l 225CF
  \l 225D0
  \l 225D1
  \l 225D2
  \l 225D3
  \l 225D4
  \l 225D5
  \l 225D6
  \l 225D7
  \l 225D8
  \l 225D9
  \l 225DA
  \l 225DB
  \l 225DC
  \l 225DD
  \l 225DE
  \l 225DF
  \l 225E0
  \l 225E1
  \l 225E2
  \l 225E3
  \l 225E4
  \l 225E5
  \l 225E6
  \l 225E7
  \l 225E8
  \l 225E9
  \l 225EA
  \l 225EB
  \l 225EC
  \l 225ED
  \l 225EE
  \l 225EF
  \l 225F0
  \l 225F1
  \l 225F2
  \l 225F3
  \l 225F4
  \l 225F5
  \l 225F6
  \l 225F7
  \l 225F8
  \l 225F9
  \l 225FA
  \l 225FB
  \l 225FC
  \l 225FD
  \l 225FE
  \l 225FF
  \l 22600
  \l 22601
  \l 22602
  \l 22603
  \l 22604
  \l 22605
  \l 22606
  \l 22607
  \l 22608
  \l 22609
  \l 2260A
  \l 2260B
  \l 2260C
  \l 2260D
  \l 2260E
  \l 2260F
  \l 22610
  \l 22611
  \l 22612
  \l 22613
  \l 22614
  \l 22615
  \l 22616
  \l 22617
  \l 22618
  \l 22619
  \l 2261A
  \l 2261B
  \l 2261C
  \l 2261D
  \l 2261E
  \l 2261F
  \l 22620
  \l 22621
  \l 22622
  \l 22623
  \l 22624
  \l 22625
  \l 22626
  \l 22627
  \l 22628
  \l 22629
  \l 2262A
  \l 2262B
  \l 2262C
  \l 2262D
  \l 2262E
  \l 2262F
  \l 22630
  \l 22631
  \l 22632
  \l 22633
  \l 22634
  \l 22635
  \l 22636
  \l 22637
  \l 22638
  \l 22639
  \l 2263A
  \l 2263B
  \l 2263C
  \l 2263D
  \l 2263E
  \l 2263F
  \l 22640
  \l 22641
  \l 22642
  \l 22643
  \l 22644
  \l 22645
  \l 22646
  \l 22647
  \l 22648
  \l 22649
  \l 2264A
  \l 2264B
  \l 2264C
  \l 2264D
  \l 2264E
  \l 2264F
  \l 22650
  \l 22651
  \l 22652
  \l 22653
  \l 22654
  \l 22655
  \l 22656
  \l 22657
  \l 22658
  \l 22659
  \l 2265A
  \l 2265B
  \l 2265C
  \l 2265D
  \l 2265E
  \l 2265F
  \l 22660
  \l 22661
  \l 22662
  \l 22663
  \l 22664
  \l 22665
  \l 22666
  \l 22667
  \l 22668
  \l 22669
  \l 2266A
  \l 2266B
  \l 2266C
  \l 2266D
  \l 2266E
  \l 2266F
  \l 22670
  \l 22671
  \l 22672
  \l 22673
  \l 22674
  \l 22675
  \l 22676
  \l 22677
  \l 22678
  \l 22679
  \l 2267A
  \l 2267B
  \l 2267C
  \l 2267D
  \l 2267E
  \l 2267F
  \l 22680
  \l 22681
  \l 22682
  \l 22683
  \l 22684
  \l 22685
  \l 22686
  \l 22687
  \l 22688
  \l 22689
  \l 2268A
  \l 2268B
  \l 2268C
  \l 2268D
  \l 2268E
  \l 2268F
  \l 22690
  \l 22691
  \l 22692
  \l 22693
  \l 22694
  \l 22695
  \l 22696
  \l 22697
  \l 22698
  \l 22699
  \l 2269A
  \l 2269B
  \l 2269C
  \l 2269D
  \l 2269E
  \l 2269F
  \l 226A0
  \l 226A1
  \l 226A2
  \l 226A3
  \l 226A4
  \l 226A5
  \l 226A6
  \l 226A7
  \l 226A8
  \l 226A9
  \l 226AA
  \l 226AB
  \l 226AC
  \l 226AD
  \l 226AE
  \l 226AF
  \l 226B0
  \l 226B1
  \l 226B2
  \l 226B3
  \l 226B4
  \l 226B5
  \l 226B6
  \l 226B7
  \l 226B8
  \l 226B9
  \l 226BA
  \l 226BB
  \l 226BC
  \l 226BD
  \l 226BE
  \l 226BF
  \l 226C0
  \l 226C1
  \l 226C2
  \l 226C3
  \l 226C4
  \l 226C5
  \l 226C6
  \l 226C7
  \l 226C8
  \l 226C9
  \l 226CA
  \l 226CB
  \l 226CC
  \l 226CD
  \l 226CE
  \l 226CF
  \l 226D0
  \l 226D1
  \l 226D2
  \l 226D3
  \l 226D4
  \l 226D5
  \l 226D6
  \l 226D7
  \l 226D8
  \l 226D9
  \l 226DA
  \l 226DB
  \l 226DC
  \l 226DD
  \l 226DE
  \l 226DF
  \l 226E0
  \l 226E1
  \l 226E2
  \l 226E3
  \l 226E4
  \l 226E5
  \l 226E6
  \l 226E7
  \l 226E8
  \l 226E9
  \l 226EA
  \l 226EB
  \l 226EC
  \l 226ED
  \l 226EE
  \l 226EF
  \l 226F0
  \l 226F1
  \l 226F2
  \l 226F3
  \l 226F4
  \l 226F5
  \l 226F6
  \l 226F7
  \l 226F8
  \l 226F9
  \l 226FA
  \l 226FB
  \l 226FC
  \l 226FD
  \l 226FE
  \l 226FF
  \l 22700
  \l 22701
  \l 22702
  \l 22703
  \l 22704
  \l 22705
  \l 22706
  \l 22707
  \l 22708
  \l 22709
  \l 2270A
  \l 2270B
  \l 2270C
  \l 2270D
  \l 2270E
  \l 2270F
  \l 22710
  \l 22711
  \l 22712
  \l 22713
  \l 22714
  \l 22715
  \l 22716
  \l 22717
  \l 22718
  \l 22719
  \l 2271A
  \l 2271B
  \l 2271C
  \l 2271D
  \l 2271E
  \l 2271F
  \l 22720
  \l 22721
  \l 22722
  \l 22723
  \l 22724
  \l 22725
  \l 22726
  \l 22727
  \l 22728
  \l 22729
  \l 2272A
  \l 2272B
  \l 2272C
  \l 2272D
  \l 2272E
  \l 2272F
  \l 22730
  \l 22731
  \l 22732
  \l 22733
  \l 22734
  \l 22735
  \l 22736
  \l 22737
  \l 22738
  \l 22739
  \l 2273A
  \l 2273B
  \l 2273C
  \l 2273D
  \l 2273E
  \l 2273F
  \l 22740
  \l 22741
  \l 22742
  \l 22743
  \l 22744
  \l 22745
  \l 22746
  \l 22747
  \l 22748
  \l 22749
  \l 2274A
  \l 2274B
  \l 2274C
  \l 2274D
  \l 2274E
  \l 2274F
  \l 22750
  \l 22751
  \l 22752
  \l 22753
  \l 22754
  \l 22755
  \l 22756
  \l 22757
  \l 22758
  \l 22759
  \l 2275A
  \l 2275B
  \l 2275C
  \l 2275D
  \l 2275E
  \l 2275F
  \l 22760
  \l 22761
  \l 22762
  \l 22763
  \l 22764
  \l 22765
  \l 22766
  \l 22767
  \l 22768
  \l 22769
  \l 2276A
  \l 2276B
  \l 2276C
  \l 2276D
  \l 2276E
  \l 2276F
  \l 22770
  \l 22771
  \l 22772
  \l 22773
  \l 22774
  \l 22775
  \l 22776
  \l 22777
  \l 22778
  \l 22779
  \l 2277A
  \l 2277B
  \l 2277C
  \l 2277D
  \l 2277E
  \l 2277F
  \l 22780
  \l 22781
  \l 22782
  \l 22783
  \l 22784
  \l 22785
  \l 22786
  \l 22787
  \l 22788
  \l 22789
  \l 2278A
  \l 2278B
  \l 2278C
  \l 2278D
  \l 2278E
  \l 2278F
  \l 22790
  \l 22791
  \l 22792
  \l 22793
  \l 22794
  \l 22795
  \l 22796
  \l 22797
  \l 22798
  \l 22799
  \l 2279A
  \l 2279B
  \l 2279C
  \l 2279D
  \l 2279E
  \l 2279F
  \l 227A0
  \l 227A1
  \l 227A2
  \l 227A3
  \l 227A4
  \l 227A5
  \l 227A6
  \l 227A7
  \l 227A8
  \l 227A9
  \l 227AA
  \l 227AB
  \l 227AC
  \l 227AD
  \l 227AE
  \l 227AF
  \l 227B0
  \l 227B1
  \l 227B2
  \l 227B3
  \l 227B4
  \l 227B5
  \l 227B6
  \l 227B7
  \l 227B8
  \l 227B9
  \l 227BA
  \l 227BB
  \l 227BC
  \l 227BD
  \l 227BE
  \l 227BF
  \l 227C0
  \l 227C1
  \l 227C2
  \l 227C3
  \l 227C4
  \l 227C5
  \l 227C6
  \l 227C7
  \l 227C8
  \l 227C9
  \l 227CA
  \l 227CB
  \l 227CC
  \l 227CD
  \l 227CE
  \l 227CF
  \l 227D0
  \l 227D1
  \l 227D2
  \l 227D3
  \l 227D4
  \l 227D5
  \l 227D6
  \l 227D7
  \l 227D8
  \l 227D9
  \l 227DA
  \l 227DB
  \l 227DC
  \l 227DD
  \l 227DE
  \l 227DF
  \l 227E0
  \l 227E1
  \l 227E2
  \l 227E3
  \l 227E4
  \l 227E5
  \l 227E6
  \l 227E7
  \l 227E8
  \l 227E9
  \l 227EA
  \l 227EB
  \l 227EC
  \l 227ED
  \l 227EE
  \l 227EF
  \l 227F0
  \l 227F1
  \l 227F2
  \l 227F3
  \l 227F4
  \l 227F5
  \l 227F6
  \l 227F7
  \l 227F8
  \l 227F9
  \l 227FA
  \l 227FB
  \l 227FC
  \l 227FD
  \l 227FE
  \l 227FF
  \l 22800
  \l 22801
  \l 22802
  \l 22803
  \l 22804
  \l 22805
  \l 22806
  \l 22807
  \l 22808
  \l 22809
  \l 2280A
  \l 2280B
  \l 2280C
  \l 2280D
  \l 2280E
  \l 2280F
  \l 22810
  \l 22811
  \l 22812
  \l 22813
  \l 22814
  \l 22815
  \l 22816
  \l 22817
  \l 22818
  \l 22819
  \l 2281A
  \l 2281B
  \l 2281C
  \l 2281D
  \l 2281E
  \l 2281F
  \l 22820
  \l 22821
  \l 22822
  \l 22823
  \l 22824
  \l 22825
  \l 22826
  \l 22827
  \l 22828
  \l 22829
  \l 2282A
  \l 2282B
  \l 2282C
  \l 2282D
  \l 2282E
  \l 2282F
  \l 22830
  \l 22831
  \l 22832
  \l 22833
  \l 22834
  \l 22835
  \l 22836
  \l 22837
  \l 22838
  \l 22839
  \l 2283A
  \l 2283B
  \l 2283C
  \l 2283D
  \l 2283E
  \l 2283F
  \l 22840
  \l 22841
  \l 22842
  \l 22843
  \l 22844
  \l 22845
  \l 22846
  \l 22847
  \l 22848
  \l 22849
  \l 2284A
  \l 2284B
  \l 2284C
  \l 2284D
  \l 2284E
  \l 2284F
  \l 22850
  \l 22851
  \l 22852
  \l 22853
  \l 22854
  \l 22855
  \l 22856
  \l 22857
  \l 22858
  \l 22859
  \l 2285A
  \l 2285B
  \l 2285C
  \l 2285D
  \l 2285E
  \l 2285F
  \l 22860
  \l 22861
  \l 22862
  \l 22863
  \l 22864
  \l 22865
  \l 22866
  \l 22867
  \l 22868
  \l 22869
  \l 2286A
  \l 2286B
  \l 2286C
  \l 2286D
  \l 2286E
  \l 2286F
  \l 22870
  \l 22871
  \l 22872
  \l 22873
  \l 22874
  \l 22875
  \l 22876
  \l 22877
  \l 22878
  \l 22879
  \l 2287A
  \l 2287B
  \l 2287C
  \l 2287D
  \l 2287E
  \l 2287F
  \l 22880
  \l 22881
  \l 22882
  \l 22883
  \l 22884
  \l 22885
  \l 22886
  \l 22887
  \l 22888
  \l 22889
  \l 2288A
  \l 2288B
  \l 2288C
  \l 2288D
  \l 2288E
  \l 2288F
  \l 22890
  \l 22891
  \l 22892
  \l 22893
  \l 22894
  \l 22895
  \l 22896
  \l 22897
  \l 22898
  \l 22899
  \l 2289A
  \l 2289B
  \l 2289C
  \l 2289D
  \l 2289E
  \l 2289F
  \l 228A0
  \l 228A1
  \l 228A2
  \l 228A3
  \l 228A4
  \l 228A5
  \l 228A6
  \l 228A7
  \l 228A8
  \l 228A9
  \l 228AA
  \l 228AB
  \l 228AC
  \l 228AD
  \l 228AE
  \l 228AF
  \l 228B0
  \l 228B1
  \l 228B2
  \l 228B3
  \l 228B4
  \l 228B5
  \l 228B6
  \l 228B7
  \l 228B8
  \l 228B9
  \l 228BA
  \l 228BB
  \l 228BC
  \l 228BD
  \l 228BE
  \l 228BF
  \l 228C0
  \l 228C1
  \l 228C2
  \l 228C3
  \l 228C4
  \l 228C5
  \l 228C6
  \l 228C7
  \l 228C8
  \l 228C9
  \l 228CA
  \l 228CB
  \l 228CC
  \l 228CD
  \l 228CE
  \l 228CF
  \l 228D0
  \l 228D1
  \l 228D2
  \l 228D3
  \l 228D4
  \l 228D5
  \l 228D6
  \l 228D7
  \l 228D8
  \l 228D9
  \l 228DA
  \l 228DB
  \l 228DC
  \l 228DD
  \l 228DE
  \l 228DF
  \l 228E0
  \l 228E1
  \l 228E2
  \l 228E3
  \l 228E4
  \l 228E5
  \l 228E6
  \l 228E7
  \l 228E8
  \l 228E9
  \l 228EA
  \l 228EB
  \l 228EC
  \l 228ED
  \l 228EE
  \l 228EF
  \l 228F0
  \l 228F1
  \l 228F2
  \l 228F3
  \l 228F4
  \l 228F5
  \l 228F6
  \l 228F7
  \l 228F8
  \l 228F9
  \l 228FA
  \l 228FB
  \l 228FC
  \l 228FD
  \l 228FE
  \l 228FF
  \l 22900
  \l 22901
  \l 22902
  \l 22903
  \l 22904
  \l 22905
  \l 22906
  \l 22907
  \l 22908
  \l 22909
  \l 2290A
  \l 2290B
  \l 2290C
  \l 2290D
  \l 2290E
  \l 2290F
  \l 22910
  \l 22911
  \l 22912
  \l 22913
  \l 22914
  \l 22915
  \l 22916
  \l 22917
  \l 22918
  \l 22919
  \l 2291A
  \l 2291B
  \l 2291C
  \l 2291D
  \l 2291E
  \l 2291F
  \l 22920
  \l 22921
  \l 22922
  \l 22923
  \l 22924
  \l 22925
  \l 22926
  \l 22927
  \l 22928
  \l 22929
  \l 2292A
  \l 2292B
  \l 2292C
  \l 2292D
  \l 2292E
  \l 2292F
  \l 22930
  \l 22931
  \l 22932
  \l 22933
  \l 22934
  \l 22935
  \l 22936
  \l 22937
  \l 22938
  \l 22939
  \l 2293A
  \l 2293B
  \l 2293C
  \l 2293D
  \l 2293E
  \l 2293F
  \l 22940
  \l 22941
  \l 22942
  \l 22943
  \l 22944
  \l 22945
  \l 22946
  \l 22947
  \l 22948
  \l 22949
  \l 2294A
  \l 2294B
  \l 2294C
  \l 2294D
  \l 2294E
  \l 2294F
  \l 22950
  \l 22951
  \l 22952
  \l 22953
  \l 22954
  \l 22955
  \l 22956
  \l 22957
  \l 22958
  \l 22959
  \l 2295A
  \l 2295B
  \l 2295C
  \l 2295D
  \l 2295E
  \l 2295F
  \l 22960
  \l 22961
  \l 22962
  \l 22963
  \l 22964
  \l 22965
  \l 22966
  \l 22967
  \l 22968
  \l 22969
  \l 2296A
  \l 2296B
  \l 2296C
  \l 2296D
  \l 2296E
  \l 2296F
  \l 22970
  \l 22971
  \l 22972
  \l 22973
  \l 22974
  \l 22975
  \l 22976
  \l 22977
  \l 22978
  \l 22979
  \l 2297A
  \l 2297B
  \l 2297C
  \l 2297D
  \l 2297E
  \l 2297F
  \l 22980
  \l 22981
  \l 22982
  \l 22983
  \l 22984
  \l 22985
  \l 22986
  \l 22987
  \l 22988
  \l 22989
  \l 2298A
  \l 2298B
  \l 2298C
  \l 2298D
  \l 2298E
  \l 2298F
  \l 22990
  \l 22991
  \l 22992
  \l 22993
  \l 22994
  \l 22995
  \l 22996
  \l 22997
  \l 22998
  \l 22999
  \l 2299A
  \l 2299B
  \l 2299C
  \l 2299D
  \l 2299E
  \l 2299F
  \l 229A0
  \l 229A1
  \l 229A2
  \l 229A3
  \l 229A4
  \l 229A5
  \l 229A6
  \l 229A7
  \l 229A8
  \l 229A9
  \l 229AA
  \l 229AB
  \l 229AC
  \l 229AD
  \l 229AE
  \l 229AF
  \l 229B0
  \l 229B1
  \l 229B2
  \l 229B3
  \l 229B4
  \l 229B5
  \l 229B6
  \l 229B7
  \l 229B8
  \l 229B9
  \l 229BA
  \l 229BB
  \l 229BC
  \l 229BD
  \l 229BE
  \l 229BF
  \l 229C0
  \l 229C1
  \l 229C2
  \l 229C3
  \l 229C4
  \l 229C5
  \l 229C6
  \l 229C7
  \l 229C8
  \l 229C9
  \l 229CA
  \l 229CB
  \l 229CC
  \l 229CD
  \l 229CE
  \l 229CF
  \l 229D0
  \l 229D1
  \l 229D2
  \l 229D3
  \l 229D4
  \l 229D5
  \l 229D6
  \l 229D7
  \l 229D8
  \l 229D9
  \l 229DA
  \l 229DB
  \l 229DC
  \l 229DD
  \l 229DE
  \l 229DF
  \l 229E0
  \l 229E1
  \l 229E2
  \l 229E3
  \l 229E4
  \l 229E5
  \l 229E6
  \l 229E7
  \l 229E8
  \l 229E9
  \l 229EA
  \l 229EB
  \l 229EC
  \l 229ED
  \l 229EE
  \l 229EF
  \l 229F0
  \l 229F1
  \l 229F2
  \l 229F3
  \l 229F4
  \l 229F5
  \l 229F6
  \l 229F7
  \l 229F8
  \l 229F9
  \l 229FA
  \l 229FB
  \l 229FC
  \l 229FD
  \l 229FE
  \l 229FF
  \l 22A00
  \l 22A01
  \l 22A02
  \l 22A03
  \l 22A04
  \l 22A05
  \l 22A06
  \l 22A07
  \l 22A08
  \l 22A09
  \l 22A0A
  \l 22A0B
  \l 22A0C
  \l 22A0D
  \l 22A0E
  \l 22A0F
  \l 22A10
  \l 22A11
  \l 22A12
  \l 22A13
  \l 22A14
  \l 22A15
  \l 22A16
  \l 22A17
  \l 22A18
  \l 22A19
  \l 22A1A
  \l 22A1B
  \l 22A1C
  \l 22A1D
  \l 22A1E
  \l 22A1F
  \l 22A20
  \l 22A21
  \l 22A22
  \l 22A23
  \l 22A24
  \l 22A25
  \l 22A26
  \l 22A27
  \l 22A28
  \l 22A29
  \l 22A2A
  \l 22A2B
  \l 22A2C
  \l 22A2D
  \l 22A2E
  \l 22A2F
  \l 22A30
  \l 22A31
  \l 22A32
  \l 22A33
  \l 22A34
  \l 22A35
  \l 22A36
  \l 22A37
  \l 22A38
  \l 22A39
  \l 22A3A
  \l 22A3B
  \l 22A3C
  \l 22A3D
  \l 22A3E
  \l 22A3F
  \l 22A40
  \l 22A41
  \l 22A42
  \l 22A43
  \l 22A44
  \l 22A45
  \l 22A46
  \l 22A47
  \l 22A48
  \l 22A49
  \l 22A4A
  \l 22A4B
  \l 22A4C
  \l 22A4D
  \l 22A4E
  \l 22A4F
  \l 22A50
  \l 22A51
  \l 22A52
  \l 22A53
  \l 22A54
  \l 22A55
  \l 22A56
  \l 22A57
  \l 22A58
  \l 22A59
  \l 22A5A
  \l 22A5B
  \l 22A5C
  \l 22A5D
  \l 22A5E
  \l 22A5F
  \l 22A60
  \l 22A61
  \l 22A62
  \l 22A63
  \l 22A64
  \l 22A65
  \l 22A66
  \l 22A67
  \l 22A68
  \l 22A69
  \l 22A6A
  \l 22A6B
  \l 22A6C
  \l 22A6D
  \l 22A6E
  \l 22A6F
  \l 22A70
  \l 22A71
  \l 22A72
  \l 22A73
  \l 22A74
  \l 22A75
  \l 22A76
  \l 22A77
  \l 22A78
  \l 22A79
  \l 22A7A
  \l 22A7B
  \l 22A7C
  \l 22A7D
  \l 22A7E
  \l 22A7F
  \l 22A80
  \l 22A81
  \l 22A82
  \l 22A83
  \l 22A84
  \l 22A85
  \l 22A86
  \l 22A87
  \l 22A88
  \l 22A89
  \l 22A8A
  \l 22A8B
  \l 22A8C
  \l 22A8D
  \l 22A8E
  \l 22A8F
  \l 22A90
  \l 22A91
  \l 22A92
  \l 22A93
  \l 22A94
  \l 22A95
  \l 22A96
  \l 22A97
  \l 22A98
  \l 22A99
  \l 22A9A
  \l 22A9B
  \l 22A9C
  \l 22A9D
  \l 22A9E
  \l 22A9F
  \l 22AA0
  \l 22AA1
  \l 22AA2
  \l 22AA3
  \l 22AA4
  \l 22AA5
  \l 22AA6
  \l 22AA7
  \l 22AA8
  \l 22AA9
  \l 22AAA
  \l 22AAB
  \l 22AAC
  \l 22AAD
  \l 22AAE
  \l 22AAF
  \l 22AB0
  \l 22AB1
  \l 22AB2
  \l 22AB3
  \l 22AB4
  \l 22AB5
  \l 22AB6
  \l 22AB7
  \l 22AB8
  \l 22AB9
  \l 22ABA
  \l 22ABB
  \l 22ABC
  \l 22ABD
  \l 22ABE
  \l 22ABF
  \l 22AC0
  \l 22AC1
  \l 22AC2
  \l 22AC3
  \l 22AC4
  \l 22AC5
  \l 22AC6
  \l 22AC7
  \l 22AC8
  \l 22AC9
  \l 22ACA
  \l 22ACB
  \l 22ACC
  \l 22ACD
  \l 22ACE
  \l 22ACF
  \l 22AD0
  \l 22AD1
  \l 22AD2
  \l 22AD3
  \l 22AD4
  \l 22AD5
  \l 22AD6
  \l 22AD7
  \l 22AD8
  \l 22AD9
  \l 22ADA
  \l 22ADB
  \l 22ADC
  \l 22ADD
  \l 22ADE
  \l 22ADF
  \l 22AE0
  \l 22AE1
  \l 22AE2
  \l 22AE3
  \l 22AE4
  \l 22AE5
  \l 22AE6
  \l 22AE7
  \l 22AE8
  \l 22AE9
  \l 22AEA
  \l 22AEB
  \l 22AEC
  \l 22AED
  \l 22AEE
  \l 22AEF
  \l 22AF0
  \l 22AF1
  \l 22AF2
  \l 22AF3
  \l 22AF4
  \l 22AF5
  \l 22AF6
  \l 22AF7
  \l 22AF8
  \l 22AF9
  \l 22AFA
  \l 22AFB
  \l 22AFC
  \l 22AFD
  \l 22AFE
  \l 22AFF
  \l 22B00
  \l 22B01
  \l 22B02
  \l 22B03
  \l 22B04
  \l 22B05
  \l 22B06
  \l 22B07
  \l 22B08
  \l 22B09
  \l 22B0A
  \l 22B0B
  \l 22B0C
  \l 22B0D
  \l 22B0E
  \l 22B0F
  \l 22B10
  \l 22B11
  \l 22B12
  \l 22B13
  \l 22B14
  \l 22B15
  \l 22B16
  \l 22B17
  \l 22B18
  \l 22B19
  \l 22B1A
  \l 22B1B
  \l 22B1C
  \l 22B1D
  \l 22B1E
  \l 22B1F
  \l 22B20
  \l 22B21
  \l 22B22
  \l 22B23
  \l 22B24
  \l 22B25
  \l 22B26
  \l 22B27
  \l 22B28
  \l 22B29
  \l 22B2A
  \l 22B2B
  \l 22B2C
  \l 22B2D
  \l 22B2E
  \l 22B2F
  \l 22B30
  \l 22B31
  \l 22B32
  \l 22B33
  \l 22B34
  \l 22B35
  \l 22B36
  \l 22B37
  \l 22B38
  \l 22B39
  \l 22B3A
  \l 22B3B
  \l 22B3C
  \l 22B3D
  \l 22B3E
  \l 22B3F
  \l 22B40
  \l 22B41
  \l 22B42
  \l 22B43
  \l 22B44
  \l 22B45
  \l 22B46
  \l 22B47
  \l 22B48
  \l 22B49
  \l 22B4A
  \l 22B4B
  \l 22B4C
  \l 22B4D
  \l 22B4E
  \l 22B4F
  \l 22B50
  \l 22B51
  \l 22B52
  \l 22B53
  \l 22B54
  \l 22B55
  \l 22B56
  \l 22B57
  \l 22B58
  \l 22B59
  \l 22B5A
  \l 22B5B
  \l 22B5C
  \l 22B5D
  \l 22B5E
  \l 22B5F
  \l 22B60
  \l 22B61
  \l 22B62
  \l 22B63
  \l 22B64
  \l 22B65
  \l 22B66
  \l 22B67
  \l 22B68
  \l 22B69
  \l 22B6A
  \l 22B6B
  \l 22B6C
  \l 22B6D
  \l 22B6E
  \l 22B6F
  \l 22B70
  \l 22B71
  \l 22B72
  \l 22B73
  \l 22B74
  \l 22B75
  \l 22B76
  \l 22B77
  \l 22B78
  \l 22B79
  \l 22B7A
  \l 22B7B
  \l 22B7C
  \l 22B7D
  \l 22B7E
  \l 22B7F
  \l 22B80
  \l 22B81
  \l 22B82
  \l 22B83
  \l 22B84
  \l 22B85
  \l 22B86
  \l 22B87
  \l 22B88
  \l 22B89
  \l 22B8A
  \l 22B8B
  \l 22B8C
  \l 22B8D
  \l 22B8E
  \l 22B8F
  \l 22B90
  \l 22B91
  \l 22B92
  \l 22B93
  \l 22B94
  \l 22B95
  \l 22B96
  \l 22B97
  \l 22B98
  \l 22B99
  \l 22B9A
  \l 22B9B
  \l 22B9C
  \l 22B9D
  \l 22B9E
  \l 22B9F
  \l 22BA0
  \l 22BA1
  \l 22BA2
  \l 22BA3
  \l 22BA4
  \l 22BA5
  \l 22BA6
  \l 22BA7
  \l 22BA8
  \l 22BA9
  \l 22BAA
  \l 22BAB
  \l 22BAC
  \l 22BAD
  \l 22BAE
  \l 22BAF
  \l 22BB0
  \l 22BB1
  \l 22BB2
  \l 22BB3
  \l 22BB4
  \l 22BB5
  \l 22BB6
  \l 22BB7
  \l 22BB8
  \l 22BB9
  \l 22BBA
  \l 22BBB
  \l 22BBC
  \l 22BBD
  \l 22BBE
  \l 22BBF
  \l 22BC0
  \l 22BC1
  \l 22BC2
  \l 22BC3
  \l 22BC4
  \l 22BC5
  \l 22BC6
  \l 22BC7
  \l 22BC8
  \l 22BC9
  \l 22BCA
  \l 22BCB
  \l 22BCC
  \l 22BCD
  \l 22BCE
  \l 22BCF
  \l 22BD0
  \l 22BD1
  \l 22BD2
  \l 22BD3
  \l 22BD4
  \l 22BD5
  \l 22BD6
  \l 22BD7
  \l 22BD8
  \l 22BD9
  \l 22BDA
  \l 22BDB
  \l 22BDC
  \l 22BDD
  \l 22BDE
  \l 22BDF
  \l 22BE0
  \l 22BE1
  \l 22BE2
  \l 22BE3
  \l 22BE4
  \l 22BE5
  \l 22BE6
  \l 22BE7
  \l 22BE8
  \l 22BE9
  \l 22BEA
  \l 22BEB
  \l 22BEC
  \l 22BED
  \l 22BEE
  \l 22BEF
  \l 22BF0
  \l 22BF1
  \l 22BF2
  \l 22BF3
  \l 22BF4
  \l 22BF5
  \l 22BF6
  \l 22BF7
  \l 22BF8
  \l 22BF9
  \l 22BFA
  \l 22BFB
  \l 22BFC
  \l 22BFD
  \l 22BFE
  \l 22BFF
  \l 22C00
  \l 22C01
  \l 22C02
  \l 22C03
  \l 22C04
  \l 22C05
  \l 22C06
  \l 22C07
  \l 22C08
  \l 22C09
  \l 22C0A
  \l 22C0B
  \l 22C0C
  \l 22C0D
  \l 22C0E
  \l 22C0F
  \l 22C10
  \l 22C11
  \l 22C12
  \l 22C13
  \l 22C14
  \l 22C15
  \l 22C16
  \l 22C17
  \l 22C18
  \l 22C19
  \l 22C1A
  \l 22C1B
  \l 22C1C
  \l 22C1D
  \l 22C1E
  \l 22C1F
  \l 22C20
  \l 22C21
  \l 22C22
  \l 22C23
  \l 22C24
  \l 22C25
  \l 22C26
  \l 22C27
  \l 22C28
  \l 22C29
  \l 22C2A
  \l 22C2B
  \l 22C2C
  \l 22C2D
  \l 22C2E
  \l 22C2F
  \l 22C30
  \l 22C31
  \l 22C32
  \l 22C33
  \l 22C34
  \l 22C35
  \l 22C36
  \l 22C37
  \l 22C38
  \l 22C39
  \l 22C3A
  \l 22C3B
  \l 22C3C
  \l 22C3D
  \l 22C3E
  \l 22C3F
  \l 22C40
  \l 22C41
  \l 22C42
  \l 22C43
  \l 22C44
  \l 22C45
  \l 22C46
  \l 22C47
  \l 22C48
  \l 22C49
  \l 22C4A
  \l 22C4B
  \l 22C4C
  \l 22C4D
  \l 22C4E
  \l 22C4F
  \l 22C50
  \l 22C51
  \l 22C52
  \l 22C53
  \l 22C54
  \l 22C55
  \l 22C56
  \l 22C57
  \l 22C58
  \l 22C59
  \l 22C5A
  \l 22C5B
  \l 22C5C
  \l 22C5D
  \l 22C5E
  \l 22C5F
  \l 22C60
  \l 22C61
  \l 22C62
  \l 22C63
  \l 22C64
  \l 22C65
  \l 22C66
  \l 22C67
  \l 22C68
  \l 22C69
  \l 22C6A
  \l 22C6B
  \l 22C6C
  \l 22C6D
  \l 22C6E
  \l 22C6F
  \l 22C70
  \l 22C71
  \l 22C72
  \l 22C73
  \l 22C74
  \l 22C75
  \l 22C76
  \l 22C77
  \l 22C78
  \l 22C79
  \l 22C7A
  \l 22C7B
  \l 22C7C
  \l 22C7D
  \l 22C7E
  \l 22C7F
  \l 22C80
  \l 22C81
  \l 22C82
  \l 22C83
  \l 22C84
  \l 22C85
  \l 22C86
  \l 22C87
  \l 22C88
  \l 22C89
  \l 22C8A
  \l 22C8B
  \l 22C8C
  \l 22C8D
  \l 22C8E
  \l 22C8F
  \l 22C90
  \l 22C91
  \l 22C92
  \l 22C93
  \l 22C94
  \l 22C95
  \l 22C96
  \l 22C97
  \l 22C98
  \l 22C99
  \l 22C9A
  \l 22C9B
  \l 22C9C
  \l 22C9D
  \l 22C9E
  \l 22C9F
  \l 22CA0
  \l 22CA1
  \l 22CA2
  \l 22CA3
  \l 22CA4
  \l 22CA5
  \l 22CA6
  \l 22CA7
  \l 22CA8
  \l 22CA9
  \l 22CAA
  \l 22CAB
  \l 22CAC
  \l 22CAD
  \l 22CAE
  \l 22CAF
  \l 22CB0
  \l 22CB1
  \l 22CB2
  \l 22CB3
  \l 22CB4
  \l 22CB5
  \l 22CB6
  \l 22CB7
  \l 22CB8
  \l 22CB9
  \l 22CBA
  \l 22CBB
  \l 22CBC
  \l 22CBD
  \l 22CBE
  \l 22CBF
  \l 22CC0
  \l 22CC1
  \l 22CC2
  \l 22CC3
  \l 22CC4
  \l 22CC5
  \l 22CC6
  \l 22CC7
  \l 22CC8
  \l 22CC9
  \l 22CCA
  \l 22CCB
  \l 22CCC
  \l 22CCD
  \l 22CCE
  \l 22CCF
  \l 22CD0
  \l 22CD1
  \l 22CD2
  \l 22CD3
  \l 22CD4
  \l 22CD5
  \l 22CD6
  \l 22CD7
  \l 22CD8
  \l 22CD9
  \l 22CDA
  \l 22CDB
  \l 22CDC
  \l 22CDD
  \l 22CDE
  \l 22CDF
  \l 22CE0
  \l 22CE1
  \l 22CE2
  \l 22CE3
  \l 22CE4
  \l 22CE5
  \l 22CE6
  \l 22CE7
  \l 22CE8
  \l 22CE9
  \l 22CEA
  \l 22CEB
  \l 22CEC
  \l 22CED
  \l 22CEE
  \l 22CEF
  \l 22CF0
  \l 22CF1
  \l 22CF2
  \l 22CF3
  \l 22CF4
  \l 22CF5
  \l 22CF6
  \l 22CF7
  \l 22CF8
  \l 22CF9
  \l 22CFA
  \l 22CFB
  \l 22CFC
  \l 22CFD
  \l 22CFE
  \l 22CFF
  \l 22D00
  \l 22D01
  \l 22D02
  \l 22D03
  \l 22D04
  \l 22D05
  \l 22D06
  \l 22D07
  \l 22D08
  \l 22D09
  \l 22D0A
  \l 22D0B
  \l 22D0C
  \l 22D0D
  \l 22D0E
  \l 22D0F
  \l 22D10
  \l 22D11
  \l 22D12
  \l 22D13
  \l 22D14
  \l 22D15
  \l 22D16
  \l 22D17
  \l 22D18
  \l 22D19
  \l 22D1A
  \l 22D1B
  \l 22D1C
  \l 22D1D
  \l 22D1E
  \l 22D1F
  \l 22D20
  \l 22D21
  \l 22D22
  \l 22D23
  \l 22D24
  \l 22D25
  \l 22D26
  \l 22D27
  \l 22D28
  \l 22D29
  \l 22D2A
  \l 22D2B
  \l 22D2C
  \l 22D2D
  \l 22D2E
  \l 22D2F
  \l 22D30
  \l 22D31
  \l 22D32
  \l 22D33
  \l 22D34
  \l 22D35
  \l 22D36
  \l 22D37
  \l 22D38
  \l 22D39
  \l 22D3A
  \l 22D3B
  \l 22D3C
  \l 22D3D
  \l 22D3E
  \l 22D3F
  \l 22D40
  \l 22D41
  \l 22D42
  \l 22D43
  \l 22D44
  \l 22D45
  \l 22D46
  \l 22D47
  \l 22D48
  \l 22D49
  \l 22D4A
  \l 22D4B
  \l 22D4C
  \l 22D4D
  \l 22D4E
  \l 22D4F
  \l 22D50
  \l 22D51
  \l 22D52
  \l 22D53
  \l 22D54
  \l 22D55
  \l 22D56
  \l 22D57
  \l 22D58
  \l 22D59
  \l 22D5A
  \l 22D5B
  \l 22D5C
  \l 22D5D
  \l 22D5E
  \l 22D5F
  \l 22D60
  \l 22D61
  \l 22D62
  \l 22D63
  \l 22D64
  \l 22D65
  \l 22D66
  \l 22D67
  \l 22D68
  \l 22D69
  \l 22D6A
  \l 22D6B
  \l 22D6C
  \l 22D6D
  \l 22D6E
  \l 22D6F
  \l 22D70
  \l 22D71
  \l 22D72
  \l 22D73
  \l 22D74
  \l 22D75
  \l 22D76
  \l 22D77
  \l 22D78
  \l 22D79
  \l 22D7A
  \l 22D7B
  \l 22D7C
  \l 22D7D
  \l 22D7E
  \l 22D7F
  \l 22D80
  \l 22D81
  \l 22D82
  \l 22D83
  \l 22D84
  \l 22D85
  \l 22D86
  \l 22D87
  \l 22D88
  \l 22D89
  \l 22D8A
  \l 22D8B
  \l 22D8C
  \l 22D8D
  \l 22D8E
  \l 22D8F
  \l 22D90
  \l 22D91
  \l 22D92
  \l 22D93
  \l 22D94
  \l 22D95
  \l 22D96
  \l 22D97
  \l 22D98
  \l 22D99
  \l 22D9A
  \l 22D9B
  \l 22D9C
  \l 22D9D
  \l 22D9E
  \l 22D9F
  \l 22DA0
  \l 22DA1
  \l 22DA2
  \l 22DA3
  \l 22DA4
  \l 22DA5
  \l 22DA6
  \l 22DA7
  \l 22DA8
  \l 22DA9
  \l 22DAA
  \l 22DAB
  \l 22DAC
  \l 22DAD
  \l 22DAE
  \l 22DAF
  \l 22DB0
  \l 22DB1
  \l 22DB2
  \l 22DB3
  \l 22DB4
  \l 22DB5
  \l 22DB6
  \l 22DB7
  \l 22DB8
  \l 22DB9
  \l 22DBA
  \l 22DBB
  \l 22DBC
  \l 22DBD
  \l 22DBE
  \l 22DBF
  \l 22DC0
  \l 22DC1
  \l 22DC2
  \l 22DC3
  \l 22DC4
  \l 22DC5
  \l 22DC6
  \l 22DC7
  \l 22DC8
  \l 22DC9
  \l 22DCA
  \l 22DCB
  \l 22DCC
  \l 22DCD
  \l 22DCE
  \l 22DCF
  \l 22DD0
  \l 22DD1
  \l 22DD2
  \l 22DD3
  \l 22DD4
  \l 22DD5
  \l 22DD6
  \l 22DD7
  \l 22DD8
  \l 22DD9
  \l 22DDA
  \l 22DDB
  \l 22DDC
  \l 22DDD
  \l 22DDE
  \l 22DDF
  \l 22DE0
  \l 22DE1
  \l 22DE2
  \l 22DE3
  \l 22DE4
  \l 22DE5
  \l 22DE6
  \l 22DE7
  \l 22DE8
  \l 22DE9
  \l 22DEA
  \l 22DEB
  \l 22DEC
  \l 22DED
  \l 22DEE
  \l 22DEF
  \l 22DF0
  \l 22DF1
  \l 22DF2
  \l 22DF3
  \l 22DF4
  \l 22DF5
  \l 22DF6
  \l 22DF7
  \l 22DF8
  \l 22DF9
  \l 22DFA
  \l 22DFB
  \l 22DFC
  \l 22DFD
  \l 22DFE
  \l 22DFF
  \l 22E00
  \l 22E01
  \l 22E02
  \l 22E03
  \l 22E04
  \l 22E05
  \l 22E06
  \l 22E07
  \l 22E08
  \l 22E09
  \l 22E0A
  \l 22E0B
  \l 22E0C
  \l 22E0D
  \l 22E0E
  \l 22E0F
  \l 22E10
  \l 22E11
  \l 22E12
  \l 22E13
  \l 22E14
  \l 22E15
  \l 22E16
  \l 22E17
  \l 22E18
  \l 22E19
  \l 22E1A
  \l 22E1B
  \l 22E1C
  \l 22E1D
  \l 22E1E
  \l 22E1F
  \l 22E20
  \l 22E21
  \l 22E22
  \l 22E23
  \l 22E24
  \l 22E25
  \l 22E26
  \l 22E27
  \l 22E28
  \l 22E29
  \l 22E2A
  \l 22E2B
  \l 22E2C
  \l 22E2D
  \l 22E2E
  \l 22E2F
  \l 22E30
  \l 22E31
  \l 22E32
  \l 22E33
  \l 22E34
  \l 22E35
  \l 22E36
  \l 22E37
  \l 22E38
  \l 22E39
  \l 22E3A
  \l 22E3B
  \l 22E3C
  \l 22E3D
  \l 22E3E
  \l 22E3F
  \l 22E40
  \l 22E41
  \l 22E42
  \l 22E43
  \l 22E44
  \l 22E45
  \l 22E46
  \l 22E47
  \l 22E48
  \l 22E49
  \l 22E4A
  \l 22E4B
  \l 22E4C
  \l 22E4D
  \l 22E4E
  \l 22E4F
  \l 22E50
  \l 22E51
  \l 22E52
  \l 22E53
  \l 22E54
  \l 22E55
  \l 22E56
  \l 22E57
  \l 22E58
  \l 22E59
  \l 22E5A
  \l 22E5B
  \l 22E5C
  \l 22E5D
  \l 22E5E
  \l 22E5F
  \l 22E60
  \l 22E61
  \l 22E62
  \l 22E63
  \l 22E64
  \l 22E65
  \l 22E66
  \l 22E67
  \l 22E68
  \l 22E69
  \l 22E6A
  \l 22E6B
  \l 22E6C
  \l 22E6D
  \l 22E6E
  \l 22E6F
  \l 22E70
  \l 22E71
  \l 22E72
  \l 22E73
  \l 22E74
  \l 22E75
  \l 22E76
  \l 22E77
  \l 22E78
  \l 22E79
  \l 22E7A
  \l 22E7B
  \l 22E7C
  \l 22E7D
  \l 22E7E
  \l 22E7F
  \l 22E80
  \l 22E81
  \l 22E82
  \l 22E83
  \l 22E84
  \l 22E85
  \l 22E86
  \l 22E87
  \l 22E88
  \l 22E89
  \l 22E8A
  \l 22E8B
  \l 22E8C
  \l 22E8D
  \l 22E8E
  \l 22E8F
  \l 22E90
  \l 22E91
  \l 22E92
  \l 22E93
  \l 22E94
  \l 22E95
  \l 22E96
  \l 22E97
  \l 22E98
  \l 22E99
  \l 22E9A
  \l 22E9B
  \l 22E9C
  \l 22E9D
  \l 22E9E
  \l 22E9F
  \l 22EA0
  \l 22EA1
  \l 22EA2
  \l 22EA3
  \l 22EA4
  \l 22EA5
  \l 22EA6
  \l 22EA7
  \l 22EA8
  \l 22EA9
  \l 22EAA
  \l 22EAB
  \l 22EAC
  \l 22EAD
  \l 22EAE
  \l 22EAF
  \l 22EB0
  \l 22EB1
  \l 22EB2
  \l 22EB3
  \l 22EB4
  \l 22EB5
  \l 22EB6
  \l 22EB7
  \l 22EB8
  \l 22EB9
  \l 22EBA
  \l 22EBB
  \l 22EBC
  \l 22EBD
  \l 22EBE
  \l 22EBF
  \l 22EC0
  \l 22EC1
  \l 22EC2
  \l 22EC3
  \l 22EC4
  \l 22EC5
  \l 22EC6
  \l 22EC7
  \l 22EC8
  \l 22EC9
  \l 22ECA
  \l 22ECB
  \l 22ECC
  \l 22ECD
  \l 22ECE
  \l 22ECF
  \l 22ED0
  \l 22ED1
  \l 22ED2
  \l 22ED3
  \l 22ED4
  \l 22ED5
  \l 22ED6
  \l 22ED7
  \l 22ED8
  \l 22ED9
  \l 22EDA
  \l 22EDB
  \l 22EDC
  \l 22EDD
  \l 22EDE
  \l 22EDF
  \l 22EE0
  \l 22EE1
  \l 22EE2
  \l 22EE3
  \l 22EE4
  \l 22EE5
  \l 22EE6
  \l 22EE7
  \l 22EE8
  \l 22EE9
  \l 22EEA
  \l 22EEB
  \l 22EEC
  \l 22EED
  \l 22EEE
  \l 22EEF
  \l 22EF0
  \l 22EF1
  \l 22EF2
  \l 22EF3
  \l 22EF4
  \l 22EF5
  \l 22EF6
  \l 22EF7
  \l 22EF8
  \l 22EF9
  \l 22EFA
  \l 22EFB
  \l 22EFC
  \l 22EFD
  \l 22EFE
  \l 22EFF
  \l 22F00
  \l 22F01
  \l 22F02
  \l 22F03
  \l 22F04
  \l 22F05
  \l 22F06
  \l 22F07
  \l 22F08
  \l 22F09
  \l 22F0A
  \l 22F0B
  \l 22F0C
  \l 22F0D
  \l 22F0E
  \l 22F0F
  \l 22F10
  \l 22F11
  \l 22F12
  \l 22F13
  \l 22F14
  \l 22F15
  \l 22F16
  \l 22F17
  \l 22F18
  \l 22F19
  \l 22F1A
  \l 22F1B
  \l 22F1C
  \l 22F1D
  \l 22F1E
  \l 22F1F
  \l 22F20
  \l 22F21
  \l 22F22
  \l 22F23
  \l 22F24
  \l 22F25
  \l 22F26
  \l 22F27
  \l 22F28
  \l 22F29
  \l 22F2A
  \l 22F2B
  \l 22F2C
  \l 22F2D
  \l 22F2E
  \l 22F2F
  \l 22F30
  \l 22F31
  \l 22F32
  \l 22F33
  \l 22F34
  \l 22F35
  \l 22F36
  \l 22F37
  \l 22F38
  \l 22F39
  \l 22F3A
  \l 22F3B
  \l 22F3C
  \l 22F3D
  \l 22F3E
  \l 22F3F
  \l 22F40
  \l 22F41
  \l 22F42
  \l 22F43
  \l 22F44
  \l 22F45
  \l 22F46
  \l 22F47
  \l 22F48
  \l 22F49
  \l 22F4A
  \l 22F4B
  \l 22F4C
  \l 22F4D
  \l 22F4E
  \l 22F4F
  \l 22F50
  \l 22F51
  \l 22F52
  \l 22F53
  \l 22F54
  \l 22F55
  \l 22F56
  \l 22F57
  \l 22F58
  \l 22F59
  \l 22F5A
  \l 22F5B
  \l 22F5C
  \l 22F5D
  \l 22F5E
  \l 22F5F
  \l 22F60
  \l 22F61
  \l 22F62
  \l 22F63
  \l 22F64
  \l 22F65
  \l 22F66
  \l 22F67
  \l 22F68
  \l 22F69
  \l 22F6A
  \l 22F6B
  \l 22F6C
  \l 22F6D
  \l 22F6E
  \l 22F6F
  \l 22F70
  \l 22F71
  \l 22F72
  \l 22F73
  \l 22F74
  \l 22F75
  \l 22F76
  \l 22F77
  \l 22F78
  \l 22F79
  \l 22F7A
  \l 22F7B
  \l 22F7C
  \l 22F7D
  \l 22F7E
  \l 22F7F
  \l 22F80
  \l 22F81
  \l 22F82
  \l 22F83
  \l 22F84
  \l 22F85
  \l 22F86
  \l 22F87
  \l 22F88
  \l 22F89
  \l 22F8A
  \l 22F8B
  \l 22F8C
  \l 22F8D
  \l 22F8E
  \l 22F8F
  \l 22F90
  \l 22F91
  \l 22F92
  \l 22F93
  \l 22F94
  \l 22F95
  \l 22F96
  \l 22F97
  \l 22F98
  \l 22F99
  \l 22F9A
  \l 22F9B
  \l 22F9C
  \l 22F9D
  \l 22F9E
  \l 22F9F
  \l 22FA0
  \l 22FA1
  \l 22FA2
  \l 22FA3
  \l 22FA4
  \l 22FA5
  \l 22FA6
  \l 22FA7
  \l 22FA8
  \l 22FA9
  \l 22FAA
  \l 22FAB
  \l 22FAC
  \l 22FAD
  \l 22FAE
  \l 22FAF
  \l 22FB0
  \l 22FB1
  \l 22FB2
  \l 22FB3
  \l 22FB4
  \l 22FB5
  \l 22FB6
  \l 22FB7
  \l 22FB8
  \l 22FB9
  \l 22FBA
  \l 22FBB
  \l 22FBC
  \l 22FBD
  \l 22FBE
  \l 22FBF
  \l 22FC0
  \l 22FC1
  \l 22FC2
  \l 22FC3
  \l 22FC4
  \l 22FC5
  \l 22FC6
  \l 22FC7
  \l 22FC8
  \l 22FC9
  \l 22FCA
  \l 22FCB
  \l 22FCC
  \l 22FCD
  \l 22FCE
  \l 22FCF
  \l 22FD0
  \l 22FD1
  \l 22FD2
  \l 22FD3
  \l 22FD4
  \l 22FD5
  \l 22FD6
  \l 22FD7
  \l 22FD8
  \l 22FD9
  \l 22FDA
  \l 22FDB
  \l 22FDC
  \l 22FDD
  \l 22FDE
  \l 22FDF
  \l 22FE0
  \l 22FE1
  \l 22FE2
  \l 22FE3
  \l 22FE4
  \l 22FE5
  \l 22FE6
  \l 22FE7
  \l 22FE8
  \l 22FE9
  \l 22FEA
  \l 22FEB
  \l 22FEC
  \l 22FED
  \l 22FEE
  \l 22FEF
  \l 22FF0
  \l 22FF1
  \l 22FF2
  \l 22FF3
  \l 22FF4
  \l 22FF5
  \l 22FF6
  \l 22FF7
  \l 22FF8
  \l 22FF9
  \l 22FFA
  \l 22FFB
  \l 22FFC
  \l 22FFD
  \l 22FFE
  \l 22FFF
  \l 23000
  \l 23001
  \l 23002
  \l 23003
  \l 23004
  \l 23005
  \l 23006
  \l 23007
  \l 23008
  \l 23009
  \l 2300A
  \l 2300B
  \l 2300C
  \l 2300D
  \l 2300E
  \l 2300F
  \l 23010
  \l 23011
  \l 23012
  \l 23013
  \l 23014
  \l 23015
  \l 23016
  \l 23017
  \l 23018
  \l 23019
  \l 2301A
  \l 2301B
  \l 2301C
  \l 2301D
  \l 2301E
  \l 2301F
  \l 23020
  \l 23021
  \l 23022
  \l 23023
  \l 23024
  \l 23025
  \l 23026
  \l 23027
  \l 23028
  \l 23029
  \l 2302A
  \l 2302B
  \l 2302C
  \l 2302D
  \l 2302E
  \l 2302F
  \l 23030
  \l 23031
  \l 23032
  \l 23033
  \l 23034
  \l 23035
  \l 23036
  \l 23037
  \l 23038
  \l 23039
  \l 2303A
  \l 2303B
  \l 2303C
  \l 2303D
  \l 2303E
  \l 2303F
  \l 23040
  \l 23041
  \l 23042
  \l 23043
  \l 23044
  \l 23045
  \l 23046
  \l 23047
  \l 23048
  \l 23049
  \l 2304A
  \l 2304B
  \l 2304C
  \l 2304D
  \l 2304E
  \l 2304F
  \l 23050
  \l 23051
  \l 23052
  \l 23053
  \l 23054
  \l 23055
  \l 23056
  \l 23057
  \l 23058
  \l 23059
  \l 2305A
  \l 2305B
  \l 2305C
  \l 2305D
  \l 2305E
  \l 2305F
  \l 23060
  \l 23061
  \l 23062
  \l 23063
  \l 23064
  \l 23065
  \l 23066
  \l 23067
  \l 23068
  \l 23069
  \l 2306A
  \l 2306B
  \l 2306C
  \l 2306D
  \l 2306E
  \l 2306F
  \l 23070
  \l 23071
  \l 23072
  \l 23073
  \l 23074
  \l 23075
  \l 23076
  \l 23077
  \l 23078
  \l 23079
  \l 2307A
  \l 2307B
  \l 2307C
  \l 2307D
  \l 2307E
  \l 2307F
  \l 23080
  \l 23081
  \l 23082
  \l 23083
  \l 23084
  \l 23085
  \l 23086
  \l 23087
  \l 23088
  \l 23089
  \l 2308A
  \l 2308B
  \l 2308C
  \l 2308D
  \l 2308E
  \l 2308F
  \l 23090
  \l 23091
  \l 23092
  \l 23093
  \l 23094
  \l 23095
  \l 23096
  \l 23097
  \l 23098
  \l 23099
  \l 2309A
  \l 2309B
  \l 2309C
  \l 2309D
  \l 2309E
  \l 2309F
  \l 230A0
  \l 230A1
  \l 230A2
  \l 230A3
  \l 230A4
  \l 230A5
  \l 230A6
  \l 230A7
  \l 230A8
  \l 230A9
  \l 230AA
  \l 230AB
  \l 230AC
  \l 230AD
  \l 230AE
  \l 230AF
  \l 230B0
  \l 230B1
  \l 230B2
  \l 230B3
  \l 230B4
  \l 230B5
  \l 230B6
  \l 230B7
  \l 230B8
  \l 230B9
  \l 230BA
  \l 230BB
  \l 230BC
  \l 230BD
  \l 230BE
  \l 230BF
  \l 230C0
  \l 230C1
  \l 230C2
  \l 230C3
  \l 230C4
  \l 230C5
  \l 230C6
  \l 230C7
  \l 230C8
  \l 230C9
  \l 230CA
  \l 230CB
  \l 230CC
  \l 230CD
  \l 230CE
  \l 230CF
  \l 230D0
  \l 230D1
  \l 230D2
  \l 230D3
  \l 230D4
  \l 230D5
  \l 230D6
  \l 230D7
  \l 230D8
  \l 230D9
  \l 230DA
  \l 230DB
  \l 230DC
  \l 230DD
  \l 230DE
  \l 230DF
  \l 230E0
  \l 230E1
  \l 230E2
  \l 230E3
  \l 230E4
  \l 230E5
  \l 230E6
  \l 230E7
  \l 230E8
  \l 230E9
  \l 230EA
  \l 230EB
  \l 230EC
  \l 230ED
  \l 230EE
  \l 230EF
  \l 230F0
  \l 230F1
  \l 230F2
  \l 230F3
  \l 230F4
  \l 230F5
  \l 230F6
  \l 230F7
  \l 230F8
  \l 230F9
  \l 230FA
  \l 230FB
  \l 230FC
  \l 230FD
  \l 230FE
  \l 230FF
  \l 23100
  \l 23101
  \l 23102
  \l 23103
  \l 23104
  \l 23105
  \l 23106
  \l 23107
  \l 23108
  \l 23109
  \l 2310A
  \l 2310B
  \l 2310C
  \l 2310D
  \l 2310E
  \l 2310F
  \l 23110
  \l 23111
  \l 23112
  \l 23113
  \l 23114
  \l 23115
  \l 23116
  \l 23117
  \l 23118
  \l 23119
  \l 2311A
  \l 2311B
  \l 2311C
  \l 2311D
  \l 2311E
  \l 2311F
  \l 23120
  \l 23121
  \l 23122
  \l 23123
  \l 23124
  \l 23125
  \l 23126
  \l 23127
  \l 23128
  \l 23129
  \l 2312A
  \l 2312B
  \l 2312C
  \l 2312D
  \l 2312E
  \l 2312F
  \l 23130
  \l 23131
  \l 23132
  \l 23133
  \l 23134
  \l 23135
  \l 23136
  \l 23137
  \l 23138
  \l 23139
  \l 2313A
  \l 2313B
  \l 2313C
  \l 2313D
  \l 2313E
  \l 2313F
  \l 23140
  \l 23141
  \l 23142
  \l 23143
  \l 23144
  \l 23145
  \l 23146
  \l 23147
  \l 23148
  \l 23149
  \l 2314A
  \l 2314B
  \l 2314C
  \l 2314D
  \l 2314E
  \l 2314F
  \l 23150
  \l 23151
  \l 23152
  \l 23153
  \l 23154
  \l 23155
  \l 23156
  \l 23157
  \l 23158
  \l 23159
  \l 2315A
  \l 2315B
  \l 2315C
  \l 2315D
  \l 2315E
  \l 2315F
  \l 23160
  \l 23161
  \l 23162
  \l 23163
  \l 23164
  \l 23165
  \l 23166
  \l 23167
  \l 23168
  \l 23169
  \l 2316A
  \l 2316B
  \l 2316C
  \l 2316D
  \l 2316E
  \l 2316F
  \l 23170
  \l 23171
  \l 23172
  \l 23173
  \l 23174
  \l 23175
  \l 23176
  \l 23177
  \l 23178
  \l 23179
  \l 2317A
  \l 2317B
  \l 2317C
  \l 2317D
  \l 2317E
  \l 2317F
  \l 23180
  \l 23181
  \l 23182
  \l 23183
  \l 23184
  \l 23185
  \l 23186
  \l 23187
  \l 23188
  \l 23189
  \l 2318A
  \l 2318B
  \l 2318C
  \l 2318D
  \l 2318E
  \l 2318F
  \l 23190
  \l 23191
  \l 23192
  \l 23193
  \l 23194
  \l 23195
  \l 23196
  \l 23197
  \l 23198
  \l 23199
  \l 2319A
  \l 2319B
  \l 2319C
  \l 2319D
  \l 2319E
  \l 2319F
  \l 231A0
  \l 231A1
  \l 231A2
  \l 231A3
  \l 231A4
  \l 231A5
  \l 231A6
  \l 231A7
  \l 231A8
  \l 231A9
  \l 231AA
  \l 231AB
  \l 231AC
  \l 231AD
  \l 231AE
  \l 231AF
  \l 231B0
  \l 231B1
  \l 231B2
  \l 231B3
  \l 231B4
  \l 231B5
  \l 231B6
  \l 231B7
  \l 231B8
  \l 231B9
  \l 231BA
  \l 231BB
  \l 231BC
  \l 231BD
  \l 231BE
  \l 231BF
  \l 231C0
  \l 231C1
  \l 231C2
  \l 231C3
  \l 231C4
  \l 231C5
  \l 231C6
  \l 231C7
  \l 231C8
  \l 231C9
  \l 231CA
  \l 231CB
  \l 231CC
  \l 231CD
  \l 231CE
  \l 231CF
  \l 231D0
  \l 231D1
  \l 231D2
  \l 231D3
  \l 231D4
  \l 231D5
  \l 231D6
  \l 231D7
  \l 231D8
  \l 231D9
  \l 231DA
  \l 231DB
  \l 231DC
  \l 231DD
  \l 231DE
  \l 231DF
  \l 231E0
  \l 231E1
  \l 231E2
  \l 231E3
  \l 231E4
  \l 231E5
  \l 231E6
  \l 231E7
  \l 231E8
  \l 231E9
  \l 231EA
  \l 231EB
  \l 231EC
  \l 231ED
  \l 231EE
  \l 231EF
  \l 231F0
  \l 231F1
  \l 231F2
  \l 231F3
  \l 231F4
  \l 231F5
  \l 231F6
  \l 231F7
  \l 231F8
  \l 231F9
  \l 231FA
  \l 231FB
  \l 231FC
  \l 231FD
  \l 231FE
  \l 231FF
  \l 23200
  \l 23201
  \l 23202
  \l 23203
  \l 23204
  \l 23205
  \l 23206
  \l 23207
  \l 23208
  \l 23209
  \l 2320A
  \l 2320B
  \l 2320C
  \l 2320D
  \l 2320E
  \l 2320F
  \l 23210
  \l 23211
  \l 23212
  \l 23213
  \l 23214
  \l 23215
  \l 23216
  \l 23217
  \l 23218
  \l 23219
  \l 2321A
  \l 2321B
  \l 2321C
  \l 2321D
  \l 2321E
  \l 2321F
  \l 23220
  \l 23221
  \l 23222
  \l 23223
  \l 23224
  \l 23225
  \l 23226
  \l 23227
  \l 23228
  \l 23229
  \l 2322A
  \l 2322B
  \l 2322C
  \l 2322D
  \l 2322E
  \l 2322F
  \l 23230
  \l 23231
  \l 23232
  \l 23233
  \l 23234
  \l 23235
  \l 23236
  \l 23237
  \l 23238
  \l 23239
  \l 2323A
  \l 2323B
  \l 2323C
  \l 2323D
  \l 2323E
  \l 2323F
  \l 23240
  \l 23241
  \l 23242
  \l 23243
  \l 23244
  \l 23245
  \l 23246
  \l 23247
  \l 23248
  \l 23249
  \l 2324A
  \l 2324B
  \l 2324C
  \l 2324D
  \l 2324E
  \l 2324F
  \l 23250
  \l 23251
  \l 23252
  \l 23253
  \l 23254
  \l 23255
  \l 23256
  \l 23257
  \l 23258
  \l 23259
  \l 2325A
  \l 2325B
  \l 2325C
  \l 2325D
  \l 2325E
  \l 2325F
  \l 23260
  \l 23261
  \l 23262
  \l 23263
  \l 23264
  \l 23265
  \l 23266
  \l 23267
  \l 23268
  \l 23269
  \l 2326A
  \l 2326B
  \l 2326C
  \l 2326D
  \l 2326E
  \l 2326F
  \l 23270
  \l 23271
  \l 23272
  \l 23273
  \l 23274
  \l 23275
  \l 23276
  \l 23277
  \l 23278
  \l 23279
  \l 2327A
  \l 2327B
  \l 2327C
  \l 2327D
  \l 2327E
  \l 2327F
  \l 23280
  \l 23281
  \l 23282
  \l 23283
  \l 23284
  \l 23285
  \l 23286
  \l 23287
  \l 23288
  \l 23289
  \l 2328A
  \l 2328B
  \l 2328C
  \l 2328D
  \l 2328E
  \l 2328F
  \l 23290
  \l 23291
  \l 23292
  \l 23293
  \l 23294
  \l 23295
  \l 23296
  \l 23297
  \l 23298
  \l 23299
  \l 2329A
  \l 2329B
  \l 2329C
  \l 2329D
  \l 2329E
  \l 2329F
  \l 232A0
  \l 232A1
  \l 232A2
  \l 232A3
  \l 232A4
  \l 232A5
  \l 232A6
  \l 232A7
  \l 232A8
  \l 232A9
  \l 232AA
  \l 232AB
  \l 232AC
  \l 232AD
  \l 232AE
  \l 232AF
  \l 232B0
  \l 232B1
  \l 232B2
  \l 232B3
  \l 232B4
  \l 232B5
  \l 232B6
  \l 232B7
  \l 232B8
  \l 232B9
  \l 232BA
  \l 232BB
  \l 232BC
  \l 232BD
  \l 232BE
  \l 232BF
  \l 232C0
  \l 232C1
  \l 232C2
  \l 232C3
  \l 232C4
  \l 232C5
  \l 232C6
  \l 232C7
  \l 232C8
  \l 232C9
  \l 232CA
  \l 232CB
  \l 232CC
  \l 232CD
  \l 232CE
  \l 232CF
  \l 232D0
  \l 232D1
  \l 232D2
  \l 232D3
  \l 232D4
  \l 232D5
  \l 232D6
  \l 232D7
  \l 232D8
  \l 232D9
  \l 232DA
  \l 232DB
  \l 232DC
  \l 232DD
  \l 232DE
  \l 232DF
  \l 232E0
  \l 232E1
  \l 232E2
  \l 232E3
  \l 232E4
  \l 232E5
  \l 232E6
  \l 232E7
  \l 232E8
  \l 232E9
  \l 232EA
  \l 232EB
  \l 232EC
  \l 232ED
  \l 232EE
  \l 232EF
  \l 232F0
  \l 232F1
  \l 232F2
  \l 232F3
  \l 232F4
  \l 232F5
  \l 232F6
  \l 232F7
  \l 232F8
  \l 232F9
  \l 232FA
  \l 232FB
  \l 232FC
  \l 232FD
  \l 232FE
  \l 232FF
  \l 23300
  \l 23301
  \l 23302
  \l 23303
  \l 23304
  \l 23305
  \l 23306
  \l 23307
  \l 23308
  \l 23309
  \l 2330A
  \l 2330B
  \l 2330C
  \l 2330D
  \l 2330E
  \l 2330F
  \l 23310
  \l 23311
  \l 23312
  \l 23313
  \l 23314
  \l 23315
  \l 23316
  \l 23317
  \l 23318
  \l 23319
  \l 2331A
  \l 2331B
  \l 2331C
  \l 2331D
  \l 2331E
  \l 2331F
  \l 23320
  \l 23321
  \l 23322
  \l 23323
  \l 23324
  \l 23325
  \l 23326
  \l 23327
  \l 23328
  \l 23329
  \l 2332A
  \l 2332B
  \l 2332C
  \l 2332D
  \l 2332E
  \l 2332F
  \l 23330
  \l 23331
  \l 23332
  \l 23333
  \l 23334
  \l 23335
  \l 23336
  \l 23337
  \l 23338
  \l 23339
  \l 2333A
  \l 2333B
  \l 2333C
  \l 2333D
  \l 2333E
  \l 2333F
  \l 23340
  \l 23341
  \l 23342
  \l 23343
  \l 23344
  \l 23345
  \l 23346
  \l 23347
  \l 23348
  \l 23349
  \l 2334A
  \l 2334B
  \l 2334C
  \l 2334D
  \l 2334E
  \l 2334F
  \l 23350
  \l 23351
  \l 23352
  \l 23353
  \l 23354
  \l 23355
  \l 23356
  \l 23357
  \l 23358
  \l 23359
  \l 2335A
  \l 2335B
  \l 2335C
  \l 2335D
  \l 2335E
  \l 2335F
  \l 23360
  \l 23361
  \l 23362
  \l 23363
  \l 23364
  \l 23365
  \l 23366
  \l 23367
  \l 23368
  \l 23369
  \l 2336A
  \l 2336B
  \l 2336C
  \l 2336D
  \l 2336E
  \l 2336F
  \l 23370
  \l 23371
  \l 23372
  \l 23373
  \l 23374
  \l 23375
  \l 23376
  \l 23377
  \l 23378
  \l 23379
  \l 2337A
  \l 2337B
  \l 2337C
  \l 2337D
  \l 2337E
  \l 2337F
  \l 23380
  \l 23381
  \l 23382
  \l 23383
  \l 23384
  \l 23385
  \l 23386
  \l 23387
  \l 23388
  \l 23389
  \l 2338A
  \l 2338B
  \l 2338C
  \l 2338D
  \l 2338E
  \l 2338F
  \l 23390
  \l 23391
  \l 23392
  \l 23393
  \l 23394
  \l 23395
  \l 23396
  \l 23397
  \l 23398
  \l 23399
  \l 2339A
  \l 2339B
  \l 2339C
  \l 2339D
  \l 2339E
  \l 2339F
  \l 233A0
  \l 233A1
  \l 233A2
  \l 233A3
  \l 233A4
  \l 233A5
  \l 233A6
  \l 233A7
  \l 233A8
  \l 233A9
  \l 233AA
  \l 233AB
  \l 233AC
  \l 233AD
  \l 233AE
  \l 233AF
  \l 233B0
  \l 233B1
  \l 233B2
  \l 233B3
  \l 233B4
  \l 233B5
  \l 233B6
  \l 233B7
  \l 233B8
  \l 233B9
  \l 233BA
  \l 233BB
  \l 233BC
  \l 233BD
  \l 233BE
  \l 233BF
  \l 233C0
  \l 233C1
  \l 233C2
  \l 233C3
  \l 233C4
  \l 233C5
  \l 233C6
  \l 233C7
  \l 233C8
  \l 233C9
  \l 233CA
  \l 233CB
  \l 233CC
  \l 233CD
  \l 233CE
  \l 233CF
  \l 233D0
  \l 233D1
  \l 233D2
  \l 233D3
  \l 233D4
  \l 233D5
  \l 233D6
  \l 233D7
  \l 233D8
  \l 233D9
  \l 233DA
  \l 233DB
  \l 233DC
  \l 233DD
  \l 233DE
  \l 233DF
  \l 233E0
  \l 233E1
  \l 233E2
  \l 233E3
  \l 233E4
  \l 233E5
  \l 233E6
  \l 233E7
  \l 233E8
  \l 233E9
  \l 233EA
  \l 233EB
  \l 233EC
  \l 233ED
  \l 233EE
  \l 233EF
  \l 233F0
  \l 233F1
  \l 233F2
  \l 233F3
  \l 233F4
  \l 233F5
  \l 233F6
  \l 233F7
  \l 233F8
  \l 233F9
  \l 233FA
  \l 233FB
  \l 233FC
  \l 233FD
  \l 233FE
  \l 233FF
  \l 23400
  \l 23401
  \l 23402
  \l 23403
  \l 23404
  \l 23405
  \l 23406
  \l 23407
  \l 23408
  \l 23409
  \l 2340A
  \l 2340B
  \l 2340C
  \l 2340D
  \l 2340E
  \l 2340F
  \l 23410
  \l 23411
  \l 23412
  \l 23413
  \l 23414
  \l 23415
  \l 23416
  \l 23417
  \l 23418
  \l 23419
  \l 2341A
  \l 2341B
  \l 2341C
  \l 2341D
  \l 2341E
  \l 2341F
  \l 23420
  \l 23421
  \l 23422
  \l 23423
  \l 23424
  \l 23425
  \l 23426
  \l 23427
  \l 23428
  \l 23429
  \l 2342A
  \l 2342B
  \l 2342C
  \l 2342D
  \l 2342E
  \l 2342F
  \l 23430
  \l 23431
  \l 23432
  \l 23433
  \l 23434
  \l 23435
  \l 23436
  \l 23437
  \l 23438
  \l 23439
  \l 2343A
  \l 2343B
  \l 2343C
  \l 2343D
  \l 2343E
  \l 2343F
  \l 23440
  \l 23441
  \l 23442
  \l 23443
  \l 23444
  \l 23445
  \l 23446
  \l 23447
  \l 23448
  \l 23449
  \l 2344A
  \l 2344B
  \l 2344C
  \l 2344D
  \l 2344E
  \l 2344F
  \l 23450
  \l 23451
  \l 23452
  \l 23453
  \l 23454
  \l 23455
  \l 23456
  \l 23457
  \l 23458
  \l 23459
  \l 2345A
  \l 2345B
  \l 2345C
  \l 2345D
  \l 2345E
  \l 2345F
  \l 23460
  \l 23461
  \l 23462
  \l 23463
  \l 23464
  \l 23465
  \l 23466
  \l 23467
  \l 23468
  \l 23469
  \l 2346A
  \l 2346B
  \l 2346C
  \l 2346D
  \l 2346E
  \l 2346F
  \l 23470
  \l 23471
  \l 23472
  \l 23473
  \l 23474
  \l 23475
  \l 23476
  \l 23477
  \l 23478
  \l 23479
  \l 2347A
  \l 2347B
  \l 2347C
  \l 2347D
  \l 2347E
  \l 2347F
  \l 23480
  \l 23481
  \l 23482
  \l 23483
  \l 23484
  \l 23485
  \l 23486
  \l 23487
  \l 23488
  \l 23489
  \l 2348A
  \l 2348B
  \l 2348C
  \l 2348D
  \l 2348E
  \l 2348F
  \l 23490
  \l 23491
  \l 23492
  \l 23493
  \l 23494
  \l 23495
  \l 23496
  \l 23497
  \l 23498
  \l 23499
  \l 2349A
  \l 2349B
  \l 2349C
  \l 2349D
  \l 2349E
  \l 2349F
  \l 234A0
  \l 234A1
  \l 234A2
  \l 234A3
  \l 234A4
  \l 234A5
  \l 234A6
  \l 234A7
  \l 234A8
  \l 234A9
  \l 234AA
  \l 234AB
  \l 234AC
  \l 234AD
  \l 234AE
  \l 234AF
  \l 234B0
  \l 234B1
  \l 234B2
  \l 234B3
  \l 234B4
  \l 234B5
  \l 234B6
  \l 234B7
  \l 234B8
  \l 234B9
  \l 234BA
  \l 234BB
  \l 234BC
  \l 234BD
  \l 234BE
  \l 234BF
  \l 234C0
  \l 234C1
  \l 234C2
  \l 234C3
  \l 234C4
  \l 234C5
  \l 234C6
  \l 234C7
  \l 234C8
  \l 234C9
  \l 234CA
  \l 234CB
  \l 234CC
  \l 234CD
  \l 234CE
  \l 234CF
  \l 234D0
  \l 234D1
  \l 234D2
  \l 234D3
  \l 234D4
  \l 234D5
  \l 234D6
  \l 234D7
  \l 234D8
  \l 234D9
  \l 234DA
  \l 234DB
  \l 234DC
  \l 234DD
  \l 234DE
  \l 234DF
  \l 234E0
  \l 234E1
  \l 234E2
  \l 234E3
  \l 234E4
  \l 234E5
  \l 234E6
  \l 234E7
  \l 234E8
  \l 234E9
  \l 234EA
  \l 234EB
  \l 234EC
  \l 234ED
  \l 234EE
  \l 234EF
  \l 234F0
  \l 234F1
  \l 234F2
  \l 234F3
  \l 234F4
  \l 234F5
  \l 234F6
  \l 234F7
  \l 234F8
  \l 234F9
  \l 234FA
  \l 234FB
  \l 234FC
  \l 234FD
  \l 234FE
  \l 234FF
  \l 23500
  \l 23501
  \l 23502
  \l 23503
  \l 23504
  \l 23505
  \l 23506
  \l 23507
  \l 23508
  \l 23509
  \l 2350A
  \l 2350B
  \l 2350C
  \l 2350D
  \l 2350E
  \l 2350F
  \l 23510
  \l 23511
  \l 23512
  \l 23513
  \l 23514
  \l 23515
  \l 23516
  \l 23517
  \l 23518
  \l 23519
  \l 2351A
  \l 2351B
  \l 2351C
  \l 2351D
  \l 2351E
  \l 2351F
  \l 23520
  \l 23521
  \l 23522
  \l 23523
  \l 23524
  \l 23525
  \l 23526
  \l 23527
  \l 23528
  \l 23529
  \l 2352A
  \l 2352B
  \l 2352C
  \l 2352D
  \l 2352E
  \l 2352F
  \l 23530
  \l 23531
  \l 23532
  \l 23533
  \l 23534
  \l 23535
  \l 23536
  \l 23537
  \l 23538
  \l 23539
  \l 2353A
  \l 2353B
  \l 2353C
  \l 2353D
  \l 2353E
  \l 2353F
  \l 23540
  \l 23541
  \l 23542
  \l 23543
  \l 23544
  \l 23545
  \l 23546
  \l 23547
  \l 23548
  \l 23549
  \l 2354A
  \l 2354B
  \l 2354C
  \l 2354D
  \l 2354E
  \l 2354F
  \l 23550
  \l 23551
  \l 23552
  \l 23553
  \l 23554
  \l 23555
  \l 23556
  \l 23557
  \l 23558
  \l 23559
  \l 2355A
  \l 2355B
  \l 2355C
  \l 2355D
  \l 2355E
  \l 2355F
  \l 23560
  \l 23561
  \l 23562
  \l 23563
  \l 23564
  \l 23565
  \l 23566
  \l 23567
  \l 23568
  \l 23569
  \l 2356A
  \l 2356B
  \l 2356C
  \l 2356D
  \l 2356E
  \l 2356F
  \l 23570
  \l 23571
  \l 23572
  \l 23573
  \l 23574
  \l 23575
  \l 23576
  \l 23577
  \l 23578
  \l 23579
  \l 2357A
  \l 2357B
  \l 2357C
  \l 2357D
  \l 2357E
  \l 2357F
  \l 23580
  \l 23581
  \l 23582
  \l 23583
  \l 23584
  \l 23585
  \l 23586
  \l 23587
  \l 23588
  \l 23589
  \l 2358A
  \l 2358B
  \l 2358C
  \l 2358D
  \l 2358E
  \l 2358F
  \l 23590
  \l 23591
  \l 23592
  \l 23593
  \l 23594
  \l 23595
  \l 23596
  \l 23597
  \l 23598
  \l 23599
  \l 2359A
  \l 2359B
  \l 2359C
  \l 2359D
  \l 2359E
  \l 2359F
  \l 235A0
  \l 235A1
  \l 235A2
  \l 235A3
  \l 235A4
  \l 235A5
  \l 235A6
  \l 235A7
  \l 235A8
  \l 235A9
  \l 235AA
  \l 235AB
  \l 235AC
  \l 235AD
  \l 235AE
  \l 235AF
  \l 235B0
  \l 235B1
  \l 235B2
  \l 235B3
  \l 235B4
  \l 235B5
  \l 235B6
  \l 235B7
  \l 235B8
  \l 235B9
  \l 235BA
  \l 235BB
  \l 235BC
  \l 235BD
  \l 235BE
  \l 235BF
  \l 235C0
  \l 235C1
  \l 235C2
  \l 235C3
  \l 235C4
  \l 235C5
  \l 235C6
  \l 235C7
  \l 235C8
  \l 235C9
  \l 235CA
  \l 235CB
  \l 235CC
  \l 235CD
  \l 235CE
  \l 235CF
  \l 235D0
  \l 235D1
  \l 235D2
  \l 235D3
  \l 235D4
  \l 235D5
  \l 235D6
  \l 235D7
  \l 235D8
  \l 235D9
  \l 235DA
  \l 235DB
  \l 235DC
  \l 235DD
  \l 235DE
  \l 235DF
  \l 235E0
  \l 235E1
  \l 235E2
  \l 235E3
  \l 235E4
  \l 235E5
  \l 235E6
  \l 235E7
  \l 235E8
  \l 235E9
  \l 235EA
  \l 235EB
  \l 235EC
  \l 235ED
  \l 235EE
  \l 235EF
  \l 235F0
  \l 235F1
  \l 235F2
  \l 235F3
  \l 235F4
  \l 235F5
  \l 235F6
  \l 235F7
  \l 235F8
  \l 235F9
  \l 235FA
  \l 235FB
  \l 235FC
  \l 235FD
  \l 235FE
  \l 235FF
  \l 23600
  \l 23601
  \l 23602
  \l 23603
  \l 23604
  \l 23605
  \l 23606
  \l 23607
  \l 23608
  \l 23609
  \l 2360A
  \l 2360B
  \l 2360C
  \l 2360D
  \l 2360E
  \l 2360F
  \l 23610
  \l 23611
  \l 23612
  \l 23613
  \l 23614
  \l 23615
  \l 23616
  \l 23617
  \l 23618
  \l 23619
  \l 2361A
  \l 2361B
  \l 2361C
  \l 2361D
  \l 2361E
  \l 2361F
  \l 23620
  \l 23621
  \l 23622
  \l 23623
  \l 23624
  \l 23625
  \l 23626
  \l 23627
  \l 23628
  \l 23629
  \l 2362A
  \l 2362B
  \l 2362C
  \l 2362D
  \l 2362E
  \l 2362F
  \l 23630
  \l 23631
  \l 23632
  \l 23633
  \l 23634
  \l 23635
  \l 23636
  \l 23637
  \l 23638
  \l 23639
  \l 2363A
  \l 2363B
  \l 2363C
  \l 2363D
  \l 2363E
  \l 2363F
  \l 23640
  \l 23641
  \l 23642
  \l 23643
  \l 23644
  \l 23645
  \l 23646
  \l 23647
  \l 23648
  \l 23649
  \l 2364A
  \l 2364B
  \l 2364C
  \l 2364D
  \l 2364E
  \l 2364F
  \l 23650
  \l 23651
  \l 23652
  \l 23653
  \l 23654
  \l 23655
  \l 23656
  \l 23657
  \l 23658
  \l 23659
  \l 2365A
  \l 2365B
  \l 2365C
  \l 2365D
  \l 2365E
  \l 2365F
  \l 23660
  \l 23661
  \l 23662
  \l 23663
  \l 23664
  \l 23665
  \l 23666
  \l 23667
  \l 23668
  \l 23669
  \l 2366A
  \l 2366B
  \l 2366C
  \l 2366D
  \l 2366E
  \l 2366F
  \l 23670
  \l 23671
  \l 23672
  \l 23673
  \l 23674
  \l 23675
  \l 23676
  \l 23677
  \l 23678
  \l 23679
  \l 2367A
  \l 2367B
  \l 2367C
  \l 2367D
  \l 2367E
  \l 2367F
  \l 23680
  \l 23681
  \l 23682
  \l 23683
  \l 23684
  \l 23685
  \l 23686
  \l 23687
  \l 23688
  \l 23689
  \l 2368A
  \l 2368B
  \l 2368C
  \l 2368D
  \l 2368E
  \l 2368F
  \l 23690
  \l 23691
  \l 23692
  \l 23693
  \l 23694
  \l 23695
  \l 23696
  \l 23697
  \l 23698
  \l 23699
  \l 2369A
  \l 2369B
  \l 2369C
  \l 2369D
  \l 2369E
  \l 2369F
  \l 236A0
  \l 236A1
  \l 236A2
  \l 236A3
  \l 236A4
  \l 236A5
  \l 236A6
  \l 236A7
  \l 236A8
  \l 236A9
  \l 236AA
  \l 236AB
  \l 236AC
  \l 236AD
  \l 236AE
  \l 236AF
  \l 236B0
  \l 236B1
  \l 236B2
  \l 236B3
  \l 236B4
  \l 236B5
  \l 236B6
  \l 236B7
  \l 236B8
  \l 236B9
  \l 236BA
  \l 236BB
  \l 236BC
  \l 236BD
  \l 236BE
  \l 236BF
  \l 236C0
  \l 236C1
  \l 236C2
  \l 236C3
  \l 236C4
  \l 236C5
  \l 236C6
  \l 236C7
  \l 236C8
  \l 236C9
  \l 236CA
  \l 236CB
  \l 236CC
  \l 236CD
  \l 236CE
  \l 236CF
  \l 236D0
  \l 236D1
  \l 236D2
  \l 236D3
  \l 236D4
  \l 236D5
  \l 236D6
  \l 236D7
  \l 236D8
  \l 236D9
  \l 236DA
  \l 236DB
  \l 236DC
  \l 236DD
  \l 236DE
  \l 236DF
  \l 236E0
  \l 236E1
  \l 236E2
  \l 236E3
  \l 236E4
  \l 236E5
  \l 236E6
  \l 236E7
  \l 236E8
  \l 236E9
  \l 236EA
  \l 236EB
  \l 236EC
  \l 236ED
  \l 236EE
  \l 236EF
  \l 236F0
  \l 236F1
  \l 236F2
  \l 236F3
  \l 236F4
  \l 236F5
  \l 236F6
  \l 236F7
  \l 236F8
  \l 236F9
  \l 236FA
  \l 236FB
  \l 236FC
  \l 236FD
  \l 236FE
  \l 236FF
  \l 23700
  \l 23701
  \l 23702
  \l 23703
  \l 23704
  \l 23705
  \l 23706
  \l 23707
  \l 23708
  \l 23709
  \l 2370A
  \l 2370B
  \l 2370C
  \l 2370D
  \l 2370E
  \l 2370F
  \l 23710
  \l 23711
  \l 23712
  \l 23713
  \l 23714
  \l 23715
  \l 23716
  \l 23717
  \l 23718
  \l 23719
  \l 2371A
  \l 2371B
  \l 2371C
  \l 2371D
  \l 2371E
  \l 2371F
  \l 23720
  \l 23721
  \l 23722
  \l 23723
  \l 23724
  \l 23725
  \l 23726
  \l 23727
  \l 23728
  \l 23729
  \l 2372A
  \l 2372B
  \l 2372C
  \l 2372D
  \l 2372E
  \l 2372F
  \l 23730
  \l 23731
  \l 23732
  \l 23733
  \l 23734
  \l 23735
  \l 23736
  \l 23737
  \l 23738
  \l 23739
  \l 2373A
  \l 2373B
  \l 2373C
  \l 2373D
  \l 2373E
  \l 2373F
  \l 23740
  \l 23741
  \l 23742
  \l 23743
  \l 23744
  \l 23745
  \l 23746
  \l 23747
  \l 23748
  \l 23749
  \l 2374A
  \l 2374B
  \l 2374C
  \l 2374D
  \l 2374E
  \l 2374F
  \l 23750
  \l 23751
  \l 23752
  \l 23753
  \l 23754
  \l 23755
  \l 23756
  \l 23757
  \l 23758
  \l 23759
  \l 2375A
  \l 2375B
  \l 2375C
  \l 2375D
  \l 2375E
  \l 2375F
  \l 23760
  \l 23761
  \l 23762
  \l 23763
  \l 23764
  \l 23765
  \l 23766
  \l 23767
  \l 23768
  \l 23769
  \l 2376A
  \l 2376B
  \l 2376C
  \l 2376D
  \l 2376E
  \l 2376F
  \l 23770
  \l 23771
  \l 23772
  \l 23773
  \l 23774
  \l 23775
  \l 23776
  \l 23777
  \l 23778
  \l 23779
  \l 2377A
  \l 2377B
  \l 2377C
  \l 2377D
  \l 2377E
  \l 2377F
  \l 23780
  \l 23781
  \l 23782
  \l 23783
  \l 23784
  \l 23785
  \l 23786
  \l 23787
  \l 23788
  \l 23789
  \l 2378A
  \l 2378B
  \l 2378C
  \l 2378D
  \l 2378E
  \l 2378F
  \l 23790
  \l 23791
  \l 23792
  \l 23793
  \l 23794
  \l 23795
  \l 23796
  \l 23797
  \l 23798
  \l 23799
  \l 2379A
  \l 2379B
  \l 2379C
  \l 2379D
  \l 2379E
  \l 2379F
  \l 237A0
  \l 237A1
  \l 237A2
  \l 237A3
  \l 237A4
  \l 237A5
  \l 237A6
  \l 237A7
  \l 237A8
  \l 237A9
  \l 237AA
  \l 237AB
  \l 237AC
  \l 237AD
  \l 237AE
  \l 237AF
  \l 237B0
  \l 237B1
  \l 237B2
  \l 237B3
  \l 237B4
  \l 237B5
  \l 237B6
  \l 237B7
  \l 237B8
  \l 237B9
  \l 237BA
  \l 237BB
  \l 237BC
  \l 237BD
  \l 237BE
  \l 237BF
  \l 237C0
  \l 237C1
  \l 237C2
  \l 237C3
  \l 237C4
  \l 237C5
  \l 237C6
  \l 237C7
  \l 237C8
  \l 237C9
  \l 237CA
  \l 237CB
  \l 237CC
  \l 237CD
  \l 237CE
  \l 237CF
  \l 237D0
  \l 237D1
  \l 237D2
  \l 237D3
  \l 237D4
  \l 237D5
  \l 237D6
  \l 237D7
  \l 237D8
  \l 237D9
  \l 237DA
  \l 237DB
  \l 237DC
  \l 237DD
  \l 237DE
  \l 237DF
  \l 237E0
  \l 237E1
  \l 237E2
  \l 237E3
  \l 237E4
  \l 237E5
  \l 237E6
  \l 237E7
  \l 237E8
  \l 237E9
  \l 237EA
  \l 237EB
  \l 237EC
  \l 237ED
  \l 237EE
  \l 237EF
  \l 237F0
  \l 237F1
  \l 237F2
  \l 237F3
  \l 237F4
  \l 237F5
  \l 237F6
  \l 237F7
  \l 237F8
  \l 237F9
  \l 237FA
  \l 237FB
  \l 237FC
  \l 237FD
  \l 237FE
  \l 237FF
  \l 23800
  \l 23801
  \l 23802
  \l 23803
  \l 23804
  \l 23805
  \l 23806
  \l 23807
  \l 23808
  \l 23809
  \l 2380A
  \l 2380B
  \l 2380C
  \l 2380D
  \l 2380E
  \l 2380F
  \l 23810
  \l 23811
  \l 23812
  \l 23813
  \l 23814
  \l 23815
  \l 23816
  \l 23817
  \l 23818
  \l 23819
  \l 2381A
  \l 2381B
  \l 2381C
  \l 2381D
  \l 2381E
  \l 2381F
  \l 23820
  \l 23821
  \l 23822
  \l 23823
  \l 23824
  \l 23825
  \l 23826
  \l 23827
  \l 23828
  \l 23829
  \l 2382A
  \l 2382B
  \l 2382C
  \l 2382D
  \l 2382E
  \l 2382F
  \l 23830
  \l 23831
  \l 23832
  \l 23833
  \l 23834
  \l 23835
  \l 23836
  \l 23837
  \l 23838
  \l 23839
  \l 2383A
  \l 2383B
  \l 2383C
  \l 2383D
  \l 2383E
  \l 2383F
  \l 23840
  \l 23841
  \l 23842
  \l 23843
  \l 23844
  \l 23845
  \l 23846
  \l 23847
  \l 23848
  \l 23849
  \l 2384A
  \l 2384B
  \l 2384C
  \l 2384D
  \l 2384E
  \l 2384F
  \l 23850
  \l 23851
  \l 23852
  \l 23853
  \l 23854
  \l 23855
  \l 23856
  \l 23857
  \l 23858
  \l 23859
  \l 2385A
  \l 2385B
  \l 2385C
  \l 2385D
  \l 2385E
  \l 2385F
  \l 23860
  \l 23861
  \l 23862
  \l 23863
  \l 23864
  \l 23865
  \l 23866
  \l 23867
  \l 23868
  \l 23869
  \l 2386A
  \l 2386B
  \l 2386C
  \l 2386D
  \l 2386E
  \l 2386F
  \l 23870
  \l 23871
  \l 23872
  \l 23873
  \l 23874
  \l 23875
  \l 23876
  \l 23877
  \l 23878
  \l 23879
  \l 2387A
  \l 2387B
  \l 2387C
  \l 2387D
  \l 2387E
  \l 2387F
  \l 23880
  \l 23881
  \l 23882
  \l 23883
  \l 23884
  \l 23885
  \l 23886
  \l 23887
  \l 23888
  \l 23889
  \l 2388A
  \l 2388B
  \l 2388C
  \l 2388D
  \l 2388E
  \l 2388F
  \l 23890
  \l 23891
  \l 23892
  \l 23893
  \l 23894
  \l 23895
  \l 23896
  \l 23897
  \l 23898
  \l 23899
  \l 2389A
  \l 2389B
  \l 2389C
  \l 2389D
  \l 2389E
  \l 2389F
  \l 238A0
  \l 238A1
  \l 238A2
  \l 238A3
  \l 238A4
  \l 238A5
  \l 238A6
  \l 238A7
  \l 238A8
  \l 238A9
  \l 238AA
  \l 238AB
  \l 238AC
  \l 238AD
  \l 238AE
  \l 238AF
  \l 238B0
  \l 238B1
  \l 238B2
  \l 238B3
  \l 238B4
  \l 238B5
  \l 238B6
  \l 238B7
  \l 238B8
  \l 238B9
  \l 238BA
  \l 238BB
  \l 238BC
  \l 238BD
  \l 238BE
  \l 238BF
  \l 238C0
  \l 238C1
  \l 238C2
  \l 238C3
  \l 238C4
  \l 238C5
  \l 238C6
  \l 238C7
  \l 238C8
  \l 238C9
  \l 238CA
  \l 238CB
  \l 238CC
  \l 238CD
  \l 238CE
  \l 238CF
  \l 238D0
  \l 238D1
  \l 238D2
  \l 238D3
  \l 238D4
  \l 238D5
  \l 238D6
  \l 238D7
  \l 238D8
  \l 238D9
  \l 238DA
  \l 238DB
  \l 238DC
  \l 238DD
  \l 238DE
  \l 238DF
  \l 238E0
  \l 238E1
  \l 238E2
  \l 238E3
  \l 238E4
  \l 238E5
  \l 238E6
  \l 238E7
  \l 238E8
  \l 238E9
  \l 238EA
  \l 238EB
  \l 238EC
  \l 238ED
  \l 238EE
  \l 238EF
  \l 238F0
  \l 238F1
  \l 238F2
  \l 238F3
  \l 238F4
  \l 238F5
  \l 238F6
  \l 238F7
  \l 238F8
  \l 238F9
  \l 238FA
  \l 238FB
  \l 238FC
  \l 238FD
  \l 238FE
  \l 238FF
  \l 23900
  \l 23901
  \l 23902
  \l 23903
  \l 23904
  \l 23905
  \l 23906
  \l 23907
  \l 23908
  \l 23909
  \l 2390A
  \l 2390B
  \l 2390C
  \l 2390D
  \l 2390E
  \l 2390F
  \l 23910
  \l 23911
  \l 23912
  \l 23913
  \l 23914
  \l 23915
  \l 23916
  \l 23917
  \l 23918
  \l 23919
  \l 2391A
  \l 2391B
  \l 2391C
  \l 2391D
  \l 2391E
  \l 2391F
  \l 23920
  \l 23921
  \l 23922
  \l 23923
  \l 23924
  \l 23925
  \l 23926
  \l 23927
  \l 23928
  \l 23929
  \l 2392A
  \l 2392B
  \l 2392C
  \l 2392D
  \l 2392E
  \l 2392F
  \l 23930
  \l 23931
  \l 23932
  \l 23933
  \l 23934
  \l 23935
  \l 23936
  \l 23937
  \l 23938
  \l 23939
  \l 2393A
  \l 2393B
  \l 2393C
  \l 2393D
  \l 2393E
  \l 2393F
  \l 23940
  \l 23941
  \l 23942
  \l 23943
  \l 23944
  \l 23945
  \l 23946
  \l 23947
  \l 23948
  \l 23949
  \l 2394A
  \l 2394B
  \l 2394C
  \l 2394D
  \l 2394E
  \l 2394F
  \l 23950
  \l 23951
  \l 23952
  \l 23953
  \l 23954
  \l 23955
  \l 23956
  \l 23957
  \l 23958
  \l 23959
  \l 2395A
  \l 2395B
  \l 2395C
  \l 2395D
  \l 2395E
  \l 2395F
  \l 23960
  \l 23961
  \l 23962
  \l 23963
  \l 23964
  \l 23965
  \l 23966
  \l 23967
  \l 23968
  \l 23969
  \l 2396A
  \l 2396B
  \l 2396C
  \l 2396D
  \l 2396E
  \l 2396F
  \l 23970
  \l 23971
  \l 23972
  \l 23973
  \l 23974
  \l 23975
  \l 23976
  \l 23977
  \l 23978
  \l 23979
  \l 2397A
  \l 2397B
  \l 2397C
  \l 2397D
  \l 2397E
  \l 2397F
  \l 23980
  \l 23981
  \l 23982
  \l 23983
  \l 23984
  \l 23985
  \l 23986
  \l 23987
  \l 23988
  \l 23989
  \l 2398A
  \l 2398B
  \l 2398C
  \l 2398D
  \l 2398E
  \l 2398F
  \l 23990
  \l 23991
  \l 23992
  \l 23993
  \l 23994
  \l 23995
  \l 23996
  \l 23997
  \l 23998
  \l 23999
  \l 2399A
  \l 2399B
  \l 2399C
  \l 2399D
  \l 2399E
  \l 2399F
  \l 239A0
  \l 239A1
  \l 239A2
  \l 239A3
  \l 239A4
  \l 239A5
  \l 239A6
  \l 239A7
  \l 239A8
  \l 239A9
  \l 239AA
  \l 239AB
  \l 239AC
  \l 239AD
  \l 239AE
  \l 239AF
  \l 239B0
  \l 239B1
  \l 239B2
  \l 239B3
  \l 239B4
  \l 239B5
  \l 239B6
  \l 239B7
  \l 239B8
  \l 239B9
  \l 239BA
  \l 239BB
  \l 239BC
  \l 239BD
  \l 239BE
  \l 239BF
  \l 239C0
  \l 239C1
  \l 239C2
  \l 239C3
  \l 239C4
  \l 239C5
  \l 239C6
  \l 239C7
  \l 239C8
  \l 239C9
  \l 239CA
  \l 239CB
  \l 239CC
  \l 239CD
  \l 239CE
  \l 239CF
  \l 239D0
  \l 239D1
  \l 239D2
  \l 239D3
  \l 239D4
  \l 239D5
  \l 239D6
  \l 239D7
  \l 239D8
  \l 239D9
  \l 239DA
  \l 239DB
  \l 239DC
  \l 239DD
  \l 239DE
  \l 239DF
  \l 239E0
  \l 239E1
  \l 239E2
  \l 239E3
  \l 239E4
  \l 239E5
  \l 239E6
  \l 239E7
  \l 239E8
  \l 239E9
  \l 239EA
  \l 239EB
  \l 239EC
  \l 239ED
  \l 239EE
  \l 239EF
  \l 239F0
  \l 239F1
  \l 239F2
  \l 239F3
  \l 239F4
  \l 239F5
  \l 239F6
  \l 239F7
  \l 239F8
  \l 239F9
  \l 239FA
  \l 239FB
  \l 239FC
  \l 239FD
  \l 239FE
  \l 239FF
  \l 23A00
  \l 23A01
  \l 23A02
  \l 23A03
  \l 23A04
  \l 23A05
  \l 23A06
  \l 23A07
  \l 23A08
  \l 23A09
  \l 23A0A
  \l 23A0B
  \l 23A0C
  \l 23A0D
  \l 23A0E
  \l 23A0F
  \l 23A10
  \l 23A11
  \l 23A12
  \l 23A13
  \l 23A14
  \l 23A15
  \l 23A16
  \l 23A17
  \l 23A18
  \l 23A19
  \l 23A1A
  \l 23A1B
  \l 23A1C
  \l 23A1D
  \l 23A1E
  \l 23A1F
  \l 23A20
  \l 23A21
  \l 23A22
  \l 23A23
  \l 23A24
  \l 23A25
  \l 23A26
  \l 23A27
  \l 23A28
  \l 23A29
  \l 23A2A
  \l 23A2B
  \l 23A2C
  \l 23A2D
  \l 23A2E
  \l 23A2F
  \l 23A30
  \l 23A31
  \l 23A32
  \l 23A33
  \l 23A34
  \l 23A35
  \l 23A36
  \l 23A37
  \l 23A38
  \l 23A39
  \l 23A3A
  \l 23A3B
  \l 23A3C
  \l 23A3D
  \l 23A3E
  \l 23A3F
  \l 23A40
  \l 23A41
  \l 23A42
  \l 23A43
  \l 23A44
  \l 23A45
  \l 23A46
  \l 23A47
  \l 23A48
  \l 23A49
  \l 23A4A
  \l 23A4B
  \l 23A4C
  \l 23A4D
  \l 23A4E
  \l 23A4F
  \l 23A50
  \l 23A51
  \l 23A52
  \l 23A53
  \l 23A54
  \l 23A55
  \l 23A56
  \l 23A57
  \l 23A58
  \l 23A59
  \l 23A5A
  \l 23A5B
  \l 23A5C
  \l 23A5D
  \l 23A5E
  \l 23A5F
  \l 23A60
  \l 23A61
  \l 23A62
  \l 23A63
  \l 23A64
  \l 23A65
  \l 23A66
  \l 23A67
  \l 23A68
  \l 23A69
  \l 23A6A
  \l 23A6B
  \l 23A6C
  \l 23A6D
  \l 23A6E
  \l 23A6F
  \l 23A70
  \l 23A71
  \l 23A72
  \l 23A73
  \l 23A74
  \l 23A75
  \l 23A76
  \l 23A77
  \l 23A78
  \l 23A79
  \l 23A7A
  \l 23A7B
  \l 23A7C
  \l 23A7D
  \l 23A7E
  \l 23A7F
  \l 23A80
  \l 23A81
  \l 23A82
  \l 23A83
  \l 23A84
  \l 23A85
  \l 23A86
  \l 23A87
  \l 23A88
  \l 23A89
  \l 23A8A
  \l 23A8B
  \l 23A8C
  \l 23A8D
  \l 23A8E
  \l 23A8F
  \l 23A90
  \l 23A91
  \l 23A92
  \l 23A93
  \l 23A94
  \l 23A95
  \l 23A96
  \l 23A97
  \l 23A98
  \l 23A99
  \l 23A9A
  \l 23A9B
  \l 23A9C
  \l 23A9D
  \l 23A9E
  \l 23A9F
  \l 23AA0
  \l 23AA1
  \l 23AA2
  \l 23AA3
  \l 23AA4
  \l 23AA5
  \l 23AA6
  \l 23AA7
  \l 23AA8
  \l 23AA9
  \l 23AAA
  \l 23AAB
  \l 23AAC
  \l 23AAD
  \l 23AAE
  \l 23AAF
  \l 23AB0
  \l 23AB1
  \l 23AB2
  \l 23AB3
  \l 23AB4
  \l 23AB5
  \l 23AB6
  \l 23AB7
  \l 23AB8
  \l 23AB9
  \l 23ABA
  \l 23ABB
  \l 23ABC
  \l 23ABD
  \l 23ABE
  \l 23ABF
  \l 23AC0
  \l 23AC1
  \l 23AC2
  \l 23AC3
  \l 23AC4
  \l 23AC5
  \l 23AC6
  \l 23AC7
  \l 23AC8
  \l 23AC9
  \l 23ACA
  \l 23ACB
  \l 23ACC
  \l 23ACD
  \l 23ACE
  \l 23ACF
  \l 23AD0
  \l 23AD1
  \l 23AD2
  \l 23AD3
  \l 23AD4
  \l 23AD5
  \l 23AD6
  \l 23AD7
  \l 23AD8
  \l 23AD9
  \l 23ADA
  \l 23ADB
  \l 23ADC
  \l 23ADD
  \l 23ADE
  \l 23ADF
  \l 23AE0
  \l 23AE1
  \l 23AE2
  \l 23AE3
  \l 23AE4
  \l 23AE5
  \l 23AE6
  \l 23AE7
  \l 23AE8
  \l 23AE9
  \l 23AEA
  \l 23AEB
  \l 23AEC
  \l 23AED
  \l 23AEE
  \l 23AEF
  \l 23AF0
  \l 23AF1
  \l 23AF2
  \l 23AF3
  \l 23AF4
  \l 23AF5
  \l 23AF6
  \l 23AF7
  \l 23AF8
  \l 23AF9
  \l 23AFA
  \l 23AFB
  \l 23AFC
  \l 23AFD
  \l 23AFE
  \l 23AFF
  \l 23B00
  \l 23B01
  \l 23B02
  \l 23B03
  \l 23B04
  \l 23B05
  \l 23B06
  \l 23B07
  \l 23B08
  \l 23B09
  \l 23B0A
  \l 23B0B
  \l 23B0C
  \l 23B0D
  \l 23B0E
  \l 23B0F
  \l 23B10
  \l 23B11
  \l 23B12
  \l 23B13
  \l 23B14
  \l 23B15
  \l 23B16
  \l 23B17
  \l 23B18
  \l 23B19
  \l 23B1A
  \l 23B1B
  \l 23B1C
  \l 23B1D
  \l 23B1E
  \l 23B1F
  \l 23B20
  \l 23B21
  \l 23B22
  \l 23B23
  \l 23B24
  \l 23B25
  \l 23B26
  \l 23B27
  \l 23B28
  \l 23B29
  \l 23B2A
  \l 23B2B
  \l 23B2C
  \l 23B2D
  \l 23B2E
  \l 23B2F
  \l 23B30
  \l 23B31
  \l 23B32
  \l 23B33
  \l 23B34
  \l 23B35
  \l 23B36
  \l 23B37
  \l 23B38
  \l 23B39
  \l 23B3A
  \l 23B3B
  \l 23B3C
  \l 23B3D
  \l 23B3E
  \l 23B3F
  \l 23B40
  \l 23B41
  \l 23B42
  \l 23B43
  \l 23B44
  \l 23B45
  \l 23B46
  \l 23B47
  \l 23B48
  \l 23B49
  \l 23B4A
  \l 23B4B
  \l 23B4C
  \l 23B4D
  \l 23B4E
  \l 23B4F
  \l 23B50
  \l 23B51
  \l 23B52
  \l 23B53
  \l 23B54
  \l 23B55
  \l 23B56
  \l 23B57
  \l 23B58
  \l 23B59
  \l 23B5A
  \l 23B5B
  \l 23B5C
  \l 23B5D
  \l 23B5E
  \l 23B5F
  \l 23B60
  \l 23B61
  \l 23B62
  \l 23B63
  \l 23B64
  \l 23B65
  \l 23B66
  \l 23B67
  \l 23B68
  \l 23B69
  \l 23B6A
  \l 23B6B
  \l 23B6C
  \l 23B6D
  \l 23B6E
  \l 23B6F
  \l 23B70
  \l 23B71
  \l 23B72
  \l 23B73
  \l 23B74
  \l 23B75
  \l 23B76
  \l 23B77
  \l 23B78
  \l 23B79
  \l 23B7A
  \l 23B7B
  \l 23B7C
  \l 23B7D
  \l 23B7E
  \l 23B7F
  \l 23B80
  \l 23B81
  \l 23B82
  \l 23B83
  \l 23B84
  \l 23B85
  \l 23B86
  \l 23B87
  \l 23B88
  \l 23B89
  \l 23B8A
  \l 23B8B
  \l 23B8C
  \l 23B8D
  \l 23B8E
  \l 23B8F
  \l 23B90
  \l 23B91
  \l 23B92
  \l 23B93
  \l 23B94
  \l 23B95
  \l 23B96
  \l 23B97
  \l 23B98
  \l 23B99
  \l 23B9A
  \l 23B9B
  \l 23B9C
  \l 23B9D
  \l 23B9E
  \l 23B9F
  \l 23BA0
  \l 23BA1
  \l 23BA2
  \l 23BA3
  \l 23BA4
  \l 23BA5
  \l 23BA6
  \l 23BA7
  \l 23BA8
  \l 23BA9
  \l 23BAA
  \l 23BAB
  \l 23BAC
  \l 23BAD
  \l 23BAE
  \l 23BAF
  \l 23BB0
  \l 23BB1
  \l 23BB2
  \l 23BB3
  \l 23BB4
  \l 23BB5
  \l 23BB6
  \l 23BB7
  \l 23BB8
  \l 23BB9
  \l 23BBA
  \l 23BBB
  \l 23BBC
  \l 23BBD
  \l 23BBE
  \l 23BBF
  \l 23BC0
  \l 23BC1
  \l 23BC2
  \l 23BC3
  \l 23BC4
  \l 23BC5
  \l 23BC6
  \l 23BC7
  \l 23BC8
  \l 23BC9
  \l 23BCA
  \l 23BCB
  \l 23BCC
  \l 23BCD
  \l 23BCE
  \l 23BCF
  \l 23BD0
  \l 23BD1
  \l 23BD2
  \l 23BD3
  \l 23BD4
  \l 23BD5
  \l 23BD6
  \l 23BD7
  \l 23BD8
  \l 23BD9
  \l 23BDA
  \l 23BDB
  \l 23BDC
  \l 23BDD
  \l 23BDE
  \l 23BDF
  \l 23BE0
  \l 23BE1
  \l 23BE2
  \l 23BE3
  \l 23BE4
  \l 23BE5
  \l 23BE6
  \l 23BE7
  \l 23BE8
  \l 23BE9
  \l 23BEA
  \l 23BEB
  \l 23BEC
  \l 23BED
  \l 23BEE
  \l 23BEF
  \l 23BF0
  \l 23BF1
  \l 23BF2
  \l 23BF3
  \l 23BF4
  \l 23BF5
  \l 23BF6
  \l 23BF7
  \l 23BF8
  \l 23BF9
  \l 23BFA
  \l 23BFB
  \l 23BFC
  \l 23BFD
  \l 23BFE
  \l 23BFF
  \l 23C00
  \l 23C01
  \l 23C02
  \l 23C03
  \l 23C04
  \l 23C05
  \l 23C06
  \l 23C07
  \l 23C08
  \l 23C09
  \l 23C0A
  \l 23C0B
  \l 23C0C
  \l 23C0D
  \l 23C0E
  \l 23C0F
  \l 23C10
  \l 23C11
  \l 23C12
  \l 23C13
  \l 23C14
  \l 23C15
  \l 23C16
  \l 23C17
  \l 23C18
  \l 23C19
  \l 23C1A
  \l 23C1B
  \l 23C1C
  \l 23C1D
  \l 23C1E
  \l 23C1F
  \l 23C20
  \l 23C21
  \l 23C22
  \l 23C23
  \l 23C24
  \l 23C25
  \l 23C26
  \l 23C27
  \l 23C28
  \l 23C29
  \l 23C2A
  \l 23C2B
  \l 23C2C
  \l 23C2D
  \l 23C2E
  \l 23C2F
  \l 23C30
  \l 23C31
  \l 23C32
  \l 23C33
  \l 23C34
  \l 23C35
  \l 23C36
  \l 23C37
  \l 23C38
  \l 23C39
  \l 23C3A
  \l 23C3B
  \l 23C3C
  \l 23C3D
  \l 23C3E
  \l 23C3F
  \l 23C40
  \l 23C41
  \l 23C42
  \l 23C43
  \l 23C44
  \l 23C45
  \l 23C46
  \l 23C47
  \l 23C48
  \l 23C49
  \l 23C4A
  \l 23C4B
  \l 23C4C
  \l 23C4D
  \l 23C4E
  \l 23C4F
  \l 23C50
  \l 23C51
  \l 23C52
  \l 23C53
  \l 23C54
  \l 23C55
  \l 23C56
  \l 23C57
  \l 23C58
  \l 23C59
  \l 23C5A
  \l 23C5B
  \l 23C5C
  \l 23C5D
  \l 23C5E
  \l 23C5F
  \l 23C60
  \l 23C61
  \l 23C62
  \l 23C63
  \l 23C64
  \l 23C65
  \l 23C66
  \l 23C67
  \l 23C68
  \l 23C69
  \l 23C6A
  \l 23C6B
  \l 23C6C
  \l 23C6D
  \l 23C6E
  \l 23C6F
  \l 23C70
  \l 23C71
  \l 23C72
  \l 23C73
  \l 23C74
  \l 23C75
  \l 23C76
  \l 23C77
  \l 23C78
  \l 23C79
  \l 23C7A
  \l 23C7B
  \l 23C7C
  \l 23C7D
  \l 23C7E
  \l 23C7F
  \l 23C80
  \l 23C81
  \l 23C82
  \l 23C83
  \l 23C84
  \l 23C85
  \l 23C86
  \l 23C87
  \l 23C88
  \l 23C89
  \l 23C8A
  \l 23C8B
  \l 23C8C
  \l 23C8D
  \l 23C8E
  \l 23C8F
  \l 23C90
  \l 23C91
  \l 23C92
  \l 23C93
  \l 23C94
  \l 23C95
  \l 23C96
  \l 23C97
  \l 23C98
  \l 23C99
  \l 23C9A
  \l 23C9B
  \l 23C9C
  \l 23C9D
  \l 23C9E
  \l 23C9F
  \l 23CA0
  \l 23CA1
  \l 23CA2
  \l 23CA3
  \l 23CA4
  \l 23CA5
  \l 23CA6
  \l 23CA7
  \l 23CA8
  \l 23CA9
  \l 23CAA
  \l 23CAB
  \l 23CAC
  \l 23CAD
  \l 23CAE
  \l 23CAF
  \l 23CB0
  \l 23CB1
  \l 23CB2
  \l 23CB3
  \l 23CB4
  \l 23CB5
  \l 23CB6
  \l 23CB7
  \l 23CB8
  \l 23CB9
  \l 23CBA
  \l 23CBB
  \l 23CBC
  \l 23CBD
  \l 23CBE
  \l 23CBF
  \l 23CC0
  \l 23CC1
  \l 23CC2
  \l 23CC3
  \l 23CC4
  \l 23CC5
  \l 23CC6
  \l 23CC7
  \l 23CC8
  \l 23CC9
  \l 23CCA
  \l 23CCB
  \l 23CCC
  \l 23CCD
  \l 23CCE
  \l 23CCF
  \l 23CD0
  \l 23CD1
  \l 23CD2
  \l 23CD3
  \l 23CD4
  \l 23CD5
  \l 23CD6
  \l 23CD7
  \l 23CD8
  \l 23CD9
  \l 23CDA
  \l 23CDB
  \l 23CDC
  \l 23CDD
  \l 23CDE
  \l 23CDF
  \l 23CE0
  \l 23CE1
  \l 23CE2
  \l 23CE3
  \l 23CE4
  \l 23CE5
  \l 23CE6
  \l 23CE7
  \l 23CE8
  \l 23CE9
  \l 23CEA
  \l 23CEB
  \l 23CEC
  \l 23CED
  \l 23CEE
  \l 23CEF
  \l 23CF0
  \l 23CF1
  \l 23CF2
  \l 23CF3
  \l 23CF4
  \l 23CF5
  \l 23CF6
  \l 23CF7
  \l 23CF8
  \l 23CF9
  \l 23CFA
  \l 23CFB
  \l 23CFC
  \l 23CFD
  \l 23CFE
  \l 23CFF
  \l 23D00
  \l 23D01
  \l 23D02
  \l 23D03
  \l 23D04
  \l 23D05
  \l 23D06
  \l 23D07
  \l 23D08
  \l 23D09
  \l 23D0A
  \l 23D0B
  \l 23D0C
  \l 23D0D
  \l 23D0E
  \l 23D0F
  \l 23D10
  \l 23D11
  \l 23D12
  \l 23D13
  \l 23D14
  \l 23D15
  \l 23D16
  \l 23D17
  \l 23D18
  \l 23D19
  \l 23D1A
  \l 23D1B
  \l 23D1C
  \l 23D1D
  \l 23D1E
  \l 23D1F
  \l 23D20
  \l 23D21
  \l 23D22
  \l 23D23
  \l 23D24
  \l 23D25
  \l 23D26
  \l 23D27
  \l 23D28
  \l 23D29
  \l 23D2A
  \l 23D2B
  \l 23D2C
  \l 23D2D
  \l 23D2E
  \l 23D2F
  \l 23D30
  \l 23D31
  \l 23D32
  \l 23D33
  \l 23D34
  \l 23D35
  \l 23D36
  \l 23D37
  \l 23D38
  \l 23D39
  \l 23D3A
  \l 23D3B
  \l 23D3C
  \l 23D3D
  \l 23D3E
  \l 23D3F
  \l 23D40
  \l 23D41
  \l 23D42
  \l 23D43
  \l 23D44
  \l 23D45
  \l 23D46
  \l 23D47
  \l 23D48
  \l 23D49
  \l 23D4A
  \l 23D4B
  \l 23D4C
  \l 23D4D
  \l 23D4E
  \l 23D4F
  \l 23D50
  \l 23D51
  \l 23D52
  \l 23D53
  \l 23D54
  \l 23D55
  \l 23D56
  \l 23D57
  \l 23D58
  \l 23D59
  \l 23D5A
  \l 23D5B
  \l 23D5C
  \l 23D5D
  \l 23D5E
  \l 23D5F
  \l 23D60
  \l 23D61
  \l 23D62
  \l 23D63
  \l 23D64
  \l 23D65
  \l 23D66
  \l 23D67
  \l 23D68
  \l 23D69
  \l 23D6A
  \l 23D6B
  \l 23D6C
  \l 23D6D
  \l 23D6E
  \l 23D6F
  \l 23D70
  \l 23D71
  \l 23D72
  \l 23D73
  \l 23D74
  \l 23D75
  \l 23D76
  \l 23D77
  \l 23D78
  \l 23D79
  \l 23D7A
  \l 23D7B
  \l 23D7C
  \l 23D7D
  \l 23D7E
  \l 23D7F
  \l 23D80
  \l 23D81
  \l 23D82
  \l 23D83
  \l 23D84
  \l 23D85
  \l 23D86
  \l 23D87
  \l 23D88
  \l 23D89
  \l 23D8A
  \l 23D8B
  \l 23D8C
  \l 23D8D
  \l 23D8E
  \l 23D8F
  \l 23D90
  \l 23D91
  \l 23D92
  \l 23D93
  \l 23D94
  \l 23D95
  \l 23D96
  \l 23D97
  \l 23D98
  \l 23D99
  \l 23D9A
  \l 23D9B
  \l 23D9C
  \l 23D9D
  \l 23D9E
  \l 23D9F
  \l 23DA0
  \l 23DA1
  \l 23DA2
  \l 23DA3
  \l 23DA4
  \l 23DA5
  \l 23DA6
  \l 23DA7
  \l 23DA8
  \l 23DA9
  \l 23DAA
  \l 23DAB
  \l 23DAC
  \l 23DAD
  \l 23DAE
  \l 23DAF
  \l 23DB0
  \l 23DB1
  \l 23DB2
  \l 23DB3
  \l 23DB4
  \l 23DB5
  \l 23DB6
  \l 23DB7
  \l 23DB8
  \l 23DB9
  \l 23DBA
  \l 23DBB
  \l 23DBC
  \l 23DBD
  \l 23DBE
  \l 23DBF
  \l 23DC0
  \l 23DC1
  \l 23DC2
  \l 23DC3
  \l 23DC4
  \l 23DC5
  \l 23DC6
  \l 23DC7
  \l 23DC8
  \l 23DC9
  \l 23DCA
  \l 23DCB
  \l 23DCC
  \l 23DCD
  \l 23DCE
  \l 23DCF
  \l 23DD0
  \l 23DD1
  \l 23DD2
  \l 23DD3
  \l 23DD4
  \l 23DD5
  \l 23DD6
  \l 23DD7
  \l 23DD8
  \l 23DD9
  \l 23DDA
  \l 23DDB
  \l 23DDC
  \l 23DDD
  \l 23DDE
  \l 23DDF
  \l 23DE0
  \l 23DE1
  \l 23DE2
  \l 23DE3
  \l 23DE4
  \l 23DE5
  \l 23DE6
  \l 23DE7
  \l 23DE8
  \l 23DE9
  \l 23DEA
  \l 23DEB
  \l 23DEC
  \l 23DED
  \l 23DEE
  \l 23DEF
  \l 23DF0
  \l 23DF1
  \l 23DF2
  \l 23DF3
  \l 23DF4
  \l 23DF5
  \l 23DF6
  \l 23DF7
  \l 23DF8
  \l 23DF9
  \l 23DFA
  \l 23DFB
  \l 23DFC
  \l 23DFD
  \l 23DFE
  \l 23DFF
  \l 23E00
  \l 23E01
  \l 23E02
  \l 23E03
  \l 23E04
  \l 23E05
  \l 23E06
  \l 23E07
  \l 23E08
  \l 23E09
  \l 23E0A
  \l 23E0B
  \l 23E0C
  \l 23E0D
  \l 23E0E
  \l 23E0F
  \l 23E10
  \l 23E11
  \l 23E12
  \l 23E13
  \l 23E14
  \l 23E15
  \l 23E16
  \l 23E17
  \l 23E18
  \l 23E19
  \l 23E1A
  \l 23E1B
  \l 23E1C
  \l 23E1D
  \l 23E1E
  \l 23E1F
  \l 23E20
  \l 23E21
  \l 23E22
  \l 23E23
  \l 23E24
  \l 23E25
  \l 23E26
  \l 23E27
  \l 23E28
  \l 23E29
  \l 23E2A
  \l 23E2B
  \l 23E2C
  \l 23E2D
  \l 23E2E
  \l 23E2F
  \l 23E30
  \l 23E31
  \l 23E32
  \l 23E33
  \l 23E34
  \l 23E35
  \l 23E36
  \l 23E37
  \l 23E38
  \l 23E39
  \l 23E3A
  \l 23E3B
  \l 23E3C
  \l 23E3D
  \l 23E3E
  \l 23E3F
  \l 23E40
  \l 23E41
  \l 23E42
  \l 23E43
  \l 23E44
  \l 23E45
  \l 23E46
  \l 23E47
  \l 23E48
  \l 23E49
  \l 23E4A
  \l 23E4B
  \l 23E4C
  \l 23E4D
  \l 23E4E
  \l 23E4F
  \l 23E50
  \l 23E51
  \l 23E52
  \l 23E53
  \l 23E54
  \l 23E55
  \l 23E56
  \l 23E57
  \l 23E58
  \l 23E59
  \l 23E5A
  \l 23E5B
  \l 23E5C
  \l 23E5D
  \l 23E5E
  \l 23E5F
  \l 23E60
  \l 23E61
  \l 23E62
  \l 23E63
  \l 23E64
  \l 23E65
  \l 23E66
  \l 23E67
  \l 23E68
  \l 23E69
  \l 23E6A
  \l 23E6B
  \l 23E6C
  \l 23E6D
  \l 23E6E
  \l 23E6F
  \l 23E70
  \l 23E71
  \l 23E72
  \l 23E73
  \l 23E74
  \l 23E75
  \l 23E76
  \l 23E77
  \l 23E78
  \l 23E79
  \l 23E7A
  \l 23E7B
  \l 23E7C
  \l 23E7D
  \l 23E7E
  \l 23E7F
  \l 23E80
  \l 23E81
  \l 23E82
  \l 23E83
  \l 23E84
  \l 23E85
  \l 23E86
  \l 23E87
  \l 23E88
  \l 23E89
  \l 23E8A
  \l 23E8B
  \l 23E8C
  \l 23E8D
  \l 23E8E
  \l 23E8F
  \l 23E90
  \l 23E91
  \l 23E92
  \l 23E93
  \l 23E94
  \l 23E95
  \l 23E96
  \l 23E97
  \l 23E98
  \l 23E99
  \l 23E9A
  \l 23E9B
  \l 23E9C
  \l 23E9D
  \l 23E9E
  \l 23E9F
  \l 23EA0
  \l 23EA1
  \l 23EA2
  \l 23EA3
  \l 23EA4
  \l 23EA5
  \l 23EA6
  \l 23EA7
  \l 23EA8
  \l 23EA9
  \l 23EAA
  \l 23EAB
  \l 23EAC
  \l 23EAD
  \l 23EAE
  \l 23EAF
  \l 23EB0
  \l 23EB1
  \l 23EB2
  \l 23EB3
  \l 23EB4
  \l 23EB5
  \l 23EB6
  \l 23EB7
  \l 23EB8
  \l 23EB9
  \l 23EBA
  \l 23EBB
  \l 23EBC
  \l 23EBD
  \l 23EBE
  \l 23EBF
  \l 23EC0
  \l 23EC1
  \l 23EC2
  \l 23EC3
  \l 23EC4
  \l 23EC5
  \l 23EC6
  \l 23EC7
  \l 23EC8
  \l 23EC9
  \l 23ECA
  \l 23ECB
  \l 23ECC
  \l 23ECD
  \l 23ECE
  \l 23ECF
  \l 23ED0
  \l 23ED1
  \l 23ED2
  \l 23ED3
  \l 23ED4
  \l 23ED5
  \l 23ED6
  \l 23ED7
  \l 23ED8
  \l 23ED9
  \l 23EDA
  \l 23EDB
  \l 23EDC
  \l 23EDD
  \l 23EDE
  \l 23EDF
  \l 23EE0
  \l 23EE1
  \l 23EE2
  \l 23EE3
  \l 23EE4
  \l 23EE5
  \l 23EE6
  \l 23EE7
  \l 23EE8
  \l 23EE9
  \l 23EEA
  \l 23EEB
  \l 23EEC
  \l 23EED
  \l 23EEE
  \l 23EEF
  \l 23EF0
  \l 23EF1
  \l 23EF2
  \l 23EF3
  \l 23EF4
  \l 23EF5
  \l 23EF6
  \l 23EF7
  \l 23EF8
  \l 23EF9
  \l 23EFA
  \l 23EFB
  \l 23EFC
  \l 23EFD
  \l 23EFE
  \l 23EFF
  \l 23F00
  \l 23F01
  \l 23F02
  \l 23F03
  \l 23F04
  \l 23F05
  \l 23F06
  \l 23F07
  \l 23F08
  \l 23F09
  \l 23F0A
  \l 23F0B
  \l 23F0C
  \l 23F0D
  \l 23F0E
  \l 23F0F
  \l 23F10
  \l 23F11
  \l 23F12
  \l 23F13
  \l 23F14
  \l 23F15
  \l 23F16
  \l 23F17
  \l 23F18
  \l 23F19
  \l 23F1A
  \l 23F1B
  \l 23F1C
  \l 23F1D
  \l 23F1E
  \l 23F1F
  \l 23F20
  \l 23F21
  \l 23F22
  \l 23F23
  \l 23F24
  \l 23F25
  \l 23F26
  \l 23F27
  \l 23F28
  \l 23F29
  \l 23F2A
  \l 23F2B
  \l 23F2C
  \l 23F2D
  \l 23F2E
  \l 23F2F
  \l 23F30
  \l 23F31
  \l 23F32
  \l 23F33
  \l 23F34
  \l 23F35
  \l 23F36
  \l 23F37
  \l 23F38
  \l 23F39
  \l 23F3A
  \l 23F3B
  \l 23F3C
  \l 23F3D
  \l 23F3E
  \l 23F3F
  \l 23F40
  \l 23F41
  \l 23F42
  \l 23F43
  \l 23F44
  \l 23F45
  \l 23F46
  \l 23F47
  \l 23F48
  \l 23F49
  \l 23F4A
  \l 23F4B
  \l 23F4C
  \l 23F4D
  \l 23F4E
  \l 23F4F
  \l 23F50
  \l 23F51
  \l 23F52
  \l 23F53
  \l 23F54
  \l 23F55
  \l 23F56
  \l 23F57
  \l 23F58
  \l 23F59
  \l 23F5A
  \l 23F5B
  \l 23F5C
  \l 23F5D
  \l 23F5E
  \l 23F5F
  \l 23F60
  \l 23F61
  \l 23F62
  \l 23F63
  \l 23F64
  \l 23F65
  \l 23F66
  \l 23F67
  \l 23F68
  \l 23F69
  \l 23F6A
  \l 23F6B
  \l 23F6C
  \l 23F6D
  \l 23F6E
  \l 23F6F
  \l 23F70
  \l 23F71
  \l 23F72
  \l 23F73
  \l 23F74
  \l 23F75
  \l 23F76
  \l 23F77
  \l 23F78
  \l 23F79
  \l 23F7A
  \l 23F7B
  \l 23F7C
  \l 23F7D
  \l 23F7E
  \l 23F7F
  \l 23F80
  \l 23F81
  \l 23F82
  \l 23F83
  \l 23F84
  \l 23F85
  \l 23F86
  \l 23F87
  \l 23F88
  \l 23F89
  \l 23F8A
  \l 23F8B
  \l 23F8C
  \l 23F8D
  \l 23F8E
  \l 23F8F
  \l 23F90
  \l 23F91
  \l 23F92
  \l 23F93
  \l 23F94
  \l 23F95
  \l 23F96
  \l 23F97
  \l 23F98
  \l 23F99
  \l 23F9A
  \l 23F9B
  \l 23F9C
  \l 23F9D
  \l 23F9E
  \l 23F9F
  \l 23FA0
  \l 23FA1
  \l 23FA2
  \l 23FA3
  \l 23FA4
  \l 23FA5
  \l 23FA6
  \l 23FA7
  \l 23FA8
  \l 23FA9
  \l 23FAA
  \l 23FAB
  \l 23FAC
  \l 23FAD
  \l 23FAE
  \l 23FAF
  \l 23FB0
  \l 23FB1
  \l 23FB2
  \l 23FB3
  \l 23FB4
  \l 23FB5
  \l 23FB6
  \l 23FB7
  \l 23FB8
  \l 23FB9
  \l 23FBA
  \l 23FBB
  \l 23FBC
  \l 23FBD
  \l 23FBE
  \l 23FBF
  \l 23FC0
  \l 23FC1
  \l 23FC2
  \l 23FC3
  \l 23FC4
  \l 23FC5
  \l 23FC6
  \l 23FC7
  \l 23FC8
  \l 23FC9
  \l 23FCA
  \l 23FCB
  \l 23FCC
  \l 23FCD
  \l 23FCE
  \l 23FCF
  \l 23FD0
  \l 23FD1
  \l 23FD2
  \l 23FD3
  \l 23FD4
  \l 23FD5
  \l 23FD6
  \l 23FD7
  \l 23FD8
  \l 23FD9
  \l 23FDA
  \l 23FDB
  \l 23FDC
  \l 23FDD
  \l 23FDE
  \l 23FDF
  \l 23FE0
  \l 23FE1
  \l 23FE2
  \l 23FE3
  \l 23FE4
  \l 23FE5
  \l 23FE6
  \l 23FE7
  \l 23FE8
  \l 23FE9
  \l 23FEA
  \l 23FEB
  \l 23FEC
  \l 23FED
  \l 23FEE
  \l 23FEF
  \l 23FF0
  \l 23FF1
  \l 23FF2
  \l 23FF3
  \l 23FF4
  \l 23FF5
  \l 23FF6
  \l 23FF7
  \l 23FF8
  \l 23FF9
  \l 23FFA
  \l 23FFB
  \l 23FFC
  \l 23FFD
  \l 23FFE
  \l 23FFF
  \l 24000
  \l 24001
  \l 24002
  \l 24003
  \l 24004
  \l 24005
  \l 24006
  \l 24007
  \l 24008
  \l 24009
  \l 2400A
  \l 2400B
  \l 2400C
  \l 2400D
  \l 2400E
  \l 2400F
  \l 24010
  \l 24011
  \l 24012
  \l 24013
  \l 24014
  \l 24015
  \l 24016
  \l 24017
  \l 24018
  \l 24019
  \l 2401A
  \l 2401B
  \l 2401C
  \l 2401D
  \l 2401E
  \l 2401F
  \l 24020
  \l 24021
  \l 24022
  \l 24023
  \l 24024
  \l 24025
  \l 24026
  \l 24027
  \l 24028
  \l 24029
  \l 2402A
  \l 2402B
  \l 2402C
  \l 2402D
  \l 2402E
  \l 2402F
  \l 24030
  \l 24031
  \l 24032
  \l 24033
  \l 24034
  \l 24035
  \l 24036
  \l 24037
  \l 24038
  \l 24039
  \l 2403A
  \l 2403B
  \l 2403C
  \l 2403D
  \l 2403E
  \l 2403F
  \l 24040
  \l 24041
  \l 24042
  \l 24043
  \l 24044
  \l 24045
  \l 24046
  \l 24047
  \l 24048
  \l 24049
  \l 2404A
  \l 2404B
  \l 2404C
  \l 2404D
  \l 2404E
  \l 2404F
  \l 24050
  \l 24051
  \l 24052
  \l 24053
  \l 24054
  \l 24055
  \l 24056
  \l 24057
  \l 24058
  \l 24059
  \l 2405A
  \l 2405B
  \l 2405C
  \l 2405D
  \l 2405E
  \l 2405F
  \l 24060
  \l 24061
  \l 24062
  \l 24063
  \l 24064
  \l 24065
  \l 24066
  \l 24067
  \l 24068
  \l 24069
  \l 2406A
  \l 2406B
  \l 2406C
  \l 2406D
  \l 2406E
  \l 2406F
  \l 24070
  \l 24071
  \l 24072
  \l 24073
  \l 24074
  \l 24075
  \l 24076
  \l 24077
  \l 24078
  \l 24079
  \l 2407A
  \l 2407B
  \l 2407C
  \l 2407D
  \l 2407E
  \l 2407F
  \l 24080
  \l 24081
  \l 24082
  \l 24083
  \l 24084
  \l 24085
  \l 24086
  \l 24087
  \l 24088
  \l 24089
  \l 2408A
  \l 2408B
  \l 2408C
  \l 2408D
  \l 2408E
  \l 2408F
  \l 24090
  \l 24091
  \l 24092
  \l 24093
  \l 24094
  \l 24095
  \l 24096
  \l 24097
  \l 24098
  \l 24099
  \l 2409A
  \l 2409B
  \l 2409C
  \l 2409D
  \l 2409E
  \l 2409F
  \l 240A0
  \l 240A1
  \l 240A2
  \l 240A3
  \l 240A4
  \l 240A5
  \l 240A6
  \l 240A7
  \l 240A8
  \l 240A9
  \l 240AA
  \l 240AB
  \l 240AC
  \l 240AD
  \l 240AE
  \l 240AF
  \l 240B0
  \l 240B1
  \l 240B2
  \l 240B3
  \l 240B4
  \l 240B5
  \l 240B6
  \l 240B7
  \l 240B8
  \l 240B9
  \l 240BA
  \l 240BB
  \l 240BC
  \l 240BD
  \l 240BE
  \l 240BF
  \l 240C0
  \l 240C1
  \l 240C2
  \l 240C3
  \l 240C4
  \l 240C5
  \l 240C6
  \l 240C7
  \l 240C8
  \l 240C9
  \l 240CA
  \l 240CB
  \l 240CC
  \l 240CD
  \l 240CE
  \l 240CF
  \l 240D0
  \l 240D1
  \l 240D2
  \l 240D3
  \l 240D4
  \l 240D5
  \l 240D6
  \l 240D7
  \l 240D8
  \l 240D9
  \l 240DA
  \l 240DB
  \l 240DC
  \l 240DD
  \l 240DE
  \l 240DF
  \l 240E0
  \l 240E1
  \l 240E2
  \l 240E3
  \l 240E4
  \l 240E5
  \l 240E6
  \l 240E7
  \l 240E8
  \l 240E9
  \l 240EA
  \l 240EB
  \l 240EC
  \l 240ED
  \l 240EE
  \l 240EF
  \l 240F0
  \l 240F1
  \l 240F2
  \l 240F3
  \l 240F4
  \l 240F5
  \l 240F6
  \l 240F7
  \l 240F8
  \l 240F9
  \l 240FA
  \l 240FB
  \l 240FC
  \l 240FD
  \l 240FE
  \l 240FF
  \l 24100
  \l 24101
  \l 24102
  \l 24103
  \l 24104
  \l 24105
  \l 24106
  \l 24107
  \l 24108
  \l 24109
  \l 2410A
  \l 2410B
  \l 2410C
  \l 2410D
  \l 2410E
  \l 2410F
  \l 24110
  \l 24111
  \l 24112
  \l 24113
  \l 24114
  \l 24115
  \l 24116
  \l 24117
  \l 24118
  \l 24119
  \l 2411A
  \l 2411B
  \l 2411C
  \l 2411D
  \l 2411E
  \l 2411F
  \l 24120
  \l 24121
  \l 24122
  \l 24123
  \l 24124
  \l 24125
  \l 24126
  \l 24127
  \l 24128
  \l 24129
  \l 2412A
  \l 2412B
  \l 2412C
  \l 2412D
  \l 2412E
  \l 2412F
  \l 24130
  \l 24131
  \l 24132
  \l 24133
  \l 24134
  \l 24135
  \l 24136
  \l 24137
  \l 24138
  \l 24139
  \l 2413A
  \l 2413B
  \l 2413C
  \l 2413D
  \l 2413E
  \l 2413F
  \l 24140
  \l 24141
  \l 24142
  \l 24143
  \l 24144
  \l 24145
  \l 24146
  \l 24147
  \l 24148
  \l 24149
  \l 2414A
  \l 2414B
  \l 2414C
  \l 2414D
  \l 2414E
  \l 2414F
  \l 24150
  \l 24151
  \l 24152
  \l 24153
  \l 24154
  \l 24155
  \l 24156
  \l 24157
  \l 24158
  \l 24159
  \l 2415A
  \l 2415B
  \l 2415C
  \l 2415D
  \l 2415E
  \l 2415F
  \l 24160
  \l 24161
  \l 24162
  \l 24163
  \l 24164
  \l 24165
  \l 24166
  \l 24167
  \l 24168
  \l 24169
  \l 2416A
  \l 2416B
  \l 2416C
  \l 2416D
  \l 2416E
  \l 2416F
  \l 24170
  \l 24171
  \l 24172
  \l 24173
  \l 24174
  \l 24175
  \l 24176
  \l 24177
  \l 24178
  \l 24179
  \l 2417A
  \l 2417B
  \l 2417C
  \l 2417D
  \l 2417E
  \l 2417F
  \l 24180
  \l 24181
  \l 24182
  \l 24183
  \l 24184
  \l 24185
  \l 24186
  \l 24187
  \l 24188
  \l 24189
  \l 2418A
  \l 2418B
  \l 2418C
  \l 2418D
  \l 2418E
  \l 2418F
  \l 24190
  \l 24191
  \l 24192
  \l 24193
  \l 24194
  \l 24195
  \l 24196
  \l 24197
  \l 24198
  \l 24199
  \l 2419A
  \l 2419B
  \l 2419C
  \l 2419D
  \l 2419E
  \l 2419F
  \l 241A0
  \l 241A1
  \l 241A2
  \l 241A3
  \l 241A4
  \l 241A5
  \l 241A6
  \l 241A7
  \l 241A8
  \l 241A9
  \l 241AA
  \l 241AB
  \l 241AC
  \l 241AD
  \l 241AE
  \l 241AF
  \l 241B0
  \l 241B1
  \l 241B2
  \l 241B3
  \l 241B4
  \l 241B5
  \l 241B6
  \l 241B7
  \l 241B8
  \l 241B9
  \l 241BA
  \l 241BB
  \l 241BC
  \l 241BD
  \l 241BE
  \l 241BF
  \l 241C0
  \l 241C1
  \l 241C2
  \l 241C3
  \l 241C4
  \l 241C5
  \l 241C6
  \l 241C7
  \l 241C8
  \l 241C9
  \l 241CA
  \l 241CB
  \l 241CC
  \l 241CD
  \l 241CE
  \l 241CF
  \l 241D0
  \l 241D1
  \l 241D2
  \l 241D3
  \l 241D4
  \l 241D5
  \l 241D6
  \l 241D7
  \l 241D8
  \l 241D9
  \l 241DA
  \l 241DB
  \l 241DC
  \l 241DD
  \l 241DE
  \l 241DF
  \l 241E0
  \l 241E1
  \l 241E2
  \l 241E3
  \l 241E4
  \l 241E5
  \l 241E6
  \l 241E7
  \l 241E8
  \l 241E9
  \l 241EA
  \l 241EB
  \l 241EC
  \l 241ED
  \l 241EE
  \l 241EF
  \l 241F0
  \l 241F1
  \l 241F2
  \l 241F3
  \l 241F4
  \l 241F5
  \l 241F6
  \l 241F7
  \l 241F8
  \l 241F9
  \l 241FA
  \l 241FB
  \l 241FC
  \l 241FD
  \l 241FE
  \l 241FF
  \l 24200
  \l 24201
  \l 24202
  \l 24203
  \l 24204
  \l 24205
  \l 24206
  \l 24207
  \l 24208
  \l 24209
  \l 2420A
  \l 2420B
  \l 2420C
  \l 2420D
  \l 2420E
  \l 2420F
  \l 24210
  \l 24211
  \l 24212
  \l 24213
  \l 24214
  \l 24215
  \l 24216
  \l 24217
  \l 24218
  \l 24219
  \l 2421A
  \l 2421B
  \l 2421C
  \l 2421D
  \l 2421E
  \l 2421F
  \l 24220
  \l 24221
  \l 24222
  \l 24223
  \l 24224
  \l 24225
  \l 24226
  \l 24227
  \l 24228
  \l 24229
  \l 2422A
  \l 2422B
  \l 2422C
  \l 2422D
  \l 2422E
  \l 2422F
  \l 24230
  \l 24231
  \l 24232
  \l 24233
  \l 24234
  \l 24235
  \l 24236
  \l 24237
  \l 24238
  \l 24239
  \l 2423A
  \l 2423B
  \l 2423C
  \l 2423D
  \l 2423E
  \l 2423F
  \l 24240
  \l 24241
  \l 24242
  \l 24243
  \l 24244
  \l 24245
  \l 24246
  \l 24247
  \l 24248
  \l 24249
  \l 2424A
  \l 2424B
  \l 2424C
  \l 2424D
  \l 2424E
  \l 2424F
  \l 24250
  \l 24251
  \l 24252
  \l 24253
  \l 24254
  \l 24255
  \l 24256
  \l 24257
  \l 24258
  \l 24259
  \l 2425A
  \l 2425B
  \l 2425C
  \l 2425D
  \l 2425E
  \l 2425F
  \l 24260
  \l 24261
  \l 24262
  \l 24263
  \l 24264
  \l 24265
  \l 24266
  \l 24267
  \l 24268
  \l 24269
  \l 2426A
  \l 2426B
  \l 2426C
  \l 2426D
  \l 2426E
  \l 2426F
  \l 24270
  \l 24271
  \l 24272
  \l 24273
  \l 24274
  \l 24275
  \l 24276
  \l 24277
  \l 24278
  \l 24279
  \l 2427A
  \l 2427B
  \l 2427C
  \l 2427D
  \l 2427E
  \l 2427F
  \l 24280
  \l 24281
  \l 24282
  \l 24283
  \l 24284
  \l 24285
  \l 24286
  \l 24287
  \l 24288
  \l 24289
  \l 2428A
  \l 2428B
  \l 2428C
  \l 2428D
  \l 2428E
  \l 2428F
  \l 24290
  \l 24291
  \l 24292
  \l 24293
  \l 24294
  \l 24295
  \l 24296
  \l 24297
  \l 24298
  \l 24299
  \l 2429A
  \l 2429B
  \l 2429C
  \l 2429D
  \l 2429E
  \l 2429F
  \l 242A0
  \l 242A1
  \l 242A2
  \l 242A3
  \l 242A4
  \l 242A5
  \l 242A6
  \l 242A7
  \l 242A8
  \l 242A9
  \l 242AA
  \l 242AB
  \l 242AC
  \l 242AD
  \l 242AE
  \l 242AF
  \l 242B0
  \l 242B1
  \l 242B2
  \l 242B3
  \l 242B4
  \l 242B5
  \l 242B6
  \l 242B7
  \l 242B8
  \l 242B9
  \l 242BA
  \l 242BB
  \l 242BC
  \l 242BD
  \l 242BE
  \l 242BF
  \l 242C0
  \l 242C1
  \l 242C2
  \l 242C3
  \l 242C4
  \l 242C5
  \l 242C6
  \l 242C7
  \l 242C8
  \l 242C9
  \l 242CA
  \l 242CB
  \l 242CC
  \l 242CD
  \l 242CE
  \l 242CF
  \l 242D0
  \l 242D1
  \l 242D2
  \l 242D3
  \l 242D4
  \l 242D5
  \l 242D6
  \l 242D7
  \l 242D8
  \l 242D9
  \l 242DA
  \l 242DB
  \l 242DC
  \l 242DD
  \l 242DE
  \l 242DF
  \l 242E0
  \l 242E1
  \l 242E2
  \l 242E3
  \l 242E4
  \l 242E5
  \l 242E6
  \l 242E7
  \l 242E8
  \l 242E9
  \l 242EA
  \l 242EB
  \l 242EC
  \l 242ED
  \l 242EE
  \l 242EF
  \l 242F0
  \l 242F1
  \l 242F2
  \l 242F3
  \l 242F4
  \l 242F5
  \l 242F6
  \l 242F7
  \l 242F8
  \l 242F9
  \l 242FA
  \l 242FB
  \l 242FC
  \l 242FD
  \l 242FE
  \l 242FF
  \l 24300
  \l 24301
  \l 24302
  \l 24303
  \l 24304
  \l 24305
  \l 24306
  \l 24307
  \l 24308
  \l 24309
  \l 2430A
  \l 2430B
  \l 2430C
  \l 2430D
  \l 2430E
  \l 2430F
  \l 24310
  \l 24311
  \l 24312
  \l 24313
  \l 24314
  \l 24315
  \l 24316
  \l 24317
  \l 24318
  \l 24319
  \l 2431A
  \l 2431B
  \l 2431C
  \l 2431D
  \l 2431E
  \l 2431F
  \l 24320
  \l 24321
  \l 24322
  \l 24323
  \l 24324
  \l 24325
  \l 24326
  \l 24327
  \l 24328
  \l 24329
  \l 2432A
  \l 2432B
  \l 2432C
  \l 2432D
  \l 2432E
  \l 2432F
  \l 24330
  \l 24331
  \l 24332
  \l 24333
  \l 24334
  \l 24335
  \l 24336
  \l 24337
  \l 24338
  \l 24339
  \l 2433A
  \l 2433B
  \l 2433C
  \l 2433D
  \l 2433E
  \l 2433F
  \l 24340
  \l 24341
  \l 24342
  \l 24343
  \l 24344
  \l 24345
  \l 24346
  \l 24347
  \l 24348
  \l 24349
  \l 2434A
  \l 2434B
  \l 2434C
  \l 2434D
  \l 2434E
  \l 2434F
  \l 24350
  \l 24351
  \l 24352
  \l 24353
  \l 24354
  \l 24355
  \l 24356
  \l 24357
  \l 24358
  \l 24359
  \l 2435A
  \l 2435B
  \l 2435C
  \l 2435D
  \l 2435E
  \l 2435F
  \l 24360
  \l 24361
  \l 24362
  \l 24363
  \l 24364
  \l 24365
  \l 24366
  \l 24367
  \l 24368
  \l 24369
  \l 2436A
  \l 2436B
  \l 2436C
  \l 2436D
  \l 2436E
  \l 2436F
  \l 24370
  \l 24371
  \l 24372
  \l 24373
  \l 24374
  \l 24375
  \l 24376
  \l 24377
  \l 24378
  \l 24379
  \l 2437A
  \l 2437B
  \l 2437C
  \l 2437D
  \l 2437E
  \l 2437F
  \l 24380
  \l 24381
  \l 24382
  \l 24383
  \l 24384
  \l 24385
  \l 24386
  \l 24387
  \l 24388
  \l 24389
  \l 2438A
  \l 2438B
  \l 2438C
  \l 2438D
  \l 2438E
  \l 2438F
  \l 24390
  \l 24391
  \l 24392
  \l 24393
  \l 24394
  \l 24395
  \l 24396
  \l 24397
  \l 24398
  \l 24399
  \l 2439A
  \l 2439B
  \l 2439C
  \l 2439D
  \l 2439E
  \l 2439F
  \l 243A0
  \l 243A1
  \l 243A2
  \l 243A3
  \l 243A4
  \l 243A5
  \l 243A6
  \l 243A7
  \l 243A8
  \l 243A9
  \l 243AA
  \l 243AB
  \l 243AC
  \l 243AD
  \l 243AE
  \l 243AF
  \l 243B0
  \l 243B1
  \l 243B2
  \l 243B3
  \l 243B4
  \l 243B5
  \l 243B6
  \l 243B7
  \l 243B8
  \l 243B9
  \l 243BA
  \l 243BB
  \l 243BC
  \l 243BD
  \l 243BE
  \l 243BF
  \l 243C0
  \l 243C1
  \l 243C2
  \l 243C3
  \l 243C4
  \l 243C5
  \l 243C6
  \l 243C7
  \l 243C8
  \l 243C9
  \l 243CA
  \l 243CB
  \l 243CC
  \l 243CD
  \l 243CE
  \l 243CF
  \l 243D0
  \l 243D1
  \l 243D2
  \l 243D3
  \l 243D4
  \l 243D5
  \l 243D6
  \l 243D7
  \l 243D8
  \l 243D9
  \l 243DA
  \l 243DB
  \l 243DC
  \l 243DD
  \l 243DE
  \l 243DF
  \l 243E0
  \l 243E1
  \l 243E2
  \l 243E3
  \l 243E4
  \l 243E5
  \l 243E6
  \l 243E7
  \l 243E8
  \l 243E9
  \l 243EA
  \l 243EB
  \l 243EC
  \l 243ED
  \l 243EE
  \l 243EF
  \l 243F0
  \l 243F1
  \l 243F2
  \l 243F3
  \l 243F4
  \l 243F5
  \l 243F6
  \l 243F7
  \l 243F8
  \l 243F9
  \l 243FA
  \l 243FB
  \l 243FC
  \l 243FD
  \l 243FE
  \l 243FF
  \l 24400
  \l 24401
  \l 24402
  \l 24403
  \l 24404
  \l 24405
  \l 24406
  \l 24407
  \l 24408
  \l 24409
  \l 2440A
  \l 2440B
  \l 2440C
  \l 2440D
  \l 2440E
  \l 2440F
  \l 24410
  \l 24411
  \l 24412
  \l 24413
  \l 24414
  \l 24415
  \l 24416
  \l 24417
  \l 24418
  \l 24419
  \l 2441A
  \l 2441B
  \l 2441C
  \l 2441D
  \l 2441E
  \l 2441F
  \l 24420
  \l 24421
  \l 24422
  \l 24423
  \l 24424
  \l 24425
  \l 24426
  \l 24427
  \l 24428
  \l 24429
  \l 2442A
  \l 2442B
  \l 2442C
  \l 2442D
  \l 2442E
  \l 2442F
  \l 24430
  \l 24431
  \l 24432
  \l 24433
  \l 24434
  \l 24435
  \l 24436
  \l 24437
  \l 24438
  \l 24439
  \l 2443A
  \l 2443B
  \l 2443C
  \l 2443D
  \l 2443E
  \l 2443F
  \l 24440
  \l 24441
  \l 24442
  \l 24443
  \l 24444
  \l 24445
  \l 24446
  \l 24447
  \l 24448
  \l 24449
  \l 2444A
  \l 2444B
  \l 2444C
  \l 2444D
  \l 2444E
  \l 2444F
  \l 24450
  \l 24451
  \l 24452
  \l 24453
  \l 24454
  \l 24455
  \l 24456
  \l 24457
  \l 24458
  \l 24459
  \l 2445A
  \l 2445B
  \l 2445C
  \l 2445D
  \l 2445E
  \l 2445F
  \l 24460
  \l 24461
  \l 24462
  \l 24463
  \l 24464
  \l 24465
  \l 24466
  \l 24467
  \l 24468
  \l 24469
  \l 2446A
  \l 2446B
  \l 2446C
  \l 2446D
  \l 2446E
  \l 2446F
  \l 24470
  \l 24471
  \l 24472
  \l 24473
  \l 24474
  \l 24475
  \l 24476
  \l 24477
  \l 24478
  \l 24479
  \l 2447A
  \l 2447B
  \l 2447C
  \l 2447D
  \l 2447E
  \l 2447F
  \l 24480
  \l 24481
  \l 24482
  \l 24483
  \l 24484
  \l 24485
  \l 24486
  \l 24487
  \l 24488
  \l 24489
  \l 2448A
  \l 2448B
  \l 2448C
  \l 2448D
  \l 2448E
  \l 2448F
  \l 24490
  \l 24491
  \l 24492
  \l 24493
  \l 24494
  \l 24495
  \l 24496
  \l 24497
  \l 24498
  \l 24499
  \l 2449A
  \l 2449B
  \l 2449C
  \l 2449D
  \l 2449E
  \l 2449F
  \l 244A0
  \l 244A1
  \l 244A2
  \l 244A3
  \l 244A4
  \l 244A5
  \l 244A6
  \l 244A7
  \l 244A8
  \l 244A9
  \l 244AA
  \l 244AB
  \l 244AC
  \l 244AD
  \l 244AE
  \l 244AF
  \l 244B0
  \l 244B1
  \l 244B2
  \l 244B3
  \l 244B4
  \l 244B5
  \l 244B6
  \l 244B7
  \l 244B8
  \l 244B9
  \l 244BA
  \l 244BB
  \l 244BC
  \l 244BD
  \l 244BE
  \l 244BF
  \l 244C0
  \l 244C1
  \l 244C2
  \l 244C3
  \l 244C4
  \l 244C5
  \l 244C6
  \l 244C7
  \l 244C8
  \l 244C9
  \l 244CA
  \l 244CB
  \l 244CC
  \l 244CD
  \l 244CE
  \l 244CF
  \l 244D0
  \l 244D1
  \l 244D2
  \l 244D3
  \l 244D4
  \l 244D5
  \l 244D6
  \l 244D7
  \l 244D8
  \l 244D9
  \l 244DA
  \l 244DB
  \l 244DC
  \l 244DD
  \l 244DE
  \l 244DF
  \l 244E0
  \l 244E1
  \l 244E2
  \l 244E3
  \l 244E4
  \l 244E5
  \l 244E6
  \l 244E7
  \l 244E8
  \l 244E9
  \l 244EA
  \l 244EB
  \l 244EC
  \l 244ED
  \l 244EE
  \l 244EF
  \l 244F0
  \l 244F1
  \l 244F2
  \l 244F3
  \l 244F4
  \l 244F5
  \l 244F6
  \l 244F7
  \l 244F8
  \l 244F9
  \l 244FA
  \l 244FB
  \l 244FC
  \l 244FD
  \l 244FE
  \l 244FF
  \l 24500
  \l 24501
  \l 24502
  \l 24503
  \l 24504
  \l 24505
  \l 24506
  \l 24507
  \l 24508
  \l 24509
  \l 2450A
  \l 2450B
  \l 2450C
  \l 2450D
  \l 2450E
  \l 2450F
  \l 24510
  \l 24511
  \l 24512
  \l 24513
  \l 24514
  \l 24515
  \l 24516
  \l 24517
  \l 24518
  \l 24519
  \l 2451A
  \l 2451B
  \l 2451C
  \l 2451D
  \l 2451E
  \l 2451F
  \l 24520
  \l 24521
  \l 24522
  \l 24523
  \l 24524
  \l 24525
  \l 24526
  \l 24527
  \l 24528
  \l 24529
  \l 2452A
  \l 2452B
  \l 2452C
  \l 2452D
  \l 2452E
  \l 2452F
  \l 24530
  \l 24531
  \l 24532
  \l 24533
  \l 24534
  \l 24535
  \l 24536
  \l 24537
  \l 24538
  \l 24539
  \l 2453A
  \l 2453B
  \l 2453C
  \l 2453D
  \l 2453E
  \l 2453F
  \l 24540
  \l 24541
  \l 24542
  \l 24543
  \l 24544
  \l 24545
  \l 24546
  \l 24547
  \l 24548
  \l 24549
  \l 2454A
  \l 2454B
  \l 2454C
  \l 2454D
  \l 2454E
  \l 2454F
  \l 24550
  \l 24551
  \l 24552
  \l 24553
  \l 24554
  \l 24555
  \l 24556
  \l 24557
  \l 24558
  \l 24559
  \l 2455A
  \l 2455B
  \l 2455C
  \l 2455D
  \l 2455E
  \l 2455F
  \l 24560
  \l 24561
  \l 24562
  \l 24563
  \l 24564
  \l 24565
  \l 24566
  \l 24567
  \l 24568
  \l 24569
  \l 2456A
  \l 2456B
  \l 2456C
  \l 2456D
  \l 2456E
  \l 2456F
  \l 24570
  \l 24571
  \l 24572
  \l 24573
  \l 24574
  \l 24575
  \l 24576
  \l 24577
  \l 24578
  \l 24579
  \l 2457A
  \l 2457B
  \l 2457C
  \l 2457D
  \l 2457E
  \l 2457F
  \l 24580
  \l 24581
  \l 24582
  \l 24583
  \l 24584
  \l 24585
  \l 24586
  \l 24587
  \l 24588
  \l 24589
  \l 2458A
  \l 2458B
  \l 2458C
  \l 2458D
  \l 2458E
  \l 2458F
  \l 24590
  \l 24591
  \l 24592
  \l 24593
  \l 24594
  \l 24595
  \l 24596
  \l 24597
  \l 24598
  \l 24599
  \l 2459A
  \l 2459B
  \l 2459C
  \l 2459D
  \l 2459E
  \l 2459F
  \l 245A0
  \l 245A1
  \l 245A2
  \l 245A3
  \l 245A4
  \l 245A5
  \l 245A6
  \l 245A7
  \l 245A8
  \l 245A9
  \l 245AA
  \l 245AB
  \l 245AC
  \l 245AD
  \l 245AE
  \l 245AF
  \l 245B0
  \l 245B1
  \l 245B2
  \l 245B3
  \l 245B4
  \l 245B5
  \l 245B6
  \l 245B7
  \l 245B8
  \l 245B9
  \l 245BA
  \l 245BB
  \l 245BC
  \l 245BD
  \l 245BE
  \l 245BF
  \l 245C0
  \l 245C1
  \l 245C2
  \l 245C3
  \l 245C4
  \l 245C5
  \l 245C6
  \l 245C7
  \l 245C8
  \l 245C9
  \l 245CA
  \l 245CB
  \l 245CC
  \l 245CD
  \l 245CE
  \l 245CF
  \l 245D0
  \l 245D1
  \l 245D2
  \l 245D3
  \l 245D4
  \l 245D5
  \l 245D6
  \l 245D7
  \l 245D8
  \l 245D9
  \l 245DA
  \l 245DB
  \l 245DC
  \l 245DD
  \l 245DE
  \l 245DF
  \l 245E0
  \l 245E1
  \l 245E2
  \l 245E3
  \l 245E4
  \l 245E5
  \l 245E6
  \l 245E7
  \l 245E8
  \l 245E9
  \l 245EA
  \l 245EB
  \l 245EC
  \l 245ED
  \l 245EE
  \l 245EF
  \l 245F0
  \l 245F1
  \l 245F2
  \l 245F3
  \l 245F4
  \l 245F5
  \l 245F6
  \l 245F7
  \l 245F8
  \l 245F9
  \l 245FA
  \l 245FB
  \l 245FC
  \l 245FD
  \l 245FE
  \l 245FF
  \l 24600
  \l 24601
  \l 24602
  \l 24603
  \l 24604
  \l 24605
  \l 24606
  \l 24607
  \l 24608
  \l 24609
  \l 2460A
  \l 2460B
  \l 2460C
  \l 2460D
  \l 2460E
  \l 2460F
  \l 24610
  \l 24611
  \l 24612
  \l 24613
  \l 24614
  \l 24615
  \l 24616
  \l 24617
  \l 24618
  \l 24619
  \l 2461A
  \l 2461B
  \l 2461C
  \l 2461D
  \l 2461E
  \l 2461F
  \l 24620
  \l 24621
  \l 24622
  \l 24623
  \l 24624
  \l 24625
  \l 24626
  \l 24627
  \l 24628
  \l 24629
  \l 2462A
  \l 2462B
  \l 2462C
  \l 2462D
  \l 2462E
  \l 2462F
  \l 24630
  \l 24631
  \l 24632
  \l 24633
  \l 24634
  \l 24635
  \l 24636
  \l 24637
  \l 24638
  \l 24639
  \l 2463A
  \l 2463B
  \l 2463C
  \l 2463D
  \l 2463E
  \l 2463F
  \l 24640
  \l 24641
  \l 24642
  \l 24643
  \l 24644
  \l 24645
  \l 24646
  \l 24647
  \l 24648
  \l 24649
  \l 2464A
  \l 2464B
  \l 2464C
  \l 2464D
  \l 2464E
  \l 2464F
  \l 24650
  \l 24651
  \l 24652
  \l 24653
  \l 24654
  \l 24655
  \l 24656
  \l 24657
  \l 24658
  \l 24659
  \l 2465A
  \l 2465B
  \l 2465C
  \l 2465D
  \l 2465E
  \l 2465F
  \l 24660
  \l 24661
  \l 24662
  \l 24663
  \l 24664
  \l 24665
  \l 24666
  \l 24667
  \l 24668
  \l 24669
  \l 2466A
  \l 2466B
  \l 2466C
  \l 2466D
  \l 2466E
  \l 2466F
  \l 24670
  \l 24671
  \l 24672
  \l 24673
  \l 24674
  \l 24675
  \l 24676
  \l 24677
  \l 24678
  \l 24679
  \l 2467A
  \l 2467B
  \l 2467C
  \l 2467D
  \l 2467E
  \l 2467F
  \l 24680
  \l 24681
  \l 24682
  \l 24683
  \l 24684
  \l 24685
  \l 24686
  \l 24687
  \l 24688
  \l 24689
  \l 2468A
  \l 2468B
  \l 2468C
  \l 2468D
  \l 2468E
  \l 2468F
  \l 24690
  \l 24691
  \l 24692
  \l 24693
  \l 24694
  \l 24695
  \l 24696
  \l 24697
  \l 24698
  \l 24699
  \l 2469A
  \l 2469B
  \l 2469C
  \l 2469D
  \l 2469E
  \l 2469F
  \l 246A0
  \l 246A1
  \l 246A2
  \l 246A3
  \l 246A4
  \l 246A5
  \l 246A6
  \l 246A7
  \l 246A8
  \l 246A9
  \l 246AA
  \l 246AB
  \l 246AC
  \l 246AD
  \l 246AE
  \l 246AF
  \l 246B0
  \l 246B1
  \l 246B2
  \l 246B3
  \l 246B4
  \l 246B5
  \l 246B6
  \l 246B7
  \l 246B8
  \l 246B9
  \l 246BA
  \l 246BB
  \l 246BC
  \l 246BD
  \l 246BE
  \l 246BF
  \l 246C0
  \l 246C1
  \l 246C2
  \l 246C3
  \l 246C4
  \l 246C5
  \l 246C6
  \l 246C7
  \l 246C8
  \l 246C9
  \l 246CA
  \l 246CB
  \l 246CC
  \l 246CD
  \l 246CE
  \l 246CF
  \l 246D0
  \l 246D1
  \l 246D2
  \l 246D3
  \l 246D4
  \l 246D5
  \l 246D6
  \l 246D7
  \l 246D8
  \l 246D9
  \l 246DA
  \l 246DB
  \l 246DC
  \l 246DD
  \l 246DE
  \l 246DF
  \l 246E0
  \l 246E1
  \l 246E2
  \l 246E3
  \l 246E4
  \l 246E5
  \l 246E6
  \l 246E7
  \l 246E8
  \l 246E9
  \l 246EA
  \l 246EB
  \l 246EC
  \l 246ED
  \l 246EE
  \l 246EF
  \l 246F0
  \l 246F1
  \l 246F2
  \l 246F3
  \l 246F4
  \l 246F5
  \l 246F6
  \l 246F7
  \l 246F8
  \l 246F9
  \l 246FA
  \l 246FB
  \l 246FC
  \l 246FD
  \l 246FE
  \l 246FF
  \l 24700
  \l 24701
  \l 24702
  \l 24703
  \l 24704
  \l 24705
  \l 24706
  \l 24707
  \l 24708
  \l 24709
  \l 2470A
  \l 2470B
  \l 2470C
  \l 2470D
  \l 2470E
  \l 2470F
  \l 24710
  \l 24711
  \l 24712
  \l 24713
  \l 24714
  \l 24715
  \l 24716
  \l 24717
  \l 24718
  \l 24719
  \l 2471A
  \l 2471B
  \l 2471C
  \l 2471D
  \l 2471E
  \l 2471F
  \l 24720
  \l 24721
  \l 24722
  \l 24723
  \l 24724
  \l 24725
  \l 24726
  \l 24727
  \l 24728
  \l 24729
  \l 2472A
  \l 2472B
  \l 2472C
  \l 2472D
  \l 2472E
  \l 2472F
  \l 24730
  \l 24731
  \l 24732
  \l 24733
  \l 24734
  \l 24735
  \l 24736
  \l 24737
  \l 24738
  \l 24739
  \l 2473A
  \l 2473B
  \l 2473C
  \l 2473D
  \l 2473E
  \l 2473F
  \l 24740
  \l 24741
  \l 24742
  \l 24743
  \l 24744
  \l 24745
  \l 24746
  \l 24747
  \l 24748
  \l 24749
  \l 2474A
  \l 2474B
  \l 2474C
  \l 2474D
  \l 2474E
  \l 2474F
  \l 24750
  \l 24751
  \l 24752
  \l 24753
  \l 24754
  \l 24755
  \l 24756
  \l 24757
  \l 24758
  \l 24759
  \l 2475A
  \l 2475B
  \l 2475C
  \l 2475D
  \l 2475E
  \l 2475F
  \l 24760
  \l 24761
  \l 24762
  \l 24763
  \l 24764
  \l 24765
  \l 24766
  \l 24767
  \l 24768
  \l 24769
  \l 2476A
  \l 2476B
  \l 2476C
  \l 2476D
  \l 2476E
  \l 2476F
  \l 24770
  \l 24771
  \l 24772
  \l 24773
  \l 24774
  \l 24775
  \l 24776
  \l 24777
  \l 24778
  \l 24779
  \l 2477A
  \l 2477B
  \l 2477C
  \l 2477D
  \l 2477E
  \l 2477F
  \l 24780
  \l 24781
  \l 24782
  \l 24783
  \l 24784
  \l 24785
  \l 24786
  \l 24787
  \l 24788
  \l 24789
  \l 2478A
  \l 2478B
  \l 2478C
  \l 2478D
  \l 2478E
  \l 2478F
  \l 24790
  \l 24791
  \l 24792
  \l 24793
  \l 24794
  \l 24795
  \l 24796
  \l 24797
  \l 24798
  \l 24799
  \l 2479A
  \l 2479B
  \l 2479C
  \l 2479D
  \l 2479E
  \l 2479F
  \l 247A0
  \l 247A1
  \l 247A2
  \l 247A3
  \l 247A4
  \l 247A5
  \l 247A6
  \l 247A7
  \l 247A8
  \l 247A9
  \l 247AA
  \l 247AB
  \l 247AC
  \l 247AD
  \l 247AE
  \l 247AF
  \l 247B0
  \l 247B1
  \l 247B2
  \l 247B3
  \l 247B4
  \l 247B5
  \l 247B6
  \l 247B7
  \l 247B8
  \l 247B9
  \l 247BA
  \l 247BB
  \l 247BC
  \l 247BD
  \l 247BE
  \l 247BF
  \l 247C0
  \l 247C1
  \l 247C2
  \l 247C3
  \l 247C4
  \l 247C5
  \l 247C6
  \l 247C7
  \l 247C8
  \l 247C9
  \l 247CA
  \l 247CB
  \l 247CC
  \l 247CD
  \l 247CE
  \l 247CF
  \l 247D0
  \l 247D1
  \l 247D2
  \l 247D3
  \l 247D4
  \l 247D5
  \l 247D6
  \l 247D7
  \l 247D8
  \l 247D9
  \l 247DA
  \l 247DB
  \l 247DC
  \l 247DD
  \l 247DE
  \l 247DF
  \l 247E0
  \l 247E1
  \l 247E2
  \l 247E3
  \l 247E4
  \l 247E5
  \l 247E6
  \l 247E7
  \l 247E8
  \l 247E9
  \l 247EA
  \l 247EB
  \l 247EC
  \l 247ED
  \l 247EE
  \l 247EF
  \l 247F0
  \l 247F1
  \l 247F2
  \l 247F3
  \l 247F4
  \l 247F5
  \l 247F6
  \l 247F7
  \l 247F8
  \l 247F9
  \l 247FA
  \l 247FB
  \l 247FC
  \l 247FD
  \l 247FE
  \l 247FF
  \l 24800
  \l 24801
  \l 24802
  \l 24803
  \l 24804
  \l 24805
  \l 24806
  \l 24807
  \l 24808
  \l 24809
  \l 2480A
  \l 2480B
  \l 2480C
  \l 2480D
  \l 2480E
  \l 2480F
  \l 24810
  \l 24811
  \l 24812
  \l 24813
  \l 24814
  \l 24815
  \l 24816
  \l 24817
  \l 24818
  \l 24819
  \l 2481A
  \l 2481B
  \l 2481C
  \l 2481D
  \l 2481E
  \l 2481F
  \l 24820
  \l 24821
  \l 24822
  \l 24823
  \l 24824
  \l 24825
  \l 24826
  \l 24827
  \l 24828
  \l 24829
  \l 2482A
  \l 2482B
  \l 2482C
  \l 2482D
  \l 2482E
  \l 2482F
  \l 24830
  \l 24831
  \l 24832
  \l 24833
  \l 24834
  \l 24835
  \l 24836
  \l 24837
  \l 24838
  \l 24839
  \l 2483A
  \l 2483B
  \l 2483C
  \l 2483D
  \l 2483E
  \l 2483F
  \l 24840
  \l 24841
  \l 24842
  \l 24843
  \l 24844
  \l 24845
  \l 24846
  \l 24847
  \l 24848
  \l 24849
  \l 2484A
  \l 2484B
  \l 2484C
  \l 2484D
  \l 2484E
  \l 2484F
  \l 24850
  \l 24851
  \l 24852
  \l 24853
  \l 24854
  \l 24855
  \l 24856
  \l 24857
  \l 24858
  \l 24859
  \l 2485A
  \l 2485B
  \l 2485C
  \l 2485D
  \l 2485E
  \l 2485F
  \l 24860
  \l 24861
  \l 24862
  \l 24863
  \l 24864
  \l 24865
  \l 24866
  \l 24867
  \l 24868
  \l 24869
  \l 2486A
  \l 2486B
  \l 2486C
  \l 2486D
  \l 2486E
  \l 2486F
  \l 24870
  \l 24871
  \l 24872
  \l 24873
  \l 24874
  \l 24875
  \l 24876
  \l 24877
  \l 24878
  \l 24879
  \l 2487A
  \l 2487B
  \l 2487C
  \l 2487D
  \l 2487E
  \l 2487F
  \l 24880
  \l 24881
  \l 24882
  \l 24883
  \l 24884
  \l 24885
  \l 24886
  \l 24887
  \l 24888
  \l 24889
  \l 2488A
  \l 2488B
  \l 2488C
  \l 2488D
  \l 2488E
  \l 2488F
  \l 24890
  \l 24891
  \l 24892
  \l 24893
  \l 24894
  \l 24895
  \l 24896
  \l 24897
  \l 24898
  \l 24899
  \l 2489A
  \l 2489B
  \l 2489C
  \l 2489D
  \l 2489E
  \l 2489F
  \l 248A0
  \l 248A1
  \l 248A2
  \l 248A3
  \l 248A4
  \l 248A5
  \l 248A6
  \l 248A7
  \l 248A8
  \l 248A9
  \l 248AA
  \l 248AB
  \l 248AC
  \l 248AD
  \l 248AE
  \l 248AF
  \l 248B0
  \l 248B1
  \l 248B2
  \l 248B3
  \l 248B4
  \l 248B5
  \l 248B6
  \l 248B7
  \l 248B8
  \l 248B9
  \l 248BA
  \l 248BB
  \l 248BC
  \l 248BD
  \l 248BE
  \l 248BF
  \l 248C0
  \l 248C1
  \l 248C2
  \l 248C3
  \l 248C4
  \l 248C5
  \l 248C6
  \l 248C7
  \l 248C8
  \l 248C9
  \l 248CA
  \l 248CB
  \l 248CC
  \l 248CD
  \l 248CE
  \l 248CF
  \l 248D0
  \l 248D1
  \l 248D2
  \l 248D3
  \l 248D4
  \l 248D5
  \l 248D6
  \l 248D7
  \l 248D8
  \l 248D9
  \l 248DA
  \l 248DB
  \l 248DC
  \l 248DD
  \l 248DE
  \l 248DF
  \l 248E0
  \l 248E1
  \l 248E2
  \l 248E3
  \l 248E4
  \l 248E5
  \l 248E6
  \l 248E7
  \l 248E8
  \l 248E9
  \l 248EA
  \l 248EB
  \l 248EC
  \l 248ED
  \l 248EE
  \l 248EF
  \l 248F0
  \l 248F1
  \l 248F2
  \l 248F3
  \l 248F4
  \l 248F5
  \l 248F6
  \l 248F7
  \l 248F8
  \l 248F9
  \l 248FA
  \l 248FB
  \l 248FC
  \l 248FD
  \l 248FE
  \l 248FF
  \l 24900
  \l 24901
  \l 24902
  \l 24903
  \l 24904
  \l 24905
  \l 24906
  \l 24907
  \l 24908
  \l 24909
  \l 2490A
  \l 2490B
  \l 2490C
  \l 2490D
  \l 2490E
  \l 2490F
  \l 24910
  \l 24911
  \l 24912
  \l 24913
  \l 24914
  \l 24915
  \l 24916
  \l 24917
  \l 24918
  \l 24919
  \l 2491A
  \l 2491B
  \l 2491C
  \l 2491D
  \l 2491E
  \l 2491F
  \l 24920
  \l 24921
  \l 24922
  \l 24923
  \l 24924
  \l 24925
  \l 24926
  \l 24927
  \l 24928
  \l 24929
  \l 2492A
  \l 2492B
  \l 2492C
  \l 2492D
  \l 2492E
  \l 2492F
  \l 24930
  \l 24931
  \l 24932
  \l 24933
  \l 24934
  \l 24935
  \l 24936
  \l 24937
  \l 24938
  \l 24939
  \l 2493A
  \l 2493B
  \l 2493C
  \l 2493D
  \l 2493E
  \l 2493F
  \l 24940
  \l 24941
  \l 24942
  \l 24943
  \l 24944
  \l 24945
  \l 24946
  \l 24947
  \l 24948
  \l 24949
  \l 2494A
  \l 2494B
  \l 2494C
  \l 2494D
  \l 2494E
  \l 2494F
  \l 24950
  \l 24951
  \l 24952
  \l 24953
  \l 24954
  \l 24955
  \l 24956
  \l 24957
  \l 24958
  \l 24959
  \l 2495A
  \l 2495B
  \l 2495C
  \l 2495D
  \l 2495E
  \l 2495F
  \l 24960
  \l 24961
  \l 24962
  \l 24963
  \l 24964
  \l 24965
  \l 24966
  \l 24967
  \l 24968
  \l 24969
  \l 2496A
  \l 2496B
  \l 2496C
  \l 2496D
  \l 2496E
  \l 2496F
  \l 24970
  \l 24971
  \l 24972
  \l 24973
  \l 24974
  \l 24975
  \l 24976
  \l 24977
  \l 24978
  \l 24979
  \l 2497A
  \l 2497B
  \l 2497C
  \l 2497D
  \l 2497E
  \l 2497F
  \l 24980
  \l 24981
  \l 24982
  \l 24983
  \l 24984
  \l 24985
  \l 24986
  \l 24987
  \l 24988
  \l 24989
  \l 2498A
  \l 2498B
  \l 2498C
  \l 2498D
  \l 2498E
  \l 2498F
  \l 24990
  \l 24991
  \l 24992
  \l 24993
  \l 24994
  \l 24995
  \l 24996
  \l 24997
  \l 24998
  \l 24999
  \l 2499A
  \l 2499B
  \l 2499C
  \l 2499D
  \l 2499E
  \l 2499F
  \l 249A0
  \l 249A1
  \l 249A2
  \l 249A3
  \l 249A4
  \l 249A5
  \l 249A6
  \l 249A7
  \l 249A8
  \l 249A9
  \l 249AA
  \l 249AB
  \l 249AC
  \l 249AD
  \l 249AE
  \l 249AF
  \l 249B0
  \l 249B1
  \l 249B2
  \l 249B3
  \l 249B4
  \l 249B5
  \l 249B6
  \l 249B7
  \l 249B8
  \l 249B9
  \l 249BA
  \l 249BB
  \l 249BC
  \l 249BD
  \l 249BE
  \l 249BF
  \l 249C0
  \l 249C1
  \l 249C2
  \l 249C3
  \l 249C4
  \l 249C5
  \l 249C6
  \l 249C7
  \l 249C8
  \l 249C9
  \l 249CA
  \l 249CB
  \l 249CC
  \l 249CD
  \l 249CE
  \l 249CF
  \l 249D0
  \l 249D1
  \l 249D2
  \l 249D3
  \l 249D4
  \l 249D5
  \l 249D6
  \l 249D7
  \l 249D8
  \l 249D9
  \l 249DA
  \l 249DB
  \l 249DC
  \l 249DD
  \l 249DE
  \l 249DF
  \l 249E0
  \l 249E1
  \l 249E2
  \l 249E3
  \l 249E4
  \l 249E5
  \l 249E6
  \l 249E7
  \l 249E8
  \l 249E9
  \l 249EA
  \l 249EB
  \l 249EC
  \l 249ED
  \l 249EE
  \l 249EF
  \l 249F0
  \l 249F1
  \l 249F2
  \l 249F3
  \l 249F4
  \l 249F5
  \l 249F6
  \l 249F7
  \l 249F8
  \l 249F9
  \l 249FA
  \l 249FB
  \l 249FC
  \l 249FD
  \l 249FE
  \l 249FF
  \l 24A00
  \l 24A01
  \l 24A02
  \l 24A03
  \l 24A04
  \l 24A05
  \l 24A06
  \l 24A07
  \l 24A08
  \l 24A09
  \l 24A0A
  \l 24A0B
  \l 24A0C
  \l 24A0D
  \l 24A0E
  \l 24A0F
  \l 24A10
  \l 24A11
  \l 24A12
  \l 24A13
  \l 24A14
  \l 24A15
  \l 24A16
  \l 24A17
  \l 24A18
  \l 24A19
  \l 24A1A
  \l 24A1B
  \l 24A1C
  \l 24A1D
  \l 24A1E
  \l 24A1F
  \l 24A20
  \l 24A21
  \l 24A22
  \l 24A23
  \l 24A24
  \l 24A25
  \l 24A26
  \l 24A27
  \l 24A28
  \l 24A29
  \l 24A2A
  \l 24A2B
  \l 24A2C
  \l 24A2D
  \l 24A2E
  \l 24A2F
  \l 24A30
  \l 24A31
  \l 24A32
  \l 24A33
  \l 24A34
  \l 24A35
  \l 24A36
  \l 24A37
  \l 24A38
  \l 24A39
  \l 24A3A
  \l 24A3B
  \l 24A3C
  \l 24A3D
  \l 24A3E
  \l 24A3F
  \l 24A40
  \l 24A41
  \l 24A42
  \l 24A43
  \l 24A44
  \l 24A45
  \l 24A46
  \l 24A47
  \l 24A48
  \l 24A49
  \l 24A4A
  \l 24A4B
  \l 24A4C
  \l 24A4D
  \l 24A4E
  \l 24A4F
  \l 24A50
  \l 24A51
  \l 24A52
  \l 24A53
  \l 24A54
  \l 24A55
  \l 24A56
  \l 24A57
  \l 24A58
  \l 24A59
  \l 24A5A
  \l 24A5B
  \l 24A5C
  \l 24A5D
  \l 24A5E
  \l 24A5F
  \l 24A60
  \l 24A61
  \l 24A62
  \l 24A63
  \l 24A64
  \l 24A65
  \l 24A66
  \l 24A67
  \l 24A68
  \l 24A69
  \l 24A6A
  \l 24A6B
  \l 24A6C
  \l 24A6D
  \l 24A6E
  \l 24A6F
  \l 24A70
  \l 24A71
  \l 24A72
  \l 24A73
  \l 24A74
  \l 24A75
  \l 24A76
  \l 24A77
  \l 24A78
  \l 24A79
  \l 24A7A
  \l 24A7B
  \l 24A7C
  \l 24A7D
  \l 24A7E
  \l 24A7F
  \l 24A80
  \l 24A81
  \l 24A82
  \l 24A83
  \l 24A84
  \l 24A85
  \l 24A86
  \l 24A87
  \l 24A88
  \l 24A89
  \l 24A8A
  \l 24A8B
  \l 24A8C
  \l 24A8D
  \l 24A8E
  \l 24A8F
  \l 24A90
  \l 24A91
  \l 24A92
  \l 24A93
  \l 24A94
  \l 24A95
  \l 24A96
  \l 24A97
  \l 24A98
  \l 24A99
  \l 24A9A
  \l 24A9B
  \l 24A9C
  \l 24A9D
  \l 24A9E
  \l 24A9F
  \l 24AA0
  \l 24AA1
  \l 24AA2
  \l 24AA3
  \l 24AA4
  \l 24AA5
  \l 24AA6
  \l 24AA7
  \l 24AA8
  \l 24AA9
  \l 24AAA
  \l 24AAB
  \l 24AAC
  \l 24AAD
  \l 24AAE
  \l 24AAF
  \l 24AB0
  \l 24AB1
  \l 24AB2
  \l 24AB3
  \l 24AB4
  \l 24AB5
  \l 24AB6
  \l 24AB7
  \l 24AB8
  \l 24AB9
  \l 24ABA
  \l 24ABB
  \l 24ABC
  \l 24ABD
  \l 24ABE
  \l 24ABF
  \l 24AC0
  \l 24AC1
  \l 24AC2
  \l 24AC3
  \l 24AC4
  \l 24AC5
  \l 24AC6
  \l 24AC7
  \l 24AC8
  \l 24AC9
  \l 24ACA
  \l 24ACB
  \l 24ACC
  \l 24ACD
  \l 24ACE
  \l 24ACF
  \l 24AD0
  \l 24AD1
  \l 24AD2
  \l 24AD3
  \l 24AD4
  \l 24AD5
  \l 24AD6
  \l 24AD7
  \l 24AD8
  \l 24AD9
  \l 24ADA
  \l 24ADB
  \l 24ADC
  \l 24ADD
  \l 24ADE
  \l 24ADF
  \l 24AE0
  \l 24AE1
  \l 24AE2
  \l 24AE3
  \l 24AE4
  \l 24AE5
  \l 24AE6
  \l 24AE7
  \l 24AE8
  \l 24AE9
  \l 24AEA
  \l 24AEB
  \l 24AEC
  \l 24AED
  \l 24AEE
  \l 24AEF
  \l 24AF0
  \l 24AF1
  \l 24AF2
  \l 24AF3
  \l 24AF4
  \l 24AF5
  \l 24AF6
  \l 24AF7
  \l 24AF8
  \l 24AF9
  \l 24AFA
  \l 24AFB
  \l 24AFC
  \l 24AFD
  \l 24AFE
  \l 24AFF
  \l 24B00
  \l 24B01
  \l 24B02
  \l 24B03
  \l 24B04
  \l 24B05
  \l 24B06
  \l 24B07
  \l 24B08
  \l 24B09
  \l 24B0A
  \l 24B0B
  \l 24B0C
  \l 24B0D
  \l 24B0E
  \l 24B0F
  \l 24B10
  \l 24B11
  \l 24B12
  \l 24B13
  \l 24B14
  \l 24B15
  \l 24B16
  \l 24B17
  \l 24B18
  \l 24B19
  \l 24B1A
  \l 24B1B
  \l 24B1C
  \l 24B1D
  \l 24B1E
  \l 24B1F
  \l 24B20
  \l 24B21
  \l 24B22
  \l 24B23
  \l 24B24
  \l 24B25
  \l 24B26
  \l 24B27
  \l 24B28
  \l 24B29
  \l 24B2A
  \l 24B2B
  \l 24B2C
  \l 24B2D
  \l 24B2E
  \l 24B2F
  \l 24B30
  \l 24B31
  \l 24B32
  \l 24B33
  \l 24B34
  \l 24B35
  \l 24B36
  \l 24B37
  \l 24B38
  \l 24B39
  \l 24B3A
  \l 24B3B
  \l 24B3C
  \l 24B3D
  \l 24B3E
  \l 24B3F
  \l 24B40
  \l 24B41
  \l 24B42
  \l 24B43
  \l 24B44
  \l 24B45
  \l 24B46
  \l 24B47
  \l 24B48
  \l 24B49
  \l 24B4A
  \l 24B4B
  \l 24B4C
  \l 24B4D
  \l 24B4E
  \l 24B4F
  \l 24B50
  \l 24B51
  \l 24B52
  \l 24B53
  \l 24B54
  \l 24B55
  \l 24B56
  \l 24B57
  \l 24B58
  \l 24B59
  \l 24B5A
  \l 24B5B
  \l 24B5C
  \l 24B5D
  \l 24B5E
  \l 24B5F
  \l 24B60
  \l 24B61
  \l 24B62
  \l 24B63
  \l 24B64
  \l 24B65
  \l 24B66
  \l 24B67
  \l 24B68
  \l 24B69
  \l 24B6A
  \l 24B6B
  \l 24B6C
  \l 24B6D
  \l 24B6E
  \l 24B6F
  \l 24B70
  \l 24B71
  \l 24B72
  \l 24B73
  \l 24B74
  \l 24B75
  \l 24B76
  \l 24B77
  \l 24B78
  \l 24B79
  \l 24B7A
  \l 24B7B
  \l 24B7C
  \l 24B7D
  \l 24B7E
  \l 24B7F
  \l 24B80
  \l 24B81
  \l 24B82
  \l 24B83
  \l 24B84
  \l 24B85
  \l 24B86
  \l 24B87
  \l 24B88
  \l 24B89
  \l 24B8A
  \l 24B8B
  \l 24B8C
  \l 24B8D
  \l 24B8E
  \l 24B8F
  \l 24B90
  \l 24B91
  \l 24B92
  \l 24B93
  \l 24B94
  \l 24B95
  \l 24B96
  \l 24B97
  \l 24B98
  \l 24B99
  \l 24B9A
  \l 24B9B
  \l 24B9C
  \l 24B9D
  \l 24B9E
  \l 24B9F
  \l 24BA0
  \l 24BA1
  \l 24BA2
  \l 24BA3
  \l 24BA4
  \l 24BA5
  \l 24BA6
  \l 24BA7
  \l 24BA8
  \l 24BA9
  \l 24BAA
  \l 24BAB
  \l 24BAC
  \l 24BAD
  \l 24BAE
  \l 24BAF
  \l 24BB0
  \l 24BB1
  \l 24BB2
  \l 24BB3
  \l 24BB4
  \l 24BB5
  \l 24BB6
  \l 24BB7
  \l 24BB8
  \l 24BB9
  \l 24BBA
  \l 24BBB
  \l 24BBC
  \l 24BBD
  \l 24BBE
  \l 24BBF
  \l 24BC0
  \l 24BC1
  \l 24BC2
  \l 24BC3
  \l 24BC4
  \l 24BC5
  \l 24BC6
  \l 24BC7
  \l 24BC8
  \l 24BC9
  \l 24BCA
  \l 24BCB
  \l 24BCC
  \l 24BCD
  \l 24BCE
  \l 24BCF
  \l 24BD0
  \l 24BD1
  \l 24BD2
  \l 24BD3
  \l 24BD4
  \l 24BD5
  \l 24BD6
  \l 24BD7
  \l 24BD8
  \l 24BD9
  \l 24BDA
  \l 24BDB
  \l 24BDC
  \l 24BDD
  \l 24BDE
  \l 24BDF
  \l 24BE0
  \l 24BE1
  \l 24BE2
  \l 24BE3
  \l 24BE4
  \l 24BE5
  \l 24BE6
  \l 24BE7
  \l 24BE8
  \l 24BE9
  \l 24BEA
  \l 24BEB
  \l 24BEC
  \l 24BED
  \l 24BEE
  \l 24BEF
  \l 24BF0
  \l 24BF1
  \l 24BF2
  \l 24BF3
  \l 24BF4
  \l 24BF5
  \l 24BF6
  \l 24BF7
  \l 24BF8
  \l 24BF9
  \l 24BFA
  \l 24BFB
  \l 24BFC
  \l 24BFD
  \l 24BFE
  \l 24BFF
  \l 24C00
  \l 24C01
  \l 24C02
  \l 24C03
  \l 24C04
  \l 24C05
  \l 24C06
  \l 24C07
  \l 24C08
  \l 24C09
  \l 24C0A
  \l 24C0B
  \l 24C0C
  \l 24C0D
  \l 24C0E
  \l 24C0F
  \l 24C10
  \l 24C11
  \l 24C12
  \l 24C13
  \l 24C14
  \l 24C15
  \l 24C16
  \l 24C17
  \l 24C18
  \l 24C19
  \l 24C1A
  \l 24C1B
  \l 24C1C
  \l 24C1D
  \l 24C1E
  \l 24C1F
  \l 24C20
  \l 24C21
  \l 24C22
  \l 24C23
  \l 24C24
  \l 24C25
  \l 24C26
  \l 24C27
  \l 24C28
  \l 24C29
  \l 24C2A
  \l 24C2B
  \l 24C2C
  \l 24C2D
  \l 24C2E
  \l 24C2F
  \l 24C30
  \l 24C31
  \l 24C32
  \l 24C33
  \l 24C34
  \l 24C35
  \l 24C36
  \l 24C37
  \l 24C38
  \l 24C39
  \l 24C3A
  \l 24C3B
  \l 24C3C
  \l 24C3D
  \l 24C3E
  \l 24C3F
  \l 24C40
  \l 24C41
  \l 24C42
  \l 24C43
  \l 24C44
  \l 24C45
  \l 24C46
  \l 24C47
  \l 24C48
  \l 24C49
  \l 24C4A
  \l 24C4B
  \l 24C4C
  \l 24C4D
  \l 24C4E
  \l 24C4F
  \l 24C50
  \l 24C51
  \l 24C52
  \l 24C53
  \l 24C54
  \l 24C55
  \l 24C56
  \l 24C57
  \l 24C58
  \l 24C59
  \l 24C5A
  \l 24C5B
  \l 24C5C
  \l 24C5D
  \l 24C5E
  \l 24C5F
  \l 24C60
  \l 24C61
  \l 24C62
  \l 24C63
  \l 24C64
  \l 24C65
  \l 24C66
  \l 24C67
  \l 24C68
  \l 24C69
  \l 24C6A
  \l 24C6B
  \l 24C6C
  \l 24C6D
  \l 24C6E
  \l 24C6F
  \l 24C70
  \l 24C71
  \l 24C72
  \l 24C73
  \l 24C74
  \l 24C75
  \l 24C76
  \l 24C77
  \l 24C78
  \l 24C79
  \l 24C7A
  \l 24C7B
  \l 24C7C
  \l 24C7D
  \l 24C7E
  \l 24C7F
  \l 24C80
  \l 24C81
  \l 24C82
  \l 24C83
  \l 24C84
  \l 24C85
  \l 24C86
  \l 24C87
  \l 24C88
  \l 24C89
  \l 24C8A
  \l 24C8B
  \l 24C8C
  \l 24C8D
  \l 24C8E
  \l 24C8F
  \l 24C90
  \l 24C91
  \l 24C92
  \l 24C93
  \l 24C94
  \l 24C95
  \l 24C96
  \l 24C97
  \l 24C98
  \l 24C99
  \l 24C9A
  \l 24C9B
  \l 24C9C
  \l 24C9D
  \l 24C9E
  \l 24C9F
  \l 24CA0
  \l 24CA1
  \l 24CA2
  \l 24CA3
  \l 24CA4
  \l 24CA5
  \l 24CA6
  \l 24CA7
  \l 24CA8
  \l 24CA9
  \l 24CAA
  \l 24CAB
  \l 24CAC
  \l 24CAD
  \l 24CAE
  \l 24CAF
  \l 24CB0
  \l 24CB1
  \l 24CB2
  \l 24CB3
  \l 24CB4
  \l 24CB5
  \l 24CB6
  \l 24CB7
  \l 24CB8
  \l 24CB9
  \l 24CBA
  \l 24CBB
  \l 24CBC
  \l 24CBD
  \l 24CBE
  \l 24CBF
  \l 24CC0
  \l 24CC1
  \l 24CC2
  \l 24CC3
  \l 24CC4
  \l 24CC5
  \l 24CC6
  \l 24CC7
  \l 24CC8
  \l 24CC9
  \l 24CCA
  \l 24CCB
  \l 24CCC
  \l 24CCD
  \l 24CCE
  \l 24CCF
  \l 24CD0
  \l 24CD1
  \l 24CD2
  \l 24CD3
  \l 24CD4
  \l 24CD5
  \l 24CD6
  \l 24CD7
  \l 24CD8
  \l 24CD9
  \l 24CDA
  \l 24CDB
  \l 24CDC
  \l 24CDD
  \l 24CDE
  \l 24CDF
  \l 24CE0
  \l 24CE1
  \l 24CE2
  \l 24CE3
  \l 24CE4
  \l 24CE5
  \l 24CE6
  \l 24CE7
  \l 24CE8
  \l 24CE9
  \l 24CEA
  \l 24CEB
  \l 24CEC
  \l 24CED
  \l 24CEE
  \l 24CEF
  \l 24CF0
  \l 24CF1
  \l 24CF2
  \l 24CF3
  \l 24CF4
  \l 24CF5
  \l 24CF6
  \l 24CF7
  \l 24CF8
  \l 24CF9
  \l 24CFA
  \l 24CFB
  \l 24CFC
  \l 24CFD
  \l 24CFE
  \l 24CFF
  \l 24D00
  \l 24D01
  \l 24D02
  \l 24D03
  \l 24D04
  \l 24D05
  \l 24D06
  \l 24D07
  \l 24D08
  \l 24D09
  \l 24D0A
  \l 24D0B
  \l 24D0C
  \l 24D0D
  \l 24D0E
  \l 24D0F
  \l 24D10
  \l 24D11
  \l 24D12
  \l 24D13
  \l 24D14
  \l 24D15
  \l 24D16
  \l 24D17
  \l 24D18
  \l 24D19
  \l 24D1A
  \l 24D1B
  \l 24D1C
  \l 24D1D
  \l 24D1E
  \l 24D1F
  \l 24D20
  \l 24D21
  \l 24D22
  \l 24D23
  \l 24D24
  \l 24D25
  \l 24D26
  \l 24D27
  \l 24D28
  \l 24D29
  \l 24D2A
  \l 24D2B
  \l 24D2C
  \l 24D2D
  \l 24D2E
  \l 24D2F
  \l 24D30
  \l 24D31
  \l 24D32
  \l 24D33
  \l 24D34
  \l 24D35
  \l 24D36
  \l 24D37
  \l 24D38
  \l 24D39
  \l 24D3A
  \l 24D3B
  \l 24D3C
  \l 24D3D
  \l 24D3E
  \l 24D3F
  \l 24D40
  \l 24D41
  \l 24D42
  \l 24D43
  \l 24D44
  \l 24D45
  \l 24D46
  \l 24D47
  \l 24D48
  \l 24D49
  \l 24D4A
  \l 24D4B
  \l 24D4C
  \l 24D4D
  \l 24D4E
  \l 24D4F
  \l 24D50
  \l 24D51
  \l 24D52
  \l 24D53
  \l 24D54
  \l 24D55
  \l 24D56
  \l 24D57
  \l 24D58
  \l 24D59
  \l 24D5A
  \l 24D5B
  \l 24D5C
  \l 24D5D
  \l 24D5E
  \l 24D5F
  \l 24D60
  \l 24D61
  \l 24D62
  \l 24D63
  \l 24D64
  \l 24D65
  \l 24D66
  \l 24D67
  \l 24D68
  \l 24D69
  \l 24D6A
  \l 24D6B
  \l 24D6C
  \l 24D6D
  \l 24D6E
  \l 24D6F
  \l 24D70
  \l 24D71
  \l 24D72
  \l 24D73
  \l 24D74
  \l 24D75
  \l 24D76
  \l 24D77
  \l 24D78
  \l 24D79
  \l 24D7A
  \l 24D7B
  \l 24D7C
  \l 24D7D
  \l 24D7E
  \l 24D7F
  \l 24D80
  \l 24D81
  \l 24D82
  \l 24D83
  \l 24D84
  \l 24D85
  \l 24D86
  \l 24D87
  \l 24D88
  \l 24D89
  \l 24D8A
  \l 24D8B
  \l 24D8C
  \l 24D8D
  \l 24D8E
  \l 24D8F
  \l 24D90
  \l 24D91
  \l 24D92
  \l 24D93
  \l 24D94
  \l 24D95
  \l 24D96
  \l 24D97
  \l 24D98
  \l 24D99
  \l 24D9A
  \l 24D9B
  \l 24D9C
  \l 24D9D
  \l 24D9E
  \l 24D9F
  \l 24DA0
  \l 24DA1
  \l 24DA2
  \l 24DA3
  \l 24DA4
  \l 24DA5
  \l 24DA6
  \l 24DA7
  \l 24DA8
  \l 24DA9
  \l 24DAA
  \l 24DAB
  \l 24DAC
  \l 24DAD
  \l 24DAE
  \l 24DAF
  \l 24DB0
  \l 24DB1
  \l 24DB2
  \l 24DB3
  \l 24DB4
  \l 24DB5
  \l 24DB6
  \l 24DB7
  \l 24DB8
  \l 24DB9
  \l 24DBA
  \l 24DBB
  \l 24DBC
  \l 24DBD
  \l 24DBE
  \l 24DBF
  \l 24DC0
  \l 24DC1
  \l 24DC2
  \l 24DC3
  \l 24DC4
  \l 24DC5
  \l 24DC6
  \l 24DC7
  \l 24DC8
  \l 24DC9
  \l 24DCA
  \l 24DCB
  \l 24DCC
  \l 24DCD
  \l 24DCE
  \l 24DCF
  \l 24DD0
  \l 24DD1
  \l 24DD2
  \l 24DD3
  \l 24DD4
  \l 24DD5
  \l 24DD6
  \l 24DD7
  \l 24DD8
  \l 24DD9
  \l 24DDA
  \l 24DDB
  \l 24DDC
  \l 24DDD
  \l 24DDE
  \l 24DDF
  \l 24DE0
  \l 24DE1
  \l 24DE2
  \l 24DE3
  \l 24DE4
  \l 24DE5
  \l 24DE6
  \l 24DE7
  \l 24DE8
  \l 24DE9
  \l 24DEA
  \l 24DEB
  \l 24DEC
  \l 24DED
  \l 24DEE
  \l 24DEF
  \l 24DF0
  \l 24DF1
  \l 24DF2
  \l 24DF3
  \l 24DF4
  \l 24DF5
  \l 24DF6
  \l 24DF7
  \l 24DF8
  \l 24DF9
  \l 24DFA
  \l 24DFB
  \l 24DFC
  \l 24DFD
  \l 24DFE
  \l 24DFF
  \l 24E00
  \l 24E01
  \l 24E02
  \l 24E03
  \l 24E04
  \l 24E05
  \l 24E06
  \l 24E07
  \l 24E08
  \l 24E09
  \l 24E0A
  \l 24E0B
  \l 24E0C
  \l 24E0D
  \l 24E0E
  \l 24E0F
  \l 24E10
  \l 24E11
  \l 24E12
  \l 24E13
  \l 24E14
  \l 24E15
  \l 24E16
  \l 24E17
  \l 24E18
  \l 24E19
  \l 24E1A
  \l 24E1B
  \l 24E1C
  \l 24E1D
  \l 24E1E
  \l 24E1F
  \l 24E20
  \l 24E21
  \l 24E22
  \l 24E23
  \l 24E24
  \l 24E25
  \l 24E26
  \l 24E27
  \l 24E28
  \l 24E29
  \l 24E2A
  \l 24E2B
  \l 24E2C
  \l 24E2D
  \l 24E2E
  \l 24E2F
  \l 24E30
  \l 24E31
  \l 24E32
  \l 24E33
  \l 24E34
  \l 24E35
  \l 24E36
  \l 24E37
  \l 24E38
  \l 24E39
  \l 24E3A
  \l 24E3B
  \l 24E3C
  \l 24E3D
  \l 24E3E
  \l 24E3F
  \l 24E40
  \l 24E41
  \l 24E42
  \l 24E43
  \l 24E44
  \l 24E45
  \l 24E46
  \l 24E47
  \l 24E48
  \l 24E49
  \l 24E4A
  \l 24E4B
  \l 24E4C
  \l 24E4D
  \l 24E4E
  \l 24E4F
  \l 24E50
  \l 24E51
  \l 24E52
  \l 24E53
  \l 24E54
  \l 24E55
  \l 24E56
  \l 24E57
  \l 24E58
  \l 24E59
  \l 24E5A
  \l 24E5B
  \l 24E5C
  \l 24E5D
  \l 24E5E
  \l 24E5F
  \l 24E60
  \l 24E61
  \l 24E62
  \l 24E63
  \l 24E64
  \l 24E65
  \l 24E66
  \l 24E67
  \l 24E68
  \l 24E69
  \l 24E6A
  \l 24E6B
  \l 24E6C
  \l 24E6D
  \l 24E6E
  \l 24E6F
  \l 24E70
  \l 24E71
  \l 24E72
  \l 24E73
  \l 24E74
  \l 24E75
  \l 24E76
  \l 24E77
  \l 24E78
  \l 24E79
  \l 24E7A
  \l 24E7B
  \l 24E7C
  \l 24E7D
  \l 24E7E
  \l 24E7F
  \l 24E80
  \l 24E81
  \l 24E82
  \l 24E83
  \l 24E84
  \l 24E85
  \l 24E86
  \l 24E87
  \l 24E88
  \l 24E89
  \l 24E8A
  \l 24E8B
  \l 24E8C
  \l 24E8D
  \l 24E8E
  \l 24E8F
  \l 24E90
  \l 24E91
  \l 24E92
  \l 24E93
  \l 24E94
  \l 24E95
  \l 24E96
  \l 24E97
  \l 24E98
  \l 24E99
  \l 24E9A
  \l 24E9B
  \l 24E9C
  \l 24E9D
  \l 24E9E
  \l 24E9F
  \l 24EA0
  \l 24EA1
  \l 24EA2
  \l 24EA3
  \l 24EA4
  \l 24EA5
  \l 24EA6
  \l 24EA7
  \l 24EA8
  \l 24EA9
  \l 24EAA
  \l 24EAB
  \l 24EAC
  \l 24EAD
  \l 24EAE
  \l 24EAF
  \l 24EB0
  \l 24EB1
  \l 24EB2
  \l 24EB3
  \l 24EB4
  \l 24EB5
  \l 24EB6
  \l 24EB7
  \l 24EB8
  \l 24EB9
  \l 24EBA
  \l 24EBB
  \l 24EBC
  \l 24EBD
  \l 24EBE
  \l 24EBF
  \l 24EC0
  \l 24EC1
  \l 24EC2
  \l 24EC3
  \l 24EC4
  \l 24EC5
  \l 24EC6
  \l 24EC7
  \l 24EC8
  \l 24EC9
  \l 24ECA
  \l 24ECB
  \l 24ECC
  \l 24ECD
  \l 24ECE
  \l 24ECF
  \l 24ED0
  \l 24ED1
  \l 24ED2
  \l 24ED3
  \l 24ED4
  \l 24ED5
  \l 24ED6
  \l 24ED7
  \l 24ED8
  \l 24ED9
  \l 24EDA
  \l 24EDB
  \l 24EDC
  \l 24EDD
  \l 24EDE
  \l 24EDF
  \l 24EE0
  \l 24EE1
  \l 24EE2
  \l 24EE3
  \l 24EE4
  \l 24EE5
  \l 24EE6
  \l 24EE7
  \l 24EE8
  \l 24EE9
  \l 24EEA
  \l 24EEB
  \l 24EEC
  \l 24EED
  \l 24EEE
  \l 24EEF
  \l 24EF0
  \l 24EF1
  \l 24EF2
  \l 24EF3
  \l 24EF4
  \l 24EF5
  \l 24EF6
  \l 24EF7
  \l 24EF8
  \l 24EF9
  \l 24EFA
  \l 24EFB
  \l 24EFC
  \l 24EFD
  \l 24EFE
  \l 24EFF
  \l 24F00
  \l 24F01
  \l 24F02
  \l 24F03
  \l 24F04
  \l 24F05
  \l 24F06
  \l 24F07
  \l 24F08
  \l 24F09
  \l 24F0A
  \l 24F0B
  \l 24F0C
  \l 24F0D
  \l 24F0E
  \l 24F0F
  \l 24F10
  \l 24F11
  \l 24F12
  \l 24F13
  \l 24F14
  \l 24F15
  \l 24F16
  \l 24F17
  \l 24F18
  \l 24F19
  \l 24F1A
  \l 24F1B
  \l 24F1C
  \l 24F1D
  \l 24F1E
  \l 24F1F
  \l 24F20
  \l 24F21
  \l 24F22
  \l 24F23
  \l 24F24
  \l 24F25
  \l 24F26
  \l 24F27
  \l 24F28
  \l 24F29
  \l 24F2A
  \l 24F2B
  \l 24F2C
  \l 24F2D
  \l 24F2E
  \l 24F2F
  \l 24F30
  \l 24F31
  \l 24F32
  \l 24F33
  \l 24F34
  \l 24F35
  \l 24F36
  \l 24F37
  \l 24F38
  \l 24F39
  \l 24F3A
  \l 24F3B
  \l 24F3C
  \l 24F3D
  \l 24F3E
  \l 24F3F
  \l 24F40
  \l 24F41
  \l 24F42
  \l 24F43
  \l 24F44
  \l 24F45
  \l 24F46
  \l 24F47
  \l 24F48
  \l 24F49
  \l 24F4A
  \l 24F4B
  \l 24F4C
  \l 24F4D
  \l 24F4E
  \l 24F4F
  \l 24F50
  \l 24F51
  \l 24F52
  \l 24F53
  \l 24F54
  \l 24F55
  \l 24F56
  \l 24F57
  \l 24F58
  \l 24F59
  \l 24F5A
  \l 24F5B
  \l 24F5C
  \l 24F5D
  \l 24F5E
  \l 24F5F
  \l 24F60
  \l 24F61
  \l 24F62
  \l 24F63
  \l 24F64
  \l 24F65
  \l 24F66
  \l 24F67
  \l 24F68
  \l 24F69
  \l 24F6A
  \l 24F6B
  \l 24F6C
  \l 24F6D
  \l 24F6E
  \l 24F6F
  \l 24F70
  \l 24F71
  \l 24F72
  \l 24F73
  \l 24F74
  \l 24F75
  \l 24F76
  \l 24F77
  \l 24F78
  \l 24F79
  \l 24F7A
  \l 24F7B
  \l 24F7C
  \l 24F7D
  \l 24F7E
  \l 24F7F
  \l 24F80
  \l 24F81
  \l 24F82
  \l 24F83
  \l 24F84
  \l 24F85
  \l 24F86
  \l 24F87
  \l 24F88
  \l 24F89
  \l 24F8A
  \l 24F8B
  \l 24F8C
  \l 24F8D
  \l 24F8E
  \l 24F8F
  \l 24F90
  \l 24F91
  \l 24F92
  \l 24F93
  \l 24F94
  \l 24F95
  \l 24F96
  \l 24F97
  \l 24F98
  \l 24F99
  \l 24F9A
  \l 24F9B
  \l 24F9C
  \l 24F9D
  \l 24F9E
  \l 24F9F
  \l 24FA0
  \l 24FA1
  \l 24FA2
  \l 24FA3
  \l 24FA4
  \l 24FA5
  \l 24FA6
  \l 24FA7
  \l 24FA8
  \l 24FA9
  \l 24FAA
  \l 24FAB
  \l 24FAC
  \l 24FAD
  \l 24FAE
  \l 24FAF
  \l 24FB0
  \l 24FB1
  \l 24FB2
  \l 24FB3
  \l 24FB4
  \l 24FB5
  \l 24FB6
  \l 24FB7
  \l 24FB8
  \l 24FB9
  \l 24FBA
  \l 24FBB
  \l 24FBC
  \l 24FBD
  \l 24FBE
  \l 24FBF
  \l 24FC0
  \l 24FC1
  \l 24FC2
  \l 24FC3
  \l 24FC4
  \l 24FC5
  \l 24FC6
  \l 24FC7
  \l 24FC8
  \l 24FC9
  \l 24FCA
  \l 24FCB
  \l 24FCC
  \l 24FCD
  \l 24FCE
  \l 24FCF
  \l 24FD0
  \l 24FD1
  \l 24FD2
  \l 24FD3
  \l 24FD4
  \l 24FD5
  \l 24FD6
  \l 24FD7
  \l 24FD8
  \l 24FD9
  \l 24FDA
  \l 24FDB
  \l 24FDC
  \l 24FDD
  \l 24FDE
  \l 24FDF
  \l 24FE0
  \l 24FE1
  \l 24FE2
  \l 24FE3
  \l 24FE4
  \l 24FE5
  \l 24FE6
  \l 24FE7
  \l 24FE8
  \l 24FE9
  \l 24FEA
  \l 24FEB
  \l 24FEC
  \l 24FED
  \l 24FEE
  \l 24FEF
  \l 24FF0
  \l 24FF1
  \l 24FF2
  \l 24FF3
  \l 24FF4
  \l 24FF5
  \l 24FF6
  \l 24FF7
  \l 24FF8
  \l 24FF9
  \l 24FFA
  \l 24FFB
  \l 24FFC
  \l 24FFD
  \l 24FFE
  \l 24FFF
  \l 25000
  \l 25001
  \l 25002
  \l 25003
  \l 25004
  \l 25005
  \l 25006
  \l 25007
  \l 25008
  \l 25009
  \l 2500A
  \l 2500B
  \l 2500C
  \l 2500D
  \l 2500E
  \l 2500F
  \l 25010
  \l 25011
  \l 25012
  \l 25013
  \l 25014
  \l 25015
  \l 25016
  \l 25017
  \l 25018
  \l 25019
  \l 2501A
  \l 2501B
  \l 2501C
  \l 2501D
  \l 2501E
  \l 2501F
  \l 25020
  \l 25021
  \l 25022
  \l 25023
  \l 25024
  \l 25025
  \l 25026
  \l 25027
  \l 25028
  \l 25029
  \l 2502A
  \l 2502B
  \l 2502C
  \l 2502D
  \l 2502E
  \l 2502F
  \l 25030
  \l 25031
  \l 25032
  \l 25033
  \l 25034
  \l 25035
  \l 25036
  \l 25037
  \l 25038
  \l 25039
  \l 2503A
  \l 2503B
  \l 2503C
  \l 2503D
  \l 2503E
  \l 2503F
  \l 25040
  \l 25041
  \l 25042
  \l 25043
  \l 25044
  \l 25045
  \l 25046
  \l 25047
  \l 25048
  \l 25049
  \l 2504A
  \l 2504B
  \l 2504C
  \l 2504D
  \l 2504E
  \l 2504F
  \l 25050
  \l 25051
  \l 25052
  \l 25053
  \l 25054
  \l 25055
  \l 25056
  \l 25057
  \l 25058
  \l 25059
  \l 2505A
  \l 2505B
  \l 2505C
  \l 2505D
  \l 2505E
  \l 2505F
  \l 25060
  \l 25061
  \l 25062
  \l 25063
  \l 25064
  \l 25065
  \l 25066
  \l 25067
  \l 25068
  \l 25069
  \l 2506A
  \l 2506B
  \l 2506C
  \l 2506D
  \l 2506E
  \l 2506F
  \l 25070
  \l 25071
  \l 25072
  \l 25073
  \l 25074
  \l 25075
  \l 25076
  \l 25077
  \l 25078
  \l 25079
  \l 2507A
  \l 2507B
  \l 2507C
  \l 2507D
  \l 2507E
  \l 2507F
  \l 25080
  \l 25081
  \l 25082
  \l 25083
  \l 25084
  \l 25085
  \l 25086
  \l 25087
  \l 25088
  \l 25089
  \l 2508A
  \l 2508B
  \l 2508C
  \l 2508D
  \l 2508E
  \l 2508F
  \l 25090
  \l 25091
  \l 25092
  \l 25093
  \l 25094
  \l 25095
  \l 25096
  \l 25097
  \l 25098
  \l 25099
  \l 2509A
  \l 2509B
  \l 2509C
  \l 2509D
  \l 2509E
  \l 2509F
  \l 250A0
  \l 250A1
  \l 250A2
  \l 250A3
  \l 250A4
  \l 250A5
  \l 250A6
  \l 250A7
  \l 250A8
  \l 250A9
  \l 250AA
  \l 250AB
  \l 250AC
  \l 250AD
  \l 250AE
  \l 250AF
  \l 250B0
  \l 250B1
  \l 250B2
  \l 250B3
  \l 250B4
  \l 250B5
  \l 250B6
  \l 250B7
  \l 250B8
  \l 250B9
  \l 250BA
  \l 250BB
  \l 250BC
  \l 250BD
  \l 250BE
  \l 250BF
  \l 250C0
  \l 250C1
  \l 250C2
  \l 250C3
  \l 250C4
  \l 250C5
  \l 250C6
  \l 250C7
  \l 250C8
  \l 250C9
  \l 250CA
  \l 250CB
  \l 250CC
  \l 250CD
  \l 250CE
  \l 250CF
  \l 250D0
  \l 250D1
  \l 250D2
  \l 250D3
  \l 250D4
  \l 250D5
  \l 250D6
  \l 250D7
  \l 250D8
  \l 250D9
  \l 250DA
  \l 250DB
  \l 250DC
  \l 250DD
  \l 250DE
  \l 250DF
  \l 250E0
  \l 250E1
  \l 250E2
  \l 250E3
  \l 250E4
  \l 250E5
  \l 250E6
  \l 250E7
  \l 250E8
  \l 250E9
  \l 250EA
  \l 250EB
  \l 250EC
  \l 250ED
  \l 250EE
  \l 250EF
  \l 250F0
  \l 250F1
  \l 250F2
  \l 250F3
  \l 250F4
  \l 250F5
  \l 250F6
  \l 250F7
  \l 250F8
  \l 250F9
  \l 250FA
  \l 250FB
  \l 250FC
  \l 250FD
  \l 250FE
  \l 250FF
  \l 25100
  \l 25101
  \l 25102
  \l 25103
  \l 25104
  \l 25105
  \l 25106
  \l 25107
  \l 25108
  \l 25109
  \l 2510A
  \l 2510B
  \l 2510C
  \l 2510D
  \l 2510E
  \l 2510F
  \l 25110
  \l 25111
  \l 25112
  \l 25113
  \l 25114
  \l 25115
  \l 25116
  \l 25117
  \l 25118
  \l 25119
  \l 2511A
  \l 2511B
  \l 2511C
  \l 2511D
  \l 2511E
  \l 2511F
  \l 25120
  \l 25121
  \l 25122
  \l 25123
  \l 25124
  \l 25125
  \l 25126
  \l 25127
  \l 25128
  \l 25129
  \l 2512A
  \l 2512B
  \l 2512C
  \l 2512D
  \l 2512E
  \l 2512F
  \l 25130
  \l 25131
  \l 25132
  \l 25133
  \l 25134
  \l 25135
  \l 25136
  \l 25137
  \l 25138
  \l 25139
  \l 2513A
  \l 2513B
  \l 2513C
  \l 2513D
  \l 2513E
  \l 2513F
  \l 25140
  \l 25141
  \l 25142
  \l 25143
  \l 25144
  \l 25145
  \l 25146
  \l 25147
  \l 25148
  \l 25149
  \l 2514A
  \l 2514B
  \l 2514C
  \l 2514D
  \l 2514E
  \l 2514F
  \l 25150
  \l 25151
  \l 25152
  \l 25153
  \l 25154
  \l 25155
  \l 25156
  \l 25157
  \l 25158
  \l 25159
  \l 2515A
  \l 2515B
  \l 2515C
  \l 2515D
  \l 2515E
  \l 2515F
  \l 25160
  \l 25161
  \l 25162
  \l 25163
  \l 25164
  \l 25165
  \l 25166
  \l 25167
  \l 25168
  \l 25169
  \l 2516A
  \l 2516B
  \l 2516C
  \l 2516D
  \l 2516E
  \l 2516F
  \l 25170
  \l 25171
  \l 25172
  \l 25173
  \l 25174
  \l 25175
  \l 25176
  \l 25177
  \l 25178
  \l 25179
  \l 2517A
  \l 2517B
  \l 2517C
  \l 2517D
  \l 2517E
  \l 2517F
  \l 25180
  \l 25181
  \l 25182
  \l 25183
  \l 25184
  \l 25185
  \l 25186
  \l 25187
  \l 25188
  \l 25189
  \l 2518A
  \l 2518B
  \l 2518C
  \l 2518D
  \l 2518E
  \l 2518F
  \l 25190
  \l 25191
  \l 25192
  \l 25193
  \l 25194
  \l 25195
  \l 25196
  \l 25197
  \l 25198
  \l 25199
  \l 2519A
  \l 2519B
  \l 2519C
  \l 2519D
  \l 2519E
  \l 2519F
  \l 251A0
  \l 251A1
  \l 251A2
  \l 251A3
  \l 251A4
  \l 251A5
  \l 251A6
  \l 251A7
  \l 251A8
  \l 251A9
  \l 251AA
  \l 251AB
  \l 251AC
  \l 251AD
  \l 251AE
  \l 251AF
  \l 251B0
  \l 251B1
  \l 251B2
  \l 251B3
  \l 251B4
  \l 251B5
  \l 251B6
  \l 251B7
  \l 251B8
  \l 251B9
  \l 251BA
  \l 251BB
  \l 251BC
  \l 251BD
  \l 251BE
  \l 251BF
  \l 251C0
  \l 251C1
  \l 251C2
  \l 251C3
  \l 251C4
  \l 251C5
  \l 251C6
  \l 251C7
  \l 251C8
  \l 251C9
  \l 251CA
  \l 251CB
  \l 251CC
  \l 251CD
  \l 251CE
  \l 251CF
  \l 251D0
  \l 251D1
  \l 251D2
  \l 251D3
  \l 251D4
  \l 251D5
  \l 251D6
  \l 251D7
  \l 251D8
  \l 251D9
  \l 251DA
  \l 251DB
  \l 251DC
  \l 251DD
  \l 251DE
  \l 251DF
  \l 251E0
  \l 251E1
  \l 251E2
  \l 251E3
  \l 251E4
  \l 251E5
  \l 251E6
  \l 251E7
  \l 251E8
  \l 251E9
  \l 251EA
  \l 251EB
  \l 251EC
  \l 251ED
  \l 251EE
  \l 251EF
  \l 251F0
  \l 251F1
  \l 251F2
  \l 251F3
  \l 251F4
  \l 251F5
  \l 251F6
  \l 251F7
  \l 251F8
  \l 251F9
  \l 251FA
  \l 251FB
  \l 251FC
  \l 251FD
  \l 251FE
  \l 251FF
  \l 25200
  \l 25201
  \l 25202
  \l 25203
  \l 25204
  \l 25205
  \l 25206
  \l 25207
  \l 25208
  \l 25209
  \l 2520A
  \l 2520B
  \l 2520C
  \l 2520D
  \l 2520E
  \l 2520F
  \l 25210
  \l 25211
  \l 25212
  \l 25213
  \l 25214
  \l 25215
  \l 25216
  \l 25217
  \l 25218
  \l 25219
  \l 2521A
  \l 2521B
  \l 2521C
  \l 2521D
  \l 2521E
  \l 2521F
  \l 25220
  \l 25221
  \l 25222
  \l 25223
  \l 25224
  \l 25225
  \l 25226
  \l 25227
  \l 25228
  \l 25229
  \l 2522A
  \l 2522B
  \l 2522C
  \l 2522D
  \l 2522E
  \l 2522F
  \l 25230
  \l 25231
  \l 25232
  \l 25233
  \l 25234
  \l 25235
  \l 25236
  \l 25237
  \l 25238
  \l 25239
  \l 2523A
  \l 2523B
  \l 2523C
  \l 2523D
  \l 2523E
  \l 2523F
  \l 25240
  \l 25241
  \l 25242
  \l 25243
  \l 25244
  \l 25245
  \l 25246
  \l 25247
  \l 25248
  \l 25249
  \l 2524A
  \l 2524B
  \l 2524C
  \l 2524D
  \l 2524E
  \l 2524F
  \l 25250
  \l 25251
  \l 25252
  \l 25253
  \l 25254
  \l 25255
  \l 25256
  \l 25257
  \l 25258
  \l 25259
  \l 2525A
  \l 2525B
  \l 2525C
  \l 2525D
  \l 2525E
  \l 2525F
  \l 25260
  \l 25261
  \l 25262
  \l 25263
  \l 25264
  \l 25265
  \l 25266
  \l 25267
  \l 25268
  \l 25269
  \l 2526A
  \l 2526B
  \l 2526C
  \l 2526D
  \l 2526E
  \l 2526F
  \l 25270
  \l 25271
  \l 25272
  \l 25273
  \l 25274
  \l 25275
  \l 25276
  \l 25277
  \l 25278
  \l 25279
  \l 2527A
  \l 2527B
  \l 2527C
  \l 2527D
  \l 2527E
  \l 2527F
  \l 25280
  \l 25281
  \l 25282
  \l 25283
  \l 25284
  \l 25285
  \l 25286
  \l 25287
  \l 25288
  \l 25289
  \l 2528A
  \l 2528B
  \l 2528C
  \l 2528D
  \l 2528E
  \l 2528F
  \l 25290
  \l 25291
  \l 25292
  \l 25293
  \l 25294
  \l 25295
  \l 25296
  \l 25297
  \l 25298
  \l 25299
  \l 2529A
  \l 2529B
  \l 2529C
  \l 2529D
  \l 2529E
  \l 2529F
  \l 252A0
  \l 252A1
  \l 252A2
  \l 252A3
  \l 252A4
  \l 252A5
  \l 252A6
  \l 252A7
  \l 252A8
  \l 252A9
  \l 252AA
  \l 252AB
  \l 252AC
  \l 252AD
  \l 252AE
  \l 252AF
  \l 252B0
  \l 252B1
  \l 252B2
  \l 252B3
  \l 252B4
  \l 252B5
  \l 252B6
  \l 252B7
  \l 252B8
  \l 252B9
  \l 252BA
  \l 252BB
  \l 252BC
  \l 252BD
  \l 252BE
  \l 252BF
  \l 252C0
  \l 252C1
  \l 252C2
  \l 252C3
  \l 252C4
  \l 252C5
  \l 252C6
  \l 252C7
  \l 252C8
  \l 252C9
  \l 252CA
  \l 252CB
  \l 252CC
  \l 252CD
  \l 252CE
  \l 252CF
  \l 252D0
  \l 252D1
  \l 252D2
  \l 252D3
  \l 252D4
  \l 252D5
  \l 252D6
  \l 252D7
  \l 252D8
  \l 252D9
  \l 252DA
  \l 252DB
  \l 252DC
  \l 252DD
  \l 252DE
  \l 252DF
  \l 252E0
  \l 252E1
  \l 252E2
  \l 252E3
  \l 252E4
  \l 252E5
  \l 252E6
  \l 252E7
  \l 252E8
  \l 252E9
  \l 252EA
  \l 252EB
  \l 252EC
  \l 252ED
  \l 252EE
  \l 252EF
  \l 252F0
  \l 252F1
  \l 252F2
  \l 252F3
  \l 252F4
  \l 252F5
  \l 252F6
  \l 252F7
  \l 252F8
  \l 252F9
  \l 252FA
  \l 252FB
  \l 252FC
  \l 252FD
  \l 252FE
  \l 252FF
  \l 25300
  \l 25301
  \l 25302
  \l 25303
  \l 25304
  \l 25305
  \l 25306
  \l 25307
  \l 25308
  \l 25309
  \l 2530A
  \l 2530B
  \l 2530C
  \l 2530D
  \l 2530E
  \l 2530F
  \l 25310
  \l 25311
  \l 25312
  \l 25313
  \l 25314
  \l 25315
  \l 25316
  \l 25317
  \l 25318
  \l 25319
  \l 2531A
  \l 2531B
  \l 2531C
  \l 2531D
  \l 2531E
  \l 2531F
  \l 25320
  \l 25321
  \l 25322
  \l 25323
  \l 25324
  \l 25325
  \l 25326
  \l 25327
  \l 25328
  \l 25329
  \l 2532A
  \l 2532B
  \l 2532C
  \l 2532D
  \l 2532E
  \l 2532F
  \l 25330
  \l 25331
  \l 25332
  \l 25333
  \l 25334
  \l 25335
  \l 25336
  \l 25337
  \l 25338
  \l 25339
  \l 2533A
  \l 2533B
  \l 2533C
  \l 2533D
  \l 2533E
  \l 2533F
  \l 25340
  \l 25341
  \l 25342
  \l 25343
  \l 25344
  \l 25345
  \l 25346
  \l 25347
  \l 25348
  \l 25349
  \l 2534A
  \l 2534B
  \l 2534C
  \l 2534D
  \l 2534E
  \l 2534F
  \l 25350
  \l 25351
  \l 25352
  \l 25353
  \l 25354
  \l 25355
  \l 25356
  \l 25357
  \l 25358
  \l 25359
  \l 2535A
  \l 2535B
  \l 2535C
  \l 2535D
  \l 2535E
  \l 2535F
  \l 25360
  \l 25361
  \l 25362
  \l 25363
  \l 25364
  \l 25365
  \l 25366
  \l 25367
  \l 25368
  \l 25369
  \l 2536A
  \l 2536B
  \l 2536C
  \l 2536D
  \l 2536E
  \l 2536F
  \l 25370
  \l 25371
  \l 25372
  \l 25373
  \l 25374
  \l 25375
  \l 25376
  \l 25377
  \l 25378
  \l 25379
  \l 2537A
  \l 2537B
  \l 2537C
  \l 2537D
  \l 2537E
  \l 2537F
  \l 25380
  \l 25381
  \l 25382
  \l 25383
  \l 25384
  \l 25385
  \l 25386
  \l 25387
  \l 25388
  \l 25389
  \l 2538A
  \l 2538B
  \l 2538C
  \l 2538D
  \l 2538E
  \l 2538F
  \l 25390
  \l 25391
  \l 25392
  \l 25393
  \l 25394
  \l 25395
  \l 25396
  \l 25397
  \l 25398
  \l 25399
  \l 2539A
  \l 2539B
  \l 2539C
  \l 2539D
  \l 2539E
  \l 2539F
  \l 253A0
  \l 253A1
  \l 253A2
  \l 253A3
  \l 253A4
  \l 253A5
  \l 253A6
  \l 253A7
  \l 253A8
  \l 253A9
  \l 253AA
  \l 253AB
  \l 253AC
  \l 253AD
  \l 253AE
  \l 253AF
  \l 253B0
  \l 253B1
  \l 253B2
  \l 253B3
  \l 253B4
  \l 253B5
  \l 253B6
  \l 253B7
  \l 253B8
  \l 253B9
  \l 253BA
  \l 253BB
  \l 253BC
  \l 253BD
  \l 253BE
  \l 253BF
  \l 253C0
  \l 253C1
  \l 253C2
  \l 253C3
  \l 253C4
  \l 253C5
  \l 253C6
  \l 253C7
  \l 253C8
  \l 253C9
  \l 253CA
  \l 253CB
  \l 253CC
  \l 253CD
  \l 253CE
  \l 253CF
  \l 253D0
  \l 253D1
  \l 253D2
  \l 253D3
  \l 253D4
  \l 253D5
  \l 253D6
  \l 253D7
  \l 253D8
  \l 253D9
  \l 253DA
  \l 253DB
  \l 253DC
  \l 253DD
  \l 253DE
  \l 253DF
  \l 253E0
  \l 253E1
  \l 253E2
  \l 253E3
  \l 253E4
  \l 253E5
  \l 253E6
  \l 253E7
  \l 253E8
  \l 253E9
  \l 253EA
  \l 253EB
  \l 253EC
  \l 253ED
  \l 253EE
  \l 253EF
  \l 253F0
  \l 253F1
  \l 253F2
  \l 253F3
  \l 253F4
  \l 253F5
  \l 253F6
  \l 253F7
  \l 253F8
  \l 253F9
  \l 253FA
  \l 253FB
  \l 253FC
  \l 253FD
  \l 253FE
  \l 253FF
  \l 25400
  \l 25401
  \l 25402
  \l 25403
  \l 25404
  \l 25405
  \l 25406
  \l 25407
  \l 25408
  \l 25409
  \l 2540A
  \l 2540B
  \l 2540C
  \l 2540D
  \l 2540E
  \l 2540F
  \l 25410
  \l 25411
  \l 25412
  \l 25413
  \l 25414
  \l 25415
  \l 25416
  \l 25417
  \l 25418
  \l 25419
  \l 2541A
  \l 2541B
  \l 2541C
  \l 2541D
  \l 2541E
  \l 2541F
  \l 25420
  \l 25421
  \l 25422
  \l 25423
  \l 25424
  \l 25425
  \l 25426
  \l 25427
  \l 25428
  \l 25429
  \l 2542A
  \l 2542B
  \l 2542C
  \l 2542D
  \l 2542E
  \l 2542F
  \l 25430
  \l 25431
  \l 25432
  \l 25433
  \l 25434
  \l 25435
  \l 25436
  \l 25437
  \l 25438
  \l 25439
  \l 2543A
  \l 2543B
  \l 2543C
  \l 2543D
  \l 2543E
  \l 2543F
  \l 25440
  \l 25441
  \l 25442
  \l 25443
  \l 25444
  \l 25445
  \l 25446
  \l 25447
  \l 25448
  \l 25449
  \l 2544A
  \l 2544B
  \l 2544C
  \l 2544D
  \l 2544E
  \l 2544F
  \l 25450
  \l 25451
  \l 25452
  \l 25453
  \l 25454
  \l 25455
  \l 25456
  \l 25457
  \l 25458
  \l 25459
  \l 2545A
  \l 2545B
  \l 2545C
  \l 2545D
  \l 2545E
  \l 2545F
  \l 25460
  \l 25461
  \l 25462
  \l 25463
  \l 25464
  \l 25465
  \l 25466
  \l 25467
  \l 25468
  \l 25469
  \l 2546A
  \l 2546B
  \l 2546C
  \l 2546D
  \l 2546E
  \l 2546F
  \l 25470
  \l 25471
  \l 25472
  \l 25473
  \l 25474
  \l 25475
  \l 25476
  \l 25477
  \l 25478
  \l 25479
  \l 2547A
  \l 2547B
  \l 2547C
  \l 2547D
  \l 2547E
  \l 2547F
  \l 25480
  \l 25481
  \l 25482
  \l 25483
  \l 25484
  \l 25485
  \l 25486
  \l 25487
  \l 25488
  \l 25489
  \l 2548A
  \l 2548B
  \l 2548C
  \l 2548D
  \l 2548E
  \l 2548F
  \l 25490
  \l 25491
  \l 25492
  \l 25493
  \l 25494
  \l 25495
  \l 25496
  \l 25497
  \l 25498
  \l 25499
  \l 2549A
  \l 2549B
  \l 2549C
  \l 2549D
  \l 2549E
  \l 2549F
  \l 254A0
  \l 254A1
  \l 254A2
  \l 254A3
  \l 254A4
  \l 254A5
  \l 254A6
  \l 254A7
  \l 254A8
  \l 254A9
  \l 254AA
  \l 254AB
  \l 254AC
  \l 254AD
  \l 254AE
  \l 254AF
  \l 254B0
  \l 254B1
  \l 254B2
  \l 254B3
  \l 254B4
  \l 254B5
  \l 254B6
  \l 254B7
  \l 254B8
  \l 254B9
  \l 254BA
  \l 254BB
  \l 254BC
  \l 254BD
  \l 254BE
  \l 254BF
  \l 254C0
  \l 254C1
  \l 254C2
  \l 254C3
  \l 254C4
  \l 254C5
  \l 254C6
  \l 254C7
  \l 254C8
  \l 254C9
  \l 254CA
  \l 254CB
  \l 254CC
  \l 254CD
  \l 254CE
  \l 254CF
  \l 254D0
  \l 254D1
  \l 254D2
  \l 254D3
  \l 254D4
  \l 254D5
  \l 254D6
  \l 254D7
  \l 254D8
  \l 254D9
  \l 254DA
  \l 254DB
  \l 254DC
  \l 254DD
  \l 254DE
  \l 254DF
  \l 254E0
  \l 254E1
  \l 254E2
  \l 254E3
  \l 254E4
  \l 254E5
  \l 254E6
  \l 254E7
  \l 254E8
  \l 254E9
  \l 254EA
  \l 254EB
  \l 254EC
  \l 254ED
  \l 254EE
  \l 254EF
  \l 254F0
  \l 254F1
  \l 254F2
  \l 254F3
  \l 254F4
  \l 254F5
  \l 254F6
  \l 254F7
  \l 254F8
  \l 254F9
  \l 254FA
  \l 254FB
  \l 254FC
  \l 254FD
  \l 254FE
  \l 254FF
  \l 25500
  \l 25501
  \l 25502
  \l 25503
  \l 25504
  \l 25505
  \l 25506
  \l 25507
  \l 25508
  \l 25509
  \l 2550A
  \l 2550B
  \l 2550C
  \l 2550D
  \l 2550E
  \l 2550F
  \l 25510
  \l 25511
  \l 25512
  \l 25513
  \l 25514
  \l 25515
  \l 25516
  \l 25517
  \l 25518
  \l 25519
  \l 2551A
  \l 2551B
  \l 2551C
  \l 2551D
  \l 2551E
  \l 2551F
  \l 25520
  \l 25521
  \l 25522
  \l 25523
  \l 25524
  \l 25525
  \l 25526
  \l 25527
  \l 25528
  \l 25529
  \l 2552A
  \l 2552B
  \l 2552C
  \l 2552D
  \l 2552E
  \l 2552F
  \l 25530
  \l 25531
  \l 25532
  \l 25533
  \l 25534
  \l 25535
  \l 25536
  \l 25537
  \l 25538
  \l 25539
  \l 2553A
  \l 2553B
  \l 2553C
  \l 2553D
  \l 2553E
  \l 2553F
  \l 25540
  \l 25541
  \l 25542
  \l 25543
  \l 25544
  \l 25545
  \l 25546
  \l 25547
  \l 25548
  \l 25549
  \l 2554A
  \l 2554B
  \l 2554C
  \l 2554D
  \l 2554E
  \l 2554F
  \l 25550
  \l 25551
  \l 25552
  \l 25553
  \l 25554
  \l 25555
  \l 25556
  \l 25557
  \l 25558
  \l 25559
  \l 2555A
  \l 2555B
  \l 2555C
  \l 2555D
  \l 2555E
  \l 2555F
  \l 25560
  \l 25561
  \l 25562
  \l 25563
  \l 25564
  \l 25565
  \l 25566
  \l 25567
  \l 25568
  \l 25569
  \l 2556A
  \l 2556B
  \l 2556C
  \l 2556D
  \l 2556E
  \l 2556F
  \l 25570
  \l 25571
  \l 25572
  \l 25573
  \l 25574
  \l 25575
  \l 25576
  \l 25577
  \l 25578
  \l 25579
  \l 2557A
  \l 2557B
  \l 2557C
  \l 2557D
  \l 2557E
  \l 2557F
  \l 25580
  \l 25581
  \l 25582
  \l 25583
  \l 25584
  \l 25585
  \l 25586
  \l 25587
  \l 25588
  \l 25589
  \l 2558A
  \l 2558B
  \l 2558C
  \l 2558D
  \l 2558E
  \l 2558F
  \l 25590
  \l 25591
  \l 25592
  \l 25593
  \l 25594
  \l 25595
  \l 25596
  \l 25597
  \l 25598
  \l 25599
  \l 2559A
  \l 2559B
  \l 2559C
  \l 2559D
  \l 2559E
  \l 2559F
  \l 255A0
  \l 255A1
  \l 255A2
  \l 255A3
  \l 255A4
  \l 255A5
  \l 255A6
  \l 255A7
  \l 255A8
  \l 255A9
  \l 255AA
  \l 255AB
  \l 255AC
  \l 255AD
  \l 255AE
  \l 255AF
  \l 255B0
  \l 255B1
  \l 255B2
  \l 255B3
  \l 255B4
  \l 255B5
  \l 255B6
  \l 255B7
  \l 255B8
  \l 255B9
  \l 255BA
  \l 255BB
  \l 255BC
  \l 255BD
  \l 255BE
  \l 255BF
  \l 255C0
  \l 255C1
  \l 255C2
  \l 255C3
  \l 255C4
  \l 255C5
  \l 255C6
  \l 255C7
  \l 255C8
  \l 255C9
  \l 255CA
  \l 255CB
  \l 255CC
  \l 255CD
  \l 255CE
  \l 255CF
  \l 255D0
  \l 255D1
  \l 255D2
  \l 255D3
  \l 255D4
  \l 255D5
  \l 255D6
  \l 255D7
  \l 255D8
  \l 255D9
  \l 255DA
  \l 255DB
  \l 255DC
  \l 255DD
  \l 255DE
  \l 255DF
  \l 255E0
  \l 255E1
  \l 255E2
  \l 255E3
  \l 255E4
  \l 255E5
  \l 255E6
  \l 255E7
  \l 255E8
  \l 255E9
  \l 255EA
  \l 255EB
  \l 255EC
  \l 255ED
  \l 255EE
  \l 255EF
  \l 255F0
  \l 255F1
  \l 255F2
  \l 255F3
  \l 255F4
  \l 255F5
  \l 255F6
  \l 255F7
  \l 255F8
  \l 255F9
  \l 255FA
  \l 255FB
  \l 255FC
  \l 255FD
  \l 255FE
  \l 255FF
  \l 25600
  \l 25601
  \l 25602
  \l 25603
  \l 25604
  \l 25605
  \l 25606
  \l 25607
  \l 25608
  \l 25609
  \l 2560A
  \l 2560B
  \l 2560C
  \l 2560D
  \l 2560E
  \l 2560F
  \l 25610
  \l 25611
  \l 25612
  \l 25613
  \l 25614
  \l 25615
  \l 25616
  \l 25617
  \l 25618
  \l 25619
  \l 2561A
  \l 2561B
  \l 2561C
  \l 2561D
  \l 2561E
  \l 2561F
  \l 25620
  \l 25621
  \l 25622
  \l 25623
  \l 25624
  \l 25625
  \l 25626
  \l 25627
  \l 25628
  \l 25629
  \l 2562A
  \l 2562B
  \l 2562C
  \l 2562D
  \l 2562E
  \l 2562F
  \l 25630
  \l 25631
  \l 25632
  \l 25633
  \l 25634
  \l 25635
  \l 25636
  \l 25637
  \l 25638
  \l 25639
  \l 2563A
  \l 2563B
  \l 2563C
  \l 2563D
  \l 2563E
  \l 2563F
  \l 25640
  \l 25641
  \l 25642
  \l 25643
  \l 25644
  \l 25645
  \l 25646
  \l 25647
  \l 25648
  \l 25649
  \l 2564A
  \l 2564B
  \l 2564C
  \l 2564D
  \l 2564E
  \l 2564F
  \l 25650
  \l 25651
  \l 25652
  \l 25653
  \l 25654
  \l 25655
  \l 25656
  \l 25657
  \l 25658
  \l 25659
  \l 2565A
  \l 2565B
  \l 2565C
  \l 2565D
  \l 2565E
  \l 2565F
  \l 25660
  \l 25661
  \l 25662
  \l 25663
  \l 25664
  \l 25665
  \l 25666
  \l 25667
  \l 25668
  \l 25669
  \l 2566A
  \l 2566B
  \l 2566C
  \l 2566D
  \l 2566E
  \l 2566F
  \l 25670
  \l 25671
  \l 25672
  \l 25673
  \l 25674
  \l 25675
  \l 25676
  \l 25677
  \l 25678
  \l 25679
  \l 2567A
  \l 2567B
  \l 2567C
  \l 2567D
  \l 2567E
  \l 2567F
  \l 25680
  \l 25681
  \l 25682
  \l 25683
  \l 25684
  \l 25685
  \l 25686
  \l 25687
  \l 25688
  \l 25689
  \l 2568A
  \l 2568B
  \l 2568C
  \l 2568D
  \l 2568E
  \l 2568F
  \l 25690
  \l 25691
  \l 25692
  \l 25693
  \l 25694
  \l 25695
  \l 25696
  \l 25697
  \l 25698
  \l 25699
  \l 2569A
  \l 2569B
  \l 2569C
  \l 2569D
  \l 2569E
  \l 2569F
  \l 256A0
  \l 256A1
  \l 256A2
  \l 256A3
  \l 256A4
  \l 256A5
  \l 256A6
  \l 256A7
  \l 256A8
  \l 256A9
  \l 256AA
  \l 256AB
  \l 256AC
  \l 256AD
  \l 256AE
  \l 256AF
  \l 256B0
  \l 256B1
  \l 256B2
  \l 256B3
  \l 256B4
  \l 256B5
  \l 256B6
  \l 256B7
  \l 256B8
  \l 256B9
  \l 256BA
  \l 256BB
  \l 256BC
  \l 256BD
  \l 256BE
  \l 256BF
  \l 256C0
  \l 256C1
  \l 256C2
  \l 256C3
  \l 256C4
  \l 256C5
  \l 256C6
  \l 256C7
  \l 256C8
  \l 256C9
  \l 256CA
  \l 256CB
  \l 256CC
  \l 256CD
  \l 256CE
  \l 256CF
  \l 256D0
  \l 256D1
  \l 256D2
  \l 256D3
  \l 256D4
  \l 256D5
  \l 256D6
  \l 256D7
  \l 256D8
  \l 256D9
  \l 256DA
  \l 256DB
  \l 256DC
  \l 256DD
  \l 256DE
  \l 256DF
  \l 256E0
  \l 256E1
  \l 256E2
  \l 256E3
  \l 256E4
  \l 256E5
  \l 256E6
  \l 256E7
  \l 256E8
  \l 256E9
  \l 256EA
  \l 256EB
  \l 256EC
  \l 256ED
  \l 256EE
  \l 256EF
  \l 256F0
  \l 256F1
  \l 256F2
  \l 256F3
  \l 256F4
  \l 256F5
  \l 256F6
  \l 256F7
  \l 256F8
  \l 256F9
  \l 256FA
  \l 256FB
  \l 256FC
  \l 256FD
  \l 256FE
  \l 256FF
  \l 25700
  \l 25701
  \l 25702
  \l 25703
  \l 25704
  \l 25705
  \l 25706
  \l 25707
  \l 25708
  \l 25709
  \l 2570A
  \l 2570B
  \l 2570C
  \l 2570D
  \l 2570E
  \l 2570F
  \l 25710
  \l 25711
  \l 25712
  \l 25713
  \l 25714
  \l 25715
  \l 25716
  \l 25717
  \l 25718
  \l 25719
  \l 2571A
  \l 2571B
  \l 2571C
  \l 2571D
  \l 2571E
  \l 2571F
  \l 25720
  \l 25721
  \l 25722
  \l 25723
  \l 25724
  \l 25725
  \l 25726
  \l 25727
  \l 25728
  \l 25729
  \l 2572A
  \l 2572B
  \l 2572C
  \l 2572D
  \l 2572E
  \l 2572F
  \l 25730
  \l 25731
  \l 25732
  \l 25733
  \l 25734
  \l 25735
  \l 25736
  \l 25737
  \l 25738
  \l 25739
  \l 2573A
  \l 2573B
  \l 2573C
  \l 2573D
  \l 2573E
  \l 2573F
  \l 25740
  \l 25741
  \l 25742
  \l 25743
  \l 25744
  \l 25745
  \l 25746
  \l 25747
  \l 25748
  \l 25749
  \l 2574A
  \l 2574B
  \l 2574C
  \l 2574D
  \l 2574E
  \l 2574F
  \l 25750
  \l 25751
  \l 25752
  \l 25753
  \l 25754
  \l 25755
  \l 25756
  \l 25757
  \l 25758
  \l 25759
  \l 2575A
  \l 2575B
  \l 2575C
  \l 2575D
  \l 2575E
  \l 2575F
  \l 25760
  \l 25761
  \l 25762
  \l 25763
  \l 25764
  \l 25765
  \l 25766
  \l 25767
  \l 25768
  \l 25769
  \l 2576A
  \l 2576B
  \l 2576C
  \l 2576D
  \l 2576E
  \l 2576F
  \l 25770
  \l 25771
  \l 25772
  \l 25773
  \l 25774
  \l 25775
  \l 25776
  \l 25777
  \l 25778
  \l 25779
  \l 2577A
  \l 2577B
  \l 2577C
  \l 2577D
  \l 2577E
  \l 2577F
  \l 25780
  \l 25781
  \l 25782
  \l 25783
  \l 25784
  \l 25785
  \l 25786
  \l 25787
  \l 25788
  \l 25789
  \l 2578A
  \l 2578B
  \l 2578C
  \l 2578D
  \l 2578E
  \l 2578F
  \l 25790
  \l 25791
  \l 25792
  \l 25793
  \l 25794
  \l 25795
  \l 25796
  \l 25797
  \l 25798
  \l 25799
  \l 2579A
  \l 2579B
  \l 2579C
  \l 2579D
  \l 2579E
  \l 2579F
  \l 257A0
  \l 257A1
  \l 257A2
  \l 257A3
  \l 257A4
  \l 257A5
  \l 257A6
  \l 257A7
  \l 257A8
  \l 257A9
  \l 257AA
  \l 257AB
  \l 257AC
  \l 257AD
  \l 257AE
  \l 257AF
  \l 257B0
  \l 257B1
  \l 257B2
  \l 257B3
  \l 257B4
  \l 257B5
  \l 257B6
  \l 257B7
  \l 257B8
  \l 257B9
  \l 257BA
  \l 257BB
  \l 257BC
  \l 257BD
  \l 257BE
  \l 257BF
  \l 257C0
  \l 257C1
  \l 257C2
  \l 257C3
  \l 257C4
  \l 257C5
  \l 257C6
  \l 257C7
  \l 257C8
  \l 257C9
  \l 257CA
  \l 257CB
  \l 257CC
  \l 257CD
  \l 257CE
  \l 257CF
  \l 257D0
  \l 257D1
  \l 257D2
  \l 257D3
  \l 257D4
  \l 257D5
  \l 257D6
  \l 257D7
  \l 257D8
  \l 257D9
  \l 257DA
  \l 257DB
  \l 257DC
  \l 257DD
  \l 257DE
  \l 257DF
  \l 257E0
  \l 257E1
  \l 257E2
  \l 257E3
  \l 257E4
  \l 257E5
  \l 257E6
  \l 257E7
  \l 257E8
  \l 257E9
  \l 257EA
  \l 257EB
  \l 257EC
  \l 257ED
  \l 257EE
  \l 257EF
  \l 257F0
  \l 257F1
  \l 257F2
  \l 257F3
  \l 257F4
  \l 257F5
  \l 257F6
  \l 257F7
  \l 257F8
  \l 257F9
  \l 257FA
  \l 257FB
  \l 257FC
  \l 257FD
  \l 257FE
  \l 257FF
  \l 25800
  \l 25801
  \l 25802
  \l 25803
  \l 25804
  \l 25805
  \l 25806
  \l 25807
  \l 25808
  \l 25809
  \l 2580A
  \l 2580B
  \l 2580C
  \l 2580D
  \l 2580E
  \l 2580F
  \l 25810
  \l 25811
  \l 25812
  \l 25813
  \l 25814
  \l 25815
  \l 25816
  \l 25817
  \l 25818
  \l 25819
  \l 2581A
  \l 2581B
  \l 2581C
  \l 2581D
  \l 2581E
  \l 2581F
  \l 25820
  \l 25821
  \l 25822
  \l 25823
  \l 25824
  \l 25825
  \l 25826
  \l 25827
  \l 25828
  \l 25829
  \l 2582A
  \l 2582B
  \l 2582C
  \l 2582D
  \l 2582E
  \l 2582F
  \l 25830
  \l 25831
  \l 25832
  \l 25833
  \l 25834
  \l 25835
  \l 25836
  \l 25837
  \l 25838
  \l 25839
  \l 2583A
  \l 2583B
  \l 2583C
  \l 2583D
  \l 2583E
  \l 2583F
  \l 25840
  \l 25841
  \l 25842
  \l 25843
  \l 25844
  \l 25845
  \l 25846
  \l 25847
  \l 25848
  \l 25849
  \l 2584A
  \l 2584B
  \l 2584C
  \l 2584D
  \l 2584E
  \l 2584F
  \l 25850
  \l 25851
  \l 25852
  \l 25853
  \l 25854
  \l 25855
  \l 25856
  \l 25857
  \l 25858
  \l 25859
  \l 2585A
  \l 2585B
  \l 2585C
  \l 2585D
  \l 2585E
  \l 2585F
  \l 25860
  \l 25861
  \l 25862
  \l 25863
  \l 25864
  \l 25865
  \l 25866
  \l 25867
  \l 25868
  \l 25869
  \l 2586A
  \l 2586B
  \l 2586C
  \l 2586D
  \l 2586E
  \l 2586F
  \l 25870
  \l 25871
  \l 25872
  \l 25873
  \l 25874
  \l 25875
  \l 25876
  \l 25877
  \l 25878
  \l 25879
  \l 2587A
  \l 2587B
  \l 2587C
  \l 2587D
  \l 2587E
  \l 2587F
  \l 25880
  \l 25881
  \l 25882
  \l 25883
  \l 25884
  \l 25885
  \l 25886
  \l 25887
  \l 25888
  \l 25889
  \l 2588A
  \l 2588B
  \l 2588C
  \l 2588D
  \l 2588E
  \l 2588F
  \l 25890
  \l 25891
  \l 25892
  \l 25893
  \l 25894
  \l 25895
  \l 25896
  \l 25897
  \l 25898
  \l 25899
  \l 2589A
  \l 2589B
  \l 2589C
  \l 2589D
  \l 2589E
  \l 2589F
  \l 258A0
  \l 258A1
  \l 258A2
  \l 258A3
  \l 258A4
  \l 258A5
  \l 258A6
  \l 258A7
  \l 258A8
  \l 258A9
  \l 258AA
  \l 258AB
  \l 258AC
  \l 258AD
  \l 258AE
  \l 258AF
  \l 258B0
  \l 258B1
  \l 258B2
  \l 258B3
  \l 258B4
  \l 258B5
  \l 258B6
  \l 258B7
  \l 258B8
  \l 258B9
  \l 258BA
  \l 258BB
  \l 258BC
  \l 258BD
  \l 258BE
  \l 258BF
  \l 258C0
  \l 258C1
  \l 258C2
  \l 258C3
  \l 258C4
  \l 258C5
  \l 258C6
  \l 258C7
  \l 258C8
  \l 258C9
  \l 258CA
  \l 258CB
  \l 258CC
  \l 258CD
  \l 258CE
  \l 258CF
  \l 258D0
  \l 258D1
  \l 258D2
  \l 258D3
  \l 258D4
  \l 258D5
  \l 258D6
  \l 258D7
  \l 258D8
  \l 258D9
  \l 258DA
  \l 258DB
  \l 258DC
  \l 258DD
  \l 258DE
  \l 258DF
  \l 258E0
  \l 258E1
  \l 258E2
  \l 258E3
  \l 258E4
  \l 258E5
  \l 258E6
  \l 258E7
  \l 258E8
  \l 258E9
  \l 258EA
  \l 258EB
  \l 258EC
  \l 258ED
  \l 258EE
  \l 258EF
  \l 258F0
  \l 258F1
  \l 258F2
  \l 258F3
  \l 258F4
  \l 258F5
  \l 258F6
  \l 258F7
  \l 258F8
  \l 258F9
  \l 258FA
  \l 258FB
  \l 258FC
  \l 258FD
  \l 258FE
  \l 258FF
  \l 25900
  \l 25901
  \l 25902
  \l 25903
  \l 25904
  \l 25905
  \l 25906
  \l 25907
  \l 25908
  \l 25909
  \l 2590A
  \l 2590B
  \l 2590C
  \l 2590D
  \l 2590E
  \l 2590F
  \l 25910
  \l 25911
  \l 25912
  \l 25913
  \l 25914
  \l 25915
  \l 25916
  \l 25917
  \l 25918
  \l 25919
  \l 2591A
  \l 2591B
  \l 2591C
  \l 2591D
  \l 2591E
  \l 2591F
  \l 25920
  \l 25921
  \l 25922
  \l 25923
  \l 25924
  \l 25925
  \l 25926
  \l 25927
  \l 25928
  \l 25929
  \l 2592A
  \l 2592B
  \l 2592C
  \l 2592D
  \l 2592E
  \l 2592F
  \l 25930
  \l 25931
  \l 25932
  \l 25933
  \l 25934
  \l 25935
  \l 25936
  \l 25937
  \l 25938
  \l 25939
  \l 2593A
  \l 2593B
  \l 2593C
  \l 2593D
  \l 2593E
  \l 2593F
  \l 25940
  \l 25941
  \l 25942
  \l 25943
  \l 25944
  \l 25945
  \l 25946
  \l 25947
  \l 25948
  \l 25949
  \l 2594A
  \l 2594B
  \l 2594C
  \l 2594D
  \l 2594E
  \l 2594F
  \l 25950
  \l 25951
  \l 25952
  \l 25953
  \l 25954
  \l 25955
  \l 25956
  \l 25957
  \l 25958
  \l 25959
  \l 2595A
  \l 2595B
  \l 2595C
  \l 2595D
  \l 2595E
  \l 2595F
  \l 25960
  \l 25961
  \l 25962
  \l 25963
  \l 25964
  \l 25965
  \l 25966
  \l 25967
  \l 25968
  \l 25969
  \l 2596A
  \l 2596B
  \l 2596C
  \l 2596D
  \l 2596E
  \l 2596F
  \l 25970
  \l 25971
  \l 25972
  \l 25973
  \l 25974
  \l 25975
  \l 25976
  \l 25977
  \l 25978
  \l 25979
  \l 2597A
  \l 2597B
  \l 2597C
  \l 2597D
  \l 2597E
  \l 2597F
  \l 25980
  \l 25981
  \l 25982
  \l 25983
  \l 25984
  \l 25985
  \l 25986
  \l 25987
  \l 25988
  \l 25989
  \l 2598A
  \l 2598B
  \l 2598C
  \l 2598D
  \l 2598E
  \l 2598F
  \l 25990
  \l 25991
  \l 25992
  \l 25993
  \l 25994
  \l 25995
  \l 25996
  \l 25997
  \l 25998
  \l 25999
  \l 2599A
  \l 2599B
  \l 2599C
  \l 2599D
  \l 2599E
  \l 2599F
  \l 259A0
  \l 259A1
  \l 259A2
  \l 259A3
  \l 259A4
  \l 259A5
  \l 259A6
  \l 259A7
  \l 259A8
  \l 259A9
  \l 259AA
  \l 259AB
  \l 259AC
  \l 259AD
  \l 259AE
  \l 259AF
  \l 259B0
  \l 259B1
  \l 259B2
  \l 259B3
  \l 259B4
  \l 259B5
  \l 259B6
  \l 259B7
  \l 259B8
  \l 259B9
  \l 259BA
  \l 259BB
  \l 259BC
  \l 259BD
  \l 259BE
  \l 259BF
  \l 259C0
  \l 259C1
  \l 259C2
  \l 259C3
  \l 259C4
  \l 259C5
  \l 259C6
  \l 259C7
  \l 259C8
  \l 259C9
  \l 259CA
  \l 259CB
  \l 259CC
  \l 259CD
  \l 259CE
  \l 259CF
  \l 259D0
  \l 259D1
  \l 259D2
  \l 259D3
  \l 259D4
  \l 259D5
  \l 259D6
  \l 259D7
  \l 259D8
  \l 259D9
  \l 259DA
  \l 259DB
  \l 259DC
  \l 259DD
  \l 259DE
  \l 259DF
  \l 259E0
  \l 259E1
  \l 259E2
  \l 259E3
  \l 259E4
  \l 259E5
  \l 259E6
  \l 259E7
  \l 259E8
  \l 259E9
  \l 259EA
  \l 259EB
  \l 259EC
  \l 259ED
  \l 259EE
  \l 259EF
  \l 259F0
  \l 259F1
  \l 259F2
  \l 259F3
  \l 259F4
  \l 259F5
  \l 259F6
  \l 259F7
  \l 259F8
  \l 259F9
  \l 259FA
  \l 259FB
  \l 259FC
  \l 259FD
  \l 259FE
  \l 259FF
  \l 25A00
  \l 25A01
  \l 25A02
  \l 25A03
  \l 25A04
  \l 25A05
  \l 25A06
  \l 25A07
  \l 25A08
  \l 25A09
  \l 25A0A
  \l 25A0B
  \l 25A0C
  \l 25A0D
  \l 25A0E
  \l 25A0F
  \l 25A10
  \l 25A11
  \l 25A12
  \l 25A13
  \l 25A14
  \l 25A15
  \l 25A16
  \l 25A17
  \l 25A18
  \l 25A19
  \l 25A1A
  \l 25A1B
  \l 25A1C
  \l 25A1D
  \l 25A1E
  \l 25A1F
  \l 25A20
  \l 25A21
  \l 25A22
  \l 25A23
  \l 25A24
  \l 25A25
  \l 25A26
  \l 25A27
  \l 25A28
  \l 25A29
  \l 25A2A
  \l 25A2B
  \l 25A2C
  \l 25A2D
  \l 25A2E
  \l 25A2F
  \l 25A30
  \l 25A31
  \l 25A32
  \l 25A33
  \l 25A34
  \l 25A35
  \l 25A36
  \l 25A37
  \l 25A38
  \l 25A39
  \l 25A3A
  \l 25A3B
  \l 25A3C
  \l 25A3D
  \l 25A3E
  \l 25A3F
  \l 25A40
  \l 25A41
  \l 25A42
  \l 25A43
  \l 25A44
  \l 25A45
  \l 25A46
  \l 25A47
  \l 25A48
  \l 25A49
  \l 25A4A
  \l 25A4B
  \l 25A4C
  \l 25A4D
  \l 25A4E
  \l 25A4F
  \l 25A50
  \l 25A51
  \l 25A52
  \l 25A53
  \l 25A54
  \l 25A55
  \l 25A56
  \l 25A57
  \l 25A58
  \l 25A59
  \l 25A5A
  \l 25A5B
  \l 25A5C
  \l 25A5D
  \l 25A5E
  \l 25A5F
  \l 25A60
  \l 25A61
  \l 25A62
  \l 25A63
  \l 25A64
  \l 25A65
  \l 25A66
  \l 25A67
  \l 25A68
  \l 25A69
  \l 25A6A
  \l 25A6B
  \l 25A6C
  \l 25A6D
  \l 25A6E
  \l 25A6F
  \l 25A70
  \l 25A71
  \l 25A72
  \l 25A73
  \l 25A74
  \l 25A75
  \l 25A76
  \l 25A77
  \l 25A78
  \l 25A79
  \l 25A7A
  \l 25A7B
  \l 25A7C
  \l 25A7D
  \l 25A7E
  \l 25A7F
  \l 25A80
  \l 25A81
  \l 25A82
  \l 25A83
  \l 25A84
  \l 25A85
  \l 25A86
  \l 25A87
  \l 25A88
  \l 25A89
  \l 25A8A
  \l 25A8B
  \l 25A8C
  \l 25A8D
  \l 25A8E
  \l 25A8F
  \l 25A90
  \l 25A91
  \l 25A92
  \l 25A93
  \l 25A94
  \l 25A95
  \l 25A96
  \l 25A97
  \l 25A98
  \l 25A99
  \l 25A9A
  \l 25A9B
  \l 25A9C
  \l 25A9D
  \l 25A9E
  \l 25A9F
  \l 25AA0
  \l 25AA1
  \l 25AA2
  \l 25AA3
  \l 25AA4
  \l 25AA5
  \l 25AA6
  \l 25AA7
  \l 25AA8
  \l 25AA9
  \l 25AAA
  \l 25AAB
  \l 25AAC
  \l 25AAD
  \l 25AAE
  \l 25AAF
  \l 25AB0
  \l 25AB1
  \l 25AB2
  \l 25AB3
  \l 25AB4
  \l 25AB5
  \l 25AB6
  \l 25AB7
  \l 25AB8
  \l 25AB9
  \l 25ABA
  \l 25ABB
  \l 25ABC
  \l 25ABD
  \l 25ABE
  \l 25ABF
  \l 25AC0
  \l 25AC1
  \l 25AC2
  \l 25AC3
  \l 25AC4
  \l 25AC5
  \l 25AC6
  \l 25AC7
  \l 25AC8
  \l 25AC9
  \l 25ACA
  \l 25ACB
  \l 25ACC
  \l 25ACD
  \l 25ACE
  \l 25ACF
  \l 25AD0
  \l 25AD1
  \l 25AD2
  \l 25AD3
  \l 25AD4
  \l 25AD5
  \l 25AD6
  \l 25AD7
  \l 25AD8
  \l 25AD9
  \l 25ADA
  \l 25ADB
  \l 25ADC
  \l 25ADD
  \l 25ADE
  \l 25ADF
  \l 25AE0
  \l 25AE1
  \l 25AE2
  \l 25AE3
  \l 25AE4
  \l 25AE5
  \l 25AE6
  \l 25AE7
  \l 25AE8
  \l 25AE9
  \l 25AEA
  \l 25AEB
  \l 25AEC
  \l 25AED
  \l 25AEE
  \l 25AEF
  \l 25AF0
  \l 25AF1
  \l 25AF2
  \l 25AF3
  \l 25AF4
  \l 25AF5
  \l 25AF6
  \l 25AF7
  \l 25AF8
  \l 25AF9
  \l 25AFA
  \l 25AFB
  \l 25AFC
  \l 25AFD
  \l 25AFE
  \l 25AFF
  \l 25B00
  \l 25B01
  \l 25B02
  \l 25B03
  \l 25B04
  \l 25B05
  \l 25B06
  \l 25B07
  \l 25B08
  \l 25B09
  \l 25B0A
  \l 25B0B
  \l 25B0C
  \l 25B0D
  \l 25B0E
  \l 25B0F
  \l 25B10
  \l 25B11
  \l 25B12
  \l 25B13
  \l 25B14
  \l 25B15
  \l 25B16
  \l 25B17
  \l 25B18
  \l 25B19
  \l 25B1A
  \l 25B1B
  \l 25B1C
  \l 25B1D
  \l 25B1E
  \l 25B1F
  \l 25B20
  \l 25B21
  \l 25B22
  \l 25B23
  \l 25B24
  \l 25B25
  \l 25B26
  \l 25B27
  \l 25B28
  \l 25B29
  \l 25B2A
  \l 25B2B
  \l 25B2C
  \l 25B2D
  \l 25B2E
  \l 25B2F
  \l 25B30
  \l 25B31
  \l 25B32
  \l 25B33
  \l 25B34
  \l 25B35
  \l 25B36
  \l 25B37
  \l 25B38
  \l 25B39
  \l 25B3A
  \l 25B3B
  \l 25B3C
  \l 25B3D
  \l 25B3E
  \l 25B3F
  \l 25B40
  \l 25B41
  \l 25B42
  \l 25B43
  \l 25B44
  \l 25B45
  \l 25B46
  \l 25B47
  \l 25B48
  \l 25B49
  \l 25B4A
  \l 25B4B
  \l 25B4C
  \l 25B4D
  \l 25B4E
  \l 25B4F
  \l 25B50
  \l 25B51
  \l 25B52
  \l 25B53
  \l 25B54
  \l 25B55
  \l 25B56
  \l 25B57
  \l 25B58
  \l 25B59
  \l 25B5A
  \l 25B5B
  \l 25B5C
  \l 25B5D
  \l 25B5E
  \l 25B5F
  \l 25B60
  \l 25B61
  \l 25B62
  \l 25B63
  \l 25B64
  \l 25B65
  \l 25B66
  \l 25B67
  \l 25B68
  \l 25B69
  \l 25B6A
  \l 25B6B
  \l 25B6C
  \l 25B6D
  \l 25B6E
  \l 25B6F
  \l 25B70
  \l 25B71
  \l 25B72
  \l 25B73
  \l 25B74
  \l 25B75
  \l 25B76
  \l 25B77
  \l 25B78
  \l 25B79
  \l 25B7A
  \l 25B7B
  \l 25B7C
  \l 25B7D
  \l 25B7E
  \l 25B7F
  \l 25B80
  \l 25B81
  \l 25B82
  \l 25B83
  \l 25B84
  \l 25B85
  \l 25B86
  \l 25B87
  \l 25B88
  \l 25B89
  \l 25B8A
  \l 25B8B
  \l 25B8C
  \l 25B8D
  \l 25B8E
  \l 25B8F
  \l 25B90
  \l 25B91
  \l 25B92
  \l 25B93
  \l 25B94
  \l 25B95
  \l 25B96
  \l 25B97
  \l 25B98
  \l 25B99
  \l 25B9A
  \l 25B9B
  \l 25B9C
  \l 25B9D
  \l 25B9E
  \l 25B9F
  \l 25BA0
  \l 25BA1
  \l 25BA2
  \l 25BA3
  \l 25BA4
  \l 25BA5
  \l 25BA6
  \l 25BA7
  \l 25BA8
  \l 25BA9
  \l 25BAA
  \l 25BAB
  \l 25BAC
  \l 25BAD
  \l 25BAE
  \l 25BAF
  \l 25BB0
  \l 25BB1
  \l 25BB2
  \l 25BB3
  \l 25BB4
  \l 25BB5
  \l 25BB6
  \l 25BB7
  \l 25BB8
  \l 25BB9
  \l 25BBA
  \l 25BBB
  \l 25BBC
  \l 25BBD
  \l 25BBE
  \l 25BBF
  \l 25BC0
  \l 25BC1
  \l 25BC2
  \l 25BC3
  \l 25BC4
  \l 25BC5
  \l 25BC6
  \l 25BC7
  \l 25BC8
  \l 25BC9
  \l 25BCA
  \l 25BCB
  \l 25BCC
  \l 25BCD
  \l 25BCE
  \l 25BCF
  \l 25BD0
  \l 25BD1
  \l 25BD2
  \l 25BD3
  \l 25BD4
  \l 25BD5
  \l 25BD6
  \l 25BD7
  \l 25BD8
  \l 25BD9
  \l 25BDA
  \l 25BDB
  \l 25BDC
  \l 25BDD
  \l 25BDE
  \l 25BDF
  \l 25BE0
  \l 25BE1
  \l 25BE2
  \l 25BE3
  \l 25BE4
  \l 25BE5
  \l 25BE6
  \l 25BE7
  \l 25BE8
  \l 25BE9
  \l 25BEA
  \l 25BEB
  \l 25BEC
  \l 25BED
  \l 25BEE
  \l 25BEF
  \l 25BF0
  \l 25BF1
  \l 25BF2
  \l 25BF3
  \l 25BF4
  \l 25BF5
  \l 25BF6
  \l 25BF7
  \l 25BF8
  \l 25BF9
  \l 25BFA
  \l 25BFB
  \l 25BFC
  \l 25BFD
  \l 25BFE
  \l 25BFF
  \l 25C00
  \l 25C01
  \l 25C02
  \l 25C03
  \l 25C04
  \l 25C05
  \l 25C06
  \l 25C07
  \l 25C08
  \l 25C09
  \l 25C0A
  \l 25C0B
  \l 25C0C
  \l 25C0D
  \l 25C0E
  \l 25C0F
  \l 25C10
  \l 25C11
  \l 25C12
  \l 25C13
  \l 25C14
  \l 25C15
  \l 25C16
  \l 25C17
  \l 25C18
  \l 25C19
  \l 25C1A
  \l 25C1B
  \l 25C1C
  \l 25C1D
  \l 25C1E
  \l 25C1F
  \l 25C20
  \l 25C21
  \l 25C22
  \l 25C23
  \l 25C24
  \l 25C25
  \l 25C26
  \l 25C27
  \l 25C28
  \l 25C29
  \l 25C2A
  \l 25C2B
  \l 25C2C
  \l 25C2D
  \l 25C2E
  \l 25C2F
  \l 25C30
  \l 25C31
  \l 25C32
  \l 25C33
  \l 25C34
  \l 25C35
  \l 25C36
  \l 25C37
  \l 25C38
  \l 25C39
  \l 25C3A
  \l 25C3B
  \l 25C3C
  \l 25C3D
  \l 25C3E
  \l 25C3F
  \l 25C40
  \l 25C41
  \l 25C42
  \l 25C43
  \l 25C44
  \l 25C45
  \l 25C46
  \l 25C47
  \l 25C48
  \l 25C49
  \l 25C4A
  \l 25C4B
  \l 25C4C
  \l 25C4D
  \l 25C4E
  \l 25C4F
  \l 25C50
  \l 25C51
  \l 25C52
  \l 25C53
  \l 25C54
  \l 25C55
  \l 25C56
  \l 25C57
  \l 25C58
  \l 25C59
  \l 25C5A
  \l 25C5B
  \l 25C5C
  \l 25C5D
  \l 25C5E
  \l 25C5F
  \l 25C60
  \l 25C61
  \l 25C62
  \l 25C63
  \l 25C64
  \l 25C65
  \l 25C66
  \l 25C67
  \l 25C68
  \l 25C69
  \l 25C6A
  \l 25C6B
  \l 25C6C
  \l 25C6D
  \l 25C6E
  \l 25C6F
  \l 25C70
  \l 25C71
  \l 25C72
  \l 25C73
  \l 25C74
  \l 25C75
  \l 25C76
  \l 25C77
  \l 25C78
  \l 25C79
  \l 25C7A
  \l 25C7B
  \l 25C7C
  \l 25C7D
  \l 25C7E
  \l 25C7F
  \l 25C80
  \l 25C81
  \l 25C82
  \l 25C83
  \l 25C84
  \l 25C85
  \l 25C86
  \l 25C87
  \l 25C88
  \l 25C89
  \l 25C8A
  \l 25C8B
  \l 25C8C
  \l 25C8D
  \l 25C8E
  \l 25C8F
  \l 25C90
  \l 25C91
  \l 25C92
  \l 25C93
  \l 25C94
  \l 25C95
  \l 25C96
  \l 25C97
  \l 25C98
  \l 25C99
  \l 25C9A
  \l 25C9B
  \l 25C9C
  \l 25C9D
  \l 25C9E
  \l 25C9F
  \l 25CA0
  \l 25CA1
  \l 25CA2
  \l 25CA3
  \l 25CA4
  \l 25CA5
  \l 25CA6
  \l 25CA7
  \l 25CA8
  \l 25CA9
  \l 25CAA
  \l 25CAB
  \l 25CAC
  \l 25CAD
  \l 25CAE
  \l 25CAF
  \l 25CB0
  \l 25CB1
  \l 25CB2
  \l 25CB3
  \l 25CB4
  \l 25CB5
  \l 25CB6
  \l 25CB7
  \l 25CB8
  \l 25CB9
  \l 25CBA
  \l 25CBB
  \l 25CBC
  \l 25CBD
  \l 25CBE
  \l 25CBF
  \l 25CC0
  \l 25CC1
  \l 25CC2
  \l 25CC3
  \l 25CC4
  \l 25CC5
  \l 25CC6
  \l 25CC7
  \l 25CC8
  \l 25CC9
  \l 25CCA
  \l 25CCB
  \l 25CCC
  \l 25CCD
  \l 25CCE
  \l 25CCF
  \l 25CD0
  \l 25CD1
  \l 25CD2
  \l 25CD3
  \l 25CD4
  \l 25CD5
  \l 25CD6
  \l 25CD7
  \l 25CD8
  \l 25CD9
  \l 25CDA
  \l 25CDB
  \l 25CDC
  \l 25CDD
  \l 25CDE
  \l 25CDF
  \l 25CE0
  \l 25CE1
  \l 25CE2
  \l 25CE3
  \l 25CE4
  \l 25CE5
  \l 25CE6
  \l 25CE7
  \l 25CE8
  \l 25CE9
  \l 25CEA
  \l 25CEB
  \l 25CEC
  \l 25CED
  \l 25CEE
  \l 25CEF
  \l 25CF0
  \l 25CF1
  \l 25CF2
  \l 25CF3
  \l 25CF4
  \l 25CF5
  \l 25CF6
  \l 25CF7
  \l 25CF8
  \l 25CF9
  \l 25CFA
  \l 25CFB
  \l 25CFC
  \l 25CFD
  \l 25CFE
  \l 25CFF
  \l 25D00
  \l 25D01
  \l 25D02
  \l 25D03
  \l 25D04
  \l 25D05
  \l 25D06
  \l 25D07
  \l 25D08
  \l 25D09
  \l 25D0A
  \l 25D0B
  \l 25D0C
  \l 25D0D
  \l 25D0E
  \l 25D0F
  \l 25D10
  \l 25D11
  \l 25D12
  \l 25D13
  \l 25D14
  \l 25D15
  \l 25D16
  \l 25D17
  \l 25D18
  \l 25D19
  \l 25D1A
  \l 25D1B
  \l 25D1C
  \l 25D1D
  \l 25D1E
  \l 25D1F
  \l 25D20
  \l 25D21
  \l 25D22
  \l 25D23
  \l 25D24
  \l 25D25
  \l 25D26
  \l 25D27
  \l 25D28
  \l 25D29
  \l 25D2A
  \l 25D2B
  \l 25D2C
  \l 25D2D
  \l 25D2E
  \l 25D2F
  \l 25D30
  \l 25D31
  \l 25D32
  \l 25D33
  \l 25D34
  \l 25D35
  \l 25D36
  \l 25D37
  \l 25D38
  \l 25D39
  \l 25D3A
  \l 25D3B
  \l 25D3C
  \l 25D3D
  \l 25D3E
  \l 25D3F
  \l 25D40
  \l 25D41
  \l 25D42
  \l 25D43
  \l 25D44
  \l 25D45
  \l 25D46
  \l 25D47
  \l 25D48
  \l 25D49
  \l 25D4A
  \l 25D4B
  \l 25D4C
  \l 25D4D
  \l 25D4E
  \l 25D4F
  \l 25D50
  \l 25D51
  \l 25D52
  \l 25D53
  \l 25D54
  \l 25D55
  \l 25D56
  \l 25D57
  \l 25D58
  \l 25D59
  \l 25D5A
  \l 25D5B
  \l 25D5C
  \l 25D5D
  \l 25D5E
  \l 25D5F
  \l 25D60
  \l 25D61
  \l 25D62
  \l 25D63
  \l 25D64
  \l 25D65
  \l 25D66
  \l 25D67
  \l 25D68
  \l 25D69
  \l 25D6A
  \l 25D6B
  \l 25D6C
  \l 25D6D
  \l 25D6E
  \l 25D6F
  \l 25D70
  \l 25D71
  \l 25D72
  \l 25D73
  \l 25D74
  \l 25D75
  \l 25D76
  \l 25D77
  \l 25D78
  \l 25D79
  \l 25D7A
  \l 25D7B
  \l 25D7C
  \l 25D7D
  \l 25D7E
  \l 25D7F
  \l 25D80
  \l 25D81
  \l 25D82
  \l 25D83
  \l 25D84
  \l 25D85
  \l 25D86
  \l 25D87
  \l 25D88
  \l 25D89
  \l 25D8A
  \l 25D8B
  \l 25D8C
  \l 25D8D
  \l 25D8E
  \l 25D8F
  \l 25D90
  \l 25D91
  \l 25D92
  \l 25D93
  \l 25D94
  \l 25D95
  \l 25D96
  \l 25D97
  \l 25D98
  \l 25D99
  \l 25D9A
  \l 25D9B
  \l 25D9C
  \l 25D9D
  \l 25D9E
  \l 25D9F
  \l 25DA0
  \l 25DA1
  \l 25DA2
  \l 25DA3
  \l 25DA4
  \l 25DA5
  \l 25DA6
  \l 25DA7
  \l 25DA8
  \l 25DA9
  \l 25DAA
  \l 25DAB
  \l 25DAC
  \l 25DAD
  \l 25DAE
  \l 25DAF
  \l 25DB0
  \l 25DB1
  \l 25DB2
  \l 25DB3
  \l 25DB4
  \l 25DB5
  \l 25DB6
  \l 25DB7
  \l 25DB8
  \l 25DB9
  \l 25DBA
  \l 25DBB
  \l 25DBC
  \l 25DBD
  \l 25DBE
  \l 25DBF
  \l 25DC0
  \l 25DC1
  \l 25DC2
  \l 25DC3
  \l 25DC4
  \l 25DC5
  \l 25DC6
  \l 25DC7
  \l 25DC8
  \l 25DC9
  \l 25DCA
  \l 25DCB
  \l 25DCC
  \l 25DCD
  \l 25DCE
  \l 25DCF
  \l 25DD0
  \l 25DD1
  \l 25DD2
  \l 25DD3
  \l 25DD4
  \l 25DD5
  \l 25DD6
  \l 25DD7
  \l 25DD8
  \l 25DD9
  \l 25DDA
  \l 25DDB
  \l 25DDC
  \l 25DDD
  \l 25DDE
  \l 25DDF
  \l 25DE0
  \l 25DE1
  \l 25DE2
  \l 25DE3
  \l 25DE4
  \l 25DE5
  \l 25DE6
  \l 25DE7
  \l 25DE8
  \l 25DE9
  \l 25DEA
  \l 25DEB
  \l 25DEC
  \l 25DED
  \l 25DEE
  \l 25DEF
  \l 25DF0
  \l 25DF1
  \l 25DF2
  \l 25DF3
  \l 25DF4
  \l 25DF5
  \l 25DF6
  \l 25DF7
  \l 25DF8
  \l 25DF9
  \l 25DFA
  \l 25DFB
  \l 25DFC
  \l 25DFD
  \l 25DFE
  \l 25DFF
  \l 25E00
  \l 25E01
  \l 25E02
  \l 25E03
  \l 25E04
  \l 25E05
  \l 25E06
  \l 25E07
  \l 25E08
  \l 25E09
  \l 25E0A
  \l 25E0B
  \l 25E0C
  \l 25E0D
  \l 25E0E
  \l 25E0F
  \l 25E10
  \l 25E11
  \l 25E12
  \l 25E13
  \l 25E14
  \l 25E15
  \l 25E16
  \l 25E17
  \l 25E18
  \l 25E19
  \l 25E1A
  \l 25E1B
  \l 25E1C
  \l 25E1D
  \l 25E1E
  \l 25E1F
  \l 25E20
  \l 25E21
  \l 25E22
  \l 25E23
  \l 25E24
  \l 25E25
  \l 25E26
  \l 25E27
  \l 25E28
  \l 25E29
  \l 25E2A
  \l 25E2B
  \l 25E2C
  \l 25E2D
  \l 25E2E
  \l 25E2F
  \l 25E30
  \l 25E31
  \l 25E32
  \l 25E33
  \l 25E34
  \l 25E35
  \l 25E36
  \l 25E37
  \l 25E38
  \l 25E39
  \l 25E3A
  \l 25E3B
  \l 25E3C
  \l 25E3D
  \l 25E3E
  \l 25E3F
  \l 25E40
  \l 25E41
  \l 25E42
  \l 25E43
  \l 25E44
  \l 25E45
  \l 25E46
  \l 25E47
  \l 25E48
  \l 25E49
  \l 25E4A
  \l 25E4B
  \l 25E4C
  \l 25E4D
  \l 25E4E
  \l 25E4F
  \l 25E50
  \l 25E51
  \l 25E52
  \l 25E53
  \l 25E54
  \l 25E55
  \l 25E56
  \l 25E57
  \l 25E58
  \l 25E59
  \l 25E5A
  \l 25E5B
  \l 25E5C
  \l 25E5D
  \l 25E5E
  \l 25E5F
  \l 25E60
  \l 25E61
  \l 25E62
  \l 25E63
  \l 25E64
  \l 25E65
  \l 25E66
  \l 25E67
  \l 25E68
  \l 25E69
  \l 25E6A
  \l 25E6B
  \l 25E6C
  \l 25E6D
  \l 25E6E
  \l 25E6F
  \l 25E70
  \l 25E71
  \l 25E72
  \l 25E73
  \l 25E74
  \l 25E75
  \l 25E76
  \l 25E77
  \l 25E78
  \l 25E79
  \l 25E7A
  \l 25E7B
  \l 25E7C
  \l 25E7D
  \l 25E7E
  \l 25E7F
  \l 25E80
  \l 25E81
  \l 25E82
  \l 25E83
  \l 25E84
  \l 25E85
  \l 25E86
  \l 25E87
  \l 25E88
  \l 25E89
  \l 25E8A
  \l 25E8B
  \l 25E8C
  \l 25E8D
  \l 25E8E
  \l 25E8F
  \l 25E90
  \l 25E91
  \l 25E92
  \l 25E93
  \l 25E94
  \l 25E95
  \l 25E96
  \l 25E97
  \l 25E98
  \l 25E99
  \l 25E9A
  \l 25E9B
  \l 25E9C
  \l 25E9D
  \l 25E9E
  \l 25E9F
  \l 25EA0
  \l 25EA1
  \l 25EA2
  \l 25EA3
  \l 25EA4
  \l 25EA5
  \l 25EA6
  \l 25EA7
  \l 25EA8
  \l 25EA9
  \l 25EAA
  \l 25EAB
  \l 25EAC
  \l 25EAD
  \l 25EAE
  \l 25EAF
  \l 25EB0
  \l 25EB1
  \l 25EB2
  \l 25EB3
  \l 25EB4
  \l 25EB5
  \l 25EB6
  \l 25EB7
  \l 25EB8
  \l 25EB9
  \l 25EBA
  \l 25EBB
  \l 25EBC
  \l 25EBD
  \l 25EBE
  \l 25EBF
  \l 25EC0
  \l 25EC1
  \l 25EC2
  \l 25EC3
  \l 25EC4
  \l 25EC5
  \l 25EC6
  \l 25EC7
  \l 25EC8
  \l 25EC9
  \l 25ECA
  \l 25ECB
  \l 25ECC
  \l 25ECD
  \l 25ECE
  \l 25ECF
  \l 25ED0
  \l 25ED1
  \l 25ED2
  \l 25ED3
  \l 25ED4
  \l 25ED5
  \l 25ED6
  \l 25ED7
  \l 25ED8
  \l 25ED9
  \l 25EDA
  \l 25EDB
  \l 25EDC
  \l 25EDD
  \l 25EDE
  \l 25EDF
  \l 25EE0
  \l 25EE1
  \l 25EE2
  \l 25EE3
  \l 25EE4
  \l 25EE5
  \l 25EE6
  \l 25EE7
  \l 25EE8
  \l 25EE9
  \l 25EEA
  \l 25EEB
  \l 25EEC
  \l 25EED
  \l 25EEE
  \l 25EEF
  \l 25EF0
  \l 25EF1
  \l 25EF2
  \l 25EF3
  \l 25EF4
  \l 25EF5
  \l 25EF6
  \l 25EF7
  \l 25EF8
  \l 25EF9
  \l 25EFA
  \l 25EFB
  \l 25EFC
  \l 25EFD
  \l 25EFE
  \l 25EFF
  \l 25F00
  \l 25F01
  \l 25F02
  \l 25F03
  \l 25F04
  \l 25F05
  \l 25F06
  \l 25F07
  \l 25F08
  \l 25F09
  \l 25F0A
  \l 25F0B
  \l 25F0C
  \l 25F0D
  \l 25F0E
  \l 25F0F
  \l 25F10
  \l 25F11
  \l 25F12
  \l 25F13
  \l 25F14
  \l 25F15
  \l 25F16
  \l 25F17
  \l 25F18
  \l 25F19
  \l 25F1A
  \l 25F1B
  \l 25F1C
  \l 25F1D
  \l 25F1E
  \l 25F1F
  \l 25F20
  \l 25F21
  \l 25F22
  \l 25F23
  \l 25F24
  \l 25F25
  \l 25F26
  \l 25F27
  \l 25F28
  \l 25F29
  \l 25F2A
  \l 25F2B
  \l 25F2C
  \l 25F2D
  \l 25F2E
  \l 25F2F
  \l 25F30
  \l 25F31
  \l 25F32
  \l 25F33
  \l 25F34
  \l 25F35
  \l 25F36
  \l 25F37
  \l 25F38
  \l 25F39
  \l 25F3A
  \l 25F3B
  \l 25F3C
  \l 25F3D
  \l 25F3E
  \l 25F3F
  \l 25F40
  \l 25F41
  \l 25F42
  \l 25F43
  \l 25F44
  \l 25F45
  \l 25F46
  \l 25F47
  \l 25F48
  \l 25F49
  \l 25F4A
  \l 25F4B
  \l 25F4C
  \l 25F4D
  \l 25F4E
  \l 25F4F
  \l 25F50
  \l 25F51
  \l 25F52
  \l 25F53
  \l 25F54
  \l 25F55
  \l 25F56
  \l 25F57
  \l 25F58
  \l 25F59
  \l 25F5A
  \l 25F5B
  \l 25F5C
  \l 25F5D
  \l 25F5E
  \l 25F5F
  \l 25F60
  \l 25F61
  \l 25F62
  \l 25F63
  \l 25F64
  \l 25F65
  \l 25F66
  \l 25F67
  \l 25F68
  \l 25F69
  \l 25F6A
  \l 25F6B
  \l 25F6C
  \l 25F6D
  \l 25F6E
  \l 25F6F
  \l 25F70
  \l 25F71
  \l 25F72
  \l 25F73
  \l 25F74
  \l 25F75
  \l 25F76
  \l 25F77
  \l 25F78
  \l 25F79
  \l 25F7A
  \l 25F7B
  \l 25F7C
  \l 25F7D
  \l 25F7E
  \l 25F7F
  \l 25F80
  \l 25F81
  \l 25F82
  \l 25F83
  \l 25F84
  \l 25F85
  \l 25F86
  \l 25F87
  \l 25F88
  \l 25F89
  \l 25F8A
  \l 25F8B
  \l 25F8C
  \l 25F8D
  \l 25F8E
  \l 25F8F
  \l 25F90
  \l 25F91
  \l 25F92
  \l 25F93
  \l 25F94
  \l 25F95
  \l 25F96
  \l 25F97
  \l 25F98
  \l 25F99
  \l 25F9A
  \l 25F9B
  \l 25F9C
  \l 25F9D
  \l 25F9E
  \l 25F9F
  \l 25FA0
  \l 25FA1
  \l 25FA2
  \l 25FA3
  \l 25FA4
  \l 25FA5
  \l 25FA6
  \l 25FA7
  \l 25FA8
  \l 25FA9
  \l 25FAA
  \l 25FAB
  \l 25FAC
  \l 25FAD
  \l 25FAE
  \l 25FAF
  \l 25FB0
  \l 25FB1
  \l 25FB2
  \l 25FB3
  \l 25FB4
  \l 25FB5
  \l 25FB6
  \l 25FB7
  \l 25FB8
  \l 25FB9
  \l 25FBA
  \l 25FBB
  \l 25FBC
  \l 25FBD
  \l 25FBE
  \l 25FBF
  \l 25FC0
  \l 25FC1
  \l 25FC2
  \l 25FC3
  \l 25FC4
  \l 25FC5
  \l 25FC6
  \l 25FC7
  \l 25FC8
  \l 25FC9
  \l 25FCA
  \l 25FCB
  \l 25FCC
  \l 25FCD
  \l 25FCE
  \l 25FCF
  \l 25FD0
  \l 25FD1
  \l 25FD2
  \l 25FD3
  \l 25FD4
  \l 25FD5
  \l 25FD6
  \l 25FD7
  \l 25FD8
  \l 25FD9
  \l 25FDA
  \l 25FDB
  \l 25FDC
  \l 25FDD
  \l 25FDE
  \l 25FDF
  \l 25FE0
  \l 25FE1
  \l 25FE2
  \l 25FE3
  \l 25FE4
  \l 25FE5
  \l 25FE6
  \l 25FE7
  \l 25FE8
  \l 25FE9
  \l 25FEA
  \l 25FEB
  \l 25FEC
  \l 25FED
  \l 25FEE
  \l 25FEF
  \l 25FF0
  \l 25FF1
  \l 25FF2
  \l 25FF3
  \l 25FF4
  \l 25FF5
  \l 25FF6
  \l 25FF7
  \l 25FF8
  \l 25FF9
  \l 25FFA
  \l 25FFB
  \l 25FFC
  \l 25FFD
  \l 25FFE
  \l 25FFF
  \l 26000
  \l 26001
  \l 26002
  \l 26003
  \l 26004
  \l 26005
  \l 26006
  \l 26007
  \l 26008
  \l 26009
  \l 2600A
  \l 2600B
  \l 2600C
  \l 2600D
  \l 2600E
  \l 2600F
  \l 26010
  \l 26011
  \l 26012
  \l 26013
  \l 26014
  \l 26015
  \l 26016
  \l 26017
  \l 26018
  \l 26019
  \l 2601A
  \l 2601B
  \l 2601C
  \l 2601D
  \l 2601E
  \l 2601F
  \l 26020
  \l 26021
  \l 26022
  \l 26023
  \l 26024
  \l 26025
  \l 26026
  \l 26027
  \l 26028
  \l 26029
  \l 2602A
  \l 2602B
  \l 2602C
  \l 2602D
  \l 2602E
  \l 2602F
  \l 26030
  \l 26031
  \l 26032
  \l 26033
  \l 26034
  \l 26035
  \l 26036
  \l 26037
  \l 26038
  \l 26039
  \l 2603A
  \l 2603B
  \l 2603C
  \l 2603D
  \l 2603E
  \l 2603F
  \l 26040
  \l 26041
  \l 26042
  \l 26043
  \l 26044
  \l 26045
  \l 26046
  \l 26047
  \l 26048
  \l 26049
  \l 2604A
  \l 2604B
  \l 2604C
  \l 2604D
  \l 2604E
  \l 2604F
  \l 26050
  \l 26051
  \l 26052
  \l 26053
  \l 26054
  \l 26055
  \l 26056
  \l 26057
  \l 26058
  \l 26059
  \l 2605A
  \l 2605B
  \l 2605C
  \l 2605D
  \l 2605E
  \l 2605F
  \l 26060
  \l 26061
  \l 26062
  \l 26063
  \l 26064
  \l 26065
  \l 26066
  \l 26067
  \l 26068
  \l 26069
  \l 2606A
  \l 2606B
  \l 2606C
  \l 2606D
  \l 2606E
  \l 2606F
  \l 26070
  \l 26071
  \l 26072
  \l 26073
  \l 26074
  \l 26075
  \l 26076
  \l 26077
  \l 26078
  \l 26079
  \l 2607A
  \l 2607B
  \l 2607C
  \l 2607D
  \l 2607E
  \l 2607F
  \l 26080
  \l 26081
  \l 26082
  \l 26083
  \l 26084
  \l 26085
  \l 26086
  \l 26087
  \l 26088
  \l 26089
  \l 2608A
  \l 2608B
  \l 2608C
  \l 2608D
  \l 2608E
  \l 2608F
  \l 26090
  \l 26091
  \l 26092
  \l 26093
  \l 26094
  \l 26095
  \l 26096
  \l 26097
  \l 26098
  \l 26099
  \l 2609A
  \l 2609B
  \l 2609C
  \l 2609D
  \l 2609E
  \l 2609F
  \l 260A0
  \l 260A1
  \l 260A2
  \l 260A3
  \l 260A4
  \l 260A5
  \l 260A6
  \l 260A7
  \l 260A8
  \l 260A9
  \l 260AA
  \l 260AB
  \l 260AC
  \l 260AD
  \l 260AE
  \l 260AF
  \l 260B0
  \l 260B1
  \l 260B2
  \l 260B3
  \l 260B4
  \l 260B5
  \l 260B6
  \l 260B7
  \l 260B8
  \l 260B9
  \l 260BA
  \l 260BB
  \l 260BC
  \l 260BD
  \l 260BE
  \l 260BF
  \l 260C0
  \l 260C1
  \l 260C2
  \l 260C3
  \l 260C4
  \l 260C5
  \l 260C6
  \l 260C7
  \l 260C8
  \l 260C9
  \l 260CA
  \l 260CB
  \l 260CC
  \l 260CD
  \l 260CE
  \l 260CF
  \l 260D0
  \l 260D1
  \l 260D2
  \l 260D3
  \l 260D4
  \l 260D5
  \l 260D6
  \l 260D7
  \l 260D8
  \l 260D9
  \l 260DA
  \l 260DB
  \l 260DC
  \l 260DD
  \l 260DE
  \l 260DF
  \l 260E0
  \l 260E1
  \l 260E2
  \l 260E3
  \l 260E4
  \l 260E5
  \l 260E6
  \l 260E7
  \l 260E8
  \l 260E9
  \l 260EA
  \l 260EB
  \l 260EC
  \l 260ED
  \l 260EE
  \l 260EF
  \l 260F0
  \l 260F1
  \l 260F2
  \l 260F3
  \l 260F4
  \l 260F5
  \l 260F6
  \l 260F7
  \l 260F8
  \l 260F9
  \l 260FA
  \l 260FB
  \l 260FC
  \l 260FD
  \l 260FE
  \l 260FF
  \l 26100
  \l 26101
  \l 26102
  \l 26103
  \l 26104
  \l 26105
  \l 26106
  \l 26107
  \l 26108
  \l 26109
  \l 2610A
  \l 2610B
  \l 2610C
  \l 2610D
  \l 2610E
  \l 2610F
  \l 26110
  \l 26111
  \l 26112
  \l 26113
  \l 26114
  \l 26115
  \l 26116
  \l 26117
  \l 26118
  \l 26119
  \l 2611A
  \l 2611B
  \l 2611C
  \l 2611D
  \l 2611E
  \l 2611F
  \l 26120
  \l 26121
  \l 26122
  \l 26123
  \l 26124
  \l 26125
  \l 26126
  \l 26127
  \l 26128
  \l 26129
  \l 2612A
  \l 2612B
  \l 2612C
  \l 2612D
  \l 2612E
  \l 2612F
  \l 26130
  \l 26131
  \l 26132
  \l 26133
  \l 26134
  \l 26135
  \l 26136
  \l 26137
  \l 26138
  \l 26139
  \l 2613A
  \l 2613B
  \l 2613C
  \l 2613D
  \l 2613E
  \l 2613F
  \l 26140
  \l 26141
  \l 26142
  \l 26143
  \l 26144
  \l 26145
  \l 26146
  \l 26147
  \l 26148
  \l 26149
  \l 2614A
  \l 2614B
  \l 2614C
  \l 2614D
  \l 2614E
  \l 2614F
  \l 26150
  \l 26151
  \l 26152
  \l 26153
  \l 26154
  \l 26155
  \l 26156
  \l 26157
  \l 26158
  \l 26159
  \l 2615A
  \l 2615B
  \l 2615C
  \l 2615D
  \l 2615E
  \l 2615F
  \l 26160
  \l 26161
  \l 26162
  \l 26163
  \l 26164
  \l 26165
  \l 26166
  \l 26167
  \l 26168
  \l 26169
  \l 2616A
  \l 2616B
  \l 2616C
  \l 2616D
  \l 2616E
  \l 2616F
  \l 26170
  \l 26171
  \l 26172
  \l 26173
  \l 26174
  \l 26175
  \l 26176
  \l 26177
  \l 26178
  \l 26179
  \l 2617A
  \l 2617B
  \l 2617C
  \l 2617D
  \l 2617E
  \l 2617F
  \l 26180
  \l 26181
  \l 26182
  \l 26183
  \l 26184
  \l 26185
  \l 26186
  \l 26187
  \l 26188
  \l 26189
  \l 2618A
  \l 2618B
  \l 2618C
  \l 2618D
  \l 2618E
  \l 2618F
  \l 26190
  \l 26191
  \l 26192
  \l 26193
  \l 26194
  \l 26195
  \l 26196
  \l 26197
  \l 26198
  \l 26199
  \l 2619A
  \l 2619B
  \l 2619C
  \l 2619D
  \l 2619E
  \l 2619F
  \l 261A0
  \l 261A1
  \l 261A2
  \l 261A3
  \l 261A4
  \l 261A5
  \l 261A6
  \l 261A7
  \l 261A8
  \l 261A9
  \l 261AA
  \l 261AB
  \l 261AC
  \l 261AD
  \l 261AE
  \l 261AF
  \l 261B0
  \l 261B1
  \l 261B2
  \l 261B3
  \l 261B4
  \l 261B5
  \l 261B6
  \l 261B7
  \l 261B8
  \l 261B9
  \l 261BA
  \l 261BB
  \l 261BC
  \l 261BD
  \l 261BE
  \l 261BF
  \l 261C0
  \l 261C1
  \l 261C2
  \l 261C3
  \l 261C4
  \l 261C5
  \l 261C6
  \l 261C7
  \l 261C8
  \l 261C9
  \l 261CA
  \l 261CB
  \l 261CC
  \l 261CD
  \l 261CE
  \l 261CF
  \l 261D0
  \l 261D1
  \l 261D2
  \l 261D3
  \l 261D4
  \l 261D5
  \l 261D6
  \l 261D7
  \l 261D8
  \l 261D9
  \l 261DA
  \l 261DB
  \l 261DC
  \l 261DD
  \l 261DE
  \l 261DF
  \l 261E0
  \l 261E1
  \l 261E2
  \l 261E3
  \l 261E4
  \l 261E5
  \l 261E6
  \l 261E7
  \l 261E8
  \l 261E9
  \l 261EA
  \l 261EB
  \l 261EC
  \l 261ED
  \l 261EE
  \l 261EF
  \l 261F0
  \l 261F1
  \l 261F2
  \l 261F3
  \l 261F4
  \l 261F5
  \l 261F6
  \l 261F7
  \l 261F8
  \l 261F9
  \l 261FA
  \l 261FB
  \l 261FC
  \l 261FD
  \l 261FE
  \l 261FF
  \l 26200
  \l 26201
  \l 26202
  \l 26203
  \l 26204
  \l 26205
  \l 26206
  \l 26207
  \l 26208
  \l 26209
  \l 2620A
  \l 2620B
  \l 2620C
  \l 2620D
  \l 2620E
  \l 2620F
  \l 26210
  \l 26211
  \l 26212
  \l 26213
  \l 26214
  \l 26215
  \l 26216
  \l 26217
  \l 26218
  \l 26219
  \l 2621A
  \l 2621B
  \l 2621C
  \l 2621D
  \l 2621E
  \l 2621F
  \l 26220
  \l 26221
  \l 26222
  \l 26223
  \l 26224
  \l 26225
  \l 26226
  \l 26227
  \l 26228
  \l 26229
  \l 2622A
  \l 2622B
  \l 2622C
  \l 2622D
  \l 2622E
  \l 2622F
  \l 26230
  \l 26231
  \l 26232
  \l 26233
  \l 26234
  \l 26235
  \l 26236
  \l 26237
  \l 26238
  \l 26239
  \l 2623A
  \l 2623B
  \l 2623C
  \l 2623D
  \l 2623E
  \l 2623F
  \l 26240
  \l 26241
  \l 26242
  \l 26243
  \l 26244
  \l 26245
  \l 26246
  \l 26247
  \l 26248
  \l 26249
  \l 2624A
  \l 2624B
  \l 2624C
  \l 2624D
  \l 2624E
  \l 2624F
  \l 26250
  \l 26251
  \l 26252
  \l 26253
  \l 26254
  \l 26255
  \l 26256
  \l 26257
  \l 26258
  \l 26259
  \l 2625A
  \l 2625B
  \l 2625C
  \l 2625D
  \l 2625E
  \l 2625F
  \l 26260
  \l 26261
  \l 26262
  \l 26263
  \l 26264
  \l 26265
  \l 26266
  \l 26267
  \l 26268
  \l 26269
  \l 2626A
  \l 2626B
  \l 2626C
  \l 2626D
  \l 2626E
  \l 2626F
  \l 26270
  \l 26271
  \l 26272
  \l 26273
  \l 26274
  \l 26275
  \l 26276
  \l 26277
  \l 26278
  \l 26279
  \l 2627A
  \l 2627B
  \l 2627C
  \l 2627D
  \l 2627E
  \l 2627F
  \l 26280
  \l 26281
  \l 26282
  \l 26283
  \l 26284
  \l 26285
  \l 26286
  \l 26287
  \l 26288
  \l 26289
  \l 2628A
  \l 2628B
  \l 2628C
  \l 2628D
  \l 2628E
  \l 2628F
  \l 26290
  \l 26291
  \l 26292
  \l 26293
  \l 26294
  \l 26295
  \l 26296
  \l 26297
  \l 26298
  \l 26299
  \l 2629A
  \l 2629B
  \l 2629C
  \l 2629D
  \l 2629E
  \l 2629F
  \l 262A0
  \l 262A1
  \l 262A2
  \l 262A3
  \l 262A4
  \l 262A5
  \l 262A6
  \l 262A7
  \l 262A8
  \l 262A9
  \l 262AA
  \l 262AB
  \l 262AC
  \l 262AD
  \l 262AE
  \l 262AF
  \l 262B0
  \l 262B1
  \l 262B2
  \l 262B3
  \l 262B4
  \l 262B5
  \l 262B6
  \l 262B7
  \l 262B8
  \l 262B9
  \l 262BA
  \l 262BB
  \l 262BC
  \l 262BD
  \l 262BE
  \l 262BF
  \l 262C0
  \l 262C1
  \l 262C2
  \l 262C3
  \l 262C4
  \l 262C5
  \l 262C6
  \l 262C7
  \l 262C8
  \l 262C9
  \l 262CA
  \l 262CB
  \l 262CC
  \l 262CD
  \l 262CE
  \l 262CF
  \l 262D0
  \l 262D1
  \l 262D2
  \l 262D3
  \l 262D4
  \l 262D5
  \l 262D6
  \l 262D7
  \l 262D8
  \l 262D9
  \l 262DA
  \l 262DB
  \l 262DC
  \l 262DD
  \l 262DE
  \l 262DF
  \l 262E0
  \l 262E1
  \l 262E2
  \l 262E3
  \l 262E4
  \l 262E5
  \l 262E6
  \l 262E7
  \l 262E8
  \l 262E9
  \l 262EA
  \l 262EB
  \l 262EC
  \l 262ED
  \l 262EE
  \l 262EF
  \l 262F0
  \l 262F1
  \l 262F2
  \l 262F3
  \l 262F4
  \l 262F5
  \l 262F6
  \l 262F7
  \l 262F8
  \l 262F9
  \l 262FA
  \l 262FB
  \l 262FC
  \l 262FD
  \l 262FE
  \l 262FF
  \l 26300
  \l 26301
  \l 26302
  \l 26303
  \l 26304
  \l 26305
  \l 26306
  \l 26307
  \l 26308
  \l 26309
  \l 2630A
  \l 2630B
  \l 2630C
  \l 2630D
  \l 2630E
  \l 2630F
  \l 26310
  \l 26311
  \l 26312
  \l 26313
  \l 26314
  \l 26315
  \l 26316
  \l 26317
  \l 26318
  \l 26319
  \l 2631A
  \l 2631B
  \l 2631C
  \l 2631D
  \l 2631E
  \l 2631F
  \l 26320
  \l 26321
  \l 26322
  \l 26323
  \l 26324
  \l 26325
  \l 26326
  \l 26327
  \l 26328
  \l 26329
  \l 2632A
  \l 2632B
  \l 2632C
  \l 2632D
  \l 2632E
  \l 2632F
  \l 26330
  \l 26331
  \l 26332
  \l 26333
  \l 26334
  \l 26335
  \l 26336
  \l 26337
  \l 26338
  \l 26339
  \l 2633A
  \l 2633B
  \l 2633C
  \l 2633D
  \l 2633E
  \l 2633F
  \l 26340
  \l 26341
  \l 26342
  \l 26343
  \l 26344
  \l 26345
  \l 26346
  \l 26347
  \l 26348
  \l 26349
  \l 2634A
  \l 2634B
  \l 2634C
  \l 2634D
  \l 2634E
  \l 2634F
  \l 26350
  \l 26351
  \l 26352
  \l 26353
  \l 26354
  \l 26355
  \l 26356
  \l 26357
  \l 26358
  \l 26359
  \l 2635A
  \l 2635B
  \l 2635C
  \l 2635D
  \l 2635E
  \l 2635F
  \l 26360
  \l 26361
  \l 26362
  \l 26363
  \l 26364
  \l 26365
  \l 26366
  \l 26367
  \l 26368
  \l 26369
  \l 2636A
  \l 2636B
  \l 2636C
  \l 2636D
  \l 2636E
  \l 2636F
  \l 26370
  \l 26371
  \l 26372
  \l 26373
  \l 26374
  \l 26375
  \l 26376
  \l 26377
  \l 26378
  \l 26379
  \l 2637A
  \l 2637B
  \l 2637C
  \l 2637D
  \l 2637E
  \l 2637F
  \l 26380
  \l 26381
  \l 26382
  \l 26383
  \l 26384
  \l 26385
  \l 26386
  \l 26387
  \l 26388
  \l 26389
  \l 2638A
  \l 2638B
  \l 2638C
  \l 2638D
  \l 2638E
  \l 2638F
  \l 26390
  \l 26391
  \l 26392
  \l 26393
  \l 26394
  \l 26395
  \l 26396
  \l 26397
  \l 26398
  \l 26399
  \l 2639A
  \l 2639B
  \l 2639C
  \l 2639D
  \l 2639E
  \l 2639F
  \l 263A0
  \l 263A1
  \l 263A2
  \l 263A3
  \l 263A4
  \l 263A5
  \l 263A6
  \l 263A7
  \l 263A8
  \l 263A9
  \l 263AA
  \l 263AB
  \l 263AC
  \l 263AD
  \l 263AE
  \l 263AF
  \l 263B0
  \l 263B1
  \l 263B2
  \l 263B3
  \l 263B4
  \l 263B5
  \l 263B6
  \l 263B7
  \l 263B8
  \l 263B9
  \l 263BA
  \l 263BB
  \l 263BC
  \l 263BD
  \l 263BE
  \l 263BF
  \l 263C0
  \l 263C1
  \l 263C2
  \l 263C3
  \l 263C4
  \l 263C5
  \l 263C6
  \l 263C7
  \l 263C8
  \l 263C9
  \l 263CA
  \l 263CB
  \l 263CC
  \l 263CD
  \l 263CE
  \l 263CF
  \l 263D0
  \l 263D1
  \l 263D2
  \l 263D3
  \l 263D4
  \l 263D5
  \l 263D6
  \l 263D7
  \l 263D8
  \l 263D9
  \l 263DA
  \l 263DB
  \l 263DC
  \l 263DD
  \l 263DE
  \l 263DF
  \l 263E0
  \l 263E1
  \l 263E2
  \l 263E3
  \l 263E4
  \l 263E5
  \l 263E6
  \l 263E7
  \l 263E8
  \l 263E9
  \l 263EA
  \l 263EB
  \l 263EC
  \l 263ED
  \l 263EE
  \l 263EF
  \l 263F0
  \l 263F1
  \l 263F2
  \l 263F3
  \l 263F4
  \l 263F5
  \l 263F6
  \l 263F7
  \l 263F8
  \l 263F9
  \l 263FA
  \l 263FB
  \l 263FC
  \l 263FD
  \l 263FE
  \l 263FF
  \l 26400
  \l 26401
  \l 26402
  \l 26403
  \l 26404
  \l 26405
  \l 26406
  \l 26407
  \l 26408
  \l 26409
  \l 2640A
  \l 2640B
  \l 2640C
  \l 2640D
  \l 2640E
  \l 2640F
  \l 26410
  \l 26411
  \l 26412
  \l 26413
  \l 26414
  \l 26415
  \l 26416
  \l 26417
  \l 26418
  \l 26419
  \l 2641A
  \l 2641B
  \l 2641C
  \l 2641D
  \l 2641E
  \l 2641F
  \l 26420
  \l 26421
  \l 26422
  \l 26423
  \l 26424
  \l 26425
  \l 26426
  \l 26427
  \l 26428
  \l 26429
  \l 2642A
  \l 2642B
  \l 2642C
  \l 2642D
  \l 2642E
  \l 2642F
  \l 26430
  \l 26431
  \l 26432
  \l 26433
  \l 26434
  \l 26435
  \l 26436
  \l 26437
  \l 26438
  \l 26439
  \l 2643A
  \l 2643B
  \l 2643C
  \l 2643D
  \l 2643E
  \l 2643F
  \l 26440
  \l 26441
  \l 26442
  \l 26443
  \l 26444
  \l 26445
  \l 26446
  \l 26447
  \l 26448
  \l 26449
  \l 2644A
  \l 2644B
  \l 2644C
  \l 2644D
  \l 2644E
  \l 2644F
  \l 26450
  \l 26451
  \l 26452
  \l 26453
  \l 26454
  \l 26455
  \l 26456
  \l 26457
  \l 26458
  \l 26459
  \l 2645A
  \l 2645B
  \l 2645C
  \l 2645D
  \l 2645E
  \l 2645F
  \l 26460
  \l 26461
  \l 26462
  \l 26463
  \l 26464
  \l 26465
  \l 26466
  \l 26467
  \l 26468
  \l 26469
  \l 2646A
  \l 2646B
  \l 2646C
  \l 2646D
  \l 2646E
  \l 2646F
  \l 26470
  \l 26471
  \l 26472
  \l 26473
  \l 26474
  \l 26475
  \l 26476
  \l 26477
  \l 26478
  \l 26479
  \l 2647A
  \l 2647B
  \l 2647C
  \l 2647D
  \l 2647E
  \l 2647F
  \l 26480
  \l 26481
  \l 26482
  \l 26483
  \l 26484
  \l 26485
  \l 26486
  \l 26487
  \l 26488
  \l 26489
  \l 2648A
  \l 2648B
  \l 2648C
  \l 2648D
  \l 2648E
  \l 2648F
  \l 26490
  \l 26491
  \l 26492
  \l 26493
  \l 26494
  \l 26495
  \l 26496
  \l 26497
  \l 26498
  \l 26499
  \l 2649A
  \l 2649B
  \l 2649C
  \l 2649D
  \l 2649E
  \l 2649F
  \l 264A0
  \l 264A1
  \l 264A2
  \l 264A3
  \l 264A4
  \l 264A5
  \l 264A6
  \l 264A7
  \l 264A8
  \l 264A9
  \l 264AA
  \l 264AB
  \l 264AC
  \l 264AD
  \l 264AE
  \l 264AF
  \l 264B0
  \l 264B1
  \l 264B2
  \l 264B3
  \l 264B4
  \l 264B5
  \l 264B6
  \l 264B7
  \l 264B8
  \l 264B9
  \l 264BA
  \l 264BB
  \l 264BC
  \l 264BD
  \l 264BE
  \l 264BF
  \l 264C0
  \l 264C1
  \l 264C2
  \l 264C3
  \l 264C4
  \l 264C5
  \l 264C6
  \l 264C7
  \l 264C8
  \l 264C9
  \l 264CA
  \l 264CB
  \l 264CC
  \l 264CD
  \l 264CE
  \l 264CF
  \l 264D0
  \l 264D1
  \l 264D2
  \l 264D3
  \l 264D4
  \l 264D5
  \l 264D6
  \l 264D7
  \l 264D8
  \l 264D9
  \l 264DA
  \l 264DB
  \l 264DC
  \l 264DD
  \l 264DE
  \l 264DF
  \l 264E0
  \l 264E1
  \l 264E2
  \l 264E3
  \l 264E4
  \l 264E5
  \l 264E6
  \l 264E7
  \l 264E8
  \l 264E9
  \l 264EA
  \l 264EB
  \l 264EC
  \l 264ED
  \l 264EE
  \l 264EF
  \l 264F0
  \l 264F1
  \l 264F2
  \l 264F3
  \l 264F4
  \l 264F5
  \l 264F6
  \l 264F7
  \l 264F8
  \l 264F9
  \l 264FA
  \l 264FB
  \l 264FC
  \l 264FD
  \l 264FE
  \l 264FF
  \l 26500
  \l 26501
  \l 26502
  \l 26503
  \l 26504
  \l 26505
  \l 26506
  \l 26507
  \l 26508
  \l 26509
  \l 2650A
  \l 2650B
  \l 2650C
  \l 2650D
  \l 2650E
  \l 2650F
  \l 26510
  \l 26511
  \l 26512
  \l 26513
  \l 26514
  \l 26515
  \l 26516
  \l 26517
  \l 26518
  \l 26519
  \l 2651A
  \l 2651B
  \l 2651C
  \l 2651D
  \l 2651E
  \l 2651F
  \l 26520
  \l 26521
  \l 26522
  \l 26523
  \l 26524
  \l 26525
  \l 26526
  \l 26527
  \l 26528
  \l 26529
  \l 2652A
  \l 2652B
  \l 2652C
  \l 2652D
  \l 2652E
  \l 2652F
  \l 26530
  \l 26531
  \l 26532
  \l 26533
  \l 26534
  \l 26535
  \l 26536
  \l 26537
  \l 26538
  \l 26539
  \l 2653A
  \l 2653B
  \l 2653C
  \l 2653D
  \l 2653E
  \l 2653F
  \l 26540
  \l 26541
  \l 26542
  \l 26543
  \l 26544
  \l 26545
  \l 26546
  \l 26547
  \l 26548
  \l 26549
  \l 2654A
  \l 2654B
  \l 2654C
  \l 2654D
  \l 2654E
  \l 2654F
  \l 26550
  \l 26551
  \l 26552
  \l 26553
  \l 26554
  \l 26555
  \l 26556
  \l 26557
  \l 26558
  \l 26559
  \l 2655A
  \l 2655B
  \l 2655C
  \l 2655D
  \l 2655E
  \l 2655F
  \l 26560
  \l 26561
  \l 26562
  \l 26563
  \l 26564
  \l 26565
  \l 26566
  \l 26567
  \l 26568
  \l 26569
  \l 2656A
  \l 2656B
  \l 2656C
  \l 2656D
  \l 2656E
  \l 2656F
  \l 26570
  \l 26571
  \l 26572
  \l 26573
  \l 26574
  \l 26575
  \l 26576
  \l 26577
  \l 26578
  \l 26579
  \l 2657A
  \l 2657B
  \l 2657C
  \l 2657D
  \l 2657E
  \l 2657F
  \l 26580
  \l 26581
  \l 26582
  \l 26583
  \l 26584
  \l 26585
  \l 26586
  \l 26587
  \l 26588
  \l 26589
  \l 2658A
  \l 2658B
  \l 2658C
  \l 2658D
  \l 2658E
  \l 2658F
  \l 26590
  \l 26591
  \l 26592
  \l 26593
  \l 26594
  \l 26595
  \l 26596
  \l 26597
  \l 26598
  \l 26599
  \l 2659A
  \l 2659B
  \l 2659C
  \l 2659D
  \l 2659E
  \l 2659F
  \l 265A0
  \l 265A1
  \l 265A2
  \l 265A3
  \l 265A4
  \l 265A5
  \l 265A6
  \l 265A7
  \l 265A8
  \l 265A9
  \l 265AA
  \l 265AB
  \l 265AC
  \l 265AD
  \l 265AE
  \l 265AF
  \l 265B0
  \l 265B1
  \l 265B2
  \l 265B3
  \l 265B4
  \l 265B5
  \l 265B6
  \l 265B7
  \l 265B8
  \l 265B9
  \l 265BA
  \l 265BB
  \l 265BC
  \l 265BD
  \l 265BE
  \l 265BF
  \l 265C0
  \l 265C1
  \l 265C2
  \l 265C3
  \l 265C4
  \l 265C5
  \l 265C6
  \l 265C7
  \l 265C8
  \l 265C9
  \l 265CA
  \l 265CB
  \l 265CC
  \l 265CD
  \l 265CE
  \l 265CF
  \l 265D0
  \l 265D1
  \l 265D2
  \l 265D3
  \l 265D4
  \l 265D5
  \l 265D6
  \l 265D7
  \l 265D8
  \l 265D9
  \l 265DA
  \l 265DB
  \l 265DC
  \l 265DD
  \l 265DE
  \l 265DF
  \l 265E0
  \l 265E1
  \l 265E2
  \l 265E3
  \l 265E4
  \l 265E5
  \l 265E6
  \l 265E7
  \l 265E8
  \l 265E9
  \l 265EA
  \l 265EB
  \l 265EC
  \l 265ED
  \l 265EE
  \l 265EF
  \l 265F0
  \l 265F1
  \l 265F2
  \l 265F3
  \l 265F4
  \l 265F5
  \l 265F6
  \l 265F7
  \l 265F8
  \l 265F9
  \l 265FA
  \l 265FB
  \l 265FC
  \l 265FD
  \l 265FE
  \l 265FF
  \l 26600
  \l 26601
  \l 26602
  \l 26603
  \l 26604
  \l 26605
  \l 26606
  \l 26607
  \l 26608
  \l 26609
  \l 2660A
  \l 2660B
  \l 2660C
  \l 2660D
  \l 2660E
  \l 2660F
  \l 26610
  \l 26611
  \l 26612
  \l 26613
  \l 26614
  \l 26615
  \l 26616
  \l 26617
  \l 26618
  \l 26619
  \l 2661A
  \l 2661B
  \l 2661C
  \l 2661D
  \l 2661E
  \l 2661F
  \l 26620
  \l 26621
  \l 26622
  \l 26623
  \l 26624
  \l 26625
  \l 26626
  \l 26627
  \l 26628
  \l 26629
  \l 2662A
  \l 2662B
  \l 2662C
  \l 2662D
  \l 2662E
  \l 2662F
  \l 26630
  \l 26631
  \l 26632
  \l 26633
  \l 26634
  \l 26635
  \l 26636
  \l 26637
  \l 26638
  \l 26639
  \l 2663A
  \l 2663B
  \l 2663C
  \l 2663D
  \l 2663E
  \l 2663F
  \l 26640
  \l 26641
  \l 26642
  \l 26643
  \l 26644
  \l 26645
  \l 26646
  \l 26647
  \l 26648
  \l 26649
  \l 2664A
  \l 2664B
  \l 2664C
  \l 2664D
  \l 2664E
  \l 2664F
  \l 26650
  \l 26651
  \l 26652
  \l 26653
  \l 26654
  \l 26655
  \l 26656
  \l 26657
  \l 26658
  \l 26659
  \l 2665A
  \l 2665B
  \l 2665C
  \l 2665D
  \l 2665E
  \l 2665F
  \l 26660
  \l 26661
  \l 26662
  \l 26663
  \l 26664
  \l 26665
  \l 26666
  \l 26667
  \l 26668
  \l 26669
  \l 2666A
  \l 2666B
  \l 2666C
  \l 2666D
  \l 2666E
  \l 2666F
  \l 26670
  \l 26671
  \l 26672
  \l 26673
  \l 26674
  \l 26675
  \l 26676
  \l 26677
  \l 26678
  \l 26679
  \l 2667A
  \l 2667B
  \l 2667C
  \l 2667D
  \l 2667E
  \l 2667F
  \l 26680
  \l 26681
  \l 26682
  \l 26683
  \l 26684
  \l 26685
  \l 26686
  \l 26687
  \l 26688
  \l 26689
  \l 2668A
  \l 2668B
  \l 2668C
  \l 2668D
  \l 2668E
  \l 2668F
  \l 26690
  \l 26691
  \l 26692
  \l 26693
  \l 26694
  \l 26695
  \l 26696
  \l 26697
  \l 26698
  \l 26699
  \l 2669A
  \l 2669B
  \l 2669C
  \l 2669D
  \l 2669E
  \l 2669F
  \l 266A0
  \l 266A1
  \l 266A2
  \l 266A3
  \l 266A4
  \l 266A5
  \l 266A6
  \l 266A7
  \l 266A8
  \l 266A9
  \l 266AA
  \l 266AB
  \l 266AC
  \l 266AD
  \l 266AE
  \l 266AF
  \l 266B0
  \l 266B1
  \l 266B2
  \l 266B3
  \l 266B4
  \l 266B5
  \l 266B6
  \l 266B7
  \l 266B8
  \l 266B9
  \l 266BA
  \l 266BB
  \l 266BC
  \l 266BD
  \l 266BE
  \l 266BF
  \l 266C0
  \l 266C1
  \l 266C2
  \l 266C3
  \l 266C4
  \l 266C5
  \l 266C6
  \l 266C7
  \l 266C8
  \l 266C9
  \l 266CA
  \l 266CB
  \l 266CC
  \l 266CD
  \l 266CE
  \l 266CF
  \l 266D0
  \l 266D1
  \l 266D2
  \l 266D3
  \l 266D4
  \l 266D5
  \l 266D6
  \l 266D7
  \l 266D8
  \l 266D9
  \l 266DA
  \l 266DB
  \l 266DC
  \l 266DD
  \l 266DE
  \l 266DF
  \l 266E0
  \l 266E1
  \l 266E2
  \l 266E3
  \l 266E4
  \l 266E5
  \l 266E6
  \l 266E7
  \l 266E8
  \l 266E9
  \l 266EA
  \l 266EB
  \l 266EC
  \l 266ED
  \l 266EE
  \l 266EF
  \l 266F0
  \l 266F1
  \l 266F2
  \l 266F3
  \l 266F4
  \l 266F5
  \l 266F6
  \l 266F7
  \l 266F8
  \l 266F9
  \l 266FA
  \l 266FB
  \l 266FC
  \l 266FD
  \l 266FE
  \l 266FF
  \l 26700
  \l 26701
  \l 26702
  \l 26703
  \l 26704
  \l 26705
  \l 26706
  \l 26707
  \l 26708
  \l 26709
  \l 2670A
  \l 2670B
  \l 2670C
  \l 2670D
  \l 2670E
  \l 2670F
  \l 26710
  \l 26711
  \l 26712
  \l 26713
  \l 26714
  \l 26715
  \l 26716
  \l 26717
  \l 26718
  \l 26719
  \l 2671A
  \l 2671B
  \l 2671C
  \l 2671D
  \l 2671E
  \l 2671F
  \l 26720
  \l 26721
  \l 26722
  \l 26723
  \l 26724
  \l 26725
  \l 26726
  \l 26727
  \l 26728
  \l 26729
  \l 2672A
  \l 2672B
  \l 2672C
  \l 2672D
  \l 2672E
  \l 2672F
  \l 26730
  \l 26731
  \l 26732
  \l 26733
  \l 26734
  \l 26735
  \l 26736
  \l 26737
  \l 26738
  \l 26739
  \l 2673A
  \l 2673B
  \l 2673C
  \l 2673D
  \l 2673E
  \l 2673F
  \l 26740
  \l 26741
  \l 26742
  \l 26743
  \l 26744
  \l 26745
  \l 26746
  \l 26747
  \l 26748
  \l 26749
  \l 2674A
  \l 2674B
  \l 2674C
  \l 2674D
  \l 2674E
  \l 2674F
  \l 26750
  \l 26751
  \l 26752
  \l 26753
  \l 26754
  \l 26755
  \l 26756
  \l 26757
  \l 26758
  \l 26759
  \l 2675A
  \l 2675B
  \l 2675C
  \l 2675D
  \l 2675E
  \l 2675F
  \l 26760
  \l 26761
  \l 26762
  \l 26763
  \l 26764
  \l 26765
  \l 26766
  \l 26767
  \l 26768
  \l 26769
  \l 2676A
  \l 2676B
  \l 2676C
  \l 2676D
  \l 2676E
  \l 2676F
  \l 26770
  \l 26771
  \l 26772
  \l 26773
  \l 26774
  \l 26775
  \l 26776
  \l 26777
  \l 26778
  \l 26779
  \l 2677A
  \l 2677B
  \l 2677C
  \l 2677D
  \l 2677E
  \l 2677F
  \l 26780
  \l 26781
  \l 26782
  \l 26783
  \l 26784
  \l 26785
  \l 26786
  \l 26787
  \l 26788
  \l 26789
  \l 2678A
  \l 2678B
  \l 2678C
  \l 2678D
  \l 2678E
  \l 2678F
  \l 26790
  \l 26791
  \l 26792
  \l 26793
  \l 26794
  \l 26795
  \l 26796
  \l 26797
  \l 26798
  \l 26799
  \l 2679A
  \l 2679B
  \l 2679C
  \l 2679D
  \l 2679E
  \l 2679F
  \l 267A0
  \l 267A1
  \l 267A2
  \l 267A3
  \l 267A4
  \l 267A5
  \l 267A6
  \l 267A7
  \l 267A8
  \l 267A9
  \l 267AA
  \l 267AB
  \l 267AC
  \l 267AD
  \l 267AE
  \l 267AF
  \l 267B0
  \l 267B1
  \l 267B2
  \l 267B3
  \l 267B4
  \l 267B5
  \l 267B6
  \l 267B7
  \l 267B8
  \l 267B9
  \l 267BA
  \l 267BB
  \l 267BC
  \l 267BD
  \l 267BE
  \l 267BF
  \l 267C0
  \l 267C1
  \l 267C2
  \l 267C3
  \l 267C4
  \l 267C5
  \l 267C6
  \l 267C7
  \l 267C8
  \l 267C9
  \l 267CA
  \l 267CB
  \l 267CC
  \l 267CD
  \l 267CE
  \l 267CF
  \l 267D0
  \l 267D1
  \l 267D2
  \l 267D3
  \l 267D4
  \l 267D5
  \l 267D6
  \l 267D7
  \l 267D8
  \l 267D9
  \l 267DA
  \l 267DB
  \l 267DC
  \l 267DD
  \l 267DE
  \l 267DF
  \l 267E0
  \l 267E1
  \l 267E2
  \l 267E3
  \l 267E4
  \l 267E5
  \l 267E6
  \l 267E7
  \l 267E8
  \l 267E9
  \l 267EA
  \l 267EB
  \l 267EC
  \l 267ED
  \l 267EE
  \l 267EF
  \l 267F0
  \l 267F1
  \l 267F2
  \l 267F3
  \l 267F4
  \l 267F5
  \l 267F6
  \l 267F7
  \l 267F8
  \l 267F9
  \l 267FA
  \l 267FB
  \l 267FC
  \l 267FD
  \l 267FE
  \l 267FF
  \l 26800
  \l 26801
  \l 26802
  \l 26803
  \l 26804
  \l 26805
  \l 26806
  \l 26807
  \l 26808
  \l 26809
  \l 2680A
  \l 2680B
  \l 2680C
  \l 2680D
  \l 2680E
  \l 2680F
  \l 26810
  \l 26811
  \l 26812
  \l 26813
  \l 26814
  \l 26815
  \l 26816
  \l 26817
  \l 26818
  \l 26819
  \l 2681A
  \l 2681B
  \l 2681C
  \l 2681D
  \l 2681E
  \l 2681F
  \l 26820
  \l 26821
  \l 26822
  \l 26823
  \l 26824
  \l 26825
  \l 26826
  \l 26827
  \l 26828
  \l 26829
  \l 2682A
  \l 2682B
  \l 2682C
  \l 2682D
  \l 2682E
  \l 2682F
  \l 26830
  \l 26831
  \l 26832
  \l 26833
  \l 26834
  \l 26835
  \l 26836
  \l 26837
  \l 26838
  \l 26839
  \l 2683A
  \l 2683B
  \l 2683C
  \l 2683D
  \l 2683E
  \l 2683F
  \l 26840
  \l 26841
  \l 26842
  \l 26843
  \l 26844
  \l 26845
  \l 26846
  \l 26847
  \l 26848
  \l 26849
  \l 2684A
  \l 2684B
  \l 2684C
  \l 2684D
  \l 2684E
  \l 2684F
  \l 26850
  \l 26851
  \l 26852
  \l 26853
  \l 26854
  \l 26855
  \l 26856
  \l 26857
  \l 26858
  \l 26859
  \l 2685A
  \l 2685B
  \l 2685C
  \l 2685D
  \l 2685E
  \l 2685F
  \l 26860
  \l 26861
  \l 26862
  \l 26863
  \l 26864
  \l 26865
  \l 26866
  \l 26867
  \l 26868
  \l 26869
  \l 2686A
  \l 2686B
  \l 2686C
  \l 2686D
  \l 2686E
  \l 2686F
  \l 26870
  \l 26871
  \l 26872
  \l 26873
  \l 26874
  \l 26875
  \l 26876
  \l 26877
  \l 26878
  \l 26879
  \l 2687A
  \l 2687B
  \l 2687C
  \l 2687D
  \l 2687E
  \l 2687F
  \l 26880
  \l 26881
  \l 26882
  \l 26883
  \l 26884
  \l 26885
  \l 26886
  \l 26887
  \l 26888
  \l 26889
  \l 2688A
  \l 2688B
  \l 2688C
  \l 2688D
  \l 2688E
  \l 2688F
  \l 26890
  \l 26891
  \l 26892
  \l 26893
  \l 26894
  \l 26895
  \l 26896
  \l 26897
  \l 26898
  \l 26899
  \l 2689A
  \l 2689B
  \l 2689C
  \l 2689D
  \l 2689E
  \l 2689F
  \l 268A0
  \l 268A1
  \l 268A2
  \l 268A3
  \l 268A4
  \l 268A5
  \l 268A6
  \l 268A7
  \l 268A8
  \l 268A9
  \l 268AA
  \l 268AB
  \l 268AC
  \l 268AD
  \l 268AE
  \l 268AF
  \l 268B0
  \l 268B1
  \l 268B2
  \l 268B3
  \l 268B4
  \l 268B5
  \l 268B6
  \l 268B7
  \l 268B8
  \l 268B9
  \l 268BA
  \l 268BB
  \l 268BC
  \l 268BD
  \l 268BE
  \l 268BF
  \l 268C0
  \l 268C1
  \l 268C2
  \l 268C3
  \l 268C4
  \l 268C5
  \l 268C6
  \l 268C7
  \l 268C8
  \l 268C9
  \l 268CA
  \l 268CB
  \l 268CC
  \l 268CD
  \l 268CE
  \l 268CF
  \l 268D0
  \l 268D1
  \l 268D2
  \l 268D3
  \l 268D4
  \l 268D5
  \l 268D6
  \l 268D7
  \l 268D8
  \l 268D9
  \l 268DA
  \l 268DB
  \l 268DC
  \l 268DD
  \l 268DE
  \l 268DF
  \l 268E0
  \l 268E1
  \l 268E2
  \l 268E3
  \l 268E4
  \l 268E5
  \l 268E6
  \l 268E7
  \l 268E8
  \l 268E9
  \l 268EA
  \l 268EB
  \l 268EC
  \l 268ED
  \l 268EE
  \l 268EF
  \l 268F0
  \l 268F1
  \l 268F2
  \l 268F3
  \l 268F4
  \l 268F5
  \l 268F6
  \l 268F7
  \l 268F8
  \l 268F9
  \l 268FA
  \l 268FB
  \l 268FC
  \l 268FD
  \l 268FE
  \l 268FF
  \l 26900
  \l 26901
  \l 26902
  \l 26903
  \l 26904
  \l 26905
  \l 26906
  \l 26907
  \l 26908
  \l 26909
  \l 2690A
  \l 2690B
  \l 2690C
  \l 2690D
  \l 2690E
  \l 2690F
  \l 26910
  \l 26911
  \l 26912
  \l 26913
  \l 26914
  \l 26915
  \l 26916
  \l 26917
  \l 26918
  \l 26919
  \l 2691A
  \l 2691B
  \l 2691C
  \l 2691D
  \l 2691E
  \l 2691F
  \l 26920
  \l 26921
  \l 26922
  \l 26923
  \l 26924
  \l 26925
  \l 26926
  \l 26927
  \l 26928
  \l 26929
  \l 2692A
  \l 2692B
  \l 2692C
  \l 2692D
  \l 2692E
  \l 2692F
  \l 26930
  \l 26931
  \l 26932
  \l 26933
  \l 26934
  \l 26935
  \l 26936
  \l 26937
  \l 26938
  \l 26939
  \l 2693A
  \l 2693B
  \l 2693C
  \l 2693D
  \l 2693E
  \l 2693F
  \l 26940
  \l 26941
  \l 26942
  \l 26943
  \l 26944
  \l 26945
  \l 26946
  \l 26947
  \l 26948
  \l 26949
  \l 2694A
  \l 2694B
  \l 2694C
  \l 2694D
  \l 2694E
  \l 2694F
  \l 26950
  \l 26951
  \l 26952
  \l 26953
  \l 26954
  \l 26955
  \l 26956
  \l 26957
  \l 26958
  \l 26959
  \l 2695A
  \l 2695B
  \l 2695C
  \l 2695D
  \l 2695E
  \l 2695F
  \l 26960
  \l 26961
  \l 26962
  \l 26963
  \l 26964
  \l 26965
  \l 26966
  \l 26967
  \l 26968
  \l 26969
  \l 2696A
  \l 2696B
  \l 2696C
  \l 2696D
  \l 2696E
  \l 2696F
  \l 26970
  \l 26971
  \l 26972
  \l 26973
  \l 26974
  \l 26975
  \l 26976
  \l 26977
  \l 26978
  \l 26979
  \l 2697A
  \l 2697B
  \l 2697C
  \l 2697D
  \l 2697E
  \l 2697F
  \l 26980
  \l 26981
  \l 26982
  \l 26983
  \l 26984
  \l 26985
  \l 26986
  \l 26987
  \l 26988
  \l 26989
  \l 2698A
  \l 2698B
  \l 2698C
  \l 2698D
  \l 2698E
  \l 2698F
  \l 26990
  \l 26991
  \l 26992
  \l 26993
  \l 26994
  \l 26995
  \l 26996
  \l 26997
  \l 26998
  \l 26999
  \l 2699A
  \l 2699B
  \l 2699C
  \l 2699D
  \l 2699E
  \l 2699F
  \l 269A0
  \l 269A1
  \l 269A2
  \l 269A3
  \l 269A4
  \l 269A5
  \l 269A6
  \l 269A7
  \l 269A8
  \l 269A9
  \l 269AA
  \l 269AB
  \l 269AC
  \l 269AD
  \l 269AE
  \l 269AF
  \l 269B0
  \l 269B1
  \l 269B2
  \l 269B3
  \l 269B4
  \l 269B5
  \l 269B6
  \l 269B7
  \l 269B8
  \l 269B9
  \l 269BA
  \l 269BB
  \l 269BC
  \l 269BD
  \l 269BE
  \l 269BF
  \l 269C0
  \l 269C1
  \l 269C2
  \l 269C3
  \l 269C4
  \l 269C5
  \l 269C6
  \l 269C7
  \l 269C8
  \l 269C9
  \l 269CA
  \l 269CB
  \l 269CC
  \l 269CD
  \l 269CE
  \l 269CF
  \l 269D0
  \l 269D1
  \l 269D2
  \l 269D3
  \l 269D4
  \l 269D5
  \l 269D6
  \l 269D7
  \l 269D8
  \l 269D9
  \l 269DA
  \l 269DB
  \l 269DC
  \l 269DD
  \l 269DE
  \l 269DF
  \l 269E0
  \l 269E1
  \l 269E2
  \l 269E3
  \l 269E4
  \l 269E5
  \l 269E6
  \l 269E7
  \l 269E8
  \l 269E9
  \l 269EA
  \l 269EB
  \l 269EC
  \l 269ED
  \l 269EE
  \l 269EF
  \l 269F0
  \l 269F1
  \l 269F2
  \l 269F3
  \l 269F4
  \l 269F5
  \l 269F6
  \l 269F7
  \l 269F8
  \l 269F9
  \l 269FA
  \l 269FB
  \l 269FC
  \l 269FD
  \l 269FE
  \l 269FF
  \l 26A00
  \l 26A01
  \l 26A02
  \l 26A03
  \l 26A04
  \l 26A05
  \l 26A06
  \l 26A07
  \l 26A08
  \l 26A09
  \l 26A0A
  \l 26A0B
  \l 26A0C
  \l 26A0D
  \l 26A0E
  \l 26A0F
  \l 26A10
  \l 26A11
  \l 26A12
  \l 26A13
  \l 26A14
  \l 26A15
  \l 26A16
  \l 26A17
  \l 26A18
  \l 26A19
  \l 26A1A
  \l 26A1B
  \l 26A1C
  \l 26A1D
  \l 26A1E
  \l 26A1F
  \l 26A20
  \l 26A21
  \l 26A22
  \l 26A23
  \l 26A24
  \l 26A25
  \l 26A26
  \l 26A27
  \l 26A28
  \l 26A29
  \l 26A2A
  \l 26A2B
  \l 26A2C
  \l 26A2D
  \l 26A2E
  \l 26A2F
  \l 26A30
  \l 26A31
  \l 26A32
  \l 26A33
  \l 26A34
  \l 26A35
  \l 26A36
  \l 26A37
  \l 26A38
  \l 26A39
  \l 26A3A
  \l 26A3B
  \l 26A3C
  \l 26A3D
  \l 26A3E
  \l 26A3F
  \l 26A40
  \l 26A41
  \l 26A42
  \l 26A43
  \l 26A44
  \l 26A45
  \l 26A46
  \l 26A47
  \l 26A48
  \l 26A49
  \l 26A4A
  \l 26A4B
  \l 26A4C
  \l 26A4D
  \l 26A4E
  \l 26A4F
  \l 26A50
  \l 26A51
  \l 26A52
  \l 26A53
  \l 26A54
  \l 26A55
  \l 26A56
  \l 26A57
  \l 26A58
  \l 26A59
  \l 26A5A
  \l 26A5B
  \l 26A5C
  \l 26A5D
  \l 26A5E
  \l 26A5F
  \l 26A60
  \l 26A61
  \l 26A62
  \l 26A63
  \l 26A64
  \l 26A65
  \l 26A66
  \l 26A67
  \l 26A68
  \l 26A69
  \l 26A6A
  \l 26A6B
  \l 26A6C
  \l 26A6D
  \l 26A6E
  \l 26A6F
  \l 26A70
  \l 26A71
  \l 26A72
  \l 26A73
  \l 26A74
  \l 26A75
  \l 26A76
  \l 26A77
  \l 26A78
  \l 26A79
  \l 26A7A
  \l 26A7B
  \l 26A7C
  \l 26A7D
  \l 26A7E
  \l 26A7F
  \l 26A80
  \l 26A81
  \l 26A82
  \l 26A83
  \l 26A84
  \l 26A85
  \l 26A86
  \l 26A87
  \l 26A88
  \l 26A89
  \l 26A8A
  \l 26A8B
  \l 26A8C
  \l 26A8D
  \l 26A8E
  \l 26A8F
  \l 26A90
  \l 26A91
  \l 26A92
  \l 26A93
  \l 26A94
  \l 26A95
  \l 26A96
  \l 26A97
  \l 26A98
  \l 26A99
  \l 26A9A
  \l 26A9B
  \l 26A9C
  \l 26A9D
  \l 26A9E
  \l 26A9F
  \l 26AA0
  \l 26AA1
  \l 26AA2
  \l 26AA3
  \l 26AA4
  \l 26AA5
  \l 26AA6
  \l 26AA7
  \l 26AA8
  \l 26AA9
  \l 26AAA
  \l 26AAB
  \l 26AAC
  \l 26AAD
  \l 26AAE
  \l 26AAF
  \l 26AB0
  \l 26AB1
  \l 26AB2
  \l 26AB3
  \l 26AB4
  \l 26AB5
  \l 26AB6
  \l 26AB7
  \l 26AB8
  \l 26AB9
  \l 26ABA
  \l 26ABB
  \l 26ABC
  \l 26ABD
  \l 26ABE
  \l 26ABF
  \l 26AC0
  \l 26AC1
  \l 26AC2
  \l 26AC3
  \l 26AC4
  \l 26AC5
  \l 26AC6
  \l 26AC7
  \l 26AC8
  \l 26AC9
  \l 26ACA
  \l 26ACB
  \l 26ACC
  \l 26ACD
  \l 26ACE
  \l 26ACF
  \l 26AD0
  \l 26AD1
  \l 26AD2
  \l 26AD3
  \l 26AD4
  \l 26AD5
  \l 26AD6
  \l 26AD7
  \l 26AD8
  \l 26AD9
  \l 26ADA
  \l 26ADB
  \l 26ADC
  \l 26ADD
  \l 26ADE
  \l 26ADF
  \l 26AE0
  \l 26AE1
  \l 26AE2
  \l 26AE3
  \l 26AE4
  \l 26AE5
  \l 26AE6
  \l 26AE7
  \l 26AE8
  \l 26AE9
  \l 26AEA
  \l 26AEB
  \l 26AEC
  \l 26AED
  \l 26AEE
  \l 26AEF
  \l 26AF0
  \l 26AF1
  \l 26AF2
  \l 26AF3
  \l 26AF4
  \l 26AF5
  \l 26AF6
  \l 26AF7
  \l 26AF8
  \l 26AF9
  \l 26AFA
  \l 26AFB
  \l 26AFC
  \l 26AFD
  \l 26AFE
  \l 26AFF
  \l 26B00
  \l 26B01
  \l 26B02
  \l 26B03
  \l 26B04
  \l 26B05
  \l 26B06
  \l 26B07
  \l 26B08
  \l 26B09
  \l 26B0A
  \l 26B0B
  \l 26B0C
  \l 26B0D
  \l 26B0E
  \l 26B0F
  \l 26B10
  \l 26B11
  \l 26B12
  \l 26B13
  \l 26B14
  \l 26B15
  \l 26B16
  \l 26B17
  \l 26B18
  \l 26B19
  \l 26B1A
  \l 26B1B
  \l 26B1C
  \l 26B1D
  \l 26B1E
  \l 26B1F
  \l 26B20
  \l 26B21
  \l 26B22
  \l 26B23
  \l 26B24
  \l 26B25
  \l 26B26
  \l 26B27
  \l 26B28
  \l 26B29
  \l 26B2A
  \l 26B2B
  \l 26B2C
  \l 26B2D
  \l 26B2E
  \l 26B2F
  \l 26B30
  \l 26B31
  \l 26B32
  \l 26B33
  \l 26B34
  \l 26B35
  \l 26B36
  \l 26B37
  \l 26B38
  \l 26B39
  \l 26B3A
  \l 26B3B
  \l 26B3C
  \l 26B3D
  \l 26B3E
  \l 26B3F
  \l 26B40
  \l 26B41
  \l 26B42
  \l 26B43
  \l 26B44
  \l 26B45
  \l 26B46
  \l 26B47
  \l 26B48
  \l 26B49
  \l 26B4A
  \l 26B4B
  \l 26B4C
  \l 26B4D
  \l 26B4E
  \l 26B4F
  \l 26B50
  \l 26B51
  \l 26B52
  \l 26B53
  \l 26B54
  \l 26B55
  \l 26B56
  \l 26B57
  \l 26B58
  \l 26B59
  \l 26B5A
  \l 26B5B
  \l 26B5C
  \l 26B5D
  \l 26B5E
  \l 26B5F
  \l 26B60
  \l 26B61
  \l 26B62
  \l 26B63
  \l 26B64
  \l 26B65
  \l 26B66
  \l 26B67
  \l 26B68
  \l 26B69
  \l 26B6A
  \l 26B6B
  \l 26B6C
  \l 26B6D
  \l 26B6E
  \l 26B6F
  \l 26B70
  \l 26B71
  \l 26B72
  \l 26B73
  \l 26B74
  \l 26B75
  \l 26B76
  \l 26B77
  \l 26B78
  \l 26B79
  \l 26B7A
  \l 26B7B
  \l 26B7C
  \l 26B7D
  \l 26B7E
  \l 26B7F
  \l 26B80
  \l 26B81
  \l 26B82
  \l 26B83
  \l 26B84
  \l 26B85
  \l 26B86
  \l 26B87
  \l 26B88
  \l 26B89
  \l 26B8A
  \l 26B8B
  \l 26B8C
  \l 26B8D
  \l 26B8E
  \l 26B8F
  \l 26B90
  \l 26B91
  \l 26B92
  \l 26B93
  \l 26B94
  \l 26B95
  \l 26B96
  \l 26B97
  \l 26B98
  \l 26B99
  \l 26B9A
  \l 26B9B
  \l 26B9C
  \l 26B9D
  \l 26B9E
  \l 26B9F
  \l 26BA0
  \l 26BA1
  \l 26BA2
  \l 26BA3
  \l 26BA4
  \l 26BA5
  \l 26BA6
  \l 26BA7
  \l 26BA8
  \l 26BA9
  \l 26BAA
  \l 26BAB
  \l 26BAC
  \l 26BAD
  \l 26BAE
  \l 26BAF
  \l 26BB0
  \l 26BB1
  \l 26BB2
  \l 26BB3
  \l 26BB4
  \l 26BB5
  \l 26BB6
  \l 26BB7
  \l 26BB8
  \l 26BB9
  \l 26BBA
  \l 26BBB
  \l 26BBC
  \l 26BBD
  \l 26BBE
  \l 26BBF
  \l 26BC0
  \l 26BC1
  \l 26BC2
  \l 26BC3
  \l 26BC4
  \l 26BC5
  \l 26BC6
  \l 26BC7
  \l 26BC8
  \l 26BC9
  \l 26BCA
  \l 26BCB
  \l 26BCC
  \l 26BCD
  \l 26BCE
  \l 26BCF
  \l 26BD0
  \l 26BD1
  \l 26BD2
  \l 26BD3
  \l 26BD4
  \l 26BD5
  \l 26BD6
  \l 26BD7
  \l 26BD8
  \l 26BD9
  \l 26BDA
  \l 26BDB
  \l 26BDC
  \l 26BDD
  \l 26BDE
  \l 26BDF
  \l 26BE0
  \l 26BE1
  \l 26BE2
  \l 26BE3
  \l 26BE4
  \l 26BE5
  \l 26BE6
  \l 26BE7
  \l 26BE8
  \l 26BE9
  \l 26BEA
  \l 26BEB
  \l 26BEC
  \l 26BED
  \l 26BEE
  \l 26BEF
  \l 26BF0
  \l 26BF1
  \l 26BF2
  \l 26BF3
  \l 26BF4
  \l 26BF5
  \l 26BF6
  \l 26BF7
  \l 26BF8
  \l 26BF9
  \l 26BFA
  \l 26BFB
  \l 26BFC
  \l 26BFD
  \l 26BFE
  \l 26BFF
  \l 26C00
  \l 26C01
  \l 26C02
  \l 26C03
  \l 26C04
  \l 26C05
  \l 26C06
  \l 26C07
  \l 26C08
  \l 26C09
  \l 26C0A
  \l 26C0B
  \l 26C0C
  \l 26C0D
  \l 26C0E
  \l 26C0F
  \l 26C10
  \l 26C11
  \l 26C12
  \l 26C13
  \l 26C14
  \l 26C15
  \l 26C16
  \l 26C17
  \l 26C18
  \l 26C19
  \l 26C1A
  \l 26C1B
  \l 26C1C
  \l 26C1D
  \l 26C1E
  \l 26C1F
  \l 26C20
  \l 26C21
  \l 26C22
  \l 26C23
  \l 26C24
  \l 26C25
  \l 26C26
  \l 26C27
  \l 26C28
  \l 26C29
  \l 26C2A
  \l 26C2B
  \l 26C2C
  \l 26C2D
  \l 26C2E
  \l 26C2F
  \l 26C30
  \l 26C31
  \l 26C32
  \l 26C33
  \l 26C34
  \l 26C35
  \l 26C36
  \l 26C37
  \l 26C38
  \l 26C39
  \l 26C3A
  \l 26C3B
  \l 26C3C
  \l 26C3D
  \l 26C3E
  \l 26C3F
  \l 26C40
  \l 26C41
  \l 26C42
  \l 26C43
  \l 26C44
  \l 26C45
  \l 26C46
  \l 26C47
  \l 26C48
  \l 26C49
  \l 26C4A
  \l 26C4B
  \l 26C4C
  \l 26C4D
  \l 26C4E
  \l 26C4F
  \l 26C50
  \l 26C51
  \l 26C52
  \l 26C53
  \l 26C54
  \l 26C55
  \l 26C56
  \l 26C57
  \l 26C58
  \l 26C59
  \l 26C5A
  \l 26C5B
  \l 26C5C
  \l 26C5D
  \l 26C5E
  \l 26C5F
  \l 26C60
  \l 26C61
  \l 26C62
  \l 26C63
  \l 26C64
  \l 26C65
  \l 26C66
  \l 26C67
  \l 26C68
  \l 26C69
  \l 26C6A
  \l 26C6B
  \l 26C6C
  \l 26C6D
  \l 26C6E
  \l 26C6F
  \l 26C70
  \l 26C71
  \l 26C72
  \l 26C73
  \l 26C74
  \l 26C75
  \l 26C76
  \l 26C77
  \l 26C78
  \l 26C79
  \l 26C7A
  \l 26C7B
  \l 26C7C
  \l 26C7D
  \l 26C7E
  \l 26C7F
  \l 26C80
  \l 26C81
  \l 26C82
  \l 26C83
  \l 26C84
  \l 26C85
  \l 26C86
  \l 26C87
  \l 26C88
  \l 26C89
  \l 26C8A
  \l 26C8B
  \l 26C8C
  \l 26C8D
  \l 26C8E
  \l 26C8F
  \l 26C90
  \l 26C91
  \l 26C92
  \l 26C93
  \l 26C94
  \l 26C95
  \l 26C96
  \l 26C97
  \l 26C98
  \l 26C99
  \l 26C9A
  \l 26C9B
  \l 26C9C
  \l 26C9D
  \l 26C9E
  \l 26C9F
  \l 26CA0
  \l 26CA1
  \l 26CA2
  \l 26CA3
  \l 26CA4
  \l 26CA5
  \l 26CA6
  \l 26CA7
  \l 26CA8
  \l 26CA9
  \l 26CAA
  \l 26CAB
  \l 26CAC
  \l 26CAD
  \l 26CAE
  \l 26CAF
  \l 26CB0
  \l 26CB1
  \l 26CB2
  \l 26CB3
  \l 26CB4
  \l 26CB5
  \l 26CB6
  \l 26CB7
  \l 26CB8
  \l 26CB9
  \l 26CBA
  \l 26CBB
  \l 26CBC
  \l 26CBD
  \l 26CBE
  \l 26CBF
  \l 26CC0
  \l 26CC1
  \l 26CC2
  \l 26CC3
  \l 26CC4
  \l 26CC5
  \l 26CC6
  \l 26CC7
  \l 26CC8
  \l 26CC9
  \l 26CCA
  \l 26CCB
  \l 26CCC
  \l 26CCD
  \l 26CCE
  \l 26CCF
  \l 26CD0
  \l 26CD1
  \l 26CD2
  \l 26CD3
  \l 26CD4
  \l 26CD5
  \l 26CD6
  \l 26CD7
  \l 26CD8
  \l 26CD9
  \l 26CDA
  \l 26CDB
  \l 26CDC
  \l 26CDD
  \l 26CDE
  \l 26CDF
  \l 26CE0
  \l 26CE1
  \l 26CE2
  \l 26CE3
  \l 26CE4
  \l 26CE5
  \l 26CE6
  \l 26CE7
  \l 26CE8
  \l 26CE9
  \l 26CEA
  \l 26CEB
  \l 26CEC
  \l 26CED
  \l 26CEE
  \l 26CEF
  \l 26CF0
  \l 26CF1
  \l 26CF2
  \l 26CF3
  \l 26CF4
  \l 26CF5
  \l 26CF6
  \l 26CF7
  \l 26CF8
  \l 26CF9
  \l 26CFA
  \l 26CFB
  \l 26CFC
  \l 26CFD
  \l 26CFE
  \l 26CFF
  \l 26D00
  \l 26D01
  \l 26D02
  \l 26D03
  \l 26D04
  \l 26D05
  \l 26D06
  \l 26D07
  \l 26D08
  \l 26D09
  \l 26D0A
  \l 26D0B
  \l 26D0C
  \l 26D0D
  \l 26D0E
  \l 26D0F
  \l 26D10
  \l 26D11
  \l 26D12
  \l 26D13
  \l 26D14
  \l 26D15
  \l 26D16
  \l 26D17
  \l 26D18
  \l 26D19
  \l 26D1A
  \l 26D1B
  \l 26D1C
  \l 26D1D
  \l 26D1E
  \l 26D1F
  \l 26D20
  \l 26D21
  \l 26D22
  \l 26D23
  \l 26D24
  \l 26D25
  \l 26D26
  \l 26D27
  \l 26D28
  \l 26D29
  \l 26D2A
  \l 26D2B
  \l 26D2C
  \l 26D2D
  \l 26D2E
  \l 26D2F
  \l 26D30
  \l 26D31
  \l 26D32
  \l 26D33
  \l 26D34
  \l 26D35
  \l 26D36
  \l 26D37
  \l 26D38
  \l 26D39
  \l 26D3A
  \l 26D3B
  \l 26D3C
  \l 26D3D
  \l 26D3E
  \l 26D3F
  \l 26D40
  \l 26D41
  \l 26D42
  \l 26D43
  \l 26D44
  \l 26D45
  \l 26D46
  \l 26D47
  \l 26D48
  \l 26D49
  \l 26D4A
  \l 26D4B
  \l 26D4C
  \l 26D4D
  \l 26D4E
  \l 26D4F
  \l 26D50
  \l 26D51
  \l 26D52
  \l 26D53
  \l 26D54
  \l 26D55
  \l 26D56
  \l 26D57
  \l 26D58
  \l 26D59
  \l 26D5A
  \l 26D5B
  \l 26D5C
  \l 26D5D
  \l 26D5E
  \l 26D5F
  \l 26D60
  \l 26D61
  \l 26D62
  \l 26D63
  \l 26D64
  \l 26D65
  \l 26D66
  \l 26D67
  \l 26D68
  \l 26D69
  \l 26D6A
  \l 26D6B
  \l 26D6C
  \l 26D6D
  \l 26D6E
  \l 26D6F
  \l 26D70
  \l 26D71
  \l 26D72
  \l 26D73
  \l 26D74
  \l 26D75
  \l 26D76
  \l 26D77
  \l 26D78
  \l 26D79
  \l 26D7A
  \l 26D7B
  \l 26D7C
  \l 26D7D
  \l 26D7E
  \l 26D7F
  \l 26D80
  \l 26D81
  \l 26D82
  \l 26D83
  \l 26D84
  \l 26D85
  \l 26D86
  \l 26D87
  \l 26D88
  \l 26D89
  \l 26D8A
  \l 26D8B
  \l 26D8C
  \l 26D8D
  \l 26D8E
  \l 26D8F
  \l 26D90
  \l 26D91
  \l 26D92
  \l 26D93
  \l 26D94
  \l 26D95
  \l 26D96
  \l 26D97
  \l 26D98
  \l 26D99
  \l 26D9A
  \l 26D9B
  \l 26D9C
  \l 26D9D
  \l 26D9E
  \l 26D9F
  \l 26DA0
  \l 26DA1
  \l 26DA2
  \l 26DA3
  \l 26DA4
  \l 26DA5
  \l 26DA6
  \l 26DA7
  \l 26DA8
  \l 26DA9
  \l 26DAA
  \l 26DAB
  \l 26DAC
  \l 26DAD
  \l 26DAE
  \l 26DAF
  \l 26DB0
  \l 26DB1
  \l 26DB2
  \l 26DB3
  \l 26DB4
  \l 26DB5
  \l 26DB6
  \l 26DB7
  \l 26DB8
  \l 26DB9
  \l 26DBA
  \l 26DBB
  \l 26DBC
  \l 26DBD
  \l 26DBE
  \l 26DBF
  \l 26DC0
  \l 26DC1
  \l 26DC2
  \l 26DC3
  \l 26DC4
  \l 26DC5
  \l 26DC6
  \l 26DC7
  \l 26DC8
  \l 26DC9
  \l 26DCA
  \l 26DCB
  \l 26DCC
  \l 26DCD
  \l 26DCE
  \l 26DCF
  \l 26DD0
  \l 26DD1
  \l 26DD2
  \l 26DD3
  \l 26DD4
  \l 26DD5
  \l 26DD6
  \l 26DD7
  \l 26DD8
  \l 26DD9
  \l 26DDA
  \l 26DDB
  \l 26DDC
  \l 26DDD
  \l 26DDE
  \l 26DDF
  \l 26DE0
  \l 26DE1
  \l 26DE2
  \l 26DE3
  \l 26DE4
  \l 26DE5
  \l 26DE6
  \l 26DE7
  \l 26DE8
  \l 26DE9
  \l 26DEA
  \l 26DEB
  \l 26DEC
  \l 26DED
  \l 26DEE
  \l 26DEF
  \l 26DF0
  \l 26DF1
  \l 26DF2
  \l 26DF3
  \l 26DF4
  \l 26DF5
  \l 26DF6
  \l 26DF7
  \l 26DF8
  \l 26DF9
  \l 26DFA
  \l 26DFB
  \l 26DFC
  \l 26DFD
  \l 26DFE
  \l 26DFF
  \l 26E00
  \l 26E01
  \l 26E02
  \l 26E03
  \l 26E04
  \l 26E05
  \l 26E06
  \l 26E07
  \l 26E08
  \l 26E09
  \l 26E0A
  \l 26E0B
  \l 26E0C
  \l 26E0D
  \l 26E0E
  \l 26E0F
  \l 26E10
  \l 26E11
  \l 26E12
  \l 26E13
  \l 26E14
  \l 26E15
  \l 26E16
  \l 26E17
  \l 26E18
  \l 26E19
  \l 26E1A
  \l 26E1B
  \l 26E1C
  \l 26E1D
  \l 26E1E
  \l 26E1F
  \l 26E20
  \l 26E21
  \l 26E22
  \l 26E23
  \l 26E24
  \l 26E25
  \l 26E26
  \l 26E27
  \l 26E28
  \l 26E29
  \l 26E2A
  \l 26E2B
  \l 26E2C
  \l 26E2D
  \l 26E2E
  \l 26E2F
  \l 26E30
  \l 26E31
  \l 26E32
  \l 26E33
  \l 26E34
  \l 26E35
  \l 26E36
  \l 26E37
  \l 26E38
  \l 26E39
  \l 26E3A
  \l 26E3B
  \l 26E3C
  \l 26E3D
  \l 26E3E
  \l 26E3F
  \l 26E40
  \l 26E41
  \l 26E42
  \l 26E43
  \l 26E44
  \l 26E45
  \l 26E46
  \l 26E47
  \l 26E48
  \l 26E49
  \l 26E4A
  \l 26E4B
  \l 26E4C
  \l 26E4D
  \l 26E4E
  \l 26E4F
  \l 26E50
  \l 26E51
  \l 26E52
  \l 26E53
  \l 26E54
  \l 26E55
  \l 26E56
  \l 26E57
  \l 26E58
  \l 26E59
  \l 26E5A
  \l 26E5B
  \l 26E5C
  \l 26E5D
  \l 26E5E
  \l 26E5F
  \l 26E60
  \l 26E61
  \l 26E62
  \l 26E63
  \l 26E64
  \l 26E65
  \l 26E66
  \l 26E67
  \l 26E68
  \l 26E69
  \l 26E6A
  \l 26E6B
  \l 26E6C
  \l 26E6D
  \l 26E6E
  \l 26E6F
  \l 26E70
  \l 26E71
  \l 26E72
  \l 26E73
  \l 26E74
  \l 26E75
  \l 26E76
  \l 26E77
  \l 26E78
  \l 26E79
  \l 26E7A
  \l 26E7B
  \l 26E7C
  \l 26E7D
  \l 26E7E
  \l 26E7F
  \l 26E80
  \l 26E81
  \l 26E82
  \l 26E83
  \l 26E84
  \l 26E85
  \l 26E86
  \l 26E87
  \l 26E88
  \l 26E89
  \l 26E8A
  \l 26E8B
  \l 26E8C
  \l 26E8D
  \l 26E8E
  \l 26E8F
  \l 26E90
  \l 26E91
  \l 26E92
  \l 26E93
  \l 26E94
  \l 26E95
  \l 26E96
  \l 26E97
  \l 26E98
  \l 26E99
  \l 26E9A
  \l 26E9B
  \l 26E9C
  \l 26E9D
  \l 26E9E
  \l 26E9F
  \l 26EA0
  \l 26EA1
  \l 26EA2
  \l 26EA3
  \l 26EA4
  \l 26EA5
  \l 26EA6
  \l 26EA7
  \l 26EA8
  \l 26EA9
  \l 26EAA
  \l 26EAB
  \l 26EAC
  \l 26EAD
  \l 26EAE
  \l 26EAF
  \l 26EB0
  \l 26EB1
  \l 26EB2
  \l 26EB3
  \l 26EB4
  \l 26EB5
  \l 26EB6
  \l 26EB7
  \l 26EB8
  \l 26EB9
  \l 26EBA
  \l 26EBB
  \l 26EBC
  \l 26EBD
  \l 26EBE
  \l 26EBF
  \l 26EC0
  \l 26EC1
  \l 26EC2
  \l 26EC3
  \l 26EC4
  \l 26EC5
  \l 26EC6
  \l 26EC7
  \l 26EC8
  \l 26EC9
  \l 26ECA
  \l 26ECB
  \l 26ECC
  \l 26ECD
  \l 26ECE
  \l 26ECF
  \l 26ED0
  \l 26ED1
  \l 26ED2
  \l 26ED3
  \l 26ED4
  \l 26ED5
  \l 26ED6
  \l 26ED7
  \l 26ED8
  \l 26ED9
  \l 26EDA
  \l 26EDB
  \l 26EDC
  \l 26EDD
  \l 26EDE
  \l 26EDF
  \l 26EE0
  \l 26EE1
  \l 26EE2
  \l 26EE3
  \l 26EE4
  \l 26EE5
  \l 26EE6
  \l 26EE7
  \l 26EE8
  \l 26EE9
  \l 26EEA
  \l 26EEB
  \l 26EEC
  \l 26EED
  \l 26EEE
  \l 26EEF
  \l 26EF0
  \l 26EF1
  \l 26EF2
  \l 26EF3
  \l 26EF4
  \l 26EF5
  \l 26EF6
  \l 26EF7
  \l 26EF8
  \l 26EF9
  \l 26EFA
  \l 26EFB
  \l 26EFC
  \l 26EFD
  \l 26EFE
  \l 26EFF
  \l 26F00
  \l 26F01
  \l 26F02
  \l 26F03
  \l 26F04
  \l 26F05
  \l 26F06
  \l 26F07
  \l 26F08
  \l 26F09
  \l 26F0A
  \l 26F0B
  \l 26F0C
  \l 26F0D
  \l 26F0E
  \l 26F0F
  \l 26F10
  \l 26F11
  \l 26F12
  \l 26F13
  \l 26F14
  \l 26F15
  \l 26F16
  \l 26F17
  \l 26F18
  \l 26F19
  \l 26F1A
  \l 26F1B
  \l 26F1C
  \l 26F1D
  \l 26F1E
  \l 26F1F
  \l 26F20
  \l 26F21
  \l 26F22
  \l 26F23
  \l 26F24
  \l 26F25
  \l 26F26
  \l 26F27
  \l 26F28
  \l 26F29
  \l 26F2A
  \l 26F2B
  \l 26F2C
  \l 26F2D
  \l 26F2E
  \l 26F2F
  \l 26F30
  \l 26F31
  \l 26F32
  \l 26F33
  \l 26F34
  \l 26F35
  \l 26F36
  \l 26F37
  \l 26F38
  \l 26F39
  \l 26F3A
  \l 26F3B
  \l 26F3C
  \l 26F3D
  \l 26F3E
  \l 26F3F
  \l 26F40
  \l 26F41
  \l 26F42
  \l 26F43
  \l 26F44
  \l 26F45
  \l 26F46
  \l 26F47
  \l 26F48
  \l 26F49
  \l 26F4A
  \l 26F4B
  \l 26F4C
  \l 26F4D
  \l 26F4E
  \l 26F4F
  \l 26F50
  \l 26F51
  \l 26F52
  \l 26F53
  \l 26F54
  \l 26F55
  \l 26F56
  \l 26F57
  \l 26F58
  \l 26F59
  \l 26F5A
  \l 26F5B
  \l 26F5C
  \l 26F5D
  \l 26F5E
  \l 26F5F
  \l 26F60
  \l 26F61
  \l 26F62
  \l 26F63
  \l 26F64
  \l 26F65
  \l 26F66
  \l 26F67
  \l 26F68
  \l 26F69
  \l 26F6A
  \l 26F6B
  \l 26F6C
  \l 26F6D
  \l 26F6E
  \l 26F6F
  \l 26F70
  \l 26F71
  \l 26F72
  \l 26F73
  \l 26F74
  \l 26F75
  \l 26F76
  \l 26F77
  \l 26F78
  \l 26F79
  \l 26F7A
  \l 26F7B
  \l 26F7C
  \l 26F7D
  \l 26F7E
  \l 26F7F
  \l 26F80
  \l 26F81
  \l 26F82
  \l 26F83
  \l 26F84
  \l 26F85
  \l 26F86
  \l 26F87
  \l 26F88
  \l 26F89
  \l 26F8A
  \l 26F8B
  \l 26F8C
  \l 26F8D
  \l 26F8E
  \l 26F8F
  \l 26F90
  \l 26F91
  \l 26F92
  \l 26F93
  \l 26F94
  \l 26F95
  \l 26F96
  \l 26F97
  \l 26F98
  \l 26F99
  \l 26F9A
  \l 26F9B
  \l 26F9C
  \l 26F9D
  \l 26F9E
  \l 26F9F
  \l 26FA0
  \l 26FA1
  \l 26FA2
  \l 26FA3
  \l 26FA4
  \l 26FA5
  \l 26FA6
  \l 26FA7
  \l 26FA8
  \l 26FA9
  \l 26FAA
  \l 26FAB
  \l 26FAC
  \l 26FAD
  \l 26FAE
  \l 26FAF
  \l 26FB0
  \l 26FB1
  \l 26FB2
  \l 26FB3
  \l 26FB4
  \l 26FB5
  \l 26FB6
  \l 26FB7
  \l 26FB8
  \l 26FB9
  \l 26FBA
  \l 26FBB
  \l 26FBC
  \l 26FBD
  \l 26FBE
  \l 26FBF
  \l 26FC0
  \l 26FC1
  \l 26FC2
  \l 26FC3
  \l 26FC4
  \l 26FC5
  \l 26FC6
  \l 26FC7
  \l 26FC8
  \l 26FC9
  \l 26FCA
  \l 26FCB
  \l 26FCC
  \l 26FCD
  \l 26FCE
  \l 26FCF
  \l 26FD0
  \l 26FD1
  \l 26FD2
  \l 26FD3
  \l 26FD4
  \l 26FD5
  \l 26FD6
  \l 26FD7
  \l 26FD8
  \l 26FD9
  \l 26FDA
  \l 26FDB
  \l 26FDC
  \l 26FDD
  \l 26FDE
  \l 26FDF
  \l 26FE0
  \l 26FE1
  \l 26FE2
  \l 26FE3
  \l 26FE4
  \l 26FE5
  \l 26FE6
  \l 26FE7
  \l 26FE8
  \l 26FE9
  \l 26FEA
  \l 26FEB
  \l 26FEC
  \l 26FED
  \l 26FEE
  \l 26FEF
  \l 26FF0
  \l 26FF1
  \l 26FF2
  \l 26FF3
  \l 26FF4
  \l 26FF5
  \l 26FF6
  \l 26FF7
  \l 26FF8
  \l 26FF9
  \l 26FFA
  \l 26FFB
  \l 26FFC
  \l 26FFD
  \l 26FFE
  \l 26FFF
  \l 27000
  \l 27001
  \l 27002
  \l 27003
  \l 27004
  \l 27005
  \l 27006
  \l 27007
  \l 27008
  \l 27009
  \l 2700A
  \l 2700B
  \l 2700C
  \l 2700D
  \l 2700E
  \l 2700F
  \l 27010
  \l 27011
  \l 27012
  \l 27013
  \l 27014
  \l 27015
  \l 27016
  \l 27017
  \l 27018
  \l 27019
  \l 2701A
  \l 2701B
  \l 2701C
  \l 2701D
  \l 2701E
  \l 2701F
  \l 27020
  \l 27021
  \l 27022
  \l 27023
  \l 27024
  \l 27025
  \l 27026
  \l 27027
  \l 27028
  \l 27029
  \l 2702A
  \l 2702B
  \l 2702C
  \l 2702D
  \l 2702E
  \l 2702F
  \l 27030
  \l 27031
  \l 27032
  \l 27033
  \l 27034
  \l 27035
  \l 27036
  \l 27037
  \l 27038
  \l 27039
  \l 2703A
  \l 2703B
  \l 2703C
  \l 2703D
  \l 2703E
  \l 2703F
  \l 27040
  \l 27041
  \l 27042
  \l 27043
  \l 27044
  \l 27045
  \l 27046
  \l 27047
  \l 27048
  \l 27049
  \l 2704A
  \l 2704B
  \l 2704C
  \l 2704D
  \l 2704E
  \l 2704F
  \l 27050
  \l 27051
  \l 27052
  \l 27053
  \l 27054
  \l 27055
  \l 27056
  \l 27057
  \l 27058
  \l 27059
  \l 2705A
  \l 2705B
  \l 2705C
  \l 2705D
  \l 2705E
  \l 2705F
  \l 27060
  \l 27061
  \l 27062
  \l 27063
  \l 27064
  \l 27065
  \l 27066
  \l 27067
  \l 27068
  \l 27069
  \l 2706A
  \l 2706B
  \l 2706C
  \l 2706D
  \l 2706E
  \l 2706F
  \l 27070
  \l 27071
  \l 27072
  \l 27073
  \l 27074
  \l 27075
  \l 27076
  \l 27077
  \l 27078
  \l 27079
  \l 2707A
  \l 2707B
  \l 2707C
  \l 2707D
  \l 2707E
  \l 2707F
  \l 27080
  \l 27081
  \l 27082
  \l 27083
  \l 27084
  \l 27085
  \l 27086
  \l 27087
  \l 27088
  \l 27089
  \l 2708A
  \l 2708B
  \l 2708C
  \l 2708D
  \l 2708E
  \l 2708F
  \l 27090
  \l 27091
  \l 27092
  \l 27093
  \l 27094
  \l 27095
  \l 27096
  \l 27097
  \l 27098
  \l 27099
  \l 2709A
  \l 2709B
  \l 2709C
  \l 2709D
  \l 2709E
  \l 2709F
  \l 270A0
  \l 270A1
  \l 270A2
  \l 270A3
  \l 270A4
  \l 270A5
  \l 270A6
  \l 270A7
  \l 270A8
  \l 270A9
  \l 270AA
  \l 270AB
  \l 270AC
  \l 270AD
  \l 270AE
  \l 270AF
  \l 270B0
  \l 270B1
  \l 270B2
  \l 270B3
  \l 270B4
  \l 270B5
  \l 270B6
  \l 270B7
  \l 270B8
  \l 270B9
  \l 270BA
  \l 270BB
  \l 270BC
  \l 270BD
  \l 270BE
  \l 270BF
  \l 270C0
  \l 270C1
  \l 270C2
  \l 270C3
  \l 270C4
  \l 270C5
  \l 270C6
  \l 270C7
  \l 270C8
  \l 270C9
  \l 270CA
  \l 270CB
  \l 270CC
  \l 270CD
  \l 270CE
  \l 270CF
  \l 270D0
  \l 270D1
  \l 270D2
  \l 270D3
  \l 270D4
  \l 270D5
  \l 270D6
  \l 270D7
  \l 270D8
  \l 270D9
  \l 270DA
  \l 270DB
  \l 270DC
  \l 270DD
  \l 270DE
  \l 270DF
  \l 270E0
  \l 270E1
  \l 270E2
  \l 270E3
  \l 270E4
  \l 270E5
  \l 270E6
  \l 270E7
  \l 270E8
  \l 270E9
  \l 270EA
  \l 270EB
  \l 270EC
  \l 270ED
  \l 270EE
  \l 270EF
  \l 270F0
  \l 270F1
  \l 270F2
  \l 270F3
  \l 270F4
  \l 270F5
  \l 270F6
  \l 270F7
  \l 270F8
  \l 270F9
  \l 270FA
  \l 270FB
  \l 270FC
  \l 270FD
  \l 270FE
  \l 270FF
  \l 27100
  \l 27101
  \l 27102
  \l 27103
  \l 27104
  \l 27105
  \l 27106
  \l 27107
  \l 27108
  \l 27109
  \l 2710A
  \l 2710B
  \l 2710C
  \l 2710D
  \l 2710E
  \l 2710F
  \l 27110
  \l 27111
  \l 27112
  \l 27113
  \l 27114
  \l 27115
  \l 27116
  \l 27117
  \l 27118
  \l 27119
  \l 2711A
  \l 2711B
  \l 2711C
  \l 2711D
  \l 2711E
  \l 2711F
  \l 27120
  \l 27121
  \l 27122
  \l 27123
  \l 27124
  \l 27125
  \l 27126
  \l 27127
  \l 27128
  \l 27129
  \l 2712A
  \l 2712B
  \l 2712C
  \l 2712D
  \l 2712E
  \l 2712F
  \l 27130
  \l 27131
  \l 27132
  \l 27133
  \l 27134
  \l 27135
  \l 27136
  \l 27137
  \l 27138
  \l 27139
  \l 2713A
  \l 2713B
  \l 2713C
  \l 2713D
  \l 2713E
  \l 2713F
  \l 27140
  \l 27141
  \l 27142
  \l 27143
  \l 27144
  \l 27145
  \l 27146
  \l 27147
  \l 27148
  \l 27149
  \l 2714A
  \l 2714B
  \l 2714C
  \l 2714D
  \l 2714E
  \l 2714F
  \l 27150
  \l 27151
  \l 27152
  \l 27153
  \l 27154
  \l 27155
  \l 27156
  \l 27157
  \l 27158
  \l 27159
  \l 2715A
  \l 2715B
  \l 2715C
  \l 2715D
  \l 2715E
  \l 2715F
  \l 27160
  \l 27161
  \l 27162
  \l 27163
  \l 27164
  \l 27165
  \l 27166
  \l 27167
  \l 27168
  \l 27169
  \l 2716A
  \l 2716B
  \l 2716C
  \l 2716D
  \l 2716E
  \l 2716F
  \l 27170
  \l 27171
  \l 27172
  \l 27173
  \l 27174
  \l 27175
  \l 27176
  \l 27177
  \l 27178
  \l 27179
  \l 2717A
  \l 2717B
  \l 2717C
  \l 2717D
  \l 2717E
  \l 2717F
  \l 27180
  \l 27181
  \l 27182
  \l 27183
  \l 27184
  \l 27185
  \l 27186
  \l 27187
  \l 27188
  \l 27189
  \l 2718A
  \l 2718B
  \l 2718C
  \l 2718D
  \l 2718E
  \l 2718F
  \l 27190
  \l 27191
  \l 27192
  \l 27193
  \l 27194
  \l 27195
  \l 27196
  \l 27197
  \l 27198
  \l 27199
  \l 2719A
  \l 2719B
  \l 2719C
  \l 2719D
  \l 2719E
  \l 2719F
  \l 271A0
  \l 271A1
  \l 271A2
  \l 271A3
  \l 271A4
  \l 271A5
  \l 271A6
  \l 271A7
  \l 271A8
  \l 271A9
  \l 271AA
  \l 271AB
  \l 271AC
  \l 271AD
  \l 271AE
  \l 271AF
  \l 271B0
  \l 271B1
  \l 271B2
  \l 271B3
  \l 271B4
  \l 271B5
  \l 271B6
  \l 271B7
  \l 271B8
  \l 271B9
  \l 271BA
  \l 271BB
  \l 271BC
  \l 271BD
  \l 271BE
  \l 271BF
  \l 271C0
  \l 271C1
  \l 271C2
  \l 271C3
  \l 271C4
  \l 271C5
  \l 271C6
  \l 271C7
  \l 271C8
  \l 271C9
  \l 271CA
  \l 271CB
  \l 271CC
  \l 271CD
  \l 271CE
  \l 271CF
  \l 271D0
  \l 271D1
  \l 271D2
  \l 271D3
  \l 271D4
  \l 271D5
  \l 271D6
  \l 271D7
  \l 271D8
  \l 271D9
  \l 271DA
  \l 271DB
  \l 271DC
  \l 271DD
  \l 271DE
  \l 271DF
  \l 271E0
  \l 271E1
  \l 271E2
  \l 271E3
  \l 271E4
  \l 271E5
  \l 271E6
  \l 271E7
  \l 271E8
  \l 271E9
  \l 271EA
  \l 271EB
  \l 271EC
  \l 271ED
  \l 271EE
  \l 271EF
  \l 271F0
  \l 271F1
  \l 271F2
  \l 271F3
  \l 271F4
  \l 271F5
  \l 271F6
  \l 271F7
  \l 271F8
  \l 271F9
  \l 271FA
  \l 271FB
  \l 271FC
  \l 271FD
  \l 271FE
  \l 271FF
  \l 27200
  \l 27201
  \l 27202
  \l 27203
  \l 27204
  \l 27205
  \l 27206
  \l 27207
  \l 27208
  \l 27209
  \l 2720A
  \l 2720B
  \l 2720C
  \l 2720D
  \l 2720E
  \l 2720F
  \l 27210
  \l 27211
  \l 27212
  \l 27213
  \l 27214
  \l 27215
  \l 27216
  \l 27217
  \l 27218
  \l 27219
  \l 2721A
  \l 2721B
  \l 2721C
  \l 2721D
  \l 2721E
  \l 2721F
  \l 27220
  \l 27221
  \l 27222
  \l 27223
  \l 27224
  \l 27225
  \l 27226
  \l 27227
  \l 27228
  \l 27229
  \l 2722A
  \l 2722B
  \l 2722C
  \l 2722D
  \l 2722E
  \l 2722F
  \l 27230
  \l 27231
  \l 27232
  \l 27233
  \l 27234
  \l 27235
  \l 27236
  \l 27237
  \l 27238
  \l 27239
  \l 2723A
  \l 2723B
  \l 2723C
  \l 2723D
  \l 2723E
  \l 2723F
  \l 27240
  \l 27241
  \l 27242
  \l 27243
  \l 27244
  \l 27245
  \l 27246
  \l 27247
  \l 27248
  \l 27249
  \l 2724A
  \l 2724B
  \l 2724C
  \l 2724D
  \l 2724E
  \l 2724F
  \l 27250
  \l 27251
  \l 27252
  \l 27253
  \l 27254
  \l 27255
  \l 27256
  \l 27257
  \l 27258
  \l 27259
  \l 2725A
  \l 2725B
  \l 2725C
  \l 2725D
  \l 2725E
  \l 2725F
  \l 27260
  \l 27261
  \l 27262
  \l 27263
  \l 27264
  \l 27265
  \l 27266
  \l 27267
  \l 27268
  \l 27269
  \l 2726A
  \l 2726B
  \l 2726C
  \l 2726D
  \l 2726E
  \l 2726F
  \l 27270
  \l 27271
  \l 27272
  \l 27273
  \l 27274
  \l 27275
  \l 27276
  \l 27277
  \l 27278
  \l 27279
  \l 2727A
  \l 2727B
  \l 2727C
  \l 2727D
  \l 2727E
  \l 2727F
  \l 27280
  \l 27281
  \l 27282
  \l 27283
  \l 27284
  \l 27285
  \l 27286
  \l 27287
  \l 27288
  \l 27289
  \l 2728A
  \l 2728B
  \l 2728C
  \l 2728D
  \l 2728E
  \l 2728F
  \l 27290
  \l 27291
  \l 27292
  \l 27293
  \l 27294
  \l 27295
  \l 27296
  \l 27297
  \l 27298
  \l 27299
  \l 2729A
  \l 2729B
  \l 2729C
  \l 2729D
  \l 2729E
  \l 2729F
  \l 272A0
  \l 272A1
  \l 272A2
  \l 272A3
  \l 272A4
  \l 272A5
  \l 272A6
  \l 272A7
  \l 272A8
  \l 272A9
  \l 272AA
  \l 272AB
  \l 272AC
  \l 272AD
  \l 272AE
  \l 272AF
  \l 272B0
  \l 272B1
  \l 272B2
  \l 272B3
  \l 272B4
  \l 272B5
  \l 272B6
  \l 272B7
  \l 272B8
  \l 272B9
  \l 272BA
  \l 272BB
  \l 272BC
  \l 272BD
  \l 272BE
  \l 272BF
  \l 272C0
  \l 272C1
  \l 272C2
  \l 272C3
  \l 272C4
  \l 272C5
  \l 272C6
  \l 272C7
  \l 272C8
  \l 272C9
  \l 272CA
  \l 272CB
  \l 272CC
  \l 272CD
  \l 272CE
  \l 272CF
  \l 272D0
  \l 272D1
  \l 272D2
  \l 272D3
  \l 272D4
  \l 272D5
  \l 272D6
  \l 272D7
  \l 272D8
  \l 272D9
  \l 272DA
  \l 272DB
  \l 272DC
  \l 272DD
  \l 272DE
  \l 272DF
  \l 272E0
  \l 272E1
  \l 272E2
  \l 272E3
  \l 272E4
  \l 272E5
  \l 272E6
  \l 272E7
  \l 272E8
  \l 272E9
  \l 272EA
  \l 272EB
  \l 272EC
  \l 272ED
  \l 272EE
  \l 272EF
  \l 272F0
  \l 272F1
  \l 272F2
  \l 272F3
  \l 272F4
  \l 272F5
  \l 272F6
  \l 272F7
  \l 272F8
  \l 272F9
  \l 272FA
  \l 272FB
  \l 272FC
  \l 272FD
  \l 272FE
  \l 272FF
  \l 27300
  \l 27301
  \l 27302
  \l 27303
  \l 27304
  \l 27305
  \l 27306
  \l 27307
  \l 27308
  \l 27309
  \l 2730A
  \l 2730B
  \l 2730C
  \l 2730D
  \l 2730E
  \l 2730F
  \l 27310
  \l 27311
  \l 27312
  \l 27313
  \l 27314
  \l 27315
  \l 27316
  \l 27317
  \l 27318
  \l 27319
  \l 2731A
  \l 2731B
  \l 2731C
  \l 2731D
  \l 2731E
  \l 2731F
  \l 27320
  \l 27321
  \l 27322
  \l 27323
  \l 27324
  \l 27325
  \l 27326
  \l 27327
  \l 27328
  \l 27329
  \l 2732A
  \l 2732B
  \l 2732C
  \l 2732D
  \l 2732E
  \l 2732F
  \l 27330
  \l 27331
  \l 27332
  \l 27333
  \l 27334
  \l 27335
  \l 27336
  \l 27337
  \l 27338
  \l 27339
  \l 2733A
  \l 2733B
  \l 2733C
  \l 2733D
  \l 2733E
  \l 2733F
  \l 27340
  \l 27341
  \l 27342
  \l 27343
  \l 27344
  \l 27345
  \l 27346
  \l 27347
  \l 27348
  \l 27349
  \l 2734A
  \l 2734B
  \l 2734C
  \l 2734D
  \l 2734E
  \l 2734F
  \l 27350
  \l 27351
  \l 27352
  \l 27353
  \l 27354
  \l 27355
  \l 27356
  \l 27357
  \l 27358
  \l 27359
  \l 2735A
  \l 2735B
  \l 2735C
  \l 2735D
  \l 2735E
  \l 2735F
  \l 27360
  \l 27361
  \l 27362
  \l 27363
  \l 27364
  \l 27365
  \l 27366
  \l 27367
  \l 27368
  \l 27369
  \l 2736A
  \l 2736B
  \l 2736C
  \l 2736D
  \l 2736E
  \l 2736F
  \l 27370
  \l 27371
  \l 27372
  \l 27373
  \l 27374
  \l 27375
  \l 27376
  \l 27377
  \l 27378
  \l 27379
  \l 2737A
  \l 2737B
  \l 2737C
  \l 2737D
  \l 2737E
  \l 2737F
  \l 27380
  \l 27381
  \l 27382
  \l 27383
  \l 27384
  \l 27385
  \l 27386
  \l 27387
  \l 27388
  \l 27389
  \l 2738A
  \l 2738B
  \l 2738C
  \l 2738D
  \l 2738E
  \l 2738F
  \l 27390
  \l 27391
  \l 27392
  \l 27393
  \l 27394
  \l 27395
  \l 27396
  \l 27397
  \l 27398
  \l 27399
  \l 2739A
  \l 2739B
  \l 2739C
  \l 2739D
  \l 2739E
  \l 2739F
  \l 273A0
  \l 273A1
  \l 273A2
  \l 273A3
  \l 273A4
  \l 273A5
  \l 273A6
  \l 273A7
  \l 273A8
  \l 273A9
  \l 273AA
  \l 273AB
  \l 273AC
  \l 273AD
  \l 273AE
  \l 273AF
  \l 273B0
  \l 273B1
  \l 273B2
  \l 273B3
  \l 273B4
  \l 273B5
  \l 273B6
  \l 273B7
  \l 273B8
  \l 273B9
  \l 273BA
  \l 273BB
  \l 273BC
  \l 273BD
  \l 273BE
  \l 273BF
  \l 273C0
  \l 273C1
  \l 273C2
  \l 273C3
  \l 273C4
  \l 273C5
  \l 273C6
  \l 273C7
  \l 273C8
  \l 273C9
  \l 273CA
  \l 273CB
  \l 273CC
  \l 273CD
  \l 273CE
  \l 273CF
  \l 273D0
  \l 273D1
  \l 273D2
  \l 273D3
  \l 273D4
  \l 273D5
  \l 273D6
  \l 273D7
  \l 273D8
  \l 273D9
  \l 273DA
  \l 273DB
  \l 273DC
  \l 273DD
  \l 273DE
  \l 273DF
  \l 273E0
  \l 273E1
  \l 273E2
  \l 273E3
  \l 273E4
  \l 273E5
  \l 273E6
  \l 273E7
  \l 273E8
  \l 273E9
  \l 273EA
  \l 273EB
  \l 273EC
  \l 273ED
  \l 273EE
  \l 273EF
  \l 273F0
  \l 273F1
  \l 273F2
  \l 273F3
  \l 273F4
  \l 273F5
  \l 273F6
  \l 273F7
  \l 273F8
  \l 273F9
  \l 273FA
  \l 273FB
  \l 273FC
  \l 273FD
  \l 273FE
  \l 273FF
  \l 27400
  \l 27401
  \l 27402
  \l 27403
  \l 27404
  \l 27405
  \l 27406
  \l 27407
  \l 27408
  \l 27409
  \l 2740A
  \l 2740B
  \l 2740C
  \l 2740D
  \l 2740E
  \l 2740F
  \l 27410
  \l 27411
  \l 27412
  \l 27413
  \l 27414
  \l 27415
  \l 27416
  \l 27417
  \l 27418
  \l 27419
  \l 2741A
  \l 2741B
  \l 2741C
  \l 2741D
  \l 2741E
  \l 2741F
  \l 27420
  \l 27421
  \l 27422
  \l 27423
  \l 27424
  \l 27425
  \l 27426
  \l 27427
  \l 27428
  \l 27429
  \l 2742A
  \l 2742B
  \l 2742C
  \l 2742D
  \l 2742E
  \l 2742F
  \l 27430
  \l 27431
  \l 27432
  \l 27433
  \l 27434
  \l 27435
  \l 27436
  \l 27437
  \l 27438
  \l 27439
  \l 2743A
  \l 2743B
  \l 2743C
  \l 2743D
  \l 2743E
  \l 2743F
  \l 27440
  \l 27441
  \l 27442
  \l 27443
  \l 27444
  \l 27445
  \l 27446
  \l 27447
  \l 27448
  \l 27449
  \l 2744A
  \l 2744B
  \l 2744C
  \l 2744D
  \l 2744E
  \l 2744F
  \l 27450
  \l 27451
  \l 27452
  \l 27453
  \l 27454
  \l 27455
  \l 27456
  \l 27457
  \l 27458
  \l 27459
  \l 2745A
  \l 2745B
  \l 2745C
  \l 2745D
  \l 2745E
  \l 2745F
  \l 27460
  \l 27461
  \l 27462
  \l 27463
  \l 27464
  \l 27465
  \l 27466
  \l 27467
  \l 27468
  \l 27469
  \l 2746A
  \l 2746B
  \l 2746C
  \l 2746D
  \l 2746E
  \l 2746F
  \l 27470
  \l 27471
  \l 27472
  \l 27473
  \l 27474
  \l 27475
  \l 27476
  \l 27477
  \l 27478
  \l 27479
  \l 2747A
  \l 2747B
  \l 2747C
  \l 2747D
  \l 2747E
  \l 2747F
  \l 27480
  \l 27481
  \l 27482
  \l 27483
  \l 27484
  \l 27485
  \l 27486
  \l 27487
  \l 27488
  \l 27489
  \l 2748A
  \l 2748B
  \l 2748C
  \l 2748D
  \l 2748E
  \l 2748F
  \l 27490
  \l 27491
  \l 27492
  \l 27493
  \l 27494
  \l 27495
  \l 27496
  \l 27497
  \l 27498
  \l 27499
  \l 2749A
  \l 2749B
  \l 2749C
  \l 2749D
  \l 2749E
  \l 2749F
  \l 274A0
  \l 274A1
  \l 274A2
  \l 274A3
  \l 274A4
  \l 274A5
  \l 274A6
  \l 274A7
  \l 274A8
  \l 274A9
  \l 274AA
  \l 274AB
  \l 274AC
  \l 274AD
  \l 274AE
  \l 274AF
  \l 274B0
  \l 274B1
  \l 274B2
  \l 274B3
  \l 274B4
  \l 274B5
  \l 274B6
  \l 274B7
  \l 274B8
  \l 274B9
  \l 274BA
  \l 274BB
  \l 274BC
  \l 274BD
  \l 274BE
  \l 274BF
  \l 274C0
  \l 274C1
  \l 274C2
  \l 274C3
  \l 274C4
  \l 274C5
  \l 274C6
  \l 274C7
  \l 274C8
  \l 274C9
  \l 274CA
  \l 274CB
  \l 274CC
  \l 274CD
  \l 274CE
  \l 274CF
  \l 274D0
  \l 274D1
  \l 274D2
  \l 274D3
  \l 274D4
  \l 274D5
  \l 274D6
  \l 274D7
  \l 274D8
  \l 274D9
  \l 274DA
  \l 274DB
  \l 274DC
  \l 274DD
  \l 274DE
  \l 274DF
  \l 274E0
  \l 274E1
  \l 274E2
  \l 274E3
  \l 274E4
  \l 274E5
  \l 274E6
  \l 274E7
  \l 274E8
  \l 274E9
  \l 274EA
  \l 274EB
  \l 274EC
  \l 274ED
  \l 274EE
  \l 274EF
  \l 274F0
  \l 274F1
  \l 274F2
  \l 274F3
  \l 274F4
  \l 274F5
  \l 274F6
  \l 274F7
  \l 274F8
  \l 274F9
  \l 274FA
  \l 274FB
  \l 274FC
  \l 274FD
  \l 274FE
  \l 274FF
  \l 27500
  \l 27501
  \l 27502
  \l 27503
  \l 27504
  \l 27505
  \l 27506
  \l 27507
  \l 27508
  \l 27509
  \l 2750A
  \l 2750B
  \l 2750C
  \l 2750D
  \l 2750E
  \l 2750F
  \l 27510
  \l 27511
  \l 27512
  \l 27513
  \l 27514
  \l 27515
  \l 27516
  \l 27517
  \l 27518
  \l 27519
  \l 2751A
  \l 2751B
  \l 2751C
  \l 2751D
  \l 2751E
  \l 2751F
  \l 27520
  \l 27521
  \l 27522
  \l 27523
  \l 27524
  \l 27525
  \l 27526
  \l 27527
  \l 27528
  \l 27529
  \l 2752A
  \l 2752B
  \l 2752C
  \l 2752D
  \l 2752E
  \l 2752F
  \l 27530
  \l 27531
  \l 27532
  \l 27533
  \l 27534
  \l 27535
  \l 27536
  \l 27537
  \l 27538
  \l 27539
  \l 2753A
  \l 2753B
  \l 2753C
  \l 2753D
  \l 2753E
  \l 2753F
  \l 27540
  \l 27541
  \l 27542
  \l 27543
  \l 27544
  \l 27545
  \l 27546
  \l 27547
  \l 27548
  \l 27549
  \l 2754A
  \l 2754B
  \l 2754C
  \l 2754D
  \l 2754E
  \l 2754F
  \l 27550
  \l 27551
  \l 27552
  \l 27553
  \l 27554
  \l 27555
  \l 27556
  \l 27557
  \l 27558
  \l 27559
  \l 2755A
  \l 2755B
  \l 2755C
  \l 2755D
  \l 2755E
  \l 2755F
  \l 27560
  \l 27561
  \l 27562
  \l 27563
  \l 27564
  \l 27565
  \l 27566
  \l 27567
  \l 27568
  \l 27569
  \l 2756A
  \l 2756B
  \l 2756C
  \l 2756D
  \l 2756E
  \l 2756F
  \l 27570
  \l 27571
  \l 27572
  \l 27573
  \l 27574
  \l 27575
  \l 27576
  \l 27577
  \l 27578
  \l 27579
  \l 2757A
  \l 2757B
  \l 2757C
  \l 2757D
  \l 2757E
  \l 2757F
  \l 27580
  \l 27581
  \l 27582
  \l 27583
  \l 27584
  \l 27585
  \l 27586
  \l 27587
  \l 27588
  \l 27589
  \l 2758A
  \l 2758B
  \l 2758C
  \l 2758D
  \l 2758E
  \l 2758F
  \l 27590
  \l 27591
  \l 27592
  \l 27593
  \l 27594
  \l 27595
  \l 27596
  \l 27597
  \l 27598
  \l 27599
  \l 2759A
  \l 2759B
  \l 2759C
  \l 2759D
  \l 2759E
  \l 2759F
  \l 275A0
  \l 275A1
  \l 275A2
  \l 275A3
  \l 275A4
  \l 275A5
  \l 275A6
  \l 275A7
  \l 275A8
  \l 275A9
  \l 275AA
  \l 275AB
  \l 275AC
  \l 275AD
  \l 275AE
  \l 275AF
  \l 275B0
  \l 275B1
  \l 275B2
  \l 275B3
  \l 275B4
  \l 275B5
  \l 275B6
  \l 275B7
  \l 275B8
  \l 275B9
  \l 275BA
  \l 275BB
  \l 275BC
  \l 275BD
  \l 275BE
  \l 275BF
  \l 275C0
  \l 275C1
  \l 275C2
  \l 275C3
  \l 275C4
  \l 275C5
  \l 275C6
  \l 275C7
  \l 275C8
  \l 275C9
  \l 275CA
  \l 275CB
  \l 275CC
  \l 275CD
  \l 275CE
  \l 275CF
  \l 275D0
  \l 275D1
  \l 275D2
  \l 275D3
  \l 275D4
  \l 275D5
  \l 275D6
  \l 275D7
  \l 275D8
  \l 275D9
  \l 275DA
  \l 275DB
  \l 275DC
  \l 275DD
  \l 275DE
  \l 275DF
  \l 275E0
  \l 275E1
  \l 275E2
  \l 275E3
  \l 275E4
  \l 275E5
  \l 275E6
  \l 275E7
  \l 275E8
  \l 275E9
  \l 275EA
  \l 275EB
  \l 275EC
  \l 275ED
  \l 275EE
  \l 275EF
  \l 275F0
  \l 275F1
  \l 275F2
  \l 275F3
  \l 275F4
  \l 275F5
  \l 275F6
  \l 275F7
  \l 275F8
  \l 275F9
  \l 275FA
  \l 275FB
  \l 275FC
  \l 275FD
  \l 275FE
  \l 275FF
  \l 27600
  \l 27601
  \l 27602
  \l 27603
  \l 27604
  \l 27605
  \l 27606
  \l 27607
  \l 27608
  \l 27609
  \l 2760A
  \l 2760B
  \l 2760C
  \l 2760D
  \l 2760E
  \l 2760F
  \l 27610
  \l 27611
  \l 27612
  \l 27613
  \l 27614
  \l 27615
  \l 27616
  \l 27617
  \l 27618
  \l 27619
  \l 2761A
  \l 2761B
  \l 2761C
  \l 2761D
  \l 2761E
  \l 2761F
  \l 27620
  \l 27621
  \l 27622
  \l 27623
  \l 27624
  \l 27625
  \l 27626
  \l 27627
  \l 27628
  \l 27629
  \l 2762A
  \l 2762B
  \l 2762C
  \l 2762D
  \l 2762E
  \l 2762F
  \l 27630
  \l 27631
  \l 27632
  \l 27633
  \l 27634
  \l 27635
  \l 27636
  \l 27637
  \l 27638
  \l 27639
  \l 2763A
  \l 2763B
  \l 2763C
  \l 2763D
  \l 2763E
  \l 2763F
  \l 27640
  \l 27641
  \l 27642
  \l 27643
  \l 27644
  \l 27645
  \l 27646
  \l 27647
  \l 27648
  \l 27649
  \l 2764A
  \l 2764B
  \l 2764C
  \l 2764D
  \l 2764E
  \l 2764F
  \l 27650
  \l 27651
  \l 27652
  \l 27653
  \l 27654
  \l 27655
  \l 27656
  \l 27657
  \l 27658
  \l 27659
  \l 2765A
  \l 2765B
  \l 2765C
  \l 2765D
  \l 2765E
  \l 2765F
  \l 27660
  \l 27661
  \l 27662
  \l 27663
  \l 27664
  \l 27665
  \l 27666
  \l 27667
  \l 27668
  \l 27669
  \l 2766A
  \l 2766B
  \l 2766C
  \l 2766D
  \l 2766E
  \l 2766F
  \l 27670
  \l 27671
  \l 27672
  \l 27673
  \l 27674
  \l 27675
  \l 27676
  \l 27677
  \l 27678
  \l 27679
  \l 2767A
  \l 2767B
  \l 2767C
  \l 2767D
  \l 2767E
  \l 2767F
  \l 27680
  \l 27681
  \l 27682
  \l 27683
  \l 27684
  \l 27685
  \l 27686
  \l 27687
  \l 27688
  \l 27689
  \l 2768A
  \l 2768B
  \l 2768C
  \l 2768D
  \l 2768E
  \l 2768F
  \l 27690
  \l 27691
  \l 27692
  \l 27693
  \l 27694
  \l 27695
  \l 27696
  \l 27697
  \l 27698
  \l 27699
  \l 2769A
  \l 2769B
  \l 2769C
  \l 2769D
  \l 2769E
  \l 2769F
  \l 276A0
  \l 276A1
  \l 276A2
  \l 276A3
  \l 276A4
  \l 276A5
  \l 276A6
  \l 276A7
  \l 276A8
  \l 276A9
  \l 276AA
  \l 276AB
  \l 276AC
  \l 276AD
  \l 276AE
  \l 276AF
  \l 276B0
  \l 276B1
  \l 276B2
  \l 276B3
  \l 276B4
  \l 276B5
  \l 276B6
  \l 276B7
  \l 276B8
  \l 276B9
  \l 276BA
  \l 276BB
  \l 276BC
  \l 276BD
  \l 276BE
  \l 276BF
  \l 276C0
  \l 276C1
  \l 276C2
  \l 276C3
  \l 276C4
  \l 276C5
  \l 276C6
  \l 276C7
  \l 276C8
  \l 276C9
  \l 276CA
  \l 276CB
  \l 276CC
  \l 276CD
  \l 276CE
  \l 276CF
  \l 276D0
  \l 276D1
  \l 276D2
  \l 276D3
  \l 276D4
  \l 276D5
  \l 276D6
  \l 276D7
  \l 276D8
  \l 276D9
  \l 276DA
  \l 276DB
  \l 276DC
  \l 276DD
  \l 276DE
  \l 276DF
  \l 276E0
  \l 276E1
  \l 276E2
  \l 276E3
  \l 276E4
  \l 276E5
  \l 276E6
  \l 276E7
  \l 276E8
  \l 276E9
  \l 276EA
  \l 276EB
  \l 276EC
  \l 276ED
  \l 276EE
  \l 276EF
  \l 276F0
  \l 276F1
  \l 276F2
  \l 276F3
  \l 276F4
  \l 276F5
  \l 276F6
  \l 276F7
  \l 276F8
  \l 276F9
  \l 276FA
  \l 276FB
  \l 276FC
  \l 276FD
  \l 276FE
  \l 276FF
  \l 27700
  \l 27701
  \l 27702
  \l 27703
  \l 27704
  \l 27705
  \l 27706
  \l 27707
  \l 27708
  \l 27709
  \l 2770A
  \l 2770B
  \l 2770C
  \l 2770D
  \l 2770E
  \l 2770F
  \l 27710
  \l 27711
  \l 27712
  \l 27713
  \l 27714
  \l 27715
  \l 27716
  \l 27717
  \l 27718
  \l 27719
  \l 2771A
  \l 2771B
  \l 2771C
  \l 2771D
  \l 2771E
  \l 2771F
  \l 27720
  \l 27721
  \l 27722
  \l 27723
  \l 27724
  \l 27725
  \l 27726
  \l 27727
  \l 27728
  \l 27729
  \l 2772A
  \l 2772B
  \l 2772C
  \l 2772D
  \l 2772E
  \l 2772F
  \l 27730
  \l 27731
  \l 27732
  \l 27733
  \l 27734
  \l 27735
  \l 27736
  \l 27737
  \l 27738
  \l 27739
  \l 2773A
  \l 2773B
  \l 2773C
  \l 2773D
  \l 2773E
  \l 2773F
  \l 27740
  \l 27741
  \l 27742
  \l 27743
  \l 27744
  \l 27745
  \l 27746
  \l 27747
  \l 27748
  \l 27749
  \l 2774A
  \l 2774B
  \l 2774C
  \l 2774D
  \l 2774E
  \l 2774F
  \l 27750
  \l 27751
  \l 27752
  \l 27753
  \l 27754
  \l 27755
  \l 27756
  \l 27757
  \l 27758
  \l 27759
  \l 2775A
  \l 2775B
  \l 2775C
  \l 2775D
  \l 2775E
  \l 2775F
  \l 27760
  \l 27761
  \l 27762
  \l 27763
  \l 27764
  \l 27765
  \l 27766
  \l 27767
  \l 27768
  \l 27769
  \l 2776A
  \l 2776B
  \l 2776C
  \l 2776D
  \l 2776E
  \l 2776F
  \l 27770
  \l 27771
  \l 27772
  \l 27773
  \l 27774
  \l 27775
  \l 27776
  \l 27777
  \l 27778
  \l 27779
  \l 2777A
  \l 2777B
  \l 2777C
  \l 2777D
  \l 2777E
  \l 2777F
  \l 27780
  \l 27781
  \l 27782
  \l 27783
  \l 27784
  \l 27785
  \l 27786
  \l 27787
  \l 27788
  \l 27789
  \l 2778A
  \l 2778B
  \l 2778C
  \l 2778D
  \l 2778E
  \l 2778F
  \l 27790
  \l 27791
  \l 27792
  \l 27793
  \l 27794
  \l 27795
  \l 27796
  \l 27797
  \l 27798
  \l 27799
  \l 2779A
  \l 2779B
  \l 2779C
  \l 2779D
  \l 2779E
  \l 2779F
  \l 277A0
  \l 277A1
  \l 277A2
  \l 277A3
  \l 277A4
  \l 277A5
  \l 277A6
  \l 277A7
  \l 277A8
  \l 277A9
  \l 277AA
  \l 277AB
  \l 277AC
  \l 277AD
  \l 277AE
  \l 277AF
  \l 277B0
  \l 277B1
  \l 277B2
  \l 277B3
  \l 277B4
  \l 277B5
  \l 277B6
  \l 277B7
  \l 277B8
  \l 277B9
  \l 277BA
  \l 277BB
  \l 277BC
  \l 277BD
  \l 277BE
  \l 277BF
  \l 277C0
  \l 277C1
  \l 277C2
  \l 277C3
  \l 277C4
  \l 277C5
  \l 277C6
  \l 277C7
  \l 277C8
  \l 277C9
  \l 277CA
  \l 277CB
  \l 277CC
  \l 277CD
  \l 277CE
  \l 277CF
  \l 277D0
  \l 277D1
  \l 277D2
  \l 277D3
  \l 277D4
  \l 277D5
  \l 277D6
  \l 277D7
  \l 277D8
  \l 277D9
  \l 277DA
  \l 277DB
  \l 277DC
  \l 277DD
  \l 277DE
  \l 277DF
  \l 277E0
  \l 277E1
  \l 277E2
  \l 277E3
  \l 277E4
  \l 277E5
  \l 277E6
  \l 277E7
  \l 277E8
  \l 277E9
  \l 277EA
  \l 277EB
  \l 277EC
  \l 277ED
  \l 277EE
  \l 277EF
  \l 277F0
  \l 277F1
  \l 277F2
  \l 277F3
  \l 277F4
  \l 277F5
  \l 277F6
  \l 277F7
  \l 277F8
  \l 277F9
  \l 277FA
  \l 277FB
  \l 277FC
  \l 277FD
  \l 277FE
  \l 277FF
  \l 27800
  \l 27801
  \l 27802
  \l 27803
  \l 27804
  \l 27805
  \l 27806
  \l 27807
  \l 27808
  \l 27809
  \l 2780A
  \l 2780B
  \l 2780C
  \l 2780D
  \l 2780E
  \l 2780F
  \l 27810
  \l 27811
  \l 27812
  \l 27813
  \l 27814
  \l 27815
  \l 27816
  \l 27817
  \l 27818
  \l 27819
  \l 2781A
  \l 2781B
  \l 2781C
  \l 2781D
  \l 2781E
  \l 2781F
  \l 27820
  \l 27821
  \l 27822
  \l 27823
  \l 27824
  \l 27825
  \l 27826
  \l 27827
  \l 27828
  \l 27829
  \l 2782A
  \l 2782B
  \l 2782C
  \l 2782D
  \l 2782E
  \l 2782F
  \l 27830
  \l 27831
  \l 27832
  \l 27833
  \l 27834
  \l 27835
  \l 27836
  \l 27837
  \l 27838
  \l 27839
  \l 2783A
  \l 2783B
  \l 2783C
  \l 2783D
  \l 2783E
  \l 2783F
  \l 27840
  \l 27841
  \l 27842
  \l 27843
  \l 27844
  \l 27845
  \l 27846
  \l 27847
  \l 27848
  \l 27849
  \l 2784A
  \l 2784B
  \l 2784C
  \l 2784D
  \l 2784E
  \l 2784F
  \l 27850
  \l 27851
  \l 27852
  \l 27853
  \l 27854
  \l 27855
  \l 27856
  \l 27857
  \l 27858
  \l 27859
  \l 2785A
  \l 2785B
  \l 2785C
  \l 2785D
  \l 2785E
  \l 2785F
  \l 27860
  \l 27861
  \l 27862
  \l 27863
  \l 27864
  \l 27865
  \l 27866
  \l 27867
  \l 27868
  \l 27869
  \l 2786A
  \l 2786B
  \l 2786C
  \l 2786D
  \l 2786E
  \l 2786F
  \l 27870
  \l 27871
  \l 27872
  \l 27873
  \l 27874
  \l 27875
  \l 27876
  \l 27877
  \l 27878
  \l 27879
  \l 2787A
  \l 2787B
  \l 2787C
  \l 2787D
  \l 2787E
  \l 2787F
  \l 27880
  \l 27881
  \l 27882
  \l 27883
  \l 27884
  \l 27885
  \l 27886
  \l 27887
  \l 27888
  \l 27889
  \l 2788A
  \l 2788B
  \l 2788C
  \l 2788D
  \l 2788E
  \l 2788F
  \l 27890
  \l 27891
  \l 27892
  \l 27893
  \l 27894
  \l 27895
  \l 27896
  \l 27897
  \l 27898
  \l 27899
  \l 2789A
  \l 2789B
  \l 2789C
  \l 2789D
  \l 2789E
  \l 2789F
  \l 278A0
  \l 278A1
  \l 278A2
  \l 278A3
  \l 278A4
  \l 278A5
  \l 278A6
  \l 278A7
  \l 278A8
  \l 278A9
  \l 278AA
  \l 278AB
  \l 278AC
  \l 278AD
  \l 278AE
  \l 278AF
  \l 278B0
  \l 278B1
  \l 278B2
  \l 278B3
  \l 278B4
  \l 278B5
  \l 278B6
  \l 278B7
  \l 278B8
  \l 278B9
  \l 278BA
  \l 278BB
  \l 278BC
  \l 278BD
  \l 278BE
  \l 278BF
  \l 278C0
  \l 278C1
  \l 278C2
  \l 278C3
  \l 278C4
  \l 278C5
  \l 278C6
  \l 278C7
  \l 278C8
  \l 278C9
  \l 278CA
  \l 278CB
  \l 278CC
  \l 278CD
  \l 278CE
  \l 278CF
  \l 278D0
  \l 278D1
  \l 278D2
  \l 278D3
  \l 278D4
  \l 278D5
  \l 278D6
  \l 278D7
  \l 278D8
  \l 278D9
  \l 278DA
  \l 278DB
  \l 278DC
  \l 278DD
  \l 278DE
  \l 278DF
  \l 278E0
  \l 278E1
  \l 278E2
  \l 278E3
  \l 278E4
  \l 278E5
  \l 278E6
  \l 278E7
  \l 278E8
  \l 278E9
  \l 278EA
  \l 278EB
  \l 278EC
  \l 278ED
  \l 278EE
  \l 278EF
  \l 278F0
  \l 278F1
  \l 278F2
  \l 278F3
  \l 278F4
  \l 278F5
  \l 278F6
  \l 278F7
  \l 278F8
  \l 278F9
  \l 278FA
  \l 278FB
  \l 278FC
  \l 278FD
  \l 278FE
  \l 278FF
  \l 27900
  \l 27901
  \l 27902
  \l 27903
  \l 27904
  \l 27905
  \l 27906
  \l 27907
  \l 27908
  \l 27909
  \l 2790A
  \l 2790B
  \l 2790C
  \l 2790D
  \l 2790E
  \l 2790F
  \l 27910
  \l 27911
  \l 27912
  \l 27913
  \l 27914
  \l 27915
  \l 27916
  \l 27917
  \l 27918
  \l 27919
  \l 2791A
  \l 2791B
  \l 2791C
  \l 2791D
  \l 2791E
  \l 2791F
  \l 27920
  \l 27921
  \l 27922
  \l 27923
  \l 27924
  \l 27925
  \l 27926
  \l 27927
  \l 27928
  \l 27929
  \l 2792A
  \l 2792B
  \l 2792C
  \l 2792D
  \l 2792E
  \l 2792F
  \l 27930
  \l 27931
  \l 27932
  \l 27933
  \l 27934
  \l 27935
  \l 27936
  \l 27937
  \l 27938
  \l 27939
  \l 2793A
  \l 2793B
  \l 2793C
  \l 2793D
  \l 2793E
  \l 2793F
  \l 27940
  \l 27941
  \l 27942
  \l 27943
  \l 27944
  \l 27945
  \l 27946
  \l 27947
  \l 27948
  \l 27949
  \l 2794A
  \l 2794B
  \l 2794C
  \l 2794D
  \l 2794E
  \l 2794F
  \l 27950
  \l 27951
  \l 27952
  \l 27953
  \l 27954
  \l 27955
  \l 27956
  \l 27957
  \l 27958
  \l 27959
  \l 2795A
  \l 2795B
  \l 2795C
  \l 2795D
  \l 2795E
  \l 2795F
  \l 27960
  \l 27961
  \l 27962
  \l 27963
  \l 27964
  \l 27965
  \l 27966
  \l 27967
  \l 27968
  \l 27969
  \l 2796A
  \l 2796B
  \l 2796C
  \l 2796D
  \l 2796E
  \l 2796F
  \l 27970
  \l 27971
  \l 27972
  \l 27973
  \l 27974
  \l 27975
  \l 27976
  \l 27977
  \l 27978
  \l 27979
  \l 2797A
  \l 2797B
  \l 2797C
  \l 2797D
  \l 2797E
  \l 2797F
  \l 27980
  \l 27981
  \l 27982
  \l 27983
  \l 27984
  \l 27985
  \l 27986
  \l 27987
  \l 27988
  \l 27989
  \l 2798A
  \l 2798B
  \l 2798C
  \l 2798D
  \l 2798E
  \l 2798F
  \l 27990
  \l 27991
  \l 27992
  \l 27993
  \l 27994
  \l 27995
  \l 27996
  \l 27997
  \l 27998
  \l 27999
  \l 2799A
  \l 2799B
  \l 2799C
  \l 2799D
  \l 2799E
  \l 2799F
  \l 279A0
  \l 279A1
  \l 279A2
  \l 279A3
  \l 279A4
  \l 279A5
  \l 279A6
  \l 279A7
  \l 279A8
  \l 279A9
  \l 279AA
  \l 279AB
  \l 279AC
  \l 279AD
  \l 279AE
  \l 279AF
  \l 279B0
  \l 279B1
  \l 279B2
  \l 279B3
  \l 279B4
  \l 279B5
  \l 279B6
  \l 279B7
  \l 279B8
  \l 279B9
  \l 279BA
  \l 279BB
  \l 279BC
  \l 279BD
  \l 279BE
  \l 279BF
  \l 279C0
  \l 279C1
  \l 279C2
  \l 279C3
  \l 279C4
  \l 279C5
  \l 279C6
  \l 279C7
  \l 279C8
  \l 279C9
  \l 279CA
  \l 279CB
  \l 279CC
  \l 279CD
  \l 279CE
  \l 279CF
  \l 279D0
  \l 279D1
  \l 279D2
  \l 279D3
  \l 279D4
  \l 279D5
  \l 279D6
  \l 279D7
  \l 279D8
  \l 279D9
  \l 279DA
  \l 279DB
  \l 279DC
  \l 279DD
  \l 279DE
  \l 279DF
  \l 279E0
  \l 279E1
  \l 279E2
  \l 279E3
  \l 279E4
  \l 279E5
  \l 279E6
  \l 279E7
  \l 279E8
  \l 279E9
  \l 279EA
  \l 279EB
  \l 279EC
  \l 279ED
  \l 279EE
  \l 279EF
  \l 279F0
  \l 279F1
  \l 279F2
  \l 279F3
  \l 279F4
  \l 279F5
  \l 279F6
  \l 279F7
  \l 279F8
  \l 279F9
  \l 279FA
  \l 279FB
  \l 279FC
  \l 279FD
  \l 279FE
  \l 279FF
  \l 27A00
  \l 27A01
  \l 27A02
  \l 27A03
  \l 27A04
  \l 27A05
  \l 27A06
  \l 27A07
  \l 27A08
  \l 27A09
  \l 27A0A
  \l 27A0B
  \l 27A0C
  \l 27A0D
  \l 27A0E
  \l 27A0F
  \l 27A10
  \l 27A11
  \l 27A12
  \l 27A13
  \l 27A14
  \l 27A15
  \l 27A16
  \l 27A17
  \l 27A18
  \l 27A19
  \l 27A1A
  \l 27A1B
  \l 27A1C
  \l 27A1D
  \l 27A1E
  \l 27A1F
  \l 27A20
  \l 27A21
  \l 27A22
  \l 27A23
  \l 27A24
  \l 27A25
  \l 27A26
  \l 27A27
  \l 27A28
  \l 27A29
  \l 27A2A
  \l 27A2B
  \l 27A2C
  \l 27A2D
  \l 27A2E
  \l 27A2F
  \l 27A30
  \l 27A31
  \l 27A32
  \l 27A33
  \l 27A34
  \l 27A35
  \l 27A36
  \l 27A37
  \l 27A38
  \l 27A39
  \l 27A3A
  \l 27A3B
  \l 27A3C
  \l 27A3D
  \l 27A3E
  \l 27A3F
  \l 27A40
  \l 27A41
  \l 27A42
  \l 27A43
  \l 27A44
  \l 27A45
  \l 27A46
  \l 27A47
  \l 27A48
  \l 27A49
  \l 27A4A
  \l 27A4B
  \l 27A4C
  \l 27A4D
  \l 27A4E
  \l 27A4F
  \l 27A50
  \l 27A51
  \l 27A52
  \l 27A53
  \l 27A54
  \l 27A55
  \l 27A56
  \l 27A57
  \l 27A58
  \l 27A59
  \l 27A5A
  \l 27A5B
  \l 27A5C
  \l 27A5D
  \l 27A5E
  \l 27A5F
  \l 27A60
  \l 27A61
  \l 27A62
  \l 27A63
  \l 27A64
  \l 27A65
  \l 27A66
  \l 27A67
  \l 27A68
  \l 27A69
  \l 27A6A
  \l 27A6B
  \l 27A6C
  \l 27A6D
  \l 27A6E
  \l 27A6F
  \l 27A70
  \l 27A71
  \l 27A72
  \l 27A73
  \l 27A74
  \l 27A75
  \l 27A76
  \l 27A77
  \l 27A78
  \l 27A79
  \l 27A7A
  \l 27A7B
  \l 27A7C
  \l 27A7D
  \l 27A7E
  \l 27A7F
  \l 27A80
  \l 27A81
  \l 27A82
  \l 27A83
  \l 27A84
  \l 27A85
  \l 27A86
  \l 27A87
  \l 27A88
  \l 27A89
  \l 27A8A
  \l 27A8B
  \l 27A8C
  \l 27A8D
  \l 27A8E
  \l 27A8F
  \l 27A90
  \l 27A91
  \l 27A92
  \l 27A93
  \l 27A94
  \l 27A95
  \l 27A96
  \l 27A97
  \l 27A98
  \l 27A99
  \l 27A9A
  \l 27A9B
  \l 27A9C
  \l 27A9D
  \l 27A9E
  \l 27A9F
  \l 27AA0
  \l 27AA1
  \l 27AA2
  \l 27AA3
  \l 27AA4
  \l 27AA5
  \l 27AA6
  \l 27AA7
  \l 27AA8
  \l 27AA9
  \l 27AAA
  \l 27AAB
  \l 27AAC
  \l 27AAD
  \l 27AAE
  \l 27AAF
  \l 27AB0
  \l 27AB1
  \l 27AB2
  \l 27AB3
  \l 27AB4
  \l 27AB5
  \l 27AB6
  \l 27AB7
  \l 27AB8
  \l 27AB9
  \l 27ABA
  \l 27ABB
  \l 27ABC
  \l 27ABD
  \l 27ABE
  \l 27ABF
  \l 27AC0
  \l 27AC1
  \l 27AC2
  \l 27AC3
  \l 27AC4
  \l 27AC5
  \l 27AC6
  \l 27AC7
  \l 27AC8
  \l 27AC9
  \l 27ACA
  \l 27ACB
  \l 27ACC
  \l 27ACD
  \l 27ACE
  \l 27ACF
  \l 27AD0
  \l 27AD1
  \l 27AD2
  \l 27AD3
  \l 27AD4
  \l 27AD5
  \l 27AD6
  \l 27AD7
  \l 27AD8
  \l 27AD9
  \l 27ADA
  \l 27ADB
  \l 27ADC
  \l 27ADD
  \l 27ADE
  \l 27ADF
  \l 27AE0
  \l 27AE1
  \l 27AE2
  \l 27AE3
  \l 27AE4
  \l 27AE5
  \l 27AE6
  \l 27AE7
  \l 27AE8
  \l 27AE9
  \l 27AEA
  \l 27AEB
  \l 27AEC
  \l 27AED
  \l 27AEE
  \l 27AEF
  \l 27AF0
  \l 27AF1
  \l 27AF2
  \l 27AF3
  \l 27AF4
  \l 27AF5
  \l 27AF6
  \l 27AF7
  \l 27AF8
  \l 27AF9
  \l 27AFA
  \l 27AFB
  \l 27AFC
  \l 27AFD
  \l 27AFE
  \l 27AFF
  \l 27B00
  \l 27B01
  \l 27B02
  \l 27B03
  \l 27B04
  \l 27B05
  \l 27B06
  \l 27B07
  \l 27B08
  \l 27B09
  \l 27B0A
  \l 27B0B
  \l 27B0C
  \l 27B0D
  \l 27B0E
  \l 27B0F
  \l 27B10
  \l 27B11
  \l 27B12
  \l 27B13
  \l 27B14
  \l 27B15
  \l 27B16
  \l 27B17
  \l 27B18
  \l 27B19
  \l 27B1A
  \l 27B1B
  \l 27B1C
  \l 27B1D
  \l 27B1E
  \l 27B1F
  \l 27B20
  \l 27B21
  \l 27B22
  \l 27B23
  \l 27B24
  \l 27B25
  \l 27B26
  \l 27B27
  \l 27B28
  \l 27B29
  \l 27B2A
  \l 27B2B
  \l 27B2C
  \l 27B2D
  \l 27B2E
  \l 27B2F
  \l 27B30
  \l 27B31
  \l 27B32
  \l 27B33
  \l 27B34
  \l 27B35
  \l 27B36
  \l 27B37
  \l 27B38
  \l 27B39
  \l 27B3A
  \l 27B3B
  \l 27B3C
  \l 27B3D
  \l 27B3E
  \l 27B3F
  \l 27B40
  \l 27B41
  \l 27B42
  \l 27B43
  \l 27B44
  \l 27B45
  \l 27B46
  \l 27B47
  \l 27B48
  \l 27B49
  \l 27B4A
  \l 27B4B
  \l 27B4C
  \l 27B4D
  \l 27B4E
  \l 27B4F
  \l 27B50
  \l 27B51
  \l 27B52
  \l 27B53
  \l 27B54
  \l 27B55
  \l 27B56
  \l 27B57
  \l 27B58
  \l 27B59
  \l 27B5A
  \l 27B5B
  \l 27B5C
  \l 27B5D
  \l 27B5E
  \l 27B5F
  \l 27B60
  \l 27B61
  \l 27B62
  \l 27B63
  \l 27B64
  \l 27B65
  \l 27B66
  \l 27B67
  \l 27B68
  \l 27B69
  \l 27B6A
  \l 27B6B
  \l 27B6C
  \l 27B6D
  \l 27B6E
  \l 27B6F
  \l 27B70
  \l 27B71
  \l 27B72
  \l 27B73
  \l 27B74
  \l 27B75
  \l 27B76
  \l 27B77
  \l 27B78
  \l 27B79
  \l 27B7A
  \l 27B7B
  \l 27B7C
  \l 27B7D
  \l 27B7E
  \l 27B7F
  \l 27B80
  \l 27B81
  \l 27B82
  \l 27B83
  \l 27B84
  \l 27B85
  \l 27B86
  \l 27B87
  \l 27B88
  \l 27B89
  \l 27B8A
  \l 27B8B
  \l 27B8C
  \l 27B8D
  \l 27B8E
  \l 27B8F
  \l 27B90
  \l 27B91
  \l 27B92
  \l 27B93
  \l 27B94
  \l 27B95
  \l 27B96
  \l 27B97
  \l 27B98
  \l 27B99
  \l 27B9A
  \l 27B9B
  \l 27B9C
  \l 27B9D
  \l 27B9E
  \l 27B9F
  \l 27BA0
  \l 27BA1
  \l 27BA2
  \l 27BA3
  \l 27BA4
  \l 27BA5
  \l 27BA6
  \l 27BA7
  \l 27BA8
  \l 27BA9
  \l 27BAA
  \l 27BAB
  \l 27BAC
  \l 27BAD
  \l 27BAE
  \l 27BAF
  \l 27BB0
  \l 27BB1
  \l 27BB2
  \l 27BB3
  \l 27BB4
  \l 27BB5
  \l 27BB6
  \l 27BB7
  \l 27BB8
  \l 27BB9
  \l 27BBA
  \l 27BBB
  \l 27BBC
  \l 27BBD
  \l 27BBE
  \l 27BBF
  \l 27BC0
  \l 27BC1
  \l 27BC2
  \l 27BC3
  \l 27BC4
  \l 27BC5
  \l 27BC6
  \l 27BC7
  \l 27BC8
  \l 27BC9
  \l 27BCA
  \l 27BCB
  \l 27BCC
  \l 27BCD
  \l 27BCE
  \l 27BCF
  \l 27BD0
  \l 27BD1
  \l 27BD2
  \l 27BD3
  \l 27BD4
  \l 27BD5
  \l 27BD6
  \l 27BD7
  \l 27BD8
  \l 27BD9
  \l 27BDA
  \l 27BDB
  \l 27BDC
  \l 27BDD
  \l 27BDE
  \l 27BDF
  \l 27BE0
  \l 27BE1
  \l 27BE2
  \l 27BE3
  \l 27BE4
  \l 27BE5
  \l 27BE6
  \l 27BE7
  \l 27BE8
  \l 27BE9
  \l 27BEA
  \l 27BEB
  \l 27BEC
  \l 27BED
  \l 27BEE
  \l 27BEF
  \l 27BF0
  \l 27BF1
  \l 27BF2
  \l 27BF3
  \l 27BF4
  \l 27BF5
  \l 27BF6
  \l 27BF7
  \l 27BF8
  \l 27BF9
  \l 27BFA
  \l 27BFB
  \l 27BFC
  \l 27BFD
  \l 27BFE
  \l 27BFF
  \l 27C00
  \l 27C01
  \l 27C02
  \l 27C03
  \l 27C04
  \l 27C05
  \l 27C06
  \l 27C07
  \l 27C08
  \l 27C09
  \l 27C0A
  \l 27C0B
  \l 27C0C
  \l 27C0D
  \l 27C0E
  \l 27C0F
  \l 27C10
  \l 27C11
  \l 27C12
  \l 27C13
  \l 27C14
  \l 27C15
  \l 27C16
  \l 27C17
  \l 27C18
  \l 27C19
  \l 27C1A
  \l 27C1B
  \l 27C1C
  \l 27C1D
  \l 27C1E
  \l 27C1F
  \l 27C20
  \l 27C21
  \l 27C22
  \l 27C23
  \l 27C24
  \l 27C25
  \l 27C26
  \l 27C27
  \l 27C28
  \l 27C29
  \l 27C2A
  \l 27C2B
  \l 27C2C
  \l 27C2D
  \l 27C2E
  \l 27C2F
  \l 27C30
  \l 27C31
  \l 27C32
  \l 27C33
  \l 27C34
  \l 27C35
  \l 27C36
  \l 27C37
  \l 27C38
  \l 27C39
  \l 27C3A
  \l 27C3B
  \l 27C3C
  \l 27C3D
  \l 27C3E
  \l 27C3F
  \l 27C40
  \l 27C41
  \l 27C42
  \l 27C43
  \l 27C44
  \l 27C45
  \l 27C46
  \l 27C47
  \l 27C48
  \l 27C49
  \l 27C4A
  \l 27C4B
  \l 27C4C
  \l 27C4D
  \l 27C4E
  \l 27C4F
  \l 27C50
  \l 27C51
  \l 27C52
  \l 27C53
  \l 27C54
  \l 27C55
  \l 27C56
  \l 27C57
  \l 27C58
  \l 27C59
  \l 27C5A
  \l 27C5B
  \l 27C5C
  \l 27C5D
  \l 27C5E
  \l 27C5F
  \l 27C60
  \l 27C61
  \l 27C62
  \l 27C63
  \l 27C64
  \l 27C65
  \l 27C66
  \l 27C67
  \l 27C68
  \l 27C69
  \l 27C6A
  \l 27C6B
  \l 27C6C
  \l 27C6D
  \l 27C6E
  \l 27C6F
  \l 27C70
  \l 27C71
  \l 27C72
  \l 27C73
  \l 27C74
  \l 27C75
  \l 27C76
  \l 27C77
  \l 27C78
  \l 27C79
  \l 27C7A
  \l 27C7B
  \l 27C7C
  \l 27C7D
  \l 27C7E
  \l 27C7F
  \l 27C80
  \l 27C81
  \l 27C82
  \l 27C83
  \l 27C84
  \l 27C85
  \l 27C86
  \l 27C87
  \l 27C88
  \l 27C89
  \l 27C8A
  \l 27C8B
  \l 27C8C
  \l 27C8D
  \l 27C8E
  \l 27C8F
  \l 27C90
  \l 27C91
  \l 27C92
  \l 27C93
  \l 27C94
  \l 27C95
  \l 27C96
  \l 27C97
  \l 27C98
  \l 27C99
  \l 27C9A
  \l 27C9B
  \l 27C9C
  \l 27C9D
  \l 27C9E
  \l 27C9F
  \l 27CA0
  \l 27CA1
  \l 27CA2
  \l 27CA3
  \l 27CA4
  \l 27CA5
  \l 27CA6
  \l 27CA7
  \l 27CA8
  \l 27CA9
  \l 27CAA
  \l 27CAB
  \l 27CAC
  \l 27CAD
  \l 27CAE
  \l 27CAF
  \l 27CB0
  \l 27CB1
  \l 27CB2
  \l 27CB3
  \l 27CB4
  \l 27CB5
  \l 27CB6
  \l 27CB7
  \l 27CB8
  \l 27CB9
  \l 27CBA
  \l 27CBB
  \l 27CBC
  \l 27CBD
  \l 27CBE
  \l 27CBF
  \l 27CC0
  \l 27CC1
  \l 27CC2
  \l 27CC3
  \l 27CC4
  \l 27CC5
  \l 27CC6
  \l 27CC7
  \l 27CC8
  \l 27CC9
  \l 27CCA
  \l 27CCB
  \l 27CCC
  \l 27CCD
  \l 27CCE
  \l 27CCF
  \l 27CD0
  \l 27CD1
  \l 27CD2
  \l 27CD3
  \l 27CD4
  \l 27CD5
  \l 27CD6
  \l 27CD7
  \l 27CD8
  \l 27CD9
  \l 27CDA
  \l 27CDB
  \l 27CDC
  \l 27CDD
  \l 27CDE
  \l 27CDF
  \l 27CE0
  \l 27CE1
  \l 27CE2
  \l 27CE3
  \l 27CE4
  \l 27CE5
  \l 27CE6
  \l 27CE7
  \l 27CE8
  \l 27CE9
  \l 27CEA
  \l 27CEB
  \l 27CEC
  \l 27CED
  \l 27CEE
  \l 27CEF
  \l 27CF0
  \l 27CF1
  \l 27CF2
  \l 27CF3
  \l 27CF4
  \l 27CF5
  \l 27CF6
  \l 27CF7
  \l 27CF8
  \l 27CF9
  \l 27CFA
  \l 27CFB
  \l 27CFC
  \l 27CFD
  \l 27CFE
  \l 27CFF
  \l 27D00
  \l 27D01
  \l 27D02
  \l 27D03
  \l 27D04
  \l 27D05
  \l 27D06
  \l 27D07
  \l 27D08
  \l 27D09
  \l 27D0A
  \l 27D0B
  \l 27D0C
  \l 27D0D
  \l 27D0E
  \l 27D0F
  \l 27D10
  \l 27D11
  \l 27D12
  \l 27D13
  \l 27D14
  \l 27D15
  \l 27D16
  \l 27D17
  \l 27D18
  \l 27D19
  \l 27D1A
  \l 27D1B
  \l 27D1C
  \l 27D1D
  \l 27D1E
  \l 27D1F
  \l 27D20
  \l 27D21
  \l 27D22
  \l 27D23
  \l 27D24
  \l 27D25
  \l 27D26
  \l 27D27
  \l 27D28
  \l 27D29
  \l 27D2A
  \l 27D2B
  \l 27D2C
  \l 27D2D
  \l 27D2E
  \l 27D2F
  \l 27D30
  \l 27D31
  \l 27D32
  \l 27D33
  \l 27D34
  \l 27D35
  \l 27D36
  \l 27D37
  \l 27D38
  \l 27D39
  \l 27D3A
  \l 27D3B
  \l 27D3C
  \l 27D3D
  \l 27D3E
  \l 27D3F
  \l 27D40
  \l 27D41
  \l 27D42
  \l 27D43
  \l 27D44
  \l 27D45
  \l 27D46
  \l 27D47
  \l 27D48
  \l 27D49
  \l 27D4A
  \l 27D4B
  \l 27D4C
  \l 27D4D
  \l 27D4E
  \l 27D4F
  \l 27D50
  \l 27D51
  \l 27D52
  \l 27D53
  \l 27D54
  \l 27D55
  \l 27D56
  \l 27D57
  \l 27D58
  \l 27D59
  \l 27D5A
  \l 27D5B
  \l 27D5C
  \l 27D5D
  \l 27D5E
  \l 27D5F
  \l 27D60
  \l 27D61
  \l 27D62
  \l 27D63
  \l 27D64
  \l 27D65
  \l 27D66
  \l 27D67
  \l 27D68
  \l 27D69
  \l 27D6A
  \l 27D6B
  \l 27D6C
  \l 27D6D
  \l 27D6E
  \l 27D6F
  \l 27D70
  \l 27D71
  \l 27D72
  \l 27D73
  \l 27D74
  \l 27D75
  \l 27D76
  \l 27D77
  \l 27D78
  \l 27D79
  \l 27D7A
  \l 27D7B
  \l 27D7C
  \l 27D7D
  \l 27D7E
  \l 27D7F
  \l 27D80
  \l 27D81
  \l 27D82
  \l 27D83
  \l 27D84
  \l 27D85
  \l 27D86
  \l 27D87
  \l 27D88
  \l 27D89
  \l 27D8A
  \l 27D8B
  \l 27D8C
  \l 27D8D
  \l 27D8E
  \l 27D8F
  \l 27D90
  \l 27D91
  \l 27D92
  \l 27D93
  \l 27D94
  \l 27D95
  \l 27D96
  \l 27D97
  \l 27D98
  \l 27D99
  \l 27D9A
  \l 27D9B
  \l 27D9C
  \l 27D9D
  \l 27D9E
  \l 27D9F
  \l 27DA0
  \l 27DA1
  \l 27DA2
  \l 27DA3
  \l 27DA4
  \l 27DA5
  \l 27DA6
  \l 27DA7
  \l 27DA8
  \l 27DA9
  \l 27DAA
  \l 27DAB
  \l 27DAC
  \l 27DAD
  \l 27DAE
  \l 27DAF
  \l 27DB0
  \l 27DB1
  \l 27DB2
  \l 27DB3
  \l 27DB4
  \l 27DB5
  \l 27DB6
  \l 27DB7
  \l 27DB8
  \l 27DB9
  \l 27DBA
  \l 27DBB
  \l 27DBC
  \l 27DBD
  \l 27DBE
  \l 27DBF
  \l 27DC0
  \l 27DC1
  \l 27DC2
  \l 27DC3
  \l 27DC4
  \l 27DC5
  \l 27DC6
  \l 27DC7
  \l 27DC8
  \l 27DC9
  \l 27DCA
  \l 27DCB
  \l 27DCC
  \l 27DCD
  \l 27DCE
  \l 27DCF
  \l 27DD0
  \l 27DD1
  \l 27DD2
  \l 27DD3
  \l 27DD4
  \l 27DD5
  \l 27DD6
  \l 27DD7
  \l 27DD8
  \l 27DD9
  \l 27DDA
  \l 27DDB
  \l 27DDC
  \l 27DDD
  \l 27DDE
  \l 27DDF
  \l 27DE0
  \l 27DE1
  \l 27DE2
  \l 27DE3
  \l 27DE4
  \l 27DE5
  \l 27DE6
  \l 27DE7
  \l 27DE8
  \l 27DE9
  \l 27DEA
  \l 27DEB
  \l 27DEC
  \l 27DED
  \l 27DEE
  \l 27DEF
  \l 27DF0
  \l 27DF1
  \l 27DF2
  \l 27DF3
  \l 27DF4
  \l 27DF5
  \l 27DF6
  \l 27DF7
  \l 27DF8
  \l 27DF9
  \l 27DFA
  \l 27DFB
  \l 27DFC
  \l 27DFD
  \l 27DFE
  \l 27DFF
  \l 27E00
  \l 27E01
  \l 27E02
  \l 27E03
  \l 27E04
  \l 27E05
  \l 27E06
  \l 27E07
  \l 27E08
  \l 27E09
  \l 27E0A
  \l 27E0B
  \l 27E0C
  \l 27E0D
  \l 27E0E
  \l 27E0F
  \l 27E10
  \l 27E11
  \l 27E12
  \l 27E13
  \l 27E14
  \l 27E15
  \l 27E16
  \l 27E17
  \l 27E18
  \l 27E19
  \l 27E1A
  \l 27E1B
  \l 27E1C
  \l 27E1D
  \l 27E1E
  \l 27E1F
  \l 27E20
  \l 27E21
  \l 27E22
  \l 27E23
  \l 27E24
  \l 27E25
  \l 27E26
  \l 27E27
  \l 27E28
  \l 27E29
  \l 27E2A
  \l 27E2B
  \l 27E2C
  \l 27E2D
  \l 27E2E
  \l 27E2F
  \l 27E30
  \l 27E31
  \l 27E32
  \l 27E33
  \l 27E34
  \l 27E35
  \l 27E36
  \l 27E37
  \l 27E38
  \l 27E39
  \l 27E3A
  \l 27E3B
  \l 27E3C
  \l 27E3D
  \l 27E3E
  \l 27E3F
  \l 27E40
  \l 27E41
  \l 27E42
  \l 27E43
  \l 27E44
  \l 27E45
  \l 27E46
  \l 27E47
  \l 27E48
  \l 27E49
  \l 27E4A
  \l 27E4B
  \l 27E4C
  \l 27E4D
  \l 27E4E
  \l 27E4F
  \l 27E50
  \l 27E51
  \l 27E52
  \l 27E53
  \l 27E54
  \l 27E55
  \l 27E56
  \l 27E57
  \l 27E58
  \l 27E59
  \l 27E5A
  \l 27E5B
  \l 27E5C
  \l 27E5D
  \l 27E5E
  \l 27E5F
  \l 27E60
  \l 27E61
  \l 27E62
  \l 27E63
  \l 27E64
  \l 27E65
  \l 27E66
  \l 27E67
  \l 27E68
  \l 27E69
  \l 27E6A
  \l 27E6B
  \l 27E6C
  \l 27E6D
  \l 27E6E
  \l 27E6F
  \l 27E70
  \l 27E71
  \l 27E72
  \l 27E73
  \l 27E74
  \l 27E75
  \l 27E76
  \l 27E77
  \l 27E78
  \l 27E79
  \l 27E7A
  \l 27E7B
  \l 27E7C
  \l 27E7D
  \l 27E7E
  \l 27E7F
  \l 27E80
  \l 27E81
  \l 27E82
  \l 27E83
  \l 27E84
  \l 27E85
  \l 27E86
  \l 27E87
  \l 27E88
  \l 27E89
  \l 27E8A
  \l 27E8B
  \l 27E8C
  \l 27E8D
  \l 27E8E
  \l 27E8F
  \l 27E90
  \l 27E91
  \l 27E92
  \l 27E93
  \l 27E94
  \l 27E95
  \l 27E96
  \l 27E97
  \l 27E98
  \l 27E99
  \l 27E9A
  \l 27E9B
  \l 27E9C
  \l 27E9D
  \l 27E9E
  \l 27E9F
  \l 27EA0
  \l 27EA1
  \l 27EA2
  \l 27EA3
  \l 27EA4
  \l 27EA5
  \l 27EA6
  \l 27EA7
  \l 27EA8
  \l 27EA9
  \l 27EAA
  \l 27EAB
  \l 27EAC
  \l 27EAD
  \l 27EAE
  \l 27EAF
  \l 27EB0
  \l 27EB1
  \l 27EB2
  \l 27EB3
  \l 27EB4
  \l 27EB5
  \l 27EB6
  \l 27EB7
  \l 27EB8
  \l 27EB9
  \l 27EBA
  \l 27EBB
  \l 27EBC
  \l 27EBD
  \l 27EBE
  \l 27EBF
  \l 27EC0
  \l 27EC1
  \l 27EC2
  \l 27EC3
  \l 27EC4
  \l 27EC5
  \l 27EC6
  \l 27EC7
  \l 27EC8
  \l 27EC9
  \l 27ECA
  \l 27ECB
  \l 27ECC
  \l 27ECD
  \l 27ECE
  \l 27ECF
  \l 27ED0
  \l 27ED1
  \l 27ED2
  \l 27ED3
  \l 27ED4
  \l 27ED5
  \l 27ED6
  \l 27ED7
  \l 27ED8
  \l 27ED9
  \l 27EDA
  \l 27EDB
  \l 27EDC
  \l 27EDD
  \l 27EDE
  \l 27EDF
  \l 27EE0
  \l 27EE1
  \l 27EE2
  \l 27EE3
  \l 27EE4
  \l 27EE5
  \l 27EE6
  \l 27EE7
  \l 27EE8
  \l 27EE9
  \l 27EEA
  \l 27EEB
  \l 27EEC
  \l 27EED
  \l 27EEE
  \l 27EEF
  \l 27EF0
  \l 27EF1
  \l 27EF2
  \l 27EF3
  \l 27EF4
  \l 27EF5
  \l 27EF6
  \l 27EF7
  \l 27EF8
  \l 27EF9
  \l 27EFA
  \l 27EFB
  \l 27EFC
  \l 27EFD
  \l 27EFE
  \l 27EFF
  \l 27F00
  \l 27F01
  \l 27F02
  \l 27F03
  \l 27F04
  \l 27F05
  \l 27F06
  \l 27F07
  \l 27F08
  \l 27F09
  \l 27F0A
  \l 27F0B
  \l 27F0C
  \l 27F0D
  \l 27F0E
  \l 27F0F
  \l 27F10
  \l 27F11
  \l 27F12
  \l 27F13
  \l 27F14
  \l 27F15
  \l 27F16
  \l 27F17
  \l 27F18
  \l 27F19
  \l 27F1A
  \l 27F1B
  \l 27F1C
  \l 27F1D
  \l 27F1E
  \l 27F1F
  \l 27F20
  \l 27F21
  \l 27F22
  \l 27F23
  \l 27F24
  \l 27F25
  \l 27F26
  \l 27F27
  \l 27F28
  \l 27F29
  \l 27F2A
  \l 27F2B
  \l 27F2C
  \l 27F2D
  \l 27F2E
  \l 27F2F
  \l 27F30
  \l 27F31
  \l 27F32
  \l 27F33
  \l 27F34
  \l 27F35
  \l 27F36
  \l 27F37
  \l 27F38
  \l 27F39
  \l 27F3A
  \l 27F3B
  \l 27F3C
  \l 27F3D
  \l 27F3E
  \l 27F3F
  \l 27F40
  \l 27F41
  \l 27F42
  \l 27F43
  \l 27F44
  \l 27F45
  \l 27F46
  \l 27F47
  \l 27F48
  \l 27F49
  \l 27F4A
  \l 27F4B
  \l 27F4C
  \l 27F4D
  \l 27F4E
  \l 27F4F
  \l 27F50
  \l 27F51
  \l 27F52
  \l 27F53
  \l 27F54
  \l 27F55
  \l 27F56
  \l 27F57
  \l 27F58
  \l 27F59
  \l 27F5A
  \l 27F5B
  \l 27F5C
  \l 27F5D
  \l 27F5E
  \l 27F5F
  \l 27F60
  \l 27F61
  \l 27F62
  \l 27F63
  \l 27F64
  \l 27F65
  \l 27F66
  \l 27F67
  \l 27F68
  \l 27F69
  \l 27F6A
  \l 27F6B
  \l 27F6C
  \l 27F6D
  \l 27F6E
  \l 27F6F
  \l 27F70
  \l 27F71
  \l 27F72
  \l 27F73
  \l 27F74
  \l 27F75
  \l 27F76
  \l 27F77
  \l 27F78
  \l 27F79
  \l 27F7A
  \l 27F7B
  \l 27F7C
  \l 27F7D
  \l 27F7E
  \l 27F7F
  \l 27F80
  \l 27F81
  \l 27F82
  \l 27F83
  \l 27F84
  \l 27F85
  \l 27F86
  \l 27F87
  \l 27F88
  \l 27F89
  \l 27F8A
  \l 27F8B
  \l 27F8C
  \l 27F8D
  \l 27F8E
  \l 27F8F
  \l 27F90
  \l 27F91
  \l 27F92
  \l 27F93
  \l 27F94
  \l 27F95
  \l 27F96
  \l 27F97
  \l 27F98
  \l 27F99
  \l 27F9A
  \l 27F9B
  \l 27F9C
  \l 27F9D
  \l 27F9E
  \l 27F9F
  \l 27FA0
  \l 27FA1
  \l 27FA2
  \l 27FA3
  \l 27FA4
  \l 27FA5
  \l 27FA6
  \l 27FA7
  \l 27FA8
  \l 27FA9
  \l 27FAA
  \l 27FAB
  \l 27FAC
  \l 27FAD
  \l 27FAE
  \l 27FAF
  \l 27FB0
  \l 27FB1
  \l 27FB2
  \l 27FB3
  \l 27FB4
  \l 27FB5
  \l 27FB6
  \l 27FB7
  \l 27FB8
  \l 27FB9
  \l 27FBA
  \l 27FBB
  \l 27FBC
  \l 27FBD
  \l 27FBE
  \l 27FBF
  \l 27FC0
  \l 27FC1
  \l 27FC2
  \l 27FC3
  \l 27FC4
  \l 27FC5
  \l 27FC6
  \l 27FC7
  \l 27FC8
  \l 27FC9
  \l 27FCA
  \l 27FCB
  \l 27FCC
  \l 27FCD
  \l 27FCE
  \l 27FCF
  \l 27FD0
  \l 27FD1
  \l 27FD2
  \l 27FD3
  \l 27FD4
  \l 27FD5
  \l 27FD6
  \l 27FD7
  \l 27FD8
  \l 27FD9
  \l 27FDA
  \l 27FDB
  \l 27FDC
  \l 27FDD
  \l 27FDE
  \l 27FDF
  \l 27FE0
  \l 27FE1
  \l 27FE2
  \l 27FE3
  \l 27FE4
  \l 27FE5
  \l 27FE6
  \l 27FE7
  \l 27FE8
  \l 27FE9
  \l 27FEA
  \l 27FEB
  \l 27FEC
  \l 27FED
  \l 27FEE
  \l 27FEF
  \l 27FF0
  \l 27FF1
  \l 27FF2
  \l 27FF3
  \l 27FF4
  \l 27FF5
  \l 27FF6
  \l 27FF7
  \l 27FF8
  \l 27FF9
  \l 27FFA
  \l 27FFB
  \l 27FFC
  \l 27FFD
  \l 27FFE
  \l 27FFF
  \l 28000
  \l 28001
  \l 28002
  \l 28003
  \l 28004
  \l 28005
  \l 28006
  \l 28007
  \l 28008
  \l 28009
  \l 2800A
  \l 2800B
  \l 2800C
  \l 2800D
  \l 2800E
  \l 2800F
  \l 28010
  \l 28011
  \l 28012
  \l 28013
  \l 28014
  \l 28015
  \l 28016
  \l 28017
  \l 28018
  \l 28019
  \l 2801A
  \l 2801B
  \l 2801C
  \l 2801D
  \l 2801E
  \l 2801F
  \l 28020
  \l 28021
  \l 28022
  \l 28023
  \l 28024
  \l 28025
  \l 28026
  \l 28027
  \l 28028
  \l 28029
  \l 2802A
  \l 2802B
  \l 2802C
  \l 2802D
  \l 2802E
  \l 2802F
  \l 28030
  \l 28031
  \l 28032
  \l 28033
  \l 28034
  \l 28035
  \l 28036
  \l 28037
  \l 28038
  \l 28039
  \l 2803A
  \l 2803B
  \l 2803C
  \l 2803D
  \l 2803E
  \l 2803F
  \l 28040
  \l 28041
  \l 28042
  \l 28043
  \l 28044
  \l 28045
  \l 28046
  \l 28047
  \l 28048
  \l 28049
  \l 2804A
  \l 2804B
  \l 2804C
  \l 2804D
  \l 2804E
  \l 2804F
  \l 28050
  \l 28051
  \l 28052
  \l 28053
  \l 28054
  \l 28055
  \l 28056
  \l 28057
  \l 28058
  \l 28059
  \l 2805A
  \l 2805B
  \l 2805C
  \l 2805D
  \l 2805E
  \l 2805F
  \l 28060
  \l 28061
  \l 28062
  \l 28063
  \l 28064
  \l 28065
  \l 28066
  \l 28067
  \l 28068
  \l 28069
  \l 2806A
  \l 2806B
  \l 2806C
  \l 2806D
  \l 2806E
  \l 2806F
  \l 28070
  \l 28071
  \l 28072
  \l 28073
  \l 28074
  \l 28075
  \l 28076
  \l 28077
  \l 28078
  \l 28079
  \l 2807A
  \l 2807B
  \l 2807C
  \l 2807D
  \l 2807E
  \l 2807F
  \l 28080
  \l 28081
  \l 28082
  \l 28083
  \l 28084
  \l 28085
  \l 28086
  \l 28087
  \l 28088
  \l 28089
  \l 2808A
  \l 2808B
  \l 2808C
  \l 2808D
  \l 2808E
  \l 2808F
  \l 28090
  \l 28091
  \l 28092
  \l 28093
  \l 28094
  \l 28095
  \l 28096
  \l 28097
  \l 28098
  \l 28099
  \l 2809A
  \l 2809B
  \l 2809C
  \l 2809D
  \l 2809E
  \l 2809F
  \l 280A0
  \l 280A1
  \l 280A2
  \l 280A3
  \l 280A4
  \l 280A5
  \l 280A6
  \l 280A7
  \l 280A8
  \l 280A9
  \l 280AA
  \l 280AB
  \l 280AC
  \l 280AD
  \l 280AE
  \l 280AF
  \l 280B0
  \l 280B1
  \l 280B2
  \l 280B3
  \l 280B4
  \l 280B5
  \l 280B6
  \l 280B7
  \l 280B8
  \l 280B9
  \l 280BA
  \l 280BB
  \l 280BC
  \l 280BD
  \l 280BE
  \l 280BF
  \l 280C0
  \l 280C1
  \l 280C2
  \l 280C3
  \l 280C4
  \l 280C5
  \l 280C6
  \l 280C7
  \l 280C8
  \l 280C9
  \l 280CA
  \l 280CB
  \l 280CC
  \l 280CD
  \l 280CE
  \l 280CF
  \l 280D0
  \l 280D1
  \l 280D2
  \l 280D3
  \l 280D4
  \l 280D5
  \l 280D6
  \l 280D7
  \l 280D8
  \l 280D9
  \l 280DA
  \l 280DB
  \l 280DC
  \l 280DD
  \l 280DE
  \l 280DF
  \l 280E0
  \l 280E1
  \l 280E2
  \l 280E3
  \l 280E4
  \l 280E5
  \l 280E6
  \l 280E7
  \l 280E8
  \l 280E9
  \l 280EA
  \l 280EB
  \l 280EC
  \l 280ED
  \l 280EE
  \l 280EF
  \l 280F0
  \l 280F1
  \l 280F2
  \l 280F3
  \l 280F4
  \l 280F5
  \l 280F6
  \l 280F7
  \l 280F8
  \l 280F9
  \l 280FA
  \l 280FB
  \l 280FC
  \l 280FD
  \l 280FE
  \l 280FF
  \l 28100
  \l 28101
  \l 28102
  \l 28103
  \l 28104
  \l 28105
  \l 28106
  \l 28107
  \l 28108
  \l 28109
  \l 2810A
  \l 2810B
  \l 2810C
  \l 2810D
  \l 2810E
  \l 2810F
  \l 28110
  \l 28111
  \l 28112
  \l 28113
  \l 28114
  \l 28115
  \l 28116
  \l 28117
  \l 28118
  \l 28119
  \l 2811A
  \l 2811B
  \l 2811C
  \l 2811D
  \l 2811E
  \l 2811F
  \l 28120
  \l 28121
  \l 28122
  \l 28123
  \l 28124
  \l 28125
  \l 28126
  \l 28127
  \l 28128
  \l 28129
  \l 2812A
  \l 2812B
  \l 2812C
  \l 2812D
  \l 2812E
  \l 2812F
  \l 28130
  \l 28131
  \l 28132
  \l 28133
  \l 28134
  \l 28135
  \l 28136
  \l 28137
  \l 28138
  \l 28139
  \l 2813A
  \l 2813B
  \l 2813C
  \l 2813D
  \l 2813E
  \l 2813F
  \l 28140
  \l 28141
  \l 28142
  \l 28143
  \l 28144
  \l 28145
  \l 28146
  \l 28147
  \l 28148
  \l 28149
  \l 2814A
  \l 2814B
  \l 2814C
  \l 2814D
  \l 2814E
  \l 2814F
  \l 28150
  \l 28151
  \l 28152
  \l 28153
  \l 28154
  \l 28155
  \l 28156
  \l 28157
  \l 28158
  \l 28159
  \l 2815A
  \l 2815B
  \l 2815C
  \l 2815D
  \l 2815E
  \l 2815F
  \l 28160
  \l 28161
  \l 28162
  \l 28163
  \l 28164
  \l 28165
  \l 28166
  \l 28167
  \l 28168
  \l 28169
  \l 2816A
  \l 2816B
  \l 2816C
  \l 2816D
  \l 2816E
  \l 2816F
  \l 28170
  \l 28171
  \l 28172
  \l 28173
  \l 28174
  \l 28175
  \l 28176
  \l 28177
  \l 28178
  \l 28179
  \l 2817A
  \l 2817B
  \l 2817C
  \l 2817D
  \l 2817E
  \l 2817F
  \l 28180
  \l 28181
  \l 28182
  \l 28183
  \l 28184
  \l 28185
  \l 28186
  \l 28187
  \l 28188
  \l 28189
  \l 2818A
  \l 2818B
  \l 2818C
  \l 2818D
  \l 2818E
  \l 2818F
  \l 28190
  \l 28191
  \l 28192
  \l 28193
  \l 28194
  \l 28195
  \l 28196
  \l 28197
  \l 28198
  \l 28199
  \l 2819A
  \l 2819B
  \l 2819C
  \l 2819D
  \l 2819E
  \l 2819F
  \l 281A0
  \l 281A1
  \l 281A2
  \l 281A3
  \l 281A4
  \l 281A5
  \l 281A6
  \l 281A7
  \l 281A8
  \l 281A9
  \l 281AA
  \l 281AB
  \l 281AC
  \l 281AD
  \l 281AE
  \l 281AF
  \l 281B0
  \l 281B1
  \l 281B2
  \l 281B3
  \l 281B4
  \l 281B5
  \l 281B6
  \l 281B7
  \l 281B8
  \l 281B9
  \l 281BA
  \l 281BB
  \l 281BC
  \l 281BD
  \l 281BE
  \l 281BF
  \l 281C0
  \l 281C1
  \l 281C2
  \l 281C3
  \l 281C4
  \l 281C5
  \l 281C6
  \l 281C7
  \l 281C8
  \l 281C9
  \l 281CA
  \l 281CB
  \l 281CC
  \l 281CD
  \l 281CE
  \l 281CF
  \l 281D0
  \l 281D1
  \l 281D2
  \l 281D3
  \l 281D4
  \l 281D5
  \l 281D6
  \l 281D7
  \l 281D8
  \l 281D9
  \l 281DA
  \l 281DB
  \l 281DC
  \l 281DD
  \l 281DE
  \l 281DF
  \l 281E0
  \l 281E1
  \l 281E2
  \l 281E3
  \l 281E4
  \l 281E5
  \l 281E6
  \l 281E7
  \l 281E8
  \l 281E9
  \l 281EA
  \l 281EB
  \l 281EC
  \l 281ED
  \l 281EE
  \l 281EF
  \l 281F0
  \l 281F1
  \l 281F2
  \l 281F3
  \l 281F4
  \l 281F5
  \l 281F6
  \l 281F7
  \l 281F8
  \l 281F9
  \l 281FA
  \l 281FB
  \l 281FC
  \l 281FD
  \l 281FE
  \l 281FF
  \l 28200
  \l 28201
  \l 28202
  \l 28203
  \l 28204
  \l 28205
  \l 28206
  \l 28207
  \l 28208
  \l 28209
  \l 2820A
  \l 2820B
  \l 2820C
  \l 2820D
  \l 2820E
  \l 2820F
  \l 28210
  \l 28211
  \l 28212
  \l 28213
  \l 28214
  \l 28215
  \l 28216
  \l 28217
  \l 28218
  \l 28219
  \l 2821A
  \l 2821B
  \l 2821C
  \l 2821D
  \l 2821E
  \l 2821F
  \l 28220
  \l 28221
  \l 28222
  \l 28223
  \l 28224
  \l 28225
  \l 28226
  \l 28227
  \l 28228
  \l 28229
  \l 2822A
  \l 2822B
  \l 2822C
  \l 2822D
  \l 2822E
  \l 2822F
  \l 28230
  \l 28231
  \l 28232
  \l 28233
  \l 28234
  \l 28235
  \l 28236
  \l 28237
  \l 28238
  \l 28239
  \l 2823A
  \l 2823B
  \l 2823C
  \l 2823D
  \l 2823E
  \l 2823F
  \l 28240
  \l 28241
  \l 28242
  \l 28243
  \l 28244
  \l 28245
  \l 28246
  \l 28247
  \l 28248
  \l 28249
  \l 2824A
  \l 2824B
  \l 2824C
  \l 2824D
  \l 2824E
  \l 2824F
  \l 28250
  \l 28251
  \l 28252
  \l 28253
  \l 28254
  \l 28255
  \l 28256
  \l 28257
  \l 28258
  \l 28259
  \l 2825A
  \l 2825B
  \l 2825C
  \l 2825D
  \l 2825E
  \l 2825F
  \l 28260
  \l 28261
  \l 28262
  \l 28263
  \l 28264
  \l 28265
  \l 28266
  \l 28267
  \l 28268
  \l 28269
  \l 2826A
  \l 2826B
  \l 2826C
  \l 2826D
  \l 2826E
  \l 2826F
  \l 28270
  \l 28271
  \l 28272
  \l 28273
  \l 28274
  \l 28275
  \l 28276
  \l 28277
  \l 28278
  \l 28279
  \l 2827A
  \l 2827B
  \l 2827C
  \l 2827D
  \l 2827E
  \l 2827F
  \l 28280
  \l 28281
  \l 28282
  \l 28283
  \l 28284
  \l 28285
  \l 28286
  \l 28287
  \l 28288
  \l 28289
  \l 2828A
  \l 2828B
  \l 2828C
  \l 2828D
  \l 2828E
  \l 2828F
  \l 28290
  \l 28291
  \l 28292
  \l 28293
  \l 28294
  \l 28295
  \l 28296
  \l 28297
  \l 28298
  \l 28299
  \l 2829A
  \l 2829B
  \l 2829C
  \l 2829D
  \l 2829E
  \l 2829F
  \l 282A0
  \l 282A1
  \l 282A2
  \l 282A3
  \l 282A4
  \l 282A5
  \l 282A6
  \l 282A7
  \l 282A8
  \l 282A9
  \l 282AA
  \l 282AB
  \l 282AC
  \l 282AD
  \l 282AE
  \l 282AF
  \l 282B0
  \l 282B1
  \l 282B2
  \l 282B3
  \l 282B4
  \l 282B5
  \l 282B6
  \l 282B7
  \l 282B8
  \l 282B9
  \l 282BA
  \l 282BB
  \l 282BC
  \l 282BD
  \l 282BE
  \l 282BF
  \l 282C0
  \l 282C1
  \l 282C2
  \l 282C3
  \l 282C4
  \l 282C5
  \l 282C6
  \l 282C7
  \l 282C8
  \l 282C9
  \l 282CA
  \l 282CB
  \l 282CC
  \l 282CD
  \l 282CE
  \l 282CF
  \l 282D0
  \l 282D1
  \l 282D2
  \l 282D3
  \l 282D4
  \l 282D5
  \l 282D6
  \l 282D7
  \l 282D8
  \l 282D9
  \l 282DA
  \l 282DB
  \l 282DC
  \l 282DD
  \l 282DE
  \l 282DF
  \l 282E0
  \l 282E1
  \l 282E2
  \l 282E3
  \l 282E4
  \l 282E5
  \l 282E6
  \l 282E7
  \l 282E8
  \l 282E9
  \l 282EA
  \l 282EB
  \l 282EC
  \l 282ED
  \l 282EE
  \l 282EF
  \l 282F0
  \l 282F1
  \l 282F2
  \l 282F3
  \l 282F4
  \l 282F5
  \l 282F6
  \l 282F7
  \l 282F8
  \l 282F9
  \l 282FA
  \l 282FB
  \l 282FC
  \l 282FD
  \l 282FE
  \l 282FF
  \l 28300
  \l 28301
  \l 28302
  \l 28303
  \l 28304
  \l 28305
  \l 28306
  \l 28307
  \l 28308
  \l 28309
  \l 2830A
  \l 2830B
  \l 2830C
  \l 2830D
  \l 2830E
  \l 2830F
  \l 28310
  \l 28311
  \l 28312
  \l 28313
  \l 28314
  \l 28315
  \l 28316
  \l 28317
  \l 28318
  \l 28319
  \l 2831A
  \l 2831B
  \l 2831C
  \l 2831D
  \l 2831E
  \l 2831F
  \l 28320
  \l 28321
  \l 28322
  \l 28323
  \l 28324
  \l 28325
  \l 28326
  \l 28327
  \l 28328
  \l 28329
  \l 2832A
  \l 2832B
  \l 2832C
  \l 2832D
  \l 2832E
  \l 2832F
  \l 28330
  \l 28331
  \l 28332
  \l 28333
  \l 28334
  \l 28335
  \l 28336
  \l 28337
  \l 28338
  \l 28339
  \l 2833A
  \l 2833B
  \l 2833C
  \l 2833D
  \l 2833E
  \l 2833F
  \l 28340
  \l 28341
  \l 28342
  \l 28343
  \l 28344
  \l 28345
  \l 28346
  \l 28347
  \l 28348
  \l 28349
  \l 2834A
  \l 2834B
  \l 2834C
  \l 2834D
  \l 2834E
  \l 2834F
  \l 28350
  \l 28351
  \l 28352
  \l 28353
  \l 28354
  \l 28355
  \l 28356
  \l 28357
  \l 28358
  \l 28359
  \l 2835A
  \l 2835B
  \l 2835C
  \l 2835D
  \l 2835E
  \l 2835F
  \l 28360
  \l 28361
  \l 28362
  \l 28363
  \l 28364
  \l 28365
  \l 28366
  \l 28367
  \l 28368
  \l 28369
  \l 2836A
  \l 2836B
  \l 2836C
  \l 2836D
  \l 2836E
  \l 2836F
  \l 28370
  \l 28371
  \l 28372
  \l 28373
  \l 28374
  \l 28375
  \l 28376
  \l 28377
  \l 28378
  \l 28379
  \l 2837A
  \l 2837B
  \l 2837C
  \l 2837D
  \l 2837E
  \l 2837F
  \l 28380
  \l 28381
  \l 28382
  \l 28383
  \l 28384
  \l 28385
  \l 28386
  \l 28387
  \l 28388
  \l 28389
  \l 2838A
  \l 2838B
  \l 2838C
  \l 2838D
  \l 2838E
  \l 2838F
  \l 28390
  \l 28391
  \l 28392
  \l 28393
  \l 28394
  \l 28395
  \l 28396
  \l 28397
  \l 28398
  \l 28399
  \l 2839A
  \l 2839B
  \l 2839C
  \l 2839D
  \l 2839E
  \l 2839F
  \l 283A0
  \l 283A1
  \l 283A2
  \l 283A3
  \l 283A4
  \l 283A5
  \l 283A6
  \l 283A7
  \l 283A8
  \l 283A9
  \l 283AA
  \l 283AB
  \l 283AC
  \l 283AD
  \l 283AE
  \l 283AF
  \l 283B0
  \l 283B1
  \l 283B2
  \l 283B3
  \l 283B4
  \l 283B5
  \l 283B6
  \l 283B7
  \l 283B8
  \l 283B9
  \l 283BA
  \l 283BB
  \l 283BC
  \l 283BD
  \l 283BE
  \l 283BF
  \l 283C0
  \l 283C1
  \l 283C2
  \l 283C3
  \l 283C4
  \l 283C5
  \l 283C6
  \l 283C7
  \l 283C8
  \l 283C9
  \l 283CA
  \l 283CB
  \l 283CC
  \l 283CD
  \l 283CE
  \l 283CF
  \l 283D0
  \l 283D1
  \l 283D2
  \l 283D3
  \l 283D4
  \l 283D5
  \l 283D6
  \l 283D7
  \l 283D8
  \l 283D9
  \l 283DA
  \l 283DB
  \l 283DC
  \l 283DD
  \l 283DE
  \l 283DF
  \l 283E0
  \l 283E1
  \l 283E2
  \l 283E3
  \l 283E4
  \l 283E5
  \l 283E6
  \l 283E7
  \l 283E8
  \l 283E9
  \l 283EA
  \l 283EB
  \l 283EC
  \l 283ED
  \l 283EE
  \l 283EF
  \l 283F0
  \l 283F1
  \l 283F2
  \l 283F3
  \l 283F4
  \l 283F5
  \l 283F6
  \l 283F7
  \l 283F8
  \l 283F9
  \l 283FA
  \l 283FB
  \l 283FC
  \l 283FD
  \l 283FE
  \l 283FF
  \l 28400
  \l 28401
  \l 28402
  \l 28403
  \l 28404
  \l 28405
  \l 28406
  \l 28407
  \l 28408
  \l 28409
  \l 2840A
  \l 2840B
  \l 2840C
  \l 2840D
  \l 2840E
  \l 2840F
  \l 28410
  \l 28411
  \l 28412
  \l 28413
  \l 28414
  \l 28415
  \l 28416
  \l 28417
  \l 28418
  \l 28419
  \l 2841A
  \l 2841B
  \l 2841C
  \l 2841D
  \l 2841E
  \l 2841F
  \l 28420
  \l 28421
  \l 28422
  \l 28423
  \l 28424
  \l 28425
  \l 28426
  \l 28427
  \l 28428
  \l 28429
  \l 2842A
  \l 2842B
  \l 2842C
  \l 2842D
  \l 2842E
  \l 2842F
  \l 28430
  \l 28431
  \l 28432
  \l 28433
  \l 28434
  \l 28435
  \l 28436
  \l 28437
  \l 28438
  \l 28439
  \l 2843A
  \l 2843B
  \l 2843C
  \l 2843D
  \l 2843E
  \l 2843F
  \l 28440
  \l 28441
  \l 28442
  \l 28443
  \l 28444
  \l 28445
  \l 28446
  \l 28447
  \l 28448
  \l 28449
  \l 2844A
  \l 2844B
  \l 2844C
  \l 2844D
  \l 2844E
  \l 2844F
  \l 28450
  \l 28451
  \l 28452
  \l 28453
  \l 28454
  \l 28455
  \l 28456
  \l 28457
  \l 28458
  \l 28459
  \l 2845A
  \l 2845B
  \l 2845C
  \l 2845D
  \l 2845E
  \l 2845F
  \l 28460
  \l 28461
  \l 28462
  \l 28463
  \l 28464
  \l 28465
  \l 28466
  \l 28467
  \l 28468
  \l 28469
  \l 2846A
  \l 2846B
  \l 2846C
  \l 2846D
  \l 2846E
  \l 2846F
  \l 28470
  \l 28471
  \l 28472
  \l 28473
  \l 28474
  \l 28475
  \l 28476
  \l 28477
  \l 28478
  \l 28479
  \l 2847A
  \l 2847B
  \l 2847C
  \l 2847D
  \l 2847E
  \l 2847F
  \l 28480
  \l 28481
  \l 28482
  \l 28483
  \l 28484
  \l 28485
  \l 28486
  \l 28487
  \l 28488
  \l 28489
  \l 2848A
  \l 2848B
  \l 2848C
  \l 2848D
  \l 2848E
  \l 2848F
  \l 28490
  \l 28491
  \l 28492
  \l 28493
  \l 28494
  \l 28495
  \l 28496
  \l 28497
  \l 28498
  \l 28499
  \l 2849A
  \l 2849B
  \l 2849C
  \l 2849D
  \l 2849E
  \l 2849F
  \l 284A0
  \l 284A1
  \l 284A2
  \l 284A3
  \l 284A4
  \l 284A5
  \l 284A6
  \l 284A7
  \l 284A8
  \l 284A9
  \l 284AA
  \l 284AB
  \l 284AC
  \l 284AD
  \l 284AE
  \l 284AF
  \l 284B0
  \l 284B1
  \l 284B2
  \l 284B3
  \l 284B4
  \l 284B5
  \l 284B6
  \l 284B7
  \l 284B8
  \l 284B9
  \l 284BA
  \l 284BB
  \l 284BC
  \l 284BD
  \l 284BE
  \l 284BF
  \l 284C0
  \l 284C1
  \l 284C2
  \l 284C3
  \l 284C4
  \l 284C5
  \l 284C6
  \l 284C7
  \l 284C8
  \l 284C9
  \l 284CA
  \l 284CB
  \l 284CC
  \l 284CD
  \l 284CE
  \l 284CF
  \l 284D0
  \l 284D1
  \l 284D2
  \l 284D3
  \l 284D4
  \l 284D5
  \l 284D6
  \l 284D7
  \l 284D8
  \l 284D9
  \l 284DA
  \l 284DB
  \l 284DC
  \l 284DD
  \l 284DE
  \l 284DF
  \l 284E0
  \l 284E1
  \l 284E2
  \l 284E3
  \l 284E4
  \l 284E5
  \l 284E6
  \l 284E7
  \l 284E8
  \l 284E9
  \l 284EA
  \l 284EB
  \l 284EC
  \l 284ED
  \l 284EE
  \l 284EF
  \l 284F0
  \l 284F1
  \l 284F2
  \l 284F3
  \l 284F4
  \l 284F5
  \l 284F6
  \l 284F7
  \l 284F8
  \l 284F9
  \l 284FA
  \l 284FB
  \l 284FC
  \l 284FD
  \l 284FE
  \l 284FF
  \l 28500
  \l 28501
  \l 28502
  \l 28503
  \l 28504
  \l 28505
  \l 28506
  \l 28507
  \l 28508
  \l 28509
  \l 2850A
  \l 2850B
  \l 2850C
  \l 2850D
  \l 2850E
  \l 2850F
  \l 28510
  \l 28511
  \l 28512
  \l 28513
  \l 28514
  \l 28515
  \l 28516
  \l 28517
  \l 28518
  \l 28519
  \l 2851A
  \l 2851B
  \l 2851C
  \l 2851D
  \l 2851E
  \l 2851F
  \l 28520
  \l 28521
  \l 28522
  \l 28523
  \l 28524
  \l 28525
  \l 28526
  \l 28527
  \l 28528
  \l 28529
  \l 2852A
  \l 2852B
  \l 2852C
  \l 2852D
  \l 2852E
  \l 2852F
  \l 28530
  \l 28531
  \l 28532
  \l 28533
  \l 28534
  \l 28535
  \l 28536
  \l 28537
  \l 28538
  \l 28539
  \l 2853A
  \l 2853B
  \l 2853C
  \l 2853D
  \l 2853E
  \l 2853F
  \l 28540
  \l 28541
  \l 28542
  \l 28543
  \l 28544
  \l 28545
  \l 28546
  \l 28547
  \l 28548
  \l 28549
  \l 2854A
  \l 2854B
  \l 2854C
  \l 2854D
  \l 2854E
  \l 2854F
  \l 28550
  \l 28551
  \l 28552
  \l 28553
  \l 28554
  \l 28555
  \l 28556
  \l 28557
  \l 28558
  \l 28559
  \l 2855A
  \l 2855B
  \l 2855C
  \l 2855D
  \l 2855E
  \l 2855F
  \l 28560
  \l 28561
  \l 28562
  \l 28563
  \l 28564
  \l 28565
  \l 28566
  \l 28567
  \l 28568
  \l 28569
  \l 2856A
  \l 2856B
  \l 2856C
  \l 2856D
  \l 2856E
  \l 2856F
  \l 28570
  \l 28571
  \l 28572
  \l 28573
  \l 28574
  \l 28575
  \l 28576
  \l 28577
  \l 28578
  \l 28579
  \l 2857A
  \l 2857B
  \l 2857C
  \l 2857D
  \l 2857E
  \l 2857F
  \l 28580
  \l 28581
  \l 28582
  \l 28583
  \l 28584
  \l 28585
  \l 28586
  \l 28587
  \l 28588
  \l 28589
  \l 2858A
  \l 2858B
  \l 2858C
  \l 2858D
  \l 2858E
  \l 2858F
  \l 28590
  \l 28591
  \l 28592
  \l 28593
  \l 28594
  \l 28595
  \l 28596
  \l 28597
  \l 28598
  \l 28599
  \l 2859A
  \l 2859B
  \l 2859C
  \l 2859D
  \l 2859E
  \l 2859F
  \l 285A0
  \l 285A1
  \l 285A2
  \l 285A3
  \l 285A4
  \l 285A5
  \l 285A6
  \l 285A7
  \l 285A8
  \l 285A9
  \l 285AA
  \l 285AB
  \l 285AC
  \l 285AD
  \l 285AE
  \l 285AF
  \l 285B0
  \l 285B1
  \l 285B2
  \l 285B3
  \l 285B4
  \l 285B5
  \l 285B6
  \l 285B7
  \l 285B8
  \l 285B9
  \l 285BA
  \l 285BB
  \l 285BC
  \l 285BD
  \l 285BE
  \l 285BF
  \l 285C0
  \l 285C1
  \l 285C2
  \l 285C3
  \l 285C4
  \l 285C5
  \l 285C6
  \l 285C7
  \l 285C8
  \l 285C9
  \l 285CA
  \l 285CB
  \l 285CC
  \l 285CD
  \l 285CE
  \l 285CF
  \l 285D0
  \l 285D1
  \l 285D2
  \l 285D3
  \l 285D4
  \l 285D5
  \l 285D6
  \l 285D7
  \l 285D8
  \l 285D9
  \l 285DA
  \l 285DB
  \l 285DC
  \l 285DD
  \l 285DE
  \l 285DF
  \l 285E0
  \l 285E1
  \l 285E2
  \l 285E3
  \l 285E4
  \l 285E5
  \l 285E6
  \l 285E7
  \l 285E8
  \l 285E9
  \l 285EA
  \l 285EB
  \l 285EC
  \l 285ED
  \l 285EE
  \l 285EF
  \l 285F0
  \l 285F1
  \l 285F2
  \l 285F3
  \l 285F4
  \l 285F5
  \l 285F6
  \l 285F7
  \l 285F8
  \l 285F9
  \l 285FA
  \l 285FB
  \l 285FC
  \l 285FD
  \l 285FE
  \l 285FF
  \l 28600
  \l 28601
  \l 28602
  \l 28603
  \l 28604
  \l 28605
  \l 28606
  \l 28607
  \l 28608
  \l 28609
  \l 2860A
  \l 2860B
  \l 2860C
  \l 2860D
  \l 2860E
  \l 2860F
  \l 28610
  \l 28611
  \l 28612
  \l 28613
  \l 28614
  \l 28615
  \l 28616
  \l 28617
  \l 28618
  \l 28619
  \l 2861A
  \l 2861B
  \l 2861C
  \l 2861D
  \l 2861E
  \l 2861F
  \l 28620
  \l 28621
  \l 28622
  \l 28623
  \l 28624
  \l 28625
  \l 28626
  \l 28627
  \l 28628
  \l 28629
  \l 2862A
  \l 2862B
  \l 2862C
  \l 2862D
  \l 2862E
  \l 2862F
  \l 28630
  \l 28631
  \l 28632
  \l 28633
  \l 28634
  \l 28635
  \l 28636
  \l 28637
  \l 28638
  \l 28639
  \l 2863A
  \l 2863B
  \l 2863C
  \l 2863D
  \l 2863E
  \l 2863F
  \l 28640
  \l 28641
  \l 28642
  \l 28643
  \l 28644
  \l 28645
  \l 28646
  \l 28647
  \l 28648
  \l 28649
  \l 2864A
  \l 2864B
  \l 2864C
  \l 2864D
  \l 2864E
  \l 2864F
  \l 28650
  \l 28651
  \l 28652
  \l 28653
  \l 28654
  \l 28655
  \l 28656
  \l 28657
  \l 28658
  \l 28659
  \l 2865A
  \l 2865B
  \l 2865C
  \l 2865D
  \l 2865E
  \l 2865F
  \l 28660
  \l 28661
  \l 28662
  \l 28663
  \l 28664
  \l 28665
  \l 28666
  \l 28667
  \l 28668
  \l 28669
  \l 2866A
  \l 2866B
  \l 2866C
  \l 2866D
  \l 2866E
  \l 2866F
  \l 28670
  \l 28671
  \l 28672
  \l 28673
  \l 28674
  \l 28675
  \l 28676
  \l 28677
  \l 28678
  \l 28679
  \l 2867A
  \l 2867B
  \l 2867C
  \l 2867D
  \l 2867E
  \l 2867F
  \l 28680
  \l 28681
  \l 28682
  \l 28683
  \l 28684
  \l 28685
  \l 28686
  \l 28687
  \l 28688
  \l 28689
  \l 2868A
  \l 2868B
  \l 2868C
  \l 2868D
  \l 2868E
  \l 2868F
  \l 28690
  \l 28691
  \l 28692
  \l 28693
  \l 28694
  \l 28695
  \l 28696
  \l 28697
  \l 28698
  \l 28699
  \l 2869A
  \l 2869B
  \l 2869C
  \l 2869D
  \l 2869E
  \l 2869F
  \l 286A0
  \l 286A1
  \l 286A2
  \l 286A3
  \l 286A4
  \l 286A5
  \l 286A6
  \l 286A7
  \l 286A8
  \l 286A9
  \l 286AA
  \l 286AB
  \l 286AC
  \l 286AD
  \l 286AE
  \l 286AF
  \l 286B0
  \l 286B1
  \l 286B2
  \l 286B3
  \l 286B4
  \l 286B5
  \l 286B6
  \l 286B7
  \l 286B8
  \l 286B9
  \l 286BA
  \l 286BB
  \l 286BC
  \l 286BD
  \l 286BE
  \l 286BF
  \l 286C0
  \l 286C1
  \l 286C2
  \l 286C3
  \l 286C4
  \l 286C5
  \l 286C6
  \l 286C7
  \l 286C8
  \l 286C9
  \l 286CA
  \l 286CB
  \l 286CC
  \l 286CD
  \l 286CE
  \l 286CF
  \l 286D0
  \l 286D1
  \l 286D2
  \l 286D3
  \l 286D4
  \l 286D5
  \l 286D6
  \l 286D7
  \l 286D8
  \l 286D9
  \l 286DA
  \l 286DB
  \l 286DC
  \l 286DD
  \l 286DE
  \l 286DF
  \l 286E0
  \l 286E1
  \l 286E2
  \l 286E3
  \l 286E4
  \l 286E5
  \l 286E6
  \l 286E7
  \l 286E8
  \l 286E9
  \l 286EA
  \l 286EB
  \l 286EC
  \l 286ED
  \l 286EE
  \l 286EF
  \l 286F0
  \l 286F1
  \l 286F2
  \l 286F3
  \l 286F4
  \l 286F5
  \l 286F6
  \l 286F7
  \l 286F8
  \l 286F9
  \l 286FA
  \l 286FB
  \l 286FC
  \l 286FD
  \l 286FE
  \l 286FF
  \l 28700
  \l 28701
  \l 28702
  \l 28703
  \l 28704
  \l 28705
  \l 28706
  \l 28707
  \l 28708
  \l 28709
  \l 2870A
  \l 2870B
  \l 2870C
  \l 2870D
  \l 2870E
  \l 2870F
  \l 28710
  \l 28711
  \l 28712
  \l 28713
  \l 28714
  \l 28715
  \l 28716
  \l 28717
  \l 28718
  \l 28719
  \l 2871A
  \l 2871B
  \l 2871C
  \l 2871D
  \l 2871E
  \l 2871F
  \l 28720
  \l 28721
  \l 28722
  \l 28723
  \l 28724
  \l 28725
  \l 28726
  \l 28727
  \l 28728
  \l 28729
  \l 2872A
  \l 2872B
  \l 2872C
  \l 2872D
  \l 2872E
  \l 2872F
  \l 28730
  \l 28731
  \l 28732
  \l 28733
  \l 28734
  \l 28735
  \l 28736
  \l 28737
  \l 28738
  \l 28739
  \l 2873A
  \l 2873B
  \l 2873C
  \l 2873D
  \l 2873E
  \l 2873F
  \l 28740
  \l 28741
  \l 28742
  \l 28743
  \l 28744
  \l 28745
  \l 28746
  \l 28747
  \l 28748
  \l 28749
  \l 2874A
  \l 2874B
  \l 2874C
  \l 2874D
  \l 2874E
  \l 2874F
  \l 28750
  \l 28751
  \l 28752
  \l 28753
  \l 28754
  \l 28755
  \l 28756
  \l 28757
  \l 28758
  \l 28759
  \l 2875A
  \l 2875B
  \l 2875C
  \l 2875D
  \l 2875E
  \l 2875F
  \l 28760
  \l 28761
  \l 28762
  \l 28763
  \l 28764
  \l 28765
  \l 28766
  \l 28767
  \l 28768
  \l 28769
  \l 2876A
  \l 2876B
  \l 2876C
  \l 2876D
  \l 2876E
  \l 2876F
  \l 28770
  \l 28771
  \l 28772
  \l 28773
  \l 28774
  \l 28775
  \l 28776
  \l 28777
  \l 28778
  \l 28779
  \l 2877A
  \l 2877B
  \l 2877C
  \l 2877D
  \l 2877E
  \l 2877F
  \l 28780
  \l 28781
  \l 28782
  \l 28783
  \l 28784
  \l 28785
  \l 28786
  \l 28787
  \l 28788
  \l 28789
  \l 2878A
  \l 2878B
  \l 2878C
  \l 2878D
  \l 2878E
  \l 2878F
  \l 28790
  \l 28791
  \l 28792
  \l 28793
  \l 28794
  \l 28795
  \l 28796
  \l 28797
  \l 28798
  \l 28799
  \l 2879A
  \l 2879B
  \l 2879C
  \l 2879D
  \l 2879E
  \l 2879F
  \l 287A0
  \l 287A1
  \l 287A2
  \l 287A3
  \l 287A4
  \l 287A5
  \l 287A6
  \l 287A7
  \l 287A8
  \l 287A9
  \l 287AA
  \l 287AB
  \l 287AC
  \l 287AD
  \l 287AE
  \l 287AF
  \l 287B0
  \l 287B1
  \l 287B2
  \l 287B3
  \l 287B4
  \l 287B5
  \l 287B6
  \l 287B7
  \l 287B8
  \l 287B9
  \l 287BA
  \l 287BB
  \l 287BC
  \l 287BD
  \l 287BE
  \l 287BF
  \l 287C0
  \l 287C1
  \l 287C2
  \l 287C3
  \l 287C4
  \l 287C5
  \l 287C6
  \l 287C7
  \l 287C8
  \l 287C9
  \l 287CA
  \l 287CB
  \l 287CC
  \l 287CD
  \l 287CE
  \l 287CF
  \l 287D0
  \l 287D1
  \l 287D2
  \l 287D3
  \l 287D4
  \l 287D5
  \l 287D6
  \l 287D7
  \l 287D8
  \l 287D9
  \l 287DA
  \l 287DB
  \l 287DC
  \l 287DD
  \l 287DE
  \l 287DF
  \l 287E0
  \l 287E1
  \l 287E2
  \l 287E3
  \l 287E4
  \l 287E5
  \l 287E6
  \l 287E7
  \l 287E8
  \l 287E9
  \l 287EA
  \l 287EB
  \l 287EC
  \l 287ED
  \l 287EE
  \l 287EF
  \l 287F0
  \l 287F1
  \l 287F2
  \l 287F3
  \l 287F4
  \l 287F5
  \l 287F6
  \l 287F7
  \l 287F8
  \l 287F9
  \l 287FA
  \l 287FB
  \l 287FC
  \l 287FD
  \l 287FE
  \l 287FF
  \l 28800
  \l 28801
  \l 28802
  \l 28803
  \l 28804
  \l 28805
  \l 28806
  \l 28807
  \l 28808
  \l 28809
  \l 2880A
  \l 2880B
  \l 2880C
  \l 2880D
  \l 2880E
  \l 2880F
  \l 28810
  \l 28811
  \l 28812
  \l 28813
  \l 28814
  \l 28815
  \l 28816
  \l 28817
  \l 28818
  \l 28819
  \l 2881A
  \l 2881B
  \l 2881C
  \l 2881D
  \l 2881E
  \l 2881F
  \l 28820
  \l 28821
  \l 28822
  \l 28823
  \l 28824
  \l 28825
  \l 28826
  \l 28827
  \l 28828
  \l 28829
  \l 2882A
  \l 2882B
  \l 2882C
  \l 2882D
  \l 2882E
  \l 2882F
  \l 28830
  \l 28831
  \l 28832
  \l 28833
  \l 28834
  \l 28835
  \l 28836
  \l 28837
  \l 28838
  \l 28839
  \l 2883A
  \l 2883B
  \l 2883C
  \l 2883D
  \l 2883E
  \l 2883F
  \l 28840
  \l 28841
  \l 28842
  \l 28843
  \l 28844
  \l 28845
  \l 28846
  \l 28847
  \l 28848
  \l 28849
  \l 2884A
  \l 2884B
  \l 2884C
  \l 2884D
  \l 2884E
  \l 2884F
  \l 28850
  \l 28851
  \l 28852
  \l 28853
  \l 28854
  \l 28855
  \l 28856
  \l 28857
  \l 28858
  \l 28859
  \l 2885A
  \l 2885B
  \l 2885C
  \l 2885D
  \l 2885E
  \l 2885F
  \l 28860
  \l 28861
  \l 28862
  \l 28863
  \l 28864
  \l 28865
  \l 28866
  \l 28867
  \l 28868
  \l 28869
  \l 2886A
  \l 2886B
  \l 2886C
  \l 2886D
  \l 2886E
  \l 2886F
  \l 28870
  \l 28871
  \l 28872
  \l 28873
  \l 28874
  \l 28875
  \l 28876
  \l 28877
  \l 28878
  \l 28879
  \l 2887A
  \l 2887B
  \l 2887C
  \l 2887D
  \l 2887E
  \l 2887F
  \l 28880
  \l 28881
  \l 28882
  \l 28883
  \l 28884
  \l 28885
  \l 28886
  \l 28887
  \l 28888
  \l 28889
  \l 2888A
  \l 2888B
  \l 2888C
  \l 2888D
  \l 2888E
  \l 2888F
  \l 28890
  \l 28891
  \l 28892
  \l 28893
  \l 28894
  \l 28895
  \l 28896
  \l 28897
  \l 28898
  \l 28899
  \l 2889A
  \l 2889B
  \l 2889C
  \l 2889D
  \l 2889E
  \l 2889F
  \l 288A0
  \l 288A1
  \l 288A2
  \l 288A3
  \l 288A4
  \l 288A5
  \l 288A6
  \l 288A7
  \l 288A8
  \l 288A9
  \l 288AA
  \l 288AB
  \l 288AC
  \l 288AD
  \l 288AE
  \l 288AF
  \l 288B0
  \l 288B1
  \l 288B2
  \l 288B3
  \l 288B4
  \l 288B5
  \l 288B6
  \l 288B7
  \l 288B8
  \l 288B9
  \l 288BA
  \l 288BB
  \l 288BC
  \l 288BD
  \l 288BE
  \l 288BF
  \l 288C0
  \l 288C1
  \l 288C2
  \l 288C3
  \l 288C4
  \l 288C5
  \l 288C6
  \l 288C7
  \l 288C8
  \l 288C9
  \l 288CA
  \l 288CB
  \l 288CC
  \l 288CD
  \l 288CE
  \l 288CF
  \l 288D0
  \l 288D1
  \l 288D2
  \l 288D3
  \l 288D4
  \l 288D5
  \l 288D6
  \l 288D7
  \l 288D8
  \l 288D9
  \l 288DA
  \l 288DB
  \l 288DC
  \l 288DD
  \l 288DE
  \l 288DF
  \l 288E0
  \l 288E1
  \l 288E2
  \l 288E3
  \l 288E4
  \l 288E5
  \l 288E6
  \l 288E7
  \l 288E8
  \l 288E9
  \l 288EA
  \l 288EB
  \l 288EC
  \l 288ED
  \l 288EE
  \l 288EF
  \l 288F0
  \l 288F1
  \l 288F2
  \l 288F3
  \l 288F4
  \l 288F5
  \l 288F6
  \l 288F7
  \l 288F8
  \l 288F9
  \l 288FA
  \l 288FB
  \l 288FC
  \l 288FD
  \l 288FE
  \l 288FF
  \l 28900
  \l 28901
  \l 28902
  \l 28903
  \l 28904
  \l 28905
  \l 28906
  \l 28907
  \l 28908
  \l 28909
  \l 2890A
  \l 2890B
  \l 2890C
  \l 2890D
  \l 2890E
  \l 2890F
  \l 28910
  \l 28911
  \l 28912
  \l 28913
  \l 28914
  \l 28915
  \l 28916
  \l 28917
  \l 28918
  \l 28919
  \l 2891A
  \l 2891B
  \l 2891C
  \l 2891D
  \l 2891E
  \l 2891F
  \l 28920
  \l 28921
  \l 28922
  \l 28923
  \l 28924
  \l 28925
  \l 28926
  \l 28927
  \l 28928
  \l 28929
  \l 2892A
  \l 2892B
  \l 2892C
  \l 2892D
  \l 2892E
  \l 2892F
  \l 28930
  \l 28931
  \l 28932
  \l 28933
  \l 28934
  \l 28935
  \l 28936
  \l 28937
  \l 28938
  \l 28939
  \l 2893A
  \l 2893B
  \l 2893C
  \l 2893D
  \l 2893E
  \l 2893F
  \l 28940
  \l 28941
  \l 28942
  \l 28943
  \l 28944
  \l 28945
  \l 28946
  \l 28947
  \l 28948
  \l 28949
  \l 2894A
  \l 2894B
  \l 2894C
  \l 2894D
  \l 2894E
  \l 2894F
  \l 28950
  \l 28951
  \l 28952
  \l 28953
  \l 28954
  \l 28955
  \l 28956
  \l 28957
  \l 28958
  \l 28959
  \l 2895A
  \l 2895B
  \l 2895C
  \l 2895D
  \l 2895E
  \l 2895F
  \l 28960
  \l 28961
  \l 28962
  \l 28963
  \l 28964
  \l 28965
  \l 28966
  \l 28967
  \l 28968
  \l 28969
  \l 2896A
  \l 2896B
  \l 2896C
  \l 2896D
  \l 2896E
  \l 2896F
  \l 28970
  \l 28971
  \l 28972
  \l 28973
  \l 28974
  \l 28975
  \l 28976
  \l 28977
  \l 28978
  \l 28979
  \l 2897A
  \l 2897B
  \l 2897C
  \l 2897D
  \l 2897E
  \l 2897F
  \l 28980
  \l 28981
  \l 28982
  \l 28983
  \l 28984
  \l 28985
  \l 28986
  \l 28987
  \l 28988
  \l 28989
  \l 2898A
  \l 2898B
  \l 2898C
  \l 2898D
  \l 2898E
  \l 2898F
  \l 28990
  \l 28991
  \l 28992
  \l 28993
  \l 28994
  \l 28995
  \l 28996
  \l 28997
  \l 28998
  \l 28999
  \l 2899A
  \l 2899B
  \l 2899C
  \l 2899D
  \l 2899E
  \l 2899F
  \l 289A0
  \l 289A1
  \l 289A2
  \l 289A3
  \l 289A4
  \l 289A5
  \l 289A6
  \l 289A7
  \l 289A8
  \l 289A9
  \l 289AA
  \l 289AB
  \l 289AC
  \l 289AD
  \l 289AE
  \l 289AF
  \l 289B0
  \l 289B1
  \l 289B2
  \l 289B3
  \l 289B4
  \l 289B5
  \l 289B6
  \l 289B7
  \l 289B8
  \l 289B9
  \l 289BA
  \l 289BB
  \l 289BC
  \l 289BD
  \l 289BE
  \l 289BF
  \l 289C0
  \l 289C1
  \l 289C2
  \l 289C3
  \l 289C4
  \l 289C5
  \l 289C6
  \l 289C7
  \l 289C8
  \l 289C9
  \l 289CA
  \l 289CB
  \l 289CC
  \l 289CD
  \l 289CE
  \l 289CF
  \l 289D0
  \l 289D1
  \l 289D2
  \l 289D3
  \l 289D4
  \l 289D5
  \l 289D6
  \l 289D7
  \l 289D8
  \l 289D9
  \l 289DA
  \l 289DB
  \l 289DC
  \l 289DD
  \l 289DE
  \l 289DF
  \l 289E0
  \l 289E1
  \l 289E2
  \l 289E3
  \l 289E4
  \l 289E5
  \l 289E6
  \l 289E7
  \l 289E8
  \l 289E9
  \l 289EA
  \l 289EB
  \l 289EC
  \l 289ED
  \l 289EE
  \l 289EF
  \l 289F0
  \l 289F1
  \l 289F2
  \l 289F3
  \l 289F4
  \l 289F5
  \l 289F6
  \l 289F7
  \l 289F8
  \l 289F9
  \l 289FA
  \l 289FB
  \l 289FC
  \l 289FD
  \l 289FE
  \l 289FF
  \l 28A00
  \l 28A01
  \l 28A02
  \l 28A03
  \l 28A04
  \l 28A05
  \l 28A06
  \l 28A07
  \l 28A08
  \l 28A09
  \l 28A0A
  \l 28A0B
  \l 28A0C
  \l 28A0D
  \l 28A0E
  \l 28A0F
  \l 28A10
  \l 28A11
  \l 28A12
  \l 28A13
  \l 28A14
  \l 28A15
  \l 28A16
  \l 28A17
  \l 28A18
  \l 28A19
  \l 28A1A
  \l 28A1B
  \l 28A1C
  \l 28A1D
  \l 28A1E
  \l 28A1F
  \l 28A20
  \l 28A21
  \l 28A22
  \l 28A23
  \l 28A24
  \l 28A25
  \l 28A26
  \l 28A27
  \l 28A28
  \l 28A29
  \l 28A2A
  \l 28A2B
  \l 28A2C
  \l 28A2D
  \l 28A2E
  \l 28A2F
  \l 28A30
  \l 28A31
  \l 28A32
  \l 28A33
  \l 28A34
  \l 28A35
  \l 28A36
  \l 28A37
  \l 28A38
  \l 28A39
  \l 28A3A
  \l 28A3B
  \l 28A3C
  \l 28A3D
  \l 28A3E
  \l 28A3F
  \l 28A40
  \l 28A41
  \l 28A42
  \l 28A43
  \l 28A44
  \l 28A45
  \l 28A46
  \l 28A47
  \l 28A48
  \l 28A49
  \l 28A4A
  \l 28A4B
  \l 28A4C
  \l 28A4D
  \l 28A4E
  \l 28A4F
  \l 28A50
  \l 28A51
  \l 28A52
  \l 28A53
  \l 28A54
  \l 28A55
  \l 28A56
  \l 28A57
  \l 28A58
  \l 28A59
  \l 28A5A
  \l 28A5B
  \l 28A5C
  \l 28A5D
  \l 28A5E
  \l 28A5F
  \l 28A60
  \l 28A61
  \l 28A62
  \l 28A63
  \l 28A64
  \l 28A65
  \l 28A66
  \l 28A67
  \l 28A68
  \l 28A69
  \l 28A6A
  \l 28A6B
  \l 28A6C
  \l 28A6D
  \l 28A6E
  \l 28A6F
  \l 28A70
  \l 28A71
  \l 28A72
  \l 28A73
  \l 28A74
  \l 28A75
  \l 28A76
  \l 28A77
  \l 28A78
  \l 28A79
  \l 28A7A
  \l 28A7B
  \l 28A7C
  \l 28A7D
  \l 28A7E
  \l 28A7F
  \l 28A80
  \l 28A81
  \l 28A82
  \l 28A83
  \l 28A84
  \l 28A85
  \l 28A86
  \l 28A87
  \l 28A88
  \l 28A89
  \l 28A8A
  \l 28A8B
  \l 28A8C
  \l 28A8D
  \l 28A8E
  \l 28A8F
  \l 28A90
  \l 28A91
  \l 28A92
  \l 28A93
  \l 28A94
  \l 28A95
  \l 28A96
  \l 28A97
  \l 28A98
  \l 28A99
  \l 28A9A
  \l 28A9B
  \l 28A9C
  \l 28A9D
  \l 28A9E
  \l 28A9F
  \l 28AA0
  \l 28AA1
  \l 28AA2
  \l 28AA3
  \l 28AA4
  \l 28AA5
  \l 28AA6
  \l 28AA7
  \l 28AA8
  \l 28AA9
  \l 28AAA
  \l 28AAB
  \l 28AAC
  \l 28AAD
  \l 28AAE
  \l 28AAF
  \l 28AB0
  \l 28AB1
  \l 28AB2
  \l 28AB3
  \l 28AB4
  \l 28AB5
  \l 28AB6
  \l 28AB7
  \l 28AB8
  \l 28AB9
  \l 28ABA
  \l 28ABB
  \l 28ABC
  \l 28ABD
  \l 28ABE
  \l 28ABF
  \l 28AC0
  \l 28AC1
  \l 28AC2
  \l 28AC3
  \l 28AC4
  \l 28AC5
  \l 28AC6
  \l 28AC7
  \l 28AC8
  \l 28AC9
  \l 28ACA
  \l 28ACB
  \l 28ACC
  \l 28ACD
  \l 28ACE
  \l 28ACF
  \l 28AD0
  \l 28AD1
  \l 28AD2
  \l 28AD3
  \l 28AD4
  \l 28AD5
  \l 28AD6
  \l 28AD7
  \l 28AD8
  \l 28AD9
  \l 28ADA
  \l 28ADB
  \l 28ADC
  \l 28ADD
  \l 28ADE
  \l 28ADF
  \l 28AE0
  \l 28AE1
  \l 28AE2
  \l 28AE3
  \l 28AE4
  \l 28AE5
  \l 28AE6
  \l 28AE7
  \l 28AE8
  \l 28AE9
  \l 28AEA
  \l 28AEB
  \l 28AEC
  \l 28AED
  \l 28AEE
  \l 28AEF
  \l 28AF0
  \l 28AF1
  \l 28AF2
  \l 28AF3
  \l 28AF4
  \l 28AF5
  \l 28AF6
  \l 28AF7
  \l 28AF8
  \l 28AF9
  \l 28AFA
  \l 28AFB
  \l 28AFC
  \l 28AFD
  \l 28AFE
  \l 28AFF
  \l 28B00
  \l 28B01
  \l 28B02
  \l 28B03
  \l 28B04
  \l 28B05
  \l 28B06
  \l 28B07
  \l 28B08
  \l 28B09
  \l 28B0A
  \l 28B0B
  \l 28B0C
  \l 28B0D
  \l 28B0E
  \l 28B0F
  \l 28B10
  \l 28B11
  \l 28B12
  \l 28B13
  \l 28B14
  \l 28B15
  \l 28B16
  \l 28B17
  \l 28B18
  \l 28B19
  \l 28B1A
  \l 28B1B
  \l 28B1C
  \l 28B1D
  \l 28B1E
  \l 28B1F
  \l 28B20
  \l 28B21
  \l 28B22
  \l 28B23
  \l 28B24
  \l 28B25
  \l 28B26
  \l 28B27
  \l 28B28
  \l 28B29
  \l 28B2A
  \l 28B2B
  \l 28B2C
  \l 28B2D
  \l 28B2E
  \l 28B2F
  \l 28B30
  \l 28B31
  \l 28B32
  \l 28B33
  \l 28B34
  \l 28B35
  \l 28B36
  \l 28B37
  \l 28B38
  \l 28B39
  \l 28B3A
  \l 28B3B
  \l 28B3C
  \l 28B3D
  \l 28B3E
  \l 28B3F
  \l 28B40
  \l 28B41
  \l 28B42
  \l 28B43
  \l 28B44
  \l 28B45
  \l 28B46
  \l 28B47
  \l 28B48
  \l 28B49
  \l 28B4A
  \l 28B4B
  \l 28B4C
  \l 28B4D
  \l 28B4E
  \l 28B4F
  \l 28B50
  \l 28B51
  \l 28B52
  \l 28B53
  \l 28B54
  \l 28B55
  \l 28B56
  \l 28B57
  \l 28B58
  \l 28B59
  \l 28B5A
  \l 28B5B
  \l 28B5C
  \l 28B5D
  \l 28B5E
  \l 28B5F
  \l 28B60
  \l 28B61
  \l 28B62
  \l 28B63
  \l 28B64
  \l 28B65
  \l 28B66
  \l 28B67
  \l 28B68
  \l 28B69
  \l 28B6A
  \l 28B6B
  \l 28B6C
  \l 28B6D
  \l 28B6E
  \l 28B6F
  \l 28B70
  \l 28B71
  \l 28B72
  \l 28B73
  \l 28B74
  \l 28B75
  \l 28B76
  \l 28B77
  \l 28B78
  \l 28B79
  \l 28B7A
  \l 28B7B
  \l 28B7C
  \l 28B7D
  \l 28B7E
  \l 28B7F
  \l 28B80
  \l 28B81
  \l 28B82
  \l 28B83
  \l 28B84
  \l 28B85
  \l 28B86
  \l 28B87
  \l 28B88
  \l 28B89
  \l 28B8A
  \l 28B8B
  \l 28B8C
  \l 28B8D
  \l 28B8E
  \l 28B8F
  \l 28B90
  \l 28B91
  \l 28B92
  \l 28B93
  \l 28B94
  \l 28B95
  \l 28B96
  \l 28B97
  \l 28B98
  \l 28B99
  \l 28B9A
  \l 28B9B
  \l 28B9C
  \l 28B9D
  \l 28B9E
  \l 28B9F
  \l 28BA0
  \l 28BA1
  \l 28BA2
  \l 28BA3
  \l 28BA4
  \l 28BA5
  \l 28BA6
  \l 28BA7
  \l 28BA8
  \l 28BA9
  \l 28BAA
  \l 28BAB
  \l 28BAC
  \l 28BAD
  \l 28BAE
  \l 28BAF
  \l 28BB0
  \l 28BB1
  \l 28BB2
  \l 28BB3
  \l 28BB4
  \l 28BB5
  \l 28BB6
  \l 28BB7
  \l 28BB8
  \l 28BB9
  \l 28BBA
  \l 28BBB
  \l 28BBC
  \l 28BBD
  \l 28BBE
  \l 28BBF
  \l 28BC0
  \l 28BC1
  \l 28BC2
  \l 28BC3
  \l 28BC4
  \l 28BC5
  \l 28BC6
  \l 28BC7
  \l 28BC8
  \l 28BC9
  \l 28BCA
  \l 28BCB
  \l 28BCC
  \l 28BCD
  \l 28BCE
  \l 28BCF
  \l 28BD0
  \l 28BD1
  \l 28BD2
  \l 28BD3
  \l 28BD4
  \l 28BD5
  \l 28BD6
  \l 28BD7
  \l 28BD8
  \l 28BD9
  \l 28BDA
  \l 28BDB
  \l 28BDC
  \l 28BDD
  \l 28BDE
  \l 28BDF
  \l 28BE0
  \l 28BE1
  \l 28BE2
  \l 28BE3
  \l 28BE4
  \l 28BE5
  \l 28BE6
  \l 28BE7
  \l 28BE8
  \l 28BE9
  \l 28BEA
  \l 28BEB
  \l 28BEC
  \l 28BED
  \l 28BEE
  \l 28BEF
  \l 28BF0
  \l 28BF1
  \l 28BF2
  \l 28BF3
  \l 28BF4
  \l 28BF5
  \l 28BF6
  \l 28BF7
  \l 28BF8
  \l 28BF9
  \l 28BFA
  \l 28BFB
  \l 28BFC
  \l 28BFD
  \l 28BFE
  \l 28BFF
  \l 28C00
  \l 28C01
  \l 28C02
  \l 28C03
  \l 28C04
  \l 28C05
  \l 28C06
  \l 28C07
  \l 28C08
  \l 28C09
  \l 28C0A
  \l 28C0B
  \l 28C0C
  \l 28C0D
  \l 28C0E
  \l 28C0F
  \l 28C10
  \l 28C11
  \l 28C12
  \l 28C13
  \l 28C14
  \l 28C15
  \l 28C16
  \l 28C17
  \l 28C18
  \l 28C19
  \l 28C1A
  \l 28C1B
  \l 28C1C
  \l 28C1D
  \l 28C1E
  \l 28C1F
  \l 28C20
  \l 28C21
  \l 28C22
  \l 28C23
  \l 28C24
  \l 28C25
  \l 28C26
  \l 28C27
  \l 28C28
  \l 28C29
  \l 28C2A
  \l 28C2B
  \l 28C2C
  \l 28C2D
  \l 28C2E
  \l 28C2F
  \l 28C30
  \l 28C31
  \l 28C32
  \l 28C33
  \l 28C34
  \l 28C35
  \l 28C36
  \l 28C37
  \l 28C38
  \l 28C39
  \l 28C3A
  \l 28C3B
  \l 28C3C
  \l 28C3D
  \l 28C3E
  \l 28C3F
  \l 28C40
  \l 28C41
  \l 28C42
  \l 28C43
  \l 28C44
  \l 28C45
  \l 28C46
  \l 28C47
  \l 28C48
  \l 28C49
  \l 28C4A
  \l 28C4B
  \l 28C4C
  \l 28C4D
  \l 28C4E
  \l 28C4F
  \l 28C50
  \l 28C51
  \l 28C52
  \l 28C53
  \l 28C54
  \l 28C55
  \l 28C56
  \l 28C57
  \l 28C58
  \l 28C59
  \l 28C5A
  \l 28C5B
  \l 28C5C
  \l 28C5D
  \l 28C5E
  \l 28C5F
  \l 28C60
  \l 28C61
  \l 28C62
  \l 28C63
  \l 28C64
  \l 28C65
  \l 28C66
  \l 28C67
  \l 28C68
  \l 28C69
  \l 28C6A
  \l 28C6B
  \l 28C6C
  \l 28C6D
  \l 28C6E
  \l 28C6F
  \l 28C70
  \l 28C71
  \l 28C72
  \l 28C73
  \l 28C74
  \l 28C75
  \l 28C76
  \l 28C77
  \l 28C78
  \l 28C79
  \l 28C7A
  \l 28C7B
  \l 28C7C
  \l 28C7D
  \l 28C7E
  \l 28C7F
  \l 28C80
  \l 28C81
  \l 28C82
  \l 28C83
  \l 28C84
  \l 28C85
  \l 28C86
  \l 28C87
  \l 28C88
  \l 28C89
  \l 28C8A
  \l 28C8B
  \l 28C8C
  \l 28C8D
  \l 28C8E
  \l 28C8F
  \l 28C90
  \l 28C91
  \l 28C92
  \l 28C93
  \l 28C94
  \l 28C95
  \l 28C96
  \l 28C97
  \l 28C98
  \l 28C99
  \l 28C9A
  \l 28C9B
  \l 28C9C
  \l 28C9D
  \l 28C9E
  \l 28C9F
  \l 28CA0
  \l 28CA1
  \l 28CA2
  \l 28CA3
  \l 28CA4
  \l 28CA5
  \l 28CA6
  \l 28CA7
  \l 28CA8
  \l 28CA9
  \l 28CAA
  \l 28CAB
  \l 28CAC
  \l 28CAD
  \l 28CAE
  \l 28CAF
  \l 28CB0
  \l 28CB1
  \l 28CB2
  \l 28CB3
  \l 28CB4
  \l 28CB5
  \l 28CB6
  \l 28CB7
  \l 28CB8
  \l 28CB9
  \l 28CBA
  \l 28CBB
  \l 28CBC
  \l 28CBD
  \l 28CBE
  \l 28CBF
  \l 28CC0
  \l 28CC1
  \l 28CC2
  \l 28CC3
  \l 28CC4
  \l 28CC5
  \l 28CC6
  \l 28CC7
  \l 28CC8
  \l 28CC9
  \l 28CCA
  \l 28CCB
  \l 28CCC
  \l 28CCD
  \l 28CCE
  \l 28CCF
  \l 28CD0
  \l 28CD1
  \l 28CD2
  \l 28CD3
  \l 28CD4
  \l 28CD5
  \l 28CD6
  \l 28CD7
  \l 28CD8
  \l 28CD9
  \l 28CDA
  \l 28CDB
  \l 28CDC
  \l 28CDD
  \l 28CDE
  \l 28CDF
  \l 28CE0
  \l 28CE1
  \l 28CE2
  \l 28CE3
  \l 28CE4
  \l 28CE5
  \l 28CE6
  \l 28CE7
  \l 28CE8
  \l 28CE9
  \l 28CEA
  \l 28CEB
  \l 28CEC
  \l 28CED
  \l 28CEE
  \l 28CEF
  \l 28CF0
  \l 28CF1
  \l 28CF2
  \l 28CF3
  \l 28CF4
  \l 28CF5
  \l 28CF6
  \l 28CF7
  \l 28CF8
  \l 28CF9
  \l 28CFA
  \l 28CFB
  \l 28CFC
  \l 28CFD
  \l 28CFE
  \l 28CFF
  \l 28D00
  \l 28D01
  \l 28D02
  \l 28D03
  \l 28D04
  \l 28D05
  \l 28D06
  \l 28D07
  \l 28D08
  \l 28D09
  \l 28D0A
  \l 28D0B
  \l 28D0C
  \l 28D0D
  \l 28D0E
  \l 28D0F
  \l 28D10
  \l 28D11
  \l 28D12
  \l 28D13
  \l 28D14
  \l 28D15
  \l 28D16
  \l 28D17
  \l 28D18
  \l 28D19
  \l 28D1A
  \l 28D1B
  \l 28D1C
  \l 28D1D
  \l 28D1E
  \l 28D1F
  \l 28D20
  \l 28D21
  \l 28D22
  \l 28D23
  \l 28D24
  \l 28D25
  \l 28D26
  \l 28D27
  \l 28D28
  \l 28D29
  \l 28D2A
  \l 28D2B
  \l 28D2C
  \l 28D2D
  \l 28D2E
  \l 28D2F
  \l 28D30
  \l 28D31
  \l 28D32
  \l 28D33
  \l 28D34
  \l 28D35
  \l 28D36
  \l 28D37
  \l 28D38
  \l 28D39
  \l 28D3A
  \l 28D3B
  \l 28D3C
  \l 28D3D
  \l 28D3E
  \l 28D3F
  \l 28D40
  \l 28D41
  \l 28D42
  \l 28D43
  \l 28D44
  \l 28D45
  \l 28D46
  \l 28D47
  \l 28D48
  \l 28D49
  \l 28D4A
  \l 28D4B
  \l 28D4C
  \l 28D4D
  \l 28D4E
  \l 28D4F
  \l 28D50
  \l 28D51
  \l 28D52
  \l 28D53
  \l 28D54
  \l 28D55
  \l 28D56
  \l 28D57
  \l 28D58
  \l 28D59
  \l 28D5A
  \l 28D5B
  \l 28D5C
  \l 28D5D
  \l 28D5E
  \l 28D5F
  \l 28D60
  \l 28D61
  \l 28D62
  \l 28D63
  \l 28D64
  \l 28D65
  \l 28D66
  \l 28D67
  \l 28D68
  \l 28D69
  \l 28D6A
  \l 28D6B
  \l 28D6C
  \l 28D6D
  \l 28D6E
  \l 28D6F
  \l 28D70
  \l 28D71
  \l 28D72
  \l 28D73
  \l 28D74
  \l 28D75
  \l 28D76
  \l 28D77
  \l 28D78
  \l 28D79
  \l 28D7A
  \l 28D7B
  \l 28D7C
  \l 28D7D
  \l 28D7E
  \l 28D7F
  \l 28D80
  \l 28D81
  \l 28D82
  \l 28D83
  \l 28D84
  \l 28D85
  \l 28D86
  \l 28D87
  \l 28D88
  \l 28D89
  \l 28D8A
  \l 28D8B
  \l 28D8C
  \l 28D8D
  \l 28D8E
  \l 28D8F
  \l 28D90
  \l 28D91
  \l 28D92
  \l 28D93
  \l 28D94
  \l 28D95
  \l 28D96
  \l 28D97
  \l 28D98
  \l 28D99
  \l 28D9A
  \l 28D9B
  \l 28D9C
  \l 28D9D
  \l 28D9E
  \l 28D9F
  \l 28DA0
  \l 28DA1
  \l 28DA2
  \l 28DA3
  \l 28DA4
  \l 28DA5
  \l 28DA6
  \l 28DA7
  \l 28DA8
  \l 28DA9
  \l 28DAA
  \l 28DAB
  \l 28DAC
  \l 28DAD
  \l 28DAE
  \l 28DAF
  \l 28DB0
  \l 28DB1
  \l 28DB2
  \l 28DB3
  \l 28DB4
  \l 28DB5
  \l 28DB6
  \l 28DB7
  \l 28DB8
  \l 28DB9
  \l 28DBA
  \l 28DBB
  \l 28DBC
  \l 28DBD
  \l 28DBE
  \l 28DBF
  \l 28DC0
  \l 28DC1
  \l 28DC2
  \l 28DC3
  \l 28DC4
  \l 28DC5
  \l 28DC6
  \l 28DC7
  \l 28DC8
  \l 28DC9
  \l 28DCA
  \l 28DCB
  \l 28DCC
  \l 28DCD
  \l 28DCE
  \l 28DCF
  \l 28DD0
  \l 28DD1
  \l 28DD2
  \l 28DD3
  \l 28DD4
  \l 28DD5
  \l 28DD6
  \l 28DD7
  \l 28DD8
  \l 28DD9
  \l 28DDA
  \l 28DDB
  \l 28DDC
  \l 28DDD
  \l 28DDE
  \l 28DDF
  \l 28DE0
  \l 28DE1
  \l 28DE2
  \l 28DE3
  \l 28DE4
  \l 28DE5
  \l 28DE6
  \l 28DE7
  \l 28DE8
  \l 28DE9
  \l 28DEA
  \l 28DEB
  \l 28DEC
  \l 28DED
  \l 28DEE
  \l 28DEF
  \l 28DF0
  \l 28DF1
  \l 28DF2
  \l 28DF3
  \l 28DF4
  \l 28DF5
  \l 28DF6
  \l 28DF7
  \l 28DF8
  \l 28DF9
  \l 28DFA
  \l 28DFB
  \l 28DFC
  \l 28DFD
  \l 28DFE
  \l 28DFF
  \l 28E00
  \l 28E01
  \l 28E02
  \l 28E03
  \l 28E04
  \l 28E05
  \l 28E06
  \l 28E07
  \l 28E08
  \l 28E09
  \l 28E0A
  \l 28E0B
  \l 28E0C
  \l 28E0D
  \l 28E0E
  \l 28E0F
  \l 28E10
  \l 28E11
  \l 28E12
  \l 28E13
  \l 28E14
  \l 28E15
  \l 28E16
  \l 28E17
  \l 28E18
  \l 28E19
  \l 28E1A
  \l 28E1B
  \l 28E1C
  \l 28E1D
  \l 28E1E
  \l 28E1F
  \l 28E20
  \l 28E21
  \l 28E22
  \l 28E23
  \l 28E24
  \l 28E25
  \l 28E26
  \l 28E27
  \l 28E28
  \l 28E29
  \l 28E2A
  \l 28E2B
  \l 28E2C
  \l 28E2D
  \l 28E2E
  \l 28E2F
  \l 28E30
  \l 28E31
  \l 28E32
  \l 28E33
  \l 28E34
  \l 28E35
  \l 28E36
  \l 28E37
  \l 28E38
  \l 28E39
  \l 28E3A
  \l 28E3B
  \l 28E3C
  \l 28E3D
  \l 28E3E
  \l 28E3F
  \l 28E40
  \l 28E41
  \l 28E42
  \l 28E43
  \l 28E44
  \l 28E45
  \l 28E46
  \l 28E47
  \l 28E48
  \l 28E49
  \l 28E4A
  \l 28E4B
  \l 28E4C
  \l 28E4D
  \l 28E4E
  \l 28E4F
  \l 28E50
  \l 28E51
  \l 28E52
  \l 28E53
  \l 28E54
  \l 28E55
  \l 28E56
  \l 28E57
  \l 28E58
  \l 28E59
  \l 28E5A
  \l 28E5B
  \l 28E5C
  \l 28E5D
  \l 28E5E
  \l 28E5F
  \l 28E60
  \l 28E61
  \l 28E62
  \l 28E63
  \l 28E64
  \l 28E65
  \l 28E66
  \l 28E67
  \l 28E68
  \l 28E69
  \l 28E6A
  \l 28E6B
  \l 28E6C
  \l 28E6D
  \l 28E6E
  \l 28E6F
  \l 28E70
  \l 28E71
  \l 28E72
  \l 28E73
  \l 28E74
  \l 28E75
  \l 28E76
  \l 28E77
  \l 28E78
  \l 28E79
  \l 28E7A
  \l 28E7B
  \l 28E7C
  \l 28E7D
  \l 28E7E
  \l 28E7F
  \l 28E80
  \l 28E81
  \l 28E82
  \l 28E83
  \l 28E84
  \l 28E85
  \l 28E86
  \l 28E87
  \l 28E88
  \l 28E89
  \l 28E8A
  \l 28E8B
  \l 28E8C
  \l 28E8D
  \l 28E8E
  \l 28E8F
  \l 28E90
  \l 28E91
  \l 28E92
  \l 28E93
  \l 28E94
  \l 28E95
  \l 28E96
  \l 28E97
  \l 28E98
  \l 28E99
  \l 28E9A
  \l 28E9B
  \l 28E9C
  \l 28E9D
  \l 28E9E
  \l 28E9F
  \l 28EA0
  \l 28EA1
  \l 28EA2
  \l 28EA3
  \l 28EA4
  \l 28EA5
  \l 28EA6
  \l 28EA7
  \l 28EA8
  \l 28EA9
  \l 28EAA
  \l 28EAB
  \l 28EAC
  \l 28EAD
  \l 28EAE
  \l 28EAF
  \l 28EB0
  \l 28EB1
  \l 28EB2
  \l 28EB3
  \l 28EB4
  \l 28EB5
  \l 28EB6
  \l 28EB7
  \l 28EB8
  \l 28EB9
  \l 28EBA
  \l 28EBB
  \l 28EBC
  \l 28EBD
  \l 28EBE
  \l 28EBF
  \l 28EC0
  \l 28EC1
  \l 28EC2
  \l 28EC3
  \l 28EC4
  \l 28EC5
  \l 28EC6
  \l 28EC7
  \l 28EC8
  \l 28EC9
  \l 28ECA
  \l 28ECB
  \l 28ECC
  \l 28ECD
  \l 28ECE
  \l 28ECF
  \l 28ED0
  \l 28ED1
  \l 28ED2
  \l 28ED3
  \l 28ED4
  \l 28ED5
  \l 28ED6
  \l 28ED7
  \l 28ED8
  \l 28ED9
  \l 28EDA
  \l 28EDB
  \l 28EDC
  \l 28EDD
  \l 28EDE
  \l 28EDF
  \l 28EE0
  \l 28EE1
  \l 28EE2
  \l 28EE3
  \l 28EE4
  \l 28EE5
  \l 28EE6
  \l 28EE7
  \l 28EE8
  \l 28EE9
  \l 28EEA
  \l 28EEB
  \l 28EEC
  \l 28EED
  \l 28EEE
  \l 28EEF
  \l 28EF0
  \l 28EF1
  \l 28EF2
  \l 28EF3
  \l 28EF4
  \l 28EF5
  \l 28EF6
  \l 28EF7
  \l 28EF8
  \l 28EF9
  \l 28EFA
  \l 28EFB
  \l 28EFC
  \l 28EFD
  \l 28EFE
  \l 28EFF
  \l 28F00
  \l 28F01
  \l 28F02
  \l 28F03
  \l 28F04
  \l 28F05
  \l 28F06
  \l 28F07
  \l 28F08
  \l 28F09
  \l 28F0A
  \l 28F0B
  \l 28F0C
  \l 28F0D
  \l 28F0E
  \l 28F0F
  \l 28F10
  \l 28F11
  \l 28F12
  \l 28F13
  \l 28F14
  \l 28F15
  \l 28F16
  \l 28F17
  \l 28F18
  \l 28F19
  \l 28F1A
  \l 28F1B
  \l 28F1C
  \l 28F1D
  \l 28F1E
  \l 28F1F
  \l 28F20
  \l 28F21
  \l 28F22
  \l 28F23
  \l 28F24
  \l 28F25
  \l 28F26
  \l 28F27
  \l 28F28
  \l 28F29
  \l 28F2A
  \l 28F2B
  \l 28F2C
  \l 28F2D
  \l 28F2E
  \l 28F2F
  \l 28F30
  \l 28F31
  \l 28F32
  \l 28F33
  \l 28F34
  \l 28F35
  \l 28F36
  \l 28F37
  \l 28F38
  \l 28F39
  \l 28F3A
  \l 28F3B
  \l 28F3C
  \l 28F3D
  \l 28F3E
  \l 28F3F
  \l 28F40
  \l 28F41
  \l 28F42
  \l 28F43
  \l 28F44
  \l 28F45
  \l 28F46
  \l 28F47
  \l 28F48
  \l 28F49
  \l 28F4A
  \l 28F4B
  \l 28F4C
  \l 28F4D
  \l 28F4E
  \l 28F4F
  \l 28F50
  \l 28F51
  \l 28F52
  \l 28F53
  \l 28F54
  \l 28F55
  \l 28F56
  \l 28F57
  \l 28F58
  \l 28F59
  \l 28F5A
  \l 28F5B
  \l 28F5C
  \l 28F5D
  \l 28F5E
  \l 28F5F
  \l 28F60
  \l 28F61
  \l 28F62
  \l 28F63
  \l 28F64
  \l 28F65
  \l 28F66
  \l 28F67
  \l 28F68
  \l 28F69
  \l 28F6A
  \l 28F6B
  \l 28F6C
  \l 28F6D
  \l 28F6E
  \l 28F6F
  \l 28F70
  \l 28F71
  \l 28F72
  \l 28F73
  \l 28F74
  \l 28F75
  \l 28F76
  \l 28F77
  \l 28F78
  \l 28F79
  \l 28F7A
  \l 28F7B
  \l 28F7C
  \l 28F7D
  \l 28F7E
  \l 28F7F
  \l 28F80
  \l 28F81
  \l 28F82
  \l 28F83
  \l 28F84
  \l 28F85
  \l 28F86
  \l 28F87
  \l 28F88
  \l 28F89
  \l 28F8A
  \l 28F8B
  \l 28F8C
  \l 28F8D
  \l 28F8E
  \l 28F8F
  \l 28F90
  \l 28F91
  \l 28F92
  \l 28F93
  \l 28F94
  \l 28F95
  \l 28F96
  \l 28F97
  \l 28F98
  \l 28F99
  \l 28F9A
  \l 28F9B
  \l 28F9C
  \l 28F9D
  \l 28F9E
  \l 28F9F
  \l 28FA0
  \l 28FA1
  \l 28FA2
  \l 28FA3
  \l 28FA4
  \l 28FA5
  \l 28FA6
  \l 28FA7
  \l 28FA8
  \l 28FA9
  \l 28FAA
  \l 28FAB
  \l 28FAC
  \l 28FAD
  \l 28FAE
  \l 28FAF
  \l 28FB0
  \l 28FB1
  \l 28FB2
  \l 28FB3
  \l 28FB4
  \l 28FB5
  \l 28FB6
  \l 28FB7
  \l 28FB8
  \l 28FB9
  \l 28FBA
  \l 28FBB
  \l 28FBC
  \l 28FBD
  \l 28FBE
  \l 28FBF
  \l 28FC0
  \l 28FC1
  \l 28FC2
  \l 28FC3
  \l 28FC4
  \l 28FC5
  \l 28FC6
  \l 28FC7
  \l 28FC8
  \l 28FC9
  \l 28FCA
  \l 28FCB
  \l 28FCC
  \l 28FCD
  \l 28FCE
  \l 28FCF
  \l 28FD0
  \l 28FD1
  \l 28FD2
  \l 28FD3
  \l 28FD4
  \l 28FD5
  \l 28FD6
  \l 28FD7
  \l 28FD8
  \l 28FD9
  \l 28FDA
  \l 28FDB
  \l 28FDC
  \l 28FDD
  \l 28FDE
  \l 28FDF
  \l 28FE0
  \l 28FE1
  \l 28FE2
  \l 28FE3
  \l 28FE4
  \l 28FE5
  \l 28FE6
  \l 28FE7
  \l 28FE8
  \l 28FE9
  \l 28FEA
  \l 28FEB
  \l 28FEC
  \l 28FED
  \l 28FEE
  \l 28FEF
  \l 28FF0
  \l 28FF1
  \l 28FF2
  \l 28FF3
  \l 28FF4
  \l 28FF5
  \l 28FF6
  \l 28FF7
  \l 28FF8
  \l 28FF9
  \l 28FFA
  \l 28FFB
  \l 28FFC
  \l 28FFD
  \l 28FFE
  \l 28FFF
  \l 29000
  \l 29001
  \l 29002
  \l 29003
  \l 29004
  \l 29005
  \l 29006
  \l 29007
  \l 29008
  \l 29009
  \l 2900A
  \l 2900B
  \l 2900C
  \l 2900D
  \l 2900E
  \l 2900F
  \l 29010
  \l 29011
  \l 29012
  \l 29013
  \l 29014
  \l 29015
  \l 29016
  \l 29017
  \l 29018
  \l 29019
  \l 2901A
  \l 2901B
  \l 2901C
  \l 2901D
  \l 2901E
  \l 2901F
  \l 29020
  \l 29021
  \l 29022
  \l 29023
  \l 29024
  \l 29025
  \l 29026
  \l 29027
  \l 29028
  \l 29029
  \l 2902A
  \l 2902B
  \l 2902C
  \l 2902D
  \l 2902E
  \l 2902F
  \l 29030
  \l 29031
  \l 29032
  \l 29033
  \l 29034
  \l 29035
  \l 29036
  \l 29037
  \l 29038
  \l 29039
  \l 2903A
  \l 2903B
  \l 2903C
  \l 2903D
  \l 2903E
  \l 2903F
  \l 29040
  \l 29041
  \l 29042
  \l 29043
  \l 29044
  \l 29045
  \l 29046
  \l 29047
  \l 29048
  \l 29049
  \l 2904A
  \l 2904B
  \l 2904C
  \l 2904D
  \l 2904E
  \l 2904F
  \l 29050
  \l 29051
  \l 29052
  \l 29053
  \l 29054
  \l 29055
  \l 29056
  \l 29057
  \l 29058
  \l 29059
  \l 2905A
  \l 2905B
  \l 2905C
  \l 2905D
  \l 2905E
  \l 2905F
  \l 29060
  \l 29061
  \l 29062
  \l 29063
  \l 29064
  \l 29065
  \l 29066
  \l 29067
  \l 29068
  \l 29069
  \l 2906A
  \l 2906B
  \l 2906C
  \l 2906D
  \l 2906E
  \l 2906F
  \l 29070
  \l 29071
  \l 29072
  \l 29073
  \l 29074
  \l 29075
  \l 29076
  \l 29077
  \l 29078
  \l 29079
  \l 2907A
  \l 2907B
  \l 2907C
  \l 2907D
  \l 2907E
  \l 2907F
  \l 29080
  \l 29081
  \l 29082
  \l 29083
  \l 29084
  \l 29085
  \l 29086
  \l 29087
  \l 29088
  \l 29089
  \l 2908A
  \l 2908B
  \l 2908C
  \l 2908D
  \l 2908E
  \l 2908F
  \l 29090
  \l 29091
  \l 29092
  \l 29093
  \l 29094
  \l 29095
  \l 29096
  \l 29097
  \l 29098
  \l 29099
  \l 2909A
  \l 2909B
  \l 2909C
  \l 2909D
  \l 2909E
  \l 2909F
  \l 290A0
  \l 290A1
  \l 290A2
  \l 290A3
  \l 290A4
  \l 290A5
  \l 290A6
  \l 290A7
  \l 290A8
  \l 290A9
  \l 290AA
  \l 290AB
  \l 290AC
  \l 290AD
  \l 290AE
  \l 290AF
  \l 290B0
  \l 290B1
  \l 290B2
  \l 290B3
  \l 290B4
  \l 290B5
  \l 290B6
  \l 290B7
  \l 290B8
  \l 290B9
  \l 290BA
  \l 290BB
  \l 290BC
  \l 290BD
  \l 290BE
  \l 290BF
  \l 290C0
  \l 290C1
  \l 290C2
  \l 290C3
  \l 290C4
  \l 290C5
  \l 290C6
  \l 290C7
  \l 290C8
  \l 290C9
  \l 290CA
  \l 290CB
  \l 290CC
  \l 290CD
  \l 290CE
  \l 290CF
  \l 290D0
  \l 290D1
  \l 290D2
  \l 290D3
  \l 290D4
  \l 290D5
  \l 290D6
  \l 290D7
  \l 290D8
  \l 290D9
  \l 290DA
  \l 290DB
  \l 290DC
  \l 290DD
  \l 290DE
  \l 290DF
  \l 290E0
  \l 290E1
  \l 290E2
  \l 290E3
  \l 290E4
  \l 290E5
  \l 290E6
  \l 290E7
  \l 290E8
  \l 290E9
  \l 290EA
  \l 290EB
  \l 290EC
  \l 290ED
  \l 290EE
  \l 290EF
  \l 290F0
  \l 290F1
  \l 290F2
  \l 290F3
  \l 290F4
  \l 290F5
  \l 290F6
  \l 290F7
  \l 290F8
  \l 290F9
  \l 290FA
  \l 290FB
  \l 290FC
  \l 290FD
  \l 290FE
  \l 290FF
  \l 29100
  \l 29101
  \l 29102
  \l 29103
  \l 29104
  \l 29105
  \l 29106
  \l 29107
  \l 29108
  \l 29109
  \l 2910A
  \l 2910B
  \l 2910C
  \l 2910D
  \l 2910E
  \l 2910F
  \l 29110
  \l 29111
  \l 29112
  \l 29113
  \l 29114
  \l 29115
  \l 29116
  \l 29117
  \l 29118
  \l 29119
  \l 2911A
  \l 2911B
  \l 2911C
  \l 2911D
  \l 2911E
  \l 2911F
  \l 29120
  \l 29121
  \l 29122
  \l 29123
  \l 29124
  \l 29125
  \l 29126
  \l 29127
  \l 29128
  \l 29129
  \l 2912A
  \l 2912B
  \l 2912C
  \l 2912D
  \l 2912E
  \l 2912F
  \l 29130
  \l 29131
  \l 29132
  \l 29133
  \l 29134
  \l 29135
  \l 29136
  \l 29137
  \l 29138
  \l 29139
  \l 2913A
  \l 2913B
  \l 2913C
  \l 2913D
  \l 2913E
  \l 2913F
  \l 29140
  \l 29141
  \l 29142
  \l 29143
  \l 29144
  \l 29145
  \l 29146
  \l 29147
  \l 29148
  \l 29149
  \l 2914A
  \l 2914B
  \l 2914C
  \l 2914D
  \l 2914E
  \l 2914F
  \l 29150
  \l 29151
  \l 29152
  \l 29153
  \l 29154
  \l 29155
  \l 29156
  \l 29157
  \l 29158
  \l 29159
  \l 2915A
  \l 2915B
  \l 2915C
  \l 2915D
  \l 2915E
  \l 2915F
  \l 29160
  \l 29161
  \l 29162
  \l 29163
  \l 29164
  \l 29165
  \l 29166
  \l 29167
  \l 29168
  \l 29169
  \l 2916A
  \l 2916B
  \l 2916C
  \l 2916D
  \l 2916E
  \l 2916F
  \l 29170
  \l 29171
  \l 29172
  \l 29173
  \l 29174
  \l 29175
  \l 29176
  \l 29177
  \l 29178
  \l 29179
  \l 2917A
  \l 2917B
  \l 2917C
  \l 2917D
  \l 2917E
  \l 2917F
  \l 29180
  \l 29181
  \l 29182
  \l 29183
  \l 29184
  \l 29185
  \l 29186
  \l 29187
  \l 29188
  \l 29189
  \l 2918A
  \l 2918B
  \l 2918C
  \l 2918D
  \l 2918E
  \l 2918F
  \l 29190
  \l 29191
  \l 29192
  \l 29193
  \l 29194
  \l 29195
  \l 29196
  \l 29197
  \l 29198
  \l 29199
  \l 2919A
  \l 2919B
  \l 2919C
  \l 2919D
  \l 2919E
  \l 2919F
  \l 291A0
  \l 291A1
  \l 291A2
  \l 291A3
  \l 291A4
  \l 291A5
  \l 291A6
  \l 291A7
  \l 291A8
  \l 291A9
  \l 291AA
  \l 291AB
  \l 291AC
  \l 291AD
  \l 291AE
  \l 291AF
  \l 291B0
  \l 291B1
  \l 291B2
  \l 291B3
  \l 291B4
  \l 291B5
  \l 291B6
  \l 291B7
  \l 291B8
  \l 291B9
  \l 291BA
  \l 291BB
  \l 291BC
  \l 291BD
  \l 291BE
  \l 291BF
  \l 291C0
  \l 291C1
  \l 291C2
  \l 291C3
  \l 291C4
  \l 291C5
  \l 291C6
  \l 291C7
  \l 291C8
  \l 291C9
  \l 291CA
  \l 291CB
  \l 291CC
  \l 291CD
  \l 291CE
  \l 291CF
  \l 291D0
  \l 291D1
  \l 291D2
  \l 291D3
  \l 291D4
  \l 291D5
  \l 291D6
  \l 291D7
  \l 291D8
  \l 291D9
  \l 291DA
  \l 291DB
  \l 291DC
  \l 291DD
  \l 291DE
  \l 291DF
  \l 291E0
  \l 291E1
  \l 291E2
  \l 291E3
  \l 291E4
  \l 291E5
  \l 291E6
  \l 291E7
  \l 291E8
  \l 291E9
  \l 291EA
  \l 291EB
  \l 291EC
  \l 291ED
  \l 291EE
  \l 291EF
  \l 291F0
  \l 291F1
  \l 291F2
  \l 291F3
  \l 291F4
  \l 291F5
  \l 291F6
  \l 291F7
  \l 291F8
  \l 291F9
  \l 291FA
  \l 291FB
  \l 291FC
  \l 291FD
  \l 291FE
  \l 291FF
  \l 29200
  \l 29201
  \l 29202
  \l 29203
  \l 29204
  \l 29205
  \l 29206
  \l 29207
  \l 29208
  \l 29209
  \l 2920A
  \l 2920B
  \l 2920C
  \l 2920D
  \l 2920E
  \l 2920F
  \l 29210
  \l 29211
  \l 29212
  \l 29213
  \l 29214
  \l 29215
  \l 29216
  \l 29217
  \l 29218
  \l 29219
  \l 2921A
  \l 2921B
  \l 2921C
  \l 2921D
  \l 2921E
  \l 2921F
  \l 29220
  \l 29221
  \l 29222
  \l 29223
  \l 29224
  \l 29225
  \l 29226
  \l 29227
  \l 29228
  \l 29229
  \l 2922A
  \l 2922B
  \l 2922C
  \l 2922D
  \l 2922E
  \l 2922F
  \l 29230
  \l 29231
  \l 29232
  \l 29233
  \l 29234
  \l 29235
  \l 29236
  \l 29237
  \l 29238
  \l 29239
  \l 2923A
  \l 2923B
  \l 2923C
  \l 2923D
  \l 2923E
  \l 2923F
  \l 29240
  \l 29241
  \l 29242
  \l 29243
  \l 29244
  \l 29245
  \l 29246
  \l 29247
  \l 29248
  \l 29249
  \l 2924A
  \l 2924B
  \l 2924C
  \l 2924D
  \l 2924E
  \l 2924F
  \l 29250
  \l 29251
  \l 29252
  \l 29253
  \l 29254
  \l 29255
  \l 29256
  \l 29257
  \l 29258
  \l 29259
  \l 2925A
  \l 2925B
  \l 2925C
  \l 2925D
  \l 2925E
  \l 2925F
  \l 29260
  \l 29261
  \l 29262
  \l 29263
  \l 29264
  \l 29265
  \l 29266
  \l 29267
  \l 29268
  \l 29269
  \l 2926A
  \l 2926B
  \l 2926C
  \l 2926D
  \l 2926E
  \l 2926F
  \l 29270
  \l 29271
  \l 29272
  \l 29273
  \l 29274
  \l 29275
  \l 29276
  \l 29277
  \l 29278
  \l 29279
  \l 2927A
  \l 2927B
  \l 2927C
  \l 2927D
  \l 2927E
  \l 2927F
  \l 29280
  \l 29281
  \l 29282
  \l 29283
  \l 29284
  \l 29285
  \l 29286
  \l 29287
  \l 29288
  \l 29289
  \l 2928A
  \l 2928B
  \l 2928C
  \l 2928D
  \l 2928E
  \l 2928F
  \l 29290
  \l 29291
  \l 29292
  \l 29293
  \l 29294
  \l 29295
  \l 29296
  \l 29297
  \l 29298
  \l 29299
  \l 2929A
  \l 2929B
  \l 2929C
  \l 2929D
  \l 2929E
  \l 2929F
  \l 292A0
  \l 292A1
  \l 292A2
  \l 292A3
  \l 292A4
  \l 292A5
  \l 292A6
  \l 292A7
  \l 292A8
  \l 292A9
  \l 292AA
  \l 292AB
  \l 292AC
  \l 292AD
  \l 292AE
  \l 292AF
  \l 292B0
  \l 292B1
  \l 292B2
  \l 292B3
  \l 292B4
  \l 292B5
  \l 292B6
  \l 292B7
  \l 292B8
  \l 292B9
  \l 292BA
  \l 292BB
  \l 292BC
  \l 292BD
  \l 292BE
  \l 292BF
  \l 292C0
  \l 292C1
  \l 292C2
  \l 292C3
  \l 292C4
  \l 292C5
  \l 292C6
  \l 292C7
  \l 292C8
  \l 292C9
  \l 292CA
  \l 292CB
  \l 292CC
  \l 292CD
  \l 292CE
  \l 292CF
  \l 292D0
  \l 292D1
  \l 292D2
  \l 292D3
  \l 292D4
  \l 292D5
  \l 292D6
  \l 292D7
  \l 292D8
  \l 292D9
  \l 292DA
  \l 292DB
  \l 292DC
  \l 292DD
  \l 292DE
  \l 292DF
  \l 292E0
  \l 292E1
  \l 292E2
  \l 292E3
  \l 292E4
  \l 292E5
  \l 292E6
  \l 292E7
  \l 292E8
  \l 292E9
  \l 292EA
  \l 292EB
  \l 292EC
  \l 292ED
  \l 292EE
  \l 292EF
  \l 292F0
  \l 292F1
  \l 292F2
  \l 292F3
  \l 292F4
  \l 292F5
  \l 292F6
  \l 292F7
  \l 292F8
  \l 292F9
  \l 292FA
  \l 292FB
  \l 292FC
  \l 292FD
  \l 292FE
  \l 292FF
  \l 29300
  \l 29301
  \l 29302
  \l 29303
  \l 29304
  \l 29305
  \l 29306
  \l 29307
  \l 29308
  \l 29309
  \l 2930A
  \l 2930B
  \l 2930C
  \l 2930D
  \l 2930E
  \l 2930F
  \l 29310
  \l 29311
  \l 29312
  \l 29313
  \l 29314
  \l 29315
  \l 29316
  \l 29317
  \l 29318
  \l 29319
  \l 2931A
  \l 2931B
  \l 2931C
  \l 2931D
  \l 2931E
  \l 2931F
  \l 29320
  \l 29321
  \l 29322
  \l 29323
  \l 29324
  \l 29325
  \l 29326
  \l 29327
  \l 29328
  \l 29329
  \l 2932A
  \l 2932B
  \l 2932C
  \l 2932D
  \l 2932E
  \l 2932F
  \l 29330
  \l 29331
  \l 29332
  \l 29333
  \l 29334
  \l 29335
  \l 29336
  \l 29337
  \l 29338
  \l 29339
  \l 2933A
  \l 2933B
  \l 2933C
  \l 2933D
  \l 2933E
  \l 2933F
  \l 29340
  \l 29341
  \l 29342
  \l 29343
  \l 29344
  \l 29345
  \l 29346
  \l 29347
  \l 29348
  \l 29349
  \l 2934A
  \l 2934B
  \l 2934C
  \l 2934D
  \l 2934E
  \l 2934F
  \l 29350
  \l 29351
  \l 29352
  \l 29353
  \l 29354
  \l 29355
  \l 29356
  \l 29357
  \l 29358
  \l 29359
  \l 2935A
  \l 2935B
  \l 2935C
  \l 2935D
  \l 2935E
  \l 2935F
  \l 29360
  \l 29361
  \l 29362
  \l 29363
  \l 29364
  \l 29365
  \l 29366
  \l 29367
  \l 29368
  \l 29369
  \l 2936A
  \l 2936B
  \l 2936C
  \l 2936D
  \l 2936E
  \l 2936F
  \l 29370
  \l 29371
  \l 29372
  \l 29373
  \l 29374
  \l 29375
  \l 29376
  \l 29377
  \l 29378
  \l 29379
  \l 2937A
  \l 2937B
  \l 2937C
  \l 2937D
  \l 2937E
  \l 2937F
  \l 29380
  \l 29381
  \l 29382
  \l 29383
  \l 29384
  \l 29385
  \l 29386
  \l 29387
  \l 29388
  \l 29389
  \l 2938A
  \l 2938B
  \l 2938C
  \l 2938D
  \l 2938E
  \l 2938F
  \l 29390
  \l 29391
  \l 29392
  \l 29393
  \l 29394
  \l 29395
  \l 29396
  \l 29397
  \l 29398
  \l 29399
  \l 2939A
  \l 2939B
  \l 2939C
  \l 2939D
  \l 2939E
  \l 2939F
  \l 293A0
  \l 293A1
  \l 293A2
  \l 293A3
  \l 293A4
  \l 293A5
  \l 293A6
  \l 293A7
  \l 293A8
  \l 293A9
  \l 293AA
  \l 293AB
  \l 293AC
  \l 293AD
  \l 293AE
  \l 293AF
  \l 293B0
  \l 293B1
  \l 293B2
  \l 293B3
  \l 293B4
  \l 293B5
  \l 293B6
  \l 293B7
  \l 293B8
  \l 293B9
  \l 293BA
  \l 293BB
  \l 293BC
  \l 293BD
  \l 293BE
  \l 293BF
  \l 293C0
  \l 293C1
  \l 293C2
  \l 293C3
  \l 293C4
  \l 293C5
  \l 293C6
  \l 293C7
  \l 293C8
  \l 293C9
  \l 293CA
  \l 293CB
  \l 293CC
  \l 293CD
  \l 293CE
  \l 293CF
  \l 293D0
  \l 293D1
  \l 293D2
  \l 293D3
  \l 293D4
  \l 293D5
  \l 293D6
  \l 293D7
  \l 293D8
  \l 293D9
  \l 293DA
  \l 293DB
  \l 293DC
  \l 293DD
  \l 293DE
  \l 293DF
  \l 293E0
  \l 293E1
  \l 293E2
  \l 293E3
  \l 293E4
  \l 293E5
  \l 293E6
  \l 293E7
  \l 293E8
  \l 293E9
  \l 293EA
  \l 293EB
  \l 293EC
  \l 293ED
  \l 293EE
  \l 293EF
  \l 293F0
  \l 293F1
  \l 293F2
  \l 293F3
  \l 293F4
  \l 293F5
  \l 293F6
  \l 293F7
  \l 293F8
  \l 293F9
  \l 293FA
  \l 293FB
  \l 293FC
  \l 293FD
  \l 293FE
  \l 293FF
  \l 29400
  \l 29401
  \l 29402
  \l 29403
  \l 29404
  \l 29405
  \l 29406
  \l 29407
  \l 29408
  \l 29409
  \l 2940A
  \l 2940B
  \l 2940C
  \l 2940D
  \l 2940E
  \l 2940F
  \l 29410
  \l 29411
  \l 29412
  \l 29413
  \l 29414
  \l 29415
  \l 29416
  \l 29417
  \l 29418
  \l 29419
  \l 2941A
  \l 2941B
  \l 2941C
  \l 2941D
  \l 2941E
  \l 2941F
  \l 29420
  \l 29421
  \l 29422
  \l 29423
  \l 29424
  \l 29425
  \l 29426
  \l 29427
  \l 29428
  \l 29429
  \l 2942A
  \l 2942B
  \l 2942C
  \l 2942D
  \l 2942E
  \l 2942F
  \l 29430
  \l 29431
  \l 29432
  \l 29433
  \l 29434
  \l 29435
  \l 29436
  \l 29437
  \l 29438
  \l 29439
  \l 2943A
  \l 2943B
  \l 2943C
  \l 2943D
  \l 2943E
  \l 2943F
  \l 29440
  \l 29441
  \l 29442
  \l 29443
  \l 29444
  \l 29445
  \l 29446
  \l 29447
  \l 29448
  \l 29449
  \l 2944A
  \l 2944B
  \l 2944C
  \l 2944D
  \l 2944E
  \l 2944F
  \l 29450
  \l 29451
  \l 29452
  \l 29453
  \l 29454
  \l 29455
  \l 29456
  \l 29457
  \l 29458
  \l 29459
  \l 2945A
  \l 2945B
  \l 2945C
  \l 2945D
  \l 2945E
  \l 2945F
  \l 29460
  \l 29461
  \l 29462
  \l 29463
  \l 29464
  \l 29465
  \l 29466
  \l 29467
  \l 29468
  \l 29469
  \l 2946A
  \l 2946B
  \l 2946C
  \l 2946D
  \l 2946E
  \l 2946F
  \l 29470
  \l 29471
  \l 29472
  \l 29473
  \l 29474
  \l 29475
  \l 29476
  \l 29477
  \l 29478
  \l 29479
  \l 2947A
  \l 2947B
  \l 2947C
  \l 2947D
  \l 2947E
  \l 2947F
  \l 29480
  \l 29481
  \l 29482
  \l 29483
  \l 29484
  \l 29485
  \l 29486
  \l 29487
  \l 29488
  \l 29489
  \l 2948A
  \l 2948B
  \l 2948C
  \l 2948D
  \l 2948E
  \l 2948F
  \l 29490
  \l 29491
  \l 29492
  \l 29493
  \l 29494
  \l 29495
  \l 29496
  \l 29497
  \l 29498
  \l 29499
  \l 2949A
  \l 2949B
  \l 2949C
  \l 2949D
  \l 2949E
  \l 2949F
  \l 294A0
  \l 294A1
  \l 294A2
  \l 294A3
  \l 294A4
  \l 294A5
  \l 294A6
  \l 294A7
  \l 294A8
  \l 294A9
  \l 294AA
  \l 294AB
  \l 294AC
  \l 294AD
  \l 294AE
  \l 294AF
  \l 294B0
  \l 294B1
  \l 294B2
  \l 294B3
  \l 294B4
  \l 294B5
  \l 294B6
  \l 294B7
  \l 294B8
  \l 294B9
  \l 294BA
  \l 294BB
  \l 294BC
  \l 294BD
  \l 294BE
  \l 294BF
  \l 294C0
  \l 294C1
  \l 294C2
  \l 294C3
  \l 294C4
  \l 294C5
  \l 294C6
  \l 294C7
  \l 294C8
  \l 294C9
  \l 294CA
  \l 294CB
  \l 294CC
  \l 294CD
  \l 294CE
  \l 294CF
  \l 294D0
  \l 294D1
  \l 294D2
  \l 294D3
  \l 294D4
  \l 294D5
  \l 294D6
  \l 294D7
  \l 294D8
  \l 294D9
  \l 294DA
  \l 294DB
  \l 294DC
  \l 294DD
  \l 294DE
  \l 294DF
  \l 294E0
  \l 294E1
  \l 294E2
  \l 294E3
  \l 294E4
  \l 294E5
  \l 294E6
  \l 294E7
  \l 294E8
  \l 294E9
  \l 294EA
  \l 294EB
  \l 294EC
  \l 294ED
  \l 294EE
  \l 294EF
  \l 294F0
  \l 294F1
  \l 294F2
  \l 294F3
  \l 294F4
  \l 294F5
  \l 294F6
  \l 294F7
  \l 294F8
  \l 294F9
  \l 294FA
  \l 294FB
  \l 294FC
  \l 294FD
  \l 294FE
  \l 294FF
  \l 29500
  \l 29501
  \l 29502
  \l 29503
  \l 29504
  \l 29505
  \l 29506
  \l 29507
  \l 29508
  \l 29509
  \l 2950A
  \l 2950B
  \l 2950C
  \l 2950D
  \l 2950E
  \l 2950F
  \l 29510
  \l 29511
  \l 29512
  \l 29513
  \l 29514
  \l 29515
  \l 29516
  \l 29517
  \l 29518
  \l 29519
  \l 2951A
  \l 2951B
  \l 2951C
  \l 2951D
  \l 2951E
  \l 2951F
  \l 29520
  \l 29521
  \l 29522
  \l 29523
  \l 29524
  \l 29525
  \l 29526
  \l 29527
  \l 29528
  \l 29529
  \l 2952A
  \l 2952B
  \l 2952C
  \l 2952D
  \l 2952E
  \l 2952F
  \l 29530
  \l 29531
  \l 29532
  \l 29533
  \l 29534
  \l 29535
  \l 29536
  \l 29537
  \l 29538
  \l 29539
  \l 2953A
  \l 2953B
  \l 2953C
  \l 2953D
  \l 2953E
  \l 2953F
  \l 29540
  \l 29541
  \l 29542
  \l 29543
  \l 29544
  \l 29545
  \l 29546
  \l 29547
  \l 29548
  \l 29549
  \l 2954A
  \l 2954B
  \l 2954C
  \l 2954D
  \l 2954E
  \l 2954F
  \l 29550
  \l 29551
  \l 29552
  \l 29553
  \l 29554
  \l 29555
  \l 29556
  \l 29557
  \l 29558
  \l 29559
  \l 2955A
  \l 2955B
  \l 2955C
  \l 2955D
  \l 2955E
  \l 2955F
  \l 29560
  \l 29561
  \l 29562
  \l 29563
  \l 29564
  \l 29565
  \l 29566
  \l 29567
  \l 29568
  \l 29569
  \l 2956A
  \l 2956B
  \l 2956C
  \l 2956D
  \l 2956E
  \l 2956F
  \l 29570
  \l 29571
  \l 29572
  \l 29573
  \l 29574
  \l 29575
  \l 29576
  \l 29577
  \l 29578
  \l 29579
  \l 2957A
  \l 2957B
  \l 2957C
  \l 2957D
  \l 2957E
  \l 2957F
  \l 29580
  \l 29581
  \l 29582
  \l 29583
  \l 29584
  \l 29585
  \l 29586
  \l 29587
  \l 29588
  \l 29589
  \l 2958A
  \l 2958B
  \l 2958C
  \l 2958D
  \l 2958E
  \l 2958F
  \l 29590
  \l 29591
  \l 29592
  \l 29593
  \l 29594
  \l 29595
  \l 29596
  \l 29597
  \l 29598
  \l 29599
  \l 2959A
  \l 2959B
  \l 2959C
  \l 2959D
  \l 2959E
  \l 2959F
  \l 295A0
  \l 295A1
  \l 295A2
  \l 295A3
  \l 295A4
  \l 295A5
  \l 295A6
  \l 295A7
  \l 295A8
  \l 295A9
  \l 295AA
  \l 295AB
  \l 295AC
  \l 295AD
  \l 295AE
  \l 295AF
  \l 295B0
  \l 295B1
  \l 295B2
  \l 295B3
  \l 295B4
  \l 295B5
  \l 295B6
  \l 295B7
  \l 295B8
  \l 295B9
  \l 295BA
  \l 295BB
  \l 295BC
  \l 295BD
  \l 295BE
  \l 295BF
  \l 295C0
  \l 295C1
  \l 295C2
  \l 295C3
  \l 295C4
  \l 295C5
  \l 295C6
  \l 295C7
  \l 295C8
  \l 295C9
  \l 295CA
  \l 295CB
  \l 295CC
  \l 295CD
  \l 295CE
  \l 295CF
  \l 295D0
  \l 295D1
  \l 295D2
  \l 295D3
  \l 295D4
  \l 295D5
  \l 295D6
  \l 295D7
  \l 295D8
  \l 295D9
  \l 295DA
  \l 295DB
  \l 295DC
  \l 295DD
  \l 295DE
  \l 295DF
  \l 295E0
  \l 295E1
  \l 295E2
  \l 295E3
  \l 295E4
  \l 295E5
  \l 295E6
  \l 295E7
  \l 295E8
  \l 295E9
  \l 295EA
  \l 295EB
  \l 295EC
  \l 295ED
  \l 295EE
  \l 295EF
  \l 295F0
  \l 295F1
  \l 295F2
  \l 295F3
  \l 295F4
  \l 295F5
  \l 295F6
  \l 295F7
  \l 295F8
  \l 295F9
  \l 295FA
  \l 295FB
  \l 295FC
  \l 295FD
  \l 295FE
  \l 295FF
  \l 29600
  \l 29601
  \l 29602
  \l 29603
  \l 29604
  \l 29605
  \l 29606
  \l 29607
  \l 29608
  \l 29609
  \l 2960A
  \l 2960B
  \l 2960C
  \l 2960D
  \l 2960E
  \l 2960F
  \l 29610
  \l 29611
  \l 29612
  \l 29613
  \l 29614
  \l 29615
  \l 29616
  \l 29617
  \l 29618
  \l 29619
  \l 2961A
  \l 2961B
  \l 2961C
  \l 2961D
  \l 2961E
  \l 2961F
  \l 29620
  \l 29621
  \l 29622
  \l 29623
  \l 29624
  \l 29625
  \l 29626
  \l 29627
  \l 29628
  \l 29629
  \l 2962A
  \l 2962B
  \l 2962C
  \l 2962D
  \l 2962E
  \l 2962F
  \l 29630
  \l 29631
  \l 29632
  \l 29633
  \l 29634
  \l 29635
  \l 29636
  \l 29637
  \l 29638
  \l 29639
  \l 2963A
  \l 2963B
  \l 2963C
  \l 2963D
  \l 2963E
  \l 2963F
  \l 29640
  \l 29641
  \l 29642
  \l 29643
  \l 29644
  \l 29645
  \l 29646
  \l 29647
  \l 29648
  \l 29649
  \l 2964A
  \l 2964B
  \l 2964C
  \l 2964D
  \l 2964E
  \l 2964F
  \l 29650
  \l 29651
  \l 29652
  \l 29653
  \l 29654
  \l 29655
  \l 29656
  \l 29657
  \l 29658
  \l 29659
  \l 2965A
  \l 2965B
  \l 2965C
  \l 2965D
  \l 2965E
  \l 2965F
  \l 29660
  \l 29661
  \l 29662
  \l 29663
  \l 29664
  \l 29665
  \l 29666
  \l 29667
  \l 29668
  \l 29669
  \l 2966A
  \l 2966B
  \l 2966C
  \l 2966D
  \l 2966E
  \l 2966F
  \l 29670
  \l 29671
  \l 29672
  \l 29673
  \l 29674
  \l 29675
  \l 29676
  \l 29677
  \l 29678
  \l 29679
  \l 2967A
  \l 2967B
  \l 2967C
  \l 2967D
  \l 2967E
  \l 2967F
  \l 29680
  \l 29681
  \l 29682
  \l 29683
  \l 29684
  \l 29685
  \l 29686
  \l 29687
  \l 29688
  \l 29689
  \l 2968A
  \l 2968B
  \l 2968C
  \l 2968D
  \l 2968E
  \l 2968F
  \l 29690
  \l 29691
  \l 29692
  \l 29693
  \l 29694
  \l 29695
  \l 29696
  \l 29697
  \l 29698
  \l 29699
  \l 2969A
  \l 2969B
  \l 2969C
  \l 2969D
  \l 2969E
  \l 2969F
  \l 296A0
  \l 296A1
  \l 296A2
  \l 296A3
  \l 296A4
  \l 296A5
  \l 296A6
  \l 296A7
  \l 296A8
  \l 296A9
  \l 296AA
  \l 296AB
  \l 296AC
  \l 296AD
  \l 296AE
  \l 296AF
  \l 296B0
  \l 296B1
  \l 296B2
  \l 296B3
  \l 296B4
  \l 296B5
  \l 296B6
  \l 296B7
  \l 296B8
  \l 296B9
  \l 296BA
  \l 296BB
  \l 296BC
  \l 296BD
  \l 296BE
  \l 296BF
  \l 296C0
  \l 296C1
  \l 296C2
  \l 296C3
  \l 296C4
  \l 296C5
  \l 296C6
  \l 296C7
  \l 296C8
  \l 296C9
  \l 296CA
  \l 296CB
  \l 296CC
  \l 296CD
  \l 296CE
  \l 296CF
  \l 296D0
  \l 296D1
  \l 296D2
  \l 296D3
  \l 296D4
  \l 296D5
  \l 296D6
  \l 296D7
  \l 296D8
  \l 296D9
  \l 296DA
  \l 296DB
  \l 296DC
  \l 296DD
  \l 296DE
  \l 296DF
  \l 296E0
  \l 296E1
  \l 296E2
  \l 296E3
  \l 296E4
  \l 296E5
  \l 296E6
  \l 296E7
  \l 296E8
  \l 296E9
  \l 296EA
  \l 296EB
  \l 296EC
  \l 296ED
  \l 296EE
  \l 296EF
  \l 296F0
  \l 296F1
  \l 296F2
  \l 296F3
  \l 296F4
  \l 296F5
  \l 296F6
  \l 296F7
  \l 296F8
  \l 296F9
  \l 296FA
  \l 296FB
  \l 296FC
  \l 296FD
  \l 296FE
  \l 296FF
  \l 29700
  \l 29701
  \l 29702
  \l 29703
  \l 29704
  \l 29705
  \l 29706
  \l 29707
  \l 29708
  \l 29709
  \l 2970A
  \l 2970B
  \l 2970C
  \l 2970D
  \l 2970E
  \l 2970F
  \l 29710
  \l 29711
  \l 29712
  \l 29713
  \l 29714
  \l 29715
  \l 29716
  \l 29717
  \l 29718
  \l 29719
  \l 2971A
  \l 2971B
  \l 2971C
  \l 2971D
  \l 2971E
  \l 2971F
  \l 29720
  \l 29721
  \l 29722
  \l 29723
  \l 29724
  \l 29725
  \l 29726
  \l 29727
  \l 29728
  \l 29729
  \l 2972A
  \l 2972B
  \l 2972C
  \l 2972D
  \l 2972E
  \l 2972F
  \l 29730
  \l 29731
  \l 29732
  \l 29733
  \l 29734
  \l 29735
  \l 29736
  \l 29737
  \l 29738
  \l 29739
  \l 2973A
  \l 2973B
  \l 2973C
  \l 2973D
  \l 2973E
  \l 2973F
  \l 29740
  \l 29741
  \l 29742
  \l 29743
  \l 29744
  \l 29745
  \l 29746
  \l 29747
  \l 29748
  \l 29749
  \l 2974A
  \l 2974B
  \l 2974C
  \l 2974D
  \l 2974E
  \l 2974F
  \l 29750
  \l 29751
  \l 29752
  \l 29753
  \l 29754
  \l 29755
  \l 29756
  \l 29757
  \l 29758
  \l 29759
  \l 2975A
  \l 2975B
  \l 2975C
  \l 2975D
  \l 2975E
  \l 2975F
  \l 29760
  \l 29761
  \l 29762
  \l 29763
  \l 29764
  \l 29765
  \l 29766
  \l 29767
  \l 29768
  \l 29769
  \l 2976A
  \l 2976B
  \l 2976C
  \l 2976D
  \l 2976E
  \l 2976F
  \l 29770
  \l 29771
  \l 29772
  \l 29773
  \l 29774
  \l 29775
  \l 29776
  \l 29777
  \l 29778
  \l 29779
  \l 2977A
  \l 2977B
  \l 2977C
  \l 2977D
  \l 2977E
  \l 2977F
  \l 29780
  \l 29781
  \l 29782
  \l 29783
  \l 29784
  \l 29785
  \l 29786
  \l 29787
  \l 29788
  \l 29789
  \l 2978A
  \l 2978B
  \l 2978C
  \l 2978D
  \l 2978E
  \l 2978F
  \l 29790
  \l 29791
  \l 29792
  \l 29793
  \l 29794
  \l 29795
  \l 29796
  \l 29797
  \l 29798
  \l 29799
  \l 2979A
  \l 2979B
  \l 2979C
  \l 2979D
  \l 2979E
  \l 2979F
  \l 297A0
  \l 297A1
  \l 297A2
  \l 297A3
  \l 297A4
  \l 297A5
  \l 297A6
  \l 297A7
  \l 297A8
  \l 297A9
  \l 297AA
  \l 297AB
  \l 297AC
  \l 297AD
  \l 297AE
  \l 297AF
  \l 297B0
  \l 297B1
  \l 297B2
  \l 297B3
  \l 297B4
  \l 297B5
  \l 297B6
  \l 297B7
  \l 297B8
  \l 297B9
  \l 297BA
  \l 297BB
  \l 297BC
  \l 297BD
  \l 297BE
  \l 297BF
  \l 297C0
  \l 297C1
  \l 297C2
  \l 297C3
  \l 297C4
  \l 297C5
  \l 297C6
  \l 297C7
  \l 297C8
  \l 297C9
  \l 297CA
  \l 297CB
  \l 297CC
  \l 297CD
  \l 297CE
  \l 297CF
  \l 297D0
  \l 297D1
  \l 297D2
  \l 297D3
  \l 297D4
  \l 297D5
  \l 297D6
  \l 297D7
  \l 297D8
  \l 297D9
  \l 297DA
  \l 297DB
  \l 297DC
  \l 297DD
  \l 297DE
  \l 297DF
  \l 297E0
  \l 297E1
  \l 297E2
  \l 297E3
  \l 297E4
  \l 297E5
  \l 297E6
  \l 297E7
  \l 297E8
  \l 297E9
  \l 297EA
  \l 297EB
  \l 297EC
  \l 297ED
  \l 297EE
  \l 297EF
  \l 297F0
  \l 297F1
  \l 297F2
  \l 297F3
  \l 297F4
  \l 297F5
  \l 297F6
  \l 297F7
  \l 297F8
  \l 297F9
  \l 297FA
  \l 297FB
  \l 297FC
  \l 297FD
  \l 297FE
  \l 297FF
  \l 29800
  \l 29801
  \l 29802
  \l 29803
  \l 29804
  \l 29805
  \l 29806
  \l 29807
  \l 29808
  \l 29809
  \l 2980A
  \l 2980B
  \l 2980C
  \l 2980D
  \l 2980E
  \l 2980F
  \l 29810
  \l 29811
  \l 29812
  \l 29813
  \l 29814
  \l 29815
  \l 29816
  \l 29817
  \l 29818
  \l 29819
  \l 2981A
  \l 2981B
  \l 2981C
  \l 2981D
  \l 2981E
  \l 2981F
  \l 29820
  \l 29821
  \l 29822
  \l 29823
  \l 29824
  \l 29825
  \l 29826
  \l 29827
  \l 29828
  \l 29829
  \l 2982A
  \l 2982B
  \l 2982C
  \l 2982D
  \l 2982E
  \l 2982F
  \l 29830
  \l 29831
  \l 29832
  \l 29833
  \l 29834
  \l 29835
  \l 29836
  \l 29837
  \l 29838
  \l 29839
  \l 2983A
  \l 2983B
  \l 2983C
  \l 2983D
  \l 2983E
  \l 2983F
  \l 29840
  \l 29841
  \l 29842
  \l 29843
  \l 29844
  \l 29845
  \l 29846
  \l 29847
  \l 29848
  \l 29849
  \l 2984A
  \l 2984B
  \l 2984C
  \l 2984D
  \l 2984E
  \l 2984F
  \l 29850
  \l 29851
  \l 29852
  \l 29853
  \l 29854
  \l 29855
  \l 29856
  \l 29857
  \l 29858
  \l 29859
  \l 2985A
  \l 2985B
  \l 2985C
  \l 2985D
  \l 2985E
  \l 2985F
  \l 29860
  \l 29861
  \l 29862
  \l 29863
  \l 29864
  \l 29865
  \l 29866
  \l 29867
  \l 29868
  \l 29869
  \l 2986A
  \l 2986B
  \l 2986C
  \l 2986D
  \l 2986E
  \l 2986F
  \l 29870
  \l 29871
  \l 29872
  \l 29873
  \l 29874
  \l 29875
  \l 29876
  \l 29877
  \l 29878
  \l 29879
  \l 2987A
  \l 2987B
  \l 2987C
  \l 2987D
  \l 2987E
  \l 2987F
  \l 29880
  \l 29881
  \l 29882
  \l 29883
  \l 29884
  \l 29885
  \l 29886
  \l 29887
  \l 29888
  \l 29889
  \l 2988A
  \l 2988B
  \l 2988C
  \l 2988D
  \l 2988E
  \l 2988F
  \l 29890
  \l 29891
  \l 29892
  \l 29893
  \l 29894
  \l 29895
  \l 29896
  \l 29897
  \l 29898
  \l 29899
  \l 2989A
  \l 2989B
  \l 2989C
  \l 2989D
  \l 2989E
  \l 2989F
  \l 298A0
  \l 298A1
  \l 298A2
  \l 298A3
  \l 298A4
  \l 298A5
  \l 298A6
  \l 298A7
  \l 298A8
  \l 298A9
  \l 298AA
  \l 298AB
  \l 298AC
  \l 298AD
  \l 298AE
  \l 298AF
  \l 298B0
  \l 298B1
  \l 298B2
  \l 298B3
  \l 298B4
  \l 298B5
  \l 298B6
  \l 298B7
  \l 298B8
  \l 298B9
  \l 298BA
  \l 298BB
  \l 298BC
  \l 298BD
  \l 298BE
  \l 298BF
  \l 298C0
  \l 298C1
  \l 298C2
  \l 298C3
  \l 298C4
  \l 298C5
  \l 298C6
  \l 298C7
  \l 298C8
  \l 298C9
  \l 298CA
  \l 298CB
  \l 298CC
  \l 298CD
  \l 298CE
  \l 298CF
  \l 298D0
  \l 298D1
  \l 298D2
  \l 298D3
  \l 298D4
  \l 298D5
  \l 298D6
  \l 298D7
  \l 298D8
  \l 298D9
  \l 298DA
  \l 298DB
  \l 298DC
  \l 298DD
  \l 298DE
  \l 298DF
  \l 298E0
  \l 298E1
  \l 298E2
  \l 298E3
  \l 298E4
  \l 298E5
  \l 298E6
  \l 298E7
  \l 298E8
  \l 298E9
  \l 298EA
  \l 298EB
  \l 298EC
  \l 298ED
  \l 298EE
  \l 298EF
  \l 298F0
  \l 298F1
  \l 298F2
  \l 298F3
  \l 298F4
  \l 298F5
  \l 298F6
  \l 298F7
  \l 298F8
  \l 298F9
  \l 298FA
  \l 298FB
  \l 298FC
  \l 298FD
  \l 298FE
  \l 298FF
  \l 29900
  \l 29901
  \l 29902
  \l 29903
  \l 29904
  \l 29905
  \l 29906
  \l 29907
  \l 29908
  \l 29909
  \l 2990A
  \l 2990B
  \l 2990C
  \l 2990D
  \l 2990E
  \l 2990F
  \l 29910
  \l 29911
  \l 29912
  \l 29913
  \l 29914
  \l 29915
  \l 29916
  \l 29917
  \l 29918
  \l 29919
  \l 2991A
  \l 2991B
  \l 2991C
  \l 2991D
  \l 2991E
  \l 2991F
  \l 29920
  \l 29921
  \l 29922
  \l 29923
  \l 29924
  \l 29925
  \l 29926
  \l 29927
  \l 29928
  \l 29929
  \l 2992A
  \l 2992B
  \l 2992C
  \l 2992D
  \l 2992E
  \l 2992F
  \l 29930
  \l 29931
  \l 29932
  \l 29933
  \l 29934
  \l 29935
  \l 29936
  \l 29937
  \l 29938
  \l 29939
  \l 2993A
  \l 2993B
  \l 2993C
  \l 2993D
  \l 2993E
  \l 2993F
  \l 29940
  \l 29941
  \l 29942
  \l 29943
  \l 29944
  \l 29945
  \l 29946
  \l 29947
  \l 29948
  \l 29949
  \l 2994A
  \l 2994B
  \l 2994C
  \l 2994D
  \l 2994E
  \l 2994F
  \l 29950
  \l 29951
  \l 29952
  \l 29953
  \l 29954
  \l 29955
  \l 29956
  \l 29957
  \l 29958
  \l 29959
  \l 2995A
  \l 2995B
  \l 2995C
  \l 2995D
  \l 2995E
  \l 2995F
  \l 29960
  \l 29961
  \l 29962
  \l 29963
  \l 29964
  \l 29965
  \l 29966
  \l 29967
  \l 29968
  \l 29969
  \l 2996A
  \l 2996B
  \l 2996C
  \l 2996D
  \l 2996E
  \l 2996F
  \l 29970
  \l 29971
  \l 29972
  \l 29973
  \l 29974
  \l 29975
  \l 29976
  \l 29977
  \l 29978
  \l 29979
  \l 2997A
  \l 2997B
  \l 2997C
  \l 2997D
  \l 2997E
  \l 2997F
  \l 29980
  \l 29981
  \l 29982
  \l 29983
  \l 29984
  \l 29985
  \l 29986
  \l 29987
  \l 29988
  \l 29989
  \l 2998A
  \l 2998B
  \l 2998C
  \l 2998D
  \l 2998E
  \l 2998F
  \l 29990
  \l 29991
  \l 29992
  \l 29993
  \l 29994
  \l 29995
  \l 29996
  \l 29997
  \l 29998
  \l 29999
  \l 2999A
  \l 2999B
  \l 2999C
  \l 2999D
  \l 2999E
  \l 2999F
  \l 299A0
  \l 299A1
  \l 299A2
  \l 299A3
  \l 299A4
  \l 299A5
  \l 299A6
  \l 299A7
  \l 299A8
  \l 299A9
  \l 299AA
  \l 299AB
  \l 299AC
  \l 299AD
  \l 299AE
  \l 299AF
  \l 299B0
  \l 299B1
  \l 299B2
  \l 299B3
  \l 299B4
  \l 299B5
  \l 299B6
  \l 299B7
  \l 299B8
  \l 299B9
  \l 299BA
  \l 299BB
  \l 299BC
  \l 299BD
  \l 299BE
  \l 299BF
  \l 299C0
  \l 299C1
  \l 299C2
  \l 299C3
  \l 299C4
  \l 299C5
  \l 299C6
  \l 299C7
  \l 299C8
  \l 299C9
  \l 299CA
  \l 299CB
  \l 299CC
  \l 299CD
  \l 299CE
  \l 299CF
  \l 299D0
  \l 299D1
  \l 299D2
  \l 299D3
  \l 299D4
  \l 299D5
  \l 299D6
  \l 299D7
  \l 299D8
  \l 299D9
  \l 299DA
  \l 299DB
  \l 299DC
  \l 299DD
  \l 299DE
  \l 299DF
  \l 299E0
  \l 299E1
  \l 299E2
  \l 299E3
  \l 299E4
  \l 299E5
  \l 299E6
  \l 299E7
  \l 299E8
  \l 299E9
  \l 299EA
  \l 299EB
  \l 299EC
  \l 299ED
  \l 299EE
  \l 299EF
  \l 299F0
  \l 299F1
  \l 299F2
  \l 299F3
  \l 299F4
  \l 299F5
  \l 299F6
  \l 299F7
  \l 299F8
  \l 299F9
  \l 299FA
  \l 299FB
  \l 299FC
  \l 299FD
  \l 299FE
  \l 299FF
  \l 29A00
  \l 29A01
  \l 29A02
  \l 29A03
  \l 29A04
  \l 29A05
  \l 29A06
  \l 29A07
  \l 29A08
  \l 29A09
  \l 29A0A
  \l 29A0B
  \l 29A0C
  \l 29A0D
  \l 29A0E
  \l 29A0F
  \l 29A10
  \l 29A11
  \l 29A12
  \l 29A13
  \l 29A14
  \l 29A15
  \l 29A16
  \l 29A17
  \l 29A18
  \l 29A19
  \l 29A1A
  \l 29A1B
  \l 29A1C
  \l 29A1D
  \l 29A1E
  \l 29A1F
  \l 29A20
  \l 29A21
  \l 29A22
  \l 29A23
  \l 29A24
  \l 29A25
  \l 29A26
  \l 29A27
  \l 29A28
  \l 29A29
  \l 29A2A
  \l 29A2B
  \l 29A2C
  \l 29A2D
  \l 29A2E
  \l 29A2F
  \l 29A30
  \l 29A31
  \l 29A32
  \l 29A33
  \l 29A34
  \l 29A35
  \l 29A36
  \l 29A37
  \l 29A38
  \l 29A39
  \l 29A3A
  \l 29A3B
  \l 29A3C
  \l 29A3D
  \l 29A3E
  \l 29A3F
  \l 29A40
  \l 29A41
  \l 29A42
  \l 29A43
  \l 29A44
  \l 29A45
  \l 29A46
  \l 29A47
  \l 29A48
  \l 29A49
  \l 29A4A
  \l 29A4B
  \l 29A4C
  \l 29A4D
  \l 29A4E
  \l 29A4F
  \l 29A50
  \l 29A51
  \l 29A52
  \l 29A53
  \l 29A54
  \l 29A55
  \l 29A56
  \l 29A57
  \l 29A58
  \l 29A59
  \l 29A5A
  \l 29A5B
  \l 29A5C
  \l 29A5D
  \l 29A5E
  \l 29A5F
  \l 29A60
  \l 29A61
  \l 29A62
  \l 29A63
  \l 29A64
  \l 29A65
  \l 29A66
  \l 29A67
  \l 29A68
  \l 29A69
  \l 29A6A
  \l 29A6B
  \l 29A6C
  \l 29A6D
  \l 29A6E
  \l 29A6F
  \l 29A70
  \l 29A71
  \l 29A72
  \l 29A73
  \l 29A74
  \l 29A75
  \l 29A76
  \l 29A77
  \l 29A78
  \l 29A79
  \l 29A7A
  \l 29A7B
  \l 29A7C
  \l 29A7D
  \l 29A7E
  \l 29A7F
  \l 29A80
  \l 29A81
  \l 29A82
  \l 29A83
  \l 29A84
  \l 29A85
  \l 29A86
  \l 29A87
  \l 29A88
  \l 29A89
  \l 29A8A
  \l 29A8B
  \l 29A8C
  \l 29A8D
  \l 29A8E
  \l 29A8F
  \l 29A90
  \l 29A91
  \l 29A92
  \l 29A93
  \l 29A94
  \l 29A95
  \l 29A96
  \l 29A97
  \l 29A98
  \l 29A99
  \l 29A9A
  \l 29A9B
  \l 29A9C
  \l 29A9D
  \l 29A9E
  \l 29A9F
  \l 29AA0
  \l 29AA1
  \l 29AA2
  \l 29AA3
  \l 29AA4
  \l 29AA5
  \l 29AA6
  \l 29AA7
  \l 29AA8
  \l 29AA9
  \l 29AAA
  \l 29AAB
  \l 29AAC
  \l 29AAD
  \l 29AAE
  \l 29AAF
  \l 29AB0
  \l 29AB1
  \l 29AB2
  \l 29AB3
  \l 29AB4
  \l 29AB5
  \l 29AB6
  \l 29AB7
  \l 29AB8
  \l 29AB9
  \l 29ABA
  \l 29ABB
  \l 29ABC
  \l 29ABD
  \l 29ABE
  \l 29ABF
  \l 29AC0
  \l 29AC1
  \l 29AC2
  \l 29AC3
  \l 29AC4
  \l 29AC5
  \l 29AC6
  \l 29AC7
  \l 29AC8
  \l 29AC9
  \l 29ACA
  \l 29ACB
  \l 29ACC
  \l 29ACD
  \l 29ACE
  \l 29ACF
  \l 29AD0
  \l 29AD1
  \l 29AD2
  \l 29AD3
  \l 29AD4
  \l 29AD5
  \l 29AD6
  \l 29AD7
  \l 29AD8
  \l 29AD9
  \l 29ADA
  \l 29ADB
  \l 29ADC
  \l 29ADD
  \l 29ADE
  \l 29ADF
  \l 29AE0
  \l 29AE1
  \l 29AE2
  \l 29AE3
  \l 29AE4
  \l 29AE5
  \l 29AE6
  \l 29AE7
  \l 29AE8
  \l 29AE9
  \l 29AEA
  \l 29AEB
  \l 29AEC
  \l 29AED
  \l 29AEE
  \l 29AEF
  \l 29AF0
  \l 29AF1
  \l 29AF2
  \l 29AF3
  \l 29AF4
  \l 29AF5
  \l 29AF6
  \l 29AF7
  \l 29AF8
  \l 29AF9
  \l 29AFA
  \l 29AFB
  \l 29AFC
  \l 29AFD
  \l 29AFE
  \l 29AFF
  \l 29B00
  \l 29B01
  \l 29B02
  \l 29B03
  \l 29B04
  \l 29B05
  \l 29B06
  \l 29B07
  \l 29B08
  \l 29B09
  \l 29B0A
  \l 29B0B
  \l 29B0C
  \l 29B0D
  \l 29B0E
  \l 29B0F
  \l 29B10
  \l 29B11
  \l 29B12
  \l 29B13
  \l 29B14
  \l 29B15
  \l 29B16
  \l 29B17
  \l 29B18
  \l 29B19
  \l 29B1A
  \l 29B1B
  \l 29B1C
  \l 29B1D
  \l 29B1E
  \l 29B1F
  \l 29B20
  \l 29B21
  \l 29B22
  \l 29B23
  \l 29B24
  \l 29B25
  \l 29B26
  \l 29B27
  \l 29B28
  \l 29B29
  \l 29B2A
  \l 29B2B
  \l 29B2C
  \l 29B2D
  \l 29B2E
  \l 29B2F
  \l 29B30
  \l 29B31
  \l 29B32
  \l 29B33
  \l 29B34
  \l 29B35
  \l 29B36
  \l 29B37
  \l 29B38
  \l 29B39
  \l 29B3A
  \l 29B3B
  \l 29B3C
  \l 29B3D
  \l 29B3E
  \l 29B3F
  \l 29B40
  \l 29B41
  \l 29B42
  \l 29B43
  \l 29B44
  \l 29B45
  \l 29B46
  \l 29B47
  \l 29B48
  \l 29B49
  \l 29B4A
  \l 29B4B
  \l 29B4C
  \l 29B4D
  \l 29B4E
  \l 29B4F
  \l 29B50
  \l 29B51
  \l 29B52
  \l 29B53
  \l 29B54
  \l 29B55
  \l 29B56
  \l 29B57
  \l 29B58
  \l 29B59
  \l 29B5A
  \l 29B5B
  \l 29B5C
  \l 29B5D
  \l 29B5E
  \l 29B5F
  \l 29B60
  \l 29B61
  \l 29B62
  \l 29B63
  \l 29B64
  \l 29B65
  \l 29B66
  \l 29B67
  \l 29B68
  \l 29B69
  \l 29B6A
  \l 29B6B
  \l 29B6C
  \l 29B6D
  \l 29B6E
  \l 29B6F
  \l 29B70
  \l 29B71
  \l 29B72
  \l 29B73
  \l 29B74
  \l 29B75
  \l 29B76
  \l 29B77
  \l 29B78
  \l 29B79
  \l 29B7A
  \l 29B7B
  \l 29B7C
  \l 29B7D
  \l 29B7E
  \l 29B7F
  \l 29B80
  \l 29B81
  \l 29B82
  \l 29B83
  \l 29B84
  \l 29B85
  \l 29B86
  \l 29B87
  \l 29B88
  \l 29B89
  \l 29B8A
  \l 29B8B
  \l 29B8C
  \l 29B8D
  \l 29B8E
  \l 29B8F
  \l 29B90
  \l 29B91
  \l 29B92
  \l 29B93
  \l 29B94
  \l 29B95
  \l 29B96
  \l 29B97
  \l 29B98
  \l 29B99
  \l 29B9A
  \l 29B9B
  \l 29B9C
  \l 29B9D
  \l 29B9E
  \l 29B9F
  \l 29BA0
  \l 29BA1
  \l 29BA2
  \l 29BA3
  \l 29BA4
  \l 29BA5
  \l 29BA6
  \l 29BA7
  \l 29BA8
  \l 29BA9
  \l 29BAA
  \l 29BAB
  \l 29BAC
  \l 29BAD
  \l 29BAE
  \l 29BAF
  \l 29BB0
  \l 29BB1
  \l 29BB2
  \l 29BB3
  \l 29BB4
  \l 29BB5
  \l 29BB6
  \l 29BB7
  \l 29BB8
  \l 29BB9
  \l 29BBA
  \l 29BBB
  \l 29BBC
  \l 29BBD
  \l 29BBE
  \l 29BBF
  \l 29BC0
  \l 29BC1
  \l 29BC2
  \l 29BC3
  \l 29BC4
  \l 29BC5
  \l 29BC6
  \l 29BC7
  \l 29BC8
  \l 29BC9
  \l 29BCA
  \l 29BCB
  \l 29BCC
  \l 29BCD
  \l 29BCE
  \l 29BCF
  \l 29BD0
  \l 29BD1
  \l 29BD2
  \l 29BD3
  \l 29BD4
  \l 29BD5
  \l 29BD6
  \l 29BD7
  \l 29BD8
  \l 29BD9
  \l 29BDA
  \l 29BDB
  \l 29BDC
  \l 29BDD
  \l 29BDE
  \l 29BDF
  \l 29BE0
  \l 29BE1
  \l 29BE2
  \l 29BE3
  \l 29BE4
  \l 29BE5
  \l 29BE6
  \l 29BE7
  \l 29BE8
  \l 29BE9
  \l 29BEA
  \l 29BEB
  \l 29BEC
  \l 29BED
  \l 29BEE
  \l 29BEF
  \l 29BF0
  \l 29BF1
  \l 29BF2
  \l 29BF3
  \l 29BF4
  \l 29BF5
  \l 29BF6
  \l 29BF7
  \l 29BF8
  \l 29BF9
  \l 29BFA
  \l 29BFB
  \l 29BFC
  \l 29BFD
  \l 29BFE
  \l 29BFF
  \l 29C00
  \l 29C01
  \l 29C02
  \l 29C03
  \l 29C04
  \l 29C05
  \l 29C06
  \l 29C07
  \l 29C08
  \l 29C09
  \l 29C0A
  \l 29C0B
  \l 29C0C
  \l 29C0D
  \l 29C0E
  \l 29C0F
  \l 29C10
  \l 29C11
  \l 29C12
  \l 29C13
  \l 29C14
  \l 29C15
  \l 29C16
  \l 29C17
  \l 29C18
  \l 29C19
  \l 29C1A
  \l 29C1B
  \l 29C1C
  \l 29C1D
  \l 29C1E
  \l 29C1F
  \l 29C20
  \l 29C21
  \l 29C22
  \l 29C23
  \l 29C24
  \l 29C25
  \l 29C26
  \l 29C27
  \l 29C28
  \l 29C29
  \l 29C2A
  \l 29C2B
  \l 29C2C
  \l 29C2D
  \l 29C2E
  \l 29C2F
  \l 29C30
  \l 29C31
  \l 29C32
  \l 29C33
  \l 29C34
  \l 29C35
  \l 29C36
  \l 29C37
  \l 29C38
  \l 29C39
  \l 29C3A
  \l 29C3B
  \l 29C3C
  \l 29C3D
  \l 29C3E
  \l 29C3F
  \l 29C40
  \l 29C41
  \l 29C42
  \l 29C43
  \l 29C44
  \l 29C45
  \l 29C46
  \l 29C47
  \l 29C48
  \l 29C49
  \l 29C4A
  \l 29C4B
  \l 29C4C
  \l 29C4D
  \l 29C4E
  \l 29C4F
  \l 29C50
  \l 29C51
  \l 29C52
  \l 29C53
  \l 29C54
  \l 29C55
  \l 29C56
  \l 29C57
  \l 29C58
  \l 29C59
  \l 29C5A
  \l 29C5B
  \l 29C5C
  \l 29C5D
  \l 29C5E
  \l 29C5F
  \l 29C60
  \l 29C61
  \l 29C62
  \l 29C63
  \l 29C64
  \l 29C65
  \l 29C66
  \l 29C67
  \l 29C68
  \l 29C69
  \l 29C6A
  \l 29C6B
  \l 29C6C
  \l 29C6D
  \l 29C6E
  \l 29C6F
  \l 29C70
  \l 29C71
  \l 29C72
  \l 29C73
  \l 29C74
  \l 29C75
  \l 29C76
  \l 29C77
  \l 29C78
  \l 29C79
  \l 29C7A
  \l 29C7B
  \l 29C7C
  \l 29C7D
  \l 29C7E
  \l 29C7F
  \l 29C80
  \l 29C81
  \l 29C82
  \l 29C83
  \l 29C84
  \l 29C85
  \l 29C86
  \l 29C87
  \l 29C88
  \l 29C89
  \l 29C8A
  \l 29C8B
  \l 29C8C
  \l 29C8D
  \l 29C8E
  \l 29C8F
  \l 29C90
  \l 29C91
  \l 29C92
  \l 29C93
  \l 29C94
  \l 29C95
  \l 29C96
  \l 29C97
  \l 29C98
  \l 29C99
  \l 29C9A
  \l 29C9B
  \l 29C9C
  \l 29C9D
  \l 29C9E
  \l 29C9F
  \l 29CA0
  \l 29CA1
  \l 29CA2
  \l 29CA3
  \l 29CA4
  \l 29CA5
  \l 29CA6
  \l 29CA7
  \l 29CA8
  \l 29CA9
  \l 29CAA
  \l 29CAB
  \l 29CAC
  \l 29CAD
  \l 29CAE
  \l 29CAF
  \l 29CB0
  \l 29CB1
  \l 29CB2
  \l 29CB3
  \l 29CB4
  \l 29CB5
  \l 29CB6
  \l 29CB7
  \l 29CB8
  \l 29CB9
  \l 29CBA
  \l 29CBB
  \l 29CBC
  \l 29CBD
  \l 29CBE
  \l 29CBF
  \l 29CC0
  \l 29CC1
  \l 29CC2
  \l 29CC3
  \l 29CC4
  \l 29CC5
  \l 29CC6
  \l 29CC7
  \l 29CC8
  \l 29CC9
  \l 29CCA
  \l 29CCB
  \l 29CCC
  \l 29CCD
  \l 29CCE
  \l 29CCF
  \l 29CD0
  \l 29CD1
  \l 29CD2
  \l 29CD3
  \l 29CD4
  \l 29CD5
  \l 29CD6
  \l 29CD7
  \l 29CD8
  \l 29CD9
  \l 29CDA
  \l 29CDB
  \l 29CDC
  \l 29CDD
  \l 29CDE
  \l 29CDF
  \l 29CE0
  \l 29CE1
  \l 29CE2
  \l 29CE3
  \l 29CE4
  \l 29CE5
  \l 29CE6
  \l 29CE7
  \l 29CE8
  \l 29CE9
  \l 29CEA
  \l 29CEB
  \l 29CEC
  \l 29CED
  \l 29CEE
  \l 29CEF
  \l 29CF0
  \l 29CF1
  \l 29CF2
  \l 29CF3
  \l 29CF4
  \l 29CF5
  \l 29CF6
  \l 29CF7
  \l 29CF8
  \l 29CF9
  \l 29CFA
  \l 29CFB
  \l 29CFC
  \l 29CFD
  \l 29CFE
  \l 29CFF
  \l 29D00
  \l 29D01
  \l 29D02
  \l 29D03
  \l 29D04
  \l 29D05
  \l 29D06
  \l 29D07
  \l 29D08
  \l 29D09
  \l 29D0A
  \l 29D0B
  \l 29D0C
  \l 29D0D
  \l 29D0E
  \l 29D0F
  \l 29D10
  \l 29D11
  \l 29D12
  \l 29D13
  \l 29D14
  \l 29D15
  \l 29D16
  \l 29D17
  \l 29D18
  \l 29D19
  \l 29D1A
  \l 29D1B
  \l 29D1C
  \l 29D1D
  \l 29D1E
  \l 29D1F
  \l 29D20
  \l 29D21
  \l 29D22
  \l 29D23
  \l 29D24
  \l 29D25
  \l 29D26
  \l 29D27
  \l 29D28
  \l 29D29
  \l 29D2A
  \l 29D2B
  \l 29D2C
  \l 29D2D
  \l 29D2E
  \l 29D2F
  \l 29D30
  \l 29D31
  \l 29D32
  \l 29D33
  \l 29D34
  \l 29D35
  \l 29D36
  \l 29D37
  \l 29D38
  \l 29D39
  \l 29D3A
  \l 29D3B
  \l 29D3C
  \l 29D3D
  \l 29D3E
  \l 29D3F
  \l 29D40
  \l 29D41
  \l 29D42
  \l 29D43
  \l 29D44
  \l 29D45
  \l 29D46
  \l 29D47
  \l 29D48
  \l 29D49
  \l 29D4A
  \l 29D4B
  \l 29D4C
  \l 29D4D
  \l 29D4E
  \l 29D4F
  \l 29D50
  \l 29D51
  \l 29D52
  \l 29D53
  \l 29D54
  \l 29D55
  \l 29D56
  \l 29D57
  \l 29D58
  \l 29D59
  \l 29D5A
  \l 29D5B
  \l 29D5C
  \l 29D5D
  \l 29D5E
  \l 29D5F
  \l 29D60
  \l 29D61
  \l 29D62
  \l 29D63
  \l 29D64
  \l 29D65
  \l 29D66
  \l 29D67
  \l 29D68
  \l 29D69
  \l 29D6A
  \l 29D6B
  \l 29D6C
  \l 29D6D
  \l 29D6E
  \l 29D6F
  \l 29D70
  \l 29D71
  \l 29D72
  \l 29D73
  \l 29D74
  \l 29D75
  \l 29D76
  \l 29D77
  \l 29D78
  \l 29D79
  \l 29D7A
  \l 29D7B
  \l 29D7C
  \l 29D7D
  \l 29D7E
  \l 29D7F
  \l 29D80
  \l 29D81
  \l 29D82
  \l 29D83
  \l 29D84
  \l 29D85
  \l 29D86
  \l 29D87
  \l 29D88
  \l 29D89
  \l 29D8A
  \l 29D8B
  \l 29D8C
  \l 29D8D
  \l 29D8E
  \l 29D8F
  \l 29D90
  \l 29D91
  \l 29D92
  \l 29D93
  \l 29D94
  \l 29D95
  \l 29D96
  \l 29D97
  \l 29D98
  \l 29D99
  \l 29D9A
  \l 29D9B
  \l 29D9C
  \l 29D9D
  \l 29D9E
  \l 29D9F
  \l 29DA0
  \l 29DA1
  \l 29DA2
  \l 29DA3
  \l 29DA4
  \l 29DA5
  \l 29DA6
  \l 29DA7
  \l 29DA8
  \l 29DA9
  \l 29DAA
  \l 29DAB
  \l 29DAC
  \l 29DAD
  \l 29DAE
  \l 29DAF
  \l 29DB0
  \l 29DB1
  \l 29DB2
  \l 29DB3
  \l 29DB4
  \l 29DB5
  \l 29DB6
  \l 29DB7
  \l 29DB8
  \l 29DB9
  \l 29DBA
  \l 29DBB
  \l 29DBC
  \l 29DBD
  \l 29DBE
  \l 29DBF
  \l 29DC0
  \l 29DC1
  \l 29DC2
  \l 29DC3
  \l 29DC4
  \l 29DC5
  \l 29DC6
  \l 29DC7
  \l 29DC8
  \l 29DC9
  \l 29DCA
  \l 29DCB
  \l 29DCC
  \l 29DCD
  \l 29DCE
  \l 29DCF
  \l 29DD0
  \l 29DD1
  \l 29DD2
  \l 29DD3
  \l 29DD4
  \l 29DD5
  \l 29DD6
  \l 29DD7
  \l 29DD8
  \l 29DD9
  \l 29DDA
  \l 29DDB
  \l 29DDC
  \l 29DDD
  \l 29DDE
  \l 29DDF
  \l 29DE0
  \l 29DE1
  \l 29DE2
  \l 29DE3
  \l 29DE4
  \l 29DE5
  \l 29DE6
  \l 29DE7
  \l 29DE8
  \l 29DE9
  \l 29DEA
  \l 29DEB
  \l 29DEC
  \l 29DED
  \l 29DEE
  \l 29DEF
  \l 29DF0
  \l 29DF1
  \l 29DF2
  \l 29DF3
  \l 29DF4
  \l 29DF5
  \l 29DF6
  \l 29DF7
  \l 29DF8
  \l 29DF9
  \l 29DFA
  \l 29DFB
  \l 29DFC
  \l 29DFD
  \l 29DFE
  \l 29DFF
  \l 29E00
  \l 29E01
  \l 29E02
  \l 29E03
  \l 29E04
  \l 29E05
  \l 29E06
  \l 29E07
  \l 29E08
  \l 29E09
  \l 29E0A
  \l 29E0B
  \l 29E0C
  \l 29E0D
  \l 29E0E
  \l 29E0F
  \l 29E10
  \l 29E11
  \l 29E12
  \l 29E13
  \l 29E14
  \l 29E15
  \l 29E16
  \l 29E17
  \l 29E18
  \l 29E19
  \l 29E1A
  \l 29E1B
  \l 29E1C
  \l 29E1D
  \l 29E1E
  \l 29E1F
  \l 29E20
  \l 29E21
  \l 29E22
  \l 29E23
  \l 29E24
  \l 29E25
  \l 29E26
  \l 29E27
  \l 29E28
  \l 29E29
  \l 29E2A
  \l 29E2B
  \l 29E2C
  \l 29E2D
  \l 29E2E
  \l 29E2F
  \l 29E30
  \l 29E31
  \l 29E32
  \l 29E33
  \l 29E34
  \l 29E35
  \l 29E36
  \l 29E37
  \l 29E38
  \l 29E39
  \l 29E3A
  \l 29E3B
  \l 29E3C
  \l 29E3D
  \l 29E3E
  \l 29E3F
  \l 29E40
  \l 29E41
  \l 29E42
  \l 29E43
  \l 29E44
  \l 29E45
  \l 29E46
  \l 29E47
  \l 29E48
  \l 29E49
  \l 29E4A
  \l 29E4B
  \l 29E4C
  \l 29E4D
  \l 29E4E
  \l 29E4F
  \l 29E50
  \l 29E51
  \l 29E52
  \l 29E53
  \l 29E54
  \l 29E55
  \l 29E56
  \l 29E57
  \l 29E58
  \l 29E59
  \l 29E5A
  \l 29E5B
  \l 29E5C
  \l 29E5D
  \l 29E5E
  \l 29E5F
  \l 29E60
  \l 29E61
  \l 29E62
  \l 29E63
  \l 29E64
  \l 29E65
  \l 29E66
  \l 29E67
  \l 29E68
  \l 29E69
  \l 29E6A
  \l 29E6B
  \l 29E6C
  \l 29E6D
  \l 29E6E
  \l 29E6F
  \l 29E70
  \l 29E71
  \l 29E72
  \l 29E73
  \l 29E74
  \l 29E75
  \l 29E76
  \l 29E77
  \l 29E78
  \l 29E79
  \l 29E7A
  \l 29E7B
  \l 29E7C
  \l 29E7D
  \l 29E7E
  \l 29E7F
  \l 29E80
  \l 29E81
  \l 29E82
  \l 29E83
  \l 29E84
  \l 29E85
  \l 29E86
  \l 29E87
  \l 29E88
  \l 29E89
  \l 29E8A
  \l 29E8B
  \l 29E8C
  \l 29E8D
  \l 29E8E
  \l 29E8F
  \l 29E90
  \l 29E91
  \l 29E92
  \l 29E93
  \l 29E94
  \l 29E95
  \l 29E96
  \l 29E97
  \l 29E98
  \l 29E99
  \l 29E9A
  \l 29E9B
  \l 29E9C
  \l 29E9D
  \l 29E9E
  \l 29E9F
  \l 29EA0
  \l 29EA1
  \l 29EA2
  \l 29EA3
  \l 29EA4
  \l 29EA5
  \l 29EA6
  \l 29EA7
  \l 29EA8
  \l 29EA9
  \l 29EAA
  \l 29EAB
  \l 29EAC
  \l 29EAD
  \l 29EAE
  \l 29EAF
  \l 29EB0
  \l 29EB1
  \l 29EB2
  \l 29EB3
  \l 29EB4
  \l 29EB5
  \l 29EB6
  \l 29EB7
  \l 29EB8
  \l 29EB9
  \l 29EBA
  \l 29EBB
  \l 29EBC
  \l 29EBD
  \l 29EBE
  \l 29EBF
  \l 29EC0
  \l 29EC1
  \l 29EC2
  \l 29EC3
  \l 29EC4
  \l 29EC5
  \l 29EC6
  \l 29EC7
  \l 29EC8
  \l 29EC9
  \l 29ECA
  \l 29ECB
  \l 29ECC
  \l 29ECD
  \l 29ECE
  \l 29ECF
  \l 29ED0
  \l 29ED1
  \l 29ED2
  \l 29ED3
  \l 29ED4
  \l 29ED5
  \l 29ED6
  \l 29ED7
  \l 29ED8
  \l 29ED9
  \l 29EDA
  \l 29EDB
  \l 29EDC
  \l 29EDD
  \l 29EDE
  \l 29EDF
  \l 29EE0
  \l 29EE1
  \l 29EE2
  \l 29EE3
  \l 29EE4
  \l 29EE5
  \l 29EE6
  \l 29EE7
  \l 29EE8
  \l 29EE9
  \l 29EEA
  \l 29EEB
  \l 29EEC
  \l 29EED
  \l 29EEE
  \l 29EEF
  \l 29EF0
  \l 29EF1
  \l 29EF2
  \l 29EF3
  \l 29EF4
  \l 29EF5
  \l 29EF6
  \l 29EF7
  \l 29EF8
  \l 29EF9
  \l 29EFA
  \l 29EFB
  \l 29EFC
  \l 29EFD
  \l 29EFE
  \l 29EFF
  \l 29F00
  \l 29F01
  \l 29F02
  \l 29F03
  \l 29F04
  \l 29F05
  \l 29F06
  \l 29F07
  \l 29F08
  \l 29F09
  \l 29F0A
  \l 29F0B
  \l 29F0C
  \l 29F0D
  \l 29F0E
  \l 29F0F
  \l 29F10
  \l 29F11
  \l 29F12
  \l 29F13
  \l 29F14
  \l 29F15
  \l 29F16
  \l 29F17
  \l 29F18
  \l 29F19
  \l 29F1A
  \l 29F1B
  \l 29F1C
  \l 29F1D
  \l 29F1E
  \l 29F1F
  \l 29F20
  \l 29F21
  \l 29F22
  \l 29F23
  \l 29F24
  \l 29F25
  \l 29F26
  \l 29F27
  \l 29F28
  \l 29F29
  \l 29F2A
  \l 29F2B
  \l 29F2C
  \l 29F2D
  \l 29F2E
  \l 29F2F
  \l 29F30
  \l 29F31
  \l 29F32
  \l 29F33
  \l 29F34
  \l 29F35
  \l 29F36
  \l 29F37
  \l 29F38
  \l 29F39
  \l 29F3A
  \l 29F3B
  \l 29F3C
  \l 29F3D
  \l 29F3E
  \l 29F3F
  \l 29F40
  \l 29F41
  \l 29F42
  \l 29F43
  \l 29F44
  \l 29F45
  \l 29F46
  \l 29F47
  \l 29F48
  \l 29F49
  \l 29F4A
  \l 29F4B
  \l 29F4C
  \l 29F4D
  \l 29F4E
  \l 29F4F
  \l 29F50
  \l 29F51
  \l 29F52
  \l 29F53
  \l 29F54
  \l 29F55
  \l 29F56
  \l 29F57
  \l 29F58
  \l 29F59
  \l 29F5A
  \l 29F5B
  \l 29F5C
  \l 29F5D
  \l 29F5E
  \l 29F5F
  \l 29F60
  \l 29F61
  \l 29F62
  \l 29F63
  \l 29F64
  \l 29F65
  \l 29F66
  \l 29F67
  \l 29F68
  \l 29F69
  \l 29F6A
  \l 29F6B
  \l 29F6C
  \l 29F6D
  \l 29F6E
  \l 29F6F
  \l 29F70
  \l 29F71
  \l 29F72
  \l 29F73
  \l 29F74
  \l 29F75
  \l 29F76
  \l 29F77
  \l 29F78
  \l 29F79
  \l 29F7A
  \l 29F7B
  \l 29F7C
  \l 29F7D
  \l 29F7E
  \l 29F7F
  \l 29F80
  \l 29F81
  \l 29F82
  \l 29F83
  \l 29F84
  \l 29F85
  \l 29F86
  \l 29F87
  \l 29F88
  \l 29F89
  \l 29F8A
  \l 29F8B
  \l 29F8C
  \l 29F8D
  \l 29F8E
  \l 29F8F
  \l 29F90
  \l 29F91
  \l 29F92
  \l 29F93
  \l 29F94
  \l 29F95
  \l 29F96
  \l 29F97
  \l 29F98
  \l 29F99
  \l 29F9A
  \l 29F9B
  \l 29F9C
  \l 29F9D
  \l 29F9E
  \l 29F9F
  \l 29FA0
  \l 29FA1
  \l 29FA2
  \l 29FA3
  \l 29FA4
  \l 29FA5
  \l 29FA6
  \l 29FA7
  \l 29FA8
  \l 29FA9
  \l 29FAA
  \l 29FAB
  \l 29FAC
  \l 29FAD
  \l 29FAE
  \l 29FAF
  \l 29FB0
  \l 29FB1
  \l 29FB2
  \l 29FB3
  \l 29FB4
  \l 29FB5
  \l 29FB6
  \l 29FB7
  \l 29FB8
  \l 29FB9
  \l 29FBA
  \l 29FBB
  \l 29FBC
  \l 29FBD
  \l 29FBE
  \l 29FBF
  \l 29FC0
  \l 29FC1
  \l 29FC2
  \l 29FC3
  \l 29FC4
  \l 29FC5
  \l 29FC6
  \l 29FC7
  \l 29FC8
  \l 29FC9
  \l 29FCA
  \l 29FCB
  \l 29FCC
  \l 29FCD
  \l 29FCE
  \l 29FCF
  \l 29FD0
  \l 29FD1
  \l 29FD2
  \l 29FD3
  \l 29FD4
  \l 29FD5
  \l 29FD6
  \l 29FD7
  \l 29FD8
  \l 29FD9
  \l 29FDA
  \l 29FDB
  \l 29FDC
  \l 29FDD
  \l 29FDE
  \l 29FDF
  \l 29FE0
  \l 29FE1
  \l 29FE2
  \l 29FE3
  \l 29FE4
  \l 29FE5
  \l 29FE6
  \l 29FE7
  \l 29FE8
  \l 29FE9
  \l 29FEA
  \l 29FEB
  \l 29FEC
  \l 29FED
  \l 29FEE
  \l 29FEF
  \l 29FF0
  \l 29FF1
  \l 29FF2
  \l 29FF3
  \l 29FF4
  \l 29FF5
  \l 29FF6
  \l 29FF7
  \l 29FF8
  \l 29FF9
  \l 29FFA
  \l 29FFB
  \l 29FFC
  \l 29FFD
  \l 29FFE
  \l 29FFF
  \l 2A000
  \l 2A001
  \l 2A002
  \l 2A003
  \l 2A004
  \l 2A005
  \l 2A006
  \l 2A007
  \l 2A008
  \l 2A009
  \l 2A00A
  \l 2A00B
  \l 2A00C
  \l 2A00D
  \l 2A00E
  \l 2A00F
  \l 2A010
  \l 2A011
  \l 2A012
  \l 2A013
  \l 2A014
  \l 2A015
  \l 2A016
  \l 2A017
  \l 2A018
  \l 2A019
  \l 2A01A
  \l 2A01B
  \l 2A01C
  \l 2A01D
  \l 2A01E
  \l 2A01F
  \l 2A020
  \l 2A021
  \l 2A022
  \l 2A023
  \l 2A024
  \l 2A025
  \l 2A026
  \l 2A027
  \l 2A028
  \l 2A029
  \l 2A02A
  \l 2A02B
  \l 2A02C
  \l 2A02D
  \l 2A02E
  \l 2A02F
  \l 2A030
  \l 2A031
  \l 2A032
  \l 2A033
  \l 2A034
  \l 2A035
  \l 2A036
  \l 2A037
  \l 2A038
  \l 2A039
  \l 2A03A
  \l 2A03B
  \l 2A03C
  \l 2A03D
  \l 2A03E
  \l 2A03F
  \l 2A040
  \l 2A041
  \l 2A042
  \l 2A043
  \l 2A044
  \l 2A045
  \l 2A046
  \l 2A047
  \l 2A048
  \l 2A049
  \l 2A04A
  \l 2A04B
  \l 2A04C
  \l 2A04D
  \l 2A04E
  \l 2A04F
  \l 2A050
  \l 2A051
  \l 2A052
  \l 2A053
  \l 2A054
  \l 2A055
  \l 2A056
  \l 2A057
  \l 2A058
  \l 2A059
  \l 2A05A
  \l 2A05B
  \l 2A05C
  \l 2A05D
  \l 2A05E
  \l 2A05F
  \l 2A060
  \l 2A061
  \l 2A062
  \l 2A063
  \l 2A064
  \l 2A065
  \l 2A066
  \l 2A067
  \l 2A068
  \l 2A069
  \l 2A06A
  \l 2A06B
  \l 2A06C
  \l 2A06D
  \l 2A06E
  \l 2A06F
  \l 2A070
  \l 2A071
  \l 2A072
  \l 2A073
  \l 2A074
  \l 2A075
  \l 2A076
  \l 2A077
  \l 2A078
  \l 2A079
  \l 2A07A
  \l 2A07B
  \l 2A07C
  \l 2A07D
  \l 2A07E
  \l 2A07F
  \l 2A080
  \l 2A081
  \l 2A082
  \l 2A083
  \l 2A084
  \l 2A085
  \l 2A086
  \l 2A087
  \l 2A088
  \l 2A089
  \l 2A08A
  \l 2A08B
  \l 2A08C
  \l 2A08D
  \l 2A08E
  \l 2A08F
  \l 2A090
  \l 2A091
  \l 2A092
  \l 2A093
  \l 2A094
  \l 2A095
  \l 2A096
  \l 2A097
  \l 2A098
  \l 2A099
  \l 2A09A
  \l 2A09B
  \l 2A09C
  \l 2A09D
  \l 2A09E
  \l 2A09F
  \l 2A0A0
  \l 2A0A1
  \l 2A0A2
  \l 2A0A3
  \l 2A0A4
  \l 2A0A5
  \l 2A0A6
  \l 2A0A7
  \l 2A0A8
  \l 2A0A9
  \l 2A0AA
  \l 2A0AB
  \l 2A0AC
  \l 2A0AD
  \l 2A0AE
  \l 2A0AF
  \l 2A0B0
  \l 2A0B1
  \l 2A0B2
  \l 2A0B3
  \l 2A0B4
  \l 2A0B5
  \l 2A0B6
  \l 2A0B7
  \l 2A0B8
  \l 2A0B9
  \l 2A0BA
  \l 2A0BB
  \l 2A0BC
  \l 2A0BD
  \l 2A0BE
  \l 2A0BF
  \l 2A0C0
  \l 2A0C1
  \l 2A0C2
  \l 2A0C3
  \l 2A0C4
  \l 2A0C5
  \l 2A0C6
  \l 2A0C7
  \l 2A0C8
  \l 2A0C9
  \l 2A0CA
  \l 2A0CB
  \l 2A0CC
  \l 2A0CD
  \l 2A0CE
  \l 2A0CF
  \l 2A0D0
  \l 2A0D1
  \l 2A0D2
  \l 2A0D3
  \l 2A0D4
  \l 2A0D5
  \l 2A0D6
  \l 2A0D7
  \l 2A0D8
  \l 2A0D9
  \l 2A0DA
  \l 2A0DB
  \l 2A0DC
  \l 2A0DD
  \l 2A0DE
  \l 2A0DF
  \l 2A0E0
  \l 2A0E1
  \l 2A0E2
  \l 2A0E3
  \l 2A0E4
  \l 2A0E5
  \l 2A0E6
  \l 2A0E7
  \l 2A0E8
  \l 2A0E9
  \l 2A0EA
  \l 2A0EB
  \l 2A0EC
  \l 2A0ED
  \l 2A0EE
  \l 2A0EF
  \l 2A0F0
  \l 2A0F1
  \l 2A0F2
  \l 2A0F3
  \l 2A0F4
  \l 2A0F5
  \l 2A0F6
  \l 2A0F7
  \l 2A0F8
  \l 2A0F9
  \l 2A0FA
  \l 2A0FB
  \l 2A0FC
  \l 2A0FD
  \l 2A0FE
  \l 2A0FF
  \l 2A100
  \l 2A101
  \l 2A102
  \l 2A103
  \l 2A104
  \l 2A105
  \l 2A106
  \l 2A107
  \l 2A108
  \l 2A109
  \l 2A10A
  \l 2A10B
  \l 2A10C
  \l 2A10D
  \l 2A10E
  \l 2A10F
  \l 2A110
  \l 2A111
  \l 2A112
  \l 2A113
  \l 2A114
  \l 2A115
  \l 2A116
  \l 2A117
  \l 2A118
  \l 2A119
  \l 2A11A
  \l 2A11B
  \l 2A11C
  \l 2A11D
  \l 2A11E
  \l 2A11F
  \l 2A120
  \l 2A121
  \l 2A122
  \l 2A123
  \l 2A124
  \l 2A125
  \l 2A126
  \l 2A127
  \l 2A128
  \l 2A129
  \l 2A12A
  \l 2A12B
  \l 2A12C
  \l 2A12D
  \l 2A12E
  \l 2A12F
  \l 2A130
  \l 2A131
  \l 2A132
  \l 2A133
  \l 2A134
  \l 2A135
  \l 2A136
  \l 2A137
  \l 2A138
  \l 2A139
  \l 2A13A
  \l 2A13B
  \l 2A13C
  \l 2A13D
  \l 2A13E
  \l 2A13F
  \l 2A140
  \l 2A141
  \l 2A142
  \l 2A143
  \l 2A144
  \l 2A145
  \l 2A146
  \l 2A147
  \l 2A148
  \l 2A149
  \l 2A14A
  \l 2A14B
  \l 2A14C
  \l 2A14D
  \l 2A14E
  \l 2A14F
  \l 2A150
  \l 2A151
  \l 2A152
  \l 2A153
  \l 2A154
  \l 2A155
  \l 2A156
  \l 2A157
  \l 2A158
  \l 2A159
  \l 2A15A
  \l 2A15B
  \l 2A15C
  \l 2A15D
  \l 2A15E
  \l 2A15F
  \l 2A160
  \l 2A161
  \l 2A162
  \l 2A163
  \l 2A164
  \l 2A165
  \l 2A166
  \l 2A167
  \l 2A168
  \l 2A169
  \l 2A16A
  \l 2A16B
  \l 2A16C
  \l 2A16D
  \l 2A16E
  \l 2A16F
  \l 2A170
  \l 2A171
  \l 2A172
  \l 2A173
  \l 2A174
  \l 2A175
  \l 2A176
  \l 2A177
  \l 2A178
  \l 2A179
  \l 2A17A
  \l 2A17B
  \l 2A17C
  \l 2A17D
  \l 2A17E
  \l 2A17F
  \l 2A180
  \l 2A181
  \l 2A182
  \l 2A183
  \l 2A184
  \l 2A185
  \l 2A186
  \l 2A187
  \l 2A188
  \l 2A189
  \l 2A18A
  \l 2A18B
  \l 2A18C
  \l 2A18D
  \l 2A18E
  \l 2A18F
  \l 2A190
  \l 2A191
  \l 2A192
  \l 2A193
  \l 2A194
  \l 2A195
  \l 2A196
  \l 2A197
  \l 2A198
  \l 2A199
  \l 2A19A
  \l 2A19B
  \l 2A19C
  \l 2A19D
  \l 2A19E
  \l 2A19F
  \l 2A1A0
  \l 2A1A1
  \l 2A1A2
  \l 2A1A3
  \l 2A1A4
  \l 2A1A5
  \l 2A1A6
  \l 2A1A7
  \l 2A1A8
  \l 2A1A9
  \l 2A1AA
  \l 2A1AB
  \l 2A1AC
  \l 2A1AD
  \l 2A1AE
  \l 2A1AF
  \l 2A1B0
  \l 2A1B1
  \l 2A1B2
  \l 2A1B3
  \l 2A1B4
  \l 2A1B5
  \l 2A1B6
  \l 2A1B7
  \l 2A1B8
  \l 2A1B9
  \l 2A1BA
  \l 2A1BB
  \l 2A1BC
  \l 2A1BD
  \l 2A1BE
  \l 2A1BF
  \l 2A1C0
  \l 2A1C1
  \l 2A1C2
  \l 2A1C3
  \l 2A1C4
  \l 2A1C5
  \l 2A1C6
  \l 2A1C7
  \l 2A1C8
  \l 2A1C9
  \l 2A1CA
  \l 2A1CB
  \l 2A1CC
  \l 2A1CD
  \l 2A1CE
  \l 2A1CF
  \l 2A1D0
  \l 2A1D1
  \l 2A1D2
  \l 2A1D3
  \l 2A1D4
  \l 2A1D5
  \l 2A1D6
  \l 2A1D7
  \l 2A1D8
  \l 2A1D9
  \l 2A1DA
  \l 2A1DB
  \l 2A1DC
  \l 2A1DD
  \l 2A1DE
  \l 2A1DF
  \l 2A1E0
  \l 2A1E1
  \l 2A1E2
  \l 2A1E3
  \l 2A1E4
  \l 2A1E5
  \l 2A1E6
  \l 2A1E7
  \l 2A1E8
  \l 2A1E9
  \l 2A1EA
  \l 2A1EB
  \l 2A1EC
  \l 2A1ED
  \l 2A1EE
  \l 2A1EF
  \l 2A1F0
  \l 2A1F1
  \l 2A1F2
  \l 2A1F3
  \l 2A1F4
  \l 2A1F5
  \l 2A1F6
  \l 2A1F7
  \l 2A1F8
  \l 2A1F9
  \l 2A1FA
  \l 2A1FB
  \l 2A1FC
  \l 2A1FD
  \l 2A1FE
  \l 2A1FF
  \l 2A200
  \l 2A201
  \l 2A202
  \l 2A203
  \l 2A204
  \l 2A205
  \l 2A206
  \l 2A207
  \l 2A208
  \l 2A209
  \l 2A20A
  \l 2A20B
  \l 2A20C
  \l 2A20D
  \l 2A20E
  \l 2A20F
  \l 2A210
  \l 2A211
  \l 2A212
  \l 2A213
  \l 2A214
  \l 2A215
  \l 2A216
  \l 2A217
  \l 2A218
  \l 2A219
  \l 2A21A
  \l 2A21B
  \l 2A21C
  \l 2A21D
  \l 2A21E
  \l 2A21F
  \l 2A220
  \l 2A221
  \l 2A222
  \l 2A223
  \l 2A224
  \l 2A225
  \l 2A226
  \l 2A227
  \l 2A228
  \l 2A229
  \l 2A22A
  \l 2A22B
  \l 2A22C
  \l 2A22D
  \l 2A22E
  \l 2A22F
  \l 2A230
  \l 2A231
  \l 2A232
  \l 2A233
  \l 2A234
  \l 2A235
  \l 2A236
  \l 2A237
  \l 2A238
  \l 2A239
  \l 2A23A
  \l 2A23B
  \l 2A23C
  \l 2A23D
  \l 2A23E
  \l 2A23F
  \l 2A240
  \l 2A241
  \l 2A242
  \l 2A243
  \l 2A244
  \l 2A245
  \l 2A246
  \l 2A247
  \l 2A248
  \l 2A249
  \l 2A24A
  \l 2A24B
  \l 2A24C
  \l 2A24D
  \l 2A24E
  \l 2A24F
  \l 2A250
  \l 2A251
  \l 2A252
  \l 2A253
  \l 2A254
  \l 2A255
  \l 2A256
  \l 2A257
  \l 2A258
  \l 2A259
  \l 2A25A
  \l 2A25B
  \l 2A25C
  \l 2A25D
  \l 2A25E
  \l 2A25F
  \l 2A260
  \l 2A261
  \l 2A262
  \l 2A263
  \l 2A264
  \l 2A265
  \l 2A266
  \l 2A267
  \l 2A268
  \l 2A269
  \l 2A26A
  \l 2A26B
  \l 2A26C
  \l 2A26D
  \l 2A26E
  \l 2A26F
  \l 2A270
  \l 2A271
  \l 2A272
  \l 2A273
  \l 2A274
  \l 2A275
  \l 2A276
  \l 2A277
  \l 2A278
  \l 2A279
  \l 2A27A
  \l 2A27B
  \l 2A27C
  \l 2A27D
  \l 2A27E
  \l 2A27F
  \l 2A280
  \l 2A281
  \l 2A282
  \l 2A283
  \l 2A284
  \l 2A285
  \l 2A286
  \l 2A287
  \l 2A288
  \l 2A289
  \l 2A28A
  \l 2A28B
  \l 2A28C
  \l 2A28D
  \l 2A28E
  \l 2A28F
  \l 2A290
  \l 2A291
  \l 2A292
  \l 2A293
  \l 2A294
  \l 2A295
  \l 2A296
  \l 2A297
  \l 2A298
  \l 2A299
  \l 2A29A
  \l 2A29B
  \l 2A29C
  \l 2A29D
  \l 2A29E
  \l 2A29F
  \l 2A2A0
  \l 2A2A1
  \l 2A2A2
  \l 2A2A3
  \l 2A2A4
  \l 2A2A5
  \l 2A2A6
  \l 2A2A7
  \l 2A2A8
  \l 2A2A9
  \l 2A2AA
  \l 2A2AB
  \l 2A2AC
  \l 2A2AD
  \l 2A2AE
  \l 2A2AF
  \l 2A2B0
  \l 2A2B1
  \l 2A2B2
  \l 2A2B3
  \l 2A2B4
  \l 2A2B5
  \l 2A2B6
  \l 2A2B7
  \l 2A2B8
  \l 2A2B9
  \l 2A2BA
  \l 2A2BB
  \l 2A2BC
  \l 2A2BD
  \l 2A2BE
  \l 2A2BF
  \l 2A2C0
  \l 2A2C1
  \l 2A2C2
  \l 2A2C3
  \l 2A2C4
  \l 2A2C5
  \l 2A2C6
  \l 2A2C7
  \l 2A2C8
  \l 2A2C9
  \l 2A2CA
  \l 2A2CB
  \l 2A2CC
  \l 2A2CD
  \l 2A2CE
  \l 2A2CF
  \l 2A2D0
  \l 2A2D1
  \l 2A2D2
  \l 2A2D3
  \l 2A2D4
  \l 2A2D5
  \l 2A2D6
  \l 2A2D7
  \l 2A2D8
  \l 2A2D9
  \l 2A2DA
  \l 2A2DB
  \l 2A2DC
  \l 2A2DD
  \l 2A2DE
  \l 2A2DF
  \l 2A2E0
  \l 2A2E1
  \l 2A2E2
  \l 2A2E3
  \l 2A2E4
  \l 2A2E5
  \l 2A2E6
  \l 2A2E7
  \l 2A2E8
  \l 2A2E9
  \l 2A2EA
  \l 2A2EB
  \l 2A2EC
  \l 2A2ED
  \l 2A2EE
  \l 2A2EF
  \l 2A2F0
  \l 2A2F1
  \l 2A2F2
  \l 2A2F3
  \l 2A2F4
  \l 2A2F5
  \l 2A2F6
  \l 2A2F7
  \l 2A2F8
  \l 2A2F9
  \l 2A2FA
  \l 2A2FB
  \l 2A2FC
  \l 2A2FD
  \l 2A2FE
  \l 2A2FF
  \l 2A300
  \l 2A301
  \l 2A302
  \l 2A303
  \l 2A304
  \l 2A305
  \l 2A306
  \l 2A307
  \l 2A308
  \l 2A309
  \l 2A30A
  \l 2A30B
  \l 2A30C
  \l 2A30D
  \l 2A30E
  \l 2A30F
  \l 2A310
  \l 2A311
  \l 2A312
  \l 2A313
  \l 2A314
  \l 2A315
  \l 2A316
  \l 2A317
  \l 2A318
  \l 2A319
  \l 2A31A
  \l 2A31B
  \l 2A31C
  \l 2A31D
  \l 2A31E
  \l 2A31F
  \l 2A320
  \l 2A321
  \l 2A322
  \l 2A323
  \l 2A324
  \l 2A325
  \l 2A326
  \l 2A327
  \l 2A328
  \l 2A329
  \l 2A32A
  \l 2A32B
  \l 2A32C
  \l 2A32D
  \l 2A32E
  \l 2A32F
  \l 2A330
  \l 2A331
  \l 2A332
  \l 2A333
  \l 2A334
  \l 2A335
  \l 2A336
  \l 2A337
  \l 2A338
  \l 2A339
  \l 2A33A
  \l 2A33B
  \l 2A33C
  \l 2A33D
  \l 2A33E
  \l 2A33F
  \l 2A340
  \l 2A341
  \l 2A342
  \l 2A343
  \l 2A344
  \l 2A345
  \l 2A346
  \l 2A347
  \l 2A348
  \l 2A349
  \l 2A34A
  \l 2A34B
  \l 2A34C
  \l 2A34D
  \l 2A34E
  \l 2A34F
  \l 2A350
  \l 2A351
  \l 2A352
  \l 2A353
  \l 2A354
  \l 2A355
  \l 2A356
  \l 2A357
  \l 2A358
  \l 2A359
  \l 2A35A
  \l 2A35B
  \l 2A35C
  \l 2A35D
  \l 2A35E
  \l 2A35F
  \l 2A360
  \l 2A361
  \l 2A362
  \l 2A363
  \l 2A364
  \l 2A365
  \l 2A366
  \l 2A367
  \l 2A368
  \l 2A369
  \l 2A36A
  \l 2A36B
  \l 2A36C
  \l 2A36D
  \l 2A36E
  \l 2A36F
  \l 2A370
  \l 2A371
  \l 2A372
  \l 2A373
  \l 2A374
  \l 2A375
  \l 2A376
  \l 2A377
  \l 2A378
  \l 2A379
  \l 2A37A
  \l 2A37B
  \l 2A37C
  \l 2A37D
  \l 2A37E
  \l 2A37F
  \l 2A380
  \l 2A381
  \l 2A382
  \l 2A383
  \l 2A384
  \l 2A385
  \l 2A386
  \l 2A387
  \l 2A388
  \l 2A389
  \l 2A38A
  \l 2A38B
  \l 2A38C
  \l 2A38D
  \l 2A38E
  \l 2A38F
  \l 2A390
  \l 2A391
  \l 2A392
  \l 2A393
  \l 2A394
  \l 2A395
  \l 2A396
  \l 2A397
  \l 2A398
  \l 2A399
  \l 2A39A
  \l 2A39B
  \l 2A39C
  \l 2A39D
  \l 2A39E
  \l 2A39F
  \l 2A3A0
  \l 2A3A1
  \l 2A3A2
  \l 2A3A3
  \l 2A3A4
  \l 2A3A5
  \l 2A3A6
  \l 2A3A7
  \l 2A3A8
  \l 2A3A9
  \l 2A3AA
  \l 2A3AB
  \l 2A3AC
  \l 2A3AD
  \l 2A3AE
  \l 2A3AF
  \l 2A3B0
  \l 2A3B1
  \l 2A3B2
  \l 2A3B3
  \l 2A3B4
  \l 2A3B5
  \l 2A3B6
  \l 2A3B7
  \l 2A3B8
  \l 2A3B9
  \l 2A3BA
  \l 2A3BB
  \l 2A3BC
  \l 2A3BD
  \l 2A3BE
  \l 2A3BF
  \l 2A3C0
  \l 2A3C1
  \l 2A3C2
  \l 2A3C3
  \l 2A3C4
  \l 2A3C5
  \l 2A3C6
  \l 2A3C7
  \l 2A3C8
  \l 2A3C9
  \l 2A3CA
  \l 2A3CB
  \l 2A3CC
  \l 2A3CD
  \l 2A3CE
  \l 2A3CF
  \l 2A3D0
  \l 2A3D1
  \l 2A3D2
  \l 2A3D3
  \l 2A3D4
  \l 2A3D5
  \l 2A3D6
  \l 2A3D7
  \l 2A3D8
  \l 2A3D9
  \l 2A3DA
  \l 2A3DB
  \l 2A3DC
  \l 2A3DD
  \l 2A3DE
  \l 2A3DF
  \l 2A3E0
  \l 2A3E1
  \l 2A3E2
  \l 2A3E3
  \l 2A3E4
  \l 2A3E5
  \l 2A3E6
  \l 2A3E7
  \l 2A3E8
  \l 2A3E9
  \l 2A3EA
  \l 2A3EB
  \l 2A3EC
  \l 2A3ED
  \l 2A3EE
  \l 2A3EF
  \l 2A3F0
  \l 2A3F1
  \l 2A3F2
  \l 2A3F3
  \l 2A3F4
  \l 2A3F5
  \l 2A3F6
  \l 2A3F7
  \l 2A3F8
  \l 2A3F9
  \l 2A3FA
  \l 2A3FB
  \l 2A3FC
  \l 2A3FD
  \l 2A3FE
  \l 2A3FF
  \l 2A400
  \l 2A401
  \l 2A402
  \l 2A403
  \l 2A404
  \l 2A405
  \l 2A406
  \l 2A407
  \l 2A408
  \l 2A409
  \l 2A40A
  \l 2A40B
  \l 2A40C
  \l 2A40D
  \l 2A40E
  \l 2A40F
  \l 2A410
  \l 2A411
  \l 2A412
  \l 2A413
  \l 2A414
  \l 2A415
  \l 2A416
  \l 2A417
  \l 2A418
  \l 2A419
  \l 2A41A
  \l 2A41B
  \l 2A41C
  \l 2A41D
  \l 2A41E
  \l 2A41F
  \l 2A420
  \l 2A421
  \l 2A422
  \l 2A423
  \l 2A424
  \l 2A425
  \l 2A426
  \l 2A427
  \l 2A428
  \l 2A429
  \l 2A42A
  \l 2A42B
  \l 2A42C
  \l 2A42D
  \l 2A42E
  \l 2A42F
  \l 2A430
  \l 2A431
  \l 2A432
  \l 2A433
  \l 2A434
  \l 2A435
  \l 2A436
  \l 2A437
  \l 2A438
  \l 2A439
  \l 2A43A
  \l 2A43B
  \l 2A43C
  \l 2A43D
  \l 2A43E
  \l 2A43F
  \l 2A440
  \l 2A441
  \l 2A442
  \l 2A443
  \l 2A444
  \l 2A445
  \l 2A446
  \l 2A447
  \l 2A448
  \l 2A449
  \l 2A44A
  \l 2A44B
  \l 2A44C
  \l 2A44D
  \l 2A44E
  \l 2A44F
  \l 2A450
  \l 2A451
  \l 2A452
  \l 2A453
  \l 2A454
  \l 2A455
  \l 2A456
  \l 2A457
  \l 2A458
  \l 2A459
  \l 2A45A
  \l 2A45B
  \l 2A45C
  \l 2A45D
  \l 2A45E
  \l 2A45F
  \l 2A460
  \l 2A461
  \l 2A462
  \l 2A463
  \l 2A464
  \l 2A465
  \l 2A466
  \l 2A467
  \l 2A468
  \l 2A469
  \l 2A46A
  \l 2A46B
  \l 2A46C
  \l 2A46D
  \l 2A46E
  \l 2A46F
  \l 2A470
  \l 2A471
  \l 2A472
  \l 2A473
  \l 2A474
  \l 2A475
  \l 2A476
  \l 2A477
  \l 2A478
  \l 2A479
  \l 2A47A
  \l 2A47B
  \l 2A47C
  \l 2A47D
  \l 2A47E
  \l 2A47F
  \l 2A480
  \l 2A481
  \l 2A482
  \l 2A483
  \l 2A484
  \l 2A485
  \l 2A486
  \l 2A487
  \l 2A488
  \l 2A489
  \l 2A48A
  \l 2A48B
  \l 2A48C
  \l 2A48D
  \l 2A48E
  \l 2A48F
  \l 2A490
  \l 2A491
  \l 2A492
  \l 2A493
  \l 2A494
  \l 2A495
  \l 2A496
  \l 2A497
  \l 2A498
  \l 2A499
  \l 2A49A
  \l 2A49B
  \l 2A49C
  \l 2A49D
  \l 2A49E
  \l 2A49F
  \l 2A4A0
  \l 2A4A1
  \l 2A4A2
  \l 2A4A3
  \l 2A4A4
  \l 2A4A5
  \l 2A4A6
  \l 2A4A7
  \l 2A4A8
  \l 2A4A9
  \l 2A4AA
  \l 2A4AB
  \l 2A4AC
  \l 2A4AD
  \l 2A4AE
  \l 2A4AF
  \l 2A4B0
  \l 2A4B1
  \l 2A4B2
  \l 2A4B3
  \l 2A4B4
  \l 2A4B5
  \l 2A4B6
  \l 2A4B7
  \l 2A4B8
  \l 2A4B9
  \l 2A4BA
  \l 2A4BB
  \l 2A4BC
  \l 2A4BD
  \l 2A4BE
  \l 2A4BF
  \l 2A4C0
  \l 2A4C1
  \l 2A4C2
  \l 2A4C3
  \l 2A4C4
  \l 2A4C5
  \l 2A4C6
  \l 2A4C7
  \l 2A4C8
  \l 2A4C9
  \l 2A4CA
  \l 2A4CB
  \l 2A4CC
  \l 2A4CD
  \l 2A4CE
  \l 2A4CF
  \l 2A4D0
  \l 2A4D1
  \l 2A4D2
  \l 2A4D3
  \l 2A4D4
  \l 2A4D5
  \l 2A4D6
  \l 2A4D7
  \l 2A4D8
  \l 2A4D9
  \l 2A4DA
  \l 2A4DB
  \l 2A4DC
  \l 2A4DD
  \l 2A4DE
  \l 2A4DF
  \l 2A4E0
  \l 2A4E1
  \l 2A4E2
  \l 2A4E3
  \l 2A4E4
  \l 2A4E5
  \l 2A4E6
  \l 2A4E7
  \l 2A4E8
  \l 2A4E9
  \l 2A4EA
  \l 2A4EB
  \l 2A4EC
  \l 2A4ED
  \l 2A4EE
  \l 2A4EF
  \l 2A4F0
  \l 2A4F1
  \l 2A4F2
  \l 2A4F3
  \l 2A4F4
  \l 2A4F5
  \l 2A4F6
  \l 2A4F7
  \l 2A4F8
  \l 2A4F9
  \l 2A4FA
  \l 2A4FB
  \l 2A4FC
  \l 2A4FD
  \l 2A4FE
  \l 2A4FF
  \l 2A500
  \l 2A501
  \l 2A502
  \l 2A503
  \l 2A504
  \l 2A505
  \l 2A506
  \l 2A507
  \l 2A508
  \l 2A509
  \l 2A50A
  \l 2A50B
  \l 2A50C
  \l 2A50D
  \l 2A50E
  \l 2A50F
  \l 2A510
  \l 2A511
  \l 2A512
  \l 2A513
  \l 2A514
  \l 2A515
  \l 2A516
  \l 2A517
  \l 2A518
  \l 2A519
  \l 2A51A
  \l 2A51B
  \l 2A51C
  \l 2A51D
  \l 2A51E
  \l 2A51F
  \l 2A520
  \l 2A521
  \l 2A522
  \l 2A523
  \l 2A524
  \l 2A525
  \l 2A526
  \l 2A527
  \l 2A528
  \l 2A529
  \l 2A52A
  \l 2A52B
  \l 2A52C
  \l 2A52D
  \l 2A52E
  \l 2A52F
  \l 2A530
  \l 2A531
  \l 2A532
  \l 2A533
  \l 2A534
  \l 2A535
  \l 2A536
  \l 2A537
  \l 2A538
  \l 2A539
  \l 2A53A
  \l 2A53B
  \l 2A53C
  \l 2A53D
  \l 2A53E
  \l 2A53F
  \l 2A540
  \l 2A541
  \l 2A542
  \l 2A543
  \l 2A544
  \l 2A545
  \l 2A546
  \l 2A547
  \l 2A548
  \l 2A549
  \l 2A54A
  \l 2A54B
  \l 2A54C
  \l 2A54D
  \l 2A54E
  \l 2A54F
  \l 2A550
  \l 2A551
  \l 2A552
  \l 2A553
  \l 2A554
  \l 2A555
  \l 2A556
  \l 2A557
  \l 2A558
  \l 2A559
  \l 2A55A
  \l 2A55B
  \l 2A55C
  \l 2A55D
  \l 2A55E
  \l 2A55F
  \l 2A560
  \l 2A561
  \l 2A562
  \l 2A563
  \l 2A564
  \l 2A565
  \l 2A566
  \l 2A567
  \l 2A568
  \l 2A569
  \l 2A56A
  \l 2A56B
  \l 2A56C
  \l 2A56D
  \l 2A56E
  \l 2A56F
  \l 2A570
  \l 2A571
  \l 2A572
  \l 2A573
  \l 2A574
  \l 2A575
  \l 2A576
  \l 2A577
  \l 2A578
  \l 2A579
  \l 2A57A
  \l 2A57B
  \l 2A57C
  \l 2A57D
  \l 2A57E
  \l 2A57F
  \l 2A580
  \l 2A581
  \l 2A582
  \l 2A583
  \l 2A584
  \l 2A585
  \l 2A586
  \l 2A587
  \l 2A588
  \l 2A589
  \l 2A58A
  \l 2A58B
  \l 2A58C
  \l 2A58D
  \l 2A58E
  \l 2A58F
  \l 2A590
  \l 2A591
  \l 2A592
  \l 2A593
  \l 2A594
  \l 2A595
  \l 2A596
  \l 2A597
  \l 2A598
  \l 2A599
  \l 2A59A
  \l 2A59B
  \l 2A59C
  \l 2A59D
  \l 2A59E
  \l 2A59F
  \l 2A5A0
  \l 2A5A1
  \l 2A5A2
  \l 2A5A3
  \l 2A5A4
  \l 2A5A5
  \l 2A5A6
  \l 2A5A7
  \l 2A5A8
  \l 2A5A9
  \l 2A5AA
  \l 2A5AB
  \l 2A5AC
  \l 2A5AD
  \l 2A5AE
  \l 2A5AF
  \l 2A5B0
  \l 2A5B1
  \l 2A5B2
  \l 2A5B3
  \l 2A5B4
  \l 2A5B5
  \l 2A5B6
  \l 2A5B7
  \l 2A5B8
  \l 2A5B9
  \l 2A5BA
  \l 2A5BB
  \l 2A5BC
  \l 2A5BD
  \l 2A5BE
  \l 2A5BF
  \l 2A5C0
  \l 2A5C1
  \l 2A5C2
  \l 2A5C3
  \l 2A5C4
  \l 2A5C5
  \l 2A5C6
  \l 2A5C7
  \l 2A5C8
  \l 2A5C9
  \l 2A5CA
  \l 2A5CB
  \l 2A5CC
  \l 2A5CD
  \l 2A5CE
  \l 2A5CF
  \l 2A5D0
  \l 2A5D1
  \l 2A5D2
  \l 2A5D3
  \l 2A5D4
  \l 2A5D5
  \l 2A5D6
  \l 2A5D7
  \l 2A5D8
  \l 2A5D9
  \l 2A5DA
  \l 2A5DB
  \l 2A5DC
  \l 2A5DD
  \l 2A5DE
  \l 2A5DF
  \l 2A5E0
  \l 2A5E1
  \l 2A5E2
  \l 2A5E3
  \l 2A5E4
  \l 2A5E5
  \l 2A5E6
  \l 2A5E7
  \l 2A5E8
  \l 2A5E9
  \l 2A5EA
  \l 2A5EB
  \l 2A5EC
  \l 2A5ED
  \l 2A5EE
  \l 2A5EF
  \l 2A5F0
  \l 2A5F1
  \l 2A5F2
  \l 2A5F3
  \l 2A5F4
  \l 2A5F5
  \l 2A5F6
  \l 2A5F7
  \l 2A5F8
  \l 2A5F9
  \l 2A5FA
  \l 2A5FB
  \l 2A5FC
  \l 2A5FD
  \l 2A5FE
  \l 2A5FF
  \l 2A600
  \l 2A601
  \l 2A602
  \l 2A603
  \l 2A604
  \l 2A605
  \l 2A606
  \l 2A607
  \l 2A608
  \l 2A609
  \l 2A60A
  \l 2A60B
  \l 2A60C
  \l 2A60D
  \l 2A60E
  \l 2A60F
  \l 2A610
  \l 2A611
  \l 2A612
  \l 2A613
  \l 2A614
  \l 2A615
  \l 2A616
  \l 2A617
  \l 2A618
  \l 2A619
  \l 2A61A
  \l 2A61B
  \l 2A61C
  \l 2A61D
  \l 2A61E
  \l 2A61F
  \l 2A620
  \l 2A621
  \l 2A622
  \l 2A623
  \l 2A624
  \l 2A625
  \l 2A626
  \l 2A627
  \l 2A628
  \l 2A629
  \l 2A62A
  \l 2A62B
  \l 2A62C
  \l 2A62D
  \l 2A62E
  \l 2A62F
  \l 2A630
  \l 2A631
  \l 2A632
  \l 2A633
  \l 2A634
  \l 2A635
  \l 2A636
  \l 2A637
  \l 2A638
  \l 2A639
  \l 2A63A
  \l 2A63B
  \l 2A63C
  \l 2A63D
  \l 2A63E
  \l 2A63F
  \l 2A640
  \l 2A641
  \l 2A642
  \l 2A643
  \l 2A644
  \l 2A645
  \l 2A646
  \l 2A647
  \l 2A648
  \l 2A649
  \l 2A64A
  \l 2A64B
  \l 2A64C
  \l 2A64D
  \l 2A64E
  \l 2A64F
  \l 2A650
  \l 2A651
  \l 2A652
  \l 2A653
  \l 2A654
  \l 2A655
  \l 2A656
  \l 2A657
  \l 2A658
  \l 2A659
  \l 2A65A
  \l 2A65B
  \l 2A65C
  \l 2A65D
  \l 2A65E
  \l 2A65F
  \l 2A660
  \l 2A661
  \l 2A662
  \l 2A663
  \l 2A664
  \l 2A665
  \l 2A666
  \l 2A667
  \l 2A668
  \l 2A669
  \l 2A66A
  \l 2A66B
  \l 2A66C
  \l 2A66D
  \l 2A66E
  \l 2A66F
  \l 2A670
  \l 2A671
  \l 2A672
  \l 2A673
  \l 2A674
  \l 2A675
  \l 2A676
  \l 2A677
  \l 2A678
  \l 2A679
  \l 2A67A
  \l 2A67B
  \l 2A67C
  \l 2A67D
  \l 2A67E
  \l 2A67F
  \l 2A680
  \l 2A681
  \l 2A682
  \l 2A683
  \l 2A684
  \l 2A685
  \l 2A686
  \l 2A687
  \l 2A688
  \l 2A689
  \l 2A68A
  \l 2A68B
  \l 2A68C
  \l 2A68D
  \l 2A68E
  \l 2A68F
  \l 2A690
  \l 2A691
  \l 2A692
  \l 2A693
  \l 2A694
  \l 2A695
  \l 2A696
  \l 2A697
  \l 2A698
  \l 2A699
  \l 2A69A
  \l 2A69B
  \l 2A69C
  \l 2A69D
  \l 2A69E
  \l 2A69F
  \l 2A6A0
  \l 2A6A1
  \l 2A6A2
  \l 2A6A3
  \l 2A6A4
  \l 2A6A5
  \l 2A6A6
  \l 2A6A7
  \l 2A6A8
  \l 2A6A9
  \l 2A6AA
  \l 2A6AB
  \l 2A6AC
  \l 2A6AD
  \l 2A6AE
  \l 2A6AF
  \l 2A6B0
  \l 2A6B1
  \l 2A6B2
  \l 2A6B3
  \l 2A6B4
  \l 2A6B5
  \l 2A6B6
  \l 2A6B7
  \l 2A6B8
  \l 2A6B9
  \l 2A6BA
  \l 2A6BB
  \l 2A6BC
  \l 2A6BD
  \l 2A6BE
  \l 2A6BF
  \l 2A6C0
  \l 2A6C1
  \l 2A6C2
  \l 2A6C3
  \l 2A6C4
  \l 2A6C5
  \l 2A6C6
  \l 2A6C7
  \l 2A6C8
  \l 2A6C9
  \l 2A6CA
  \l 2A6CB
  \l 2A6CC
  \l 2A6CD
  \l 2A6CE
  \l 2A6CF
  \l 2A6D0
  \l 2A6D1
  \l 2A6D2
  \l 2A6D3
  \l 2A6D4
  \l 2A6D5
  \l 2A6D6
  \l 2A700
  \l 2A701
  \l 2A702
  \l 2A703
  \l 2A704
  \l 2A705
  \l 2A706
  \l 2A707
  \l 2A708
  \l 2A709
  \l 2A70A
  \l 2A70B
  \l 2A70C
  \l 2A70D
  \l 2A70E
  \l 2A70F
  \l 2A710
  \l 2A711
  \l 2A712
  \l 2A713
  \l 2A714
  \l 2A715
  \l 2A716
  \l 2A717
  \l 2A718
  \l 2A719
  \l 2A71A
  \l 2A71B
  \l 2A71C
  \l 2A71D
  \l 2A71E
  \l 2A71F
  \l 2A720
  \l 2A721
  \l 2A722
  \l 2A723
  \l 2A724
  \l 2A725
  \l 2A726
  \l 2A727
  \l 2A728
  \l 2A729
  \l 2A72A
  \l 2A72B
  \l 2A72C
  \l 2A72D
  \l 2A72E
  \l 2A72F
  \l 2A730
  \l 2A731
  \l 2A732
  \l 2A733
  \l 2A734
  \l 2A735
  \l 2A736
  \l 2A737
  \l 2A738
  \l 2A739
  \l 2A73A
  \l 2A73B
  \l 2A73C
  \l 2A73D
  \l 2A73E
  \l 2A73F
  \l 2A740
  \l 2A741
  \l 2A742
  \l 2A743
  \l 2A744
  \l 2A745
  \l 2A746
  \l 2A747
  \l 2A748
  \l 2A749
  \l 2A74A
  \l 2A74B
  \l 2A74C
  \l 2A74D
  \l 2A74E
  \l 2A74F
  \l 2A750
  \l 2A751
  \l 2A752
  \l 2A753
  \l 2A754
  \l 2A755
  \l 2A756
  \l 2A757
  \l 2A758
  \l 2A759
  \l 2A75A
  \l 2A75B
  \l 2A75C
  \l 2A75D
  \l 2A75E
  \l 2A75F
  \l 2A760
  \l 2A761
  \l 2A762
  \l 2A763
  \l 2A764
  \l 2A765
  \l 2A766
  \l 2A767
  \l 2A768
  \l 2A769
  \l 2A76A
  \l 2A76B
  \l 2A76C
  \l 2A76D
  \l 2A76E
  \l 2A76F
  \l 2A770
  \l 2A771
  \l 2A772
  \l 2A773
  \l 2A774
  \l 2A775
  \l 2A776
  \l 2A777
  \l 2A778
  \l 2A779
  \l 2A77A
  \l 2A77B
  \l 2A77C
  \l 2A77D
  \l 2A77E
  \l 2A77F
  \l 2A780
  \l 2A781
  \l 2A782
  \l 2A783
  \l 2A784
  \l 2A785
  \l 2A786
  \l 2A787
  \l 2A788
  \l 2A789
  \l 2A78A
  \l 2A78B
  \l 2A78C
  \l 2A78D
  \l 2A78E
  \l 2A78F
  \l 2A790
  \l 2A791
  \l 2A792
  \l 2A793
  \l 2A794
  \l 2A795
  \l 2A796
  \l 2A797
  \l 2A798
  \l 2A799
  \l 2A79A
  \l 2A79B
  \l 2A79C
  \l 2A79D
  \l 2A79E
  \l 2A79F
  \l 2A7A0
  \l 2A7A1
  \l 2A7A2
  \l 2A7A3
  \l 2A7A4
  \l 2A7A5
  \l 2A7A6
  \l 2A7A7
  \l 2A7A8
  \l 2A7A9
  \l 2A7AA
  \l 2A7AB
  \l 2A7AC
  \l 2A7AD
  \l 2A7AE
  \l 2A7AF
  \l 2A7B0
  \l 2A7B1
  \l 2A7B2
  \l 2A7B3
  \l 2A7B4
  \l 2A7B5
  \l 2A7B6
  \l 2A7B7
  \l 2A7B8
  \l 2A7B9
  \l 2A7BA
  \l 2A7BB
  \l 2A7BC
  \l 2A7BD
  \l 2A7BE
  \l 2A7BF
  \l 2A7C0
  \l 2A7C1
  \l 2A7C2
  \l 2A7C3
  \l 2A7C4
  \l 2A7C5
  \l 2A7C6
  \l 2A7C7
  \l 2A7C8
  \l 2A7C9
  \l 2A7CA
  \l 2A7CB
  \l 2A7CC
  \l 2A7CD
  \l 2A7CE
  \l 2A7CF
  \l 2A7D0
  \l 2A7D1
  \l 2A7D2
  \l 2A7D3
  \l 2A7D4
  \l 2A7D5
  \l 2A7D6
  \l 2A7D7
  \l 2A7D8
  \l 2A7D9
  \l 2A7DA
  \l 2A7DB
  \l 2A7DC
  \l 2A7DD
  \l 2A7DE
  \l 2A7DF
  \l 2A7E0
  \l 2A7E1
  \l 2A7E2
  \l 2A7E3
  \l 2A7E4
  \l 2A7E5
  \l 2A7E6
  \l 2A7E7
  \l 2A7E8
  \l 2A7E9
  \l 2A7EA
  \l 2A7EB
  \l 2A7EC
  \l 2A7ED
  \l 2A7EE
  \l 2A7EF
  \l 2A7F0
  \l 2A7F1
  \l 2A7F2
  \l 2A7F3
  \l 2A7F4
  \l 2A7F5
  \l 2A7F6
  \l 2A7F7
  \l 2A7F8
  \l 2A7F9
  \l 2A7FA
  \l 2A7FB
  \l 2A7FC
  \l 2A7FD
  \l 2A7FE
  \l 2A7FF
  \l 2A800
  \l 2A801
  \l 2A802
  \l 2A803
  \l 2A804
  \l 2A805
  \l 2A806
  \l 2A807
  \l 2A808
  \l 2A809
  \l 2A80A
  \l 2A80B
  \l 2A80C
  \l 2A80D
  \l 2A80E
  \l 2A80F
  \l 2A810
  \l 2A811
  \l 2A812
  \l 2A813
  \l 2A814
  \l 2A815
  \l 2A816
  \l 2A817
  \l 2A818
  \l 2A819
  \l 2A81A
  \l 2A81B
  \l 2A81C
  \l 2A81D
  \l 2A81E
  \l 2A81F
  \l 2A820
  \l 2A821
  \l 2A822
  \l 2A823
  \l 2A824
  \l 2A825
  \l 2A826
  \l 2A827
  \l 2A828
  \l 2A829
  \l 2A82A
  \l 2A82B
  \l 2A82C
  \l 2A82D
  \l 2A82E
  \l 2A82F
  \l 2A830
  \l 2A831
  \l 2A832
  \l 2A833
  \l 2A834
  \l 2A835
  \l 2A836
  \l 2A837
  \l 2A838
  \l 2A839
  \l 2A83A
  \l 2A83B
  \l 2A83C
  \l 2A83D
  \l 2A83E
  \l 2A83F
  \l 2A840
  \l 2A841
  \l 2A842
  \l 2A843
  \l 2A844
  \l 2A845
  \l 2A846
  \l 2A847
  \l 2A848
  \l 2A849
  \l 2A84A
  \l 2A84B
  \l 2A84C
  \l 2A84D
  \l 2A84E
  \l 2A84F
  \l 2A850
  \l 2A851
  \l 2A852
  \l 2A853
  \l 2A854
  \l 2A855
  \l 2A856
  \l 2A857
  \l 2A858
  \l 2A859
  \l 2A85A
  \l 2A85B
  \l 2A85C
  \l 2A85D
  \l 2A85E
  \l 2A85F
  \l 2A860
  \l 2A861
  \l 2A862
  \l 2A863
  \l 2A864
  \l 2A865
  \l 2A866
  \l 2A867
  \l 2A868
  \l 2A869
  \l 2A86A
  \l 2A86B
  \l 2A86C
  \l 2A86D
  \l 2A86E
  \l 2A86F
  \l 2A870
  \l 2A871
  \l 2A872
  \l 2A873
  \l 2A874
  \l 2A875
  \l 2A876
  \l 2A877
  \l 2A878
  \l 2A879
  \l 2A87A
  \l 2A87B
  \l 2A87C
  \l 2A87D
  \l 2A87E
  \l 2A87F
  \l 2A880
  \l 2A881
  \l 2A882
  \l 2A883
  \l 2A884
  \l 2A885
  \l 2A886
  \l 2A887
  \l 2A888
  \l 2A889
  \l 2A88A
  \l 2A88B
  \l 2A88C
  \l 2A88D
  \l 2A88E
  \l 2A88F
  \l 2A890
  \l 2A891
  \l 2A892
  \l 2A893
  \l 2A894
  \l 2A895
  \l 2A896
  \l 2A897
  \l 2A898
  \l 2A899
  \l 2A89A
  \l 2A89B
  \l 2A89C
  \l 2A89D
  \l 2A89E
  \l 2A89F
  \l 2A8A0
  \l 2A8A1
  \l 2A8A2
  \l 2A8A3
  \l 2A8A4
  \l 2A8A5
  \l 2A8A6
  \l 2A8A7
  \l 2A8A8
  \l 2A8A9
  \l 2A8AA
  \l 2A8AB
  \l 2A8AC
  \l 2A8AD
  \l 2A8AE
  \l 2A8AF
  \l 2A8B0
  \l 2A8B1
  \l 2A8B2
  \l 2A8B3
  \l 2A8B4
  \l 2A8B5
  \l 2A8B6
  \l 2A8B7
  \l 2A8B8
  \l 2A8B9
  \l 2A8BA
  \l 2A8BB
  \l 2A8BC
  \l 2A8BD
  \l 2A8BE
  \l 2A8BF
  \l 2A8C0
  \l 2A8C1
  \l 2A8C2
  \l 2A8C3
  \l 2A8C4
  \l 2A8C5
  \l 2A8C6
  \l 2A8C7
  \l 2A8C8
  \l 2A8C9
  \l 2A8CA
  \l 2A8CB
  \l 2A8CC
  \l 2A8CD
  \l 2A8CE
  \l 2A8CF
  \l 2A8D0
  \l 2A8D1
  \l 2A8D2
  \l 2A8D3
  \l 2A8D4
  \l 2A8D5
  \l 2A8D6
  \l 2A8D7
  \l 2A8D8
  \l 2A8D9
  \l 2A8DA
  \l 2A8DB
  \l 2A8DC
  \l 2A8DD
  \l 2A8DE
  \l 2A8DF
  \l 2A8E0
  \l 2A8E1
  \l 2A8E2
  \l 2A8E3
  \l 2A8E4
  \l 2A8E5
  \l 2A8E6
  \l 2A8E7
  \l 2A8E8
  \l 2A8E9
  \l 2A8EA
  \l 2A8EB
  \l 2A8EC
  \l 2A8ED
  \l 2A8EE
  \l 2A8EF
  \l 2A8F0
  \l 2A8F1
  \l 2A8F2
  \l 2A8F3
  \l 2A8F4
  \l 2A8F5
  \l 2A8F6
  \l 2A8F7
  \l 2A8F8
  \l 2A8F9
  \l 2A8FA
  \l 2A8FB
  \l 2A8FC
  \l 2A8FD
  \l 2A8FE
  \l 2A8FF
  \l 2A900
  \l 2A901
  \l 2A902
  \l 2A903
  \l 2A904
  \l 2A905
  \l 2A906
  \l 2A907
  \l 2A908
  \l 2A909
  \l 2A90A
  \l 2A90B
  \l 2A90C
  \l 2A90D
  \l 2A90E
  \l 2A90F
  \l 2A910
  \l 2A911
  \l 2A912
  \l 2A913
  \l 2A914
  \l 2A915
  \l 2A916
  \l 2A917
  \l 2A918
  \l 2A919
  \l 2A91A
  \l 2A91B
  \l 2A91C
  \l 2A91D
  \l 2A91E
  \l 2A91F
  \l 2A920
  \l 2A921
  \l 2A922
  \l 2A923
  \l 2A924
  \l 2A925
  \l 2A926
  \l 2A927
  \l 2A928
  \l 2A929
  \l 2A92A
  \l 2A92B
  \l 2A92C
  \l 2A92D
  \l 2A92E
  \l 2A92F
  \l 2A930
  \l 2A931
  \l 2A932
  \l 2A933
  \l 2A934
  \l 2A935
  \l 2A936
  \l 2A937
  \l 2A938
  \l 2A939
  \l 2A93A
  \l 2A93B
  \l 2A93C
  \l 2A93D
  \l 2A93E
  \l 2A93F
  \l 2A940
  \l 2A941
  \l 2A942
  \l 2A943
  \l 2A944
  \l 2A945
  \l 2A946
  \l 2A947
  \l 2A948
  \l 2A949
  \l 2A94A
  \l 2A94B
  \l 2A94C
  \l 2A94D
  \l 2A94E
  \l 2A94F
  \l 2A950
  \l 2A951
  \l 2A952
  \l 2A953
  \l 2A954
  \l 2A955
  \l 2A956
  \l 2A957
  \l 2A958
  \l 2A959
  \l 2A95A
  \l 2A95B
  \l 2A95C
  \l 2A95D
  \l 2A95E
  \l 2A95F
  \l 2A960
  \l 2A961
  \l 2A962
  \l 2A963
  \l 2A964
  \l 2A965
  \l 2A966
  \l 2A967
  \l 2A968
  \l 2A969
  \l 2A96A
  \l 2A96B
  \l 2A96C
  \l 2A96D
  \l 2A96E
  \l 2A96F
  \l 2A970
  \l 2A971
  \l 2A972
  \l 2A973
  \l 2A974
  \l 2A975
  \l 2A976
  \l 2A977
  \l 2A978
  \l 2A979
  \l 2A97A
  \l 2A97B
  \l 2A97C
  \l 2A97D
  \l 2A97E
  \l 2A97F
  \l 2A980
  \l 2A981
  \l 2A982
  \l 2A983
  \l 2A984
  \l 2A985
  \l 2A986
  \l 2A987
  \l 2A988
  \l 2A989
  \l 2A98A
  \l 2A98B
  \l 2A98C
  \l 2A98D
  \l 2A98E
  \l 2A98F
  \l 2A990
  \l 2A991
  \l 2A992
  \l 2A993
  \l 2A994
  \l 2A995
  \l 2A996
  \l 2A997
  \l 2A998
  \l 2A999
  \l 2A99A
  \l 2A99B
  \l 2A99C
  \l 2A99D
  \l 2A99E
  \l 2A99F
  \l 2A9A0
  \l 2A9A1
  \l 2A9A2
  \l 2A9A3
  \l 2A9A4
  \l 2A9A5
  \l 2A9A6
  \l 2A9A7
  \l 2A9A8
  \l 2A9A9
  \l 2A9AA
  \l 2A9AB
  \l 2A9AC
  \l 2A9AD
  \l 2A9AE
  \l 2A9AF
  \l 2A9B0
  \l 2A9B1
  \l 2A9B2
  \l 2A9B3
  \l 2A9B4
  \l 2A9B5
  \l 2A9B6
  \l 2A9B7
  \l 2A9B8
  \l 2A9B9
  \l 2A9BA
  \l 2A9BB
  \l 2A9BC
  \l 2A9BD
  \l 2A9BE
  \l 2A9BF
  \l 2A9C0
  \l 2A9C1
  \l 2A9C2
  \l 2A9C3
  \l 2A9C4
  \l 2A9C5
  \l 2A9C6
  \l 2A9C7
  \l 2A9C8
  \l 2A9C9
  \l 2A9CA
  \l 2A9CB
  \l 2A9CC
  \l 2A9CD
  \l 2A9CE
  \l 2A9CF
  \l 2A9D0
  \l 2A9D1
  \l 2A9D2
  \l 2A9D3
  \l 2A9D4
  \l 2A9D5
  \l 2A9D6
  \l 2A9D7
  \l 2A9D8
  \l 2A9D9
  \l 2A9DA
  \l 2A9DB
  \l 2A9DC
  \l 2A9DD
  \l 2A9DE
  \l 2A9DF
  \l 2A9E0
  \l 2A9E1
  \l 2A9E2
  \l 2A9E3
  \l 2A9E4
  \l 2A9E5
  \l 2A9E6
  \l 2A9E7
  \l 2A9E8
  \l 2A9E9
  \l 2A9EA
  \l 2A9EB
  \l 2A9EC
  \l 2A9ED
  \l 2A9EE
  \l 2A9EF
  \l 2A9F0
  \l 2A9F1
  \l 2A9F2
  \l 2A9F3
  \l 2A9F4
  \l 2A9F5
  \l 2A9F6
  \l 2A9F7
  \l 2A9F8
  \l 2A9F9
  \l 2A9FA
  \l 2A9FB
  \l 2A9FC
  \l 2A9FD
  \l 2A9FE
  \l 2A9FF
  \l 2AA00
  \l 2AA01
  \l 2AA02
  \l 2AA03
  \l 2AA04
  \l 2AA05
  \l 2AA06
  \l 2AA07
  \l 2AA08
  \l 2AA09
  \l 2AA0A
  \l 2AA0B
  \l 2AA0C
  \l 2AA0D
  \l 2AA0E
  \l 2AA0F
  \l 2AA10
  \l 2AA11
  \l 2AA12
  \l 2AA13
  \l 2AA14
  \l 2AA15
  \l 2AA16
  \l 2AA17
  \l 2AA18
  \l 2AA19
  \l 2AA1A
  \l 2AA1B
  \l 2AA1C
  \l 2AA1D
  \l 2AA1E
  \l 2AA1F
  \l 2AA20
  \l 2AA21
  \l 2AA22
  \l 2AA23
  \l 2AA24
  \l 2AA25
  \l 2AA26
  \l 2AA27
  \l 2AA28
  \l 2AA29
  \l 2AA2A
  \l 2AA2B
  \l 2AA2C
  \l 2AA2D
  \l 2AA2E
  \l 2AA2F
  \l 2AA30
  \l 2AA31
  \l 2AA32
  \l 2AA33
  \l 2AA34
  \l 2AA35
  \l 2AA36
  \l 2AA37
  \l 2AA38
  \l 2AA39
  \l 2AA3A
  \l 2AA3B
  \l 2AA3C
  \l 2AA3D
  \l 2AA3E
  \l 2AA3F
  \l 2AA40
  \l 2AA41
  \l 2AA42
  \l 2AA43
  \l 2AA44
  \l 2AA45
  \l 2AA46
  \l 2AA47
  \l 2AA48
  \l 2AA49
  \l 2AA4A
  \l 2AA4B
  \l 2AA4C
  \l 2AA4D
  \l 2AA4E
  \l 2AA4F
  \l 2AA50
  \l 2AA51
  \l 2AA52
  \l 2AA53
  \l 2AA54
  \l 2AA55
  \l 2AA56
  \l 2AA57
  \l 2AA58
  \l 2AA59
  \l 2AA5A
  \l 2AA5B
  \l 2AA5C
  \l 2AA5D
  \l 2AA5E
  \l 2AA5F
  \l 2AA60
  \l 2AA61
  \l 2AA62
  \l 2AA63
  \l 2AA64
  \l 2AA65
  \l 2AA66
  \l 2AA67
  \l 2AA68
  \l 2AA69
  \l 2AA6A
  \l 2AA6B
  \l 2AA6C
  \l 2AA6D
  \l 2AA6E
  \l 2AA6F
  \l 2AA70
  \l 2AA71
  \l 2AA72
  \l 2AA73
  \l 2AA74
  \l 2AA75
  \l 2AA76
  \l 2AA77
  \l 2AA78
  \l 2AA79
  \l 2AA7A
  \l 2AA7B
  \l 2AA7C
  \l 2AA7D
  \l 2AA7E
  \l 2AA7F
  \l 2AA80
  \l 2AA81
  \l 2AA82
  \l 2AA83
  \l 2AA84
  \l 2AA85
  \l 2AA86
  \l 2AA87
  \l 2AA88
  \l 2AA89
  \l 2AA8A
  \l 2AA8B
  \l 2AA8C
  \l 2AA8D
  \l 2AA8E
  \l 2AA8F
  \l 2AA90
  \l 2AA91
  \l 2AA92
  \l 2AA93
  \l 2AA94
  \l 2AA95
  \l 2AA96
  \l 2AA97
  \l 2AA98
  \l 2AA99
  \l 2AA9A
  \l 2AA9B
  \l 2AA9C
  \l 2AA9D
  \l 2AA9E
  \l 2AA9F
  \l 2AAA0
  \l 2AAA1
  \l 2AAA2
  \l 2AAA3
  \l 2AAA4
  \l 2AAA5
  \l 2AAA6
  \l 2AAA7
  \l 2AAA8
  \l 2AAA9
  \l 2AAAA
  \l 2AAAB
  \l 2AAAC
  \l 2AAAD
  \l 2AAAE
  \l 2AAAF
  \l 2AAB0
  \l 2AAB1
  \l 2AAB2
  \l 2AAB3
  \l 2AAB4
  \l 2AAB5
  \l 2AAB6
  \l 2AAB7
  \l 2AAB8
  \l 2AAB9
  \l 2AABA
  \l 2AABB
  \l 2AABC
  \l 2AABD
  \l 2AABE
  \l 2AABF
  \l 2AAC0
  \l 2AAC1
  \l 2AAC2
  \l 2AAC3
  \l 2AAC4
  \l 2AAC5
  \l 2AAC6
  \l 2AAC7
  \l 2AAC8
  \l 2AAC9
  \l 2AACA
  \l 2AACB
  \l 2AACC
  \l 2AACD
  \l 2AACE
  \l 2AACF
  \l 2AAD0
  \l 2AAD1
  \l 2AAD2
  \l 2AAD3
  \l 2AAD4
  \l 2AAD5
  \l 2AAD6
  \l 2AAD7
  \l 2AAD8
  \l 2AAD9
  \l 2AADA
  \l 2AADB
  \l 2AADC
  \l 2AADD
  \l 2AADE
  \l 2AADF
  \l 2AAE0
  \l 2AAE1
  \l 2AAE2
  \l 2AAE3
  \l 2AAE4
  \l 2AAE5
  \l 2AAE6
  \l 2AAE7
  \l 2AAE8
  \l 2AAE9
  \l 2AAEA
  \l 2AAEB
  \l 2AAEC
  \l 2AAED
  \l 2AAEE
  \l 2AAEF
  \l 2AAF0
  \l 2AAF1
  \l 2AAF2
  \l 2AAF3
  \l 2AAF4
  \l 2AAF5
  \l 2AAF6
  \l 2AAF7
  \l 2AAF8
  \l 2AAF9
  \l 2AAFA
  \l 2AAFB
  \l 2AAFC
  \l 2AAFD
  \l 2AAFE
  \l 2AAFF
  \l 2AB00
  \l 2AB01
  \l 2AB02
  \l 2AB03
  \l 2AB04
  \l 2AB05
  \l 2AB06
  \l 2AB07
  \l 2AB08
  \l 2AB09
  \l 2AB0A
  \l 2AB0B
  \l 2AB0C
  \l 2AB0D
  \l 2AB0E
  \l 2AB0F
  \l 2AB10
  \l 2AB11
  \l 2AB12
  \l 2AB13
  \l 2AB14
  \l 2AB15
  \l 2AB16
  \l 2AB17
  \l 2AB18
  \l 2AB19
  \l 2AB1A
  \l 2AB1B
  \l 2AB1C
  \l 2AB1D
  \l 2AB1E
  \l 2AB1F
  \l 2AB20
  \l 2AB21
  \l 2AB22
  \l 2AB23
  \l 2AB24
  \l 2AB25
  \l 2AB26
  \l 2AB27
  \l 2AB28
  \l 2AB29
  \l 2AB2A
  \l 2AB2B
  \l 2AB2C
  \l 2AB2D
  \l 2AB2E
  \l 2AB2F
  \l 2AB30
  \l 2AB31
  \l 2AB32
  \l 2AB33
  \l 2AB34
  \l 2AB35
  \l 2AB36
  \l 2AB37
  \l 2AB38
  \l 2AB39
  \l 2AB3A
  \l 2AB3B
  \l 2AB3C
  \l 2AB3D
  \l 2AB3E
  \l 2AB3F
  \l 2AB40
  \l 2AB41
  \l 2AB42
  \l 2AB43
  \l 2AB44
  \l 2AB45
  \l 2AB46
  \l 2AB47
  \l 2AB48
  \l 2AB49
  \l 2AB4A
  \l 2AB4B
  \l 2AB4C
  \l 2AB4D
  \l 2AB4E
  \l 2AB4F
  \l 2AB50
  \l 2AB51
  \l 2AB52
  \l 2AB53
  \l 2AB54
  \l 2AB55
  \l 2AB56
  \l 2AB57
  \l 2AB58
  \l 2AB59
  \l 2AB5A
  \l 2AB5B
  \l 2AB5C
  \l 2AB5D
  \l 2AB5E
  \l 2AB5F
  \l 2AB60
  \l 2AB61
  \l 2AB62
  \l 2AB63
  \l 2AB64
  \l 2AB65
  \l 2AB66
  \l 2AB67
  \l 2AB68
  \l 2AB69
  \l 2AB6A
  \l 2AB6B
  \l 2AB6C
  \l 2AB6D
  \l 2AB6E
  \l 2AB6F
  \l 2AB70
  \l 2AB71
  \l 2AB72
  \l 2AB73
  \l 2AB74
  \l 2AB75
  \l 2AB76
  \l 2AB77
  \l 2AB78
  \l 2AB79
  \l 2AB7A
  \l 2AB7B
  \l 2AB7C
  \l 2AB7D
  \l 2AB7E
  \l 2AB7F
  \l 2AB80
  \l 2AB81
  \l 2AB82
  \l 2AB83
  \l 2AB84
  \l 2AB85
  \l 2AB86
  \l 2AB87
  \l 2AB88
  \l 2AB89
  \l 2AB8A
  \l 2AB8B
  \l 2AB8C
  \l 2AB8D
  \l 2AB8E
  \l 2AB8F
  \l 2AB90
  \l 2AB91
  \l 2AB92
  \l 2AB93
  \l 2AB94
  \l 2AB95
  \l 2AB96
  \l 2AB97
  \l 2AB98
  \l 2AB99
  \l 2AB9A
  \l 2AB9B
  \l 2AB9C
  \l 2AB9D
  \l 2AB9E
  \l 2AB9F
  \l 2ABA0
  \l 2ABA1
  \l 2ABA2
  \l 2ABA3
  \l 2ABA4
  \l 2ABA5
  \l 2ABA6
  \l 2ABA7
  \l 2ABA8
  \l 2ABA9
  \l 2ABAA
  \l 2ABAB
  \l 2ABAC
  \l 2ABAD
  \l 2ABAE
  \l 2ABAF
  \l 2ABB0
  \l 2ABB1
  \l 2ABB2
  \l 2ABB3
  \l 2ABB4
  \l 2ABB5
  \l 2ABB6
  \l 2ABB7
  \l 2ABB8
  \l 2ABB9
  \l 2ABBA
  \l 2ABBB
  \l 2ABBC
  \l 2ABBD
  \l 2ABBE
  \l 2ABBF
  \l 2ABC0
  \l 2ABC1
  \l 2ABC2
  \l 2ABC3
  \l 2ABC4
  \l 2ABC5
  \l 2ABC6
  \l 2ABC7
  \l 2ABC8
  \l 2ABC9
  \l 2ABCA
  \l 2ABCB
  \l 2ABCC
  \l 2ABCD
  \l 2ABCE
  \l 2ABCF
  \l 2ABD0
  \l 2ABD1
  \l 2ABD2
  \l 2ABD3
  \l 2ABD4
  \l 2ABD5
  \l 2ABD6
  \l 2ABD7
  \l 2ABD8
  \l 2ABD9
  \l 2ABDA
  \l 2ABDB
  \l 2ABDC
  \l 2ABDD
  \l 2ABDE
  \l 2ABDF
  \l 2ABE0
  \l 2ABE1
  \l 2ABE2
  \l 2ABE3
  \l 2ABE4
  \l 2ABE5
  \l 2ABE6
  \l 2ABE7
  \l 2ABE8
  \l 2ABE9
  \l 2ABEA
  \l 2ABEB
  \l 2ABEC
  \l 2ABED
  \l 2ABEE
  \l 2ABEF
  \l 2ABF0
  \l 2ABF1
  \l 2ABF2
  \l 2ABF3
  \l 2ABF4
  \l 2ABF5
  \l 2ABF6
  \l 2ABF7
  \l 2ABF8
  \l 2ABF9
  \l 2ABFA
  \l 2ABFB
  \l 2ABFC
  \l 2ABFD
  \l 2ABFE
  \l 2ABFF
  \l 2AC00
  \l 2AC01
  \l 2AC02
  \l 2AC03
  \l 2AC04
  \l 2AC05
  \l 2AC06
  \l 2AC07
  \l 2AC08
  \l 2AC09
  \l 2AC0A
  \l 2AC0B
  \l 2AC0C
  \l 2AC0D
  \l 2AC0E
  \l 2AC0F
  \l 2AC10
  \l 2AC11
  \l 2AC12
  \l 2AC13
  \l 2AC14
  \l 2AC15
  \l 2AC16
  \l 2AC17
  \l 2AC18
  \l 2AC19
  \l 2AC1A
  \l 2AC1B
  \l 2AC1C
  \l 2AC1D
  \l 2AC1E
  \l 2AC1F
  \l 2AC20
  \l 2AC21
  \l 2AC22
  \l 2AC23
  \l 2AC24
  \l 2AC25
  \l 2AC26
  \l 2AC27
  \l 2AC28
  \l 2AC29
  \l 2AC2A
  \l 2AC2B
  \l 2AC2C
  \l 2AC2D
  \l 2AC2E
  \l 2AC2F
  \l 2AC30
  \l 2AC31
  \l 2AC32
  \l 2AC33
  \l 2AC34
  \l 2AC35
  \l 2AC36
  \l 2AC37
  \l 2AC38
  \l 2AC39
  \l 2AC3A
  \l 2AC3B
  \l 2AC3C
  \l 2AC3D
  \l 2AC3E
  \l 2AC3F
  \l 2AC40
  \l 2AC41
  \l 2AC42
  \l 2AC43
  \l 2AC44
  \l 2AC45
  \l 2AC46
  \l 2AC47
  \l 2AC48
  \l 2AC49
  \l 2AC4A
  \l 2AC4B
  \l 2AC4C
  \l 2AC4D
  \l 2AC4E
  \l 2AC4F
  \l 2AC50
  \l 2AC51
  \l 2AC52
  \l 2AC53
  \l 2AC54
  \l 2AC55
  \l 2AC56
  \l 2AC57
  \l 2AC58
  \l 2AC59
  \l 2AC5A
  \l 2AC5B
  \l 2AC5C
  \l 2AC5D
  \l 2AC5E
  \l 2AC5F
  \l 2AC60
  \l 2AC61
  \l 2AC62
  \l 2AC63
  \l 2AC64
  \l 2AC65
  \l 2AC66
  \l 2AC67
  \l 2AC68
  \l 2AC69
  \l 2AC6A
  \l 2AC6B
  \l 2AC6C
  \l 2AC6D
  \l 2AC6E
  \l 2AC6F
  \l 2AC70
  \l 2AC71
  \l 2AC72
  \l 2AC73
  \l 2AC74
  \l 2AC75
  \l 2AC76
  \l 2AC77
  \l 2AC78
  \l 2AC79
  \l 2AC7A
  \l 2AC7B
  \l 2AC7C
  \l 2AC7D
  \l 2AC7E
  \l 2AC7F
  \l 2AC80
  \l 2AC81
  \l 2AC82
  \l 2AC83
  \l 2AC84
  \l 2AC85
  \l 2AC86
  \l 2AC87
  \l 2AC88
  \l 2AC89
  \l 2AC8A
  \l 2AC8B
  \l 2AC8C
  \l 2AC8D
  \l 2AC8E
  \l 2AC8F
  \l 2AC90
  \l 2AC91
  \l 2AC92
  \l 2AC93
  \l 2AC94
  \l 2AC95
  \l 2AC96
  \l 2AC97
  \l 2AC98
  \l 2AC99
  \l 2AC9A
  \l 2AC9B
  \l 2AC9C
  \l 2AC9D
  \l 2AC9E
  \l 2AC9F
  \l 2ACA0
  \l 2ACA1
  \l 2ACA2
  \l 2ACA3
  \l 2ACA4
  \l 2ACA5
  \l 2ACA6
  \l 2ACA7
  \l 2ACA8
  \l 2ACA9
  \l 2ACAA
  \l 2ACAB
  \l 2ACAC
  \l 2ACAD
  \l 2ACAE
  \l 2ACAF
  \l 2ACB0
  \l 2ACB1
  \l 2ACB2
  \l 2ACB3
  \l 2ACB4
  \l 2ACB5
  \l 2ACB6
  \l 2ACB7
  \l 2ACB8
  \l 2ACB9
  \l 2ACBA
  \l 2ACBB
  \l 2ACBC
  \l 2ACBD
  \l 2ACBE
  \l 2ACBF
  \l 2ACC0
  \l 2ACC1
  \l 2ACC2
  \l 2ACC3
  \l 2ACC4
  \l 2ACC5
  \l 2ACC6
  \l 2ACC7
  \l 2ACC8
  \l 2ACC9
  \l 2ACCA
  \l 2ACCB
  \l 2ACCC
  \l 2ACCD
  \l 2ACCE
  \l 2ACCF
  \l 2ACD0
  \l 2ACD1
  \l 2ACD2
  \l 2ACD3
  \l 2ACD4
  \l 2ACD5
  \l 2ACD6
  \l 2ACD7
  \l 2ACD8
  \l 2ACD9
  \l 2ACDA
  \l 2ACDB
  \l 2ACDC
  \l 2ACDD
  \l 2ACDE
  \l 2ACDF
  \l 2ACE0
  \l 2ACE1
  \l 2ACE2
  \l 2ACE3
  \l 2ACE4
  \l 2ACE5
  \l 2ACE6
  \l 2ACE7
  \l 2ACE8
  \l 2ACE9
  \l 2ACEA
  \l 2ACEB
  \l 2ACEC
  \l 2ACED
  \l 2ACEE
  \l 2ACEF
  \l 2ACF0
  \l 2ACF1
  \l 2ACF2
  \l 2ACF3
  \l 2ACF4
  \l 2ACF5
  \l 2ACF6
  \l 2ACF7
  \l 2ACF8
  \l 2ACF9
  \l 2ACFA
  \l 2ACFB
  \l 2ACFC
  \l 2ACFD
  \l 2ACFE
  \l 2ACFF
  \l 2AD00
  \l 2AD01
  \l 2AD02
  \l 2AD03
  \l 2AD04
  \l 2AD05
  \l 2AD06
  \l 2AD07
  \l 2AD08
  \l 2AD09
  \l 2AD0A
  \l 2AD0B
  \l 2AD0C
  \l 2AD0D
  \l 2AD0E
  \l 2AD0F
  \l 2AD10
  \l 2AD11
  \l 2AD12
  \l 2AD13
  \l 2AD14
  \l 2AD15
  \l 2AD16
  \l 2AD17
  \l 2AD18
  \l 2AD19
  \l 2AD1A
  \l 2AD1B
  \l 2AD1C
  \l 2AD1D
  \l 2AD1E
  \l 2AD1F
  \l 2AD20
  \l 2AD21
  \l 2AD22
  \l 2AD23
  \l 2AD24
  \l 2AD25
  \l 2AD26
  \l 2AD27
  \l 2AD28
  \l 2AD29
  \l 2AD2A
  \l 2AD2B
  \l 2AD2C
  \l 2AD2D
  \l 2AD2E
  \l 2AD2F
  \l 2AD30
  \l 2AD31
  \l 2AD32
  \l 2AD33
  \l 2AD34
  \l 2AD35
  \l 2AD36
  \l 2AD37
  \l 2AD38
  \l 2AD39
  \l 2AD3A
  \l 2AD3B
  \l 2AD3C
  \l 2AD3D
  \l 2AD3E
  \l 2AD3F
  \l 2AD40
  \l 2AD41
  \l 2AD42
  \l 2AD43
  \l 2AD44
  \l 2AD45
  \l 2AD46
  \l 2AD47
  \l 2AD48
  \l 2AD49
  \l 2AD4A
  \l 2AD4B
  \l 2AD4C
  \l 2AD4D
  \l 2AD4E
  \l 2AD4F
  \l 2AD50
  \l 2AD51
  \l 2AD52
  \l 2AD53
  \l 2AD54
  \l 2AD55
  \l 2AD56
  \l 2AD57
  \l 2AD58
  \l 2AD59
  \l 2AD5A
  \l 2AD5B
  \l 2AD5C
  \l 2AD5D
  \l 2AD5E
  \l 2AD5F
  \l 2AD60
  \l 2AD61
  \l 2AD62
  \l 2AD63
  \l 2AD64
  \l 2AD65
  \l 2AD66
  \l 2AD67
  \l 2AD68
  \l 2AD69
  \l 2AD6A
  \l 2AD6B
  \l 2AD6C
  \l 2AD6D
  \l 2AD6E
  \l 2AD6F
  \l 2AD70
  \l 2AD71
  \l 2AD72
  \l 2AD73
  \l 2AD74
  \l 2AD75
  \l 2AD76
  \l 2AD77
  \l 2AD78
  \l 2AD79
  \l 2AD7A
  \l 2AD7B
  \l 2AD7C
  \l 2AD7D
  \l 2AD7E
  \l 2AD7F
  \l 2AD80
  \l 2AD81
  \l 2AD82
  \l 2AD83
  \l 2AD84
  \l 2AD85
  \l 2AD86
  \l 2AD87
  \l 2AD88
  \l 2AD89
  \l 2AD8A
  \l 2AD8B
  \l 2AD8C
  \l 2AD8D
  \l 2AD8E
  \l 2AD8F
  \l 2AD90
  \l 2AD91
  \l 2AD92
  \l 2AD93
  \l 2AD94
  \l 2AD95
  \l 2AD96
  \l 2AD97
  \l 2AD98
  \l 2AD99
  \l 2AD9A
  \l 2AD9B
  \l 2AD9C
  \l 2AD9D
  \l 2AD9E
  \l 2AD9F
  \l 2ADA0
  \l 2ADA1
  \l 2ADA2
  \l 2ADA3
  \l 2ADA4
  \l 2ADA5
  \l 2ADA6
  \l 2ADA7
  \l 2ADA8
  \l 2ADA9
  \l 2ADAA
  \l 2ADAB
  \l 2ADAC
  \l 2ADAD
  \l 2ADAE
  \l 2ADAF
  \l 2ADB0
  \l 2ADB1
  \l 2ADB2
  \l 2ADB3
  \l 2ADB4
  \l 2ADB5
  \l 2ADB6
  \l 2ADB7
  \l 2ADB8
  \l 2ADB9
  \l 2ADBA
  \l 2ADBB
  \l 2ADBC
  \l 2ADBD
  \l 2ADBE
  \l 2ADBF
  \l 2ADC0
  \l 2ADC1
  \l 2ADC2
  \l 2ADC3
  \l 2ADC4
  \l 2ADC5
  \l 2ADC6
  \l 2ADC7
  \l 2ADC8
  \l 2ADC9
  \l 2ADCA
  \l 2ADCB
  \l 2ADCC
  \l 2ADCD
  \l 2ADCE
  \l 2ADCF
  \l 2ADD0
  \l 2ADD1
  \l 2ADD2
  \l 2ADD3
  \l 2ADD4
  \l 2ADD5
  \l 2ADD6
  \l 2ADD7
  \l 2ADD8
  \l 2ADD9
  \l 2ADDA
  \l 2ADDB
  \l 2ADDC
  \l 2ADDD
  \l 2ADDE
  \l 2ADDF
  \l 2ADE0
  \l 2ADE1
  \l 2ADE2
  \l 2ADE3
  \l 2ADE4
  \l 2ADE5
  \l 2ADE6
  \l 2ADE7
  \l 2ADE8
  \l 2ADE9
  \l 2ADEA
  \l 2ADEB
  \l 2ADEC
  \l 2ADED
  \l 2ADEE
  \l 2ADEF
  \l 2ADF0
  \l 2ADF1
  \l 2ADF2
  \l 2ADF3
  \l 2ADF4
  \l 2ADF5
  \l 2ADF6
  \l 2ADF7
  \l 2ADF8
  \l 2ADF9
  \l 2ADFA
  \l 2ADFB
  \l 2ADFC
  \l 2ADFD
  \l 2ADFE
  \l 2ADFF
  \l 2AE00
  \l 2AE01
  \l 2AE02
  \l 2AE03
  \l 2AE04
  \l 2AE05
  \l 2AE06
  \l 2AE07
  \l 2AE08
  \l 2AE09
  \l 2AE0A
  \l 2AE0B
  \l 2AE0C
  \l 2AE0D
  \l 2AE0E
  \l 2AE0F
  \l 2AE10
  \l 2AE11
  \l 2AE12
  \l 2AE13
  \l 2AE14
  \l 2AE15
  \l 2AE16
  \l 2AE17
  \l 2AE18
  \l 2AE19
  \l 2AE1A
  \l 2AE1B
  \l 2AE1C
  \l 2AE1D
  \l 2AE1E
  \l 2AE1F
  \l 2AE20
  \l 2AE21
  \l 2AE22
  \l 2AE23
  \l 2AE24
  \l 2AE25
  \l 2AE26
  \l 2AE27
  \l 2AE28
  \l 2AE29
  \l 2AE2A
  \l 2AE2B
  \l 2AE2C
  \l 2AE2D
  \l 2AE2E
  \l 2AE2F
  \l 2AE30
  \l 2AE31
  \l 2AE32
  \l 2AE33
  \l 2AE34
  \l 2AE35
  \l 2AE36
  \l 2AE37
  \l 2AE38
  \l 2AE39
  \l 2AE3A
  \l 2AE3B
  \l 2AE3C
  \l 2AE3D
  \l 2AE3E
  \l 2AE3F
  \l 2AE40
  \l 2AE41
  \l 2AE42
  \l 2AE43
  \l 2AE44
  \l 2AE45
  \l 2AE46
  \l 2AE47
  \l 2AE48
  \l 2AE49
  \l 2AE4A
  \l 2AE4B
  \l 2AE4C
  \l 2AE4D
  \l 2AE4E
  \l 2AE4F
  \l 2AE50
  \l 2AE51
  \l 2AE52
  \l 2AE53
  \l 2AE54
  \l 2AE55
  \l 2AE56
  \l 2AE57
  \l 2AE58
  \l 2AE59
  \l 2AE5A
  \l 2AE5B
  \l 2AE5C
  \l 2AE5D
  \l 2AE5E
  \l 2AE5F
  \l 2AE60
  \l 2AE61
  \l 2AE62
  \l 2AE63
  \l 2AE64
  \l 2AE65
  \l 2AE66
  \l 2AE67
  \l 2AE68
  \l 2AE69
  \l 2AE6A
  \l 2AE6B
  \l 2AE6C
  \l 2AE6D
  \l 2AE6E
  \l 2AE6F
  \l 2AE70
  \l 2AE71
  \l 2AE72
  \l 2AE73
  \l 2AE74
  \l 2AE75
  \l 2AE76
  \l 2AE77
  \l 2AE78
  \l 2AE79
  \l 2AE7A
  \l 2AE7B
  \l 2AE7C
  \l 2AE7D
  \l 2AE7E
  \l 2AE7F
  \l 2AE80
  \l 2AE81
  \l 2AE82
  \l 2AE83
  \l 2AE84
  \l 2AE85
  \l 2AE86
  \l 2AE87
  \l 2AE88
  \l 2AE89
  \l 2AE8A
  \l 2AE8B
  \l 2AE8C
  \l 2AE8D
  \l 2AE8E
  \l 2AE8F
  \l 2AE90
  \l 2AE91
  \l 2AE92
  \l 2AE93
  \l 2AE94
  \l 2AE95
  \l 2AE96
  \l 2AE97
  \l 2AE98
  \l 2AE99
  \l 2AE9A
  \l 2AE9B
  \l 2AE9C
  \l 2AE9D
  \l 2AE9E
  \l 2AE9F
  \l 2AEA0
  \l 2AEA1
  \l 2AEA2
  \l 2AEA3
  \l 2AEA4
  \l 2AEA5
  \l 2AEA6
  \l 2AEA7
  \l 2AEA8
  \l 2AEA9
  \l 2AEAA
  \l 2AEAB
  \l 2AEAC
  \l 2AEAD
  \l 2AEAE
  \l 2AEAF
  \l 2AEB0
  \l 2AEB1
  \l 2AEB2
  \l 2AEB3
  \l 2AEB4
  \l 2AEB5
  \l 2AEB6
  \l 2AEB7
  \l 2AEB8
  \l 2AEB9
  \l 2AEBA
  \l 2AEBB
  \l 2AEBC
  \l 2AEBD
  \l 2AEBE
  \l 2AEBF
  \l 2AEC0
  \l 2AEC1
  \l 2AEC2
  \l 2AEC3
  \l 2AEC4
  \l 2AEC5
  \l 2AEC6
  \l 2AEC7
  \l 2AEC8
  \l 2AEC9
  \l 2AECA
  \l 2AECB
  \l 2AECC
  \l 2AECD
  \l 2AECE
  \l 2AECF
  \l 2AED0
  \l 2AED1
  \l 2AED2
  \l 2AED3
  \l 2AED4
  \l 2AED5
  \l 2AED6
  \l 2AED7
  \l 2AED8
  \l 2AED9
  \l 2AEDA
  \l 2AEDB
  \l 2AEDC
  \l 2AEDD
  \l 2AEDE
  \l 2AEDF
  \l 2AEE0
  \l 2AEE1
  \l 2AEE2
  \l 2AEE3
  \l 2AEE4
  \l 2AEE5
  \l 2AEE6
  \l 2AEE7
  \l 2AEE8
  \l 2AEE9
  \l 2AEEA
  \l 2AEEB
  \l 2AEEC
  \l 2AEED
  \l 2AEEE
  \l 2AEEF
  \l 2AEF0
  \l 2AEF1
  \l 2AEF2
  \l 2AEF3
  \l 2AEF4
  \l 2AEF5
  \l 2AEF6
  \l 2AEF7
  \l 2AEF8
  \l 2AEF9
  \l 2AEFA
  \l 2AEFB
  \l 2AEFC
  \l 2AEFD
  \l 2AEFE
  \l 2AEFF
  \l 2AF00
  \l 2AF01
  \l 2AF02
  \l 2AF03
  \l 2AF04
  \l 2AF05
  \l 2AF06
  \l 2AF07
  \l 2AF08
  \l 2AF09
  \l 2AF0A
  \l 2AF0B
  \l 2AF0C
  \l 2AF0D
  \l 2AF0E
  \l 2AF0F
  \l 2AF10
  \l 2AF11
  \l 2AF12
  \l 2AF13
  \l 2AF14
  \l 2AF15
  \l 2AF16
  \l 2AF17
  \l 2AF18
  \l 2AF19
  \l 2AF1A
  \l 2AF1B
  \l 2AF1C
  \l 2AF1D
  \l 2AF1E
  \l 2AF1F
  \l 2AF20
  \l 2AF21
  \l 2AF22
  \l 2AF23
  \l 2AF24
  \l 2AF25
  \l 2AF26
  \l 2AF27
  \l 2AF28
  \l 2AF29
  \l 2AF2A
  \l 2AF2B
  \l 2AF2C
  \l 2AF2D
  \l 2AF2E
  \l 2AF2F
  \l 2AF30
  \l 2AF31
  \l 2AF32
  \l 2AF33
  \l 2AF34
  \l 2AF35
  \l 2AF36
  \l 2AF37
  \l 2AF38
  \l 2AF39
  \l 2AF3A
  \l 2AF3B
  \l 2AF3C
  \l 2AF3D
  \l 2AF3E
  \l 2AF3F
  \l 2AF40
  \l 2AF41
  \l 2AF42
  \l 2AF43
  \l 2AF44
  \l 2AF45
  \l 2AF46
  \l 2AF47
  \l 2AF48
  \l 2AF49
  \l 2AF4A
  \l 2AF4B
  \l 2AF4C
  \l 2AF4D
  \l 2AF4E
  \l 2AF4F
  \l 2AF50
  \l 2AF51
  \l 2AF52
  \l 2AF53
  \l 2AF54
  \l 2AF55
  \l 2AF56
  \l 2AF57
  \l 2AF58
  \l 2AF59
  \l 2AF5A
  \l 2AF5B
  \l 2AF5C
  \l 2AF5D
  \l 2AF5E
  \l 2AF5F
  \l 2AF60
  \l 2AF61
  \l 2AF62
  \l 2AF63
  \l 2AF64
  \l 2AF65
  \l 2AF66
  \l 2AF67
  \l 2AF68
  \l 2AF69
  \l 2AF6A
  \l 2AF6B
  \l 2AF6C
  \l 2AF6D
  \l 2AF6E
  \l 2AF6F
  \l 2AF70
  \l 2AF71
  \l 2AF72
  \l 2AF73
  \l 2AF74
  \l 2AF75
  \l 2AF76
  \l 2AF77
  \l 2AF78
  \l 2AF79
  \l 2AF7A
  \l 2AF7B
  \l 2AF7C
  \l 2AF7D
  \l 2AF7E
  \l 2AF7F
  \l 2AF80
  \l 2AF81
  \l 2AF82
  \l 2AF83
  \l 2AF84
  \l 2AF85
  \l 2AF86
  \l 2AF87
  \l 2AF88
  \l 2AF89
  \l 2AF8A
  \l 2AF8B
  \l 2AF8C
  \l 2AF8D
  \l 2AF8E
  \l 2AF8F
  \l 2AF90
  \l 2AF91
  \l 2AF92
  \l 2AF93
  \l 2AF94
  \l 2AF95
  \l 2AF96
  \l 2AF97
  \l 2AF98
  \l 2AF99
  \l 2AF9A
  \l 2AF9B
  \l 2AF9C
  \l 2AF9D
  \l 2AF9E
  \l 2AF9F
  \l 2AFA0
  \l 2AFA1
  \l 2AFA2
  \l 2AFA3
  \l 2AFA4
  \l 2AFA5
  \l 2AFA6
  \l 2AFA7
  \l 2AFA8
  \l 2AFA9
  \l 2AFAA
  \l 2AFAB
  \l 2AFAC
  \l 2AFAD
  \l 2AFAE
  \l 2AFAF
  \l 2AFB0
  \l 2AFB1
  \l 2AFB2
  \l 2AFB3
  \l 2AFB4
  \l 2AFB5
  \l 2AFB6
  \l 2AFB7
  \l 2AFB8
  \l 2AFB9
  \l 2AFBA
  \l 2AFBB
  \l 2AFBC
  \l 2AFBD
  \l 2AFBE
  \l 2AFBF
  \l 2AFC0
  \l 2AFC1
  \l 2AFC2
  \l 2AFC3
  \l 2AFC4
  \l 2AFC5
  \l 2AFC6
  \l 2AFC7
  \l 2AFC8
  \l 2AFC9
  \l 2AFCA
  \l 2AFCB
  \l 2AFCC
  \l 2AFCD
  \l 2AFCE
  \l 2AFCF
  \l 2AFD0
  \l 2AFD1
  \l 2AFD2
  \l 2AFD3
  \l 2AFD4
  \l 2AFD5
  \l 2AFD6
  \l 2AFD7
  \l 2AFD8
  \l 2AFD9
  \l 2AFDA
  \l 2AFDB
  \l 2AFDC
  \l 2AFDD
  \l 2AFDE
  \l 2AFDF
  \l 2AFE0
  \l 2AFE1
  \l 2AFE2
  \l 2AFE3
  \l 2AFE4
  \l 2AFE5
  \l 2AFE6
  \l 2AFE7
  \l 2AFE8
  \l 2AFE9
  \l 2AFEA
  \l 2AFEB
  \l 2AFEC
  \l 2AFED
  \l 2AFEE
  \l 2AFEF
  \l 2AFF0
  \l 2AFF1
  \l 2AFF2
  \l 2AFF3
  \l 2AFF4
  \l 2AFF5
  \l 2AFF6
  \l 2AFF7
  \l 2AFF8
  \l 2AFF9
  \l 2AFFA
  \l 2AFFB
  \l 2AFFC
  \l 2AFFD
  \l 2AFFE
  \l 2AFFF
  \l 2B000
  \l 2B001
  \l 2B002
  \l 2B003
  \l 2B004
  \l 2B005
  \l 2B006
  \l 2B007
  \l 2B008
  \l 2B009
  \l 2B00A
  \l 2B00B
  \l 2B00C
  \l 2B00D
  \l 2B00E
  \l 2B00F
  \l 2B010
  \l 2B011
  \l 2B012
  \l 2B013
  \l 2B014
  \l 2B015
  \l 2B016
  \l 2B017
  \l 2B018
  \l 2B019
  \l 2B01A
  \l 2B01B
  \l 2B01C
  \l 2B01D
  \l 2B01E
  \l 2B01F
  \l 2B020
  \l 2B021
  \l 2B022
  \l 2B023
  \l 2B024
  \l 2B025
  \l 2B026
  \l 2B027
  \l 2B028
  \l 2B029
  \l 2B02A
  \l 2B02B
  \l 2B02C
  \l 2B02D
  \l 2B02E
  \l 2B02F
  \l 2B030
  \l 2B031
  \l 2B032
  \l 2B033
  \l 2B034
  \l 2B035
  \l 2B036
  \l 2B037
  \l 2B038
  \l 2B039
  \l 2B03A
  \l 2B03B
  \l 2B03C
  \l 2B03D
  \l 2B03E
  \l 2B03F
  \l 2B040
  \l 2B041
  \l 2B042
  \l 2B043
  \l 2B044
  \l 2B045
  \l 2B046
  \l 2B047
  \l 2B048
  \l 2B049
  \l 2B04A
  \l 2B04B
  \l 2B04C
  \l 2B04D
  \l 2B04E
  \l 2B04F
  \l 2B050
  \l 2B051
  \l 2B052
  \l 2B053
  \l 2B054
  \l 2B055
  \l 2B056
  \l 2B057
  \l 2B058
  \l 2B059
  \l 2B05A
  \l 2B05B
  \l 2B05C
  \l 2B05D
  \l 2B05E
  \l 2B05F
  \l 2B060
  \l 2B061
  \l 2B062
  \l 2B063
  \l 2B064
  \l 2B065
  \l 2B066
  \l 2B067
  \l 2B068
  \l 2B069
  \l 2B06A
  \l 2B06B
  \l 2B06C
  \l 2B06D
  \l 2B06E
  \l 2B06F
  \l 2B070
  \l 2B071
  \l 2B072
  \l 2B073
  \l 2B074
  \l 2B075
  \l 2B076
  \l 2B077
  \l 2B078
  \l 2B079
  \l 2B07A
  \l 2B07B
  \l 2B07C
  \l 2B07D
  \l 2B07E
  \l 2B07F
  \l 2B080
  \l 2B081
  \l 2B082
  \l 2B083
  \l 2B084
  \l 2B085
  \l 2B086
  \l 2B087
  \l 2B088
  \l 2B089
  \l 2B08A
  \l 2B08B
  \l 2B08C
  \l 2B08D
  \l 2B08E
  \l 2B08F
  \l 2B090
  \l 2B091
  \l 2B092
  \l 2B093
  \l 2B094
  \l 2B095
  \l 2B096
  \l 2B097
  \l 2B098
  \l 2B099
  \l 2B09A
  \l 2B09B
  \l 2B09C
  \l 2B09D
  \l 2B09E
  \l 2B09F
  \l 2B0A0
  \l 2B0A1
  \l 2B0A2
  \l 2B0A3
  \l 2B0A4
  \l 2B0A5
  \l 2B0A6
  \l 2B0A7
  \l 2B0A8
  \l 2B0A9
  \l 2B0AA
  \l 2B0AB
  \l 2B0AC
  \l 2B0AD
  \l 2B0AE
  \l 2B0AF
  \l 2B0B0
  \l 2B0B1
  \l 2B0B2
  \l 2B0B3
  \l 2B0B4
  \l 2B0B5
  \l 2B0B6
  \l 2B0B7
  \l 2B0B8
  \l 2B0B9
  \l 2B0BA
  \l 2B0BB
  \l 2B0BC
  \l 2B0BD
  \l 2B0BE
  \l 2B0BF
  \l 2B0C0
  \l 2B0C1
  \l 2B0C2
  \l 2B0C3
  \l 2B0C4
  \l 2B0C5
  \l 2B0C6
  \l 2B0C7
  \l 2B0C8
  \l 2B0C9
  \l 2B0CA
  \l 2B0CB
  \l 2B0CC
  \l 2B0CD
  \l 2B0CE
  \l 2B0CF
  \l 2B0D0
  \l 2B0D1
  \l 2B0D2
  \l 2B0D3
  \l 2B0D4
  \l 2B0D5
  \l 2B0D6
  \l 2B0D7
  \l 2B0D8
  \l 2B0D9
  \l 2B0DA
  \l 2B0DB
  \l 2B0DC
  \l 2B0DD
  \l 2B0DE
  \l 2B0DF
  \l 2B0E0
  \l 2B0E1
  \l 2B0E2
  \l 2B0E3
  \l 2B0E4
  \l 2B0E5
  \l 2B0E6
  \l 2B0E7
  \l 2B0E8
  \l 2B0E9
  \l 2B0EA
  \l 2B0EB
  \l 2B0EC
  \l 2B0ED
  \l 2B0EE
  \l 2B0EF
  \l 2B0F0
  \l 2B0F1
  \l 2B0F2
  \l 2B0F3
  \l 2B0F4
  \l 2B0F5
  \l 2B0F6
  \l 2B0F7
  \l 2B0F8
  \l 2B0F9
  \l 2B0FA
  \l 2B0FB
  \l 2B0FC
  \l 2B0FD
  \l 2B0FE
  \l 2B0FF
  \l 2B100
  \l 2B101
  \l 2B102
  \l 2B103
  \l 2B104
  \l 2B105
  \l 2B106
  \l 2B107
  \l 2B108
  \l 2B109
  \l 2B10A
  \l 2B10B
  \l 2B10C
  \l 2B10D
  \l 2B10E
  \l 2B10F
  \l 2B110
  \l 2B111
  \l 2B112
  \l 2B113
  \l 2B114
  \l 2B115
  \l 2B116
  \l 2B117
  \l 2B118
  \l 2B119
  \l 2B11A
  \l 2B11B
  \l 2B11C
  \l 2B11D
  \l 2B11E
  \l 2B11F
  \l 2B120
  \l 2B121
  \l 2B122
  \l 2B123
  \l 2B124
  \l 2B125
  \l 2B126
  \l 2B127
  \l 2B128
  \l 2B129
  \l 2B12A
  \l 2B12B
  \l 2B12C
  \l 2B12D
  \l 2B12E
  \l 2B12F
  \l 2B130
  \l 2B131
  \l 2B132
  \l 2B133
  \l 2B134
  \l 2B135
  \l 2B136
  \l 2B137
  \l 2B138
  \l 2B139
  \l 2B13A
  \l 2B13B
  \l 2B13C
  \l 2B13D
  \l 2B13E
  \l 2B13F
  \l 2B140
  \l 2B141
  \l 2B142
  \l 2B143
  \l 2B144
  \l 2B145
  \l 2B146
  \l 2B147
  \l 2B148
  \l 2B149
  \l 2B14A
  \l 2B14B
  \l 2B14C
  \l 2B14D
  \l 2B14E
  \l 2B14F
  \l 2B150
  \l 2B151
  \l 2B152
  \l 2B153
  \l 2B154
  \l 2B155
  \l 2B156
  \l 2B157
  \l 2B158
  \l 2B159
  \l 2B15A
  \l 2B15B
  \l 2B15C
  \l 2B15D
  \l 2B15E
  \l 2B15F
  \l 2B160
  \l 2B161
  \l 2B162
  \l 2B163
  \l 2B164
  \l 2B165
  \l 2B166
  \l 2B167
  \l 2B168
  \l 2B169
  \l 2B16A
  \l 2B16B
  \l 2B16C
  \l 2B16D
  \l 2B16E
  \l 2B16F
  \l 2B170
  \l 2B171
  \l 2B172
  \l 2B173
  \l 2B174
  \l 2B175
  \l 2B176
  \l 2B177
  \l 2B178
  \l 2B179
  \l 2B17A
  \l 2B17B
  \l 2B17C
  \l 2B17D
  \l 2B17E
  \l 2B17F
  \l 2B180
  \l 2B181
  \l 2B182
  \l 2B183
  \l 2B184
  \l 2B185
  \l 2B186
  \l 2B187
  \l 2B188
  \l 2B189
  \l 2B18A
  \l 2B18B
  \l 2B18C
  \l 2B18D
  \l 2B18E
  \l 2B18F
  \l 2B190
  \l 2B191
  \l 2B192
  \l 2B193
  \l 2B194
  \l 2B195
  \l 2B196
  \l 2B197
  \l 2B198
  \l 2B199
  \l 2B19A
  \l 2B19B
  \l 2B19C
  \l 2B19D
  \l 2B19E
  \l 2B19F
  \l 2B1A0
  \l 2B1A1
  \l 2B1A2
  \l 2B1A3
  \l 2B1A4
  \l 2B1A5
  \l 2B1A6
  \l 2B1A7
  \l 2B1A8
  \l 2B1A9
  \l 2B1AA
  \l 2B1AB
  \l 2B1AC
  \l 2B1AD
  \l 2B1AE
  \l 2B1AF
  \l 2B1B0
  \l 2B1B1
  \l 2B1B2
  \l 2B1B3
  \l 2B1B4
  \l 2B1B5
  \l 2B1B6
  \l 2B1B7
  \l 2B1B8
  \l 2B1B9
  \l 2B1BA
  \l 2B1BB
  \l 2B1BC
  \l 2B1BD
  \l 2B1BE
  \l 2B1BF
  \l 2B1C0
  \l 2B1C1
  \l 2B1C2
  \l 2B1C3
  \l 2B1C4
  \l 2B1C5
  \l 2B1C6
  \l 2B1C7
  \l 2B1C8
  \l 2B1C9
  \l 2B1CA
  \l 2B1CB
  \l 2B1CC
  \l 2B1CD
  \l 2B1CE
  \l 2B1CF
  \l 2B1D0
  \l 2B1D1
  \l 2B1D2
  \l 2B1D3
  \l 2B1D4
  \l 2B1D5
  \l 2B1D6
  \l 2B1D7
  \l 2B1D8
  \l 2B1D9
  \l 2B1DA
  \l 2B1DB
  \l 2B1DC
  \l 2B1DD
  \l 2B1DE
  \l 2B1DF
  \l 2B1E0
  \l 2B1E1
  \l 2B1E2
  \l 2B1E3
  \l 2B1E4
  \l 2B1E5
  \l 2B1E6
  \l 2B1E7
  \l 2B1E8
  \l 2B1E9
  \l 2B1EA
  \l 2B1EB
  \l 2B1EC
  \l 2B1ED
  \l 2B1EE
  \l 2B1EF
  \l 2B1F0
  \l 2B1F1
  \l 2B1F2
  \l 2B1F3
  \l 2B1F4
  \l 2B1F5
  \l 2B1F6
  \l 2B1F7
  \l 2B1F8
  \l 2B1F9
  \l 2B1FA
  \l 2B1FB
  \l 2B1FC
  \l 2B1FD
  \l 2B1FE
  \l 2B1FF
  \l 2B200
  \l 2B201
  \l 2B202
  \l 2B203
  \l 2B204
  \l 2B205
  \l 2B206
  \l 2B207
  \l 2B208
  \l 2B209
  \l 2B20A
  \l 2B20B
  \l 2B20C
  \l 2B20D
  \l 2B20E
  \l 2B20F
  \l 2B210
  \l 2B211
  \l 2B212
  \l 2B213
  \l 2B214
  \l 2B215
  \l 2B216
  \l 2B217
  \l 2B218
  \l 2B219
  \l 2B21A
  \l 2B21B
  \l 2B21C
  \l 2B21D
  \l 2B21E
  \l 2B21F
  \l 2B220
  \l 2B221
  \l 2B222
  \l 2B223
  \l 2B224
  \l 2B225
  \l 2B226
  \l 2B227
  \l 2B228
  \l 2B229
  \l 2B22A
  \l 2B22B
  \l 2B22C
  \l 2B22D
  \l 2B22E
  \l 2B22F
  \l 2B230
  \l 2B231
  \l 2B232
  \l 2B233
  \l 2B234
  \l 2B235
  \l 2B236
  \l 2B237
  \l 2B238
  \l 2B239
  \l 2B23A
  \l 2B23B
  \l 2B23C
  \l 2B23D
  \l 2B23E
  \l 2B23F
  \l 2B240
  \l 2B241
  \l 2B242
  \l 2B243
  \l 2B244
  \l 2B245
  \l 2B246
  \l 2B247
  \l 2B248
  \l 2B249
  \l 2B24A
  \l 2B24B
  \l 2B24C
  \l 2B24D
  \l 2B24E
  \l 2B24F
  \l 2B250
  \l 2B251
  \l 2B252
  \l 2B253
  \l 2B254
  \l 2B255
  \l 2B256
  \l 2B257
  \l 2B258
  \l 2B259
  \l 2B25A
  \l 2B25B
  \l 2B25C
  \l 2B25D
  \l 2B25E
  \l 2B25F
  \l 2B260
  \l 2B261
  \l 2B262
  \l 2B263
  \l 2B264
  \l 2B265
  \l 2B266
  \l 2B267
  \l 2B268
  \l 2B269
  \l 2B26A
  \l 2B26B
  \l 2B26C
  \l 2B26D
  \l 2B26E
  \l 2B26F
  \l 2B270
  \l 2B271
  \l 2B272
  \l 2B273
  \l 2B274
  \l 2B275
  \l 2B276
  \l 2B277
  \l 2B278
  \l 2B279
  \l 2B27A
  \l 2B27B
  \l 2B27C
  \l 2B27D
  \l 2B27E
  \l 2B27F
  \l 2B280
  \l 2B281
  \l 2B282
  \l 2B283
  \l 2B284
  \l 2B285
  \l 2B286
  \l 2B287
  \l 2B288
  \l 2B289
  \l 2B28A
  \l 2B28B
  \l 2B28C
  \l 2B28D
  \l 2B28E
  \l 2B28F
  \l 2B290
  \l 2B291
  \l 2B292
  \l 2B293
  \l 2B294
  \l 2B295
  \l 2B296
  \l 2B297
  \l 2B298
  \l 2B299
  \l 2B29A
  \l 2B29B
  \l 2B29C
  \l 2B29D
  \l 2B29E
  \l 2B29F
  \l 2B2A0
  \l 2B2A1
  \l 2B2A2
  \l 2B2A3
  \l 2B2A4
  \l 2B2A5
  \l 2B2A6
  \l 2B2A7
  \l 2B2A8
  \l 2B2A9
  \l 2B2AA
  \l 2B2AB
  \l 2B2AC
  \l 2B2AD
  \l 2B2AE
  \l 2B2AF
  \l 2B2B0
  \l 2B2B1
  \l 2B2B2
  \l 2B2B3
  \l 2B2B4
  \l 2B2B5
  \l 2B2B6
  \l 2B2B7
  \l 2B2B8
  \l 2B2B9
  \l 2B2BA
  \l 2B2BB
  \l 2B2BC
  \l 2B2BD
  \l 2B2BE
  \l 2B2BF
  \l 2B2C0
  \l 2B2C1
  \l 2B2C2
  \l 2B2C3
  \l 2B2C4
  \l 2B2C5
  \l 2B2C6
  \l 2B2C7
  \l 2B2C8
  \l 2B2C9
  \l 2B2CA
  \l 2B2CB
  \l 2B2CC
  \l 2B2CD
  \l 2B2CE
  \l 2B2CF
  \l 2B2D0
  \l 2B2D1
  \l 2B2D2
  \l 2B2D3
  \l 2B2D4
  \l 2B2D5
  \l 2B2D6
  \l 2B2D7
  \l 2B2D8
  \l 2B2D9
  \l 2B2DA
  \l 2B2DB
  \l 2B2DC
  \l 2B2DD
  \l 2B2DE
  \l 2B2DF
  \l 2B2E0
  \l 2B2E1
  \l 2B2E2
  \l 2B2E3
  \l 2B2E4
  \l 2B2E5
  \l 2B2E6
  \l 2B2E7
  \l 2B2E8
  \l 2B2E9
  \l 2B2EA
  \l 2B2EB
  \l 2B2EC
  \l 2B2ED
  \l 2B2EE
  \l 2B2EF
  \l 2B2F0
  \l 2B2F1
  \l 2B2F2
  \l 2B2F3
  \l 2B2F4
  \l 2B2F5
  \l 2B2F6
  \l 2B2F7
  \l 2B2F8
  \l 2B2F9
  \l 2B2FA
  \l 2B2FB
  \l 2B2FC
  \l 2B2FD
  \l 2B2FE
  \l 2B2FF
  \l 2B300
  \l 2B301
  \l 2B302
  \l 2B303
  \l 2B304
  \l 2B305
  \l 2B306
  \l 2B307
  \l 2B308
  \l 2B309
  \l 2B30A
  \l 2B30B
  \l 2B30C
  \l 2B30D
  \l 2B30E
  \l 2B30F
  \l 2B310
  \l 2B311
  \l 2B312
  \l 2B313
  \l 2B314
  \l 2B315
  \l 2B316
  \l 2B317
  \l 2B318
  \l 2B319
  \l 2B31A
  \l 2B31B
  \l 2B31C
  \l 2B31D
  \l 2B31E
  \l 2B31F
  \l 2B320
  \l 2B321
  \l 2B322
  \l 2B323
  \l 2B324
  \l 2B325
  \l 2B326
  \l 2B327
  \l 2B328
  \l 2B329
  \l 2B32A
  \l 2B32B
  \l 2B32C
  \l 2B32D
  \l 2B32E
  \l 2B32F
  \l 2B330
  \l 2B331
  \l 2B332
  \l 2B333
  \l 2B334
  \l 2B335
  \l 2B336
  \l 2B337
  \l 2B338
  \l 2B339
  \l 2B33A
  \l 2B33B
  \l 2B33C
  \l 2B33D
  \l 2B33E
  \l 2B33F
  \l 2B340
  \l 2B341
  \l 2B342
  \l 2B343
  \l 2B344
  \l 2B345
  \l 2B346
  \l 2B347
  \l 2B348
  \l 2B349
  \l 2B34A
  \l 2B34B
  \l 2B34C
  \l 2B34D
  \l 2B34E
  \l 2B34F
  \l 2B350
  \l 2B351
  \l 2B352
  \l 2B353
  \l 2B354
  \l 2B355
  \l 2B356
  \l 2B357
  \l 2B358
  \l 2B359
  \l 2B35A
  \l 2B35B
  \l 2B35C
  \l 2B35D
  \l 2B35E
  \l 2B35F
  \l 2B360
  \l 2B361
  \l 2B362
  \l 2B363
  \l 2B364
  \l 2B365
  \l 2B366
  \l 2B367
  \l 2B368
  \l 2B369
  \l 2B36A
  \l 2B36B
  \l 2B36C
  \l 2B36D
  \l 2B36E
  \l 2B36F
  \l 2B370
  \l 2B371
  \l 2B372
  \l 2B373
  \l 2B374
  \l 2B375
  \l 2B376
  \l 2B377
  \l 2B378
  \l 2B379
  \l 2B37A
  \l 2B37B
  \l 2B37C
  \l 2B37D
  \l 2B37E
  \l 2B37F
  \l 2B380
  \l 2B381
  \l 2B382
  \l 2B383
  \l 2B384
  \l 2B385
  \l 2B386
  \l 2B387
  \l 2B388
  \l 2B389
  \l 2B38A
  \l 2B38B
  \l 2B38C
  \l 2B38D
  \l 2B38E
  \l 2B38F
  \l 2B390
  \l 2B391
  \l 2B392
  \l 2B393
  \l 2B394
  \l 2B395
  \l 2B396
  \l 2B397
  \l 2B398
  \l 2B399
  \l 2B39A
  \l 2B39B
  \l 2B39C
  \l 2B39D
  \l 2B39E
  \l 2B39F
  \l 2B3A0
  \l 2B3A1
  \l 2B3A2
  \l 2B3A3
  \l 2B3A4
  \l 2B3A5
  \l 2B3A6
  \l 2B3A7
  \l 2B3A8
  \l 2B3A9
  \l 2B3AA
  \l 2B3AB
  \l 2B3AC
  \l 2B3AD
  \l 2B3AE
  \l 2B3AF
  \l 2B3B0
  \l 2B3B1
  \l 2B3B2
  \l 2B3B3
  \l 2B3B4
  \l 2B3B5
  \l 2B3B6
  \l 2B3B7
  \l 2B3B8
  \l 2B3B9
  \l 2B3BA
  \l 2B3BB
  \l 2B3BC
  \l 2B3BD
  \l 2B3BE
  \l 2B3BF
  \l 2B3C0
  \l 2B3C1
  \l 2B3C2
  \l 2B3C3
  \l 2B3C4
  \l 2B3C5
  \l 2B3C6
  \l 2B3C7
  \l 2B3C8
  \l 2B3C9
  \l 2B3CA
  \l 2B3CB
  \l 2B3CC
  \l 2B3CD
  \l 2B3CE
  \l 2B3CF
  \l 2B3D0
  \l 2B3D1
  \l 2B3D2
  \l 2B3D3
  \l 2B3D4
  \l 2B3D5
  \l 2B3D6
  \l 2B3D7
  \l 2B3D8
  \l 2B3D9
  \l 2B3DA
  \l 2B3DB
  \l 2B3DC
  \l 2B3DD
  \l 2B3DE
  \l 2B3DF
  \l 2B3E0
  \l 2B3E1
  \l 2B3E2
  \l 2B3E3
  \l 2B3E4
  \l 2B3E5
  \l 2B3E6
  \l 2B3E7
  \l 2B3E8
  \l 2B3E9
  \l 2B3EA
  \l 2B3EB
  \l 2B3EC
  \l 2B3ED
  \l 2B3EE
  \l 2B3EF
  \l 2B3F0
  \l 2B3F1
  \l 2B3F2
  \l 2B3F3
  \l 2B3F4
  \l 2B3F5
  \l 2B3F6
  \l 2B3F7
  \l 2B3F8
  \l 2B3F9
  \l 2B3FA
  \l 2B3FB
  \l 2B3FC
  \l 2B3FD
  \l 2B3FE
  \l 2B3FF
  \l 2B400
  \l 2B401
  \l 2B402
  \l 2B403
  \l 2B404
  \l 2B405
  \l 2B406
  \l 2B407
  \l 2B408
  \l 2B409
  \l 2B40A
  \l 2B40B
  \l 2B40C
  \l 2B40D
  \l 2B40E
  \l 2B40F
  \l 2B410
  \l 2B411
  \l 2B412
  \l 2B413
  \l 2B414
  \l 2B415
  \l 2B416
  \l 2B417
  \l 2B418
  \l 2B419
  \l 2B41A
  \l 2B41B
  \l 2B41C
  \l 2B41D
  \l 2B41E
  \l 2B41F
  \l 2B420
  \l 2B421
  \l 2B422
  \l 2B423
  \l 2B424
  \l 2B425
  \l 2B426
  \l 2B427
  \l 2B428
  \l 2B429
  \l 2B42A
  \l 2B42B
  \l 2B42C
  \l 2B42D
  \l 2B42E
  \l 2B42F
  \l 2B430
  \l 2B431
  \l 2B432
  \l 2B433
  \l 2B434
  \l 2B435
  \l 2B436
  \l 2B437
  \l 2B438
  \l 2B439
  \l 2B43A
  \l 2B43B
  \l 2B43C
  \l 2B43D
  \l 2B43E
  \l 2B43F
  \l 2B440
  \l 2B441
  \l 2B442
  \l 2B443
  \l 2B444
  \l 2B445
  \l 2B446
  \l 2B447
  \l 2B448
  \l 2B449
  \l 2B44A
  \l 2B44B
  \l 2B44C
  \l 2B44D
  \l 2B44E
  \l 2B44F
  \l 2B450
  \l 2B451
  \l 2B452
  \l 2B453
  \l 2B454
  \l 2B455
  \l 2B456
  \l 2B457
  \l 2B458
  \l 2B459
  \l 2B45A
  \l 2B45B
  \l 2B45C
  \l 2B45D
  \l 2B45E
  \l 2B45F
  \l 2B460
  \l 2B461
  \l 2B462
  \l 2B463
  \l 2B464
  \l 2B465
  \l 2B466
  \l 2B467
  \l 2B468
  \l 2B469
  \l 2B46A
  \l 2B46B
  \l 2B46C
  \l 2B46D
  \l 2B46E
  \l 2B46F
  \l 2B470
  \l 2B471
  \l 2B472
  \l 2B473
  \l 2B474
  \l 2B475
  \l 2B476
  \l 2B477
  \l 2B478
  \l 2B479
  \l 2B47A
  \l 2B47B
  \l 2B47C
  \l 2B47D
  \l 2B47E
  \l 2B47F
  \l 2B480
  \l 2B481
  \l 2B482
  \l 2B483
  \l 2B484
  \l 2B485
  \l 2B486
  \l 2B487
  \l 2B488
  \l 2B489
  \l 2B48A
  \l 2B48B
  \l 2B48C
  \l 2B48D
  \l 2B48E
  \l 2B48F
  \l 2B490
  \l 2B491
  \l 2B492
  \l 2B493
  \l 2B494
  \l 2B495
  \l 2B496
  \l 2B497
  \l 2B498
  \l 2B499
  \l 2B49A
  \l 2B49B
  \l 2B49C
  \l 2B49D
  \l 2B49E
  \l 2B49F
  \l 2B4A0
  \l 2B4A1
  \l 2B4A2
  \l 2B4A3
  \l 2B4A4
  \l 2B4A5
  \l 2B4A6
  \l 2B4A7
  \l 2B4A8
  \l 2B4A9
  \l 2B4AA
  \l 2B4AB
  \l 2B4AC
  \l 2B4AD
  \l 2B4AE
  \l 2B4AF
  \l 2B4B0
  \l 2B4B1
  \l 2B4B2
  \l 2B4B3
  \l 2B4B4
  \l 2B4B5
  \l 2B4B6
  \l 2B4B7
  \l 2B4B8
  \l 2B4B9
  \l 2B4BA
  \l 2B4BB
  \l 2B4BC
  \l 2B4BD
  \l 2B4BE
  \l 2B4BF
  \l 2B4C0
  \l 2B4C1
  \l 2B4C2
  \l 2B4C3
  \l 2B4C4
  \l 2B4C5
  \l 2B4C6
  \l 2B4C7
  \l 2B4C8
  \l 2B4C9
  \l 2B4CA
  \l 2B4CB
  \l 2B4CC
  \l 2B4CD
  \l 2B4CE
  \l 2B4CF
  \l 2B4D0
  \l 2B4D1
  \l 2B4D2
  \l 2B4D3
  \l 2B4D4
  \l 2B4D5
  \l 2B4D6
  \l 2B4D7
  \l 2B4D8
  \l 2B4D9
  \l 2B4DA
  \l 2B4DB
  \l 2B4DC
  \l 2B4DD
  \l 2B4DE
  \l 2B4DF
  \l 2B4E0
  \l 2B4E1
  \l 2B4E2
  \l 2B4E3
  \l 2B4E4
  \l 2B4E5
  \l 2B4E6
  \l 2B4E7
  \l 2B4E8
  \l 2B4E9
  \l 2B4EA
  \l 2B4EB
  \l 2B4EC
  \l 2B4ED
  \l 2B4EE
  \l 2B4EF
  \l 2B4F0
  \l 2B4F1
  \l 2B4F2
  \l 2B4F3
  \l 2B4F4
  \l 2B4F5
  \l 2B4F6
  \l 2B4F7
  \l 2B4F8
  \l 2B4F9
  \l 2B4FA
  \l 2B4FB
  \l 2B4FC
  \l 2B4FD
  \l 2B4FE
  \l 2B4FF
  \l 2B500
  \l 2B501
  \l 2B502
  \l 2B503
  \l 2B504
  \l 2B505
  \l 2B506
  \l 2B507
  \l 2B508
  \l 2B509
  \l 2B50A
  \l 2B50B
  \l 2B50C
  \l 2B50D
  \l 2B50E
  \l 2B50F
  \l 2B510
  \l 2B511
  \l 2B512
  \l 2B513
  \l 2B514
  \l 2B515
  \l 2B516
  \l 2B517
  \l 2B518
  \l 2B519
  \l 2B51A
  \l 2B51B
  \l 2B51C
  \l 2B51D
  \l 2B51E
  \l 2B51F
  \l 2B520
  \l 2B521
  \l 2B522
  \l 2B523
  \l 2B524
  \l 2B525
  \l 2B526
  \l 2B527
  \l 2B528
  \l 2B529
  \l 2B52A
  \l 2B52B
  \l 2B52C
  \l 2B52D
  \l 2B52E
  \l 2B52F
  \l 2B530
  \l 2B531
  \l 2B532
  \l 2B533
  \l 2B534
  \l 2B535
  \l 2B536
  \l 2B537
  \l 2B538
  \l 2B539
  \l 2B53A
  \l 2B53B
  \l 2B53C
  \l 2B53D
  \l 2B53E
  \l 2B53F
  \l 2B540
  \l 2B541
  \l 2B542
  \l 2B543
  \l 2B544
  \l 2B545
  \l 2B546
  \l 2B547
  \l 2B548
  \l 2B549
  \l 2B54A
  \l 2B54B
  \l 2B54C
  \l 2B54D
  \l 2B54E
  \l 2B54F
  \l 2B550
  \l 2B551
  \l 2B552
  \l 2B553
  \l 2B554
  \l 2B555
  \l 2B556
  \l 2B557
  \l 2B558
  \l 2B559
  \l 2B55A
  \l 2B55B
  \l 2B55C
  \l 2B55D
  \l 2B55E
  \l 2B55F
  \l 2B560
  \l 2B561
  \l 2B562
  \l 2B563
  \l 2B564
  \l 2B565
  \l 2B566
  \l 2B567
  \l 2B568
  \l 2B569
  \l 2B56A
  \l 2B56B
  \l 2B56C
  \l 2B56D
  \l 2B56E
  \l 2B56F
  \l 2B570
  \l 2B571
  \l 2B572
  \l 2B573
  \l 2B574
  \l 2B575
  \l 2B576
  \l 2B577
  \l 2B578
  \l 2B579
  \l 2B57A
  \l 2B57B
  \l 2B57C
  \l 2B57D
  \l 2B57E
  \l 2B57F
  \l 2B580
  \l 2B581
  \l 2B582
  \l 2B583
  \l 2B584
  \l 2B585
  \l 2B586
  \l 2B587
  \l 2B588
  \l 2B589
  \l 2B58A
  \l 2B58B
  \l 2B58C
  \l 2B58D
  \l 2B58E
  \l 2B58F
  \l 2B590
  \l 2B591
  \l 2B592
  \l 2B593
  \l 2B594
  \l 2B595
  \l 2B596
  \l 2B597
  \l 2B598
  \l 2B599
  \l 2B59A
  \l 2B59B
  \l 2B59C
  \l 2B59D
  \l 2B59E
  \l 2B59F
  \l 2B5A0
  \l 2B5A1
  \l 2B5A2
  \l 2B5A3
  \l 2B5A4
  \l 2B5A5
  \l 2B5A6
  \l 2B5A7
  \l 2B5A8
  \l 2B5A9
  \l 2B5AA
  \l 2B5AB
  \l 2B5AC
  \l 2B5AD
  \l 2B5AE
  \l 2B5AF
  \l 2B5B0
  \l 2B5B1
  \l 2B5B2
  \l 2B5B3
  \l 2B5B4
  \l 2B5B5
  \l 2B5B6
  \l 2B5B7
  \l 2B5B8
  \l 2B5B9
  \l 2B5BA
  \l 2B5BB
  \l 2B5BC
  \l 2B5BD
  \l 2B5BE
  \l 2B5BF
  \l 2B5C0
  \l 2B5C1
  \l 2B5C2
  \l 2B5C3
  \l 2B5C4
  \l 2B5C5
  \l 2B5C6
  \l 2B5C7
  \l 2B5C8
  \l 2B5C9
  \l 2B5CA
  \l 2B5CB
  \l 2B5CC
  \l 2B5CD
  \l 2B5CE
  \l 2B5CF
  \l 2B5D0
  \l 2B5D1
  \l 2B5D2
  \l 2B5D3
  \l 2B5D4
  \l 2B5D5
  \l 2B5D6
  \l 2B5D7
  \l 2B5D8
  \l 2B5D9
  \l 2B5DA
  \l 2B5DB
  \l 2B5DC
  \l 2B5DD
  \l 2B5DE
  \l 2B5DF
  \l 2B5E0
  \l 2B5E1
  \l 2B5E2
  \l 2B5E3
  \l 2B5E4
  \l 2B5E5
  \l 2B5E6
  \l 2B5E7
  \l 2B5E8
  \l 2B5E9
  \l 2B5EA
  \l 2B5EB
  \l 2B5EC
  \l 2B5ED
  \l 2B5EE
  \l 2B5EF
  \l 2B5F0
  \l 2B5F1
  \l 2B5F2
  \l 2B5F3
  \l 2B5F4
  \l 2B5F5
  \l 2B5F6
  \l 2B5F7
  \l 2B5F8
  \l 2B5F9
  \l 2B5FA
  \l 2B5FB
  \l 2B5FC
  \l 2B5FD
  \l 2B5FE
  \l 2B5FF
  \l 2B600
  \l 2B601
  \l 2B602
  \l 2B603
  \l 2B604
  \l 2B605
  \l 2B606
  \l 2B607
  \l 2B608
  \l 2B609
  \l 2B60A
  \l 2B60B
  \l 2B60C
  \l 2B60D
  \l 2B60E
  \l 2B60F
  \l 2B610
  \l 2B611
  \l 2B612
  \l 2B613
  \l 2B614
  \l 2B615
  \l 2B616
  \l 2B617
  \l 2B618
  \l 2B619
  \l 2B61A
  \l 2B61B
  \l 2B61C
  \l 2B61D
  \l 2B61E
  \l 2B61F
  \l 2B620
  \l 2B621
  \l 2B622
  \l 2B623
  \l 2B624
  \l 2B625
  \l 2B626
  \l 2B627
  \l 2B628
  \l 2B629
  \l 2B62A
  \l 2B62B
  \l 2B62C
  \l 2B62D
  \l 2B62E
  \l 2B62F
  \l 2B630
  \l 2B631
  \l 2B632
  \l 2B633
  \l 2B634
  \l 2B635
  \l 2B636
  \l 2B637
  \l 2B638
  \l 2B639
  \l 2B63A
  \l 2B63B
  \l 2B63C
  \l 2B63D
  \l 2B63E
  \l 2B63F
  \l 2B640
  \l 2B641
  \l 2B642
  \l 2B643
  \l 2B644
  \l 2B645
  \l 2B646
  \l 2B647
  \l 2B648
  \l 2B649
  \l 2B64A
  \l 2B64B
  \l 2B64C
  \l 2B64D
  \l 2B64E
  \l 2B64F
  \l 2B650
  \l 2B651
  \l 2B652
  \l 2B653
  \l 2B654
  \l 2B655
  \l 2B656
  \l 2B657
  \l 2B658
  \l 2B659
  \l 2B65A
  \l 2B65B
  \l 2B65C
  \l 2B65D
  \l 2B65E
  \l 2B65F
  \l 2B660
  \l 2B661
  \l 2B662
  \l 2B663
  \l 2B664
  \l 2B665
  \l 2B666
  \l 2B667
  \l 2B668
  \l 2B669
  \l 2B66A
  \l 2B66B
  \l 2B66C
  \l 2B66D
  \l 2B66E
  \l 2B66F
  \l 2B670
  \l 2B671
  \l 2B672
  \l 2B673
  \l 2B674
  \l 2B675
  \l 2B676
  \l 2B677
  \l 2B678
  \l 2B679
  \l 2B67A
  \l 2B67B
  \l 2B67C
  \l 2B67D
  \l 2B67E
  \l 2B67F
  \l 2B680
  \l 2B681
  \l 2B682
  \l 2B683
  \l 2B684
  \l 2B685
  \l 2B686
  \l 2B687
  \l 2B688
  \l 2B689
  \l 2B68A
  \l 2B68B
  \l 2B68C
  \l 2B68D
  \l 2B68E
  \l 2B68F
  \l 2B690
  \l 2B691
  \l 2B692
  \l 2B693
  \l 2B694
  \l 2B695
  \l 2B696
  \l 2B697
  \l 2B698
  \l 2B699
  \l 2B69A
  \l 2B69B
  \l 2B69C
  \l 2B69D
  \l 2B69E
  \l 2B69F
  \l 2B6A0
  \l 2B6A1
  \l 2B6A2
  \l 2B6A3
  \l 2B6A4
  \l 2B6A5
  \l 2B6A6
  \l 2B6A7
  \l 2B6A8
  \l 2B6A9
  \l 2B6AA
  \l 2B6AB
  \l 2B6AC
  \l 2B6AD
  \l 2B6AE
  \l 2B6AF
  \l 2B6B0
  \l 2B6B1
  \l 2B6B2
  \l 2B6B3
  \l 2B6B4
  \l 2B6B5
  \l 2B6B6
  \l 2B6B7
  \l 2B6B8
  \l 2B6B9
  \l 2B6BA
  \l 2B6BB
  \l 2B6BC
  \l 2B6BD
  \l 2B6BE
  \l 2B6BF
  \l 2B6C0
  \l 2B6C1
  \l 2B6C2
  \l 2B6C3
  \l 2B6C4
  \l 2B6C5
  \l 2B6C6
  \l 2B6C7
  \l 2B6C8
  \l 2B6C9
  \l 2B6CA
  \l 2B6CB
  \l 2B6CC
  \l 2B6CD
  \l 2B6CE
  \l 2B6CF
  \l 2B6D0
  \l 2B6D1
  \l 2B6D2
  \l 2B6D3
  \l 2B6D4
  \l 2B6D5
  \l 2B6D6
  \l 2B6D7
  \l 2B6D8
  \l 2B6D9
  \l 2B6DA
  \l 2B6DB
  \l 2B6DC
  \l 2B6DD
  \l 2B6DE
  \l 2B6DF
  \l 2B6E0
  \l 2B6E1
  \l 2B6E2
  \l 2B6E3
  \l 2B6E4
  \l 2B6E5
  \l 2B6E6
  \l 2B6E7
  \l 2B6E8
  \l 2B6E9
  \l 2B6EA
  \l 2B6EB
  \l 2B6EC
  \l 2B6ED
  \l 2B6EE
  \l 2B6EF
  \l 2B6F0
  \l 2B6F1
  \l 2B6F2
  \l 2B6F3
  \l 2B6F4
  \l 2B6F5
  \l 2B6F6
  \l 2B6F7
  \l 2B6F8
  \l 2B6F9
  \l 2B6FA
  \l 2B6FB
  \l 2B6FC
  \l 2B6FD
  \l 2B6FE
  \l 2B6FF
  \l 2B700
  \l 2B701
  \l 2B702
  \l 2B703
  \l 2B704
  \l 2B705
  \l 2B706
  \l 2B707
  \l 2B708
  \l 2B709
  \l 2B70A
  \l 2B70B
  \l 2B70C
  \l 2B70D
  \l 2B70E
  \l 2B70F
  \l 2B710
  \l 2B711
  \l 2B712
  \l 2B713
  \l 2B714
  \l 2B715
  \l 2B716
  \l 2B717
  \l 2B718
  \l 2B719
  \l 2B71A
  \l 2B71B
  \l 2B71C
  \l 2B71D
  \l 2B71E
  \l 2B71F
  \l 2B720
  \l 2B721
  \l 2B722
  \l 2B723
  \l 2B724
  \l 2B725
  \l 2B726
  \l 2B727
  \l 2B728
  \l 2B729
  \l 2B72A
  \l 2B72B
  \l 2B72C
  \l 2B72D
  \l 2B72E
  \l 2B72F
  \l 2B730
  \l 2B731
  \l 2B732
  \l 2B733
  \l 2B734
  \l 2B740
  \l 2B741
  \l 2B742
  \l 2B743
  \l 2B744
  \l 2B745
  \l 2B746
  \l 2B747
  \l 2B748
  \l 2B749
  \l 2B74A
  \l 2B74B
  \l 2B74C
  \l 2B74D
  \l 2B74E
  \l 2B74F
  \l 2B750
  \l 2B751
  \l 2B752
  \l 2B753
  \l 2B754
  \l 2B755
  \l 2B756
  \l 2B757
  \l 2B758
  \l 2B759
  \l 2B75A
  \l 2B75B
  \l 2B75C
  \l 2B75D
  \l 2B75E
  \l 2B75F
  \l 2B760
  \l 2B761
  \l 2B762
  \l 2B763
  \l 2B764
  \l 2B765
  \l 2B766
  \l 2B767
  \l 2B768
  \l 2B769
  \l 2B76A
  \l 2B76B
  \l 2B76C
  \l 2B76D
  \l 2B76E
  \l 2B76F
  \l 2B770
  \l 2B771
  \l 2B772
  \l 2B773
  \l 2B774
  \l 2B775
  \l 2B776
  \l 2B777
  \l 2B778
  \l 2B779
  \l 2B77A
  \l 2B77B
  \l 2B77C
  \l 2B77D
  \l 2B77E
  \l 2B77F
  \l 2B780
  \l 2B781
  \l 2B782
  \l 2B783
  \l 2B784
  \l 2B785
  \l 2B786
  \l 2B787
  \l 2B788
  \l 2B789
  \l 2B78A
  \l 2B78B
  \l 2B78C
  \l 2B78D
  \l 2B78E
  \l 2B78F
  \l 2B790
  \l 2B791
  \l 2B792
  \l 2B793
  \l 2B794
  \l 2B795
  \l 2B796
  \l 2B797
  \l 2B798
  \l 2B799
  \l 2B79A
  \l 2B79B
  \l 2B79C
  \l 2B79D
  \l 2B79E
  \l 2B79F
  \l 2B7A0
  \l 2B7A1
  \l 2B7A2
  \l 2B7A3
  \l 2B7A4
  \l 2B7A5
  \l 2B7A6
  \l 2B7A7
  \l 2B7A8
  \l 2B7A9
  \l 2B7AA
  \l 2B7AB
  \l 2B7AC
  \l 2B7AD
  \l 2B7AE
  \l 2B7AF
  \l 2B7B0
  \l 2B7B1
  \l 2B7B2
  \l 2B7B3
  \l 2B7B4
  \l 2B7B5
  \l 2B7B6
  \l 2B7B7
  \l 2B7B8
  \l 2B7B9
  \l 2B7BA
  \l 2B7BB
  \l 2B7BC
  \l 2B7BD
  \l 2B7BE
  \l 2B7BF
  \l 2B7C0
  \l 2B7C1
  \l 2B7C2
  \l 2B7C3
  \l 2B7C4
  \l 2B7C5
  \l 2B7C6
  \l 2B7C7
  \l 2B7C8
  \l 2B7C9
  \l 2B7CA
  \l 2B7CB
  \l 2B7CC
  \l 2B7CD
  \l 2B7CE
  \l 2B7CF
  \l 2B7D0
  \l 2B7D1
  \l 2B7D2
  \l 2B7D3
  \l 2B7D4
  \l 2B7D5
  \l 2B7D6
  \l 2B7D7
  \l 2B7D8
  \l 2B7D9
  \l 2B7DA
  \l 2B7DB
  \l 2B7DC
  \l 2B7DD
  \l 2B7DE
  \l 2B7DF
  \l 2B7E0
  \l 2B7E1
  \l 2B7E2
  \l 2B7E3
  \l 2B7E4
  \l 2B7E5
  \l 2B7E6
  \l 2B7E7
  \l 2B7E8
  \l 2B7E9
  \l 2B7EA
  \l 2B7EB
  \l 2B7EC
  \l 2B7ED
  \l 2B7EE
  \l 2B7EF
  \l 2B7F0
  \l 2B7F1
  \l 2B7F2
  \l 2B7F3
  \l 2B7F4
  \l 2B7F5
  \l 2B7F6
  \l 2B7F7
  \l 2B7F8
  \l 2B7F9
  \l 2B7FA
  \l 2B7FB
  \l 2B7FC
  \l 2B7FD
  \l 2B7FE
  \l 2B7FF
  \l 2B800
  \l 2B801
  \l 2B802
  \l 2B803
  \l 2B804
  \l 2B805
  \l 2B806
  \l 2B807
  \l 2B808
  \l 2B809
  \l 2B80A
  \l 2B80B
  \l 2B80C
  \l 2B80D
  \l 2B80E
  \l 2B80F
  \l 2B810
  \l 2B811
  \l 2B812
  \l 2B813
  \l 2B814
  \l 2B815
  \l 2B816
  \l 2B817
  \l 2B818
  \l 2B819
  \l 2B81A
  \l 2B81B
  \l 2B81C
  \l 2B81D
  \l 2F800
  \l 2F801
  \l 2F802
  \l 2F803
  \l 2F804
  \l 2F805
  \l 2F806
  \l 2F807
  \l 2F808
  \l 2F809
  \l 2F80A
  \l 2F80B
  \l 2F80C
  \l 2F80D
  \l 2F80E
  \l 2F80F
  \l 2F810
  \l 2F811
  \l 2F812
  \l 2F813
  \l 2F814
  \l 2F815
  \l 2F816
  \l 2F817
  \l 2F818
  \l 2F819
  \l 2F81A
  \l 2F81B
  \l 2F81C
  \l 2F81D
  \l 2F81E
  \l 2F81F
  \l 2F820
  \l 2F821
  \l 2F822
  \l 2F823
  \l 2F824
  \l 2F825
  \l 2F826
  \l 2F827
  \l 2F828
  \l 2F829
  \l 2F82A
  \l 2F82B
  \l 2F82C
  \l 2F82D
  \l 2F82E
  \l 2F82F
  \l 2F830
  \l 2F831
  \l 2F832
  \l 2F833
  \l 2F834
  \l 2F835
  \l 2F836
  \l 2F837
  \l 2F838
  \l 2F839
  \l 2F83A
  \l 2F83B
  \l 2F83C
  \l 2F83D
  \l 2F83E
  \l 2F83F
  \l 2F840
  \l 2F841
  \l 2F842
  \l 2F843
  \l 2F844
  \l 2F845
  \l 2F846
  \l 2F847
  \l 2F848
  \l 2F849
  \l 2F84A
  \l 2F84B
  \l 2F84C
  \l 2F84D
  \l 2F84E
  \l 2F84F
  \l 2F850
  \l 2F851
  \l 2F852
  \l 2F853
  \l 2F854
  \l 2F855
  \l 2F856
  \l 2F857
  \l 2F858
  \l 2F859
  \l 2F85A
  \l 2F85B
  \l 2F85C
  \l 2F85D
  \l 2F85E
  \l 2F85F
  \l 2F860
  \l 2F861
  \l 2F862
  \l 2F863
  \l 2F864
  \l 2F865
  \l 2F866
  \l 2F867
  \l 2F868
  \l 2F869
  \l 2F86A
  \l 2F86B
  \l 2F86C
  \l 2F86D
  \l 2F86E
  \l 2F86F
  \l 2F870
  \l 2F871
  \l 2F872
  \l 2F873
  \l 2F874
  \l 2F875
  \l 2F876
  \l 2F877
  \l 2F878
  \l 2F879
  \l 2F87A
  \l 2F87B
  \l 2F87C
  \l 2F87D
  \l 2F87E
  \l 2F87F
  \l 2F880
  \l 2F881
  \l 2F882
  \l 2F883
  \l 2F884
  \l 2F885
  \l 2F886
  \l 2F887
  \l 2F888
  \l 2F889
  \l 2F88A
  \l 2F88B
  \l 2F88C
  \l 2F88D
  \l 2F88E
  \l 2F88F
  \l 2F890
  \l 2F891
  \l 2F892
  \l 2F893
  \l 2F894
  \l 2F895
  \l 2F896
  \l 2F897
  \l 2F898
  \l 2F899
  \l 2F89A
  \l 2F89B
  \l 2F89C
  \l 2F89D
  \l 2F89E
  \l 2F89F
  \l 2F8A0
  \l 2F8A1
  \l 2F8A2
  \l 2F8A3
  \l 2F8A4
  \l 2F8A5
  \l 2F8A6
  \l 2F8A7
  \l 2F8A8
  \l 2F8A9
  \l 2F8AA
  \l 2F8AB
  \l 2F8AC
  \l 2F8AD
  \l 2F8AE
  \l 2F8AF
  \l 2F8B0
  \l 2F8B1
  \l 2F8B2
  \l 2F8B3
  \l 2F8B4
  \l 2F8B5
  \l 2F8B6
  \l 2F8B7
  \l 2F8B8
  \l 2F8B9
  \l 2F8BA
  \l 2F8BB
  \l 2F8BC
  \l 2F8BD
  \l 2F8BE
  \l 2F8BF
  \l 2F8C0
  \l 2F8C1
  \l 2F8C2
  \l 2F8C3
  \l 2F8C4
  \l 2F8C5
  \l 2F8C6
  \l 2F8C7
  \l 2F8C8
  \l 2F8C9
  \l 2F8CA
  \l 2F8CB
  \l 2F8CC
  \l 2F8CD
  \l 2F8CE
  \l 2F8CF
  \l 2F8D0
  \l 2F8D1
  \l 2F8D2
  \l 2F8D3
  \l 2F8D4
  \l 2F8D5
  \l 2F8D6
  \l 2F8D7
  \l 2F8D8
  \l 2F8D9
  \l 2F8DA
  \l 2F8DB
  \l 2F8DC
  \l 2F8DD
  \l 2F8DE
  \l 2F8DF
  \l 2F8E0
  \l 2F8E1
  \l 2F8E2
  \l 2F8E3
  \l 2F8E4
  \l 2F8E5
  \l 2F8E6
  \l 2F8E7
  \l 2F8E8
  \l 2F8E9
  \l 2F8EA
  \l 2F8EB
  \l 2F8EC
  \l 2F8ED
  \l 2F8EE
  \l 2F8EF
  \l 2F8F0
  \l 2F8F1
  \l 2F8F2
  \l 2F8F3
  \l 2F8F4
  \l 2F8F5
  \l 2F8F6
  \l 2F8F7
  \l 2F8F8
  \l 2F8F9
  \l 2F8FA
  \l 2F8FB
  \l 2F8FC
  \l 2F8FD
  \l 2F8FE
  \l 2F8FF
  \l 2F900
  \l 2F901
  \l 2F902
  \l 2F903
  \l 2F904
  \l 2F905
  \l 2F906
  \l 2F907
  \l 2F908
  \l 2F909
  \l 2F90A
  \l 2F90B
  \l 2F90C
  \l 2F90D
  \l 2F90E
  \l 2F90F
  \l 2F910
  \l 2F911
  \l 2F912
  \l 2F913
  \l 2F914
  \l 2F915
  \l 2F916
  \l 2F917
  \l 2F918
  \l 2F919
  \l 2F91A
  \l 2F91B
  \l 2F91C
  \l 2F91D
  \l 2F91E
  \l 2F91F
  \l 2F920
  \l 2F921
  \l 2F922
  \l 2F923
  \l 2F924
  \l 2F925
  \l 2F926
  \l 2F927
  \l 2F928
  \l 2F929
  \l 2F92A
  \l 2F92B
  \l 2F92C
  \l 2F92D
  \l 2F92E
  \l 2F92F
  \l 2F930
  \l 2F931
  \l 2F932
  \l 2F933
  \l 2F934
  \l 2F935
  \l 2F936
  \l 2F937
  \l 2F938
  \l 2F939
  \l 2F93A
  \l 2F93B
  \l 2F93C
  \l 2F93D
  \l 2F93E
  \l 2F93F
  \l 2F940
  \l 2F941
  \l 2F942
  \l 2F943
  \l 2F944
  \l 2F945
  \l 2F946
  \l 2F947
  \l 2F948
  \l 2F949
  \l 2F94A
  \l 2F94B
  \l 2F94C
  \l 2F94D
  \l 2F94E
  \l 2F94F
  \l 2F950
  \l 2F951
  \l 2F952
  \l 2F953
  \l 2F954
  \l 2F955
  \l 2F956
  \l 2F957
  \l 2F958
  \l 2F959
  \l 2F95A
  \l 2F95B
  \l 2F95C
  \l 2F95D
  \l 2F95E
  \l 2F95F
  \l 2F960
  \l 2F961
  \l 2F962
  \l 2F963
  \l 2F964
  \l 2F965
  \l 2F966
  \l 2F967
  \l 2F968
  \l 2F969
  \l 2F96A
  \l 2F96B
  \l 2F96C
  \l 2F96D
  \l 2F96E
  \l 2F96F
  \l 2F970
  \l 2F971
  \l 2F972
  \l 2F973
  \l 2F974
  \l 2F975
  \l 2F976
  \l 2F977
  \l 2F978
  \l 2F979
  \l 2F97A
  \l 2F97B
  \l 2F97C
  \l 2F97D
  \l 2F97E
  \l 2F97F
  \l 2F980
  \l 2F981
  \l 2F982
  \l 2F983
  \l 2F984
  \l 2F985
  \l 2F986
  \l 2F987
  \l 2F988
  \l 2F989
  \l 2F98A
  \l 2F98B
  \l 2F98C
  \l 2F98D
  \l 2F98E
  \l 2F98F
  \l 2F990
  \l 2F991
  \l 2F992
  \l 2F993
  \l 2F994
  \l 2F995
  \l 2F996
  \l 2F997
  \l 2F998
  \l 2F999
  \l 2F99A
  \l 2F99B
  \l 2F99C
  \l 2F99D
  \l 2F99E
  \l 2F99F
  \l 2F9A0
  \l 2F9A1
  \l 2F9A2
  \l 2F9A3
  \l 2F9A4
  \l 2F9A5
  \l 2F9A6
  \l 2F9A7
  \l 2F9A8
  \l 2F9A9
  \l 2F9AA
  \l 2F9AB
  \l 2F9AC
  \l 2F9AD
  \l 2F9AE
  \l 2F9AF
  \l 2F9B0
  \l 2F9B1
  \l 2F9B2
  \l 2F9B3
  \l 2F9B4
  \l 2F9B5
  \l 2F9B6
  \l 2F9B7
  \l 2F9B8
  \l 2F9B9
  \l 2F9BA
  \l 2F9BB
  \l 2F9BC
  \l 2F9BD
  \l 2F9BE
  \l 2F9BF
  \l 2F9C0
  \l 2F9C1
  \l 2F9C2
  \l 2F9C3
  \l 2F9C4
  \l 2F9C5
  \l 2F9C6
  \l 2F9C7
  \l 2F9C8
  \l 2F9C9
  \l 2F9CA
  \l 2F9CB
  \l 2F9CC
  \l 2F9CD
  \l 2F9CE
  \l 2F9CF
  \l 2F9D0
  \l 2F9D1
  \l 2F9D2
  \l 2F9D3
  \l 2F9D4
  \l 2F9D5
  \l 2F9D6
  \l 2F9D7
  \l 2F9D8
  \l 2F9D9
  \l 2F9DA
  \l 2F9DB
  \l 2F9DC
  \l 2F9DD
  \l 2F9DE
  \l 2F9DF
  \l 2F9E0
  \l 2F9E1
  \l 2F9E2
  \l 2F9E3
  \l 2F9E4
  \l 2F9E5
  \l 2F9E6
  \l 2F9E7
  \l 2F9E8
  \l 2F9E9
  \l 2F9EA
  \l 2F9EB
  \l 2F9EC
  \l 2F9ED
  \l 2F9EE
  \l 2F9EF
  \l 2F9F0
  \l 2F9F1
  \l 2F9F2
  \l 2F9F3
  \l 2F9F4
  \l 2F9F5
  \l 2F9F6
  \l 2F9F7
  \l 2F9F8
  \l 2F9F9
  \l 2F9FA
  \l 2F9FB
  \l 2F9FC
  \l 2F9FD
  \l 2F9FE
  \l 2F9FF
  \l 2FA00
  \l 2FA01
  \l 2FA02
  \l 2FA03
  \l 2FA04
  \l 2FA05
  \l 2FA06
  \l 2FA07
  \l 2FA08
  \l 2FA09
  \l 2FA0A
  \l 2FA0B
  \l 2FA0C
  \l 2FA0D
  \l 2FA0E
  \l 2FA0F
  \l 2FA10
  \l 2FA11
  \l 2FA12
  \l 2FA13
  \l 2FA14
  \l 2FA15
  \l 2FA16
  \l 2FA17
  \l 2FA18
  \l 2FA19
  \l 2FA1A
  \l 2FA1B
  \l 2FA1C
  \l 2FA1D
  \l E0100
  \l E0101
  \l E0102
  \l E0103
  \l E0104
  \l E0105
  \l E0106
  \l E0107
  \l E0108
  \l E0109
  \l E010A
  \l E010B
  \l E010C
  \l E010D
  \l E010E
  \l E010F
  \l E0110
  \l E0111
  \l E0112
  \l E0113
  \l E0114
  \l E0115
  \l E0116
  \l E0117
  \l E0118
  \l E0119
  \l E011A
  \l E011B
  \l E011C
  \l E011D
  \l E011E
  \l E011F
  \l E0120
  \l E0121
  \l E0122
  \l E0123
  \l E0124
  \l E0125
  \l E0126
  \l E0127
  \l E0128
  \l E0129
  \l E012A
  \l E012B
  \l E012C
  \l E012D
  \l E012E
  \l E012F
  \l E0130
  \l E0131
  \l E0132
  \l E0133
  \l E0134
  \l E0135
  \l E0136
  \l E0137
  \l E0138
  \l E0139
  \l E013A
  \l E013B
  \l E013C
  \l E013D
  \l E013E
  \l E013F
  \l E0140
  \l E0141
  \l E0142
  \l E0143
  \l E0144
  \l E0145
  \l E0146
  \l E0147
  \l E0148
  \l E0149
  \l E014A
  \l E014B
  \l E014C
  \l E014D
  \l E014E
  \l E014F
  \l E0150
  \l E0151
  \l E0152
  \l E0153
  \l E0154
  \l E0155
  \l E0156
  \l E0157
  \l E0158
  \l E0159
  \l E015A
  \l E015B
  \l E015C
  \l E015D
  \l E015E
  \l E015F
  \l E0160
  \l E0161
  \l E0162
  \l E0163
  \l E0164
  \l E0165
  \l E0166
  \l E0167
  \l E0168
  \l E0169
  \l E016A
  \l E016B
  \l E016C
  \l E016D
  \l E016E
  \l E016F
  \l E0170
  \l E0171
  \l E0172
  \l E0173
  \l E0174
  \l E0175
  \l E0176
  \l E0177
  \l E0178
  \l E0179
  \l E017A
  \l E017B
  \l E017C
  \l E017D
  \l E017E
  \l E017F
  \l E0180
  \l E0181
  \l E0182
  \l E0183
  \l E0184
  \l E0185
  \l E0186
  \l E0187
  \l E0188
  \l E0189
  \l E018A
  \l E018B
  \l E018C
  \l E018D
  \l E018E
  \l E018F
  \l E0190
  \l E0191
  \l E0192
  \l E0193
  \l E0194
  \l E0195
  \l E0196
  \l E0197
  \l E0198
  \l E0199
  \l E019A
  \l E019B
  \l E019C
  \l E019D
  \l E019E
  \l E019F
  \l E01A0
  \l E01A1
  \l E01A2
  \l E01A3
  \l E01A4
  \l E01A5
  \l E01A6
  \l E01A7
  \l E01A8
  \l E01A9
  \l E01AA
  \l E01AB
  \l E01AC
  \l E01AD
  \l E01AE
  \l E01AF
  \l E01B0
  \l E01B1
  \l E01B2
  \l E01B3
  \l E01B4
  \l E01B5
  \l E01B6
  \l E01B7
  \l E01B8
  \l E01B9
  \l E01BA
  \l E01BB
  \l E01BC
  \l E01BD
  \l E01BE
  \l E01BF
  \l E01C0
  \l E01C1
  \l E01C2
  \l E01C3
  \l E01C4
  \l E01C5
  \l E01C6
  \l E01C7
  \l E01C8
  \l E01C9
  \l E01CA
  \l E01CB
  \l E01CC
  \l E01CD
  \l E01CE
  \l E01CF
  \l E01D0
  \l E01D1
  \l E01D2
  \l E01D3
  \l E01D4
  \l E01D5
  \l E01D6
  \l E01D7
  \l E01D8
  \l E01D9
  \l E01DA
  \l E01DB
  \l E01DC
  \l E01DD
  \l E01DE
  \l E01DF
  \l E01E0
  \l E01E1
  \l E01E2
  \l E01E3
  \l E01E4
  \l E01E5
  \l E01E6
  \l E01E7
  \l E01E8
  \l E01E9
  \l E01EA
  \l E01EB
  \l E01EC
  \l E01ED
  \l E01EE
  \l E01EF
\endgroup
\global\sfcode"2019=0 %
\global\sfcode"201D=0 %
\begingroup
  \ifx\XeTeXchartoks\XeTeXcharclass
    \endgroup\expandafter\endinput
  \else
    \def\setclass#1#2#3{%
      \ifnum#1>#2 %
        \expandafter\gobble
      \else
        \expandafter\firstofone
      \fi
        {%
          \global\XeTeXcharclass#1=#3 %
          \expandafter\setclass\expandafter
            {\number\numexpr#1+1\relax}{#2}{#3}%
        }%
    }%
    \def\gobble#1{}
    \def\firstofone#1{#1}
    \def\ID#1 #2 {\setclass{"#1}{"#2}{1}}
    \def\OP#1 {\setclass{"#1}{"#1}{2}}
    \def\CL#1 {\setclass{"#1}{"#1}{3}}
    \def\EX#1 {\setclass{"#1}{"#1}{3}}
    \def\IS#1 {\setclass{"#1}{"#1}{3}}
    \def\NS#1 {\setclass{"#1}{"#1}{3}}
    \def\CM#1 {\setclass{"#1}{"#1}{256}}
  \fi
\ID 231A 231B
  \OP 2329
  \CL 232A
\ID 23F0 23F3
\ID 2600 2603
\ID 2614 2615
\ID 2618 2618
\ID 261A 261F
\ID 2639 263B
\ID 2668 2668
\ID 267F 267F
\ID 26BD 26C8
\ID 26CD 26CD
\ID 26CF 26D1
\ID 26D3 26D4
\ID 26D8 26D9
\ID 26DC 26DC
\ID 26DF 26E1
\ID 26EA 26EA
\ID 26F1 26F5
\ID 26F7 26FA
\ID 26FD 26FF
\ID 2700 2704
\ID 2708 270D
\ID 2E80 2E99
\ID 2E9B 2EF3
\ID 2F00 2FD5
\ID 2FF0 2FFB
  \CL 3001
  \CL 3002
\ID 3003 3003
\ID 3004 3004
  \NS 3005
\ID 3006 3006
\ID 3007 3007
  \OP 3008
  \CL 3009
  \OP 300A
  \CL 300B
  \OP 300C
  \CL 300D
  \OP 300E
  \CL 300F
  \OP 3010
  \CL 3011
\ID 3012 3013
  \OP 3014
  \CL 3015
  \OP 3016
  \CL 3017
  \OP 3018
  \CL 3019
  \OP 301A
  \CL 301B
  \NS 301C
  \OP 301D
  \CL 301E
  \CL 301F
\ID 3020 3020
\ID 3021 3029
  \CM 302A
  \CM 302B
  \CM 302C
  \CM 302D
  \CM 302E
  \CM 302F
\ID 3030 3030
\ID 3031 3034
  \CM 3035
\ID 3036 3037
\ID 3038 303A
  \NS 303B
  \NS 303C
\ID 303D 303D
\ID 303E 303F
\ID 3042 3042
\ID 3044 3044
\ID 3046 3046
\ID 3048 3048
\ID 304A 3062
\ID 3064 3082
\ID 3084 3084
\ID 3086 3086
\ID 3088 308D
\ID 308F 3094
  \CM 3099
  \CM 309A
  \NS 309B
  \NS 309C
  \NS 309D
  \NS 309E
\ID 309F 309F
  \NS 30A0
\ID 30A2 30A2
\ID 30A4 30A4
\ID 30A6 30A6
\ID 30A8 30A8
\ID 30AA 30C2
\ID 30C4 30E2
\ID 30E4 30E4
\ID 30E6 30E6
\ID 30E8 30ED
\ID 30EF 30F4
\ID 30F7 30FA
  \NS 30FB
  \NS 30FD
  \NS 30FE
\ID 30FF 30FF
\ID 3105 312D
\ID 3131 318E
\ID 3190 3191
\ID 3192 3195
\ID 3196 319F
\ID 31A0 31BA
\ID 31C0 31E3
\ID 3200 321E
\ID 3220 3229
\ID 322A 3247
\ID 3250 3250
\ID 3251 325F
\ID 3260 327F
\ID 3280 3289
\ID 328A 32B0
\ID 32B1 32BF
\ID 32C0 32FE
\ID 3300 33FF
\ID 3400 4DB5
\ID 4DB6 4DBF
\ID 4E00 9FCC
\ID 9FCD 9FFF
\ID A000 A014
  \NS A015
\ID A016 A48C
\ID A490 A4C6
\ID F900 FA6D
\ID FA6E FA6F
\ID FA70 FAD9
\ID FADA FAFF
  \IS FE10
  \CL FE11
  \CL FE12
  \IS FE13
  \IS FE14
  \EX FE15
  \EX FE16
  \OP FE17
  \CL FE18
\ID FE30 FE30
\ID FE31 FE32
\ID FE33 FE34
  \OP FE35
  \CL FE36
  \OP FE37
  \CL FE38
  \OP FE39
  \CL FE3A
  \OP FE3B
  \CL FE3C
  \OP FE3D
  \CL FE3E
  \OP FE3F
  \CL FE40
  \OP FE41
  \CL FE42
  \OP FE43
  \CL FE44
\ID FE45 FE46
  \OP FE47
  \CL FE48
\ID FE49 FE4C
\ID FE4D FE4F
  \CL FE50
\ID FE51 FE51
  \CL FE52
  \NS FE54
  \NS FE55
  \EX FE56
  \EX FE57
\ID FE58 FE58
  \OP FE59
  \CL FE5A
  \OP FE5B
  \CL FE5C
  \OP FE5D
  \CL FE5E
\ID FE5F FE61
\ID FE62 FE62
\ID FE63 FE63
\ID FE64 FE66
\ID FE68 FE68
\ID FE6B FE6B
  \EX FF01
\ID FF02 FF03
\ID FF06 FF07
  \OP FF08
  \CL FF09
\ID FF0A FF0A
\ID FF0B FF0B
  \CL FF0C
\ID FF0D FF0D
  \CL FF0E
\ID FF0F FF0F
\ID FF10 FF19
  \NS FF1A
  \NS FF1B
\ID FF1C FF1E
  \EX FF1F
\ID FF20 FF20
\ID FF21 FF3A
  \OP FF3B
\ID FF3C FF3C
  \CL FF3D
\ID FF3E FF3E
\ID FF3F FF3F
\ID FF40 FF40
\ID FF41 FF5A
  \OP FF5B
\ID FF5C FF5C
  \CL FF5D
\ID FF5E FF5E
  \OP FF5F
  \CL FF60
  \CL FF61
  \OP FF62
  \CL FF63
  \CL FF64
  \NS FF65
  \NS FF9E
  \NS FF9F
\ID FFE2 FFE2
\ID FFE3 FFE3
\ID FFE4 FFE4
\ID 1B000 1B001
\ID 1F000 1F02B
\ID 1F030 1F093
\ID 1F0A0 1F0AE
\ID 1F0B1 1F0BF
\ID 1F0C1 1F0CF
\ID 1F0D1 1F0F5
\ID 1F200 1F202
\ID 1F210 1F23A
\ID 1F240 1F248
\ID 1F250 1F251
\ID 1F300 1F32C
\ID 1F330 1F37D
\ID 1F380 1F39B
\ID 1F39E 1F3B4
\ID 1F3B7 1F3BB
\ID 1F3BD 1F3CE
\ID 1F3D4 1F3F7
\ID 1F400 1F49F
\ID 1F4A1 1F4A1
\ID 1F4A3 1F4A3
\ID 1F4A5 1F4AE
\ID 1F4B0 1F4B0
\ID 1F4B3 1F4FE
\ID 1F507 1F516
\ID 1F525 1F531
\ID 1F54A 1F54A
\ID 1F550 1F579
\ID 1F57B 1F5A3
\ID 1F5A5 1F5D3
\ID 1F5DC 1F5F3
\ID 1F5FA 1F5FF
\ID 1F600 1F642
\ID 1F645 1F64F
\ID 1F680 1F6CF
\ID 1F6E0 1F6EC
\ID 1F6F0 1F6F3
\ID 20000 2A6D6
\ID 2A6D7 2A6FF
\ID 2A700 2B734
\ID 2B735 2B73F
\ID 2B740 2B81D
\ID 2B81E 2F7FF
\ID 2F800 2FA1D
\ID 2FA1E 2FFFD
\ID 30000 3FFFD
\endgroup
\gdef\xtxHanGlue{\hskip0pt plus 0.1em\relax}
\gdef\xtxHanSpace{\hskip0.2em plus 0.2em minus 0.1em\relax}
\global\XeTeXinterchartoks 0 1 = {\xtxHanSpace}
\global\XeTeXinterchartoks 0 2 = {\xtxHanSpace}
\global\XeTeXinterchartoks 0 3 = {\nobreak\xtxHanSpace}
\global\XeTeXinterchartoks 1 0 = {\xtxHanSpace}
\global\XeTeXinterchartoks 2 0 = {\nobreak\xtxHanSpace}
\global\XeTeXinterchartoks 3 0 = {\xtxHanSpace}
\global\XeTeXinterchartoks 1 1 = {\xtxHanGlue}
\global\XeTeXinterchartoks 1 2 = {\xtxHanGlue}
\global\XeTeXinterchartoks 1 3 = {\nobreak\xtxHanGlue}
\global\XeTeXinterchartoks 2 1 = {\nobreak\xtxHanGlue}
\global\XeTeXinterchartoks 2 2 = {\nobreak\xtxHanGlue}
\global\XeTeXinterchartoks 2 3 = {\xtxHanGlue}
\global\XeTeXinterchartoks 3 1 = {\xtxHanGlue}
\global\XeTeXinterchartoks 3 2 = {\xtxHanGlue}
\global\XeTeXinterchartoks 3 3 = {\nobreak\xtxHanGlue}
%
  \let\endgroup\ENDGROUP
  \@firstofone{%
    \catcode64=12 %
    \savecatcodetable\catcodetable@latex
    \catcode64=11 %
    \savecatcodetable\catcodetable@atletter
   }
\endgroup
%    \end{macrocode}
% \end{macro}
% \end{macro}
% \end{macro}
% \end{macro}
%
% \subsection{Named Lua functions}
%
% \begin{macro}{\newluafunction}
% \changes{v1.0a}{0000/00/00}{Macro added}
%   Much the same story for allocating Lua\TeX{} functions except here they are
%   just numbers so are allocated in the same way as boxes. Lua index from~$1$
%   so once again slot~$0$ is skipped.
%    \begin{macrocode}
\ifx\e@alloc@luafunction@count\@undefined
  \countdef\e@alloc@luafunction@count=260
\fi
\def\newluafunction{%
  \e@alloc\luafunction\e@alloc@chardef
    \e@alloc@luafunction@count\m@ne\e@alloc@top
}
\e@alloc@luafunction@count=\z@
%    \end{macrocode}
% \end{macro}
%
% \subsection{Custom whatsits}
%
% \begin{macro}{\newwhatsit}
% \changes{v1.0a}{0000/00/00}{Macro added}
%   These are only settable from Lua but for consistency are definable
%   here.
%    \begin{macrocode}
\ifx\e@alloc@whatsit@count\@undefined
  \countdef\e@alloc@whatsit@count=261
\fi
\def\newwhatsit#1{%
  \e@alloc\whatsit\e@alloc@chardef
    \e@alloc@whatsit@count\m@ne\e@alloc@top#1%
}
\e@alloc@whatsit@count=\z@
%    \end{macrocode}
% \end{macro}
%
% \subsection{Lua loader}
%
% Load the Lua code at the start of every job.
% For the conversion of \TeX{} into numbers at the Lua side we need some
% known registers: for convenience we use a set of systematic names, which
% means using a group around the Lua loader.
%    \begin{macrocode}
%<2ekernel>\everyjob\expandafter{%
%<2ekernel>  \the\everyjob
  \begingroup
    \attributedef\attributezero=0 %
    \chardef     \charzero     =0 %
%    \end{macrocode}
% Note name change required on older luatex, for hash table access.
%    \begin{macrocode}
    \countdef    \CountZero    =0 % 
    \dimendef    \dimenzero    =0 %
    \mathchardef \mathcharzero =0 %
    \muskipdef   \muskipzero   =0 %
    \skipdef     \skipzero     =0 %
    \toksdef     \tokszero     =0 %
    \directlua{require("ltluatex")}
  \endgroup
%<2ekernel>}
%<latexrelease>\EndIncludeInRelease
%    \end{macrocode}
%
%    \begin{macrocode}
%<latexrelease>\IncludeInRelease{0000/00/00}
%<latexrelease>                 {\newluafunction}{LuaTeX}%
%<latexrelease>\let\e@alloc@attribute@count\@undefined
%<latexrelease>\let\newattribute\@undefined
%<latexrelease>\let\setattribute\@undefined
%<latexrelease>\let\unsetattribute\@undefined
%<latexrelease>\let\e@alloc@ccodetable@count\@undefined
%<latexrelease>\let\newcatcodetable\@undefined
%<latexrelease>\let\catcodetable@initex\@undefined
%<latexrelease>\let\catcodetable@string\@undefined
%<latexrelease>\let\catcodetable@latex\@undefined
%<latexrelease>\let\catcodetable@atletter\@undefined
%<latexrelease>\let\e@alloc@luafunction@count\@undefined
%<latexrelease>\let\newluafunction\@undefined
%<latexrelease>\let\e@alloc@luafunction@count\@undefined
%<latexrelease>\let\newwhatsit\@undefined
%<latexrelease>\let\e@alloc@whatsit@count\@undefined
%<latexrelease>\EndIncludeInRelease
%    \end{macrocode}
%
%    \begin{macrocode}
%<2ekernel|latexrelease>\fi
%</2ekernel|tex|latexrelease>
%    \end{macrocode}
%
% \subsection{Lua module preliminaries}
%
% \begingroup
%
%  \begingroup\lccode`~=`_
%  \lowercase{\endgroup\let~}_
%  \catcode`_=12
%
%    \begin{macrocode}
%<*lua>
%    \end{macrocode}
%
% Some set up for the Lua module which is needed for all of the Lua
% functionality added here.
%
% \begin{macro}{luatexbase}
% \changes{v1.0a}{0000/00/00}{Table added}
%   Set up the table for the returned functions. This is used to expose
%   all of the public functions.
%    \begin{macrocode}
luatexbase       = luatexbase or { }
local luatexbase = luatexbase
%    \end{macrocode}
% \end{macro}
%
% Some Lua best practice: use local versions of functions where possible.
%    \begin{macrocode}
local string_gsub      = string.gsub
local tex_count        = tex.count
local tex_setattribute = tex.setattribute
local tex_setcount     = tex.setcount
local texio_write_nl   = texio.write_nl
%    \end{macrocode}
%
% \subsection{Lua module utilities}
%
% \subsubsection{Module tracking}
%
% \begin{macro}{luatexbase.modules}
% \changes{v1.0a}{0000/00/00}{Function modified}
%   To allow tracking of module usage, a structure is provided to store
%   information and to return it.
%    \begin{macrocode}
local modules = modules or { }
%    \end{macrocode}
% \end{macro}
%
% \begin{macro}{luatexbase.provides_module}
% \changes{v1.0a}{0000/00/00}{Function added}
%   Modelled on |\ProvidesPackage|, we store much the same information but
%   with a little more structure.
%    \begin{macrocode}
local function provides_module(info)
  if not (info and info.name) then
    luatexbase_error("Missing module name for provides_modules")
    return
  end
  local function spaced(text)
    return text and (" " .. text) or ""
  end
  texio_write_nl(
    "log",
    "Lua module: " .. info.name
      .. spaced(info.date)
      .. spaced(info.version)
      .. spaced(info.description)
  )
  modules[info.name] = info
end
luatexbase.provides_module = provides_module
%    \end{macrocode}
% \end{macro}
%
% \subsubsection{Module messages}
%
% There are various warnings and errors that need to be given. For warnings
% we can get exactly the same formatting as from \TeX{}. For errors we have to
% make some changes. Here we give the text of the error in the \LaTeX{} format
% then force an error from Lua to halt the run. Splitting the message text is
% done using |\n| which takes the place of |\MessageBreak|.
%
% First an auxiliary for the formatting: this measures up the message
% leader so we always get the correct indent.
%    \begin{macrocode}
local function msg_format(mod, msg_type, text)
  local leader = ""
  if type == "Error" then
    leader = "! "
  end
  local cont
  if mod == "LaTeX" then
    cont = string_gsub(leader, ".", " ")
    leader = leader .. "LaTeX: "
  else
    first_head = leader .. "Module "  .. msg_type 
    cont = "(" .. mod .. ")"
      .. string_gsub(first_head, ".", " ")
    first_head =  leader .. "Module "  .. mod .. " " .. msg_type  .. ":"
  end
  return first_head .. " "
    .. string_gsub(
         text .. " on input line "
         .. tex.inputlineno, "\n", "\n" .. cont .. " "
      )
   .. "\n"
end
%    \end{macrocode}
%
% \begin{macro}{luatexbase.module_info}
% \changes{v1.0a}{0000/00/00}{Function added}
% \begin{macro}{luatexbase.module_warning}
% \changes{v1.0a}{0000/00/00}{Function added}
% \begin{macro}{luatexbase.module_error}
% \changes{v1.0a}{0000/00/00}{Function added}
%   Write messages.
%    \begin{macrocode}
local function module_info(mod, text)
  local i
  for _,i in ipairs(msg_format(mod, "Info", text):explode("\n")) do
    texio_write_nl("log", i)
  end
end
luatexbase.module_info = module_info
local function module_warning(mod, text)
  local i
  for _,i in ipairs(msg_format(mod, "Warning", text):explode("\n")) do
    texio_write_nl("term and log", i)
  end
end
luatexbase.module_warning = module_warning
local function module_error(mod, text)
  local i
  for _,i in ipairs(msg_format(mod, "Error", text):explode("\n")) do
    texio_write_nl("term and log", i)
  end
  texio_write_nl("term and log", "\n")
  error("See " .. mod .. " Error")
end
luatexbase.module_error = module_error
%    \end{macrocode}
% \end{macro}
% \end{macro}
% \end{macro}
%
% Dedicated versions for the rest of the code here.
%    \begin{macrocode}
local function luatexbase_warning(text)
  module_warning("luatexbase", text)
end
local function luatexbase_error(text)
  module_error("luatexbase", text)
end
%    \end{macrocode}
%
%
% \subsection{Accessing register numbers from Lua}
%
% Collect up the data from the \TeX{} level into a Lua table: from
% version~0.80, Lua\TeX{} makes that easy.
%    \begin{macrocode}
local luaregisterbasetable = { }
local registermap = {
  attributezero = "assign_attr"    ,
  charzero      = "char_given"     ,
  CountZero     = "assign_int"     ,
  dimenzero     = "assign_dimen"   ,
  mathcharzero  = "math_given"     ,
  muskipzero    = "assign_mu_skip" ,
  skipzero      = "assign_skip"    ,
  tokszero      = "assign_toks"    ,
}
local i, j
local createtoken
if tex.luatexversion >79 then
 createtoken   = newtoken.create
end
local hashtokens    = tex.hashtokens
local luatexversion = tex.luatexversion
for i,j in pairs (registermap) do
  if luatexversion < 80 then
    luaregisterbasetable[hashtokens()[i][1]] =
      hashtokens()[i][2]
  else
    luaregisterbasetable[j] = createtoken(i).mode
  end
end
%    \end{macrocode}
%
% \begin{macro}{luatexbase.registernumber}
%   Working out the correct return value can be done in two ways. For older
%   Lua\TeX{} releases it has to be extracted from the |hashtokens|. On the
%   other hand, newer Lua\TeX{}'s have |newtoken|, and whilst |.mode| isn't
%   currently documented, Hans Hagen pointed to this approach so we should be
%   OK.
%    \begin{macrocode}
local registernumber
if luatexversion < 80 then
  function registernumber(name)
    local nt = hashtokens()[name]
    if(nt and luaregisterbasetable[nt[1]]) then
      return nt[2] - luaregisterbasetable[nt[1]]
    else
      return false
    end
  end
else
  function registernumber(name)
    local nt = createtoken(name)
    if(luaregisterbasetable[nt.cmdname]) then
      return nt.mode - luaregisterbasetable[nt.cmdname]
    else
      return false
    end
  end
end
luatexbase.registernumber = registernumber
%    \end{macrocode}
% \end{macro}
%
% \subsection{Attribute allocation}
%
% \begin{macro}{luatexbase.new_attribute}
% \changes{v1.0a}{0000/00/00}{Function added}
%   As attributes are used for Lua manipulations its useful to be able
%   to assign from this end.
%    \begin{macrocode}
local attributes=setmetatable(
{},
{
__index = function(t,key)
return registernumber(key) or nil
end}
)
luatexbase.attributes=attributes
%    \end{macrocode}
%
%    \begin{macrocode}
local function new_attribute(name)
  tex_setcount("global", "e@alloc@attribute@count",
                          tex_count["e@alloc@attribute@count"] + 1)
  if tex_count["e@alloc@attribute@count"] > 65534 then
    luatexbase_error("No room for a new \\attribute")
    return -1
  end
  attributes[name]= tex_count["e@alloc@attribute@count"]
  texio_write_nl("Lua-only attribute " .. name .. " = " ..
                 tex_count["e@alloc@attribute@count"])
  return tex_count["e@alloc@attribute@count"]
end
luatexbase.new_attribute = new_attribute
%    \end{macrocode}
% \end{macro}
%
% \subsection{Custom whatsit allocation}
%
% \begin{macro}{luatexbase.new_whatsit}
% Much the same as for attribute allocation in Lua
%    \begin{macrocode}
local function new_whatsit(name)
  tex_setcount("global", "e@alloc@whatsit@count", 
                         tex_count["e@alloc@whatsit@count"] + 1)
  if tex_count["e@alloc@whatsit@count"] > 65534 then
    luatexbase_error("No room for a new custom whatsit")
    return -1
  end
  texio_write_nl("Custom whatsit " .. name .. " = " ..
                 tex_count["e@alloc@whatsit@count"])
  return tex_count["e@alloc@whatsit@count"]
end
luatexbase.new_whatsit = new_whatsit
%    \end{macrocode}
% \end{macro}
%
% \subsection{Bytecode register allocation}
%
% \begin{macro}{luatexbase.new_bytecode}
% Much the same as for attribute allocation in Lua.
% Currently we maintain the allocation count purely in lua, not using
% a \TeX\ counter.
% The optional \meta{name} argument is used in the log if given.
%    \begin{macrocode}
local bytecode_count = 0
local function new_bytecode(name)
  bytecode_count = bytecode_count + 1
  if bytecode_count > 65534 then
    luatexbase_error("No room for a new bytecode")
    return -1
  end
  texio_write_nl("Lua bytecode " .. (name or "") .. " = " .. 
                 bytecode_count .. "\n")
  return bytecode_count
end
luatexbase.new_bytecode = new_bytecode
%    \end{macrocode}
% \end{macro}
%
% \subsection{Lua chunk name allocation}
%
% \begin{macro}{luatexbase.new_chunkname}
% As for bytecode registers but here we also store the name in the
% |lua.name| table.
%    \begin{macrocode}
local chunkname_count = 0
local function new_chunkname(name)
  chunkname_count = chunkname_count + 1
  if chunkname_count > 65534 then
    luatexbase_error("No room for a new chunkname")
    return -1
  end
  lua.name[chunkname_count]=name
  texio_write_nl("Lua chunkname " .. name .. " = " .. 
                 chunkname_count .. "\n")
  return chunkname_count
end
luatexbase.new_chunkname = new_chunkname
%    \end{macrocode}
% \end{macro}
%
% \subsection{Lua callback management}
%
% The native mechanism for callbacks in Lua allows only one per function.
% That is extremely restrictive and so a mechanism is needed to add and
% remove callbacks from the appropriate hooks.
%
% \subsubsection{Housekeeping}
%
% The main table: keys are callback names, and values are the associated lists
% of functions. More precisely, the entries in the list are tables holding the
% actual function as |func| and the identifying description as |description|.
% Only callbacks with a non-empty list of functions have an entry in this
% list.
%    \begin{macrocode}
local callbacklist = callbacklist or { }
%    \end{macrocode}
%
% Numerical codes for callback types, and name-to-value association (the
% table keys are strings, the values are numbers).
%    \begin{macrocode}
local list, data, exclusive, simple = 1, 2, 3, 4
local types = {
  list      = list,
  data      = data,
  exclusive = exclusive,
  simple    = simple,
}
%    \end{macrocode}
%
% Now, list all predefined callbacks with their current type, based on the
% Lua\TeX{} manual version~0.80. A full list of the currently-available
% callbacks can be obtained using
%  \begin{verbatim}
%    \directlua{
%      for i,_ in pairs(callback.list()) do
%        texio.write_nl("- " .. i)
%      end
%    }
%    \bye
%  \end{verbatim}
% in plain Lua\TeX{}. (Some undocumented callbacks are omitted as they are
% to be removed.)
%    \begin{macrocode}
local callbacktypes = callbacktypes or {
%    \end{macrocode}
%   Section 4.1.1: file discovery callbacks.
%    \begin{macrocode}
  find_read_file     = exclusive,
  find_write_file    = exclusive,
  find_font_file     = data,
  find_output_file   = data,
  find_format_file   = data,
  find_vf_file       = data,
  find_map_file      = data,
  find_enc_file      = data,
  find_sfd_file      = data,
  find_pk_file       = data,
  find_data_file     = data,
  find_opentype_file = data,
  find_truetype_file = data,
  find_type1_file    = data,
  find_image_file    = data,
%    \end{macrocode}
% Section 4.1.2: file reading callbacks.
%    \begin{macrocode}
  open_read_file     = exclusive,
  read_font_file     = exclusive,
  read_vf_file       = exclusive,
  read_map_file      = exclusive,
  read_enc_file      = exclusive,
  read_sfd_file      = exclusive,
  read_pk_file       = exclusive,
  read_data_file     = exclusive,
  read_truetype_file = exclusive,
  read_type1_file    = exclusive,
  read_opentype_file = exclusive,
%    \end{macrocode}
% Section 4.1.3: data processing callbacks.
%    \begin{macrocode}
  process_input_buffer  = data,
  process_output_buffer = data,
  process_jobname       = data,
  token_filter          = exclusive,
%    \end{macrocode}
% Section 4.1.4: node list processing callbacks.
%    \begin{macrocode}
  buildpage_filter      = simple,
  pre_linebreak_filter  = list,
  linebreak_filter      = list,
  post_linebreak_filter = list,
  hpack_filter          = list,
  vpack_filter          = list,
  pre_output_filter     = list,
  hyphenate             = simple,
  ligaturing            = simple,
  kerning               = simple,
  mlist_to_hlist        = list,
%    \end{macrocode}
% Section 4.1.5: information reporting callbacks.
%    \begin{macrocode}
  pre_dump            = simple,
  start_run           = simple,
  stop_run            = simple,
  start_page_number   = simple,
  stop_page_number    = simple,
  show_error_hook     = simple,
  show_error_message  = simple,
  show_lua_error_hook = simple,
  start_file          = simple,
  stop_file           = simple,
%    \end{macrocode}
% Section 4.1.6: PDF-related callbacks.
%    \begin{macrocode}
  finish_pdffile = data,
  finish_pdfpage = data,
%    \end{macrocode}
% Section 4.1.7: font-related callbacks.
%    \begin{macrocode}
  define_font = exclusive,
%    \end{macrocode}
% Undocumented callbacks which are likely to get documented.
%    \begin{macrocode}
  find_cidmap_file           = data,
  pdf_stream_filter_callback = data,
}
luatexbase.callbacktypes=callbacktypes
%    \end{macrocode}
%
% \begin{macro}{callback.register}
% \changes{v1.0a}{0000/00/00}{Function modified}
%   Save the original function for registering callbacks and prevent the
%   original being used. The original is saved in a place that remains
%   available so other more sophisticated code can override the approach
%   taken by the kernel if desired.
%    \begin{macrocode}
local callback_register = callback_register or callback.register
function callback.register()
  luatexbase_error("Attempt to use callback.register() directly.")
end
%    \end{macrocode}
% \end{macro}
%
% \subsubsection{Handlers}
%
% The handler function is registered into the callback when the
% first function is added to this callback's list. Then, when the callback
% is called, then handler takes care of running all functions in the list.
% When the last function is removed from the callback's list, the handler
% is unregistered.
%
% More precisely, the functions below are used to generate a specialized
% function (closure) for a given callback, which is the actual handler.
%
% Handler for |data| callbacks.
%    \begin{macrocode}
local function data_handler(name)
  return function(data, ...)
    local i
    for _,i in ipairs(callbacklist[name]) do
      data = i.func(data,...)
    end
    return data
  end
end
%    \end{macrocode}
% Handler for |exclusive| callbacks. We can assume |callbacklist[name]| is not
% empty: otherwise, the function wouldn't be registered in the callback any
% more.
%    \begin{macrocode}
local function exclusive_handler(name)
  return function(...)
    return callbacklist[name][1].func(...)
  end
end
%    \end{macrocode}
% Handler for |list| callbacks.
%    \begin{macrocode}
local function list_handler(name)
  return function(head, ...)
    local ret
    local alltrue = true
    local i
    for _,i in ipairs(callbacklist[name]) do
      ret = i.func(head, ...)
      if ret == false then
        luatexbase_warning(
          "Function `i.description' returned false\n"
            .. "in callback `name'"
         )
         break
      end
      if ret ~= true then
        alltrue = false
        head = ret
      end
    end
    return alltrue and true or head
  end
end
%    \end{macrocode}
% Handler for |simple| callbacks.
%    \begin{macrocode}
local function simple_handler(name)
  return function(...)
    local i
    for _,i in ipairs(callbacklist[name]) do
      i.func(...)
    end
  end
end
%    \end{macrocode}
%
% Keep a handlers table for indexed access.
%    \begin{macrocode}
local handlers = {
  [data]      = data_handler,
  [exclusive] = exclusive_handler,
  [list]      = list_handler,
  [simple]    = simple_handler,
}
%    \end{macrocode}
%
% \subsubsection{Public functions for callback management}
%
% Defining user callbacks perhaps should be in package code,
% but impacts on |add_to_callback|.
% If a default function is not required, may may be declared as |false|.
% First we need a list of user callbacks.
%    \begin{macrocode}
local user_callbacks_defaults = { }
%    \end{macrocode}
%
% \begin{macro}{luatexbase.create_callback}
% \changes{v1.0a}{0000/00/00}{Function added}
%   The allocator itself.
%    \begin{macrocode}
local function create_callback(name, ctype, default)
  if not name or
    name == "" or
    callbacktypes[name] or
    not(default == false or  type(default) == "function")
    then
      luatexbase_error("Unable to create callback " .. name)
  end
  user_callbacks_defaults[name] = default
  callbacktypes[name] = types[ctype]
end
luatexbase.create_callback = create_callback
%    \end{macrocode}
% \end{macro}
%
% \begin{macro}{luatexbase.call_callback}
% \changes{v1.0a}{0000/00/00}{Function added}
%  Call a user defined callback. First check arguments.
%    \begin{macrocode}
local function call_callback(name,...)
  if not name or
    name == "" or
    user_callbacks_defaults[name] == nil
    then
        luatexbase_error("Unable to call callback " .. name)
  end
  local l = callbacklist[name]
  local f
  if not l then
    f = user_callbacks_defaults[name]
    if l == false then
	   return nil
	 end
  else
    f = handlers[callbacktypes[name]](name)
  end
  return f(...)
end
luatexbase.call_callback=call_callback
%    \end{macrocode}
% \end{macro}
%
% \begin{macro}{luatexbase.add_to_callback}
% \changes{v1.0a}{0000/00/00}{Function added}
%   Add a function to a callback. First check arguments.
%    \begin{macrocode}
local function add_to_callback(name, func, description)
  if
    not name or
    name == "" or
    not callbacktypes[name] or
    type(func) ~= "function" or
    not description or
    description == "" then
    luatexbase_error(
      "Unable to register callback.\n\n"
        .. "Correct usage:\n"
        .. "add_to_callback(<callback>, <function>, <description>)"
    )
    return
  end
%    \end{macrocode}
%   Then test if this callback is already in use. If not, initialise its list
%   and register the proper handler.
%    \begin{macrocode}
  local l = callbacklist[name]
  if l == nil then
    l = { }
    callbacklist[name] = l
%    \end{macrocode}
% If it is not a user defined callback use the primitive callback register.
%    \begin{macrocode}
    if user_callbacks_defaults[name] == nil then
      callback_register(name, handlers[callbacktypes[name]](name))
    end
  end
%    \end{macrocode}
%  Actually register the function and give an error if more than one
%  |exclusive| one is registered.
%    \begin{macrocode}
  local f = {
    func        = func,
    description = description,
  }
  local priority = #l + 1
  if callbacktypes[name] == exclusive then
    if #l == 1 then
      luatexbase_error(
        "Cannot add second callback to exclusive function `" ..
        name .. "'.")
    end
  end
  table.insert(l, priority, f)
%    \end{macrocode}
%  Keep user informed.
%    \begin{macrocode}
  texio_write_nl(
    "Inserting `" .. description .. "' at position "
      .. priority .. " in `" .. name .. "'."
  )
end
luatexbase.add_to_callback = add_to_callback
%    \end{macrocode}
% \end{macro}
%
% \begin{macro}{luatexbase.remove_from_callback}
% \changes{v1.0a}{0000/00/00}{Function added}
%   Remove a function from a callback. First check arguments.
%    \begin{macrocode}
local function remove_from_callback(name, description)
  if
    not name or
    name == "" or
    not callbacktypes[name] or
    not description or
    description == "" then
    luatexbase_error(
      "Unable to remove function from callback.\n\n"
        .. "Correct usage:\n"
        .. "remove_to_callback(<callback>, <description>)"
    )
    return
  end
  local l = callbacklist[name]
  if not l then
    luatexbase_error(
      "No callback list for `" .. name .. "'.")
  end
%    \end{macrocode}
%  Loop over the callback's function list until we find a matching entry.
%  Remove it and check if the list is empty: if so, unregister the
%   callback handler.
%    \begin{macrocode}
  local index = false
  local i,j
  local cb = {}
  for i,j in ipairs(l) do
    if j.description == description then
      index = i
      break
    end
  end
  if not index then
    luatexbase_error(
      "No callback `" .. description .. "' registered for `" ..
      name .. "'.")
    return
  end
  cb = l[index]
  table.remove(l, index)
  texio_write_nl(
    "Removing  `" .. description .. "' from `" .. name .. "'."
  )
  if #l == 0 then
    callbacklist[name] = nil
    callback_register(name, nil)
  end
  return cb.func,cb.description
end
luatexbase.remove_from_callback = remove_from_callback
%    \end{macrocode}
% \end{macro}
%
% \begin{macro}{luatexbase.in_callback}
% \changes{v1.0a}{0000/00/00}{Function added}
%   Look for a function description in a callback.
%    \begin{macrocode}
local function in_callback(name, description)
  if not name
    or name == ""
    or not callbacktypes[name]
    or not description then
      return false
  end
  local i
  for _, i in pairs(callbacklist[name]) do
    if i.description == description then
      return true
    end
  end
  return false
end
luatexbase.in_callback = in_callback
%    \end{macrocode}
% \end{macro}
%
% \begin{macro}{luatexbase.disable_callback}
% \changes{v1.0a}{0000/00/00}{Function added}
%   As we subvert the engine interface we need to provide a way to access
%   this functionality.
%    \begin{macrocode}
local function disable_callback(name)
  if(callbacklist[name] == nil) then
    callback_register(name, false)
  else
    luatexbase_error("Callback list for " .. name .. " not empty")
  end
end
luatexbase.disable_callback = disable_callback
%    \end{macrocode}
% \end{macro}
%
% \begin{macro}{luatexbase.in_callback}
% \changes{v1.0a}{0000/00/00}{Function added}
%   List the descriptions of functions registed for the given callback.
%    \begin{macrocode}
local function callback_descriptions (name)
  local d = {}
  if not name
    or name == ""
    or not callbacktypes[name]
    then
    return d
  else
  local i
  for k, i in pairs(callbacklist[name] or {}) do
    d[k]= i.description
    end
  end
  return d
end
luatexbase.callback_descriptions =callback_descriptions 
%    \end{macrocode}
% \end{macro}
% \endgroup
%
%    \begin{macrocode}
%</lua>
%    \end{macrocode}
%
% Reset the catcode of |@|.
%    \begin{macrocode}
%<tex>\catcode`\@=\etatcatcode\relax
%    \end{macrocode}
%
%
% \Finale
|. This inputs |ltluatex.tex| which inputs
% |etex.src| (or |etex.sty| if used with \LaTeX)
% if it is not already input, and then defines some internal commands to
% allow the \textsf{ltluatex} interface to be defined.
%
% The \textsf{luatexbase} package interface may also be used in plain \TeX,
% as before, by inputting the package |\input luatexbase.sty|. The new
% version of \textsf{luatexbase} is based on this \textsf{ltluatex}
% code but implements a compatibility layer providing the interface
% of the original package.
%
% \section{Lua functionality}
%
% \begingroup
%
% \begingroup\lccode`~=`_
% \lowercase{\endgroup\let~}_
% \catcode`_=12
%
% \subsection{Allocators in Lua}
%
% \DescribeMacro{new_attribute}
% |luatexbase.new_attribute(|\meta{attribute}|)|\\
% Returns an allocation number for the \meta{attribute}, indexed from~$1$.
% The attribute will be initialised with the marker value |-"7FFFFFFF|
% (`unset'). The attribute allocation sequence is shared with the \TeX{}
% code but this function does \emph{not} define a token using
% |\attributedef|.
% The attribute name is recorded in the |attributes| table. A
% metatable is provided so that the table syntax can be used
% consistently for attributes declared in \TeX\ or Lua.
%
% \noindent
% \DescribeMacro{new_whatsit}
% |luatexbase.new_whatsit(|\meta{whatsit}|)|\\
% Returns an allocation number for the custom \meta{whatsit}, indexed from~$1$.
%
% \noindent
% \DescribeMacro{new_bytecode}
% |luatexbase.new_bytecode(|\meta{bytecode}|)|\\
% Returns an allocation number for a bytecode register, indexed from~$1$.
% The optional \meta{name} argument is just used for logging.
%
% \noindent
% \DescribeMacro{new_chunkname}
% |luatexbase.new_chunkname(|\meta{chunkname}|)|\\
% Returns an allocation number for a Lua chunk name for use with 
% |\directlua| and |\latelua|, indexed from~$1$.
% The number is returned and also \meta{name} argument is added to the
% |lua.name| array at that index.
%
% These functions all require access to a named \TeX{} count register
% to manage their allocations. The standard names are those defined
% above for access from \TeX{}, \emph{e.g.}~\string\e@alloc@attribute@count,
% but these can be adjusted by defining the variable
% \texttt{\meta{type}\_count\_name} before loading |ltluatex.lua|, for example
% \begin{verbatim}
% local attribute_count_name = "attributetracker"
% require("ltluatex")
% \end{verbatim}
% would use a \TeX{} |\count| (|\countdef|'d token) called |attributetracker|
% in place of \string\e@alloc@attribute@count.
%
% \subsection{Lua access to \TeX{} register numbers}
%
% \DescribeMacro{registernumber}
% |luatexbase.registernumer(|\meta{name}|)|\\
% Sometimes (notably in the case of Lua attributes) it is necessary to
% access a register \emph{by number} that has been allocated by \TeX{}.
% This package provides a function to look up the relevant number
% using Lua\TeX{}'s internal tables. After for example
% |\newattribute\myattrib|, |\myattrib| would be defined by (say)
% |\myattrib=\attribute15|.  |luatexbase.registernumer("myattrib")|
% would then return the register number, $15$ in this case. If the string passed
% as argument does not correspond to a token defined by |\attributedef|,
% |\countdef| or similar commands, the Lua value |false| is returned.
%
% As an example, consider the input:
%\begin{verbatim}
% \newcommand\test[1]{%
% \typeout{#1: \expandafter\meaning\csname#1\endcsname^^J
% \space\space\space\space
% \directlua{tex.write(luatexbase.registernumber("#1") or "bad input")}%
% }}
%
% \test{undefinedrubbish}
%
% \test{space}
%
% \test{hbox}
%
% \test{@MM}
%
% \test{@tempdima}
% \test{@tempdimb}
%
% \test{strutbox}
%
% \test{sixt@@n}
%
% \attrbutedef\myattr=12
% \myattr=200
% \test{myattr}
%
%\end{verbatim}
%
% If the demonstration code is processed with Lua\LaTeX{} then the following
% would be produced in the log and terminal output.
%\begin{verbatim}
% undefinedrubbish: \relax
%      bad input
% space: macro:->
%      bad input
% hbox: \hbox
%      bad input
% @MM: \mathchar"4E20
%      20000
% @tempdima: \dimen14
%      14
% @tempdimb: \dimen15
%      15
% strutbox: \char"B
%      11
% sixt@@n: \char"10
%      16
% myattr: \attribute12
%      12
%\end{verbatim}
%
% Notice how undefined commands, or commands unrelated to registers
% do not produce an error, just return |false| and so print
% |bad input| here. Note also that commands defined by |\newbox| work and
% return the number of the box register even though the actual command
% holding this number is a |\chardef| defined token (there is no
% |\boxdef|).
%
% \subsection{Module utilities}
%
% \DescribeMacro{provides_module}
% |luatexbase.provides_module(|\meta{info}|)|\\
% This function is used by modules to identify themselves; the |info| should be
% a table containing information about the module. The required field
% |name| must contain the name of the module. It is recommended to provide a
% field |date| in the usual \LaTeX{} format |yyyy/mm/dd|. Optional fields
% |version| (a string) and |description| may be used if present. This
% information will be recorded in the log. Other fields are ignored.
%
% \noindent
% \DescribeMacro{module_info}
% \DescribeMacro{module_warning}
% \DescribeMacro{module_error}
% |luatexbase.module_info(|\meta{module}, \meta{text}|)|\\
% |luatexbase.module_warning(|\meta{module}, \meta{text}|)|\\
% |luatexbase.module_error(|\meta{module}, \meta{text}|)|\\
% These functions are similar to \LaTeX{}'s |\PackageError|, |\PackageWarning|
% and |\PackageInfo| in the way they format the output.  No automatic line
% breaking is done, you may still use |\n| as usual for that, and the name of
% the package will be prepended to each output line.
%
% Note that |luatexbase.module_error| raises an actual Lua error with |error()|,
% which currently means a call stack will be dumped. While this may not
% look pretty, at least it provides useful information for tracking the
% error down.
%
% \subsection{Callback management}
%
% \noindent
% \DescribeMacro{add_to_callback}
% |luatexbase.add_to_callback(|^^A
% \meta{callback}, \meta{function}, \meta{description}|)|
% Registers the \meta{function} into the \meta{callback} with a textual
% \meta{description} of the function. Functions are inserted into the callback
% in the order loaded.
%
% \noindent
% \DescribeMacro{remove_from_callback}
% |luatexbase.remove_from_callback(|\meta{callback}, \meta{description}|)|
% Removes the callback function with \meta{description} from the \meta{callback}.
% The removed function and its description 
% are returned as the results of this function.
%
% \noindent
% \DescribeMacro{in_callback}
% |luatexbase.in_callback(|\meta{callback}, \meta{description}|)|
% Checks if the \meta{description} matches one of the functions added
% to the list for the \meta{callback}, returning a boolean value.
%
% \noindent
% \DescribeMacro{disable_callback}
% |luatexbase.disable_callback(|\meta{callback}|)|
% Sets the \meta{callback} to \texttt{false} as described in the Lua\TeX{}
% manual for the underlying \texttt{callback.register} built-in. Callbacks
% will only be set to false (and thus be skipped entirely) if there are
% no functions registered using the callback.
%
% \noindent
% \DescribeMacro{callback_descriptions}
% A list of the descriptions of functions registered to the specified
% callback is returned. |{}| is returned if there are no functions registered.
%
% \noindent
% \DescribeMacro{create_callback}
% |luatexbase.create_callback(|\meta{name},meta{type},\meta{default}|)|
% Defines a user defined callback. The last argument is a default
% function or |false|.
%
% \noindent
% \DescribeMacro{call_callback}
% |luatexbase.call_callback(|\meta{name},\ldots|)|
% Calls a user defined callback with the supplied arguments.
%
% \endgroup
%
% \StopEventually{}
%
% \section{Implementation}
%
%    \begin{macrocode}
%<*2ekernel|tex|latexrelease>
%<2ekernel|latexrelease>\ifx\directlua\@undefined\else
%    \end{macrocode}
%
%
% \changes{v1.0j}{2015/12/02}{Remove nonlocal iteration variables (PHG)}
% \changes{v1.0j}{2015/12/02}{Assorted typos fixed (PHG)}
% \changes{v1.0j}{2015/12/02}{Remove unreachable code after calls to error() (PHG)}
% \subsection{Minimum Lua\TeX{} version}
%
% Lua\TeX{} has changed a lot over time. In the kernel support for ancient
% versions is not provided: trying to build a format with a very old binary
% therefore gives some information in the log and loading stops. The cut-off
% selected here relates to the tree-searching behaviour of |require()|:
% from version~0.60, Lua\TeX{} will correctly find Lua files in the |texmf|
% tree without `help'.
%    \begin{macrocode}
%<latexrelease>\IncludeInRelease{2015/10/01}
%<latexrelease>                 {\newluafunction}{LuaTeX}%
\ifnum\luatexversion<60 %
  \wlog{***************************************************}
  \wlog{* LuaTeX version too old for ltluatex support *}
  \wlog{***************************************************}
  \expandafter\endinput
\fi
%    \end{macrocode}
%
% \subsection{Older \LaTeX{}/Plain \TeX\ setup}
% 
%    \begin{macrocode}
%<*tex>
%    \end{macrocode}
%
% Older \LaTeX{} formats don't have the primitives with `native' names:
% sort that out. If they already exist this will still be safe.
%    \begin{macrocode}
\directlua{tex.enableprimitives("",tex.extraprimitives("luatex"))}
%    \end{macrocode}
%
%    \begin{macrocode}
\ifx\e@alloc\@undefined
%    \end{macrocode}
%
% In pre-2014 \LaTeX{}, or plain \TeX{}, load |etex.{sty,src}|.
%    \begin{macrocode}
  \ifx\documentclass\@undefined
    \ifx\loccount\@undefined
      \input{etex.src}%
    \fi
    \catcode`\@=11 %
    \outer\expandafter\def\csname newfam\endcsname
                          {\alloc@8\fam\chardef\et@xmaxfam}
  \else
    \RequirePackage{etex}
    \expandafter\def\csname newfam\endcsname
                    {\alloc@8\fam\chardef\et@xmaxfam}
    \expandafter\let\expandafter\new@mathgroup\csname newfam\endcsname
  \fi
%    \end{macrocode}
%
% \subsubsection{Fixes to \texttt{etex.src}/\texttt{etex.sty}}
%
% These could and probably should be made directly in an
% update to |etex.src| which already has some Lua\TeX-specific
% code, but does not define the correct range for Lua\TeX.
%
% 2015-07-13 higher range in luatex.
%    \begin{macrocode}
\edef \et@xmaxregs {\ifx\directlua\@undefined 32768\else 65536\fi}
%    \end{macrocode}
% luatex/xetex also allow more math fam.
%    \begin{macrocode}
\edef \et@xmaxfam {\ifx\Umathchar\@undefined\sixt@@n\else\@cclvi\fi}
%    \end{macrocode}
%
%    \begin{macrocode}
\count 270=\et@xmaxregs % locally allocates \count registers
\count 271=\et@xmaxregs % ditto for \dimen registers
\count 272=\et@xmaxregs % ditto for \skip registers
\count 273=\et@xmaxregs % ditto for \muskip registers
\count 274=\et@xmaxregs % ditto for \box registers
\count 275=\et@xmaxregs % ditto for \toks registers
\count 276=\et@xmaxregs % ditto for \marks classes
%    \end{macrocode}
%
% and 256 or 16 fam. (Done above due to plain/\LaTeX\ differences in
% \textsf{ltluatex}.)
%    \begin{macrocode}
% \outer\def\newfam{\alloc@8\fam\chardef\et@xmaxfam}
%    \end{macrocode}
%
% End of proposed changes to \texttt{etex.src}
%
% \subsubsection{luatex specific settings}
% 
% Switch to global cf |luatex.sty| to leave room for inserts
% not really needed for luatex but possibly most compatible
% with existing use.
%    \begin{macrocode}
\expandafter\let\csname newcount\expandafter\expandafter\endcsname
                \csname globcount\endcsname
\expandafter\let\csname newdimen\expandafter\expandafter\endcsname
                \csname globdimen\endcsname
\expandafter\let\csname newskip\expandafter\expandafter\endcsname
                \csname globskip\endcsname
\expandafter\let\csname newbox\expandafter\expandafter\endcsname
                \csname globbox\endcsname
%    \end{macrocode}
%
% Define|\e@alloc| as in latex (the existing macros in |etex.src|
% hard to extend to further register types as they assume specific
% 26x and 27x count range. For compatibility the existing register
% allocation is not changed.
%
%    \begin{macrocode}
\chardef\e@alloc@top=65535
\let\e@alloc@chardef\chardef
%    \end{macrocode}
%
%    \begin{macrocode}
\def\e@alloc#1#2#3#4#5#6{%
  \global\advance#3\@ne
  \e@ch@ck{#3}{#4}{#5}#1%
  \allocationnumber#3\relax
  \global#2#6\allocationnumber
  \wlog{\string#6=\string#1\the\allocationnumber}}%
%    \end{macrocode}
%
%    \begin{macrocode}
\gdef\e@ch@ck#1#2#3#4{%
  \ifnum#1<#2\else
    \ifnum#1=#2\relax
      #1\@cclvi
      \ifx\count#4\advance#1 10 \fi
    \fi
    \ifnum#1<#3\relax
    \else
      \errmessage{No room for a new \string#4}%
    \fi
  \fi}%
%    \end{macrocode}
%
% Two simple \LaTeX\ macros used in |ltlatex.sty|.
%    \begin{macrocode}
\long\def\@gobble#1{}
\long\def\@firstofone#1{#1}
%    \end{macrocode}
%
% Fix up allocations not to clash with |etex.src|.
%
%    \begin{macrocode}
\expandafter\csname newcount\endcsname\e@alloc@attribute@count
\expandafter\csname newcount\endcsname\e@alloc@ccodetable@count
\expandafter\csname newcount\endcsname\e@alloc@luafunction@count
\expandafter\csname newcount\endcsname\e@alloc@whatsit@count
\expandafter\csname newcount\endcsname\e@alloc@bytecode@count
\expandafter\csname newcount\endcsname\e@alloc@luachunk@count
%    \end{macrocode}
%
% End of conditional setup for plain \TeX\ / old \LaTeX.
%    \begin{macrocode}
\fi
%</tex>
%    \end{macrocode}
%
%
% \subsection{Attributes}
%
% \begin{macro}{\newattribute}
% \changes{v1.0a}{2015/09/24}{Macro added}
%   As is generally the case for the Lua\TeX{} registers we start here
%   from~$1$. Notably, some code assumes that |\attribute0| is never used so
%   this is important in this case.
%    \begin{macrocode}
\ifx\e@alloc@attribute@count\@undefined
  \countdef\e@alloc@attribute@count=258
\fi
\def\newattribute#1{%
  \e@alloc\attribute\attributedef
    \e@alloc@attribute@count\m@ne\e@alloc@top#1%
}
\e@alloc@attribute@count=\z@
%    \end{macrocode}
% \end{macro}
%
% \begin{macro}{\setattribute}
% \begin{macro}{\unsetattribute}
%   Handy utilities.
%    \begin{macrocode}
\def\setattribute#1#2{#1=\numexpr#2\relax}
\def\unsetattribute#1{#1=-"7FFFFFFF\relax}
%    \end{macrocode}
% \end{macro}
% \end{macro}
%
% \subsection{Category code tables}
%
% \begin{macro}{\newcatcodetable}
% \changes{v1.0a}{2015/09/24}{Macro added}
%   Category code tables are allocated with a limit half of that used by Lua\TeX{}
%   for everything else. At the end of allocation there needs to be an
%   initialisation step. Table~$0$ is already taken (it's the global one for
%   current use) so the allocation starts at~$1$.
%    \begin{macrocode}
\ifx\e@alloc@ccodetable@count\@undefined
  \countdef\e@alloc@ccodetable@count=259
\fi
\def\newcatcodetable#1{%
  \e@alloc\catcodetable\chardef
    \e@alloc@ccodetable@count\m@ne{"8000}#1%
  \initcatcodetable\allocationnumber
}
\e@alloc@ccodetable@count=\z@
%    \end{macrocode}
% \end{macro}
%
% \changes{v1.0l}{2015/12/18}{Load Unicode data from source}
% \begin{macro}{\catcodetable@initex}
% \changes{v1.0a}{2015/09/24}{Macro added}
% \begin{macro}{\catcodetable@string}
% \changes{v1.0a}{2015/09/24}{Macro added}
% \begin{macro}{\catcodetable@latex}
% \changes{v1.0a}{2015/09/24}{Macro added}
% \begin{macro}{\catcodetable@atletter}
% \changes{v1.0a}{2015/09/24}{Macro added}
%   Save a small set of standard tables. The Unicode data is read
%   here in using a parser simplified from that in |load-unicode-data|:
%   only the nature of letters needs to be detected.
%    \begin{macrocode}
\newcatcodetable\catcodetable@initex
\newcatcodetable\catcodetable@string
\begingroup
  \def\setrangecatcode#1#2#3{%
    \ifnum#1>#2 %
      \expandafter\@gobble
    \else
      \expandafter\@firstofone
    \fi
      {%
        \catcode#1=#3 %
        \expandafter\setrangecatcode\expandafter
          {\number\numexpr#1 + 1\relax}{#2}{#3}
      }%
  }
  \@firstofone{%
    \catcodetable\catcodetable@initex
      \catcode0=12 %
      \catcode13=12 %
      \catcode37=12 %
      \setrangecatcode{65}{90}{12}%
      \setrangecatcode{97}{122}{12}%
      \catcode92=12 %
      \catcode127=12 %
      \savecatcodetable\catcodetable@string
    \endgroup
  }%
\newcatcodetable\catcodetable@latex
\newcatcodetable\catcodetable@atletter
\begingroup
  \def\parseunicodedataI#1;#2;#3;#4\relax{%
    \parseunicodedataII#1;#3;#2 First>\relax
  }%
  \def\parseunicodedataII#1;#2;#3 First>#4\relax{%
    \ifx\relax#4\relax
      \expandafter\parseunicodedataIII
    \else
      \expandafter\parseunicodedataIV
    \fi
      {#1}#2\relax%
  }%
  \def\parseunicodedataIII#1#2#3\relax{%
    \ifnum 0%
      \if L#21\fi
      \if M#21\fi
      >0 %
      \catcode"#1=11 %
    \fi
  }%
  \def\parseunicodedataIV#1#2#3\relax{%
    \read\unicoderead to \unicodedataline
    \if L#2%
      \count0="#1 %
      \expandafter\parseunicodedataV\unicodedataline\relax
    \fi
  }%
  \def\parseunicodedataV#1;#2\relax{%
    \loop
      \unless\ifnum\count0>"#1 %
        \catcode\count0=11 %
        \advance\count0 by 1 %
    \repeat
  }%
  \def\storedpar{\par}%
  \chardef\unicoderead=\numexpr\count16 + 1\relax
  \openin\unicoderead=UnicodeData.txt %
  \loop\unless\ifeof\unicoderead %
    \read\unicoderead to \unicodedataline  
    \unless\ifx\unicodedataline\storedpar
      \expandafter\parseunicodedataI\unicodedataline\relax
    \fi
  \repeat
  \closein\unicoderead
  \@firstofone{%
    \catcode64=12 %
    \savecatcodetable\catcodetable@latex
    \catcode64=11 %
    \savecatcodetable\catcodetable@atletter
   }
\endgroup
%    \end{macrocode}
% \end{macro}
% \end{macro}
% \end{macro}
% \end{macro}
%
% \subsection{Named Lua functions}
%
% \begin{macro}{\newluafunction}
% \changes{v1.0a}{2015/09/24}{Macro added}
%   Much the same story for allocating Lua\TeX{} functions except here they are
%   just numbers so they are allocated in the same way as boxes.
%   Lua indexes from~$1$ so once again slot~$0$ is skipped.
%    \begin{macrocode}
\ifx\e@alloc@luafunction@count\@undefined
  \countdef\e@alloc@luafunction@count=260
\fi
\def\newluafunction{%
  \e@alloc\luafunction\e@alloc@chardef
    \e@alloc@luafunction@count\m@ne\e@alloc@top
}
\e@alloc@luafunction@count=\z@
%    \end{macrocode}
% \end{macro}
%
% \subsection{Custom whatsits}
%
% \begin{macro}{\newwhatsit}
% \changes{v1.0a}{2015/09/24}{Macro added}
%   These are only settable from Lua but for consistency are definable
%   here.
%    \begin{macrocode}
\ifx\e@alloc@whatsit@count\@undefined
  \countdef\e@alloc@whatsit@count=261
\fi
\def\newwhatsit#1{%
  \e@alloc\whatsit\e@alloc@chardef
    \e@alloc@whatsit@count\m@ne\e@alloc@top#1%
}
\e@alloc@whatsit@count=\z@
%    \end{macrocode}
% \end{macro}
%
% \subsection{Lua bytecode registers}
%
% \begin{macro}{\newluabytecode}
% \changes{v1.0a}{2015/09/24}{Macro added}
%   These are only settable from Lua but for consistency are definable
%   here.
%    \begin{macrocode}
\ifx\e@alloc@bytecode@count\@undefined
  \countdef\e@alloc@bytecode@count=262
\fi
\def\newluabytecode#1{%
  \e@alloc\luabytecode\e@alloc@chardef
    \e@alloc@bytecode@count\m@ne\e@alloc@top#1%
}
\e@alloc@bytecode@count=\z@
%    \end{macrocode}
% \end{macro}
%
% \subsection{Lua chunk registers}

% \begin{macro}{\newluachunkname}
% \changes{v1.0a}{2015/09/24}{Macro added}
% As for bytecode registers, but in addition we need to add a string
% to the \verb|lua.name| table to use in stack tracing. We use the
% name of the command passed to the allocator, with no backslash.
%    \begin{macrocode}
\ifx\e@alloc@luachunk@count\@undefined
  \countdef\e@alloc@luachunk@count=263
\fi
\def\newluachunkname#1{%
  \e@alloc\luachunk\e@alloc@chardef
    \e@alloc@luachunk@count\m@ne\e@alloc@top#1%
    {\escapechar\m@ne
    \directlua{lua.name[\the\allocationnumber]="\string#1"}}%
}
\e@alloc@luachunk@count=\z@
%    \end{macrocode}
% \end{macro}
%
% \subsection{Lua loader}
%
% Load the Lua code at the start of every job.
% For the conversion of \TeX{} into numbers at the Lua side we need some
% known registers: for convenience we use a set of systematic names, which
% means using a group around the Lua loader.
%    \begin{macrocode}
%<2ekernel>\everyjob\expandafter{%
%<2ekernel>  \the\everyjob
  \begingroup
    \attributedef\attributezero=0 %
    \chardef     \charzero     =0 %
%    \end{macrocode}
% Note name change required on older luatex, for hash table access.
%    \begin{macrocode}
    \countdef    \CountZero    =0 % 
    \dimendef    \dimenzero    =0 %
    \mathchardef \mathcharzero =0 %
    \muskipdef   \muskipzero   =0 %
    \skipdef     \skipzero     =0 %
    \toksdef     \tokszero     =0 %
    \directlua{require("ltluatex")}
  \endgroup
%<2ekernel>}
%<latexrelease>\EndIncludeInRelease
%    \end{macrocode}
%
% \changes{v1.0b}{2015/10/02}{Fix backing out of \TeX{} code}
% \changes{v1.0c}{2015/10/02}{Allow backing out of Lua code}
%    \begin{macrocode}
%<latexrelease>\IncludeInRelease{0000/00/00}
%<latexrelease>                 {\newluafunction}{LuaTeX}%
%<latexrelease>\let\e@alloc@attribute@count\@undefined
%<latexrelease>\let\newattribute\@undefined
%<latexrelease>\let\setattribute\@undefined
%<latexrelease>\let\unsetattribute\@undefined
%<latexrelease>\let\e@alloc@ccodetable@count\@undefined
%<latexrelease>\let\newcatcodetable\@undefined
%<latexrelease>\let\catcodetable@initex\@undefined
%<latexrelease>\let\catcodetable@string\@undefined
%<latexrelease>\let\catcodetable@latex\@undefined
%<latexrelease>\let\catcodetable@atletter\@undefined
%<latexrelease>\let\e@alloc@luafunction@count\@undefined
%<latexrelease>\let\newluafunction\@undefined
%<latexrelease>\let\e@alloc@luafunction@count\@undefined
%<latexrelease>\let\newwhatsit\@undefined
%<latexrelease>\let\e@alloc@whatsit@count\@undefined
%<latexrelease>\let\newluabytecode\@undefined
%<latexrelease>\let\e@alloc@bytecode@count\@undefined
%<latexrelease>\let\newluachunkname\@undefined
%<latexrelease>\let\e@alloc@luachunk@count\@undefined
%<latexrelease>\directlua{luatexbase.uninstall()}
%<latexrelease>\EndIncludeInRelease
%    \end{macrocode}
%
% In \verb|\everyjob|, if luaotfload is available, load it and switch to TU.
%    \begin{macrocode}
%<latexrelease>\IncludeInRelease{2017/01/01}%
%<latexrelease>                 {\fontencoding}{TU in everyjob}%
%<latexrelease>\fontencoding{TU}\let\encodingdefault\f@encoding
%<latexrelease>\ifx\directlua\@undefined\else
%<2ekernel>\everyjob\expandafter{%
%<2ekernel>  \the\everyjob
%<*2ekernel,latexrelease>
  \directlua{%
  if xpcall(function ()%
             require('luaotfload-main')%
            end,texio.write_nl) then %
  local _void = luaotfload.main ()%
  else %
  texio.write_nl('Error in luaotfload: reverting to OT1')%
  tex.print('\string\\def\string\\encodingdefault{OT1}')%
  end %
  }%
  \let\f@encoding\encodingdefault
  \expandafter\let\csname ver@luaotfload.sty\endcsname\fmtversion
%</2ekernel,latexrelease>
%<latexrelease>\fi
%<2ekernel>  }
%<latexrelease>\EndIncludeInRelease
%<latexrelease>\IncludeInRelease{0000/00/00}%
%<latexrelease>                 {\fontencoding}{TU in everyjob}%
%<latexrelease>\fontencoding{OT1}\let\encodingdefault\f@encoding
%<latexrelease>\EndIncludeInRelease
%    \end{macrocode}
%
%    \begin{macrocode}
%<2ekernel|latexrelease>\fi
%</2ekernel|tex|latexrelease>
%    \end{macrocode}
%
% \subsection{Lua module preliminaries}
%
% \begingroup
%
%  \begingroup\lccode`~=`_
%  \lowercase{\endgroup\let~}_
%  \catcode`_=12
%
%    \begin{macrocode}
%<*lua>
%    \end{macrocode}
%
% Some set up for the Lua module which is needed for all of the Lua
% functionality added here.
%
% \begin{macro}{luatexbase}
% \changes{v1.0a}{2015/09/24}{Table added}
%   Set up the table for the returned functions. This is used to expose
%   all of the public functions.
%    \begin{macrocode}
luatexbase       = luatexbase or { }
local luatexbase = luatexbase
%    \end{macrocode}
% \end{macro}
%
% Some Lua best practice: use local versions of functions where possible.
%    \begin{macrocode}
local string_gsub      = string.gsub
local tex_count        = tex.count
local tex_setattribute = tex.setattribute
local tex_setcount     = tex.setcount
local texio_write_nl   = texio.write_nl
%    \end{macrocode}
% \changes{v1.0i}{2015/11/29}{Declare this as local before used in the module error definitions (PHG)}
%    \begin{macrocode}
local luatexbase_warning
local luatexbase_error
%    \end{macrocode}
%
% \subsection{Lua module utilities}
%
% \subsubsection{Module tracking}
%
% \begin{macro}{modules}
% \changes{v1.0a}{2015/09/24}{Function modified}
%   To allow tracking of module usage, a structure is provided to store
%   information and to return it.
%    \begin{macrocode}
local modules = modules or { }
%    \end{macrocode}
% \end{macro}
%
% \begin{macro}{provides_module}
% \changes{v1.0a}{2015/09/24}{Function added}
% \changes{v1.0f}{2015/10/03}{use luatexbase\_log}
% Local function to write to the log.
%    \begin{macrocode}
local function luatexbase_log(text)
  texio_write_nl("log", text)
end
%    \end{macrocode}
%
%   Modelled on |\ProvidesPackage|, we store much the same information but
%   with a little more structure.
%    \begin{macrocode}
local function provides_module(info)
  if not (info and info.name) then
    luatexbase_error("Missing module name for provides_module")
  end
  local function spaced(text)
    return text and (" " .. text) or ""
  end
  luatexbase_log(
    "Lua module: " .. info.name
      .. spaced(info.date)
      .. spaced(info.version)
      .. spaced(info.description)
  )
  modules[info.name] = info
end
luatexbase.provides_module = provides_module
%    \end{macrocode}
% \end{macro}
%
% \subsubsection{Module messages}
%
% There are various warnings and errors that need to be given. For warnings
% we can get exactly the same formatting as from \TeX{}. For errors we have to
% make some changes. Here we give the text of the error in the \LaTeX{} format
% then force an error from Lua to halt the run. Splitting the message text is
% done using |\n| which takes the place of |\MessageBreak|.
%
% First an auxiliary for the formatting: this measures up the message
% leader so we always get the correct indent.
% \changes{v1.0j}{2015/12/02}{Declaration/use of first\_head fixed (PHG)}
%    \begin{macrocode}
local function msg_format(mod, msg_type, text)
  local leader = ""
  local cont
  local first_head
  if mod == "LaTeX" then
    cont = string_gsub(leader, ".", " ")
    first_head = leader .. "LaTeX: "
  else
    first_head = leader .. "Module "  .. msg_type
    cont = "(" .. mod .. ")"
      .. string_gsub(first_head, ".", " ")
    first_head =  leader .. "Module "  .. mod .. " " .. msg_type  .. ":"
  end
  if msg_type == "Error" then
    first_head = "\n" .. first_head
  end
  if string.sub(text,-1) ~= "\n" then
    text = text .. " "
  end
  return first_head .. " "
    .. string_gsub(
         text 
	 .. "on input line "
         .. tex.inputlineno, "\n", "\n" .. cont .. " "
      )
   .. "\n"
end
%    \end{macrocode}
%
% \begin{macro}{module_info}
% \changes{v1.0a}{2015/09/24}{Function added}
% \begin{macro}{module_warning}
% \changes{v1.0a}{2015/09/24}{Function added}
% \begin{macro}{module_error}
% \changes{v1.0a}{2015/09/24}{Function added}
%   Write messages.
%    \begin{macrocode}
local function module_info(mod, text)
  texio_write_nl("log", msg_format(mod, "Info", text))
end
luatexbase.module_info = module_info
local function module_warning(mod, text)
  texio_write_nl("term and log",msg_format(mod, "Warning", text))
end
luatexbase.module_warning = module_warning
local function module_error(mod, text)
  error(msg_format(mod, "Error", text))
end
luatexbase.module_error = module_error
%    \end{macrocode}
% \end{macro}
% \end{macro}
% \end{macro}
%
% Dedicated versions for the rest of the code here.
%    \begin{macrocode}
function luatexbase_warning(text)
  module_warning("luatexbase", text)
end
function luatexbase_error(text)
  module_error("luatexbase", text)
end
%    \end{macrocode}
%
%
% \subsection{Accessing register numbers from Lua}
%
% \changes{v1.0g}{2015/11/14}{Track Lua\TeX{} changes for
%   \texttt{(new)token.create}}
% Collect up the data from the \TeX{} level into a Lua table: from
% version~0.80, Lua\TeX{} makes that easy.
% \changes{v1.0j}{2015/12/02}{Adjust hashtokens to store the result of tex.hashtokens()), not the function (PHG)}
%    \begin{macrocode}
local luaregisterbasetable = { }
local registermap = {
  attributezero = "assign_attr"    ,
  charzero      = "char_given"     ,
  CountZero     = "assign_int"     ,
  dimenzero     = "assign_dimen"   ,
  mathcharzero  = "math_given"     ,
  muskipzero    = "assign_mu_skip" ,
  skipzero      = "assign_skip"    ,
  tokszero      = "assign_toks"    ,
}
local createtoken
if tex.luatexversion > 81 then
  createtoken = token.create
elseif tex.luatexversion > 79 then
  createtoken = newtoken.create 
end
local hashtokens    = tex.hashtokens()
local luatexversion = tex.luatexversion
for i,j in pairs (registermap) do
  if luatexversion < 80 then
    luaregisterbasetable[hashtokens[i][1]] =
      hashtokens[i][2]
  else
    luaregisterbasetable[j] = createtoken(i).mode
  end
end
%    \end{macrocode}
%
% \begin{macro}{registernumber}
%   Working out the correct return value can be done in two ways. For older
%   Lua\TeX{} releases it has to be extracted from the |hashtokens|. On the
%   other hand, newer Lua\TeX{}'s have |newtoken|, and whilst |.mode| isn't
%   currently documented, Hans Hagen pointed to this approach so we should be
%   OK.
%    \begin{macrocode}
local registernumber
if luatexversion < 80 then
  function registernumber(name)
    local nt = hashtokens[name]
    if(nt and luaregisterbasetable[nt[1]]) then
      return nt[2] - luaregisterbasetable[nt[1]]
    else
      return false
    end
  end
else
  function registernumber(name)
    local nt = createtoken(name)
    if(luaregisterbasetable[nt.cmdname]) then
      return nt.mode - luaregisterbasetable[nt.cmdname]
    else
      return false
    end
  end
end
luatexbase.registernumber = registernumber
%    \end{macrocode}
% \end{macro}
%
% \subsection{Attribute allocation}
%
% \begin{macro}{new_attribute}
% \changes{v1.0a}{2015/09/24}{Function added}
% \changes{v1.1c}{2017/02/18}{Parameterise count used in tracking}
%   As attributes are used for Lua manipulations its useful to be able
%   to assign from this end.
%    \begin{macrocode}
local attributes=setmetatable(
{},
{
__index = function(t,key)
return registernumber(key) or nil
end}
)
luatexbase.attributes = attributes
%    \end{macrocode}
%
%    \begin{macrocode}
local attribute_count_name =
                     attribute_count_name or "e@alloc@attribute@count"
local function new_attribute(name)
  tex_setcount("global", attribute_count_name,
                          tex_count[attribute_count_name] + 1)
  if tex_count[attribute_count_name] > 65534 then
    luatexbase_error("No room for a new \\attribute")
  end
  attributes[name]= tex_count[attribute_count_name]
  luatexbase_log("Lua-only attribute " .. name .. " = " ..
                 tex_count[attribute_count_name])
  return tex_count[attribute_count_name]
end
luatexbase.new_attribute = new_attribute
%    \end{macrocode}
% \end{macro}
%
% \subsection{Custom whatsit allocation}
%
% \begin{macro}{new_whatsit}
% \changes{v1.1c}{2017/02/18}{Parameterise count used in tracking}
% Much the same as for attribute allocation in Lua.
%    \begin{macrocode}
local whatsit_count_name = whatsit_count_name or "e@alloc@whatsit@count"
local function new_whatsit(name)
  tex_setcount("global", whatsit_count_name, 
                         tex_count[whatsit_count_name] + 1)
  if tex_count[whatsit_count_name] > 65534 then
    luatexbase_error("No room for a new custom whatsit")
  end
  luatexbase_log("Custom whatsit " .. (name or "") .. " = " ..
                 tex_count[whatsit_count_name])
  return tex_count[whatsit_count_name]
end
luatexbase.new_whatsit = new_whatsit
%    \end{macrocode}
% \end{macro}
%
% \subsection{Bytecode register allocation}
%
% \begin{macro}{new_bytecode}
% \changes{v1.1c}{2017/02/18}{Parameterise count used in tracking}
% Much the same as for attribute allocation in Lua.
% The optional \meta{name} argument is used in the log if given.
%    \begin{macrocode}
local bytecode_count_name =
                         bytecode_count_name or "e@alloc@bytecode@count"
local function new_bytecode(name)
  tex_setcount("global", bytecode_count_name, 
                         tex_count[bytecode_count_name] + 1)
  if tex_count[bytecode_count_name] > 65534 then
    luatexbase_error("No room for a new bytecode register")
  end
  luatexbase_log("Lua bytecode " .. (name or "") .. " = " ..
                 tex_count[bytecode_count_name])
  return tex_count[bytecode_count_name]
end
luatexbase.new_bytecode = new_bytecode
%    \end{macrocode}
% \end{macro}
%
% \subsection{Lua chunk name allocation}
%
% \begin{macro}{new_chunkname}
% \changes{v1.1c}{2017/02/18}{Parameterise count used in tracking}
% As for bytecode registers but also store the name in the
% |lua.name| table.
%    \begin{macrocode}
local chunkname_count_name =
                        chunkname_count_name or "e@alloc@luachunk@count"
local function new_chunkname(name)
  tex_setcount("global", chunkname_count_name, 
                         tex_count[chunkname_count_name] + 1)
  local chunkname_count = tex_count[chunkname_count_name]
  chunkname_count = chunkname_count + 1
  if chunkname_count > 65534 then
    luatexbase_error("No room for a new chunkname")
  end
  lua.name[chunkname_count]=name
  luatexbase_log("Lua chunkname " .. (name or "") .. " = " .. 
                 chunkname_count .. "\n")
  return chunkname_count
end
luatexbase.new_chunkname = new_chunkname
%    \end{macrocode}
% \end{macro}
%
% \subsection{Lua callback management}
%
% The native mechanism for callbacks in Lua\TeX\ allows only one per function.
% That is extremely restrictive and so a mechanism is needed to add and
% remove callbacks from the appropriate hooks.
%
% \subsubsection{Housekeeping}
%
% The main table: keys are callback names, and values are the associated lists
% of functions. More precisely, the entries in the list are tables holding the
% actual function as |func| and the identifying description as |description|.
% Only callbacks with a non-empty list of functions have an entry in this
% list.
%    \begin{macrocode}
local callbacklist = callbacklist or { }
%    \end{macrocode}
%
% Numerical codes for callback types, and name-to-value association (the
% table keys are strings, the values are numbers).
%    \begin{macrocode}
local list, data, exclusive, simple = 1, 2, 3, 4
local types = {
  list      = list,
  data      = data,
  exclusive = exclusive,
  simple    = simple,
}
%    \end{macrocode}
%
% Now, list all predefined callbacks with their current type, based on the
% Lua\TeX{} manual version~1.01. A full list of the currently-available
% callbacks can be obtained using
%  \begin{verbatim}
%    \directlua{
%      for i,_ in pairs(callback.list()) do
%        texio.write_nl("- " .. i)
%      end
%    }
%    \bye
%  \end{verbatim}
% in plain Lua\TeX{}. (Some undocumented callbacks are omitted as they are
% to be removed.)
%    \begin{macrocode}
local callbacktypes = callbacktypes or {
%    \end{macrocode}
%   Section 8.2: file discovery callbacks.
%    \begin{macrocode}
  find_read_file     = exclusive,
  find_write_file    = exclusive,
  find_font_file     = data,
  find_output_file   = data,
  find_format_file   = data,
  find_vf_file       = data,
  find_map_file      = data,
  find_enc_file      = data,
  find_sfd_file      = data,
  find_pk_file       = data,
  find_data_file     = data,
  find_opentype_file = data,
  find_truetype_file = data,
  find_type1_file    = data,
  find_image_file    = data,
%    \end{macrocode}
%
%    \begin{macrocode}
  open_read_file     = exclusive,
  read_font_file     = exclusive,
  read_vf_file       = exclusive,
  read_map_file      = exclusive,
  read_enc_file      = exclusive,
  read_sfd_file      = exclusive,
  read_pk_file       = exclusive,
  read_data_file     = exclusive,
  read_truetype_file = exclusive,
  read_type1_file    = exclusive,
  read_opentype_file = exclusive,
%    \end{macrocode}
% \changes{v1.0m}{2016/02/11}{read\_cidmap\_file added}
% Not currently used by luatex but included for completeness.
% may be used by a font handler.
%    \begin{macrocode}
  find_cidmap_file   = data,
  read_cidmap_file   = exclusive,
%    \end{macrocode}
% Section 8.3: data processing callbacks.
% \changes{v1.0m}{2016/02/11}{token\_filter removed}
%    \begin{macrocode}
  process_input_buffer  = data,
  process_output_buffer = data,
  process_jobname       = data,
%    \end{macrocode}
% Section 8.4: node list processing callbacks.
% \changes{v1.0m}{2016/02/11}
% {process\_rule, [hv]pack\_quality  append\_to\_vlist\_filter added}
% \changes{v1.0n}{2016/03/13}{insert\_local\_par added}
% \changes{v1.0n}{2016/03/13}{contribute\_filter added}
%    \begin{macrocode}
  contribute_filter      = simple,
  buildpage_filter       = simple,
  build_page_insert      = exclusive,
  pre_linebreak_filter   = list,
  linebreak_filter       = list,
  append_to_vlist_filter = list,
  post_linebreak_filter  = list,
  hpack_filter           = list,
  vpack_filter           = list,
  hpack_quality          = list,
  vpack_quality          = list,
  pre_output_filter      = list,
  process_rule           = list,
  hyphenate              = simple,
  ligaturing             = simple,
  kerning                = simple,
  insert_local_par       = simple,
  mlist_to_hlist         = list,
%    \end{macrocode}
% Section 8.5: information reporting callbacks.
% \changes{v1.0m}{2016/02/11}{show\_warning\_message added}
% \changes{v1.0p}{2016/11/17}{call\_edit added}
%    \begin{macrocode}
  pre_dump             = simple,
  start_run            = simple,
  stop_run             = simple,
  start_page_number    = simple,
  stop_page_number     = simple,
  show_error_hook      = simple,
  show_warning_message = simple,
  show_error_message   = simple,
  show_lua_error_hook  = simple,
  start_file           = simple,
  stop_file            = simple,
  call_edit            = simple,
%    \end{macrocode}
% Section 8.6: PDF-related callbacks.
%    \begin{macrocode}
  finish_pdffile = data,
  finish_pdfpage = data,
%    \end{macrocode}
% Section 8.7: font-related callbacks.
% \changes{v1.1e}{2017/03/28}{glyph\_stream\_provider added}
%    \begin{macrocode}
  define_font = exclusive,
  glyph_stream_provider = exclusive,
%    \end{macrocode}
% \changes{v1.0m}{2016/02/11}{pdf\_stream\_filter\_callback removed}
%    \begin{macrocode}
}
luatexbase.callbacktypes=callbacktypes
%    \end{macrocode}
%
% \begin{macro}{callback.register}
% \changes{v1.0a}{2015/09/24}{Function modified}
%   Save the original function for registering callbacks and prevent the
%   original being used. The original is saved in a place that remains
%   available so other more sophisticated code can override the approach
%   taken by the kernel if desired.
%    \begin{macrocode}
local callback_register = callback_register or callback.register
function callback.register()
  luatexbase_error("Attempt to use callback.register() directly\n")
end
%    \end{macrocode}
% \end{macro}
%
% \subsubsection{Handlers}
%
% The handler function is registered into the callback when the
% first function is added to this callback's list. Then, when the callback
% is called, the handler takes care of running all functions in the list.
% When the last function is removed from the callback's list, the handler
% is unregistered.
%
% More precisely, the functions below are used to generate a specialized
% function (closure) for a given callback, which is the actual handler.
%
%
% The way the functions are combined together depends on
% the type of the callback. There are currently 4 types of callback, depending
% on the calling convention of the functions the callback can hold:
% \begin{description}
%   \item[simple] is for functions that don't return anything: they are called
%     in order, all with the same argument;
%   \item[data] is for functions receiving a piece of data of any type
%     except node list head (and possibly other arguments) and returning it
%     (possibly modified): the functions are called in order, and each is
%     passed the return value of the previous (and the other arguments
%     untouched, if any). The return value is that of the last function;
%   \item[list] is a specialized variant of \emph{data} for functions
%     filtering node lists. Such functions may return either the head of a
%     modified node list, or the boolean values |true| or |false|. The
%     functions are chained the same way as for \emph{data} except that for
%     the following. If
%     one function returns |false|, then |false| is immediately returned and
%     the following functions are \emph{not} called. If one function returns
%     |true|, then the same head is passed to the next function. If all
%     functions return |true|, then |true| is returned, otherwise the return
%     value of the last function not returning |true| is used.
%   \item[exclusive] is for functions with more complex signatures; functions in
%     this type of callback are \emph{not} combined: An error is raised if
%     a second callback is registered..
% \end{description}
%
% Handler for |data| callbacks.
%    \begin{macrocode}
local function data_handler(name)
  return function(data, ...)
    for _,i in ipairs(callbacklist[name]) do
      data = i.func(data,...)
    end
    return data
  end
end
%    \end{macrocode}
% Handler for |exclusive| callbacks. We can assume |callbacklist[name]| is not
% empty: otherwise, the function wouldn't be registered in the callback any
% more.
%    \begin{macrocode}
local function exclusive_handler(name)
  return function(...)
    return callbacklist[name][1].func(...)
  end
end
%    \end{macrocode}
% Handler for |list| callbacks.
% \changes{v1.0k}{2015/12/02}{resolve name and i.description (PHG)}
%    \begin{macrocode}
local function list_handler(name)
  return function(head, ...)
    local ret
    local alltrue = true
    for _,i in ipairs(callbacklist[name]) do
      ret = i.func(head, ...)
      if ret == false then
        luatexbase_warning(
          "Function `" .. i.description .. "' returned false\n"
            .. "in callback `" .. name .."'"
         )
         break
      end
      if ret ~= true then
        alltrue = false
        head = ret
      end
    end
    return alltrue and true or head
  end
end
%    \end{macrocode}
% Handler for |simple| callbacks.
%    \begin{macrocode}
local function simple_handler(name)
  return function(...)
    for _,i in ipairs(callbacklist[name]) do
      i.func(...)
    end
  end
end
%    \end{macrocode}
%
% Keep a handlers table for indexed access.
%    \begin{macrocode}
local handlers = {
  [data]      = data_handler,
  [exclusive] = exclusive_handler,
  [list]      = list_handler,
  [simple]    = simple_handler,
}
%    \end{macrocode}
%
% \subsubsection{Public functions for callback management}
%
% Defining user callbacks perhaps should be in package code,
% but impacts on |add_to_callback|.
% If a default function is not required, it may be declared as |false|.
% First we need a list of user callbacks.
%    \begin{macrocode}
local user_callbacks_defaults = { }
%    \end{macrocode}
%
% \begin{macro}{create_callback}
% \changes{v1.0a}{2015/09/24}{Function added}
% \changes{v1.0i}{2015/11/29}{Check name is not nil in error message (PHG)}
% \changes{v1.0k}{2015/12/02}{Give more specific error messages (PHG)}
%   The allocator itself.
%    \begin{macrocode}
local function create_callback(name, ctype, default)
  if not name  or name  == ""
  or not ctype or ctype == ""
  then
    luatexbase_error("Unable to create callback:\n" ..
                     "valid callback name and type required")
  end
  if callbacktypes[name] then
    luatexbase_error("Unable to create callback `" .. name ..
                     "':\ncallback is already defined")
  end
  if default ~= false and type (default) ~= "function" then
    luatexbase_error("Unable to create callback `" .. name ..
                     ":\ndefault is not a function")
   end
  user_callbacks_defaults[name] = default
  callbacktypes[name] = types[ctype]
end
luatexbase.create_callback = create_callback
%    \end{macrocode}
% \end{macro}
%
% \begin{macro}{call_callback}
% \changes{v1.0a}{2015/09/24}{Function added}
% \changes{v1.0i}{2015/11/29}{Check name is not nil in error message (PHG)}
% \changes{v1.0k}{2015/12/02}{Give more specific error messages (PHG)}
%  Call a user defined callback. First check arguments.
%    \begin{macrocode}
local function call_callback(name,...)
  if not name or name == "" then
    luatexbase_error("Unable to create callback:\n" ..
                     "valid callback name required")
  end
  if user_callbacks_defaults[name] == nil then
    luatexbase_error("Unable to call callback `" .. name
                     .. "':\nunknown or empty")
   end
  local l = callbacklist[name]
  local f
  if not l then
    f = user_callbacks_defaults[name]
    if l == false then
	   return nil
	 end
  else
    f = handlers[callbacktypes[name]](name)
  end
  return f(...)
end
luatexbase.call_callback=call_callback
%    \end{macrocode}
% \end{macro}
%
% \begin{macro}{add_to_callback}
% \changes{v1.0a}{2015/09/24}{Function added}
%   Add a function to a callback. First check arguments.
% \changes{v1.0k}{2015/12/02}{Give more specific error messages (PHG)}
%    \begin{macrocode}
local function add_to_callback(name, func, description)
  if not name or name == "" then
    luatexbase_error("Unable to register callback:\n" ..
                     "valid callback name required")
  end
  if not callbacktypes[name] or
    type(func) ~= "function" or
    not description or
    description == "" then
    luatexbase_error(
      "Unable to register callback.\n\n"
        .. "Correct usage:\n"
        .. "add_to_callback(<callback>, <function>, <description>)"
    )
  end
%    \end{macrocode}
%   Then test if this callback is already in use. If not, initialise its list
%   and register the proper handler.
%    \begin{macrocode}
  local l = callbacklist[name]
  if l == nil then
    l = { }
    callbacklist[name] = l
%    \end{macrocode}
% If it is not a user defined callback use the primitive callback register.
%    \begin{macrocode}
    if user_callbacks_defaults[name] == nil then
      callback_register(name, handlers[callbacktypes[name]](name))
    end
  end
%    \end{macrocode}
%  Actually register the function and give an error if more than one
%  |exclusive| one is registered.
%    \begin{macrocode}
  local f = {
    func        = func,
    description = description,
  }
  local priority = #l + 1
  if callbacktypes[name] == exclusive then
    if #l == 1 then
      luatexbase_error(
        "Cannot add second callback to exclusive function\n`" ..
        name .. "'")
    end
  end
  table.insert(l, priority, f)
%    \end{macrocode}
%  Keep user informed.
%    \begin{macrocode}
  luatexbase_log(
    "Inserting `" .. description .. "' at position "
      .. priority .. " in `" .. name .. "'."
  )
end
luatexbase.add_to_callback = add_to_callback
%    \end{macrocode}
% \end{macro}
%
% \begin{macro}{remove_from_callback}
% \changes{v1.0a}{2015/09/24}{Function added}
% \changes{v1.0k}{2015/12/02}{adjust initialisation of cb local (PHG)}
% \changes{v1.0k}{2015/12/02}{Give more specific error messages (PHG)}
%   Remove a function from a callback. First check arguments.
%    \begin{macrocode}
local function remove_from_callback(name, description)
  if not name or name == "" then
    luatexbase_error("Unable to remove function from callback:\n" ..
                     "valid callback name required")
  end
  if not callbacktypes[name] or
    not description or
    description == "" then
    luatexbase_error(
      "Unable to remove function from callback.\n\n"
        .. "Correct usage:\n"
        .. "remove_from_callback(<callback>, <description>)"
    )
  end
  local l = callbacklist[name]
  if not l then
    luatexbase_error(
      "No callback list for `" .. name .. "'\n")
  end
%    \end{macrocode}
%  Loop over the callback's function list until we find a matching entry.
%  Remove it and check if the list is empty: if so, unregister the
%   callback handler.
%    \begin{macrocode}
  local index = false
  for i,j in ipairs(l) do
    if j.description == description then
      index = i
      break
    end
  end
  if not index then
    luatexbase_error(
      "No callback `" .. description .. "' registered for `" ..
      name .. "'\n")
  end
  local cb = l[index]
  table.remove(l, index)
  luatexbase_log(
    "Removing  `" .. description .. "' from `" .. name .. "'."
  )
  if #l == 0 then
    callbacklist[name] = nil
    callback_register(name, nil)
  end
  return cb.func,cb.description
end
luatexbase.remove_from_callback = remove_from_callback
%    \end{macrocode}
% \end{macro}
%
% \begin{macro}{in_callback}
% \changes{v1.0a}{2015/09/24}{Function added}
% \changes{v1.0h}{2015/11/27}{Guard against undefined list latex/4445}
%   Look for a function description in a callback.
%    \begin{macrocode}
local function in_callback(name, description)
  if not name
    or name == ""
    or not callbacklist[name]
    or not callbacktypes[name]
    or not description then
      return false
  end
  for _, i in pairs(callbacklist[name]) do
    if i.description == description then
      return true
    end
  end
  return false
end
luatexbase.in_callback = in_callback
%    \end{macrocode}
% \end{macro}
%
% \begin{macro}{disable_callback}
% \changes{v1.0a}{2015/09/24}{Function added}
%   As we subvert the engine interface we need to provide a way to access
%   this functionality.
%    \begin{macrocode}
local function disable_callback(name)
  if(callbacklist[name] == nil) then
    callback_register(name, false)
  else
    luatexbase_error("Callback list for " .. name .. " not empty")
  end
end
luatexbase.disable_callback = disable_callback
%    \end{macrocode}
% \end{macro}
%
% \begin{macro}{callback_descriptions}
% \changes{v1.0a}{2015/09/24}{Function added}
% \changes{v1.0h}{2015/11/27}{Match test in in-callback latex/4445}
%   List the descriptions of functions registered for the given callback.
%    \begin{macrocode}
local function callback_descriptions (name)
  local d = {}
  if not name
    or name == ""
    or not callbacklist[name]
    or not callbacktypes[name]
    then
    return d
  else
  for k, i in pairs(callbacklist[name]) do
    d[k]= i.description
    end
  end
  return d
end
luatexbase.callback_descriptions =callback_descriptions 
%    \end{macrocode}
% \end{macro}
%
% \begin{macro}{uninstall}
% \changes{v1.0e}{2015/10/02}{Function added}
%   Unlike at the \TeX{} level, we have to provide a back-out mechanism here
%   at the same time as the rest of the code. This is not meant for use by
%   anything other than \textsf{latexrelease}: as such this is
%   \emph{deliberately} not documented for users!
%    \begin{macrocode}
local function uninstall()
  module_info(
    "luatexbase",
    "Uninstalling kernel luatexbase code"
  )
  callback.register = callback_register
  luatexbase = nil
end
luatexbase.uninstall = uninstall
%    \end{macrocode}
% \end{macro}
% \endgroup
%
%    \begin{macrocode}
%</lua>
%    \end{macrocode}
%
% Reset the catcode of |@|.
%    \begin{macrocode}
%<tex>\catcode`\@=\etatcatcode\relax
%    \end{macrocode}
%
%
% \Finale

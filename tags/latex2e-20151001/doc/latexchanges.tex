% \iffalse meta-comment
%
% Copyright 2015
% The LaTeX3 Project and any individual authors listed elsewhere
% in this file. 
% 
% This file is part of the LaTeX base system.
% -------------------------------------------
% 
% It may be distributed and/or modified under the
% conditions of the LaTeX Project Public License, either version 1.3c
% of this license or (at your option) any later version.
% The latest version of this license is in
%    http://www.latex-project.org/lppl.txt
% and version 1.3c or later is part of all distributions of LaTeX 
% version 2005/12/01 or later.
% 
% This file has the LPPL maintenance status "maintained".
% 
% The list of all files belonging to the LaTeX base distribution is
% given in the file `manifest.txt'. See also `legal.txt' for additional
% information.
% 
% The list of derived (unpacked) files belonging to the distribution 
% and covered by LPPL is defined by the unpacking scripts (with 
% extension .ins) which are part of the distribution.
% 
% \fi
% Filename: latexchanges.tex
 
\NeedsTeXFormat{LaTeX2e}
 
\documentclass{ltxguide}
 
\title{Changes to the \LaTeXe{} format}
 
\author{\copyright~Copyright 2015, \LaTeX3 Project Team.\\
   All rights reserved.}
 
\date{2 October 2015}

% a few commands from doc 
\newcommand\Lpack[1]{\mbox{\textsf{#1}}}
\newcommand\DescribeMacro[1]{\texttt{\string#1}}

\makeatletter % -- provide command introduced in new release
              %    so this typesets with an old format

\DeclareTextCommandDefault\textcommabelow[1]
  {\hmode@bgroup\ooalign{\null#1\crcr\hidewidth\raise-.31ex
   \hbox{\check@mathfonts\fontsize\ssf@size\z@
   \math@fontsfalse\selectfont,}\hidewidth}\egroup}
\makeatother
 
\begin{document}
 
\maketitle
 
\tableofcontents

\newpage
 
\section{Introduction}
 
This document describes changes that have been made to the \LaTeX\ format since the 2014/05/01 \LaTeX\ release.
%This document details all the changes to documented functions since the 2014/05/01 \LaTeX\ release.

As announced in \LaTeX\ News~22, the 2015 \LaTeX, release adopts a new
policy. Improvements and bug fixes will be made to the format sources,
with the \Lpack{latexrelease} package being available to revert
changes to use definitions from an earlier format.

\section{Changes  introduced in 2015/10/01 patch~2}

This release fixes the behaviour of the allocation mechanism if the switch
from the standard to extended pool takes place within a group.

\section{Changes  introduced in 2015/10/01 patch~1}
This release allows \Lpack{latexrelease} to revert the 
Lua\TeX{}-specific
changes, in particular fixing an incorrect date in part of the mechanism and
adding a method to disable callback management entirely.

\section{Changes  introduced in \LaTeX\ 2015/10/01}

\subsection{LuaTeX allocation}
Almost all changes at this release relate to incorporating allocation macros for luatex into the format
as done for etex and xetex in 2015/01/01. For details see |ltluatex.dtx| or \LaTeX\ News 22.
|ltluatex.dtx| now forms a new chapter (N) in the documented sources, |source2e.tex|.


\subsection{Increased number of floats}
The default float list has been increased from 18 to 52 registers if
e\TeX\ is available.
The list can be increased further using |\extrafloats| however this
default allocation uses classic registers below 256 so the registers are
also available for |\newinsert| as described below.

\subsection{Improved \texttt{\textbackslash newinsert}}
The command |\newinsert| has been extended to take registers from the lists of free float
registers once the classic register allocation is used up. This should make it highly unlikely
to get ``no room'' errors on register allocation assuming the format is used with
an e-\TeX\ based \TeX\ engine.

\subsection{New accent, \texttt{\textbackslash textcommabelow} (pr/4414)}
The command |\textcommabelow| has been added. This is mainly intended for Romanian letters
\textcommabelow{S}\,\textcommabelow{s}\,\textcommabelow{T}\,\textcommabelow{t}.

\subsection{Unicode 8}
The file |unicode-letters.def| used to initialise character data in
Unicode \TeX\ variants has been regenerated from data files updated
to Unicode~8.

\section{Changes  introduced in 2015/01/01 patch~2}
There were no changes to the format at this release, but the sources were fixed to fix bug latex/4434
affecting bottom float positioning if the \Lpack{latexrelease} package was used.

\section{Changes  introduced in 2015/01/01 patch~1}
This release re-introduces the ``Patch Level'' scheme for identifying
releases between main ``dated'' releases. Early \LaTeXe\ releases
included a mechanism whereby updates could be provided by a 
\emph{patch
  file}. This was mainly intended to allow updates to be made without
downloading the full sources again, which was an important
consideration with download speeds and costs at the time.

The new mechanism incorporates any changes directly into the sources,
but having the patch level identified in the banner allows the
\LaTeX\ release to be identified, even if (as in this case) most of the
changes do not affect the format but affect other base packages such
as \Lpack{latexrelease} and \Lpack{inputenc}. The patch level is shown
in the banner at the start of the job, but does not affect the date
handling of the |\IncludeInRelease| mechanism.

Apart from re-arranging the version banner, the only change in the
format is that |\newtoks| was accidentally defined twice, using the
old and new allocation scheme described in Section \ref{e@alloc}.
The old definition is now only in the \Lpack{latexrelease} package,
for use when emulating old formats.




\section
[Changes between  \LaTeX\ releases 2014/05/01 and 2015/01/01]
{Changes between \LaTeX\ releases 2014/05/01 and 2015/01/01\footnote{Much of this text
is taken from \Lpack{fixltx2e} package which was formerly used to make such changes available separately.}}

\subsection{Support for \LaTeX\ version changes}
\DescribeMacro{\includeInRelease}\arg{date}\oarg{date}\arg{label}\arg{message}\arg{code}

\smallskip

The |\includeInRelease| command has been added to support backward and
forward compatibility for the \LaTeX\ format. It supports the
declaration of conditional code that can be loaded based on options
given to the \Lpack{latexrelease} package. Its use is described in
detail in the \Lpack{latexrelease} package documentation.

\subsection{New Allocation Code\label{e@alloc}}
Previously |\newcount| and related commands were based on classic TeX
and only allocated in the range 0--255. This was extended (in
different ways) for e-\TeX\ in the |etex| package and in the
|xelatex.ini| and |lualatex.ini| files used in those formats. Related
to this the number of boxes allocated to store floats was
limited. This was extended to a certain extent in the |morefloats|
package (by Don Hosek and H.-Martin Muench) but the new allocation
incorporates float allocation directly and supports much larger float
lists using the extended registers.

The new code allocates registers in the full extended range (
$2^{15}-1$ for etex and xelatex, $2^{16}-1$ for lualatex.
In addition a new command |\extrafloats| is provided.

\DescribeMacro{\extrafloats}\arg{number}
\smallskip

This allocates additional registers for the \LaTeX\ float system to
hold figures and tables etc. Similar functionality has been
available via the \Lpack{morefloats} package but this is a different
implementation using extended e-\TeX\ registers when available so
allows many more registers to be reserved for floats as they are
allocated from a pool of 32 or 64 thousand rather than 256 registers,
depending on the engine in use.

\DescribeMacro{\newmarks}\arg{command}
\smallskip

e-\TeX\ only, previously available via the \Lpack{etex} package.
Allocates commands to use the extended  e-\TeX\ mark mechanism.

\DescribeMacro{\newXeTeXintercharclass}\arg{command}
\smallskip

Xe-\TeX\ only, previously in the Xe-\LaTeX\ format, but added via
|xelatex.ini| not part of the core release. 
Allocates commands to use the xe-\TeX\ character class mechanism.


\subsection{e-\TeX\ tracing if available}
|\loggingall| (Usually used via |\tracingall|) is extended to enable
additional e-\TeX\ tracing if e-\TeX\ is available. (|\tracingall|
extension has been available as part of the |etex| package previously).

Also based on code from the |etex| package, a command |\tracingnone|
is added to reverse the effects of |\tracingall| and turn off all primitive
\TeX\ tracing. A new command |\hideoutput| has been  added which resets the
tracing parameters set by |\showoutput|.




\subsection{\texttt{\textbackslash textsubscript} not defined in
   latex.ltx (pr/3492)}

\begin{verbatim}
>>Number:         3492
>>Category:       latex
>>Synopsis:       \textsubscript not defined in latex.ltx
>>Arrival-Date:   Tue Jan 14 23:01:00 CET 2003
>>Originator:     Ionel Mugurel Ciobica

I use \textsubscript much more often than \textsuperscript, and
\textsubscript it is not defined in latex.ltx. Could you please
consider including the definition of \textsubscript in the latex.ltx
for the next versions of LaTeX.    Thank you.
\end{verbatim}

\subsection{\texttt{\textbackslash @} discards spaces when moving
            (pr/3039)}

\begin{verbatim}
>>Number:         3039
>>Category:       latex
>>Synopsis:       \@ discards spaces when moving
>>Arrival-Date:   Sat May 22 09:01:06 1999
>>Originator:     Donald Arseneau
>>Description:
The \@ command expands to \spacefactor\@m in auxiliary files,
which then ignores following spaces when it is reprocessed.
\end{verbatim}

\subsection{1-col fig can come before earlier 2-col fig
           (pr/2346)}

\begin{verbatim}
>>Number:         2346
>>Category:       latex
>>Synopsis:       2-col: 1-col fig can come before earlier 2-col fig
>>Arrival-Date:   Wed Dec 18 15:41:07 1996
>>Originator:     bil kleb
>>Description:
as documented in Lamport's book, p. 198, concerning figure
placement, "a figure will not be printed before an earlier
figure, and a table will not be printed before an earlier
table."  however, there is a footnote stating, "However,
in two-column page style, a single-column figure can come before
an earlier double-column figure, and vice versa."

This twocolumn behavior is undesirable---at least by me and
most professional organizations i publish in.  ed snyzter developed
a hack fix for 2.09 several years ago which links the two
counters, but i have not run across a similar "fix" for 2e...
\end{verbatim}

Originally fixed in package \Lpack{fix2col} which was merged into
this package. Documentation and code from this package have been
merged into this file.

\subsubsection{Notes on the Implementation Strategy}

The standard output routine maintains two lists of floats that have
been `deferred' for later consideration. One list for single column
floats, and one for double column floats (which are always
immediately put onto their deferred list). This mechanism means
that \LaTeX\ `knows' which type of float is contained in each box
by the list that it is processing, but having two lists means
that there is no mechanism for preserving the order between the
floats in each list.

The solution to this problem consists of two small changes to
the output routine.

Firstly, abandon the `double column float list' |\@dbldeferlist|
and change every command where it is used so that instead the
same |\@deferlist| is used as for single column floats.
That one change ensures that double and single column floats
stay in the same sequence, but as \LaTeX\ no longer `knows'
whether a float is double or single column, it will happily
insert a double float into a single column, overprinting the
other column, or the margin.

The second change is to provide an alternative mechanism for
recording the two column floats. \LaTeX\ already has a compact
mechanism for recording float information, an integer count register
assigned to each float records information about the `type' of float
`figure', `table' and the position information `htp' etc.

The type information is stored in the `high' bits, one bit position
(above `32') allocated to each float type. The `low' bits store
information about the allowed positions, one bit each allocated for
|h t b p|.  In the \LaTeX2.09 system, the bit corresponding to `16'
formed a `boundary' between these two sets of information, and it
was never actually used by the system. Ed Sznyter's
\Lpack{fixfloats} package not unreasonably used this position to
store the double column information, setting the bit for double
column floats. Then at each point in the output routine at which a
float is committed to a certain region, an additional check must be
made to check that the float is (or is not) double column. If it
spans the wrong number of columns it is deferred rather than being
added.

Unfortunately the bit `16' is not available in \LaTeXe. It is used
to encode the extra float position possibility `|!|' that was added
in that system. It would be possible to use position `32' and to
move the flags for `table', `figure',\ldots\ up one position, to
start at 64, but this would mean that in principle one less float
type would be supported, and more importantly is likely to break
any other packages that assume anything about the output routine
internals. So here I instead use another mechanism for flagging
double column floats: By default all floats have depth 0pt.
This package arranges that double column ones have depth 1sp.
This information may then be used in the same manner as in
the \Lpack{fixfloats} package, to defer any floats that are not of
the correct column spanning type.



\subsection{Infinite glue found (pr/4023 and pr/2346)}

The fix for pr/2346 did not work as intended when used in conjunction
with |\enlargethispage| as the latter introduced an infinite negative
glue at the bottom of the page. That in turn make a |\vsplit|
operation to get at the column marks invalid.

\subsection{Wrong header for twocolumn (pr/2613)}

\begin{verbatim}
>>Number:         2613
>>Category:       latex
>>Synopsis:       wrong headline for twocolumn
>>Arrival-Date:   Mon Sep 22 16:41:09 1997
>>Originator:   Daniel Reischert
>>Description:
When setting the document in two columns
the headline shows the top mark of the second column,
but it should show the top mark of the first column.
\end{verbatim}

Originally fixed in package \Lpack{fix2col} which was merged into
this package. Documentation and code from this package have been
merged into this file.

\subsubsection{Notes on the Implementation Strategy}

The standard \LaTeX\ twocolumn system works internally by making
each column a separate `page' that is passed independently to \TeX's
page breaker. (Unlike say the \Lpack{multicol} package, where all
columns are gathered together and then split into columns later,
using |\vsplit|.) This means that the primitive \TeX\ marks that are
normally used for header information, are globally reset after the
first column. By default \LaTeX\ does nothing about this.
A good solution is provided by Piet van Oostrum (building on earlier
work of Joe Pallas) in his \Lpack{fixmarks} package.

After the first column box has been collected the mark information
for that box is saved, so that any |\firstmark| can be
`artificially' used to set the page-level marks after the second
column has been collected. (The second column |\firstmark| is not
normally required.) Unfortunately \TeX\ does not provide a direct
way of knowing if any marks are in the page, |\firstmark| always has a
value from previous pages, even if there is no mark in this page.
The solution is to make a copy of the box and then |\vsplit| it
so that any marks show up as |\splitfirstmark|.

The use of |\vsplit| does mean that the output routine will globally
change the value of |\splitfirstmark| and
|\splitbotmark|. The \Lpack{fixmarks} package goes to some trouble
to save and restore these values so that the output routine does
\emph{not} change the values. This part of \Lpack{fixmarks} is not
copied here as it is quite costly (having to be run on every page) and
there is no reason why anyone writing code using |\vsplit| should
allow the output routine to be triggered before the split marks have
been accessed.



\subsection{\texttt{\textbackslash setlength} produces error if
  used with registers like \texttt{\textbackslash dimen0} (pr/3066)}

\begin{verbatim}
>>Number:         3066
>>Category:       latex
>>Synopsis:       \setlength{\dimen0}{10pt}
>>Arrival-Date:   Tue Jul  6 15:01:06 1999
>>Originator:     Heiko Oberdiek
>>Description:
The current implementation of \setlength causes an error,
because the length specification isn't terminated properly.
More safe:
\def\setlength#1#2{#1=#2\relax}
\end{verbatim}

\subsection{Fewer fragile commands}

\begin{verbatim}
>>Number:         3816
>>Category:       latex
>>Synopsis:       Argument of \@sect has an extra }.
>>Arrival-Date:   Sat Oct 22 23:11:01 +0200 2005
>>Originator:     Susanne Wunsch

Use of a \raisebox in \section{} produces the error message
mentioned in the subject.

PR latex/1738 described a similar problem, which has been solved
10 years ago. Protecting the \raisebox with \protect solved my
problem as well, but wouldn't it make sense to have a similar fix
as in the PR?

It is particularly confusing, that an unprotected \raisebox in a
\section*-environment works fine, while in a \section-environment
produces error.
\end{verbatim}

While not technically a bug, in this day and age there are few
reasons why commands taking optional arguments should not be robust.

\subsubsection{Notes on the implementation strategy}

Rather than changing the kernel macros to be robust, we have decided
to add the macro \DescribeMacro{\MakeRobust}|\MakeRobust| in
\Lpack{fixltx2e} so that users can easily turn fragile macros into
robust ones. A macro |\foo| is made robust by doing the simple
|\MakeRobust{\foo}|. \Lpack{fixltx2e} makes the following kernel
macros robust: |\(|, |\)|, |\[|, |\]|,
 |\makebox|, |\savebox|,
|\framebox|, |\parbox|, |\rule| and |\raisebox|.

\ldots TODO \ldots  fleqn vesion of |\[\]| 

\subsection{\texttt{\textbackslash addpenalty} ruins flush-bottom (pr/3073)}

\begin{verbatim}
>>Number:         3073
>>Category:       latex
>>Synopsis:       \addpenalty ruins flush-bottom
>>Arrival-Date:   Sat Jul 17 05:11:05 1999
>>Originator:     Donald Arseneau
>>Description:
Just to keep in mind for further development eh?
A page break at an \addpenalty after \vspace does *not*
give a flush-bottom page.  (The intent of \addpenalty is
apparently just to preserve the flush bottom by putting
the breakpoint `above' the skip.)
\end{verbatim}

\subsection{Within counters only reset next level down (pr4393)}

This is actually implicitly documented behavior in the \LaTeX{}
Manual that states that |\stepcounter| resets all counters marked
``within''. However it means that if, for example, theorems are
numbered within sections and you start a new chapter in a book, the
section counter is reset to zero but the theorem counter is not
until the first section appears. Thus a theorem directly within the
chapter body (without a new section) would show an incremented
number relative to the last theorem of the previous chapter.


For this reason we are now resetting all levels of within in one go
even if that means that some of these resets may happen several times
unnecessarily.


\subsection{Check the optional arguments of floats}

By default LaTeX silently ignores unknown letters in the optional
arguments of floats. |\begin{figure}[tB]| the |B| is ignored so it
acts like |\begin{figure}[t]| However |\begin{figure}[B]| does
\emph{not} act like |\begin{figure}[]| as the check for an empty
argument, or unsupplied argument, is earlier. |[]| causes the
default float placement to be used, but |[B]| means that \emph{no}
float area is allowed and so the float will not be placed until the
next |\clearpage| or end of document, no warning is given.

This package adds a check on each letter, and if it not one of
|!tbhp| then an error is given and the code acts as if |p| had been
used, so that the float may be placed somewhere.

\subsection{\texttt{\textbackslash DeclareMathSizes} only take pts.
    (pr/3693)}

\begin{verbatim}
>>Number:         3693
>>Category:       latex
>>Synopsis:       \DeclareMathSizes only take pts.
>>Arrival-Date:   Fri Jun 11 16:21:00 CEST 2004
>>Originator:     Morten Hoegholm

The last three arguments of \@DeclareMathSizes cannot take a dimension
as argument, making it inconsistent with the rest of the font changing
commands and itself, as the second argument can take a dimension
specification.
\end{verbatim}

\subsection{No hyphenation in first word after float environment (pr/3498)}

\begin{verbatim}
>>Number:         3498
>>Category:       latex
>>Synopsis:       No hyphenation in first word after float environment
>>Arrival-Date:   Thu Jan 30 13:21:00 CET 2003
>>Originator:    Harald Harders

If a float environment (figure, table) is written within a paragraph,
the first word after the environment is not hyphenated.
\end{verbatim}

\subsection{\texttt{\textbackslash fnsymbol} should use text symbols
    (pr/3400)}

\begin{verbatim}
>>Number:         3400
>>Category:       latex
>>Synopsis:       \fnsymbol should use text symbols if possible
>>Arrival-Date:   Fri Jan 04 20:41:00 CET 2002
>>Originator:     Walter Schmidt

The \fnsymbol command can be used in both text and math
mode.  The symbols produced are, however, always taken from
the math fonts.  As a result, they may not match the text
fonts, even if the symbols are actually available, for
instance from the TS1 encoding.  Since \fnsymbol is
primarily used for footnotes in text, this should be fixed,
IMO.
\end{verbatim}





\subsection{\texttt{\textbackslash footnotemark[x]} crashes with fixltx2e.sty
   (pr/3752)}

\begin{verbatim}
>>Number:         3752
>>Category:       tools
>>Synopsis:       feature \footnotemark[x] crashes with fixltx2e.sty
>>Arrival-Date:   Fri Dec 17 10:11:00 +0100 2004
>>Originator:     Stefan Pofahl

 If I use /fnsymbol together with fixltx2e.sty I can not use
 optional parameter [num]
 \footnotemark[1] is not showing the mark number 1 but
 the mark \value{footnote}.
\end{verbatim}
This bug was related to pr/3400, where |\@fnsymbol| was made robust.

\subsubsection{Notes on the implementation strategy}

Pr/3400 made |\@fnsymbol| decide between text-mode and math-mode,
which requires a certain level of robustness somewhere as the
decision between text and math must be made at typesetting time and
not when inside |\protected@edef| or similar commands. One way of
dealing with this is to make sure the value seen by |\@fnsymbol| is
a fully expanded number, which could be handled by code such as
\begin{verbatim}
\def\fnsymbol#1{\expandafter\@fnsymbol
  \expandafter{\the\csname c@#1\endcsname}}
\end{verbatim}
This would be a good solution if everybody used the high level
commands only by writing code like |\fnsymbol{footnote}|. Unfortunately
many classes (including the standard classes) and packages use the
internal forms directly as in |\@fnsymbol\c@footnote| so the easy
solution of changing |\fnsymbol| would break code that had worked for
the past 20~years.

Therefore the implementation here makes |\@fnsymbol| itself a
non-robust command again and instead uses a new robust command
\DescribeMacro{\TextOrMath}|\TextOrMath|, which will take care of
typesetting either the math or the text symbol. In order to do so,
we face an age old problem and unsolvable problem in \TeX: A
reliable test for math mode that doesn't destroy
kerning. Fortunately this problem can be solved when using e\TeX\ so
if you use this as engine for your \LaTeX\ format, as recommended by
the \LaTeX3 Project, you will get a fully functioning |\TextOrMath|
command with no side effects. If you use regular \TeX\ as engine for
your \LaTeX\ format then we have to choose between the lesser of two
evils: 1)~breaking ligatures and preventing kerning or 2)~face the
risk of choosing text-mode at the beginning of an alignment cell,
which was supposed to be math-mode. We have decided upon 1) as is
customary for regular robust commands in \LaTeX.


 
\end{document}
